\subsection{Presentation Goals}
\begin{frame}
    \frametitle{Presentation Goals}
    % Confession: I am not a social scientist. A significant part of preparing for this prelim involved reading and developing ideas
    % that feel original to me but may have a 


    I have the following goals for this presentation:

    \begin{enumerate}
        \item \boldorange{Motivate} why social science and quantitative modeling \textit{must} be more strongly integrated 
        (based on the relations among three types of uncertainty).
        \item \boldorange{Demonstrate} how \texttt{Osier} currently accomplishes this goal.
        \item \boldorange{Propose} future work to enhance \texttt{Osier}'s capabilities and validate its usage.
    \end{enumerate}

    and I hope to show the \boldorange{layered novelty} of this work as a corrolary of the above.

\end{frame}
\subsection{Proposal Overview}
\begin{frame}
    \frametitle{Proposal Overview}

    I propose to:

    \begin{enumerate}
        % \item \textcolor{blue}{\textbf{Deepen}} the theoretical foundations of this work.
        \item \boldblue{Deepen} the theoretical foundations of this work.
        \item \boldblue{Develop} an optimization tool (\texttt{Osier}) that
        \begin{itemize}
            \item addresses three related uncertainties,
            \item closes the gap between technical expertise and public preferences,
            \item enhances justice outcomes related to energy planning.
        \end{itemize}
        \item \boldblue{Validate} this tool by conducting a case study of energy planning processes
        in the Champaign-Urbana region.
    \end{enumerate}

\end{frame}

% \begin{frame}
%     \frametitle{moral relativism}

%     Avoiding moral relativism. For example, in an effort to be inclusive and create a more deliberative democracy,
%     we cannot include voices whose normative premise is antithetical (i.e., exclusionary) to an inclusive normative
%     premise.\\~\\

%     Motivating question: If a more inclusive decision making process produces more legitimate (i.e., ``just'') outcomes,
%     how can members of the public adequately participate in deliberation on issues perceived by experts as highly technical?\\~\\

%     For example, what role could the public possibly play in choosing a reactor design --- let alone whether a nuclear reactor
%     is the right energy technology for their community --- without understanding nuclear energy and energy systems writ large?\\~\\

%     The instinctual response from the nuclear industry, specifically, and scientists and engineers, generally, is to ``debunk''
%     myths about nuclear energy or educate the public on the relevant science based on a theory of science communication called
%     the ``deficit model.'' This is a pathological response from a neurotic and self-conscious discipline that is continually
%     reproduced in spite of the clear failures of this strategy. The result is confusion among many nuclear engineers about
%     why the ignorant public refuses to update their Bayesian priors on nuclear energy after receiving authoritative information
%     about its benefits. While there are trends suggesting the public perception of nuclear energy has improved in recent years,
%     attributing a causal relationship between this improvement and any direct action on the part of the nuclear industry is
%     dubious at best. If you will forgive my speculation, I would hypothesize this shift is influenced more greatly by the 
%     increasing impacts of climate change and the public's growing desperation for solutions, rather than a wholehearted approval
%     of nuclear technology.\\~\\
    
%     Instead, what if we recognized that the public has preferences that could be translated and incorporated into 
% \end{frame}
