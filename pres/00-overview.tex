\subsection{Presentation Goals}
\begin{frame}
    \frametitle{Presentation Goals}
    Confession: I am not a social scientist. A significant part of preparing for this prelim involved reading and developing ideas
    that feel original to me but may have a 


    I have the following goals for this presentation:

    \begin{enumerate}
        \item \boldorange{Motivate} why social science and quantitative modeling \textit{must} be more strongly integrated 
        (based on the relations among three types of uncertainty).
        \item \boldorange{Demonstrate} how \texttt{Osier} currently accomplishes this goal.
        \item \boldorange{Propose} future work to enhance \texttt{Osier}'s capabilities and validate its usage.
    \end{enumerate}

    and I hope to show the \boldorange{layered novelty} of this work as a corrolary of the above.

\end{frame}
\subsection{Proposal Overview}
\begin{frame}
    \frametitle{Proposal Overview}

    I propose to:

    \begin{enumerate}
        % \item \textcolor{blue}{\textbf{Deepen}} the theoretical foundations of this work.
        \item \boldblue{Deepen} the theoretical foundations of this work.
        \item \boldblue{Develop} an optimization tool (\texttt{Osier}) that
        \begin{itemize}
            \item addresses three related uncertainties,
            \item closes the gap between technical expertise and public preferences,
            \item enhances justice outcomes related to energy planning.
        \end{itemize}
        \item \boldblue{Validate} this tool by conducting a case study of energy planning processes
        in the Champaign-Urbana region.
    \end{enumerate}

\end{frame}

\begin{frame}
    \frametitle{moral relativism}

    Avoiding moral relativism. For example, in an effort to be inclusive and create a more deliberative democracy,
    we cannot include voices whose normative premise is antithetical (i.e., exclusionary) to an inclusive normative
    premise.

\end{frame}
