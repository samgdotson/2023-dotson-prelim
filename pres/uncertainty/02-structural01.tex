\begin{frame}
    \frametitle{Structural Uncertainty}

    \begin{columns}
        \column[t]{4cm}
        \begin{figure}
            \centering
            \resizebox{\columnwidth}{!}{
            \begin{tikzpicture}[nodes={text depth=0.25ex,text height=1.25ex distance=1.7cm}]
                    \tikzstyle{every node}=[font=\small]
                    \tikzstyle{vertex} = [circle, draw=black, fill=illiniblue]
                    \tikzstyle{hidden} = [draw=none]
                    \tikzstyle{edge} = [<->, very thick]
                    
                    % \node[vertex](v1) at (0,5) {\textbf{Normative}};
                    \node[vertex](v2) at (4,0) {\textbf{Structural}};

        
            \end{tikzpicture}
            }
            % \caption{Parametric Uncertainty}
            % \label{fig:triarchic-uncertainty}
        \end{figure}

        \column[t]{6cm}
        \begin{definition}[Structural Uncertainty]
            [R]efers to the imperfect and incomplete nature of the equations describing the system \cite{decarolis_using_2011}.
        \end{definition}
        
        This type of uncertainty will \textit{always} persist.
    \end{columns}

\end{frame}

\begin{frame}
    \frametitle{Examples of Structural Uncertainty}

    \begin{itemize}
        \item Objective functions (most typical)\pause
        \item Spatiotemporal resolution\pause
        \item Physics fidelity\pause
        \item Solution method
    \end{itemize}

\end{frame}

\begin{frame}
    \frametitle{Addressing Structural Uncertainty}

    \begin{columns}
        \column[t]{4cm}
        \boldblue{Idea:} Look for alternatives in the ``near-optimal'' space.\\~\\

        How? Modeling-to-generate-alternatives (MGA)
        \begin{enumerate}
            \item \boldorange{Relax} the objective function.
            \item \boldorange{Search} for maximally different solutions in the design space.
            \item \boldorange{Iterate} until enough solutions have been generated.
        \end{enumerate}
    
        \column[t]{6cm}
        \begin{figure}
            \centering
            \resizebox{\columnwidth}{!}{%% Creator: Matplotlib, PGF backend
%%
%% To include the figure in your LaTeX document, write
%%   \input{<filename>.pgf}
%%
%% Make sure the required packages are loaded in your preamble
%%   \usepackage{pgf}
%%
%% Also ensure that all the required font packages are loaded; for instance,
%% the lmodern package is sometimes necessary when using math font.
%%   \usepackage{lmodern}
%%
%% Figures using additional raster images can only be included by \input if
%% they are in the same directory as the main LaTeX file. For loading figures
%% from other directories you can use the `import` package
%%   \usepackage{import}
%%
%% and then include the figures with
%%   \import{<path to file>}{<filename>.pgf}
%%
%% Matplotlib used the following preamble
%%   
%%   \makeatletter\@ifpackageloaded{underscore}{}{\usepackage[strings]{underscore}}\makeatother
%%
\begingroup%
\makeatletter%
\begin{pgfpicture}%
\pgfpathrectangle{\pgfpointorigin}{\pgfqpoint{8.900000in}{6.910000in}}%
\pgfusepath{use as bounding box, clip}%
\begin{pgfscope}%
\pgfsetbuttcap%
\pgfsetmiterjoin%
\definecolor{currentfill}{rgb}{0.827451,0.827451,0.827451}%
\pgfsetfillcolor{currentfill}%
\pgfsetlinewidth{0.000000pt}%
\definecolor{currentstroke}{rgb}{0.000000,0.000000,0.000000}%
\pgfsetstrokecolor{currentstroke}%
\pgfsetdash{}{0pt}%
\pgfpathmoveto{\pgfqpoint{0.000000in}{0.000000in}}%
\pgfpathlineto{\pgfqpoint{8.900000in}{0.000000in}}%
\pgfpathlineto{\pgfqpoint{8.900000in}{6.910000in}}%
\pgfpathlineto{\pgfqpoint{0.000000in}{6.910000in}}%
\pgfpathlineto{\pgfqpoint{0.000000in}{0.000000in}}%
\pgfpathclose%
\pgfusepath{fill}%
\end{pgfscope}%
\begin{pgfscope}%
\pgfsetbuttcap%
\pgfsetmiterjoin%
\definecolor{currentfill}{rgb}{1.000000,1.000000,1.000000}%
\pgfsetfillcolor{currentfill}%
\pgfsetlinewidth{0.000000pt}%
\definecolor{currentstroke}{rgb}{0.000000,0.000000,0.000000}%
\pgfsetstrokecolor{currentstroke}%
\pgfsetstrokeopacity{0.000000}%
\pgfsetdash{}{0pt}%
\pgfpathmoveto{\pgfqpoint{0.716355in}{0.643904in}}%
\pgfpathlineto{\pgfqpoint{8.800000in}{0.643904in}}%
\pgfpathlineto{\pgfqpoint{8.800000in}{6.059445in}}%
\pgfpathlineto{\pgfqpoint{0.716355in}{6.059445in}}%
\pgfpathlineto{\pgfqpoint{0.716355in}{0.643904in}}%
\pgfpathclose%
\pgfusepath{fill}%
\end{pgfscope}%
\begin{pgfscope}%
\pgfpathrectangle{\pgfqpoint{0.716355in}{0.643904in}}{\pgfqpoint{8.083645in}{5.415542in}}%
\pgfusepath{clip}%
\pgfsetbuttcap%
\pgfsetroundjoin%
\definecolor{currentfill}{rgb}{1.000000,1.000000,1.000000}%
\pgfsetfillcolor{currentfill}%
\pgfsetlinewidth{1.003750pt}%
\definecolor{currentstroke}{rgb}{0.000000,0.000000,0.000000}%
\pgfsetstrokecolor{currentstroke}%
\pgfsetdash{}{0pt}%
\pgfsys@defobject{currentmarker}{\pgfqpoint{-0.069444in}{-0.069444in}}{\pgfqpoint{0.069444in}{0.069444in}}{%
\pgfpathmoveto{\pgfqpoint{0.000000in}{-0.069444in}}%
\pgfpathcurveto{\pgfqpoint{0.018417in}{-0.069444in}}{\pgfqpoint{0.036082in}{-0.062127in}}{\pgfqpoint{0.049105in}{-0.049105in}}%
\pgfpathcurveto{\pgfqpoint{0.062127in}{-0.036082in}}{\pgfqpoint{0.069444in}{-0.018417in}}{\pgfqpoint{0.069444in}{0.000000in}}%
\pgfpathcurveto{\pgfqpoint{0.069444in}{0.018417in}}{\pgfqpoint{0.062127in}{0.036082in}}{\pgfqpoint{0.049105in}{0.049105in}}%
\pgfpathcurveto{\pgfqpoint{0.036082in}{0.062127in}}{\pgfqpoint{0.018417in}{0.069444in}}{\pgfqpoint{0.000000in}{0.069444in}}%
\pgfpathcurveto{\pgfqpoint{-0.018417in}{0.069444in}}{\pgfqpoint{-0.036082in}{0.062127in}}{\pgfqpoint{-0.049105in}{0.049105in}}%
\pgfpathcurveto{\pgfqpoint{-0.062127in}{0.036082in}}{\pgfqpoint{-0.069444in}{0.018417in}}{\pgfqpoint{-0.069444in}{0.000000in}}%
\pgfpathcurveto{\pgfqpoint{-0.069444in}{-0.018417in}}{\pgfqpoint{-0.062127in}{-0.036082in}}{\pgfqpoint{-0.049105in}{-0.049105in}}%
\pgfpathcurveto{\pgfqpoint{-0.036082in}{-0.062127in}}{\pgfqpoint{-0.018417in}{-0.069444in}}{\pgfqpoint{0.000000in}{-0.069444in}}%
\pgfpathlineto{\pgfqpoint{0.000000in}{-0.069444in}}%
\pgfpathclose%
\pgfusepath{stroke,fill}%
}%
\begin{pgfscope}%
\pgfsys@transformshift{8.065123in}{0.643904in}%
\pgfsys@useobject{currentmarker}{}%
\end{pgfscope}%
\end{pgfscope}%
\begin{pgfscope}%
\pgfpathrectangle{\pgfqpoint{0.716355in}{0.643904in}}{\pgfqpoint{8.083645in}{5.415542in}}%
\pgfusepath{clip}%
\pgfsetbuttcap%
\pgfsetroundjoin%
\definecolor{currentfill}{rgb}{1.000000,1.000000,1.000000}%
\pgfsetfillcolor{currentfill}%
\pgfsetlinewidth{1.003750pt}%
\definecolor{currentstroke}{rgb}{0.000000,0.000000,0.000000}%
\pgfsetstrokecolor{currentstroke}%
\pgfsetdash{}{0pt}%
\pgfsys@defobject{currentmarker}{\pgfqpoint{-0.069444in}{-0.069444in}}{\pgfqpoint{0.069444in}{0.069444in}}{%
\pgfpathmoveto{\pgfqpoint{0.000000in}{-0.069444in}}%
\pgfpathcurveto{\pgfqpoint{0.018417in}{-0.069444in}}{\pgfqpoint{0.036082in}{-0.062127in}}{\pgfqpoint{0.049105in}{-0.049105in}}%
\pgfpathcurveto{\pgfqpoint{0.062127in}{-0.036082in}}{\pgfqpoint{0.069444in}{-0.018417in}}{\pgfqpoint{0.069444in}{0.000000in}}%
\pgfpathcurveto{\pgfqpoint{0.069444in}{0.018417in}}{\pgfqpoint{0.062127in}{0.036082in}}{\pgfqpoint{0.049105in}{0.049105in}}%
\pgfpathcurveto{\pgfqpoint{0.036082in}{0.062127in}}{\pgfqpoint{0.018417in}{0.069444in}}{\pgfqpoint{0.000000in}{0.069444in}}%
\pgfpathcurveto{\pgfqpoint{-0.018417in}{0.069444in}}{\pgfqpoint{-0.036082in}{0.062127in}}{\pgfqpoint{-0.049105in}{0.049105in}}%
\pgfpathcurveto{\pgfqpoint{-0.062127in}{0.036082in}}{\pgfqpoint{-0.069444in}{0.018417in}}{\pgfqpoint{-0.069444in}{0.000000in}}%
\pgfpathcurveto{\pgfqpoint{-0.069444in}{-0.018417in}}{\pgfqpoint{-0.062127in}{-0.036082in}}{\pgfqpoint{-0.049105in}{-0.049105in}}%
\pgfpathcurveto{\pgfqpoint{-0.036082in}{-0.062127in}}{\pgfqpoint{-0.018417in}{-0.069444in}}{\pgfqpoint{0.000000in}{-0.069444in}}%
\pgfpathlineto{\pgfqpoint{0.000000in}{-0.069444in}}%
\pgfpathclose%
\pgfusepath{stroke,fill}%
}%
\begin{pgfscope}%
\pgfsys@transformshift{0.716355in}{5.567123in}%
\pgfsys@useobject{currentmarker}{}%
\end{pgfscope}%
\end{pgfscope}%
\begin{pgfscope}%
\pgfpathrectangle{\pgfqpoint{0.716355in}{0.643904in}}{\pgfqpoint{8.083645in}{5.415542in}}%
\pgfusepath{clip}%
\pgfsetbuttcap%
\pgfsetroundjoin%
\definecolor{currentfill}{rgb}{0.000000,0.000000,0.000000}%
\pgfsetfillcolor{currentfill}%
\pgfsetlinewidth{1.003750pt}%
\definecolor{currentstroke}{rgb}{0.000000,0.000000,0.000000}%
\pgfsetstrokecolor{currentstroke}%
\pgfsetdash{}{0pt}%
\pgfsys@defobject{currentmarker}{\pgfqpoint{-0.069444in}{-0.069444in}}{\pgfqpoint{0.069444in}{0.069444in}}{%
\pgfpathmoveto{\pgfqpoint{0.000000in}{-0.069444in}}%
\pgfpathcurveto{\pgfqpoint{0.018417in}{-0.069444in}}{\pgfqpoint{0.036082in}{-0.062127in}}{\pgfqpoint{0.049105in}{-0.049105in}}%
\pgfpathcurveto{\pgfqpoint{0.062127in}{-0.036082in}}{\pgfqpoint{0.069444in}{-0.018417in}}{\pgfqpoint{0.069444in}{0.000000in}}%
\pgfpathcurveto{\pgfqpoint{0.069444in}{0.018417in}}{\pgfqpoint{0.062127in}{0.036082in}}{\pgfqpoint{0.049105in}{0.049105in}}%
\pgfpathcurveto{\pgfqpoint{0.036082in}{0.062127in}}{\pgfqpoint{0.018417in}{0.069444in}}{\pgfqpoint{0.000000in}{0.069444in}}%
\pgfpathcurveto{\pgfqpoint{-0.018417in}{0.069444in}}{\pgfqpoint{-0.036082in}{0.062127in}}{\pgfqpoint{-0.049105in}{0.049105in}}%
\pgfpathcurveto{\pgfqpoint{-0.062127in}{0.036082in}}{\pgfqpoint{-0.069444in}{0.018417in}}{\pgfqpoint{-0.069444in}{0.000000in}}%
\pgfpathcurveto{\pgfqpoint{-0.069444in}{-0.018417in}}{\pgfqpoint{-0.062127in}{-0.036082in}}{\pgfqpoint{-0.049105in}{-0.049105in}}%
\pgfpathcurveto{\pgfqpoint{-0.036082in}{-0.062127in}}{\pgfqpoint{-0.018417in}{-0.069444in}}{\pgfqpoint{0.000000in}{-0.069444in}}%
\pgfpathlineto{\pgfqpoint{0.000000in}{-0.069444in}}%
\pgfpathclose%
\pgfusepath{stroke,fill}%
}%
\begin{pgfscope}%
\pgfsys@transformshift{3.655862in}{3.597836in}%
\pgfsys@useobject{currentmarker}{}%
\end{pgfscope}%
\end{pgfscope}%
\begin{pgfscope}%
\pgfpathrectangle{\pgfqpoint{0.716355in}{0.643904in}}{\pgfqpoint{8.083645in}{5.415542in}}%
\pgfusepath{clip}%
\pgfsetrectcap%
\pgfsetroundjoin%
\pgfsetlinewidth{0.803000pt}%
\definecolor{currentstroke}{rgb}{0.690196,0.690196,0.690196}%
\pgfsetstrokecolor{currentstroke}%
\pgfsetstrokeopacity{0.300000}%
\pgfsetdash{}{0pt}%
\pgfpathmoveto{\pgfqpoint{0.716355in}{0.643904in}}%
\pgfpathlineto{\pgfqpoint{0.716355in}{6.059445in}}%
\pgfusepath{stroke}%
\end{pgfscope}%
\begin{pgfscope}%
\pgfsetbuttcap%
\pgfsetroundjoin%
\definecolor{currentfill}{rgb}{0.000000,0.000000,0.000000}%
\pgfsetfillcolor{currentfill}%
\pgfsetlinewidth{0.803000pt}%
\definecolor{currentstroke}{rgb}{0.000000,0.000000,0.000000}%
\pgfsetstrokecolor{currentstroke}%
\pgfsetdash{}{0pt}%
\pgfsys@defobject{currentmarker}{\pgfqpoint{0.000000in}{-0.048611in}}{\pgfqpoint{0.000000in}{0.000000in}}{%
\pgfpathmoveto{\pgfqpoint{0.000000in}{0.000000in}}%
\pgfpathlineto{\pgfqpoint{0.000000in}{-0.048611in}}%
\pgfusepath{stroke,fill}%
}%
\begin{pgfscope}%
\pgfsys@transformshift{0.716355in}{0.643904in}%
\pgfsys@useobject{currentmarker}{}%
\end{pgfscope}%
\end{pgfscope}%
\begin{pgfscope}%
\definecolor{textcolor}{rgb}{0.000000,0.000000,0.000000}%
\pgfsetstrokecolor{textcolor}%
\pgfsetfillcolor{textcolor}%
\pgftext[x=0.716355in,y=0.546682in,,top]{\color{textcolor}\rmfamily\fontsize{14.000000}{16.800000}\selectfont 0.0}%
\end{pgfscope}%
\begin{pgfscope}%
\pgfpathrectangle{\pgfqpoint{0.716355in}{0.643904in}}{\pgfqpoint{8.083645in}{5.415542in}}%
\pgfusepath{clip}%
\pgfsetrectcap%
\pgfsetroundjoin%
\pgfsetlinewidth{0.803000pt}%
\definecolor{currentstroke}{rgb}{0.690196,0.690196,0.690196}%
\pgfsetstrokecolor{currentstroke}%
\pgfsetstrokeopacity{0.300000}%
\pgfsetdash{}{0pt}%
\pgfpathmoveto{\pgfqpoint{2.186108in}{0.643904in}}%
\pgfpathlineto{\pgfqpoint{2.186108in}{6.059445in}}%
\pgfusepath{stroke}%
\end{pgfscope}%
\begin{pgfscope}%
\pgfsetbuttcap%
\pgfsetroundjoin%
\definecolor{currentfill}{rgb}{0.000000,0.000000,0.000000}%
\pgfsetfillcolor{currentfill}%
\pgfsetlinewidth{0.803000pt}%
\definecolor{currentstroke}{rgb}{0.000000,0.000000,0.000000}%
\pgfsetstrokecolor{currentstroke}%
\pgfsetdash{}{0pt}%
\pgfsys@defobject{currentmarker}{\pgfqpoint{0.000000in}{-0.048611in}}{\pgfqpoint{0.000000in}{0.000000in}}{%
\pgfpathmoveto{\pgfqpoint{0.000000in}{0.000000in}}%
\pgfpathlineto{\pgfqpoint{0.000000in}{-0.048611in}}%
\pgfusepath{stroke,fill}%
}%
\begin{pgfscope}%
\pgfsys@transformshift{2.186108in}{0.643904in}%
\pgfsys@useobject{currentmarker}{}%
\end{pgfscope}%
\end{pgfscope}%
\begin{pgfscope}%
\definecolor{textcolor}{rgb}{0.000000,0.000000,0.000000}%
\pgfsetstrokecolor{textcolor}%
\pgfsetfillcolor{textcolor}%
\pgftext[x=2.186108in,y=0.546682in,,top]{\color{textcolor}\rmfamily\fontsize{14.000000}{16.800000}\selectfont 0.2}%
\end{pgfscope}%
\begin{pgfscope}%
\pgfpathrectangle{\pgfqpoint{0.716355in}{0.643904in}}{\pgfqpoint{8.083645in}{5.415542in}}%
\pgfusepath{clip}%
\pgfsetrectcap%
\pgfsetroundjoin%
\pgfsetlinewidth{0.803000pt}%
\definecolor{currentstroke}{rgb}{0.690196,0.690196,0.690196}%
\pgfsetstrokecolor{currentstroke}%
\pgfsetstrokeopacity{0.300000}%
\pgfsetdash{}{0pt}%
\pgfpathmoveto{\pgfqpoint{3.655862in}{0.643904in}}%
\pgfpathlineto{\pgfqpoint{3.655862in}{6.059445in}}%
\pgfusepath{stroke}%
\end{pgfscope}%
\begin{pgfscope}%
\pgfsetbuttcap%
\pgfsetroundjoin%
\definecolor{currentfill}{rgb}{0.000000,0.000000,0.000000}%
\pgfsetfillcolor{currentfill}%
\pgfsetlinewidth{0.803000pt}%
\definecolor{currentstroke}{rgb}{0.000000,0.000000,0.000000}%
\pgfsetstrokecolor{currentstroke}%
\pgfsetdash{}{0pt}%
\pgfsys@defobject{currentmarker}{\pgfqpoint{0.000000in}{-0.048611in}}{\pgfqpoint{0.000000in}{0.000000in}}{%
\pgfpathmoveto{\pgfqpoint{0.000000in}{0.000000in}}%
\pgfpathlineto{\pgfqpoint{0.000000in}{-0.048611in}}%
\pgfusepath{stroke,fill}%
}%
\begin{pgfscope}%
\pgfsys@transformshift{3.655862in}{0.643904in}%
\pgfsys@useobject{currentmarker}{}%
\end{pgfscope}%
\end{pgfscope}%
\begin{pgfscope}%
\definecolor{textcolor}{rgb}{0.000000,0.000000,0.000000}%
\pgfsetstrokecolor{textcolor}%
\pgfsetfillcolor{textcolor}%
\pgftext[x=3.655862in,y=0.546682in,,top]{\color{textcolor}\rmfamily\fontsize{14.000000}{16.800000}\selectfont 0.4}%
\end{pgfscope}%
\begin{pgfscope}%
\pgfpathrectangle{\pgfqpoint{0.716355in}{0.643904in}}{\pgfqpoint{8.083645in}{5.415542in}}%
\pgfusepath{clip}%
\pgfsetrectcap%
\pgfsetroundjoin%
\pgfsetlinewidth{0.803000pt}%
\definecolor{currentstroke}{rgb}{0.690196,0.690196,0.690196}%
\pgfsetstrokecolor{currentstroke}%
\pgfsetstrokeopacity{0.300000}%
\pgfsetdash{}{0pt}%
\pgfpathmoveto{\pgfqpoint{5.125616in}{0.643904in}}%
\pgfpathlineto{\pgfqpoint{5.125616in}{6.059445in}}%
\pgfusepath{stroke}%
\end{pgfscope}%
\begin{pgfscope}%
\pgfsetbuttcap%
\pgfsetroundjoin%
\definecolor{currentfill}{rgb}{0.000000,0.000000,0.000000}%
\pgfsetfillcolor{currentfill}%
\pgfsetlinewidth{0.803000pt}%
\definecolor{currentstroke}{rgb}{0.000000,0.000000,0.000000}%
\pgfsetstrokecolor{currentstroke}%
\pgfsetdash{}{0pt}%
\pgfsys@defobject{currentmarker}{\pgfqpoint{0.000000in}{-0.048611in}}{\pgfqpoint{0.000000in}{0.000000in}}{%
\pgfpathmoveto{\pgfqpoint{0.000000in}{0.000000in}}%
\pgfpathlineto{\pgfqpoint{0.000000in}{-0.048611in}}%
\pgfusepath{stroke,fill}%
}%
\begin{pgfscope}%
\pgfsys@transformshift{5.125616in}{0.643904in}%
\pgfsys@useobject{currentmarker}{}%
\end{pgfscope}%
\end{pgfscope}%
\begin{pgfscope}%
\definecolor{textcolor}{rgb}{0.000000,0.000000,0.000000}%
\pgfsetstrokecolor{textcolor}%
\pgfsetfillcolor{textcolor}%
\pgftext[x=5.125616in,y=0.546682in,,top]{\color{textcolor}\rmfamily\fontsize{14.000000}{16.800000}\selectfont 0.6}%
\end{pgfscope}%
\begin{pgfscope}%
\pgfpathrectangle{\pgfqpoint{0.716355in}{0.643904in}}{\pgfqpoint{8.083645in}{5.415542in}}%
\pgfusepath{clip}%
\pgfsetrectcap%
\pgfsetroundjoin%
\pgfsetlinewidth{0.803000pt}%
\definecolor{currentstroke}{rgb}{0.690196,0.690196,0.690196}%
\pgfsetstrokecolor{currentstroke}%
\pgfsetstrokeopacity{0.300000}%
\pgfsetdash{}{0pt}%
\pgfpathmoveto{\pgfqpoint{6.595369in}{0.643904in}}%
\pgfpathlineto{\pgfqpoint{6.595369in}{6.059445in}}%
\pgfusepath{stroke}%
\end{pgfscope}%
\begin{pgfscope}%
\pgfsetbuttcap%
\pgfsetroundjoin%
\definecolor{currentfill}{rgb}{0.000000,0.000000,0.000000}%
\pgfsetfillcolor{currentfill}%
\pgfsetlinewidth{0.803000pt}%
\definecolor{currentstroke}{rgb}{0.000000,0.000000,0.000000}%
\pgfsetstrokecolor{currentstroke}%
\pgfsetdash{}{0pt}%
\pgfsys@defobject{currentmarker}{\pgfqpoint{0.000000in}{-0.048611in}}{\pgfqpoint{0.000000in}{0.000000in}}{%
\pgfpathmoveto{\pgfqpoint{0.000000in}{0.000000in}}%
\pgfpathlineto{\pgfqpoint{0.000000in}{-0.048611in}}%
\pgfusepath{stroke,fill}%
}%
\begin{pgfscope}%
\pgfsys@transformshift{6.595369in}{0.643904in}%
\pgfsys@useobject{currentmarker}{}%
\end{pgfscope}%
\end{pgfscope}%
\begin{pgfscope}%
\definecolor{textcolor}{rgb}{0.000000,0.000000,0.000000}%
\pgfsetstrokecolor{textcolor}%
\pgfsetfillcolor{textcolor}%
\pgftext[x=6.595369in,y=0.546682in,,top]{\color{textcolor}\rmfamily\fontsize{14.000000}{16.800000}\selectfont 0.8}%
\end{pgfscope}%
\begin{pgfscope}%
\pgfpathrectangle{\pgfqpoint{0.716355in}{0.643904in}}{\pgfqpoint{8.083645in}{5.415542in}}%
\pgfusepath{clip}%
\pgfsetrectcap%
\pgfsetroundjoin%
\pgfsetlinewidth{0.803000pt}%
\definecolor{currentstroke}{rgb}{0.690196,0.690196,0.690196}%
\pgfsetstrokecolor{currentstroke}%
\pgfsetstrokeopacity{0.300000}%
\pgfsetdash{}{0pt}%
\pgfpathmoveto{\pgfqpoint{8.065123in}{0.643904in}}%
\pgfpathlineto{\pgfqpoint{8.065123in}{6.059445in}}%
\pgfusepath{stroke}%
\end{pgfscope}%
\begin{pgfscope}%
\pgfsetbuttcap%
\pgfsetroundjoin%
\definecolor{currentfill}{rgb}{0.000000,0.000000,0.000000}%
\pgfsetfillcolor{currentfill}%
\pgfsetlinewidth{0.803000pt}%
\definecolor{currentstroke}{rgb}{0.000000,0.000000,0.000000}%
\pgfsetstrokecolor{currentstroke}%
\pgfsetdash{}{0pt}%
\pgfsys@defobject{currentmarker}{\pgfqpoint{0.000000in}{-0.048611in}}{\pgfqpoint{0.000000in}{0.000000in}}{%
\pgfpathmoveto{\pgfqpoint{0.000000in}{0.000000in}}%
\pgfpathlineto{\pgfqpoint{0.000000in}{-0.048611in}}%
\pgfusepath{stroke,fill}%
}%
\begin{pgfscope}%
\pgfsys@transformshift{8.065123in}{0.643904in}%
\pgfsys@useobject{currentmarker}{}%
\end{pgfscope}%
\end{pgfscope}%
\begin{pgfscope}%
\definecolor{textcolor}{rgb}{0.000000,0.000000,0.000000}%
\pgfsetstrokecolor{textcolor}%
\pgfsetfillcolor{textcolor}%
\pgftext[x=8.065123in,y=0.546682in,,top]{\color{textcolor}\rmfamily\fontsize{14.000000}{16.800000}\selectfont 1.0}%
\end{pgfscope}%
\begin{pgfscope}%
\definecolor{textcolor}{rgb}{0.000000,0.000000,0.000000}%
\pgfsetstrokecolor{textcolor}%
\pgfsetfillcolor{textcolor}%
\pgftext[x=4.758177in,y=0.313349in,,top]{\color{textcolor}\rmfamily\fontsize{16.000000}{19.200000}\selectfont x1}%
\end{pgfscope}%
\begin{pgfscope}%
\pgfpathrectangle{\pgfqpoint{0.716355in}{0.643904in}}{\pgfqpoint{8.083645in}{5.415542in}}%
\pgfusepath{clip}%
\pgfsetrectcap%
\pgfsetroundjoin%
\pgfsetlinewidth{0.803000pt}%
\definecolor{currentstroke}{rgb}{0.690196,0.690196,0.690196}%
\pgfsetstrokecolor{currentstroke}%
\pgfsetstrokeopacity{0.300000}%
\pgfsetdash{}{0pt}%
\pgfpathmoveto{\pgfqpoint{0.716355in}{0.643904in}}%
\pgfpathlineto{\pgfqpoint{8.800000in}{0.643904in}}%
\pgfusepath{stroke}%
\end{pgfscope}%
\begin{pgfscope}%
\pgfsetbuttcap%
\pgfsetroundjoin%
\definecolor{currentfill}{rgb}{0.000000,0.000000,0.000000}%
\pgfsetfillcolor{currentfill}%
\pgfsetlinewidth{0.803000pt}%
\definecolor{currentstroke}{rgb}{0.000000,0.000000,0.000000}%
\pgfsetstrokecolor{currentstroke}%
\pgfsetdash{}{0pt}%
\pgfsys@defobject{currentmarker}{\pgfqpoint{-0.048611in}{0.000000in}}{\pgfqpoint{-0.000000in}{0.000000in}}{%
\pgfpathmoveto{\pgfqpoint{-0.000000in}{0.000000in}}%
\pgfpathlineto{\pgfqpoint{-0.048611in}{0.000000in}}%
\pgfusepath{stroke,fill}%
}%
\begin{pgfscope}%
\pgfsys@transformshift{0.716355in}{0.643904in}%
\pgfsys@useobject{currentmarker}{}%
\end{pgfscope}%
\end{pgfscope}%
\begin{pgfscope}%
\definecolor{textcolor}{rgb}{0.000000,0.000000,0.000000}%
\pgfsetstrokecolor{textcolor}%
\pgfsetfillcolor{textcolor}%
\pgftext[x=0.368904in, y=0.574459in, left, base]{\color{textcolor}\rmfamily\fontsize{14.000000}{16.800000}\selectfont 0.0}%
\end{pgfscope}%
\begin{pgfscope}%
\pgfpathrectangle{\pgfqpoint{0.716355in}{0.643904in}}{\pgfqpoint{8.083645in}{5.415542in}}%
\pgfusepath{clip}%
\pgfsetrectcap%
\pgfsetroundjoin%
\pgfsetlinewidth{0.803000pt}%
\definecolor{currentstroke}{rgb}{0.690196,0.690196,0.690196}%
\pgfsetstrokecolor{currentstroke}%
\pgfsetstrokeopacity{0.300000}%
\pgfsetdash{}{0pt}%
\pgfpathmoveto{\pgfqpoint{0.716355in}{1.628548in}}%
\pgfpathlineto{\pgfqpoint{8.800000in}{1.628548in}}%
\pgfusepath{stroke}%
\end{pgfscope}%
\begin{pgfscope}%
\pgfsetbuttcap%
\pgfsetroundjoin%
\definecolor{currentfill}{rgb}{0.000000,0.000000,0.000000}%
\pgfsetfillcolor{currentfill}%
\pgfsetlinewidth{0.803000pt}%
\definecolor{currentstroke}{rgb}{0.000000,0.000000,0.000000}%
\pgfsetstrokecolor{currentstroke}%
\pgfsetdash{}{0pt}%
\pgfsys@defobject{currentmarker}{\pgfqpoint{-0.048611in}{0.000000in}}{\pgfqpoint{-0.000000in}{0.000000in}}{%
\pgfpathmoveto{\pgfqpoint{-0.000000in}{0.000000in}}%
\pgfpathlineto{\pgfqpoint{-0.048611in}{0.000000in}}%
\pgfusepath{stroke,fill}%
}%
\begin{pgfscope}%
\pgfsys@transformshift{0.716355in}{1.628548in}%
\pgfsys@useobject{currentmarker}{}%
\end{pgfscope}%
\end{pgfscope}%
\begin{pgfscope}%
\definecolor{textcolor}{rgb}{0.000000,0.000000,0.000000}%
\pgfsetstrokecolor{textcolor}%
\pgfsetfillcolor{textcolor}%
\pgftext[x=0.368904in, y=1.559103in, left, base]{\color{textcolor}\rmfamily\fontsize{14.000000}{16.800000}\selectfont 0.2}%
\end{pgfscope}%
\begin{pgfscope}%
\pgfpathrectangle{\pgfqpoint{0.716355in}{0.643904in}}{\pgfqpoint{8.083645in}{5.415542in}}%
\pgfusepath{clip}%
\pgfsetrectcap%
\pgfsetroundjoin%
\pgfsetlinewidth{0.803000pt}%
\definecolor{currentstroke}{rgb}{0.690196,0.690196,0.690196}%
\pgfsetstrokecolor{currentstroke}%
\pgfsetstrokeopacity{0.300000}%
\pgfsetdash{}{0pt}%
\pgfpathmoveto{\pgfqpoint{0.716355in}{2.613192in}}%
\pgfpathlineto{\pgfqpoint{8.800000in}{2.613192in}}%
\pgfusepath{stroke}%
\end{pgfscope}%
\begin{pgfscope}%
\pgfsetbuttcap%
\pgfsetroundjoin%
\definecolor{currentfill}{rgb}{0.000000,0.000000,0.000000}%
\pgfsetfillcolor{currentfill}%
\pgfsetlinewidth{0.803000pt}%
\definecolor{currentstroke}{rgb}{0.000000,0.000000,0.000000}%
\pgfsetstrokecolor{currentstroke}%
\pgfsetdash{}{0pt}%
\pgfsys@defobject{currentmarker}{\pgfqpoint{-0.048611in}{0.000000in}}{\pgfqpoint{-0.000000in}{0.000000in}}{%
\pgfpathmoveto{\pgfqpoint{-0.000000in}{0.000000in}}%
\pgfpathlineto{\pgfqpoint{-0.048611in}{0.000000in}}%
\pgfusepath{stroke,fill}%
}%
\begin{pgfscope}%
\pgfsys@transformshift{0.716355in}{2.613192in}%
\pgfsys@useobject{currentmarker}{}%
\end{pgfscope}%
\end{pgfscope}%
\begin{pgfscope}%
\definecolor{textcolor}{rgb}{0.000000,0.000000,0.000000}%
\pgfsetstrokecolor{textcolor}%
\pgfsetfillcolor{textcolor}%
\pgftext[x=0.368904in, y=2.543747in, left, base]{\color{textcolor}\rmfamily\fontsize{14.000000}{16.800000}\selectfont 0.4}%
\end{pgfscope}%
\begin{pgfscope}%
\pgfpathrectangle{\pgfqpoint{0.716355in}{0.643904in}}{\pgfqpoint{8.083645in}{5.415542in}}%
\pgfusepath{clip}%
\pgfsetrectcap%
\pgfsetroundjoin%
\pgfsetlinewidth{0.803000pt}%
\definecolor{currentstroke}{rgb}{0.690196,0.690196,0.690196}%
\pgfsetstrokecolor{currentstroke}%
\pgfsetstrokeopacity{0.300000}%
\pgfsetdash{}{0pt}%
\pgfpathmoveto{\pgfqpoint{0.716355in}{3.597836in}}%
\pgfpathlineto{\pgfqpoint{8.800000in}{3.597836in}}%
\pgfusepath{stroke}%
\end{pgfscope}%
\begin{pgfscope}%
\pgfsetbuttcap%
\pgfsetroundjoin%
\definecolor{currentfill}{rgb}{0.000000,0.000000,0.000000}%
\pgfsetfillcolor{currentfill}%
\pgfsetlinewidth{0.803000pt}%
\definecolor{currentstroke}{rgb}{0.000000,0.000000,0.000000}%
\pgfsetstrokecolor{currentstroke}%
\pgfsetdash{}{0pt}%
\pgfsys@defobject{currentmarker}{\pgfqpoint{-0.048611in}{0.000000in}}{\pgfqpoint{-0.000000in}{0.000000in}}{%
\pgfpathmoveto{\pgfqpoint{-0.000000in}{0.000000in}}%
\pgfpathlineto{\pgfqpoint{-0.048611in}{0.000000in}}%
\pgfusepath{stroke,fill}%
}%
\begin{pgfscope}%
\pgfsys@transformshift{0.716355in}{3.597836in}%
\pgfsys@useobject{currentmarker}{}%
\end{pgfscope}%
\end{pgfscope}%
\begin{pgfscope}%
\definecolor{textcolor}{rgb}{0.000000,0.000000,0.000000}%
\pgfsetstrokecolor{textcolor}%
\pgfsetfillcolor{textcolor}%
\pgftext[x=0.368904in, y=3.528391in, left, base]{\color{textcolor}\rmfamily\fontsize{14.000000}{16.800000}\selectfont 0.6}%
\end{pgfscope}%
\begin{pgfscope}%
\pgfpathrectangle{\pgfqpoint{0.716355in}{0.643904in}}{\pgfqpoint{8.083645in}{5.415542in}}%
\pgfusepath{clip}%
\pgfsetrectcap%
\pgfsetroundjoin%
\pgfsetlinewidth{0.803000pt}%
\definecolor{currentstroke}{rgb}{0.690196,0.690196,0.690196}%
\pgfsetstrokecolor{currentstroke}%
\pgfsetstrokeopacity{0.300000}%
\pgfsetdash{}{0pt}%
\pgfpathmoveto{\pgfqpoint{0.716355in}{4.582480in}}%
\pgfpathlineto{\pgfqpoint{8.800000in}{4.582480in}}%
\pgfusepath{stroke}%
\end{pgfscope}%
\begin{pgfscope}%
\pgfsetbuttcap%
\pgfsetroundjoin%
\definecolor{currentfill}{rgb}{0.000000,0.000000,0.000000}%
\pgfsetfillcolor{currentfill}%
\pgfsetlinewidth{0.803000pt}%
\definecolor{currentstroke}{rgb}{0.000000,0.000000,0.000000}%
\pgfsetstrokecolor{currentstroke}%
\pgfsetdash{}{0pt}%
\pgfsys@defobject{currentmarker}{\pgfqpoint{-0.048611in}{0.000000in}}{\pgfqpoint{-0.000000in}{0.000000in}}{%
\pgfpathmoveto{\pgfqpoint{-0.000000in}{0.000000in}}%
\pgfpathlineto{\pgfqpoint{-0.048611in}{0.000000in}}%
\pgfusepath{stroke,fill}%
}%
\begin{pgfscope}%
\pgfsys@transformshift{0.716355in}{4.582480in}%
\pgfsys@useobject{currentmarker}{}%
\end{pgfscope}%
\end{pgfscope}%
\begin{pgfscope}%
\definecolor{textcolor}{rgb}{0.000000,0.000000,0.000000}%
\pgfsetstrokecolor{textcolor}%
\pgfsetfillcolor{textcolor}%
\pgftext[x=0.368904in, y=4.513035in, left, base]{\color{textcolor}\rmfamily\fontsize{14.000000}{16.800000}\selectfont 0.8}%
\end{pgfscope}%
\begin{pgfscope}%
\pgfpathrectangle{\pgfqpoint{0.716355in}{0.643904in}}{\pgfqpoint{8.083645in}{5.415542in}}%
\pgfusepath{clip}%
\pgfsetrectcap%
\pgfsetroundjoin%
\pgfsetlinewidth{0.803000pt}%
\definecolor{currentstroke}{rgb}{0.690196,0.690196,0.690196}%
\pgfsetstrokecolor{currentstroke}%
\pgfsetstrokeopacity{0.300000}%
\pgfsetdash{}{0pt}%
\pgfpathmoveto{\pgfqpoint{0.716355in}{5.567123in}}%
\pgfpathlineto{\pgfqpoint{8.800000in}{5.567123in}}%
\pgfusepath{stroke}%
\end{pgfscope}%
\begin{pgfscope}%
\pgfsetbuttcap%
\pgfsetroundjoin%
\definecolor{currentfill}{rgb}{0.000000,0.000000,0.000000}%
\pgfsetfillcolor{currentfill}%
\pgfsetlinewidth{0.803000pt}%
\definecolor{currentstroke}{rgb}{0.000000,0.000000,0.000000}%
\pgfsetstrokecolor{currentstroke}%
\pgfsetdash{}{0pt}%
\pgfsys@defobject{currentmarker}{\pgfqpoint{-0.048611in}{0.000000in}}{\pgfqpoint{-0.000000in}{0.000000in}}{%
\pgfpathmoveto{\pgfqpoint{-0.000000in}{0.000000in}}%
\pgfpathlineto{\pgfqpoint{-0.048611in}{0.000000in}}%
\pgfusepath{stroke,fill}%
}%
\begin{pgfscope}%
\pgfsys@transformshift{0.716355in}{5.567123in}%
\pgfsys@useobject{currentmarker}{}%
\end{pgfscope}%
\end{pgfscope}%
\begin{pgfscope}%
\definecolor{textcolor}{rgb}{0.000000,0.000000,0.000000}%
\pgfsetstrokecolor{textcolor}%
\pgfsetfillcolor{textcolor}%
\pgftext[x=0.368904in, y=5.497679in, left, base]{\color{textcolor}\rmfamily\fontsize{14.000000}{16.800000}\selectfont 1.0}%
\end{pgfscope}%
\begin{pgfscope}%
\definecolor{textcolor}{rgb}{0.000000,0.000000,0.000000}%
\pgfsetstrokecolor{textcolor}%
\pgfsetfillcolor{textcolor}%
\pgftext[x=0.313349in,y=3.351675in,,bottom,rotate=90.000000]{\color{textcolor}\rmfamily\fontsize{16.000000}{19.200000}\selectfont x2}%
\end{pgfscope}%
\begin{pgfscope}%
\pgfpathrectangle{\pgfqpoint{0.716355in}{0.643904in}}{\pgfqpoint{8.083645in}{5.415542in}}%
\pgfusepath{clip}%
\pgfsetrectcap%
\pgfsetroundjoin%
\pgfsetlinewidth{3.011250pt}%
\definecolor{currentstroke}{rgb}{0.000000,0.000000,0.000000}%
\pgfsetstrokecolor{currentstroke}%
\pgfsetdash{}{0pt}%
\pgfpathmoveto{\pgfqpoint{0.716355in}{5.567123in}}%
\pgfpathlineto{\pgfqpoint{0.941317in}{5.416413in}}%
\pgfpathlineto{\pgfqpoint{1.166279in}{5.265702in}}%
\pgfpathlineto{\pgfqpoint{1.391241in}{5.114991in}}%
\pgfpathlineto{\pgfqpoint{1.616204in}{4.964280in}}%
\pgfpathlineto{\pgfqpoint{1.841166in}{4.813569in}}%
\pgfpathlineto{\pgfqpoint{2.066128in}{4.662859in}}%
\pgfpathlineto{\pgfqpoint{2.291091in}{4.512148in}}%
\pgfpathlineto{\pgfqpoint{2.516053in}{4.361437in}}%
\pgfpathlineto{\pgfqpoint{2.741015in}{4.210726in}}%
\pgfpathlineto{\pgfqpoint{2.965978in}{4.060015in}}%
\pgfpathlineto{\pgfqpoint{3.190940in}{3.909305in}}%
\pgfpathlineto{\pgfqpoint{3.415902in}{3.758594in}}%
\pgfpathlineto{\pgfqpoint{3.640865in}{3.607883in}}%
\pgfpathlineto{\pgfqpoint{3.865827in}{3.457172in}}%
\pgfpathlineto{\pgfqpoint{4.090789in}{3.306461in}}%
\pgfpathlineto{\pgfqpoint{4.315751in}{3.155751in}}%
\pgfpathlineto{\pgfqpoint{4.540714in}{3.005040in}}%
\pgfpathlineto{\pgfqpoint{4.765676in}{2.854329in}}%
\pgfpathlineto{\pgfqpoint{4.990638in}{2.703618in}}%
\pgfpathlineto{\pgfqpoint{5.215601in}{2.552907in}}%
\pgfpathlineto{\pgfqpoint{5.440563in}{2.402196in}}%
\pgfpathlineto{\pgfqpoint{5.665525in}{2.251486in}}%
\pgfpathlineto{\pgfqpoint{5.890488in}{2.100775in}}%
\pgfpathlineto{\pgfqpoint{6.115450in}{1.950064in}}%
\pgfpathlineto{\pgfqpoint{6.340412in}{1.799353in}}%
\pgfpathlineto{\pgfqpoint{6.565374in}{1.648642in}}%
\pgfpathlineto{\pgfqpoint{6.790337in}{1.497932in}}%
\pgfpathlineto{\pgfqpoint{7.015299in}{1.347221in}}%
\pgfpathlineto{\pgfqpoint{7.240261in}{1.196510in}}%
\pgfpathlineto{\pgfqpoint{7.465224in}{1.045799in}}%
\pgfpathlineto{\pgfqpoint{7.690186in}{0.895088in}}%
\pgfpathlineto{\pgfqpoint{7.915148in}{0.744378in}}%
\pgfpathlineto{\pgfqpoint{8.068855in}{0.641404in}}%
\pgfusepath{stroke}%
\end{pgfscope}%
\begin{pgfscope}%
\pgfpathrectangle{\pgfqpoint{0.716355in}{0.643904in}}{\pgfqpoint{8.083645in}{5.415542in}}%
\pgfusepath{clip}%
\pgfsetrectcap%
\pgfsetroundjoin%
\pgfsetlinewidth{3.011250pt}%
\definecolor{currentstroke}{rgb}{0.501961,0.501961,0.501961}%
\pgfsetstrokecolor{currentstroke}%
\pgfsetdash{}{0pt}%
\pgfpathmoveto{\pgfqpoint{0.716355in}{4.582480in}}%
\pgfpathlineto{\pgfqpoint{0.941317in}{4.461911in}}%
\pgfpathlineto{\pgfqpoint{1.166279in}{4.341342in}}%
\pgfpathlineto{\pgfqpoint{1.391241in}{4.220774in}}%
\pgfpathlineto{\pgfqpoint{1.616204in}{4.100205in}}%
\pgfpathlineto{\pgfqpoint{1.841166in}{3.979636in}}%
\pgfpathlineto{\pgfqpoint{2.066128in}{3.859068in}}%
\pgfpathlineto{\pgfqpoint{2.291091in}{3.738499in}}%
\pgfpathlineto{\pgfqpoint{2.516053in}{3.617930in}}%
\pgfpathlineto{\pgfqpoint{2.741015in}{3.497362in}}%
\pgfpathlineto{\pgfqpoint{2.965978in}{3.376793in}}%
\pgfpathlineto{\pgfqpoint{3.190940in}{3.256224in}}%
\pgfpathlineto{\pgfqpoint{3.415902in}{3.135656in}}%
\pgfpathlineto{\pgfqpoint{3.640865in}{3.015087in}}%
\pgfpathlineto{\pgfqpoint{3.865827in}{2.894518in}}%
\pgfpathlineto{\pgfqpoint{4.090789in}{2.773950in}}%
\pgfpathlineto{\pgfqpoint{4.315751in}{2.653381in}}%
\pgfpathlineto{\pgfqpoint{4.540714in}{2.532813in}}%
\pgfpathlineto{\pgfqpoint{4.765676in}{2.412244in}}%
\pgfpathlineto{\pgfqpoint{4.990638in}{2.291675in}}%
\pgfpathlineto{\pgfqpoint{5.215601in}{2.171107in}}%
\pgfpathlineto{\pgfqpoint{5.440563in}{2.050538in}}%
\pgfpathlineto{\pgfqpoint{5.665525in}{1.929969in}}%
\pgfpathlineto{\pgfqpoint{5.890488in}{1.809401in}}%
\pgfpathlineto{\pgfqpoint{6.115450in}{1.688832in}}%
\pgfpathlineto{\pgfqpoint{6.340412in}{1.568263in}}%
\pgfpathlineto{\pgfqpoint{6.565374in}{1.447695in}}%
\pgfpathlineto{\pgfqpoint{6.790337in}{1.327126in}}%
\pgfpathlineto{\pgfqpoint{7.015299in}{1.206557in}}%
\pgfpathlineto{\pgfqpoint{7.240261in}{1.085989in}}%
\pgfpathlineto{\pgfqpoint{7.465224in}{0.965420in}}%
\pgfpathlineto{\pgfqpoint{7.690186in}{0.844851in}}%
\pgfpathlineto{\pgfqpoint{7.915148in}{0.724283in}}%
\pgfpathlineto{\pgfqpoint{8.069788in}{0.641404in}}%
\pgfusepath{stroke}%
\end{pgfscope}%
\begin{pgfscope}%
\pgfpathrectangle{\pgfqpoint{0.716355in}{0.643904in}}{\pgfqpoint{8.083645in}{5.415542in}}%
\pgfusepath{clip}%
\pgfsetbuttcap%
\pgfsetroundjoin%
\pgfsetlinewidth{3.011250pt}%
\definecolor{currentstroke}{rgb}{0.501961,0.501961,0.501961}%
\pgfsetstrokecolor{currentstroke}%
\pgfsetdash{{11.100000pt}{4.800000pt}}{0.000000pt}%
\pgfpathmoveto{\pgfqpoint{0.716355in}{5.173266in}}%
\pgfpathlineto{\pgfqpoint{0.975061in}{5.034612in}}%
\pgfpathlineto{\pgfqpoint{1.233768in}{4.895958in}}%
\pgfpathlineto{\pgfqpoint{1.492475in}{4.757304in}}%
\pgfpathlineto{\pgfqpoint{1.751181in}{4.618650in}}%
\pgfpathlineto{\pgfqpoint{2.009888in}{4.479996in}}%
\pgfpathlineto{\pgfqpoint{2.268594in}{4.341342in}}%
\pgfpathlineto{\pgfqpoint{2.527301in}{4.202688in}}%
\pgfpathlineto{\pgfqpoint{2.786008in}{4.064034in}}%
\pgfpathlineto{\pgfqpoint{3.044714in}{3.925380in}}%
\pgfpathlineto{\pgfqpoint{3.303421in}{3.786726in}}%
\pgfpathlineto{\pgfqpoint{3.562128in}{3.648073in}}%
\pgfpathlineto{\pgfqpoint{3.820834in}{3.509419in}}%
\pgfpathlineto{\pgfqpoint{4.079541in}{3.370765in}}%
\pgfpathlineto{\pgfqpoint{4.338248in}{3.232111in}}%
\pgfpathlineto{\pgfqpoint{4.596954in}{3.093457in}}%
\pgfpathlineto{\pgfqpoint{4.855661in}{2.954803in}}%
\pgfpathlineto{\pgfqpoint{5.114368in}{2.816149in}}%
\pgfpathlineto{\pgfqpoint{5.373074in}{2.677495in}}%
\pgfpathlineto{\pgfqpoint{5.631781in}{2.538841in}}%
\pgfpathlineto{\pgfqpoint{5.890488in}{2.400187in}}%
\pgfpathlineto{\pgfqpoint{6.149194in}{2.261533in}}%
\pgfpathlineto{\pgfqpoint{6.407901in}{2.122879in}}%
\pgfpathlineto{\pgfqpoint{6.666607in}{1.984225in}}%
\pgfpathlineto{\pgfqpoint{6.925314in}{1.845571in}}%
\pgfpathlineto{\pgfqpoint{7.184021in}{1.706917in}}%
\pgfpathlineto{\pgfqpoint{7.442727in}{1.568263in}}%
\pgfpathlineto{\pgfqpoint{7.701434in}{1.429609in}}%
\pgfpathlineto{\pgfqpoint{7.960141in}{1.290955in}}%
\pgfpathlineto{\pgfqpoint{8.218847in}{1.152302in}}%
\pgfpathlineto{\pgfqpoint{8.477554in}{1.013648in}}%
\pgfpathlineto{\pgfqpoint{8.736261in}{0.874994in}}%
\pgfpathlineto{\pgfqpoint{8.802500in}{0.839493in}}%
\pgfusepath{stroke}%
\end{pgfscope}%
\begin{pgfscope}%
\pgfsetrectcap%
\pgfsetmiterjoin%
\pgfsetlinewidth{0.803000pt}%
\definecolor{currentstroke}{rgb}{0.000000,0.000000,0.000000}%
\pgfsetstrokecolor{currentstroke}%
\pgfsetdash{}{0pt}%
\pgfpathmoveto{\pgfqpoint{0.716355in}{0.643904in}}%
\pgfpathlineto{\pgfqpoint{0.716355in}{6.059445in}}%
\pgfusepath{stroke}%
\end{pgfscope}%
\begin{pgfscope}%
\pgfsetrectcap%
\pgfsetmiterjoin%
\pgfsetlinewidth{0.803000pt}%
\definecolor{currentstroke}{rgb}{0.000000,0.000000,0.000000}%
\pgfsetstrokecolor{currentstroke}%
\pgfsetdash{}{0pt}%
\pgfpathmoveto{\pgfqpoint{8.800000in}{0.643904in}}%
\pgfpathlineto{\pgfqpoint{8.800000in}{6.059445in}}%
\pgfusepath{stroke}%
\end{pgfscope}%
\begin{pgfscope}%
\pgfsetrectcap%
\pgfsetmiterjoin%
\pgfsetlinewidth{0.803000pt}%
\definecolor{currentstroke}{rgb}{0.000000,0.000000,0.000000}%
\pgfsetstrokecolor{currentstroke}%
\pgfsetdash{}{0pt}%
\pgfpathmoveto{\pgfqpoint{0.716355in}{0.643904in}}%
\pgfpathlineto{\pgfqpoint{8.800000in}{0.643904in}}%
\pgfusepath{stroke}%
\end{pgfscope}%
\begin{pgfscope}%
\pgfsetrectcap%
\pgfsetmiterjoin%
\pgfsetlinewidth{0.803000pt}%
\definecolor{currentstroke}{rgb}{0.000000,0.000000,0.000000}%
\pgfsetstrokecolor{currentstroke}%
\pgfsetdash{}{0pt}%
\pgfpathmoveto{\pgfqpoint{0.716355in}{6.059445in}}%
\pgfpathlineto{\pgfqpoint{8.800000in}{6.059445in}}%
\pgfusepath{stroke}%
\end{pgfscope}%
\begin{pgfscope}%
\pgfsetroundcap%
\pgfsetroundjoin%
\definecolor{currentfill}{rgb}{0.000000,0.000000,0.000000}%
\pgfsetfillcolor{currentfill}%
\pgfsetlinewidth{1.003750pt}%
\definecolor{currentstroke}{rgb}{0.000000,0.000000,0.000000}%
\pgfsetstrokecolor{currentstroke}%
\pgfsetdash{}{0pt}%
\pgfpathmoveto{\pgfqpoint{2.186476in}{3.828408in}}%
\pgfpathquadraticcurveto{\pgfqpoint{2.255003in}{3.975317in}}{\pgfqpoint{2.323531in}{4.122226in}}%
\pgfpathlineto{\pgfqpoint{2.298357in}{4.133968in}}%
\pgfpathquadraticcurveto{\pgfqpoint{2.352464in}{4.200681in}}{\pgfqpoint{2.406571in}{4.267393in}}%
\pgfpathquadraticcurveto{\pgfqpoint{2.390225in}{4.183067in}}{\pgfqpoint{2.373878in}{4.098740in}}%
\pgfpathlineto{\pgfqpoint{2.348704in}{4.110483in}}%
\pgfpathquadraticcurveto{\pgfqpoint{2.280177in}{3.963574in}}{\pgfqpoint{2.211650in}{3.816666in}}%
\pgfpathlineto{\pgfqpoint{2.186476in}{3.828408in}}%
\pgfpathlineto{\pgfqpoint{2.186476in}{3.828408in}}%
\pgfpathclose%
\pgfusepath{stroke,fill}%
\end{pgfscope}%
\begin{pgfscope}%
\definecolor{textcolor}{rgb}{0.000000,0.000000,0.000000}%
\pgfsetstrokecolor{textcolor}%
\pgfsetfillcolor{textcolor}%
\pgftext[x=2.186108in,y=3.794764in,right,top]{\color{textcolor}\rmfamily\fontsize{16.000000}{19.200000}\selectfont slack}%
\end{pgfscope}%
\begin{pgfscope}%
\pgfsetroundcap%
\pgfsetroundjoin%
\definecolor{currentfill}{rgb}{0.000000,0.000000,0.000000}%
\pgfsetfillcolor{currentfill}%
\pgfsetlinewidth{1.003750pt}%
\definecolor{currentstroke}{rgb}{0.000000,0.000000,0.000000}%
\pgfsetstrokecolor{currentstroke}%
\pgfsetdash{}{0pt}%
\pgfpathmoveto{\pgfqpoint{5.862620in}{3.583947in}}%
\pgfpathquadraticcurveto{\pgfqpoint{4.842576in}{3.583947in}}{\pgfqpoint{3.822532in}{3.583947in}}%
\pgfpathlineto{\pgfqpoint{3.822532in}{3.542280in}}%
\pgfpathquadraticcurveto{\pgfqpoint{3.739197in}{3.570058in}}{\pgfqpoint{3.655862in}{3.597836in}}%
\pgfpathquadraticcurveto{\pgfqpoint{3.739197in}{3.625613in}}{\pgfqpoint{3.822532in}{3.653391in}}%
\pgfpathlineto{\pgfqpoint{3.822532in}{3.611724in}}%
\pgfpathquadraticcurveto{\pgfqpoint{4.842576in}{3.611724in}}{\pgfqpoint{5.862620in}{3.611724in}}%
\pgfpathlineto{\pgfqpoint{5.862620in}{3.583947in}}%
\pgfpathlineto{\pgfqpoint{5.862620in}{3.583947in}}%
\pgfpathclose%
\pgfusepath{stroke,fill}%
\end{pgfscope}%
\begin{pgfscope}%
\definecolor{textcolor}{rgb}{0.000000,0.000000,0.000000}%
\pgfsetstrokecolor{textcolor}%
\pgfsetfillcolor{textcolor}%
\pgftext[x=6.595369in,y=3.597836in,,]{\color{textcolor}\rmfamily\fontsize{16.000000}{19.200000}\selectfont MGA Solution}%
\end{pgfscope}%
\begin{pgfscope}%
\pgfsetroundcap%
\pgfsetroundjoin%
\definecolor{currentfill}{rgb}{0.000000,0.000000,0.000000}%
\pgfsetfillcolor{currentfill}%
\pgfsetlinewidth{1.003750pt}%
\definecolor{currentstroke}{rgb}{0.000000,0.000000,0.000000}%
\pgfsetstrokecolor{currentstroke}%
\pgfsetdash{}{0pt}%
\pgfpathmoveto{\pgfqpoint{6.052183in}{1.037958in}}%
\pgfpathquadraticcurveto{\pgfqpoint{6.978054in}{0.863210in}}{\pgfqpoint{7.903925in}{0.688462in}}%
\pgfpathlineto{\pgfqpoint{7.911652in}{0.729406in}}%
\pgfpathquadraticcurveto{\pgfqpoint{7.988388in}{0.686655in}}{\pgfqpoint{8.065123in}{0.643904in}}%
\pgfpathquadraticcurveto{\pgfqpoint{7.978084in}{0.632063in}}{\pgfqpoint{7.891045in}{0.620223in}}%
\pgfpathlineto{\pgfqpoint{7.898773in}{0.661166in}}%
\pgfpathquadraticcurveto{\pgfqpoint{6.972902in}{0.835914in}}{\pgfqpoint{6.047031in}{1.010662in}}%
\pgfpathlineto{\pgfqpoint{6.052183in}{1.037958in}}%
\pgfpathlineto{\pgfqpoint{6.052183in}{1.037958in}}%
\pgfpathclose%
\pgfusepath{stroke,fill}%
\end{pgfscope}%
\begin{pgfscope}%
\definecolor{textcolor}{rgb}{0.000000,0.000000,0.000000}%
\pgfsetstrokecolor{textcolor}%
\pgfsetfillcolor{textcolor}%
\pgftext[x=5.125616in,y=1.136226in,,]{\color{textcolor}\rmfamily\fontsize{16.000000}{19.200000}\selectfont Optimum Solution}%
\end{pgfscope}%
\begin{pgfscope}%
\definecolor{textcolor}{rgb}{0.000000,0.000000,0.000000}%
\pgfsetstrokecolor{textcolor}%
\pgfsetfillcolor{textcolor}%
\pgftext[x=4.758177in,y=6.142779in,,base]{\color{textcolor}\rmfamily\fontsize{20.000000}{24.000000}\selectfont Design Space}%
\end{pgfscope}%
\begin{pgfscope}%
\pgfsetbuttcap%
\pgfsetmiterjoin%
\definecolor{currentfill}{rgb}{0.300000,0.300000,0.300000}%
\pgfsetfillcolor{currentfill}%
\pgfsetfillopacity{0.500000}%
\pgfsetlinewidth{1.003750pt}%
\definecolor{currentstroke}{rgb}{0.300000,0.300000,0.300000}%
\pgfsetstrokecolor{currentstroke}%
\pgfsetstrokeopacity{0.500000}%
\pgfsetdash{}{0pt}%
\pgfpathmoveto{\pgfqpoint{5.086793in}{4.593088in}}%
\pgfpathlineto{\pgfqpoint{8.633333in}{4.593088in}}%
\pgfpathquadraticcurveto{\pgfqpoint{8.688889in}{4.593088in}}{\pgfqpoint{8.688889in}{4.648644in}}%
\pgfpathlineto{\pgfqpoint{8.688889in}{5.837223in}}%
\pgfpathquadraticcurveto{\pgfqpoint{8.688889in}{5.892779in}}{\pgfqpoint{8.633333in}{5.892779in}}%
\pgfpathlineto{\pgfqpoint{5.086793in}{5.892779in}}%
\pgfpathquadraticcurveto{\pgfqpoint{5.031237in}{5.892779in}}{\pgfqpoint{5.031237in}{5.837223in}}%
\pgfpathlineto{\pgfqpoint{5.031237in}{4.648644in}}%
\pgfpathquadraticcurveto{\pgfqpoint{5.031237in}{4.593088in}}{\pgfqpoint{5.086793in}{4.593088in}}%
\pgfpathlineto{\pgfqpoint{5.086793in}{4.593088in}}%
\pgfpathclose%
\pgfusepath{stroke,fill}%
\end{pgfscope}%
\begin{pgfscope}%
\pgfsetbuttcap%
\pgfsetmiterjoin%
\definecolor{currentfill}{rgb}{1.000000,1.000000,1.000000}%
\pgfsetfillcolor{currentfill}%
\pgfsetlinewidth{1.003750pt}%
\definecolor{currentstroke}{rgb}{0.000000,0.000000,0.000000}%
\pgfsetstrokecolor{currentstroke}%
\pgfsetdash{}{0pt}%
\pgfpathmoveto{\pgfqpoint{5.059015in}{4.620866in}}%
\pgfpathlineto{\pgfqpoint{8.605556in}{4.620866in}}%
\pgfpathquadraticcurveto{\pgfqpoint{8.661111in}{4.620866in}}{\pgfqpoint{8.661111in}{4.676422in}}%
\pgfpathlineto{\pgfqpoint{8.661111in}{5.865001in}}%
\pgfpathquadraticcurveto{\pgfqpoint{8.661111in}{5.920557in}}{\pgfqpoint{8.605556in}{5.920557in}}%
\pgfpathlineto{\pgfqpoint{5.059015in}{5.920557in}}%
\pgfpathquadraticcurveto{\pgfqpoint{5.003460in}{5.920557in}}{\pgfqpoint{5.003460in}{5.865001in}}%
\pgfpathlineto{\pgfqpoint{5.003460in}{4.676422in}}%
\pgfpathquadraticcurveto{\pgfqpoint{5.003460in}{4.620866in}}{\pgfqpoint{5.059015in}{4.620866in}}%
\pgfpathlineto{\pgfqpoint{5.059015in}{4.620866in}}%
\pgfpathclose%
\pgfusepath{stroke,fill}%
\end{pgfscope}%
\begin{pgfscope}%
\pgfsetrectcap%
\pgfsetroundjoin%
\pgfsetlinewidth{3.011250pt}%
\definecolor{currentstroke}{rgb}{0.000000,0.000000,0.000000}%
\pgfsetstrokecolor{currentstroke}%
\pgfsetdash{}{0pt}%
\pgfpathmoveto{\pgfqpoint{5.114571in}{5.706629in}}%
\pgfpathlineto{\pgfqpoint{5.392349in}{5.706629in}}%
\pgfpathlineto{\pgfqpoint{5.670126in}{5.706629in}}%
\pgfusepath{stroke}%
\end{pgfscope}%
\begin{pgfscope}%
\definecolor{textcolor}{rgb}{0.000000,0.000000,0.000000}%
\pgfsetstrokecolor{textcolor}%
\pgfsetfillcolor{textcolor}%
\pgftext[x=5.892349in,y=5.609407in,left,base]{\color{textcolor}\rmfamily\fontsize{20.000000}{24.000000}\selectfont x\(\displaystyle _1\) + x\(\displaystyle _2\) = 1}%
\end{pgfscope}%
\begin{pgfscope}%
\pgfsetrectcap%
\pgfsetroundjoin%
\pgfsetlinewidth{3.011250pt}%
\definecolor{currentstroke}{rgb}{0.501961,0.501961,0.501961}%
\pgfsetstrokecolor{currentstroke}%
\pgfsetdash{}{0pt}%
\pgfpathmoveto{\pgfqpoint{5.114571in}{5.295929in}}%
\pgfpathlineto{\pgfqpoint{5.392349in}{5.295929in}}%
\pgfpathlineto{\pgfqpoint{5.670126in}{5.295929in}}%
\pgfusepath{stroke}%
\end{pgfscope}%
\begin{pgfscope}%
\definecolor{textcolor}{rgb}{0.000000,0.000000,0.000000}%
\pgfsetstrokecolor{textcolor}%
\pgfsetfillcolor{textcolor}%
\pgftext[x=5.892349in,y=5.198707in,left,base]{\color{textcolor}\rmfamily\fontsize{20.000000}{24.000000}\selectfont min(c\(\displaystyle _1\)x\(\displaystyle _1\) + c\(\displaystyle _2\)x\(\displaystyle _2\))}%
\end{pgfscope}%
\begin{pgfscope}%
\pgfsetbuttcap%
\pgfsetroundjoin%
\pgfsetlinewidth{3.011250pt}%
\definecolor{currentstroke}{rgb}{0.501961,0.501961,0.501961}%
\pgfsetstrokecolor{currentstroke}%
\pgfsetdash{{11.100000pt}{4.800000pt}}{0.000000pt}%
\pgfpathmoveto{\pgfqpoint{5.114571in}{4.885228in}}%
\pgfpathlineto{\pgfqpoint{5.392349in}{4.885228in}}%
\pgfpathlineto{\pgfqpoint{5.670126in}{4.885228in}}%
\pgfusepath{stroke}%
\end{pgfscope}%
\begin{pgfscope}%
\definecolor{textcolor}{rgb}{0.000000,0.000000,0.000000}%
\pgfsetstrokecolor{textcolor}%
\pgfsetfillcolor{textcolor}%
\pgftext[x=5.892349in,y=4.788006in,left,base]{\color{textcolor}\rmfamily\fontsize{20.000000}{24.000000}\selectfont c\(\displaystyle _1\)x\(\displaystyle _1\) + c\(\displaystyle _2\)x\(\displaystyle _2\) \(\displaystyle \leq\) c\(\displaystyle _1\)\(\displaystyle \cdot\)slack }%
\end{pgfscope}%
\begin{pgfscope}%
\definecolor{textcolor}{rgb}{0.000000,0.000000,0.000000}%
\pgfsetstrokecolor{textcolor}%
\pgfsetfillcolor{textcolor}%
\pgftext[x=4.450000in,y=6.810000in,,top]{\color{textcolor}\rmfamily\fontsize{24.000000}{28.800000}\selectfont Modeling-to-Generate-Alternatives}%
\end{pgfscope}%
\end{pgfpicture}%
\makeatother%
\endgroup%
}
            \caption{Illustration of the MGA algorithm.}
            \label{fig:standard-mga}
        \end{figure}
    \end{columns}

\end{frame}


\begin{frame}
    \frametitle{How \texttt{Osier} handles structural uncertainty}

    \begin{columns}
        \column[t]{4cm}
        \begin{enumerate}
            \item \boldorange{Reduce} uncertainty by introducing more objectives.
        \end{enumerate}
        \column[t]{6cm}
        \begin{figure}
            \centering
            \resizebox{\columnwidth}{!}{%% Creator: Matplotlib, PGF backend
%%
%% To include the figure in your LaTeX document, write
%%   \input{<filename>.pgf}
%%
%% Make sure the required packages are loaded in your preamble
%%   \usepackage{pgf}
%%
%% Also ensure that all the required font packages are loaded; for instance,
%% the lmodern package is sometimes necessary when using math font.
%%   \usepackage{lmodern}
%%
%% Figures using additional raster images can only be included by \input if
%% they are in the same directory as the main LaTeX file. For loading figures
%% from other directories you can use the `import` package
%%   \usepackage{import}
%%
%% and then include the figures with
%%   \import{<path to file>}{<filename>.pgf}
%%
%% Matplotlib used the following preamble
%%   
%%   \makeatletter\@ifpackageloaded{underscore}{}{\usepackage[strings]{underscore}}\makeatother
%%
\begingroup%
\makeatletter%
\begin{pgfpicture}%
\pgfpathrectangle{\pgfpointorigin}{\pgfqpoint{7.900000in}{5.930000in}}%
\pgfusepath{use as bounding box, clip}%
\begin{pgfscope}%
\pgfsetbuttcap%
\pgfsetmiterjoin%
\definecolor{currentfill}{rgb}{0.827451,0.827451,0.827451}%
\pgfsetfillcolor{currentfill}%
\pgfsetlinewidth{0.000000pt}%
\definecolor{currentstroke}{rgb}{0.000000,0.000000,0.000000}%
\pgfsetstrokecolor{currentstroke}%
\pgfsetdash{}{0pt}%
\pgfpathmoveto{\pgfqpoint{0.000000in}{0.000000in}}%
\pgfpathlineto{\pgfqpoint{7.900000in}{0.000000in}}%
\pgfpathlineto{\pgfqpoint{7.900000in}{5.930000in}}%
\pgfpathlineto{\pgfqpoint{0.000000in}{5.930000in}}%
\pgfpathlineto{\pgfqpoint{0.000000in}{0.000000in}}%
\pgfpathclose%
\pgfusepath{fill}%
\end{pgfscope}%
\begin{pgfscope}%
\pgfsetbuttcap%
\pgfsetmiterjoin%
\definecolor{currentfill}{rgb}{1.000000,1.000000,1.000000}%
\pgfsetfillcolor{currentfill}%
\pgfsetlinewidth{0.000000pt}%
\definecolor{currentstroke}{rgb}{0.000000,0.000000,0.000000}%
\pgfsetstrokecolor{currentstroke}%
\pgfsetstrokeopacity{0.000000}%
\pgfsetdash{}{0pt}%
\pgfpathmoveto{\pgfqpoint{0.882794in}{0.589583in}}%
\pgfpathlineto{\pgfqpoint{7.800000in}{0.589583in}}%
\pgfpathlineto{\pgfqpoint{7.800000in}{5.059445in}}%
\pgfpathlineto{\pgfqpoint{0.882794in}{5.059445in}}%
\pgfpathlineto{\pgfqpoint{0.882794in}{0.589583in}}%
\pgfpathclose%
\pgfusepath{fill}%
\end{pgfscope}%
\begin{pgfscope}%
\pgfpathrectangle{\pgfqpoint{0.882794in}{0.589583in}}{\pgfqpoint{6.917206in}{4.469862in}}%
\pgfusepath{clip}%
\pgfsetbuttcap%
\pgfsetmiterjoin%
\definecolor{currentfill}{rgb}{0.827451,0.827451,0.827451}%
\pgfsetfillcolor{currentfill}%
\pgfsetfillopacity{0.500000}%
\pgfsetlinewidth{0.000000pt}%
\definecolor{currentstroke}{rgb}{0.000000,0.000000,0.000000}%
\pgfsetstrokecolor{currentstroke}%
\pgfsetstrokeopacity{0.500000}%
\pgfsetdash{}{0pt}%
\pgfpathmoveto{\pgfqpoint{0.882794in}{4.258160in}}%
\pgfpathlineto{\pgfqpoint{0.887684in}{4.217744in}}%
\pgfpathlineto{\pgfqpoint{0.892574in}{4.178135in}}%
\pgfpathlineto{\pgfqpoint{0.897463in}{4.139310in}}%
\pgfpathlineto{\pgfqpoint{0.902353in}{4.101245in}}%
\pgfpathlineto{\pgfqpoint{0.907243in}{4.063919in}}%
\pgfpathlineto{\pgfqpoint{0.912132in}{4.027309in}}%
\pgfpathlineto{\pgfqpoint{0.917022in}{3.991396in}}%
\pgfpathlineto{\pgfqpoint{0.921912in}{3.956159in}}%
\pgfpathlineto{\pgfqpoint{0.926801in}{3.921580in}}%
\pgfpathlineto{\pgfqpoint{0.931691in}{3.887641in}}%
\pgfpathlineto{\pgfqpoint{0.936581in}{3.854324in}}%
\pgfpathlineto{\pgfqpoint{0.941470in}{3.821612in}}%
\pgfpathlineto{\pgfqpoint{0.946360in}{3.789489in}}%
\pgfpathlineto{\pgfqpoint{0.951250in}{3.757938in}}%
\pgfpathlineto{\pgfqpoint{0.956139in}{3.726946in}}%
\pgfpathlineto{\pgfqpoint{0.961029in}{3.696496in}}%
\pgfpathlineto{\pgfqpoint{0.965919in}{3.666575in}}%
\pgfpathlineto{\pgfqpoint{0.970808in}{3.637170in}}%
\pgfpathlineto{\pgfqpoint{0.975698in}{3.608267in}}%
\pgfpathlineto{\pgfqpoint{0.980588in}{3.579853in}}%
\pgfpathlineto{\pgfqpoint{0.985477in}{3.551916in}}%
\pgfpathlineto{\pgfqpoint{0.990367in}{3.524445in}}%
\pgfpathlineto{\pgfqpoint{0.995257in}{3.497427in}}%
\pgfpathlineto{\pgfqpoint{1.000146in}{3.470851in}}%
\pgfpathlineto{\pgfqpoint{1.005036in}{3.444707in}}%
\pgfpathlineto{\pgfqpoint{1.009926in}{3.418984in}}%
\pgfpathlineto{\pgfqpoint{1.014815in}{3.393672in}}%
\pgfpathlineto{\pgfqpoint{1.019705in}{3.368762in}}%
\pgfpathlineto{\pgfqpoint{1.024595in}{3.344244in}}%
\pgfpathlineto{\pgfqpoint{1.029484in}{3.320108in}}%
\pgfpathlineto{\pgfqpoint{1.034374in}{3.296346in}}%
\pgfpathlineto{\pgfqpoint{1.039264in}{3.272950in}}%
\pgfpathlineto{\pgfqpoint{1.044153in}{3.249910in}}%
\pgfpathlineto{\pgfqpoint{1.049043in}{3.227219in}}%
\pgfpathlineto{\pgfqpoint{1.053933in}{3.204869in}}%
\pgfpathlineto{\pgfqpoint{1.058822in}{3.182852in}}%
\pgfpathlineto{\pgfqpoint{1.063712in}{3.161161in}}%
\pgfpathlineto{\pgfqpoint{1.068602in}{3.139788in}}%
\pgfpathlineto{\pgfqpoint{1.073491in}{3.118727in}}%
\pgfpathlineto{\pgfqpoint{1.078381in}{3.097972in}}%
\pgfpathlineto{\pgfqpoint{1.083271in}{3.077514in}}%
\pgfpathlineto{\pgfqpoint{1.088160in}{3.057349in}}%
\pgfpathlineto{\pgfqpoint{1.093050in}{3.037469in}}%
\pgfpathlineto{\pgfqpoint{1.097940in}{3.017869in}}%
\pgfpathlineto{\pgfqpoint{1.102829in}{2.998543in}}%
\pgfpathlineto{\pgfqpoint{1.107719in}{2.979486in}}%
\pgfpathlineto{\pgfqpoint{1.112609in}{2.960690in}}%
\pgfpathlineto{\pgfqpoint{1.117498in}{2.942152in}}%
\pgfpathlineto{\pgfqpoint{1.122388in}{2.923866in}}%
\pgfpathlineto{\pgfqpoint{1.127278in}{2.905827in}}%
\pgfpathlineto{\pgfqpoint{1.132167in}{2.888030in}}%
\pgfpathlineto{\pgfqpoint{1.137057in}{2.870470in}}%
\pgfpathlineto{\pgfqpoint{1.141947in}{2.853142in}}%
\pgfpathlineto{\pgfqpoint{1.146837in}{2.836043in}}%
\pgfpathlineto{\pgfqpoint{1.151726in}{2.819166in}}%
\pgfpathlineto{\pgfqpoint{1.156616in}{2.802508in}}%
\pgfpathlineto{\pgfqpoint{1.161506in}{2.786066in}}%
\pgfpathlineto{\pgfqpoint{1.166395in}{2.769834in}}%
\pgfpathlineto{\pgfqpoint{1.171285in}{2.753808in}}%
\pgfpathlineto{\pgfqpoint{1.176175in}{2.737985in}}%
\pgfpathlineto{\pgfqpoint{1.181064in}{2.722361in}}%
\pgfpathlineto{\pgfqpoint{1.185954in}{2.706932in}}%
\pgfpathlineto{\pgfqpoint{1.190844in}{2.691695in}}%
\pgfpathlineto{\pgfqpoint{1.195733in}{2.676645in}}%
\pgfpathlineto{\pgfqpoint{1.200623in}{2.661781in}}%
\pgfpathlineto{\pgfqpoint{1.205513in}{2.647097in}}%
\pgfpathlineto{\pgfqpoint{1.210402in}{2.632591in}}%
\pgfpathlineto{\pgfqpoint{1.215292in}{2.618260in}}%
\pgfpathlineto{\pgfqpoint{1.220182in}{2.604100in}}%
\pgfpathlineto{\pgfqpoint{1.225071in}{2.590108in}}%
\pgfpathlineto{\pgfqpoint{1.229961in}{2.576283in}}%
\pgfpathlineto{\pgfqpoint{1.234851in}{2.562619in}}%
\pgfpathlineto{\pgfqpoint{1.239740in}{2.549116in}}%
\pgfpathlineto{\pgfqpoint{1.244630in}{2.535769in}}%
\pgfpathlineto{\pgfqpoint{1.249520in}{2.522576in}}%
\pgfpathlineto{\pgfqpoint{1.254409in}{2.509535in}}%
\pgfpathlineto{\pgfqpoint{1.259299in}{2.496643in}}%
\pgfpathlineto{\pgfqpoint{1.264189in}{2.483897in}}%
\pgfpathlineto{\pgfqpoint{1.269078in}{2.471295in}}%
\pgfpathlineto{\pgfqpoint{1.273968in}{2.458835in}}%
\pgfpathlineto{\pgfqpoint{1.278858in}{2.446514in}}%
\pgfpathlineto{\pgfqpoint{1.283747in}{2.434329in}}%
\pgfpathlineto{\pgfqpoint{1.288637in}{2.422280in}}%
\pgfpathlineto{\pgfqpoint{1.293527in}{2.410362in}}%
\pgfpathlineto{\pgfqpoint{1.298416in}{2.398575in}}%
\pgfpathlineto{\pgfqpoint{1.303306in}{2.386915in}}%
\pgfpathlineto{\pgfqpoint{1.308196in}{2.375382in}}%
\pgfpathlineto{\pgfqpoint{1.313085in}{2.363972in}}%
\pgfpathlineto{\pgfqpoint{1.317975in}{2.352685in}}%
\pgfpathlineto{\pgfqpoint{1.322865in}{2.341517in}}%
\pgfpathlineto{\pgfqpoint{1.327754in}{2.330468in}}%
\pgfpathlineto{\pgfqpoint{1.332644in}{2.319535in}}%
\pgfpathlineto{\pgfqpoint{1.337534in}{2.308716in}}%
\pgfpathlineto{\pgfqpoint{1.342423in}{2.298010in}}%
\pgfpathlineto{\pgfqpoint{1.347313in}{2.287414in}}%
\pgfpathlineto{\pgfqpoint{1.352203in}{2.276928in}}%
\pgfpathlineto{\pgfqpoint{1.357092in}{2.266549in}}%
\pgfpathlineto{\pgfqpoint{1.361982in}{2.256277in}}%
\pgfpathlineto{\pgfqpoint{1.366872in}{2.246108in}}%
\pgfpathlineto{\pgfqpoint{1.371761in}{2.236042in}}%
\pgfpathlineto{\pgfqpoint{1.376651in}{2.226077in}}%
\pgfpathlineto{\pgfqpoint{1.381541in}{2.216212in}}%
\pgfpathlineto{\pgfqpoint{1.386430in}{2.206445in}}%
\pgfpathlineto{\pgfqpoint{1.391320in}{2.196774in}}%
\pgfpathlineto{\pgfqpoint{1.396210in}{2.187199in}}%
\pgfpathlineto{\pgfqpoint{1.401099in}{2.177717in}}%
\pgfpathlineto{\pgfqpoint{1.405989in}{2.168328in}}%
\pgfpathlineto{\pgfqpoint{1.410879in}{2.159030in}}%
\pgfpathlineto{\pgfqpoint{1.415768in}{2.149822in}}%
\pgfpathlineto{\pgfqpoint{1.420658in}{2.140702in}}%
\pgfpathlineto{\pgfqpoint{1.425548in}{2.131670in}}%
\pgfpathlineto{\pgfqpoint{1.430437in}{2.122723in}}%
\pgfpathlineto{\pgfqpoint{1.435327in}{2.113861in}}%
\pgfpathlineto{\pgfqpoint{1.440217in}{2.105083in}}%
\pgfpathlineto{\pgfqpoint{1.445106in}{2.096387in}}%
\pgfpathlineto{\pgfqpoint{1.449996in}{2.087773in}}%
\pgfpathlineto{\pgfqpoint{1.454886in}{2.079238in}}%
\pgfpathlineto{\pgfqpoint{1.459775in}{2.070782in}}%
\pgfpathlineto{\pgfqpoint{1.464665in}{2.062405in}}%
\pgfpathlineto{\pgfqpoint{1.469555in}{2.054104in}}%
\pgfpathlineto{\pgfqpoint{1.474444in}{2.045879in}}%
\pgfpathlineto{\pgfqpoint{1.479334in}{2.037728in}}%
\pgfpathlineto{\pgfqpoint{1.484224in}{2.029652in}}%
\pgfpathlineto{\pgfqpoint{1.489113in}{2.021648in}}%
\pgfpathlineto{\pgfqpoint{1.494003in}{2.013716in}}%
\pgfpathlineto{\pgfqpoint{1.498893in}{2.005854in}}%
\pgfpathlineto{\pgfqpoint{1.503782in}{1.998063in}}%
\pgfpathlineto{\pgfqpoint{1.508672in}{1.990340in}}%
\pgfpathlineto{\pgfqpoint{1.513562in}{1.982685in}}%
\pgfpathlineto{\pgfqpoint{1.518451in}{1.975098in}}%
\pgfpathlineto{\pgfqpoint{1.523341in}{1.967576in}}%
\pgfpathlineto{\pgfqpoint{1.528231in}{1.960120in}}%
\pgfpathlineto{\pgfqpoint{1.533120in}{1.952729in}}%
\pgfpathlineto{\pgfqpoint{1.538010in}{1.945401in}}%
\pgfpathlineto{\pgfqpoint{1.542900in}{1.938136in}}%
\pgfpathlineto{\pgfqpoint{1.547789in}{1.930933in}}%
\pgfpathlineto{\pgfqpoint{1.552679in}{1.923792in}}%
\pgfpathlineto{\pgfqpoint{1.557569in}{1.916711in}}%
\pgfpathlineto{\pgfqpoint{1.562458in}{1.909689in}}%
\pgfpathlineto{\pgfqpoint{1.567348in}{1.902727in}}%
\pgfpathlineto{\pgfqpoint{1.572238in}{1.895823in}}%
\pgfpathlineto{\pgfqpoint{1.577128in}{1.888976in}}%
\pgfpathlineto{\pgfqpoint{1.582017in}{1.882186in}}%
\pgfpathlineto{\pgfqpoint{1.586907in}{1.875452in}}%
\pgfpathlineto{\pgfqpoint{1.591797in}{1.868774in}}%
\pgfpathlineto{\pgfqpoint{1.596686in}{1.862150in}}%
\pgfpathlineto{\pgfqpoint{1.601576in}{1.855581in}}%
\pgfpathlineto{\pgfqpoint{1.606466in}{1.849064in}}%
\pgfpathlineto{\pgfqpoint{1.611355in}{1.842601in}}%
\pgfpathlineto{\pgfqpoint{1.616245in}{1.836189in}}%
\pgfpathlineto{\pgfqpoint{1.621135in}{1.829830in}}%
\pgfpathlineto{\pgfqpoint{1.626024in}{1.823520in}}%
\pgfpathlineto{\pgfqpoint{1.630914in}{1.817262in}}%
\pgfpathlineto{\pgfqpoint{1.635804in}{1.811052in}}%
\pgfpathlineto{\pgfqpoint{1.640693in}{1.804892in}}%
\pgfpathlineto{\pgfqpoint{1.645583in}{1.798781in}}%
\pgfpathlineto{\pgfqpoint{1.650473in}{1.792717in}}%
\pgfpathlineto{\pgfqpoint{1.655362in}{1.786700in}}%
\pgfpathlineto{\pgfqpoint{1.660252in}{1.780731in}}%
\pgfpathlineto{\pgfqpoint{1.665142in}{1.774807in}}%
\pgfpathlineto{\pgfqpoint{1.670031in}{1.768930in}}%
\pgfpathlineto{\pgfqpoint{1.674921in}{1.763097in}}%
\pgfpathlineto{\pgfqpoint{1.679811in}{1.757309in}}%
\pgfpathlineto{\pgfqpoint{1.684700in}{1.751566in}}%
\pgfpathlineto{\pgfqpoint{1.689590in}{1.745866in}}%
\pgfpathlineto{\pgfqpoint{1.694480in}{1.740209in}}%
\pgfpathlineto{\pgfqpoint{1.699369in}{1.734595in}}%
\pgfpathlineto{\pgfqpoint{1.704259in}{1.729023in}}%
\pgfpathlineto{\pgfqpoint{1.709149in}{1.723493in}}%
\pgfpathlineto{\pgfqpoint{1.714038in}{1.718004in}}%
\pgfpathlineto{\pgfqpoint{1.718928in}{1.712556in}}%
\pgfpathlineto{\pgfqpoint{1.723818in}{1.707148in}}%
\pgfpathlineto{\pgfqpoint{1.728707in}{1.701780in}}%
\pgfpathlineto{\pgfqpoint{1.733597in}{1.696451in}}%
\pgfpathlineto{\pgfqpoint{1.738487in}{1.691162in}}%
\pgfpathlineto{\pgfqpoint{1.743376in}{1.685911in}}%
\pgfpathlineto{\pgfqpoint{1.748266in}{1.680698in}}%
\pgfpathlineto{\pgfqpoint{1.753156in}{1.675523in}}%
\pgfpathlineto{\pgfqpoint{1.758045in}{1.670386in}}%
\pgfpathlineto{\pgfqpoint{1.762935in}{1.665285in}}%
\pgfpathlineto{\pgfqpoint{1.767825in}{1.660221in}}%
\pgfpathlineto{\pgfqpoint{1.772714in}{1.655193in}}%
\pgfpathlineto{\pgfqpoint{1.777604in}{1.650201in}}%
\pgfpathlineto{\pgfqpoint{1.782494in}{1.645244in}}%
\pgfpathlineto{\pgfqpoint{1.787383in}{1.640322in}}%
\pgfpathlineto{\pgfqpoint{1.792273in}{1.635435in}}%
\pgfpathlineto{\pgfqpoint{1.797163in}{1.630582in}}%
\pgfpathlineto{\pgfqpoint{1.802052in}{1.625763in}}%
\pgfpathlineto{\pgfqpoint{1.806942in}{1.620978in}}%
\pgfpathlineto{\pgfqpoint{1.811832in}{1.616225in}}%
\pgfpathlineto{\pgfqpoint{1.816721in}{1.611506in}}%
\pgfpathlineto{\pgfqpoint{1.821611in}{1.606819in}}%
\pgfpathlineto{\pgfqpoint{1.826501in}{1.602165in}}%
\pgfpathlineto{\pgfqpoint{1.831390in}{1.597542in}}%
\pgfpathlineto{\pgfqpoint{1.836280in}{1.592951in}}%
\pgfpathlineto{\pgfqpoint{1.841170in}{1.588391in}}%
\pgfpathlineto{\pgfqpoint{1.846059in}{1.583862in}}%
\pgfpathlineto{\pgfqpoint{1.850949in}{1.579363in}}%
\pgfpathlineto{\pgfqpoint{1.855839in}{1.574895in}}%
\pgfpathlineto{\pgfqpoint{1.860728in}{1.570457in}}%
\pgfpathlineto{\pgfqpoint{1.865618in}{1.566048in}}%
\pgfpathlineto{\pgfqpoint{1.870508in}{1.561669in}}%
\pgfpathlineto{\pgfqpoint{1.875397in}{1.557319in}}%
\pgfpathlineto{\pgfqpoint{1.880287in}{1.552998in}}%
\pgfpathlineto{\pgfqpoint{1.885177in}{1.548705in}}%
\pgfpathlineto{\pgfqpoint{1.890066in}{1.544441in}}%
\pgfpathlineto{\pgfqpoint{1.894956in}{1.540204in}}%
\pgfpathlineto{\pgfqpoint{1.899846in}{1.535995in}}%
\pgfpathlineto{\pgfqpoint{1.904735in}{1.531814in}}%
\pgfpathlineto{\pgfqpoint{1.909625in}{1.527659in}}%
\pgfpathlineto{\pgfqpoint{1.914515in}{1.523532in}}%
\pgfpathlineto{\pgfqpoint{1.919404in}{1.519431in}}%
\pgfpathlineto{\pgfqpoint{1.924294in}{1.515356in}}%
\pgfpathlineto{\pgfqpoint{1.929184in}{1.511308in}}%
\pgfpathlineto{\pgfqpoint{1.934073in}{1.507286in}}%
\pgfpathlineto{\pgfqpoint{1.938963in}{1.503289in}}%
\pgfpathlineto{\pgfqpoint{1.943853in}{1.499317in}}%
\pgfpathlineto{\pgfqpoint{1.948742in}{1.495371in}}%
\pgfpathlineto{\pgfqpoint{1.953632in}{1.491449in}}%
\pgfpathlineto{\pgfqpoint{1.958522in}{1.487552in}}%
\pgfpathlineto{\pgfqpoint{1.963411in}{1.483680in}}%
\pgfpathlineto{\pgfqpoint{1.968301in}{1.479832in}}%
\pgfpathlineto{\pgfqpoint{1.973191in}{1.476007in}}%
\pgfpathlineto{\pgfqpoint{1.978080in}{1.472207in}}%
\pgfpathlineto{\pgfqpoint{1.982970in}{1.468430in}}%
\pgfpathlineto{\pgfqpoint{1.987860in}{1.464676in}}%
\pgfpathlineto{\pgfqpoint{1.992749in}{1.460945in}}%
\pgfpathlineto{\pgfqpoint{1.997639in}{1.457237in}}%
\pgfpathlineto{\pgfqpoint{2.002529in}{1.453552in}}%
\pgfpathlineto{\pgfqpoint{2.007418in}{1.449890in}}%
\pgfpathlineto{\pgfqpoint{2.012308in}{1.446249in}}%
\pgfpathlineto{\pgfqpoint{2.017198in}{1.442631in}}%
\pgfpathlineto{\pgfqpoint{2.022088in}{1.439035in}}%
\pgfpathlineto{\pgfqpoint{2.026977in}{1.435460in}}%
\pgfpathlineto{\pgfqpoint{2.031867in}{1.431906in}}%
\pgfpathlineto{\pgfqpoint{2.036757in}{1.428374in}}%
\pgfpathlineto{\pgfqpoint{2.041646in}{1.424863in}}%
\pgfpathlineto{\pgfqpoint{2.046536in}{1.421373in}}%
\pgfpathlineto{\pgfqpoint{2.051426in}{1.417904in}}%
\pgfpathlineto{\pgfqpoint{2.056315in}{1.414455in}}%
\pgfpathlineto{\pgfqpoint{2.061205in}{1.411026in}}%
\pgfpathlineto{\pgfqpoint{2.066095in}{1.407618in}}%
\pgfpathlineto{\pgfqpoint{2.070984in}{1.404230in}}%
\pgfpathlineto{\pgfqpoint{2.075874in}{1.400861in}}%
\pgfpathlineto{\pgfqpoint{2.080764in}{1.397512in}}%
\pgfpathlineto{\pgfqpoint{2.085653in}{1.394183in}}%
\pgfpathlineto{\pgfqpoint{2.090543in}{1.390872in}}%
\pgfpathlineto{\pgfqpoint{2.095433in}{1.387581in}}%
\pgfpathlineto{\pgfqpoint{2.100322in}{1.384309in}}%
\pgfpathlineto{\pgfqpoint{2.105212in}{1.381056in}}%
\pgfpathlineto{\pgfqpoint{2.110102in}{1.377821in}}%
\pgfpathlineto{\pgfqpoint{2.114991in}{1.374605in}}%
\pgfpathlineto{\pgfqpoint{2.119881in}{1.371407in}}%
\pgfpathlineto{\pgfqpoint{2.124771in}{1.368227in}}%
\pgfpathlineto{\pgfqpoint{2.129660in}{1.365066in}}%
\pgfpathlineto{\pgfqpoint{2.134550in}{1.361922in}}%
\pgfpathlineto{\pgfqpoint{2.139440in}{1.358796in}}%
\pgfpathlineto{\pgfqpoint{2.144329in}{1.355687in}}%
\pgfpathlineto{\pgfqpoint{2.149219in}{1.352596in}}%
\pgfpathlineto{\pgfqpoint{2.154109in}{1.349522in}}%
\pgfpathlineto{\pgfqpoint{2.158998in}{1.346465in}}%
\pgfpathlineto{\pgfqpoint{2.163888in}{1.343425in}}%
\pgfpathlineto{\pgfqpoint{2.168778in}{1.340402in}}%
\pgfpathlineto{\pgfqpoint{2.173667in}{1.337396in}}%
\pgfpathlineto{\pgfqpoint{2.178557in}{1.334406in}}%
\pgfpathlineto{\pgfqpoint{2.183447in}{1.331433in}}%
\pgfpathlineto{\pgfqpoint{2.188336in}{1.328476in}}%
\pgfpathlineto{\pgfqpoint{2.193226in}{1.325535in}}%
\pgfpathlineto{\pgfqpoint{2.198116in}{1.322610in}}%
\pgfpathlineto{\pgfqpoint{2.203005in}{1.319701in}}%
\pgfpathlineto{\pgfqpoint{2.207895in}{1.316808in}}%
\pgfpathlineto{\pgfqpoint{2.212785in}{1.313930in}}%
\pgfpathlineto{\pgfqpoint{2.217674in}{1.311068in}}%
\pgfpathlineto{\pgfqpoint{2.222564in}{1.308222in}}%
\pgfpathlineto{\pgfqpoint{2.227454in}{1.305390in}}%
\pgfpathlineto{\pgfqpoint{2.232343in}{1.302574in}}%
\pgfpathlineto{\pgfqpoint{2.237233in}{1.299773in}}%
\pgfpathlineto{\pgfqpoint{2.242123in}{1.296986in}}%
\pgfpathlineto{\pgfqpoint{2.247012in}{1.294215in}}%
\pgfpathlineto{\pgfqpoint{2.251902in}{1.291458in}}%
\pgfpathlineto{\pgfqpoint{2.256792in}{1.288715in}}%
\pgfpathlineto{\pgfqpoint{2.261681in}{1.285987in}}%
\pgfpathlineto{\pgfqpoint{2.266571in}{1.283274in}}%
\pgfpathlineto{\pgfqpoint{2.271461in}{1.280574in}}%
\pgfpathlineto{\pgfqpoint{2.276350in}{1.277889in}}%
\pgfpathlineto{\pgfqpoint{2.281240in}{1.275218in}}%
\pgfpathlineto{\pgfqpoint{2.286130in}{1.272560in}}%
\pgfpathlineto{\pgfqpoint{2.291019in}{1.269916in}}%
\pgfpathlineto{\pgfqpoint{2.295909in}{1.267286in}}%
\pgfpathlineto{\pgfqpoint{2.300799in}{1.264670in}}%
\pgfpathlineto{\pgfqpoint{2.305688in}{1.262067in}}%
\pgfpathlineto{\pgfqpoint{2.310578in}{1.259477in}}%
\pgfpathlineto{\pgfqpoint{2.315468in}{1.256900in}}%
\pgfpathlineto{\pgfqpoint{2.320357in}{1.254337in}}%
\pgfpathlineto{\pgfqpoint{2.325247in}{1.251787in}}%
\pgfpathlineto{\pgfqpoint{2.330137in}{1.249249in}}%
\pgfpathlineto{\pgfqpoint{2.335026in}{1.246725in}}%
\pgfpathlineto{\pgfqpoint{2.339916in}{1.244213in}}%
\pgfpathlineto{\pgfqpoint{2.344806in}{1.241714in}}%
\pgfpathlineto{\pgfqpoint{2.349695in}{1.239227in}}%
\pgfpathlineto{\pgfqpoint{2.354585in}{1.236753in}}%
\pgfpathlineto{\pgfqpoint{2.359475in}{1.234291in}}%
\pgfpathlineto{\pgfqpoint{2.364364in}{1.231842in}}%
\pgfpathlineto{\pgfqpoint{2.369254in}{1.229405in}}%
\pgfpathlineto{\pgfqpoint{2.374144in}{1.226979in}}%
\pgfpathlineto{\pgfqpoint{2.379033in}{1.224566in}}%
\pgfpathlineto{\pgfqpoint{2.383923in}{1.222165in}}%
\pgfpathlineto{\pgfqpoint{2.388813in}{1.219776in}}%
\pgfpathlineto{\pgfqpoint{2.393702in}{1.217398in}}%
\pgfpathlineto{\pgfqpoint{2.398592in}{1.215032in}}%
\pgfpathlineto{\pgfqpoint{2.403482in}{1.212677in}}%
\pgfpathlineto{\pgfqpoint{2.408371in}{1.210334in}}%
\pgfpathlineto{\pgfqpoint{2.413261in}{1.208003in}}%
\pgfpathlineto{\pgfqpoint{2.418151in}{1.205682in}}%
\pgfpathlineto{\pgfqpoint{2.423040in}{1.203373in}}%
\pgfpathlineto{\pgfqpoint{2.427930in}{1.201075in}}%
\pgfpathlineto{\pgfqpoint{2.432820in}{1.198788in}}%
\pgfpathlineto{\pgfqpoint{2.437709in}{1.196513in}}%
\pgfpathlineto{\pgfqpoint{2.442599in}{1.194248in}}%
\pgfpathlineto{\pgfqpoint{2.447489in}{1.191993in}}%
\pgfpathlineto{\pgfqpoint{2.452379in}{1.189750in}}%
\pgfpathlineto{\pgfqpoint{2.457268in}{1.187517in}}%
\pgfpathlineto{\pgfqpoint{2.462158in}{1.185295in}}%
\pgfpathlineto{\pgfqpoint{2.467048in}{1.183084in}}%
\pgfpathlineto{\pgfqpoint{2.471937in}{1.180883in}}%
\pgfpathlineto{\pgfqpoint{2.476827in}{1.178692in}}%
\pgfpathlineto{\pgfqpoint{2.481717in}{1.176512in}}%
\pgfpathlineto{\pgfqpoint{2.486606in}{1.174341in}}%
\pgfpathlineto{\pgfqpoint{2.491496in}{1.172181in}}%
\pgfpathlineto{\pgfqpoint{2.496386in}{1.170031in}}%
\pgfpathlineto{\pgfqpoint{2.501275in}{1.167891in}}%
\pgfpathlineto{\pgfqpoint{2.506165in}{1.165762in}}%
\pgfpathlineto{\pgfqpoint{2.511055in}{1.163641in}}%
\pgfpathlineto{\pgfqpoint{2.515944in}{1.161531in}}%
\pgfpathlineto{\pgfqpoint{2.520834in}{1.159431in}}%
\pgfpathlineto{\pgfqpoint{2.525724in}{1.157340in}}%
\pgfpathlineto{\pgfqpoint{2.530613in}{1.155259in}}%
\pgfpathlineto{\pgfqpoint{2.535503in}{1.153187in}}%
\pgfpathlineto{\pgfqpoint{2.540393in}{1.151125in}}%
\pgfpathlineto{\pgfqpoint{2.545282in}{1.149072in}}%
\pgfpathlineto{\pgfqpoint{2.550172in}{1.147028in}}%
\pgfpathlineto{\pgfqpoint{2.555062in}{1.144994in}}%
\pgfpathlineto{\pgfqpoint{2.559951in}{1.142969in}}%
\pgfpathlineto{\pgfqpoint{2.564841in}{1.140954in}}%
\pgfpathlineto{\pgfqpoint{2.569731in}{1.138947in}}%
\pgfpathlineto{\pgfqpoint{2.574620in}{1.136949in}}%
\pgfpathlineto{\pgfqpoint{2.579510in}{1.134961in}}%
\pgfpathlineto{\pgfqpoint{2.584400in}{1.132981in}}%
\pgfpathlineto{\pgfqpoint{2.589289in}{1.131010in}}%
\pgfpathlineto{\pgfqpoint{2.594179in}{1.129048in}}%
\pgfpathlineto{\pgfqpoint{2.599069in}{1.127094in}}%
\pgfpathlineto{\pgfqpoint{2.603958in}{1.125150in}}%
\pgfpathlineto{\pgfqpoint{2.608848in}{1.123214in}}%
\pgfpathlineto{\pgfqpoint{2.613738in}{1.121286in}}%
\pgfpathlineto{\pgfqpoint{2.618627in}{1.119367in}}%
\pgfpathlineto{\pgfqpoint{2.623517in}{1.117457in}}%
\pgfpathlineto{\pgfqpoint{2.628407in}{1.115554in}}%
\pgfpathlineto{\pgfqpoint{2.633296in}{1.113661in}}%
\pgfpathlineto{\pgfqpoint{2.638186in}{1.111775in}}%
\pgfpathlineto{\pgfqpoint{2.643076in}{1.109898in}}%
\pgfpathlineto{\pgfqpoint{2.647965in}{1.108029in}}%
\pgfpathlineto{\pgfqpoint{2.652855in}{1.106168in}}%
\pgfpathlineto{\pgfqpoint{2.657745in}{1.104315in}}%
\pgfpathlineto{\pgfqpoint{2.662634in}{1.102470in}}%
\pgfpathlineto{\pgfqpoint{2.667524in}{1.100633in}}%
\pgfpathlineto{\pgfqpoint{2.672414in}{1.098804in}}%
\pgfpathlineto{\pgfqpoint{2.677303in}{1.096983in}}%
\pgfpathlineto{\pgfqpoint{2.682193in}{1.095170in}}%
\pgfpathlineto{\pgfqpoint{2.687083in}{1.093364in}}%
\pgfpathlineto{\pgfqpoint{2.691972in}{1.091566in}}%
\pgfpathlineto{\pgfqpoint{2.696862in}{1.089776in}}%
\pgfpathlineto{\pgfqpoint{2.701752in}{1.087993in}}%
\pgfpathlineto{\pgfqpoint{2.706641in}{1.086218in}}%
\pgfpathlineto{\pgfqpoint{2.711531in}{1.084451in}}%
\pgfpathlineto{\pgfqpoint{2.716421in}{1.082691in}}%
\pgfpathlineto{\pgfqpoint{2.721310in}{1.080938in}}%
\pgfpathlineto{\pgfqpoint{2.726200in}{1.079193in}}%
\pgfpathlineto{\pgfqpoint{2.731090in}{1.077455in}}%
\pgfpathlineto{\pgfqpoint{2.735979in}{1.075724in}}%
\pgfpathlineto{\pgfqpoint{2.740869in}{1.074001in}}%
\pgfpathlineto{\pgfqpoint{2.745759in}{1.072285in}}%
\pgfpathlineto{\pgfqpoint{2.750648in}{1.070576in}}%
\pgfpathlineto{\pgfqpoint{2.755538in}{1.068874in}}%
\pgfpathlineto{\pgfqpoint{2.760428in}{1.067179in}}%
\pgfpathlineto{\pgfqpoint{2.765317in}{1.065491in}}%
\pgfpathlineto{\pgfqpoint{2.770207in}{1.063810in}}%
\pgfpathlineto{\pgfqpoint{2.775097in}{1.062136in}}%
\pgfpathlineto{\pgfqpoint{2.779986in}{1.060469in}}%
\pgfpathlineto{\pgfqpoint{2.784876in}{1.058809in}}%
\pgfpathlineto{\pgfqpoint{2.789766in}{1.057156in}}%
\pgfpathlineto{\pgfqpoint{2.794655in}{1.055509in}}%
\pgfpathlineto{\pgfqpoint{2.799545in}{1.053869in}}%
\pgfpathlineto{\pgfqpoint{2.804435in}{1.052236in}}%
\pgfpathlineto{\pgfqpoint{2.809324in}{1.050609in}}%
\pgfpathlineto{\pgfqpoint{2.814214in}{1.048989in}}%
\pgfpathlineto{\pgfqpoint{2.819104in}{1.047376in}}%
\pgfpathlineto{\pgfqpoint{2.823993in}{1.045769in}}%
\pgfpathlineto{\pgfqpoint{2.828883in}{1.044168in}}%
\pgfpathlineto{\pgfqpoint{2.833773in}{1.042574in}}%
\pgfpathlineto{\pgfqpoint{2.838662in}{1.040987in}}%
\pgfpathlineto{\pgfqpoint{2.843552in}{1.039406in}}%
\pgfpathlineto{\pgfqpoint{2.848442in}{1.037831in}}%
\pgfpathlineto{\pgfqpoint{2.853331in}{1.036262in}}%
\pgfpathlineto{\pgfqpoint{2.858221in}{1.034700in}}%
\pgfpathlineto{\pgfqpoint{2.863111in}{1.033143in}}%
\pgfpathlineto{\pgfqpoint{2.868000in}{1.031593in}}%
\pgfpathlineto{\pgfqpoint{2.872890in}{1.030049in}}%
\pgfpathlineto{\pgfqpoint{2.877780in}{1.028512in}}%
\pgfpathlineto{\pgfqpoint{2.882669in}{1.026980in}}%
\pgfpathlineto{\pgfqpoint{2.887559in}{1.025454in}}%
\pgfpathlineto{\pgfqpoint{2.892449in}{1.023935in}}%
\pgfpathlineto{\pgfqpoint{2.897339in}{1.022421in}}%
\pgfpathlineto{\pgfqpoint{2.902228in}{1.020913in}}%
\pgfpathlineto{\pgfqpoint{2.907118in}{1.019411in}}%
\pgfpathlineto{\pgfqpoint{2.912008in}{1.017915in}}%
\pgfpathlineto{\pgfqpoint{2.916897in}{1.016425in}}%
\pgfpathlineto{\pgfqpoint{2.921787in}{1.014940in}}%
\pgfpathlineto{\pgfqpoint{2.926677in}{1.013462in}}%
\pgfpathlineto{\pgfqpoint{2.931566in}{1.011989in}}%
\pgfpathlineto{\pgfqpoint{2.936456in}{1.010521in}}%
\pgfpathlineto{\pgfqpoint{2.941346in}{1.009060in}}%
\pgfpathlineto{\pgfqpoint{2.946235in}{1.007604in}}%
\pgfpathlineto{\pgfqpoint{2.951125in}{1.006153in}}%
\pgfpathlineto{\pgfqpoint{2.956015in}{1.004708in}}%
\pgfpathlineto{\pgfqpoint{2.960904in}{1.003269in}}%
\pgfpathlineto{\pgfqpoint{2.965794in}{1.001835in}}%
\pgfpathlineto{\pgfqpoint{2.970684in}{1.000407in}}%
\pgfpathlineto{\pgfqpoint{2.975573in}{0.998984in}}%
\pgfpathlineto{\pgfqpoint{2.980463in}{0.997566in}}%
\pgfpathlineto{\pgfqpoint{2.985353in}{0.996154in}}%
\pgfpathlineto{\pgfqpoint{2.990242in}{0.994747in}}%
\pgfpathlineto{\pgfqpoint{2.995132in}{0.993345in}}%
\pgfpathlineto{\pgfqpoint{3.000022in}{0.991949in}}%
\pgfpathlineto{\pgfqpoint{3.004911in}{0.990558in}}%
\pgfpathlineto{\pgfqpoint{3.009801in}{0.989172in}}%
\pgfpathlineto{\pgfqpoint{3.014691in}{0.987792in}}%
\pgfpathlineto{\pgfqpoint{3.019580in}{0.986416in}}%
\pgfpathlineto{\pgfqpoint{3.024470in}{0.985046in}}%
\pgfpathlineto{\pgfqpoint{3.029360in}{0.983681in}}%
\pgfpathlineto{\pgfqpoint{3.034249in}{0.982320in}}%
\pgfpathlineto{\pgfqpoint{3.039139in}{0.980965in}}%
\pgfpathlineto{\pgfqpoint{3.044029in}{0.979615in}}%
\pgfpathlineto{\pgfqpoint{3.048918in}{0.978270in}}%
\pgfpathlineto{\pgfqpoint{3.053808in}{0.976930in}}%
\pgfpathlineto{\pgfqpoint{3.058698in}{0.975595in}}%
\pgfpathlineto{\pgfqpoint{3.063587in}{0.974264in}}%
\pgfpathlineto{\pgfqpoint{3.068477in}{0.972939in}}%
\pgfpathlineto{\pgfqpoint{3.073367in}{0.971618in}}%
\pgfpathlineto{\pgfqpoint{3.078256in}{0.970303in}}%
\pgfpathlineto{\pgfqpoint{3.083146in}{0.968992in}}%
\pgfpathlineto{\pgfqpoint{3.088036in}{0.967685in}}%
\pgfpathlineto{\pgfqpoint{3.092925in}{0.966384in}}%
\pgfpathlineto{\pgfqpoint{3.097815in}{0.965087in}}%
\pgfpathlineto{\pgfqpoint{3.102705in}{0.963795in}}%
\pgfpathlineto{\pgfqpoint{3.107594in}{0.962508in}}%
\pgfpathlineto{\pgfqpoint{3.112484in}{0.961225in}}%
\pgfpathlineto{\pgfqpoint{3.117374in}{0.959947in}}%
\pgfpathlineto{\pgfqpoint{3.122263in}{0.958674in}}%
\pgfpathlineto{\pgfqpoint{3.127153in}{0.957405in}}%
\pgfpathlineto{\pgfqpoint{3.132043in}{0.956140in}}%
\pgfpathlineto{\pgfqpoint{3.136932in}{0.954881in}}%
\pgfpathlineto{\pgfqpoint{3.141822in}{0.953625in}}%
\pgfpathlineto{\pgfqpoint{3.146712in}{0.952374in}}%
\pgfpathlineto{\pgfqpoint{3.151601in}{0.951128in}}%
\pgfpathlineto{\pgfqpoint{3.156491in}{0.949886in}}%
\pgfpathlineto{\pgfqpoint{3.161381in}{0.948649in}}%
\pgfpathlineto{\pgfqpoint{3.166270in}{0.947415in}}%
\pgfpathlineto{\pgfqpoint{3.171160in}{0.946186in}}%
\pgfpathlineto{\pgfqpoint{3.176050in}{0.944962in}}%
\pgfpathlineto{\pgfqpoint{3.180939in}{0.943742in}}%
\pgfpathlineto{\pgfqpoint{3.185829in}{0.942526in}}%
\pgfpathlineto{\pgfqpoint{3.190719in}{0.941314in}}%
\pgfpathlineto{\pgfqpoint{3.195608in}{0.940107in}}%
\pgfpathlineto{\pgfqpoint{3.200498in}{0.938904in}}%
\pgfpathlineto{\pgfqpoint{3.205388in}{0.937705in}}%
\pgfpathlineto{\pgfqpoint{3.210277in}{0.936510in}}%
\pgfpathlineto{\pgfqpoint{3.215167in}{0.935319in}}%
\pgfpathlineto{\pgfqpoint{3.220057in}{0.934133in}}%
\pgfpathlineto{\pgfqpoint{3.224946in}{0.932950in}}%
\pgfpathlineto{\pgfqpoint{3.229836in}{0.931772in}}%
\pgfpathlineto{\pgfqpoint{3.234726in}{0.930598in}}%
\pgfpathlineto{\pgfqpoint{3.239615in}{0.929428in}}%
\pgfpathlineto{\pgfqpoint{3.244505in}{0.928261in}}%
\pgfpathlineto{\pgfqpoint{3.249395in}{0.927099in}}%
\pgfpathlineto{\pgfqpoint{3.254284in}{0.925941in}}%
\pgfpathlineto{\pgfqpoint{3.259174in}{0.924787in}}%
\pgfpathlineto{\pgfqpoint{3.264064in}{0.923637in}}%
\pgfpathlineto{\pgfqpoint{3.268953in}{0.922490in}}%
\pgfpathlineto{\pgfqpoint{3.273843in}{0.921348in}}%
\pgfpathlineto{\pgfqpoint{3.278733in}{0.920209in}}%
\pgfpathlineto{\pgfqpoint{3.283622in}{0.919075in}}%
\pgfpathlineto{\pgfqpoint{3.288512in}{0.917944in}}%
\pgfpathlineto{\pgfqpoint{3.293402in}{0.916817in}}%
\pgfpathlineto{\pgfqpoint{3.298291in}{0.915693in}}%
\pgfpathlineto{\pgfqpoint{3.303181in}{0.914574in}}%
\pgfpathlineto{\pgfqpoint{3.308071in}{0.913458in}}%
\pgfpathlineto{\pgfqpoint{3.312960in}{0.912346in}}%
\pgfpathlineto{\pgfqpoint{3.317850in}{0.911238in}}%
\pgfpathlineto{\pgfqpoint{3.322740in}{0.910134in}}%
\pgfpathlineto{\pgfqpoint{3.327630in}{0.909033in}}%
\pgfpathlineto{\pgfqpoint{3.332519in}{0.907936in}}%
\pgfpathlineto{\pgfqpoint{3.337409in}{0.906843in}}%
\pgfpathlineto{\pgfqpoint{3.342299in}{0.905753in}}%
\pgfpathlineto{\pgfqpoint{3.347188in}{0.904667in}}%
\pgfpathlineto{\pgfqpoint{3.352078in}{0.903584in}}%
\pgfpathlineto{\pgfqpoint{3.356968in}{0.902505in}}%
\pgfpathlineto{\pgfqpoint{3.361857in}{0.901430in}}%
\pgfpathlineto{\pgfqpoint{3.366747in}{0.900358in}}%
\pgfpathlineto{\pgfqpoint{3.371637in}{0.899289in}}%
\pgfpathlineto{\pgfqpoint{3.376526in}{0.898225in}}%
\pgfpathlineto{\pgfqpoint{3.381416in}{0.897163in}}%
\pgfpathlineto{\pgfqpoint{3.386306in}{0.896105in}}%
\pgfpathlineto{\pgfqpoint{3.391195in}{0.895051in}}%
\pgfpathlineto{\pgfqpoint{3.396085in}{0.894000in}}%
\pgfpathlineto{\pgfqpoint{3.400975in}{0.892952in}}%
\pgfpathlineto{\pgfqpoint{3.405864in}{0.891908in}}%
\pgfpathlineto{\pgfqpoint{3.410754in}{0.890868in}}%
\pgfpathlineto{\pgfqpoint{3.415644in}{0.889830in}}%
\pgfpathlineto{\pgfqpoint{3.420533in}{0.888796in}}%
\pgfpathlineto{\pgfqpoint{3.425423in}{0.887766in}}%
\pgfpathlineto{\pgfqpoint{3.430313in}{0.886738in}}%
\pgfpathlineto{\pgfqpoint{3.435202in}{0.885714in}}%
\pgfpathlineto{\pgfqpoint{3.440092in}{0.884694in}}%
\pgfpathlineto{\pgfqpoint{3.444982in}{0.883676in}}%
\pgfpathlineto{\pgfqpoint{3.449871in}{0.882662in}}%
\pgfpathlineto{\pgfqpoint{3.454761in}{0.881651in}}%
\pgfpathlineto{\pgfqpoint{3.459651in}{0.880644in}}%
\pgfpathlineto{\pgfqpoint{3.464540in}{0.879639in}}%
\pgfpathlineto{\pgfqpoint{3.469430in}{0.878638in}}%
\pgfpathlineto{\pgfqpoint{3.474320in}{0.877640in}}%
\pgfpathlineto{\pgfqpoint{3.479209in}{0.876645in}}%
\pgfpathlineto{\pgfqpoint{3.484099in}{0.875653in}}%
\pgfpathlineto{\pgfqpoint{3.488989in}{0.874665in}}%
\pgfpathlineto{\pgfqpoint{3.493878in}{0.873679in}}%
\pgfpathlineto{\pgfqpoint{3.498768in}{0.872697in}}%
\pgfpathlineto{\pgfqpoint{3.503658in}{0.871718in}}%
\pgfpathlineto{\pgfqpoint{3.508547in}{0.870741in}}%
\pgfpathlineto{\pgfqpoint{3.513437in}{0.869768in}}%
\pgfpathlineto{\pgfqpoint{3.518327in}{0.868798in}}%
\pgfpathlineto{\pgfqpoint{3.523216in}{0.867831in}}%
\pgfpathlineto{\pgfqpoint{3.528106in}{0.866867in}}%
\pgfpathlineto{\pgfqpoint{3.532996in}{0.865906in}}%
\pgfpathlineto{\pgfqpoint{3.537885in}{0.864948in}}%
\pgfpathlineto{\pgfqpoint{3.542775in}{0.863994in}}%
\pgfpathlineto{\pgfqpoint{3.547665in}{0.863042in}}%
\pgfpathlineto{\pgfqpoint{3.552554in}{0.862092in}}%
\pgfpathlineto{\pgfqpoint{3.557444in}{0.861146in}}%
\pgfpathlineto{\pgfqpoint{3.562334in}{0.860203in}}%
\pgfpathlineto{\pgfqpoint{3.567223in}{0.859263in}}%
\pgfpathlineto{\pgfqpoint{3.572113in}{0.858326in}}%
\pgfpathlineto{\pgfqpoint{3.577003in}{0.857391in}}%
\pgfpathlineto{\pgfqpoint{3.581892in}{0.856459in}}%
\pgfpathlineto{\pgfqpoint{3.586782in}{0.855531in}}%
\pgfpathlineto{\pgfqpoint{3.591672in}{0.854605in}}%
\pgfpathlineto{\pgfqpoint{3.596561in}{0.853682in}}%
\pgfpathlineto{\pgfqpoint{3.601451in}{0.852762in}}%
\pgfpathlineto{\pgfqpoint{3.606341in}{0.851844in}}%
\pgfpathlineto{\pgfqpoint{3.611230in}{0.850930in}}%
\pgfpathlineto{\pgfqpoint{3.616120in}{0.850018in}}%
\pgfpathlineto{\pgfqpoint{3.621010in}{0.849109in}}%
\pgfpathlineto{\pgfqpoint{3.625899in}{0.848202in}}%
\pgfpathlineto{\pgfqpoint{3.630789in}{0.847299in}}%
\pgfpathlineto{\pgfqpoint{3.635679in}{0.846398in}}%
\pgfpathlineto{\pgfqpoint{3.640568in}{0.845500in}}%
\pgfpathlineto{\pgfqpoint{3.645458in}{0.844604in}}%
\pgfpathlineto{\pgfqpoint{3.650348in}{0.843712in}}%
\pgfpathlineto{\pgfqpoint{3.655237in}{0.842822in}}%
\pgfpathlineto{\pgfqpoint{3.660127in}{0.841934in}}%
\pgfpathlineto{\pgfqpoint{3.665017in}{0.841050in}}%
\pgfpathlineto{\pgfqpoint{3.669906in}{0.840168in}}%
\pgfpathlineto{\pgfqpoint{3.674796in}{0.839288in}}%
\pgfpathlineto{\pgfqpoint{3.679686in}{0.838411in}}%
\pgfpathlineto{\pgfqpoint{3.684575in}{0.837537in}}%
\pgfpathlineto{\pgfqpoint{3.689465in}{0.836666in}}%
\pgfpathlineto{\pgfqpoint{3.694355in}{0.835797in}}%
\pgfpathlineto{\pgfqpoint{3.699244in}{0.834930in}}%
\pgfpathlineto{\pgfqpoint{3.704134in}{0.834066in}}%
\pgfpathlineto{\pgfqpoint{3.709024in}{0.833205in}}%
\pgfpathlineto{\pgfqpoint{3.713913in}{0.832346in}}%
\pgfpathlineto{\pgfqpoint{3.718803in}{0.831490in}}%
\pgfpathlineto{\pgfqpoint{3.723693in}{0.830637in}}%
\pgfpathlineto{\pgfqpoint{3.728582in}{0.829785in}}%
\pgfpathlineto{\pgfqpoint{3.733472in}{0.828937in}}%
\pgfpathlineto{\pgfqpoint{3.738362in}{0.828091in}}%
\pgfpathlineto{\pgfqpoint{3.743251in}{0.827247in}}%
\pgfpathlineto{\pgfqpoint{3.748141in}{0.826406in}}%
\pgfpathlineto{\pgfqpoint{3.753031in}{0.825567in}}%
\pgfpathlineto{\pgfqpoint{3.757921in}{0.824731in}}%
\pgfpathlineto{\pgfqpoint{3.762810in}{0.823897in}}%
\pgfpathlineto{\pgfqpoint{3.767700in}{0.823065in}}%
\pgfpathlineto{\pgfqpoint{3.772590in}{0.822236in}}%
\pgfpathlineto{\pgfqpoint{3.777479in}{0.821409in}}%
\pgfpathlineto{\pgfqpoint{3.782369in}{0.820585in}}%
\pgfpathlineto{\pgfqpoint{3.787259in}{0.819763in}}%
\pgfpathlineto{\pgfqpoint{3.792148in}{0.818944in}}%
\pgfpathlineto{\pgfqpoint{3.797038in}{0.818127in}}%
\pgfpathlineto{\pgfqpoint{3.801928in}{0.817312in}}%
\pgfpathlineto{\pgfqpoint{3.806817in}{0.816499in}}%
\pgfpathlineto{\pgfqpoint{3.811707in}{0.815689in}}%
\pgfpathlineto{\pgfqpoint{3.816597in}{0.814881in}}%
\pgfpathlineto{\pgfqpoint{3.821486in}{0.814076in}}%
\pgfpathlineto{\pgfqpoint{3.826376in}{0.813273in}}%
\pgfpathlineto{\pgfqpoint{3.831266in}{0.812472in}}%
\pgfpathlineto{\pgfqpoint{3.836155in}{0.811673in}}%
\pgfpathlineto{\pgfqpoint{3.841045in}{0.810877in}}%
\pgfpathlineto{\pgfqpoint{3.845935in}{0.810083in}}%
\pgfpathlineto{\pgfqpoint{3.850824in}{0.809291in}}%
\pgfpathlineto{\pgfqpoint{3.855714in}{0.808501in}}%
\pgfpathlineto{\pgfqpoint{3.860604in}{0.807714in}}%
\pgfpathlineto{\pgfqpoint{3.865493in}{0.806929in}}%
\pgfpathlineto{\pgfqpoint{3.870383in}{0.806146in}}%
\pgfpathlineto{\pgfqpoint{3.875273in}{0.805365in}}%
\pgfpathlineto{\pgfqpoint{3.880162in}{0.804587in}}%
\pgfpathlineto{\pgfqpoint{3.885052in}{0.803810in}}%
\pgfpathlineto{\pgfqpoint{3.889942in}{0.803036in}}%
\pgfpathlineto{\pgfqpoint{3.894831in}{0.802264in}}%
\pgfpathlineto{\pgfqpoint{3.899721in}{0.801494in}}%
\pgfpathlineto{\pgfqpoint{3.904611in}{0.800727in}}%
\pgfpathlineto{\pgfqpoint{3.909500in}{0.799961in}}%
\pgfpathlineto{\pgfqpoint{3.914390in}{0.799198in}}%
\pgfpathlineto{\pgfqpoint{3.919280in}{0.798436in}}%
\pgfpathlineto{\pgfqpoint{3.924169in}{0.797677in}}%
\pgfpathlineto{\pgfqpoint{3.929059in}{0.796920in}}%
\pgfpathlineto{\pgfqpoint{3.933949in}{0.796165in}}%
\pgfpathlineto{\pgfqpoint{3.938838in}{0.795412in}}%
\pgfpathlineto{\pgfqpoint{3.943728in}{0.794661in}}%
\pgfpathlineto{\pgfqpoint{3.948618in}{0.793913in}}%
\pgfpathlineto{\pgfqpoint{3.953507in}{0.793166in}}%
\pgfpathlineto{\pgfqpoint{3.958397in}{0.792421in}}%
\pgfpathlineto{\pgfqpoint{3.963287in}{0.791679in}}%
\pgfpathlineto{\pgfqpoint{3.968176in}{0.790938in}}%
\pgfpathlineto{\pgfqpoint{3.973066in}{0.790200in}}%
\pgfpathlineto{\pgfqpoint{3.977956in}{0.789463in}}%
\pgfpathlineto{\pgfqpoint{3.982845in}{0.788729in}}%
\pgfpathlineto{\pgfqpoint{3.987735in}{0.787996in}}%
\pgfpathlineto{\pgfqpoint{3.992625in}{0.787266in}}%
\pgfpathlineto{\pgfqpoint{3.997514in}{0.786537in}}%
\pgfpathlineto{\pgfqpoint{4.002404in}{0.785811in}}%
\pgfpathlineto{\pgfqpoint{4.007294in}{0.785086in}}%
\pgfpathlineto{\pgfqpoint{4.012183in}{0.784364in}}%
\pgfpathlineto{\pgfqpoint{4.017073in}{0.783643in}}%
\pgfpathlineto{\pgfqpoint{4.021963in}{0.782924in}}%
\pgfpathlineto{\pgfqpoint{4.026852in}{0.782208in}}%
\pgfpathlineto{\pgfqpoint{4.031742in}{0.781493in}}%
\pgfpathlineto{\pgfqpoint{4.036632in}{0.780780in}}%
\pgfpathlineto{\pgfqpoint{4.041521in}{0.780069in}}%
\pgfpathlineto{\pgfqpoint{4.046411in}{0.779360in}}%
\pgfpathlineto{\pgfqpoint{4.051301in}{0.778653in}}%
\pgfpathlineto{\pgfqpoint{4.056190in}{0.777948in}}%
\pgfpathlineto{\pgfqpoint{4.061080in}{0.777244in}}%
\pgfpathlineto{\pgfqpoint{4.065970in}{0.776543in}}%
\pgfpathlineto{\pgfqpoint{4.070859in}{0.775843in}}%
\pgfpathlineto{\pgfqpoint{4.075749in}{0.775145in}}%
\pgfpathlineto{\pgfqpoint{4.080639in}{0.774449in}}%
\pgfpathlineto{\pgfqpoint{4.085528in}{0.773755in}}%
\pgfpathlineto{\pgfqpoint{4.090418in}{0.773063in}}%
\pgfpathlineto{\pgfqpoint{4.095308in}{0.772373in}}%
\pgfpathlineto{\pgfqpoint{4.100197in}{0.771684in}}%
\pgfpathlineto{\pgfqpoint{4.105087in}{0.770997in}}%
\pgfpathlineto{\pgfqpoint{4.109977in}{0.770313in}}%
\pgfpathlineto{\pgfqpoint{4.114866in}{0.769629in}}%
\pgfpathlineto{\pgfqpoint{4.119756in}{0.768948in}}%
\pgfpathlineto{\pgfqpoint{4.124646in}{0.768269in}}%
\pgfpathlineto{\pgfqpoint{4.129535in}{0.767591in}}%
\pgfpathlineto{\pgfqpoint{4.134425in}{0.766915in}}%
\pgfpathlineto{\pgfqpoint{4.139315in}{0.766241in}}%
\pgfpathlineto{\pgfqpoint{4.144204in}{0.765568in}}%
\pgfpathlineto{\pgfqpoint{4.149094in}{0.764897in}}%
\pgfpathlineto{\pgfqpoint{4.153984in}{0.764228in}}%
\pgfpathlineto{\pgfqpoint{4.158873in}{0.763561in}}%
\pgfpathlineto{\pgfqpoint{4.163763in}{0.762896in}}%
\pgfpathlineto{\pgfqpoint{4.168653in}{0.762232in}}%
\pgfpathlineto{\pgfqpoint{4.173542in}{0.761570in}}%
\pgfpathlineto{\pgfqpoint{4.178432in}{0.760910in}}%
\pgfpathlineto{\pgfqpoint{4.183322in}{0.760251in}}%
\pgfpathlineto{\pgfqpoint{4.188211in}{0.759594in}}%
\pgfpathlineto{\pgfqpoint{4.193101in}{0.758939in}}%
\pgfpathlineto{\pgfqpoint{4.197991in}{0.758286in}}%
\pgfpathlineto{\pgfqpoint{4.202881in}{0.757634in}}%
\pgfpathlineto{\pgfqpoint{4.207770in}{0.756984in}}%
\pgfpathlineto{\pgfqpoint{4.212660in}{0.756335in}}%
\pgfpathlineto{\pgfqpoint{4.217550in}{0.755688in}}%
\pgfpathlineto{\pgfqpoint{4.222439in}{0.755043in}}%
\pgfpathlineto{\pgfqpoint{4.227329in}{0.754400in}}%
\pgfpathlineto{\pgfqpoint{4.232219in}{0.753758in}}%
\pgfpathlineto{\pgfqpoint{4.237108in}{0.753118in}}%
\pgfpathlineto{\pgfqpoint{4.241998in}{0.752479in}}%
\pgfpathlineto{\pgfqpoint{4.246888in}{0.751842in}}%
\pgfpathlineto{\pgfqpoint{4.251777in}{0.751207in}}%
\pgfpathlineto{\pgfqpoint{4.256667in}{0.750573in}}%
\pgfpathlineto{\pgfqpoint{4.261557in}{0.749941in}}%
\pgfpathlineto{\pgfqpoint{4.266446in}{0.749310in}}%
\pgfpathlineto{\pgfqpoint{4.271336in}{0.748681in}}%
\pgfpathlineto{\pgfqpoint{4.276226in}{0.748054in}}%
\pgfpathlineto{\pgfqpoint{4.281115in}{0.747428in}}%
\pgfpathlineto{\pgfqpoint{4.286005in}{0.746804in}}%
\pgfpathlineto{\pgfqpoint{4.290895in}{0.746181in}}%
\pgfpathlineto{\pgfqpoint{4.295784in}{0.745560in}}%
\pgfpathlineto{\pgfqpoint{4.300674in}{0.744941in}}%
\pgfpathlineto{\pgfqpoint{4.305564in}{0.744323in}}%
\pgfpathlineto{\pgfqpoint{4.310453in}{0.743706in}}%
\pgfpathlineto{\pgfqpoint{4.315343in}{0.743091in}}%
\pgfpathlineto{\pgfqpoint{4.320233in}{0.742478in}}%
\pgfpathlineto{\pgfqpoint{4.325122in}{0.741866in}}%
\pgfpathlineto{\pgfqpoint{4.330012in}{0.741256in}}%
\pgfpathlineto{\pgfqpoint{4.334902in}{0.740647in}}%
\pgfpathlineto{\pgfqpoint{4.339791in}{0.740040in}}%
\pgfpathlineto{\pgfqpoint{4.344681in}{0.739434in}}%
\pgfpathlineto{\pgfqpoint{4.349571in}{0.738830in}}%
\pgfpathlineto{\pgfqpoint{4.354460in}{0.738227in}}%
\pgfpathlineto{\pgfqpoint{4.359350in}{0.737626in}}%
\pgfpathlineto{\pgfqpoint{4.364240in}{0.737026in}}%
\pgfpathlineto{\pgfqpoint{4.369129in}{0.736428in}}%
\pgfpathlineto{\pgfqpoint{4.374019in}{0.735831in}}%
\pgfpathlineto{\pgfqpoint{4.378909in}{0.735236in}}%
\pgfpathlineto{\pgfqpoint{4.383798in}{0.734642in}}%
\pgfpathlineto{\pgfqpoint{4.388688in}{0.734049in}}%
\pgfpathlineto{\pgfqpoint{4.393578in}{0.733458in}}%
\pgfpathlineto{\pgfqpoint{4.398467in}{0.732869in}}%
\pgfpathlineto{\pgfqpoint{4.403357in}{0.732281in}}%
\pgfpathlineto{\pgfqpoint{4.408247in}{0.731694in}}%
\pgfpathlineto{\pgfqpoint{4.413136in}{0.731109in}}%
\pgfpathlineto{\pgfqpoint{4.418026in}{0.730525in}}%
\pgfpathlineto{\pgfqpoint{4.422916in}{0.729943in}}%
\pgfpathlineto{\pgfqpoint{4.427805in}{0.729362in}}%
\pgfpathlineto{\pgfqpoint{4.432695in}{0.728782in}}%
\pgfpathlineto{\pgfqpoint{4.437585in}{0.728204in}}%
\pgfpathlineto{\pgfqpoint{4.442474in}{0.727627in}}%
\pgfpathlineto{\pgfqpoint{4.447364in}{0.727052in}}%
\pgfpathlineto{\pgfqpoint{4.452254in}{0.726478in}}%
\pgfpathlineto{\pgfqpoint{4.457143in}{0.725905in}}%
\pgfpathlineto{\pgfqpoint{4.462033in}{0.725334in}}%
\pgfpathlineto{\pgfqpoint{4.466923in}{0.724764in}}%
\pgfpathlineto{\pgfqpoint{4.471812in}{0.724196in}}%
\pgfpathlineto{\pgfqpoint{4.476702in}{0.723629in}}%
\pgfpathlineto{\pgfqpoint{4.481592in}{0.723063in}}%
\pgfpathlineto{\pgfqpoint{4.486481in}{0.722499in}}%
\pgfpathlineto{\pgfqpoint{4.491371in}{0.721936in}}%
\pgfpathlineto{\pgfqpoint{4.496261in}{0.721374in}}%
\pgfpathlineto{\pgfqpoint{4.501150in}{0.720814in}}%
\pgfpathlineto{\pgfqpoint{4.506040in}{0.720255in}}%
\pgfpathlineto{\pgfqpoint{4.510930in}{0.719697in}}%
\pgfpathlineto{\pgfqpoint{4.515819in}{0.719141in}}%
\pgfpathlineto{\pgfqpoint{4.520709in}{0.718586in}}%
\pgfpathlineto{\pgfqpoint{4.525599in}{0.718032in}}%
\pgfpathlineto{\pgfqpoint{4.530488in}{0.717479in}}%
\pgfpathlineto{\pgfqpoint{4.535378in}{0.716928in}}%
\pgfpathlineto{\pgfqpoint{4.540268in}{0.716379in}}%
\pgfpathlineto{\pgfqpoint{4.545157in}{0.715830in}}%
\pgfpathlineto{\pgfqpoint{4.550047in}{0.715283in}}%
\pgfpathlineto{\pgfqpoint{4.554937in}{0.714737in}}%
\pgfpathlineto{\pgfqpoint{4.559826in}{0.714192in}}%
\pgfpathlineto{\pgfqpoint{4.564716in}{0.713649in}}%
\pgfpathlineto{\pgfqpoint{4.569606in}{0.713107in}}%
\pgfpathlineto{\pgfqpoint{4.574495in}{0.712566in}}%
\pgfpathlineto{\pgfqpoint{4.579385in}{0.712027in}}%
\pgfpathlineto{\pgfqpoint{4.584275in}{0.711488in}}%
\pgfpathlineto{\pgfqpoint{4.589164in}{0.710951in}}%
\pgfpathlineto{\pgfqpoint{4.594054in}{0.710416in}}%
\pgfpathlineto{\pgfqpoint{4.598944in}{0.709881in}}%
\pgfpathlineto{\pgfqpoint{4.603833in}{0.709348in}}%
\pgfpathlineto{\pgfqpoint{4.608723in}{0.708816in}}%
\pgfpathlineto{\pgfqpoint{4.613613in}{0.708285in}}%
\pgfpathlineto{\pgfqpoint{4.618502in}{0.707755in}}%
\pgfpathlineto{\pgfqpoint{4.623392in}{0.707227in}}%
\pgfpathlineto{\pgfqpoint{4.628282in}{0.706700in}}%
\pgfpathlineto{\pgfqpoint{4.633172in}{0.706174in}}%
\pgfpathlineto{\pgfqpoint{4.638061in}{0.705649in}}%
\pgfpathlineto{\pgfqpoint{4.642951in}{0.705126in}}%
\pgfpathlineto{\pgfqpoint{4.647841in}{0.704604in}}%
\pgfpathlineto{\pgfqpoint{4.652730in}{0.704082in}}%
\pgfpathlineto{\pgfqpoint{4.657620in}{0.703563in}}%
\pgfpathlineto{\pgfqpoint{4.662510in}{0.703044in}}%
\pgfpathlineto{\pgfqpoint{4.667399in}{0.702526in}}%
\pgfpathlineto{\pgfqpoint{4.672289in}{0.702010in}}%
\pgfpathlineto{\pgfqpoint{4.677179in}{0.701495in}}%
\pgfpathlineto{\pgfqpoint{4.682068in}{0.700981in}}%
\pgfpathlineto{\pgfqpoint{4.686958in}{0.700468in}}%
\pgfpathlineto{\pgfqpoint{4.691848in}{0.699957in}}%
\pgfpathlineto{\pgfqpoint{4.696737in}{0.699446in}}%
\pgfpathlineto{\pgfqpoint{4.701627in}{0.698937in}}%
\pgfpathlineto{\pgfqpoint{4.706517in}{0.698429in}}%
\pgfpathlineto{\pgfqpoint{4.711406in}{0.697922in}}%
\pgfpathlineto{\pgfqpoint{4.716296in}{0.697416in}}%
\pgfpathlineto{\pgfqpoint{4.721186in}{0.696912in}}%
\pgfpathlineto{\pgfqpoint{4.726075in}{0.696408in}}%
\pgfpathlineto{\pgfqpoint{4.730965in}{0.695906in}}%
\pgfpathlineto{\pgfqpoint{4.735855in}{0.695404in}}%
\pgfpathlineto{\pgfqpoint{4.740744in}{0.694904in}}%
\pgfpathlineto{\pgfqpoint{4.745634in}{0.694405in}}%
\pgfpathlineto{\pgfqpoint{4.750524in}{0.693907in}}%
\pgfpathlineto{\pgfqpoint{4.755413in}{0.693411in}}%
\pgfpathlineto{\pgfqpoint{4.760303in}{0.692915in}}%
\pgfpathlineto{\pgfqpoint{4.765193in}{0.692420in}}%
\pgfpathlineto{\pgfqpoint{4.770082in}{0.691927in}}%
\pgfpathlineto{\pgfqpoint{4.774972in}{0.691435in}}%
\pgfpathlineto{\pgfqpoint{4.779862in}{0.690943in}}%
\pgfpathlineto{\pgfqpoint{4.784751in}{0.690453in}}%
\pgfpathlineto{\pgfqpoint{4.789641in}{0.689964in}}%
\pgfpathlineto{\pgfqpoint{4.794531in}{0.689476in}}%
\pgfpathlineto{\pgfqpoint{4.799420in}{0.688990in}}%
\pgfpathlineto{\pgfqpoint{4.804310in}{0.688504in}}%
\pgfpathlineto{\pgfqpoint{4.809200in}{0.688019in}}%
\pgfpathlineto{\pgfqpoint{4.814089in}{0.687535in}}%
\pgfpathlineto{\pgfqpoint{4.818979in}{0.687053in}}%
\pgfpathlineto{\pgfqpoint{4.823869in}{0.686571in}}%
\pgfpathlineto{\pgfqpoint{4.828758in}{0.686091in}}%
\pgfpathlineto{\pgfqpoint{4.833648in}{0.685612in}}%
\pgfpathlineto{\pgfqpoint{4.838538in}{0.685133in}}%
\pgfpathlineto{\pgfqpoint{4.843427in}{0.684656in}}%
\pgfpathlineto{\pgfqpoint{4.848317in}{0.684180in}}%
\pgfpathlineto{\pgfqpoint{4.853207in}{0.683705in}}%
\pgfpathlineto{\pgfqpoint{4.858096in}{0.683231in}}%
\pgfpathlineto{\pgfqpoint{4.862986in}{0.682758in}}%
\pgfpathlineto{\pgfqpoint{4.867876in}{0.682286in}}%
\pgfpathlineto{\pgfqpoint{4.872765in}{0.681815in}}%
\pgfpathlineto{\pgfqpoint{4.877655in}{0.681345in}}%
\pgfpathlineto{\pgfqpoint{4.882545in}{0.680876in}}%
\pgfpathlineto{\pgfqpoint{4.887434in}{0.680408in}}%
\pgfpathlineto{\pgfqpoint{4.892324in}{0.679941in}}%
\pgfpathlineto{\pgfqpoint{4.897214in}{0.679475in}}%
\pgfpathlineto{\pgfqpoint{4.902103in}{0.679010in}}%
\pgfpathlineto{\pgfqpoint{4.906993in}{0.678546in}}%
\pgfpathlineto{\pgfqpoint{4.911883in}{0.678083in}}%
\pgfpathlineto{\pgfqpoint{4.916772in}{0.677621in}}%
\pgfpathlineto{\pgfqpoint{4.921662in}{0.677161in}}%
\pgfpathlineto{\pgfqpoint{4.926552in}{0.676701in}}%
\pgfpathlineto{\pgfqpoint{4.931441in}{0.676242in}}%
\pgfpathlineto{\pgfqpoint{4.936331in}{0.675784in}}%
\pgfpathlineto{\pgfqpoint{4.941221in}{0.675327in}}%
\pgfpathlineto{\pgfqpoint{4.946110in}{0.674871in}}%
\pgfpathlineto{\pgfqpoint{4.951000in}{0.674416in}}%
\pgfpathlineto{\pgfqpoint{4.955890in}{0.673962in}}%
\pgfpathlineto{\pgfqpoint{4.960779in}{0.673509in}}%
\pgfpathlineto{\pgfqpoint{4.965669in}{0.673057in}}%
\pgfpathlineto{\pgfqpoint{4.970559in}{0.672606in}}%
\pgfpathlineto{\pgfqpoint{4.975448in}{0.672156in}}%
\pgfpathlineto{\pgfqpoint{4.980338in}{0.671707in}}%
\pgfpathlineto{\pgfqpoint{4.985228in}{0.671259in}}%
\pgfpathlineto{\pgfqpoint{4.990117in}{0.670812in}}%
\pgfpathlineto{\pgfqpoint{4.995007in}{0.670366in}}%
\pgfpathlineto{\pgfqpoint{4.999897in}{0.669920in}}%
\pgfpathlineto{\pgfqpoint{5.004786in}{0.669476in}}%
\pgfpathlineto{\pgfqpoint{5.009676in}{0.669033in}}%
\pgfpathlineto{\pgfqpoint{5.014566in}{0.668590in}}%
\pgfpathlineto{\pgfqpoint{5.019455in}{0.668149in}}%
\pgfpathlineto{\pgfqpoint{5.024345in}{0.667708in}}%
\pgfpathlineto{\pgfqpoint{5.029235in}{0.667268in}}%
\pgfpathlineto{\pgfqpoint{5.034124in}{0.666830in}}%
\pgfpathlineto{\pgfqpoint{5.039014in}{0.666392in}}%
\pgfpathlineto{\pgfqpoint{5.043904in}{0.665955in}}%
\pgfpathlineto{\pgfqpoint{5.048793in}{0.665519in}}%
\pgfpathlineto{\pgfqpoint{5.053683in}{0.665084in}}%
\pgfpathlineto{\pgfqpoint{5.058573in}{0.664650in}}%
\pgfpathlineto{\pgfqpoint{5.063462in}{0.664216in}}%
\pgfpathlineto{\pgfqpoint{5.068352in}{0.663784in}}%
\pgfpathlineto{\pgfqpoint{5.073242in}{0.663353in}}%
\pgfpathlineto{\pgfqpoint{5.078132in}{0.662922in}}%
\pgfpathlineto{\pgfqpoint{5.083021in}{0.662493in}}%
\pgfpathlineto{\pgfqpoint{5.087911in}{0.662064in}}%
\pgfpathlineto{\pgfqpoint{5.092801in}{0.661636in}}%
\pgfpathlineto{\pgfqpoint{5.097690in}{0.661209in}}%
\pgfpathlineto{\pgfqpoint{5.102580in}{0.660783in}}%
\pgfpathlineto{\pgfqpoint{5.107470in}{0.660358in}}%
\pgfpathlineto{\pgfqpoint{5.112359in}{0.659934in}}%
\pgfpathlineto{\pgfqpoint{5.117249in}{0.659510in}}%
\pgfpathlineto{\pgfqpoint{5.122139in}{0.659088in}}%
\pgfpathlineto{\pgfqpoint{5.127028in}{0.658666in}}%
\pgfpathlineto{\pgfqpoint{5.131918in}{0.658245in}}%
\pgfpathlineto{\pgfqpoint{5.136808in}{0.657826in}}%
\pgfpathlineto{\pgfqpoint{5.141697in}{0.657407in}}%
\pgfpathlineto{\pgfqpoint{5.146587in}{0.656988in}}%
\pgfpathlineto{\pgfqpoint{5.151477in}{0.656571in}}%
\pgfpathlineto{\pgfqpoint{5.156366in}{0.656155in}}%
\pgfpathlineto{\pgfqpoint{5.161256in}{0.655739in}}%
\pgfpathlineto{\pgfqpoint{5.166146in}{0.655324in}}%
\pgfpathlineto{\pgfqpoint{5.171035in}{0.654910in}}%
\pgfpathlineto{\pgfqpoint{5.175925in}{0.654497in}}%
\pgfpathlineto{\pgfqpoint{5.180815in}{0.654085in}}%
\pgfpathlineto{\pgfqpoint{5.185704in}{0.653674in}}%
\pgfpathlineto{\pgfqpoint{5.190594in}{0.653263in}}%
\pgfpathlineto{\pgfqpoint{5.195484in}{0.652854in}}%
\pgfpathlineto{\pgfqpoint{5.200373in}{0.652445in}}%
\pgfpathlineto{\pgfqpoint{5.205263in}{0.652037in}}%
\pgfpathlineto{\pgfqpoint{5.210153in}{0.651630in}}%
\pgfpathlineto{\pgfqpoint{5.215042in}{0.651223in}}%
\pgfpathlineto{\pgfqpoint{5.219932in}{0.650818in}}%
\pgfpathlineto{\pgfqpoint{5.224822in}{0.650413in}}%
\pgfpathlineto{\pgfqpoint{5.229711in}{0.650009in}}%
\pgfpathlineto{\pgfqpoint{5.234601in}{0.649606in}}%
\pgfpathlineto{\pgfqpoint{5.239491in}{0.649204in}}%
\pgfpathlineto{\pgfqpoint{5.244380in}{0.648802in}}%
\pgfpathlineto{\pgfqpoint{5.249270in}{0.648402in}}%
\pgfpathlineto{\pgfqpoint{5.254160in}{0.648002in}}%
\pgfpathlineto{\pgfqpoint{5.259049in}{0.647603in}}%
\pgfpathlineto{\pgfqpoint{5.263939in}{0.647205in}}%
\pgfpathlineto{\pgfqpoint{5.268829in}{0.646807in}}%
\pgfpathlineto{\pgfqpoint{5.273718in}{0.646411in}}%
\pgfpathlineto{\pgfqpoint{5.278608in}{0.646015in}}%
\pgfpathlineto{\pgfqpoint{5.283498in}{0.645620in}}%
\pgfpathlineto{\pgfqpoint{5.288387in}{0.645226in}}%
\pgfpathlineto{\pgfqpoint{5.293277in}{0.644832in}}%
\pgfpathlineto{\pgfqpoint{5.298167in}{0.644440in}}%
\pgfpathlineto{\pgfqpoint{5.303056in}{0.644048in}}%
\pgfpathlineto{\pgfqpoint{5.307946in}{0.643657in}}%
\pgfpathlineto{\pgfqpoint{5.312836in}{0.643266in}}%
\pgfpathlineto{\pgfqpoint{5.317725in}{0.642877in}}%
\pgfpathlineto{\pgfqpoint{5.322615in}{0.642488in}}%
\pgfpathlineto{\pgfqpoint{5.327505in}{0.642100in}}%
\pgfpathlineto{\pgfqpoint{5.332394in}{0.641713in}}%
\pgfpathlineto{\pgfqpoint{5.337284in}{0.641326in}}%
\pgfpathlineto{\pgfqpoint{5.342174in}{0.640941in}}%
\pgfpathlineto{\pgfqpoint{5.347063in}{0.640556in}}%
\pgfpathlineto{\pgfqpoint{5.351953in}{0.640172in}}%
\pgfpathlineto{\pgfqpoint{5.356843in}{0.639788in}}%
\pgfpathlineto{\pgfqpoint{5.361732in}{0.639405in}}%
\pgfpathlineto{\pgfqpoint{5.366622in}{0.639024in}}%
\pgfpathlineto{\pgfqpoint{5.371512in}{0.638642in}}%
\pgfpathlineto{\pgfqpoint{5.376401in}{0.638262in}}%
\pgfpathlineto{\pgfqpoint{5.381291in}{0.637882in}}%
\pgfpathlineto{\pgfqpoint{5.386181in}{0.637504in}}%
\pgfpathlineto{\pgfqpoint{5.391070in}{0.637125in}}%
\pgfpathlineto{\pgfqpoint{5.395960in}{0.636748in}}%
\pgfpathlineto{\pgfqpoint{5.400850in}{0.636371in}}%
\pgfpathlineto{\pgfqpoint{5.405739in}{0.635995in}}%
\pgfpathlineto{\pgfqpoint{5.410629in}{0.635620in}}%
\pgfpathlineto{\pgfqpoint{5.415519in}{0.635246in}}%
\pgfpathlineto{\pgfqpoint{5.420408in}{0.634872in}}%
\pgfpathlineto{\pgfqpoint{5.425298in}{0.634499in}}%
\pgfpathlineto{\pgfqpoint{5.430188in}{0.634127in}}%
\pgfpathlineto{\pgfqpoint{5.435077in}{0.633755in}}%
\pgfpathlineto{\pgfqpoint{5.439967in}{0.633384in}}%
\pgfpathlineto{\pgfqpoint{5.444857in}{0.633014in}}%
\pgfpathlineto{\pgfqpoint{5.449746in}{0.632645in}}%
\pgfpathlineto{\pgfqpoint{5.454636in}{0.632276in}}%
\pgfpathlineto{\pgfqpoint{5.459526in}{0.631908in}}%
\pgfpathlineto{\pgfqpoint{5.464415in}{0.631541in}}%
\pgfpathlineto{\pgfqpoint{5.469305in}{0.631174in}}%
\pgfpathlineto{\pgfqpoint{5.474195in}{0.630808in}}%
\pgfpathlineto{\pgfqpoint{5.479084in}{0.630443in}}%
\pgfpathlineto{\pgfqpoint{5.483974in}{0.630079in}}%
\pgfpathlineto{\pgfqpoint{5.488864in}{0.629715in}}%
\pgfpathlineto{\pgfqpoint{5.493753in}{0.629352in}}%
\pgfpathlineto{\pgfqpoint{5.498643in}{0.628989in}}%
\pgfpathlineto{\pgfqpoint{5.503533in}{0.628628in}}%
\pgfpathlineto{\pgfqpoint{5.508423in}{0.628267in}}%
\pgfpathlineto{\pgfqpoint{5.513312in}{0.627907in}}%
\pgfpathlineto{\pgfqpoint{5.518202in}{0.627547in}}%
\pgfpathlineto{\pgfqpoint{5.523092in}{0.627188in}}%
\pgfpathlineto{\pgfqpoint{5.527981in}{0.626830in}}%
\pgfpathlineto{\pgfqpoint{5.532871in}{0.626472in}}%
\pgfpathlineto{\pgfqpoint{5.537761in}{0.626115in}}%
\pgfpathlineto{\pgfqpoint{5.542650in}{0.625759in}}%
\pgfpathlineto{\pgfqpoint{5.547540in}{0.625404in}}%
\pgfpathlineto{\pgfqpoint{5.552430in}{0.625049in}}%
\pgfpathlineto{\pgfqpoint{5.557319in}{0.624695in}}%
\pgfpathlineto{\pgfqpoint{5.562209in}{0.624341in}}%
\pgfpathlineto{\pgfqpoint{5.567099in}{0.623989in}}%
\pgfpathlineto{\pgfqpoint{5.571988in}{0.623636in}}%
\pgfpathlineto{\pgfqpoint{5.576878in}{0.623285in}}%
\pgfpathlineto{\pgfqpoint{5.581768in}{0.622934in}}%
\pgfpathlineto{\pgfqpoint{5.586657in}{0.622584in}}%
\pgfpathlineto{\pgfqpoint{5.591547in}{0.622234in}}%
\pgfpathlineto{\pgfqpoint{5.596437in}{0.621886in}}%
\pgfpathlineto{\pgfqpoint{5.601326in}{0.621537in}}%
\pgfpathlineto{\pgfqpoint{5.606216in}{0.621190in}}%
\pgfpathlineto{\pgfqpoint{5.611106in}{0.620843in}}%
\pgfpathlineto{\pgfqpoint{5.615995in}{0.620497in}}%
\pgfpathlineto{\pgfqpoint{5.620885in}{0.620151in}}%
\pgfpathlineto{\pgfqpoint{5.625775in}{0.619806in}}%
\pgfpathlineto{\pgfqpoint{5.630664in}{0.619462in}}%
\pgfpathlineto{\pgfqpoint{5.635554in}{0.619118in}}%
\pgfpathlineto{\pgfqpoint{5.640444in}{0.618775in}}%
\pgfpathlineto{\pgfqpoint{5.645333in}{0.618433in}}%
\pgfpathlineto{\pgfqpoint{5.650223in}{0.618091in}}%
\pgfpathlineto{\pgfqpoint{5.655113in}{0.617750in}}%
\pgfpathlineto{\pgfqpoint{5.660002in}{0.617410in}}%
\pgfpathlineto{\pgfqpoint{5.664892in}{0.617070in}}%
\pgfpathlineto{\pgfqpoint{5.669782in}{0.616731in}}%
\pgfpathlineto{\pgfqpoint{5.674671in}{0.616392in}}%
\pgfpathlineto{\pgfqpoint{5.679561in}{0.616054in}}%
\pgfpathlineto{\pgfqpoint{5.684451in}{0.615717in}}%
\pgfpathlineto{\pgfqpoint{5.689340in}{0.615380in}}%
\pgfpathlineto{\pgfqpoint{5.694230in}{0.615044in}}%
\pgfpathlineto{\pgfqpoint{5.699120in}{0.614709in}}%
\pgfpathlineto{\pgfqpoint{5.704009in}{0.614374in}}%
\pgfpathlineto{\pgfqpoint{5.708899in}{0.614040in}}%
\pgfpathlineto{\pgfqpoint{5.713789in}{0.613706in}}%
\pgfpathlineto{\pgfqpoint{5.718678in}{0.613373in}}%
\pgfpathlineto{\pgfqpoint{5.723568in}{0.613041in}}%
\pgfpathlineto{\pgfqpoint{5.728458in}{0.612709in}}%
\pgfpathlineto{\pgfqpoint{5.733347in}{0.612378in}}%
\pgfpathlineto{\pgfqpoint{5.738237in}{0.612048in}}%
\pgfpathlineto{\pgfqpoint{5.743127in}{0.611718in}}%
\pgfpathlineto{\pgfqpoint{5.748016in}{0.611388in}}%
\pgfpathlineto{\pgfqpoint{5.752906in}{0.611060in}}%
\pgfpathlineto{\pgfqpoint{5.757796in}{0.610732in}}%
\pgfpathlineto{\pgfqpoint{5.762685in}{0.610404in}}%
\pgfpathlineto{\pgfqpoint{5.767575in}{0.610077in}}%
\pgfpathlineto{\pgfqpoint{5.767575in}{0.610077in}}%
\pgfpathlineto{\pgfqpoint{5.775652in}{0.609548in}}%
\pgfpathlineto{\pgfqpoint{5.783728in}{0.609040in}}%
\pgfpathlineto{\pgfqpoint{5.791805in}{0.608550in}}%
\pgfpathlineto{\pgfqpoint{5.799882in}{0.608078in}}%
\pgfpathlineto{\pgfqpoint{5.807958in}{0.607622in}}%
\pgfpathlineto{\pgfqpoint{5.816035in}{0.607182in}}%
\pgfpathlineto{\pgfqpoint{5.824111in}{0.606756in}}%
\pgfpathlineto{\pgfqpoint{5.832188in}{0.606343in}}%
\pgfpathlineto{\pgfqpoint{5.840265in}{0.605943in}}%
\pgfpathlineto{\pgfqpoint{5.848341in}{0.605555in}}%
\pgfpathlineto{\pgfqpoint{5.856418in}{0.605178in}}%
\pgfpathlineto{\pgfqpoint{5.864495in}{0.604812in}}%
\pgfpathlineto{\pgfqpoint{5.872571in}{0.604456in}}%
\pgfpathlineto{\pgfqpoint{5.880648in}{0.604110in}}%
\pgfpathlineto{\pgfqpoint{5.888725in}{0.603773in}}%
\pgfpathlineto{\pgfqpoint{5.896801in}{0.603445in}}%
\pgfpathlineto{\pgfqpoint{5.904878in}{0.603125in}}%
\pgfpathlineto{\pgfqpoint{5.912954in}{0.602813in}}%
\pgfpathlineto{\pgfqpoint{5.921031in}{0.602508in}}%
\pgfpathlineto{\pgfqpoint{5.929108in}{0.602211in}}%
\pgfpathlineto{\pgfqpoint{5.937184in}{0.601921in}}%
\pgfpathlineto{\pgfqpoint{5.945261in}{0.601637in}}%
\pgfpathlineto{\pgfqpoint{5.953338in}{0.601360in}}%
\pgfpathlineto{\pgfqpoint{5.961414in}{0.601088in}}%
\pgfpathlineto{\pgfqpoint{5.969491in}{0.600823in}}%
\pgfpathlineto{\pgfqpoint{5.977567in}{0.600564in}}%
\pgfpathlineto{\pgfqpoint{5.985644in}{0.600309in}}%
\pgfpathlineto{\pgfqpoint{5.993721in}{0.600060in}}%
\pgfpathlineto{\pgfqpoint{6.001797in}{0.599817in}}%
\pgfpathlineto{\pgfqpoint{6.009874in}{0.599578in}}%
\pgfpathlineto{\pgfqpoint{6.017951in}{0.599343in}}%
\pgfpathlineto{\pgfqpoint{6.026027in}{0.599113in}}%
\pgfpathlineto{\pgfqpoint{6.034104in}{0.598888in}}%
\pgfpathlineto{\pgfqpoint{6.042181in}{0.598667in}}%
\pgfpathlineto{\pgfqpoint{6.050257in}{0.598450in}}%
\pgfpathlineto{\pgfqpoint{6.058334in}{0.598237in}}%
\pgfpathlineto{\pgfqpoint{6.066410in}{0.598028in}}%
\pgfpathlineto{\pgfqpoint{6.074487in}{0.597822in}}%
\pgfpathlineto{\pgfqpoint{6.082564in}{0.597620in}}%
\pgfpathlineto{\pgfqpoint{6.090640in}{0.597422in}}%
\pgfpathlineto{\pgfqpoint{6.098717in}{0.597227in}}%
\pgfpathlineto{\pgfqpoint{6.106794in}{0.597035in}}%
\pgfpathlineto{\pgfqpoint{6.114870in}{0.596846in}}%
\pgfpathlineto{\pgfqpoint{6.122947in}{0.596661in}}%
\pgfpathlineto{\pgfqpoint{6.131023in}{0.596478in}}%
\pgfpathlineto{\pgfqpoint{6.139100in}{0.596299in}}%
\pgfpathlineto{\pgfqpoint{6.147177in}{0.596122in}}%
\pgfpathlineto{\pgfqpoint{6.155253in}{0.595948in}}%
\pgfpathlineto{\pgfqpoint{6.163330in}{0.595777in}}%
\pgfpathlineto{\pgfqpoint{6.171407in}{0.595609in}}%
\pgfpathlineto{\pgfqpoint{6.179483in}{0.595443in}}%
\pgfpathlineto{\pgfqpoint{6.187560in}{0.595279in}}%
\pgfpathlineto{\pgfqpoint{6.195637in}{0.595118in}}%
\pgfpathlineto{\pgfqpoint{6.203713in}{0.594959in}}%
\pgfpathlineto{\pgfqpoint{6.211790in}{0.594803in}}%
\pgfpathlineto{\pgfqpoint{6.219866in}{0.594649in}}%
\pgfpathlineto{\pgfqpoint{6.227943in}{0.594497in}}%
\pgfpathlineto{\pgfqpoint{6.236020in}{0.594347in}}%
\pgfpathlineto{\pgfqpoint{6.244096in}{0.594199in}}%
\pgfpathlineto{\pgfqpoint{6.252173in}{0.594054in}}%
\pgfpathlineto{\pgfqpoint{6.260250in}{0.593910in}}%
\pgfpathlineto{\pgfqpoint{6.268326in}{0.593768in}}%
\pgfpathlineto{\pgfqpoint{6.276403in}{0.593628in}}%
\pgfpathlineto{\pgfqpoint{6.284479in}{0.593490in}}%
\pgfpathlineto{\pgfqpoint{6.292556in}{0.593354in}}%
\pgfpathlineto{\pgfqpoint{6.300633in}{0.593220in}}%
\pgfpathlineto{\pgfqpoint{6.308709in}{0.593087in}}%
\pgfpathlineto{\pgfqpoint{6.316786in}{0.592957in}}%
\pgfpathlineto{\pgfqpoint{6.324863in}{0.592827in}}%
\pgfpathlineto{\pgfqpoint{6.332939in}{0.592700in}}%
\pgfpathlineto{\pgfqpoint{6.341016in}{0.592574in}}%
\pgfpathlineto{\pgfqpoint{6.349093in}{0.592449in}}%
\pgfpathlineto{\pgfqpoint{6.357169in}{0.592326in}}%
\pgfpathlineto{\pgfqpoint{6.365246in}{0.592205in}}%
\pgfpathlineto{\pgfqpoint{6.373322in}{0.592085in}}%
\pgfpathlineto{\pgfqpoint{6.381399in}{0.591967in}}%
\pgfpathlineto{\pgfqpoint{6.389476in}{0.591850in}}%
\pgfpathlineto{\pgfqpoint{6.397552in}{0.591734in}}%
\pgfpathlineto{\pgfqpoint{6.405629in}{0.591620in}}%
\pgfpathlineto{\pgfqpoint{6.413706in}{0.591507in}}%
\pgfpathlineto{\pgfqpoint{6.421782in}{0.591395in}}%
\pgfpathlineto{\pgfqpoint{6.429859in}{0.591285in}}%
\pgfpathlineto{\pgfqpoint{6.437935in}{0.591175in}}%
\pgfpathlineto{\pgfqpoint{6.446012in}{0.591068in}}%
\pgfpathlineto{\pgfqpoint{6.454089in}{0.590961in}}%
\pgfpathlineto{\pgfqpoint{6.462165in}{0.590855in}}%
\pgfpathlineto{\pgfqpoint{6.470242in}{0.590751in}}%
\pgfpathlineto{\pgfqpoint{6.478319in}{0.590648in}}%
\pgfpathlineto{\pgfqpoint{6.486395in}{0.590546in}}%
\pgfpathlineto{\pgfqpoint{6.494472in}{0.590445in}}%
\pgfpathlineto{\pgfqpoint{6.502549in}{0.590345in}}%
\pgfpathlineto{\pgfqpoint{6.510625in}{0.590246in}}%
\pgfpathlineto{\pgfqpoint{6.518702in}{0.590148in}}%
\pgfpathlineto{\pgfqpoint{6.526778in}{0.590052in}}%
\pgfpathlineto{\pgfqpoint{6.534855in}{0.589956in}}%
\pgfpathlineto{\pgfqpoint{6.542932in}{0.589861in}}%
\pgfpathlineto{\pgfqpoint{6.551008in}{0.589768in}}%
\pgfpathlineto{\pgfqpoint{6.559085in}{0.589675in}}%
\pgfpathlineto{\pgfqpoint{6.567162in}{0.589583in}}%
\pgfpathlineto{\pgfqpoint{7.800000in}{0.657153in}}%
\pgfpathlineto{\pgfqpoint{7.790308in}{0.657264in}}%
\pgfpathlineto{\pgfqpoint{7.780616in}{0.657375in}}%
\pgfpathlineto{\pgfqpoint{7.770924in}{0.657487in}}%
\pgfpathlineto{\pgfqpoint{7.761232in}{0.657601in}}%
\pgfpathlineto{\pgfqpoint{7.751540in}{0.657716in}}%
\pgfpathlineto{\pgfqpoint{7.741848in}{0.657832in}}%
\pgfpathlineto{\pgfqpoint{7.732156in}{0.657949in}}%
\pgfpathlineto{\pgfqpoint{7.722464in}{0.658068in}}%
\pgfpathlineto{\pgfqpoint{7.712772in}{0.658188in}}%
\pgfpathlineto{\pgfqpoint{7.703080in}{0.658309in}}%
\pgfpathlineto{\pgfqpoint{7.693388in}{0.658431in}}%
\pgfpathlineto{\pgfqpoint{7.683697in}{0.658555in}}%
\pgfpathlineto{\pgfqpoint{7.674005in}{0.658680in}}%
\pgfpathlineto{\pgfqpoint{7.664313in}{0.658807in}}%
\pgfpathlineto{\pgfqpoint{7.654621in}{0.658935in}}%
\pgfpathlineto{\pgfqpoint{7.644929in}{0.659064in}}%
\pgfpathlineto{\pgfqpoint{7.635237in}{0.659195in}}%
\pgfpathlineto{\pgfqpoint{7.625545in}{0.659328in}}%
\pgfpathlineto{\pgfqpoint{7.615853in}{0.659462in}}%
\pgfpathlineto{\pgfqpoint{7.606161in}{0.659597in}}%
\pgfpathlineto{\pgfqpoint{7.596469in}{0.659734in}}%
\pgfpathlineto{\pgfqpoint{7.586777in}{0.659873in}}%
\pgfpathlineto{\pgfqpoint{7.577085in}{0.660014in}}%
\pgfpathlineto{\pgfqpoint{7.567393in}{0.660156in}}%
\pgfpathlineto{\pgfqpoint{7.557701in}{0.660300in}}%
\pgfpathlineto{\pgfqpoint{7.548009in}{0.660445in}}%
\pgfpathlineto{\pgfqpoint{7.538317in}{0.660593in}}%
\pgfpathlineto{\pgfqpoint{7.528625in}{0.660742in}}%
\pgfpathlineto{\pgfqpoint{7.518933in}{0.660893in}}%
\pgfpathlineto{\pgfqpoint{7.509241in}{0.661047in}}%
\pgfpathlineto{\pgfqpoint{7.499549in}{0.661202in}}%
\pgfpathlineto{\pgfqpoint{7.489857in}{0.661359in}}%
\pgfpathlineto{\pgfqpoint{7.480165in}{0.661518in}}%
\pgfpathlineto{\pgfqpoint{7.470473in}{0.661679in}}%
\pgfpathlineto{\pgfqpoint{7.460781in}{0.661842in}}%
\pgfpathlineto{\pgfqpoint{7.451090in}{0.662008in}}%
\pgfpathlineto{\pgfqpoint{7.441398in}{0.662176in}}%
\pgfpathlineto{\pgfqpoint{7.431706in}{0.662346in}}%
\pgfpathlineto{\pgfqpoint{7.422014in}{0.662518in}}%
\pgfpathlineto{\pgfqpoint{7.412322in}{0.662693in}}%
\pgfpathlineto{\pgfqpoint{7.402630in}{0.662870in}}%
\pgfpathlineto{\pgfqpoint{7.392938in}{0.663050in}}%
\pgfpathlineto{\pgfqpoint{7.383246in}{0.663232in}}%
\pgfpathlineto{\pgfqpoint{7.373554in}{0.663417in}}%
\pgfpathlineto{\pgfqpoint{7.363862in}{0.663605in}}%
\pgfpathlineto{\pgfqpoint{7.354170in}{0.663795in}}%
\pgfpathlineto{\pgfqpoint{7.344478in}{0.663989in}}%
\pgfpathlineto{\pgfqpoint{7.334786in}{0.664185in}}%
\pgfpathlineto{\pgfqpoint{7.325094in}{0.664384in}}%
\pgfpathlineto{\pgfqpoint{7.315402in}{0.664586in}}%
\pgfpathlineto{\pgfqpoint{7.305710in}{0.664792in}}%
\pgfpathlineto{\pgfqpoint{7.296018in}{0.665000in}}%
\pgfpathlineto{\pgfqpoint{7.286326in}{0.665212in}}%
\pgfpathlineto{\pgfqpoint{7.276634in}{0.665428in}}%
\pgfpathlineto{\pgfqpoint{7.266942in}{0.665647in}}%
\pgfpathlineto{\pgfqpoint{7.257250in}{0.665869in}}%
\pgfpathlineto{\pgfqpoint{7.247558in}{0.666095in}}%
\pgfpathlineto{\pgfqpoint{7.237866in}{0.666326in}}%
\pgfpathlineto{\pgfqpoint{7.228174in}{0.666560in}}%
\pgfpathlineto{\pgfqpoint{7.218483in}{0.666798in}}%
\pgfpathlineto{\pgfqpoint{7.208791in}{0.667040in}}%
\pgfpathlineto{\pgfqpoint{7.199099in}{0.667287in}}%
\pgfpathlineto{\pgfqpoint{7.189407in}{0.667538in}}%
\pgfpathlineto{\pgfqpoint{7.179715in}{0.667794in}}%
\pgfpathlineto{\pgfqpoint{7.170023in}{0.668054in}}%
\pgfpathlineto{\pgfqpoint{7.160331in}{0.668319in}}%
\pgfpathlineto{\pgfqpoint{7.150639in}{0.668590in}}%
\pgfpathlineto{\pgfqpoint{7.140947in}{0.668866in}}%
\pgfpathlineto{\pgfqpoint{7.131255in}{0.669147in}}%
\pgfpathlineto{\pgfqpoint{7.121563in}{0.669434in}}%
\pgfpathlineto{\pgfqpoint{7.111871in}{0.669726in}}%
\pgfpathlineto{\pgfqpoint{7.102179in}{0.670025in}}%
\pgfpathlineto{\pgfqpoint{7.092487in}{0.670330in}}%
\pgfpathlineto{\pgfqpoint{7.082795in}{0.670642in}}%
\pgfpathlineto{\pgfqpoint{7.073103in}{0.670960in}}%
\pgfpathlineto{\pgfqpoint{7.063411in}{0.671285in}}%
\pgfpathlineto{\pgfqpoint{7.053719in}{0.671618in}}%
\pgfpathlineto{\pgfqpoint{7.044027in}{0.671958in}}%
\pgfpathlineto{\pgfqpoint{7.034335in}{0.672307in}}%
\pgfpathlineto{\pgfqpoint{7.024643in}{0.672663in}}%
\pgfpathlineto{\pgfqpoint{7.014951in}{0.673029in}}%
\pgfpathlineto{\pgfqpoint{7.005259in}{0.673403in}}%
\pgfpathlineto{\pgfqpoint{6.995567in}{0.673787in}}%
\pgfpathlineto{\pgfqpoint{6.985876in}{0.674181in}}%
\pgfpathlineto{\pgfqpoint{6.976184in}{0.674586in}}%
\pgfpathlineto{\pgfqpoint{6.966492in}{0.675001in}}%
\pgfpathlineto{\pgfqpoint{6.956800in}{0.675428in}}%
\pgfpathlineto{\pgfqpoint{6.947108in}{0.675867in}}%
\pgfpathlineto{\pgfqpoint{6.937416in}{0.676319in}}%
\pgfpathlineto{\pgfqpoint{6.927724in}{0.676785in}}%
\pgfpathlineto{\pgfqpoint{6.918032in}{0.677265in}}%
\pgfpathlineto{\pgfqpoint{6.908340in}{0.677760in}}%
\pgfpathlineto{\pgfqpoint{6.898648in}{0.678272in}}%
\pgfpathlineto{\pgfqpoint{6.888956in}{0.678800in}}%
\pgfpathlineto{\pgfqpoint{6.879264in}{0.679347in}}%
\pgfpathlineto{\pgfqpoint{6.869572in}{0.679914in}}%
\pgfpathlineto{\pgfqpoint{6.859880in}{0.680502in}}%
\pgfpathlineto{\pgfqpoint{6.850188in}{0.681112in}}%
\pgfpathlineto{\pgfqpoint{6.840496in}{0.681746in}}%
\pgfpathlineto{\pgfqpoint{6.840496in}{0.681746in}}%
\pgfpathlineto{\pgfqpoint{6.834629in}{0.682139in}}%
\pgfpathlineto{\pgfqpoint{6.828761in}{0.682532in}}%
\pgfpathlineto{\pgfqpoint{6.822893in}{0.682925in}}%
\pgfpathlineto{\pgfqpoint{6.817026in}{0.683320in}}%
\pgfpathlineto{\pgfqpoint{6.811158in}{0.683715in}}%
\pgfpathlineto{\pgfqpoint{6.805291in}{0.684111in}}%
\pgfpathlineto{\pgfqpoint{6.799423in}{0.684508in}}%
\pgfpathlineto{\pgfqpoint{6.793555in}{0.684905in}}%
\pgfpathlineto{\pgfqpoint{6.787688in}{0.685303in}}%
\pgfpathlineto{\pgfqpoint{6.781820in}{0.685702in}}%
\pgfpathlineto{\pgfqpoint{6.775953in}{0.686101in}}%
\pgfpathlineto{\pgfqpoint{6.770085in}{0.686502in}}%
\pgfpathlineto{\pgfqpoint{6.764217in}{0.686903in}}%
\pgfpathlineto{\pgfqpoint{6.758350in}{0.687305in}}%
\pgfpathlineto{\pgfqpoint{6.752482in}{0.687707in}}%
\pgfpathlineto{\pgfqpoint{6.746614in}{0.688110in}}%
\pgfpathlineto{\pgfqpoint{6.740747in}{0.688514in}}%
\pgfpathlineto{\pgfqpoint{6.734879in}{0.688919in}}%
\pgfpathlineto{\pgfqpoint{6.729012in}{0.689325in}}%
\pgfpathlineto{\pgfqpoint{6.723144in}{0.689731in}}%
\pgfpathlineto{\pgfqpoint{6.717276in}{0.690138in}}%
\pgfpathlineto{\pgfqpoint{6.711409in}{0.690546in}}%
\pgfpathlineto{\pgfqpoint{6.705541in}{0.690954in}}%
\pgfpathlineto{\pgfqpoint{6.699674in}{0.691363in}}%
\pgfpathlineto{\pgfqpoint{6.693806in}{0.691773in}}%
\pgfpathlineto{\pgfqpoint{6.687938in}{0.692184in}}%
\pgfpathlineto{\pgfqpoint{6.682071in}{0.692596in}}%
\pgfpathlineto{\pgfqpoint{6.676203in}{0.693008in}}%
\pgfpathlineto{\pgfqpoint{6.670336in}{0.693421in}}%
\pgfpathlineto{\pgfqpoint{6.664468in}{0.693835in}}%
\pgfpathlineto{\pgfqpoint{6.658600in}{0.694250in}}%
\pgfpathlineto{\pgfqpoint{6.652733in}{0.694665in}}%
\pgfpathlineto{\pgfqpoint{6.646865in}{0.695082in}}%
\pgfpathlineto{\pgfqpoint{6.640998in}{0.695499in}}%
\pgfpathlineto{\pgfqpoint{6.635130in}{0.695916in}}%
\pgfpathlineto{\pgfqpoint{6.629262in}{0.696335in}}%
\pgfpathlineto{\pgfqpoint{6.623395in}{0.696754in}}%
\pgfpathlineto{\pgfqpoint{6.617527in}{0.697175in}}%
\pgfpathlineto{\pgfqpoint{6.611660in}{0.697596in}}%
\pgfpathlineto{\pgfqpoint{6.605792in}{0.698017in}}%
\pgfpathlineto{\pgfqpoint{6.599924in}{0.698440in}}%
\pgfpathlineto{\pgfqpoint{6.594057in}{0.698863in}}%
\pgfpathlineto{\pgfqpoint{6.588189in}{0.699288in}}%
\pgfpathlineto{\pgfqpoint{6.582322in}{0.699712in}}%
\pgfpathlineto{\pgfqpoint{6.576454in}{0.700138in}}%
\pgfpathlineto{\pgfqpoint{6.570586in}{0.700565in}}%
\pgfpathlineto{\pgfqpoint{6.564719in}{0.700992in}}%
\pgfpathlineto{\pgfqpoint{6.558851in}{0.701420in}}%
\pgfpathlineto{\pgfqpoint{6.552984in}{0.701850in}}%
\pgfpathlineto{\pgfqpoint{6.547116in}{0.702279in}}%
\pgfpathlineto{\pgfqpoint{6.541248in}{0.702710in}}%
\pgfpathlineto{\pgfqpoint{6.535381in}{0.703142in}}%
\pgfpathlineto{\pgfqpoint{6.529513in}{0.703574in}}%
\pgfpathlineto{\pgfqpoint{6.523646in}{0.704007in}}%
\pgfpathlineto{\pgfqpoint{6.517778in}{0.704441in}}%
\pgfpathlineto{\pgfqpoint{6.511910in}{0.704876in}}%
\pgfpathlineto{\pgfqpoint{6.506043in}{0.705312in}}%
\pgfpathlineto{\pgfqpoint{6.500175in}{0.705748in}}%
\pgfpathlineto{\pgfqpoint{6.494308in}{0.706185in}}%
\pgfpathlineto{\pgfqpoint{6.488440in}{0.706624in}}%
\pgfpathlineto{\pgfqpoint{6.482572in}{0.707063in}}%
\pgfpathlineto{\pgfqpoint{6.476705in}{0.707502in}}%
\pgfpathlineto{\pgfqpoint{6.470837in}{0.707943in}}%
\pgfpathlineto{\pgfqpoint{6.464969in}{0.708385in}}%
\pgfpathlineto{\pgfqpoint{6.459102in}{0.708827in}}%
\pgfpathlineto{\pgfqpoint{6.453234in}{0.709271in}}%
\pgfpathlineto{\pgfqpoint{6.447367in}{0.709715in}}%
\pgfpathlineto{\pgfqpoint{6.441499in}{0.710160in}}%
\pgfpathlineto{\pgfqpoint{6.435631in}{0.710606in}}%
\pgfpathlineto{\pgfqpoint{6.429764in}{0.711052in}}%
\pgfpathlineto{\pgfqpoint{6.423896in}{0.711500in}}%
\pgfpathlineto{\pgfqpoint{6.418029in}{0.711948in}}%
\pgfpathlineto{\pgfqpoint{6.412161in}{0.712398in}}%
\pgfpathlineto{\pgfqpoint{6.406293in}{0.712848in}}%
\pgfpathlineto{\pgfqpoint{6.400426in}{0.713299in}}%
\pgfpathlineto{\pgfqpoint{6.394558in}{0.713751in}}%
\pgfpathlineto{\pgfqpoint{6.388691in}{0.714204in}}%
\pgfpathlineto{\pgfqpoint{6.382823in}{0.714658in}}%
\pgfpathlineto{\pgfqpoint{6.376955in}{0.715113in}}%
\pgfpathlineto{\pgfqpoint{6.371088in}{0.715568in}}%
\pgfpathlineto{\pgfqpoint{6.365220in}{0.716025in}}%
\pgfpathlineto{\pgfqpoint{6.359353in}{0.716482in}}%
\pgfpathlineto{\pgfqpoint{6.353485in}{0.716940in}}%
\pgfpathlineto{\pgfqpoint{6.347617in}{0.717399in}}%
\pgfpathlineto{\pgfqpoint{6.341750in}{0.717860in}}%
\pgfpathlineto{\pgfqpoint{6.335882in}{0.718321in}}%
\pgfpathlineto{\pgfqpoint{6.330015in}{0.718782in}}%
\pgfpathlineto{\pgfqpoint{6.324147in}{0.719245in}}%
\pgfpathlineto{\pgfqpoint{6.318279in}{0.719709in}}%
\pgfpathlineto{\pgfqpoint{6.312412in}{0.720174in}}%
\pgfpathlineto{\pgfqpoint{6.306544in}{0.720639in}}%
\pgfpathlineto{\pgfqpoint{6.300677in}{0.721106in}}%
\pgfpathlineto{\pgfqpoint{6.294809in}{0.721573in}}%
\pgfpathlineto{\pgfqpoint{6.288941in}{0.722042in}}%
\pgfpathlineto{\pgfqpoint{6.283074in}{0.722511in}}%
\pgfpathlineto{\pgfqpoint{6.277206in}{0.722981in}}%
\pgfpathlineto{\pgfqpoint{6.271339in}{0.723452in}}%
\pgfpathlineto{\pgfqpoint{6.265471in}{0.723925in}}%
\pgfpathlineto{\pgfqpoint{6.259603in}{0.724398in}}%
\pgfpathlineto{\pgfqpoint{6.253736in}{0.724872in}}%
\pgfpathlineto{\pgfqpoint{6.247868in}{0.725347in}}%
\pgfpathlineto{\pgfqpoint{6.242001in}{0.725823in}}%
\pgfpathlineto{\pgfqpoint{6.236133in}{0.726299in}}%
\pgfpathlineto{\pgfqpoint{6.230265in}{0.726777in}}%
\pgfpathlineto{\pgfqpoint{6.224398in}{0.727256in}}%
\pgfpathlineto{\pgfqpoint{6.218530in}{0.727736in}}%
\pgfpathlineto{\pgfqpoint{6.212662in}{0.728217in}}%
\pgfpathlineto{\pgfqpoint{6.206795in}{0.728698in}}%
\pgfpathlineto{\pgfqpoint{6.200927in}{0.729181in}}%
\pgfpathlineto{\pgfqpoint{6.195060in}{0.729665in}}%
\pgfpathlineto{\pgfqpoint{6.189192in}{0.730149in}}%
\pgfpathlineto{\pgfqpoint{6.183324in}{0.730635in}}%
\pgfpathlineto{\pgfqpoint{6.177457in}{0.731122in}}%
\pgfpathlineto{\pgfqpoint{6.171589in}{0.731609in}}%
\pgfpathlineto{\pgfqpoint{6.165722in}{0.732098in}}%
\pgfpathlineto{\pgfqpoint{6.159854in}{0.732587in}}%
\pgfpathlineto{\pgfqpoint{6.153986in}{0.733078in}}%
\pgfpathlineto{\pgfqpoint{6.148119in}{0.733570in}}%
\pgfpathlineto{\pgfqpoint{6.142251in}{0.734062in}}%
\pgfpathlineto{\pgfqpoint{6.136384in}{0.734556in}}%
\pgfpathlineto{\pgfqpoint{6.130516in}{0.735051in}}%
\pgfpathlineto{\pgfqpoint{6.124648in}{0.735546in}}%
\pgfpathlineto{\pgfqpoint{6.118781in}{0.736043in}}%
\pgfpathlineto{\pgfqpoint{6.112913in}{0.736541in}}%
\pgfpathlineto{\pgfqpoint{6.107046in}{0.737039in}}%
\pgfpathlineto{\pgfqpoint{6.101178in}{0.737539in}}%
\pgfpathlineto{\pgfqpoint{6.095310in}{0.738040in}}%
\pgfpathlineto{\pgfqpoint{6.089443in}{0.738542in}}%
\pgfpathlineto{\pgfqpoint{6.083575in}{0.739044in}}%
\pgfpathlineto{\pgfqpoint{6.077708in}{0.739548in}}%
\pgfpathlineto{\pgfqpoint{6.071840in}{0.740053in}}%
\pgfpathlineto{\pgfqpoint{6.065972in}{0.740559in}}%
\pgfpathlineto{\pgfqpoint{6.060105in}{0.741066in}}%
\pgfpathlineto{\pgfqpoint{6.054237in}{0.741574in}}%
\pgfpathlineto{\pgfqpoint{6.048370in}{0.742083in}}%
\pgfpathlineto{\pgfqpoint{6.042502in}{0.742594in}}%
\pgfpathlineto{\pgfqpoint{6.036634in}{0.743105in}}%
\pgfpathlineto{\pgfqpoint{6.030767in}{0.743617in}}%
\pgfpathlineto{\pgfqpoint{6.024899in}{0.744130in}}%
\pgfpathlineto{\pgfqpoint{6.019032in}{0.744645in}}%
\pgfpathlineto{\pgfqpoint{6.013164in}{0.745160in}}%
\pgfpathlineto{\pgfqpoint{6.007296in}{0.745677in}}%
\pgfpathlineto{\pgfqpoint{6.001429in}{0.746195in}}%
\pgfpathlineto{\pgfqpoint{5.995561in}{0.746714in}}%
\pgfpathlineto{\pgfqpoint{5.989694in}{0.747233in}}%
\pgfpathlineto{\pgfqpoint{5.983826in}{0.747754in}}%
\pgfpathlineto{\pgfqpoint{5.977958in}{0.748276in}}%
\pgfpathlineto{\pgfqpoint{5.972091in}{0.748800in}}%
\pgfpathlineto{\pgfqpoint{5.966223in}{0.749324in}}%
\pgfpathlineto{\pgfqpoint{5.960355in}{0.749849in}}%
\pgfpathlineto{\pgfqpoint{5.954488in}{0.750376in}}%
\pgfpathlineto{\pgfqpoint{5.948620in}{0.750903in}}%
\pgfpathlineto{\pgfqpoint{5.942753in}{0.751432in}}%
\pgfpathlineto{\pgfqpoint{5.936885in}{0.751962in}}%
\pgfpathlineto{\pgfqpoint{5.931017in}{0.752493in}}%
\pgfpathlineto{\pgfqpoint{5.925150in}{0.753025in}}%
\pgfpathlineto{\pgfqpoint{5.919282in}{0.753558in}}%
\pgfpathlineto{\pgfqpoint{5.913415in}{0.754092in}}%
\pgfpathlineto{\pgfqpoint{5.907547in}{0.754628in}}%
\pgfpathlineto{\pgfqpoint{5.901679in}{0.755165in}}%
\pgfpathlineto{\pgfqpoint{5.895812in}{0.755702in}}%
\pgfpathlineto{\pgfqpoint{5.889944in}{0.756241in}}%
\pgfpathlineto{\pgfqpoint{5.884077in}{0.756781in}}%
\pgfpathlineto{\pgfqpoint{5.878209in}{0.757323in}}%
\pgfpathlineto{\pgfqpoint{5.872341in}{0.757865in}}%
\pgfpathlineto{\pgfqpoint{5.866474in}{0.758409in}}%
\pgfpathlineto{\pgfqpoint{5.860606in}{0.758953in}}%
\pgfpathlineto{\pgfqpoint{5.854739in}{0.759499in}}%
\pgfpathlineto{\pgfqpoint{5.848871in}{0.760046in}}%
\pgfpathlineto{\pgfqpoint{5.843003in}{0.760595in}}%
\pgfpathlineto{\pgfqpoint{5.837136in}{0.761144in}}%
\pgfpathlineto{\pgfqpoint{5.831268in}{0.761695in}}%
\pgfpathlineto{\pgfqpoint{5.825401in}{0.762246in}}%
\pgfpathlineto{\pgfqpoint{5.819533in}{0.762800in}}%
\pgfpathlineto{\pgfqpoint{5.813665in}{0.763354in}}%
\pgfpathlineto{\pgfqpoint{5.807798in}{0.763909in}}%
\pgfpathlineto{\pgfqpoint{5.801930in}{0.764466in}}%
\pgfpathlineto{\pgfqpoint{5.796063in}{0.765024in}}%
\pgfpathlineto{\pgfqpoint{5.790195in}{0.765583in}}%
\pgfpathlineto{\pgfqpoint{5.784327in}{0.766143in}}%
\pgfpathlineto{\pgfqpoint{5.778460in}{0.766704in}}%
\pgfpathlineto{\pgfqpoint{5.772592in}{0.767267in}}%
\pgfpathlineto{\pgfqpoint{5.766725in}{0.767831in}}%
\pgfpathlineto{\pgfqpoint{5.760857in}{0.768396in}}%
\pgfpathlineto{\pgfqpoint{5.754989in}{0.768963in}}%
\pgfpathlineto{\pgfqpoint{5.749122in}{0.769531in}}%
\pgfpathlineto{\pgfqpoint{5.743254in}{0.770099in}}%
\pgfpathlineto{\pgfqpoint{5.737387in}{0.770670in}}%
\pgfpathlineto{\pgfqpoint{5.731519in}{0.771241in}}%
\pgfpathlineto{\pgfqpoint{5.725651in}{0.771814in}}%
\pgfpathlineto{\pgfqpoint{5.719784in}{0.772388in}}%
\pgfpathlineto{\pgfqpoint{5.713916in}{0.772963in}}%
\pgfpathlineto{\pgfqpoint{5.708049in}{0.773539in}}%
\pgfpathlineto{\pgfqpoint{5.702181in}{0.774117in}}%
\pgfpathlineto{\pgfqpoint{5.696313in}{0.774696in}}%
\pgfpathlineto{\pgfqpoint{5.690446in}{0.775277in}}%
\pgfpathlineto{\pgfqpoint{5.684578in}{0.775858in}}%
\pgfpathlineto{\pgfqpoint{5.678710in}{0.776441in}}%
\pgfpathlineto{\pgfqpoint{5.672843in}{0.777025in}}%
\pgfpathlineto{\pgfqpoint{5.666975in}{0.777611in}}%
\pgfpathlineto{\pgfqpoint{5.661108in}{0.778198in}}%
\pgfpathlineto{\pgfqpoint{5.655240in}{0.778786in}}%
\pgfpathlineto{\pgfqpoint{5.649372in}{0.779375in}}%
\pgfpathlineto{\pgfqpoint{5.643505in}{0.779966in}}%
\pgfpathlineto{\pgfqpoint{5.637637in}{0.780558in}}%
\pgfpathlineto{\pgfqpoint{5.631770in}{0.781152in}}%
\pgfpathlineto{\pgfqpoint{5.625902in}{0.781746in}}%
\pgfpathlineto{\pgfqpoint{5.620034in}{0.782343in}}%
\pgfpathlineto{\pgfqpoint{5.614167in}{0.782940in}}%
\pgfpathlineto{\pgfqpoint{5.608299in}{0.783539in}}%
\pgfpathlineto{\pgfqpoint{5.602432in}{0.784139in}}%
\pgfpathlineto{\pgfqpoint{5.596564in}{0.784740in}}%
\pgfpathlineto{\pgfqpoint{5.590696in}{0.785343in}}%
\pgfpathlineto{\pgfqpoint{5.584829in}{0.785948in}}%
\pgfpathlineto{\pgfqpoint{5.578961in}{0.786553in}}%
\pgfpathlineto{\pgfqpoint{5.573094in}{0.787160in}}%
\pgfpathlineto{\pgfqpoint{5.567226in}{0.787768in}}%
\pgfpathlineto{\pgfqpoint{5.561358in}{0.788378in}}%
\pgfpathlineto{\pgfqpoint{5.555491in}{0.788989in}}%
\pgfpathlineto{\pgfqpoint{5.549623in}{0.789602in}}%
\pgfpathlineto{\pgfqpoint{5.543756in}{0.790216in}}%
\pgfpathlineto{\pgfqpoint{5.537888in}{0.790831in}}%
\pgfpathlineto{\pgfqpoint{5.532020in}{0.791448in}}%
\pgfpathlineto{\pgfqpoint{5.526153in}{0.792066in}}%
\pgfpathlineto{\pgfqpoint{5.520285in}{0.792685in}}%
\pgfpathlineto{\pgfqpoint{5.514418in}{0.793306in}}%
\pgfpathlineto{\pgfqpoint{5.508550in}{0.793929in}}%
\pgfpathlineto{\pgfqpoint{5.502682in}{0.794553in}}%
\pgfpathlineto{\pgfqpoint{5.496815in}{0.795178in}}%
\pgfpathlineto{\pgfqpoint{5.490947in}{0.795805in}}%
\pgfpathlineto{\pgfqpoint{5.485080in}{0.796433in}}%
\pgfpathlineto{\pgfqpoint{5.479212in}{0.797063in}}%
\pgfpathlineto{\pgfqpoint{5.473344in}{0.797694in}}%
\pgfpathlineto{\pgfqpoint{5.467477in}{0.798326in}}%
\pgfpathlineto{\pgfqpoint{5.461609in}{0.798960in}}%
\pgfpathlineto{\pgfqpoint{5.455742in}{0.799596in}}%
\pgfpathlineto{\pgfqpoint{5.449874in}{0.800233in}}%
\pgfpathlineto{\pgfqpoint{5.444006in}{0.800871in}}%
\pgfpathlineto{\pgfqpoint{5.438139in}{0.801511in}}%
\pgfpathlineto{\pgfqpoint{5.432271in}{0.802152in}}%
\pgfpathlineto{\pgfqpoint{5.426403in}{0.802795in}}%
\pgfpathlineto{\pgfqpoint{5.420536in}{0.803440in}}%
\pgfpathlineto{\pgfqpoint{5.414668in}{0.804086in}}%
\pgfpathlineto{\pgfqpoint{5.408801in}{0.804733in}}%
\pgfpathlineto{\pgfqpoint{5.402933in}{0.805382in}}%
\pgfpathlineto{\pgfqpoint{5.397065in}{0.806032in}}%
\pgfpathlineto{\pgfqpoint{5.391198in}{0.806685in}}%
\pgfpathlineto{\pgfqpoint{5.385330in}{0.807338in}}%
\pgfpathlineto{\pgfqpoint{5.379463in}{0.807993in}}%
\pgfpathlineto{\pgfqpoint{5.373595in}{0.808650in}}%
\pgfpathlineto{\pgfqpoint{5.367727in}{0.809308in}}%
\pgfpathlineto{\pgfqpoint{5.361860in}{0.809968in}}%
\pgfpathlineto{\pgfqpoint{5.355992in}{0.810629in}}%
\pgfpathlineto{\pgfqpoint{5.350125in}{0.811292in}}%
\pgfpathlineto{\pgfqpoint{5.344257in}{0.811956in}}%
\pgfpathlineto{\pgfqpoint{5.338389in}{0.812622in}}%
\pgfpathlineto{\pgfqpoint{5.332522in}{0.813290in}}%
\pgfpathlineto{\pgfqpoint{5.326654in}{0.813959in}}%
\pgfpathlineto{\pgfqpoint{5.320787in}{0.814630in}}%
\pgfpathlineto{\pgfqpoint{5.314919in}{0.815303in}}%
\pgfpathlineto{\pgfqpoint{5.309051in}{0.815977in}}%
\pgfpathlineto{\pgfqpoint{5.303184in}{0.816652in}}%
\pgfpathlineto{\pgfqpoint{5.297316in}{0.817329in}}%
\pgfpathlineto{\pgfqpoint{5.291449in}{0.818008in}}%
\pgfpathlineto{\pgfqpoint{5.285581in}{0.818689in}}%
\pgfpathlineto{\pgfqpoint{5.279713in}{0.819371in}}%
\pgfpathlineto{\pgfqpoint{5.273846in}{0.820055in}}%
\pgfpathlineto{\pgfqpoint{5.267978in}{0.820740in}}%
\pgfpathlineto{\pgfqpoint{5.262111in}{0.821427in}}%
\pgfpathlineto{\pgfqpoint{5.256243in}{0.822116in}}%
\pgfpathlineto{\pgfqpoint{5.250375in}{0.822807in}}%
\pgfpathlineto{\pgfqpoint{5.244508in}{0.823499in}}%
\pgfpathlineto{\pgfqpoint{5.238640in}{0.824192in}}%
\pgfpathlineto{\pgfqpoint{5.232773in}{0.824888in}}%
\pgfpathlineto{\pgfqpoint{5.226905in}{0.825585in}}%
\pgfpathlineto{\pgfqpoint{5.221037in}{0.826284in}}%
\pgfpathlineto{\pgfqpoint{5.215170in}{0.826984in}}%
\pgfpathlineto{\pgfqpoint{5.209302in}{0.827687in}}%
\pgfpathlineto{\pgfqpoint{5.203435in}{0.828391in}}%
\pgfpathlineto{\pgfqpoint{5.197567in}{0.829096in}}%
\pgfpathlineto{\pgfqpoint{5.191699in}{0.829804in}}%
\pgfpathlineto{\pgfqpoint{5.185832in}{0.830513in}}%
\pgfpathlineto{\pgfqpoint{5.179964in}{0.831224in}}%
\pgfpathlineto{\pgfqpoint{5.174096in}{0.831937in}}%
\pgfpathlineto{\pgfqpoint{5.168229in}{0.832651in}}%
\pgfpathlineto{\pgfqpoint{5.162361in}{0.833367in}}%
\pgfpathlineto{\pgfqpoint{5.156494in}{0.834085in}}%
\pgfpathlineto{\pgfqpoint{5.150626in}{0.834805in}}%
\pgfpathlineto{\pgfqpoint{5.144758in}{0.835526in}}%
\pgfpathlineto{\pgfqpoint{5.138891in}{0.836250in}}%
\pgfpathlineto{\pgfqpoint{5.133023in}{0.836975in}}%
\pgfpathlineto{\pgfqpoint{5.127156in}{0.837702in}}%
\pgfpathlineto{\pgfqpoint{5.121288in}{0.838430in}}%
\pgfpathlineto{\pgfqpoint{5.115420in}{0.839161in}}%
\pgfpathlineto{\pgfqpoint{5.109553in}{0.839893in}}%
\pgfpathlineto{\pgfqpoint{5.103685in}{0.840627in}}%
\pgfpathlineto{\pgfqpoint{5.097818in}{0.841363in}}%
\pgfpathlineto{\pgfqpoint{5.091950in}{0.842101in}}%
\pgfpathlineto{\pgfqpoint{5.086082in}{0.842841in}}%
\pgfpathlineto{\pgfqpoint{5.080215in}{0.843583in}}%
\pgfpathlineto{\pgfqpoint{5.074347in}{0.844326in}}%
\pgfpathlineto{\pgfqpoint{5.068480in}{0.845071in}}%
\pgfpathlineto{\pgfqpoint{5.062612in}{0.845818in}}%
\pgfpathlineto{\pgfqpoint{5.056744in}{0.846567in}}%
\pgfpathlineto{\pgfqpoint{5.050877in}{0.847318in}}%
\pgfpathlineto{\pgfqpoint{5.045009in}{0.848071in}}%
\pgfpathlineto{\pgfqpoint{5.039142in}{0.848826in}}%
\pgfpathlineto{\pgfqpoint{5.033274in}{0.849583in}}%
\pgfpathlineto{\pgfqpoint{5.027406in}{0.850341in}}%
\pgfpathlineto{\pgfqpoint{5.021539in}{0.851102in}}%
\pgfpathlineto{\pgfqpoint{5.015671in}{0.851864in}}%
\pgfpathlineto{\pgfqpoint{5.009804in}{0.852628in}}%
\pgfpathlineto{\pgfqpoint{5.003936in}{0.853395in}}%
\pgfpathlineto{\pgfqpoint{4.998068in}{0.854163in}}%
\pgfpathlineto{\pgfqpoint{4.992201in}{0.854933in}}%
\pgfpathlineto{\pgfqpoint{4.986333in}{0.855705in}}%
\pgfpathlineto{\pgfqpoint{4.980466in}{0.856480in}}%
\pgfpathlineto{\pgfqpoint{4.974598in}{0.857256in}}%
\pgfpathlineto{\pgfqpoint{4.968730in}{0.858034in}}%
\pgfpathlineto{\pgfqpoint{4.962863in}{0.858814in}}%
\pgfpathlineto{\pgfqpoint{4.956995in}{0.859596in}}%
\pgfpathlineto{\pgfqpoint{4.951128in}{0.860381in}}%
\pgfpathlineto{\pgfqpoint{4.945260in}{0.861167in}}%
\pgfpathlineto{\pgfqpoint{4.939392in}{0.861955in}}%
\pgfpathlineto{\pgfqpoint{4.933525in}{0.862745in}}%
\pgfpathlineto{\pgfqpoint{4.927657in}{0.863538in}}%
\pgfpathlineto{\pgfqpoint{4.921790in}{0.864332in}}%
\pgfpathlineto{\pgfqpoint{4.915922in}{0.865129in}}%
\pgfpathlineto{\pgfqpoint{4.910054in}{0.865927in}}%
\pgfpathlineto{\pgfqpoint{4.904187in}{0.866728in}}%
\pgfpathlineto{\pgfqpoint{4.898319in}{0.867531in}}%
\pgfpathlineto{\pgfqpoint{4.892451in}{0.868335in}}%
\pgfpathlineto{\pgfqpoint{4.886584in}{0.869142in}}%
\pgfpathlineto{\pgfqpoint{4.880716in}{0.869951in}}%
\pgfpathlineto{\pgfqpoint{4.874849in}{0.870763in}}%
\pgfpathlineto{\pgfqpoint{4.868981in}{0.871576in}}%
\pgfpathlineto{\pgfqpoint{4.863113in}{0.872391in}}%
\pgfpathlineto{\pgfqpoint{4.857246in}{0.873209in}}%
\pgfpathlineto{\pgfqpoint{4.851378in}{0.874029in}}%
\pgfpathlineto{\pgfqpoint{4.845511in}{0.874851in}}%
\pgfpathlineto{\pgfqpoint{4.839643in}{0.875675in}}%
\pgfpathlineto{\pgfqpoint{4.833775in}{0.876501in}}%
\pgfpathlineto{\pgfqpoint{4.827908in}{0.877329in}}%
\pgfpathlineto{\pgfqpoint{4.822040in}{0.878160in}}%
\pgfpathlineto{\pgfqpoint{4.816173in}{0.878993in}}%
\pgfpathlineto{\pgfqpoint{4.810305in}{0.879828in}}%
\pgfpathlineto{\pgfqpoint{4.804437in}{0.880665in}}%
\pgfpathlineto{\pgfqpoint{4.798570in}{0.881505in}}%
\pgfpathlineto{\pgfqpoint{4.792702in}{0.882347in}}%
\pgfpathlineto{\pgfqpoint{4.786835in}{0.883191in}}%
\pgfpathlineto{\pgfqpoint{4.780967in}{0.884037in}}%
\pgfpathlineto{\pgfqpoint{4.775099in}{0.884886in}}%
\pgfpathlineto{\pgfqpoint{4.769232in}{0.885737in}}%
\pgfpathlineto{\pgfqpoint{4.763364in}{0.886590in}}%
\pgfpathlineto{\pgfqpoint{4.757497in}{0.887445in}}%
\pgfpathlineto{\pgfqpoint{4.751629in}{0.888303in}}%
\pgfpathlineto{\pgfqpoint{4.745761in}{0.889163in}}%
\pgfpathlineto{\pgfqpoint{4.739894in}{0.890025in}}%
\pgfpathlineto{\pgfqpoint{4.734026in}{0.890890in}}%
\pgfpathlineto{\pgfqpoint{4.728159in}{0.891757in}}%
\pgfpathlineto{\pgfqpoint{4.722291in}{0.892627in}}%
\pgfpathlineto{\pgfqpoint{4.716423in}{0.893499in}}%
\pgfpathlineto{\pgfqpoint{4.710556in}{0.894373in}}%
\pgfpathlineto{\pgfqpoint{4.704688in}{0.895249in}}%
\pgfpathlineto{\pgfqpoint{4.698821in}{0.896128in}}%
\pgfpathlineto{\pgfqpoint{4.692953in}{0.897010in}}%
\pgfpathlineto{\pgfqpoint{4.687085in}{0.897893in}}%
\pgfpathlineto{\pgfqpoint{4.681218in}{0.898780in}}%
\pgfpathlineto{\pgfqpoint{4.675350in}{0.899668in}}%
\pgfpathlineto{\pgfqpoint{4.669483in}{0.900559in}}%
\pgfpathlineto{\pgfqpoint{4.663615in}{0.901453in}}%
\pgfpathlineto{\pgfqpoint{4.657747in}{0.902349in}}%
\pgfpathlineto{\pgfqpoint{4.651880in}{0.903247in}}%
\pgfpathlineto{\pgfqpoint{4.646012in}{0.904148in}}%
\pgfpathlineto{\pgfqpoint{4.640144in}{0.905052in}}%
\pgfpathlineto{\pgfqpoint{4.634277in}{0.905958in}}%
\pgfpathlineto{\pgfqpoint{4.628409in}{0.906866in}}%
\pgfpathlineto{\pgfqpoint{4.622542in}{0.907777in}}%
\pgfpathlineto{\pgfqpoint{4.616674in}{0.908691in}}%
\pgfpathlineto{\pgfqpoint{4.610806in}{0.909607in}}%
\pgfpathlineto{\pgfqpoint{4.604939in}{0.910526in}}%
\pgfpathlineto{\pgfqpoint{4.599071in}{0.911447in}}%
\pgfpathlineto{\pgfqpoint{4.593204in}{0.912371in}}%
\pgfpathlineto{\pgfqpoint{4.587336in}{0.913297in}}%
\pgfpathlineto{\pgfqpoint{4.581468in}{0.914226in}}%
\pgfpathlineto{\pgfqpoint{4.575601in}{0.915158in}}%
\pgfpathlineto{\pgfqpoint{4.569733in}{0.916092in}}%
\pgfpathlineto{\pgfqpoint{4.563866in}{0.917029in}}%
\pgfpathlineto{\pgfqpoint{4.557998in}{0.917968in}}%
\pgfpathlineto{\pgfqpoint{4.552130in}{0.918910in}}%
\pgfpathlineto{\pgfqpoint{4.546263in}{0.919855in}}%
\pgfpathlineto{\pgfqpoint{4.540395in}{0.920803in}}%
\pgfpathlineto{\pgfqpoint{4.534528in}{0.921753in}}%
\pgfpathlineto{\pgfqpoint{4.528660in}{0.922706in}}%
\pgfpathlineto{\pgfqpoint{4.522792in}{0.923661in}}%
\pgfpathlineto{\pgfqpoint{4.516925in}{0.924620in}}%
\pgfpathlineto{\pgfqpoint{4.511057in}{0.925581in}}%
\pgfpathlineto{\pgfqpoint{4.505190in}{0.926545in}}%
\pgfpathlineto{\pgfqpoint{4.499322in}{0.927511in}}%
\pgfpathlineto{\pgfqpoint{4.493454in}{0.928481in}}%
\pgfpathlineto{\pgfqpoint{4.487587in}{0.929453in}}%
\pgfpathlineto{\pgfqpoint{4.481719in}{0.930428in}}%
\pgfpathlineto{\pgfqpoint{4.475852in}{0.931406in}}%
\pgfpathlineto{\pgfqpoint{4.469984in}{0.932386in}}%
\pgfpathlineto{\pgfqpoint{4.464116in}{0.933370in}}%
\pgfpathlineto{\pgfqpoint{4.458249in}{0.934356in}}%
\pgfpathlineto{\pgfqpoint{4.452381in}{0.935345in}}%
\pgfpathlineto{\pgfqpoint{4.446514in}{0.936337in}}%
\pgfpathlineto{\pgfqpoint{4.440646in}{0.937332in}}%
\pgfpathlineto{\pgfqpoint{4.434778in}{0.938330in}}%
\pgfpathlineto{\pgfqpoint{4.428911in}{0.939330in}}%
\pgfpathlineto{\pgfqpoint{4.423043in}{0.940334in}}%
\pgfpathlineto{\pgfqpoint{4.417176in}{0.941340in}}%
\pgfpathlineto{\pgfqpoint{4.411308in}{0.942350in}}%
\pgfpathlineto{\pgfqpoint{4.405440in}{0.943362in}}%
\pgfpathlineto{\pgfqpoint{4.399573in}{0.944378in}}%
\pgfpathlineto{\pgfqpoint{4.393705in}{0.945396in}}%
\pgfpathlineto{\pgfqpoint{4.387837in}{0.946418in}}%
\pgfpathlineto{\pgfqpoint{4.381970in}{0.947442in}}%
\pgfpathlineto{\pgfqpoint{4.376102in}{0.948469in}}%
\pgfpathlineto{\pgfqpoint{4.370235in}{0.949500in}}%
\pgfpathlineto{\pgfqpoint{4.364367in}{0.950533in}}%
\pgfpathlineto{\pgfqpoint{4.358499in}{0.951570in}}%
\pgfpathlineto{\pgfqpoint{4.352632in}{0.952610in}}%
\pgfpathlineto{\pgfqpoint{4.346764in}{0.953653in}}%
\pgfpathlineto{\pgfqpoint{4.340897in}{0.954698in}}%
\pgfpathlineto{\pgfqpoint{4.335029in}{0.955747in}}%
\pgfpathlineto{\pgfqpoint{4.329161in}{0.956800in}}%
\pgfpathlineto{\pgfqpoint{4.323294in}{0.957855in}}%
\pgfpathlineto{\pgfqpoint{4.317426in}{0.958913in}}%
\pgfpathlineto{\pgfqpoint{4.311559in}{0.959975in}}%
\pgfpathlineto{\pgfqpoint{4.305691in}{0.961040in}}%
\pgfpathlineto{\pgfqpoint{4.299823in}{0.962108in}}%
\pgfpathlineto{\pgfqpoint{4.293956in}{0.963179in}}%
\pgfpathlineto{\pgfqpoint{4.288088in}{0.964253in}}%
\pgfpathlineto{\pgfqpoint{4.282221in}{0.965331in}}%
\pgfpathlineto{\pgfqpoint{4.276353in}{0.966412in}}%
\pgfpathlineto{\pgfqpoint{4.270485in}{0.967497in}}%
\pgfpathlineto{\pgfqpoint{4.264618in}{0.968584in}}%
\pgfpathlineto{\pgfqpoint{4.258750in}{0.969675in}}%
\pgfpathlineto{\pgfqpoint{4.252883in}{0.970769in}}%
\pgfpathlineto{\pgfqpoint{4.247015in}{0.971867in}}%
\pgfpathlineto{\pgfqpoint{4.241147in}{0.972968in}}%
\pgfpathlineto{\pgfqpoint{4.235280in}{0.974072in}}%
\pgfpathlineto{\pgfqpoint{4.229412in}{0.975180in}}%
\pgfpathlineto{\pgfqpoint{4.223545in}{0.976291in}}%
\pgfpathlineto{\pgfqpoint{4.217677in}{0.977405in}}%
\pgfpathlineto{\pgfqpoint{4.211809in}{0.978523in}}%
\pgfpathlineto{\pgfqpoint{4.205942in}{0.979644in}}%
\pgfpathlineto{\pgfqpoint{4.200074in}{0.980769in}}%
\pgfpathlineto{\pgfqpoint{4.194207in}{0.981898in}}%
\pgfpathlineto{\pgfqpoint{4.188339in}{0.983029in}}%
\pgfpathlineto{\pgfqpoint{4.182471in}{0.984165in}}%
\pgfpathlineto{\pgfqpoint{4.176604in}{0.985304in}}%
\pgfpathlineto{\pgfqpoint{4.170736in}{0.986446in}}%
\pgfpathlineto{\pgfqpoint{4.164869in}{0.987592in}}%
\pgfpathlineto{\pgfqpoint{4.159001in}{0.988741in}}%
\pgfpathlineto{\pgfqpoint{4.153133in}{0.989895in}}%
\pgfpathlineto{\pgfqpoint{4.147266in}{0.991051in}}%
\pgfpathlineto{\pgfqpoint{4.141398in}{0.992212in}}%
\pgfpathlineto{\pgfqpoint{4.135530in}{0.993376in}}%
\pgfpathlineto{\pgfqpoint{4.129663in}{0.994543in}}%
\pgfpathlineto{\pgfqpoint{4.123795in}{0.995715in}}%
\pgfpathlineto{\pgfqpoint{4.117928in}{0.996890in}}%
\pgfpathlineto{\pgfqpoint{4.112060in}{0.998069in}}%
\pgfpathlineto{\pgfqpoint{4.106192in}{0.999251in}}%
\pgfpathlineto{\pgfqpoint{4.100325in}{1.000438in}}%
\pgfpathlineto{\pgfqpoint{4.094457in}{1.001628in}}%
\pgfpathlineto{\pgfqpoint{4.088590in}{1.002822in}}%
\pgfpathlineto{\pgfqpoint{4.082722in}{1.004019in}}%
\pgfpathlineto{\pgfqpoint{4.076854in}{1.005221in}}%
\pgfpathlineto{\pgfqpoint{4.070987in}{1.006426in}}%
\pgfpathlineto{\pgfqpoint{4.065119in}{1.007635in}}%
\pgfpathlineto{\pgfqpoint{4.059252in}{1.008848in}}%
\pgfpathlineto{\pgfqpoint{4.053384in}{1.010065in}}%
\pgfpathlineto{\pgfqpoint{4.047516in}{1.011286in}}%
\pgfpathlineto{\pgfqpoint{4.041649in}{1.012511in}}%
\pgfpathlineto{\pgfqpoint{4.035781in}{1.013740in}}%
\pgfpathlineto{\pgfqpoint{4.029914in}{1.014973in}}%
\pgfpathlineto{\pgfqpoint{4.024046in}{1.016209in}}%
\pgfpathlineto{\pgfqpoint{4.018178in}{1.017450in}}%
\pgfpathlineto{\pgfqpoint{4.012311in}{1.018695in}}%
\pgfpathlineto{\pgfqpoint{4.006443in}{1.019944in}}%
\pgfpathlineto{\pgfqpoint{4.000576in}{1.021197in}}%
\pgfpathlineto{\pgfqpoint{3.994708in}{1.022454in}}%
\pgfpathlineto{\pgfqpoint{3.988840in}{1.023715in}}%
\pgfpathlineto{\pgfqpoint{3.982973in}{1.024980in}}%
\pgfpathlineto{\pgfqpoint{3.977105in}{1.026250in}}%
\pgfpathlineto{\pgfqpoint{3.971238in}{1.027523in}}%
\pgfpathlineto{\pgfqpoint{3.965370in}{1.028801in}}%
\pgfpathlineto{\pgfqpoint{3.959502in}{1.030083in}}%
\pgfpathlineto{\pgfqpoint{3.953635in}{1.031369in}}%
\pgfpathlineto{\pgfqpoint{3.947767in}{1.032660in}}%
\pgfpathlineto{\pgfqpoint{3.941900in}{1.033955in}}%
\pgfpathlineto{\pgfqpoint{3.936032in}{1.035254in}}%
\pgfpathlineto{\pgfqpoint{3.930164in}{1.036557in}}%
\pgfpathlineto{\pgfqpoint{3.924297in}{1.037865in}}%
\pgfpathlineto{\pgfqpoint{3.918429in}{1.039177in}}%
\pgfpathlineto{\pgfqpoint{3.912562in}{1.040493in}}%
\pgfpathlineto{\pgfqpoint{3.906694in}{1.041814in}}%
\pgfpathlineto{\pgfqpoint{3.900826in}{1.043140in}}%
\pgfpathlineto{\pgfqpoint{3.894959in}{1.044470in}}%
\pgfpathlineto{\pgfqpoint{3.889091in}{1.045804in}}%
\pgfpathlineto{\pgfqpoint{3.883224in}{1.047143in}}%
\pgfpathlineto{\pgfqpoint{3.877356in}{1.048486in}}%
\pgfpathlineto{\pgfqpoint{3.871488in}{1.049834in}}%
\pgfpathlineto{\pgfqpoint{3.865621in}{1.051186in}}%
\pgfpathlineto{\pgfqpoint{3.859753in}{1.052543in}}%
\pgfpathlineto{\pgfqpoint{3.853885in}{1.053905in}}%
\pgfpathlineto{\pgfqpoint{3.848018in}{1.055271in}}%
\pgfpathlineto{\pgfqpoint{3.842150in}{1.056642in}}%
\pgfpathlineto{\pgfqpoint{3.836283in}{1.058018in}}%
\pgfpathlineto{\pgfqpoint{3.830415in}{1.059398in}}%
\pgfpathlineto{\pgfqpoint{3.824547in}{1.060783in}}%
\pgfpathlineto{\pgfqpoint{3.818680in}{1.062173in}}%
\pgfpathlineto{\pgfqpoint{3.812812in}{1.063567in}}%
\pgfpathlineto{\pgfqpoint{3.806945in}{1.064967in}}%
\pgfpathlineto{\pgfqpoint{3.801077in}{1.066371in}}%
\pgfpathlineto{\pgfqpoint{3.795209in}{1.067780in}}%
\pgfpathlineto{\pgfqpoint{3.789342in}{1.069194in}}%
\pgfpathlineto{\pgfqpoint{3.783474in}{1.070613in}}%
\pgfpathlineto{\pgfqpoint{3.777607in}{1.072037in}}%
\pgfpathlineto{\pgfqpoint{3.771739in}{1.073466in}}%
\pgfpathlineto{\pgfqpoint{3.765871in}{1.074899in}}%
\pgfpathlineto{\pgfqpoint{3.760004in}{1.076338in}}%
\pgfpathlineto{\pgfqpoint{3.754136in}{1.077782in}}%
\pgfpathlineto{\pgfqpoint{3.748269in}{1.079231in}}%
\pgfpathlineto{\pgfqpoint{3.742401in}{1.080685in}}%
\pgfpathlineto{\pgfqpoint{3.736533in}{1.082144in}}%
\pgfpathlineto{\pgfqpoint{3.730666in}{1.083608in}}%
\pgfpathlineto{\pgfqpoint{3.724798in}{1.085077in}}%
\pgfpathlineto{\pgfqpoint{3.718931in}{1.086552in}}%
\pgfpathlineto{\pgfqpoint{3.713063in}{1.088032in}}%
\pgfpathlineto{\pgfqpoint{3.707195in}{1.089517in}}%
\pgfpathlineto{\pgfqpoint{3.701328in}{1.091007in}}%
\pgfpathlineto{\pgfqpoint{3.695460in}{1.092503in}}%
\pgfpathlineto{\pgfqpoint{3.689593in}{1.094004in}}%
\pgfpathlineto{\pgfqpoint{3.683725in}{1.095511in}}%
\pgfpathlineto{\pgfqpoint{3.677857in}{1.097022in}}%
\pgfpathlineto{\pgfqpoint{3.671990in}{1.098540in}}%
\pgfpathlineto{\pgfqpoint{3.666122in}{1.100062in}}%
\pgfpathlineto{\pgfqpoint{3.660255in}{1.101590in}}%
\pgfpathlineto{\pgfqpoint{3.654387in}{1.103124in}}%
\pgfpathlineto{\pgfqpoint{3.648519in}{1.104663in}}%
\pgfpathlineto{\pgfqpoint{3.642652in}{1.106208in}}%
\pgfpathlineto{\pgfqpoint{3.636784in}{1.107758in}}%
\pgfpathlineto{\pgfqpoint{3.630917in}{1.109315in}}%
\pgfpathlineto{\pgfqpoint{3.625049in}{1.110876in}}%
\pgfpathlineto{\pgfqpoint{3.619181in}{1.112444in}}%
\pgfpathlineto{\pgfqpoint{3.613314in}{1.114017in}}%
\pgfpathlineto{\pgfqpoint{3.607446in}{1.115596in}}%
\pgfpathlineto{\pgfqpoint{3.601578in}{1.117180in}}%
\pgfpathlineto{\pgfqpoint{3.595711in}{1.118771in}}%
\pgfpathlineto{\pgfqpoint{3.589843in}{1.120367in}}%
\pgfpathlineto{\pgfqpoint{3.583976in}{1.121970in}}%
\pgfpathlineto{\pgfqpoint{3.578108in}{1.123578in}}%
\pgfpathlineto{\pgfqpoint{3.572240in}{1.125192in}}%
\pgfpathlineto{\pgfqpoint{3.566373in}{1.126812in}}%
\pgfpathlineto{\pgfqpoint{3.560505in}{1.128438in}}%
\pgfpathlineto{\pgfqpoint{3.554638in}{1.130070in}}%
\pgfpathlineto{\pgfqpoint{3.548770in}{1.131709in}}%
\pgfpathlineto{\pgfqpoint{3.542902in}{1.133353in}}%
\pgfpathlineto{\pgfqpoint{3.537035in}{1.135004in}}%
\pgfpathlineto{\pgfqpoint{3.531167in}{1.136660in}}%
\pgfpathlineto{\pgfqpoint{3.525300in}{1.138323in}}%
\pgfpathlineto{\pgfqpoint{3.519432in}{1.139993in}}%
\pgfpathlineto{\pgfqpoint{3.513564in}{1.141668in}}%
\pgfpathlineto{\pgfqpoint{3.507697in}{1.143350in}}%
\pgfpathlineto{\pgfqpoint{3.501829in}{1.145038in}}%
\pgfpathlineto{\pgfqpoint{3.495962in}{1.146733in}}%
\pgfpathlineto{\pgfqpoint{3.490094in}{1.148434in}}%
\pgfpathlineto{\pgfqpoint{3.484226in}{1.150142in}}%
\pgfpathlineto{\pgfqpoint{3.478359in}{1.151856in}}%
\pgfpathlineto{\pgfqpoint{3.472491in}{1.153576in}}%
\pgfpathlineto{\pgfqpoint{3.466624in}{1.155304in}}%
\pgfpathlineto{\pgfqpoint{3.460756in}{1.157038in}}%
\pgfpathlineto{\pgfqpoint{3.454888in}{1.158778in}}%
\pgfpathlineto{\pgfqpoint{3.449021in}{1.160525in}}%
\pgfpathlineto{\pgfqpoint{3.443153in}{1.162279in}}%
\pgfpathlineto{\pgfqpoint{3.437286in}{1.164040in}}%
\pgfpathlineto{\pgfqpoint{3.431418in}{1.165808in}}%
\pgfpathlineto{\pgfqpoint{3.425550in}{1.167582in}}%
\pgfpathlineto{\pgfqpoint{3.419683in}{1.169364in}}%
\pgfpathlineto{\pgfqpoint{3.413815in}{1.171152in}}%
\pgfpathlineto{\pgfqpoint{3.407948in}{1.172947in}}%
\pgfpathlineto{\pgfqpoint{3.402080in}{1.174749in}}%
\pgfpathlineto{\pgfqpoint{3.396212in}{1.176559in}}%
\pgfpathlineto{\pgfqpoint{3.390345in}{1.178375in}}%
\pgfpathlineto{\pgfqpoint{3.384477in}{1.180199in}}%
\pgfpathlineto{\pgfqpoint{3.378610in}{1.182030in}}%
\pgfpathlineto{\pgfqpoint{3.372742in}{1.183868in}}%
\pgfpathlineto{\pgfqpoint{3.366874in}{1.185713in}}%
\pgfpathlineto{\pgfqpoint{3.361007in}{1.187566in}}%
\pgfpathlineto{\pgfqpoint{3.355139in}{1.189426in}}%
\pgfpathlineto{\pgfqpoint{3.349271in}{1.191293in}}%
\pgfpathlineto{\pgfqpoint{3.343404in}{1.193168in}}%
\pgfpathlineto{\pgfqpoint{3.337536in}{1.195051in}}%
\pgfpathlineto{\pgfqpoint{3.331669in}{1.196940in}}%
\pgfpathlineto{\pgfqpoint{3.325801in}{1.198838in}}%
\pgfpathlineto{\pgfqpoint{3.319933in}{1.200743in}}%
\pgfpathlineto{\pgfqpoint{3.314066in}{1.202656in}}%
\pgfpathlineto{\pgfqpoint{3.308198in}{1.204576in}}%
\pgfpathlineto{\pgfqpoint{3.302331in}{1.206505in}}%
\pgfpathlineto{\pgfqpoint{3.296463in}{1.208441in}}%
\pgfpathlineto{\pgfqpoint{3.290595in}{1.210385in}}%
\pgfpathlineto{\pgfqpoint{3.284728in}{1.212337in}}%
\pgfpathlineto{\pgfqpoint{3.278860in}{1.214297in}}%
\pgfpathlineto{\pgfqpoint{3.272993in}{1.216265in}}%
\pgfpathlineto{\pgfqpoint{3.267125in}{1.218241in}}%
\pgfpathlineto{\pgfqpoint{3.261257in}{1.220225in}}%
\pgfpathlineto{\pgfqpoint{3.255390in}{1.222217in}}%
\pgfpathlineto{\pgfqpoint{3.249522in}{1.224217in}}%
\pgfpathlineto{\pgfqpoint{3.243655in}{1.226226in}}%
\pgfpathlineto{\pgfqpoint{3.237787in}{1.228243in}}%
\pgfpathlineto{\pgfqpoint{3.231919in}{1.230269in}}%
\pgfpathlineto{\pgfqpoint{3.226052in}{1.232303in}}%
\pgfpathlineto{\pgfqpoint{3.220184in}{1.234345in}}%
\pgfpathlineto{\pgfqpoint{3.214317in}{1.236396in}}%
\pgfpathlineto{\pgfqpoint{3.208449in}{1.238455in}}%
\pgfpathlineto{\pgfqpoint{3.202581in}{1.240523in}}%
\pgfpathlineto{\pgfqpoint{3.196714in}{1.242600in}}%
\pgfpathlineto{\pgfqpoint{3.190846in}{1.244685in}}%
\pgfpathlineto{\pgfqpoint{3.184979in}{1.246780in}}%
\pgfpathlineto{\pgfqpoint{3.179111in}{1.248883in}}%
\pgfpathlineto{\pgfqpoint{3.173243in}{1.250995in}}%
\pgfpathlineto{\pgfqpoint{3.167376in}{1.253116in}}%
\pgfpathlineto{\pgfqpoint{3.161508in}{1.255246in}}%
\pgfpathlineto{\pgfqpoint{3.155641in}{1.257385in}}%
\pgfpathlineto{\pgfqpoint{3.149773in}{1.259533in}}%
\pgfpathlineto{\pgfqpoint{3.143905in}{1.261690in}}%
\pgfpathlineto{\pgfqpoint{3.138038in}{1.263857in}}%
\pgfpathlineto{\pgfqpoint{3.132170in}{1.266033in}}%
\pgfpathlineto{\pgfqpoint{3.126303in}{1.268219in}}%
\pgfpathlineto{\pgfqpoint{3.120435in}{1.270413in}}%
\pgfpathlineto{\pgfqpoint{3.114567in}{1.272618in}}%
\pgfpathlineto{\pgfqpoint{3.108700in}{1.274832in}}%
\pgfpathlineto{\pgfqpoint{3.102832in}{1.277055in}}%
\pgfpathlineto{\pgfqpoint{3.096965in}{1.279288in}}%
\pgfpathlineto{\pgfqpoint{3.091097in}{1.281531in}}%
\pgfpathlineto{\pgfqpoint{3.085229in}{1.283784in}}%
\pgfpathlineto{\pgfqpoint{3.079362in}{1.286047in}}%
\pgfpathlineto{\pgfqpoint{3.073494in}{1.288319in}}%
\pgfpathlineto{\pgfqpoint{3.067626in}{1.290602in}}%
\pgfpathlineto{\pgfqpoint{3.061759in}{1.292894in}}%
\pgfpathlineto{\pgfqpoint{3.055891in}{1.295197in}}%
\pgfpathlineto{\pgfqpoint{3.050024in}{1.297510in}}%
\pgfpathlineto{\pgfqpoint{3.044156in}{1.299833in}}%
\pgfpathlineto{\pgfqpoint{3.038288in}{1.302167in}}%
\pgfpathlineto{\pgfqpoint{3.032421in}{1.304511in}}%
\pgfpathlineto{\pgfqpoint{3.026553in}{1.306866in}}%
\pgfpathlineto{\pgfqpoint{3.020686in}{1.309231in}}%
\pgfpathlineto{\pgfqpoint{3.014818in}{1.311606in}}%
\pgfpathlineto{\pgfqpoint{3.008950in}{1.313993in}}%
\pgfpathlineto{\pgfqpoint{3.003083in}{1.316390in}}%
\pgfpathlineto{\pgfqpoint{2.997215in}{1.318798in}}%
\pgfpathlineto{\pgfqpoint{2.991348in}{1.321217in}}%
\pgfpathlineto{\pgfqpoint{2.985480in}{1.323647in}}%
\pgfpathlineto{\pgfqpoint{2.979612in}{1.326088in}}%
\pgfpathlineto{\pgfqpoint{2.973745in}{1.328540in}}%
\pgfpathlineto{\pgfqpoint{2.967877in}{1.331003in}}%
\pgfpathlineto{\pgfqpoint{2.962010in}{1.333478in}}%
\pgfpathlineto{\pgfqpoint{2.956142in}{1.335964in}}%
\pgfpathlineto{\pgfqpoint{2.950274in}{1.338462in}}%
\pgfpathlineto{\pgfqpoint{2.944407in}{1.340971in}}%
\pgfpathlineto{\pgfqpoint{2.938539in}{1.343491in}}%
\pgfpathlineto{\pgfqpoint{2.932672in}{1.346023in}}%
\pgfpathlineto{\pgfqpoint{2.926804in}{1.348568in}}%
\pgfpathlineto{\pgfqpoint{2.920936in}{1.351124in}}%
\pgfpathlineto{\pgfqpoint{2.915069in}{1.353691in}}%
\pgfpathlineto{\pgfqpoint{2.909201in}{1.356271in}}%
\pgfpathlineto{\pgfqpoint{2.903334in}{1.358863in}}%
\pgfpathlineto{\pgfqpoint{2.897466in}{1.361468in}}%
\pgfpathlineto{\pgfqpoint{2.891598in}{1.364084in}}%
\pgfpathlineto{\pgfqpoint{2.885731in}{1.366713in}}%
\pgfpathlineto{\pgfqpoint{2.879863in}{1.369354in}}%
\pgfpathlineto{\pgfqpoint{2.873996in}{1.372008in}}%
\pgfpathlineto{\pgfqpoint{2.868128in}{1.374675in}}%
\pgfpathlineto{\pgfqpoint{2.862260in}{1.377354in}}%
\pgfpathlineto{\pgfqpoint{2.856393in}{1.380046in}}%
\pgfpathlineto{\pgfqpoint{2.850525in}{1.382751in}}%
\pgfpathlineto{\pgfqpoint{2.844658in}{1.385469in}}%
\pgfpathlineto{\pgfqpoint{2.838790in}{1.388200in}}%
\pgfpathlineto{\pgfqpoint{2.832922in}{1.390944in}}%
\pgfpathlineto{\pgfqpoint{2.827055in}{1.393702in}}%
\pgfpathlineto{\pgfqpoint{2.821187in}{1.396472in}}%
\pgfpathlineto{\pgfqpoint{2.815319in}{1.399257in}}%
\pgfpathlineto{\pgfqpoint{2.809452in}{1.402055in}}%
\pgfpathlineto{\pgfqpoint{2.803584in}{1.404866in}}%
\pgfpathlineto{\pgfqpoint{2.797717in}{1.407692in}}%
\pgfpathlineto{\pgfqpoint{2.791849in}{1.410531in}}%
\pgfpathlineto{\pgfqpoint{2.785981in}{1.413384in}}%
\pgfpathlineto{\pgfqpoint{2.780114in}{1.416252in}}%
\pgfpathlineto{\pgfqpoint{2.774246in}{1.419133in}}%
\pgfpathlineto{\pgfqpoint{2.768379in}{1.422029in}}%
\pgfpathlineto{\pgfqpoint{2.762511in}{1.424939in}}%
\pgfpathlineto{\pgfqpoint{2.756643in}{1.427864in}}%
\pgfpathlineto{\pgfqpoint{2.750776in}{1.430803in}}%
\pgfpathlineto{\pgfqpoint{2.744908in}{1.433758in}}%
\pgfpathlineto{\pgfqpoint{2.739041in}{1.436727in}}%
\pgfpathlineto{\pgfqpoint{2.733173in}{1.439710in}}%
\pgfpathlineto{\pgfqpoint{2.727305in}{1.442709in}}%
\pgfpathlineto{\pgfqpoint{2.721438in}{1.445724in}}%
\pgfpathlineto{\pgfqpoint{2.715570in}{1.448753in}}%
\pgfpathlineto{\pgfqpoint{2.709703in}{1.451798in}}%
\pgfpathlineto{\pgfqpoint{2.703835in}{1.454858in}}%
\pgfpathlineto{\pgfqpoint{2.697967in}{1.457934in}}%
\pgfpathlineto{\pgfqpoint{2.692100in}{1.461026in}}%
\pgfpathlineto{\pgfqpoint{2.686232in}{1.464134in}}%
\pgfpathlineto{\pgfqpoint{2.680365in}{1.467257in}}%
\pgfpathlineto{\pgfqpoint{2.674497in}{1.470397in}}%
\pgfpathlineto{\pgfqpoint{2.668629in}{1.473553in}}%
\pgfpathlineto{\pgfqpoint{2.662762in}{1.476726in}}%
\pgfpathlineto{\pgfqpoint{2.656894in}{1.479915in}}%
\pgfpathlineto{\pgfqpoint{2.651027in}{1.483120in}}%
\pgfpathlineto{\pgfqpoint{2.645159in}{1.486343in}}%
\pgfpathlineto{\pgfqpoint{2.639291in}{1.489582in}}%
\pgfpathlineto{\pgfqpoint{2.633424in}{1.492839in}}%
\pgfpathlineto{\pgfqpoint{2.627556in}{1.496112in}}%
\pgfpathlineto{\pgfqpoint{2.621689in}{1.499403in}}%
\pgfpathlineto{\pgfqpoint{2.615821in}{1.502711in}}%
\pgfpathlineto{\pgfqpoint{2.609953in}{1.506037in}}%
\pgfpathlineto{\pgfqpoint{2.604086in}{1.509381in}}%
\pgfpathlineto{\pgfqpoint{2.598218in}{1.512743in}}%
\pgfpathlineto{\pgfqpoint{2.592351in}{1.516122in}}%
\pgfpathlineto{\pgfqpoint{2.586483in}{1.519520in}}%
\pgfpathlineto{\pgfqpoint{2.580615in}{1.522936in}}%
\pgfpathlineto{\pgfqpoint{2.574748in}{1.526370in}}%
\pgfpathlineto{\pgfqpoint{2.568880in}{1.529823in}}%
\pgfpathlineto{\pgfqpoint{2.563012in}{1.533295in}}%
\pgfpathlineto{\pgfqpoint{2.557145in}{1.536786in}}%
\pgfpathlineto{\pgfqpoint{2.551277in}{1.540296in}}%
\pgfpathlineto{\pgfqpoint{2.545410in}{1.543825in}}%
\pgfpathlineto{\pgfqpoint{2.539542in}{1.547373in}}%
\pgfpathlineto{\pgfqpoint{2.533674in}{1.550941in}}%
\pgfpathlineto{\pgfqpoint{2.527807in}{1.554529in}}%
\pgfpathlineto{\pgfqpoint{2.521939in}{1.558137in}}%
\pgfpathlineto{\pgfqpoint{2.516072in}{1.561764in}}%
\pgfpathlineto{\pgfqpoint{2.510204in}{1.565412in}}%
\pgfpathlineto{\pgfqpoint{2.504336in}{1.569080in}}%
\pgfpathlineto{\pgfqpoint{2.498469in}{1.572769in}}%
\pgfpathlineto{\pgfqpoint{2.492601in}{1.576478in}}%
\pgfpathlineto{\pgfqpoint{2.486734in}{1.580209in}}%
\pgfpathlineto{\pgfqpoint{2.480866in}{1.583960in}}%
\pgfpathlineto{\pgfqpoint{2.474998in}{1.587733in}}%
\pgfpathlineto{\pgfqpoint{2.469131in}{1.591527in}}%
\pgfpathlineto{\pgfqpoint{2.463263in}{1.595342in}}%
\pgfpathlineto{\pgfqpoint{2.457396in}{1.599180in}}%
\pgfpathlineto{\pgfqpoint{2.451528in}{1.603039in}}%
\pgfpathlineto{\pgfqpoint{2.445660in}{1.606921in}}%
\pgfpathlineto{\pgfqpoint{2.439793in}{1.610825in}}%
\pgfpathlineto{\pgfqpoint{2.433925in}{1.614751in}}%
\pgfpathlineto{\pgfqpoint{2.428058in}{1.618701in}}%
\pgfpathlineto{\pgfqpoint{2.422190in}{1.622673in}}%
\pgfpathlineto{\pgfqpoint{2.416322in}{1.626668in}}%
\pgfpathlineto{\pgfqpoint{2.410455in}{1.630687in}}%
\pgfpathlineto{\pgfqpoint{2.404587in}{1.634729in}}%
\pgfpathlineto{\pgfqpoint{2.398720in}{1.638795in}}%
\pgfpathlineto{\pgfqpoint{2.392852in}{1.642885in}}%
\pgfpathlineto{\pgfqpoint{2.386984in}{1.647000in}}%
\pgfpathlineto{\pgfqpoint{2.381117in}{1.651138in}}%
\pgfpathlineto{\pgfqpoint{2.375249in}{1.655302in}}%
\pgfpathlineto{\pgfqpoint{2.369382in}{1.659490in}}%
\pgfpathlineto{\pgfqpoint{2.363514in}{1.663703in}}%
\pgfpathlineto{\pgfqpoint{2.357646in}{1.667941in}}%
\pgfpathlineto{\pgfqpoint{2.351779in}{1.672205in}}%
\pgfpathlineto{\pgfqpoint{2.345911in}{1.676495in}}%
\pgfpathlineto{\pgfqpoint{2.340044in}{1.680811in}}%
\pgfpathlineto{\pgfqpoint{2.334176in}{1.685153in}}%
\pgfpathlineto{\pgfqpoint{2.328308in}{1.689521in}}%
\pgfpathlineto{\pgfqpoint{2.322441in}{1.693917in}}%
\pgfpathlineto{\pgfqpoint{2.316573in}{1.698339in}}%
\pgfpathlineto{\pgfqpoint{2.310706in}{1.702788in}}%
\pgfpathlineto{\pgfqpoint{2.304838in}{1.707265in}}%
\pgfpathlineto{\pgfqpoint{2.298970in}{1.711769in}}%
\pgfpathlineto{\pgfqpoint{2.293103in}{1.716302in}}%
\pgfpathlineto{\pgfqpoint{2.287235in}{1.720862in}}%
\pgfpathlineto{\pgfqpoint{2.281367in}{1.725452in}}%
\pgfpathlineto{\pgfqpoint{2.275500in}{1.730069in}}%
\pgfpathlineto{\pgfqpoint{2.269632in}{1.734716in}}%
\pgfpathlineto{\pgfqpoint{2.263765in}{1.739393in}}%
\pgfpathlineto{\pgfqpoint{2.257897in}{1.744099in}}%
\pgfpathlineto{\pgfqpoint{2.252029in}{1.748834in}}%
\pgfpathlineto{\pgfqpoint{2.246162in}{1.753600in}}%
\pgfpathlineto{\pgfqpoint{2.240294in}{1.758396in}}%
\pgfpathlineto{\pgfqpoint{2.234427in}{1.763223in}}%
\pgfpathlineto{\pgfqpoint{2.228559in}{1.768082in}}%
\pgfpathlineto{\pgfqpoint{2.222691in}{1.772971in}}%
\pgfpathlineto{\pgfqpoint{2.216824in}{1.777892in}}%
\pgfpathlineto{\pgfqpoint{2.210956in}{1.782845in}}%
\pgfpathlineto{\pgfqpoint{2.205089in}{1.787830in}}%
\pgfpathlineto{\pgfqpoint{2.199221in}{1.792848in}}%
\pgfpathlineto{\pgfqpoint{2.193353in}{1.797899in}}%
\pgfpathlineto{\pgfqpoint{2.187486in}{1.802983in}}%
\pgfpathlineto{\pgfqpoint{2.181618in}{1.808100in}}%
\pgfpathlineto{\pgfqpoint{2.175751in}{1.813251in}}%
\pgfpathlineto{\pgfqpoint{2.169883in}{1.818437in}}%
\pgfpathlineto{\pgfqpoint{2.164015in}{1.823657in}}%
\pgfpathlineto{\pgfqpoint{2.158148in}{1.828912in}}%
\pgfpathlineto{\pgfqpoint{2.152280in}{1.834202in}}%
\pgfpathlineto{\pgfqpoint{2.146413in}{1.839528in}}%
\pgfpathlineto{\pgfqpoint{2.140545in}{1.844890in}}%
\pgfpathlineto{\pgfqpoint{2.134677in}{1.850288in}}%
\pgfpathlineto{\pgfqpoint{2.128810in}{1.855723in}}%
\pgfpathlineto{\pgfqpoint{2.122942in}{1.861195in}}%
\pgfpathlineto{\pgfqpoint{2.117075in}{1.866704in}}%
\pgfpathlineto{\pgfqpoint{2.111207in}{1.872251in}}%
\pgfpathlineto{\pgfqpoint{2.105339in}{1.877837in}}%
\pgfpathlineto{\pgfqpoint{2.099472in}{1.883461in}}%
\pgfpathlineto{\pgfqpoint{2.093604in}{1.889124in}}%
\pgfpathlineto{\pgfqpoint{2.087737in}{1.894827in}}%
\pgfpathlineto{\pgfqpoint{2.081869in}{1.900569in}}%
\pgfpathlineto{\pgfqpoint{2.076001in}{1.906352in}}%
\pgfpathlineto{\pgfqpoint{2.070134in}{1.912175in}}%
\pgfpathlineto{\pgfqpoint{2.064266in}{1.918040in}}%
\pgfpathlineto{\pgfqpoint{2.058399in}{1.923946in}}%
\pgfpathlineto{\pgfqpoint{2.052531in}{1.929894in}}%
\pgfpathlineto{\pgfqpoint{2.046663in}{1.935885in}}%
\pgfpathlineto{\pgfqpoint{2.040796in}{1.941919in}}%
\pgfpathlineto{\pgfqpoint{2.034928in}{1.947996in}}%
\pgfpathlineto{\pgfqpoint{2.029060in}{1.954116in}}%
\pgfpathlineto{\pgfqpoint{2.023193in}{1.960282in}}%
\pgfpathlineto{\pgfqpoint{2.017325in}{1.966492in}}%
\pgfpathlineto{\pgfqpoint{2.011458in}{1.972747in}}%
\pgfpathlineto{\pgfqpoint{2.005590in}{1.979048in}}%
\pgfpathlineto{\pgfqpoint{1.999722in}{1.985395in}}%
\pgfpathlineto{\pgfqpoint{1.993855in}{1.991789in}}%
\pgfpathlineto{\pgfqpoint{1.987987in}{1.998231in}}%
\pgfpathlineto{\pgfqpoint{1.982120in}{2.004720in}}%
\pgfpathlineto{\pgfqpoint{1.976252in}{2.011258in}}%
\pgfpathlineto{\pgfqpoint{1.970384in}{2.017845in}}%
\pgfpathlineto{\pgfqpoint{1.964517in}{2.024481in}}%
\pgfpathlineto{\pgfqpoint{1.958649in}{2.031168in}}%
\pgfpathlineto{\pgfqpoint{1.952782in}{2.037905in}}%
\pgfpathlineto{\pgfqpoint{1.946914in}{2.044693in}}%
\pgfpathlineto{\pgfqpoint{1.941046in}{2.051533in}}%
\pgfpathlineto{\pgfqpoint{1.935179in}{2.058425in}}%
\pgfpathlineto{\pgfqpoint{1.929311in}{2.065370in}}%
\pgfpathlineto{\pgfqpoint{1.923444in}{2.072369in}}%
\pgfpathlineto{\pgfqpoint{1.917576in}{2.079423in}}%
\pgfpathlineto{\pgfqpoint{1.911708in}{2.086531in}}%
\pgfpathlineto{\pgfqpoint{1.905841in}{2.093694in}}%
\pgfpathlineto{\pgfqpoint{1.899973in}{2.100914in}}%
\pgfpathlineto{\pgfqpoint{1.894106in}{2.108190in}}%
\pgfpathlineto{\pgfqpoint{1.888238in}{2.115524in}}%
\pgfpathlineto{\pgfqpoint{1.882370in}{2.122917in}}%
\pgfpathlineto{\pgfqpoint{1.876503in}{2.130368in}}%
\pgfpathlineto{\pgfqpoint{1.870635in}{2.137878in}}%
\pgfpathlineto{\pgfqpoint{1.864768in}{2.145449in}}%
\pgfpathlineto{\pgfqpoint{1.858900in}{2.153081in}}%
\pgfpathlineto{\pgfqpoint{1.853032in}{2.160775in}}%
\pgfpathlineto{\pgfqpoint{1.847165in}{2.168531in}}%
\pgfpathlineto{\pgfqpoint{1.841297in}{2.176351in}}%
\pgfpathlineto{\pgfqpoint{1.835430in}{2.184234in}}%
\pgfpathlineto{\pgfqpoint{1.829562in}{2.192183in}}%
\pgfpathlineto{\pgfqpoint{1.823694in}{2.200197in}}%
\pgfpathlineto{\pgfqpoint{1.817827in}{2.208277in}}%
\pgfpathlineto{\pgfqpoint{1.811959in}{2.216425in}}%
\pgfpathlineto{\pgfqpoint{1.806092in}{2.224641in}}%
\pgfpathlineto{\pgfqpoint{1.800224in}{2.232926in}}%
\pgfpathlineto{\pgfqpoint{1.794356in}{2.241281in}}%
\pgfpathlineto{\pgfqpoint{1.788489in}{2.249707in}}%
\pgfpathlineto{\pgfqpoint{1.782621in}{2.258204in}}%
\pgfpathlineto{\pgfqpoint{1.776753in}{2.266774in}}%
\pgfpathlineto{\pgfqpoint{1.770886in}{2.275417in}}%
\pgfpathlineto{\pgfqpoint{1.765018in}{2.284135in}}%
\pgfpathlineto{\pgfqpoint{1.759151in}{2.292928in}}%
\pgfpathlineto{\pgfqpoint{1.753283in}{2.301798in}}%
\pgfpathlineto{\pgfqpoint{1.747415in}{2.310745in}}%
\pgfpathlineto{\pgfqpoint{1.741548in}{2.319771in}}%
\pgfpathlineto{\pgfqpoint{1.735680in}{2.328876in}}%
\pgfpathlineto{\pgfqpoint{1.729813in}{2.338062in}}%
\pgfpathlineto{\pgfqpoint{1.723945in}{2.347329in}}%
\pgfpathlineto{\pgfqpoint{1.718077in}{2.356679in}}%
\pgfpathlineto{\pgfqpoint{1.712210in}{2.366112in}}%
\pgfpathlineto{\pgfqpoint{1.706342in}{2.375631in}}%
\pgfpathlineto{\pgfqpoint{1.700475in}{2.385236in}}%
\pgfpathlineto{\pgfqpoint{1.694607in}{2.394928in}}%
\pgfpathlineto{\pgfqpoint{1.688739in}{2.404708in}}%
\pgfpathlineto{\pgfqpoint{1.682872in}{2.414578in}}%
\pgfpathlineto{\pgfqpoint{1.677004in}{2.424539in}}%
\pgfpathlineto{\pgfqpoint{1.671137in}{2.434593in}}%
\pgfpathlineto{\pgfqpoint{1.665269in}{2.444739in}}%
\pgfpathlineto{\pgfqpoint{1.659401in}{2.454981in}}%
\pgfpathlineto{\pgfqpoint{1.653534in}{2.465318in}}%
\pgfpathlineto{\pgfqpoint{1.647666in}{2.475753in}}%
\pgfpathlineto{\pgfqpoint{1.641799in}{2.486287in}}%
\pgfpathlineto{\pgfqpoint{1.635931in}{2.496921in}}%
\pgfpathlineto{\pgfqpoint{1.630063in}{2.507657in}}%
\pgfpathlineto{\pgfqpoint{1.624196in}{2.518496in}}%
\pgfpathlineto{\pgfqpoint{1.618328in}{2.529440in}}%
\pgfpathlineto{\pgfqpoint{1.612461in}{2.540490in}}%
\pgfpathlineto{\pgfqpoint{1.606593in}{2.551647in}}%
\pgfpathlineto{\pgfqpoint{1.600725in}{2.562914in}}%
\pgfpathlineto{\pgfqpoint{1.594858in}{2.574292in}}%
\pgfpathlineto{\pgfqpoint{1.588990in}{2.585783in}}%
\pgfpathlineto{\pgfqpoint{1.583123in}{2.597387in}}%
\pgfpathlineto{\pgfqpoint{1.577255in}{2.609108in}}%
\pgfpathlineto{\pgfqpoint{1.571387in}{2.620946in}}%
\pgfpathlineto{\pgfqpoint{1.565520in}{2.632904in}}%
\pgfpathlineto{\pgfqpoint{1.559652in}{2.644983in}}%
\pgfpathlineto{\pgfqpoint{1.553785in}{2.657186in}}%
\pgfpathlineto{\pgfqpoint{1.547917in}{2.669513in}}%
\pgfpathlineto{\pgfqpoint{1.542049in}{2.681967in}}%
\pgfpathlineto{\pgfqpoint{1.536182in}{2.694551in}}%
\pgfpathlineto{\pgfqpoint{1.530314in}{2.707265in}}%
\pgfpathlineto{\pgfqpoint{1.524447in}{2.720113in}}%
\pgfpathlineto{\pgfqpoint{1.518579in}{2.733095in}}%
\pgfpathlineto{\pgfqpoint{1.512711in}{2.746215in}}%
\pgfpathlineto{\pgfqpoint{1.506844in}{2.759474in}}%
\pgfpathlineto{\pgfqpoint{1.500976in}{2.772876in}}%
\pgfpathlineto{\pgfqpoint{1.495108in}{2.786421in}}%
\pgfpathlineto{\pgfqpoint{1.489241in}{2.800112in}}%
\pgfpathlineto{\pgfqpoint{1.483373in}{2.813952in}}%
\pgfpathlineto{\pgfqpoint{1.477506in}{2.827943in}}%
\pgfpathlineto{\pgfqpoint{1.471638in}{2.842088in}}%
\pgfpathlineto{\pgfqpoint{1.465770in}{2.856389in}}%
\pgfpathlineto{\pgfqpoint{1.459903in}{2.870849in}}%
\pgfpathlineto{\pgfqpoint{1.454035in}{2.885470in}}%
\pgfpathlineto{\pgfqpoint{1.448168in}{2.900256in}}%
\pgfpathlineto{\pgfqpoint{1.442300in}{2.915208in}}%
\pgfpathlineto{\pgfqpoint{1.436432in}{2.930330in}}%
\pgfpathlineto{\pgfqpoint{1.430565in}{2.945625in}}%
\pgfpathlineto{\pgfqpoint{1.424697in}{2.961096in}}%
\pgfpathlineto{\pgfqpoint{1.418830in}{2.976745in}}%
\pgfpathlineto{\pgfqpoint{1.412962in}{2.992576in}}%
\pgfpathlineto{\pgfqpoint{1.407094in}{3.008592in}}%
\pgfpathlineto{\pgfqpoint{1.401227in}{3.024797in}}%
\pgfpathlineto{\pgfqpoint{1.395359in}{3.041193in}}%
\pgfpathlineto{\pgfqpoint{1.389492in}{3.057784in}}%
\pgfpathlineto{\pgfqpoint{1.383624in}{3.074574in}}%
\pgfpathlineto{\pgfqpoint{1.377756in}{3.091565in}}%
\pgfpathlineto{\pgfqpoint{1.371889in}{3.108763in}}%
\pgfpathlineto{\pgfqpoint{1.366021in}{3.126170in}}%
\pgfpathlineto{\pgfqpoint{1.360154in}{3.143790in}}%
\pgfpathlineto{\pgfqpoint{1.354286in}{3.161628in}}%
\pgfpathlineto{\pgfqpoint{1.348418in}{3.179688in}}%
\pgfpathlineto{\pgfqpoint{1.342551in}{3.197972in}}%
\pgfpathlineto{\pgfqpoint{1.336683in}{3.216487in}}%
\pgfpathlineto{\pgfqpoint{1.330816in}{3.235236in}}%
\pgfpathlineto{\pgfqpoint{1.324948in}{3.254223in}}%
\pgfpathlineto{\pgfqpoint{1.319080in}{3.273454in}}%
\pgfpathlineto{\pgfqpoint{1.313213in}{3.292933in}}%
\pgfpathlineto{\pgfqpoint{1.307345in}{3.312664in}}%
\pgfpathlineto{\pgfqpoint{1.301478in}{3.332653in}}%
\pgfpathlineto{\pgfqpoint{1.295610in}{3.352905in}}%
\pgfpathlineto{\pgfqpoint{1.289742in}{3.373425in}}%
\pgfpathlineto{\pgfqpoint{1.283875in}{3.394218in}}%
\pgfpathlineto{\pgfqpoint{1.278007in}{3.415290in}}%
\pgfpathlineto{\pgfqpoint{1.272140in}{3.436647in}}%
\pgfpathlineto{\pgfqpoint{1.266272in}{3.458294in}}%
\pgfpathlineto{\pgfqpoint{1.260404in}{3.480237in}}%
\pgfpathlineto{\pgfqpoint{1.254537in}{3.502482in}}%
\pgfpathlineto{\pgfqpoint{1.248669in}{3.525036in}}%
\pgfpathlineto{\pgfqpoint{1.242801in}{3.547906in}}%
\pgfpathlineto{\pgfqpoint{1.236934in}{3.571097in}}%
\pgfpathlineto{\pgfqpoint{1.231066in}{3.594617in}}%
\pgfpathlineto{\pgfqpoint{1.225199in}{3.618473in}}%
\pgfpathlineto{\pgfqpoint{1.219331in}{3.642671in}}%
\pgfpathlineto{\pgfqpoint{1.213463in}{3.667220in}}%
\pgfpathlineto{\pgfqpoint{1.207596in}{3.692127in}}%
\pgfpathlineto{\pgfqpoint{1.201728in}{3.717400in}}%
\pgfpathlineto{\pgfqpoint{1.195861in}{3.743046in}}%
\pgfpathlineto{\pgfqpoint{1.189993in}{3.769076in}}%
\pgfpathlineto{\pgfqpoint{1.184125in}{3.795496in}}%
\pgfpathlineto{\pgfqpoint{1.178258in}{3.822316in}}%
\pgfpathlineto{\pgfqpoint{1.172390in}{3.849545in}}%
\pgfpathlineto{\pgfqpoint{1.166523in}{3.877193in}}%
\pgfpathlineto{\pgfqpoint{1.160655in}{3.905269in}}%
\pgfpathlineto{\pgfqpoint{1.154787in}{3.933783in}}%
\pgfpathlineto{\pgfqpoint{1.148920in}{3.962746in}}%
\pgfpathlineto{\pgfqpoint{1.143052in}{3.992168in}}%
\pgfpathlineto{\pgfqpoint{1.137185in}{4.022061in}}%
\pgfpathlineto{\pgfqpoint{1.131317in}{4.052435in}}%
\pgfpathlineto{\pgfqpoint{1.125449in}{4.083302in}}%
\pgfpathlineto{\pgfqpoint{1.119582in}{4.114675in}}%
\pgfpathlineto{\pgfqpoint{1.113714in}{4.146566in}}%
\pgfpathlineto{\pgfqpoint{1.107847in}{4.178987in}}%
\pgfpathlineto{\pgfqpoint{1.101979in}{4.211953in}}%
\pgfpathlineto{\pgfqpoint{1.096111in}{4.245478in}}%
\pgfpathlineto{\pgfqpoint{1.090244in}{4.279574in}}%
\pgfpathlineto{\pgfqpoint{1.084376in}{4.314258in}}%
\pgfpathlineto{\pgfqpoint{1.078509in}{4.349544in}}%
\pgfpathlineto{\pgfqpoint{1.072641in}{4.385449in}}%
\pgfpathlineto{\pgfqpoint{1.066773in}{4.421988in}}%
\pgfpathlineto{\pgfqpoint{1.060906in}{4.459180in}}%
\pgfpathlineto{\pgfqpoint{1.055038in}{4.497040in}}%
\pgfpathlineto{\pgfqpoint{1.049171in}{4.535588in}}%
\pgfpathlineto{\pgfqpoint{1.043303in}{4.574843in}}%
\pgfpathlineto{\pgfqpoint{1.037435in}{4.614823in}}%
\pgfpathlineto{\pgfqpoint{1.031568in}{4.655550in}}%
\pgfpathlineto{\pgfqpoint{1.025700in}{4.697045in}}%
\pgfpathlineto{\pgfqpoint{1.019833in}{4.739328in}}%
\pgfpathlineto{\pgfqpoint{1.013965in}{4.782424in}}%
\pgfpathlineto{\pgfqpoint{1.008097in}{4.826356in}}%
\pgfpathlineto{\pgfqpoint{1.002230in}{4.871148in}}%
\pgfpathlineto{\pgfqpoint{0.996362in}{4.916826in}}%
\pgfpathlineto{\pgfqpoint{0.990494in}{4.963416in}}%
\pgfpathlineto{\pgfqpoint{0.984627in}{5.010946in}}%
\pgfpathlineto{\pgfqpoint{0.978759in}{5.059445in}}%
\pgfpathlineto{\pgfqpoint{0.882794in}{4.258160in}}%
\pgfpathclose%
\pgfusepath{fill}%
\end{pgfscope}%
\begin{pgfscope}%
\pgfpathrectangle{\pgfqpoint{0.882794in}{0.589583in}}{\pgfqpoint{6.917206in}{4.469862in}}%
\pgfusepath{clip}%
\pgfsetrectcap%
\pgfsetroundjoin%
\pgfsetlinewidth{0.803000pt}%
\definecolor{currentstroke}{rgb}{0.690196,0.690196,0.690196}%
\pgfsetstrokecolor{currentstroke}%
\pgfsetstrokeopacity{0.300000}%
\pgfsetdash{}{0pt}%
\pgfpathmoveto{\pgfqpoint{1.602532in}{0.589583in}}%
\pgfpathlineto{\pgfqpoint{1.602532in}{5.059445in}}%
\pgfusepath{stroke}%
\end{pgfscope}%
\begin{pgfscope}%
\pgfsetbuttcap%
\pgfsetroundjoin%
\definecolor{currentfill}{rgb}{0.000000,0.000000,0.000000}%
\pgfsetfillcolor{currentfill}%
\pgfsetlinewidth{0.803000pt}%
\definecolor{currentstroke}{rgb}{0.000000,0.000000,0.000000}%
\pgfsetstrokecolor{currentstroke}%
\pgfsetdash{}{0pt}%
\pgfsys@defobject{currentmarker}{\pgfqpoint{0.000000in}{-0.048611in}}{\pgfqpoint{0.000000in}{0.000000in}}{%
\pgfpathmoveto{\pgfqpoint{0.000000in}{0.000000in}}%
\pgfpathlineto{\pgfqpoint{0.000000in}{-0.048611in}}%
\pgfusepath{stroke,fill}%
}%
\begin{pgfscope}%
\pgfsys@transformshift{1.602532in}{0.589583in}%
\pgfsys@useobject{currentmarker}{}%
\end{pgfscope}%
\end{pgfscope}%
\begin{pgfscope}%
\definecolor{textcolor}{rgb}{0.000000,0.000000,0.000000}%
\pgfsetstrokecolor{textcolor}%
\pgfsetfillcolor{textcolor}%
\pgftext[x=1.602532in,y=0.492361in,,top]{\color{textcolor}\rmfamily\fontsize{10.000000}{12.000000}\selectfont \(\displaystyle {0.01}\)}%
\end{pgfscope}%
\begin{pgfscope}%
\pgfpathrectangle{\pgfqpoint{0.882794in}{0.589583in}}{\pgfqpoint{6.917206in}{4.469862in}}%
\pgfusepath{clip}%
\pgfsetrectcap%
\pgfsetroundjoin%
\pgfsetlinewidth{0.803000pt}%
\definecolor{currentstroke}{rgb}{0.690196,0.690196,0.690196}%
\pgfsetstrokecolor{currentstroke}%
\pgfsetstrokeopacity{0.300000}%
\pgfsetdash{}{0pt}%
\pgfpathmoveto{\pgfqpoint{2.802094in}{0.589583in}}%
\pgfpathlineto{\pgfqpoint{2.802094in}{5.059445in}}%
\pgfusepath{stroke}%
\end{pgfscope}%
\begin{pgfscope}%
\pgfsetbuttcap%
\pgfsetroundjoin%
\definecolor{currentfill}{rgb}{0.000000,0.000000,0.000000}%
\pgfsetfillcolor{currentfill}%
\pgfsetlinewidth{0.803000pt}%
\definecolor{currentstroke}{rgb}{0.000000,0.000000,0.000000}%
\pgfsetstrokecolor{currentstroke}%
\pgfsetdash{}{0pt}%
\pgfsys@defobject{currentmarker}{\pgfqpoint{0.000000in}{-0.048611in}}{\pgfqpoint{0.000000in}{0.000000in}}{%
\pgfpathmoveto{\pgfqpoint{0.000000in}{0.000000in}}%
\pgfpathlineto{\pgfqpoint{0.000000in}{-0.048611in}}%
\pgfusepath{stroke,fill}%
}%
\begin{pgfscope}%
\pgfsys@transformshift{2.802094in}{0.589583in}%
\pgfsys@useobject{currentmarker}{}%
\end{pgfscope}%
\end{pgfscope}%
\begin{pgfscope}%
\definecolor{textcolor}{rgb}{0.000000,0.000000,0.000000}%
\pgfsetstrokecolor{textcolor}%
\pgfsetfillcolor{textcolor}%
\pgftext[x=2.802094in,y=0.492361in,,top]{\color{textcolor}\rmfamily\fontsize{10.000000}{12.000000}\selectfont \(\displaystyle {0.02}\)}%
\end{pgfscope}%
\begin{pgfscope}%
\pgfpathrectangle{\pgfqpoint{0.882794in}{0.589583in}}{\pgfqpoint{6.917206in}{4.469862in}}%
\pgfusepath{clip}%
\pgfsetrectcap%
\pgfsetroundjoin%
\pgfsetlinewidth{0.803000pt}%
\definecolor{currentstroke}{rgb}{0.690196,0.690196,0.690196}%
\pgfsetstrokecolor{currentstroke}%
\pgfsetstrokeopacity{0.300000}%
\pgfsetdash{}{0pt}%
\pgfpathmoveto{\pgfqpoint{4.001656in}{0.589583in}}%
\pgfpathlineto{\pgfqpoint{4.001656in}{5.059445in}}%
\pgfusepath{stroke}%
\end{pgfscope}%
\begin{pgfscope}%
\pgfsetbuttcap%
\pgfsetroundjoin%
\definecolor{currentfill}{rgb}{0.000000,0.000000,0.000000}%
\pgfsetfillcolor{currentfill}%
\pgfsetlinewidth{0.803000pt}%
\definecolor{currentstroke}{rgb}{0.000000,0.000000,0.000000}%
\pgfsetstrokecolor{currentstroke}%
\pgfsetdash{}{0pt}%
\pgfsys@defobject{currentmarker}{\pgfqpoint{0.000000in}{-0.048611in}}{\pgfqpoint{0.000000in}{0.000000in}}{%
\pgfpathmoveto{\pgfqpoint{0.000000in}{0.000000in}}%
\pgfpathlineto{\pgfqpoint{0.000000in}{-0.048611in}}%
\pgfusepath{stroke,fill}%
}%
\begin{pgfscope}%
\pgfsys@transformshift{4.001656in}{0.589583in}%
\pgfsys@useobject{currentmarker}{}%
\end{pgfscope}%
\end{pgfscope}%
\begin{pgfscope}%
\definecolor{textcolor}{rgb}{0.000000,0.000000,0.000000}%
\pgfsetstrokecolor{textcolor}%
\pgfsetfillcolor{textcolor}%
\pgftext[x=4.001656in,y=0.492361in,,top]{\color{textcolor}\rmfamily\fontsize{10.000000}{12.000000}\selectfont \(\displaystyle {0.03}\)}%
\end{pgfscope}%
\begin{pgfscope}%
\pgfpathrectangle{\pgfqpoint{0.882794in}{0.589583in}}{\pgfqpoint{6.917206in}{4.469862in}}%
\pgfusepath{clip}%
\pgfsetrectcap%
\pgfsetroundjoin%
\pgfsetlinewidth{0.803000pt}%
\definecolor{currentstroke}{rgb}{0.690196,0.690196,0.690196}%
\pgfsetstrokecolor{currentstroke}%
\pgfsetstrokeopacity{0.300000}%
\pgfsetdash{}{0pt}%
\pgfpathmoveto{\pgfqpoint{5.201219in}{0.589583in}}%
\pgfpathlineto{\pgfqpoint{5.201219in}{5.059445in}}%
\pgfusepath{stroke}%
\end{pgfscope}%
\begin{pgfscope}%
\pgfsetbuttcap%
\pgfsetroundjoin%
\definecolor{currentfill}{rgb}{0.000000,0.000000,0.000000}%
\pgfsetfillcolor{currentfill}%
\pgfsetlinewidth{0.803000pt}%
\definecolor{currentstroke}{rgb}{0.000000,0.000000,0.000000}%
\pgfsetstrokecolor{currentstroke}%
\pgfsetdash{}{0pt}%
\pgfsys@defobject{currentmarker}{\pgfqpoint{0.000000in}{-0.048611in}}{\pgfqpoint{0.000000in}{0.000000in}}{%
\pgfpathmoveto{\pgfqpoint{0.000000in}{0.000000in}}%
\pgfpathlineto{\pgfqpoint{0.000000in}{-0.048611in}}%
\pgfusepath{stroke,fill}%
}%
\begin{pgfscope}%
\pgfsys@transformshift{5.201219in}{0.589583in}%
\pgfsys@useobject{currentmarker}{}%
\end{pgfscope}%
\end{pgfscope}%
\begin{pgfscope}%
\definecolor{textcolor}{rgb}{0.000000,0.000000,0.000000}%
\pgfsetstrokecolor{textcolor}%
\pgfsetfillcolor{textcolor}%
\pgftext[x=5.201219in,y=0.492361in,,top]{\color{textcolor}\rmfamily\fontsize{10.000000}{12.000000}\selectfont \(\displaystyle {0.04}\)}%
\end{pgfscope}%
\begin{pgfscope}%
\pgfpathrectangle{\pgfqpoint{0.882794in}{0.589583in}}{\pgfqpoint{6.917206in}{4.469862in}}%
\pgfusepath{clip}%
\pgfsetrectcap%
\pgfsetroundjoin%
\pgfsetlinewidth{0.803000pt}%
\definecolor{currentstroke}{rgb}{0.690196,0.690196,0.690196}%
\pgfsetstrokecolor{currentstroke}%
\pgfsetstrokeopacity{0.300000}%
\pgfsetdash{}{0pt}%
\pgfpathmoveto{\pgfqpoint{6.400781in}{0.589583in}}%
\pgfpathlineto{\pgfqpoint{6.400781in}{5.059445in}}%
\pgfusepath{stroke}%
\end{pgfscope}%
\begin{pgfscope}%
\pgfsetbuttcap%
\pgfsetroundjoin%
\definecolor{currentfill}{rgb}{0.000000,0.000000,0.000000}%
\pgfsetfillcolor{currentfill}%
\pgfsetlinewidth{0.803000pt}%
\definecolor{currentstroke}{rgb}{0.000000,0.000000,0.000000}%
\pgfsetstrokecolor{currentstroke}%
\pgfsetdash{}{0pt}%
\pgfsys@defobject{currentmarker}{\pgfqpoint{0.000000in}{-0.048611in}}{\pgfqpoint{0.000000in}{0.000000in}}{%
\pgfpathmoveto{\pgfqpoint{0.000000in}{0.000000in}}%
\pgfpathlineto{\pgfqpoint{0.000000in}{-0.048611in}}%
\pgfusepath{stroke,fill}%
}%
\begin{pgfscope}%
\pgfsys@transformshift{6.400781in}{0.589583in}%
\pgfsys@useobject{currentmarker}{}%
\end{pgfscope}%
\end{pgfscope}%
\begin{pgfscope}%
\definecolor{textcolor}{rgb}{0.000000,0.000000,0.000000}%
\pgfsetstrokecolor{textcolor}%
\pgfsetfillcolor{textcolor}%
\pgftext[x=6.400781in,y=0.492361in,,top]{\color{textcolor}\rmfamily\fontsize{10.000000}{12.000000}\selectfont \(\displaystyle {0.05}\)}%
\end{pgfscope}%
\begin{pgfscope}%
\pgfpathrectangle{\pgfqpoint{0.882794in}{0.589583in}}{\pgfqpoint{6.917206in}{4.469862in}}%
\pgfusepath{clip}%
\pgfsetrectcap%
\pgfsetroundjoin%
\pgfsetlinewidth{0.803000pt}%
\definecolor{currentstroke}{rgb}{0.690196,0.690196,0.690196}%
\pgfsetstrokecolor{currentstroke}%
\pgfsetstrokeopacity{0.300000}%
\pgfsetdash{}{0pt}%
\pgfpathmoveto{\pgfqpoint{7.600343in}{0.589583in}}%
\pgfpathlineto{\pgfqpoint{7.600343in}{5.059445in}}%
\pgfusepath{stroke}%
\end{pgfscope}%
\begin{pgfscope}%
\pgfsetbuttcap%
\pgfsetroundjoin%
\definecolor{currentfill}{rgb}{0.000000,0.000000,0.000000}%
\pgfsetfillcolor{currentfill}%
\pgfsetlinewidth{0.803000pt}%
\definecolor{currentstroke}{rgb}{0.000000,0.000000,0.000000}%
\pgfsetstrokecolor{currentstroke}%
\pgfsetdash{}{0pt}%
\pgfsys@defobject{currentmarker}{\pgfqpoint{0.000000in}{-0.048611in}}{\pgfqpoint{0.000000in}{0.000000in}}{%
\pgfpathmoveto{\pgfqpoint{0.000000in}{0.000000in}}%
\pgfpathlineto{\pgfqpoint{0.000000in}{-0.048611in}}%
\pgfusepath{stroke,fill}%
}%
\begin{pgfscope}%
\pgfsys@transformshift{7.600343in}{0.589583in}%
\pgfsys@useobject{currentmarker}{}%
\end{pgfscope}%
\end{pgfscope}%
\begin{pgfscope}%
\definecolor{textcolor}{rgb}{0.000000,0.000000,0.000000}%
\pgfsetstrokecolor{textcolor}%
\pgfsetfillcolor{textcolor}%
\pgftext[x=7.600343in,y=0.492361in,,top]{\color{textcolor}\rmfamily\fontsize{10.000000}{12.000000}\selectfont \(\displaystyle {0.06}\)}%
\end{pgfscope}%
\begin{pgfscope}%
\definecolor{textcolor}{rgb}{0.000000,0.000000,0.000000}%
\pgfsetstrokecolor{textcolor}%
\pgfsetfillcolor{textcolor}%
\pgftext[x=4.341397in,y=0.313349in,,top]{\color{textcolor}\rmfamily\fontsize{16.000000}{19.200000}\selectfont f1}%
\end{pgfscope}%
\begin{pgfscope}%
\pgfpathrectangle{\pgfqpoint{0.882794in}{0.589583in}}{\pgfqpoint{6.917206in}{4.469862in}}%
\pgfusepath{clip}%
\pgfsetrectcap%
\pgfsetroundjoin%
\pgfsetlinewidth{0.803000pt}%
\definecolor{currentstroke}{rgb}{0.690196,0.690196,0.690196}%
\pgfsetstrokecolor{currentstroke}%
\pgfsetstrokeopacity{0.300000}%
\pgfsetdash{}{0pt}%
\pgfpathmoveto{\pgfqpoint{0.882794in}{1.053017in}}%
\pgfpathlineto{\pgfqpoint{7.800000in}{1.053017in}}%
\pgfusepath{stroke}%
\end{pgfscope}%
\begin{pgfscope}%
\pgfsetbuttcap%
\pgfsetroundjoin%
\definecolor{currentfill}{rgb}{0.000000,0.000000,0.000000}%
\pgfsetfillcolor{currentfill}%
\pgfsetlinewidth{0.803000pt}%
\definecolor{currentstroke}{rgb}{0.000000,0.000000,0.000000}%
\pgfsetstrokecolor{currentstroke}%
\pgfsetdash{}{0pt}%
\pgfsys@defobject{currentmarker}{\pgfqpoint{-0.048611in}{0.000000in}}{\pgfqpoint{-0.000000in}{0.000000in}}{%
\pgfpathmoveto{\pgfqpoint{-0.000000in}{0.000000in}}%
\pgfpathlineto{\pgfqpoint{-0.048611in}{0.000000in}}%
\pgfusepath{stroke,fill}%
}%
\begin{pgfscope}%
\pgfsys@transformshift{0.882794in}{1.053017in}%
\pgfsys@useobject{currentmarker}{}%
\end{pgfscope}%
\end{pgfscope}%
\begin{pgfscope}%
\definecolor{textcolor}{rgb}{0.000000,0.000000,0.000000}%
\pgfsetstrokecolor{textcolor}%
\pgfsetfillcolor{textcolor}%
\pgftext[x=0.438349in, y=1.004792in, left, base]{\color{textcolor}\rmfamily\fontsize{10.000000}{12.000000}\selectfont \(\displaystyle {20000}\)}%
\end{pgfscope}%
\begin{pgfscope}%
\pgfpathrectangle{\pgfqpoint{0.882794in}{0.589583in}}{\pgfqpoint{6.917206in}{4.469862in}}%
\pgfusepath{clip}%
\pgfsetrectcap%
\pgfsetroundjoin%
\pgfsetlinewidth{0.803000pt}%
\definecolor{currentstroke}{rgb}{0.690196,0.690196,0.690196}%
\pgfsetstrokecolor{currentstroke}%
\pgfsetstrokeopacity{0.300000}%
\pgfsetdash{}{0pt}%
\pgfpathmoveto{\pgfqpoint{0.882794in}{1.854303in}}%
\pgfpathlineto{\pgfqpoint{7.800000in}{1.854303in}}%
\pgfusepath{stroke}%
\end{pgfscope}%
\begin{pgfscope}%
\pgfsetbuttcap%
\pgfsetroundjoin%
\definecolor{currentfill}{rgb}{0.000000,0.000000,0.000000}%
\pgfsetfillcolor{currentfill}%
\pgfsetlinewidth{0.803000pt}%
\definecolor{currentstroke}{rgb}{0.000000,0.000000,0.000000}%
\pgfsetstrokecolor{currentstroke}%
\pgfsetdash{}{0pt}%
\pgfsys@defobject{currentmarker}{\pgfqpoint{-0.048611in}{0.000000in}}{\pgfqpoint{-0.000000in}{0.000000in}}{%
\pgfpathmoveto{\pgfqpoint{-0.000000in}{0.000000in}}%
\pgfpathlineto{\pgfqpoint{-0.048611in}{0.000000in}}%
\pgfusepath{stroke,fill}%
}%
\begin{pgfscope}%
\pgfsys@transformshift{0.882794in}{1.854303in}%
\pgfsys@useobject{currentmarker}{}%
\end{pgfscope}%
\end{pgfscope}%
\begin{pgfscope}%
\definecolor{textcolor}{rgb}{0.000000,0.000000,0.000000}%
\pgfsetstrokecolor{textcolor}%
\pgfsetfillcolor{textcolor}%
\pgftext[x=0.438349in, y=1.806077in, left, base]{\color{textcolor}\rmfamily\fontsize{10.000000}{12.000000}\selectfont \(\displaystyle {40000}\)}%
\end{pgfscope}%
\begin{pgfscope}%
\pgfpathrectangle{\pgfqpoint{0.882794in}{0.589583in}}{\pgfqpoint{6.917206in}{4.469862in}}%
\pgfusepath{clip}%
\pgfsetrectcap%
\pgfsetroundjoin%
\pgfsetlinewidth{0.803000pt}%
\definecolor{currentstroke}{rgb}{0.690196,0.690196,0.690196}%
\pgfsetstrokecolor{currentstroke}%
\pgfsetstrokeopacity{0.300000}%
\pgfsetdash{}{0pt}%
\pgfpathmoveto{\pgfqpoint{0.882794in}{2.655588in}}%
\pgfpathlineto{\pgfqpoint{7.800000in}{2.655588in}}%
\pgfusepath{stroke}%
\end{pgfscope}%
\begin{pgfscope}%
\pgfsetbuttcap%
\pgfsetroundjoin%
\definecolor{currentfill}{rgb}{0.000000,0.000000,0.000000}%
\pgfsetfillcolor{currentfill}%
\pgfsetlinewidth{0.803000pt}%
\definecolor{currentstroke}{rgb}{0.000000,0.000000,0.000000}%
\pgfsetstrokecolor{currentstroke}%
\pgfsetdash{}{0pt}%
\pgfsys@defobject{currentmarker}{\pgfqpoint{-0.048611in}{0.000000in}}{\pgfqpoint{-0.000000in}{0.000000in}}{%
\pgfpathmoveto{\pgfqpoint{-0.000000in}{0.000000in}}%
\pgfpathlineto{\pgfqpoint{-0.048611in}{0.000000in}}%
\pgfusepath{stroke,fill}%
}%
\begin{pgfscope}%
\pgfsys@transformshift{0.882794in}{2.655588in}%
\pgfsys@useobject{currentmarker}{}%
\end{pgfscope}%
\end{pgfscope}%
\begin{pgfscope}%
\definecolor{textcolor}{rgb}{0.000000,0.000000,0.000000}%
\pgfsetstrokecolor{textcolor}%
\pgfsetfillcolor{textcolor}%
\pgftext[x=0.438349in, y=2.607363in, left, base]{\color{textcolor}\rmfamily\fontsize{10.000000}{12.000000}\selectfont \(\displaystyle {60000}\)}%
\end{pgfscope}%
\begin{pgfscope}%
\pgfpathrectangle{\pgfqpoint{0.882794in}{0.589583in}}{\pgfqpoint{6.917206in}{4.469862in}}%
\pgfusepath{clip}%
\pgfsetrectcap%
\pgfsetroundjoin%
\pgfsetlinewidth{0.803000pt}%
\definecolor{currentstroke}{rgb}{0.690196,0.690196,0.690196}%
\pgfsetstrokecolor{currentstroke}%
\pgfsetstrokeopacity{0.300000}%
\pgfsetdash{}{0pt}%
\pgfpathmoveto{\pgfqpoint{0.882794in}{3.456874in}}%
\pgfpathlineto{\pgfqpoint{7.800000in}{3.456874in}}%
\pgfusepath{stroke}%
\end{pgfscope}%
\begin{pgfscope}%
\pgfsetbuttcap%
\pgfsetroundjoin%
\definecolor{currentfill}{rgb}{0.000000,0.000000,0.000000}%
\pgfsetfillcolor{currentfill}%
\pgfsetlinewidth{0.803000pt}%
\definecolor{currentstroke}{rgb}{0.000000,0.000000,0.000000}%
\pgfsetstrokecolor{currentstroke}%
\pgfsetdash{}{0pt}%
\pgfsys@defobject{currentmarker}{\pgfqpoint{-0.048611in}{0.000000in}}{\pgfqpoint{-0.000000in}{0.000000in}}{%
\pgfpathmoveto{\pgfqpoint{-0.000000in}{0.000000in}}%
\pgfpathlineto{\pgfqpoint{-0.048611in}{0.000000in}}%
\pgfusepath{stroke,fill}%
}%
\begin{pgfscope}%
\pgfsys@transformshift{0.882794in}{3.456874in}%
\pgfsys@useobject{currentmarker}{}%
\end{pgfscope}%
\end{pgfscope}%
\begin{pgfscope}%
\definecolor{textcolor}{rgb}{0.000000,0.000000,0.000000}%
\pgfsetstrokecolor{textcolor}%
\pgfsetfillcolor{textcolor}%
\pgftext[x=0.438349in, y=3.408649in, left, base]{\color{textcolor}\rmfamily\fontsize{10.000000}{12.000000}\selectfont \(\displaystyle {80000}\)}%
\end{pgfscope}%
\begin{pgfscope}%
\pgfpathrectangle{\pgfqpoint{0.882794in}{0.589583in}}{\pgfqpoint{6.917206in}{4.469862in}}%
\pgfusepath{clip}%
\pgfsetrectcap%
\pgfsetroundjoin%
\pgfsetlinewidth{0.803000pt}%
\definecolor{currentstroke}{rgb}{0.690196,0.690196,0.690196}%
\pgfsetstrokecolor{currentstroke}%
\pgfsetstrokeopacity{0.300000}%
\pgfsetdash{}{0pt}%
\pgfpathmoveto{\pgfqpoint{0.882794in}{4.258160in}}%
\pgfpathlineto{\pgfqpoint{7.800000in}{4.258160in}}%
\pgfusepath{stroke}%
\end{pgfscope}%
\begin{pgfscope}%
\pgfsetbuttcap%
\pgfsetroundjoin%
\definecolor{currentfill}{rgb}{0.000000,0.000000,0.000000}%
\pgfsetfillcolor{currentfill}%
\pgfsetlinewidth{0.803000pt}%
\definecolor{currentstroke}{rgb}{0.000000,0.000000,0.000000}%
\pgfsetstrokecolor{currentstroke}%
\pgfsetdash{}{0pt}%
\pgfsys@defobject{currentmarker}{\pgfqpoint{-0.048611in}{0.000000in}}{\pgfqpoint{-0.000000in}{0.000000in}}{%
\pgfpathmoveto{\pgfqpoint{-0.000000in}{0.000000in}}%
\pgfpathlineto{\pgfqpoint{-0.048611in}{0.000000in}}%
\pgfusepath{stroke,fill}%
}%
\begin{pgfscope}%
\pgfsys@transformshift{0.882794in}{4.258160in}%
\pgfsys@useobject{currentmarker}{}%
\end{pgfscope}%
\end{pgfscope}%
\begin{pgfscope}%
\definecolor{textcolor}{rgb}{0.000000,0.000000,0.000000}%
\pgfsetstrokecolor{textcolor}%
\pgfsetfillcolor{textcolor}%
\pgftext[x=0.368904in, y=4.209934in, left, base]{\color{textcolor}\rmfamily\fontsize{10.000000}{12.000000}\selectfont \(\displaystyle {100000}\)}%
\end{pgfscope}%
\begin{pgfscope}%
\pgfpathrectangle{\pgfqpoint{0.882794in}{0.589583in}}{\pgfqpoint{6.917206in}{4.469862in}}%
\pgfusepath{clip}%
\pgfsetrectcap%
\pgfsetroundjoin%
\pgfsetlinewidth{0.803000pt}%
\definecolor{currentstroke}{rgb}{0.690196,0.690196,0.690196}%
\pgfsetstrokecolor{currentstroke}%
\pgfsetstrokeopacity{0.300000}%
\pgfsetdash{}{0pt}%
\pgfpathmoveto{\pgfqpoint{0.882794in}{5.059445in}}%
\pgfpathlineto{\pgfqpoint{7.800000in}{5.059445in}}%
\pgfusepath{stroke}%
\end{pgfscope}%
\begin{pgfscope}%
\pgfsetbuttcap%
\pgfsetroundjoin%
\definecolor{currentfill}{rgb}{0.000000,0.000000,0.000000}%
\pgfsetfillcolor{currentfill}%
\pgfsetlinewidth{0.803000pt}%
\definecolor{currentstroke}{rgb}{0.000000,0.000000,0.000000}%
\pgfsetstrokecolor{currentstroke}%
\pgfsetdash{}{0pt}%
\pgfsys@defobject{currentmarker}{\pgfqpoint{-0.048611in}{0.000000in}}{\pgfqpoint{-0.000000in}{0.000000in}}{%
\pgfpathmoveto{\pgfqpoint{-0.000000in}{0.000000in}}%
\pgfpathlineto{\pgfqpoint{-0.048611in}{0.000000in}}%
\pgfusepath{stroke,fill}%
}%
\begin{pgfscope}%
\pgfsys@transformshift{0.882794in}{5.059445in}%
\pgfsys@useobject{currentmarker}{}%
\end{pgfscope}%
\end{pgfscope}%
\begin{pgfscope}%
\definecolor{textcolor}{rgb}{0.000000,0.000000,0.000000}%
\pgfsetstrokecolor{textcolor}%
\pgfsetfillcolor{textcolor}%
\pgftext[x=0.368904in, y=5.011220in, left, base]{\color{textcolor}\rmfamily\fontsize{10.000000}{12.000000}\selectfont \(\displaystyle {120000}\)}%
\end{pgfscope}%
\begin{pgfscope}%
\definecolor{textcolor}{rgb}{0.000000,0.000000,0.000000}%
\pgfsetstrokecolor{textcolor}%
\pgfsetfillcolor{textcolor}%
\pgftext[x=0.313349in,y=2.824514in,,bottom,rotate=90.000000]{\color{textcolor}\rmfamily\fontsize{16.000000}{19.200000}\selectfont f2}%
\end{pgfscope}%
\begin{pgfscope}%
\pgfpathrectangle{\pgfqpoint{0.882794in}{0.589583in}}{\pgfqpoint{6.917206in}{4.469862in}}%
\pgfusepath{clip}%
\pgfsetrectcap%
\pgfsetroundjoin%
\pgfsetlinewidth{1.505625pt}%
\definecolor{currentstroke}{rgb}{0.827451,0.827451,0.827451}%
\pgfsetstrokecolor{currentstroke}%
\pgfsetstrokeopacity{0.500000}%
\pgfsetdash{}{0pt}%
\pgfpathmoveto{\pgfqpoint{0.978759in}{5.059445in}}%
\pgfpathlineto{\pgfqpoint{0.990494in}{4.963416in}}%
\pgfpathlineto{\pgfqpoint{1.002230in}{4.871148in}}%
\pgfpathlineto{\pgfqpoint{1.013965in}{4.782424in}}%
\pgfpathlineto{\pgfqpoint{1.025700in}{4.697045in}}%
\pgfpathlineto{\pgfqpoint{1.037435in}{4.614823in}}%
\pgfpathlineto{\pgfqpoint{1.049171in}{4.535588in}}%
\pgfpathlineto{\pgfqpoint{1.060906in}{4.459180in}}%
\pgfpathlineto{\pgfqpoint{1.072641in}{4.385449in}}%
\pgfpathlineto{\pgfqpoint{1.084376in}{4.314258in}}%
\pgfpathlineto{\pgfqpoint{1.096111in}{4.245478in}}%
\pgfpathlineto{\pgfqpoint{1.107847in}{4.178987in}}%
\pgfpathlineto{\pgfqpoint{1.119582in}{4.114675in}}%
\pgfpathlineto{\pgfqpoint{1.131317in}{4.052435in}}%
\pgfpathlineto{\pgfqpoint{1.143052in}{3.992168in}}%
\pgfpathlineto{\pgfqpoint{1.154787in}{3.933783in}}%
\pgfpathlineto{\pgfqpoint{1.166523in}{3.877193in}}%
\pgfpathlineto{\pgfqpoint{1.178258in}{3.822316in}}%
\pgfpathlineto{\pgfqpoint{1.189993in}{3.769076in}}%
\pgfpathlineto{\pgfqpoint{1.201728in}{3.717400in}}%
\pgfpathlineto{\pgfqpoint{1.213463in}{3.667220in}}%
\pgfpathlineto{\pgfqpoint{1.225199in}{3.618473in}}%
\pgfpathlineto{\pgfqpoint{1.236934in}{3.571097in}}%
\pgfpathlineto{\pgfqpoint{1.248669in}{3.525036in}}%
\pgfpathlineto{\pgfqpoint{1.260404in}{3.480237in}}%
\pgfpathlineto{\pgfqpoint{1.272140in}{3.436647in}}%
\pgfpathlineto{\pgfqpoint{1.283875in}{3.394218in}}%
\pgfpathlineto{\pgfqpoint{1.295610in}{3.352905in}}%
\pgfpathlineto{\pgfqpoint{1.307345in}{3.312664in}}%
\pgfpathlineto{\pgfqpoint{1.319080in}{3.273454in}}%
\pgfpathlineto{\pgfqpoint{1.330816in}{3.235236in}}%
\pgfpathlineto{\pgfqpoint{1.342551in}{3.197972in}}%
\pgfpathlineto{\pgfqpoint{1.354286in}{3.161628in}}%
\pgfpathlineto{\pgfqpoint{1.366021in}{3.126170in}}%
\pgfpathlineto{\pgfqpoint{1.377756in}{3.091565in}}%
\pgfpathlineto{\pgfqpoint{1.389492in}{3.057784in}}%
\pgfpathlineto{\pgfqpoint{1.401227in}{3.024797in}}%
\pgfpathlineto{\pgfqpoint{1.412962in}{2.992576in}}%
\pgfpathlineto{\pgfqpoint{1.424697in}{2.961096in}}%
\pgfpathlineto{\pgfqpoint{1.436432in}{2.930330in}}%
\pgfpathlineto{\pgfqpoint{1.454035in}{2.885470in}}%
\pgfpathlineto{\pgfqpoint{1.471638in}{2.842088in}}%
\pgfpathlineto{\pgfqpoint{1.489241in}{2.800112in}}%
\pgfpathlineto{\pgfqpoint{1.506844in}{2.759474in}}%
\pgfpathlineto{\pgfqpoint{1.524447in}{2.720113in}}%
\pgfpathlineto{\pgfqpoint{1.542049in}{2.681967in}}%
\pgfpathlineto{\pgfqpoint{1.559652in}{2.644983in}}%
\pgfpathlineto{\pgfqpoint{1.577255in}{2.609108in}}%
\pgfpathlineto{\pgfqpoint{1.594858in}{2.574292in}}%
\pgfpathlineto{\pgfqpoint{1.612461in}{2.540490in}}%
\pgfpathlineto{\pgfqpoint{1.630063in}{2.507657in}}%
\pgfpathlineto{\pgfqpoint{1.647666in}{2.475753in}}%
\pgfpathlineto{\pgfqpoint{1.665269in}{2.444739in}}%
\pgfpathlineto{\pgfqpoint{1.682872in}{2.414578in}}%
\pgfpathlineto{\pgfqpoint{1.700475in}{2.385236in}}%
\pgfpathlineto{\pgfqpoint{1.718077in}{2.356679in}}%
\pgfpathlineto{\pgfqpoint{1.735680in}{2.328876in}}%
\pgfpathlineto{\pgfqpoint{1.753283in}{2.301798in}}%
\pgfpathlineto{\pgfqpoint{1.770886in}{2.275417in}}%
\pgfpathlineto{\pgfqpoint{1.788489in}{2.249707in}}%
\pgfpathlineto{\pgfqpoint{1.806092in}{2.224641in}}%
\pgfpathlineto{\pgfqpoint{1.823694in}{2.200197in}}%
\pgfpathlineto{\pgfqpoint{1.841297in}{2.176351in}}%
\pgfpathlineto{\pgfqpoint{1.858900in}{2.153081in}}%
\pgfpathlineto{\pgfqpoint{1.876503in}{2.130368in}}%
\pgfpathlineto{\pgfqpoint{1.894106in}{2.108190in}}%
\pgfpathlineto{\pgfqpoint{1.911708in}{2.086531in}}%
\pgfpathlineto{\pgfqpoint{1.929311in}{2.065370in}}%
\pgfpathlineto{\pgfqpoint{1.946914in}{2.044693in}}%
\pgfpathlineto{\pgfqpoint{1.964517in}{2.024481in}}%
\pgfpathlineto{\pgfqpoint{1.982120in}{2.004720in}}%
\pgfpathlineto{\pgfqpoint{1.999722in}{1.985395in}}%
\pgfpathlineto{\pgfqpoint{2.017325in}{1.966492in}}%
\pgfpathlineto{\pgfqpoint{2.040796in}{1.941919in}}%
\pgfpathlineto{\pgfqpoint{2.064266in}{1.918040in}}%
\pgfpathlineto{\pgfqpoint{2.087737in}{1.894827in}}%
\pgfpathlineto{\pgfqpoint{2.111207in}{1.872251in}}%
\pgfpathlineto{\pgfqpoint{2.134677in}{1.850288in}}%
\pgfpathlineto{\pgfqpoint{2.158148in}{1.828912in}}%
\pgfpathlineto{\pgfqpoint{2.181618in}{1.808100in}}%
\pgfpathlineto{\pgfqpoint{2.205089in}{1.787830in}}%
\pgfpathlineto{\pgfqpoint{2.228559in}{1.768082in}}%
\pgfpathlineto{\pgfqpoint{2.252029in}{1.748834in}}%
\pgfpathlineto{\pgfqpoint{2.275500in}{1.730069in}}%
\pgfpathlineto{\pgfqpoint{2.298970in}{1.711769in}}%
\pgfpathlineto{\pgfqpoint{2.322441in}{1.693917in}}%
\pgfpathlineto{\pgfqpoint{2.345911in}{1.676495in}}%
\pgfpathlineto{\pgfqpoint{2.369382in}{1.659490in}}%
\pgfpathlineto{\pgfqpoint{2.392852in}{1.642885in}}%
\pgfpathlineto{\pgfqpoint{2.416322in}{1.626668in}}%
\pgfpathlineto{\pgfqpoint{2.439793in}{1.610825in}}%
\pgfpathlineto{\pgfqpoint{2.469131in}{1.591527in}}%
\pgfpathlineto{\pgfqpoint{2.498469in}{1.572769in}}%
\pgfpathlineto{\pgfqpoint{2.527807in}{1.554529in}}%
\pgfpathlineto{\pgfqpoint{2.557145in}{1.536786in}}%
\pgfpathlineto{\pgfqpoint{2.586483in}{1.519520in}}%
\pgfpathlineto{\pgfqpoint{2.615821in}{1.502711in}}%
\pgfpathlineto{\pgfqpoint{2.645159in}{1.486343in}}%
\pgfpathlineto{\pgfqpoint{2.674497in}{1.470397in}}%
\pgfpathlineto{\pgfqpoint{2.703835in}{1.454858in}}%
\pgfpathlineto{\pgfqpoint{2.733173in}{1.439710in}}%
\pgfpathlineto{\pgfqpoint{2.762511in}{1.424939in}}%
\pgfpathlineto{\pgfqpoint{2.791849in}{1.410531in}}%
\pgfpathlineto{\pgfqpoint{2.821187in}{1.396472in}}%
\pgfpathlineto{\pgfqpoint{2.856393in}{1.380046in}}%
\pgfpathlineto{\pgfqpoint{2.891598in}{1.364084in}}%
\pgfpathlineto{\pgfqpoint{2.926804in}{1.348568in}}%
\pgfpathlineto{\pgfqpoint{2.962010in}{1.333478in}}%
\pgfpathlineto{\pgfqpoint{2.997215in}{1.318798in}}%
\pgfpathlineto{\pgfqpoint{3.032421in}{1.304511in}}%
\pgfpathlineto{\pgfqpoint{3.067626in}{1.290602in}}%
\pgfpathlineto{\pgfqpoint{3.102832in}{1.277055in}}%
\pgfpathlineto{\pgfqpoint{3.138038in}{1.263857in}}%
\pgfpathlineto{\pgfqpoint{3.179111in}{1.248883in}}%
\pgfpathlineto{\pgfqpoint{3.220184in}{1.234345in}}%
\pgfpathlineto{\pgfqpoint{3.261257in}{1.220225in}}%
\pgfpathlineto{\pgfqpoint{3.302331in}{1.206505in}}%
\pgfpathlineto{\pgfqpoint{3.343404in}{1.193168in}}%
\pgfpathlineto{\pgfqpoint{3.384477in}{1.180199in}}%
\pgfpathlineto{\pgfqpoint{3.425550in}{1.167582in}}%
\pgfpathlineto{\pgfqpoint{3.472491in}{1.153576in}}%
\pgfpathlineto{\pgfqpoint{3.519432in}{1.139993in}}%
\pgfpathlineto{\pgfqpoint{3.566373in}{1.126812in}}%
\pgfpathlineto{\pgfqpoint{3.613314in}{1.114017in}}%
\pgfpathlineto{\pgfqpoint{3.660255in}{1.101590in}}%
\pgfpathlineto{\pgfqpoint{3.707195in}{1.089517in}}%
\pgfpathlineto{\pgfqpoint{3.760004in}{1.076338in}}%
\pgfpathlineto{\pgfqpoint{3.812812in}{1.063567in}}%
\pgfpathlineto{\pgfqpoint{3.865621in}{1.051186in}}%
\pgfpathlineto{\pgfqpoint{3.918429in}{1.039177in}}%
\pgfpathlineto{\pgfqpoint{3.971238in}{1.027523in}}%
\pgfpathlineto{\pgfqpoint{4.029914in}{1.014973in}}%
\pgfpathlineto{\pgfqpoint{4.088590in}{1.002822in}}%
\pgfpathlineto{\pgfqpoint{4.147266in}{0.991051in}}%
\pgfpathlineto{\pgfqpoint{4.205942in}{0.979644in}}%
\pgfpathlineto{\pgfqpoint{4.270485in}{0.967497in}}%
\pgfpathlineto{\pgfqpoint{4.335029in}{0.955747in}}%
\pgfpathlineto{\pgfqpoint{4.399573in}{0.944378in}}%
\pgfpathlineto{\pgfqpoint{4.464116in}{0.933370in}}%
\pgfpathlineto{\pgfqpoint{4.534528in}{0.921753in}}%
\pgfpathlineto{\pgfqpoint{4.604939in}{0.910526in}}%
\pgfpathlineto{\pgfqpoint{4.675350in}{0.899668in}}%
\pgfpathlineto{\pgfqpoint{4.751629in}{0.888303in}}%
\pgfpathlineto{\pgfqpoint{4.827908in}{0.877329in}}%
\pgfpathlineto{\pgfqpoint{4.904187in}{0.866728in}}%
\pgfpathlineto{\pgfqpoint{4.986333in}{0.855705in}}%
\pgfpathlineto{\pgfqpoint{5.068480in}{0.845071in}}%
\pgfpathlineto{\pgfqpoint{5.150626in}{0.834805in}}%
\pgfpathlineto{\pgfqpoint{5.238640in}{0.824192in}}%
\pgfpathlineto{\pgfqpoint{5.326654in}{0.813959in}}%
\pgfpathlineto{\pgfqpoint{5.414668in}{0.804086in}}%
\pgfpathlineto{\pgfqpoint{5.508550in}{0.793929in}}%
\pgfpathlineto{\pgfqpoint{5.602432in}{0.784139in}}%
\pgfpathlineto{\pgfqpoint{5.702181in}{0.774117in}}%
\pgfpathlineto{\pgfqpoint{5.801930in}{0.764466in}}%
\pgfpathlineto{\pgfqpoint{5.907547in}{0.754628in}}%
\pgfpathlineto{\pgfqpoint{6.013164in}{0.745160in}}%
\pgfpathlineto{\pgfqpoint{6.124648in}{0.735546in}}%
\pgfpathlineto{\pgfqpoint{6.236133in}{0.726299in}}%
\pgfpathlineto{\pgfqpoint{6.353485in}{0.716940in}}%
\pgfpathlineto{\pgfqpoint{6.470837in}{0.707943in}}%
\pgfpathlineto{\pgfqpoint{6.594057in}{0.698863in}}%
\pgfpathlineto{\pgfqpoint{6.723144in}{0.689731in}}%
\pgfpathlineto{\pgfqpoint{6.850188in}{0.681112in}}%
\pgfpathlineto{\pgfqpoint{6.888956in}{0.678800in}}%
\pgfpathlineto{\pgfqpoint{6.937416in}{0.676319in}}%
\pgfpathlineto{\pgfqpoint{6.995567in}{0.673787in}}%
\pgfpathlineto{\pgfqpoint{7.053719in}{0.671618in}}%
\pgfpathlineto{\pgfqpoint{7.121563in}{0.669434in}}%
\pgfpathlineto{\pgfqpoint{7.199099in}{0.667287in}}%
\pgfpathlineto{\pgfqpoint{7.296018in}{0.665000in}}%
\pgfpathlineto{\pgfqpoint{7.402630in}{0.662870in}}%
\pgfpathlineto{\pgfqpoint{7.528625in}{0.660742in}}%
\pgfpathlineto{\pgfqpoint{7.674005in}{0.658680in}}%
\pgfpathlineto{\pgfqpoint{7.800000in}{0.657153in}}%
\pgfpathlineto{\pgfqpoint{7.800000in}{0.657153in}}%
\pgfusepath{stroke}%
\end{pgfscope}%
\begin{pgfscope}%
\pgfpathrectangle{\pgfqpoint{0.882794in}{0.589583in}}{\pgfqpoint{6.917206in}{4.469862in}}%
\pgfusepath{clip}%
\pgfsetrectcap%
\pgfsetroundjoin%
\pgfsetlinewidth{3.011250pt}%
\definecolor{currentstroke}{rgb}{0.000000,0.000000,0.000000}%
\pgfsetstrokecolor{currentstroke}%
\pgfsetdash{}{0pt}%
\pgfpathmoveto{\pgfqpoint{0.882794in}{4.258160in}}%
\pgfpathlineto{\pgfqpoint{0.892574in}{4.178135in}}%
\pgfpathlineto{\pgfqpoint{0.902353in}{4.101245in}}%
\pgfpathlineto{\pgfqpoint{0.912132in}{4.027309in}}%
\pgfpathlineto{\pgfqpoint{0.921912in}{3.956159in}}%
\pgfpathlineto{\pgfqpoint{0.931691in}{3.887641in}}%
\pgfpathlineto{\pgfqpoint{0.941470in}{3.821612in}}%
\pgfpathlineto{\pgfqpoint{0.951250in}{3.757938in}}%
\pgfpathlineto{\pgfqpoint{0.961029in}{3.696496in}}%
\pgfpathlineto{\pgfqpoint{0.970808in}{3.637170in}}%
\pgfpathlineto{\pgfqpoint{0.980588in}{3.579853in}}%
\pgfpathlineto{\pgfqpoint{0.990367in}{3.524445in}}%
\pgfpathlineto{\pgfqpoint{1.000146in}{3.470851in}}%
\pgfpathlineto{\pgfqpoint{1.009926in}{3.418984in}}%
\pgfpathlineto{\pgfqpoint{1.019705in}{3.368762in}}%
\pgfpathlineto{\pgfqpoint{1.029484in}{3.320108in}}%
\pgfpathlineto{\pgfqpoint{1.039264in}{3.272950in}}%
\pgfpathlineto{\pgfqpoint{1.049043in}{3.227219in}}%
\pgfpathlineto{\pgfqpoint{1.058822in}{3.182852in}}%
\pgfpathlineto{\pgfqpoint{1.068602in}{3.139788in}}%
\pgfpathlineto{\pgfqpoint{1.078381in}{3.097972in}}%
\pgfpathlineto{\pgfqpoint{1.088160in}{3.057349in}}%
\pgfpathlineto{\pgfqpoint{1.097940in}{3.017869in}}%
\pgfpathlineto{\pgfqpoint{1.107719in}{2.979486in}}%
\pgfpathlineto{\pgfqpoint{1.117498in}{2.942152in}}%
\pgfpathlineto{\pgfqpoint{1.127278in}{2.905827in}}%
\pgfpathlineto{\pgfqpoint{1.137057in}{2.870470in}}%
\pgfpathlineto{\pgfqpoint{1.151726in}{2.819166in}}%
\pgfpathlineto{\pgfqpoint{1.166395in}{2.769834in}}%
\pgfpathlineto{\pgfqpoint{1.181064in}{2.722361in}}%
\pgfpathlineto{\pgfqpoint{1.195733in}{2.676645in}}%
\pgfpathlineto{\pgfqpoint{1.210402in}{2.632591in}}%
\pgfpathlineto{\pgfqpoint{1.225071in}{2.590108in}}%
\pgfpathlineto{\pgfqpoint{1.239740in}{2.549116in}}%
\pgfpathlineto{\pgfqpoint{1.254409in}{2.509535in}}%
\pgfpathlineto{\pgfqpoint{1.269078in}{2.471295in}}%
\pgfpathlineto{\pgfqpoint{1.283747in}{2.434329in}}%
\pgfpathlineto{\pgfqpoint{1.298416in}{2.398575in}}%
\pgfpathlineto{\pgfqpoint{1.313085in}{2.363972in}}%
\pgfpathlineto{\pgfqpoint{1.327754in}{2.330468in}}%
\pgfpathlineto{\pgfqpoint{1.342423in}{2.298010in}}%
\pgfpathlineto{\pgfqpoint{1.357092in}{2.266549in}}%
\pgfpathlineto{\pgfqpoint{1.371761in}{2.236042in}}%
\pgfpathlineto{\pgfqpoint{1.386430in}{2.206445in}}%
\pgfpathlineto{\pgfqpoint{1.401099in}{2.177717in}}%
\pgfpathlineto{\pgfqpoint{1.415768in}{2.149822in}}%
\pgfpathlineto{\pgfqpoint{1.430437in}{2.122723in}}%
\pgfpathlineto{\pgfqpoint{1.445106in}{2.096387in}}%
\pgfpathlineto{\pgfqpoint{1.459775in}{2.070782in}}%
\pgfpathlineto{\pgfqpoint{1.474444in}{2.045879in}}%
\pgfpathlineto{\pgfqpoint{1.489113in}{2.021648in}}%
\pgfpathlineto{\pgfqpoint{1.503782in}{1.998063in}}%
\pgfpathlineto{\pgfqpoint{1.518451in}{1.975098in}}%
\pgfpathlineto{\pgfqpoint{1.533120in}{1.952729in}}%
\pgfpathlineto{\pgfqpoint{1.547789in}{1.930933in}}%
\pgfpathlineto{\pgfqpoint{1.562458in}{1.909689in}}%
\pgfpathlineto{\pgfqpoint{1.577128in}{1.888976in}}%
\pgfpathlineto{\pgfqpoint{1.591797in}{1.868774in}}%
\pgfpathlineto{\pgfqpoint{1.606466in}{1.849064in}}%
\pgfpathlineto{\pgfqpoint{1.621135in}{1.829830in}}%
\pgfpathlineto{\pgfqpoint{1.640693in}{1.804892in}}%
\pgfpathlineto{\pgfqpoint{1.660252in}{1.780731in}}%
\pgfpathlineto{\pgfqpoint{1.679811in}{1.757309in}}%
\pgfpathlineto{\pgfqpoint{1.699369in}{1.734595in}}%
\pgfpathlineto{\pgfqpoint{1.718928in}{1.712556in}}%
\pgfpathlineto{\pgfqpoint{1.738487in}{1.691162in}}%
\pgfpathlineto{\pgfqpoint{1.758045in}{1.670386in}}%
\pgfpathlineto{\pgfqpoint{1.777604in}{1.650201in}}%
\pgfpathlineto{\pgfqpoint{1.797163in}{1.630582in}}%
\pgfpathlineto{\pgfqpoint{1.816721in}{1.611506in}}%
\pgfpathlineto{\pgfqpoint{1.836280in}{1.592951in}}%
\pgfpathlineto{\pgfqpoint{1.855839in}{1.574895in}}%
\pgfpathlineto{\pgfqpoint{1.875397in}{1.557319in}}%
\pgfpathlineto{\pgfqpoint{1.894956in}{1.540204in}}%
\pgfpathlineto{\pgfqpoint{1.914515in}{1.523532in}}%
\pgfpathlineto{\pgfqpoint{1.934073in}{1.507286in}}%
\pgfpathlineto{\pgfqpoint{1.953632in}{1.491449in}}%
\pgfpathlineto{\pgfqpoint{1.973191in}{1.476007in}}%
\pgfpathlineto{\pgfqpoint{1.997639in}{1.457237in}}%
\pgfpathlineto{\pgfqpoint{2.022088in}{1.439035in}}%
\pgfpathlineto{\pgfqpoint{2.046536in}{1.421373in}}%
\pgfpathlineto{\pgfqpoint{2.070984in}{1.404230in}}%
\pgfpathlineto{\pgfqpoint{2.095433in}{1.387581in}}%
\pgfpathlineto{\pgfqpoint{2.119881in}{1.371407in}}%
\pgfpathlineto{\pgfqpoint{2.144329in}{1.355687in}}%
\pgfpathlineto{\pgfqpoint{2.168778in}{1.340402in}}%
\pgfpathlineto{\pgfqpoint{2.193226in}{1.325535in}}%
\pgfpathlineto{\pgfqpoint{2.217674in}{1.311068in}}%
\pgfpathlineto{\pgfqpoint{2.242123in}{1.296986in}}%
\pgfpathlineto{\pgfqpoint{2.266571in}{1.283274in}}%
\pgfpathlineto{\pgfqpoint{2.295909in}{1.267286in}}%
\pgfpathlineto{\pgfqpoint{2.325247in}{1.251787in}}%
\pgfpathlineto{\pgfqpoint{2.354585in}{1.236753in}}%
\pgfpathlineto{\pgfqpoint{2.383923in}{1.222165in}}%
\pgfpathlineto{\pgfqpoint{2.413261in}{1.208003in}}%
\pgfpathlineto{\pgfqpoint{2.442599in}{1.194248in}}%
\pgfpathlineto{\pgfqpoint{2.471937in}{1.180883in}}%
\pgfpathlineto{\pgfqpoint{2.501275in}{1.167891in}}%
\pgfpathlineto{\pgfqpoint{2.530613in}{1.155259in}}%
\pgfpathlineto{\pgfqpoint{2.564841in}{1.140954in}}%
\pgfpathlineto{\pgfqpoint{2.599069in}{1.127094in}}%
\pgfpathlineto{\pgfqpoint{2.633296in}{1.113661in}}%
\pgfpathlineto{\pgfqpoint{2.667524in}{1.100633in}}%
\pgfpathlineto{\pgfqpoint{2.701752in}{1.087993in}}%
\pgfpathlineto{\pgfqpoint{2.735979in}{1.075724in}}%
\pgfpathlineto{\pgfqpoint{2.770207in}{1.063810in}}%
\pgfpathlineto{\pgfqpoint{2.809324in}{1.050609in}}%
\pgfpathlineto{\pgfqpoint{2.848442in}{1.037831in}}%
\pgfpathlineto{\pgfqpoint{2.887559in}{1.025454in}}%
\pgfpathlineto{\pgfqpoint{2.926677in}{1.013462in}}%
\pgfpathlineto{\pgfqpoint{2.965794in}{1.001835in}}%
\pgfpathlineto{\pgfqpoint{3.009801in}{0.989172in}}%
\pgfpathlineto{\pgfqpoint{3.053808in}{0.976930in}}%
\pgfpathlineto{\pgfqpoint{3.097815in}{0.965087in}}%
\pgfpathlineto{\pgfqpoint{3.141822in}{0.953625in}}%
\pgfpathlineto{\pgfqpoint{3.185829in}{0.942526in}}%
\pgfpathlineto{\pgfqpoint{3.234726in}{0.930598in}}%
\pgfpathlineto{\pgfqpoint{3.283622in}{0.919075in}}%
\pgfpathlineto{\pgfqpoint{3.332519in}{0.907936in}}%
\pgfpathlineto{\pgfqpoint{3.381416in}{0.897163in}}%
\pgfpathlineto{\pgfqpoint{3.435202in}{0.885714in}}%
\pgfpathlineto{\pgfqpoint{3.488989in}{0.874665in}}%
\pgfpathlineto{\pgfqpoint{3.542775in}{0.863994in}}%
\pgfpathlineto{\pgfqpoint{3.596561in}{0.853682in}}%
\pgfpathlineto{\pgfqpoint{3.655237in}{0.842822in}}%
\pgfpathlineto{\pgfqpoint{3.713913in}{0.832346in}}%
\pgfpathlineto{\pgfqpoint{3.772590in}{0.822236in}}%
\pgfpathlineto{\pgfqpoint{3.836155in}{0.811673in}}%
\pgfpathlineto{\pgfqpoint{3.899721in}{0.801494in}}%
\pgfpathlineto{\pgfqpoint{3.963287in}{0.791679in}}%
\pgfpathlineto{\pgfqpoint{4.031742in}{0.781493in}}%
\pgfpathlineto{\pgfqpoint{4.100197in}{0.771684in}}%
\pgfpathlineto{\pgfqpoint{4.168653in}{0.762232in}}%
\pgfpathlineto{\pgfqpoint{4.241998in}{0.752479in}}%
\pgfpathlineto{\pgfqpoint{4.315343in}{0.743091in}}%
\pgfpathlineto{\pgfqpoint{4.393578in}{0.733458in}}%
\pgfpathlineto{\pgfqpoint{4.471812in}{0.724196in}}%
\pgfpathlineto{\pgfqpoint{4.554937in}{0.714737in}}%
\pgfpathlineto{\pgfqpoint{4.638061in}{0.705649in}}%
\pgfpathlineto{\pgfqpoint{4.726075in}{0.696408in}}%
\pgfpathlineto{\pgfqpoint{4.814089in}{0.687535in}}%
\pgfpathlineto{\pgfqpoint{4.906993in}{0.678546in}}%
\pgfpathlineto{\pgfqpoint{4.999897in}{0.669920in}}%
\pgfpathlineto{\pgfqpoint{5.097690in}{0.661209in}}%
\pgfpathlineto{\pgfqpoint{5.200373in}{0.652445in}}%
\pgfpathlineto{\pgfqpoint{5.303056in}{0.644048in}}%
\pgfpathlineto{\pgfqpoint{5.410629in}{0.635620in}}%
\pgfpathlineto{\pgfqpoint{5.523092in}{0.627188in}}%
\pgfpathlineto{\pgfqpoint{5.640444in}{0.618775in}}%
\pgfpathlineto{\pgfqpoint{5.757796in}{0.610732in}}%
\pgfpathlineto{\pgfqpoint{5.799882in}{0.608078in}}%
\pgfpathlineto{\pgfqpoint{5.840265in}{0.605943in}}%
\pgfpathlineto{\pgfqpoint{5.888725in}{0.603773in}}%
\pgfpathlineto{\pgfqpoint{5.945261in}{0.601637in}}%
\pgfpathlineto{\pgfqpoint{6.009874in}{0.599578in}}%
\pgfpathlineto{\pgfqpoint{6.082564in}{0.597620in}}%
\pgfpathlineto{\pgfqpoint{6.171407in}{0.595609in}}%
\pgfpathlineto{\pgfqpoint{6.276403in}{0.593628in}}%
\pgfpathlineto{\pgfqpoint{6.397552in}{0.591734in}}%
\pgfpathlineto{\pgfqpoint{6.542932in}{0.589861in}}%
\pgfpathlineto{\pgfqpoint{6.567162in}{0.589583in}}%
\pgfpathlineto{\pgfqpoint{6.567162in}{0.589583in}}%
\pgfusepath{stroke}%
\end{pgfscope}%
\begin{pgfscope}%
\pgfsetrectcap%
\pgfsetmiterjoin%
\pgfsetlinewidth{0.803000pt}%
\definecolor{currentstroke}{rgb}{0.000000,0.000000,0.000000}%
\pgfsetstrokecolor{currentstroke}%
\pgfsetdash{}{0pt}%
\pgfpathmoveto{\pgfqpoint{0.882794in}{0.589583in}}%
\pgfpathlineto{\pgfqpoint{0.882794in}{5.059445in}}%
\pgfusepath{stroke}%
\end{pgfscope}%
\begin{pgfscope}%
\pgfsetrectcap%
\pgfsetmiterjoin%
\pgfsetlinewidth{0.803000pt}%
\definecolor{currentstroke}{rgb}{0.000000,0.000000,0.000000}%
\pgfsetstrokecolor{currentstroke}%
\pgfsetdash{}{0pt}%
\pgfpathmoveto{\pgfqpoint{7.800000in}{0.589583in}}%
\pgfpathlineto{\pgfqpoint{7.800000in}{5.059445in}}%
\pgfusepath{stroke}%
\end{pgfscope}%
\begin{pgfscope}%
\pgfsetrectcap%
\pgfsetmiterjoin%
\pgfsetlinewidth{0.803000pt}%
\definecolor{currentstroke}{rgb}{0.000000,0.000000,0.000000}%
\pgfsetstrokecolor{currentstroke}%
\pgfsetdash{}{0pt}%
\pgfpathmoveto{\pgfqpoint{0.882794in}{0.589583in}}%
\pgfpathlineto{\pgfqpoint{7.800000in}{0.589583in}}%
\pgfusepath{stroke}%
\end{pgfscope}%
\begin{pgfscope}%
\pgfsetrectcap%
\pgfsetmiterjoin%
\pgfsetlinewidth{0.803000pt}%
\definecolor{currentstroke}{rgb}{0.000000,0.000000,0.000000}%
\pgfsetstrokecolor{currentstroke}%
\pgfsetdash{}{0pt}%
\pgfpathmoveto{\pgfqpoint{0.882794in}{5.059445in}}%
\pgfpathlineto{\pgfqpoint{7.800000in}{5.059445in}}%
\pgfusepath{stroke}%
\end{pgfscope}%
\begin{pgfscope}%
\definecolor{textcolor}{rgb}{0.000000,0.000000,0.000000}%
\pgfsetstrokecolor{textcolor}%
\pgfsetfillcolor{textcolor}%
\pgftext[x=4.341397in,y=5.142779in,,base]{\color{textcolor}\rmfamily\fontsize{20.000000}{24.000000}\selectfont Objective Space}%
\end{pgfscope}%
\begin{pgfscope}%
\pgfsetbuttcap%
\pgfsetmiterjoin%
\definecolor{currentfill}{rgb}{1.000000,1.000000,1.000000}%
\pgfsetfillcolor{currentfill}%
\pgfsetfillopacity{0.800000}%
\pgfsetlinewidth{1.003750pt}%
\definecolor{currentstroke}{rgb}{0.800000,0.800000,0.800000}%
\pgfsetstrokecolor{currentstroke}%
\pgfsetstrokeopacity{0.800000}%
\pgfsetdash{}{0pt}%
\pgfpathmoveto{\pgfqpoint{4.495358in}{4.047310in}}%
\pgfpathlineto{\pgfqpoint{7.605556in}{4.047310in}}%
\pgfpathquadraticcurveto{\pgfqpoint{7.661111in}{4.047310in}}{\pgfqpoint{7.661111in}{4.102866in}}%
\pgfpathlineto{\pgfqpoint{7.661111in}{4.865001in}}%
\pgfpathquadraticcurveto{\pgfqpoint{7.661111in}{4.920557in}}{\pgfqpoint{7.605556in}{4.920557in}}%
\pgfpathlineto{\pgfqpoint{4.495358in}{4.920557in}}%
\pgfpathquadraticcurveto{\pgfqpoint{4.439802in}{4.920557in}}{\pgfqpoint{4.439802in}{4.865001in}}%
\pgfpathlineto{\pgfqpoint{4.439802in}{4.102866in}}%
\pgfpathquadraticcurveto{\pgfqpoint{4.439802in}{4.047310in}}{\pgfqpoint{4.495358in}{4.047310in}}%
\pgfpathlineto{\pgfqpoint{4.495358in}{4.047310in}}%
\pgfpathclose%
\pgfusepath{stroke,fill}%
\end{pgfscope}%
\begin{pgfscope}%
\pgfsetrectcap%
\pgfsetroundjoin%
\pgfsetlinewidth{3.011250pt}%
\definecolor{currentstroke}{rgb}{0.000000,0.000000,0.000000}%
\pgfsetstrokecolor{currentstroke}%
\pgfsetdash{}{0pt}%
\pgfpathmoveto{\pgfqpoint{4.550914in}{4.706629in}}%
\pgfpathlineto{\pgfqpoint{4.828691in}{4.706629in}}%
\pgfpathlineto{\pgfqpoint{5.106469in}{4.706629in}}%
\pgfusepath{stroke}%
\end{pgfscope}%
\begin{pgfscope}%
\definecolor{textcolor}{rgb}{0.000000,0.000000,0.000000}%
\pgfsetstrokecolor{textcolor}%
\pgfsetfillcolor{textcolor}%
\pgftext[x=5.328691in,y=4.609407in,left,base]{\color{textcolor}\rmfamily\fontsize{20.000000}{24.000000}\selectfont Pareto-front}%
\end{pgfscope}%
\begin{pgfscope}%
\pgfsetbuttcap%
\pgfsetmiterjoin%
\definecolor{currentfill}{rgb}{0.827451,0.827451,0.827451}%
\pgfsetfillcolor{currentfill}%
\pgfsetfillopacity{0.500000}%
\pgfsetlinewidth{0.000000pt}%
\definecolor{currentstroke}{rgb}{0.000000,0.000000,0.000000}%
\pgfsetstrokecolor{currentstroke}%
\pgfsetstrokeopacity{0.500000}%
\pgfsetdash{}{0pt}%
\pgfpathmoveto{\pgfqpoint{4.550914in}{4.214450in}}%
\pgfpathlineto{\pgfqpoint{5.106469in}{4.214450in}}%
\pgfpathlineto{\pgfqpoint{5.106469in}{4.408895in}}%
\pgfpathlineto{\pgfqpoint{4.550914in}{4.408895in}}%
\pgfpathlineto{\pgfqpoint{4.550914in}{4.214450in}}%
\pgfpathclose%
\pgfusepath{fill}%
\end{pgfscope}%
\begin{pgfscope}%
\definecolor{textcolor}{rgb}{0.000000,0.000000,0.000000}%
\pgfsetstrokecolor{textcolor}%
\pgfsetfillcolor{textcolor}%
\pgftext[x=5.328691in,y=4.214450in,left,base]{\color{textcolor}\rmfamily\fontsize{20.000000}{24.000000}\selectfont Near-optimal space}%
\end{pgfscope}%
\begin{pgfscope}%
\definecolor{textcolor}{rgb}{0.000000,0.000000,0.000000}%
\pgfsetstrokecolor{textcolor}%
\pgfsetfillcolor{textcolor}%
\pgftext[x=3.950000in,y=5.830000in,,top]{\color{textcolor}\rmfamily\fontsize{24.000000}{28.800000}\selectfont Multi-objective MGA}%
\end{pgfscope}%
\end{pgfpicture}%
\makeatother%
\endgroup%
}
            \caption{Near optimal space for a multi-objective problem.}
            \label{fig:near-opt}
        \end{figure}
    \end{columns}

\end{frame}

\begin{frame}
    \frametitle{How \texttt{Osier} handles structural uncertainty}

    \begin{columns}
        \column[t]{4cm}

        \column[t]{6cm}
        \begin{figure}
            \centering
            \resizebox{\columnwidth}{!}{%% Creator: Matplotlib, PGF backend
%%
%% To include the figure in your LaTeX document, write
%%   \input{<filename>.pgf}
%%
%% Make sure the required packages are loaded in your preamble
%%   \usepackage{pgf}
%%
%% Also ensure that all the required font packages are loaded; for instance,
%% the lmodern package is sometimes necessary when using math font.
%%   \usepackage{lmodern}
%%
%% Figures using additional raster images can only be included by \input if
%% they are in the same directory as the main LaTeX file. For loading figures
%% from other directories you can use the `import` package
%%   \usepackage{import}
%%
%% and then include the figures with
%%   \import{<path to file>}{<filename>.pgf}
%%
%% Matplotlib used the following preamble
%%   
%%   \makeatletter\@ifpackageloaded{underscore}{}{\usepackage[strings]{underscore}}\makeatother
%%
\begingroup%
\makeatletter%
\begin{pgfpicture}%
\pgfpathrectangle{\pgfpointorigin}{\pgfqpoint{7.900000in}{5.930000in}}%
\pgfusepath{use as bounding box, clip}%
\begin{pgfscope}%
\pgfsetbuttcap%
\pgfsetmiterjoin%
\definecolor{currentfill}{rgb}{0.827451,0.827451,0.827451}%
\pgfsetfillcolor{currentfill}%
\pgfsetlinewidth{0.000000pt}%
\definecolor{currentstroke}{rgb}{0.000000,0.000000,0.000000}%
\pgfsetstrokecolor{currentstroke}%
\pgfsetdash{}{0pt}%
\pgfpathmoveto{\pgfqpoint{0.000000in}{0.000000in}}%
\pgfpathlineto{\pgfqpoint{7.900000in}{0.000000in}}%
\pgfpathlineto{\pgfqpoint{7.900000in}{5.930000in}}%
\pgfpathlineto{\pgfqpoint{0.000000in}{5.930000in}}%
\pgfpathlineto{\pgfqpoint{0.000000in}{0.000000in}}%
\pgfpathclose%
\pgfusepath{fill}%
\end{pgfscope}%
\begin{pgfscope}%
\pgfsetbuttcap%
\pgfsetmiterjoin%
\definecolor{currentfill}{rgb}{1.000000,1.000000,1.000000}%
\pgfsetfillcolor{currentfill}%
\pgfsetlinewidth{0.000000pt}%
\definecolor{currentstroke}{rgb}{0.000000,0.000000,0.000000}%
\pgfsetstrokecolor{currentstroke}%
\pgfsetstrokeopacity{0.000000}%
\pgfsetdash{}{0pt}%
\pgfpathmoveto{\pgfqpoint{0.882794in}{0.589583in}}%
\pgfpathlineto{\pgfqpoint{7.800000in}{0.589583in}}%
\pgfpathlineto{\pgfqpoint{7.800000in}{5.059445in}}%
\pgfpathlineto{\pgfqpoint{0.882794in}{5.059445in}}%
\pgfpathlineto{\pgfqpoint{0.882794in}{0.589583in}}%
\pgfpathclose%
\pgfusepath{fill}%
\end{pgfscope}%
\begin{pgfscope}%
\pgfpathrectangle{\pgfqpoint{0.882794in}{0.589583in}}{\pgfqpoint{6.917206in}{4.469862in}}%
\pgfusepath{clip}%
\pgfsetbuttcap%
\pgfsetmiterjoin%
\definecolor{currentfill}{rgb}{0.827451,0.827451,0.827451}%
\pgfsetfillcolor{currentfill}%
\pgfsetfillopacity{0.500000}%
\pgfsetlinewidth{0.000000pt}%
\definecolor{currentstroke}{rgb}{0.000000,0.000000,0.000000}%
\pgfsetstrokecolor{currentstroke}%
\pgfsetstrokeopacity{0.500000}%
\pgfsetdash{}{0pt}%
\pgfpathmoveto{\pgfqpoint{0.882794in}{3.480739in}}%
\pgfpathlineto{\pgfqpoint{0.887684in}{3.448888in}}%
\pgfpathlineto{\pgfqpoint{0.892574in}{3.417673in}}%
\pgfpathlineto{\pgfqpoint{0.897463in}{3.387075in}}%
\pgfpathlineto{\pgfqpoint{0.902353in}{3.357077in}}%
\pgfpathlineto{\pgfqpoint{0.907243in}{3.327660in}}%
\pgfpathlineto{\pgfqpoint{0.912132in}{3.298809in}}%
\pgfpathlineto{\pgfqpoint{0.917022in}{3.270506in}}%
\pgfpathlineto{\pgfqpoint{0.921912in}{3.242736in}}%
\pgfpathlineto{\pgfqpoint{0.926801in}{3.215485in}}%
\pgfpathlineto{\pgfqpoint{0.931691in}{3.188738in}}%
\pgfpathlineto{\pgfqpoint{0.936581in}{3.162482in}}%
\pgfpathlineto{\pgfqpoint{0.941470in}{3.136702in}}%
\pgfpathlineto{\pgfqpoint{0.946360in}{3.111386in}}%
\pgfpathlineto{\pgfqpoint{0.951250in}{3.086521in}}%
\pgfpathlineto{\pgfqpoint{0.956139in}{3.062096in}}%
\pgfpathlineto{\pgfqpoint{0.961029in}{3.038100in}}%
\pgfpathlineto{\pgfqpoint{0.965919in}{3.014520in}}%
\pgfpathlineto{\pgfqpoint{0.970808in}{2.991346in}}%
\pgfpathlineto{\pgfqpoint{0.975698in}{2.968567in}}%
\pgfpathlineto{\pgfqpoint{0.980588in}{2.946175in}}%
\pgfpathlineto{\pgfqpoint{0.985477in}{2.924158in}}%
\pgfpathlineto{\pgfqpoint{0.990367in}{2.902508in}}%
\pgfpathlineto{\pgfqpoint{0.995257in}{2.881216in}}%
\pgfpathlineto{\pgfqpoint{1.000146in}{2.860272in}}%
\pgfpathlineto{\pgfqpoint{1.005036in}{2.839668in}}%
\pgfpathlineto{\pgfqpoint{1.009926in}{2.819396in}}%
\pgfpathlineto{\pgfqpoint{1.014815in}{2.799448in}}%
\pgfpathlineto{\pgfqpoint{1.019705in}{2.779817in}}%
\pgfpathlineto{\pgfqpoint{1.024595in}{2.760494in}}%
\pgfpathlineto{\pgfqpoint{1.029484in}{2.741473in}}%
\pgfpathlineto{\pgfqpoint{1.034374in}{2.722747in}}%
\pgfpathlineto{\pgfqpoint{1.039264in}{2.704308in}}%
\pgfpathlineto{\pgfqpoint{1.044153in}{2.686151in}}%
\pgfpathlineto{\pgfqpoint{1.049043in}{2.668268in}}%
\pgfpathlineto{\pgfqpoint{1.053933in}{2.650654in}}%
\pgfpathlineto{\pgfqpoint{1.058822in}{2.633303in}}%
\pgfpathlineto{\pgfqpoint{1.063712in}{2.616209in}}%
\pgfpathlineto{\pgfqpoint{1.068602in}{2.599366in}}%
\pgfpathlineto{\pgfqpoint{1.073491in}{2.582768in}}%
\pgfpathlineto{\pgfqpoint{1.078381in}{2.566411in}}%
\pgfpathlineto{\pgfqpoint{1.083271in}{2.550288in}}%
\pgfpathlineto{\pgfqpoint{1.088160in}{2.534396in}}%
\pgfpathlineto{\pgfqpoint{1.093050in}{2.518729in}}%
\pgfpathlineto{\pgfqpoint{1.097940in}{2.503283in}}%
\pgfpathlineto{\pgfqpoint{1.102829in}{2.488052in}}%
\pgfpathlineto{\pgfqpoint{1.107719in}{2.473033in}}%
\pgfpathlineto{\pgfqpoint{1.112609in}{2.458221in}}%
\pgfpathlineto{\pgfqpoint{1.117498in}{2.443611in}}%
\pgfpathlineto{\pgfqpoint{1.122388in}{2.429201in}}%
\pgfpathlineto{\pgfqpoint{1.127278in}{2.414984in}}%
\pgfpathlineto{\pgfqpoint{1.132167in}{2.400958in}}%
\pgfpathlineto{\pgfqpoint{1.137057in}{2.387120in}}%
\pgfpathlineto{\pgfqpoint{1.141947in}{2.373464in}}%
\pgfpathlineto{\pgfqpoint{1.146837in}{2.359988in}}%
\pgfpathlineto{\pgfqpoint{1.151726in}{2.346688in}}%
\pgfpathlineto{\pgfqpoint{1.156616in}{2.333560in}}%
\pgfpathlineto{\pgfqpoint{1.161506in}{2.320602in}}%
\pgfpathlineto{\pgfqpoint{1.166395in}{2.307809in}}%
\pgfpathlineto{\pgfqpoint{1.171285in}{2.295180in}}%
\pgfpathlineto{\pgfqpoint{1.176175in}{2.282710in}}%
\pgfpathlineto{\pgfqpoint{1.181064in}{2.270397in}}%
\pgfpathlineto{\pgfqpoint{1.185954in}{2.258238in}}%
\pgfpathlineto{\pgfqpoint{1.190844in}{2.246229in}}%
\pgfpathlineto{\pgfqpoint{1.195733in}{2.234369in}}%
\pgfpathlineto{\pgfqpoint{1.200623in}{2.222654in}}%
\pgfpathlineto{\pgfqpoint{1.205513in}{2.211082in}}%
\pgfpathlineto{\pgfqpoint{1.210402in}{2.199650in}}%
\pgfpathlineto{\pgfqpoint{1.215292in}{2.188356in}}%
\pgfpathlineto{\pgfqpoint{1.220182in}{2.177197in}}%
\pgfpathlineto{\pgfqpoint{1.225071in}{2.166170in}}%
\pgfpathlineto{\pgfqpoint{1.229961in}{2.155274in}}%
\pgfpathlineto{\pgfqpoint{1.234851in}{2.144506in}}%
\pgfpathlineto{\pgfqpoint{1.239740in}{2.133864in}}%
\pgfpathlineto{\pgfqpoint{1.244630in}{2.123346in}}%
\pgfpathlineto{\pgfqpoint{1.249520in}{2.112949in}}%
\pgfpathlineto{\pgfqpoint{1.254409in}{2.102672in}}%
\pgfpathlineto{\pgfqpoint{1.259299in}{2.092511in}}%
\pgfpathlineto{\pgfqpoint{1.264189in}{2.082467in}}%
\pgfpathlineto{\pgfqpoint{1.269078in}{2.072535in}}%
\pgfpathlineto{\pgfqpoint{1.273968in}{2.062716in}}%
\pgfpathlineto{\pgfqpoint{1.278858in}{2.053005in}}%
\pgfpathlineto{\pgfqpoint{1.283747in}{2.043403in}}%
\pgfpathlineto{\pgfqpoint{1.288637in}{2.033907in}}%
\pgfpathlineto{\pgfqpoint{1.293527in}{2.024515in}}%
\pgfpathlineto{\pgfqpoint{1.298416in}{2.015225in}}%
\pgfpathlineto{\pgfqpoint{1.303306in}{2.006036in}}%
\pgfpathlineto{\pgfqpoint{1.308196in}{1.996947in}}%
\pgfpathlineto{\pgfqpoint{1.313085in}{1.987955in}}%
\pgfpathlineto{\pgfqpoint{1.317975in}{1.979060in}}%
\pgfpathlineto{\pgfqpoint{1.322865in}{1.970259in}}%
\pgfpathlineto{\pgfqpoint{1.327754in}{1.961551in}}%
\pgfpathlineto{\pgfqpoint{1.332644in}{1.952935in}}%
\pgfpathlineto{\pgfqpoint{1.337534in}{1.944409in}}%
\pgfpathlineto{\pgfqpoint{1.342423in}{1.935971in}}%
\pgfpathlineto{\pgfqpoint{1.347313in}{1.927621in}}%
\pgfpathlineto{\pgfqpoint{1.352203in}{1.919357in}}%
\pgfpathlineto{\pgfqpoint{1.357092in}{1.911178in}}%
\pgfpathlineto{\pgfqpoint{1.361982in}{1.903082in}}%
\pgfpathlineto{\pgfqpoint{1.366872in}{1.895068in}}%
\pgfpathlineto{\pgfqpoint{1.371761in}{1.887135in}}%
\pgfpathlineto{\pgfqpoint{1.376651in}{1.879282in}}%
\pgfpathlineto{\pgfqpoint{1.381541in}{1.871507in}}%
\pgfpathlineto{\pgfqpoint{1.386430in}{1.863810in}}%
\pgfpathlineto{\pgfqpoint{1.391320in}{1.856189in}}%
\pgfpathlineto{\pgfqpoint{1.396210in}{1.848643in}}%
\pgfpathlineto{\pgfqpoint{1.401099in}{1.841170in}}%
\pgfpathlineto{\pgfqpoint{1.405989in}{1.833771in}}%
\pgfpathlineto{\pgfqpoint{1.410879in}{1.826443in}}%
\pgfpathlineto{\pgfqpoint{1.415768in}{1.819186in}}%
\pgfpathlineto{\pgfqpoint{1.420658in}{1.811999in}}%
\pgfpathlineto{\pgfqpoint{1.425548in}{1.804881in}}%
\pgfpathlineto{\pgfqpoint{1.430437in}{1.797830in}}%
\pgfpathlineto{\pgfqpoint{1.435327in}{1.790846in}}%
\pgfpathlineto{\pgfqpoint{1.440217in}{1.783928in}}%
\pgfpathlineto{\pgfqpoint{1.445106in}{1.777075in}}%
\pgfpathlineto{\pgfqpoint{1.449996in}{1.770286in}}%
\pgfpathlineto{\pgfqpoint{1.454886in}{1.763560in}}%
\pgfpathlineto{\pgfqpoint{1.459775in}{1.756896in}}%
\pgfpathlineto{\pgfqpoint{1.464665in}{1.750294in}}%
\pgfpathlineto{\pgfqpoint{1.469555in}{1.743752in}}%
\pgfpathlineto{\pgfqpoint{1.474444in}{1.737270in}}%
\pgfpathlineto{\pgfqpoint{1.479334in}{1.730847in}}%
\pgfpathlineto{\pgfqpoint{1.484224in}{1.724482in}}%
\pgfpathlineto{\pgfqpoint{1.489113in}{1.718174in}}%
\pgfpathlineto{\pgfqpoint{1.494003in}{1.711923in}}%
\pgfpathlineto{\pgfqpoint{1.498893in}{1.705727in}}%
\pgfpathlineto{\pgfqpoint{1.503782in}{1.699587in}}%
\pgfpathlineto{\pgfqpoint{1.508672in}{1.693501in}}%
\pgfpathlineto{\pgfqpoint{1.513562in}{1.687468in}}%
\pgfpathlineto{\pgfqpoint{1.518451in}{1.681489in}}%
\pgfpathlineto{\pgfqpoint{1.523341in}{1.675561in}}%
\pgfpathlineto{\pgfqpoint{1.528231in}{1.669685in}}%
\pgfpathlineto{\pgfqpoint{1.533120in}{1.663860in}}%
\pgfpathlineto{\pgfqpoint{1.538010in}{1.658085in}}%
\pgfpathlineto{\pgfqpoint{1.542900in}{1.652360in}}%
\pgfpathlineto{\pgfqpoint{1.547789in}{1.646683in}}%
\pgfpathlineto{\pgfqpoint{1.552679in}{1.641055in}}%
\pgfpathlineto{\pgfqpoint{1.557569in}{1.635475in}}%
\pgfpathlineto{\pgfqpoint{1.562458in}{1.629941in}}%
\pgfpathlineto{\pgfqpoint{1.567348in}{1.624454in}}%
\pgfpathlineto{\pgfqpoint{1.572238in}{1.619013in}}%
\pgfpathlineto{\pgfqpoint{1.577128in}{1.613617in}}%
\pgfpathlineto{\pgfqpoint{1.582017in}{1.608266in}}%
\pgfpathlineto{\pgfqpoint{1.586907in}{1.602959in}}%
\pgfpathlineto{\pgfqpoint{1.591797in}{1.597696in}}%
\pgfpathlineto{\pgfqpoint{1.596686in}{1.592476in}}%
\pgfpathlineto{\pgfqpoint{1.601576in}{1.587299in}}%
\pgfpathlineto{\pgfqpoint{1.606466in}{1.582163in}}%
\pgfpathlineto{\pgfqpoint{1.611355in}{1.577070in}}%
\pgfpathlineto{\pgfqpoint{1.616245in}{1.572017in}}%
\pgfpathlineto{\pgfqpoint{1.621135in}{1.567005in}}%
\pgfpathlineto{\pgfqpoint{1.626024in}{1.562033in}}%
\pgfpathlineto{\pgfqpoint{1.630914in}{1.557100in}}%
\pgfpathlineto{\pgfqpoint{1.635804in}{1.552207in}}%
\pgfpathlineto{\pgfqpoint{1.640693in}{1.547352in}}%
\pgfpathlineto{\pgfqpoint{1.645583in}{1.542535in}}%
\pgfpathlineto{\pgfqpoint{1.650473in}{1.537757in}}%
\pgfpathlineto{\pgfqpoint{1.655362in}{1.533015in}}%
\pgfpathlineto{\pgfqpoint{1.660252in}{1.528311in}}%
\pgfpathlineto{\pgfqpoint{1.665142in}{1.523642in}}%
\pgfpathlineto{\pgfqpoint{1.670031in}{1.519010in}}%
\pgfpathlineto{\pgfqpoint{1.674921in}{1.514414in}}%
\pgfpathlineto{\pgfqpoint{1.679811in}{1.509853in}}%
\pgfpathlineto{\pgfqpoint{1.684700in}{1.505326in}}%
\pgfpathlineto{\pgfqpoint{1.689590in}{1.500834in}}%
\pgfpathlineto{\pgfqpoint{1.694480in}{1.496376in}}%
\pgfpathlineto{\pgfqpoint{1.699369in}{1.491952in}}%
\pgfpathlineto{\pgfqpoint{1.704259in}{1.487560in}}%
\pgfpathlineto{\pgfqpoint{1.709149in}{1.483202in}}%
\pgfpathlineto{\pgfqpoint{1.714038in}{1.478876in}}%
\pgfpathlineto{\pgfqpoint{1.718928in}{1.474583in}}%
\pgfpathlineto{\pgfqpoint{1.723818in}{1.470321in}}%
\pgfpathlineto{\pgfqpoint{1.728707in}{1.466090in}}%
\pgfpathlineto{\pgfqpoint{1.733597in}{1.461891in}}%
\pgfpathlineto{\pgfqpoint{1.738487in}{1.457722in}}%
\pgfpathlineto{\pgfqpoint{1.743376in}{1.453584in}}%
\pgfpathlineto{\pgfqpoint{1.748266in}{1.449476in}}%
\pgfpathlineto{\pgfqpoint{1.753156in}{1.445398in}}%
\pgfpathlineto{\pgfqpoint{1.758045in}{1.441349in}}%
\pgfpathlineto{\pgfqpoint{1.762935in}{1.437329in}}%
\pgfpathlineto{\pgfqpoint{1.767825in}{1.433338in}}%
\pgfpathlineto{\pgfqpoint{1.772714in}{1.429376in}}%
\pgfpathlineto{\pgfqpoint{1.777604in}{1.425441in}}%
\pgfpathlineto{\pgfqpoint{1.782494in}{1.421535in}}%
\pgfpathlineto{\pgfqpoint{1.787383in}{1.417656in}}%
\pgfpathlineto{\pgfqpoint{1.792273in}{1.413805in}}%
\pgfpathlineto{\pgfqpoint{1.797163in}{1.409980in}}%
\pgfpathlineto{\pgfqpoint{1.802052in}{1.406183in}}%
\pgfpathlineto{\pgfqpoint{1.806942in}{1.402411in}}%
\pgfpathlineto{\pgfqpoint{1.811832in}{1.398666in}}%
\pgfpathlineto{\pgfqpoint{1.816721in}{1.394947in}}%
\pgfpathlineto{\pgfqpoint{1.821611in}{1.391253in}}%
\pgfpathlineto{\pgfqpoint{1.826501in}{1.387585in}}%
\pgfpathlineto{\pgfqpoint{1.831390in}{1.383942in}}%
\pgfpathlineto{\pgfqpoint{1.836280in}{1.380324in}}%
\pgfpathlineto{\pgfqpoint{1.841170in}{1.376730in}}%
\pgfpathlineto{\pgfqpoint{1.846059in}{1.373161in}}%
\pgfpathlineto{\pgfqpoint{1.850949in}{1.369616in}}%
\pgfpathlineto{\pgfqpoint{1.855839in}{1.366094in}}%
\pgfpathlineto{\pgfqpoint{1.860728in}{1.362597in}}%
\pgfpathlineto{\pgfqpoint{1.865618in}{1.359122in}}%
\pgfpathlineto{\pgfqpoint{1.870508in}{1.355671in}}%
\pgfpathlineto{\pgfqpoint{1.875397in}{1.352243in}}%
\pgfpathlineto{\pgfqpoint{1.880287in}{1.348837in}}%
\pgfpathlineto{\pgfqpoint{1.885177in}{1.345454in}}%
\pgfpathlineto{\pgfqpoint{1.890066in}{1.342094in}}%
\pgfpathlineto{\pgfqpoint{1.894956in}{1.338755in}}%
\pgfpathlineto{\pgfqpoint{1.899846in}{1.335438in}}%
\pgfpathlineto{\pgfqpoint{1.904735in}{1.332142in}}%
\pgfpathlineto{\pgfqpoint{1.909625in}{1.328868in}}%
\pgfpathlineto{\pgfqpoint{1.914515in}{1.325616in}}%
\pgfpathlineto{\pgfqpoint{1.919404in}{1.322384in}}%
\pgfpathlineto{\pgfqpoint{1.924294in}{1.319173in}}%
\pgfpathlineto{\pgfqpoint{1.929184in}{1.315982in}}%
\pgfpathlineto{\pgfqpoint{1.934073in}{1.312812in}}%
\pgfpathlineto{\pgfqpoint{1.938963in}{1.309662in}}%
\pgfpathlineto{\pgfqpoint{1.943853in}{1.306532in}}%
\pgfpathlineto{\pgfqpoint{1.948742in}{1.303422in}}%
\pgfpathlineto{\pgfqpoint{1.953632in}{1.300332in}}%
\pgfpathlineto{\pgfqpoint{1.958522in}{1.297261in}}%
\pgfpathlineto{\pgfqpoint{1.963411in}{1.294209in}}%
\pgfpathlineto{\pgfqpoint{1.968301in}{1.291176in}}%
\pgfpathlineto{\pgfqpoint{1.973191in}{1.288162in}}%
\pgfpathlineto{\pgfqpoint{1.978080in}{1.285167in}}%
\pgfpathlineto{\pgfqpoint{1.982970in}{1.282190in}}%
\pgfpathlineto{\pgfqpoint{1.987860in}{1.279232in}}%
\pgfpathlineto{\pgfqpoint{1.992749in}{1.276292in}}%
\pgfpathlineto{\pgfqpoint{1.997639in}{1.273370in}}%
\pgfpathlineto{\pgfqpoint{2.002529in}{1.270466in}}%
\pgfpathlineto{\pgfqpoint{2.007418in}{1.267579in}}%
\pgfpathlineto{\pgfqpoint{2.012308in}{1.264710in}}%
\pgfpathlineto{\pgfqpoint{2.017198in}{1.261859in}}%
\pgfpathlineto{\pgfqpoint{2.022088in}{1.259024in}}%
\pgfpathlineto{\pgfqpoint{2.026977in}{1.256207in}}%
\pgfpathlineto{\pgfqpoint{2.031867in}{1.253407in}}%
\pgfpathlineto{\pgfqpoint{2.036757in}{1.250623in}}%
\pgfpathlineto{\pgfqpoint{2.041646in}{1.247856in}}%
\pgfpathlineto{\pgfqpoint{2.046536in}{1.245106in}}%
\pgfpathlineto{\pgfqpoint{2.051426in}{1.242372in}}%
\pgfpathlineto{\pgfqpoint{2.056315in}{1.239654in}}%
\pgfpathlineto{\pgfqpoint{2.061205in}{1.236952in}}%
\pgfpathlineto{\pgfqpoint{2.066095in}{1.234265in}}%
\pgfpathlineto{\pgfqpoint{2.070984in}{1.231595in}}%
\pgfpathlineto{\pgfqpoint{2.075874in}{1.228940in}}%
\pgfpathlineto{\pgfqpoint{2.080764in}{1.226301in}}%
\pgfpathlineto{\pgfqpoint{2.085653in}{1.223677in}}%
\pgfpathlineto{\pgfqpoint{2.090543in}{1.221068in}}%
\pgfpathlineto{\pgfqpoint{2.095433in}{1.218475in}}%
\pgfpathlineto{\pgfqpoint{2.100322in}{1.215896in}}%
\pgfpathlineto{\pgfqpoint{2.105212in}{1.213332in}}%
\pgfpathlineto{\pgfqpoint{2.110102in}{1.210783in}}%
\pgfpathlineto{\pgfqpoint{2.114991in}{1.208248in}}%
\pgfpathlineto{\pgfqpoint{2.119881in}{1.205728in}}%
\pgfpathlineto{\pgfqpoint{2.124771in}{1.203222in}}%
\pgfpathlineto{\pgfqpoint{2.129660in}{1.200730in}}%
\pgfpathlineto{\pgfqpoint{2.134550in}{1.198253in}}%
\pgfpathlineto{\pgfqpoint{2.139440in}{1.195789in}}%
\pgfpathlineto{\pgfqpoint{2.144329in}{1.193339in}}%
\pgfpathlineto{\pgfqpoint{2.149219in}{1.190903in}}%
\pgfpathlineto{\pgfqpoint{2.154109in}{1.188481in}}%
\pgfpathlineto{\pgfqpoint{2.158998in}{1.186072in}}%
\pgfpathlineto{\pgfqpoint{2.163888in}{1.183676in}}%
\pgfpathlineto{\pgfqpoint{2.168778in}{1.181294in}}%
\pgfpathlineto{\pgfqpoint{2.173667in}{1.178924in}}%
\pgfpathlineto{\pgfqpoint{2.178557in}{1.176568in}}%
\pgfpathlineto{\pgfqpoint{2.183447in}{1.174225in}}%
\pgfpathlineto{\pgfqpoint{2.188336in}{1.171895in}}%
\pgfpathlineto{\pgfqpoint{2.193226in}{1.169577in}}%
\pgfpathlineto{\pgfqpoint{2.198116in}{1.167272in}}%
\pgfpathlineto{\pgfqpoint{2.203005in}{1.164979in}}%
\pgfpathlineto{\pgfqpoint{2.207895in}{1.162699in}}%
\pgfpathlineto{\pgfqpoint{2.212785in}{1.160432in}}%
\pgfpathlineto{\pgfqpoint{2.217674in}{1.158176in}}%
\pgfpathlineto{\pgfqpoint{2.222564in}{1.155933in}}%
\pgfpathlineto{\pgfqpoint{2.227454in}{1.153701in}}%
\pgfpathlineto{\pgfqpoint{2.232343in}{1.151482in}}%
\pgfpathlineto{\pgfqpoint{2.237233in}{1.149274in}}%
\pgfpathlineto{\pgfqpoint{2.242123in}{1.147078in}}%
\pgfpathlineto{\pgfqpoint{2.247012in}{1.144894in}}%
\pgfpathlineto{\pgfqpoint{2.251902in}{1.142721in}}%
\pgfpathlineto{\pgfqpoint{2.256792in}{1.140560in}}%
\pgfpathlineto{\pgfqpoint{2.261681in}{1.138410in}}%
\pgfpathlineto{\pgfqpoint{2.266571in}{1.136271in}}%
\pgfpathlineto{\pgfqpoint{2.271461in}{1.134144in}}%
\pgfpathlineto{\pgfqpoint{2.276350in}{1.132028in}}%
\pgfpathlineto{\pgfqpoint{2.281240in}{1.129922in}}%
\pgfpathlineto{\pgfqpoint{2.286130in}{1.127828in}}%
\pgfpathlineto{\pgfqpoint{2.291019in}{1.125745in}}%
\pgfpathlineto{\pgfqpoint{2.295909in}{1.123672in}}%
\pgfpathlineto{\pgfqpoint{2.300799in}{1.121610in}}%
\pgfpathlineto{\pgfqpoint{2.305688in}{1.119558in}}%
\pgfpathlineto{\pgfqpoint{2.310578in}{1.117517in}}%
\pgfpathlineto{\pgfqpoint{2.315468in}{1.115487in}}%
\pgfpathlineto{\pgfqpoint{2.320357in}{1.113467in}}%
\pgfpathlineto{\pgfqpoint{2.325247in}{1.111457in}}%
\pgfpathlineto{\pgfqpoint{2.330137in}{1.109457in}}%
\pgfpathlineto{\pgfqpoint{2.335026in}{1.107468in}}%
\pgfpathlineto{\pgfqpoint{2.339916in}{1.105488in}}%
\pgfpathlineto{\pgfqpoint{2.344806in}{1.103519in}}%
\pgfpathlineto{\pgfqpoint{2.349695in}{1.101559in}}%
\pgfpathlineto{\pgfqpoint{2.354585in}{1.099609in}}%
\pgfpathlineto{\pgfqpoint{2.359475in}{1.097669in}}%
\pgfpathlineto{\pgfqpoint{2.364364in}{1.095739in}}%
\pgfpathlineto{\pgfqpoint{2.369254in}{1.093818in}}%
\pgfpathlineto{\pgfqpoint{2.374144in}{1.091907in}}%
\pgfpathlineto{\pgfqpoint{2.379033in}{1.090005in}}%
\pgfpathlineto{\pgfqpoint{2.383923in}{1.088112in}}%
\pgfpathlineto{\pgfqpoint{2.388813in}{1.086229in}}%
\pgfpathlineto{\pgfqpoint{2.393702in}{1.084355in}}%
\pgfpathlineto{\pgfqpoint{2.398592in}{1.082491in}}%
\pgfpathlineto{\pgfqpoint{2.403482in}{1.080635in}}%
\pgfpathlineto{\pgfqpoint{2.408371in}{1.078789in}}%
\pgfpathlineto{\pgfqpoint{2.413261in}{1.076951in}}%
\pgfpathlineto{\pgfqpoint{2.418151in}{1.075123in}}%
\pgfpathlineto{\pgfqpoint{2.423040in}{1.073303in}}%
\pgfpathlineto{\pgfqpoint{2.427930in}{1.071492in}}%
\pgfpathlineto{\pgfqpoint{2.432820in}{1.069690in}}%
\pgfpathlineto{\pgfqpoint{2.437709in}{1.067896in}}%
\pgfpathlineto{\pgfqpoint{2.442599in}{1.066111in}}%
\pgfpathlineto{\pgfqpoint{2.447489in}{1.064335in}}%
\pgfpathlineto{\pgfqpoint{2.452379in}{1.062567in}}%
\pgfpathlineto{\pgfqpoint{2.457268in}{1.060807in}}%
\pgfpathlineto{\pgfqpoint{2.462158in}{1.059056in}}%
\pgfpathlineto{\pgfqpoint{2.467048in}{1.057313in}}%
\pgfpathlineto{\pgfqpoint{2.471937in}{1.055578in}}%
\pgfpathlineto{\pgfqpoint{2.476827in}{1.053852in}}%
\pgfpathlineto{\pgfqpoint{2.481717in}{1.052134in}}%
\pgfpathlineto{\pgfqpoint{2.486606in}{1.050423in}}%
\pgfpathlineto{\pgfqpoint{2.491496in}{1.048721in}}%
\pgfpathlineto{\pgfqpoint{2.496386in}{1.047027in}}%
\pgfpathlineto{\pgfqpoint{2.501275in}{1.045340in}}%
\pgfpathlineto{\pgfqpoint{2.506165in}{1.043662in}}%
\pgfpathlineto{\pgfqpoint{2.511055in}{1.041991in}}%
\pgfpathlineto{\pgfqpoint{2.515944in}{1.040328in}}%
\pgfpathlineto{\pgfqpoint{2.520834in}{1.038672in}}%
\pgfpathlineto{\pgfqpoint{2.525724in}{1.037025in}}%
\pgfpathlineto{\pgfqpoint{2.530613in}{1.035384in}}%
\pgfpathlineto{\pgfqpoint{2.535503in}{1.033752in}}%
\pgfpathlineto{\pgfqpoint{2.540393in}{1.032126in}}%
\pgfpathlineto{\pgfqpoint{2.545282in}{1.030509in}}%
\pgfpathlineto{\pgfqpoint{2.550172in}{1.028898in}}%
\pgfpathlineto{\pgfqpoint{2.555062in}{1.027295in}}%
\pgfpathlineto{\pgfqpoint{2.559951in}{1.025699in}}%
\pgfpathlineto{\pgfqpoint{2.564841in}{1.024111in}}%
\pgfpathlineto{\pgfqpoint{2.569731in}{1.022529in}}%
\pgfpathlineto{\pgfqpoint{2.574620in}{1.020955in}}%
\pgfpathlineto{\pgfqpoint{2.579510in}{1.019388in}}%
\pgfpathlineto{\pgfqpoint{2.584400in}{1.017828in}}%
\pgfpathlineto{\pgfqpoint{2.589289in}{1.016274in}}%
\pgfpathlineto{\pgfqpoint{2.594179in}{1.014728in}}%
\pgfpathlineto{\pgfqpoint{2.599069in}{1.013189in}}%
\pgfpathlineto{\pgfqpoint{2.603958in}{1.011656in}}%
\pgfpathlineto{\pgfqpoint{2.608848in}{1.010130in}}%
\pgfpathlineto{\pgfqpoint{2.613738in}{1.008611in}}%
\pgfpathlineto{\pgfqpoint{2.618627in}{1.007099in}}%
\pgfpathlineto{\pgfqpoint{2.623517in}{1.005593in}}%
\pgfpathlineto{\pgfqpoint{2.628407in}{1.004094in}}%
\pgfpathlineto{\pgfqpoint{2.633296in}{1.002602in}}%
\pgfpathlineto{\pgfqpoint{2.638186in}{1.001116in}}%
\pgfpathlineto{\pgfqpoint{2.643076in}{0.999636in}}%
\pgfpathlineto{\pgfqpoint{2.647965in}{0.998163in}}%
\pgfpathlineto{\pgfqpoint{2.652855in}{0.996697in}}%
\pgfpathlineto{\pgfqpoint{2.657745in}{0.995236in}}%
\pgfpathlineto{\pgfqpoint{2.662634in}{0.993782in}}%
\pgfpathlineto{\pgfqpoint{2.667524in}{0.992335in}}%
\pgfpathlineto{\pgfqpoint{2.672414in}{0.990893in}}%
\pgfpathlineto{\pgfqpoint{2.677303in}{0.989458in}}%
\pgfpathlineto{\pgfqpoint{2.682193in}{0.988029in}}%
\pgfpathlineto{\pgfqpoint{2.687083in}{0.986606in}}%
\pgfpathlineto{\pgfqpoint{2.691972in}{0.985189in}}%
\pgfpathlineto{\pgfqpoint{2.696862in}{0.983778in}}%
\pgfpathlineto{\pgfqpoint{2.701752in}{0.982373in}}%
\pgfpathlineto{\pgfqpoint{2.706641in}{0.980975in}}%
\pgfpathlineto{\pgfqpoint{2.711531in}{0.979582in}}%
\pgfpathlineto{\pgfqpoint{2.716421in}{0.978195in}}%
\pgfpathlineto{\pgfqpoint{2.721310in}{0.976813in}}%
\pgfpathlineto{\pgfqpoint{2.726200in}{0.975438in}}%
\pgfpathlineto{\pgfqpoint{2.731090in}{0.974068in}}%
\pgfpathlineto{\pgfqpoint{2.735979in}{0.972705in}}%
\pgfpathlineto{\pgfqpoint{2.740869in}{0.971346in}}%
\pgfpathlineto{\pgfqpoint{2.745759in}{0.969994in}}%
\pgfpathlineto{\pgfqpoint{2.750648in}{0.968647in}}%
\pgfpathlineto{\pgfqpoint{2.755538in}{0.967306in}}%
\pgfpathlineto{\pgfqpoint{2.760428in}{0.965970in}}%
\pgfpathlineto{\pgfqpoint{2.765317in}{0.964640in}}%
\pgfpathlineto{\pgfqpoint{2.770207in}{0.963315in}}%
\pgfpathlineto{\pgfqpoint{2.775097in}{0.961996in}}%
\pgfpathlineto{\pgfqpoint{2.779986in}{0.960682in}}%
\pgfpathlineto{\pgfqpoint{2.784876in}{0.959374in}}%
\pgfpathlineto{\pgfqpoint{2.789766in}{0.958071in}}%
\pgfpathlineto{\pgfqpoint{2.794655in}{0.956773in}}%
\pgfpathlineto{\pgfqpoint{2.799545in}{0.955481in}}%
\pgfpathlineto{\pgfqpoint{2.804435in}{0.954194in}}%
\pgfpathlineto{\pgfqpoint{2.809324in}{0.952912in}}%
\pgfpathlineto{\pgfqpoint{2.814214in}{0.951635in}}%
\pgfpathlineto{\pgfqpoint{2.819104in}{0.950363in}}%
\pgfpathlineto{\pgfqpoint{2.823993in}{0.949097in}}%
\pgfpathlineto{\pgfqpoint{2.828883in}{0.947836in}}%
\pgfpathlineto{\pgfqpoint{2.833773in}{0.946580in}}%
\pgfpathlineto{\pgfqpoint{2.838662in}{0.945328in}}%
\pgfpathlineto{\pgfqpoint{2.843552in}{0.944082in}}%
\pgfpathlineto{\pgfqpoint{2.848442in}{0.942841in}}%
\pgfpathlineto{\pgfqpoint{2.853331in}{0.941605in}}%
\pgfpathlineto{\pgfqpoint{2.858221in}{0.940373in}}%
\pgfpathlineto{\pgfqpoint{2.863111in}{0.939147in}}%
\pgfpathlineto{\pgfqpoint{2.868000in}{0.937925in}}%
\pgfpathlineto{\pgfqpoint{2.872890in}{0.936709in}}%
\pgfpathlineto{\pgfqpoint{2.877780in}{0.935497in}}%
\pgfpathlineto{\pgfqpoint{2.882669in}{0.934290in}}%
\pgfpathlineto{\pgfqpoint{2.887559in}{0.933087in}}%
\pgfpathlineto{\pgfqpoint{2.892449in}{0.931890in}}%
\pgfpathlineto{\pgfqpoint{2.897339in}{0.930697in}}%
\pgfpathlineto{\pgfqpoint{2.902228in}{0.929508in}}%
\pgfpathlineto{\pgfqpoint{2.907118in}{0.928325in}}%
\pgfpathlineto{\pgfqpoint{2.912008in}{0.927146in}}%
\pgfpathlineto{\pgfqpoint{2.916897in}{0.925971in}}%
\pgfpathlineto{\pgfqpoint{2.921787in}{0.924801in}}%
\pgfpathlineto{\pgfqpoint{2.926677in}{0.923636in}}%
\pgfpathlineto{\pgfqpoint{2.931566in}{0.922475in}}%
\pgfpathlineto{\pgfqpoint{2.936456in}{0.921319in}}%
\pgfpathlineto{\pgfqpoint{2.941346in}{0.920167in}}%
\pgfpathlineto{\pgfqpoint{2.946235in}{0.919019in}}%
\pgfpathlineto{\pgfqpoint{2.951125in}{0.917876in}}%
\pgfpathlineto{\pgfqpoint{2.956015in}{0.916738in}}%
\pgfpathlineto{\pgfqpoint{2.960904in}{0.915603in}}%
\pgfpathlineto{\pgfqpoint{2.965794in}{0.914473in}}%
\pgfpathlineto{\pgfqpoint{2.970684in}{0.913348in}}%
\pgfpathlineto{\pgfqpoint{2.975573in}{0.912226in}}%
\pgfpathlineto{\pgfqpoint{2.980463in}{0.911109in}}%
\pgfpathlineto{\pgfqpoint{2.985353in}{0.909996in}}%
\pgfpathlineto{\pgfqpoint{2.990242in}{0.908887in}}%
\pgfpathlineto{\pgfqpoint{2.995132in}{0.907783in}}%
\pgfpathlineto{\pgfqpoint{3.000022in}{0.906682in}}%
\pgfpathlineto{\pgfqpoint{3.004911in}{0.905586in}}%
\pgfpathlineto{\pgfqpoint{3.009801in}{0.904494in}}%
\pgfpathlineto{\pgfqpoint{3.014691in}{0.903406in}}%
\pgfpathlineto{\pgfqpoint{3.019580in}{0.902322in}}%
\pgfpathlineto{\pgfqpoint{3.024470in}{0.901242in}}%
\pgfpathlineto{\pgfqpoint{3.029360in}{0.900166in}}%
\pgfpathlineto{\pgfqpoint{3.034249in}{0.899094in}}%
\pgfpathlineto{\pgfqpoint{3.039139in}{0.898026in}}%
\pgfpathlineto{\pgfqpoint{3.044029in}{0.896962in}}%
\pgfpathlineto{\pgfqpoint{3.048918in}{0.895902in}}%
\pgfpathlineto{\pgfqpoint{3.053808in}{0.894846in}}%
\pgfpathlineto{\pgfqpoint{3.058698in}{0.893794in}}%
\pgfpathlineto{\pgfqpoint{3.063587in}{0.892745in}}%
\pgfpathlineto{\pgfqpoint{3.068477in}{0.891701in}}%
\pgfpathlineto{\pgfqpoint{3.073367in}{0.890660in}}%
\pgfpathlineto{\pgfqpoint{3.078256in}{0.889623in}}%
\pgfpathlineto{\pgfqpoint{3.083146in}{0.888590in}}%
\pgfpathlineto{\pgfqpoint{3.088036in}{0.887560in}}%
\pgfpathlineto{\pgfqpoint{3.092925in}{0.886535in}}%
\pgfpathlineto{\pgfqpoint{3.097815in}{0.885513in}}%
\pgfpathlineto{\pgfqpoint{3.102705in}{0.884495in}}%
\pgfpathlineto{\pgfqpoint{3.107594in}{0.883480in}}%
\pgfpathlineto{\pgfqpoint{3.112484in}{0.882469in}}%
\pgfpathlineto{\pgfqpoint{3.117374in}{0.881462in}}%
\pgfpathlineto{\pgfqpoint{3.122263in}{0.880458in}}%
\pgfpathlineto{\pgfqpoint{3.127153in}{0.879458in}}%
\pgfpathlineto{\pgfqpoint{3.132043in}{0.878462in}}%
\pgfpathlineto{\pgfqpoint{3.136932in}{0.877469in}}%
\pgfpathlineto{\pgfqpoint{3.141822in}{0.876480in}}%
\pgfpathlineto{\pgfqpoint{3.146712in}{0.875494in}}%
\pgfpathlineto{\pgfqpoint{3.151601in}{0.874512in}}%
\pgfpathlineto{\pgfqpoint{3.156491in}{0.873533in}}%
\pgfpathlineto{\pgfqpoint{3.161381in}{0.872558in}}%
\pgfpathlineto{\pgfqpoint{3.166270in}{0.871586in}}%
\pgfpathlineto{\pgfqpoint{3.171160in}{0.870617in}}%
\pgfpathlineto{\pgfqpoint{3.176050in}{0.869652in}}%
\pgfpathlineto{\pgfqpoint{3.180939in}{0.868691in}}%
\pgfpathlineto{\pgfqpoint{3.185829in}{0.867733in}}%
\pgfpathlineto{\pgfqpoint{3.190719in}{0.866778in}}%
\pgfpathlineto{\pgfqpoint{3.195608in}{0.865826in}}%
\pgfpathlineto{\pgfqpoint{3.200498in}{0.864878in}}%
\pgfpathlineto{\pgfqpoint{3.205388in}{0.863933in}}%
\pgfpathlineto{\pgfqpoint{3.210277in}{0.862991in}}%
\pgfpathlineto{\pgfqpoint{3.215167in}{0.862053in}}%
\pgfpathlineto{\pgfqpoint{3.220057in}{0.861118in}}%
\pgfpathlineto{\pgfqpoint{3.224946in}{0.860186in}}%
\pgfpathlineto{\pgfqpoint{3.229836in}{0.859258in}}%
\pgfpathlineto{\pgfqpoint{3.234726in}{0.858332in}}%
\pgfpathlineto{\pgfqpoint{3.239615in}{0.857410in}}%
\pgfpathlineto{\pgfqpoint{3.244505in}{0.856491in}}%
\pgfpathlineto{\pgfqpoint{3.249395in}{0.855575in}}%
\pgfpathlineto{\pgfqpoint{3.254284in}{0.854662in}}%
\pgfpathlineto{\pgfqpoint{3.259174in}{0.853753in}}%
\pgfpathlineto{\pgfqpoint{3.264064in}{0.852846in}}%
\pgfpathlineto{\pgfqpoint{3.268953in}{0.851943in}}%
\pgfpathlineto{\pgfqpoint{3.273843in}{0.851042in}}%
\pgfpathlineto{\pgfqpoint{3.278733in}{0.850145in}}%
\pgfpathlineto{\pgfqpoint{3.283622in}{0.849251in}}%
\pgfpathlineto{\pgfqpoint{3.288512in}{0.848360in}}%
\pgfpathlineto{\pgfqpoint{3.293402in}{0.847471in}}%
\pgfpathlineto{\pgfqpoint{3.298291in}{0.846586in}}%
\pgfpathlineto{\pgfqpoint{3.303181in}{0.845704in}}%
\pgfpathlineto{\pgfqpoint{3.308071in}{0.844825in}}%
\pgfpathlineto{\pgfqpoint{3.312960in}{0.843949in}}%
\pgfpathlineto{\pgfqpoint{3.317850in}{0.843075in}}%
\pgfpathlineto{\pgfqpoint{3.322740in}{0.842205in}}%
\pgfpathlineto{\pgfqpoint{3.327630in}{0.841337in}}%
\pgfpathlineto{\pgfqpoint{3.332519in}{0.840473in}}%
\pgfpathlineto{\pgfqpoint{3.337409in}{0.839611in}}%
\pgfpathlineto{\pgfqpoint{3.342299in}{0.838752in}}%
\pgfpathlineto{\pgfqpoint{3.347188in}{0.837896in}}%
\pgfpathlineto{\pgfqpoint{3.352078in}{0.837043in}}%
\pgfpathlineto{\pgfqpoint{3.356968in}{0.836193in}}%
\pgfpathlineto{\pgfqpoint{3.361857in}{0.835345in}}%
\pgfpathlineto{\pgfqpoint{3.366747in}{0.834500in}}%
\pgfpathlineto{\pgfqpoint{3.371637in}{0.833658in}}%
\pgfpathlineto{\pgfqpoint{3.376526in}{0.832819in}}%
\pgfpathlineto{\pgfqpoint{3.381416in}{0.831983in}}%
\pgfpathlineto{\pgfqpoint{3.386306in}{0.831149in}}%
\pgfpathlineto{\pgfqpoint{3.391195in}{0.830318in}}%
\pgfpathlineto{\pgfqpoint{3.396085in}{0.829490in}}%
\pgfpathlineto{\pgfqpoint{3.400975in}{0.828664in}}%
\pgfpathlineto{\pgfqpoint{3.405864in}{0.827842in}}%
\pgfpathlineto{\pgfqpoint{3.410754in}{0.827021in}}%
\pgfpathlineto{\pgfqpoint{3.415644in}{0.826204in}}%
\pgfpathlineto{\pgfqpoint{3.420533in}{0.825389in}}%
\pgfpathlineto{\pgfqpoint{3.425423in}{0.824577in}}%
\pgfpathlineto{\pgfqpoint{3.430313in}{0.823767in}}%
\pgfpathlineto{\pgfqpoint{3.435202in}{0.822960in}}%
\pgfpathlineto{\pgfqpoint{3.440092in}{0.822156in}}%
\pgfpathlineto{\pgfqpoint{3.444982in}{0.821354in}}%
\pgfpathlineto{\pgfqpoint{3.449871in}{0.820555in}}%
\pgfpathlineto{\pgfqpoint{3.454761in}{0.819758in}}%
\pgfpathlineto{\pgfqpoint{3.459651in}{0.818964in}}%
\pgfpathlineto{\pgfqpoint{3.464540in}{0.818172in}}%
\pgfpathlineto{\pgfqpoint{3.469430in}{0.817383in}}%
\pgfpathlineto{\pgfqpoint{3.474320in}{0.816597in}}%
\pgfpathlineto{\pgfqpoint{3.479209in}{0.815813in}}%
\pgfpathlineto{\pgfqpoint{3.484099in}{0.815031in}}%
\pgfpathlineto{\pgfqpoint{3.488989in}{0.814252in}}%
\pgfpathlineto{\pgfqpoint{3.493878in}{0.813475in}}%
\pgfpathlineto{\pgfqpoint{3.498768in}{0.812701in}}%
\pgfpathlineto{\pgfqpoint{3.503658in}{0.811930in}}%
\pgfpathlineto{\pgfqpoint{3.508547in}{0.811160in}}%
\pgfpathlineto{\pgfqpoint{3.513437in}{0.810393in}}%
\pgfpathlineto{\pgfqpoint{3.518327in}{0.809629in}}%
\pgfpathlineto{\pgfqpoint{3.523216in}{0.808867in}}%
\pgfpathlineto{\pgfqpoint{3.528106in}{0.808107in}}%
\pgfpathlineto{\pgfqpoint{3.532996in}{0.807350in}}%
\pgfpathlineto{\pgfqpoint{3.537885in}{0.806595in}}%
\pgfpathlineto{\pgfqpoint{3.542775in}{0.805842in}}%
\pgfpathlineto{\pgfqpoint{3.547665in}{0.805092in}}%
\pgfpathlineto{\pgfqpoint{3.552554in}{0.804344in}}%
\pgfpathlineto{\pgfqpoint{3.557444in}{0.803598in}}%
\pgfpathlineto{\pgfqpoint{3.562334in}{0.802855in}}%
\pgfpathlineto{\pgfqpoint{3.567223in}{0.802114in}}%
\pgfpathlineto{\pgfqpoint{3.572113in}{0.801375in}}%
\pgfpathlineto{\pgfqpoint{3.577003in}{0.800639in}}%
\pgfpathlineto{\pgfqpoint{3.581892in}{0.799905in}}%
\pgfpathlineto{\pgfqpoint{3.586782in}{0.799173in}}%
\pgfpathlineto{\pgfqpoint{3.591672in}{0.798443in}}%
\pgfpathlineto{\pgfqpoint{3.596561in}{0.797716in}}%
\pgfpathlineto{\pgfqpoint{3.601451in}{0.796991in}}%
\pgfpathlineto{\pgfqpoint{3.606341in}{0.796268in}}%
\pgfpathlineto{\pgfqpoint{3.611230in}{0.795547in}}%
\pgfpathlineto{\pgfqpoint{3.616120in}{0.794828in}}%
\pgfpathlineto{\pgfqpoint{3.621010in}{0.794112in}}%
\pgfpathlineto{\pgfqpoint{3.625899in}{0.793397in}}%
\pgfpathlineto{\pgfqpoint{3.630789in}{0.792685in}}%
\pgfpathlineto{\pgfqpoint{3.635679in}{0.791975in}}%
\pgfpathlineto{\pgfqpoint{3.640568in}{0.791268in}}%
\pgfpathlineto{\pgfqpoint{3.645458in}{0.790562in}}%
\pgfpathlineto{\pgfqpoint{3.650348in}{0.789858in}}%
\pgfpathlineto{\pgfqpoint{3.655237in}{0.789157in}}%
\pgfpathlineto{\pgfqpoint{3.660127in}{0.788458in}}%
\pgfpathlineto{\pgfqpoint{3.665017in}{0.787760in}}%
\pgfpathlineto{\pgfqpoint{3.669906in}{0.787065in}}%
\pgfpathlineto{\pgfqpoint{3.674796in}{0.786372in}}%
\pgfpathlineto{\pgfqpoint{3.679686in}{0.785681in}}%
\pgfpathlineto{\pgfqpoint{3.684575in}{0.784992in}}%
\pgfpathlineto{\pgfqpoint{3.689465in}{0.784306in}}%
\pgfpathlineto{\pgfqpoint{3.694355in}{0.783621in}}%
\pgfpathlineto{\pgfqpoint{3.699244in}{0.782938in}}%
\pgfpathlineto{\pgfqpoint{3.704134in}{0.782257in}}%
\pgfpathlineto{\pgfqpoint{3.709024in}{0.781578in}}%
\pgfpathlineto{\pgfqpoint{3.713913in}{0.780902in}}%
\pgfpathlineto{\pgfqpoint{3.718803in}{0.780227in}}%
\pgfpathlineto{\pgfqpoint{3.723693in}{0.779554in}}%
\pgfpathlineto{\pgfqpoint{3.728582in}{0.778883in}}%
\pgfpathlineto{\pgfqpoint{3.733472in}{0.778214in}}%
\pgfpathlineto{\pgfqpoint{3.738362in}{0.777548in}}%
\pgfpathlineto{\pgfqpoint{3.743251in}{0.776883in}}%
\pgfpathlineto{\pgfqpoint{3.748141in}{0.776220in}}%
\pgfpathlineto{\pgfqpoint{3.753031in}{0.775559in}}%
\pgfpathlineto{\pgfqpoint{3.757921in}{0.774900in}}%
\pgfpathlineto{\pgfqpoint{3.762810in}{0.774242in}}%
\pgfpathlineto{\pgfqpoint{3.767700in}{0.773587in}}%
\pgfpathlineto{\pgfqpoint{3.772590in}{0.772934in}}%
\pgfpathlineto{\pgfqpoint{3.777479in}{0.772282in}}%
\pgfpathlineto{\pgfqpoint{3.782369in}{0.771633in}}%
\pgfpathlineto{\pgfqpoint{3.787259in}{0.770985in}}%
\pgfpathlineto{\pgfqpoint{3.792148in}{0.770339in}}%
\pgfpathlineto{\pgfqpoint{3.797038in}{0.769695in}}%
\pgfpathlineto{\pgfqpoint{3.801928in}{0.769053in}}%
\pgfpathlineto{\pgfqpoint{3.806817in}{0.768413in}}%
\pgfpathlineto{\pgfqpoint{3.811707in}{0.767774in}}%
\pgfpathlineto{\pgfqpoint{3.816597in}{0.767138in}}%
\pgfpathlineto{\pgfqpoint{3.821486in}{0.766503in}}%
\pgfpathlineto{\pgfqpoint{3.826376in}{0.765870in}}%
\pgfpathlineto{\pgfqpoint{3.831266in}{0.765239in}}%
\pgfpathlineto{\pgfqpoint{3.836155in}{0.764609in}}%
\pgfpathlineto{\pgfqpoint{3.841045in}{0.763982in}}%
\pgfpathlineto{\pgfqpoint{3.845935in}{0.763356in}}%
\pgfpathlineto{\pgfqpoint{3.850824in}{0.762732in}}%
\pgfpathlineto{\pgfqpoint{3.855714in}{0.762110in}}%
\pgfpathlineto{\pgfqpoint{3.860604in}{0.761489in}}%
\pgfpathlineto{\pgfqpoint{3.865493in}{0.760870in}}%
\pgfpathlineto{\pgfqpoint{3.870383in}{0.760253in}}%
\pgfpathlineto{\pgfqpoint{3.875273in}{0.759638in}}%
\pgfpathlineto{\pgfqpoint{3.880162in}{0.759024in}}%
\pgfpathlineto{\pgfqpoint{3.885052in}{0.758413in}}%
\pgfpathlineto{\pgfqpoint{3.889942in}{0.757802in}}%
\pgfpathlineto{\pgfqpoint{3.894831in}{0.757194in}}%
\pgfpathlineto{\pgfqpoint{3.899721in}{0.756587in}}%
\pgfpathlineto{\pgfqpoint{3.904611in}{0.755982in}}%
\pgfpathlineto{\pgfqpoint{3.909500in}{0.755379in}}%
\pgfpathlineto{\pgfqpoint{3.914390in}{0.754777in}}%
\pgfpathlineto{\pgfqpoint{3.919280in}{0.754177in}}%
\pgfpathlineto{\pgfqpoint{3.924169in}{0.753579in}}%
\pgfpathlineto{\pgfqpoint{3.929059in}{0.752983in}}%
\pgfpathlineto{\pgfqpoint{3.933949in}{0.752388in}}%
\pgfpathlineto{\pgfqpoint{3.938838in}{0.751794in}}%
\pgfpathlineto{\pgfqpoint{3.943728in}{0.751203in}}%
\pgfpathlineto{\pgfqpoint{3.948618in}{0.750612in}}%
\pgfpathlineto{\pgfqpoint{3.953507in}{0.750024in}}%
\pgfpathlineto{\pgfqpoint{3.958397in}{0.749437in}}%
\pgfpathlineto{\pgfqpoint{3.963287in}{0.748852in}}%
\pgfpathlineto{\pgfqpoint{3.968176in}{0.748268in}}%
\pgfpathlineto{\pgfqpoint{3.973066in}{0.747686in}}%
\pgfpathlineto{\pgfqpoint{3.977956in}{0.747106in}}%
\pgfpathlineto{\pgfqpoint{3.982845in}{0.746527in}}%
\pgfpathlineto{\pgfqpoint{3.987735in}{0.745950in}}%
\pgfpathlineto{\pgfqpoint{3.992625in}{0.745374in}}%
\pgfpathlineto{\pgfqpoint{3.997514in}{0.744800in}}%
\pgfpathlineto{\pgfqpoint{4.002404in}{0.744228in}}%
\pgfpathlineto{\pgfqpoint{4.007294in}{0.743657in}}%
\pgfpathlineto{\pgfqpoint{4.012183in}{0.743087in}}%
\pgfpathlineto{\pgfqpoint{4.017073in}{0.742519in}}%
\pgfpathlineto{\pgfqpoint{4.021963in}{0.741953in}}%
\pgfpathlineto{\pgfqpoint{4.026852in}{0.741388in}}%
\pgfpathlineto{\pgfqpoint{4.031742in}{0.740825in}}%
\pgfpathlineto{\pgfqpoint{4.036632in}{0.740263in}}%
\pgfpathlineto{\pgfqpoint{4.041521in}{0.739703in}}%
\pgfpathlineto{\pgfqpoint{4.046411in}{0.739144in}}%
\pgfpathlineto{\pgfqpoint{4.051301in}{0.738586in}}%
\pgfpathlineto{\pgfqpoint{4.056190in}{0.738031in}}%
\pgfpathlineto{\pgfqpoint{4.061080in}{0.737476in}}%
\pgfpathlineto{\pgfqpoint{4.065970in}{0.736923in}}%
\pgfpathlineto{\pgfqpoint{4.070859in}{0.736372in}}%
\pgfpathlineto{\pgfqpoint{4.075749in}{0.735822in}}%
\pgfpathlineto{\pgfqpoint{4.080639in}{0.735274in}}%
\pgfpathlineto{\pgfqpoint{4.085528in}{0.734727in}}%
\pgfpathlineto{\pgfqpoint{4.090418in}{0.734181in}}%
\pgfpathlineto{\pgfqpoint{4.095308in}{0.733637in}}%
\pgfpathlineto{\pgfqpoint{4.100197in}{0.733095in}}%
\pgfpathlineto{\pgfqpoint{4.105087in}{0.732553in}}%
\pgfpathlineto{\pgfqpoint{4.109977in}{0.732014in}}%
\pgfpathlineto{\pgfqpoint{4.114866in}{0.731475in}}%
\pgfpathlineto{\pgfqpoint{4.119756in}{0.730938in}}%
\pgfpathlineto{\pgfqpoint{4.124646in}{0.730403in}}%
\pgfpathlineto{\pgfqpoint{4.129535in}{0.729869in}}%
\pgfpathlineto{\pgfqpoint{4.134425in}{0.729336in}}%
\pgfpathlineto{\pgfqpoint{4.139315in}{0.728804in}}%
\pgfpathlineto{\pgfqpoint{4.144204in}{0.728275in}}%
\pgfpathlineto{\pgfqpoint{4.149094in}{0.727746in}}%
\pgfpathlineto{\pgfqpoint{4.153984in}{0.727219in}}%
\pgfpathlineto{\pgfqpoint{4.158873in}{0.726693in}}%
\pgfpathlineto{\pgfqpoint{4.163763in}{0.726169in}}%
\pgfpathlineto{\pgfqpoint{4.168653in}{0.725645in}}%
\pgfpathlineto{\pgfqpoint{4.173542in}{0.725124in}}%
\pgfpathlineto{\pgfqpoint{4.178432in}{0.724603in}}%
\pgfpathlineto{\pgfqpoint{4.183322in}{0.724084in}}%
\pgfpathlineto{\pgfqpoint{4.188211in}{0.723567in}}%
\pgfpathlineto{\pgfqpoint{4.193101in}{0.723050in}}%
\pgfpathlineto{\pgfqpoint{4.197991in}{0.722535in}}%
\pgfpathlineto{\pgfqpoint{4.202881in}{0.722022in}}%
\pgfpathlineto{\pgfqpoint{4.207770in}{0.721509in}}%
\pgfpathlineto{\pgfqpoint{4.212660in}{0.720998in}}%
\pgfpathlineto{\pgfqpoint{4.217550in}{0.720488in}}%
\pgfpathlineto{\pgfqpoint{4.222439in}{0.719980in}}%
\pgfpathlineto{\pgfqpoint{4.227329in}{0.719473in}}%
\pgfpathlineto{\pgfqpoint{4.232219in}{0.718967in}}%
\pgfpathlineto{\pgfqpoint{4.237108in}{0.718462in}}%
\pgfpathlineto{\pgfqpoint{4.241998in}{0.717959in}}%
\pgfpathlineto{\pgfqpoint{4.246888in}{0.717457in}}%
\pgfpathlineto{\pgfqpoint{4.251777in}{0.716956in}}%
\pgfpathlineto{\pgfqpoint{4.256667in}{0.716457in}}%
\pgfpathlineto{\pgfqpoint{4.261557in}{0.715959in}}%
\pgfpathlineto{\pgfqpoint{4.266446in}{0.715462in}}%
\pgfpathlineto{\pgfqpoint{4.271336in}{0.714966in}}%
\pgfpathlineto{\pgfqpoint{4.276226in}{0.714472in}}%
\pgfpathlineto{\pgfqpoint{4.281115in}{0.713979in}}%
\pgfpathlineto{\pgfqpoint{4.286005in}{0.713487in}}%
\pgfpathlineto{\pgfqpoint{4.290895in}{0.712996in}}%
\pgfpathlineto{\pgfqpoint{4.295784in}{0.712507in}}%
\pgfpathlineto{\pgfqpoint{4.300674in}{0.712018in}}%
\pgfpathlineto{\pgfqpoint{4.305564in}{0.711531in}}%
\pgfpathlineto{\pgfqpoint{4.310453in}{0.711046in}}%
\pgfpathlineto{\pgfqpoint{4.315343in}{0.710561in}}%
\pgfpathlineto{\pgfqpoint{4.320233in}{0.710078in}}%
\pgfpathlineto{\pgfqpoint{4.325122in}{0.709595in}}%
\pgfpathlineto{\pgfqpoint{4.330012in}{0.709114in}}%
\pgfpathlineto{\pgfqpoint{4.334902in}{0.708635in}}%
\pgfpathlineto{\pgfqpoint{4.339791in}{0.708156in}}%
\pgfpathlineto{\pgfqpoint{4.344681in}{0.707679in}}%
\pgfpathlineto{\pgfqpoint{4.349571in}{0.707203in}}%
\pgfpathlineto{\pgfqpoint{4.354460in}{0.706728in}}%
\pgfpathlineto{\pgfqpoint{4.359350in}{0.706254in}}%
\pgfpathlineto{\pgfqpoint{4.364240in}{0.705781in}}%
\pgfpathlineto{\pgfqpoint{4.369129in}{0.705310in}}%
\pgfpathlineto{\pgfqpoint{4.374019in}{0.704839in}}%
\pgfpathlineto{\pgfqpoint{4.378909in}{0.704370in}}%
\pgfpathlineto{\pgfqpoint{4.383798in}{0.703902in}}%
\pgfpathlineto{\pgfqpoint{4.388688in}{0.703435in}}%
\pgfpathlineto{\pgfqpoint{4.393578in}{0.702969in}}%
\pgfpathlineto{\pgfqpoint{4.398467in}{0.702505in}}%
\pgfpathlineto{\pgfqpoint{4.403357in}{0.702041in}}%
\pgfpathlineto{\pgfqpoint{4.408247in}{0.701579in}}%
\pgfpathlineto{\pgfqpoint{4.413136in}{0.701118in}}%
\pgfpathlineto{\pgfqpoint{4.418026in}{0.700658in}}%
\pgfpathlineto{\pgfqpoint{4.422916in}{0.700199in}}%
\pgfpathlineto{\pgfqpoint{4.427805in}{0.699741in}}%
\pgfpathlineto{\pgfqpoint{4.432695in}{0.699284in}}%
\pgfpathlineto{\pgfqpoint{4.437585in}{0.698828in}}%
\pgfpathlineto{\pgfqpoint{4.442474in}{0.698374in}}%
\pgfpathlineto{\pgfqpoint{4.447364in}{0.697920in}}%
\pgfpathlineto{\pgfqpoint{4.452254in}{0.697468in}}%
\pgfpathlineto{\pgfqpoint{4.457143in}{0.697017in}}%
\pgfpathlineto{\pgfqpoint{4.462033in}{0.696567in}}%
\pgfpathlineto{\pgfqpoint{4.466923in}{0.696118in}}%
\pgfpathlineto{\pgfqpoint{4.471812in}{0.695670in}}%
\pgfpathlineto{\pgfqpoint{4.476702in}{0.695223in}}%
\pgfpathlineto{\pgfqpoint{4.481592in}{0.694777in}}%
\pgfpathlineto{\pgfqpoint{4.486481in}{0.694332in}}%
\pgfpathlineto{\pgfqpoint{4.491371in}{0.693888in}}%
\pgfpathlineto{\pgfqpoint{4.496261in}{0.693446in}}%
\pgfpathlineto{\pgfqpoint{4.501150in}{0.693004in}}%
\pgfpathlineto{\pgfqpoint{4.506040in}{0.692564in}}%
\pgfpathlineto{\pgfqpoint{4.510930in}{0.692124in}}%
\pgfpathlineto{\pgfqpoint{4.515819in}{0.691686in}}%
\pgfpathlineto{\pgfqpoint{4.520709in}{0.691248in}}%
\pgfpathlineto{\pgfqpoint{4.525599in}{0.690812in}}%
\pgfpathlineto{\pgfqpoint{4.530488in}{0.690376in}}%
\pgfpathlineto{\pgfqpoint{4.535378in}{0.689942in}}%
\pgfpathlineto{\pgfqpoint{4.540268in}{0.689509in}}%
\pgfpathlineto{\pgfqpoint{4.545157in}{0.689077in}}%
\pgfpathlineto{\pgfqpoint{4.550047in}{0.688645in}}%
\pgfpathlineto{\pgfqpoint{4.554937in}{0.688215in}}%
\pgfpathlineto{\pgfqpoint{4.559826in}{0.687786in}}%
\pgfpathlineto{\pgfqpoint{4.564716in}{0.687358in}}%
\pgfpathlineto{\pgfqpoint{4.569606in}{0.686931in}}%
\pgfpathlineto{\pgfqpoint{4.574495in}{0.686504in}}%
\pgfpathlineto{\pgfqpoint{4.579385in}{0.686079in}}%
\pgfpathlineto{\pgfqpoint{4.584275in}{0.685655in}}%
\pgfpathlineto{\pgfqpoint{4.589164in}{0.685232in}}%
\pgfpathlineto{\pgfqpoint{4.594054in}{0.684809in}}%
\pgfpathlineto{\pgfqpoint{4.598944in}{0.684388in}}%
\pgfpathlineto{\pgfqpoint{4.603833in}{0.683968in}}%
\pgfpathlineto{\pgfqpoint{4.608723in}{0.683549in}}%
\pgfpathlineto{\pgfqpoint{4.613613in}{0.683130in}}%
\pgfpathlineto{\pgfqpoint{4.618502in}{0.682713in}}%
\pgfpathlineto{\pgfqpoint{4.623392in}{0.682297in}}%
\pgfpathlineto{\pgfqpoint{4.628282in}{0.681881in}}%
\pgfpathlineto{\pgfqpoint{4.633172in}{0.681467in}}%
\pgfpathlineto{\pgfqpoint{4.638061in}{0.681053in}}%
\pgfpathlineto{\pgfqpoint{4.642951in}{0.680641in}}%
\pgfpathlineto{\pgfqpoint{4.647841in}{0.680229in}}%
\pgfpathlineto{\pgfqpoint{4.652730in}{0.679819in}}%
\pgfpathlineto{\pgfqpoint{4.657620in}{0.679409in}}%
\pgfpathlineto{\pgfqpoint{4.662510in}{0.679000in}}%
\pgfpathlineto{\pgfqpoint{4.667399in}{0.678592in}}%
\pgfpathlineto{\pgfqpoint{4.672289in}{0.678185in}}%
\pgfpathlineto{\pgfqpoint{4.677179in}{0.677779in}}%
\pgfpathlineto{\pgfqpoint{4.682068in}{0.677374in}}%
\pgfpathlineto{\pgfqpoint{4.686958in}{0.676970in}}%
\pgfpathlineto{\pgfqpoint{4.691848in}{0.676567in}}%
\pgfpathlineto{\pgfqpoint{4.696737in}{0.676165in}}%
\pgfpathlineto{\pgfqpoint{4.701627in}{0.675763in}}%
\pgfpathlineto{\pgfqpoint{4.706517in}{0.675363in}}%
\pgfpathlineto{\pgfqpoint{4.711406in}{0.674964in}}%
\pgfpathlineto{\pgfqpoint{4.716296in}{0.674565in}}%
\pgfpathlineto{\pgfqpoint{4.721186in}{0.674167in}}%
\pgfpathlineto{\pgfqpoint{4.726075in}{0.673770in}}%
\pgfpathlineto{\pgfqpoint{4.730965in}{0.673374in}}%
\pgfpathlineto{\pgfqpoint{4.735855in}{0.672979in}}%
\pgfpathlineto{\pgfqpoint{4.740744in}{0.672585in}}%
\pgfpathlineto{\pgfqpoint{4.745634in}{0.672192in}}%
\pgfpathlineto{\pgfqpoint{4.750524in}{0.671800in}}%
\pgfpathlineto{\pgfqpoint{4.755413in}{0.671408in}}%
\pgfpathlineto{\pgfqpoint{4.760303in}{0.671018in}}%
\pgfpathlineto{\pgfqpoint{4.765193in}{0.670628in}}%
\pgfpathlineto{\pgfqpoint{4.770082in}{0.670239in}}%
\pgfpathlineto{\pgfqpoint{4.774972in}{0.669851in}}%
\pgfpathlineto{\pgfqpoint{4.779862in}{0.669464in}}%
\pgfpathlineto{\pgfqpoint{4.784751in}{0.669078in}}%
\pgfpathlineto{\pgfqpoint{4.789641in}{0.668692in}}%
\pgfpathlineto{\pgfqpoint{4.794531in}{0.668308in}}%
\pgfpathlineto{\pgfqpoint{4.799420in}{0.667924in}}%
\pgfpathlineto{\pgfqpoint{4.804310in}{0.667541in}}%
\pgfpathlineto{\pgfqpoint{4.809200in}{0.667159in}}%
\pgfpathlineto{\pgfqpoint{4.814089in}{0.666778in}}%
\pgfpathlineto{\pgfqpoint{4.818979in}{0.666398in}}%
\pgfpathlineto{\pgfqpoint{4.823869in}{0.666018in}}%
\pgfpathlineto{\pgfqpoint{4.828758in}{0.665640in}}%
\pgfpathlineto{\pgfqpoint{4.833648in}{0.665262in}}%
\pgfpathlineto{\pgfqpoint{4.838538in}{0.664885in}}%
\pgfpathlineto{\pgfqpoint{4.843427in}{0.664509in}}%
\pgfpathlineto{\pgfqpoint{4.848317in}{0.664134in}}%
\pgfpathlineto{\pgfqpoint{4.853207in}{0.663759in}}%
\pgfpathlineto{\pgfqpoint{4.858096in}{0.663385in}}%
\pgfpathlineto{\pgfqpoint{4.862986in}{0.663013in}}%
\pgfpathlineto{\pgfqpoint{4.867876in}{0.662641in}}%
\pgfpathlineto{\pgfqpoint{4.872765in}{0.662269in}}%
\pgfpathlineto{\pgfqpoint{4.877655in}{0.661899in}}%
\pgfpathlineto{\pgfqpoint{4.882545in}{0.661530in}}%
\pgfpathlineto{\pgfqpoint{4.887434in}{0.661161in}}%
\pgfpathlineto{\pgfqpoint{4.892324in}{0.660793in}}%
\pgfpathlineto{\pgfqpoint{4.897214in}{0.660426in}}%
\pgfpathlineto{\pgfqpoint{4.902103in}{0.660059in}}%
\pgfpathlineto{\pgfqpoint{4.906993in}{0.659694in}}%
\pgfpathlineto{\pgfqpoint{4.911883in}{0.659329in}}%
\pgfpathlineto{\pgfqpoint{4.916772in}{0.658965in}}%
\pgfpathlineto{\pgfqpoint{4.921662in}{0.658602in}}%
\pgfpathlineto{\pgfqpoint{4.926552in}{0.658239in}}%
\pgfpathlineto{\pgfqpoint{4.931441in}{0.657878in}}%
\pgfpathlineto{\pgfqpoint{4.936331in}{0.657517in}}%
\pgfpathlineto{\pgfqpoint{4.941221in}{0.657157in}}%
\pgfpathlineto{\pgfqpoint{4.946110in}{0.656798in}}%
\pgfpathlineto{\pgfqpoint{4.951000in}{0.656439in}}%
\pgfpathlineto{\pgfqpoint{4.955890in}{0.656081in}}%
\pgfpathlineto{\pgfqpoint{4.960779in}{0.655724in}}%
\pgfpathlineto{\pgfqpoint{4.965669in}{0.655368in}}%
\pgfpathlineto{\pgfqpoint{4.970559in}{0.655013in}}%
\pgfpathlineto{\pgfqpoint{4.975448in}{0.654658in}}%
\pgfpathlineto{\pgfqpoint{4.980338in}{0.654304in}}%
\pgfpathlineto{\pgfqpoint{4.985228in}{0.653951in}}%
\pgfpathlineto{\pgfqpoint{4.990117in}{0.653598in}}%
\pgfpathlineto{\pgfqpoint{4.995007in}{0.653247in}}%
\pgfpathlineto{\pgfqpoint{4.999897in}{0.652896in}}%
\pgfpathlineto{\pgfqpoint{5.004786in}{0.652546in}}%
\pgfpathlineto{\pgfqpoint{5.009676in}{0.652196in}}%
\pgfpathlineto{\pgfqpoint{5.014566in}{0.651847in}}%
\pgfpathlineto{\pgfqpoint{5.019455in}{0.651499in}}%
\pgfpathlineto{\pgfqpoint{5.024345in}{0.651152in}}%
\pgfpathlineto{\pgfqpoint{5.029235in}{0.650806in}}%
\pgfpathlineto{\pgfqpoint{5.034124in}{0.650460in}}%
\pgfpathlineto{\pgfqpoint{5.039014in}{0.650115in}}%
\pgfpathlineto{\pgfqpoint{5.043904in}{0.649771in}}%
\pgfpathlineto{\pgfqpoint{5.048793in}{0.649427in}}%
\pgfpathlineto{\pgfqpoint{5.053683in}{0.649084in}}%
\pgfpathlineto{\pgfqpoint{5.058573in}{0.648742in}}%
\pgfpathlineto{\pgfqpoint{5.063462in}{0.648401in}}%
\pgfpathlineto{\pgfqpoint{5.068352in}{0.648060in}}%
\pgfpathlineto{\pgfqpoint{5.073242in}{0.647720in}}%
\pgfpathlineto{\pgfqpoint{5.078132in}{0.647381in}}%
\pgfpathlineto{\pgfqpoint{5.083021in}{0.647042in}}%
\pgfpathlineto{\pgfqpoint{5.087911in}{0.646704in}}%
\pgfpathlineto{\pgfqpoint{5.092801in}{0.646367in}}%
\pgfpathlineto{\pgfqpoint{5.097690in}{0.646031in}}%
\pgfpathlineto{\pgfqpoint{5.102580in}{0.645695in}}%
\pgfpathlineto{\pgfqpoint{5.107470in}{0.645360in}}%
\pgfpathlineto{\pgfqpoint{5.112359in}{0.645025in}}%
\pgfpathlineto{\pgfqpoint{5.117249in}{0.644692in}}%
\pgfpathlineto{\pgfqpoint{5.122139in}{0.644359in}}%
\pgfpathlineto{\pgfqpoint{5.127028in}{0.644027in}}%
\pgfpathlineto{\pgfqpoint{5.131918in}{0.643695in}}%
\pgfpathlineto{\pgfqpoint{5.136808in}{0.643364in}}%
\pgfpathlineto{\pgfqpoint{5.141697in}{0.643034in}}%
\pgfpathlineto{\pgfqpoint{5.146587in}{0.642704in}}%
\pgfpathlineto{\pgfqpoint{5.151477in}{0.642375in}}%
\pgfpathlineto{\pgfqpoint{5.156366in}{0.642047in}}%
\pgfpathlineto{\pgfqpoint{5.161256in}{0.641720in}}%
\pgfpathlineto{\pgfqpoint{5.166146in}{0.641393in}}%
\pgfpathlineto{\pgfqpoint{5.171035in}{0.641067in}}%
\pgfpathlineto{\pgfqpoint{5.175925in}{0.640741in}}%
\pgfpathlineto{\pgfqpoint{5.180815in}{0.640416in}}%
\pgfpathlineto{\pgfqpoint{5.185704in}{0.640092in}}%
\pgfpathlineto{\pgfqpoint{5.190594in}{0.639769in}}%
\pgfpathlineto{\pgfqpoint{5.195484in}{0.639446in}}%
\pgfpathlineto{\pgfqpoint{5.200373in}{0.639124in}}%
\pgfpathlineto{\pgfqpoint{5.205263in}{0.638802in}}%
\pgfpathlineto{\pgfqpoint{5.210153in}{0.638481in}}%
\pgfpathlineto{\pgfqpoint{5.215042in}{0.638161in}}%
\pgfpathlineto{\pgfqpoint{5.219932in}{0.637841in}}%
\pgfpathlineto{\pgfqpoint{5.224822in}{0.637522in}}%
\pgfpathlineto{\pgfqpoint{5.229711in}{0.637204in}}%
\pgfpathlineto{\pgfqpoint{5.234601in}{0.636886in}}%
\pgfpathlineto{\pgfqpoint{5.239491in}{0.636569in}}%
\pgfpathlineto{\pgfqpoint{5.244380in}{0.636253in}}%
\pgfpathlineto{\pgfqpoint{5.249270in}{0.635937in}}%
\pgfpathlineto{\pgfqpoint{5.254160in}{0.635622in}}%
\pgfpathlineto{\pgfqpoint{5.259049in}{0.635308in}}%
\pgfpathlineto{\pgfqpoint{5.263939in}{0.634994in}}%
\pgfpathlineto{\pgfqpoint{5.268829in}{0.634681in}}%
\pgfpathlineto{\pgfqpoint{5.273718in}{0.634368in}}%
\pgfpathlineto{\pgfqpoint{5.278608in}{0.634056in}}%
\pgfpathlineto{\pgfqpoint{5.283498in}{0.633745in}}%
\pgfpathlineto{\pgfqpoint{5.288387in}{0.633434in}}%
\pgfpathlineto{\pgfqpoint{5.293277in}{0.633124in}}%
\pgfpathlineto{\pgfqpoint{5.298167in}{0.632815in}}%
\pgfpathlineto{\pgfqpoint{5.303056in}{0.632506in}}%
\pgfpathlineto{\pgfqpoint{5.307946in}{0.632198in}}%
\pgfpathlineto{\pgfqpoint{5.312836in}{0.631890in}}%
\pgfpathlineto{\pgfqpoint{5.317725in}{0.631583in}}%
\pgfpathlineto{\pgfqpoint{5.322615in}{0.631277in}}%
\pgfpathlineto{\pgfqpoint{5.327505in}{0.630971in}}%
\pgfpathlineto{\pgfqpoint{5.332394in}{0.630666in}}%
\pgfpathlineto{\pgfqpoint{5.337284in}{0.630361in}}%
\pgfpathlineto{\pgfqpoint{5.342174in}{0.630057in}}%
\pgfpathlineto{\pgfqpoint{5.347063in}{0.629754in}}%
\pgfpathlineto{\pgfqpoint{5.351953in}{0.629451in}}%
\pgfpathlineto{\pgfqpoint{5.356843in}{0.629149in}}%
\pgfpathlineto{\pgfqpoint{5.361732in}{0.628847in}}%
\pgfpathlineto{\pgfqpoint{5.366622in}{0.628546in}}%
\pgfpathlineto{\pgfqpoint{5.371512in}{0.628246in}}%
\pgfpathlineto{\pgfqpoint{5.376401in}{0.627946in}}%
\pgfpathlineto{\pgfqpoint{5.381291in}{0.627647in}}%
\pgfpathlineto{\pgfqpoint{5.386181in}{0.627349in}}%
\pgfpathlineto{\pgfqpoint{5.391070in}{0.627050in}}%
\pgfpathlineto{\pgfqpoint{5.395960in}{0.626753in}}%
\pgfpathlineto{\pgfqpoint{5.400850in}{0.626456in}}%
\pgfpathlineto{\pgfqpoint{5.405739in}{0.626160in}}%
\pgfpathlineto{\pgfqpoint{5.410629in}{0.625864in}}%
\pgfpathlineto{\pgfqpoint{5.415519in}{0.625569in}}%
\pgfpathlineto{\pgfqpoint{5.420408in}{0.625275in}}%
\pgfpathlineto{\pgfqpoint{5.425298in}{0.624981in}}%
\pgfpathlineto{\pgfqpoint{5.430188in}{0.624687in}}%
\pgfpathlineto{\pgfqpoint{5.435077in}{0.624394in}}%
\pgfpathlineto{\pgfqpoint{5.439967in}{0.624102in}}%
\pgfpathlineto{\pgfqpoint{5.444857in}{0.623810in}}%
\pgfpathlineto{\pgfqpoint{5.449746in}{0.623519in}}%
\pgfpathlineto{\pgfqpoint{5.454636in}{0.623229in}}%
\pgfpathlineto{\pgfqpoint{5.459526in}{0.622939in}}%
\pgfpathlineto{\pgfqpoint{5.464415in}{0.622649in}}%
\pgfpathlineto{\pgfqpoint{5.469305in}{0.622360in}}%
\pgfpathlineto{\pgfqpoint{5.474195in}{0.622072in}}%
\pgfpathlineto{\pgfqpoint{5.479084in}{0.621784in}}%
\pgfpathlineto{\pgfqpoint{5.483974in}{0.621497in}}%
\pgfpathlineto{\pgfqpoint{5.488864in}{0.621210in}}%
\pgfpathlineto{\pgfqpoint{5.493753in}{0.620924in}}%
\pgfpathlineto{\pgfqpoint{5.498643in}{0.620639in}}%
\pgfpathlineto{\pgfqpoint{5.503533in}{0.620354in}}%
\pgfpathlineto{\pgfqpoint{5.508423in}{0.620069in}}%
\pgfpathlineto{\pgfqpoint{5.513312in}{0.619785in}}%
\pgfpathlineto{\pgfqpoint{5.518202in}{0.619502in}}%
\pgfpathlineto{\pgfqpoint{5.523092in}{0.619219in}}%
\pgfpathlineto{\pgfqpoint{5.527981in}{0.618937in}}%
\pgfpathlineto{\pgfqpoint{5.532871in}{0.618655in}}%
\pgfpathlineto{\pgfqpoint{5.537761in}{0.618374in}}%
\pgfpathlineto{\pgfqpoint{5.542650in}{0.618093in}}%
\pgfpathlineto{\pgfqpoint{5.547540in}{0.617813in}}%
\pgfpathlineto{\pgfqpoint{5.552430in}{0.617533in}}%
\pgfpathlineto{\pgfqpoint{5.557319in}{0.617254in}}%
\pgfpathlineto{\pgfqpoint{5.562209in}{0.616976in}}%
\pgfpathlineto{\pgfqpoint{5.567099in}{0.616698in}}%
\pgfpathlineto{\pgfqpoint{5.571988in}{0.616420in}}%
\pgfpathlineto{\pgfqpoint{5.576878in}{0.616143in}}%
\pgfpathlineto{\pgfqpoint{5.581768in}{0.615867in}}%
\pgfpathlineto{\pgfqpoint{5.586657in}{0.615591in}}%
\pgfpathlineto{\pgfqpoint{5.591547in}{0.615315in}}%
\pgfpathlineto{\pgfqpoint{5.596437in}{0.615040in}}%
\pgfpathlineto{\pgfqpoint{5.601326in}{0.614766in}}%
\pgfpathlineto{\pgfqpoint{5.606216in}{0.614492in}}%
\pgfpathlineto{\pgfqpoint{5.611106in}{0.614219in}}%
\pgfpathlineto{\pgfqpoint{5.615995in}{0.613946in}}%
\pgfpathlineto{\pgfqpoint{5.620885in}{0.613673in}}%
\pgfpathlineto{\pgfqpoint{5.625775in}{0.613402in}}%
\pgfpathlineto{\pgfqpoint{5.630664in}{0.613130in}}%
\pgfpathlineto{\pgfqpoint{5.635554in}{0.612860in}}%
\pgfpathlineto{\pgfqpoint{5.640444in}{0.612589in}}%
\pgfpathlineto{\pgfqpoint{5.645333in}{0.612319in}}%
\pgfpathlineto{\pgfqpoint{5.650223in}{0.612050in}}%
\pgfpathlineto{\pgfqpoint{5.655113in}{0.611781in}}%
\pgfpathlineto{\pgfqpoint{5.660002in}{0.611513in}}%
\pgfpathlineto{\pgfqpoint{5.664892in}{0.611245in}}%
\pgfpathlineto{\pgfqpoint{5.669782in}{0.610978in}}%
\pgfpathlineto{\pgfqpoint{5.674671in}{0.610711in}}%
\pgfpathlineto{\pgfqpoint{5.679561in}{0.610445in}}%
\pgfpathlineto{\pgfqpoint{5.684451in}{0.610179in}}%
\pgfpathlineto{\pgfqpoint{5.689340in}{0.609914in}}%
\pgfpathlineto{\pgfqpoint{5.694230in}{0.609649in}}%
\pgfpathlineto{\pgfqpoint{5.699120in}{0.609384in}}%
\pgfpathlineto{\pgfqpoint{5.704009in}{0.609121in}}%
\pgfpathlineto{\pgfqpoint{5.708899in}{0.608857in}}%
\pgfpathlineto{\pgfqpoint{5.713789in}{0.608594in}}%
\pgfpathlineto{\pgfqpoint{5.718678in}{0.608332in}}%
\pgfpathlineto{\pgfqpoint{5.723568in}{0.608070in}}%
\pgfpathlineto{\pgfqpoint{5.728458in}{0.607809in}}%
\pgfpathlineto{\pgfqpoint{5.733347in}{0.607548in}}%
\pgfpathlineto{\pgfqpoint{5.738237in}{0.607287in}}%
\pgfpathlineto{\pgfqpoint{5.743127in}{0.607027in}}%
\pgfpathlineto{\pgfqpoint{5.748016in}{0.606768in}}%
\pgfpathlineto{\pgfqpoint{5.752906in}{0.606509in}}%
\pgfpathlineto{\pgfqpoint{5.757796in}{0.606250in}}%
\pgfpathlineto{\pgfqpoint{5.762685in}{0.605992in}}%
\pgfpathlineto{\pgfqpoint{5.767575in}{0.605734in}}%
\pgfpathlineto{\pgfqpoint{5.767575in}{0.605734in}}%
\pgfpathlineto{\pgfqpoint{5.775652in}{0.605317in}}%
\pgfpathlineto{\pgfqpoint{5.783728in}{0.604917in}}%
\pgfpathlineto{\pgfqpoint{5.791805in}{0.604531in}}%
\pgfpathlineto{\pgfqpoint{5.799882in}{0.604159in}}%
\pgfpathlineto{\pgfqpoint{5.807958in}{0.603799in}}%
\pgfpathlineto{\pgfqpoint{5.816035in}{0.603452in}}%
\pgfpathlineto{\pgfqpoint{5.824111in}{0.603116in}}%
\pgfpathlineto{\pgfqpoint{5.832188in}{0.602791in}}%
\pgfpathlineto{\pgfqpoint{5.840265in}{0.602476in}}%
\pgfpathlineto{\pgfqpoint{5.848341in}{0.602170in}}%
\pgfpathlineto{\pgfqpoint{5.856418in}{0.601873in}}%
\pgfpathlineto{\pgfqpoint{5.864495in}{0.601585in}}%
\pgfpathlineto{\pgfqpoint{5.872571in}{0.601304in}}%
\pgfpathlineto{\pgfqpoint{5.880648in}{0.601032in}}%
\pgfpathlineto{\pgfqpoint{5.888725in}{0.600766in}}%
\pgfpathlineto{\pgfqpoint{5.896801in}{0.600507in}}%
\pgfpathlineto{\pgfqpoint{5.904878in}{0.600255in}}%
\pgfpathlineto{\pgfqpoint{5.912954in}{0.600009in}}%
\pgfpathlineto{\pgfqpoint{5.921031in}{0.599769in}}%
\pgfpathlineto{\pgfqpoint{5.929108in}{0.599535in}}%
\pgfpathlineto{\pgfqpoint{5.937184in}{0.599306in}}%
\pgfpathlineto{\pgfqpoint{5.945261in}{0.599083in}}%
\pgfpathlineto{\pgfqpoint{5.953338in}{0.598864in}}%
\pgfpathlineto{\pgfqpoint{5.961414in}{0.598650in}}%
\pgfpathlineto{\pgfqpoint{5.969491in}{0.598441in}}%
\pgfpathlineto{\pgfqpoint{5.977567in}{0.598237in}}%
\pgfpathlineto{\pgfqpoint{5.985644in}{0.598036in}}%
\pgfpathlineto{\pgfqpoint{5.993721in}{0.597840in}}%
\pgfpathlineto{\pgfqpoint{6.001797in}{0.597648in}}%
\pgfpathlineto{\pgfqpoint{6.009874in}{0.597460in}}%
\pgfpathlineto{\pgfqpoint{6.017951in}{0.597275in}}%
\pgfpathlineto{\pgfqpoint{6.026027in}{0.597094in}}%
\pgfpathlineto{\pgfqpoint{6.034104in}{0.596916in}}%
\pgfpathlineto{\pgfqpoint{6.042181in}{0.596742in}}%
\pgfpathlineto{\pgfqpoint{6.050257in}{0.596571in}}%
\pgfpathlineto{\pgfqpoint{6.058334in}{0.596403in}}%
\pgfpathlineto{\pgfqpoint{6.066410in}{0.596238in}}%
\pgfpathlineto{\pgfqpoint{6.074487in}{0.596076in}}%
\pgfpathlineto{\pgfqpoint{6.082564in}{0.595917in}}%
\pgfpathlineto{\pgfqpoint{6.090640in}{0.595761in}}%
\pgfpathlineto{\pgfqpoint{6.098717in}{0.595607in}}%
\pgfpathlineto{\pgfqpoint{6.106794in}{0.595456in}}%
\pgfpathlineto{\pgfqpoint{6.114870in}{0.595307in}}%
\pgfpathlineto{\pgfqpoint{6.122947in}{0.595161in}}%
\pgfpathlineto{\pgfqpoint{6.131023in}{0.595017in}}%
\pgfpathlineto{\pgfqpoint{6.139100in}{0.594876in}}%
\pgfpathlineto{\pgfqpoint{6.147177in}{0.594736in}}%
\pgfpathlineto{\pgfqpoint{6.155253in}{0.594599in}}%
\pgfpathlineto{\pgfqpoint{6.163330in}{0.594465in}}%
\pgfpathlineto{\pgfqpoint{6.171407in}{0.594332in}}%
\pgfpathlineto{\pgfqpoint{6.179483in}{0.594201in}}%
\pgfpathlineto{\pgfqpoint{6.187560in}{0.594072in}}%
\pgfpathlineto{\pgfqpoint{6.195637in}{0.593945in}}%
\pgfpathlineto{\pgfqpoint{6.203713in}{0.593820in}}%
\pgfpathlineto{\pgfqpoint{6.211790in}{0.593697in}}%
\pgfpathlineto{\pgfqpoint{6.219866in}{0.593575in}}%
\pgfpathlineto{\pgfqpoint{6.227943in}{0.593456in}}%
\pgfpathlineto{\pgfqpoint{6.236020in}{0.593337in}}%
\pgfpathlineto{\pgfqpoint{6.244096in}{0.593221in}}%
\pgfpathlineto{\pgfqpoint{6.252173in}{0.593106in}}%
\pgfpathlineto{\pgfqpoint{6.260250in}{0.592993in}}%
\pgfpathlineto{\pgfqpoint{6.268326in}{0.592881in}}%
\pgfpathlineto{\pgfqpoint{6.276403in}{0.592771in}}%
\pgfpathlineto{\pgfqpoint{6.284479in}{0.592662in}}%
\pgfpathlineto{\pgfqpoint{6.292556in}{0.592555in}}%
\pgfpathlineto{\pgfqpoint{6.300633in}{0.592449in}}%
\pgfpathlineto{\pgfqpoint{6.308709in}{0.592345in}}%
\pgfpathlineto{\pgfqpoint{6.316786in}{0.592242in}}%
\pgfpathlineto{\pgfqpoint{6.324863in}{0.592140in}}%
\pgfpathlineto{\pgfqpoint{6.332939in}{0.592039in}}%
\pgfpathlineto{\pgfqpoint{6.341016in}{0.591940in}}%
\pgfpathlineto{\pgfqpoint{6.349093in}{0.591842in}}%
\pgfpathlineto{\pgfqpoint{6.357169in}{0.591745in}}%
\pgfpathlineto{\pgfqpoint{6.365246in}{0.591649in}}%
\pgfpathlineto{\pgfqpoint{6.373322in}{0.591555in}}%
\pgfpathlineto{\pgfqpoint{6.381399in}{0.591462in}}%
\pgfpathlineto{\pgfqpoint{6.389476in}{0.591369in}}%
\pgfpathlineto{\pgfqpoint{6.397552in}{0.591278in}}%
\pgfpathlineto{\pgfqpoint{6.405629in}{0.591188in}}%
\pgfpathlineto{\pgfqpoint{6.413706in}{0.591099in}}%
\pgfpathlineto{\pgfqpoint{6.421782in}{0.591011in}}%
\pgfpathlineto{\pgfqpoint{6.429859in}{0.590924in}}%
\pgfpathlineto{\pgfqpoint{6.437935in}{0.590838in}}%
\pgfpathlineto{\pgfqpoint{6.446012in}{0.590753in}}%
\pgfpathlineto{\pgfqpoint{6.454089in}{0.590669in}}%
\pgfpathlineto{\pgfqpoint{6.462165in}{0.590586in}}%
\pgfpathlineto{\pgfqpoint{6.470242in}{0.590504in}}%
\pgfpathlineto{\pgfqpoint{6.478319in}{0.590422in}}%
\pgfpathlineto{\pgfqpoint{6.486395in}{0.590342in}}%
\pgfpathlineto{\pgfqpoint{6.494472in}{0.590262in}}%
\pgfpathlineto{\pgfqpoint{6.502549in}{0.590184in}}%
\pgfpathlineto{\pgfqpoint{6.510625in}{0.590106in}}%
\pgfpathlineto{\pgfqpoint{6.518702in}{0.590029in}}%
\pgfpathlineto{\pgfqpoint{6.526778in}{0.589952in}}%
\pgfpathlineto{\pgfqpoint{6.534855in}{0.589877in}}%
\pgfpathlineto{\pgfqpoint{6.542932in}{0.589802in}}%
\pgfpathlineto{\pgfqpoint{6.551008in}{0.589729in}}%
\pgfpathlineto{\pgfqpoint{6.559085in}{0.589655in}}%
\pgfpathlineto{\pgfqpoint{6.567162in}{0.589583in}}%
\pgfpathlineto{\pgfqpoint{7.800000in}{0.642834in}}%
\pgfpathlineto{\pgfqpoint{7.790308in}{0.642921in}}%
\pgfpathlineto{\pgfqpoint{7.780616in}{0.643009in}}%
\pgfpathlineto{\pgfqpoint{7.770924in}{0.643098in}}%
\pgfpathlineto{\pgfqpoint{7.761232in}{0.643187in}}%
\pgfpathlineto{\pgfqpoint{7.751540in}{0.643278in}}%
\pgfpathlineto{\pgfqpoint{7.741848in}{0.643369in}}%
\pgfpathlineto{\pgfqpoint{7.732156in}{0.643461in}}%
\pgfpathlineto{\pgfqpoint{7.722464in}{0.643555in}}%
\pgfpathlineto{\pgfqpoint{7.712772in}{0.643649in}}%
\pgfpathlineto{\pgfqpoint{7.703080in}{0.643745in}}%
\pgfpathlineto{\pgfqpoint{7.693388in}{0.643841in}}%
\pgfpathlineto{\pgfqpoint{7.683697in}{0.643939in}}%
\pgfpathlineto{\pgfqpoint{7.674005in}{0.644038in}}%
\pgfpathlineto{\pgfqpoint{7.664313in}{0.644137in}}%
\pgfpathlineto{\pgfqpoint{7.654621in}{0.644238in}}%
\pgfpathlineto{\pgfqpoint{7.644929in}{0.644340in}}%
\pgfpathlineto{\pgfqpoint{7.635237in}{0.644443in}}%
\pgfpathlineto{\pgfqpoint{7.625545in}{0.644548in}}%
\pgfpathlineto{\pgfqpoint{7.615853in}{0.644653in}}%
\pgfpathlineto{\pgfqpoint{7.606161in}{0.644760in}}%
\pgfpathlineto{\pgfqpoint{7.596469in}{0.644868in}}%
\pgfpathlineto{\pgfqpoint{7.586777in}{0.644978in}}%
\pgfpathlineto{\pgfqpoint{7.577085in}{0.645089in}}%
\pgfpathlineto{\pgfqpoint{7.567393in}{0.645201in}}%
\pgfpathlineto{\pgfqpoint{7.557701in}{0.645314in}}%
\pgfpathlineto{\pgfqpoint{7.548009in}{0.645429in}}%
\pgfpathlineto{\pgfqpoint{7.538317in}{0.645545in}}%
\pgfpathlineto{\pgfqpoint{7.528625in}{0.645663in}}%
\pgfpathlineto{\pgfqpoint{7.518933in}{0.645782in}}%
\pgfpathlineto{\pgfqpoint{7.509241in}{0.645902in}}%
\pgfpathlineto{\pgfqpoint{7.499549in}{0.646025in}}%
\pgfpathlineto{\pgfqpoint{7.489857in}{0.646148in}}%
\pgfpathlineto{\pgfqpoint{7.480165in}{0.646274in}}%
\pgfpathlineto{\pgfqpoint{7.470473in}{0.646401in}}%
\pgfpathlineto{\pgfqpoint{7.460781in}{0.646530in}}%
\pgfpathlineto{\pgfqpoint{7.451090in}{0.646660in}}%
\pgfpathlineto{\pgfqpoint{7.441398in}{0.646792in}}%
\pgfpathlineto{\pgfqpoint{7.431706in}{0.646926in}}%
\pgfpathlineto{\pgfqpoint{7.422014in}{0.647062in}}%
\pgfpathlineto{\pgfqpoint{7.412322in}{0.647200in}}%
\pgfpathlineto{\pgfqpoint{7.402630in}{0.647340in}}%
\pgfpathlineto{\pgfqpoint{7.392938in}{0.647481in}}%
\pgfpathlineto{\pgfqpoint{7.383246in}{0.647625in}}%
\pgfpathlineto{\pgfqpoint{7.373554in}{0.647771in}}%
\pgfpathlineto{\pgfqpoint{7.363862in}{0.647919in}}%
\pgfpathlineto{\pgfqpoint{7.354170in}{0.648069in}}%
\pgfpathlineto{\pgfqpoint{7.344478in}{0.648221in}}%
\pgfpathlineto{\pgfqpoint{7.334786in}{0.648376in}}%
\pgfpathlineto{\pgfqpoint{7.325094in}{0.648533in}}%
\pgfpathlineto{\pgfqpoint{7.315402in}{0.648692in}}%
\pgfpathlineto{\pgfqpoint{7.305710in}{0.648854in}}%
\pgfpathlineto{\pgfqpoint{7.296018in}{0.649018in}}%
\pgfpathlineto{\pgfqpoint{7.286326in}{0.649185in}}%
\pgfpathlineto{\pgfqpoint{7.276634in}{0.649355in}}%
\pgfpathlineto{\pgfqpoint{7.266942in}{0.649528in}}%
\pgfpathlineto{\pgfqpoint{7.257250in}{0.649703in}}%
\pgfpathlineto{\pgfqpoint{7.247558in}{0.649881in}}%
\pgfpathlineto{\pgfqpoint{7.237866in}{0.650063in}}%
\pgfpathlineto{\pgfqpoint{7.228174in}{0.650247in}}%
\pgfpathlineto{\pgfqpoint{7.218483in}{0.650435in}}%
\pgfpathlineto{\pgfqpoint{7.208791in}{0.650626in}}%
\pgfpathlineto{\pgfqpoint{7.199099in}{0.650820in}}%
\pgfpathlineto{\pgfqpoint{7.189407in}{0.651018in}}%
\pgfpathlineto{\pgfqpoint{7.179715in}{0.651220in}}%
\pgfpathlineto{\pgfqpoint{7.170023in}{0.651425in}}%
\pgfpathlineto{\pgfqpoint{7.160331in}{0.651634in}}%
\pgfpathlineto{\pgfqpoint{7.150639in}{0.651847in}}%
\pgfpathlineto{\pgfqpoint{7.140947in}{0.652065in}}%
\pgfpathlineto{\pgfqpoint{7.131255in}{0.652286in}}%
\pgfpathlineto{\pgfqpoint{7.121563in}{0.652512in}}%
\pgfpathlineto{\pgfqpoint{7.111871in}{0.652743in}}%
\pgfpathlineto{\pgfqpoint{7.102179in}{0.652978in}}%
\pgfpathlineto{\pgfqpoint{7.092487in}{0.653219in}}%
\pgfpathlineto{\pgfqpoint{7.082795in}{0.653464in}}%
\pgfpathlineto{\pgfqpoint{7.073103in}{0.653715in}}%
\pgfpathlineto{\pgfqpoint{7.063411in}{0.653971in}}%
\pgfpathlineto{\pgfqpoint{7.053719in}{0.654234in}}%
\pgfpathlineto{\pgfqpoint{7.044027in}{0.654502in}}%
\pgfpathlineto{\pgfqpoint{7.034335in}{0.654776in}}%
\pgfpathlineto{\pgfqpoint{7.024643in}{0.655058in}}%
\pgfpathlineto{\pgfqpoint{7.014951in}{0.655346in}}%
\pgfpathlineto{\pgfqpoint{7.005259in}{0.655641in}}%
\pgfpathlineto{\pgfqpoint{6.995567in}{0.655943in}}%
\pgfpathlineto{\pgfqpoint{6.985876in}{0.656254in}}%
\pgfpathlineto{\pgfqpoint{6.976184in}{0.656573in}}%
\pgfpathlineto{\pgfqpoint{6.966492in}{0.656900in}}%
\pgfpathlineto{\pgfqpoint{6.956800in}{0.657236in}}%
\pgfpathlineto{\pgfqpoint{6.947108in}{0.657583in}}%
\pgfpathlineto{\pgfqpoint{6.937416in}{0.657939in}}%
\pgfpathlineto{\pgfqpoint{6.927724in}{0.658306in}}%
\pgfpathlineto{\pgfqpoint{6.918032in}{0.658684in}}%
\pgfpathlineto{\pgfqpoint{6.908340in}{0.659074in}}%
\pgfpathlineto{\pgfqpoint{6.898648in}{0.659477in}}%
\pgfpathlineto{\pgfqpoint{6.888956in}{0.659894in}}%
\pgfpathlineto{\pgfqpoint{6.879264in}{0.660325in}}%
\pgfpathlineto{\pgfqpoint{6.869572in}{0.660772in}}%
\pgfpathlineto{\pgfqpoint{6.859880in}{0.661235in}}%
\pgfpathlineto{\pgfqpoint{6.850188in}{0.661716in}}%
\pgfpathlineto{\pgfqpoint{6.840496in}{0.662216in}}%
\pgfpathlineto{\pgfqpoint{6.840496in}{0.662216in}}%
\pgfpathlineto{\pgfqpoint{6.834629in}{0.662525in}}%
\pgfpathlineto{\pgfqpoint{6.828761in}{0.662835in}}%
\pgfpathlineto{\pgfqpoint{6.822893in}{0.663145in}}%
\pgfpathlineto{\pgfqpoint{6.817026in}{0.663456in}}%
\pgfpathlineto{\pgfqpoint{6.811158in}{0.663767in}}%
\pgfpathlineto{\pgfqpoint{6.805291in}{0.664079in}}%
\pgfpathlineto{\pgfqpoint{6.799423in}{0.664392in}}%
\pgfpathlineto{\pgfqpoint{6.793555in}{0.664705in}}%
\pgfpathlineto{\pgfqpoint{6.787688in}{0.665019in}}%
\pgfpathlineto{\pgfqpoint{6.781820in}{0.665333in}}%
\pgfpathlineto{\pgfqpoint{6.775953in}{0.665648in}}%
\pgfpathlineto{\pgfqpoint{6.770085in}{0.665963in}}%
\pgfpathlineto{\pgfqpoint{6.764217in}{0.666279in}}%
\pgfpathlineto{\pgfqpoint{6.758350in}{0.666596in}}%
\pgfpathlineto{\pgfqpoint{6.752482in}{0.666913in}}%
\pgfpathlineto{\pgfqpoint{6.746614in}{0.667231in}}%
\pgfpathlineto{\pgfqpoint{6.740747in}{0.667549in}}%
\pgfpathlineto{\pgfqpoint{6.734879in}{0.667868in}}%
\pgfpathlineto{\pgfqpoint{6.729012in}{0.668188in}}%
\pgfpathlineto{\pgfqpoint{6.723144in}{0.668508in}}%
\pgfpathlineto{\pgfqpoint{6.717276in}{0.668829in}}%
\pgfpathlineto{\pgfqpoint{6.711409in}{0.669150in}}%
\pgfpathlineto{\pgfqpoint{6.705541in}{0.669472in}}%
\pgfpathlineto{\pgfqpoint{6.699674in}{0.669795in}}%
\pgfpathlineto{\pgfqpoint{6.693806in}{0.670118in}}%
\pgfpathlineto{\pgfqpoint{6.687938in}{0.670442in}}%
\pgfpathlineto{\pgfqpoint{6.682071in}{0.670766in}}%
\pgfpathlineto{\pgfqpoint{6.676203in}{0.671091in}}%
\pgfpathlineto{\pgfqpoint{6.670336in}{0.671417in}}%
\pgfpathlineto{\pgfqpoint{6.664468in}{0.671743in}}%
\pgfpathlineto{\pgfqpoint{6.658600in}{0.672070in}}%
\pgfpathlineto{\pgfqpoint{6.652733in}{0.672397in}}%
\pgfpathlineto{\pgfqpoint{6.646865in}{0.672725in}}%
\pgfpathlineto{\pgfqpoint{6.640998in}{0.673054in}}%
\pgfpathlineto{\pgfqpoint{6.635130in}{0.673383in}}%
\pgfpathlineto{\pgfqpoint{6.629262in}{0.673713in}}%
\pgfpathlineto{\pgfqpoint{6.623395in}{0.674043in}}%
\pgfpathlineto{\pgfqpoint{6.617527in}{0.674375in}}%
\pgfpathlineto{\pgfqpoint{6.611660in}{0.674706in}}%
\pgfpathlineto{\pgfqpoint{6.605792in}{0.675039in}}%
\pgfpathlineto{\pgfqpoint{6.599924in}{0.675372in}}%
\pgfpathlineto{\pgfqpoint{6.594057in}{0.675705in}}%
\pgfpathlineto{\pgfqpoint{6.588189in}{0.676040in}}%
\pgfpathlineto{\pgfqpoint{6.582322in}{0.676375in}}%
\pgfpathlineto{\pgfqpoint{6.576454in}{0.676710in}}%
\pgfpathlineto{\pgfqpoint{6.570586in}{0.677046in}}%
\pgfpathlineto{\pgfqpoint{6.564719in}{0.677383in}}%
\pgfpathlineto{\pgfqpoint{6.558851in}{0.677721in}}%
\pgfpathlineto{\pgfqpoint{6.552984in}{0.678059in}}%
\pgfpathlineto{\pgfqpoint{6.547116in}{0.678398in}}%
\pgfpathlineto{\pgfqpoint{6.541248in}{0.678737in}}%
\pgfpathlineto{\pgfqpoint{6.535381in}{0.679077in}}%
\pgfpathlineto{\pgfqpoint{6.529513in}{0.679418in}}%
\pgfpathlineto{\pgfqpoint{6.523646in}{0.679759in}}%
\pgfpathlineto{\pgfqpoint{6.517778in}{0.680101in}}%
\pgfpathlineto{\pgfqpoint{6.511910in}{0.680444in}}%
\pgfpathlineto{\pgfqpoint{6.506043in}{0.680787in}}%
\pgfpathlineto{\pgfqpoint{6.500175in}{0.681131in}}%
\pgfpathlineto{\pgfqpoint{6.494308in}{0.681476in}}%
\pgfpathlineto{\pgfqpoint{6.488440in}{0.681821in}}%
\pgfpathlineto{\pgfqpoint{6.482572in}{0.682167in}}%
\pgfpathlineto{\pgfqpoint{6.476705in}{0.682514in}}%
\pgfpathlineto{\pgfqpoint{6.470837in}{0.682861in}}%
\pgfpathlineto{\pgfqpoint{6.464969in}{0.683209in}}%
\pgfpathlineto{\pgfqpoint{6.459102in}{0.683558in}}%
\pgfpathlineto{\pgfqpoint{6.453234in}{0.683907in}}%
\pgfpathlineto{\pgfqpoint{6.447367in}{0.684257in}}%
\pgfpathlineto{\pgfqpoint{6.441499in}{0.684608in}}%
\pgfpathlineto{\pgfqpoint{6.435631in}{0.684959in}}%
\pgfpathlineto{\pgfqpoint{6.429764in}{0.685311in}}%
\pgfpathlineto{\pgfqpoint{6.423896in}{0.685664in}}%
\pgfpathlineto{\pgfqpoint{6.418029in}{0.686018in}}%
\pgfpathlineto{\pgfqpoint{6.412161in}{0.686372in}}%
\pgfpathlineto{\pgfqpoint{6.406293in}{0.686727in}}%
\pgfpathlineto{\pgfqpoint{6.400426in}{0.687082in}}%
\pgfpathlineto{\pgfqpoint{6.394558in}{0.687438in}}%
\pgfpathlineto{\pgfqpoint{6.388691in}{0.687795in}}%
\pgfpathlineto{\pgfqpoint{6.382823in}{0.688153in}}%
\pgfpathlineto{\pgfqpoint{6.376955in}{0.688511in}}%
\pgfpathlineto{\pgfqpoint{6.371088in}{0.688870in}}%
\pgfpathlineto{\pgfqpoint{6.365220in}{0.689230in}}%
\pgfpathlineto{\pgfqpoint{6.359353in}{0.689590in}}%
\pgfpathlineto{\pgfqpoint{6.353485in}{0.689952in}}%
\pgfpathlineto{\pgfqpoint{6.347617in}{0.690313in}}%
\pgfpathlineto{\pgfqpoint{6.341750in}{0.690676in}}%
\pgfpathlineto{\pgfqpoint{6.335882in}{0.691039in}}%
\pgfpathlineto{\pgfqpoint{6.330015in}{0.691403in}}%
\pgfpathlineto{\pgfqpoint{6.324147in}{0.691768in}}%
\pgfpathlineto{\pgfqpoint{6.318279in}{0.692134in}}%
\pgfpathlineto{\pgfqpoint{6.312412in}{0.692500in}}%
\pgfpathlineto{\pgfqpoint{6.306544in}{0.692867in}}%
\pgfpathlineto{\pgfqpoint{6.300677in}{0.693234in}}%
\pgfpathlineto{\pgfqpoint{6.294809in}{0.693603in}}%
\pgfpathlineto{\pgfqpoint{6.288941in}{0.693972in}}%
\pgfpathlineto{\pgfqpoint{6.283074in}{0.694342in}}%
\pgfpathlineto{\pgfqpoint{6.277206in}{0.694712in}}%
\pgfpathlineto{\pgfqpoint{6.271339in}{0.695084in}}%
\pgfpathlineto{\pgfqpoint{6.265471in}{0.695456in}}%
\pgfpathlineto{\pgfqpoint{6.259603in}{0.695829in}}%
\pgfpathlineto{\pgfqpoint{6.253736in}{0.696202in}}%
\pgfpathlineto{\pgfqpoint{6.247868in}{0.696576in}}%
\pgfpathlineto{\pgfqpoint{6.242001in}{0.696952in}}%
\pgfpathlineto{\pgfqpoint{6.236133in}{0.697327in}}%
\pgfpathlineto{\pgfqpoint{6.230265in}{0.697704in}}%
\pgfpathlineto{\pgfqpoint{6.224398in}{0.698081in}}%
\pgfpathlineto{\pgfqpoint{6.218530in}{0.698459in}}%
\pgfpathlineto{\pgfqpoint{6.212662in}{0.698838in}}%
\pgfpathlineto{\pgfqpoint{6.206795in}{0.699218in}}%
\pgfpathlineto{\pgfqpoint{6.200927in}{0.699598in}}%
\pgfpathlineto{\pgfqpoint{6.195060in}{0.699980in}}%
\pgfpathlineto{\pgfqpoint{6.189192in}{0.700361in}}%
\pgfpathlineto{\pgfqpoint{6.183324in}{0.700744in}}%
\pgfpathlineto{\pgfqpoint{6.177457in}{0.701128in}}%
\pgfpathlineto{\pgfqpoint{6.171589in}{0.701512in}}%
\pgfpathlineto{\pgfqpoint{6.165722in}{0.701897in}}%
\pgfpathlineto{\pgfqpoint{6.159854in}{0.702283in}}%
\pgfpathlineto{\pgfqpoint{6.153986in}{0.702670in}}%
\pgfpathlineto{\pgfqpoint{6.148119in}{0.703057in}}%
\pgfpathlineto{\pgfqpoint{6.142251in}{0.703445in}}%
\pgfpathlineto{\pgfqpoint{6.136384in}{0.703834in}}%
\pgfpathlineto{\pgfqpoint{6.130516in}{0.704224in}}%
\pgfpathlineto{\pgfqpoint{6.124648in}{0.704615in}}%
\pgfpathlineto{\pgfqpoint{6.118781in}{0.705006in}}%
\pgfpathlineto{\pgfqpoint{6.112913in}{0.705398in}}%
\pgfpathlineto{\pgfqpoint{6.107046in}{0.705791in}}%
\pgfpathlineto{\pgfqpoint{6.101178in}{0.706185in}}%
\pgfpathlineto{\pgfqpoint{6.095310in}{0.706580in}}%
\pgfpathlineto{\pgfqpoint{6.089443in}{0.706975in}}%
\pgfpathlineto{\pgfqpoint{6.083575in}{0.707372in}}%
\pgfpathlineto{\pgfqpoint{6.077708in}{0.707769in}}%
\pgfpathlineto{\pgfqpoint{6.071840in}{0.708167in}}%
\pgfpathlineto{\pgfqpoint{6.065972in}{0.708565in}}%
\pgfpathlineto{\pgfqpoint{6.060105in}{0.708965in}}%
\pgfpathlineto{\pgfqpoint{6.054237in}{0.709365in}}%
\pgfpathlineto{\pgfqpoint{6.048370in}{0.709766in}}%
\pgfpathlineto{\pgfqpoint{6.042502in}{0.710169in}}%
\pgfpathlineto{\pgfqpoint{6.036634in}{0.710571in}}%
\pgfpathlineto{\pgfqpoint{6.030767in}{0.710975in}}%
\pgfpathlineto{\pgfqpoint{6.024899in}{0.711380in}}%
\pgfpathlineto{\pgfqpoint{6.019032in}{0.711785in}}%
\pgfpathlineto{\pgfqpoint{6.013164in}{0.712192in}}%
\pgfpathlineto{\pgfqpoint{6.007296in}{0.712599in}}%
\pgfpathlineto{\pgfqpoint{6.001429in}{0.713007in}}%
\pgfpathlineto{\pgfqpoint{5.995561in}{0.713415in}}%
\pgfpathlineto{\pgfqpoint{5.989694in}{0.713825in}}%
\pgfpathlineto{\pgfqpoint{5.983826in}{0.714236in}}%
\pgfpathlineto{\pgfqpoint{5.977958in}{0.714647in}}%
\pgfpathlineto{\pgfqpoint{5.972091in}{0.715059in}}%
\pgfpathlineto{\pgfqpoint{5.966223in}{0.715473in}}%
\pgfpathlineto{\pgfqpoint{5.960355in}{0.715887in}}%
\pgfpathlineto{\pgfqpoint{5.954488in}{0.716302in}}%
\pgfpathlineto{\pgfqpoint{5.948620in}{0.716717in}}%
\pgfpathlineto{\pgfqpoint{5.942753in}{0.717134in}}%
\pgfpathlineto{\pgfqpoint{5.936885in}{0.717552in}}%
\pgfpathlineto{\pgfqpoint{5.931017in}{0.717970in}}%
\pgfpathlineto{\pgfqpoint{5.925150in}{0.718389in}}%
\pgfpathlineto{\pgfqpoint{5.919282in}{0.718810in}}%
\pgfpathlineto{\pgfqpoint{5.913415in}{0.719231in}}%
\pgfpathlineto{\pgfqpoint{5.907547in}{0.719653in}}%
\pgfpathlineto{\pgfqpoint{5.901679in}{0.720076in}}%
\pgfpathlineto{\pgfqpoint{5.895812in}{0.720499in}}%
\pgfpathlineto{\pgfqpoint{5.889944in}{0.720924in}}%
\pgfpathlineto{\pgfqpoint{5.884077in}{0.721350in}}%
\pgfpathlineto{\pgfqpoint{5.878209in}{0.721776in}}%
\pgfpathlineto{\pgfqpoint{5.872341in}{0.722204in}}%
\pgfpathlineto{\pgfqpoint{5.866474in}{0.722632in}}%
\pgfpathlineto{\pgfqpoint{5.860606in}{0.723061in}}%
\pgfpathlineto{\pgfqpoint{5.854739in}{0.723492in}}%
\pgfpathlineto{\pgfqpoint{5.848871in}{0.723923in}}%
\pgfpathlineto{\pgfqpoint{5.843003in}{0.724355in}}%
\pgfpathlineto{\pgfqpoint{5.837136in}{0.724788in}}%
\pgfpathlineto{\pgfqpoint{5.831268in}{0.725222in}}%
\pgfpathlineto{\pgfqpoint{5.825401in}{0.725657in}}%
\pgfpathlineto{\pgfqpoint{5.819533in}{0.726093in}}%
\pgfpathlineto{\pgfqpoint{5.813665in}{0.726529in}}%
\pgfpathlineto{\pgfqpoint{5.807798in}{0.726967in}}%
\pgfpathlineto{\pgfqpoint{5.801930in}{0.727406in}}%
\pgfpathlineto{\pgfqpoint{5.796063in}{0.727845in}}%
\pgfpathlineto{\pgfqpoint{5.790195in}{0.728286in}}%
\pgfpathlineto{\pgfqpoint{5.784327in}{0.728728in}}%
\pgfpathlineto{\pgfqpoint{5.778460in}{0.729170in}}%
\pgfpathlineto{\pgfqpoint{5.772592in}{0.729614in}}%
\pgfpathlineto{\pgfqpoint{5.766725in}{0.730058in}}%
\pgfpathlineto{\pgfqpoint{5.760857in}{0.730503in}}%
\pgfpathlineto{\pgfqpoint{5.754989in}{0.730950in}}%
\pgfpathlineto{\pgfqpoint{5.749122in}{0.731397in}}%
\pgfpathlineto{\pgfqpoint{5.743254in}{0.731846in}}%
\pgfpathlineto{\pgfqpoint{5.737387in}{0.732295in}}%
\pgfpathlineto{\pgfqpoint{5.731519in}{0.732745in}}%
\pgfpathlineto{\pgfqpoint{5.725651in}{0.733197in}}%
\pgfpathlineto{\pgfqpoint{5.719784in}{0.733649in}}%
\pgfpathlineto{\pgfqpoint{5.713916in}{0.734102in}}%
\pgfpathlineto{\pgfqpoint{5.708049in}{0.734557in}}%
\pgfpathlineto{\pgfqpoint{5.702181in}{0.735012in}}%
\pgfpathlineto{\pgfqpoint{5.696313in}{0.735468in}}%
\pgfpathlineto{\pgfqpoint{5.690446in}{0.735926in}}%
\pgfpathlineto{\pgfqpoint{5.684578in}{0.736384in}}%
\pgfpathlineto{\pgfqpoint{5.678710in}{0.736843in}}%
\pgfpathlineto{\pgfqpoint{5.672843in}{0.737304in}}%
\pgfpathlineto{\pgfqpoint{5.666975in}{0.737765in}}%
\pgfpathlineto{\pgfqpoint{5.661108in}{0.738228in}}%
\pgfpathlineto{\pgfqpoint{5.655240in}{0.738691in}}%
\pgfpathlineto{\pgfqpoint{5.649372in}{0.739156in}}%
\pgfpathlineto{\pgfqpoint{5.643505in}{0.739621in}}%
\pgfpathlineto{\pgfqpoint{5.637637in}{0.740088in}}%
\pgfpathlineto{\pgfqpoint{5.631770in}{0.740556in}}%
\pgfpathlineto{\pgfqpoint{5.625902in}{0.741024in}}%
\pgfpathlineto{\pgfqpoint{5.620034in}{0.741494in}}%
\pgfpathlineto{\pgfqpoint{5.614167in}{0.741965in}}%
\pgfpathlineto{\pgfqpoint{5.608299in}{0.742437in}}%
\pgfpathlineto{\pgfqpoint{5.602432in}{0.742910in}}%
\pgfpathlineto{\pgfqpoint{5.596564in}{0.743384in}}%
\pgfpathlineto{\pgfqpoint{5.590696in}{0.743859in}}%
\pgfpathlineto{\pgfqpoint{5.584829in}{0.744335in}}%
\pgfpathlineto{\pgfqpoint{5.578961in}{0.744813in}}%
\pgfpathlineto{\pgfqpoint{5.573094in}{0.745291in}}%
\pgfpathlineto{\pgfqpoint{5.567226in}{0.745770in}}%
\pgfpathlineto{\pgfqpoint{5.561358in}{0.746251in}}%
\pgfpathlineto{\pgfqpoint{5.555491in}{0.746732in}}%
\pgfpathlineto{\pgfqpoint{5.549623in}{0.747215in}}%
\pgfpathlineto{\pgfqpoint{5.543756in}{0.747699in}}%
\pgfpathlineto{\pgfqpoint{5.537888in}{0.748184in}}%
\pgfpathlineto{\pgfqpoint{5.532020in}{0.748670in}}%
\pgfpathlineto{\pgfqpoint{5.526153in}{0.749157in}}%
\pgfpathlineto{\pgfqpoint{5.520285in}{0.749645in}}%
\pgfpathlineto{\pgfqpoint{5.514418in}{0.750135in}}%
\pgfpathlineto{\pgfqpoint{5.508550in}{0.750625in}}%
\pgfpathlineto{\pgfqpoint{5.502682in}{0.751117in}}%
\pgfpathlineto{\pgfqpoint{5.496815in}{0.751610in}}%
\pgfpathlineto{\pgfqpoint{5.490947in}{0.752104in}}%
\pgfpathlineto{\pgfqpoint{5.485080in}{0.752599in}}%
\pgfpathlineto{\pgfqpoint{5.479212in}{0.753095in}}%
\pgfpathlineto{\pgfqpoint{5.473344in}{0.753592in}}%
\pgfpathlineto{\pgfqpoint{5.467477in}{0.754091in}}%
\pgfpathlineto{\pgfqpoint{5.461609in}{0.754590in}}%
\pgfpathlineto{\pgfqpoint{5.455742in}{0.755091in}}%
\pgfpathlineto{\pgfqpoint{5.449874in}{0.755593in}}%
\pgfpathlineto{\pgfqpoint{5.444006in}{0.756096in}}%
\pgfpathlineto{\pgfqpoint{5.438139in}{0.756601in}}%
\pgfpathlineto{\pgfqpoint{5.432271in}{0.757106in}}%
\pgfpathlineto{\pgfqpoint{5.426403in}{0.757613in}}%
\pgfpathlineto{\pgfqpoint{5.420536in}{0.758121in}}%
\pgfpathlineto{\pgfqpoint{5.414668in}{0.758630in}}%
\pgfpathlineto{\pgfqpoint{5.408801in}{0.759140in}}%
\pgfpathlineto{\pgfqpoint{5.402933in}{0.759651in}}%
\pgfpathlineto{\pgfqpoint{5.397065in}{0.760164in}}%
\pgfpathlineto{\pgfqpoint{5.391198in}{0.760678in}}%
\pgfpathlineto{\pgfqpoint{5.385330in}{0.761193in}}%
\pgfpathlineto{\pgfqpoint{5.379463in}{0.761709in}}%
\pgfpathlineto{\pgfqpoint{5.373595in}{0.762227in}}%
\pgfpathlineto{\pgfqpoint{5.367727in}{0.762745in}}%
\pgfpathlineto{\pgfqpoint{5.361860in}{0.763265in}}%
\pgfpathlineto{\pgfqpoint{5.355992in}{0.763786in}}%
\pgfpathlineto{\pgfqpoint{5.350125in}{0.764309in}}%
\pgfpathlineto{\pgfqpoint{5.344257in}{0.764833in}}%
\pgfpathlineto{\pgfqpoint{5.338389in}{0.765357in}}%
\pgfpathlineto{\pgfqpoint{5.332522in}{0.765884in}}%
\pgfpathlineto{\pgfqpoint{5.326654in}{0.766411in}}%
\pgfpathlineto{\pgfqpoint{5.320787in}{0.766940in}}%
\pgfpathlineto{\pgfqpoint{5.314919in}{0.767470in}}%
\pgfpathlineto{\pgfqpoint{5.309051in}{0.768001in}}%
\pgfpathlineto{\pgfqpoint{5.303184in}{0.768533in}}%
\pgfpathlineto{\pgfqpoint{5.297316in}{0.769067in}}%
\pgfpathlineto{\pgfqpoint{5.291449in}{0.769602in}}%
\pgfpathlineto{\pgfqpoint{5.285581in}{0.770138in}}%
\pgfpathlineto{\pgfqpoint{5.279713in}{0.770676in}}%
\pgfpathlineto{\pgfqpoint{5.273846in}{0.771215in}}%
\pgfpathlineto{\pgfqpoint{5.267978in}{0.771755in}}%
\pgfpathlineto{\pgfqpoint{5.262111in}{0.772296in}}%
\pgfpathlineto{\pgfqpoint{5.256243in}{0.772839in}}%
\pgfpathlineto{\pgfqpoint{5.250375in}{0.773383in}}%
\pgfpathlineto{\pgfqpoint{5.244508in}{0.773929in}}%
\pgfpathlineto{\pgfqpoint{5.238640in}{0.774476in}}%
\pgfpathlineto{\pgfqpoint{5.232773in}{0.775024in}}%
\pgfpathlineto{\pgfqpoint{5.226905in}{0.775573in}}%
\pgfpathlineto{\pgfqpoint{5.221037in}{0.776124in}}%
\pgfpathlineto{\pgfqpoint{5.215170in}{0.776676in}}%
\pgfpathlineto{\pgfqpoint{5.209302in}{0.777229in}}%
\pgfpathlineto{\pgfqpoint{5.203435in}{0.777784in}}%
\pgfpathlineto{\pgfqpoint{5.197567in}{0.778340in}}%
\pgfpathlineto{\pgfqpoint{5.191699in}{0.778898in}}%
\pgfpathlineto{\pgfqpoint{5.185832in}{0.779457in}}%
\pgfpathlineto{\pgfqpoint{5.179964in}{0.780017in}}%
\pgfpathlineto{\pgfqpoint{5.174096in}{0.780579in}}%
\pgfpathlineto{\pgfqpoint{5.168229in}{0.781142in}}%
\pgfpathlineto{\pgfqpoint{5.162361in}{0.781706in}}%
\pgfpathlineto{\pgfqpoint{5.156494in}{0.782272in}}%
\pgfpathlineto{\pgfqpoint{5.150626in}{0.782839in}}%
\pgfpathlineto{\pgfqpoint{5.144758in}{0.783408in}}%
\pgfpathlineto{\pgfqpoint{5.138891in}{0.783978in}}%
\pgfpathlineto{\pgfqpoint{5.133023in}{0.784549in}}%
\pgfpathlineto{\pgfqpoint{5.127156in}{0.785122in}}%
\pgfpathlineto{\pgfqpoint{5.121288in}{0.785696in}}%
\pgfpathlineto{\pgfqpoint{5.115420in}{0.786272in}}%
\pgfpathlineto{\pgfqpoint{5.109553in}{0.786849in}}%
\pgfpathlineto{\pgfqpoint{5.103685in}{0.787428in}}%
\pgfpathlineto{\pgfqpoint{5.097818in}{0.788008in}}%
\pgfpathlineto{\pgfqpoint{5.091950in}{0.788589in}}%
\pgfpathlineto{\pgfqpoint{5.086082in}{0.789172in}}%
\pgfpathlineto{\pgfqpoint{5.080215in}{0.789757in}}%
\pgfpathlineto{\pgfqpoint{5.074347in}{0.790343in}}%
\pgfpathlineto{\pgfqpoint{5.068480in}{0.790930in}}%
\pgfpathlineto{\pgfqpoint{5.062612in}{0.791519in}}%
\pgfpathlineto{\pgfqpoint{5.056744in}{0.792109in}}%
\pgfpathlineto{\pgfqpoint{5.050877in}{0.792701in}}%
\pgfpathlineto{\pgfqpoint{5.045009in}{0.793294in}}%
\pgfpathlineto{\pgfqpoint{5.039142in}{0.793889in}}%
\pgfpathlineto{\pgfqpoint{5.033274in}{0.794485in}}%
\pgfpathlineto{\pgfqpoint{5.027406in}{0.795083in}}%
\pgfpathlineto{\pgfqpoint{5.021539in}{0.795682in}}%
\pgfpathlineto{\pgfqpoint{5.015671in}{0.796283in}}%
\pgfpathlineto{\pgfqpoint{5.009804in}{0.796886in}}%
\pgfpathlineto{\pgfqpoint{5.003936in}{0.797490in}}%
\pgfpathlineto{\pgfqpoint{4.998068in}{0.798095in}}%
\pgfpathlineto{\pgfqpoint{4.992201in}{0.798702in}}%
\pgfpathlineto{\pgfqpoint{4.986333in}{0.799311in}}%
\pgfpathlineto{\pgfqpoint{4.980466in}{0.799921in}}%
\pgfpathlineto{\pgfqpoint{4.974598in}{0.800532in}}%
\pgfpathlineto{\pgfqpoint{4.968730in}{0.801146in}}%
\pgfpathlineto{\pgfqpoint{4.962863in}{0.801761in}}%
\pgfpathlineto{\pgfqpoint{4.956995in}{0.802377in}}%
\pgfpathlineto{\pgfqpoint{4.951128in}{0.802995in}}%
\pgfpathlineto{\pgfqpoint{4.945260in}{0.803615in}}%
\pgfpathlineto{\pgfqpoint{4.939392in}{0.804236in}}%
\pgfpathlineto{\pgfqpoint{4.933525in}{0.804859in}}%
\pgfpathlineto{\pgfqpoint{4.927657in}{0.805483in}}%
\pgfpathlineto{\pgfqpoint{4.921790in}{0.806109in}}%
\pgfpathlineto{\pgfqpoint{4.915922in}{0.806737in}}%
\pgfpathlineto{\pgfqpoint{4.910054in}{0.807366in}}%
\pgfpathlineto{\pgfqpoint{4.904187in}{0.807997in}}%
\pgfpathlineto{\pgfqpoint{4.898319in}{0.808630in}}%
\pgfpathlineto{\pgfqpoint{4.892451in}{0.809264in}}%
\pgfpathlineto{\pgfqpoint{4.886584in}{0.809900in}}%
\pgfpathlineto{\pgfqpoint{4.880716in}{0.810538in}}%
\pgfpathlineto{\pgfqpoint{4.874849in}{0.811177in}}%
\pgfpathlineto{\pgfqpoint{4.868981in}{0.811818in}}%
\pgfpathlineto{\pgfqpoint{4.863113in}{0.812461in}}%
\pgfpathlineto{\pgfqpoint{4.857246in}{0.813105in}}%
\pgfpathlineto{\pgfqpoint{4.851378in}{0.813751in}}%
\pgfpathlineto{\pgfqpoint{4.845511in}{0.814399in}}%
\pgfpathlineto{\pgfqpoint{4.839643in}{0.815048in}}%
\pgfpathlineto{\pgfqpoint{4.833775in}{0.815699in}}%
\pgfpathlineto{\pgfqpoint{4.827908in}{0.816352in}}%
\pgfpathlineto{\pgfqpoint{4.822040in}{0.817007in}}%
\pgfpathlineto{\pgfqpoint{4.816173in}{0.817663in}}%
\pgfpathlineto{\pgfqpoint{4.810305in}{0.818321in}}%
\pgfpathlineto{\pgfqpoint{4.804437in}{0.818981in}}%
\pgfpathlineto{\pgfqpoint{4.798570in}{0.819643in}}%
\pgfpathlineto{\pgfqpoint{4.792702in}{0.820306in}}%
\pgfpathlineto{\pgfqpoint{4.786835in}{0.820971in}}%
\pgfpathlineto{\pgfqpoint{4.780967in}{0.821638in}}%
\pgfpathlineto{\pgfqpoint{4.775099in}{0.822307in}}%
\pgfpathlineto{\pgfqpoint{4.769232in}{0.822978in}}%
\pgfpathlineto{\pgfqpoint{4.763364in}{0.823650in}}%
\pgfpathlineto{\pgfqpoint{4.757497in}{0.824324in}}%
\pgfpathlineto{\pgfqpoint{4.751629in}{0.825000in}}%
\pgfpathlineto{\pgfqpoint{4.745761in}{0.825678in}}%
\pgfpathlineto{\pgfqpoint{4.739894in}{0.826358in}}%
\pgfpathlineto{\pgfqpoint{4.734026in}{0.827039in}}%
\pgfpathlineto{\pgfqpoint{4.728159in}{0.827723in}}%
\pgfpathlineto{\pgfqpoint{4.722291in}{0.828408in}}%
\pgfpathlineto{\pgfqpoint{4.716423in}{0.829095in}}%
\pgfpathlineto{\pgfqpoint{4.710556in}{0.829784in}}%
\pgfpathlineto{\pgfqpoint{4.704688in}{0.830475in}}%
\pgfpathlineto{\pgfqpoint{4.698821in}{0.831167in}}%
\pgfpathlineto{\pgfqpoint{4.692953in}{0.831862in}}%
\pgfpathlineto{\pgfqpoint{4.687085in}{0.832558in}}%
\pgfpathlineto{\pgfqpoint{4.681218in}{0.833257in}}%
\pgfpathlineto{\pgfqpoint{4.675350in}{0.833957in}}%
\pgfpathlineto{\pgfqpoint{4.669483in}{0.834659in}}%
\pgfpathlineto{\pgfqpoint{4.663615in}{0.835363in}}%
\pgfpathlineto{\pgfqpoint{4.657747in}{0.836070in}}%
\pgfpathlineto{\pgfqpoint{4.651880in}{0.836778in}}%
\pgfpathlineto{\pgfqpoint{4.646012in}{0.837488in}}%
\pgfpathlineto{\pgfqpoint{4.640144in}{0.838200in}}%
\pgfpathlineto{\pgfqpoint{4.634277in}{0.838914in}}%
\pgfpathlineto{\pgfqpoint{4.628409in}{0.839630in}}%
\pgfpathlineto{\pgfqpoint{4.622542in}{0.840348in}}%
\pgfpathlineto{\pgfqpoint{4.616674in}{0.841068in}}%
\pgfpathlineto{\pgfqpoint{4.610806in}{0.841790in}}%
\pgfpathlineto{\pgfqpoint{4.604939in}{0.842514in}}%
\pgfpathlineto{\pgfqpoint{4.599071in}{0.843240in}}%
\pgfpathlineto{\pgfqpoint{4.593204in}{0.843968in}}%
\pgfpathlineto{\pgfqpoint{4.587336in}{0.844698in}}%
\pgfpathlineto{\pgfqpoint{4.581468in}{0.845430in}}%
\pgfpathlineto{\pgfqpoint{4.575601in}{0.846164in}}%
\pgfpathlineto{\pgfqpoint{4.569733in}{0.846900in}}%
\pgfpathlineto{\pgfqpoint{4.563866in}{0.847639in}}%
\pgfpathlineto{\pgfqpoint{4.557998in}{0.848379in}}%
\pgfpathlineto{\pgfqpoint{4.552130in}{0.849121in}}%
\pgfpathlineto{\pgfqpoint{4.546263in}{0.849866in}}%
\pgfpathlineto{\pgfqpoint{4.540395in}{0.850613in}}%
\pgfpathlineto{\pgfqpoint{4.534528in}{0.851362in}}%
\pgfpathlineto{\pgfqpoint{4.528660in}{0.852113in}}%
\pgfpathlineto{\pgfqpoint{4.522792in}{0.852866in}}%
\pgfpathlineto{\pgfqpoint{4.516925in}{0.853621in}}%
\pgfpathlineto{\pgfqpoint{4.511057in}{0.854378in}}%
\pgfpathlineto{\pgfqpoint{4.505190in}{0.855138in}}%
\pgfpathlineto{\pgfqpoint{4.499322in}{0.855900in}}%
\pgfpathlineto{\pgfqpoint{4.493454in}{0.856664in}}%
\pgfpathlineto{\pgfqpoint{4.487587in}{0.857430in}}%
\pgfpathlineto{\pgfqpoint{4.481719in}{0.858198in}}%
\pgfpathlineto{\pgfqpoint{4.475852in}{0.858969in}}%
\pgfpathlineto{\pgfqpoint{4.469984in}{0.859742in}}%
\pgfpathlineto{\pgfqpoint{4.464116in}{0.860517in}}%
\pgfpathlineto{\pgfqpoint{4.458249in}{0.861294in}}%
\pgfpathlineto{\pgfqpoint{4.452381in}{0.862073in}}%
\pgfpathlineto{\pgfqpoint{4.446514in}{0.862855in}}%
\pgfpathlineto{\pgfqpoint{4.440646in}{0.863639in}}%
\pgfpathlineto{\pgfqpoint{4.434778in}{0.864426in}}%
\pgfpathlineto{\pgfqpoint{4.428911in}{0.865214in}}%
\pgfpathlineto{\pgfqpoint{4.423043in}{0.866005in}}%
\pgfpathlineto{\pgfqpoint{4.417176in}{0.866798in}}%
\pgfpathlineto{\pgfqpoint{4.411308in}{0.867594in}}%
\pgfpathlineto{\pgfqpoint{4.405440in}{0.868392in}}%
\pgfpathlineto{\pgfqpoint{4.399573in}{0.869192in}}%
\pgfpathlineto{\pgfqpoint{4.393705in}{0.869995in}}%
\pgfpathlineto{\pgfqpoint{4.387837in}{0.870800in}}%
\pgfpathlineto{\pgfqpoint{4.381970in}{0.871607in}}%
\pgfpathlineto{\pgfqpoint{4.376102in}{0.872417in}}%
\pgfpathlineto{\pgfqpoint{4.370235in}{0.873229in}}%
\pgfpathlineto{\pgfqpoint{4.364367in}{0.874043in}}%
\pgfpathlineto{\pgfqpoint{4.358499in}{0.874860in}}%
\pgfpathlineto{\pgfqpoint{4.352632in}{0.875680in}}%
\pgfpathlineto{\pgfqpoint{4.346764in}{0.876501in}}%
\pgfpathlineto{\pgfqpoint{4.340897in}{0.877326in}}%
\pgfpathlineto{\pgfqpoint{4.335029in}{0.878152in}}%
\pgfpathlineto{\pgfqpoint{4.329161in}{0.878981in}}%
\pgfpathlineto{\pgfqpoint{4.323294in}{0.879813in}}%
\pgfpathlineto{\pgfqpoint{4.317426in}{0.880647in}}%
\pgfpathlineto{\pgfqpoint{4.311559in}{0.881484in}}%
\pgfpathlineto{\pgfqpoint{4.305691in}{0.882323in}}%
\pgfpathlineto{\pgfqpoint{4.299823in}{0.883165in}}%
\pgfpathlineto{\pgfqpoint{4.293956in}{0.884009in}}%
\pgfpathlineto{\pgfqpoint{4.288088in}{0.884856in}}%
\pgfpathlineto{\pgfqpoint{4.282221in}{0.885705in}}%
\pgfpathlineto{\pgfqpoint{4.276353in}{0.886557in}}%
\pgfpathlineto{\pgfqpoint{4.270485in}{0.887412in}}%
\pgfpathlineto{\pgfqpoint{4.264618in}{0.888269in}}%
\pgfpathlineto{\pgfqpoint{4.258750in}{0.889128in}}%
\pgfpathlineto{\pgfqpoint{4.252883in}{0.889991in}}%
\pgfpathlineto{\pgfqpoint{4.247015in}{0.890856in}}%
\pgfpathlineto{\pgfqpoint{4.241147in}{0.891723in}}%
\pgfpathlineto{\pgfqpoint{4.235280in}{0.892594in}}%
\pgfpathlineto{\pgfqpoint{4.229412in}{0.893466in}}%
\pgfpathlineto{\pgfqpoint{4.223545in}{0.894342in}}%
\pgfpathlineto{\pgfqpoint{4.217677in}{0.895220in}}%
\pgfpathlineto{\pgfqpoint{4.211809in}{0.896101in}}%
\pgfpathlineto{\pgfqpoint{4.205942in}{0.896985in}}%
\pgfpathlineto{\pgfqpoint{4.200074in}{0.897872in}}%
\pgfpathlineto{\pgfqpoint{4.194207in}{0.898761in}}%
\pgfpathlineto{\pgfqpoint{4.188339in}{0.899653in}}%
\pgfpathlineto{\pgfqpoint{4.182471in}{0.900548in}}%
\pgfpathlineto{\pgfqpoint{4.176604in}{0.901445in}}%
\pgfpathlineto{\pgfqpoint{4.170736in}{0.902345in}}%
\pgfpathlineto{\pgfqpoint{4.164869in}{0.903248in}}%
\pgfpathlineto{\pgfqpoint{4.159001in}{0.904154in}}%
\pgfpathlineto{\pgfqpoint{4.153133in}{0.905063in}}%
\pgfpathlineto{\pgfqpoint{4.147266in}{0.905975in}}%
\pgfpathlineto{\pgfqpoint{4.141398in}{0.906889in}}%
\pgfpathlineto{\pgfqpoint{4.135530in}{0.907807in}}%
\pgfpathlineto{\pgfqpoint{4.129663in}{0.908727in}}%
\pgfpathlineto{\pgfqpoint{4.123795in}{0.909650in}}%
\pgfpathlineto{\pgfqpoint{4.117928in}{0.910576in}}%
\pgfpathlineto{\pgfqpoint{4.112060in}{0.911505in}}%
\pgfpathlineto{\pgfqpoint{4.106192in}{0.912437in}}%
\pgfpathlineto{\pgfqpoint{4.100325in}{0.913372in}}%
\pgfpathlineto{\pgfqpoint{4.094457in}{0.914310in}}%
\pgfpathlineto{\pgfqpoint{4.088590in}{0.915251in}}%
\pgfpathlineto{\pgfqpoint{4.082722in}{0.916195in}}%
\pgfpathlineto{\pgfqpoint{4.076854in}{0.917141in}}%
\pgfpathlineto{\pgfqpoint{4.070987in}{0.918091in}}%
\pgfpathlineto{\pgfqpoint{4.065119in}{0.919044in}}%
\pgfpathlineto{\pgfqpoint{4.059252in}{0.920000in}}%
\pgfpathlineto{\pgfqpoint{4.053384in}{0.920959in}}%
\pgfpathlineto{\pgfqpoint{4.047516in}{0.921922in}}%
\pgfpathlineto{\pgfqpoint{4.041649in}{0.922887in}}%
\pgfpathlineto{\pgfqpoint{4.035781in}{0.923855in}}%
\pgfpathlineto{\pgfqpoint{4.029914in}{0.924827in}}%
\pgfpathlineto{\pgfqpoint{4.024046in}{0.925801in}}%
\pgfpathlineto{\pgfqpoint{4.018178in}{0.926779in}}%
\pgfpathlineto{\pgfqpoint{4.012311in}{0.927760in}}%
\pgfpathlineto{\pgfqpoint{4.006443in}{0.928745in}}%
\pgfpathlineto{\pgfqpoint{4.000576in}{0.929732in}}%
\pgfpathlineto{\pgfqpoint{3.994708in}{0.930723in}}%
\pgfpathlineto{\pgfqpoint{3.988840in}{0.931716in}}%
\pgfpathlineto{\pgfqpoint{3.982973in}{0.932714in}}%
\pgfpathlineto{\pgfqpoint{3.977105in}{0.933714in}}%
\pgfpathlineto{\pgfqpoint{3.971238in}{0.934718in}}%
\pgfpathlineto{\pgfqpoint{3.965370in}{0.935725in}}%
\pgfpathlineto{\pgfqpoint{3.959502in}{0.936735in}}%
\pgfpathlineto{\pgfqpoint{3.953635in}{0.937749in}}%
\pgfpathlineto{\pgfqpoint{3.947767in}{0.938766in}}%
\pgfpathlineto{\pgfqpoint{3.941900in}{0.939786in}}%
\pgfpathlineto{\pgfqpoint{3.936032in}{0.940810in}}%
\pgfpathlineto{\pgfqpoint{3.930164in}{0.941837in}}%
\pgfpathlineto{\pgfqpoint{3.924297in}{0.942868in}}%
\pgfpathlineto{\pgfqpoint{3.918429in}{0.943902in}}%
\pgfpathlineto{\pgfqpoint{3.912562in}{0.944939in}}%
\pgfpathlineto{\pgfqpoint{3.906694in}{0.945980in}}%
\pgfpathlineto{\pgfqpoint{3.900826in}{0.947025in}}%
\pgfpathlineto{\pgfqpoint{3.894959in}{0.948073in}}%
\pgfpathlineto{\pgfqpoint{3.889091in}{0.949125in}}%
\pgfpathlineto{\pgfqpoint{3.883224in}{0.950180in}}%
\pgfpathlineto{\pgfqpoint{3.877356in}{0.951238in}}%
\pgfpathlineto{\pgfqpoint{3.871488in}{0.952300in}}%
\pgfpathlineto{\pgfqpoint{3.865621in}{0.953366in}}%
\pgfpathlineto{\pgfqpoint{3.859753in}{0.954436in}}%
\pgfpathlineto{\pgfqpoint{3.853885in}{0.955509in}}%
\pgfpathlineto{\pgfqpoint{3.848018in}{0.956586in}}%
\pgfpathlineto{\pgfqpoint{3.842150in}{0.957666in}}%
\pgfpathlineto{\pgfqpoint{3.836283in}{0.958750in}}%
\pgfpathlineto{\pgfqpoint{3.830415in}{0.959838in}}%
\pgfpathlineto{\pgfqpoint{3.824547in}{0.960929in}}%
\pgfpathlineto{\pgfqpoint{3.818680in}{0.962025in}}%
\pgfpathlineto{\pgfqpoint{3.812812in}{0.963124in}}%
\pgfpathlineto{\pgfqpoint{3.806945in}{0.964227in}}%
\pgfpathlineto{\pgfqpoint{3.801077in}{0.965333in}}%
\pgfpathlineto{\pgfqpoint{3.795209in}{0.966444in}}%
\pgfpathlineto{\pgfqpoint{3.789342in}{0.967558in}}%
\pgfpathlineto{\pgfqpoint{3.783474in}{0.968676in}}%
\pgfpathlineto{\pgfqpoint{3.777607in}{0.969798in}}%
\pgfpathlineto{\pgfqpoint{3.771739in}{0.970924in}}%
\pgfpathlineto{\pgfqpoint{3.765871in}{0.972054in}}%
\pgfpathlineto{\pgfqpoint{3.760004in}{0.973188in}}%
\pgfpathlineto{\pgfqpoint{3.754136in}{0.974326in}}%
\pgfpathlineto{\pgfqpoint{3.748269in}{0.975468in}}%
\pgfpathlineto{\pgfqpoint{3.742401in}{0.976614in}}%
\pgfpathlineto{\pgfqpoint{3.736533in}{0.977764in}}%
\pgfpathlineto{\pgfqpoint{3.730666in}{0.978918in}}%
\pgfpathlineto{\pgfqpoint{3.724798in}{0.980076in}}%
\pgfpathlineto{\pgfqpoint{3.718931in}{0.981238in}}%
\pgfpathlineto{\pgfqpoint{3.713063in}{0.982404in}}%
\pgfpathlineto{\pgfqpoint{3.707195in}{0.983574in}}%
\pgfpathlineto{\pgfqpoint{3.701328in}{0.984749in}}%
\pgfpathlineto{\pgfqpoint{3.695460in}{0.985928in}}%
\pgfpathlineto{\pgfqpoint{3.689593in}{0.987111in}}%
\pgfpathlineto{\pgfqpoint{3.683725in}{0.988298in}}%
\pgfpathlineto{\pgfqpoint{3.677857in}{0.989489in}}%
\pgfpathlineto{\pgfqpoint{3.671990in}{0.990685in}}%
\pgfpathlineto{\pgfqpoint{3.666122in}{0.991885in}}%
\pgfpathlineto{\pgfqpoint{3.660255in}{0.993089in}}%
\pgfpathlineto{\pgfqpoint{3.654387in}{0.994298in}}%
\pgfpathlineto{\pgfqpoint{3.648519in}{0.995511in}}%
\pgfpathlineto{\pgfqpoint{3.642652in}{0.996728in}}%
\pgfpathlineto{\pgfqpoint{3.636784in}{0.997950in}}%
\pgfpathlineto{\pgfqpoint{3.630917in}{0.999176in}}%
\pgfpathlineto{\pgfqpoint{3.625049in}{1.000407in}}%
\pgfpathlineto{\pgfqpoint{3.619181in}{1.001642in}}%
\pgfpathlineto{\pgfqpoint{3.613314in}{1.002882in}}%
\pgfpathlineto{\pgfqpoint{3.607446in}{1.004127in}}%
\pgfpathlineto{\pgfqpoint{3.601578in}{1.005375in}}%
\pgfpathlineto{\pgfqpoint{3.595711in}{1.006629in}}%
\pgfpathlineto{\pgfqpoint{3.589843in}{1.007887in}}%
\pgfpathlineto{\pgfqpoint{3.583976in}{1.009150in}}%
\pgfpathlineto{\pgfqpoint{3.578108in}{1.010417in}}%
\pgfpathlineto{\pgfqpoint{3.572240in}{1.011689in}}%
\pgfpathlineto{\pgfqpoint{3.566373in}{1.012966in}}%
\pgfpathlineto{\pgfqpoint{3.560505in}{1.014248in}}%
\pgfpathlineto{\pgfqpoint{3.554638in}{1.015534in}}%
\pgfpathlineto{\pgfqpoint{3.548770in}{1.016825in}}%
\pgfpathlineto{\pgfqpoint{3.542902in}{1.018121in}}%
\pgfpathlineto{\pgfqpoint{3.537035in}{1.019422in}}%
\pgfpathlineto{\pgfqpoint{3.531167in}{1.020727in}}%
\pgfpathlineto{\pgfqpoint{3.525300in}{1.022038in}}%
\pgfpathlineto{\pgfqpoint{3.519432in}{1.023353in}}%
\pgfpathlineto{\pgfqpoint{3.513564in}{1.024674in}}%
\pgfpathlineto{\pgfqpoint{3.507697in}{1.025999in}}%
\pgfpathlineto{\pgfqpoint{3.501829in}{1.027330in}}%
\pgfpathlineto{\pgfqpoint{3.495962in}{1.028665in}}%
\pgfpathlineto{\pgfqpoint{3.490094in}{1.030006in}}%
\pgfpathlineto{\pgfqpoint{3.484226in}{1.031352in}}%
\pgfpathlineto{\pgfqpoint{3.478359in}{1.032703in}}%
\pgfpathlineto{\pgfqpoint{3.472491in}{1.034059in}}%
\pgfpathlineto{\pgfqpoint{3.466624in}{1.035420in}}%
\pgfpathlineto{\pgfqpoint{3.460756in}{1.036786in}}%
\pgfpathlineto{\pgfqpoint{3.454888in}{1.038158in}}%
\pgfpathlineto{\pgfqpoint{3.449021in}{1.039535in}}%
\pgfpathlineto{\pgfqpoint{3.443153in}{1.040917in}}%
\pgfpathlineto{\pgfqpoint{3.437286in}{1.042305in}}%
\pgfpathlineto{\pgfqpoint{3.431418in}{1.043698in}}%
\pgfpathlineto{\pgfqpoint{3.425550in}{1.045096in}}%
\pgfpathlineto{\pgfqpoint{3.419683in}{1.046500in}}%
\pgfpathlineto{\pgfqpoint{3.413815in}{1.047910in}}%
\pgfpathlineto{\pgfqpoint{3.407948in}{1.049324in}}%
\pgfpathlineto{\pgfqpoint{3.402080in}{1.050745in}}%
\pgfpathlineto{\pgfqpoint{3.396212in}{1.052171in}}%
\pgfpathlineto{\pgfqpoint{3.390345in}{1.053602in}}%
\pgfpathlineto{\pgfqpoint{3.384477in}{1.055039in}}%
\pgfpathlineto{\pgfqpoint{3.378610in}{1.056482in}}%
\pgfpathlineto{\pgfqpoint{3.372742in}{1.057931in}}%
\pgfpathlineto{\pgfqpoint{3.366874in}{1.059385in}}%
\pgfpathlineto{\pgfqpoint{3.361007in}{1.060845in}}%
\pgfpathlineto{\pgfqpoint{3.355139in}{1.062311in}}%
\pgfpathlineto{\pgfqpoint{3.349271in}{1.063783in}}%
\pgfpathlineto{\pgfqpoint{3.343404in}{1.065260in}}%
\pgfpathlineto{\pgfqpoint{3.337536in}{1.066744in}}%
\pgfpathlineto{\pgfqpoint{3.331669in}{1.068233in}}%
\pgfpathlineto{\pgfqpoint{3.325801in}{1.069729in}}%
\pgfpathlineto{\pgfqpoint{3.319933in}{1.071230in}}%
\pgfpathlineto{\pgfqpoint{3.314066in}{1.072738in}}%
\pgfpathlineto{\pgfqpoint{3.308198in}{1.074251in}}%
\pgfpathlineto{\pgfqpoint{3.302331in}{1.075771in}}%
\pgfpathlineto{\pgfqpoint{3.296463in}{1.077297in}}%
\pgfpathlineto{\pgfqpoint{3.290595in}{1.078829in}}%
\pgfpathlineto{\pgfqpoint{3.284728in}{1.080367in}}%
\pgfpathlineto{\pgfqpoint{3.278860in}{1.081912in}}%
\pgfpathlineto{\pgfqpoint{3.272993in}{1.083462in}}%
\pgfpathlineto{\pgfqpoint{3.267125in}{1.085020in}}%
\pgfpathlineto{\pgfqpoint{3.261257in}{1.086583in}}%
\pgfpathlineto{\pgfqpoint{3.255390in}{1.088153in}}%
\pgfpathlineto{\pgfqpoint{3.249522in}{1.089730in}}%
\pgfpathlineto{\pgfqpoint{3.243655in}{1.091313in}}%
\pgfpathlineto{\pgfqpoint{3.237787in}{1.092903in}}%
\pgfpathlineto{\pgfqpoint{3.231919in}{1.094499in}}%
\pgfpathlineto{\pgfqpoint{3.226052in}{1.096102in}}%
\pgfpathlineto{\pgfqpoint{3.220184in}{1.097711in}}%
\pgfpathlineto{\pgfqpoint{3.214317in}{1.099327in}}%
\pgfpathlineto{\pgfqpoint{3.208449in}{1.100950in}}%
\pgfpathlineto{\pgfqpoint{3.202581in}{1.102580in}}%
\pgfpathlineto{\pgfqpoint{3.196714in}{1.104217in}}%
\pgfpathlineto{\pgfqpoint{3.190846in}{1.105860in}}%
\pgfpathlineto{\pgfqpoint{3.184979in}{1.107511in}}%
\pgfpathlineto{\pgfqpoint{3.179111in}{1.109168in}}%
\pgfpathlineto{\pgfqpoint{3.173243in}{1.110833in}}%
\pgfpathlineto{\pgfqpoint{3.167376in}{1.112504in}}%
\pgfpathlineto{\pgfqpoint{3.161508in}{1.114183in}}%
\pgfpathlineto{\pgfqpoint{3.155641in}{1.115869in}}%
\pgfpathlineto{\pgfqpoint{3.149773in}{1.117562in}}%
\pgfpathlineto{\pgfqpoint{3.143905in}{1.119262in}}%
\pgfpathlineto{\pgfqpoint{3.138038in}{1.120969in}}%
\pgfpathlineto{\pgfqpoint{3.132170in}{1.122684in}}%
\pgfpathlineto{\pgfqpoint{3.126303in}{1.124407in}}%
\pgfpathlineto{\pgfqpoint{3.120435in}{1.126136in}}%
\pgfpathlineto{\pgfqpoint{3.114567in}{1.127873in}}%
\pgfpathlineto{\pgfqpoint{3.108700in}{1.129618in}}%
\pgfpathlineto{\pgfqpoint{3.102832in}{1.131371in}}%
\pgfpathlineto{\pgfqpoint{3.096965in}{1.133130in}}%
\pgfpathlineto{\pgfqpoint{3.091097in}{1.134898in}}%
\pgfpathlineto{\pgfqpoint{3.085229in}{1.136673in}}%
\pgfpathlineto{\pgfqpoint{3.079362in}{1.138457in}}%
\pgfpathlineto{\pgfqpoint{3.073494in}{1.140248in}}%
\pgfpathlineto{\pgfqpoint{3.067626in}{1.142046in}}%
\pgfpathlineto{\pgfqpoint{3.061759in}{1.143853in}}%
\pgfpathlineto{\pgfqpoint{3.055891in}{1.145668in}}%
\pgfpathlineto{\pgfqpoint{3.050024in}{1.147491in}}%
\pgfpathlineto{\pgfqpoint{3.044156in}{1.149322in}}%
\pgfpathlineto{\pgfqpoint{3.038288in}{1.151161in}}%
\pgfpathlineto{\pgfqpoint{3.032421in}{1.153008in}}%
\pgfpathlineto{\pgfqpoint{3.026553in}{1.154864in}}%
\pgfpathlineto{\pgfqpoint{3.020686in}{1.156728in}}%
\pgfpathlineto{\pgfqpoint{3.014818in}{1.158600in}}%
\pgfpathlineto{\pgfqpoint{3.008950in}{1.160481in}}%
\pgfpathlineto{\pgfqpoint{3.003083in}{1.162370in}}%
\pgfpathlineto{\pgfqpoint{2.997215in}{1.164268in}}%
\pgfpathlineto{\pgfqpoint{2.991348in}{1.166174in}}%
\pgfpathlineto{\pgfqpoint{2.985480in}{1.168089in}}%
\pgfpathlineto{\pgfqpoint{2.979612in}{1.170013in}}%
\pgfpathlineto{\pgfqpoint{2.973745in}{1.171945in}}%
\pgfpathlineto{\pgfqpoint{2.967877in}{1.173886in}}%
\pgfpathlineto{\pgfqpoint{2.962010in}{1.175837in}}%
\pgfpathlineto{\pgfqpoint{2.956142in}{1.177796in}}%
\pgfpathlineto{\pgfqpoint{2.950274in}{1.179764in}}%
\pgfpathlineto{\pgfqpoint{2.944407in}{1.181741in}}%
\pgfpathlineto{\pgfqpoint{2.938539in}{1.183728in}}%
\pgfpathlineto{\pgfqpoint{2.932672in}{1.185724in}}%
\pgfpathlineto{\pgfqpoint{2.926804in}{1.187729in}}%
\pgfpathlineto{\pgfqpoint{2.920936in}{1.189743in}}%
\pgfpathlineto{\pgfqpoint{2.915069in}{1.191767in}}%
\pgfpathlineto{\pgfqpoint{2.909201in}{1.193800in}}%
\pgfpathlineto{\pgfqpoint{2.903334in}{1.195843in}}%
\pgfpathlineto{\pgfqpoint{2.897466in}{1.197895in}}%
\pgfpathlineto{\pgfqpoint{2.891598in}{1.199957in}}%
\pgfpathlineto{\pgfqpoint{2.885731in}{1.202029in}}%
\pgfpathlineto{\pgfqpoint{2.879863in}{1.204110in}}%
\pgfpathlineto{\pgfqpoint{2.873996in}{1.206202in}}%
\pgfpathlineto{\pgfqpoint{2.868128in}{1.208303in}}%
\pgfpathlineto{\pgfqpoint{2.862260in}{1.210415in}}%
\pgfpathlineto{\pgfqpoint{2.856393in}{1.212536in}}%
\pgfpathlineto{\pgfqpoint{2.850525in}{1.214668in}}%
\pgfpathlineto{\pgfqpoint{2.844658in}{1.216810in}}%
\pgfpathlineto{\pgfqpoint{2.838790in}{1.218962in}}%
\pgfpathlineto{\pgfqpoint{2.832922in}{1.221125in}}%
\pgfpathlineto{\pgfqpoint{2.827055in}{1.223298in}}%
\pgfpathlineto{\pgfqpoint{2.821187in}{1.225482in}}%
\pgfpathlineto{\pgfqpoint{2.815319in}{1.227676in}}%
\pgfpathlineto{\pgfqpoint{2.809452in}{1.229881in}}%
\pgfpathlineto{\pgfqpoint{2.803584in}{1.232097in}}%
\pgfpathlineto{\pgfqpoint{2.797717in}{1.234324in}}%
\pgfpathlineto{\pgfqpoint{2.791849in}{1.236561in}}%
\pgfpathlineto{\pgfqpoint{2.785981in}{1.238810in}}%
\pgfpathlineto{\pgfqpoint{2.780114in}{1.241070in}}%
\pgfpathlineto{\pgfqpoint{2.774246in}{1.243340in}}%
\pgfpathlineto{\pgfqpoint{2.768379in}{1.245623in}}%
\pgfpathlineto{\pgfqpoint{2.762511in}{1.247916in}}%
\pgfpathlineto{\pgfqpoint{2.756643in}{1.250221in}}%
\pgfpathlineto{\pgfqpoint{2.750776in}{1.252538in}}%
\pgfpathlineto{\pgfqpoint{2.744908in}{1.254866in}}%
\pgfpathlineto{\pgfqpoint{2.739041in}{1.257205in}}%
\pgfpathlineto{\pgfqpoint{2.733173in}{1.259557in}}%
\pgfpathlineto{\pgfqpoint{2.727305in}{1.261920in}}%
\pgfpathlineto{\pgfqpoint{2.721438in}{1.264296in}}%
\pgfpathlineto{\pgfqpoint{2.715570in}{1.266683in}}%
\pgfpathlineto{\pgfqpoint{2.709703in}{1.269083in}}%
\pgfpathlineto{\pgfqpoint{2.703835in}{1.271495in}}%
\pgfpathlineto{\pgfqpoint{2.697967in}{1.273919in}}%
\pgfpathlineto{\pgfqpoint{2.692100in}{1.276356in}}%
\pgfpathlineto{\pgfqpoint{2.686232in}{1.278805in}}%
\pgfpathlineto{\pgfqpoint{2.680365in}{1.281266in}}%
\pgfpathlineto{\pgfqpoint{2.674497in}{1.283741in}}%
\pgfpathlineto{\pgfqpoint{2.668629in}{1.286228in}}%
\pgfpathlineto{\pgfqpoint{2.662762in}{1.288728in}}%
\pgfpathlineto{\pgfqpoint{2.656894in}{1.291242in}}%
\pgfpathlineto{\pgfqpoint{2.651027in}{1.293768in}}%
\pgfpathlineto{\pgfqpoint{2.645159in}{1.296307in}}%
\pgfpathlineto{\pgfqpoint{2.639291in}{1.298860in}}%
\pgfpathlineto{\pgfqpoint{2.633424in}{1.301427in}}%
\pgfpathlineto{\pgfqpoint{2.627556in}{1.304007in}}%
\pgfpathlineto{\pgfqpoint{2.621689in}{1.306600in}}%
\pgfpathlineto{\pgfqpoint{2.615821in}{1.309207in}}%
\pgfpathlineto{\pgfqpoint{2.609953in}{1.311828in}}%
\pgfpathlineto{\pgfqpoint{2.604086in}{1.314463in}}%
\pgfpathlineto{\pgfqpoint{2.598218in}{1.317113in}}%
\pgfpathlineto{\pgfqpoint{2.592351in}{1.319776in}}%
\pgfpathlineto{\pgfqpoint{2.586483in}{1.322454in}}%
\pgfpathlineto{\pgfqpoint{2.580615in}{1.325146in}}%
\pgfpathlineto{\pgfqpoint{2.574748in}{1.327853in}}%
\pgfpathlineto{\pgfqpoint{2.568880in}{1.330574in}}%
\pgfpathlineto{\pgfqpoint{2.563012in}{1.333310in}}%
\pgfpathlineto{\pgfqpoint{2.557145in}{1.336061in}}%
\pgfpathlineto{\pgfqpoint{2.551277in}{1.338827in}}%
\pgfpathlineto{\pgfqpoint{2.545410in}{1.341608in}}%
\pgfpathlineto{\pgfqpoint{2.539542in}{1.344405in}}%
\pgfpathlineto{\pgfqpoint{2.533674in}{1.347217in}}%
\pgfpathlineto{\pgfqpoint{2.527807in}{1.350044in}}%
\pgfpathlineto{\pgfqpoint{2.521939in}{1.352887in}}%
\pgfpathlineto{\pgfqpoint{2.516072in}{1.355746in}}%
\pgfpathlineto{\pgfqpoint{2.510204in}{1.358621in}}%
\pgfpathlineto{\pgfqpoint{2.504336in}{1.361512in}}%
\pgfpathlineto{\pgfqpoint{2.498469in}{1.364419in}}%
\pgfpathlineto{\pgfqpoint{2.492601in}{1.367342in}}%
\pgfpathlineto{\pgfqpoint{2.486734in}{1.370282in}}%
\pgfpathlineto{\pgfqpoint{2.480866in}{1.373238in}}%
\pgfpathlineto{\pgfqpoint{2.474998in}{1.376211in}}%
\pgfpathlineto{\pgfqpoint{2.469131in}{1.379201in}}%
\pgfpathlineto{\pgfqpoint{2.463263in}{1.382208in}}%
\pgfpathlineto{\pgfqpoint{2.457396in}{1.385233in}}%
\pgfpathlineto{\pgfqpoint{2.451528in}{1.388274in}}%
\pgfpathlineto{\pgfqpoint{2.445660in}{1.391333in}}%
\pgfpathlineto{\pgfqpoint{2.439793in}{1.394410in}}%
\pgfpathlineto{\pgfqpoint{2.433925in}{1.397504in}}%
\pgfpathlineto{\pgfqpoint{2.428058in}{1.400617in}}%
\pgfpathlineto{\pgfqpoint{2.422190in}{1.403747in}}%
\pgfpathlineto{\pgfqpoint{2.416322in}{1.406896in}}%
\pgfpathlineto{\pgfqpoint{2.410455in}{1.410063in}}%
\pgfpathlineto{\pgfqpoint{2.404587in}{1.413249in}}%
\pgfpathlineto{\pgfqpoint{2.398720in}{1.416453in}}%
\pgfpathlineto{\pgfqpoint{2.392852in}{1.419677in}}%
\pgfpathlineto{\pgfqpoint{2.386984in}{1.422919in}}%
\pgfpathlineto{\pgfqpoint{2.381117in}{1.426181in}}%
\pgfpathlineto{\pgfqpoint{2.375249in}{1.429462in}}%
\pgfpathlineto{\pgfqpoint{2.369382in}{1.432762in}}%
\pgfpathlineto{\pgfqpoint{2.363514in}{1.436083in}}%
\pgfpathlineto{\pgfqpoint{2.357646in}{1.439423in}}%
\pgfpathlineto{\pgfqpoint{2.351779in}{1.442783in}}%
\pgfpathlineto{\pgfqpoint{2.345911in}{1.446164in}}%
\pgfpathlineto{\pgfqpoint{2.340044in}{1.449565in}}%
\pgfpathlineto{\pgfqpoint{2.334176in}{1.452987in}}%
\pgfpathlineto{\pgfqpoint{2.328308in}{1.456430in}}%
\pgfpathlineto{\pgfqpoint{2.322441in}{1.459894in}}%
\pgfpathlineto{\pgfqpoint{2.316573in}{1.463379in}}%
\pgfpathlineto{\pgfqpoint{2.310706in}{1.466885in}}%
\pgfpathlineto{\pgfqpoint{2.304838in}{1.470413in}}%
\pgfpathlineto{\pgfqpoint{2.298970in}{1.473963in}}%
\pgfpathlineto{\pgfqpoint{2.293103in}{1.477535in}}%
\pgfpathlineto{\pgfqpoint{2.287235in}{1.481129in}}%
\pgfpathlineto{\pgfqpoint{2.281367in}{1.484746in}}%
\pgfpathlineto{\pgfqpoint{2.275500in}{1.488385in}}%
\pgfpathlineto{\pgfqpoint{2.269632in}{1.492047in}}%
\pgfpathlineto{\pgfqpoint{2.263765in}{1.495733in}}%
\pgfpathlineto{\pgfqpoint{2.257897in}{1.499441in}}%
\pgfpathlineto{\pgfqpoint{2.252029in}{1.503173in}}%
\pgfpathlineto{\pgfqpoint{2.246162in}{1.506929in}}%
\pgfpathlineto{\pgfqpoint{2.240294in}{1.510709in}}%
\pgfpathlineto{\pgfqpoint{2.234427in}{1.514513in}}%
\pgfpathlineto{\pgfqpoint{2.228559in}{1.518342in}}%
\pgfpathlineto{\pgfqpoint{2.222691in}{1.522195in}}%
\pgfpathlineto{\pgfqpoint{2.216824in}{1.526073in}}%
\pgfpathlineto{\pgfqpoint{2.210956in}{1.529977in}}%
\pgfpathlineto{\pgfqpoint{2.205089in}{1.533906in}}%
\pgfpathlineto{\pgfqpoint{2.199221in}{1.537860in}}%
\pgfpathlineto{\pgfqpoint{2.193353in}{1.541840in}}%
\pgfpathlineto{\pgfqpoint{2.187486in}{1.545847in}}%
\pgfpathlineto{\pgfqpoint{2.181618in}{1.549880in}}%
\pgfpathlineto{\pgfqpoint{2.175751in}{1.553940in}}%
\pgfpathlineto{\pgfqpoint{2.169883in}{1.558026in}}%
\pgfpathlineto{\pgfqpoint{2.164015in}{1.562140in}}%
\pgfpathlineto{\pgfqpoint{2.158148in}{1.566281in}}%
\pgfpathlineto{\pgfqpoint{2.152280in}{1.570451in}}%
\pgfpathlineto{\pgfqpoint{2.146413in}{1.574648in}}%
\pgfpathlineto{\pgfqpoint{2.140545in}{1.578873in}}%
\pgfpathlineto{\pgfqpoint{2.134677in}{1.583128in}}%
\pgfpathlineto{\pgfqpoint{2.128810in}{1.587411in}}%
\pgfpathlineto{\pgfqpoint{2.122942in}{1.591723in}}%
\pgfpathlineto{\pgfqpoint{2.117075in}{1.596065in}}%
\pgfpathlineto{\pgfqpoint{2.111207in}{1.600437in}}%
\pgfpathlineto{\pgfqpoint{2.105339in}{1.604839in}}%
\pgfpathlineto{\pgfqpoint{2.099472in}{1.609271in}}%
\pgfpathlineto{\pgfqpoint{2.093604in}{1.613734in}}%
\pgfpathlineto{\pgfqpoint{2.087737in}{1.618228in}}%
\pgfpathlineto{\pgfqpoint{2.081869in}{1.622754in}}%
\pgfpathlineto{\pgfqpoint{2.076001in}{1.627311in}}%
\pgfpathlineto{\pgfqpoint{2.070134in}{1.631900in}}%
\pgfpathlineto{\pgfqpoint{2.064266in}{1.636522in}}%
\pgfpathlineto{\pgfqpoint{2.058399in}{1.641177in}}%
\pgfpathlineto{\pgfqpoint{2.052531in}{1.645864in}}%
\pgfpathlineto{\pgfqpoint{2.046663in}{1.650586in}}%
\pgfpathlineto{\pgfqpoint{2.040796in}{1.655341in}}%
\pgfpathlineto{\pgfqpoint{2.034928in}{1.660130in}}%
\pgfpathlineto{\pgfqpoint{2.029060in}{1.664954in}}%
\pgfpathlineto{\pgfqpoint{2.023193in}{1.669812in}}%
\pgfpathlineto{\pgfqpoint{2.017325in}{1.674706in}}%
\pgfpathlineto{\pgfqpoint{2.011458in}{1.679636in}}%
\pgfpathlineto{\pgfqpoint{2.005590in}{1.684602in}}%
\pgfpathlineto{\pgfqpoint{1.999722in}{1.689604in}}%
\pgfpathlineto{\pgfqpoint{1.993855in}{1.694643in}}%
\pgfpathlineto{\pgfqpoint{1.987987in}{1.699720in}}%
\pgfpathlineto{\pgfqpoint{1.982120in}{1.704834in}}%
\pgfpathlineto{\pgfqpoint{1.976252in}{1.709986in}}%
\pgfpathlineto{\pgfqpoint{1.970384in}{1.715177in}}%
\pgfpathlineto{\pgfqpoint{1.964517in}{1.720407in}}%
\pgfpathlineto{\pgfqpoint{1.958649in}{1.725677in}}%
\pgfpathlineto{\pgfqpoint{1.952782in}{1.730986in}}%
\pgfpathlineto{\pgfqpoint{1.946914in}{1.736336in}}%
\pgfpathlineto{\pgfqpoint{1.941046in}{1.741726in}}%
\pgfpathlineto{\pgfqpoint{1.935179in}{1.747158in}}%
\pgfpathlineto{\pgfqpoint{1.929311in}{1.752631in}}%
\pgfpathlineto{\pgfqpoint{1.923444in}{1.758147in}}%
\pgfpathlineto{\pgfqpoint{1.917576in}{1.763706in}}%
\pgfpathlineto{\pgfqpoint{1.911708in}{1.769307in}}%
\pgfpathlineto{\pgfqpoint{1.905841in}{1.774953in}}%
\pgfpathlineto{\pgfqpoint{1.899973in}{1.780643in}}%
\pgfpathlineto{\pgfqpoint{1.894106in}{1.786377in}}%
\pgfpathlineto{\pgfqpoint{1.888238in}{1.792157in}}%
\pgfpathlineto{\pgfqpoint{1.882370in}{1.797983in}}%
\pgfpathlineto{\pgfqpoint{1.876503in}{1.803855in}}%
\pgfpathlineto{\pgfqpoint{1.870635in}{1.809774in}}%
\pgfpathlineto{\pgfqpoint{1.864768in}{1.815740in}}%
\pgfpathlineto{\pgfqpoint{1.858900in}{1.821755in}}%
\pgfpathlineto{\pgfqpoint{1.853032in}{1.827818in}}%
\pgfpathlineto{\pgfqpoint{1.847165in}{1.833931in}}%
\pgfpathlineto{\pgfqpoint{1.841297in}{1.840093in}}%
\pgfpathlineto{\pgfqpoint{1.835430in}{1.846306in}}%
\pgfpathlineto{\pgfqpoint{1.829562in}{1.852570in}}%
\pgfpathlineto{\pgfqpoint{1.823694in}{1.858886in}}%
\pgfpathlineto{\pgfqpoint{1.817827in}{1.865254in}}%
\pgfpathlineto{\pgfqpoint{1.811959in}{1.871675in}}%
\pgfpathlineto{\pgfqpoint{1.806092in}{1.878150in}}%
\pgfpathlineto{\pgfqpoint{1.800224in}{1.884680in}}%
\pgfpathlineto{\pgfqpoint{1.794356in}{1.891264in}}%
\pgfpathlineto{\pgfqpoint{1.788489in}{1.897904in}}%
\pgfpathlineto{\pgfqpoint{1.782621in}{1.904601in}}%
\pgfpathlineto{\pgfqpoint{1.776753in}{1.911355in}}%
\pgfpathlineto{\pgfqpoint{1.770886in}{1.918166in}}%
\pgfpathlineto{\pgfqpoint{1.765018in}{1.925037in}}%
\pgfpathlineto{\pgfqpoint{1.759151in}{1.931967in}}%
\pgfpathlineto{\pgfqpoint{1.753283in}{1.938957in}}%
\pgfpathlineto{\pgfqpoint{1.747415in}{1.946008in}}%
\pgfpathlineto{\pgfqpoint{1.741548in}{1.953121in}}%
\pgfpathlineto{\pgfqpoint{1.735680in}{1.960296in}}%
\pgfpathlineto{\pgfqpoint{1.729813in}{1.967536in}}%
\pgfpathlineto{\pgfqpoint{1.723945in}{1.974839in}}%
\pgfpathlineto{\pgfqpoint{1.718077in}{1.982207in}}%
\pgfpathlineto{\pgfqpoint{1.712210in}{1.989642in}}%
\pgfpathlineto{\pgfqpoint{1.706342in}{1.997144in}}%
\pgfpathlineto{\pgfqpoint{1.700475in}{2.004713in}}%
\pgfpathlineto{\pgfqpoint{1.694607in}{2.012351in}}%
\pgfpathlineto{\pgfqpoint{1.688739in}{2.020059in}}%
\pgfpathlineto{\pgfqpoint{1.682872in}{2.027837in}}%
\pgfpathlineto{\pgfqpoint{1.677004in}{2.035688in}}%
\pgfpathlineto{\pgfqpoint{1.671137in}{2.043610in}}%
\pgfpathlineto{\pgfqpoint{1.665269in}{2.051607in}}%
\pgfpathlineto{\pgfqpoint{1.659401in}{2.059678in}}%
\pgfpathlineto{\pgfqpoint{1.653534in}{2.067825in}}%
\pgfpathlineto{\pgfqpoint{1.647666in}{2.076049in}}%
\pgfpathlineto{\pgfqpoint{1.641799in}{2.084350in}}%
\pgfpathlineto{\pgfqpoint{1.635931in}{2.092731in}}%
\pgfpathlineto{\pgfqpoint{1.630063in}{2.101192in}}%
\pgfpathlineto{\pgfqpoint{1.624196in}{2.109734in}}%
\pgfpathlineto{\pgfqpoint{1.618328in}{2.118358in}}%
\pgfpathlineto{\pgfqpoint{1.612461in}{2.127067in}}%
\pgfpathlineto{\pgfqpoint{1.606593in}{2.135860in}}%
\pgfpathlineto{\pgfqpoint{1.600725in}{2.144739in}}%
\pgfpathlineto{\pgfqpoint{1.594858in}{2.153706in}}%
\pgfpathlineto{\pgfqpoint{1.588990in}{2.162761in}}%
\pgfpathlineto{\pgfqpoint{1.583123in}{2.171907in}}%
\pgfpathlineto{\pgfqpoint{1.577255in}{2.181143in}}%
\pgfpathlineto{\pgfqpoint{1.571387in}{2.190473in}}%
\pgfpathlineto{\pgfqpoint{1.565520in}{2.199897in}}%
\pgfpathlineto{\pgfqpoint{1.559652in}{2.209416in}}%
\pgfpathlineto{\pgfqpoint{1.553785in}{2.219033in}}%
\pgfpathlineto{\pgfqpoint{1.547917in}{2.228748in}}%
\pgfpathlineto{\pgfqpoint{1.542049in}{2.238563in}}%
\pgfpathlineto{\pgfqpoint{1.536182in}{2.248480in}}%
\pgfpathlineto{\pgfqpoint{1.530314in}{2.258500in}}%
\pgfpathlineto{\pgfqpoint{1.524447in}{2.268625in}}%
\pgfpathlineto{\pgfqpoint{1.518579in}{2.278856in}}%
\pgfpathlineto{\pgfqpoint{1.512711in}{2.289196in}}%
\pgfpathlineto{\pgfqpoint{1.506844in}{2.299645in}}%
\pgfpathlineto{\pgfqpoint{1.500976in}{2.310207in}}%
\pgfpathlineto{\pgfqpoint{1.495108in}{2.320881in}}%
\pgfpathlineto{\pgfqpoint{1.489241in}{2.331671in}}%
\pgfpathlineto{\pgfqpoint{1.483373in}{2.342578in}}%
\pgfpathlineto{\pgfqpoint{1.477506in}{2.353605in}}%
\pgfpathlineto{\pgfqpoint{1.471638in}{2.364752in}}%
\pgfpathlineto{\pgfqpoint{1.465770in}{2.376023in}}%
\pgfpathlineto{\pgfqpoint{1.459903in}{2.387418in}}%
\pgfpathlineto{\pgfqpoint{1.454035in}{2.398941in}}%
\pgfpathlineto{\pgfqpoint{1.448168in}{2.410593in}}%
\pgfpathlineto{\pgfqpoint{1.442300in}{2.422377in}}%
\pgfpathlineto{\pgfqpoint{1.436432in}{2.434295in}}%
\pgfpathlineto{\pgfqpoint{1.430565in}{2.446348in}}%
\pgfpathlineto{\pgfqpoint{1.424697in}{2.458541in}}%
\pgfpathlineto{\pgfqpoint{1.418830in}{2.470874in}}%
\pgfpathlineto{\pgfqpoint{1.412962in}{2.483350in}}%
\pgfpathlineto{\pgfqpoint{1.407094in}{2.495972in}}%
\pgfpathlineto{\pgfqpoint{1.401227in}{2.508742in}}%
\pgfpathlineto{\pgfqpoint{1.395359in}{2.521664in}}%
\pgfpathlineto{\pgfqpoint{1.389492in}{2.534739in}}%
\pgfpathlineto{\pgfqpoint{1.383624in}{2.547971in}}%
\pgfpathlineto{\pgfqpoint{1.377756in}{2.561362in}}%
\pgfpathlineto{\pgfqpoint{1.371889in}{2.574915in}}%
\pgfpathlineto{\pgfqpoint{1.366021in}{2.588633in}}%
\pgfpathlineto{\pgfqpoint{1.360154in}{2.602520in}}%
\pgfpathlineto{\pgfqpoint{1.354286in}{2.616577in}}%
\pgfpathlineto{\pgfqpoint{1.348418in}{2.630810in}}%
\pgfpathlineto{\pgfqpoint{1.342551in}{2.645220in}}%
\pgfpathlineto{\pgfqpoint{1.336683in}{2.659811in}}%
\pgfpathlineto{\pgfqpoint{1.330816in}{2.674587in}}%
\pgfpathlineto{\pgfqpoint{1.324948in}{2.689550in}}%
\pgfpathlineto{\pgfqpoint{1.319080in}{2.704706in}}%
\pgfpathlineto{\pgfqpoint{1.313213in}{2.720057in}}%
\pgfpathlineto{\pgfqpoint{1.307345in}{2.735607in}}%
\pgfpathlineto{\pgfqpoint{1.301478in}{2.751360in}}%
\pgfpathlineto{\pgfqpoint{1.295610in}{2.767320in}}%
\pgfpathlineto{\pgfqpoint{1.289742in}{2.783491in}}%
\pgfpathlineto{\pgfqpoint{1.283875in}{2.799878in}}%
\pgfpathlineto{\pgfqpoint{1.278007in}{2.816485in}}%
\pgfpathlineto{\pgfqpoint{1.272140in}{2.833316in}}%
\pgfpathlineto{\pgfqpoint{1.266272in}{2.850375in}}%
\pgfpathlineto{\pgfqpoint{1.260404in}{2.867668in}}%
\pgfpathlineto{\pgfqpoint{1.254537in}{2.885200in}}%
\pgfpathlineto{\pgfqpoint{1.248669in}{2.902975in}}%
\pgfpathlineto{\pgfqpoint{1.242801in}{2.920998in}}%
\pgfpathlineto{\pgfqpoint{1.236934in}{2.939274in}}%
\pgfpathlineto{\pgfqpoint{1.231066in}{2.957810in}}%
\pgfpathlineto{\pgfqpoint{1.225199in}{2.976610in}}%
\pgfpathlineto{\pgfqpoint{1.219331in}{2.995681in}}%
\pgfpathlineto{\pgfqpoint{1.213463in}{3.015027in}}%
\pgfpathlineto{\pgfqpoint{1.207596in}{3.034656in}}%
\pgfpathlineto{\pgfqpoint{1.201728in}{3.054573in}}%
\pgfpathlineto{\pgfqpoint{1.195861in}{3.074785in}}%
\pgfpathlineto{\pgfqpoint{1.189993in}{3.095299in}}%
\pgfpathlineto{\pgfqpoint{1.184125in}{3.116120in}}%
\pgfpathlineto{\pgfqpoint{1.178258in}{3.137257in}}%
\pgfpathlineto{\pgfqpoint{1.172390in}{3.158716in}}%
\pgfpathlineto{\pgfqpoint{1.166523in}{3.180505in}}%
\pgfpathlineto{\pgfqpoint{1.160655in}{3.202631in}}%
\pgfpathlineto{\pgfqpoint{1.154787in}{3.225103in}}%
\pgfpathlineto{\pgfqpoint{1.148920in}{3.247928in}}%
\pgfpathlineto{\pgfqpoint{1.143052in}{3.271115in}}%
\pgfpathlineto{\pgfqpoint{1.137185in}{3.294673in}}%
\pgfpathlineto{\pgfqpoint{1.131317in}{3.318610in}}%
\pgfpathlineto{\pgfqpoint{1.125449in}{3.342936in}}%
\pgfpathlineto{\pgfqpoint{1.119582in}{3.367661in}}%
\pgfpathlineto{\pgfqpoint{1.113714in}{3.392793in}}%
\pgfpathlineto{\pgfqpoint{1.107847in}{3.418344in}}%
\pgfpathlineto{\pgfqpoint{1.101979in}{3.444325in}}%
\pgfpathlineto{\pgfqpoint{1.096111in}{3.470745in}}%
\pgfpathlineto{\pgfqpoint{1.090244in}{3.497616in}}%
\pgfpathlineto{\pgfqpoint{1.084376in}{3.524949in}}%
\pgfpathlineto{\pgfqpoint{1.078509in}{3.552758in}}%
\pgfpathlineto{\pgfqpoint{1.072641in}{3.581054in}}%
\pgfpathlineto{\pgfqpoint{1.066773in}{3.609850in}}%
\pgfpathlineto{\pgfqpoint{1.060906in}{3.639160in}}%
\pgfpathlineto{\pgfqpoint{1.055038in}{3.668998in}}%
\pgfpathlineto{\pgfqpoint{1.049171in}{3.699377in}}%
\pgfpathlineto{\pgfqpoint{1.043303in}{3.730313in}}%
\pgfpathlineto{\pgfqpoint{1.037435in}{3.761821in}}%
\pgfpathlineto{\pgfqpoint{1.031568in}{3.793917in}}%
\pgfpathlineto{\pgfqpoint{1.025700in}{3.826618in}}%
\pgfpathlineto{\pgfqpoint{1.019833in}{3.859942in}}%
\pgfpathlineto{\pgfqpoint{1.013965in}{3.893905in}}%
\pgfpathlineto{\pgfqpoint{1.008097in}{3.928527in}}%
\pgfpathlineto{\pgfqpoint{1.002230in}{3.963827in}}%
\pgfpathlineto{\pgfqpoint{0.996362in}{3.999825in}}%
\pgfpathlineto{\pgfqpoint{0.990494in}{4.036542in}}%
\pgfpathlineto{\pgfqpoint{0.984627in}{4.074000in}}%
\pgfpathlineto{\pgfqpoint{0.978759in}{4.112222in}}%
\pgfpathlineto{\pgfqpoint{0.882794in}{3.480739in}}%
\pgfpathclose%
\pgfusepath{fill}%
\end{pgfscope}%
\begin{pgfscope}%
\pgfpathrectangle{\pgfqpoint{0.882794in}{0.589583in}}{\pgfqpoint{6.917206in}{4.469862in}}%
\pgfusepath{clip}%
\pgfsetbuttcap%
\pgfsetmiterjoin%
\definecolor{currentfill}{rgb}{0.121569,0.466667,0.705882}%
\pgfsetfillcolor{currentfill}%
\pgfsetfillopacity{0.200000}%
\pgfsetlinewidth{0.000000pt}%
\definecolor{currentstroke}{rgb}{0.000000,0.000000,0.000000}%
\pgfsetstrokecolor{currentstroke}%
\pgfsetstrokeopacity{0.200000}%
\pgfsetdash{}{0pt}%
\pgfpathmoveto{\pgfqpoint{0.882794in}{3.405596in}}%
\pgfpathlineto{\pgfqpoint{0.882794in}{5.059445in}}%
\pgfpathlineto{\pgfqpoint{0.978759in}{5.059445in}}%
\pgfpathlineto{\pgfqpoint{0.978759in}{3.405596in}}%
\pgfpathlineto{\pgfqpoint{0.882794in}{3.405596in}}%
\pgfpathclose%
\pgfusepath{fill}%
\end{pgfscope}%
\begin{pgfscope}%
\pgfsetbuttcap%
\pgfsetmiterjoin%
\definecolor{currentfill}{rgb}{0.121569,0.466667,0.705882}%
\pgfsetfillcolor{currentfill}%
\pgfsetfillopacity{0.200000}%
\pgfsetlinewidth{0.000000pt}%
\definecolor{currentstroke}{rgb}{0.000000,0.000000,0.000000}%
\pgfsetstrokecolor{currentstroke}%
\pgfsetstrokeopacity{0.200000}%
\pgfsetdash{}{0pt}%
\pgfpathrectangle{\pgfqpoint{0.882794in}{0.589583in}}{\pgfqpoint{6.917206in}{4.469862in}}%
\pgfusepath{clip}%
\pgfpathmoveto{\pgfqpoint{0.882794in}{3.405596in}}%
\pgfpathlineto{\pgfqpoint{0.882794in}{5.059445in}}%
\pgfpathlineto{\pgfqpoint{0.978759in}{5.059445in}}%
\pgfpathlineto{\pgfqpoint{0.978759in}{3.405596in}}%
\pgfpathlineto{\pgfqpoint{0.882794in}{3.405596in}}%
\pgfpathclose%
\pgfusepath{clip}%
\pgfsys@defobject{currentpattern}{\pgfqpoint{0in}{0in}}{\pgfqpoint{1in}{1in}}{%
\begin{pgfscope}%
\pgfpathrectangle{\pgfqpoint{0in}{0in}}{\pgfqpoint{1in}{1in}}%
\pgfusepath{clip}%
\pgfpathmoveto{\pgfqpoint{-0.500000in}{0.500000in}}%
\pgfpathlineto{\pgfqpoint{0.500000in}{1.500000in}}%
\pgfpathmoveto{\pgfqpoint{-0.416667in}{0.416667in}}%
\pgfpathlineto{\pgfqpoint{0.583333in}{1.416667in}}%
\pgfpathmoveto{\pgfqpoint{-0.333333in}{0.333333in}}%
\pgfpathlineto{\pgfqpoint{0.666667in}{1.333333in}}%
\pgfpathmoveto{\pgfqpoint{-0.250000in}{0.250000in}}%
\pgfpathlineto{\pgfqpoint{0.750000in}{1.250000in}}%
\pgfpathmoveto{\pgfqpoint{-0.166667in}{0.166667in}}%
\pgfpathlineto{\pgfqpoint{0.833333in}{1.166667in}}%
\pgfpathmoveto{\pgfqpoint{-0.083333in}{0.083333in}}%
\pgfpathlineto{\pgfqpoint{0.916667in}{1.083333in}}%
\pgfpathmoveto{\pgfqpoint{0.000000in}{0.000000in}}%
\pgfpathlineto{\pgfqpoint{1.000000in}{1.000000in}}%
\pgfpathmoveto{\pgfqpoint{0.083333in}{-0.083333in}}%
\pgfpathlineto{\pgfqpoint{1.083333in}{0.916667in}}%
\pgfpathmoveto{\pgfqpoint{0.166667in}{-0.166667in}}%
\pgfpathlineto{\pgfqpoint{1.166667in}{0.833333in}}%
\pgfpathmoveto{\pgfqpoint{0.250000in}{-0.250000in}}%
\pgfpathlineto{\pgfqpoint{1.250000in}{0.750000in}}%
\pgfpathmoveto{\pgfqpoint{0.333333in}{-0.333333in}}%
\pgfpathlineto{\pgfqpoint{1.333333in}{0.666667in}}%
\pgfpathmoveto{\pgfqpoint{0.416667in}{-0.416667in}}%
\pgfpathlineto{\pgfqpoint{1.416667in}{0.583333in}}%
\pgfpathmoveto{\pgfqpoint{0.500000in}{-0.500000in}}%
\pgfpathlineto{\pgfqpoint{1.500000in}{0.500000in}}%
\pgfusepath{stroke}%
\end{pgfscope}%
}%
\pgfsys@transformshift{0.882794in}{3.405596in}%
\pgfsys@useobject{currentpattern}{}%
\pgfsys@transformshift{1in}{0in}%
\pgfsys@transformshift{-1in}{0in}%
\pgfsys@transformshift{0in}{1in}%
\pgfsys@useobject{currentpattern}{}%
\pgfsys@transformshift{1in}{0in}%
\pgfsys@transformshift{-1in}{0in}%
\pgfsys@transformshift{0in}{1in}%
\end{pgfscope}%
\begin{pgfscope}%
\pgfpathrectangle{\pgfqpoint{0.882794in}{0.589583in}}{\pgfqpoint{6.917206in}{4.469862in}}%
\pgfusepath{clip}%
\pgfsetrectcap%
\pgfsetroundjoin%
\pgfsetlinewidth{0.803000pt}%
\definecolor{currentstroke}{rgb}{0.690196,0.690196,0.690196}%
\pgfsetstrokecolor{currentstroke}%
\pgfsetstrokeopacity{0.300000}%
\pgfsetdash{}{0pt}%
\pgfpathmoveto{\pgfqpoint{1.602532in}{0.589583in}}%
\pgfpathlineto{\pgfqpoint{1.602532in}{5.059445in}}%
\pgfusepath{stroke}%
\end{pgfscope}%
\begin{pgfscope}%
\pgfsetbuttcap%
\pgfsetroundjoin%
\definecolor{currentfill}{rgb}{0.000000,0.000000,0.000000}%
\pgfsetfillcolor{currentfill}%
\pgfsetlinewidth{0.803000pt}%
\definecolor{currentstroke}{rgb}{0.000000,0.000000,0.000000}%
\pgfsetstrokecolor{currentstroke}%
\pgfsetdash{}{0pt}%
\pgfsys@defobject{currentmarker}{\pgfqpoint{0.000000in}{-0.048611in}}{\pgfqpoint{0.000000in}{0.000000in}}{%
\pgfpathmoveto{\pgfqpoint{0.000000in}{0.000000in}}%
\pgfpathlineto{\pgfqpoint{0.000000in}{-0.048611in}}%
\pgfusepath{stroke,fill}%
}%
\begin{pgfscope}%
\pgfsys@transformshift{1.602532in}{0.589583in}%
\pgfsys@useobject{currentmarker}{}%
\end{pgfscope}%
\end{pgfscope}%
\begin{pgfscope}%
\definecolor{textcolor}{rgb}{0.000000,0.000000,0.000000}%
\pgfsetstrokecolor{textcolor}%
\pgfsetfillcolor{textcolor}%
\pgftext[x=1.602532in,y=0.492361in,,top]{\color{textcolor}\rmfamily\fontsize{10.000000}{12.000000}\selectfont \(\displaystyle {0.01}\)}%
\end{pgfscope}%
\begin{pgfscope}%
\pgfpathrectangle{\pgfqpoint{0.882794in}{0.589583in}}{\pgfqpoint{6.917206in}{4.469862in}}%
\pgfusepath{clip}%
\pgfsetrectcap%
\pgfsetroundjoin%
\pgfsetlinewidth{0.803000pt}%
\definecolor{currentstroke}{rgb}{0.690196,0.690196,0.690196}%
\pgfsetstrokecolor{currentstroke}%
\pgfsetstrokeopacity{0.300000}%
\pgfsetdash{}{0pt}%
\pgfpathmoveto{\pgfqpoint{2.802094in}{0.589583in}}%
\pgfpathlineto{\pgfqpoint{2.802094in}{5.059445in}}%
\pgfusepath{stroke}%
\end{pgfscope}%
\begin{pgfscope}%
\pgfsetbuttcap%
\pgfsetroundjoin%
\definecolor{currentfill}{rgb}{0.000000,0.000000,0.000000}%
\pgfsetfillcolor{currentfill}%
\pgfsetlinewidth{0.803000pt}%
\definecolor{currentstroke}{rgb}{0.000000,0.000000,0.000000}%
\pgfsetstrokecolor{currentstroke}%
\pgfsetdash{}{0pt}%
\pgfsys@defobject{currentmarker}{\pgfqpoint{0.000000in}{-0.048611in}}{\pgfqpoint{0.000000in}{0.000000in}}{%
\pgfpathmoveto{\pgfqpoint{0.000000in}{0.000000in}}%
\pgfpathlineto{\pgfqpoint{0.000000in}{-0.048611in}}%
\pgfusepath{stroke,fill}%
}%
\begin{pgfscope}%
\pgfsys@transformshift{2.802094in}{0.589583in}%
\pgfsys@useobject{currentmarker}{}%
\end{pgfscope}%
\end{pgfscope}%
\begin{pgfscope}%
\definecolor{textcolor}{rgb}{0.000000,0.000000,0.000000}%
\pgfsetstrokecolor{textcolor}%
\pgfsetfillcolor{textcolor}%
\pgftext[x=2.802094in,y=0.492361in,,top]{\color{textcolor}\rmfamily\fontsize{10.000000}{12.000000}\selectfont \(\displaystyle {0.02}\)}%
\end{pgfscope}%
\begin{pgfscope}%
\pgfpathrectangle{\pgfqpoint{0.882794in}{0.589583in}}{\pgfqpoint{6.917206in}{4.469862in}}%
\pgfusepath{clip}%
\pgfsetrectcap%
\pgfsetroundjoin%
\pgfsetlinewidth{0.803000pt}%
\definecolor{currentstroke}{rgb}{0.690196,0.690196,0.690196}%
\pgfsetstrokecolor{currentstroke}%
\pgfsetstrokeopacity{0.300000}%
\pgfsetdash{}{0pt}%
\pgfpathmoveto{\pgfqpoint{4.001656in}{0.589583in}}%
\pgfpathlineto{\pgfqpoint{4.001656in}{5.059445in}}%
\pgfusepath{stroke}%
\end{pgfscope}%
\begin{pgfscope}%
\pgfsetbuttcap%
\pgfsetroundjoin%
\definecolor{currentfill}{rgb}{0.000000,0.000000,0.000000}%
\pgfsetfillcolor{currentfill}%
\pgfsetlinewidth{0.803000pt}%
\definecolor{currentstroke}{rgb}{0.000000,0.000000,0.000000}%
\pgfsetstrokecolor{currentstroke}%
\pgfsetdash{}{0pt}%
\pgfsys@defobject{currentmarker}{\pgfqpoint{0.000000in}{-0.048611in}}{\pgfqpoint{0.000000in}{0.000000in}}{%
\pgfpathmoveto{\pgfqpoint{0.000000in}{0.000000in}}%
\pgfpathlineto{\pgfqpoint{0.000000in}{-0.048611in}}%
\pgfusepath{stroke,fill}%
}%
\begin{pgfscope}%
\pgfsys@transformshift{4.001656in}{0.589583in}%
\pgfsys@useobject{currentmarker}{}%
\end{pgfscope}%
\end{pgfscope}%
\begin{pgfscope}%
\definecolor{textcolor}{rgb}{0.000000,0.000000,0.000000}%
\pgfsetstrokecolor{textcolor}%
\pgfsetfillcolor{textcolor}%
\pgftext[x=4.001656in,y=0.492361in,,top]{\color{textcolor}\rmfamily\fontsize{10.000000}{12.000000}\selectfont \(\displaystyle {0.03}\)}%
\end{pgfscope}%
\begin{pgfscope}%
\pgfpathrectangle{\pgfqpoint{0.882794in}{0.589583in}}{\pgfqpoint{6.917206in}{4.469862in}}%
\pgfusepath{clip}%
\pgfsetrectcap%
\pgfsetroundjoin%
\pgfsetlinewidth{0.803000pt}%
\definecolor{currentstroke}{rgb}{0.690196,0.690196,0.690196}%
\pgfsetstrokecolor{currentstroke}%
\pgfsetstrokeopacity{0.300000}%
\pgfsetdash{}{0pt}%
\pgfpathmoveto{\pgfqpoint{5.201219in}{0.589583in}}%
\pgfpathlineto{\pgfqpoint{5.201219in}{5.059445in}}%
\pgfusepath{stroke}%
\end{pgfscope}%
\begin{pgfscope}%
\pgfsetbuttcap%
\pgfsetroundjoin%
\definecolor{currentfill}{rgb}{0.000000,0.000000,0.000000}%
\pgfsetfillcolor{currentfill}%
\pgfsetlinewidth{0.803000pt}%
\definecolor{currentstroke}{rgb}{0.000000,0.000000,0.000000}%
\pgfsetstrokecolor{currentstroke}%
\pgfsetdash{}{0pt}%
\pgfsys@defobject{currentmarker}{\pgfqpoint{0.000000in}{-0.048611in}}{\pgfqpoint{0.000000in}{0.000000in}}{%
\pgfpathmoveto{\pgfqpoint{0.000000in}{0.000000in}}%
\pgfpathlineto{\pgfqpoint{0.000000in}{-0.048611in}}%
\pgfusepath{stroke,fill}%
}%
\begin{pgfscope}%
\pgfsys@transformshift{5.201219in}{0.589583in}%
\pgfsys@useobject{currentmarker}{}%
\end{pgfscope}%
\end{pgfscope}%
\begin{pgfscope}%
\definecolor{textcolor}{rgb}{0.000000,0.000000,0.000000}%
\pgfsetstrokecolor{textcolor}%
\pgfsetfillcolor{textcolor}%
\pgftext[x=5.201219in,y=0.492361in,,top]{\color{textcolor}\rmfamily\fontsize{10.000000}{12.000000}\selectfont \(\displaystyle {0.04}\)}%
\end{pgfscope}%
\begin{pgfscope}%
\pgfpathrectangle{\pgfqpoint{0.882794in}{0.589583in}}{\pgfqpoint{6.917206in}{4.469862in}}%
\pgfusepath{clip}%
\pgfsetrectcap%
\pgfsetroundjoin%
\pgfsetlinewidth{0.803000pt}%
\definecolor{currentstroke}{rgb}{0.690196,0.690196,0.690196}%
\pgfsetstrokecolor{currentstroke}%
\pgfsetstrokeopacity{0.300000}%
\pgfsetdash{}{0pt}%
\pgfpathmoveto{\pgfqpoint{6.400781in}{0.589583in}}%
\pgfpathlineto{\pgfqpoint{6.400781in}{5.059445in}}%
\pgfusepath{stroke}%
\end{pgfscope}%
\begin{pgfscope}%
\pgfsetbuttcap%
\pgfsetroundjoin%
\definecolor{currentfill}{rgb}{0.000000,0.000000,0.000000}%
\pgfsetfillcolor{currentfill}%
\pgfsetlinewidth{0.803000pt}%
\definecolor{currentstroke}{rgb}{0.000000,0.000000,0.000000}%
\pgfsetstrokecolor{currentstroke}%
\pgfsetdash{}{0pt}%
\pgfsys@defobject{currentmarker}{\pgfqpoint{0.000000in}{-0.048611in}}{\pgfqpoint{0.000000in}{0.000000in}}{%
\pgfpathmoveto{\pgfqpoint{0.000000in}{0.000000in}}%
\pgfpathlineto{\pgfqpoint{0.000000in}{-0.048611in}}%
\pgfusepath{stroke,fill}%
}%
\begin{pgfscope}%
\pgfsys@transformshift{6.400781in}{0.589583in}%
\pgfsys@useobject{currentmarker}{}%
\end{pgfscope}%
\end{pgfscope}%
\begin{pgfscope}%
\definecolor{textcolor}{rgb}{0.000000,0.000000,0.000000}%
\pgfsetstrokecolor{textcolor}%
\pgfsetfillcolor{textcolor}%
\pgftext[x=6.400781in,y=0.492361in,,top]{\color{textcolor}\rmfamily\fontsize{10.000000}{12.000000}\selectfont \(\displaystyle {0.05}\)}%
\end{pgfscope}%
\begin{pgfscope}%
\pgfpathrectangle{\pgfqpoint{0.882794in}{0.589583in}}{\pgfqpoint{6.917206in}{4.469862in}}%
\pgfusepath{clip}%
\pgfsetrectcap%
\pgfsetroundjoin%
\pgfsetlinewidth{0.803000pt}%
\definecolor{currentstroke}{rgb}{0.690196,0.690196,0.690196}%
\pgfsetstrokecolor{currentstroke}%
\pgfsetstrokeopacity{0.300000}%
\pgfsetdash{}{0pt}%
\pgfpathmoveto{\pgfqpoint{7.600343in}{0.589583in}}%
\pgfpathlineto{\pgfqpoint{7.600343in}{5.059445in}}%
\pgfusepath{stroke}%
\end{pgfscope}%
\begin{pgfscope}%
\pgfsetbuttcap%
\pgfsetroundjoin%
\definecolor{currentfill}{rgb}{0.000000,0.000000,0.000000}%
\pgfsetfillcolor{currentfill}%
\pgfsetlinewidth{0.803000pt}%
\definecolor{currentstroke}{rgb}{0.000000,0.000000,0.000000}%
\pgfsetstrokecolor{currentstroke}%
\pgfsetdash{}{0pt}%
\pgfsys@defobject{currentmarker}{\pgfqpoint{0.000000in}{-0.048611in}}{\pgfqpoint{0.000000in}{0.000000in}}{%
\pgfpathmoveto{\pgfqpoint{0.000000in}{0.000000in}}%
\pgfpathlineto{\pgfqpoint{0.000000in}{-0.048611in}}%
\pgfusepath{stroke,fill}%
}%
\begin{pgfscope}%
\pgfsys@transformshift{7.600343in}{0.589583in}%
\pgfsys@useobject{currentmarker}{}%
\end{pgfscope}%
\end{pgfscope}%
\begin{pgfscope}%
\definecolor{textcolor}{rgb}{0.000000,0.000000,0.000000}%
\pgfsetstrokecolor{textcolor}%
\pgfsetfillcolor{textcolor}%
\pgftext[x=7.600343in,y=0.492361in,,top]{\color{textcolor}\rmfamily\fontsize{10.000000}{12.000000}\selectfont \(\displaystyle {0.06}\)}%
\end{pgfscope}%
\begin{pgfscope}%
\definecolor{textcolor}{rgb}{0.000000,0.000000,0.000000}%
\pgfsetstrokecolor{textcolor}%
\pgfsetfillcolor{textcolor}%
\pgftext[x=4.341397in,y=0.313349in,,top]{\color{textcolor}\rmfamily\fontsize{16.000000}{19.200000}\selectfont f1}%
\end{pgfscope}%
\begin{pgfscope}%
\pgfpathrectangle{\pgfqpoint{0.882794in}{0.589583in}}{\pgfqpoint{6.917206in}{4.469862in}}%
\pgfusepath{clip}%
\pgfsetrectcap%
\pgfsetroundjoin%
\pgfsetlinewidth{0.803000pt}%
\definecolor{currentstroke}{rgb}{0.690196,0.690196,0.690196}%
\pgfsetstrokecolor{currentstroke}%
\pgfsetstrokeopacity{0.300000}%
\pgfsetdash{}{0pt}%
\pgfpathmoveto{\pgfqpoint{0.882794in}{0.954809in}}%
\pgfpathlineto{\pgfqpoint{7.800000in}{0.954809in}}%
\pgfusepath{stroke}%
\end{pgfscope}%
\begin{pgfscope}%
\pgfsetbuttcap%
\pgfsetroundjoin%
\definecolor{currentfill}{rgb}{0.000000,0.000000,0.000000}%
\pgfsetfillcolor{currentfill}%
\pgfsetlinewidth{0.803000pt}%
\definecolor{currentstroke}{rgb}{0.000000,0.000000,0.000000}%
\pgfsetstrokecolor{currentstroke}%
\pgfsetdash{}{0pt}%
\pgfsys@defobject{currentmarker}{\pgfqpoint{-0.048611in}{0.000000in}}{\pgfqpoint{-0.000000in}{0.000000in}}{%
\pgfpathmoveto{\pgfqpoint{-0.000000in}{0.000000in}}%
\pgfpathlineto{\pgfqpoint{-0.048611in}{0.000000in}}%
\pgfusepath{stroke,fill}%
}%
\begin{pgfscope}%
\pgfsys@transformshift{0.882794in}{0.954809in}%
\pgfsys@useobject{currentmarker}{}%
\end{pgfscope}%
\end{pgfscope}%
\begin{pgfscope}%
\definecolor{textcolor}{rgb}{0.000000,0.000000,0.000000}%
\pgfsetstrokecolor{textcolor}%
\pgfsetfillcolor{textcolor}%
\pgftext[x=0.438349in, y=0.906584in, left, base]{\color{textcolor}\rmfamily\fontsize{10.000000}{12.000000}\selectfont \(\displaystyle {20000}\)}%
\end{pgfscope}%
\begin{pgfscope}%
\pgfpathrectangle{\pgfqpoint{0.882794in}{0.589583in}}{\pgfqpoint{6.917206in}{4.469862in}}%
\pgfusepath{clip}%
\pgfsetrectcap%
\pgfsetroundjoin%
\pgfsetlinewidth{0.803000pt}%
\definecolor{currentstroke}{rgb}{0.690196,0.690196,0.690196}%
\pgfsetstrokecolor{currentstroke}%
\pgfsetstrokeopacity{0.300000}%
\pgfsetdash{}{0pt}%
\pgfpathmoveto{\pgfqpoint{0.882794in}{1.586292in}}%
\pgfpathlineto{\pgfqpoint{7.800000in}{1.586292in}}%
\pgfusepath{stroke}%
\end{pgfscope}%
\begin{pgfscope}%
\pgfsetbuttcap%
\pgfsetroundjoin%
\definecolor{currentfill}{rgb}{0.000000,0.000000,0.000000}%
\pgfsetfillcolor{currentfill}%
\pgfsetlinewidth{0.803000pt}%
\definecolor{currentstroke}{rgb}{0.000000,0.000000,0.000000}%
\pgfsetstrokecolor{currentstroke}%
\pgfsetdash{}{0pt}%
\pgfsys@defobject{currentmarker}{\pgfqpoint{-0.048611in}{0.000000in}}{\pgfqpoint{-0.000000in}{0.000000in}}{%
\pgfpathmoveto{\pgfqpoint{-0.000000in}{0.000000in}}%
\pgfpathlineto{\pgfqpoint{-0.048611in}{0.000000in}}%
\pgfusepath{stroke,fill}%
}%
\begin{pgfscope}%
\pgfsys@transformshift{0.882794in}{1.586292in}%
\pgfsys@useobject{currentmarker}{}%
\end{pgfscope}%
\end{pgfscope}%
\begin{pgfscope}%
\definecolor{textcolor}{rgb}{0.000000,0.000000,0.000000}%
\pgfsetstrokecolor{textcolor}%
\pgfsetfillcolor{textcolor}%
\pgftext[x=0.438349in, y=1.538066in, left, base]{\color{textcolor}\rmfamily\fontsize{10.000000}{12.000000}\selectfont \(\displaystyle {40000}\)}%
\end{pgfscope}%
\begin{pgfscope}%
\pgfpathrectangle{\pgfqpoint{0.882794in}{0.589583in}}{\pgfqpoint{6.917206in}{4.469862in}}%
\pgfusepath{clip}%
\pgfsetrectcap%
\pgfsetroundjoin%
\pgfsetlinewidth{0.803000pt}%
\definecolor{currentstroke}{rgb}{0.690196,0.690196,0.690196}%
\pgfsetstrokecolor{currentstroke}%
\pgfsetstrokeopacity{0.300000}%
\pgfsetdash{}{0pt}%
\pgfpathmoveto{\pgfqpoint{0.882794in}{2.217774in}}%
\pgfpathlineto{\pgfqpoint{7.800000in}{2.217774in}}%
\pgfusepath{stroke}%
\end{pgfscope}%
\begin{pgfscope}%
\pgfsetbuttcap%
\pgfsetroundjoin%
\definecolor{currentfill}{rgb}{0.000000,0.000000,0.000000}%
\pgfsetfillcolor{currentfill}%
\pgfsetlinewidth{0.803000pt}%
\definecolor{currentstroke}{rgb}{0.000000,0.000000,0.000000}%
\pgfsetstrokecolor{currentstroke}%
\pgfsetdash{}{0pt}%
\pgfsys@defobject{currentmarker}{\pgfqpoint{-0.048611in}{0.000000in}}{\pgfqpoint{-0.000000in}{0.000000in}}{%
\pgfpathmoveto{\pgfqpoint{-0.000000in}{0.000000in}}%
\pgfpathlineto{\pgfqpoint{-0.048611in}{0.000000in}}%
\pgfusepath{stroke,fill}%
}%
\begin{pgfscope}%
\pgfsys@transformshift{0.882794in}{2.217774in}%
\pgfsys@useobject{currentmarker}{}%
\end{pgfscope}%
\end{pgfscope}%
\begin{pgfscope}%
\definecolor{textcolor}{rgb}{0.000000,0.000000,0.000000}%
\pgfsetstrokecolor{textcolor}%
\pgfsetfillcolor{textcolor}%
\pgftext[x=0.438349in, y=2.169549in, left, base]{\color{textcolor}\rmfamily\fontsize{10.000000}{12.000000}\selectfont \(\displaystyle {60000}\)}%
\end{pgfscope}%
\begin{pgfscope}%
\pgfpathrectangle{\pgfqpoint{0.882794in}{0.589583in}}{\pgfqpoint{6.917206in}{4.469862in}}%
\pgfusepath{clip}%
\pgfsetrectcap%
\pgfsetroundjoin%
\pgfsetlinewidth{0.803000pt}%
\definecolor{currentstroke}{rgb}{0.690196,0.690196,0.690196}%
\pgfsetstrokecolor{currentstroke}%
\pgfsetstrokeopacity{0.300000}%
\pgfsetdash{}{0pt}%
\pgfpathmoveto{\pgfqpoint{0.882794in}{2.849257in}}%
\pgfpathlineto{\pgfqpoint{7.800000in}{2.849257in}}%
\pgfusepath{stroke}%
\end{pgfscope}%
\begin{pgfscope}%
\pgfsetbuttcap%
\pgfsetroundjoin%
\definecolor{currentfill}{rgb}{0.000000,0.000000,0.000000}%
\pgfsetfillcolor{currentfill}%
\pgfsetlinewidth{0.803000pt}%
\definecolor{currentstroke}{rgb}{0.000000,0.000000,0.000000}%
\pgfsetstrokecolor{currentstroke}%
\pgfsetdash{}{0pt}%
\pgfsys@defobject{currentmarker}{\pgfqpoint{-0.048611in}{0.000000in}}{\pgfqpoint{-0.000000in}{0.000000in}}{%
\pgfpathmoveto{\pgfqpoint{-0.000000in}{0.000000in}}%
\pgfpathlineto{\pgfqpoint{-0.048611in}{0.000000in}}%
\pgfusepath{stroke,fill}%
}%
\begin{pgfscope}%
\pgfsys@transformshift{0.882794in}{2.849257in}%
\pgfsys@useobject{currentmarker}{}%
\end{pgfscope}%
\end{pgfscope}%
\begin{pgfscope}%
\definecolor{textcolor}{rgb}{0.000000,0.000000,0.000000}%
\pgfsetstrokecolor{textcolor}%
\pgfsetfillcolor{textcolor}%
\pgftext[x=0.438349in, y=2.801031in, left, base]{\color{textcolor}\rmfamily\fontsize{10.000000}{12.000000}\selectfont \(\displaystyle {80000}\)}%
\end{pgfscope}%
\begin{pgfscope}%
\pgfpathrectangle{\pgfqpoint{0.882794in}{0.589583in}}{\pgfqpoint{6.917206in}{4.469862in}}%
\pgfusepath{clip}%
\pgfsetrectcap%
\pgfsetroundjoin%
\pgfsetlinewidth{0.803000pt}%
\definecolor{currentstroke}{rgb}{0.690196,0.690196,0.690196}%
\pgfsetstrokecolor{currentstroke}%
\pgfsetstrokeopacity{0.300000}%
\pgfsetdash{}{0pt}%
\pgfpathmoveto{\pgfqpoint{0.882794in}{3.480739in}}%
\pgfpathlineto{\pgfqpoint{7.800000in}{3.480739in}}%
\pgfusepath{stroke}%
\end{pgfscope}%
\begin{pgfscope}%
\pgfsetbuttcap%
\pgfsetroundjoin%
\definecolor{currentfill}{rgb}{0.000000,0.000000,0.000000}%
\pgfsetfillcolor{currentfill}%
\pgfsetlinewidth{0.803000pt}%
\definecolor{currentstroke}{rgb}{0.000000,0.000000,0.000000}%
\pgfsetstrokecolor{currentstroke}%
\pgfsetdash{}{0pt}%
\pgfsys@defobject{currentmarker}{\pgfqpoint{-0.048611in}{0.000000in}}{\pgfqpoint{-0.000000in}{0.000000in}}{%
\pgfpathmoveto{\pgfqpoint{-0.000000in}{0.000000in}}%
\pgfpathlineto{\pgfqpoint{-0.048611in}{0.000000in}}%
\pgfusepath{stroke,fill}%
}%
\begin{pgfscope}%
\pgfsys@transformshift{0.882794in}{3.480739in}%
\pgfsys@useobject{currentmarker}{}%
\end{pgfscope}%
\end{pgfscope}%
\begin{pgfscope}%
\definecolor{textcolor}{rgb}{0.000000,0.000000,0.000000}%
\pgfsetstrokecolor{textcolor}%
\pgfsetfillcolor{textcolor}%
\pgftext[x=0.368904in, y=3.432514in, left, base]{\color{textcolor}\rmfamily\fontsize{10.000000}{12.000000}\selectfont \(\displaystyle {100000}\)}%
\end{pgfscope}%
\begin{pgfscope}%
\pgfpathrectangle{\pgfqpoint{0.882794in}{0.589583in}}{\pgfqpoint{6.917206in}{4.469862in}}%
\pgfusepath{clip}%
\pgfsetrectcap%
\pgfsetroundjoin%
\pgfsetlinewidth{0.803000pt}%
\definecolor{currentstroke}{rgb}{0.690196,0.690196,0.690196}%
\pgfsetstrokecolor{currentstroke}%
\pgfsetstrokeopacity{0.300000}%
\pgfsetdash{}{0pt}%
\pgfpathmoveto{\pgfqpoint{0.882794in}{4.112222in}}%
\pgfpathlineto{\pgfqpoint{7.800000in}{4.112222in}}%
\pgfusepath{stroke}%
\end{pgfscope}%
\begin{pgfscope}%
\pgfsetbuttcap%
\pgfsetroundjoin%
\definecolor{currentfill}{rgb}{0.000000,0.000000,0.000000}%
\pgfsetfillcolor{currentfill}%
\pgfsetlinewidth{0.803000pt}%
\definecolor{currentstroke}{rgb}{0.000000,0.000000,0.000000}%
\pgfsetstrokecolor{currentstroke}%
\pgfsetdash{}{0pt}%
\pgfsys@defobject{currentmarker}{\pgfqpoint{-0.048611in}{0.000000in}}{\pgfqpoint{-0.000000in}{0.000000in}}{%
\pgfpathmoveto{\pgfqpoint{-0.000000in}{0.000000in}}%
\pgfpathlineto{\pgfqpoint{-0.048611in}{0.000000in}}%
\pgfusepath{stroke,fill}%
}%
\begin{pgfscope}%
\pgfsys@transformshift{0.882794in}{4.112222in}%
\pgfsys@useobject{currentmarker}{}%
\end{pgfscope}%
\end{pgfscope}%
\begin{pgfscope}%
\definecolor{textcolor}{rgb}{0.000000,0.000000,0.000000}%
\pgfsetstrokecolor{textcolor}%
\pgfsetfillcolor{textcolor}%
\pgftext[x=0.368904in, y=4.063996in, left, base]{\color{textcolor}\rmfamily\fontsize{10.000000}{12.000000}\selectfont \(\displaystyle {120000}\)}%
\end{pgfscope}%
\begin{pgfscope}%
\pgfpathrectangle{\pgfqpoint{0.882794in}{0.589583in}}{\pgfqpoint{6.917206in}{4.469862in}}%
\pgfusepath{clip}%
\pgfsetrectcap%
\pgfsetroundjoin%
\pgfsetlinewidth{0.803000pt}%
\definecolor{currentstroke}{rgb}{0.690196,0.690196,0.690196}%
\pgfsetstrokecolor{currentstroke}%
\pgfsetstrokeopacity{0.300000}%
\pgfsetdash{}{0pt}%
\pgfpathmoveto{\pgfqpoint{0.882794in}{4.743704in}}%
\pgfpathlineto{\pgfqpoint{7.800000in}{4.743704in}}%
\pgfusepath{stroke}%
\end{pgfscope}%
\begin{pgfscope}%
\pgfsetbuttcap%
\pgfsetroundjoin%
\definecolor{currentfill}{rgb}{0.000000,0.000000,0.000000}%
\pgfsetfillcolor{currentfill}%
\pgfsetlinewidth{0.803000pt}%
\definecolor{currentstroke}{rgb}{0.000000,0.000000,0.000000}%
\pgfsetstrokecolor{currentstroke}%
\pgfsetdash{}{0pt}%
\pgfsys@defobject{currentmarker}{\pgfqpoint{-0.048611in}{0.000000in}}{\pgfqpoint{-0.000000in}{0.000000in}}{%
\pgfpathmoveto{\pgfqpoint{-0.000000in}{0.000000in}}%
\pgfpathlineto{\pgfqpoint{-0.048611in}{0.000000in}}%
\pgfusepath{stroke,fill}%
}%
\begin{pgfscope}%
\pgfsys@transformshift{0.882794in}{4.743704in}%
\pgfsys@useobject{currentmarker}{}%
\end{pgfscope}%
\end{pgfscope}%
\begin{pgfscope}%
\definecolor{textcolor}{rgb}{0.000000,0.000000,0.000000}%
\pgfsetstrokecolor{textcolor}%
\pgfsetfillcolor{textcolor}%
\pgftext[x=0.368904in, y=4.695479in, left, base]{\color{textcolor}\rmfamily\fontsize{10.000000}{12.000000}\selectfont \(\displaystyle {140000}\)}%
\end{pgfscope}%
\begin{pgfscope}%
\definecolor{textcolor}{rgb}{0.000000,0.000000,0.000000}%
\pgfsetstrokecolor{textcolor}%
\pgfsetfillcolor{textcolor}%
\pgftext[x=0.313349in,y=2.824514in,,bottom,rotate=90.000000]{\color{textcolor}\rmfamily\fontsize{16.000000}{19.200000}\selectfont f2}%
\end{pgfscope}%
\begin{pgfscope}%
\pgfpathrectangle{\pgfqpoint{0.882794in}{0.589583in}}{\pgfqpoint{6.917206in}{4.469862in}}%
\pgfusepath{clip}%
\pgfsetrectcap%
\pgfsetroundjoin%
\pgfsetlinewidth{1.505625pt}%
\definecolor{currentstroke}{rgb}{0.827451,0.827451,0.827451}%
\pgfsetstrokecolor{currentstroke}%
\pgfsetstrokeopacity{0.500000}%
\pgfsetdash{}{0pt}%
\pgfpathmoveto{\pgfqpoint{0.978759in}{4.112222in}}%
\pgfpathlineto{\pgfqpoint{0.990494in}{4.036542in}}%
\pgfpathlineto{\pgfqpoint{1.002230in}{3.963827in}}%
\pgfpathlineto{\pgfqpoint{1.013965in}{3.893905in}}%
\pgfpathlineto{\pgfqpoint{1.025700in}{3.826618in}}%
\pgfpathlineto{\pgfqpoint{1.037435in}{3.761821in}}%
\pgfpathlineto{\pgfqpoint{1.049171in}{3.699377in}}%
\pgfpathlineto{\pgfqpoint{1.060906in}{3.639160in}}%
\pgfpathlineto{\pgfqpoint{1.072641in}{3.581054in}}%
\pgfpathlineto{\pgfqpoint{1.084376in}{3.524949in}}%
\pgfpathlineto{\pgfqpoint{1.096111in}{3.470745in}}%
\pgfpathlineto{\pgfqpoint{1.107847in}{3.418344in}}%
\pgfpathlineto{\pgfqpoint{1.119582in}{3.367661in}}%
\pgfpathlineto{\pgfqpoint{1.131317in}{3.318610in}}%
\pgfpathlineto{\pgfqpoint{1.143052in}{3.271115in}}%
\pgfpathlineto{\pgfqpoint{1.154787in}{3.225103in}}%
\pgfpathlineto{\pgfqpoint{1.166523in}{3.180505in}}%
\pgfpathlineto{\pgfqpoint{1.178258in}{3.137257in}}%
\pgfpathlineto{\pgfqpoint{1.189993in}{3.095299in}}%
\pgfpathlineto{\pgfqpoint{1.201728in}{3.054573in}}%
\pgfpathlineto{\pgfqpoint{1.213463in}{3.015027in}}%
\pgfpathlineto{\pgfqpoint{1.225199in}{2.976610in}}%
\pgfpathlineto{\pgfqpoint{1.236934in}{2.939274in}}%
\pgfpathlineto{\pgfqpoint{1.248669in}{2.902975in}}%
\pgfpathlineto{\pgfqpoint{1.260404in}{2.867668in}}%
\pgfpathlineto{\pgfqpoint{1.272140in}{2.833316in}}%
\pgfpathlineto{\pgfqpoint{1.283875in}{2.799878in}}%
\pgfpathlineto{\pgfqpoint{1.295610in}{2.767320in}}%
\pgfpathlineto{\pgfqpoint{1.307345in}{2.735607in}}%
\pgfpathlineto{\pgfqpoint{1.319080in}{2.704706in}}%
\pgfpathlineto{\pgfqpoint{1.330816in}{2.674587in}}%
\pgfpathlineto{\pgfqpoint{1.342551in}{2.645220in}}%
\pgfpathlineto{\pgfqpoint{1.354286in}{2.616577in}}%
\pgfpathlineto{\pgfqpoint{1.366021in}{2.588633in}}%
\pgfpathlineto{\pgfqpoint{1.377756in}{2.561362in}}%
\pgfpathlineto{\pgfqpoint{1.389492in}{2.534739in}}%
\pgfpathlineto{\pgfqpoint{1.401227in}{2.508742in}}%
\pgfpathlineto{\pgfqpoint{1.418830in}{2.470874in}}%
\pgfpathlineto{\pgfqpoint{1.436432in}{2.434295in}}%
\pgfpathlineto{\pgfqpoint{1.454035in}{2.398941in}}%
\pgfpathlineto{\pgfqpoint{1.471638in}{2.364752in}}%
\pgfpathlineto{\pgfqpoint{1.489241in}{2.331671in}}%
\pgfpathlineto{\pgfqpoint{1.506844in}{2.299645in}}%
\pgfpathlineto{\pgfqpoint{1.524447in}{2.268625in}}%
\pgfpathlineto{\pgfqpoint{1.542049in}{2.238563in}}%
\pgfpathlineto{\pgfqpoint{1.559652in}{2.209416in}}%
\pgfpathlineto{\pgfqpoint{1.577255in}{2.181143in}}%
\pgfpathlineto{\pgfqpoint{1.594858in}{2.153706in}}%
\pgfpathlineto{\pgfqpoint{1.612461in}{2.127067in}}%
\pgfpathlineto{\pgfqpoint{1.630063in}{2.101192in}}%
\pgfpathlineto{\pgfqpoint{1.647666in}{2.076049in}}%
\pgfpathlineto{\pgfqpoint{1.665269in}{2.051607in}}%
\pgfpathlineto{\pgfqpoint{1.682872in}{2.027837in}}%
\pgfpathlineto{\pgfqpoint{1.700475in}{2.004713in}}%
\pgfpathlineto{\pgfqpoint{1.718077in}{1.982207in}}%
\pgfpathlineto{\pgfqpoint{1.735680in}{1.960296in}}%
\pgfpathlineto{\pgfqpoint{1.753283in}{1.938957in}}%
\pgfpathlineto{\pgfqpoint{1.770886in}{1.918166in}}%
\pgfpathlineto{\pgfqpoint{1.788489in}{1.897904in}}%
\pgfpathlineto{\pgfqpoint{1.806092in}{1.878150in}}%
\pgfpathlineto{\pgfqpoint{1.823694in}{1.858886in}}%
\pgfpathlineto{\pgfqpoint{1.841297in}{1.840093in}}%
\pgfpathlineto{\pgfqpoint{1.858900in}{1.821755in}}%
\pgfpathlineto{\pgfqpoint{1.876503in}{1.803855in}}%
\pgfpathlineto{\pgfqpoint{1.894106in}{1.786377in}}%
\pgfpathlineto{\pgfqpoint{1.911708in}{1.769307in}}%
\pgfpathlineto{\pgfqpoint{1.929311in}{1.752631in}}%
\pgfpathlineto{\pgfqpoint{1.952782in}{1.730986in}}%
\pgfpathlineto{\pgfqpoint{1.976252in}{1.709986in}}%
\pgfpathlineto{\pgfqpoint{1.999722in}{1.689604in}}%
\pgfpathlineto{\pgfqpoint{2.023193in}{1.669812in}}%
\pgfpathlineto{\pgfqpoint{2.046663in}{1.650586in}}%
\pgfpathlineto{\pgfqpoint{2.070134in}{1.631900in}}%
\pgfpathlineto{\pgfqpoint{2.093604in}{1.613734in}}%
\pgfpathlineto{\pgfqpoint{2.117075in}{1.596065in}}%
\pgfpathlineto{\pgfqpoint{2.140545in}{1.578873in}}%
\pgfpathlineto{\pgfqpoint{2.164015in}{1.562140in}}%
\pgfpathlineto{\pgfqpoint{2.187486in}{1.545847in}}%
\pgfpathlineto{\pgfqpoint{2.210956in}{1.529977in}}%
\pgfpathlineto{\pgfqpoint{2.234427in}{1.514513in}}%
\pgfpathlineto{\pgfqpoint{2.257897in}{1.499441in}}%
\pgfpathlineto{\pgfqpoint{2.281367in}{1.484746in}}%
\pgfpathlineto{\pgfqpoint{2.304838in}{1.470413in}}%
\pgfpathlineto{\pgfqpoint{2.334176in}{1.452987in}}%
\pgfpathlineto{\pgfqpoint{2.363514in}{1.436083in}}%
\pgfpathlineto{\pgfqpoint{2.392852in}{1.419677in}}%
\pgfpathlineto{\pgfqpoint{2.422190in}{1.403747in}}%
\pgfpathlineto{\pgfqpoint{2.451528in}{1.388274in}}%
\pgfpathlineto{\pgfqpoint{2.480866in}{1.373238in}}%
\pgfpathlineto{\pgfqpoint{2.510204in}{1.358621in}}%
\pgfpathlineto{\pgfqpoint{2.539542in}{1.344405in}}%
\pgfpathlineto{\pgfqpoint{2.568880in}{1.330574in}}%
\pgfpathlineto{\pgfqpoint{2.598218in}{1.317113in}}%
\pgfpathlineto{\pgfqpoint{2.627556in}{1.304007in}}%
\pgfpathlineto{\pgfqpoint{2.662762in}{1.288728in}}%
\pgfpathlineto{\pgfqpoint{2.697967in}{1.273919in}}%
\pgfpathlineto{\pgfqpoint{2.733173in}{1.259557in}}%
\pgfpathlineto{\pgfqpoint{2.768379in}{1.245623in}}%
\pgfpathlineto{\pgfqpoint{2.803584in}{1.232097in}}%
\pgfpathlineto{\pgfqpoint{2.838790in}{1.218962in}}%
\pgfpathlineto{\pgfqpoint{2.873996in}{1.206202in}}%
\pgfpathlineto{\pgfqpoint{2.909201in}{1.193800in}}%
\pgfpathlineto{\pgfqpoint{2.944407in}{1.181741in}}%
\pgfpathlineto{\pgfqpoint{2.985480in}{1.168089in}}%
\pgfpathlineto{\pgfqpoint{3.026553in}{1.154864in}}%
\pgfpathlineto{\pgfqpoint{3.067626in}{1.142046in}}%
\pgfpathlineto{\pgfqpoint{3.108700in}{1.129618in}}%
\pgfpathlineto{\pgfqpoint{3.149773in}{1.117562in}}%
\pgfpathlineto{\pgfqpoint{3.190846in}{1.105860in}}%
\pgfpathlineto{\pgfqpoint{3.237787in}{1.092903in}}%
\pgfpathlineto{\pgfqpoint{3.284728in}{1.080367in}}%
\pgfpathlineto{\pgfqpoint{3.331669in}{1.068233in}}%
\pgfpathlineto{\pgfqpoint{3.378610in}{1.056482in}}%
\pgfpathlineto{\pgfqpoint{3.425550in}{1.045096in}}%
\pgfpathlineto{\pgfqpoint{3.472491in}{1.034059in}}%
\pgfpathlineto{\pgfqpoint{3.525300in}{1.022038in}}%
\pgfpathlineto{\pgfqpoint{3.578108in}{1.010417in}}%
\pgfpathlineto{\pgfqpoint{3.630917in}{0.999176in}}%
\pgfpathlineto{\pgfqpoint{3.683725in}{0.988298in}}%
\pgfpathlineto{\pgfqpoint{3.742401in}{0.976614in}}%
\pgfpathlineto{\pgfqpoint{3.801077in}{0.965333in}}%
\pgfpathlineto{\pgfqpoint{3.859753in}{0.954436in}}%
\pgfpathlineto{\pgfqpoint{3.918429in}{0.943902in}}%
\pgfpathlineto{\pgfqpoint{3.982973in}{0.932714in}}%
\pgfpathlineto{\pgfqpoint{4.047516in}{0.921922in}}%
\pgfpathlineto{\pgfqpoint{4.112060in}{0.911505in}}%
\pgfpathlineto{\pgfqpoint{4.176604in}{0.901445in}}%
\pgfpathlineto{\pgfqpoint{4.247015in}{0.890856in}}%
\pgfpathlineto{\pgfqpoint{4.317426in}{0.880647in}}%
\pgfpathlineto{\pgfqpoint{4.387837in}{0.870800in}}%
\pgfpathlineto{\pgfqpoint{4.464116in}{0.860517in}}%
\pgfpathlineto{\pgfqpoint{4.540395in}{0.850613in}}%
\pgfpathlineto{\pgfqpoint{4.616674in}{0.841068in}}%
\pgfpathlineto{\pgfqpoint{4.698821in}{0.831167in}}%
\pgfpathlineto{\pgfqpoint{4.780967in}{0.821638in}}%
\pgfpathlineto{\pgfqpoint{4.868981in}{0.811818in}}%
\pgfpathlineto{\pgfqpoint{4.956995in}{0.802377in}}%
\pgfpathlineto{\pgfqpoint{5.045009in}{0.793294in}}%
\pgfpathlineto{\pgfqpoint{5.138891in}{0.783978in}}%
\pgfpathlineto{\pgfqpoint{5.232773in}{0.775024in}}%
\pgfpathlineto{\pgfqpoint{5.332522in}{0.765884in}}%
\pgfpathlineto{\pgfqpoint{5.432271in}{0.757106in}}%
\pgfpathlineto{\pgfqpoint{5.537888in}{0.748184in}}%
\pgfpathlineto{\pgfqpoint{5.649372in}{0.739156in}}%
\pgfpathlineto{\pgfqpoint{5.760857in}{0.730503in}}%
\pgfpathlineto{\pgfqpoint{5.878209in}{0.721776in}}%
\pgfpathlineto{\pgfqpoint{5.995561in}{0.713415in}}%
\pgfpathlineto{\pgfqpoint{6.118781in}{0.705006in}}%
\pgfpathlineto{\pgfqpoint{6.247868in}{0.696576in}}%
\pgfpathlineto{\pgfqpoint{6.376955in}{0.688511in}}%
\pgfpathlineto{\pgfqpoint{6.511910in}{0.680444in}}%
\pgfpathlineto{\pgfqpoint{6.652733in}{0.672397in}}%
\pgfpathlineto{\pgfqpoint{6.799423in}{0.664392in}}%
\pgfpathlineto{\pgfqpoint{6.869572in}{0.660772in}}%
\pgfpathlineto{\pgfqpoint{6.918032in}{0.658684in}}%
\pgfpathlineto{\pgfqpoint{6.976184in}{0.656573in}}%
\pgfpathlineto{\pgfqpoint{7.044027in}{0.654502in}}%
\pgfpathlineto{\pgfqpoint{7.121563in}{0.652512in}}%
\pgfpathlineto{\pgfqpoint{7.218483in}{0.650435in}}%
\pgfpathlineto{\pgfqpoint{7.325094in}{0.648533in}}%
\pgfpathlineto{\pgfqpoint{7.451090in}{0.646660in}}%
\pgfpathlineto{\pgfqpoint{7.596469in}{0.644868in}}%
\pgfpathlineto{\pgfqpoint{7.770924in}{0.643098in}}%
\pgfpathlineto{\pgfqpoint{7.800000in}{0.642834in}}%
\pgfpathlineto{\pgfqpoint{7.800000in}{0.642834in}}%
\pgfusepath{stroke}%
\end{pgfscope}%
\begin{pgfscope}%
\pgfpathrectangle{\pgfqpoint{0.882794in}{0.589583in}}{\pgfqpoint{6.917206in}{4.469862in}}%
\pgfusepath{clip}%
\pgfsetrectcap%
\pgfsetroundjoin%
\pgfsetlinewidth{3.011250pt}%
\definecolor{currentstroke}{rgb}{0.000000,0.000000,0.000000}%
\pgfsetstrokecolor{currentstroke}%
\pgfsetdash{}{0pt}%
\pgfpathmoveto{\pgfqpoint{0.882794in}{3.480739in}}%
\pgfpathlineto{\pgfqpoint{0.892574in}{3.417673in}}%
\pgfpathlineto{\pgfqpoint{0.902353in}{3.357077in}}%
\pgfpathlineto{\pgfqpoint{0.912132in}{3.298809in}}%
\pgfpathlineto{\pgfqpoint{0.921912in}{3.242736in}}%
\pgfpathlineto{\pgfqpoint{0.931691in}{3.188738in}}%
\pgfpathlineto{\pgfqpoint{0.941470in}{3.136702in}}%
\pgfpathlineto{\pgfqpoint{0.951250in}{3.086521in}}%
\pgfpathlineto{\pgfqpoint{0.961029in}{3.038100in}}%
\pgfpathlineto{\pgfqpoint{0.970808in}{2.991346in}}%
\pgfpathlineto{\pgfqpoint{0.980588in}{2.946175in}}%
\pgfpathlineto{\pgfqpoint{0.990367in}{2.902508in}}%
\pgfpathlineto{\pgfqpoint{1.000146in}{2.860272in}}%
\pgfpathlineto{\pgfqpoint{1.009926in}{2.819396in}}%
\pgfpathlineto{\pgfqpoint{1.019705in}{2.779817in}}%
\pgfpathlineto{\pgfqpoint{1.029484in}{2.741473in}}%
\pgfpathlineto{\pgfqpoint{1.039264in}{2.704308in}}%
\pgfpathlineto{\pgfqpoint{1.049043in}{2.668268in}}%
\pgfpathlineto{\pgfqpoint{1.058822in}{2.633303in}}%
\pgfpathlineto{\pgfqpoint{1.068602in}{2.599366in}}%
\pgfpathlineto{\pgfqpoint{1.078381in}{2.566411in}}%
\pgfpathlineto{\pgfqpoint{1.088160in}{2.534396in}}%
\pgfpathlineto{\pgfqpoint{1.097940in}{2.503283in}}%
\pgfpathlineto{\pgfqpoint{1.107719in}{2.473033in}}%
\pgfpathlineto{\pgfqpoint{1.117498in}{2.443611in}}%
\pgfpathlineto{\pgfqpoint{1.127278in}{2.414984in}}%
\pgfpathlineto{\pgfqpoint{1.141947in}{2.373464in}}%
\pgfpathlineto{\pgfqpoint{1.156616in}{2.333560in}}%
\pgfpathlineto{\pgfqpoint{1.171285in}{2.295180in}}%
\pgfpathlineto{\pgfqpoint{1.185954in}{2.258238in}}%
\pgfpathlineto{\pgfqpoint{1.200623in}{2.222654in}}%
\pgfpathlineto{\pgfqpoint{1.215292in}{2.188356in}}%
\pgfpathlineto{\pgfqpoint{1.229961in}{2.155274in}}%
\pgfpathlineto{\pgfqpoint{1.244630in}{2.123346in}}%
\pgfpathlineto{\pgfqpoint{1.259299in}{2.092511in}}%
\pgfpathlineto{\pgfqpoint{1.273968in}{2.062716in}}%
\pgfpathlineto{\pgfqpoint{1.288637in}{2.033907in}}%
\pgfpathlineto{\pgfqpoint{1.303306in}{2.006036in}}%
\pgfpathlineto{\pgfqpoint{1.317975in}{1.979060in}}%
\pgfpathlineto{\pgfqpoint{1.332644in}{1.952935in}}%
\pgfpathlineto{\pgfqpoint{1.347313in}{1.927621in}}%
\pgfpathlineto{\pgfqpoint{1.361982in}{1.903082in}}%
\pgfpathlineto{\pgfqpoint{1.376651in}{1.879282in}}%
\pgfpathlineto{\pgfqpoint{1.391320in}{1.856189in}}%
\pgfpathlineto{\pgfqpoint{1.405989in}{1.833771in}}%
\pgfpathlineto{\pgfqpoint{1.420658in}{1.811999in}}%
\pgfpathlineto{\pgfqpoint{1.435327in}{1.790846in}}%
\pgfpathlineto{\pgfqpoint{1.449996in}{1.770286in}}%
\pgfpathlineto{\pgfqpoint{1.464665in}{1.750294in}}%
\pgfpathlineto{\pgfqpoint{1.479334in}{1.730847in}}%
\pgfpathlineto{\pgfqpoint{1.494003in}{1.711923in}}%
\pgfpathlineto{\pgfqpoint{1.508672in}{1.693501in}}%
\pgfpathlineto{\pgfqpoint{1.523341in}{1.675561in}}%
\pgfpathlineto{\pgfqpoint{1.538010in}{1.658085in}}%
\pgfpathlineto{\pgfqpoint{1.552679in}{1.641055in}}%
\pgfpathlineto{\pgfqpoint{1.572238in}{1.619013in}}%
\pgfpathlineto{\pgfqpoint{1.591797in}{1.597696in}}%
\pgfpathlineto{\pgfqpoint{1.611355in}{1.577070in}}%
\pgfpathlineto{\pgfqpoint{1.630914in}{1.557100in}}%
\pgfpathlineto{\pgfqpoint{1.650473in}{1.537757in}}%
\pgfpathlineto{\pgfqpoint{1.670031in}{1.519010in}}%
\pgfpathlineto{\pgfqpoint{1.689590in}{1.500834in}}%
\pgfpathlineto{\pgfqpoint{1.709149in}{1.483202in}}%
\pgfpathlineto{\pgfqpoint{1.728707in}{1.466090in}}%
\pgfpathlineto{\pgfqpoint{1.748266in}{1.449476in}}%
\pgfpathlineto{\pgfqpoint{1.767825in}{1.433338in}}%
\pgfpathlineto{\pgfqpoint{1.787383in}{1.417656in}}%
\pgfpathlineto{\pgfqpoint{1.806942in}{1.402411in}}%
\pgfpathlineto{\pgfqpoint{1.826501in}{1.387585in}}%
\pgfpathlineto{\pgfqpoint{1.846059in}{1.373161in}}%
\pgfpathlineto{\pgfqpoint{1.865618in}{1.359122in}}%
\pgfpathlineto{\pgfqpoint{1.885177in}{1.345454in}}%
\pgfpathlineto{\pgfqpoint{1.909625in}{1.328868in}}%
\pgfpathlineto{\pgfqpoint{1.934073in}{1.312812in}}%
\pgfpathlineto{\pgfqpoint{1.958522in}{1.297261in}}%
\pgfpathlineto{\pgfqpoint{1.982970in}{1.282190in}}%
\pgfpathlineto{\pgfqpoint{2.007418in}{1.267579in}}%
\pgfpathlineto{\pgfqpoint{2.031867in}{1.253407in}}%
\pgfpathlineto{\pgfqpoint{2.056315in}{1.239654in}}%
\pgfpathlineto{\pgfqpoint{2.080764in}{1.226301in}}%
\pgfpathlineto{\pgfqpoint{2.105212in}{1.213332in}}%
\pgfpathlineto{\pgfqpoint{2.129660in}{1.200730in}}%
\pgfpathlineto{\pgfqpoint{2.158998in}{1.186072in}}%
\pgfpathlineto{\pgfqpoint{2.188336in}{1.171895in}}%
\pgfpathlineto{\pgfqpoint{2.217674in}{1.158176in}}%
\pgfpathlineto{\pgfqpoint{2.247012in}{1.144894in}}%
\pgfpathlineto{\pgfqpoint{2.276350in}{1.132028in}}%
\pgfpathlineto{\pgfqpoint{2.305688in}{1.119558in}}%
\pgfpathlineto{\pgfqpoint{2.335026in}{1.107468in}}%
\pgfpathlineto{\pgfqpoint{2.364364in}{1.095739in}}%
\pgfpathlineto{\pgfqpoint{2.398592in}{1.082491in}}%
\pgfpathlineto{\pgfqpoint{2.432820in}{1.069690in}}%
\pgfpathlineto{\pgfqpoint{2.467048in}{1.057313in}}%
\pgfpathlineto{\pgfqpoint{2.501275in}{1.045340in}}%
\pgfpathlineto{\pgfqpoint{2.535503in}{1.033752in}}%
\pgfpathlineto{\pgfqpoint{2.569731in}{1.022529in}}%
\pgfpathlineto{\pgfqpoint{2.603958in}{1.011656in}}%
\pgfpathlineto{\pgfqpoint{2.643076in}{0.999636in}}%
\pgfpathlineto{\pgfqpoint{2.682193in}{0.988029in}}%
\pgfpathlineto{\pgfqpoint{2.721310in}{0.976813in}}%
\pgfpathlineto{\pgfqpoint{2.760428in}{0.965970in}}%
\pgfpathlineto{\pgfqpoint{2.799545in}{0.955481in}}%
\pgfpathlineto{\pgfqpoint{2.843552in}{0.944082in}}%
\pgfpathlineto{\pgfqpoint{2.887559in}{0.933087in}}%
\pgfpathlineto{\pgfqpoint{2.931566in}{0.922475in}}%
\pgfpathlineto{\pgfqpoint{2.975573in}{0.912226in}}%
\pgfpathlineto{\pgfqpoint{3.024470in}{0.901242in}}%
\pgfpathlineto{\pgfqpoint{3.073367in}{0.890660in}}%
\pgfpathlineto{\pgfqpoint{3.122263in}{0.880458in}}%
\pgfpathlineto{\pgfqpoint{3.171160in}{0.870617in}}%
\pgfpathlineto{\pgfqpoint{3.224946in}{0.860186in}}%
\pgfpathlineto{\pgfqpoint{3.278733in}{0.850145in}}%
\pgfpathlineto{\pgfqpoint{3.332519in}{0.840473in}}%
\pgfpathlineto{\pgfqpoint{3.391195in}{0.830318in}}%
\pgfpathlineto{\pgfqpoint{3.449871in}{0.820555in}}%
\pgfpathlineto{\pgfqpoint{3.508547in}{0.811160in}}%
\pgfpathlineto{\pgfqpoint{3.572113in}{0.801375in}}%
\pgfpathlineto{\pgfqpoint{3.635679in}{0.791975in}}%
\pgfpathlineto{\pgfqpoint{3.699244in}{0.782938in}}%
\pgfpathlineto{\pgfqpoint{3.767700in}{0.773587in}}%
\pgfpathlineto{\pgfqpoint{3.836155in}{0.764609in}}%
\pgfpathlineto{\pgfqpoint{3.909500in}{0.755379in}}%
\pgfpathlineto{\pgfqpoint{3.982845in}{0.746527in}}%
\pgfpathlineto{\pgfqpoint{4.061080in}{0.737476in}}%
\pgfpathlineto{\pgfqpoint{4.139315in}{0.728804in}}%
\pgfpathlineto{\pgfqpoint{4.222439in}{0.719980in}}%
\pgfpathlineto{\pgfqpoint{4.305564in}{0.711531in}}%
\pgfpathlineto{\pgfqpoint{4.393578in}{0.702969in}}%
\pgfpathlineto{\pgfqpoint{4.481592in}{0.694777in}}%
\pgfpathlineto{\pgfqpoint{4.574495in}{0.686504in}}%
\pgfpathlineto{\pgfqpoint{4.672289in}{0.678185in}}%
\pgfpathlineto{\pgfqpoint{4.770082in}{0.670239in}}%
\pgfpathlineto{\pgfqpoint{4.872765in}{0.662269in}}%
\pgfpathlineto{\pgfqpoint{4.980338in}{0.654304in}}%
\pgfpathlineto{\pgfqpoint{5.087911in}{0.646704in}}%
\pgfpathlineto{\pgfqpoint{5.200373in}{0.639124in}}%
\pgfpathlineto{\pgfqpoint{5.317725in}{0.631583in}}%
\pgfpathlineto{\pgfqpoint{5.439967in}{0.624102in}}%
\pgfpathlineto{\pgfqpoint{5.567099in}{0.616698in}}%
\pgfpathlineto{\pgfqpoint{5.699120in}{0.609384in}}%
\pgfpathlineto{\pgfqpoint{5.791805in}{0.604531in}}%
\pgfpathlineto{\pgfqpoint{5.840265in}{0.602476in}}%
\pgfpathlineto{\pgfqpoint{5.896801in}{0.600507in}}%
\pgfpathlineto{\pgfqpoint{5.961414in}{0.598650in}}%
\pgfpathlineto{\pgfqpoint{6.042181in}{0.596742in}}%
\pgfpathlineto{\pgfqpoint{6.131023in}{0.595017in}}%
\pgfpathlineto{\pgfqpoint{6.244096in}{0.593221in}}%
\pgfpathlineto{\pgfqpoint{6.373322in}{0.591555in}}%
\pgfpathlineto{\pgfqpoint{6.526778in}{0.589952in}}%
\pgfpathlineto{\pgfqpoint{6.567162in}{0.589583in}}%
\pgfpathlineto{\pgfqpoint{6.567162in}{0.589583in}}%
\pgfusepath{stroke}%
\end{pgfscope}%
\begin{pgfscope}%
\pgfsetrectcap%
\pgfsetmiterjoin%
\pgfsetlinewidth{0.803000pt}%
\definecolor{currentstroke}{rgb}{0.000000,0.000000,0.000000}%
\pgfsetstrokecolor{currentstroke}%
\pgfsetdash{}{0pt}%
\pgfpathmoveto{\pgfqpoint{0.882794in}{0.589583in}}%
\pgfpathlineto{\pgfqpoint{0.882794in}{5.059445in}}%
\pgfusepath{stroke}%
\end{pgfscope}%
\begin{pgfscope}%
\pgfsetrectcap%
\pgfsetmiterjoin%
\pgfsetlinewidth{0.803000pt}%
\definecolor{currentstroke}{rgb}{0.000000,0.000000,0.000000}%
\pgfsetstrokecolor{currentstroke}%
\pgfsetdash{}{0pt}%
\pgfpathmoveto{\pgfqpoint{7.800000in}{0.589583in}}%
\pgfpathlineto{\pgfqpoint{7.800000in}{5.059445in}}%
\pgfusepath{stroke}%
\end{pgfscope}%
\begin{pgfscope}%
\pgfsetrectcap%
\pgfsetmiterjoin%
\pgfsetlinewidth{0.803000pt}%
\definecolor{currentstroke}{rgb}{0.000000,0.000000,0.000000}%
\pgfsetstrokecolor{currentstroke}%
\pgfsetdash{}{0pt}%
\pgfpathmoveto{\pgfqpoint{0.882794in}{0.589583in}}%
\pgfpathlineto{\pgfqpoint{7.800000in}{0.589583in}}%
\pgfusepath{stroke}%
\end{pgfscope}%
\begin{pgfscope}%
\pgfsetrectcap%
\pgfsetmiterjoin%
\pgfsetlinewidth{0.803000pt}%
\definecolor{currentstroke}{rgb}{0.000000,0.000000,0.000000}%
\pgfsetstrokecolor{currentstroke}%
\pgfsetdash{}{0pt}%
\pgfpathmoveto{\pgfqpoint{0.882794in}{5.059445in}}%
\pgfpathlineto{\pgfqpoint{7.800000in}{5.059445in}}%
\pgfusepath{stroke}%
\end{pgfscope}%
\begin{pgfscope}%
\definecolor{textcolor}{rgb}{0.000000,0.000000,0.000000}%
\pgfsetstrokecolor{textcolor}%
\pgfsetfillcolor{textcolor}%
\pgftext[x=4.341397in,y=5.142779in,,base]{\color{textcolor}\rmfamily\fontsize{20.000000}{24.000000}\selectfont Objective Space}%
\end{pgfscope}%
\begin{pgfscope}%
\pgfsetbuttcap%
\pgfsetmiterjoin%
\definecolor{currentfill}{rgb}{1.000000,1.000000,1.000000}%
\pgfsetfillcolor{currentfill}%
\pgfsetfillopacity{0.800000}%
\pgfsetlinewidth{1.003750pt}%
\definecolor{currentstroke}{rgb}{0.800000,0.800000,0.800000}%
\pgfsetstrokecolor{currentstroke}%
\pgfsetstrokeopacity{0.800000}%
\pgfsetdash{}{0pt}%
\pgfpathmoveto{\pgfqpoint{3.876887in}{3.620866in}}%
\pgfpathlineto{\pgfqpoint{7.605556in}{3.620866in}}%
\pgfpathquadraticcurveto{\pgfqpoint{7.661111in}{3.620866in}}{\pgfqpoint{7.661111in}{3.676422in}}%
\pgfpathlineto{\pgfqpoint{7.661111in}{4.865001in}}%
\pgfpathquadraticcurveto{\pgfqpoint{7.661111in}{4.920557in}}{\pgfqpoint{7.605556in}{4.920557in}}%
\pgfpathlineto{\pgfqpoint{3.876887in}{4.920557in}}%
\pgfpathquadraticcurveto{\pgfqpoint{3.821332in}{4.920557in}}{\pgfqpoint{3.821332in}{4.865001in}}%
\pgfpathlineto{\pgfqpoint{3.821332in}{3.676422in}}%
\pgfpathquadraticcurveto{\pgfqpoint{3.821332in}{3.620866in}}{\pgfqpoint{3.876887in}{3.620866in}}%
\pgfpathlineto{\pgfqpoint{3.876887in}{3.620866in}}%
\pgfpathclose%
\pgfusepath{stroke,fill}%
\end{pgfscope}%
\begin{pgfscope}%
\pgfsetrectcap%
\pgfsetroundjoin%
\pgfsetlinewidth{3.011250pt}%
\definecolor{currentstroke}{rgb}{0.000000,0.000000,0.000000}%
\pgfsetstrokecolor{currentstroke}%
\pgfsetdash{}{0pt}%
\pgfpathmoveto{\pgfqpoint{3.932443in}{4.706629in}}%
\pgfpathlineto{\pgfqpoint{4.210221in}{4.706629in}}%
\pgfpathlineto{\pgfqpoint{4.487998in}{4.706629in}}%
\pgfusepath{stroke}%
\end{pgfscope}%
\begin{pgfscope}%
\definecolor{textcolor}{rgb}{0.000000,0.000000,0.000000}%
\pgfsetstrokecolor{textcolor}%
\pgfsetfillcolor{textcolor}%
\pgftext[x=4.710221in,y=4.609407in,left,base]{\color{textcolor}\rmfamily\fontsize{20.000000}{24.000000}\selectfont Pareto-front}%
\end{pgfscope}%
\begin{pgfscope}%
\pgfsetbuttcap%
\pgfsetmiterjoin%
\definecolor{currentfill}{rgb}{0.827451,0.827451,0.827451}%
\pgfsetfillcolor{currentfill}%
\pgfsetfillopacity{0.500000}%
\pgfsetlinewidth{0.000000pt}%
\definecolor{currentstroke}{rgb}{0.000000,0.000000,0.000000}%
\pgfsetstrokecolor{currentstroke}%
\pgfsetstrokeopacity{0.500000}%
\pgfsetdash{}{0pt}%
\pgfpathmoveto{\pgfqpoint{3.932443in}{4.214450in}}%
\pgfpathlineto{\pgfqpoint{4.487998in}{4.214450in}}%
\pgfpathlineto{\pgfqpoint{4.487998in}{4.408895in}}%
\pgfpathlineto{\pgfqpoint{3.932443in}{4.408895in}}%
\pgfpathlineto{\pgfqpoint{3.932443in}{4.214450in}}%
\pgfpathclose%
\pgfusepath{fill}%
\end{pgfscope}%
\begin{pgfscope}%
\definecolor{textcolor}{rgb}{0.000000,0.000000,0.000000}%
\pgfsetstrokecolor{textcolor}%
\pgfsetfillcolor{textcolor}%
\pgftext[x=4.710221in,y=4.214450in,left,base]{\color{textcolor}\rmfamily\fontsize{20.000000}{24.000000}\selectfont Near-optimal space}%
\end{pgfscope}%
\begin{pgfscope}%
\pgfsetbuttcap%
\pgfsetmiterjoin%
\definecolor{currentfill}{rgb}{0.121569,0.466667,0.705882}%
\pgfsetfillcolor{currentfill}%
\pgfsetfillopacity{0.200000}%
\pgfsetlinewidth{0.000000pt}%
\definecolor{currentstroke}{rgb}{0.000000,0.000000,0.000000}%
\pgfsetstrokecolor{currentstroke}%
\pgfsetstrokeopacity{0.200000}%
\pgfsetdash{}{0pt}%
\pgfpathmoveto{\pgfqpoint{3.932443in}{3.803750in}}%
\pgfpathlineto{\pgfqpoint{4.487998in}{3.803750in}}%
\pgfpathlineto{\pgfqpoint{4.487998in}{3.998194in}}%
\pgfpathlineto{\pgfqpoint{3.932443in}{3.998194in}}%
\pgfpathlineto{\pgfqpoint{3.932443in}{3.803750in}}%
\pgfpathclose%
\pgfusepath{fill}%
\end{pgfscope}%
\begin{pgfscope}%
\pgfsetbuttcap%
\pgfsetmiterjoin%
\definecolor{currentfill}{rgb}{0.121569,0.466667,0.705882}%
\pgfsetfillcolor{currentfill}%
\pgfsetfillopacity{0.200000}%
\pgfsetlinewidth{0.000000pt}%
\definecolor{currentstroke}{rgb}{0.000000,0.000000,0.000000}%
\pgfsetstrokecolor{currentstroke}%
\pgfsetstrokeopacity{0.200000}%
\pgfsetdash{}{0pt}%
\pgfpathmoveto{\pgfqpoint{3.932443in}{3.803750in}}%
\pgfpathlineto{\pgfqpoint{4.487998in}{3.803750in}}%
\pgfpathlineto{\pgfqpoint{4.487998in}{3.998194in}}%
\pgfpathlineto{\pgfqpoint{3.932443in}{3.998194in}}%
\pgfpathlineto{\pgfqpoint{3.932443in}{3.803750in}}%
\pgfpathclose%
\pgfusepath{clip}%
\pgfsys@defobject{currentpattern}{\pgfqpoint{0in}{0in}}{\pgfqpoint{1in}{1in}}{%
\begin{pgfscope}%
\pgfpathrectangle{\pgfqpoint{0in}{0in}}{\pgfqpoint{1in}{1in}}%
\pgfusepath{clip}%
\pgfpathmoveto{\pgfqpoint{-0.500000in}{0.500000in}}%
\pgfpathlineto{\pgfqpoint{0.500000in}{1.500000in}}%
\pgfpathmoveto{\pgfqpoint{-0.416667in}{0.416667in}}%
\pgfpathlineto{\pgfqpoint{0.583333in}{1.416667in}}%
\pgfpathmoveto{\pgfqpoint{-0.333333in}{0.333333in}}%
\pgfpathlineto{\pgfqpoint{0.666667in}{1.333333in}}%
\pgfpathmoveto{\pgfqpoint{-0.250000in}{0.250000in}}%
\pgfpathlineto{\pgfqpoint{0.750000in}{1.250000in}}%
\pgfpathmoveto{\pgfqpoint{-0.166667in}{0.166667in}}%
\pgfpathlineto{\pgfqpoint{0.833333in}{1.166667in}}%
\pgfpathmoveto{\pgfqpoint{-0.083333in}{0.083333in}}%
\pgfpathlineto{\pgfqpoint{0.916667in}{1.083333in}}%
\pgfpathmoveto{\pgfqpoint{0.000000in}{0.000000in}}%
\pgfpathlineto{\pgfqpoint{1.000000in}{1.000000in}}%
\pgfpathmoveto{\pgfqpoint{0.083333in}{-0.083333in}}%
\pgfpathlineto{\pgfqpoint{1.083333in}{0.916667in}}%
\pgfpathmoveto{\pgfqpoint{0.166667in}{-0.166667in}}%
\pgfpathlineto{\pgfqpoint{1.166667in}{0.833333in}}%
\pgfpathmoveto{\pgfqpoint{0.250000in}{-0.250000in}}%
\pgfpathlineto{\pgfqpoint{1.250000in}{0.750000in}}%
\pgfpathmoveto{\pgfqpoint{0.333333in}{-0.333333in}}%
\pgfpathlineto{\pgfqpoint{1.333333in}{0.666667in}}%
\pgfpathmoveto{\pgfqpoint{0.416667in}{-0.416667in}}%
\pgfpathlineto{\pgfqpoint{1.416667in}{0.583333in}}%
\pgfpathmoveto{\pgfqpoint{0.500000in}{-0.500000in}}%
\pgfpathlineto{\pgfqpoint{1.500000in}{0.500000in}}%
\pgfusepath{stroke}%
\end{pgfscope}%
}%
\pgfsys@transformshift{3.932443in}{3.803750in}%
\pgfsys@useobject{currentpattern}{}%
\pgfsys@transformshift{1in}{0in}%
\pgfsys@transformshift{-1in}{0in}%
\pgfsys@transformshift{0in}{1in}%
\end{pgfscope}%
\begin{pgfscope}%
\definecolor{textcolor}{rgb}{0.000000,0.000000,0.000000}%
\pgfsetstrokecolor{textcolor}%
\pgfsetfillcolor{textcolor}%
\pgftext[x=4.710221in,y=3.803750in,left,base]{\color{textcolor}\rmfamily\fontsize{20.000000}{24.000000}\selectfont MGA Search Space (F1)}%
\end{pgfscope}%
\begin{pgfscope}%
\definecolor{textcolor}{rgb}{0.000000,0.000000,0.000000}%
\pgfsetstrokecolor{textcolor}%
\pgfsetfillcolor{textcolor}%
\pgftext[x=3.950000in,y=5.830000in,,top]{\color{textcolor}\rmfamily\fontsize{24.000000}{28.800000}\selectfont Multi-objective MGA}%
\end{pgfscope}%
\end{pgfpicture}%
\makeatother%
\endgroup%
}
            \caption{Near optimal space for mono- and multi-objective problems. The light blue area shows
            a vertically truncated near-optimal space around the f1 objective.}
            \label{fig:near-opt-mga}
        \end{figure}
    \end{columns}

\end{frame}

\begin{frame}
    \frametitle{How \texttt{Osier} handles structural uncertainty}

    \begin{columns}
        \column[t]{4cm}

        \column[t]{6cm}
        \begin{figure}
            \centering
            \resizebox{\columnwidth}{!}{%% Creator: Matplotlib, PGF backend
%%
%% To include the figure in your LaTeX document, write
%%   \input{<filename>.pgf}
%%
%% Make sure the required packages are loaded in your preamble
%%   \usepackage{pgf}
%%
%% Also ensure that all the required font packages are loaded; for instance,
%% the lmodern package is sometimes necessary when using math font.
%%   \usepackage{lmodern}
%%
%% Figures using additional raster images can only be included by \input if
%% they are in the same directory as the main LaTeX file. For loading figures
%% from other directories you can use the `import` package
%%   \usepackage{import}
%%
%% and then include the figures with
%%   \import{<path to file>}{<filename>.pgf}
%%
%% Matplotlib used the following preamble
%%   
%%   \makeatletter\@ifpackageloaded{underscore}{}{\usepackage[strings]{underscore}}\makeatother
%%
\begingroup%
\makeatletter%
\begin{pgfpicture}%
\pgfpathrectangle{\pgfpointorigin}{\pgfqpoint{7.182794in}{5.909583in}}%
\pgfusepath{use as bounding box, clip}%
\begin{pgfscope}%
\pgfsetbuttcap%
\pgfsetmiterjoin%
\definecolor{currentfill}{rgb}{0.827451,0.827451,0.827451}%
\pgfsetfillcolor{currentfill}%
\pgfsetlinewidth{0.000000pt}%
\definecolor{currentstroke}{rgb}{0.000000,0.000000,0.000000}%
\pgfsetstrokecolor{currentstroke}%
\pgfsetdash{}{0pt}%
\pgfpathmoveto{\pgfqpoint{0.000000in}{0.000000in}}%
\pgfpathlineto{\pgfqpoint{7.182794in}{0.000000in}}%
\pgfpathlineto{\pgfqpoint{7.182794in}{5.909583in}}%
\pgfpathlineto{\pgfqpoint{0.000000in}{5.909583in}}%
\pgfpathlineto{\pgfqpoint{0.000000in}{0.000000in}}%
\pgfpathclose%
\pgfusepath{fill}%
\end{pgfscope}%
\begin{pgfscope}%
\pgfsetbuttcap%
\pgfsetmiterjoin%
\definecolor{currentfill}{rgb}{1.000000,1.000000,1.000000}%
\pgfsetfillcolor{currentfill}%
\pgfsetlinewidth{0.000000pt}%
\definecolor{currentstroke}{rgb}{0.000000,0.000000,0.000000}%
\pgfsetstrokecolor{currentstroke}%
\pgfsetstrokeopacity{0.000000}%
\pgfsetdash{}{0pt}%
\pgfpathmoveto{\pgfqpoint{0.882794in}{0.589583in}}%
\pgfpathlineto{\pgfqpoint{7.082794in}{0.589583in}}%
\pgfpathlineto{\pgfqpoint{7.082794in}{5.209583in}}%
\pgfpathlineto{\pgfqpoint{0.882794in}{5.209583in}}%
\pgfpathlineto{\pgfqpoint{0.882794in}{0.589583in}}%
\pgfpathclose%
\pgfusepath{fill}%
\end{pgfscope}%
\begin{pgfscope}%
\pgfpathrectangle{\pgfqpoint{0.882794in}{0.589583in}}{\pgfqpoint{6.200000in}{4.620000in}}%
\pgfusepath{clip}%
\pgfsetbuttcap%
\pgfsetroundjoin%
\definecolor{currentfill}{rgb}{0.121569,0.466667,0.705882}%
\pgfsetfillcolor{currentfill}%
\pgfsetlinewidth{1.003750pt}%
\definecolor{currentstroke}{rgb}{0.121569,0.466667,0.705882}%
\pgfsetstrokecolor{currentstroke}%
\pgfsetdash{}{0pt}%
\pgfsys@defobject{currentmarker}{\pgfqpoint{-0.012028in}{-0.012028in}}{\pgfqpoint{0.012028in}{0.012028in}}{%
\pgfpathmoveto{\pgfqpoint{0.000000in}{-0.012028in}}%
\pgfpathcurveto{\pgfqpoint{0.003190in}{-0.012028in}}{\pgfqpoint{0.006250in}{-0.010761in}}{\pgfqpoint{0.008505in}{-0.008505in}}%
\pgfpathcurveto{\pgfqpoint{0.010761in}{-0.006250in}}{\pgfqpoint{0.012028in}{-0.003190in}}{\pgfqpoint{0.012028in}{0.000000in}}%
\pgfpathcurveto{\pgfqpoint{0.012028in}{0.003190in}}{\pgfqpoint{0.010761in}{0.006250in}}{\pgfqpoint{0.008505in}{0.008505in}}%
\pgfpathcurveto{\pgfqpoint{0.006250in}{0.010761in}}{\pgfqpoint{0.003190in}{0.012028in}}{\pgfqpoint{0.000000in}{0.012028in}}%
\pgfpathcurveto{\pgfqpoint{-0.003190in}{0.012028in}}{\pgfqpoint{-0.006250in}{0.010761in}}{\pgfqpoint{-0.008505in}{0.008505in}}%
\pgfpathcurveto{\pgfqpoint{-0.010761in}{0.006250in}}{\pgfqpoint{-0.012028in}{0.003190in}}{\pgfqpoint{-0.012028in}{0.000000in}}%
\pgfpathcurveto{\pgfqpoint{-0.012028in}{-0.003190in}}{\pgfqpoint{-0.010761in}{-0.006250in}}{\pgfqpoint{-0.008505in}{-0.008505in}}%
\pgfpathcurveto{\pgfqpoint{-0.006250in}{-0.010761in}}{\pgfqpoint{-0.003190in}{-0.012028in}}{\pgfqpoint{0.000000in}{-0.012028in}}%
\pgfpathlineto{\pgfqpoint{0.000000in}{-0.012028in}}%
\pgfpathclose%
\pgfusepath{stroke,fill}%
}%
\begin{pgfscope}%
\pgfsys@transformshift{2.100073in}{3.359730in}%
\pgfsys@useobject{currentmarker}{}%
\end{pgfscope}%
\begin{pgfscope}%
\pgfsys@transformshift{4.217831in}{4.158870in}%
\pgfsys@useobject{currentmarker}{}%
\end{pgfscope}%
\begin{pgfscope}%
\pgfsys@transformshift{3.530482in}{2.766855in}%
\pgfsys@useobject{currentmarker}{}%
\end{pgfscope}%
\begin{pgfscope}%
\pgfsys@transformshift{3.637429in}{3.061551in}%
\pgfsys@useobject{currentmarker}{}%
\end{pgfscope}%
\begin{pgfscope}%
\pgfsys@transformshift{4.779986in}{0.381783in}%
\pgfsys@useobject{currentmarker}{}%
\end{pgfscope}%
\begin{pgfscope}%
\pgfsys@transformshift{3.591284in}{3.326871in}%
\pgfsys@useobject{currentmarker}{}%
\end{pgfscope}%
\begin{pgfscope}%
\pgfsys@transformshift{1.101111in}{2.119851in}%
\pgfsys@useobject{currentmarker}{}%
\end{pgfscope}%
\begin{pgfscope}%
\pgfsys@transformshift{3.869255in}{4.175375in}%
\pgfsys@useobject{currentmarker}{}%
\end{pgfscope}%
\begin{pgfscope}%
\pgfsys@transformshift{3.177983in}{3.095340in}%
\pgfsys@useobject{currentmarker}{}%
\end{pgfscope}%
\begin{pgfscope}%
\pgfsys@transformshift{0.967168in}{1.216505in}%
\pgfsys@useobject{currentmarker}{}%
\end{pgfscope}%
\begin{pgfscope}%
\pgfsys@transformshift{0.860046in}{3.618060in}%
\pgfsys@useobject{currentmarker}{}%
\end{pgfscope}%
\begin{pgfscope}%
\pgfsys@transformshift{5.567380in}{2.925043in}%
\pgfsys@useobject{currentmarker}{}%
\end{pgfscope}%
\begin{pgfscope}%
\pgfsys@transformshift{0.825371in}{3.062710in}%
\pgfsys@useobject{currentmarker}{}%
\end{pgfscope}%
\begin{pgfscope}%
\pgfsys@transformshift{3.288360in}{2.776506in}%
\pgfsys@useobject{currentmarker}{}%
\end{pgfscope}%
\begin{pgfscope}%
\pgfsys@transformshift{1.415178in}{3.286718in}%
\pgfsys@useobject{currentmarker}{}%
\end{pgfscope}%
\begin{pgfscope}%
\pgfsys@transformshift{5.320688in}{0.347495in}%
\pgfsys@useobject{currentmarker}{}%
\end{pgfscope}%
\begin{pgfscope}%
\pgfsys@transformshift{2.906304in}{3.373475in}%
\pgfsys@useobject{currentmarker}{}%
\end{pgfscope}%
\begin{pgfscope}%
\pgfsys@transformshift{4.565040in}{1.272252in}%
\pgfsys@useobject{currentmarker}{}%
\end{pgfscope}%
\begin{pgfscope}%
\pgfsys@transformshift{3.741220in}{0.578106in}%
\pgfsys@useobject{currentmarker}{}%
\end{pgfscope}%
\begin{pgfscope}%
\pgfsys@transformshift{1.518812in}{0.898777in}%
\pgfsys@useobject{currentmarker}{}%
\end{pgfscope}%
\begin{pgfscope}%
\pgfsys@transformshift{4.509463in}{0.838791in}%
\pgfsys@useobject{currentmarker}{}%
\end{pgfscope}%
\begin{pgfscope}%
\pgfsys@transformshift{4.037422in}{2.937484in}%
\pgfsys@useobject{currentmarker}{}%
\end{pgfscope}%
\begin{pgfscope}%
\pgfsys@transformshift{0.506122in}{1.786085in}%
\pgfsys@useobject{currentmarker}{}%
\end{pgfscope}%
\begin{pgfscope}%
\pgfsys@transformshift{4.210653in}{3.310975in}%
\pgfsys@useobject{currentmarker}{}%
\end{pgfscope}%
\begin{pgfscope}%
\pgfsys@transformshift{1.765073in}{3.096862in}%
\pgfsys@useobject{currentmarker}{}%
\end{pgfscope}%
\begin{pgfscope}%
\pgfsys@transformshift{4.584525in}{2.774513in}%
\pgfsys@useobject{currentmarker}{}%
\end{pgfscope}%
\begin{pgfscope}%
\pgfsys@transformshift{0.943999in}{3.591336in}%
\pgfsys@useobject{currentmarker}{}%
\end{pgfscope}%
\begin{pgfscope}%
\pgfsys@transformshift{0.707728in}{3.221611in}%
\pgfsys@useobject{currentmarker}{}%
\end{pgfscope}%
\begin{pgfscope}%
\pgfsys@transformshift{3.254289in}{1.262186in}%
\pgfsys@useobject{currentmarker}{}%
\end{pgfscope}%
\begin{pgfscope}%
\pgfsys@transformshift{3.446115in}{1.208192in}%
\pgfsys@useobject{currentmarker}{}%
\end{pgfscope}%
\begin{pgfscope}%
\pgfsys@transformshift{2.956831in}{1.055493in}%
\pgfsys@useobject{currentmarker}{}%
\end{pgfscope}%
\begin{pgfscope}%
\pgfsys@transformshift{3.335722in}{3.533231in}%
\pgfsys@useobject{currentmarker}{}%
\end{pgfscope}%
\begin{pgfscope}%
\pgfsys@transformshift{1.083575in}{0.583651in}%
\pgfsys@useobject{currentmarker}{}%
\end{pgfscope}%
\begin{pgfscope}%
\pgfsys@transformshift{3.232677in}{3.204910in}%
\pgfsys@useobject{currentmarker}{}%
\end{pgfscope}%
\begin{pgfscope}%
\pgfsys@transformshift{1.941356in}{0.970444in}%
\pgfsys@useobject{currentmarker}{}%
\end{pgfscope}%
\begin{pgfscope}%
\pgfsys@transformshift{3.184096in}{3.096283in}%
\pgfsys@useobject{currentmarker}{}%
\end{pgfscope}%
\begin{pgfscope}%
\pgfsys@transformshift{1.181901in}{1.944497in}%
\pgfsys@useobject{currentmarker}{}%
\end{pgfscope}%
\begin{pgfscope}%
\pgfsys@transformshift{5.204777in}{1.650985in}%
\pgfsys@useobject{currentmarker}{}%
\end{pgfscope}%
\begin{pgfscope}%
\pgfsys@transformshift{2.651719in}{0.844590in}%
\pgfsys@useobject{currentmarker}{}%
\end{pgfscope}%
\begin{pgfscope}%
\pgfsys@transformshift{2.188120in}{3.732684in}%
\pgfsys@useobject{currentmarker}{}%
\end{pgfscope}%
\begin{pgfscope}%
\pgfsys@transformshift{5.141584in}{2.523362in}%
\pgfsys@useobject{currentmarker}{}%
\end{pgfscope}%
\begin{pgfscope}%
\pgfsys@transformshift{5.556943in}{2.106391in}%
\pgfsys@useobject{currentmarker}{}%
\end{pgfscope}%
\begin{pgfscope}%
\pgfsys@transformshift{2.307494in}{3.912087in}%
\pgfsys@useobject{currentmarker}{}%
\end{pgfscope}%
\begin{pgfscope}%
\pgfsys@transformshift{5.809528in}{2.731059in}%
\pgfsys@useobject{currentmarker}{}%
\end{pgfscope}%
\begin{pgfscope}%
\pgfsys@transformshift{4.270999in}{0.466305in}%
\pgfsys@useobject{currentmarker}{}%
\end{pgfscope}%
\begin{pgfscope}%
\pgfsys@transformshift{3.577124in}{3.891412in}%
\pgfsys@useobject{currentmarker}{}%
\end{pgfscope}%
\begin{pgfscope}%
\pgfsys@transformshift{4.543302in}{2.000373in}%
\pgfsys@useobject{currentmarker}{}%
\end{pgfscope}%
\begin{pgfscope}%
\pgfsys@transformshift{1.927577in}{3.922569in}%
\pgfsys@useobject{currentmarker}{}%
\end{pgfscope}%
\begin{pgfscope}%
\pgfsys@transformshift{1.545362in}{2.123893in}%
\pgfsys@useobject{currentmarker}{}%
\end{pgfscope}%
\begin{pgfscope}%
\pgfsys@transformshift{5.650894in}{0.819100in}%
\pgfsys@useobject{currentmarker}{}%
\end{pgfscope}%
\begin{pgfscope}%
\pgfsys@transformshift{0.806057in}{2.465734in}%
\pgfsys@useobject{currentmarker}{}%
\end{pgfscope}%
\begin{pgfscope}%
\pgfsys@transformshift{2.878941in}{1.954256in}%
\pgfsys@useobject{currentmarker}{}%
\end{pgfscope}%
\begin{pgfscope}%
\pgfsys@transformshift{1.171177in}{4.038892in}%
\pgfsys@useobject{currentmarker}{}%
\end{pgfscope}%
\begin{pgfscope}%
\pgfsys@transformshift{4.833223in}{3.550342in}%
\pgfsys@useobject{currentmarker}{}%
\end{pgfscope}%
\begin{pgfscope}%
\pgfsys@transformshift{4.295352in}{2.346421in}%
\pgfsys@useobject{currentmarker}{}%
\end{pgfscope}%
\begin{pgfscope}%
\pgfsys@transformshift{3.664721in}{0.375118in}%
\pgfsys@useobject{currentmarker}{}%
\end{pgfscope}%
\begin{pgfscope}%
\pgfsys@transformshift{3.478267in}{3.399011in}%
\pgfsys@useobject{currentmarker}{}%
\end{pgfscope}%
\begin{pgfscope}%
\pgfsys@transformshift{3.530565in}{1.356179in}%
\pgfsys@useobject{currentmarker}{}%
\end{pgfscope}%
\begin{pgfscope}%
\pgfsys@transformshift{2.373598in}{2.368232in}%
\pgfsys@useobject{currentmarker}{}%
\end{pgfscope}%
\begin{pgfscope}%
\pgfsys@transformshift{1.079856in}{3.500371in}%
\pgfsys@useobject{currentmarker}{}%
\end{pgfscope}%
\begin{pgfscope}%
\pgfsys@transformshift{3.690357in}{1.767902in}%
\pgfsys@useobject{currentmarker}{}%
\end{pgfscope}%
\begin{pgfscope}%
\pgfsys@transformshift{0.471552in}{0.519812in}%
\pgfsys@useobject{currentmarker}{}%
\end{pgfscope}%
\begin{pgfscope}%
\pgfsys@transformshift{1.672303in}{3.242874in}%
\pgfsys@useobject{currentmarker}{}%
\end{pgfscope}%
\begin{pgfscope}%
\pgfsys@transformshift{1.368439in}{1.769036in}%
\pgfsys@useobject{currentmarker}{}%
\end{pgfscope}%
\begin{pgfscope}%
\pgfsys@transformshift{0.954978in}{3.355893in}%
\pgfsys@useobject{currentmarker}{}%
\end{pgfscope}%
\begin{pgfscope}%
\pgfsys@transformshift{3.087677in}{4.028780in}%
\pgfsys@useobject{currentmarker}{}%
\end{pgfscope}%
\begin{pgfscope}%
\pgfsys@transformshift{5.297726in}{0.501205in}%
\pgfsys@useobject{currentmarker}{}%
\end{pgfscope}%
\begin{pgfscope}%
\pgfsys@transformshift{5.317007in}{3.539701in}%
\pgfsys@useobject{currentmarker}{}%
\end{pgfscope}%
\begin{pgfscope}%
\pgfsys@transformshift{2.029084in}{4.217871in}%
\pgfsys@useobject{currentmarker}{}%
\end{pgfscope}%
\begin{pgfscope}%
\pgfsys@transformshift{5.719993in}{3.034418in}%
\pgfsys@useobject{currentmarker}{}%
\end{pgfscope}%
\begin{pgfscope}%
\pgfsys@transformshift{2.520368in}{4.171447in}%
\pgfsys@useobject{currentmarker}{}%
\end{pgfscope}%
\begin{pgfscope}%
\pgfsys@transformshift{3.320250in}{2.850809in}%
\pgfsys@useobject{currentmarker}{}%
\end{pgfscope}%
\begin{pgfscope}%
\pgfsys@transformshift{3.847630in}{0.466377in}%
\pgfsys@useobject{currentmarker}{}%
\end{pgfscope}%
\begin{pgfscope}%
\pgfsys@transformshift{1.107831in}{2.321797in}%
\pgfsys@useobject{currentmarker}{}%
\end{pgfscope}%
\begin{pgfscope}%
\pgfsys@transformshift{0.813134in}{1.167063in}%
\pgfsys@useobject{currentmarker}{}%
\end{pgfscope}%
\begin{pgfscope}%
\pgfsys@transformshift{5.316021in}{1.160720in}%
\pgfsys@useobject{currentmarker}{}%
\end{pgfscope}%
\begin{pgfscope}%
\pgfsys@transformshift{5.326415in}{2.830537in}%
\pgfsys@useobject{currentmarker}{}%
\end{pgfscope}%
\begin{pgfscope}%
\pgfsys@transformshift{2.660425in}{3.739571in}%
\pgfsys@useobject{currentmarker}{}%
\end{pgfscope}%
\begin{pgfscope}%
\pgfsys@transformshift{4.166851in}{3.125376in}%
\pgfsys@useobject{currentmarker}{}%
\end{pgfscope}%
\begin{pgfscope}%
\pgfsys@transformshift{4.987788in}{1.281935in}%
\pgfsys@useobject{currentmarker}{}%
\end{pgfscope}%
\begin{pgfscope}%
\pgfsys@transformshift{1.878059in}{2.766832in}%
\pgfsys@useobject{currentmarker}{}%
\end{pgfscope}%
\begin{pgfscope}%
\pgfsys@transformshift{1.060812in}{1.085980in}%
\pgfsys@useobject{currentmarker}{}%
\end{pgfscope}%
\begin{pgfscope}%
\pgfsys@transformshift{1.842360in}{1.702844in}%
\pgfsys@useobject{currentmarker}{}%
\end{pgfscope}%
\begin{pgfscope}%
\pgfsys@transformshift{5.069192in}{4.099541in}%
\pgfsys@useobject{currentmarker}{}%
\end{pgfscope}%
\begin{pgfscope}%
\pgfsys@transformshift{0.957614in}{0.603174in}%
\pgfsys@useobject{currentmarker}{}%
\end{pgfscope}%
\begin{pgfscope}%
\pgfsys@transformshift{0.899750in}{2.840238in}%
\pgfsys@useobject{currentmarker}{}%
\end{pgfscope}%
\begin{pgfscope}%
\pgfsys@transformshift{1.386584in}{1.916995in}%
\pgfsys@useobject{currentmarker}{}%
\end{pgfscope}%
\begin{pgfscope}%
\pgfsys@transformshift{2.380757in}{0.993148in}%
\pgfsys@useobject{currentmarker}{}%
\end{pgfscope}%
\begin{pgfscope}%
\pgfsys@transformshift{5.058663in}{2.491758in}%
\pgfsys@useobject{currentmarker}{}%
\end{pgfscope}%
\begin{pgfscope}%
\pgfsys@transformshift{0.642909in}{3.280817in}%
\pgfsys@useobject{currentmarker}{}%
\end{pgfscope}%
\begin{pgfscope}%
\pgfsys@transformshift{5.811124in}{2.335918in}%
\pgfsys@useobject{currentmarker}{}%
\end{pgfscope}%
\begin{pgfscope}%
\pgfsys@transformshift{3.575588in}{0.493562in}%
\pgfsys@useobject{currentmarker}{}%
\end{pgfscope}%
\begin{pgfscope}%
\pgfsys@transformshift{0.635014in}{2.436559in}%
\pgfsys@useobject{currentmarker}{}%
\end{pgfscope}%
\begin{pgfscope}%
\pgfsys@transformshift{4.291678in}{1.483455in}%
\pgfsys@useobject{currentmarker}{}%
\end{pgfscope}%
\begin{pgfscope}%
\pgfsys@transformshift{5.497314in}{2.727070in}%
\pgfsys@useobject{currentmarker}{}%
\end{pgfscope}%
\begin{pgfscope}%
\pgfsys@transformshift{4.644707in}{1.620132in}%
\pgfsys@useobject{currentmarker}{}%
\end{pgfscope}%
\begin{pgfscope}%
\pgfsys@transformshift{5.656718in}{4.003130in}%
\pgfsys@useobject{currentmarker}{}%
\end{pgfscope}%
\begin{pgfscope}%
\pgfsys@transformshift{2.780148in}{2.871162in}%
\pgfsys@useobject{currentmarker}{}%
\end{pgfscope}%
\begin{pgfscope}%
\pgfsys@transformshift{1.298492in}{3.807777in}%
\pgfsys@useobject{currentmarker}{}%
\end{pgfscope}%
\begin{pgfscope}%
\pgfsys@transformshift{0.557659in}{2.551482in}%
\pgfsys@useobject{currentmarker}{}%
\end{pgfscope}%
\begin{pgfscope}%
\pgfsys@transformshift{2.396618in}{1.803277in}%
\pgfsys@useobject{currentmarker}{}%
\end{pgfscope}%
\begin{pgfscope}%
\pgfsys@transformshift{0.613365in}{1.713496in}%
\pgfsys@useobject{currentmarker}{}%
\end{pgfscope}%
\begin{pgfscope}%
\pgfsys@transformshift{3.368599in}{0.902214in}%
\pgfsys@useobject{currentmarker}{}%
\end{pgfscope}%
\begin{pgfscope}%
\pgfsys@transformshift{4.893435in}{3.737142in}%
\pgfsys@useobject{currentmarker}{}%
\end{pgfscope}%
\begin{pgfscope}%
\pgfsys@transformshift{4.607695in}{0.927907in}%
\pgfsys@useobject{currentmarker}{}%
\end{pgfscope}%
\begin{pgfscope}%
\pgfsys@transformshift{1.788642in}{3.494098in}%
\pgfsys@useobject{currentmarker}{}%
\end{pgfscope}%
\begin{pgfscope}%
\pgfsys@transformshift{1.647411in}{1.372659in}%
\pgfsys@useobject{currentmarker}{}%
\end{pgfscope}%
\begin{pgfscope}%
\pgfsys@transformshift{4.668509in}{1.198898in}%
\pgfsys@useobject{currentmarker}{}%
\end{pgfscope}%
\begin{pgfscope}%
\pgfsys@transformshift{5.031794in}{3.045731in}%
\pgfsys@useobject{currentmarker}{}%
\end{pgfscope}%
\begin{pgfscope}%
\pgfsys@transformshift{3.229043in}{1.972049in}%
\pgfsys@useobject{currentmarker}{}%
\end{pgfscope}%
\begin{pgfscope}%
\pgfsys@transformshift{2.200674in}{3.579609in}%
\pgfsys@useobject{currentmarker}{}%
\end{pgfscope}%
\begin{pgfscope}%
\pgfsys@transformshift{5.396722in}{0.568813in}%
\pgfsys@useobject{currentmarker}{}%
\end{pgfscope}%
\begin{pgfscope}%
\pgfsys@transformshift{5.121937in}{1.332336in}%
\pgfsys@useobject{currentmarker}{}%
\end{pgfscope}%
\begin{pgfscope}%
\pgfsys@transformshift{3.269707in}{2.322776in}%
\pgfsys@useobject{currentmarker}{}%
\end{pgfscope}%
\begin{pgfscope}%
\pgfsys@transformshift{1.618602in}{0.936398in}%
\pgfsys@useobject{currentmarker}{}%
\end{pgfscope}%
\begin{pgfscope}%
\pgfsys@transformshift{4.013600in}{3.336894in}%
\pgfsys@useobject{currentmarker}{}%
\end{pgfscope}%
\begin{pgfscope}%
\pgfsys@transformshift{2.006451in}{4.036448in}%
\pgfsys@useobject{currentmarker}{}%
\end{pgfscope}%
\begin{pgfscope}%
\pgfsys@transformshift{4.600031in}{4.074283in}%
\pgfsys@useobject{currentmarker}{}%
\end{pgfscope}%
\begin{pgfscope}%
\pgfsys@transformshift{3.058098in}{0.388669in}%
\pgfsys@useobject{currentmarker}{}%
\end{pgfscope}%
\begin{pgfscope}%
\pgfsys@transformshift{4.438382in}{3.757359in}%
\pgfsys@useobject{currentmarker}{}%
\end{pgfscope}%
\begin{pgfscope}%
\pgfsys@transformshift{1.020770in}{3.589328in}%
\pgfsys@useobject{currentmarker}{}%
\end{pgfscope}%
\begin{pgfscope}%
\pgfsys@transformshift{3.424355in}{3.210978in}%
\pgfsys@useobject{currentmarker}{}%
\end{pgfscope}%
\begin{pgfscope}%
\pgfsys@transformshift{2.422921in}{1.624184in}%
\pgfsys@useobject{currentmarker}{}%
\end{pgfscope}%
\begin{pgfscope}%
\pgfsys@transformshift{1.703748in}{3.481595in}%
\pgfsys@useobject{currentmarker}{}%
\end{pgfscope}%
\begin{pgfscope}%
\pgfsys@transformshift{4.888602in}{0.976468in}%
\pgfsys@useobject{currentmarker}{}%
\end{pgfscope}%
\begin{pgfscope}%
\pgfsys@transformshift{1.459592in}{2.875074in}%
\pgfsys@useobject{currentmarker}{}%
\end{pgfscope}%
\begin{pgfscope}%
\pgfsys@transformshift{1.908647in}{0.623622in}%
\pgfsys@useobject{currentmarker}{}%
\end{pgfscope}%
\begin{pgfscope}%
\pgfsys@transformshift{0.729895in}{1.501131in}%
\pgfsys@useobject{currentmarker}{}%
\end{pgfscope}%
\begin{pgfscope}%
\pgfsys@transformshift{1.422167in}{2.296890in}%
\pgfsys@useobject{currentmarker}{}%
\end{pgfscope}%
\begin{pgfscope}%
\pgfsys@transformshift{2.072751in}{3.189423in}%
\pgfsys@useobject{currentmarker}{}%
\end{pgfscope}%
\begin{pgfscope}%
\pgfsys@transformshift{1.033658in}{1.869031in}%
\pgfsys@useobject{currentmarker}{}%
\end{pgfscope}%
\begin{pgfscope}%
\pgfsys@transformshift{1.895144in}{4.057693in}%
\pgfsys@useobject{currentmarker}{}%
\end{pgfscope}%
\begin{pgfscope}%
\pgfsys@transformshift{3.543365in}{0.745184in}%
\pgfsys@useobject{currentmarker}{}%
\end{pgfscope}%
\begin{pgfscope}%
\pgfsys@transformshift{0.807651in}{1.170396in}%
\pgfsys@useobject{currentmarker}{}%
\end{pgfscope}%
\begin{pgfscope}%
\pgfsys@transformshift{4.066184in}{1.981721in}%
\pgfsys@useobject{currentmarker}{}%
\end{pgfscope}%
\begin{pgfscope}%
\pgfsys@transformshift{0.837799in}{0.867009in}%
\pgfsys@useobject{currentmarker}{}%
\end{pgfscope}%
\begin{pgfscope}%
\pgfsys@transformshift{4.220530in}{0.662301in}%
\pgfsys@useobject{currentmarker}{}%
\end{pgfscope}%
\begin{pgfscope}%
\pgfsys@transformshift{1.580546in}{1.260231in}%
\pgfsys@useobject{currentmarker}{}%
\end{pgfscope}%
\begin{pgfscope}%
\pgfsys@transformshift{3.830798in}{1.002505in}%
\pgfsys@useobject{currentmarker}{}%
\end{pgfscope}%
\begin{pgfscope}%
\pgfsys@transformshift{0.901170in}{0.768312in}%
\pgfsys@useobject{currentmarker}{}%
\end{pgfscope}%
\begin{pgfscope}%
\pgfsys@transformshift{3.574086in}{3.517224in}%
\pgfsys@useobject{currentmarker}{}%
\end{pgfscope}%
\begin{pgfscope}%
\pgfsys@transformshift{5.655091in}{2.247952in}%
\pgfsys@useobject{currentmarker}{}%
\end{pgfscope}%
\begin{pgfscope}%
\pgfsys@transformshift{3.674885in}{0.623137in}%
\pgfsys@useobject{currentmarker}{}%
\end{pgfscope}%
\begin{pgfscope}%
\pgfsys@transformshift{4.142407in}{3.837561in}%
\pgfsys@useobject{currentmarker}{}%
\end{pgfscope}%
\begin{pgfscope}%
\pgfsys@transformshift{1.558736in}{1.870449in}%
\pgfsys@useobject{currentmarker}{}%
\end{pgfscope}%
\begin{pgfscope}%
\pgfsys@transformshift{3.425837in}{1.489090in}%
\pgfsys@useobject{currentmarker}{}%
\end{pgfscope}%
\begin{pgfscope}%
\pgfsys@transformshift{5.061292in}{1.643268in}%
\pgfsys@useobject{currentmarker}{}%
\end{pgfscope}%
\begin{pgfscope}%
\pgfsys@transformshift{0.748877in}{3.384093in}%
\pgfsys@useobject{currentmarker}{}%
\end{pgfscope}%
\begin{pgfscope}%
\pgfsys@transformshift{4.927220in}{0.519765in}%
\pgfsys@useobject{currentmarker}{}%
\end{pgfscope}%
\begin{pgfscope}%
\pgfsys@transformshift{3.337585in}{2.035396in}%
\pgfsys@useobject{currentmarker}{}%
\end{pgfscope}%
\begin{pgfscope}%
\pgfsys@transformshift{4.262713in}{3.151119in}%
\pgfsys@useobject{currentmarker}{}%
\end{pgfscope}%
\begin{pgfscope}%
\pgfsys@transformshift{5.764665in}{1.181409in}%
\pgfsys@useobject{currentmarker}{}%
\end{pgfscope}%
\begin{pgfscope}%
\pgfsys@transformshift{0.623516in}{3.455553in}%
\pgfsys@useobject{currentmarker}{}%
\end{pgfscope}%
\begin{pgfscope}%
\pgfsys@transformshift{1.849037in}{1.575144in}%
\pgfsys@useobject{currentmarker}{}%
\end{pgfscope}%
\begin{pgfscope}%
\pgfsys@transformshift{3.399646in}{0.344460in}%
\pgfsys@useobject{currentmarker}{}%
\end{pgfscope}%
\begin{pgfscope}%
\pgfsys@transformshift{4.224913in}{4.216073in}%
\pgfsys@useobject{currentmarker}{}%
\end{pgfscope}%
\begin{pgfscope}%
\pgfsys@transformshift{5.305914in}{0.382839in}%
\pgfsys@useobject{currentmarker}{}%
\end{pgfscope}%
\begin{pgfscope}%
\pgfsys@transformshift{2.380374in}{3.188931in}%
\pgfsys@useobject{currentmarker}{}%
\end{pgfscope}%
\begin{pgfscope}%
\pgfsys@transformshift{3.284572in}{2.062961in}%
\pgfsys@useobject{currentmarker}{}%
\end{pgfscope}%
\begin{pgfscope}%
\pgfsys@transformshift{2.272251in}{1.067829in}%
\pgfsys@useobject{currentmarker}{}%
\end{pgfscope}%
\begin{pgfscope}%
\pgfsys@transformshift{4.718164in}{1.593474in}%
\pgfsys@useobject{currentmarker}{}%
\end{pgfscope}%
\begin{pgfscope}%
\pgfsys@transformshift{5.232452in}{0.671322in}%
\pgfsys@useobject{currentmarker}{}%
\end{pgfscope}%
\begin{pgfscope}%
\pgfsys@transformshift{3.708529in}{2.571506in}%
\pgfsys@useobject{currentmarker}{}%
\end{pgfscope}%
\begin{pgfscope}%
\pgfsys@transformshift{2.441423in}{0.434533in}%
\pgfsys@useobject{currentmarker}{}%
\end{pgfscope}%
\begin{pgfscope}%
\pgfsys@transformshift{2.666442in}{2.321983in}%
\pgfsys@useobject{currentmarker}{}%
\end{pgfscope}%
\begin{pgfscope}%
\pgfsys@transformshift{0.524964in}{2.622710in}%
\pgfsys@useobject{currentmarker}{}%
\end{pgfscope}%
\begin{pgfscope}%
\pgfsys@transformshift{0.504311in}{1.955502in}%
\pgfsys@useobject{currentmarker}{}%
\end{pgfscope}%
\begin{pgfscope}%
\pgfsys@transformshift{4.257437in}{2.177836in}%
\pgfsys@useobject{currentmarker}{}%
\end{pgfscope}%
\begin{pgfscope}%
\pgfsys@transformshift{3.506256in}{0.434983in}%
\pgfsys@useobject{currentmarker}{}%
\end{pgfscope}%
\begin{pgfscope}%
\pgfsys@transformshift{4.393451in}{0.399120in}%
\pgfsys@useobject{currentmarker}{}%
\end{pgfscope}%
\begin{pgfscope}%
\pgfsys@transformshift{1.043252in}{1.883824in}%
\pgfsys@useobject{currentmarker}{}%
\end{pgfscope}%
\begin{pgfscope}%
\pgfsys@transformshift{5.547400in}{3.300712in}%
\pgfsys@useobject{currentmarker}{}%
\end{pgfscope}%
\begin{pgfscope}%
\pgfsys@transformshift{4.860361in}{1.022995in}%
\pgfsys@useobject{currentmarker}{}%
\end{pgfscope}%
\begin{pgfscope}%
\pgfsys@transformshift{1.228914in}{0.627692in}%
\pgfsys@useobject{currentmarker}{}%
\end{pgfscope}%
\begin{pgfscope}%
\pgfsys@transformshift{1.734014in}{3.102096in}%
\pgfsys@useobject{currentmarker}{}%
\end{pgfscope}%
\begin{pgfscope}%
\pgfsys@transformshift{0.966647in}{1.143033in}%
\pgfsys@useobject{currentmarker}{}%
\end{pgfscope}%
\begin{pgfscope}%
\pgfsys@transformshift{4.301114in}{0.835160in}%
\pgfsys@useobject{currentmarker}{}%
\end{pgfscope}%
\begin{pgfscope}%
\pgfsys@transformshift{3.972505in}{3.878057in}%
\pgfsys@useobject{currentmarker}{}%
\end{pgfscope}%
\begin{pgfscope}%
\pgfsys@transformshift{1.932954in}{2.196396in}%
\pgfsys@useobject{currentmarker}{}%
\end{pgfscope}%
\begin{pgfscope}%
\pgfsys@transformshift{2.749328in}{1.823186in}%
\pgfsys@useobject{currentmarker}{}%
\end{pgfscope}%
\begin{pgfscope}%
\pgfsys@transformshift{1.977788in}{2.676603in}%
\pgfsys@useobject{currentmarker}{}%
\end{pgfscope}%
\begin{pgfscope}%
\pgfsys@transformshift{2.146244in}{1.681164in}%
\pgfsys@useobject{currentmarker}{}%
\end{pgfscope}%
\begin{pgfscope}%
\pgfsys@transformshift{4.110883in}{3.764361in}%
\pgfsys@useobject{currentmarker}{}%
\end{pgfscope}%
\begin{pgfscope}%
\pgfsys@transformshift{1.821021in}{2.465353in}%
\pgfsys@useobject{currentmarker}{}%
\end{pgfscope}%
\begin{pgfscope}%
\pgfsys@transformshift{2.569110in}{0.522829in}%
\pgfsys@useobject{currentmarker}{}%
\end{pgfscope}%
\begin{pgfscope}%
\pgfsys@transformshift{2.584146in}{0.534749in}%
\pgfsys@useobject{currentmarker}{}%
\end{pgfscope}%
\begin{pgfscope}%
\pgfsys@transformshift{4.859776in}{1.601725in}%
\pgfsys@useobject{currentmarker}{}%
\end{pgfscope}%
\begin{pgfscope}%
\pgfsys@transformshift{1.303379in}{4.224896in}%
\pgfsys@useobject{currentmarker}{}%
\end{pgfscope}%
\begin{pgfscope}%
\pgfsys@transformshift{2.176232in}{3.464792in}%
\pgfsys@useobject{currentmarker}{}%
\end{pgfscope}%
\begin{pgfscope}%
\pgfsys@transformshift{3.892157in}{3.321169in}%
\pgfsys@useobject{currentmarker}{}%
\end{pgfscope}%
\begin{pgfscope}%
\pgfsys@transformshift{4.214250in}{3.217745in}%
\pgfsys@useobject{currentmarker}{}%
\end{pgfscope}%
\begin{pgfscope}%
\pgfsys@transformshift{0.465998in}{3.314330in}%
\pgfsys@useobject{currentmarker}{}%
\end{pgfscope}%
\begin{pgfscope}%
\pgfsys@transformshift{5.112173in}{4.135726in}%
\pgfsys@useobject{currentmarker}{}%
\end{pgfscope}%
\begin{pgfscope}%
\pgfsys@transformshift{5.699252in}{1.734392in}%
\pgfsys@useobject{currentmarker}{}%
\end{pgfscope}%
\begin{pgfscope}%
\pgfsys@transformshift{5.692551in}{3.349513in}%
\pgfsys@useobject{currentmarker}{}%
\end{pgfscope}%
\begin{pgfscope}%
\pgfsys@transformshift{4.250278in}{1.287412in}%
\pgfsys@useobject{currentmarker}{}%
\end{pgfscope}%
\begin{pgfscope}%
\pgfsys@transformshift{5.378444in}{0.905049in}%
\pgfsys@useobject{currentmarker}{}%
\end{pgfscope}%
\begin{pgfscope}%
\pgfsys@transformshift{2.558871in}{3.564858in}%
\pgfsys@useobject{currentmarker}{}%
\end{pgfscope}%
\begin{pgfscope}%
\pgfsys@transformshift{4.591936in}{0.485439in}%
\pgfsys@useobject{currentmarker}{}%
\end{pgfscope}%
\begin{pgfscope}%
\pgfsys@transformshift{3.096833in}{1.775207in}%
\pgfsys@useobject{currentmarker}{}%
\end{pgfscope}%
\begin{pgfscope}%
\pgfsys@transformshift{3.280505in}{2.984124in}%
\pgfsys@useobject{currentmarker}{}%
\end{pgfscope}%
\begin{pgfscope}%
\pgfsys@transformshift{1.972894in}{2.128720in}%
\pgfsys@useobject{currentmarker}{}%
\end{pgfscope}%
\begin{pgfscope}%
\pgfsys@transformshift{5.752437in}{1.387857in}%
\pgfsys@useobject{currentmarker}{}%
\end{pgfscope}%
\begin{pgfscope}%
\pgfsys@transformshift{4.521251in}{1.460792in}%
\pgfsys@useobject{currentmarker}{}%
\end{pgfscope}%
\begin{pgfscope}%
\pgfsys@transformshift{2.216257in}{1.463436in}%
\pgfsys@useobject{currentmarker}{}%
\end{pgfscope}%
\begin{pgfscope}%
\pgfsys@transformshift{5.587585in}{3.390857in}%
\pgfsys@useobject{currentmarker}{}%
\end{pgfscope}%
\begin{pgfscope}%
\pgfsys@transformshift{1.572276in}{1.650022in}%
\pgfsys@useobject{currentmarker}{}%
\end{pgfscope}%
\begin{pgfscope}%
\pgfsys@transformshift{1.247688in}{3.170362in}%
\pgfsys@useobject{currentmarker}{}%
\end{pgfscope}%
\begin{pgfscope}%
\pgfsys@transformshift{1.181984in}{1.201082in}%
\pgfsys@useobject{currentmarker}{}%
\end{pgfscope}%
\begin{pgfscope}%
\pgfsys@transformshift{0.888518in}{1.789203in}%
\pgfsys@useobject{currentmarker}{}%
\end{pgfscope}%
\begin{pgfscope}%
\pgfsys@transformshift{2.088585in}{0.356508in}%
\pgfsys@useobject{currentmarker}{}%
\end{pgfscope}%
\begin{pgfscope}%
\pgfsys@transformshift{1.549607in}{0.866879in}%
\pgfsys@useobject{currentmarker}{}%
\end{pgfscope}%
\begin{pgfscope}%
\pgfsys@transformshift{3.707994in}{2.969004in}%
\pgfsys@useobject{currentmarker}{}%
\end{pgfscope}%
\begin{pgfscope}%
\pgfsys@transformshift{4.889318in}{2.340027in}%
\pgfsys@useobject{currentmarker}{}%
\end{pgfscope}%
\begin{pgfscope}%
\pgfsys@transformshift{3.797553in}{1.980371in}%
\pgfsys@useobject{currentmarker}{}%
\end{pgfscope}%
\begin{pgfscope}%
\pgfsys@transformshift{3.912538in}{2.552028in}%
\pgfsys@useobject{currentmarker}{}%
\end{pgfscope}%
\begin{pgfscope}%
\pgfsys@transformshift{3.747014in}{3.535864in}%
\pgfsys@useobject{currentmarker}{}%
\end{pgfscope}%
\begin{pgfscope}%
\pgfsys@transformshift{4.134506in}{1.579195in}%
\pgfsys@useobject{currentmarker}{}%
\end{pgfscope}%
\begin{pgfscope}%
\pgfsys@transformshift{5.441179in}{3.875041in}%
\pgfsys@useobject{currentmarker}{}%
\end{pgfscope}%
\begin{pgfscope}%
\pgfsys@transformshift{1.467072in}{4.052332in}%
\pgfsys@useobject{currentmarker}{}%
\end{pgfscope}%
\begin{pgfscope}%
\pgfsys@transformshift{5.043001in}{3.848888in}%
\pgfsys@useobject{currentmarker}{}%
\end{pgfscope}%
\begin{pgfscope}%
\pgfsys@transformshift{5.593249in}{2.384397in}%
\pgfsys@useobject{currentmarker}{}%
\end{pgfscope}%
\begin{pgfscope}%
\pgfsys@transformshift{5.164951in}{3.889027in}%
\pgfsys@useobject{currentmarker}{}%
\end{pgfscope}%
\begin{pgfscope}%
\pgfsys@transformshift{4.246950in}{2.895155in}%
\pgfsys@useobject{currentmarker}{}%
\end{pgfscope}%
\begin{pgfscope}%
\pgfsys@transformshift{1.639579in}{2.286216in}%
\pgfsys@useobject{currentmarker}{}%
\end{pgfscope}%
\begin{pgfscope}%
\pgfsys@transformshift{4.058722in}{3.098120in}%
\pgfsys@useobject{currentmarker}{}%
\end{pgfscope}%
\begin{pgfscope}%
\pgfsys@transformshift{5.777080in}{4.130026in}%
\pgfsys@useobject{currentmarker}{}%
\end{pgfscope}%
\begin{pgfscope}%
\pgfsys@transformshift{5.658795in}{0.346429in}%
\pgfsys@useobject{currentmarker}{}%
\end{pgfscope}%
\begin{pgfscope}%
\pgfsys@transformshift{4.740507in}{3.244823in}%
\pgfsys@useobject{currentmarker}{}%
\end{pgfscope}%
\begin{pgfscope}%
\pgfsys@transformshift{3.034436in}{2.574779in}%
\pgfsys@useobject{currentmarker}{}%
\end{pgfscope}%
\begin{pgfscope}%
\pgfsys@transformshift{4.715617in}{3.181302in}%
\pgfsys@useobject{currentmarker}{}%
\end{pgfscope}%
\begin{pgfscope}%
\pgfsys@transformshift{5.729322in}{3.908780in}%
\pgfsys@useobject{currentmarker}{}%
\end{pgfscope}%
\begin{pgfscope}%
\pgfsys@transformshift{5.324339in}{4.109192in}%
\pgfsys@useobject{currentmarker}{}%
\end{pgfscope}%
\begin{pgfscope}%
\pgfsys@transformshift{5.736926in}{1.089218in}%
\pgfsys@useobject{currentmarker}{}%
\end{pgfscope}%
\begin{pgfscope}%
\pgfsys@transformshift{4.550353in}{1.702903in}%
\pgfsys@useobject{currentmarker}{}%
\end{pgfscope}%
\begin{pgfscope}%
\pgfsys@transformshift{1.775735in}{4.119033in}%
\pgfsys@useobject{currentmarker}{}%
\end{pgfscope}%
\begin{pgfscope}%
\pgfsys@transformshift{0.959891in}{0.701704in}%
\pgfsys@useobject{currentmarker}{}%
\end{pgfscope}%
\begin{pgfscope}%
\pgfsys@transformshift{4.292485in}{1.816380in}%
\pgfsys@useobject{currentmarker}{}%
\end{pgfscope}%
\begin{pgfscope}%
\pgfsys@transformshift{0.601398in}{1.751256in}%
\pgfsys@useobject{currentmarker}{}%
\end{pgfscope}%
\begin{pgfscope}%
\pgfsys@transformshift{2.756147in}{0.494163in}%
\pgfsys@useobject{currentmarker}{}%
\end{pgfscope}%
\begin{pgfscope}%
\pgfsys@transformshift{5.537625in}{1.839254in}%
\pgfsys@useobject{currentmarker}{}%
\end{pgfscope}%
\begin{pgfscope}%
\pgfsys@transformshift{2.394444in}{2.089902in}%
\pgfsys@useobject{currentmarker}{}%
\end{pgfscope}%
\begin{pgfscope}%
\pgfsys@transformshift{5.436274in}{1.228612in}%
\pgfsys@useobject{currentmarker}{}%
\end{pgfscope}%
\begin{pgfscope}%
\pgfsys@transformshift{2.321712in}{4.101235in}%
\pgfsys@useobject{currentmarker}{}%
\end{pgfscope}%
\begin{pgfscope}%
\pgfsys@transformshift{3.975888in}{0.404004in}%
\pgfsys@useobject{currentmarker}{}%
\end{pgfscope}%
\begin{pgfscope}%
\pgfsys@transformshift{4.063438in}{1.974921in}%
\pgfsys@useobject{currentmarker}{}%
\end{pgfscope}%
\begin{pgfscope}%
\pgfsys@transformshift{3.008617in}{3.829068in}%
\pgfsys@useobject{currentmarker}{}%
\end{pgfscope}%
\begin{pgfscope}%
\pgfsys@transformshift{2.474442in}{3.074055in}%
\pgfsys@useobject{currentmarker}{}%
\end{pgfscope}%
\begin{pgfscope}%
\pgfsys@transformshift{0.681177in}{2.174811in}%
\pgfsys@useobject{currentmarker}{}%
\end{pgfscope}%
\begin{pgfscope}%
\pgfsys@transformshift{3.565041in}{0.761039in}%
\pgfsys@useobject{currentmarker}{}%
\end{pgfscope}%
\begin{pgfscope}%
\pgfsys@transformshift{0.841968in}{1.990752in}%
\pgfsys@useobject{currentmarker}{}%
\end{pgfscope}%
\begin{pgfscope}%
\pgfsys@transformshift{4.911347in}{2.650595in}%
\pgfsys@useobject{currentmarker}{}%
\end{pgfscope}%
\begin{pgfscope}%
\pgfsys@transformshift{4.447431in}{0.676235in}%
\pgfsys@useobject{currentmarker}{}%
\end{pgfscope}%
\begin{pgfscope}%
\pgfsys@transformshift{4.499716in}{1.296223in}%
\pgfsys@useobject{currentmarker}{}%
\end{pgfscope}%
\begin{pgfscope}%
\pgfsys@transformshift{2.246709in}{1.183475in}%
\pgfsys@useobject{currentmarker}{}%
\end{pgfscope}%
\begin{pgfscope}%
\pgfsys@transformshift{4.181646in}{1.187177in}%
\pgfsys@useobject{currentmarker}{}%
\end{pgfscope}%
\begin{pgfscope}%
\pgfsys@transformshift{1.163792in}{0.503257in}%
\pgfsys@useobject{currentmarker}{}%
\end{pgfscope}%
\begin{pgfscope}%
\pgfsys@transformshift{4.454810in}{3.270245in}%
\pgfsys@useobject{currentmarker}{}%
\end{pgfscope}%
\begin{pgfscope}%
\pgfsys@transformshift{2.016379in}{2.855137in}%
\pgfsys@useobject{currentmarker}{}%
\end{pgfscope}%
\begin{pgfscope}%
\pgfsys@transformshift{5.159910in}{0.345899in}%
\pgfsys@useobject{currentmarker}{}%
\end{pgfscope}%
\begin{pgfscope}%
\pgfsys@transformshift{4.519305in}{2.870109in}%
\pgfsys@useobject{currentmarker}{}%
\end{pgfscope}%
\begin{pgfscope}%
\pgfsys@transformshift{1.761943in}{3.516990in}%
\pgfsys@useobject{currentmarker}{}%
\end{pgfscope}%
\begin{pgfscope}%
\pgfsys@transformshift{1.574730in}{2.537533in}%
\pgfsys@useobject{currentmarker}{}%
\end{pgfscope}%
\begin{pgfscope}%
\pgfsys@transformshift{0.971208in}{0.646632in}%
\pgfsys@useobject{currentmarker}{}%
\end{pgfscope}%
\begin{pgfscope}%
\pgfsys@transformshift{4.658214in}{3.671527in}%
\pgfsys@useobject{currentmarker}{}%
\end{pgfscope}%
\begin{pgfscope}%
\pgfsys@transformshift{2.874870in}{3.626965in}%
\pgfsys@useobject{currentmarker}{}%
\end{pgfscope}%
\begin{pgfscope}%
\pgfsys@transformshift{4.716752in}{4.187181in}%
\pgfsys@useobject{currentmarker}{}%
\end{pgfscope}%
\begin{pgfscope}%
\pgfsys@transformshift{2.335015in}{3.967161in}%
\pgfsys@useobject{currentmarker}{}%
\end{pgfscope}%
\begin{pgfscope}%
\pgfsys@transformshift{4.358752in}{2.312909in}%
\pgfsys@useobject{currentmarker}{}%
\end{pgfscope}%
\begin{pgfscope}%
\pgfsys@transformshift{1.579368in}{2.183812in}%
\pgfsys@useobject{currentmarker}{}%
\end{pgfscope}%
\begin{pgfscope}%
\pgfsys@transformshift{3.233227in}{0.791381in}%
\pgfsys@useobject{currentmarker}{}%
\end{pgfscope}%
\begin{pgfscope}%
\pgfsys@transformshift{0.676676in}{0.891497in}%
\pgfsys@useobject{currentmarker}{}%
\end{pgfscope}%
\begin{pgfscope}%
\pgfsys@transformshift{5.351021in}{2.653189in}%
\pgfsys@useobject{currentmarker}{}%
\end{pgfscope}%
\begin{pgfscope}%
\pgfsys@transformshift{3.090681in}{0.591486in}%
\pgfsys@useobject{currentmarker}{}%
\end{pgfscope}%
\begin{pgfscope}%
\pgfsys@transformshift{3.260604in}{0.881507in}%
\pgfsys@useobject{currentmarker}{}%
\end{pgfscope}%
\begin{pgfscope}%
\pgfsys@transformshift{2.065067in}{1.366338in}%
\pgfsys@useobject{currentmarker}{}%
\end{pgfscope}%
\begin{pgfscope}%
\pgfsys@transformshift{4.766803in}{3.555415in}%
\pgfsys@useobject{currentmarker}{}%
\end{pgfscope}%
\begin{pgfscope}%
\pgfsys@transformshift{3.440914in}{0.943759in}%
\pgfsys@useobject{currentmarker}{}%
\end{pgfscope}%
\begin{pgfscope}%
\pgfsys@transformshift{0.678117in}{2.764919in}%
\pgfsys@useobject{currentmarker}{}%
\end{pgfscope}%
\begin{pgfscope}%
\pgfsys@transformshift{4.995401in}{1.407325in}%
\pgfsys@useobject{currentmarker}{}%
\end{pgfscope}%
\begin{pgfscope}%
\pgfsys@transformshift{0.467253in}{2.500401in}%
\pgfsys@useobject{currentmarker}{}%
\end{pgfscope}%
\begin{pgfscope}%
\pgfsys@transformshift{4.799276in}{0.710987in}%
\pgfsys@useobject{currentmarker}{}%
\end{pgfscope}%
\begin{pgfscope}%
\pgfsys@transformshift{0.536071in}{1.416008in}%
\pgfsys@useobject{currentmarker}{}%
\end{pgfscope}%
\begin{pgfscope}%
\pgfsys@transformshift{1.001812in}{2.403430in}%
\pgfsys@useobject{currentmarker}{}%
\end{pgfscope}%
\begin{pgfscope}%
\pgfsys@transformshift{2.116544in}{3.937067in}%
\pgfsys@useobject{currentmarker}{}%
\end{pgfscope}%
\begin{pgfscope}%
\pgfsys@transformshift{4.151750in}{0.427153in}%
\pgfsys@useobject{currentmarker}{}%
\end{pgfscope}%
\begin{pgfscope}%
\pgfsys@transformshift{4.816480in}{3.675243in}%
\pgfsys@useobject{currentmarker}{}%
\end{pgfscope}%
\begin{pgfscope}%
\pgfsys@transformshift{1.641233in}{3.606028in}%
\pgfsys@useobject{currentmarker}{}%
\end{pgfscope}%
\begin{pgfscope}%
\pgfsys@transformshift{0.790857in}{2.428016in}%
\pgfsys@useobject{currentmarker}{}%
\end{pgfscope}%
\begin{pgfscope}%
\pgfsys@transformshift{1.213514in}{0.591917in}%
\pgfsys@useobject{currentmarker}{}%
\end{pgfscope}%
\begin{pgfscope}%
\pgfsys@transformshift{1.341020in}{2.690101in}%
\pgfsys@useobject{currentmarker}{}%
\end{pgfscope}%
\begin{pgfscope}%
\pgfsys@transformshift{0.978816in}{1.892683in}%
\pgfsys@useobject{currentmarker}{}%
\end{pgfscope}%
\begin{pgfscope}%
\pgfsys@transformshift{3.696684in}{2.773071in}%
\pgfsys@useobject{currentmarker}{}%
\end{pgfscope}%
\begin{pgfscope}%
\pgfsys@transformshift{4.323093in}{2.602214in}%
\pgfsys@useobject{currentmarker}{}%
\end{pgfscope}%
\begin{pgfscope}%
\pgfsys@transformshift{1.353420in}{2.143968in}%
\pgfsys@useobject{currentmarker}{}%
\end{pgfscope}%
\begin{pgfscope}%
\pgfsys@transformshift{0.639994in}{2.509086in}%
\pgfsys@useobject{currentmarker}{}%
\end{pgfscope}%
\begin{pgfscope}%
\pgfsys@transformshift{2.595879in}{3.895440in}%
\pgfsys@useobject{currentmarker}{}%
\end{pgfscope}%
\begin{pgfscope}%
\pgfsys@transformshift{3.163376in}{2.743041in}%
\pgfsys@useobject{currentmarker}{}%
\end{pgfscope}%
\begin{pgfscope}%
\pgfsys@transformshift{0.886457in}{3.165935in}%
\pgfsys@useobject{currentmarker}{}%
\end{pgfscope}%
\begin{pgfscope}%
\pgfsys@transformshift{2.883450in}{1.750322in}%
\pgfsys@useobject{currentmarker}{}%
\end{pgfscope}%
\begin{pgfscope}%
\pgfsys@transformshift{4.334342in}{3.500529in}%
\pgfsys@useobject{currentmarker}{}%
\end{pgfscope}%
\begin{pgfscope}%
\pgfsys@transformshift{2.804211in}{2.202644in}%
\pgfsys@useobject{currentmarker}{}%
\end{pgfscope}%
\begin{pgfscope}%
\pgfsys@transformshift{0.703792in}{0.839140in}%
\pgfsys@useobject{currentmarker}{}%
\end{pgfscope}%
\begin{pgfscope}%
\pgfsys@transformshift{2.140841in}{1.480598in}%
\pgfsys@useobject{currentmarker}{}%
\end{pgfscope}%
\begin{pgfscope}%
\pgfsys@transformshift{0.692591in}{3.120808in}%
\pgfsys@useobject{currentmarker}{}%
\end{pgfscope}%
\begin{pgfscope}%
\pgfsys@transformshift{4.970798in}{1.715109in}%
\pgfsys@useobject{currentmarker}{}%
\end{pgfscope}%
\begin{pgfscope}%
\pgfsys@transformshift{2.100909in}{3.439004in}%
\pgfsys@useobject{currentmarker}{}%
\end{pgfscope}%
\begin{pgfscope}%
\pgfsys@transformshift{5.444622in}{0.847484in}%
\pgfsys@useobject{currentmarker}{}%
\end{pgfscope}%
\begin{pgfscope}%
\pgfsys@transformshift{4.452265in}{2.596600in}%
\pgfsys@useobject{currentmarker}{}%
\end{pgfscope}%
\begin{pgfscope}%
\pgfsys@transformshift{2.138652in}{0.503969in}%
\pgfsys@useobject{currentmarker}{}%
\end{pgfscope}%
\begin{pgfscope}%
\pgfsys@transformshift{5.750050in}{3.078366in}%
\pgfsys@useobject{currentmarker}{}%
\end{pgfscope}%
\begin{pgfscope}%
\pgfsys@transformshift{1.383666in}{3.944826in}%
\pgfsys@useobject{currentmarker}{}%
\end{pgfscope}%
\begin{pgfscope}%
\pgfsys@transformshift{2.030939in}{2.021051in}%
\pgfsys@useobject{currentmarker}{}%
\end{pgfscope}%
\begin{pgfscope}%
\pgfsys@transformshift{4.585577in}{3.466532in}%
\pgfsys@useobject{currentmarker}{}%
\end{pgfscope}%
\begin{pgfscope}%
\pgfsys@transformshift{5.799634in}{4.040575in}%
\pgfsys@useobject{currentmarker}{}%
\end{pgfscope}%
\begin{pgfscope}%
\pgfsys@transformshift{2.918346in}{2.724707in}%
\pgfsys@useobject{currentmarker}{}%
\end{pgfscope}%
\begin{pgfscope}%
\pgfsys@transformshift{5.156710in}{1.511969in}%
\pgfsys@useobject{currentmarker}{}%
\end{pgfscope}%
\begin{pgfscope}%
\pgfsys@transformshift{1.346130in}{1.440205in}%
\pgfsys@useobject{currentmarker}{}%
\end{pgfscope}%
\begin{pgfscope}%
\pgfsys@transformshift{5.198910in}{0.767366in}%
\pgfsys@useobject{currentmarker}{}%
\end{pgfscope}%
\begin{pgfscope}%
\pgfsys@transformshift{3.992025in}{1.357829in}%
\pgfsys@useobject{currentmarker}{}%
\end{pgfscope}%
\begin{pgfscope}%
\pgfsys@transformshift{2.985341in}{3.363757in}%
\pgfsys@useobject{currentmarker}{}%
\end{pgfscope}%
\begin{pgfscope}%
\pgfsys@transformshift{2.587757in}{4.228981in}%
\pgfsys@useobject{currentmarker}{}%
\end{pgfscope}%
\begin{pgfscope}%
\pgfsys@transformshift{0.615135in}{2.665799in}%
\pgfsys@useobject{currentmarker}{}%
\end{pgfscope}%
\begin{pgfscope}%
\pgfsys@transformshift{4.076382in}{3.707752in}%
\pgfsys@useobject{currentmarker}{}%
\end{pgfscope}%
\begin{pgfscope}%
\pgfsys@transformshift{3.611168in}{1.452230in}%
\pgfsys@useobject{currentmarker}{}%
\end{pgfscope}%
\begin{pgfscope}%
\pgfsys@transformshift{2.512250in}{2.130784in}%
\pgfsys@useobject{currentmarker}{}%
\end{pgfscope}%
\begin{pgfscope}%
\pgfsys@transformshift{5.457745in}{0.745106in}%
\pgfsys@useobject{currentmarker}{}%
\end{pgfscope}%
\begin{pgfscope}%
\pgfsys@transformshift{2.985372in}{4.135669in}%
\pgfsys@useobject{currentmarker}{}%
\end{pgfscope}%
\begin{pgfscope}%
\pgfsys@transformshift{2.386217in}{1.733253in}%
\pgfsys@useobject{currentmarker}{}%
\end{pgfscope}%
\begin{pgfscope}%
\pgfsys@transformshift{3.310593in}{3.091598in}%
\pgfsys@useobject{currentmarker}{}%
\end{pgfscope}%
\begin{pgfscope}%
\pgfsys@transformshift{2.408424in}{3.994899in}%
\pgfsys@useobject{currentmarker}{}%
\end{pgfscope}%
\begin{pgfscope}%
\pgfsys@transformshift{3.899756in}{2.287518in}%
\pgfsys@useobject{currentmarker}{}%
\end{pgfscope}%
\begin{pgfscope}%
\pgfsys@transformshift{2.631366in}{2.795459in}%
\pgfsys@useobject{currentmarker}{}%
\end{pgfscope}%
\begin{pgfscope}%
\pgfsys@transformshift{2.040212in}{0.593821in}%
\pgfsys@useobject{currentmarker}{}%
\end{pgfscope}%
\begin{pgfscope}%
\pgfsys@transformshift{3.608835in}{3.838131in}%
\pgfsys@useobject{currentmarker}{}%
\end{pgfscope}%
\begin{pgfscope}%
\pgfsys@transformshift{1.916025in}{3.831006in}%
\pgfsys@useobject{currentmarker}{}%
\end{pgfscope}%
\begin{pgfscope}%
\pgfsys@transformshift{2.815839in}{0.567582in}%
\pgfsys@useobject{currentmarker}{}%
\end{pgfscope}%
\begin{pgfscope}%
\pgfsys@transformshift{1.725253in}{0.395118in}%
\pgfsys@useobject{currentmarker}{}%
\end{pgfscope}%
\begin{pgfscope}%
\pgfsys@transformshift{1.125341in}{4.178363in}%
\pgfsys@useobject{currentmarker}{}%
\end{pgfscope}%
\begin{pgfscope}%
\pgfsys@transformshift{3.639040in}{0.857567in}%
\pgfsys@useobject{currentmarker}{}%
\end{pgfscope}%
\begin{pgfscope}%
\pgfsys@transformshift{3.938182in}{3.502621in}%
\pgfsys@useobject{currentmarker}{}%
\end{pgfscope}%
\begin{pgfscope}%
\pgfsys@transformshift{4.205449in}{1.305540in}%
\pgfsys@useobject{currentmarker}{}%
\end{pgfscope}%
\begin{pgfscope}%
\pgfsys@transformshift{4.270832in}{3.947347in}%
\pgfsys@useobject{currentmarker}{}%
\end{pgfscope}%
\begin{pgfscope}%
\pgfsys@transformshift{1.278593in}{2.819361in}%
\pgfsys@useobject{currentmarker}{}%
\end{pgfscope}%
\begin{pgfscope}%
\pgfsys@transformshift{1.536695in}{0.379762in}%
\pgfsys@useobject{currentmarker}{}%
\end{pgfscope}%
\begin{pgfscope}%
\pgfsys@transformshift{3.361363in}{2.685227in}%
\pgfsys@useobject{currentmarker}{}%
\end{pgfscope}%
\begin{pgfscope}%
\pgfsys@transformshift{5.150778in}{0.490803in}%
\pgfsys@useobject{currentmarker}{}%
\end{pgfscope}%
\begin{pgfscope}%
\pgfsys@transformshift{0.979341in}{0.427760in}%
\pgfsys@useobject{currentmarker}{}%
\end{pgfscope}%
\begin{pgfscope}%
\pgfsys@transformshift{4.817026in}{1.619400in}%
\pgfsys@useobject{currentmarker}{}%
\end{pgfscope}%
\begin{pgfscope}%
\pgfsys@transformshift{5.233312in}{1.281616in}%
\pgfsys@useobject{currentmarker}{}%
\end{pgfscope}%
\begin{pgfscope}%
\pgfsys@transformshift{4.188024in}{2.408851in}%
\pgfsys@useobject{currentmarker}{}%
\end{pgfscope}%
\begin{pgfscope}%
\pgfsys@transformshift{0.542636in}{1.163869in}%
\pgfsys@useobject{currentmarker}{}%
\end{pgfscope}%
\begin{pgfscope}%
\pgfsys@transformshift{5.203559in}{1.223569in}%
\pgfsys@useobject{currentmarker}{}%
\end{pgfscope}%
\begin{pgfscope}%
\pgfsys@transformshift{2.953687in}{1.431423in}%
\pgfsys@useobject{currentmarker}{}%
\end{pgfscope}%
\begin{pgfscope}%
\pgfsys@transformshift{4.149409in}{2.352528in}%
\pgfsys@useobject{currentmarker}{}%
\end{pgfscope}%
\begin{pgfscope}%
\pgfsys@transformshift{1.049859in}{3.008717in}%
\pgfsys@useobject{currentmarker}{}%
\end{pgfscope}%
\begin{pgfscope}%
\pgfsys@transformshift{1.782561in}{3.831882in}%
\pgfsys@useobject{currentmarker}{}%
\end{pgfscope}%
\begin{pgfscope}%
\pgfsys@transformshift{1.617421in}{0.336639in}%
\pgfsys@useobject{currentmarker}{}%
\end{pgfscope}%
\begin{pgfscope}%
\pgfsys@transformshift{5.076161in}{0.395608in}%
\pgfsys@useobject{currentmarker}{}%
\end{pgfscope}%
\begin{pgfscope}%
\pgfsys@transformshift{5.058944in}{0.538055in}%
\pgfsys@useobject{currentmarker}{}%
\end{pgfscope}%
\begin{pgfscope}%
\pgfsys@transformshift{3.880357in}{2.513768in}%
\pgfsys@useobject{currentmarker}{}%
\end{pgfscope}%
\begin{pgfscope}%
\pgfsys@transformshift{0.629353in}{3.337434in}%
\pgfsys@useobject{currentmarker}{}%
\end{pgfscope}%
\begin{pgfscope}%
\pgfsys@transformshift{3.894421in}{2.729255in}%
\pgfsys@useobject{currentmarker}{}%
\end{pgfscope}%
\begin{pgfscope}%
\pgfsys@transformshift{4.769646in}{1.946875in}%
\pgfsys@useobject{currentmarker}{}%
\end{pgfscope}%
\begin{pgfscope}%
\pgfsys@transformshift{1.359491in}{3.451266in}%
\pgfsys@useobject{currentmarker}{}%
\end{pgfscope}%
\begin{pgfscope}%
\pgfsys@transformshift{1.755580in}{1.269812in}%
\pgfsys@useobject{currentmarker}{}%
\end{pgfscope}%
\begin{pgfscope}%
\pgfsys@transformshift{1.556303in}{0.868117in}%
\pgfsys@useobject{currentmarker}{}%
\end{pgfscope}%
\begin{pgfscope}%
\pgfsys@transformshift{4.331322in}{3.727364in}%
\pgfsys@useobject{currentmarker}{}%
\end{pgfscope}%
\begin{pgfscope}%
\pgfsys@transformshift{3.809797in}{3.362251in}%
\pgfsys@useobject{currentmarker}{}%
\end{pgfscope}%
\begin{pgfscope}%
\pgfsys@transformshift{5.269070in}{2.976650in}%
\pgfsys@useobject{currentmarker}{}%
\end{pgfscope}%
\begin{pgfscope}%
\pgfsys@transformshift{5.510156in}{0.437519in}%
\pgfsys@useobject{currentmarker}{}%
\end{pgfscope}%
\begin{pgfscope}%
\pgfsys@transformshift{1.768917in}{2.072943in}%
\pgfsys@useobject{currentmarker}{}%
\end{pgfscope}%
\begin{pgfscope}%
\pgfsys@transformshift{2.632875in}{1.385279in}%
\pgfsys@useobject{currentmarker}{}%
\end{pgfscope}%
\begin{pgfscope}%
\pgfsys@transformshift{5.298658in}{2.664385in}%
\pgfsys@useobject{currentmarker}{}%
\end{pgfscope}%
\begin{pgfscope}%
\pgfsys@transformshift{1.253062in}{2.830059in}%
\pgfsys@useobject{currentmarker}{}%
\end{pgfscope}%
\begin{pgfscope}%
\pgfsys@transformshift{2.150807in}{2.665403in}%
\pgfsys@useobject{currentmarker}{}%
\end{pgfscope}%
\begin{pgfscope}%
\pgfsys@transformshift{2.835507in}{0.365703in}%
\pgfsys@useobject{currentmarker}{}%
\end{pgfscope}%
\begin{pgfscope}%
\pgfsys@transformshift{4.456205in}{3.083337in}%
\pgfsys@useobject{currentmarker}{}%
\end{pgfscope}%
\begin{pgfscope}%
\pgfsys@transformshift{0.698361in}{1.001330in}%
\pgfsys@useobject{currentmarker}{}%
\end{pgfscope}%
\begin{pgfscope}%
\pgfsys@transformshift{4.369765in}{3.709612in}%
\pgfsys@useobject{currentmarker}{}%
\end{pgfscope}%
\begin{pgfscope}%
\pgfsys@transformshift{3.039149in}{3.718855in}%
\pgfsys@useobject{currentmarker}{}%
\end{pgfscope}%
\begin{pgfscope}%
\pgfsys@transformshift{1.234487in}{2.877743in}%
\pgfsys@useobject{currentmarker}{}%
\end{pgfscope}%
\begin{pgfscope}%
\pgfsys@transformshift{3.416861in}{1.547241in}%
\pgfsys@useobject{currentmarker}{}%
\end{pgfscope}%
\begin{pgfscope}%
\pgfsys@transformshift{4.503306in}{0.423507in}%
\pgfsys@useobject{currentmarker}{}%
\end{pgfscope}%
\begin{pgfscope}%
\pgfsys@transformshift{1.253209in}{1.304136in}%
\pgfsys@useobject{currentmarker}{}%
\end{pgfscope}%
\begin{pgfscope}%
\pgfsys@transformshift{3.655962in}{1.164463in}%
\pgfsys@useobject{currentmarker}{}%
\end{pgfscope}%
\begin{pgfscope}%
\pgfsys@transformshift{3.422245in}{3.093577in}%
\pgfsys@useobject{currentmarker}{}%
\end{pgfscope}%
\begin{pgfscope}%
\pgfsys@transformshift{1.802741in}{4.111788in}%
\pgfsys@useobject{currentmarker}{}%
\end{pgfscope}%
\begin{pgfscope}%
\pgfsys@transformshift{0.742259in}{0.368028in}%
\pgfsys@useobject{currentmarker}{}%
\end{pgfscope}%
\begin{pgfscope}%
\pgfsys@transformshift{3.206258in}{2.892818in}%
\pgfsys@useobject{currentmarker}{}%
\end{pgfscope}%
\begin{pgfscope}%
\pgfsys@transformshift{4.818618in}{0.922323in}%
\pgfsys@useobject{currentmarker}{}%
\end{pgfscope}%
\begin{pgfscope}%
\pgfsys@transformshift{5.688669in}{3.708288in}%
\pgfsys@useobject{currentmarker}{}%
\end{pgfscope}%
\begin{pgfscope}%
\pgfsys@transformshift{1.121797in}{2.292443in}%
\pgfsys@useobject{currentmarker}{}%
\end{pgfscope}%
\begin{pgfscope}%
\pgfsys@transformshift{1.303978in}{2.627521in}%
\pgfsys@useobject{currentmarker}{}%
\end{pgfscope}%
\begin{pgfscope}%
\pgfsys@transformshift{5.332706in}{2.938131in}%
\pgfsys@useobject{currentmarker}{}%
\end{pgfscope}%
\begin{pgfscope}%
\pgfsys@transformshift{2.037655in}{3.186804in}%
\pgfsys@useobject{currentmarker}{}%
\end{pgfscope}%
\begin{pgfscope}%
\pgfsys@transformshift{0.906273in}{0.523305in}%
\pgfsys@useobject{currentmarker}{}%
\end{pgfscope}%
\begin{pgfscope}%
\pgfsys@transformshift{5.002346in}{3.268046in}%
\pgfsys@useobject{currentmarker}{}%
\end{pgfscope}%
\begin{pgfscope}%
\pgfsys@transformshift{1.160222in}{3.789073in}%
\pgfsys@useobject{currentmarker}{}%
\end{pgfscope}%
\begin{pgfscope}%
\pgfsys@transformshift{1.451683in}{1.280218in}%
\pgfsys@useobject{currentmarker}{}%
\end{pgfscope}%
\begin{pgfscope}%
\pgfsys@transformshift{3.109287in}{2.114230in}%
\pgfsys@useobject{currentmarker}{}%
\end{pgfscope}%
\begin{pgfscope}%
\pgfsys@transformshift{1.637514in}{2.766197in}%
\pgfsys@useobject{currentmarker}{}%
\end{pgfscope}%
\begin{pgfscope}%
\pgfsys@transformshift{0.580904in}{2.014458in}%
\pgfsys@useobject{currentmarker}{}%
\end{pgfscope}%
\begin{pgfscope}%
\pgfsys@transformshift{3.357764in}{2.784654in}%
\pgfsys@useobject{currentmarker}{}%
\end{pgfscope}%
\begin{pgfscope}%
\pgfsys@transformshift{4.687915in}{2.029150in}%
\pgfsys@useobject{currentmarker}{}%
\end{pgfscope}%
\begin{pgfscope}%
\pgfsys@transformshift{2.193576in}{1.097342in}%
\pgfsys@useobject{currentmarker}{}%
\end{pgfscope}%
\begin{pgfscope}%
\pgfsys@transformshift{5.278299in}{2.498502in}%
\pgfsys@useobject{currentmarker}{}%
\end{pgfscope}%
\begin{pgfscope}%
\pgfsys@transformshift{2.192899in}{3.870138in}%
\pgfsys@useobject{currentmarker}{}%
\end{pgfscope}%
\begin{pgfscope}%
\pgfsys@transformshift{5.366907in}{3.172824in}%
\pgfsys@useobject{currentmarker}{}%
\end{pgfscope}%
\begin{pgfscope}%
\pgfsys@transformshift{4.381209in}{3.163932in}%
\pgfsys@useobject{currentmarker}{}%
\end{pgfscope}%
\begin{pgfscope}%
\pgfsys@transformshift{1.035456in}{0.343512in}%
\pgfsys@useobject{currentmarker}{}%
\end{pgfscope}%
\begin{pgfscope}%
\pgfsys@transformshift{4.478319in}{4.182706in}%
\pgfsys@useobject{currentmarker}{}%
\end{pgfscope}%
\begin{pgfscope}%
\pgfsys@transformshift{1.535914in}{3.834485in}%
\pgfsys@useobject{currentmarker}{}%
\end{pgfscope}%
\begin{pgfscope}%
\pgfsys@transformshift{5.746889in}{2.207447in}%
\pgfsys@useobject{currentmarker}{}%
\end{pgfscope}%
\begin{pgfscope}%
\pgfsys@transformshift{3.918587in}{3.948563in}%
\pgfsys@useobject{currentmarker}{}%
\end{pgfscope}%
\begin{pgfscope}%
\pgfsys@transformshift{4.411328in}{0.655212in}%
\pgfsys@useobject{currentmarker}{}%
\end{pgfscope}%
\begin{pgfscope}%
\pgfsys@transformshift{3.549292in}{1.101880in}%
\pgfsys@useobject{currentmarker}{}%
\end{pgfscope}%
\begin{pgfscope}%
\pgfsys@transformshift{4.518741in}{3.085773in}%
\pgfsys@useobject{currentmarker}{}%
\end{pgfscope}%
\begin{pgfscope}%
\pgfsys@transformshift{1.714424in}{0.802652in}%
\pgfsys@useobject{currentmarker}{}%
\end{pgfscope}%
\begin{pgfscope}%
\pgfsys@transformshift{1.196896in}{4.000536in}%
\pgfsys@useobject{currentmarker}{}%
\end{pgfscope}%
\begin{pgfscope}%
\pgfsys@transformshift{2.826327in}{3.168567in}%
\pgfsys@useobject{currentmarker}{}%
\end{pgfscope}%
\begin{pgfscope}%
\pgfsys@transformshift{4.234848in}{2.108576in}%
\pgfsys@useobject{currentmarker}{}%
\end{pgfscope}%
\begin{pgfscope}%
\pgfsys@transformshift{1.018515in}{4.170865in}%
\pgfsys@useobject{currentmarker}{}%
\end{pgfscope}%
\begin{pgfscope}%
\pgfsys@transformshift{4.860022in}{1.180043in}%
\pgfsys@useobject{currentmarker}{}%
\end{pgfscope}%
\begin{pgfscope}%
\pgfsys@transformshift{4.932645in}{3.207491in}%
\pgfsys@useobject{currentmarker}{}%
\end{pgfscope}%
\begin{pgfscope}%
\pgfsys@transformshift{4.276415in}{1.014436in}%
\pgfsys@useobject{currentmarker}{}%
\end{pgfscope}%
\begin{pgfscope}%
\pgfsys@transformshift{1.571130in}{3.501124in}%
\pgfsys@useobject{currentmarker}{}%
\end{pgfscope}%
\begin{pgfscope}%
\pgfsys@transformshift{5.509583in}{1.561374in}%
\pgfsys@useobject{currentmarker}{}%
\end{pgfscope}%
\begin{pgfscope}%
\pgfsys@transformshift{0.936189in}{1.676187in}%
\pgfsys@useobject{currentmarker}{}%
\end{pgfscope}%
\begin{pgfscope}%
\pgfsys@transformshift{3.754551in}{2.164481in}%
\pgfsys@useobject{currentmarker}{}%
\end{pgfscope}%
\begin{pgfscope}%
\pgfsys@transformshift{4.116647in}{3.774789in}%
\pgfsys@useobject{currentmarker}{}%
\end{pgfscope}%
\begin{pgfscope}%
\pgfsys@transformshift{2.552951in}{1.184659in}%
\pgfsys@useobject{currentmarker}{}%
\end{pgfscope}%
\begin{pgfscope}%
\pgfsys@transformshift{2.314646in}{3.860419in}%
\pgfsys@useobject{currentmarker}{}%
\end{pgfscope}%
\begin{pgfscope}%
\pgfsys@transformshift{1.205248in}{1.571260in}%
\pgfsys@useobject{currentmarker}{}%
\end{pgfscope}%
\begin{pgfscope}%
\pgfsys@transformshift{1.763458in}{0.896839in}%
\pgfsys@useobject{currentmarker}{}%
\end{pgfscope}%
\begin{pgfscope}%
\pgfsys@transformshift{0.539196in}{1.684254in}%
\pgfsys@useobject{currentmarker}{}%
\end{pgfscope}%
\begin{pgfscope}%
\pgfsys@transformshift{1.510033in}{2.554603in}%
\pgfsys@useobject{currentmarker}{}%
\end{pgfscope}%
\begin{pgfscope}%
\pgfsys@transformshift{2.073254in}{0.351643in}%
\pgfsys@useobject{currentmarker}{}%
\end{pgfscope}%
\begin{pgfscope}%
\pgfsys@transformshift{5.057300in}{3.300458in}%
\pgfsys@useobject{currentmarker}{}%
\end{pgfscope}%
\begin{pgfscope}%
\pgfsys@transformshift{2.701124in}{4.009236in}%
\pgfsys@useobject{currentmarker}{}%
\end{pgfscope}%
\begin{pgfscope}%
\pgfsys@transformshift{4.049373in}{2.309350in}%
\pgfsys@useobject{currentmarker}{}%
\end{pgfscope}%
\begin{pgfscope}%
\pgfsys@transformshift{4.502526in}{3.177584in}%
\pgfsys@useobject{currentmarker}{}%
\end{pgfscope}%
\begin{pgfscope}%
\pgfsys@transformshift{1.878142in}{2.992832in}%
\pgfsys@useobject{currentmarker}{}%
\end{pgfscope}%
\begin{pgfscope}%
\pgfsys@transformshift{5.087363in}{2.799333in}%
\pgfsys@useobject{currentmarker}{}%
\end{pgfscope}%
\begin{pgfscope}%
\pgfsys@transformshift{3.190183in}{4.172737in}%
\pgfsys@useobject{currentmarker}{}%
\end{pgfscope}%
\begin{pgfscope}%
\pgfsys@transformshift{5.307620in}{2.348725in}%
\pgfsys@useobject{currentmarker}{}%
\end{pgfscope}%
\begin{pgfscope}%
\pgfsys@transformshift{1.253366in}{0.970778in}%
\pgfsys@useobject{currentmarker}{}%
\end{pgfscope}%
\begin{pgfscope}%
\pgfsys@transformshift{5.463458in}{3.030739in}%
\pgfsys@useobject{currentmarker}{}%
\end{pgfscope}%
\begin{pgfscope}%
\pgfsys@transformshift{2.175948in}{3.530922in}%
\pgfsys@useobject{currentmarker}{}%
\end{pgfscope}%
\begin{pgfscope}%
\pgfsys@transformshift{0.726258in}{2.632393in}%
\pgfsys@useobject{currentmarker}{}%
\end{pgfscope}%
\begin{pgfscope}%
\pgfsys@transformshift{1.032956in}{2.758331in}%
\pgfsys@useobject{currentmarker}{}%
\end{pgfscope}%
\begin{pgfscope}%
\pgfsys@transformshift{2.429601in}{2.749098in}%
\pgfsys@useobject{currentmarker}{}%
\end{pgfscope}%
\begin{pgfscope}%
\pgfsys@transformshift{5.121530in}{2.737189in}%
\pgfsys@useobject{currentmarker}{}%
\end{pgfscope}%
\begin{pgfscope}%
\pgfsys@transformshift{1.433460in}{3.057071in}%
\pgfsys@useobject{currentmarker}{}%
\end{pgfscope}%
\begin{pgfscope}%
\pgfsys@transformshift{1.171236in}{3.244620in}%
\pgfsys@useobject{currentmarker}{}%
\end{pgfscope}%
\begin{pgfscope}%
\pgfsys@transformshift{1.543594in}{0.334328in}%
\pgfsys@useobject{currentmarker}{}%
\end{pgfscope}%
\begin{pgfscope}%
\pgfsys@transformshift{5.015652in}{0.590004in}%
\pgfsys@useobject{currentmarker}{}%
\end{pgfscope}%
\begin{pgfscope}%
\pgfsys@transformshift{1.633984in}{1.009443in}%
\pgfsys@useobject{currentmarker}{}%
\end{pgfscope}%
\begin{pgfscope}%
\pgfsys@transformshift{5.621842in}{0.483159in}%
\pgfsys@useobject{currentmarker}{}%
\end{pgfscope}%
\begin{pgfscope}%
\pgfsys@transformshift{2.972869in}{0.521455in}%
\pgfsys@useobject{currentmarker}{}%
\end{pgfscope}%
\begin{pgfscope}%
\pgfsys@transformshift{1.289367in}{1.787430in}%
\pgfsys@useobject{currentmarker}{}%
\end{pgfscope}%
\begin{pgfscope}%
\pgfsys@transformshift{0.971103in}{3.169156in}%
\pgfsys@useobject{currentmarker}{}%
\end{pgfscope}%
\begin{pgfscope}%
\pgfsys@transformshift{2.920999in}{1.597634in}%
\pgfsys@useobject{currentmarker}{}%
\end{pgfscope}%
\begin{pgfscope}%
\pgfsys@transformshift{1.022380in}{0.325040in}%
\pgfsys@useobject{currentmarker}{}%
\end{pgfscope}%
\begin{pgfscope}%
\pgfsys@transformshift{2.769483in}{0.479894in}%
\pgfsys@useobject{currentmarker}{}%
\end{pgfscope}%
\begin{pgfscope}%
\pgfsys@transformshift{4.477035in}{2.961837in}%
\pgfsys@useobject{currentmarker}{}%
\end{pgfscope}%
\begin{pgfscope}%
\pgfsys@transformshift{2.214094in}{1.434397in}%
\pgfsys@useobject{currentmarker}{}%
\end{pgfscope}%
\begin{pgfscope}%
\pgfsys@transformshift{3.028145in}{2.666114in}%
\pgfsys@useobject{currentmarker}{}%
\end{pgfscope}%
\begin{pgfscope}%
\pgfsys@transformshift{3.287433in}{2.158792in}%
\pgfsys@useobject{currentmarker}{}%
\end{pgfscope}%
\begin{pgfscope}%
\pgfsys@transformshift{1.406165in}{1.625087in}%
\pgfsys@useobject{currentmarker}{}%
\end{pgfscope}%
\begin{pgfscope}%
\pgfsys@transformshift{0.715347in}{2.277329in}%
\pgfsys@useobject{currentmarker}{}%
\end{pgfscope}%
\begin{pgfscope}%
\pgfsys@transformshift{0.505380in}{3.588644in}%
\pgfsys@useobject{currentmarker}{}%
\end{pgfscope}%
\begin{pgfscope}%
\pgfsys@transformshift{2.870689in}{0.810515in}%
\pgfsys@useobject{currentmarker}{}%
\end{pgfscope}%
\begin{pgfscope}%
\pgfsys@transformshift{2.342608in}{3.903892in}%
\pgfsys@useobject{currentmarker}{}%
\end{pgfscope}%
\begin{pgfscope}%
\pgfsys@transformshift{1.027327in}{2.630033in}%
\pgfsys@useobject{currentmarker}{}%
\end{pgfscope}%
\begin{pgfscope}%
\pgfsys@transformshift{2.221339in}{1.417976in}%
\pgfsys@useobject{currentmarker}{}%
\end{pgfscope}%
\begin{pgfscope}%
\pgfsys@transformshift{4.189023in}{1.454524in}%
\pgfsys@useobject{currentmarker}{}%
\end{pgfscope}%
\begin{pgfscope}%
\pgfsys@transformshift{2.719049in}{1.280831in}%
\pgfsys@useobject{currentmarker}{}%
\end{pgfscope}%
\begin{pgfscope}%
\pgfsys@transformshift{1.705006in}{1.633896in}%
\pgfsys@useobject{currentmarker}{}%
\end{pgfscope}%
\begin{pgfscope}%
\pgfsys@transformshift{1.174429in}{1.272588in}%
\pgfsys@useobject{currentmarker}{}%
\end{pgfscope}%
\begin{pgfscope}%
\pgfsys@transformshift{0.486237in}{1.216534in}%
\pgfsys@useobject{currentmarker}{}%
\end{pgfscope}%
\begin{pgfscope}%
\pgfsys@transformshift{1.146775in}{3.121838in}%
\pgfsys@useobject{currentmarker}{}%
\end{pgfscope}%
\begin{pgfscope}%
\pgfsys@transformshift{2.119785in}{1.748366in}%
\pgfsys@useobject{currentmarker}{}%
\end{pgfscope}%
\begin{pgfscope}%
\pgfsys@transformshift{4.722670in}{3.514630in}%
\pgfsys@useobject{currentmarker}{}%
\end{pgfscope}%
\begin{pgfscope}%
\pgfsys@transformshift{5.670493in}{3.779569in}%
\pgfsys@useobject{currentmarker}{}%
\end{pgfscope}%
\begin{pgfscope}%
\pgfsys@transformshift{3.155316in}{2.496925in}%
\pgfsys@useobject{currentmarker}{}%
\end{pgfscope}%
\begin{pgfscope}%
\pgfsys@transformshift{3.042541in}{2.454975in}%
\pgfsys@useobject{currentmarker}{}%
\end{pgfscope}%
\begin{pgfscope}%
\pgfsys@transformshift{1.991470in}{0.690103in}%
\pgfsys@useobject{currentmarker}{}%
\end{pgfscope}%
\begin{pgfscope}%
\pgfsys@transformshift{0.737460in}{3.102185in}%
\pgfsys@useobject{currentmarker}{}%
\end{pgfscope}%
\begin{pgfscope}%
\pgfsys@transformshift{1.721569in}{3.153864in}%
\pgfsys@useobject{currentmarker}{}%
\end{pgfscope}%
\begin{pgfscope}%
\pgfsys@transformshift{1.170863in}{2.543106in}%
\pgfsys@useobject{currentmarker}{}%
\end{pgfscope}%
\begin{pgfscope}%
\pgfsys@transformshift{5.528655in}{3.411819in}%
\pgfsys@useobject{currentmarker}{}%
\end{pgfscope}%
\begin{pgfscope}%
\pgfsys@transformshift{3.314048in}{1.103392in}%
\pgfsys@useobject{currentmarker}{}%
\end{pgfscope}%
\begin{pgfscope}%
\pgfsys@transformshift{1.135437in}{0.521558in}%
\pgfsys@useobject{currentmarker}{}%
\end{pgfscope}%
\begin{pgfscope}%
\pgfsys@transformshift{1.339342in}{0.834982in}%
\pgfsys@useobject{currentmarker}{}%
\end{pgfscope}%
\begin{pgfscope}%
\pgfsys@transformshift{2.975121in}{2.909411in}%
\pgfsys@useobject{currentmarker}{}%
\end{pgfscope}%
\begin{pgfscope}%
\pgfsys@transformshift{3.519180in}{0.794737in}%
\pgfsys@useobject{currentmarker}{}%
\end{pgfscope}%
\begin{pgfscope}%
\pgfsys@transformshift{0.810499in}{1.836044in}%
\pgfsys@useobject{currentmarker}{}%
\end{pgfscope}%
\begin{pgfscope}%
\pgfsys@transformshift{5.533007in}{0.995345in}%
\pgfsys@useobject{currentmarker}{}%
\end{pgfscope}%
\begin{pgfscope}%
\pgfsys@transformshift{0.751395in}{1.074410in}%
\pgfsys@useobject{currentmarker}{}%
\end{pgfscope}%
\begin{pgfscope}%
\pgfsys@transformshift{2.885328in}{1.498036in}%
\pgfsys@useobject{currentmarker}{}%
\end{pgfscope}%
\begin{pgfscope}%
\pgfsys@transformshift{4.153670in}{2.291416in}%
\pgfsys@useobject{currentmarker}{}%
\end{pgfscope}%
\begin{pgfscope}%
\pgfsys@transformshift{5.623129in}{3.490808in}%
\pgfsys@useobject{currentmarker}{}%
\end{pgfscope}%
\begin{pgfscope}%
\pgfsys@transformshift{1.433974in}{3.767218in}%
\pgfsys@useobject{currentmarker}{}%
\end{pgfscope}%
\begin{pgfscope}%
\pgfsys@transformshift{3.546926in}{2.384692in}%
\pgfsys@useobject{currentmarker}{}%
\end{pgfscope}%
\begin{pgfscope}%
\pgfsys@transformshift{3.149650in}{3.096355in}%
\pgfsys@useobject{currentmarker}{}%
\end{pgfscope}%
\begin{pgfscope}%
\pgfsys@transformshift{3.633475in}{3.145151in}%
\pgfsys@useobject{currentmarker}{}%
\end{pgfscope}%
\begin{pgfscope}%
\pgfsys@transformshift{1.173602in}{3.223026in}%
\pgfsys@useobject{currentmarker}{}%
\end{pgfscope}%
\begin{pgfscope}%
\pgfsys@transformshift{1.974954in}{2.374613in}%
\pgfsys@useobject{currentmarker}{}%
\end{pgfscope}%
\begin{pgfscope}%
\pgfsys@transformshift{2.515995in}{0.938660in}%
\pgfsys@useobject{currentmarker}{}%
\end{pgfscope}%
\begin{pgfscope}%
\pgfsys@transformshift{4.452879in}{4.017088in}%
\pgfsys@useobject{currentmarker}{}%
\end{pgfscope}%
\begin{pgfscope}%
\pgfsys@transformshift{1.150830in}{2.064666in}%
\pgfsys@useobject{currentmarker}{}%
\end{pgfscope}%
\begin{pgfscope}%
\pgfsys@transformshift{0.647393in}{1.356189in}%
\pgfsys@useobject{currentmarker}{}%
\end{pgfscope}%
\begin{pgfscope}%
\pgfsys@transformshift{0.782986in}{2.070229in}%
\pgfsys@useobject{currentmarker}{}%
\end{pgfscope}%
\end{pgfscope}%
\begin{pgfscope}%
\pgfpathrectangle{\pgfqpoint{0.882794in}{0.589583in}}{\pgfqpoint{6.200000in}{4.620000in}}%
\pgfusepath{clip}%
\pgfsetbuttcap%
\pgfsetroundjoin%
\definecolor{currentfill}{rgb}{0.839216,0.152941,0.156863}%
\pgfsetfillcolor{currentfill}%
\pgfsetlinewidth{1.003750pt}%
\definecolor{currentstroke}{rgb}{0.839216,0.152941,0.156863}%
\pgfsetstrokecolor{currentstroke}%
\pgfsetdash{}{0pt}%
\pgfsys@defobject{currentmarker}{\pgfqpoint{-0.031056in}{-0.031056in}}{\pgfqpoint{0.031056in}{0.031056in}}{%
\pgfpathmoveto{\pgfqpoint{0.000000in}{-0.031056in}}%
\pgfpathcurveto{\pgfqpoint{0.008236in}{-0.031056in}}{\pgfqpoint{0.016136in}{-0.027784in}}{\pgfqpoint{0.021960in}{-0.021960in}}%
\pgfpathcurveto{\pgfqpoint{0.027784in}{-0.016136in}}{\pgfqpoint{0.031056in}{-0.008236in}}{\pgfqpoint{0.031056in}{0.000000in}}%
\pgfpathcurveto{\pgfqpoint{0.031056in}{0.008236in}}{\pgfqpoint{0.027784in}{0.016136in}}{\pgfqpoint{0.021960in}{0.021960in}}%
\pgfpathcurveto{\pgfqpoint{0.016136in}{0.027784in}}{\pgfqpoint{0.008236in}{0.031056in}}{\pgfqpoint{0.000000in}{0.031056in}}%
\pgfpathcurveto{\pgfqpoint{-0.008236in}{0.031056in}}{\pgfqpoint{-0.016136in}{0.027784in}}{\pgfqpoint{-0.021960in}{0.021960in}}%
\pgfpathcurveto{\pgfqpoint{-0.027784in}{0.016136in}}{\pgfqpoint{-0.031056in}{0.008236in}}{\pgfqpoint{-0.031056in}{0.000000in}}%
\pgfpathcurveto{\pgfqpoint{-0.031056in}{-0.008236in}}{\pgfqpoint{-0.027784in}{-0.016136in}}{\pgfqpoint{-0.021960in}{-0.021960in}}%
\pgfpathcurveto{\pgfqpoint{-0.016136in}{-0.027784in}}{\pgfqpoint{-0.008236in}{-0.031056in}}{\pgfqpoint{0.000000in}{-0.031056in}}%
\pgfpathlineto{\pgfqpoint{0.000000in}{-0.031056in}}%
\pgfpathclose%
\pgfusepath{stroke,fill}%
}%
\begin{pgfscope}%
\pgfsys@transformshift{0.943999in}{3.591336in}%
\pgfsys@useobject{currentmarker}{}%
\end{pgfscope}%
\begin{pgfscope}%
\pgfsys@transformshift{2.956831in}{1.055493in}%
\pgfsys@useobject{currentmarker}{}%
\end{pgfscope}%
\begin{pgfscope}%
\pgfsys@transformshift{1.545362in}{2.123893in}%
\pgfsys@useobject{currentmarker}{}%
\end{pgfscope}%
\begin{pgfscope}%
\pgfsys@transformshift{1.079856in}{3.500371in}%
\pgfsys@useobject{currentmarker}{}%
\end{pgfscope}%
\begin{pgfscope}%
\pgfsys@transformshift{0.954978in}{3.355893in}%
\pgfsys@useobject{currentmarker}{}%
\end{pgfscope}%
\begin{pgfscope}%
\pgfsys@transformshift{1.842360in}{1.702844in}%
\pgfsys@useobject{currentmarker}{}%
\end{pgfscope}%
\begin{pgfscope}%
\pgfsys@transformshift{1.386584in}{1.916995in}%
\pgfsys@useobject{currentmarker}{}%
\end{pgfscope}%
\begin{pgfscope}%
\pgfsys@transformshift{3.368599in}{0.902214in}%
\pgfsys@useobject{currentmarker}{}%
\end{pgfscope}%
\begin{pgfscope}%
\pgfsys@transformshift{1.020770in}{3.589328in}%
\pgfsys@useobject{currentmarker}{}%
\end{pgfscope}%
\begin{pgfscope}%
\pgfsys@transformshift{1.422167in}{2.296890in}%
\pgfsys@useobject{currentmarker}{}%
\end{pgfscope}%
\begin{pgfscope}%
\pgfsys@transformshift{1.558736in}{1.870449in}%
\pgfsys@useobject{currentmarker}{}%
\end{pgfscope}%
\begin{pgfscope}%
\pgfsys@transformshift{1.849037in}{1.575144in}%
\pgfsys@useobject{currentmarker}{}%
\end{pgfscope}%
\begin{pgfscope}%
\pgfsys@transformshift{5.232452in}{0.671322in}%
\pgfsys@useobject{currentmarker}{}%
\end{pgfscope}%
\begin{pgfscope}%
\pgfsys@transformshift{4.301114in}{0.835160in}%
\pgfsys@useobject{currentmarker}{}%
\end{pgfscope}%
\begin{pgfscope}%
\pgfsys@transformshift{1.572276in}{1.650022in}%
\pgfsys@useobject{currentmarker}{}%
\end{pgfscope}%
\begin{pgfscope}%
\pgfsys@transformshift{4.447431in}{0.676235in}%
\pgfsys@useobject{currentmarker}{}%
\end{pgfscope}%
\begin{pgfscope}%
\pgfsys@transformshift{2.246709in}{1.183475in}%
\pgfsys@useobject{currentmarker}{}%
\end{pgfscope}%
\begin{pgfscope}%
\pgfsys@transformshift{3.260604in}{0.881507in}%
\pgfsys@useobject{currentmarker}{}%
\end{pgfscope}%
\begin{pgfscope}%
\pgfsys@transformshift{2.065067in}{1.366338in}%
\pgfsys@useobject{currentmarker}{}%
\end{pgfscope}%
\begin{pgfscope}%
\pgfsys@transformshift{3.440914in}{0.943759in}%
\pgfsys@useobject{currentmarker}{}%
\end{pgfscope}%
\begin{pgfscope}%
\pgfsys@transformshift{4.799276in}{0.710987in}%
\pgfsys@useobject{currentmarker}{}%
\end{pgfscope}%
\begin{pgfscope}%
\pgfsys@transformshift{1.353420in}{2.143968in}%
\pgfsys@useobject{currentmarker}{}%
\end{pgfscope}%
\begin{pgfscope}%
\pgfsys@transformshift{2.140841in}{1.480598in}%
\pgfsys@useobject{currentmarker}{}%
\end{pgfscope}%
\begin{pgfscope}%
\pgfsys@transformshift{3.639040in}{0.857567in}%
\pgfsys@useobject{currentmarker}{}%
\end{pgfscope}%
\begin{pgfscope}%
\pgfsys@transformshift{1.049859in}{3.008717in}%
\pgfsys@useobject{currentmarker}{}%
\end{pgfscope}%
\begin{pgfscope}%
\pgfsys@transformshift{1.234487in}{2.877743in}%
\pgfsys@useobject{currentmarker}{}%
\end{pgfscope}%
\begin{pgfscope}%
\pgfsys@transformshift{1.303978in}{2.627521in}%
\pgfsys@useobject{currentmarker}{}%
\end{pgfscope}%
\begin{pgfscope}%
\pgfsys@transformshift{2.552951in}{1.184659in}%
\pgfsys@useobject{currentmarker}{}%
\end{pgfscope}%
\begin{pgfscope}%
\pgfsys@transformshift{1.032956in}{2.758331in}%
\pgfsys@useobject{currentmarker}{}%
\end{pgfscope}%
\begin{pgfscope}%
\pgfsys@transformshift{0.971103in}{3.169156in}%
\pgfsys@useobject{currentmarker}{}%
\end{pgfscope}%
\begin{pgfscope}%
\pgfsys@transformshift{2.214094in}{1.434397in}%
\pgfsys@useobject{currentmarker}{}%
\end{pgfscope}%
\begin{pgfscope}%
\pgfsys@transformshift{2.221339in}{1.417976in}%
\pgfsys@useobject{currentmarker}{}%
\end{pgfscope}%
\begin{pgfscope}%
\pgfsys@transformshift{1.705006in}{1.633896in}%
\pgfsys@useobject{currentmarker}{}%
\end{pgfscope}%
\begin{pgfscope}%
\pgfsys@transformshift{1.146775in}{3.121838in}%
\pgfsys@useobject{currentmarker}{}%
\end{pgfscope}%
\begin{pgfscope}%
\pgfsys@transformshift{1.170863in}{2.543106in}%
\pgfsys@useobject{currentmarker}{}%
\end{pgfscope}%
\begin{pgfscope}%
\pgfsys@transformshift{3.519180in}{0.794737in}%
\pgfsys@useobject{currentmarker}{}%
\end{pgfscope}%
\end{pgfscope}%
\begin{pgfscope}%
\pgfpathrectangle{\pgfqpoint{0.882794in}{0.589583in}}{\pgfqpoint{6.200000in}{4.620000in}}%
\pgfusepath{clip}%
\pgfsetbuttcap%
\pgfsetmiterjoin%
\definecolor{currentfill}{rgb}{0.501961,0.501961,0.501961}%
\pgfsetfillcolor{currentfill}%
\pgfsetfillopacity{0.200000}%
\pgfsetlinewidth{1.003750pt}%
\definecolor{currentstroke}{rgb}{0.501961,0.501961,0.501961}%
\pgfsetstrokecolor{currentstroke}%
\pgfsetstrokeopacity{0.200000}%
\pgfsetdash{}{0pt}%
\pgfpathmoveto{\pgfqpoint{0.882794in}{3.577850in}}%
\pgfpathlineto{\pgfqpoint{0.887177in}{3.544929in}}%
\pgfpathlineto{\pgfqpoint{0.891560in}{3.512665in}}%
\pgfpathlineto{\pgfqpoint{0.895942in}{3.481040in}}%
\pgfpathlineto{\pgfqpoint{0.900325in}{3.450034in}}%
\pgfpathlineto{\pgfqpoint{0.904708in}{3.419629in}}%
\pgfpathlineto{\pgfqpoint{0.909090in}{3.389809in}}%
\pgfpathlineto{\pgfqpoint{0.913473in}{3.360555in}}%
\pgfpathlineto{\pgfqpoint{0.917856in}{3.331853in}}%
\pgfpathlineto{\pgfqpoint{0.922239in}{3.303687in}}%
\pgfpathlineto{\pgfqpoint{0.926621in}{3.276041in}}%
\pgfpathlineto{\pgfqpoint{0.931004in}{3.248902in}}%
\pgfpathlineto{\pgfqpoint{0.935387in}{3.222257in}}%
\pgfpathlineto{\pgfqpoint{0.939769in}{3.196090in}}%
\pgfpathlineto{\pgfqpoint{0.944152in}{3.170391in}}%
\pgfpathlineto{\pgfqpoint{0.948535in}{3.145145in}}%
\pgfpathlineto{\pgfqpoint{0.952917in}{3.120342in}}%
\pgfpathlineto{\pgfqpoint{0.957300in}{3.095970in}}%
\pgfpathlineto{\pgfqpoint{0.961683in}{3.072018in}}%
\pgfpathlineto{\pgfqpoint{0.966065in}{3.048475in}}%
\pgfpathlineto{\pgfqpoint{0.970448in}{3.025330in}}%
\pgfpathlineto{\pgfqpoint{0.974831in}{3.002574in}}%
\pgfpathlineto{\pgfqpoint{0.979213in}{2.980197in}}%
\pgfpathlineto{\pgfqpoint{0.983596in}{2.958189in}}%
\pgfpathlineto{\pgfqpoint{0.987979in}{2.936541in}}%
\pgfpathlineto{\pgfqpoint{0.992362in}{2.915246in}}%
\pgfpathlineto{\pgfqpoint{0.996744in}{2.894293in}}%
\pgfpathlineto{\pgfqpoint{1.001127in}{2.873675in}}%
\pgfpathlineto{\pgfqpoint{1.005510in}{2.853384in}}%
\pgfpathlineto{\pgfqpoint{1.009892in}{2.833413in}}%
\pgfpathlineto{\pgfqpoint{1.014275in}{2.813753in}}%
\pgfpathlineto{\pgfqpoint{1.018658in}{2.794397in}}%
\pgfpathlineto{\pgfqpoint{1.023040in}{2.775340in}}%
\pgfpathlineto{\pgfqpoint{1.027423in}{2.756572in}}%
\pgfpathlineto{\pgfqpoint{1.031806in}{2.738089in}}%
\pgfpathlineto{\pgfqpoint{1.036188in}{2.719884in}}%
\pgfpathlineto{\pgfqpoint{1.040571in}{2.701949in}}%
\pgfpathlineto{\pgfqpoint{1.044954in}{2.684281in}}%
\pgfpathlineto{\pgfqpoint{1.049336in}{2.666872in}}%
\pgfpathlineto{\pgfqpoint{1.053719in}{2.649717in}}%
\pgfpathlineto{\pgfqpoint{1.058102in}{2.632810in}}%
\pgfpathlineto{\pgfqpoint{1.062485in}{2.616146in}}%
\pgfpathlineto{\pgfqpoint{1.066867in}{2.599720in}}%
\pgfpathlineto{\pgfqpoint{1.071250in}{2.583527in}}%
\pgfpathlineto{\pgfqpoint{1.075633in}{2.567562in}}%
\pgfpathlineto{\pgfqpoint{1.080015in}{2.551820in}}%
\pgfpathlineto{\pgfqpoint{1.084398in}{2.536296in}}%
\pgfpathlineto{\pgfqpoint{1.088781in}{2.520986in}}%
\pgfpathlineto{\pgfqpoint{1.093163in}{2.505886in}}%
\pgfpathlineto{\pgfqpoint{1.097546in}{2.490991in}}%
\pgfpathlineto{\pgfqpoint{1.101929in}{2.476297in}}%
\pgfpathlineto{\pgfqpoint{1.106311in}{2.461801in}}%
\pgfpathlineto{\pgfqpoint{1.110694in}{2.447497in}}%
\pgfpathlineto{\pgfqpoint{1.115077in}{2.433382in}}%
\pgfpathlineto{\pgfqpoint{1.119459in}{2.419454in}}%
\pgfpathlineto{\pgfqpoint{1.123842in}{2.405707in}}%
\pgfpathlineto{\pgfqpoint{1.128225in}{2.392138in}}%
\pgfpathlineto{\pgfqpoint{1.132608in}{2.378745in}}%
\pgfpathlineto{\pgfqpoint{1.136990in}{2.365523in}}%
\pgfpathlineto{\pgfqpoint{1.141373in}{2.352469in}}%
\pgfpathlineto{\pgfqpoint{1.145756in}{2.339580in}}%
\pgfpathlineto{\pgfqpoint{1.150138in}{2.326853in}}%
\pgfpathlineto{\pgfqpoint{1.154521in}{2.314286in}}%
\pgfpathlineto{\pgfqpoint{1.158904in}{2.301874in}}%
\pgfpathlineto{\pgfqpoint{1.163286in}{2.289615in}}%
\pgfpathlineto{\pgfqpoint{1.167669in}{2.277507in}}%
\pgfpathlineto{\pgfqpoint{1.172052in}{2.265546in}}%
\pgfpathlineto{\pgfqpoint{1.176434in}{2.253730in}}%
\pgfpathlineto{\pgfqpoint{1.180817in}{2.242057in}}%
\pgfpathlineto{\pgfqpoint{1.185200in}{2.230523in}}%
\pgfpathlineto{\pgfqpoint{1.189582in}{2.219126in}}%
\pgfpathlineto{\pgfqpoint{1.193965in}{2.207864in}}%
\pgfpathlineto{\pgfqpoint{1.198348in}{2.196735in}}%
\pgfpathlineto{\pgfqpoint{1.202731in}{2.185735in}}%
\pgfpathlineto{\pgfqpoint{1.207113in}{2.174863in}}%
\pgfpathlineto{\pgfqpoint{1.211496in}{2.164117in}}%
\pgfpathlineto{\pgfqpoint{1.215879in}{2.153495in}}%
\pgfpathlineto{\pgfqpoint{1.220261in}{2.142993in}}%
\pgfpathlineto{\pgfqpoint{1.224644in}{2.132611in}}%
\pgfpathlineto{\pgfqpoint{1.229027in}{2.122346in}}%
\pgfpathlineto{\pgfqpoint{1.233409in}{2.112196in}}%
\pgfpathlineto{\pgfqpoint{1.237792in}{2.102160in}}%
\pgfpathlineto{\pgfqpoint{1.242175in}{2.092235in}}%
\pgfpathlineto{\pgfqpoint{1.246557in}{2.082420in}}%
\pgfpathlineto{\pgfqpoint{1.250940in}{2.072712in}}%
\pgfpathlineto{\pgfqpoint{1.255323in}{2.063111in}}%
\pgfpathlineto{\pgfqpoint{1.259705in}{2.053614in}}%
\pgfpathlineto{\pgfqpoint{1.264088in}{2.044219in}}%
\pgfpathlineto{\pgfqpoint{1.268471in}{2.034925in}}%
\pgfpathlineto{\pgfqpoint{1.272854in}{2.025731in}}%
\pgfpathlineto{\pgfqpoint{1.277236in}{2.016634in}}%
\pgfpathlineto{\pgfqpoint{1.281619in}{2.007634in}}%
\pgfpathlineto{\pgfqpoint{1.286002in}{1.998728in}}%
\pgfpathlineto{\pgfqpoint{1.290384in}{1.989916in}}%
\pgfpathlineto{\pgfqpoint{1.294767in}{1.981195in}}%
\pgfpathlineto{\pgfqpoint{1.299150in}{1.972564in}}%
\pgfpathlineto{\pgfqpoint{1.303532in}{1.964023in}}%
\pgfpathlineto{\pgfqpoint{1.307915in}{1.955569in}}%
\pgfpathlineto{\pgfqpoint{1.312298in}{1.947201in}}%
\pgfpathlineto{\pgfqpoint{1.316680in}{1.938918in}}%
\pgfpathlineto{\pgfqpoint{1.321063in}{1.930719in}}%
\pgfpathlineto{\pgfqpoint{1.325446in}{1.922602in}}%
\pgfpathlineto{\pgfqpoint{1.329829in}{1.914566in}}%
\pgfpathlineto{\pgfqpoint{1.334211in}{1.906610in}}%
\pgfpathlineto{\pgfqpoint{1.338594in}{1.898733in}}%
\pgfpathlineto{\pgfqpoint{1.342977in}{1.890933in}}%
\pgfpathlineto{\pgfqpoint{1.347359in}{1.883210in}}%
\pgfpathlineto{\pgfqpoint{1.351742in}{1.875562in}}%
\pgfpathlineto{\pgfqpoint{1.356125in}{1.867988in}}%
\pgfpathlineto{\pgfqpoint{1.360507in}{1.860487in}}%
\pgfpathlineto{\pgfqpoint{1.364890in}{1.853059in}}%
\pgfpathlineto{\pgfqpoint{1.369273in}{1.845701in}}%
\pgfpathlineto{\pgfqpoint{1.373655in}{1.838414in}}%
\pgfpathlineto{\pgfqpoint{1.378038in}{1.831195in}}%
\pgfpathlineto{\pgfqpoint{1.382421in}{1.824045in}}%
\pgfpathlineto{\pgfqpoint{1.386803in}{1.816962in}}%
\pgfpathlineto{\pgfqpoint{1.391186in}{1.809945in}}%
\pgfpathlineto{\pgfqpoint{1.395569in}{1.802993in}}%
\pgfpathlineto{\pgfqpoint{1.399952in}{1.796105in}}%
\pgfpathlineto{\pgfqpoint{1.404334in}{1.789281in}}%
\pgfpathlineto{\pgfqpoint{1.408717in}{1.782520in}}%
\pgfpathlineto{\pgfqpoint{1.413100in}{1.775820in}}%
\pgfpathlineto{\pgfqpoint{1.417482in}{1.769181in}}%
\pgfpathlineto{\pgfqpoint{1.421865in}{1.762602in}}%
\pgfpathlineto{\pgfqpoint{1.426248in}{1.756082in}}%
\pgfpathlineto{\pgfqpoint{1.430630in}{1.749621in}}%
\pgfpathlineto{\pgfqpoint{1.435013in}{1.743217in}}%
\pgfpathlineto{\pgfqpoint{1.439396in}{1.736871in}}%
\pgfpathlineto{\pgfqpoint{1.443778in}{1.730580in}}%
\pgfpathlineto{\pgfqpoint{1.448161in}{1.724345in}}%
\pgfpathlineto{\pgfqpoint{1.452544in}{1.718164in}}%
\pgfpathlineto{\pgfqpoint{1.456926in}{1.712038in}}%
\pgfpathlineto{\pgfqpoint{1.461309in}{1.705965in}}%
\pgfpathlineto{\pgfqpoint{1.465692in}{1.699944in}}%
\pgfpathlineto{\pgfqpoint{1.470075in}{1.693975in}}%
\pgfpathlineto{\pgfqpoint{1.474457in}{1.688057in}}%
\pgfpathlineto{\pgfqpoint{1.478840in}{1.682190in}}%
\pgfpathlineto{\pgfqpoint{1.483223in}{1.676373in}}%
\pgfpathlineto{\pgfqpoint{1.487605in}{1.670605in}}%
\pgfpathlineto{\pgfqpoint{1.491988in}{1.664886in}}%
\pgfpathlineto{\pgfqpoint{1.496371in}{1.659214in}}%
\pgfpathlineto{\pgfqpoint{1.500753in}{1.653590in}}%
\pgfpathlineto{\pgfqpoint{1.505136in}{1.648013in}}%
\pgfpathlineto{\pgfqpoint{1.509519in}{1.642483in}}%
\pgfpathlineto{\pgfqpoint{1.513901in}{1.636998in}}%
\pgfpathlineto{\pgfqpoint{1.518284in}{1.631558in}}%
\pgfpathlineto{\pgfqpoint{1.522667in}{1.626162in}}%
\pgfpathlineto{\pgfqpoint{1.527049in}{1.620811in}}%
\pgfpathlineto{\pgfqpoint{1.531432in}{1.615503in}}%
\pgfpathlineto{\pgfqpoint{1.535815in}{1.610238in}}%
\pgfpathlineto{\pgfqpoint{1.540198in}{1.605016in}}%
\pgfpathlineto{\pgfqpoint{1.544580in}{1.599835in}}%
\pgfpathlineto{\pgfqpoint{1.548963in}{1.594696in}}%
\pgfpathlineto{\pgfqpoint{1.553346in}{1.589598in}}%
\pgfpathlineto{\pgfqpoint{1.557728in}{1.584540in}}%
\pgfpathlineto{\pgfqpoint{1.562111in}{1.579522in}}%
\pgfpathlineto{\pgfqpoint{1.566494in}{1.574544in}}%
\pgfpathlineto{\pgfqpoint{1.570876in}{1.569605in}}%
\pgfpathlineto{\pgfqpoint{1.575259in}{1.564704in}}%
\pgfpathlineto{\pgfqpoint{1.579642in}{1.559841in}}%
\pgfpathlineto{\pgfqpoint{1.584024in}{1.555016in}}%
\pgfpathlineto{\pgfqpoint{1.588407in}{1.550229in}}%
\pgfpathlineto{\pgfqpoint{1.592790in}{1.545478in}}%
\pgfpathlineto{\pgfqpoint{1.597172in}{1.540763in}}%
\pgfpathlineto{\pgfqpoint{1.601555in}{1.536085in}}%
\pgfpathlineto{\pgfqpoint{1.605938in}{1.531442in}}%
\pgfpathlineto{\pgfqpoint{1.610321in}{1.526834in}}%
\pgfpathlineto{\pgfqpoint{1.614703in}{1.522261in}}%
\pgfpathlineto{\pgfqpoint{1.619086in}{1.517722in}}%
\pgfpathlineto{\pgfqpoint{1.623469in}{1.513218in}}%
\pgfpathlineto{\pgfqpoint{1.627851in}{1.508747in}}%
\pgfpathlineto{\pgfqpoint{1.632234in}{1.504309in}}%
\pgfpathlineto{\pgfqpoint{1.636617in}{1.499904in}}%
\pgfpathlineto{\pgfqpoint{1.640999in}{1.495531in}}%
\pgfpathlineto{\pgfqpoint{1.645382in}{1.491191in}}%
\pgfpathlineto{\pgfqpoint{1.649765in}{1.486882in}}%
\pgfpathlineto{\pgfqpoint{1.654147in}{1.482605in}}%
\pgfpathlineto{\pgfqpoint{1.658530in}{1.478359in}}%
\pgfpathlineto{\pgfqpoint{1.662913in}{1.474144in}}%
\pgfpathlineto{\pgfqpoint{1.667296in}{1.469959in}}%
\pgfpathlineto{\pgfqpoint{1.671678in}{1.465804in}}%
\pgfpathlineto{\pgfqpoint{1.676061in}{1.461679in}}%
\pgfpathlineto{\pgfqpoint{1.680444in}{1.457583in}}%
\pgfpathlineto{\pgfqpoint{1.684826in}{1.453517in}}%
\pgfpathlineto{\pgfqpoint{1.689209in}{1.449479in}}%
\pgfpathlineto{\pgfqpoint{1.693592in}{1.445470in}}%
\pgfpathlineto{\pgfqpoint{1.697974in}{1.441489in}}%
\pgfpathlineto{\pgfqpoint{1.702357in}{1.437537in}}%
\pgfpathlineto{\pgfqpoint{1.706740in}{1.433611in}}%
\pgfpathlineto{\pgfqpoint{1.711122in}{1.429713in}}%
\pgfpathlineto{\pgfqpoint{1.715505in}{1.425842in}}%
\pgfpathlineto{\pgfqpoint{1.719888in}{1.421998in}}%
\pgfpathlineto{\pgfqpoint{1.724270in}{1.418180in}}%
\pgfpathlineto{\pgfqpoint{1.728653in}{1.414389in}}%
\pgfpathlineto{\pgfqpoint{1.733036in}{1.410624in}}%
\pgfpathlineto{\pgfqpoint{1.737419in}{1.406884in}}%
\pgfpathlineto{\pgfqpoint{1.741801in}{1.403169in}}%
\pgfpathlineto{\pgfqpoint{1.746184in}{1.399480in}}%
\pgfpathlineto{\pgfqpoint{1.750567in}{1.395816in}}%
\pgfpathlineto{\pgfqpoint{1.754949in}{1.392176in}}%
\pgfpathlineto{\pgfqpoint{1.759332in}{1.388561in}}%
\pgfpathlineto{\pgfqpoint{1.763715in}{1.384970in}}%
\pgfpathlineto{\pgfqpoint{1.768097in}{1.381403in}}%
\pgfpathlineto{\pgfqpoint{1.772480in}{1.377860in}}%
\pgfpathlineto{\pgfqpoint{1.776863in}{1.374340in}}%
\pgfpathlineto{\pgfqpoint{1.781245in}{1.370843in}}%
\pgfpathlineto{\pgfqpoint{1.785628in}{1.367370in}}%
\pgfpathlineto{\pgfqpoint{1.790011in}{1.363919in}}%
\pgfpathlineto{\pgfqpoint{1.794393in}{1.360490in}}%
\pgfpathlineto{\pgfqpoint{1.798776in}{1.357084in}}%
\pgfpathlineto{\pgfqpoint{1.803159in}{1.353700in}}%
\pgfpathlineto{\pgfqpoint{1.807542in}{1.350338in}}%
\pgfpathlineto{\pgfqpoint{1.811924in}{1.346998in}}%
\pgfpathlineto{\pgfqpoint{1.816307in}{1.343679in}}%
\pgfpathlineto{\pgfqpoint{1.820690in}{1.340381in}}%
\pgfpathlineto{\pgfqpoint{1.825072in}{1.337105in}}%
\pgfpathlineto{\pgfqpoint{1.829455in}{1.333849in}}%
\pgfpathlineto{\pgfqpoint{1.833838in}{1.330614in}}%
\pgfpathlineto{\pgfqpoint{1.838220in}{1.327399in}}%
\pgfpathlineto{\pgfqpoint{1.842603in}{1.324205in}}%
\pgfpathlineto{\pgfqpoint{1.846986in}{1.321031in}}%
\pgfpathlineto{\pgfqpoint{1.851368in}{1.317876in}}%
\pgfpathlineto{\pgfqpoint{1.855751in}{1.314742in}}%
\pgfpathlineto{\pgfqpoint{1.860134in}{1.311627in}}%
\pgfpathlineto{\pgfqpoint{1.864516in}{1.308531in}}%
\pgfpathlineto{\pgfqpoint{1.868899in}{1.305454in}}%
\pgfpathlineto{\pgfqpoint{1.873282in}{1.302397in}}%
\pgfpathlineto{\pgfqpoint{1.877665in}{1.299358in}}%
\pgfpathlineto{\pgfqpoint{1.882047in}{1.296337in}}%
\pgfpathlineto{\pgfqpoint{1.886430in}{1.293336in}}%
\pgfpathlineto{\pgfqpoint{1.890813in}{1.290352in}}%
\pgfpathlineto{\pgfqpoint{1.895195in}{1.287387in}}%
\pgfpathlineto{\pgfqpoint{1.899578in}{1.284440in}}%
\pgfpathlineto{\pgfqpoint{1.903961in}{1.281510in}}%
\pgfpathlineto{\pgfqpoint{1.908343in}{1.278598in}}%
\pgfpathlineto{\pgfqpoint{1.912726in}{1.275704in}}%
\pgfpathlineto{\pgfqpoint{1.917109in}{1.272827in}}%
\pgfpathlineto{\pgfqpoint{1.921491in}{1.269967in}}%
\pgfpathlineto{\pgfqpoint{1.925874in}{1.267124in}}%
\pgfpathlineto{\pgfqpoint{1.930257in}{1.264298in}}%
\pgfpathlineto{\pgfqpoint{1.934639in}{1.261489in}}%
\pgfpathlineto{\pgfqpoint{1.939022in}{1.258696in}}%
\pgfpathlineto{\pgfqpoint{1.943405in}{1.255920in}}%
\pgfpathlineto{\pgfqpoint{1.947788in}{1.253160in}}%
\pgfpathlineto{\pgfqpoint{1.952170in}{1.250416in}}%
\pgfpathlineto{\pgfqpoint{1.956553in}{1.247688in}}%
\pgfpathlineto{\pgfqpoint{1.960936in}{1.244976in}}%
\pgfpathlineto{\pgfqpoint{1.965318in}{1.242279in}}%
\pgfpathlineto{\pgfqpoint{1.969701in}{1.239599in}}%
\pgfpathlineto{\pgfqpoint{1.974084in}{1.236933in}}%
\pgfpathlineto{\pgfqpoint{1.978466in}{1.234283in}}%
\pgfpathlineto{\pgfqpoint{1.982849in}{1.231648in}}%
\pgfpathlineto{\pgfqpoint{1.987232in}{1.229029in}}%
\pgfpathlineto{\pgfqpoint{1.991614in}{1.226424in}}%
\pgfpathlineto{\pgfqpoint{1.995997in}{1.223834in}}%
\pgfpathlineto{\pgfqpoint{2.000380in}{1.221258in}}%
\pgfpathlineto{\pgfqpoint{2.004763in}{1.218697in}}%
\pgfpathlineto{\pgfqpoint{2.009145in}{1.216151in}}%
\pgfpathlineto{\pgfqpoint{2.013528in}{1.213619in}}%
\pgfpathlineto{\pgfqpoint{2.017911in}{1.211101in}}%
\pgfpathlineto{\pgfqpoint{2.022293in}{1.208597in}}%
\pgfpathlineto{\pgfqpoint{2.026676in}{1.206107in}}%
\pgfpathlineto{\pgfqpoint{2.031059in}{1.203631in}}%
\pgfpathlineto{\pgfqpoint{2.035441in}{1.201169in}}%
\pgfpathlineto{\pgfqpoint{2.039824in}{1.198720in}}%
\pgfpathlineto{\pgfqpoint{2.044207in}{1.196284in}}%
\pgfpathlineto{\pgfqpoint{2.048589in}{1.193862in}}%
\pgfpathlineto{\pgfqpoint{2.052972in}{1.191454in}}%
\pgfpathlineto{\pgfqpoint{2.057355in}{1.189058in}}%
\pgfpathlineto{\pgfqpoint{2.061737in}{1.186676in}}%
\pgfpathlineto{\pgfqpoint{2.066120in}{1.184306in}}%
\pgfpathlineto{\pgfqpoint{2.070503in}{1.181950in}}%
\pgfpathlineto{\pgfqpoint{2.074886in}{1.179606in}}%
\pgfpathlineto{\pgfqpoint{2.079268in}{1.177274in}}%
\pgfpathlineto{\pgfqpoint{2.083651in}{1.174956in}}%
\pgfpathlineto{\pgfqpoint{2.088034in}{1.172649in}}%
\pgfpathlineto{\pgfqpoint{2.092416in}{1.170355in}}%
\pgfpathlineto{\pgfqpoint{2.096799in}{1.168073in}}%
\pgfpathlineto{\pgfqpoint{2.101182in}{1.165804in}}%
\pgfpathlineto{\pgfqpoint{2.105564in}{1.163546in}}%
\pgfpathlineto{\pgfqpoint{2.109947in}{1.161300in}}%
\pgfpathlineto{\pgfqpoint{2.114330in}{1.159067in}}%
\pgfpathlineto{\pgfqpoint{2.118712in}{1.156844in}}%
\pgfpathlineto{\pgfqpoint{2.123095in}{1.154634in}}%
\pgfpathlineto{\pgfqpoint{2.127478in}{1.152435in}}%
\pgfpathlineto{\pgfqpoint{2.131860in}{1.150248in}}%
\pgfpathlineto{\pgfqpoint{2.136243in}{1.148072in}}%
\pgfpathlineto{\pgfqpoint{2.140626in}{1.145907in}}%
\pgfpathlineto{\pgfqpoint{2.145009in}{1.143754in}}%
\pgfpathlineto{\pgfqpoint{2.149391in}{1.141611in}}%
\pgfpathlineto{\pgfqpoint{2.153774in}{1.139480in}}%
\pgfpathlineto{\pgfqpoint{2.158157in}{1.137360in}}%
\pgfpathlineto{\pgfqpoint{2.162539in}{1.135250in}}%
\pgfpathlineto{\pgfqpoint{2.166922in}{1.133151in}}%
\pgfpathlineto{\pgfqpoint{2.171305in}{1.131063in}}%
\pgfpathlineto{\pgfqpoint{2.175687in}{1.128986in}}%
\pgfpathlineto{\pgfqpoint{2.180070in}{1.126919in}}%
\pgfpathlineto{\pgfqpoint{2.184453in}{1.124863in}}%
\pgfpathlineto{\pgfqpoint{2.188835in}{1.122817in}}%
\pgfpathlineto{\pgfqpoint{2.193218in}{1.120781in}}%
\pgfpathlineto{\pgfqpoint{2.197601in}{1.118756in}}%
\pgfpathlineto{\pgfqpoint{2.201983in}{1.116740in}}%
\pgfpathlineto{\pgfqpoint{2.206366in}{1.114735in}}%
\pgfpathlineto{\pgfqpoint{2.210749in}{1.112740in}}%
\pgfpathlineto{\pgfqpoint{2.215132in}{1.110755in}}%
\pgfpathlineto{\pgfqpoint{2.219514in}{1.108779in}}%
\pgfpathlineto{\pgfqpoint{2.223897in}{1.106813in}}%
\pgfpathlineto{\pgfqpoint{2.228280in}{1.104857in}}%
\pgfpathlineto{\pgfqpoint{2.232662in}{1.102911in}}%
\pgfpathlineto{\pgfqpoint{2.237045in}{1.100974in}}%
\pgfpathlineto{\pgfqpoint{2.241428in}{1.099047in}}%
\pgfpathlineto{\pgfqpoint{2.245810in}{1.097129in}}%
\pgfpathlineto{\pgfqpoint{2.250193in}{1.095221in}}%
\pgfpathlineto{\pgfqpoint{2.254576in}{1.093321in}}%
\pgfpathlineto{\pgfqpoint{2.258958in}{1.091431in}}%
\pgfpathlineto{\pgfqpoint{2.263341in}{1.089550in}}%
\pgfpathlineto{\pgfqpoint{2.267724in}{1.087679in}}%
\pgfpathlineto{\pgfqpoint{2.272106in}{1.085816in}}%
\pgfpathlineto{\pgfqpoint{2.276489in}{1.083962in}}%
\pgfpathlineto{\pgfqpoint{2.280872in}{1.082117in}}%
\pgfpathlineto{\pgfqpoint{2.285255in}{1.080281in}}%
\pgfpathlineto{\pgfqpoint{2.289637in}{1.078454in}}%
\pgfpathlineto{\pgfqpoint{2.294020in}{1.076635in}}%
\pgfpathlineto{\pgfqpoint{2.298403in}{1.074825in}}%
\pgfpathlineto{\pgfqpoint{2.302785in}{1.073024in}}%
\pgfpathlineto{\pgfqpoint{2.307168in}{1.071231in}}%
\pgfpathlineto{\pgfqpoint{2.311551in}{1.069446in}}%
\pgfpathlineto{\pgfqpoint{2.315933in}{1.067670in}}%
\pgfpathlineto{\pgfqpoint{2.320316in}{1.065902in}}%
\pgfpathlineto{\pgfqpoint{2.324699in}{1.064143in}}%
\pgfpathlineto{\pgfqpoint{2.329081in}{1.062392in}}%
\pgfpathlineto{\pgfqpoint{2.333464in}{1.060649in}}%
\pgfpathlineto{\pgfqpoint{2.337847in}{1.058914in}}%
\pgfpathlineto{\pgfqpoint{2.342230in}{1.057187in}}%
\pgfpathlineto{\pgfqpoint{2.346612in}{1.055468in}}%
\pgfpathlineto{\pgfqpoint{2.350995in}{1.053757in}}%
\pgfpathlineto{\pgfqpoint{2.355378in}{1.052054in}}%
\pgfpathlineto{\pgfqpoint{2.359760in}{1.050358in}}%
\pgfpathlineto{\pgfqpoint{2.364143in}{1.048671in}}%
\pgfpathlineto{\pgfqpoint{2.368526in}{1.046991in}}%
\pgfpathlineto{\pgfqpoint{2.372908in}{1.045319in}}%
\pgfpathlineto{\pgfqpoint{2.377291in}{1.043654in}}%
\pgfpathlineto{\pgfqpoint{2.381674in}{1.041997in}}%
\pgfpathlineto{\pgfqpoint{2.386056in}{1.040348in}}%
\pgfpathlineto{\pgfqpoint{2.390439in}{1.038706in}}%
\pgfpathlineto{\pgfqpoint{2.394822in}{1.037071in}}%
\pgfpathlineto{\pgfqpoint{2.399204in}{1.035444in}}%
\pgfpathlineto{\pgfqpoint{2.403587in}{1.033824in}}%
\pgfpathlineto{\pgfqpoint{2.407970in}{1.032212in}}%
\pgfpathlineto{\pgfqpoint{2.412353in}{1.030606in}}%
\pgfpathlineto{\pgfqpoint{2.416735in}{1.029008in}}%
\pgfpathlineto{\pgfqpoint{2.421118in}{1.027417in}}%
\pgfpathlineto{\pgfqpoint{2.425501in}{1.025833in}}%
\pgfpathlineto{\pgfqpoint{2.429883in}{1.024256in}}%
\pgfpathlineto{\pgfqpoint{2.434266in}{1.022686in}}%
\pgfpathlineto{\pgfqpoint{2.438649in}{1.021123in}}%
\pgfpathlineto{\pgfqpoint{2.443031in}{1.019566in}}%
\pgfpathlineto{\pgfqpoint{2.447414in}{1.018017in}}%
\pgfpathlineto{\pgfqpoint{2.451797in}{1.016474in}}%
\pgfpathlineto{\pgfqpoint{2.456179in}{1.014939in}}%
\pgfpathlineto{\pgfqpoint{2.460562in}{1.013409in}}%
\pgfpathlineto{\pgfqpoint{2.464945in}{1.011887in}}%
\pgfpathlineto{\pgfqpoint{2.469327in}{1.010371in}}%
\pgfpathlineto{\pgfqpoint{2.473710in}{1.008862in}}%
\pgfpathlineto{\pgfqpoint{2.478093in}{1.007359in}}%
\pgfpathlineto{\pgfqpoint{2.482476in}{1.005863in}}%
\pgfpathlineto{\pgfqpoint{2.486858in}{1.004373in}}%
\pgfpathlineto{\pgfqpoint{2.491241in}{1.002889in}}%
\pgfpathlineto{\pgfqpoint{2.495624in}{1.001412in}}%
\pgfpathlineto{\pgfqpoint{2.500006in}{0.999942in}}%
\pgfpathlineto{\pgfqpoint{2.504389in}{0.998477in}}%
\pgfpathlineto{\pgfqpoint{2.508772in}{0.997019in}}%
\pgfpathlineto{\pgfqpoint{2.513154in}{0.995567in}}%
\pgfpathlineto{\pgfqpoint{2.517537in}{0.994121in}}%
\pgfpathlineto{\pgfqpoint{2.521920in}{0.992681in}}%
\pgfpathlineto{\pgfqpoint{2.526302in}{0.991248in}}%
\pgfpathlineto{\pgfqpoint{2.530685in}{0.989820in}}%
\pgfpathlineto{\pgfqpoint{2.535068in}{0.988398in}}%
\pgfpathlineto{\pgfqpoint{2.539450in}{0.986983in}}%
\pgfpathlineto{\pgfqpoint{2.543833in}{0.985573in}}%
\pgfpathlineto{\pgfqpoint{2.548216in}{0.984169in}}%
\pgfpathlineto{\pgfqpoint{2.552599in}{0.982771in}}%
\pgfpathlineto{\pgfqpoint{2.556981in}{0.981379in}}%
\pgfpathlineto{\pgfqpoint{2.561364in}{0.979993in}}%
\pgfpathlineto{\pgfqpoint{2.565747in}{0.978613in}}%
\pgfpathlineto{\pgfqpoint{2.570129in}{0.977238in}}%
\pgfpathlineto{\pgfqpoint{2.574512in}{0.975869in}}%
\pgfpathlineto{\pgfqpoint{2.578895in}{0.974505in}}%
\pgfpathlineto{\pgfqpoint{2.583277in}{0.973147in}}%
\pgfpathlineto{\pgfqpoint{2.587660in}{0.971795in}}%
\pgfpathlineto{\pgfqpoint{2.592043in}{0.970448in}}%
\pgfpathlineto{\pgfqpoint{2.596425in}{0.969107in}}%
\pgfpathlineto{\pgfqpoint{2.600808in}{0.967771in}}%
\pgfpathlineto{\pgfqpoint{2.605191in}{0.966440in}}%
\pgfpathlineto{\pgfqpoint{2.609573in}{0.965116in}}%
\pgfpathlineto{\pgfqpoint{2.613956in}{0.963796in}}%
\pgfpathlineto{\pgfqpoint{2.618339in}{0.962482in}}%
\pgfpathlineto{\pgfqpoint{2.622722in}{0.961173in}}%
\pgfpathlineto{\pgfqpoint{2.627104in}{0.959869in}}%
\pgfpathlineto{\pgfqpoint{2.631487in}{0.958571in}}%
\pgfpathlineto{\pgfqpoint{2.635870in}{0.957277in}}%
\pgfpathlineto{\pgfqpoint{2.640252in}{0.955989in}}%
\pgfpathlineto{\pgfqpoint{2.644635in}{0.954707in}}%
\pgfpathlineto{\pgfqpoint{2.649018in}{0.953429in}}%
\pgfpathlineto{\pgfqpoint{2.653400in}{0.952156in}}%
\pgfpathlineto{\pgfqpoint{2.657783in}{0.950888in}}%
\pgfpathlineto{\pgfqpoint{2.662166in}{0.949626in}}%
\pgfpathlineto{\pgfqpoint{2.666548in}{0.948368in}}%
\pgfpathlineto{\pgfqpoint{2.670931in}{0.947116in}}%
\pgfpathlineto{\pgfqpoint{2.675314in}{0.945868in}}%
\pgfpathlineto{\pgfqpoint{2.679697in}{0.944625in}}%
\pgfpathlineto{\pgfqpoint{2.684079in}{0.943387in}}%
\pgfpathlineto{\pgfqpoint{2.688462in}{0.942154in}}%
\pgfpathlineto{\pgfqpoint{2.692845in}{0.940926in}}%
\pgfpathlineto{\pgfqpoint{2.697227in}{0.939703in}}%
\pgfpathlineto{\pgfqpoint{2.701610in}{0.938484in}}%
\pgfpathlineto{\pgfqpoint{2.705993in}{0.937270in}}%
\pgfpathlineto{\pgfqpoint{2.710375in}{0.936061in}}%
\pgfpathlineto{\pgfqpoint{2.714758in}{0.934857in}}%
\pgfpathlineto{\pgfqpoint{2.719141in}{0.933657in}}%
\pgfpathlineto{\pgfqpoint{2.723523in}{0.932462in}}%
\pgfpathlineto{\pgfqpoint{2.727906in}{0.931271in}}%
\pgfpathlineto{\pgfqpoint{2.732289in}{0.930085in}}%
\pgfpathlineto{\pgfqpoint{2.736671in}{0.928903in}}%
\pgfpathlineto{\pgfqpoint{2.741054in}{0.927726in}}%
\pgfpathlineto{\pgfqpoint{2.745437in}{0.926554in}}%
\pgfpathlineto{\pgfqpoint{2.749820in}{0.925386in}}%
\pgfpathlineto{\pgfqpoint{2.754202in}{0.924223in}}%
\pgfpathlineto{\pgfqpoint{2.758585in}{0.923063in}}%
\pgfpathlineto{\pgfqpoint{2.762968in}{0.921909in}}%
\pgfpathlineto{\pgfqpoint{2.767350in}{0.920758in}}%
\pgfpathlineto{\pgfqpoint{2.771733in}{0.919612in}}%
\pgfpathlineto{\pgfqpoint{2.776116in}{0.918471in}}%
\pgfpathlineto{\pgfqpoint{2.780498in}{0.917333in}}%
\pgfpathlineto{\pgfqpoint{2.784881in}{0.916200in}}%
\pgfpathlineto{\pgfqpoint{2.789264in}{0.915071in}}%
\pgfpathlineto{\pgfqpoint{2.793646in}{0.913947in}}%
\pgfpathlineto{\pgfqpoint{2.798029in}{0.912826in}}%
\pgfpathlineto{\pgfqpoint{2.802412in}{0.911710in}}%
\pgfpathlineto{\pgfqpoint{2.806794in}{0.910598in}}%
\pgfpathlineto{\pgfqpoint{2.811177in}{0.909490in}}%
\pgfpathlineto{\pgfqpoint{2.815560in}{0.908386in}}%
\pgfpathlineto{\pgfqpoint{2.819943in}{0.907287in}}%
\pgfpathlineto{\pgfqpoint{2.824325in}{0.906191in}}%
\pgfpathlineto{\pgfqpoint{2.828708in}{0.905099in}}%
\pgfpathlineto{\pgfqpoint{2.833091in}{0.904012in}}%
\pgfpathlineto{\pgfqpoint{2.837473in}{0.902928in}}%
\pgfpathlineto{\pgfqpoint{2.841856in}{0.901848in}}%
\pgfpathlineto{\pgfqpoint{2.846239in}{0.900773in}}%
\pgfpathlineto{\pgfqpoint{2.850621in}{0.899701in}}%
\pgfpathlineto{\pgfqpoint{2.855004in}{0.898633in}}%
\pgfpathlineto{\pgfqpoint{2.859387in}{0.897569in}}%
\pgfpathlineto{\pgfqpoint{2.863769in}{0.896509in}}%
\pgfpathlineto{\pgfqpoint{2.868152in}{0.895453in}}%
\pgfpathlineto{\pgfqpoint{2.872535in}{0.894400in}}%
\pgfpathlineto{\pgfqpoint{2.876917in}{0.893352in}}%
\pgfpathlineto{\pgfqpoint{2.881300in}{0.892307in}}%
\pgfpathlineto{\pgfqpoint{2.885683in}{0.891266in}}%
\pgfpathlineto{\pgfqpoint{2.890066in}{0.890229in}}%
\pgfpathlineto{\pgfqpoint{2.894448in}{0.889195in}}%
\pgfpathlineto{\pgfqpoint{2.898831in}{0.888165in}}%
\pgfpathlineto{\pgfqpoint{2.903214in}{0.887139in}}%
\pgfpathlineto{\pgfqpoint{2.907596in}{0.886116in}}%
\pgfpathlineto{\pgfqpoint{2.911979in}{0.885098in}}%
\pgfpathlineto{\pgfqpoint{2.916362in}{0.884082in}}%
\pgfpathlineto{\pgfqpoint{2.920744in}{0.883071in}}%
\pgfpathlineto{\pgfqpoint{2.925127in}{0.882063in}}%
\pgfpathlineto{\pgfqpoint{2.929510in}{0.881058in}}%
\pgfpathlineto{\pgfqpoint{2.933892in}{0.880057in}}%
\pgfpathlineto{\pgfqpoint{2.938275in}{0.879060in}}%
\pgfpathlineto{\pgfqpoint{2.942658in}{0.878066in}}%
\pgfpathlineto{\pgfqpoint{2.947040in}{0.877075in}}%
\pgfpathlineto{\pgfqpoint{2.951423in}{0.876088in}}%
\pgfpathlineto{\pgfqpoint{2.955806in}{0.875105in}}%
\pgfpathlineto{\pgfqpoint{2.960189in}{0.874125in}}%
\pgfpathlineto{\pgfqpoint{2.964571in}{0.873148in}}%
\pgfpathlineto{\pgfqpoint{2.968954in}{0.872175in}}%
\pgfpathlineto{\pgfqpoint{2.973337in}{0.871205in}}%
\pgfpathlineto{\pgfqpoint{2.977719in}{0.870239in}}%
\pgfpathlineto{\pgfqpoint{2.982102in}{0.869275in}}%
\pgfpathlineto{\pgfqpoint{2.986485in}{0.868316in}}%
\pgfpathlineto{\pgfqpoint{2.990867in}{0.867359in}}%
\pgfpathlineto{\pgfqpoint{2.995250in}{0.866406in}}%
\pgfpathlineto{\pgfqpoint{2.999633in}{0.865456in}}%
\pgfpathlineto{\pgfqpoint{3.004015in}{0.864509in}}%
\pgfpathlineto{\pgfqpoint{3.008398in}{0.863566in}}%
\pgfpathlineto{\pgfqpoint{3.012781in}{0.862626in}}%
\pgfpathlineto{\pgfqpoint{3.017164in}{0.861689in}}%
\pgfpathlineto{\pgfqpoint{3.021546in}{0.860755in}}%
\pgfpathlineto{\pgfqpoint{3.025929in}{0.859824in}}%
\pgfpathlineto{\pgfqpoint{3.030312in}{0.858897in}}%
\pgfpathlineto{\pgfqpoint{3.034694in}{0.857973in}}%
\pgfpathlineto{\pgfqpoint{3.039077in}{0.857052in}}%
\pgfpathlineto{\pgfqpoint{3.043460in}{0.856134in}}%
\pgfpathlineto{\pgfqpoint{3.047842in}{0.855219in}}%
\pgfpathlineto{\pgfqpoint{3.052225in}{0.854307in}}%
\pgfpathlineto{\pgfqpoint{3.056608in}{0.853398in}}%
\pgfpathlineto{\pgfqpoint{3.060990in}{0.852492in}}%
\pgfpathlineto{\pgfqpoint{3.065373in}{0.851590in}}%
\pgfpathlineto{\pgfqpoint{3.069756in}{0.850690in}}%
\pgfpathlineto{\pgfqpoint{3.074138in}{0.849793in}}%
\pgfpathlineto{\pgfqpoint{3.078521in}{0.848900in}}%
\pgfpathlineto{\pgfqpoint{3.082904in}{0.848009in}}%
\pgfpathlineto{\pgfqpoint{3.087287in}{0.847121in}}%
\pgfpathlineto{\pgfqpoint{3.091669in}{0.846237in}}%
\pgfpathlineto{\pgfqpoint{3.096052in}{0.845355in}}%
\pgfpathlineto{\pgfqpoint{3.100435in}{0.844476in}}%
\pgfpathlineto{\pgfqpoint{3.104817in}{0.843600in}}%
\pgfpathlineto{\pgfqpoint{3.109200in}{0.842727in}}%
\pgfpathlineto{\pgfqpoint{3.113583in}{0.841857in}}%
\pgfpathlineto{\pgfqpoint{3.117965in}{0.840989in}}%
\pgfpathlineto{\pgfqpoint{3.122348in}{0.840125in}}%
\pgfpathlineto{\pgfqpoint{3.126731in}{0.839263in}}%
\pgfpathlineto{\pgfqpoint{3.131113in}{0.838404in}}%
\pgfpathlineto{\pgfqpoint{3.135496in}{0.837548in}}%
\pgfpathlineto{\pgfqpoint{3.139879in}{0.836695in}}%
\pgfpathlineto{\pgfqpoint{3.144261in}{0.835844in}}%
\pgfpathlineto{\pgfqpoint{3.148644in}{0.834997in}}%
\pgfpathlineto{\pgfqpoint{3.153027in}{0.834152in}}%
\pgfpathlineto{\pgfqpoint{3.157410in}{0.833309in}}%
\pgfpathlineto{\pgfqpoint{3.161792in}{0.832470in}}%
\pgfpathlineto{\pgfqpoint{3.166175in}{0.831633in}}%
\pgfpathlineto{\pgfqpoint{3.170558in}{0.830799in}}%
\pgfpathlineto{\pgfqpoint{3.174940in}{0.829968in}}%
\pgfpathlineto{\pgfqpoint{3.179323in}{0.829139in}}%
\pgfpathlineto{\pgfqpoint{3.183706in}{0.828313in}}%
\pgfpathlineto{\pgfqpoint{3.188088in}{0.827489in}}%
\pgfpathlineto{\pgfqpoint{3.192471in}{0.826669in}}%
\pgfpathlineto{\pgfqpoint{3.196854in}{0.825850in}}%
\pgfpathlineto{\pgfqpoint{3.201236in}{0.825035in}}%
\pgfpathlineto{\pgfqpoint{3.205619in}{0.824222in}}%
\pgfpathlineto{\pgfqpoint{3.210002in}{0.823411in}}%
\pgfpathlineto{\pgfqpoint{3.214384in}{0.822604in}}%
\pgfpathlineto{\pgfqpoint{3.218767in}{0.821798in}}%
\pgfpathlineto{\pgfqpoint{3.223150in}{0.820996in}}%
\pgfpathlineto{\pgfqpoint{3.227533in}{0.820196in}}%
\pgfpathlineto{\pgfqpoint{3.231915in}{0.819398in}}%
\pgfpathlineto{\pgfqpoint{3.236298in}{0.818603in}}%
\pgfpathlineto{\pgfqpoint{3.240681in}{0.817810in}}%
\pgfpathlineto{\pgfqpoint{3.245063in}{0.817020in}}%
\pgfpathlineto{\pgfqpoint{3.249446in}{0.816232in}}%
\pgfpathlineto{\pgfqpoint{3.253829in}{0.815447in}}%
\pgfpathlineto{\pgfqpoint{3.258211in}{0.814664in}}%
\pgfpathlineto{\pgfqpoint{3.262594in}{0.813884in}}%
\pgfpathlineto{\pgfqpoint{3.266977in}{0.813106in}}%
\pgfpathlineto{\pgfqpoint{3.271359in}{0.812331in}}%
\pgfpathlineto{\pgfqpoint{3.275742in}{0.811558in}}%
\pgfpathlineto{\pgfqpoint{3.280125in}{0.810787in}}%
\pgfpathlineto{\pgfqpoint{3.284507in}{0.810019in}}%
\pgfpathlineto{\pgfqpoint{3.288890in}{0.809253in}}%
\pgfpathlineto{\pgfqpoint{3.293273in}{0.808489in}}%
\pgfpathlineto{\pgfqpoint{3.297656in}{0.807728in}}%
\pgfpathlineto{\pgfqpoint{3.302038in}{0.806969in}}%
\pgfpathlineto{\pgfqpoint{3.306421in}{0.806213in}}%
\pgfpathlineto{\pgfqpoint{3.310804in}{0.805459in}}%
\pgfpathlineto{\pgfqpoint{3.315186in}{0.804707in}}%
\pgfpathlineto{\pgfqpoint{3.319569in}{0.803957in}}%
\pgfpathlineto{\pgfqpoint{3.323952in}{0.803210in}}%
\pgfpathlineto{\pgfqpoint{3.328334in}{0.802465in}}%
\pgfpathlineto{\pgfqpoint{3.332717in}{0.801722in}}%
\pgfpathlineto{\pgfqpoint{3.337100in}{0.800982in}}%
\pgfpathlineto{\pgfqpoint{3.341482in}{0.800243in}}%
\pgfpathlineto{\pgfqpoint{3.345865in}{0.799507in}}%
\pgfpathlineto{\pgfqpoint{3.350248in}{0.798774in}}%
\pgfpathlineto{\pgfqpoint{3.354631in}{0.798042in}}%
\pgfpathlineto{\pgfqpoint{3.359013in}{0.797313in}}%
\pgfpathlineto{\pgfqpoint{3.363396in}{0.796585in}}%
\pgfpathlineto{\pgfqpoint{3.367779in}{0.795860in}}%
\pgfpathlineto{\pgfqpoint{3.372161in}{0.795138in}}%
\pgfpathlineto{\pgfqpoint{3.376544in}{0.794417in}}%
\pgfpathlineto{\pgfqpoint{3.380927in}{0.793699in}}%
\pgfpathlineto{\pgfqpoint{3.385309in}{0.792982in}}%
\pgfpathlineto{\pgfqpoint{3.389692in}{0.792268in}}%
\pgfpathlineto{\pgfqpoint{3.394075in}{0.791556in}}%
\pgfpathlineto{\pgfqpoint{3.398457in}{0.790846in}}%
\pgfpathlineto{\pgfqpoint{3.402840in}{0.790138in}}%
\pgfpathlineto{\pgfqpoint{3.407223in}{0.789433in}}%
\pgfpathlineto{\pgfqpoint{3.411605in}{0.788729in}}%
\pgfpathlineto{\pgfqpoint{3.415988in}{0.788027in}}%
\pgfpathlineto{\pgfqpoint{3.420371in}{0.787328in}}%
\pgfpathlineto{\pgfqpoint{3.424754in}{0.786630in}}%
\pgfpathlineto{\pgfqpoint{3.429136in}{0.785935in}}%
\pgfpathlineto{\pgfqpoint{3.433519in}{0.785242in}}%
\pgfpathlineto{\pgfqpoint{3.437902in}{0.784550in}}%
\pgfpathlineto{\pgfqpoint{3.442284in}{0.783861in}}%
\pgfpathlineto{\pgfqpoint{3.446667in}{0.783174in}}%
\pgfpathlineto{\pgfqpoint{3.451050in}{0.782489in}}%
\pgfpathlineto{\pgfqpoint{3.455432in}{0.781805in}}%
\pgfpathlineto{\pgfqpoint{3.459815in}{0.781124in}}%
\pgfpathlineto{\pgfqpoint{3.464198in}{0.780445in}}%
\pgfpathlineto{\pgfqpoint{3.468580in}{0.779768in}}%
\pgfpathlineto{\pgfqpoint{3.472963in}{0.779092in}}%
\pgfpathlineto{\pgfqpoint{3.477346in}{0.778419in}}%
\pgfpathlineto{\pgfqpoint{3.481728in}{0.777748in}}%
\pgfpathlineto{\pgfqpoint{3.486111in}{0.777078in}}%
\pgfpathlineto{\pgfqpoint{3.490494in}{0.776411in}}%
\pgfpathlineto{\pgfqpoint{3.494877in}{0.775745in}}%
\pgfpathlineto{\pgfqpoint{3.499259in}{0.775081in}}%
\pgfpathlineto{\pgfqpoint{3.503642in}{0.774419in}}%
\pgfpathlineto{\pgfqpoint{3.508025in}{0.773759in}}%
\pgfpathlineto{\pgfqpoint{3.512407in}{0.773101in}}%
\pgfpathlineto{\pgfqpoint{3.516790in}{0.772445in}}%
\pgfpathlineto{\pgfqpoint{3.521173in}{0.771791in}}%
\pgfpathlineto{\pgfqpoint{3.525555in}{0.771139in}}%
\pgfpathlineto{\pgfqpoint{3.529938in}{0.770488in}}%
\pgfpathlineto{\pgfqpoint{3.534321in}{0.769840in}}%
\pgfpathlineto{\pgfqpoint{3.538703in}{0.769193in}}%
\pgfpathlineto{\pgfqpoint{3.543086in}{0.768548in}}%
\pgfpathlineto{\pgfqpoint{3.547469in}{0.767905in}}%
\pgfpathlineto{\pgfqpoint{3.551851in}{0.767263in}}%
\pgfpathlineto{\pgfqpoint{3.556234in}{0.766624in}}%
\pgfpathlineto{\pgfqpoint{3.560617in}{0.765986in}}%
\pgfpathlineto{\pgfqpoint{3.565000in}{0.765350in}}%
\pgfpathlineto{\pgfqpoint{3.569382in}{0.764716in}}%
\pgfpathlineto{\pgfqpoint{3.573765in}{0.764083in}}%
\pgfpathlineto{\pgfqpoint{3.578148in}{0.763453in}}%
\pgfpathlineto{\pgfqpoint{3.582530in}{0.762824in}}%
\pgfpathlineto{\pgfqpoint{3.586913in}{0.762197in}}%
\pgfpathlineto{\pgfqpoint{3.591296in}{0.761572in}}%
\pgfpathlineto{\pgfqpoint{3.595678in}{0.760948in}}%
\pgfpathlineto{\pgfqpoint{3.600061in}{0.760326in}}%
\pgfpathlineto{\pgfqpoint{3.604444in}{0.759706in}}%
\pgfpathlineto{\pgfqpoint{3.608826in}{0.759088in}}%
\pgfpathlineto{\pgfqpoint{3.613209in}{0.758471in}}%
\pgfpathlineto{\pgfqpoint{3.617592in}{0.757856in}}%
\pgfpathlineto{\pgfqpoint{3.621974in}{0.757243in}}%
\pgfpathlineto{\pgfqpoint{3.626357in}{0.756631in}}%
\pgfpathlineto{\pgfqpoint{3.630740in}{0.756021in}}%
\pgfpathlineto{\pgfqpoint{3.635123in}{0.755413in}}%
\pgfpathlineto{\pgfqpoint{3.639505in}{0.754807in}}%
\pgfpathlineto{\pgfqpoint{3.643888in}{0.754202in}}%
\pgfpathlineto{\pgfqpoint{3.648271in}{0.753598in}}%
\pgfpathlineto{\pgfqpoint{3.652653in}{0.752997in}}%
\pgfpathlineto{\pgfqpoint{3.657036in}{0.752397in}}%
\pgfpathlineto{\pgfqpoint{3.661419in}{0.751799in}}%
\pgfpathlineto{\pgfqpoint{3.665801in}{0.751202in}}%
\pgfpathlineto{\pgfqpoint{3.670184in}{0.750607in}}%
\pgfpathlineto{\pgfqpoint{3.674567in}{0.750014in}}%
\pgfpathlineto{\pgfqpoint{3.678949in}{0.749422in}}%
\pgfpathlineto{\pgfqpoint{3.683332in}{0.748832in}}%
\pgfpathlineto{\pgfqpoint{3.687715in}{0.748243in}}%
\pgfpathlineto{\pgfqpoint{3.692098in}{0.747656in}}%
\pgfpathlineto{\pgfqpoint{3.696480in}{0.747071in}}%
\pgfpathlineto{\pgfqpoint{3.700863in}{0.746487in}}%
\pgfpathlineto{\pgfqpoint{3.705246in}{0.745905in}}%
\pgfpathlineto{\pgfqpoint{3.709628in}{0.745324in}}%
\pgfpathlineto{\pgfqpoint{3.714011in}{0.744745in}}%
\pgfpathlineto{\pgfqpoint{3.718394in}{0.744167in}}%
\pgfpathlineto{\pgfqpoint{3.722776in}{0.743591in}}%
\pgfpathlineto{\pgfqpoint{3.727159in}{0.743017in}}%
\pgfpathlineto{\pgfqpoint{3.731542in}{0.742444in}}%
\pgfpathlineto{\pgfqpoint{3.735924in}{0.741872in}}%
\pgfpathlineto{\pgfqpoint{3.740307in}{0.741303in}}%
\pgfpathlineto{\pgfqpoint{3.744690in}{0.740734in}}%
\pgfpathlineto{\pgfqpoint{3.749072in}{0.740167in}}%
\pgfpathlineto{\pgfqpoint{3.753455in}{0.739602in}}%
\pgfpathlineto{\pgfqpoint{3.757838in}{0.739038in}}%
\pgfpathlineto{\pgfqpoint{3.762221in}{0.738476in}}%
\pgfpathlineto{\pgfqpoint{3.766603in}{0.737915in}}%
\pgfpathlineto{\pgfqpoint{3.770986in}{0.737356in}}%
\pgfpathlineto{\pgfqpoint{3.775369in}{0.736798in}}%
\pgfpathlineto{\pgfqpoint{3.779751in}{0.736241in}}%
\pgfpathlineto{\pgfqpoint{3.784134in}{0.735686in}}%
\pgfpathlineto{\pgfqpoint{3.788517in}{0.735133in}}%
\pgfpathlineto{\pgfqpoint{3.792899in}{0.734581in}}%
\pgfpathlineto{\pgfqpoint{3.797282in}{0.734030in}}%
\pgfpathlineto{\pgfqpoint{3.801665in}{0.733481in}}%
\pgfpathlineto{\pgfqpoint{3.806047in}{0.732933in}}%
\pgfpathlineto{\pgfqpoint{3.810430in}{0.732387in}}%
\pgfpathlineto{\pgfqpoint{3.814813in}{0.731842in}}%
\pgfpathlineto{\pgfqpoint{3.819195in}{0.731298in}}%
\pgfpathlineto{\pgfqpoint{3.823578in}{0.730756in}}%
\pgfpathlineto{\pgfqpoint{3.827961in}{0.730216in}}%
\pgfpathlineto{\pgfqpoint{3.832344in}{0.729676in}}%
\pgfpathlineto{\pgfqpoint{3.836726in}{0.729139in}}%
\pgfpathlineto{\pgfqpoint{3.841109in}{0.728602in}}%
\pgfpathlineto{\pgfqpoint{3.845492in}{0.728067in}}%
\pgfpathlineto{\pgfqpoint{3.849874in}{0.727533in}}%
\pgfpathlineto{\pgfqpoint{3.854257in}{0.727001in}}%
\pgfpathlineto{\pgfqpoint{3.858640in}{0.726470in}}%
\pgfpathlineto{\pgfqpoint{3.863022in}{0.725940in}}%
\pgfpathlineto{\pgfqpoint{3.867405in}{0.725412in}}%
\pgfpathlineto{\pgfqpoint{3.871788in}{0.724885in}}%
\pgfpathlineto{\pgfqpoint{3.876170in}{0.724360in}}%
\pgfpathlineto{\pgfqpoint{3.880553in}{0.723836in}}%
\pgfpathlineto{\pgfqpoint{3.884936in}{0.723313in}}%
\pgfpathlineto{\pgfqpoint{3.889318in}{0.722791in}}%
\pgfpathlineto{\pgfqpoint{3.893701in}{0.722271in}}%
\pgfpathlineto{\pgfqpoint{3.898084in}{0.721752in}}%
\pgfpathlineto{\pgfqpoint{3.902467in}{0.721235in}}%
\pgfpathlineto{\pgfqpoint{3.906849in}{0.720718in}}%
\pgfpathlineto{\pgfqpoint{3.911232in}{0.720204in}}%
\pgfpathlineto{\pgfqpoint{3.915615in}{0.719690in}}%
\pgfpathlineto{\pgfqpoint{3.919997in}{0.719178in}}%
\pgfpathlineto{\pgfqpoint{3.924380in}{0.718667in}}%
\pgfpathlineto{\pgfqpoint{3.928763in}{0.718157in}}%
\pgfpathlineto{\pgfqpoint{3.933145in}{0.717648in}}%
\pgfpathlineto{\pgfqpoint{3.937528in}{0.717141in}}%
\pgfpathlineto{\pgfqpoint{3.941911in}{0.716635in}}%
\pgfpathlineto{\pgfqpoint{3.946293in}{0.716131in}}%
\pgfpathlineto{\pgfqpoint{3.950676in}{0.715627in}}%
\pgfpathlineto{\pgfqpoint{3.955059in}{0.715125in}}%
\pgfpathlineto{\pgfqpoint{3.959441in}{0.714624in}}%
\pgfpathlineto{\pgfqpoint{3.963824in}{0.714125in}}%
\pgfpathlineto{\pgfqpoint{3.968207in}{0.713627in}}%
\pgfpathlineto{\pgfqpoint{3.972590in}{0.713129in}}%
\pgfpathlineto{\pgfqpoint{3.976972in}{0.712634in}}%
\pgfpathlineto{\pgfqpoint{3.981355in}{0.712139in}}%
\pgfpathlineto{\pgfqpoint{3.985738in}{0.711645in}}%
\pgfpathlineto{\pgfqpoint{3.990120in}{0.711153in}}%
\pgfpathlineto{\pgfqpoint{3.994503in}{0.710662in}}%
\pgfpathlineto{\pgfqpoint{3.998886in}{0.710173in}}%
\pgfpathlineto{\pgfqpoint{4.003268in}{0.709684in}}%
\pgfpathlineto{\pgfqpoint{4.007651in}{0.709197in}}%
\pgfpathlineto{\pgfqpoint{4.012034in}{0.708711in}}%
\pgfpathlineto{\pgfqpoint{4.016416in}{0.708226in}}%
\pgfpathlineto{\pgfqpoint{4.020799in}{0.707742in}}%
\pgfpathlineto{\pgfqpoint{4.025182in}{0.707259in}}%
\pgfpathlineto{\pgfqpoint{4.029565in}{0.706778in}}%
\pgfpathlineto{\pgfqpoint{4.033947in}{0.706298in}}%
\pgfpathlineto{\pgfqpoint{4.038330in}{0.705819in}}%
\pgfpathlineto{\pgfqpoint{4.042713in}{0.705341in}}%
\pgfpathlineto{\pgfqpoint{4.047095in}{0.704864in}}%
\pgfpathlineto{\pgfqpoint{4.051478in}{0.704388in}}%
\pgfpathlineto{\pgfqpoint{4.055861in}{0.703914in}}%
\pgfpathlineto{\pgfqpoint{4.060243in}{0.703441in}}%
\pgfpathlineto{\pgfqpoint{4.064626in}{0.702969in}}%
\pgfpathlineto{\pgfqpoint{4.069009in}{0.702498in}}%
\pgfpathlineto{\pgfqpoint{4.073391in}{0.702028in}}%
\pgfpathlineto{\pgfqpoint{4.077774in}{0.701559in}}%
\pgfpathlineto{\pgfqpoint{4.082157in}{0.701092in}}%
\pgfpathlineto{\pgfqpoint{4.086539in}{0.700625in}}%
\pgfpathlineto{\pgfqpoint{4.090922in}{0.700160in}}%
\pgfpathlineto{\pgfqpoint{4.095305in}{0.699696in}}%
\pgfpathlineto{\pgfqpoint{4.099688in}{0.699233in}}%
\pgfpathlineto{\pgfqpoint{4.104070in}{0.698771in}}%
\pgfpathlineto{\pgfqpoint{4.108453in}{0.698310in}}%
\pgfpathlineto{\pgfqpoint{4.112836in}{0.697851in}}%
\pgfpathlineto{\pgfqpoint{4.117218in}{0.697392in}}%
\pgfpathlineto{\pgfqpoint{4.121601in}{0.696934in}}%
\pgfpathlineto{\pgfqpoint{4.125984in}{0.696478in}}%
\pgfpathlineto{\pgfqpoint{4.130366in}{0.696023in}}%
\pgfpathlineto{\pgfqpoint{4.134749in}{0.695568in}}%
\pgfpathlineto{\pgfqpoint{4.139132in}{0.695115in}}%
\pgfpathlineto{\pgfqpoint{4.143514in}{0.694663in}}%
\pgfpathlineto{\pgfqpoint{4.147897in}{0.694212in}}%
\pgfpathlineto{\pgfqpoint{4.152280in}{0.693762in}}%
\pgfpathlineto{\pgfqpoint{4.156662in}{0.693313in}}%
\pgfpathlineto{\pgfqpoint{4.161045in}{0.692865in}}%
\pgfpathlineto{\pgfqpoint{4.165428in}{0.692419in}}%
\pgfpathlineto{\pgfqpoint{4.169811in}{0.691973in}}%
\pgfpathlineto{\pgfqpoint{4.174193in}{0.691528in}}%
\pgfpathlineto{\pgfqpoint{4.178576in}{0.691084in}}%
\pgfpathlineto{\pgfqpoint{4.182959in}{0.690642in}}%
\pgfpathlineto{\pgfqpoint{4.187341in}{0.690200in}}%
\pgfpathlineto{\pgfqpoint{4.191724in}{0.689760in}}%
\pgfpathlineto{\pgfqpoint{4.196107in}{0.689320in}}%
\pgfpathlineto{\pgfqpoint{4.200489in}{0.688882in}}%
\pgfpathlineto{\pgfqpoint{4.204872in}{0.688444in}}%
\pgfpathlineto{\pgfqpoint{4.209255in}{0.688008in}}%
\pgfpathlineto{\pgfqpoint{4.213637in}{0.687573in}}%
\pgfpathlineto{\pgfqpoint{4.218020in}{0.687138in}}%
\pgfpathlineto{\pgfqpoint{4.222403in}{0.686705in}}%
\pgfpathlineto{\pgfqpoint{4.226785in}{0.686273in}}%
\pgfpathlineto{\pgfqpoint{4.231168in}{0.685841in}}%
\pgfpathlineto{\pgfqpoint{4.235551in}{0.685411in}}%
\pgfpathlineto{\pgfqpoint{4.239934in}{0.684981in}}%
\pgfpathlineto{\pgfqpoint{4.244316in}{0.684553in}}%
\pgfpathlineto{\pgfqpoint{4.248699in}{0.684126in}}%
\pgfpathlineto{\pgfqpoint{4.253082in}{0.683699in}}%
\pgfpathlineto{\pgfqpoint{4.257464in}{0.683274in}}%
\pgfpathlineto{\pgfqpoint{4.261847in}{0.682849in}}%
\pgfpathlineto{\pgfqpoint{4.266230in}{0.682426in}}%
\pgfpathlineto{\pgfqpoint{4.270612in}{0.682003in}}%
\pgfpathlineto{\pgfqpoint{4.274995in}{0.681582in}}%
\pgfpathlineto{\pgfqpoint{4.279378in}{0.681161in}}%
\pgfpathlineto{\pgfqpoint{4.283760in}{0.680742in}}%
\pgfpathlineto{\pgfqpoint{4.288143in}{0.680323in}}%
\pgfpathlineto{\pgfqpoint{4.292526in}{0.679906in}}%
\pgfpathlineto{\pgfqpoint{4.296908in}{0.679489in}}%
\pgfpathlineto{\pgfqpoint{4.301291in}{0.679073in}}%
\pgfpathlineto{\pgfqpoint{4.305674in}{0.678658in}}%
\pgfpathlineto{\pgfqpoint{4.310057in}{0.678244in}}%
\pgfpathlineto{\pgfqpoint{4.314439in}{0.677831in}}%
\pgfpathlineto{\pgfqpoint{4.318822in}{0.677419in}}%
\pgfpathlineto{\pgfqpoint{4.323205in}{0.677008in}}%
\pgfpathlineto{\pgfqpoint{4.327587in}{0.676598in}}%
\pgfpathlineto{\pgfqpoint{4.331970in}{0.676189in}}%
\pgfpathlineto{\pgfqpoint{4.336353in}{0.675781in}}%
\pgfpathlineto{\pgfqpoint{4.340735in}{0.675373in}}%
\pgfpathlineto{\pgfqpoint{4.345118in}{0.674967in}}%
\pgfpathlineto{\pgfqpoint{4.349501in}{0.674561in}}%
\pgfpathlineto{\pgfqpoint{4.353883in}{0.674157in}}%
\pgfpathlineto{\pgfqpoint{4.358266in}{0.673753in}}%
\pgfpathlineto{\pgfqpoint{4.362649in}{0.673350in}}%
\pgfpathlineto{\pgfqpoint{4.367032in}{0.672948in}}%
\pgfpathlineto{\pgfqpoint{4.371414in}{0.672547in}}%
\pgfpathlineto{\pgfqpoint{4.375797in}{0.672147in}}%
\pgfpathlineto{\pgfqpoint{4.380180in}{0.671748in}}%
\pgfpathlineto{\pgfqpoint{4.384562in}{0.671349in}}%
\pgfpathlineto{\pgfqpoint{4.388945in}{0.670952in}}%
\pgfpathlineto{\pgfqpoint{4.393328in}{0.670555in}}%
\pgfpathlineto{\pgfqpoint{4.397710in}{0.670160in}}%
\pgfpathlineto{\pgfqpoint{4.402093in}{0.669765in}}%
\pgfpathlineto{\pgfqpoint{4.406476in}{0.669371in}}%
\pgfpathlineto{\pgfqpoint{4.410858in}{0.668978in}}%
\pgfpathlineto{\pgfqpoint{4.415241in}{0.668586in}}%
\pgfpathlineto{\pgfqpoint{4.419624in}{0.668194in}}%
\pgfpathlineto{\pgfqpoint{4.424006in}{0.667804in}}%
\pgfpathlineto{\pgfqpoint{4.428389in}{0.667414in}}%
\pgfpathlineto{\pgfqpoint{4.432772in}{0.667026in}}%
\pgfpathlineto{\pgfqpoint{4.437155in}{0.666638in}}%
\pgfpathlineto{\pgfqpoint{4.441537in}{0.666251in}}%
\pgfpathlineto{\pgfqpoint{4.445920in}{0.665864in}}%
\pgfpathlineto{\pgfqpoint{4.450303in}{0.665479in}}%
\pgfpathlineto{\pgfqpoint{4.454685in}{0.665095in}}%
\pgfpathlineto{\pgfqpoint{4.459068in}{0.664711in}}%
\pgfpathlineto{\pgfqpoint{4.463451in}{0.664328in}}%
\pgfpathlineto{\pgfqpoint{4.467833in}{0.663946in}}%
\pgfpathlineto{\pgfqpoint{4.472216in}{0.663565in}}%
\pgfpathlineto{\pgfqpoint{4.476599in}{0.663185in}}%
\pgfpathlineto{\pgfqpoint{4.480981in}{0.662805in}}%
\pgfpathlineto{\pgfqpoint{4.485364in}{0.662426in}}%
\pgfpathlineto{\pgfqpoint{4.489747in}{0.662049in}}%
\pgfpathlineto{\pgfqpoint{4.494129in}{0.661672in}}%
\pgfpathlineto{\pgfqpoint{4.498512in}{0.661295in}}%
\pgfpathlineto{\pgfqpoint{4.502895in}{0.660920in}}%
\pgfpathlineto{\pgfqpoint{4.507278in}{0.660545in}}%
\pgfpathlineto{\pgfqpoint{4.511660in}{0.660172in}}%
\pgfpathlineto{\pgfqpoint{4.516043in}{0.659799in}}%
\pgfpathlineto{\pgfqpoint{4.520426in}{0.659427in}}%
\pgfpathlineto{\pgfqpoint{4.524808in}{0.659055in}}%
\pgfpathlineto{\pgfqpoint{4.529191in}{0.658685in}}%
\pgfpathlineto{\pgfqpoint{4.533574in}{0.658315in}}%
\pgfpathlineto{\pgfqpoint{4.537956in}{0.657946in}}%
\pgfpathlineto{\pgfqpoint{4.542339in}{0.657578in}}%
\pgfpathlineto{\pgfqpoint{4.546722in}{0.657210in}}%
\pgfpathlineto{\pgfqpoint{4.551104in}{0.656844in}}%
\pgfpathlineto{\pgfqpoint{4.555487in}{0.656478in}}%
\pgfpathlineto{\pgfqpoint{4.559870in}{0.656113in}}%
\pgfpathlineto{\pgfqpoint{4.564252in}{0.655749in}}%
\pgfpathlineto{\pgfqpoint{4.568635in}{0.655385in}}%
\pgfpathlineto{\pgfqpoint{4.573018in}{0.655022in}}%
\pgfpathlineto{\pgfqpoint{4.577401in}{0.654660in}}%
\pgfpathlineto{\pgfqpoint{4.581783in}{0.654299in}}%
\pgfpathlineto{\pgfqpoint{4.586166in}{0.653939in}}%
\pgfpathlineto{\pgfqpoint{4.590549in}{0.653579in}}%
\pgfpathlineto{\pgfqpoint{4.594931in}{0.653220in}}%
\pgfpathlineto{\pgfqpoint{4.599314in}{0.652862in}}%
\pgfpathlineto{\pgfqpoint{4.603697in}{0.652505in}}%
\pgfpathlineto{\pgfqpoint{4.608079in}{0.652148in}}%
\pgfpathlineto{\pgfqpoint{4.612462in}{0.651792in}}%
\pgfpathlineto{\pgfqpoint{4.616845in}{0.651437in}}%
\pgfpathlineto{\pgfqpoint{4.621227in}{0.651083in}}%
\pgfpathlineto{\pgfqpoint{4.625610in}{0.650729in}}%
\pgfpathlineto{\pgfqpoint{4.629993in}{0.650376in}}%
\pgfpathlineto{\pgfqpoint{4.634375in}{0.650024in}}%
\pgfpathlineto{\pgfqpoint{4.638758in}{0.649673in}}%
\pgfpathlineto{\pgfqpoint{4.643141in}{0.649322in}}%
\pgfpathlineto{\pgfqpoint{4.647524in}{0.648972in}}%
\pgfpathlineto{\pgfqpoint{4.651906in}{0.648623in}}%
\pgfpathlineto{\pgfqpoint{4.656289in}{0.648274in}}%
\pgfpathlineto{\pgfqpoint{4.660672in}{0.647927in}}%
\pgfpathlineto{\pgfqpoint{4.665054in}{0.647580in}}%
\pgfpathlineto{\pgfqpoint{4.669437in}{0.647233in}}%
\pgfpathlineto{\pgfqpoint{4.673820in}{0.646888in}}%
\pgfpathlineto{\pgfqpoint{4.678202in}{0.646543in}}%
\pgfpathlineto{\pgfqpoint{4.682585in}{0.646199in}}%
\pgfpathlineto{\pgfqpoint{4.686968in}{0.645855in}}%
\pgfpathlineto{\pgfqpoint{4.691350in}{0.645513in}}%
\pgfpathlineto{\pgfqpoint{4.695733in}{0.645170in}}%
\pgfpathlineto{\pgfqpoint{4.700116in}{0.644829in}}%
\pgfpathlineto{\pgfqpoint{4.704499in}{0.644489in}}%
\pgfpathlineto{\pgfqpoint{4.708881in}{0.644149in}}%
\pgfpathlineto{\pgfqpoint{4.713264in}{0.643809in}}%
\pgfpathlineto{\pgfqpoint{4.717647in}{0.643471in}}%
\pgfpathlineto{\pgfqpoint{4.722029in}{0.643133in}}%
\pgfpathlineto{\pgfqpoint{4.726412in}{0.642796in}}%
\pgfpathlineto{\pgfqpoint{4.730795in}{0.642459in}}%
\pgfpathlineto{\pgfqpoint{4.735177in}{0.642124in}}%
\pgfpathlineto{\pgfqpoint{4.739560in}{0.641789in}}%
\pgfpathlineto{\pgfqpoint{4.743943in}{0.641454in}}%
\pgfpathlineto{\pgfqpoint{4.748325in}{0.641121in}}%
\pgfpathlineto{\pgfqpoint{4.752708in}{0.640788in}}%
\pgfpathlineto{\pgfqpoint{4.757091in}{0.640455in}}%
\pgfpathlineto{\pgfqpoint{4.761473in}{0.640124in}}%
\pgfpathlineto{\pgfqpoint{4.765856in}{0.639793in}}%
\pgfpathlineto{\pgfqpoint{4.770239in}{0.639462in}}%
\pgfpathlineto{\pgfqpoint{4.774622in}{0.639133in}}%
\pgfpathlineto{\pgfqpoint{4.779004in}{0.638804in}}%
\pgfpathlineto{\pgfqpoint{4.783387in}{0.638475in}}%
\pgfpathlineto{\pgfqpoint{4.787770in}{0.638148in}}%
\pgfpathlineto{\pgfqpoint{4.792152in}{0.637821in}}%
\pgfpathlineto{\pgfqpoint{4.796535in}{0.637494in}}%
\pgfpathlineto{\pgfqpoint{4.800918in}{0.637169in}}%
\pgfpathlineto{\pgfqpoint{4.805300in}{0.636844in}}%
\pgfpathlineto{\pgfqpoint{4.809683in}{0.636519in}}%
\pgfpathlineto{\pgfqpoint{4.814066in}{0.636196in}}%
\pgfpathlineto{\pgfqpoint{4.818448in}{0.635872in}}%
\pgfpathlineto{\pgfqpoint{4.822831in}{0.635550in}}%
\pgfpathlineto{\pgfqpoint{4.827214in}{0.635228in}}%
\pgfpathlineto{\pgfqpoint{4.831596in}{0.634907in}}%
\pgfpathlineto{\pgfqpoint{4.835979in}{0.634587in}}%
\pgfpathlineto{\pgfqpoint{4.840362in}{0.634267in}}%
\pgfpathlineto{\pgfqpoint{4.844745in}{0.633948in}}%
\pgfpathlineto{\pgfqpoint{4.849127in}{0.633629in}}%
\pgfpathlineto{\pgfqpoint{4.853510in}{0.633311in}}%
\pgfpathlineto{\pgfqpoint{4.857893in}{0.632994in}}%
\pgfpathlineto{\pgfqpoint{4.862275in}{0.632677in}}%
\pgfpathlineto{\pgfqpoint{4.866658in}{0.632361in}}%
\pgfpathlineto{\pgfqpoint{4.871041in}{0.632046in}}%
\pgfpathlineto{\pgfqpoint{4.875423in}{0.631731in}}%
\pgfpathlineto{\pgfqpoint{4.879806in}{0.631417in}}%
\pgfpathlineto{\pgfqpoint{4.884189in}{0.631103in}}%
\pgfpathlineto{\pgfqpoint{4.888571in}{0.630790in}}%
\pgfpathlineto{\pgfqpoint{4.892954in}{0.630478in}}%
\pgfpathlineto{\pgfqpoint{4.897337in}{0.630166in}}%
\pgfpathlineto{\pgfqpoint{4.901719in}{0.629855in}}%
\pgfpathlineto{\pgfqpoint{4.906102in}{0.629545in}}%
\pgfpathlineto{\pgfqpoint{4.910485in}{0.629235in}}%
\pgfpathlineto{\pgfqpoint{4.914868in}{0.628926in}}%
\pgfpathlineto{\pgfqpoint{4.919250in}{0.628617in}}%
\pgfpathlineto{\pgfqpoint{4.923633in}{0.628309in}}%
\pgfpathlineto{\pgfqpoint{4.928016in}{0.628002in}}%
\pgfpathlineto{\pgfqpoint{4.932398in}{0.627695in}}%
\pgfpathlineto{\pgfqpoint{4.936781in}{0.627388in}}%
\pgfpathlineto{\pgfqpoint{4.941164in}{0.627083in}}%
\pgfpathlineto{\pgfqpoint{4.945546in}{0.626778in}}%
\pgfpathlineto{\pgfqpoint{4.949929in}{0.626473in}}%
\pgfpathlineto{\pgfqpoint{4.954312in}{0.626170in}}%
\pgfpathlineto{\pgfqpoint{4.958694in}{0.625866in}}%
\pgfpathlineto{\pgfqpoint{4.963077in}{0.625564in}}%
\pgfpathlineto{\pgfqpoint{4.967460in}{0.625262in}}%
\pgfpathlineto{\pgfqpoint{4.971842in}{0.624960in}}%
\pgfpathlineto{\pgfqpoint{4.976225in}{0.624659in}}%
\pgfpathlineto{\pgfqpoint{4.980608in}{0.624359in}}%
\pgfpathlineto{\pgfqpoint{4.984991in}{0.624059in}}%
\pgfpathlineto{\pgfqpoint{4.989373in}{0.623760in}}%
\pgfpathlineto{\pgfqpoint{4.993756in}{0.623461in}}%
\pgfpathlineto{\pgfqpoint{4.998139in}{0.623163in}}%
\pgfpathlineto{\pgfqpoint{5.002521in}{0.622866in}}%
\pgfpathlineto{\pgfqpoint{5.006904in}{0.622569in}}%
\pgfpathlineto{\pgfqpoint{5.011287in}{0.622273in}}%
\pgfpathlineto{\pgfqpoint{5.015669in}{0.621977in}}%
\pgfpathlineto{\pgfqpoint{5.020052in}{0.621682in}}%
\pgfpathlineto{\pgfqpoint{5.024435in}{0.621387in}}%
\pgfpathlineto{\pgfqpoint{5.028817in}{0.621093in}}%
\pgfpathlineto{\pgfqpoint{5.033200in}{0.620800in}}%
\pgfpathlineto{\pgfqpoint{5.037583in}{0.620507in}}%
\pgfpathlineto{\pgfqpoint{5.041966in}{0.620214in}}%
\pgfpathlineto{\pgfqpoint{5.046348in}{0.619923in}}%
\pgfpathlineto{\pgfqpoint{5.050731in}{0.619631in}}%
\pgfpathlineto{\pgfqpoint{5.055114in}{0.619341in}}%
\pgfpathlineto{\pgfqpoint{5.059496in}{0.619051in}}%
\pgfpathlineto{\pgfqpoint{5.063879in}{0.618761in}}%
\pgfpathlineto{\pgfqpoint{5.068262in}{0.618472in}}%
\pgfpathlineto{\pgfqpoint{5.072644in}{0.618184in}}%
\pgfpathlineto{\pgfqpoint{5.077027in}{0.617896in}}%
\pgfpathlineto{\pgfqpoint{5.081410in}{0.617608in}}%
\pgfpathlineto{\pgfqpoint{5.085792in}{0.617321in}}%
\pgfpathlineto{\pgfqpoint{5.090175in}{0.617035in}}%
\pgfpathlineto{\pgfqpoint{5.094558in}{0.616749in}}%
\pgfpathlineto{\pgfqpoint{5.098940in}{0.616464in}}%
\pgfpathlineto{\pgfqpoint{5.103323in}{0.616180in}}%
\pgfpathlineto{\pgfqpoint{5.107706in}{0.615895in}}%
\pgfpathlineto{\pgfqpoint{5.112089in}{0.615612in}}%
\pgfpathlineto{\pgfqpoint{5.116471in}{0.615329in}}%
\pgfpathlineto{\pgfqpoint{5.120854in}{0.615046in}}%
\pgfpathlineto{\pgfqpoint{5.125237in}{0.614764in}}%
\pgfpathlineto{\pgfqpoint{5.129619in}{0.614483in}}%
\pgfpathlineto{\pgfqpoint{5.134002in}{0.614202in}}%
\pgfpathlineto{\pgfqpoint{5.138385in}{0.613921in}}%
\pgfpathlineto{\pgfqpoint{5.142767in}{0.613641in}}%
\pgfpathlineto{\pgfqpoint{5.147150in}{0.613362in}}%
\pgfpathlineto{\pgfqpoint{5.151533in}{0.613083in}}%
\pgfpathlineto{\pgfqpoint{5.155915in}{0.612805in}}%
\pgfpathlineto{\pgfqpoint{5.160298in}{0.612527in}}%
\pgfpathlineto{\pgfqpoint{5.164681in}{0.612250in}}%
\pgfpathlineto{\pgfqpoint{5.169063in}{0.611973in}}%
\pgfpathlineto{\pgfqpoint{5.173446in}{0.611697in}}%
\pgfpathlineto{\pgfqpoint{5.177829in}{0.611421in}}%
\pgfpathlineto{\pgfqpoint{5.182212in}{0.611146in}}%
\pgfpathlineto{\pgfqpoint{5.186594in}{0.610871in}}%
\pgfpathlineto{\pgfqpoint{5.190977in}{0.610597in}}%
\pgfpathlineto{\pgfqpoint{5.195360in}{0.610323in}}%
\pgfpathlineto{\pgfqpoint{5.199742in}{0.610050in}}%
\pgfpathlineto{\pgfqpoint{5.204125in}{0.609777in}}%
\pgfpathlineto{\pgfqpoint{5.208508in}{0.609505in}}%
\pgfpathlineto{\pgfqpoint{5.212890in}{0.609233in}}%
\pgfpathlineto{\pgfqpoint{5.217273in}{0.608962in}}%
\pgfpathlineto{\pgfqpoint{5.221656in}{0.608691in}}%
\pgfpathlineto{\pgfqpoint{5.226038in}{0.608421in}}%
\pgfpathlineto{\pgfqpoint{5.230421in}{0.608151in}}%
\pgfpathlineto{\pgfqpoint{5.234804in}{0.607882in}}%
\pgfpathlineto{\pgfqpoint{5.239186in}{0.607613in}}%
\pgfpathlineto{\pgfqpoint{5.243569in}{0.607345in}}%
\pgfpathlineto{\pgfqpoint{5.247952in}{0.607077in}}%
\pgfpathlineto{\pgfqpoint{5.252335in}{0.606810in}}%
\pgfpathlineto{\pgfqpoint{5.256717in}{0.606543in}}%
\pgfpathlineto{\pgfqpoint{5.261100in}{0.606277in}}%
\pgfpathlineto{\pgfqpoint{5.261100in}{0.606277in}}%
\pgfpathlineto{\pgfqpoint{5.268339in}{0.605846in}}%
\pgfpathlineto{\pgfqpoint{5.275578in}{0.605432in}}%
\pgfpathlineto{\pgfqpoint{5.282818in}{0.605033in}}%
\pgfpathlineto{\pgfqpoint{5.290057in}{0.604648in}}%
\pgfpathlineto{\pgfqpoint{5.297296in}{0.604277in}}%
\pgfpathlineto{\pgfqpoint{5.304535in}{0.603918in}}%
\pgfpathlineto{\pgfqpoint{5.311774in}{0.603571in}}%
\pgfpathlineto{\pgfqpoint{5.319014in}{0.603235in}}%
\pgfpathlineto{\pgfqpoint{5.326253in}{0.602909in}}%
\pgfpathlineto{\pgfqpoint{5.333492in}{0.602593in}}%
\pgfpathlineto{\pgfqpoint{5.340731in}{0.602286in}}%
\pgfpathlineto{\pgfqpoint{5.347970in}{0.601988in}}%
\pgfpathlineto{\pgfqpoint{5.355210in}{0.601698in}}%
\pgfpathlineto{\pgfqpoint{5.362449in}{0.601416in}}%
\pgfpathlineto{\pgfqpoint{5.369688in}{0.601142in}}%
\pgfpathlineto{\pgfqpoint{5.376927in}{0.600874in}}%
\pgfpathlineto{\pgfqpoint{5.384167in}{0.600613in}}%
\pgfpathlineto{\pgfqpoint{5.391406in}{0.600359in}}%
\pgfpathlineto{\pgfqpoint{5.398645in}{0.600111in}}%
\pgfpathlineto{\pgfqpoint{5.405884in}{0.599869in}}%
\pgfpathlineto{\pgfqpoint{5.413123in}{0.599633in}}%
\pgfpathlineto{\pgfqpoint{5.420363in}{0.599402in}}%
\pgfpathlineto{\pgfqpoint{5.427602in}{0.599176in}}%
\pgfpathlineto{\pgfqpoint{5.434841in}{0.598955in}}%
\pgfpathlineto{\pgfqpoint{5.442080in}{0.598739in}}%
\pgfpathlineto{\pgfqpoint{5.449319in}{0.598527in}}%
\pgfpathlineto{\pgfqpoint{5.456559in}{0.598320in}}%
\pgfpathlineto{\pgfqpoint{5.463798in}{0.598118in}}%
\pgfpathlineto{\pgfqpoint{5.471037in}{0.597919in}}%
\pgfpathlineto{\pgfqpoint{5.478276in}{0.597724in}}%
\pgfpathlineto{\pgfqpoint{5.485515in}{0.597533in}}%
\pgfpathlineto{\pgfqpoint{5.492755in}{0.597346in}}%
\pgfpathlineto{\pgfqpoint{5.499994in}{0.597163in}}%
\pgfpathlineto{\pgfqpoint{5.507233in}{0.596982in}}%
\pgfpathlineto{\pgfqpoint{5.514472in}{0.596806in}}%
\pgfpathlineto{\pgfqpoint{5.521712in}{0.596632in}}%
\pgfpathlineto{\pgfqpoint{5.528951in}{0.596462in}}%
\pgfpathlineto{\pgfqpoint{5.536190in}{0.596294in}}%
\pgfpathlineto{\pgfqpoint{5.543429in}{0.596130in}}%
\pgfpathlineto{\pgfqpoint{5.550668in}{0.595968in}}%
\pgfpathlineto{\pgfqpoint{5.557908in}{0.595809in}}%
\pgfpathlineto{\pgfqpoint{5.565147in}{0.595653in}}%
\pgfpathlineto{\pgfqpoint{5.572386in}{0.595499in}}%
\pgfpathlineto{\pgfqpoint{5.579625in}{0.595348in}}%
\pgfpathlineto{\pgfqpoint{5.586864in}{0.595200in}}%
\pgfpathlineto{\pgfqpoint{5.594104in}{0.595053in}}%
\pgfpathlineto{\pgfqpoint{5.601343in}{0.594910in}}%
\pgfpathlineto{\pgfqpoint{5.608582in}{0.594768in}}%
\pgfpathlineto{\pgfqpoint{5.615821in}{0.594628in}}%
\pgfpathlineto{\pgfqpoint{5.623061in}{0.594491in}}%
\pgfpathlineto{\pgfqpoint{5.630300in}{0.594356in}}%
\pgfpathlineto{\pgfqpoint{5.637539in}{0.594223in}}%
\pgfpathlineto{\pgfqpoint{5.644778in}{0.594092in}}%
\pgfpathlineto{\pgfqpoint{5.652017in}{0.593962in}}%
\pgfpathlineto{\pgfqpoint{5.659257in}{0.593835in}}%
\pgfpathlineto{\pgfqpoint{5.666496in}{0.593709in}}%
\pgfpathlineto{\pgfqpoint{5.673735in}{0.593586in}}%
\pgfpathlineto{\pgfqpoint{5.680974in}{0.593464in}}%
\pgfpathlineto{\pgfqpoint{5.688213in}{0.593343in}}%
\pgfpathlineto{\pgfqpoint{5.695453in}{0.593225in}}%
\pgfpathlineto{\pgfqpoint{5.702692in}{0.593108in}}%
\pgfpathlineto{\pgfqpoint{5.709931in}{0.592992in}}%
\pgfpathlineto{\pgfqpoint{5.717170in}{0.592878in}}%
\pgfpathlineto{\pgfqpoint{5.724409in}{0.592766in}}%
\pgfpathlineto{\pgfqpoint{5.731649in}{0.592655in}}%
\pgfpathlineto{\pgfqpoint{5.738888in}{0.592546in}}%
\pgfpathlineto{\pgfqpoint{5.746127in}{0.592438in}}%
\pgfpathlineto{\pgfqpoint{5.753366in}{0.592331in}}%
\pgfpathlineto{\pgfqpoint{5.760606in}{0.592226in}}%
\pgfpathlineto{\pgfqpoint{5.767845in}{0.592122in}}%
\pgfpathlineto{\pgfqpoint{5.775084in}{0.592019in}}%
\pgfpathlineto{\pgfqpoint{5.782323in}{0.591918in}}%
\pgfpathlineto{\pgfqpoint{5.789562in}{0.591818in}}%
\pgfpathlineto{\pgfqpoint{5.796802in}{0.591719in}}%
\pgfpathlineto{\pgfqpoint{5.804041in}{0.591621in}}%
\pgfpathlineto{\pgfqpoint{5.811280in}{0.591525in}}%
\pgfpathlineto{\pgfqpoint{5.818519in}{0.591429in}}%
\pgfpathlineto{\pgfqpoint{5.825758in}{0.591335in}}%
\pgfpathlineto{\pgfqpoint{5.832998in}{0.591242in}}%
\pgfpathlineto{\pgfqpoint{5.840237in}{0.591150in}}%
\pgfpathlineto{\pgfqpoint{5.847476in}{0.591059in}}%
\pgfpathlineto{\pgfqpoint{5.854715in}{0.590969in}}%
\pgfpathlineto{\pgfqpoint{5.861954in}{0.590880in}}%
\pgfpathlineto{\pgfqpoint{5.869194in}{0.590792in}}%
\pgfpathlineto{\pgfqpoint{5.876433in}{0.590705in}}%
\pgfpathlineto{\pgfqpoint{5.883672in}{0.590619in}}%
\pgfpathlineto{\pgfqpoint{5.890911in}{0.590534in}}%
\pgfpathlineto{\pgfqpoint{5.898151in}{0.590450in}}%
\pgfpathlineto{\pgfqpoint{5.905390in}{0.590367in}}%
\pgfpathlineto{\pgfqpoint{5.912629in}{0.590285in}}%
\pgfpathlineto{\pgfqpoint{5.919868in}{0.590204in}}%
\pgfpathlineto{\pgfqpoint{5.927107in}{0.590123in}}%
\pgfpathlineto{\pgfqpoint{5.934347in}{0.590044in}}%
\pgfpathlineto{\pgfqpoint{5.941586in}{0.589965in}}%
\pgfpathlineto{\pgfqpoint{5.948825in}{0.589887in}}%
\pgfpathlineto{\pgfqpoint{5.956064in}{0.589810in}}%
\pgfpathlineto{\pgfqpoint{5.963303in}{0.589733in}}%
\pgfpathlineto{\pgfqpoint{5.970543in}{0.589658in}}%
\pgfpathlineto{\pgfqpoint{5.977782in}{0.589583in}}%
\pgfpathlineto{\pgfqpoint{7.082794in}{0.644623in}}%
\pgfpathlineto{\pgfqpoint{7.074107in}{0.644713in}}%
\pgfpathlineto{\pgfqpoint{7.065420in}{0.644803in}}%
\pgfpathlineto{\pgfqpoint{7.056733in}{0.644895in}}%
\pgfpathlineto{\pgfqpoint{7.048046in}{0.644988in}}%
\pgfpathlineto{\pgfqpoint{7.039359in}{0.645081in}}%
\pgfpathlineto{\pgfqpoint{7.030672in}{0.645176in}}%
\pgfpathlineto{\pgfqpoint{7.021985in}{0.645271in}}%
\pgfpathlineto{\pgfqpoint{7.013298in}{0.645368in}}%
\pgfpathlineto{\pgfqpoint{7.004611in}{0.645465in}}%
\pgfpathlineto{\pgfqpoint{6.995924in}{0.645564in}}%
\pgfpathlineto{\pgfqpoint{6.987237in}{0.645664in}}%
\pgfpathlineto{\pgfqpoint{6.978550in}{0.645765in}}%
\pgfpathlineto{\pgfqpoint{6.969863in}{0.645867in}}%
\pgfpathlineto{\pgfqpoint{6.961176in}{0.645970in}}%
\pgfpathlineto{\pgfqpoint{6.952488in}{0.646074in}}%
\pgfpathlineto{\pgfqpoint{6.943801in}{0.646179in}}%
\pgfpathlineto{\pgfqpoint{6.935114in}{0.646286in}}%
\pgfpathlineto{\pgfqpoint{6.926427in}{0.646394in}}%
\pgfpathlineto{\pgfqpoint{6.917740in}{0.646503in}}%
\pgfpathlineto{\pgfqpoint{6.909053in}{0.646614in}}%
\pgfpathlineto{\pgfqpoint{6.900366in}{0.646725in}}%
\pgfpathlineto{\pgfqpoint{6.891679in}{0.646838in}}%
\pgfpathlineto{\pgfqpoint{6.882992in}{0.646953in}}%
\pgfpathlineto{\pgfqpoint{6.874305in}{0.647069in}}%
\pgfpathlineto{\pgfqpoint{6.865618in}{0.647186in}}%
\pgfpathlineto{\pgfqpoint{6.856931in}{0.647305in}}%
\pgfpathlineto{\pgfqpoint{6.848244in}{0.647425in}}%
\pgfpathlineto{\pgfqpoint{6.839557in}{0.647546in}}%
\pgfpathlineto{\pgfqpoint{6.830870in}{0.647669in}}%
\pgfpathlineto{\pgfqpoint{6.822183in}{0.647794in}}%
\pgfpathlineto{\pgfqpoint{6.813496in}{0.647920in}}%
\pgfpathlineto{\pgfqpoint{6.804809in}{0.648048in}}%
\pgfpathlineto{\pgfqpoint{6.796122in}{0.648178in}}%
\pgfpathlineto{\pgfqpoint{6.787434in}{0.648309in}}%
\pgfpathlineto{\pgfqpoint{6.778747in}{0.648442in}}%
\pgfpathlineto{\pgfqpoint{6.770060in}{0.648577in}}%
\pgfpathlineto{\pgfqpoint{6.761373in}{0.648714in}}%
\pgfpathlineto{\pgfqpoint{6.752686in}{0.648852in}}%
\pgfpathlineto{\pgfqpoint{6.743999in}{0.648993in}}%
\pgfpathlineto{\pgfqpoint{6.735312in}{0.649135in}}%
\pgfpathlineto{\pgfqpoint{6.726625in}{0.649280in}}%
\pgfpathlineto{\pgfqpoint{6.717938in}{0.649426in}}%
\pgfpathlineto{\pgfqpoint{6.709251in}{0.649575in}}%
\pgfpathlineto{\pgfqpoint{6.700564in}{0.649725in}}%
\pgfpathlineto{\pgfqpoint{6.691877in}{0.649878in}}%
\pgfpathlineto{\pgfqpoint{6.683190in}{0.650033in}}%
\pgfpathlineto{\pgfqpoint{6.674503in}{0.650191in}}%
\pgfpathlineto{\pgfqpoint{6.665816in}{0.650350in}}%
\pgfpathlineto{\pgfqpoint{6.657129in}{0.650513in}}%
\pgfpathlineto{\pgfqpoint{6.648442in}{0.650677in}}%
\pgfpathlineto{\pgfqpoint{6.639755in}{0.650845in}}%
\pgfpathlineto{\pgfqpoint{6.631067in}{0.651015in}}%
\pgfpathlineto{\pgfqpoint{6.622380in}{0.651187in}}%
\pgfpathlineto{\pgfqpoint{6.613693in}{0.651363in}}%
\pgfpathlineto{\pgfqpoint{6.605006in}{0.651541in}}%
\pgfpathlineto{\pgfqpoint{6.596319in}{0.651723in}}%
\pgfpathlineto{\pgfqpoint{6.587632in}{0.651907in}}%
\pgfpathlineto{\pgfqpoint{6.578945in}{0.652094in}}%
\pgfpathlineto{\pgfqpoint{6.570258in}{0.652285in}}%
\pgfpathlineto{\pgfqpoint{6.561571in}{0.652479in}}%
\pgfpathlineto{\pgfqpoint{6.552884in}{0.652676in}}%
\pgfpathlineto{\pgfqpoint{6.544197in}{0.652877in}}%
\pgfpathlineto{\pgfqpoint{6.535510in}{0.653082in}}%
\pgfpathlineto{\pgfqpoint{6.526823in}{0.653290in}}%
\pgfpathlineto{\pgfqpoint{6.518136in}{0.653502in}}%
\pgfpathlineto{\pgfqpoint{6.509449in}{0.653718in}}%
\pgfpathlineto{\pgfqpoint{6.500762in}{0.653939in}}%
\pgfpathlineto{\pgfqpoint{6.492075in}{0.654163in}}%
\pgfpathlineto{\pgfqpoint{6.483388in}{0.654392in}}%
\pgfpathlineto{\pgfqpoint{6.474701in}{0.654626in}}%
\pgfpathlineto{\pgfqpoint{6.466013in}{0.654864in}}%
\pgfpathlineto{\pgfqpoint{6.457326in}{0.655108in}}%
\pgfpathlineto{\pgfqpoint{6.448639in}{0.655356in}}%
\pgfpathlineto{\pgfqpoint{6.439952in}{0.655610in}}%
\pgfpathlineto{\pgfqpoint{6.431265in}{0.655869in}}%
\pgfpathlineto{\pgfqpoint{6.422578in}{0.656134in}}%
\pgfpathlineto{\pgfqpoint{6.413891in}{0.656405in}}%
\pgfpathlineto{\pgfqpoint{6.405204in}{0.656682in}}%
\pgfpathlineto{\pgfqpoint{6.396517in}{0.656966in}}%
\pgfpathlineto{\pgfqpoint{6.387830in}{0.657257in}}%
\pgfpathlineto{\pgfqpoint{6.379143in}{0.657554in}}%
\pgfpathlineto{\pgfqpoint{6.370456in}{0.657859in}}%
\pgfpathlineto{\pgfqpoint{6.361769in}{0.658172in}}%
\pgfpathlineto{\pgfqpoint{6.353082in}{0.658493in}}%
\pgfpathlineto{\pgfqpoint{6.344395in}{0.658823in}}%
\pgfpathlineto{\pgfqpoint{6.335708in}{0.659161in}}%
\pgfpathlineto{\pgfqpoint{6.327021in}{0.659509in}}%
\pgfpathlineto{\pgfqpoint{6.318334in}{0.659867in}}%
\pgfpathlineto{\pgfqpoint{6.309647in}{0.660235in}}%
\pgfpathlineto{\pgfqpoint{6.300959in}{0.660614in}}%
\pgfpathlineto{\pgfqpoint{6.292272in}{0.661005in}}%
\pgfpathlineto{\pgfqpoint{6.283585in}{0.661409in}}%
\pgfpathlineto{\pgfqpoint{6.274898in}{0.661825in}}%
\pgfpathlineto{\pgfqpoint{6.266211in}{0.662256in}}%
\pgfpathlineto{\pgfqpoint{6.257524in}{0.662701in}}%
\pgfpathlineto{\pgfqpoint{6.248837in}{0.663163in}}%
\pgfpathlineto{\pgfqpoint{6.240150in}{0.663641in}}%
\pgfpathlineto{\pgfqpoint{6.231463in}{0.664138in}}%
\pgfpathlineto{\pgfqpoint{6.222776in}{0.664655in}}%
\pgfpathlineto{\pgfqpoint{6.222776in}{0.664655in}}%
\pgfpathlineto{\pgfqpoint{6.217517in}{0.664975in}}%
\pgfpathlineto{\pgfqpoint{6.212258in}{0.665295in}}%
\pgfpathlineto{\pgfqpoint{6.206998in}{0.665616in}}%
\pgfpathlineto{\pgfqpoint{6.201739in}{0.665937in}}%
\pgfpathlineto{\pgfqpoint{6.196480in}{0.666259in}}%
\pgfpathlineto{\pgfqpoint{6.191221in}{0.666581in}}%
\pgfpathlineto{\pgfqpoint{6.185961in}{0.666905in}}%
\pgfpathlineto{\pgfqpoint{6.180702in}{0.667228in}}%
\pgfpathlineto{\pgfqpoint{6.175443in}{0.667552in}}%
\pgfpathlineto{\pgfqpoint{6.170184in}{0.667877in}}%
\pgfpathlineto{\pgfqpoint{6.164924in}{0.668203in}}%
\pgfpathlineto{\pgfqpoint{6.159665in}{0.668529in}}%
\pgfpathlineto{\pgfqpoint{6.154406in}{0.668856in}}%
\pgfpathlineto{\pgfqpoint{6.149147in}{0.669183in}}%
\pgfpathlineto{\pgfqpoint{6.143888in}{0.669511in}}%
\pgfpathlineto{\pgfqpoint{6.138628in}{0.669839in}}%
\pgfpathlineto{\pgfqpoint{6.133369in}{0.670168in}}%
\pgfpathlineto{\pgfqpoint{6.128110in}{0.670498in}}%
\pgfpathlineto{\pgfqpoint{6.122851in}{0.670828in}}%
\pgfpathlineto{\pgfqpoint{6.117591in}{0.671159in}}%
\pgfpathlineto{\pgfqpoint{6.112332in}{0.671491in}}%
\pgfpathlineto{\pgfqpoint{6.107073in}{0.671823in}}%
\pgfpathlineto{\pgfqpoint{6.101814in}{0.672156in}}%
\pgfpathlineto{\pgfqpoint{6.096555in}{0.672489in}}%
\pgfpathlineto{\pgfqpoint{6.091295in}{0.672823in}}%
\pgfpathlineto{\pgfqpoint{6.086036in}{0.673158in}}%
\pgfpathlineto{\pgfqpoint{6.080777in}{0.673493in}}%
\pgfpathlineto{\pgfqpoint{6.075518in}{0.673829in}}%
\pgfpathlineto{\pgfqpoint{6.070258in}{0.674165in}}%
\pgfpathlineto{\pgfqpoint{6.064999in}{0.674502in}}%
\pgfpathlineto{\pgfqpoint{6.059740in}{0.674840in}}%
\pgfpathlineto{\pgfqpoint{6.054481in}{0.675179in}}%
\pgfpathlineto{\pgfqpoint{6.049222in}{0.675518in}}%
\pgfpathlineto{\pgfqpoint{6.043962in}{0.675857in}}%
\pgfpathlineto{\pgfqpoint{6.038703in}{0.676198in}}%
\pgfpathlineto{\pgfqpoint{6.033444in}{0.676539in}}%
\pgfpathlineto{\pgfqpoint{6.028185in}{0.676880in}}%
\pgfpathlineto{\pgfqpoint{6.022925in}{0.677223in}}%
\pgfpathlineto{\pgfqpoint{6.017666in}{0.677566in}}%
\pgfpathlineto{\pgfqpoint{6.012407in}{0.677909in}}%
\pgfpathlineto{\pgfqpoint{6.007148in}{0.678253in}}%
\pgfpathlineto{\pgfqpoint{6.001888in}{0.678598in}}%
\pgfpathlineto{\pgfqpoint{5.996629in}{0.678944in}}%
\pgfpathlineto{\pgfqpoint{5.991370in}{0.679290in}}%
\pgfpathlineto{\pgfqpoint{5.986111in}{0.679637in}}%
\pgfpathlineto{\pgfqpoint{5.980852in}{0.679984in}}%
\pgfpathlineto{\pgfqpoint{5.975592in}{0.680332in}}%
\pgfpathlineto{\pgfqpoint{5.970333in}{0.680681in}}%
\pgfpathlineto{\pgfqpoint{5.965074in}{0.681031in}}%
\pgfpathlineto{\pgfqpoint{5.959815in}{0.681381in}}%
\pgfpathlineto{\pgfqpoint{5.954555in}{0.681732in}}%
\pgfpathlineto{\pgfqpoint{5.949296in}{0.682083in}}%
\pgfpathlineto{\pgfqpoint{5.944037in}{0.682435in}}%
\pgfpathlineto{\pgfqpoint{5.938778in}{0.682788in}}%
\pgfpathlineto{\pgfqpoint{5.933519in}{0.683142in}}%
\pgfpathlineto{\pgfqpoint{5.928259in}{0.683496in}}%
\pgfpathlineto{\pgfqpoint{5.923000in}{0.683851in}}%
\pgfpathlineto{\pgfqpoint{5.917741in}{0.684206in}}%
\pgfpathlineto{\pgfqpoint{5.912482in}{0.684562in}}%
\pgfpathlineto{\pgfqpoint{5.907222in}{0.684919in}}%
\pgfpathlineto{\pgfqpoint{5.901963in}{0.685277in}}%
\pgfpathlineto{\pgfqpoint{5.896704in}{0.685635in}}%
\pgfpathlineto{\pgfqpoint{5.891445in}{0.685994in}}%
\pgfpathlineto{\pgfqpoint{5.886186in}{0.686354in}}%
\pgfpathlineto{\pgfqpoint{5.880926in}{0.686714in}}%
\pgfpathlineto{\pgfqpoint{5.875667in}{0.687075in}}%
\pgfpathlineto{\pgfqpoint{5.870408in}{0.687437in}}%
\pgfpathlineto{\pgfqpoint{5.865149in}{0.687800in}}%
\pgfpathlineto{\pgfqpoint{5.859889in}{0.688163in}}%
\pgfpathlineto{\pgfqpoint{5.854630in}{0.688527in}}%
\pgfpathlineto{\pgfqpoint{5.849371in}{0.688891in}}%
\pgfpathlineto{\pgfqpoint{5.844112in}{0.689257in}}%
\pgfpathlineto{\pgfqpoint{5.838852in}{0.689623in}}%
\pgfpathlineto{\pgfqpoint{5.833593in}{0.689990in}}%
\pgfpathlineto{\pgfqpoint{5.828334in}{0.690357in}}%
\pgfpathlineto{\pgfqpoint{5.823075in}{0.690725in}}%
\pgfpathlineto{\pgfqpoint{5.817816in}{0.691094in}}%
\pgfpathlineto{\pgfqpoint{5.812556in}{0.691464in}}%
\pgfpathlineto{\pgfqpoint{5.807297in}{0.691834in}}%
\pgfpathlineto{\pgfqpoint{5.802038in}{0.692205in}}%
\pgfpathlineto{\pgfqpoint{5.796779in}{0.692577in}}%
\pgfpathlineto{\pgfqpoint{5.791519in}{0.692950in}}%
\pgfpathlineto{\pgfqpoint{5.786260in}{0.693323in}}%
\pgfpathlineto{\pgfqpoint{5.781001in}{0.693697in}}%
\pgfpathlineto{\pgfqpoint{5.775742in}{0.694072in}}%
\pgfpathlineto{\pgfqpoint{5.770483in}{0.694447in}}%
\pgfpathlineto{\pgfqpoint{5.765223in}{0.694823in}}%
\pgfpathlineto{\pgfqpoint{5.759964in}{0.695200in}}%
\pgfpathlineto{\pgfqpoint{5.754705in}{0.695578in}}%
\pgfpathlineto{\pgfqpoint{5.749446in}{0.695957in}}%
\pgfpathlineto{\pgfqpoint{5.744186in}{0.696336in}}%
\pgfpathlineto{\pgfqpoint{5.738927in}{0.696716in}}%
\pgfpathlineto{\pgfqpoint{5.733668in}{0.697097in}}%
\pgfpathlineto{\pgfqpoint{5.728409in}{0.697478in}}%
\pgfpathlineto{\pgfqpoint{5.723149in}{0.697860in}}%
\pgfpathlineto{\pgfqpoint{5.717890in}{0.698244in}}%
\pgfpathlineto{\pgfqpoint{5.712631in}{0.698627in}}%
\pgfpathlineto{\pgfqpoint{5.707372in}{0.699012in}}%
\pgfpathlineto{\pgfqpoint{5.702113in}{0.699397in}}%
\pgfpathlineto{\pgfqpoint{5.696853in}{0.699783in}}%
\pgfpathlineto{\pgfqpoint{5.691594in}{0.700170in}}%
\pgfpathlineto{\pgfqpoint{5.686335in}{0.700558in}}%
\pgfpathlineto{\pgfqpoint{5.681076in}{0.700946in}}%
\pgfpathlineto{\pgfqpoint{5.675816in}{0.701336in}}%
\pgfpathlineto{\pgfqpoint{5.670557in}{0.701726in}}%
\pgfpathlineto{\pgfqpoint{5.665298in}{0.702116in}}%
\pgfpathlineto{\pgfqpoint{5.660039in}{0.702508in}}%
\pgfpathlineto{\pgfqpoint{5.654780in}{0.702900in}}%
\pgfpathlineto{\pgfqpoint{5.649520in}{0.703294in}}%
\pgfpathlineto{\pgfqpoint{5.644261in}{0.703688in}}%
\pgfpathlineto{\pgfqpoint{5.639002in}{0.704082in}}%
\pgfpathlineto{\pgfqpoint{5.633743in}{0.704478in}}%
\pgfpathlineto{\pgfqpoint{5.628483in}{0.704874in}}%
\pgfpathlineto{\pgfqpoint{5.623224in}{0.705272in}}%
\pgfpathlineto{\pgfqpoint{5.617965in}{0.705670in}}%
\pgfpathlineto{\pgfqpoint{5.612706in}{0.706068in}}%
\pgfpathlineto{\pgfqpoint{5.607447in}{0.706468in}}%
\pgfpathlineto{\pgfqpoint{5.602187in}{0.706868in}}%
\pgfpathlineto{\pgfqpoint{5.596928in}{0.707270in}}%
\pgfpathlineto{\pgfqpoint{5.591669in}{0.707672in}}%
\pgfpathlineto{\pgfqpoint{5.586410in}{0.708075in}}%
\pgfpathlineto{\pgfqpoint{5.581150in}{0.708478in}}%
\pgfpathlineto{\pgfqpoint{5.575891in}{0.708883in}}%
\pgfpathlineto{\pgfqpoint{5.570632in}{0.709288in}}%
\pgfpathlineto{\pgfqpoint{5.565373in}{0.709695in}}%
\pgfpathlineto{\pgfqpoint{5.560113in}{0.710102in}}%
\pgfpathlineto{\pgfqpoint{5.554854in}{0.710510in}}%
\pgfpathlineto{\pgfqpoint{5.549595in}{0.710918in}}%
\pgfpathlineto{\pgfqpoint{5.544336in}{0.711328in}}%
\pgfpathlineto{\pgfqpoint{5.539077in}{0.711738in}}%
\pgfpathlineto{\pgfqpoint{5.533817in}{0.712150in}}%
\pgfpathlineto{\pgfqpoint{5.528558in}{0.712562in}}%
\pgfpathlineto{\pgfqpoint{5.523299in}{0.712975in}}%
\pgfpathlineto{\pgfqpoint{5.518040in}{0.713389in}}%
\pgfpathlineto{\pgfqpoint{5.512780in}{0.713803in}}%
\pgfpathlineto{\pgfqpoint{5.507521in}{0.714219in}}%
\pgfpathlineto{\pgfqpoint{5.502262in}{0.714635in}}%
\pgfpathlineto{\pgfqpoint{5.497003in}{0.715053in}}%
\pgfpathlineto{\pgfqpoint{5.491744in}{0.715471in}}%
\pgfpathlineto{\pgfqpoint{5.486484in}{0.715890in}}%
\pgfpathlineto{\pgfqpoint{5.481225in}{0.716310in}}%
\pgfpathlineto{\pgfqpoint{5.475966in}{0.716731in}}%
\pgfpathlineto{\pgfqpoint{5.470707in}{0.717152in}}%
\pgfpathlineto{\pgfqpoint{5.465447in}{0.717575in}}%
\pgfpathlineto{\pgfqpoint{5.460188in}{0.717998in}}%
\pgfpathlineto{\pgfqpoint{5.454929in}{0.718423in}}%
\pgfpathlineto{\pgfqpoint{5.449670in}{0.718848in}}%
\pgfpathlineto{\pgfqpoint{5.444411in}{0.719274in}}%
\pgfpathlineto{\pgfqpoint{5.439151in}{0.719701in}}%
\pgfpathlineto{\pgfqpoint{5.433892in}{0.720129in}}%
\pgfpathlineto{\pgfqpoint{5.428633in}{0.720558in}}%
\pgfpathlineto{\pgfqpoint{5.423374in}{0.720988in}}%
\pgfpathlineto{\pgfqpoint{5.418114in}{0.721418in}}%
\pgfpathlineto{\pgfqpoint{5.412855in}{0.721850in}}%
\pgfpathlineto{\pgfqpoint{5.407596in}{0.722282in}}%
\pgfpathlineto{\pgfqpoint{5.402337in}{0.722716in}}%
\pgfpathlineto{\pgfqpoint{5.397077in}{0.723150in}}%
\pgfpathlineto{\pgfqpoint{5.391818in}{0.723585in}}%
\pgfpathlineto{\pgfqpoint{5.386559in}{0.724022in}}%
\pgfpathlineto{\pgfqpoint{5.381300in}{0.724459in}}%
\pgfpathlineto{\pgfqpoint{5.376041in}{0.724897in}}%
\pgfpathlineto{\pgfqpoint{5.370781in}{0.725336in}}%
\pgfpathlineto{\pgfqpoint{5.365522in}{0.725776in}}%
\pgfpathlineto{\pgfqpoint{5.360263in}{0.726217in}}%
\pgfpathlineto{\pgfqpoint{5.355004in}{0.726658in}}%
\pgfpathlineto{\pgfqpoint{5.349744in}{0.727101in}}%
\pgfpathlineto{\pgfqpoint{5.344485in}{0.727545in}}%
\pgfpathlineto{\pgfqpoint{5.339226in}{0.727990in}}%
\pgfpathlineto{\pgfqpoint{5.333967in}{0.728435in}}%
\pgfpathlineto{\pgfqpoint{5.328708in}{0.728882in}}%
\pgfpathlineto{\pgfqpoint{5.323448in}{0.729329in}}%
\pgfpathlineto{\pgfqpoint{5.318189in}{0.729778in}}%
\pgfpathlineto{\pgfqpoint{5.312930in}{0.730227in}}%
\pgfpathlineto{\pgfqpoint{5.307671in}{0.730678in}}%
\pgfpathlineto{\pgfqpoint{5.302411in}{0.731129in}}%
\pgfpathlineto{\pgfqpoint{5.297152in}{0.731582in}}%
\pgfpathlineto{\pgfqpoint{5.291893in}{0.732035in}}%
\pgfpathlineto{\pgfqpoint{5.286634in}{0.732490in}}%
\pgfpathlineto{\pgfqpoint{5.281375in}{0.732945in}}%
\pgfpathlineto{\pgfqpoint{5.276115in}{0.733401in}}%
\pgfpathlineto{\pgfqpoint{5.270856in}{0.733859in}}%
\pgfpathlineto{\pgfqpoint{5.265597in}{0.734317in}}%
\pgfpathlineto{\pgfqpoint{5.260338in}{0.734776in}}%
\pgfpathlineto{\pgfqpoint{5.255078in}{0.735237in}}%
\pgfpathlineto{\pgfqpoint{5.249819in}{0.735698in}}%
\pgfpathlineto{\pgfqpoint{5.244560in}{0.736161in}}%
\pgfpathlineto{\pgfqpoint{5.239301in}{0.736624in}}%
\pgfpathlineto{\pgfqpoint{5.234041in}{0.737088in}}%
\pgfpathlineto{\pgfqpoint{5.228782in}{0.737554in}}%
\pgfpathlineto{\pgfqpoint{5.223523in}{0.738020in}}%
\pgfpathlineto{\pgfqpoint{5.218264in}{0.738488in}}%
\pgfpathlineto{\pgfqpoint{5.213005in}{0.738957in}}%
\pgfpathlineto{\pgfqpoint{5.207745in}{0.739426in}}%
\pgfpathlineto{\pgfqpoint{5.202486in}{0.739897in}}%
\pgfpathlineto{\pgfqpoint{5.197227in}{0.740368in}}%
\pgfpathlineto{\pgfqpoint{5.191968in}{0.740841in}}%
\pgfpathlineto{\pgfqpoint{5.186708in}{0.741315in}}%
\pgfpathlineto{\pgfqpoint{5.181449in}{0.741790in}}%
\pgfpathlineto{\pgfqpoint{5.176190in}{0.742266in}}%
\pgfpathlineto{\pgfqpoint{5.170931in}{0.742743in}}%
\pgfpathlineto{\pgfqpoint{5.165672in}{0.743221in}}%
\pgfpathlineto{\pgfqpoint{5.160412in}{0.743700in}}%
\pgfpathlineto{\pgfqpoint{5.155153in}{0.744180in}}%
\pgfpathlineto{\pgfqpoint{5.149894in}{0.744661in}}%
\pgfpathlineto{\pgfqpoint{5.144635in}{0.745143in}}%
\pgfpathlineto{\pgfqpoint{5.139375in}{0.745627in}}%
\pgfpathlineto{\pgfqpoint{5.134116in}{0.746111in}}%
\pgfpathlineto{\pgfqpoint{5.128857in}{0.746597in}}%
\pgfpathlineto{\pgfqpoint{5.123598in}{0.747083in}}%
\pgfpathlineto{\pgfqpoint{5.118339in}{0.747571in}}%
\pgfpathlineto{\pgfqpoint{5.113079in}{0.748060in}}%
\pgfpathlineto{\pgfqpoint{5.107820in}{0.748550in}}%
\pgfpathlineto{\pgfqpoint{5.102561in}{0.749041in}}%
\pgfpathlineto{\pgfqpoint{5.097302in}{0.749533in}}%
\pgfpathlineto{\pgfqpoint{5.092042in}{0.750027in}}%
\pgfpathlineto{\pgfqpoint{5.086783in}{0.750521in}}%
\pgfpathlineto{\pgfqpoint{5.081524in}{0.751016in}}%
\pgfpathlineto{\pgfqpoint{5.076265in}{0.751513in}}%
\pgfpathlineto{\pgfqpoint{5.071005in}{0.752011in}}%
\pgfpathlineto{\pgfqpoint{5.065746in}{0.752510in}}%
\pgfpathlineto{\pgfqpoint{5.060487in}{0.753010in}}%
\pgfpathlineto{\pgfqpoint{5.055228in}{0.753511in}}%
\pgfpathlineto{\pgfqpoint{5.049969in}{0.754014in}}%
\pgfpathlineto{\pgfqpoint{5.044709in}{0.754517in}}%
\pgfpathlineto{\pgfqpoint{5.039450in}{0.755022in}}%
\pgfpathlineto{\pgfqpoint{5.034191in}{0.755528in}}%
\pgfpathlineto{\pgfqpoint{5.028932in}{0.756034in}}%
\pgfpathlineto{\pgfqpoint{5.023672in}{0.756543in}}%
\pgfpathlineto{\pgfqpoint{5.018413in}{0.757052in}}%
\pgfpathlineto{\pgfqpoint{5.013154in}{0.757562in}}%
\pgfpathlineto{\pgfqpoint{5.007895in}{0.758074in}}%
\pgfpathlineto{\pgfqpoint{5.002636in}{0.758587in}}%
\pgfpathlineto{\pgfqpoint{4.997376in}{0.759101in}}%
\pgfpathlineto{\pgfqpoint{4.992117in}{0.759616in}}%
\pgfpathlineto{\pgfqpoint{4.986858in}{0.760133in}}%
\pgfpathlineto{\pgfqpoint{4.981599in}{0.760650in}}%
\pgfpathlineto{\pgfqpoint{4.976339in}{0.761169in}}%
\pgfpathlineto{\pgfqpoint{4.971080in}{0.761689in}}%
\pgfpathlineto{\pgfqpoint{4.965821in}{0.762210in}}%
\pgfpathlineto{\pgfqpoint{4.960562in}{0.762733in}}%
\pgfpathlineto{\pgfqpoint{4.955303in}{0.763257in}}%
\pgfpathlineto{\pgfqpoint{4.950043in}{0.763782in}}%
\pgfpathlineto{\pgfqpoint{4.944784in}{0.764308in}}%
\pgfpathlineto{\pgfqpoint{4.939525in}{0.764835in}}%
\pgfpathlineto{\pgfqpoint{4.934266in}{0.765364in}}%
\pgfpathlineto{\pgfqpoint{4.929006in}{0.765894in}}%
\pgfpathlineto{\pgfqpoint{4.923747in}{0.766425in}}%
\pgfpathlineto{\pgfqpoint{4.918488in}{0.766957in}}%
\pgfpathlineto{\pgfqpoint{4.913229in}{0.767491in}}%
\pgfpathlineto{\pgfqpoint{4.907969in}{0.768026in}}%
\pgfpathlineto{\pgfqpoint{4.902710in}{0.768562in}}%
\pgfpathlineto{\pgfqpoint{4.897451in}{0.769099in}}%
\pgfpathlineto{\pgfqpoint{4.892192in}{0.769638in}}%
\pgfpathlineto{\pgfqpoint{4.886933in}{0.770178in}}%
\pgfpathlineto{\pgfqpoint{4.881673in}{0.770719in}}%
\pgfpathlineto{\pgfqpoint{4.876414in}{0.771261in}}%
\pgfpathlineto{\pgfqpoint{4.871155in}{0.771805in}}%
\pgfpathlineto{\pgfqpoint{4.865896in}{0.772350in}}%
\pgfpathlineto{\pgfqpoint{4.860636in}{0.772897in}}%
\pgfpathlineto{\pgfqpoint{4.855377in}{0.773445in}}%
\pgfpathlineto{\pgfqpoint{4.850118in}{0.773994in}}%
\pgfpathlineto{\pgfqpoint{4.844859in}{0.774544in}}%
\pgfpathlineto{\pgfqpoint{4.839600in}{0.775096in}}%
\pgfpathlineto{\pgfqpoint{4.834340in}{0.775649in}}%
\pgfpathlineto{\pgfqpoint{4.829081in}{0.776203in}}%
\pgfpathlineto{\pgfqpoint{4.823822in}{0.776759in}}%
\pgfpathlineto{\pgfqpoint{4.818563in}{0.777316in}}%
\pgfpathlineto{\pgfqpoint{4.813303in}{0.777874in}}%
\pgfpathlineto{\pgfqpoint{4.808044in}{0.778434in}}%
\pgfpathlineto{\pgfqpoint{4.802785in}{0.778995in}}%
\pgfpathlineto{\pgfqpoint{4.797526in}{0.779557in}}%
\pgfpathlineto{\pgfqpoint{4.792267in}{0.780121in}}%
\pgfpathlineto{\pgfqpoint{4.787007in}{0.780686in}}%
\pgfpathlineto{\pgfqpoint{4.781748in}{0.781252in}}%
\pgfpathlineto{\pgfqpoint{4.776489in}{0.781820in}}%
\pgfpathlineto{\pgfqpoint{4.771230in}{0.782390in}}%
\pgfpathlineto{\pgfqpoint{4.765970in}{0.782960in}}%
\pgfpathlineto{\pgfqpoint{4.760711in}{0.783532in}}%
\pgfpathlineto{\pgfqpoint{4.755452in}{0.784106in}}%
\pgfpathlineto{\pgfqpoint{4.750193in}{0.784680in}}%
\pgfpathlineto{\pgfqpoint{4.744933in}{0.785257in}}%
\pgfpathlineto{\pgfqpoint{4.739674in}{0.785834in}}%
\pgfpathlineto{\pgfqpoint{4.734415in}{0.786413in}}%
\pgfpathlineto{\pgfqpoint{4.729156in}{0.786994in}}%
\pgfpathlineto{\pgfqpoint{4.723897in}{0.787576in}}%
\pgfpathlineto{\pgfqpoint{4.718637in}{0.788159in}}%
\pgfpathlineto{\pgfqpoint{4.713378in}{0.788744in}}%
\pgfpathlineto{\pgfqpoint{4.708119in}{0.789330in}}%
\pgfpathlineto{\pgfqpoint{4.702860in}{0.789918in}}%
\pgfpathlineto{\pgfqpoint{4.697600in}{0.790507in}}%
\pgfpathlineto{\pgfqpoint{4.692341in}{0.791098in}}%
\pgfpathlineto{\pgfqpoint{4.687082in}{0.791690in}}%
\pgfpathlineto{\pgfqpoint{4.681823in}{0.792284in}}%
\pgfpathlineto{\pgfqpoint{4.676564in}{0.792879in}}%
\pgfpathlineto{\pgfqpoint{4.671304in}{0.793475in}}%
\pgfpathlineto{\pgfqpoint{4.666045in}{0.794073in}}%
\pgfpathlineto{\pgfqpoint{4.660786in}{0.794673in}}%
\pgfpathlineto{\pgfqpoint{4.655527in}{0.795274in}}%
\pgfpathlineto{\pgfqpoint{4.650267in}{0.795876in}}%
\pgfpathlineto{\pgfqpoint{4.645008in}{0.796480in}}%
\pgfpathlineto{\pgfqpoint{4.639749in}{0.797086in}}%
\pgfpathlineto{\pgfqpoint{4.634490in}{0.797693in}}%
\pgfpathlineto{\pgfqpoint{4.629230in}{0.798301in}}%
\pgfpathlineto{\pgfqpoint{4.623971in}{0.798912in}}%
\pgfpathlineto{\pgfqpoint{4.618712in}{0.799523in}}%
\pgfpathlineto{\pgfqpoint{4.613453in}{0.800136in}}%
\pgfpathlineto{\pgfqpoint{4.608194in}{0.800751in}}%
\pgfpathlineto{\pgfqpoint{4.602934in}{0.801368in}}%
\pgfpathlineto{\pgfqpoint{4.597675in}{0.801986in}}%
\pgfpathlineto{\pgfqpoint{4.592416in}{0.802605in}}%
\pgfpathlineto{\pgfqpoint{4.587157in}{0.803226in}}%
\pgfpathlineto{\pgfqpoint{4.581897in}{0.803849in}}%
\pgfpathlineto{\pgfqpoint{4.576638in}{0.804473in}}%
\pgfpathlineto{\pgfqpoint{4.571379in}{0.805099in}}%
\pgfpathlineto{\pgfqpoint{4.566120in}{0.805726in}}%
\pgfpathlineto{\pgfqpoint{4.560861in}{0.806355in}}%
\pgfpathlineto{\pgfqpoint{4.555601in}{0.806986in}}%
\pgfpathlineto{\pgfqpoint{4.550342in}{0.807618in}}%
\pgfpathlineto{\pgfqpoint{4.545083in}{0.808252in}}%
\pgfpathlineto{\pgfqpoint{4.539824in}{0.808887in}}%
\pgfpathlineto{\pgfqpoint{4.534564in}{0.809524in}}%
\pgfpathlineto{\pgfqpoint{4.529305in}{0.810163in}}%
\pgfpathlineto{\pgfqpoint{4.524046in}{0.810804in}}%
\pgfpathlineto{\pgfqpoint{4.518787in}{0.811446in}}%
\pgfpathlineto{\pgfqpoint{4.513528in}{0.812090in}}%
\pgfpathlineto{\pgfqpoint{4.508268in}{0.812735in}}%
\pgfpathlineto{\pgfqpoint{4.503009in}{0.813382in}}%
\pgfpathlineto{\pgfqpoint{4.497750in}{0.814031in}}%
\pgfpathlineto{\pgfqpoint{4.492491in}{0.814681in}}%
\pgfpathlineto{\pgfqpoint{4.487231in}{0.815333in}}%
\pgfpathlineto{\pgfqpoint{4.481972in}{0.815987in}}%
\pgfpathlineto{\pgfqpoint{4.476713in}{0.816643in}}%
\pgfpathlineto{\pgfqpoint{4.471454in}{0.817300in}}%
\pgfpathlineto{\pgfqpoint{4.466194in}{0.817959in}}%
\pgfpathlineto{\pgfqpoint{4.460935in}{0.818620in}}%
\pgfpathlineto{\pgfqpoint{4.455676in}{0.819282in}}%
\pgfpathlineto{\pgfqpoint{4.450417in}{0.819947in}}%
\pgfpathlineto{\pgfqpoint{4.445158in}{0.820613in}}%
\pgfpathlineto{\pgfqpoint{4.439898in}{0.821280in}}%
\pgfpathlineto{\pgfqpoint{4.434639in}{0.821950in}}%
\pgfpathlineto{\pgfqpoint{4.429380in}{0.822621in}}%
\pgfpathlineto{\pgfqpoint{4.424121in}{0.823294in}}%
\pgfpathlineto{\pgfqpoint{4.418861in}{0.823969in}}%
\pgfpathlineto{\pgfqpoint{4.413602in}{0.824646in}}%
\pgfpathlineto{\pgfqpoint{4.408343in}{0.825324in}}%
\pgfpathlineto{\pgfqpoint{4.403084in}{0.826004in}}%
\pgfpathlineto{\pgfqpoint{4.397825in}{0.826686in}}%
\pgfpathlineto{\pgfqpoint{4.392565in}{0.827370in}}%
\pgfpathlineto{\pgfqpoint{4.387306in}{0.828056in}}%
\pgfpathlineto{\pgfqpoint{4.382047in}{0.828743in}}%
\pgfpathlineto{\pgfqpoint{4.376788in}{0.829433in}}%
\pgfpathlineto{\pgfqpoint{4.371528in}{0.830124in}}%
\pgfpathlineto{\pgfqpoint{4.366269in}{0.830817in}}%
\pgfpathlineto{\pgfqpoint{4.361010in}{0.831512in}}%
\pgfpathlineto{\pgfqpoint{4.355751in}{0.832209in}}%
\pgfpathlineto{\pgfqpoint{4.350492in}{0.832908in}}%
\pgfpathlineto{\pgfqpoint{4.345232in}{0.833608in}}%
\pgfpathlineto{\pgfqpoint{4.339973in}{0.834311in}}%
\pgfpathlineto{\pgfqpoint{4.334714in}{0.835015in}}%
\pgfpathlineto{\pgfqpoint{4.329455in}{0.835721in}}%
\pgfpathlineto{\pgfqpoint{4.324195in}{0.836430in}}%
\pgfpathlineto{\pgfqpoint{4.318936in}{0.837140in}}%
\pgfpathlineto{\pgfqpoint{4.313677in}{0.837852in}}%
\pgfpathlineto{\pgfqpoint{4.308418in}{0.838566in}}%
\pgfpathlineto{\pgfqpoint{4.303158in}{0.839282in}}%
\pgfpathlineto{\pgfqpoint{4.297899in}{0.840000in}}%
\pgfpathlineto{\pgfqpoint{4.292640in}{0.840720in}}%
\pgfpathlineto{\pgfqpoint{4.287381in}{0.841441in}}%
\pgfpathlineto{\pgfqpoint{4.282122in}{0.842165in}}%
\pgfpathlineto{\pgfqpoint{4.276862in}{0.842891in}}%
\pgfpathlineto{\pgfqpoint{4.271603in}{0.843619in}}%
\pgfpathlineto{\pgfqpoint{4.266344in}{0.844349in}}%
\pgfpathlineto{\pgfqpoint{4.261085in}{0.845081in}}%
\pgfpathlineto{\pgfqpoint{4.255825in}{0.845815in}}%
\pgfpathlineto{\pgfqpoint{4.250566in}{0.846551in}}%
\pgfpathlineto{\pgfqpoint{4.245307in}{0.847288in}}%
\pgfpathlineto{\pgfqpoint{4.240048in}{0.848028in}}%
\pgfpathlineto{\pgfqpoint{4.234789in}{0.848771in}}%
\pgfpathlineto{\pgfqpoint{4.229529in}{0.849515in}}%
\pgfpathlineto{\pgfqpoint{4.224270in}{0.850261in}}%
\pgfpathlineto{\pgfqpoint{4.219011in}{0.851009in}}%
\pgfpathlineto{\pgfqpoint{4.213752in}{0.851760in}}%
\pgfpathlineto{\pgfqpoint{4.208492in}{0.852512in}}%
\pgfpathlineto{\pgfqpoint{4.203233in}{0.853267in}}%
\pgfpathlineto{\pgfqpoint{4.197974in}{0.854023in}}%
\pgfpathlineto{\pgfqpoint{4.192715in}{0.854782in}}%
\pgfpathlineto{\pgfqpoint{4.187456in}{0.855543in}}%
\pgfpathlineto{\pgfqpoint{4.182196in}{0.856306in}}%
\pgfpathlineto{\pgfqpoint{4.176937in}{0.857072in}}%
\pgfpathlineto{\pgfqpoint{4.171678in}{0.857839in}}%
\pgfpathlineto{\pgfqpoint{4.166419in}{0.858609in}}%
\pgfpathlineto{\pgfqpoint{4.161159in}{0.859381in}}%
\pgfpathlineto{\pgfqpoint{4.155900in}{0.860155in}}%
\pgfpathlineto{\pgfqpoint{4.150641in}{0.860931in}}%
\pgfpathlineto{\pgfqpoint{4.145382in}{0.861709in}}%
\pgfpathlineto{\pgfqpoint{4.140122in}{0.862490in}}%
\pgfpathlineto{\pgfqpoint{4.134863in}{0.863273in}}%
\pgfpathlineto{\pgfqpoint{4.129604in}{0.864058in}}%
\pgfpathlineto{\pgfqpoint{4.124345in}{0.864845in}}%
\pgfpathlineto{\pgfqpoint{4.119086in}{0.865635in}}%
\pgfpathlineto{\pgfqpoint{4.113826in}{0.866427in}}%
\pgfpathlineto{\pgfqpoint{4.108567in}{0.867221in}}%
\pgfpathlineto{\pgfqpoint{4.103308in}{0.868017in}}%
\pgfpathlineto{\pgfqpoint{4.098049in}{0.868816in}}%
\pgfpathlineto{\pgfqpoint{4.092789in}{0.869617in}}%
\pgfpathlineto{\pgfqpoint{4.087530in}{0.870420in}}%
\pgfpathlineto{\pgfqpoint{4.082271in}{0.871226in}}%
\pgfpathlineto{\pgfqpoint{4.077012in}{0.872034in}}%
\pgfpathlineto{\pgfqpoint{4.071753in}{0.872844in}}%
\pgfpathlineto{\pgfqpoint{4.066493in}{0.873657in}}%
\pgfpathlineto{\pgfqpoint{4.061234in}{0.874472in}}%
\pgfpathlineto{\pgfqpoint{4.055975in}{0.875290in}}%
\pgfpathlineto{\pgfqpoint{4.050716in}{0.876110in}}%
\pgfpathlineto{\pgfqpoint{4.045456in}{0.876932in}}%
\pgfpathlineto{\pgfqpoint{4.040197in}{0.877757in}}%
\pgfpathlineto{\pgfqpoint{4.034938in}{0.878584in}}%
\pgfpathlineto{\pgfqpoint{4.029679in}{0.879413in}}%
\pgfpathlineto{\pgfqpoint{4.024420in}{0.880245in}}%
\pgfpathlineto{\pgfqpoint{4.019160in}{0.881080in}}%
\pgfpathlineto{\pgfqpoint{4.013901in}{0.881917in}}%
\pgfpathlineto{\pgfqpoint{4.008642in}{0.882756in}}%
\pgfpathlineto{\pgfqpoint{4.003383in}{0.883598in}}%
\pgfpathlineto{\pgfqpoint{3.998123in}{0.884442in}}%
\pgfpathlineto{\pgfqpoint{3.992864in}{0.885289in}}%
\pgfpathlineto{\pgfqpoint{3.987605in}{0.886139in}}%
\pgfpathlineto{\pgfqpoint{3.982346in}{0.886991in}}%
\pgfpathlineto{\pgfqpoint{3.977086in}{0.887845in}}%
\pgfpathlineto{\pgfqpoint{3.971827in}{0.888702in}}%
\pgfpathlineto{\pgfqpoint{3.966568in}{0.889562in}}%
\pgfpathlineto{\pgfqpoint{3.961309in}{0.890424in}}%
\pgfpathlineto{\pgfqpoint{3.956050in}{0.891289in}}%
\pgfpathlineto{\pgfqpoint{3.950790in}{0.892156in}}%
\pgfpathlineto{\pgfqpoint{3.945531in}{0.893026in}}%
\pgfpathlineto{\pgfqpoint{3.940272in}{0.893898in}}%
\pgfpathlineto{\pgfqpoint{3.935013in}{0.894774in}}%
\pgfpathlineto{\pgfqpoint{3.929753in}{0.895652in}}%
\pgfpathlineto{\pgfqpoint{3.924494in}{0.896532in}}%
\pgfpathlineto{\pgfqpoint{3.919235in}{0.897415in}}%
\pgfpathlineto{\pgfqpoint{3.913976in}{0.898301in}}%
\pgfpathlineto{\pgfqpoint{3.908717in}{0.899190in}}%
\pgfpathlineto{\pgfqpoint{3.903457in}{0.900081in}}%
\pgfpathlineto{\pgfqpoint{3.898198in}{0.900975in}}%
\pgfpathlineto{\pgfqpoint{3.892939in}{0.901872in}}%
\pgfpathlineto{\pgfqpoint{3.887680in}{0.902771in}}%
\pgfpathlineto{\pgfqpoint{3.882420in}{0.903674in}}%
\pgfpathlineto{\pgfqpoint{3.877161in}{0.904579in}}%
\pgfpathlineto{\pgfqpoint{3.871902in}{0.905486in}}%
\pgfpathlineto{\pgfqpoint{3.866643in}{0.906397in}}%
\pgfpathlineto{\pgfqpoint{3.861384in}{0.907310in}}%
\pgfpathlineto{\pgfqpoint{3.856124in}{0.908227in}}%
\pgfpathlineto{\pgfqpoint{3.850865in}{0.909146in}}%
\pgfpathlineto{\pgfqpoint{3.845606in}{0.910068in}}%
\pgfpathlineto{\pgfqpoint{3.840347in}{0.910992in}}%
\pgfpathlineto{\pgfqpoint{3.835087in}{0.911920in}}%
\pgfpathlineto{\pgfqpoint{3.829828in}{0.912851in}}%
\pgfpathlineto{\pgfqpoint{3.824569in}{0.913784in}}%
\pgfpathlineto{\pgfqpoint{3.819310in}{0.914721in}}%
\pgfpathlineto{\pgfqpoint{3.814050in}{0.915660in}}%
\pgfpathlineto{\pgfqpoint{3.808791in}{0.916602in}}%
\pgfpathlineto{\pgfqpoint{3.803532in}{0.917547in}}%
\pgfpathlineto{\pgfqpoint{3.798273in}{0.918495in}}%
\pgfpathlineto{\pgfqpoint{3.793014in}{0.919447in}}%
\pgfpathlineto{\pgfqpoint{3.787754in}{0.920401in}}%
\pgfpathlineto{\pgfqpoint{3.782495in}{0.921358in}}%
\pgfpathlineto{\pgfqpoint{3.777236in}{0.922318in}}%
\pgfpathlineto{\pgfqpoint{3.771977in}{0.923281in}}%
\pgfpathlineto{\pgfqpoint{3.766717in}{0.924248in}}%
\pgfpathlineto{\pgfqpoint{3.761458in}{0.925217in}}%
\pgfpathlineto{\pgfqpoint{3.756199in}{0.926190in}}%
\pgfpathlineto{\pgfqpoint{3.750940in}{0.927165in}}%
\pgfpathlineto{\pgfqpoint{3.745681in}{0.928144in}}%
\pgfpathlineto{\pgfqpoint{3.740421in}{0.929126in}}%
\pgfpathlineto{\pgfqpoint{3.735162in}{0.930111in}}%
\pgfpathlineto{\pgfqpoint{3.729903in}{0.931099in}}%
\pgfpathlineto{\pgfqpoint{3.724644in}{0.932090in}}%
\pgfpathlineto{\pgfqpoint{3.719384in}{0.933084in}}%
\pgfpathlineto{\pgfqpoint{3.714125in}{0.934082in}}%
\pgfpathlineto{\pgfqpoint{3.708866in}{0.935083in}}%
\pgfpathlineto{\pgfqpoint{3.703607in}{0.936087in}}%
\pgfpathlineto{\pgfqpoint{3.698348in}{0.937095in}}%
\pgfpathlineto{\pgfqpoint{3.693088in}{0.938105in}}%
\pgfpathlineto{\pgfqpoint{3.687829in}{0.939119in}}%
\pgfpathlineto{\pgfqpoint{3.682570in}{0.940137in}}%
\pgfpathlineto{\pgfqpoint{3.677311in}{0.941157in}}%
\pgfpathlineto{\pgfqpoint{3.672051in}{0.942181in}}%
\pgfpathlineto{\pgfqpoint{3.666792in}{0.943208in}}%
\pgfpathlineto{\pgfqpoint{3.661533in}{0.944239in}}%
\pgfpathlineto{\pgfqpoint{3.656274in}{0.945273in}}%
\pgfpathlineto{\pgfqpoint{3.651014in}{0.946310in}}%
\pgfpathlineto{\pgfqpoint{3.645755in}{0.947351in}}%
\pgfpathlineto{\pgfqpoint{3.640496in}{0.948396in}}%
\pgfpathlineto{\pgfqpoint{3.635237in}{0.949443in}}%
\pgfpathlineto{\pgfqpoint{3.629978in}{0.950495in}}%
\pgfpathlineto{\pgfqpoint{3.624718in}{0.951549in}}%
\pgfpathlineto{\pgfqpoint{3.619459in}{0.952607in}}%
\pgfpathlineto{\pgfqpoint{3.614200in}{0.953669in}}%
\pgfpathlineto{\pgfqpoint{3.608941in}{0.954734in}}%
\pgfpathlineto{\pgfqpoint{3.603681in}{0.955803in}}%
\pgfpathlineto{\pgfqpoint{3.598422in}{0.956875in}}%
\pgfpathlineto{\pgfqpoint{3.593163in}{0.957951in}}%
\pgfpathlineto{\pgfqpoint{3.587904in}{0.959031in}}%
\pgfpathlineto{\pgfqpoint{3.582645in}{0.960114in}}%
\pgfpathlineto{\pgfqpoint{3.577385in}{0.961201in}}%
\pgfpathlineto{\pgfqpoint{3.572126in}{0.962292in}}%
\pgfpathlineto{\pgfqpoint{3.566867in}{0.963386in}}%
\pgfpathlineto{\pgfqpoint{3.561608in}{0.964484in}}%
\pgfpathlineto{\pgfqpoint{3.556348in}{0.965585in}}%
\pgfpathlineto{\pgfqpoint{3.551089in}{0.966691in}}%
\pgfpathlineto{\pgfqpoint{3.545830in}{0.967800in}}%
\pgfpathlineto{\pgfqpoint{3.540571in}{0.968913in}}%
\pgfpathlineto{\pgfqpoint{3.535311in}{0.970029in}}%
\pgfpathlineto{\pgfqpoint{3.530052in}{0.971150in}}%
\pgfpathlineto{\pgfqpoint{3.524793in}{0.972274in}}%
\pgfpathlineto{\pgfqpoint{3.519534in}{0.973402in}}%
\pgfpathlineto{\pgfqpoint{3.514275in}{0.974535in}}%
\pgfpathlineto{\pgfqpoint{3.509015in}{0.975671in}}%
\pgfpathlineto{\pgfqpoint{3.503756in}{0.976810in}}%
\pgfpathlineto{\pgfqpoint{3.498497in}{0.977954in}}%
\pgfpathlineto{\pgfqpoint{3.493238in}{0.979102in}}%
\pgfpathlineto{\pgfqpoint{3.487978in}{0.980254in}}%
\pgfpathlineto{\pgfqpoint{3.482719in}{0.981410in}}%
\pgfpathlineto{\pgfqpoint{3.477460in}{0.982569in}}%
\pgfpathlineto{\pgfqpoint{3.472201in}{0.983733in}}%
\pgfpathlineto{\pgfqpoint{3.466942in}{0.984901in}}%
\pgfpathlineto{\pgfqpoint{3.461682in}{0.986073in}}%
\pgfpathlineto{\pgfqpoint{3.456423in}{0.987249in}}%
\pgfpathlineto{\pgfqpoint{3.451164in}{0.988429in}}%
\pgfpathlineto{\pgfqpoint{3.445905in}{0.989614in}}%
\pgfpathlineto{\pgfqpoint{3.440645in}{0.990802in}}%
\pgfpathlineto{\pgfqpoint{3.435386in}{0.991995in}}%
\pgfpathlineto{\pgfqpoint{3.430127in}{0.993192in}}%
\pgfpathlineto{\pgfqpoint{3.424868in}{0.994393in}}%
\pgfpathlineto{\pgfqpoint{3.419609in}{0.995598in}}%
\pgfpathlineto{\pgfqpoint{3.414349in}{0.996808in}}%
\pgfpathlineto{\pgfqpoint{3.409090in}{0.998022in}}%
\pgfpathlineto{\pgfqpoint{3.403831in}{0.999240in}}%
\pgfpathlineto{\pgfqpoint{3.398572in}{1.000463in}}%
\pgfpathlineto{\pgfqpoint{3.393312in}{1.001690in}}%
\pgfpathlineto{\pgfqpoint{3.388053in}{1.002922in}}%
\pgfpathlineto{\pgfqpoint{3.382794in}{1.004157in}}%
\pgfpathlineto{\pgfqpoint{3.377535in}{1.005398in}}%
\pgfpathlineto{\pgfqpoint{3.372275in}{1.006642in}}%
\pgfpathlineto{\pgfqpoint{3.367016in}{1.007892in}}%
\pgfpathlineto{\pgfqpoint{3.361757in}{1.009146in}}%
\pgfpathlineto{\pgfqpoint{3.356498in}{1.010404in}}%
\pgfpathlineto{\pgfqpoint{3.351239in}{1.011667in}}%
\pgfpathlineto{\pgfqpoint{3.345979in}{1.012934in}}%
\pgfpathlineto{\pgfqpoint{3.340720in}{1.014206in}}%
\pgfpathlineto{\pgfqpoint{3.335461in}{1.015483in}}%
\pgfpathlineto{\pgfqpoint{3.330202in}{1.016765in}}%
\pgfpathlineto{\pgfqpoint{3.324942in}{1.018051in}}%
\pgfpathlineto{\pgfqpoint{3.319683in}{1.019341in}}%
\pgfpathlineto{\pgfqpoint{3.314424in}{1.020637in}}%
\pgfpathlineto{\pgfqpoint{3.309165in}{1.021937in}}%
\pgfpathlineto{\pgfqpoint{3.303906in}{1.023242in}}%
\pgfpathlineto{\pgfqpoint{3.298646in}{1.024552in}}%
\pgfpathlineto{\pgfqpoint{3.293387in}{1.025867in}}%
\pgfpathlineto{\pgfqpoint{3.288128in}{1.027187in}}%
\pgfpathlineto{\pgfqpoint{3.282869in}{1.028512in}}%
\pgfpathlineto{\pgfqpoint{3.277609in}{1.029841in}}%
\pgfpathlineto{\pgfqpoint{3.272350in}{1.031176in}}%
\pgfpathlineto{\pgfqpoint{3.267091in}{1.032515in}}%
\pgfpathlineto{\pgfqpoint{3.261832in}{1.033860in}}%
\pgfpathlineto{\pgfqpoint{3.256573in}{1.035209in}}%
\pgfpathlineto{\pgfqpoint{3.251313in}{1.036564in}}%
\pgfpathlineto{\pgfqpoint{3.246054in}{1.037923in}}%
\pgfpathlineto{\pgfqpoint{3.240795in}{1.039288in}}%
\pgfpathlineto{\pgfqpoint{3.235536in}{1.040658in}}%
\pgfpathlineto{\pgfqpoint{3.230276in}{1.042033in}}%
\pgfpathlineto{\pgfqpoint{3.225017in}{1.043414in}}%
\pgfpathlineto{\pgfqpoint{3.219758in}{1.044799in}}%
\pgfpathlineto{\pgfqpoint{3.214499in}{1.046190in}}%
\pgfpathlineto{\pgfqpoint{3.209239in}{1.047587in}}%
\pgfpathlineto{\pgfqpoint{3.203980in}{1.048988in}}%
\pgfpathlineto{\pgfqpoint{3.198721in}{1.050395in}}%
\pgfpathlineto{\pgfqpoint{3.193462in}{1.051807in}}%
\pgfpathlineto{\pgfqpoint{3.188203in}{1.053225in}}%
\pgfpathlineto{\pgfqpoint{3.182943in}{1.054648in}}%
\pgfpathlineto{\pgfqpoint{3.177684in}{1.056077in}}%
\pgfpathlineto{\pgfqpoint{3.172425in}{1.057511in}}%
\pgfpathlineto{\pgfqpoint{3.167166in}{1.058951in}}%
\pgfpathlineto{\pgfqpoint{3.161906in}{1.060397in}}%
\pgfpathlineto{\pgfqpoint{3.156647in}{1.061848in}}%
\pgfpathlineto{\pgfqpoint{3.151388in}{1.063304in}}%
\pgfpathlineto{\pgfqpoint{3.146129in}{1.064767in}}%
\pgfpathlineto{\pgfqpoint{3.140870in}{1.066235in}}%
\pgfpathlineto{\pgfqpoint{3.135610in}{1.067709in}}%
\pgfpathlineto{\pgfqpoint{3.130351in}{1.069188in}}%
\pgfpathlineto{\pgfqpoint{3.125092in}{1.070674in}}%
\pgfpathlineto{\pgfqpoint{3.119833in}{1.072165in}}%
\pgfpathlineto{\pgfqpoint{3.114573in}{1.073662in}}%
\pgfpathlineto{\pgfqpoint{3.109314in}{1.075165in}}%
\pgfpathlineto{\pgfqpoint{3.104055in}{1.076674in}}%
\pgfpathlineto{\pgfqpoint{3.098796in}{1.078189in}}%
\pgfpathlineto{\pgfqpoint{3.093537in}{1.079711in}}%
\pgfpathlineto{\pgfqpoint{3.088277in}{1.081238in}}%
\pgfpathlineto{\pgfqpoint{3.083018in}{1.082771in}}%
\pgfpathlineto{\pgfqpoint{3.077759in}{1.084311in}}%
\pgfpathlineto{\pgfqpoint{3.072500in}{1.085856in}}%
\pgfpathlineto{\pgfqpoint{3.067240in}{1.087408in}}%
\pgfpathlineto{\pgfqpoint{3.061981in}{1.088966in}}%
\pgfpathlineto{\pgfqpoint{3.056722in}{1.090531in}}%
\pgfpathlineto{\pgfqpoint{3.051463in}{1.092101in}}%
\pgfpathlineto{\pgfqpoint{3.046203in}{1.093678in}}%
\pgfpathlineto{\pgfqpoint{3.040944in}{1.095262in}}%
\pgfpathlineto{\pgfqpoint{3.035685in}{1.096852in}}%
\pgfpathlineto{\pgfqpoint{3.030426in}{1.098448in}}%
\pgfpathlineto{\pgfqpoint{3.025167in}{1.100051in}}%
\pgfpathlineto{\pgfqpoint{3.019907in}{1.101661in}}%
\pgfpathlineto{\pgfqpoint{3.014648in}{1.103277in}}%
\pgfpathlineto{\pgfqpoint{3.009389in}{1.104900in}}%
\pgfpathlineto{\pgfqpoint{3.004130in}{1.106529in}}%
\pgfpathlineto{\pgfqpoint{2.998870in}{1.108166in}}%
\pgfpathlineto{\pgfqpoint{2.993611in}{1.109809in}}%
\pgfpathlineto{\pgfqpoint{2.988352in}{1.111458in}}%
\pgfpathlineto{\pgfqpoint{2.983093in}{1.113115in}}%
\pgfpathlineto{\pgfqpoint{2.977834in}{1.114779in}}%
\pgfpathlineto{\pgfqpoint{2.972574in}{1.116449in}}%
\pgfpathlineto{\pgfqpoint{2.967315in}{1.118127in}}%
\pgfpathlineto{\pgfqpoint{2.962056in}{1.119811in}}%
\pgfpathlineto{\pgfqpoint{2.956797in}{1.121503in}}%
\pgfpathlineto{\pgfqpoint{2.951537in}{1.123202in}}%
\pgfpathlineto{\pgfqpoint{2.946278in}{1.124907in}}%
\pgfpathlineto{\pgfqpoint{2.941019in}{1.126621in}}%
\pgfpathlineto{\pgfqpoint{2.935760in}{1.128341in}}%
\pgfpathlineto{\pgfqpoint{2.930501in}{1.130069in}}%
\pgfpathlineto{\pgfqpoint{2.925241in}{1.131804in}}%
\pgfpathlineto{\pgfqpoint{2.919982in}{1.133546in}}%
\pgfpathlineto{\pgfqpoint{2.914723in}{1.135296in}}%
\pgfpathlineto{\pgfqpoint{2.909464in}{1.137053in}}%
\pgfpathlineto{\pgfqpoint{2.904204in}{1.138818in}}%
\pgfpathlineto{\pgfqpoint{2.898945in}{1.140591in}}%
\pgfpathlineto{\pgfqpoint{2.893686in}{1.142371in}}%
\pgfpathlineto{\pgfqpoint{2.888427in}{1.144158in}}%
\pgfpathlineto{\pgfqpoint{2.883167in}{1.145954in}}%
\pgfpathlineto{\pgfqpoint{2.877908in}{1.147757in}}%
\pgfpathlineto{\pgfqpoint{2.872649in}{1.149569in}}%
\pgfpathlineto{\pgfqpoint{2.867390in}{1.151388in}}%
\pgfpathlineto{\pgfqpoint{2.862131in}{1.153215in}}%
\pgfpathlineto{\pgfqpoint{2.856871in}{1.155050in}}%
\pgfpathlineto{\pgfqpoint{2.851612in}{1.156893in}}%
\pgfpathlineto{\pgfqpoint{2.846353in}{1.158744in}}%
\pgfpathlineto{\pgfqpoint{2.841094in}{1.160603in}}%
\pgfpathlineto{\pgfqpoint{2.835834in}{1.162471in}}%
\pgfpathlineto{\pgfqpoint{2.830575in}{1.164346in}}%
\pgfpathlineto{\pgfqpoint{2.825316in}{1.166230in}}%
\pgfpathlineto{\pgfqpoint{2.820057in}{1.168123in}}%
\pgfpathlineto{\pgfqpoint{2.814798in}{1.170024in}}%
\pgfpathlineto{\pgfqpoint{2.809538in}{1.171933in}}%
\pgfpathlineto{\pgfqpoint{2.804279in}{1.173851in}}%
\pgfpathlineto{\pgfqpoint{2.799020in}{1.175777in}}%
\pgfpathlineto{\pgfqpoint{2.793761in}{1.177713in}}%
\pgfpathlineto{\pgfqpoint{2.788501in}{1.179656in}}%
\pgfpathlineto{\pgfqpoint{2.783242in}{1.181609in}}%
\pgfpathlineto{\pgfqpoint{2.777983in}{1.183571in}}%
\pgfpathlineto{\pgfqpoint{2.772724in}{1.185541in}}%
\pgfpathlineto{\pgfqpoint{2.767465in}{1.187520in}}%
\pgfpathlineto{\pgfqpoint{2.762205in}{1.189509in}}%
\pgfpathlineto{\pgfqpoint{2.756946in}{1.191506in}}%
\pgfpathlineto{\pgfqpoint{2.751687in}{1.193513in}}%
\pgfpathlineto{\pgfqpoint{2.746428in}{1.195528in}}%
\pgfpathlineto{\pgfqpoint{2.741168in}{1.197553in}}%
\pgfpathlineto{\pgfqpoint{2.735909in}{1.199588in}}%
\pgfpathlineto{\pgfqpoint{2.730650in}{1.201631in}}%
\pgfpathlineto{\pgfqpoint{2.725391in}{1.203685in}}%
\pgfpathlineto{\pgfqpoint{2.720131in}{1.205747in}}%
\pgfpathlineto{\pgfqpoint{2.714872in}{1.207820in}}%
\pgfpathlineto{\pgfqpoint{2.709613in}{1.209902in}}%
\pgfpathlineto{\pgfqpoint{2.704354in}{1.211993in}}%
\pgfpathlineto{\pgfqpoint{2.699095in}{1.214095in}}%
\pgfpathlineto{\pgfqpoint{2.693835in}{1.216206in}}%
\pgfpathlineto{\pgfqpoint{2.688576in}{1.218327in}}%
\pgfpathlineto{\pgfqpoint{2.683317in}{1.220459in}}%
\pgfpathlineto{\pgfqpoint{2.678058in}{1.222600in}}%
\pgfpathlineto{\pgfqpoint{2.672798in}{1.224752in}}%
\pgfpathlineto{\pgfqpoint{2.667539in}{1.226913in}}%
\pgfpathlineto{\pgfqpoint{2.662280in}{1.229085in}}%
\pgfpathlineto{\pgfqpoint{2.657021in}{1.231268in}}%
\pgfpathlineto{\pgfqpoint{2.651762in}{1.233461in}}%
\pgfpathlineto{\pgfqpoint{2.646502in}{1.235664in}}%
\pgfpathlineto{\pgfqpoint{2.641243in}{1.237878in}}%
\pgfpathlineto{\pgfqpoint{2.635984in}{1.240102in}}%
\pgfpathlineto{\pgfqpoint{2.630725in}{1.242338in}}%
\pgfpathlineto{\pgfqpoint{2.625465in}{1.244584in}}%
\pgfpathlineto{\pgfqpoint{2.620206in}{1.246841in}}%
\pgfpathlineto{\pgfqpoint{2.614947in}{1.249109in}}%
\pgfpathlineto{\pgfqpoint{2.609688in}{1.251388in}}%
\pgfpathlineto{\pgfqpoint{2.604429in}{1.253678in}}%
\pgfpathlineto{\pgfqpoint{2.599169in}{1.255980in}}%
\pgfpathlineto{\pgfqpoint{2.593910in}{1.258292in}}%
\pgfpathlineto{\pgfqpoint{2.588651in}{1.260617in}}%
\pgfpathlineto{\pgfqpoint{2.583392in}{1.262952in}}%
\pgfpathlineto{\pgfqpoint{2.578132in}{1.265299in}}%
\pgfpathlineto{\pgfqpoint{2.572873in}{1.267658in}}%
\pgfpathlineto{\pgfqpoint{2.567614in}{1.270029in}}%
\pgfpathlineto{\pgfqpoint{2.562355in}{1.272411in}}%
\pgfpathlineto{\pgfqpoint{2.557095in}{1.274806in}}%
\pgfpathlineto{\pgfqpoint{2.551836in}{1.277212in}}%
\pgfpathlineto{\pgfqpoint{2.546577in}{1.279630in}}%
\pgfpathlineto{\pgfqpoint{2.541318in}{1.282061in}}%
\pgfpathlineto{\pgfqpoint{2.536059in}{1.284504in}}%
\pgfpathlineto{\pgfqpoint{2.530799in}{1.286959in}}%
\pgfpathlineto{\pgfqpoint{2.525540in}{1.289426in}}%
\pgfpathlineto{\pgfqpoint{2.520281in}{1.291907in}}%
\pgfpathlineto{\pgfqpoint{2.515022in}{1.294399in}}%
\pgfpathlineto{\pgfqpoint{2.509762in}{1.296905in}}%
\pgfpathlineto{\pgfqpoint{2.504503in}{1.299423in}}%
\pgfpathlineto{\pgfqpoint{2.499244in}{1.301955in}}%
\pgfpathlineto{\pgfqpoint{2.493985in}{1.304499in}}%
\pgfpathlineto{\pgfqpoint{2.488726in}{1.307057in}}%
\pgfpathlineto{\pgfqpoint{2.483466in}{1.309628in}}%
\pgfpathlineto{\pgfqpoint{2.478207in}{1.312212in}}%
\pgfpathlineto{\pgfqpoint{2.472948in}{1.314810in}}%
\pgfpathlineto{\pgfqpoint{2.467689in}{1.317421in}}%
\pgfpathlineto{\pgfqpoint{2.462429in}{1.320046in}}%
\pgfpathlineto{\pgfqpoint{2.457170in}{1.322684in}}%
\pgfpathlineto{\pgfqpoint{2.451911in}{1.325337in}}%
\pgfpathlineto{\pgfqpoint{2.446652in}{1.328003in}}%
\pgfpathlineto{\pgfqpoint{2.441392in}{1.330684in}}%
\pgfpathlineto{\pgfqpoint{2.436133in}{1.333379in}}%
\pgfpathlineto{\pgfqpoint{2.430874in}{1.336088in}}%
\pgfpathlineto{\pgfqpoint{2.425615in}{1.338811in}}%
\pgfpathlineto{\pgfqpoint{2.420356in}{1.341550in}}%
\pgfpathlineto{\pgfqpoint{2.415096in}{1.344302in}}%
\pgfpathlineto{\pgfqpoint{2.409837in}{1.347070in}}%
\pgfpathlineto{\pgfqpoint{2.404578in}{1.349853in}}%
\pgfpathlineto{\pgfqpoint{2.399319in}{1.352650in}}%
\pgfpathlineto{\pgfqpoint{2.394059in}{1.355463in}}%
\pgfpathlineto{\pgfqpoint{2.388800in}{1.358291in}}%
\pgfpathlineto{\pgfqpoint{2.383541in}{1.361134in}}%
\pgfpathlineto{\pgfqpoint{2.378282in}{1.363993in}}%
\pgfpathlineto{\pgfqpoint{2.373023in}{1.366868in}}%
\pgfpathlineto{\pgfqpoint{2.367763in}{1.369758in}}%
\pgfpathlineto{\pgfqpoint{2.362504in}{1.372665in}}%
\pgfpathlineto{\pgfqpoint{2.357245in}{1.375587in}}%
\pgfpathlineto{\pgfqpoint{2.351986in}{1.378526in}}%
\pgfpathlineto{\pgfqpoint{2.346726in}{1.381480in}}%
\pgfpathlineto{\pgfqpoint{2.341467in}{1.384452in}}%
\pgfpathlineto{\pgfqpoint{2.336208in}{1.387440in}}%
\pgfpathlineto{\pgfqpoint{2.330949in}{1.390444in}}%
\pgfpathlineto{\pgfqpoint{2.325690in}{1.393466in}}%
\pgfpathlineto{\pgfqpoint{2.320430in}{1.396504in}}%
\pgfpathlineto{\pgfqpoint{2.315171in}{1.399560in}}%
\pgfpathlineto{\pgfqpoint{2.309912in}{1.402633in}}%
\pgfpathlineto{\pgfqpoint{2.304653in}{1.405724in}}%
\pgfpathlineto{\pgfqpoint{2.299393in}{1.408832in}}%
\pgfpathlineto{\pgfqpoint{2.294134in}{1.411958in}}%
\pgfpathlineto{\pgfqpoint{2.288875in}{1.415101in}}%
\pgfpathlineto{\pgfqpoint{2.283616in}{1.418263in}}%
\pgfpathlineto{\pgfqpoint{2.278356in}{1.421443in}}%
\pgfpathlineto{\pgfqpoint{2.273097in}{1.424642in}}%
\pgfpathlineto{\pgfqpoint{2.267838in}{1.427858in}}%
\pgfpathlineto{\pgfqpoint{2.262579in}{1.431094in}}%
\pgfpathlineto{\pgfqpoint{2.257320in}{1.434349in}}%
\pgfpathlineto{\pgfqpoint{2.252060in}{1.437622in}}%
\pgfpathlineto{\pgfqpoint{2.246801in}{1.440915in}}%
\pgfpathlineto{\pgfqpoint{2.241542in}{1.444227in}}%
\pgfpathlineto{\pgfqpoint{2.236283in}{1.447558in}}%
\pgfpathlineto{\pgfqpoint{2.231023in}{1.450910in}}%
\pgfpathlineto{\pgfqpoint{2.225764in}{1.454281in}}%
\pgfpathlineto{\pgfqpoint{2.220505in}{1.457672in}}%
\pgfpathlineto{\pgfqpoint{2.215246in}{1.461084in}}%
\pgfpathlineto{\pgfqpoint{2.209987in}{1.464515in}}%
\pgfpathlineto{\pgfqpoint{2.204727in}{1.467968in}}%
\pgfpathlineto{\pgfqpoint{2.199468in}{1.471441in}}%
\pgfpathlineto{\pgfqpoint{2.194209in}{1.474936in}}%
\pgfpathlineto{\pgfqpoint{2.188950in}{1.478451in}}%
\pgfpathlineto{\pgfqpoint{2.183690in}{1.481988in}}%
\pgfpathlineto{\pgfqpoint{2.178431in}{1.485546in}}%
\pgfpathlineto{\pgfqpoint{2.173172in}{1.489126in}}%
\pgfpathlineto{\pgfqpoint{2.167913in}{1.492728in}}%
\pgfpathlineto{\pgfqpoint{2.162654in}{1.496353in}}%
\pgfpathlineto{\pgfqpoint{2.157394in}{1.499999in}}%
\pgfpathlineto{\pgfqpoint{2.152135in}{1.503668in}}%
\pgfpathlineto{\pgfqpoint{2.146876in}{1.507360in}}%
\pgfpathlineto{\pgfqpoint{2.141617in}{1.511075in}}%
\pgfpathlineto{\pgfqpoint{2.136357in}{1.514813in}}%
\pgfpathlineto{\pgfqpoint{2.131098in}{1.518575in}}%
\pgfpathlineto{\pgfqpoint{2.125839in}{1.522360in}}%
\pgfpathlineto{\pgfqpoint{2.120580in}{1.526169in}}%
\pgfpathlineto{\pgfqpoint{2.115320in}{1.530002in}}%
\pgfpathlineto{\pgfqpoint{2.110061in}{1.533860in}}%
\pgfpathlineto{\pgfqpoint{2.104802in}{1.537742in}}%
\pgfpathlineto{\pgfqpoint{2.099543in}{1.541649in}}%
\pgfpathlineto{\pgfqpoint{2.094284in}{1.545581in}}%
\pgfpathlineto{\pgfqpoint{2.089024in}{1.549538in}}%
\pgfpathlineto{\pgfqpoint{2.083765in}{1.553521in}}%
\pgfpathlineto{\pgfqpoint{2.078506in}{1.557529in}}%
\pgfpathlineto{\pgfqpoint{2.073247in}{1.561564in}}%
\pgfpathlineto{\pgfqpoint{2.067987in}{1.565624in}}%
\pgfpathlineto{\pgfqpoint{2.062728in}{1.569712in}}%
\pgfpathlineto{\pgfqpoint{2.057469in}{1.573826in}}%
\pgfpathlineto{\pgfqpoint{2.052210in}{1.577967in}}%
\pgfpathlineto{\pgfqpoint{2.046951in}{1.582135in}}%
\pgfpathlineto{\pgfqpoint{2.041691in}{1.586331in}}%
\pgfpathlineto{\pgfqpoint{2.036432in}{1.590555in}}%
\pgfpathlineto{\pgfqpoint{2.031173in}{1.594807in}}%
\pgfpathlineto{\pgfqpoint{2.025914in}{1.599088in}}%
\pgfpathlineto{\pgfqpoint{2.020654in}{1.603397in}}%
\pgfpathlineto{\pgfqpoint{2.015395in}{1.607735in}}%
\pgfpathlineto{\pgfqpoint{2.010136in}{1.612103in}}%
\pgfpathlineto{\pgfqpoint{2.004877in}{1.616500in}}%
\pgfpathlineto{\pgfqpoint{1.999618in}{1.620927in}}%
\pgfpathlineto{\pgfqpoint{1.994358in}{1.625384in}}%
\pgfpathlineto{\pgfqpoint{1.989099in}{1.629872in}}%
\pgfpathlineto{\pgfqpoint{1.983840in}{1.634390in}}%
\pgfpathlineto{\pgfqpoint{1.978581in}{1.638940in}}%
\pgfpathlineto{\pgfqpoint{1.973321in}{1.643521in}}%
\pgfpathlineto{\pgfqpoint{1.968062in}{1.648134in}}%
\pgfpathlineto{\pgfqpoint{1.962803in}{1.652779in}}%
\pgfpathlineto{\pgfqpoint{1.957544in}{1.657457in}}%
\pgfpathlineto{\pgfqpoint{1.952284in}{1.662167in}}%
\pgfpathlineto{\pgfqpoint{1.947025in}{1.666911in}}%
\pgfpathlineto{\pgfqpoint{1.941766in}{1.671688in}}%
\pgfpathlineto{\pgfqpoint{1.936507in}{1.676499in}}%
\pgfpathlineto{\pgfqpoint{1.931248in}{1.681344in}}%
\pgfpathlineto{\pgfqpoint{1.925988in}{1.686223in}}%
\pgfpathlineto{\pgfqpoint{1.920729in}{1.691138in}}%
\pgfpathlineto{\pgfqpoint{1.915470in}{1.696088in}}%
\pgfpathlineto{\pgfqpoint{1.910211in}{1.701074in}}%
\pgfpathlineto{\pgfqpoint{1.904951in}{1.706096in}}%
\pgfpathlineto{\pgfqpoint{1.899692in}{1.711154in}}%
\pgfpathlineto{\pgfqpoint{1.894433in}{1.716250in}}%
\pgfpathlineto{\pgfqpoint{1.889174in}{1.721382in}}%
\pgfpathlineto{\pgfqpoint{1.883915in}{1.726552in}}%
\pgfpathlineto{\pgfqpoint{1.878655in}{1.731761in}}%
\pgfpathlineto{\pgfqpoint{1.873396in}{1.737008in}}%
\pgfpathlineto{\pgfqpoint{1.868137in}{1.742294in}}%
\pgfpathlineto{\pgfqpoint{1.862878in}{1.747619in}}%
\pgfpathlineto{\pgfqpoint{1.857618in}{1.752985in}}%
\pgfpathlineto{\pgfqpoint{1.852359in}{1.758390in}}%
\pgfpathlineto{\pgfqpoint{1.847100in}{1.763837in}}%
\pgfpathlineto{\pgfqpoint{1.841841in}{1.769324in}}%
\pgfpathlineto{\pgfqpoint{1.836582in}{1.774854in}}%
\pgfpathlineto{\pgfqpoint{1.831322in}{1.780425in}}%
\pgfpathlineto{\pgfqpoint{1.826063in}{1.786039in}}%
\pgfpathlineto{\pgfqpoint{1.820804in}{1.791697in}}%
\pgfpathlineto{\pgfqpoint{1.815545in}{1.797398in}}%
\pgfpathlineto{\pgfqpoint{1.810285in}{1.803143in}}%
\pgfpathlineto{\pgfqpoint{1.805026in}{1.808933in}}%
\pgfpathlineto{\pgfqpoint{1.799767in}{1.814768in}}%
\pgfpathlineto{\pgfqpoint{1.794508in}{1.820649in}}%
\pgfpathlineto{\pgfqpoint{1.789248in}{1.826576in}}%
\pgfpathlineto{\pgfqpoint{1.783989in}{1.832550in}}%
\pgfpathlineto{\pgfqpoint{1.778730in}{1.838571in}}%
\pgfpathlineto{\pgfqpoint{1.773471in}{1.844641in}}%
\pgfpathlineto{\pgfqpoint{1.768212in}{1.850759in}}%
\pgfpathlineto{\pgfqpoint{1.762952in}{1.856925in}}%
\pgfpathlineto{\pgfqpoint{1.757693in}{1.863142in}}%
\pgfpathlineto{\pgfqpoint{1.752434in}{1.869409in}}%
\pgfpathlineto{\pgfqpoint{1.747175in}{1.875727in}}%
\pgfpathlineto{\pgfqpoint{1.741915in}{1.882096in}}%
\pgfpathlineto{\pgfqpoint{1.736656in}{1.888518in}}%
\pgfpathlineto{\pgfqpoint{1.731397in}{1.894993in}}%
\pgfpathlineto{\pgfqpoint{1.726138in}{1.901520in}}%
\pgfpathlineto{\pgfqpoint{1.720879in}{1.908103in}}%
\pgfpathlineto{\pgfqpoint{1.715619in}{1.914739in}}%
\pgfpathlineto{\pgfqpoint{1.710360in}{1.921432in}}%
\pgfpathlineto{\pgfqpoint{1.705101in}{1.928180in}}%
\pgfpathlineto{\pgfqpoint{1.699842in}{1.934986in}}%
\pgfpathlineto{\pgfqpoint{1.694582in}{1.941849in}}%
\pgfpathlineto{\pgfqpoint{1.689323in}{1.948771in}}%
\pgfpathlineto{\pgfqpoint{1.684064in}{1.955751in}}%
\pgfpathlineto{\pgfqpoint{1.678805in}{1.962792in}}%
\pgfpathlineto{\pgfqpoint{1.673546in}{1.969893in}}%
\pgfpathlineto{\pgfqpoint{1.668286in}{1.977056in}}%
\pgfpathlineto{\pgfqpoint{1.663027in}{1.984281in}}%
\pgfpathlineto{\pgfqpoint{1.657768in}{1.991569in}}%
\pgfpathlineto{\pgfqpoint{1.652509in}{1.998921in}}%
\pgfpathlineto{\pgfqpoint{1.647249in}{2.006337in}}%
\pgfpathlineto{\pgfqpoint{1.641990in}{2.013819in}}%
\pgfpathlineto{\pgfqpoint{1.636731in}{2.021368in}}%
\pgfpathlineto{\pgfqpoint{1.631472in}{2.028984in}}%
\pgfpathlineto{\pgfqpoint{1.626212in}{2.036668in}}%
\pgfpathlineto{\pgfqpoint{1.620953in}{2.044422in}}%
\pgfpathlineto{\pgfqpoint{1.615694in}{2.052246in}}%
\pgfpathlineto{\pgfqpoint{1.610435in}{2.060140in}}%
\pgfpathlineto{\pgfqpoint{1.605176in}{2.068107in}}%
\pgfpathlineto{\pgfqpoint{1.599916in}{2.076147in}}%
\pgfpathlineto{\pgfqpoint{1.594657in}{2.084261in}}%
\pgfpathlineto{\pgfqpoint{1.589398in}{2.092450in}}%
\pgfpathlineto{\pgfqpoint{1.584139in}{2.100715in}}%
\pgfpathlineto{\pgfqpoint{1.578879in}{2.109057in}}%
\pgfpathlineto{\pgfqpoint{1.573620in}{2.117477in}}%
\pgfpathlineto{\pgfqpoint{1.568361in}{2.125977in}}%
\pgfpathlineto{\pgfqpoint{1.563102in}{2.134558in}}%
\pgfpathlineto{\pgfqpoint{1.557843in}{2.143220in}}%
\pgfpathlineto{\pgfqpoint{1.552583in}{2.151965in}}%
\pgfpathlineto{\pgfqpoint{1.547324in}{2.160794in}}%
\pgfpathlineto{\pgfqpoint{1.542065in}{2.169708in}}%
\pgfpathlineto{\pgfqpoint{1.536806in}{2.178709in}}%
\pgfpathlineto{\pgfqpoint{1.531546in}{2.187797in}}%
\pgfpathlineto{\pgfqpoint{1.526287in}{2.196975in}}%
\pgfpathlineto{\pgfqpoint{1.521028in}{2.206243in}}%
\pgfpathlineto{\pgfqpoint{1.515769in}{2.215602in}}%
\pgfpathlineto{\pgfqpoint{1.510509in}{2.225055in}}%
\pgfpathlineto{\pgfqpoint{1.505250in}{2.234602in}}%
\pgfpathlineto{\pgfqpoint{1.499991in}{2.244245in}}%
\pgfpathlineto{\pgfqpoint{1.494732in}{2.253986in}}%
\pgfpathlineto{\pgfqpoint{1.489473in}{2.263825in}}%
\pgfpathlineto{\pgfqpoint{1.484213in}{2.273764in}}%
\pgfpathlineto{\pgfqpoint{1.478954in}{2.283806in}}%
\pgfpathlineto{\pgfqpoint{1.473695in}{2.293951in}}%
\pgfpathlineto{\pgfqpoint{1.468436in}{2.304200in}}%
\pgfpathlineto{\pgfqpoint{1.463176in}{2.314557in}}%
\pgfpathlineto{\pgfqpoint{1.457917in}{2.325022in}}%
\pgfpathlineto{\pgfqpoint{1.452658in}{2.335597in}}%
\pgfpathlineto{\pgfqpoint{1.447399in}{2.346284in}}%
\pgfpathlineto{\pgfqpoint{1.442140in}{2.357084in}}%
\pgfpathlineto{\pgfqpoint{1.436880in}{2.368000in}}%
\pgfpathlineto{\pgfqpoint{1.431621in}{2.379034in}}%
\pgfpathlineto{\pgfqpoint{1.426362in}{2.390186in}}%
\pgfpathlineto{\pgfqpoint{1.421103in}{2.401460in}}%
\pgfpathlineto{\pgfqpoint{1.415843in}{2.412856in}}%
\pgfpathlineto{\pgfqpoint{1.410584in}{2.424378in}}%
\pgfpathlineto{\pgfqpoint{1.405325in}{2.436027in}}%
\pgfpathlineto{\pgfqpoint{1.400066in}{2.447805in}}%
\pgfpathlineto{\pgfqpoint{1.394807in}{2.459715in}}%
\pgfpathlineto{\pgfqpoint{1.389547in}{2.471759in}}%
\pgfpathlineto{\pgfqpoint{1.384288in}{2.483939in}}%
\pgfpathlineto{\pgfqpoint{1.379029in}{2.496256in}}%
\pgfpathlineto{\pgfqpoint{1.373770in}{2.508715in}}%
\pgfpathlineto{\pgfqpoint{1.368510in}{2.521317in}}%
\pgfpathlineto{\pgfqpoint{1.363251in}{2.534064in}}%
\pgfpathlineto{\pgfqpoint{1.357992in}{2.546959in}}%
\pgfpathlineto{\pgfqpoint{1.352733in}{2.560005in}}%
\pgfpathlineto{\pgfqpoint{1.347473in}{2.573205in}}%
\pgfpathlineto{\pgfqpoint{1.342214in}{2.586560in}}%
\pgfpathlineto{\pgfqpoint{1.336955in}{2.600075in}}%
\pgfpathlineto{\pgfqpoint{1.331696in}{2.613751in}}%
\pgfpathlineto{\pgfqpoint{1.326437in}{2.627592in}}%
\pgfpathlineto{\pgfqpoint{1.321177in}{2.641600in}}%
\pgfpathlineto{\pgfqpoint{1.315918in}{2.655779in}}%
\pgfpathlineto{\pgfqpoint{1.310659in}{2.670132in}}%
\pgfpathlineto{\pgfqpoint{1.305400in}{2.684662in}}%
\pgfpathlineto{\pgfqpoint{1.300140in}{2.699372in}}%
\pgfpathlineto{\pgfqpoint{1.294881in}{2.714266in}}%
\pgfpathlineto{\pgfqpoint{1.289622in}{2.729347in}}%
\pgfpathlineto{\pgfqpoint{1.284363in}{2.744619in}}%
\pgfpathlineto{\pgfqpoint{1.279104in}{2.760086in}}%
\pgfpathlineto{\pgfqpoint{1.273844in}{2.775750in}}%
\pgfpathlineto{\pgfqpoint{1.268585in}{2.791617in}}%
\pgfpathlineto{\pgfqpoint{1.263326in}{2.807689in}}%
\pgfpathlineto{\pgfqpoint{1.258067in}{2.823971in}}%
\pgfpathlineto{\pgfqpoint{1.252807in}{2.840468in}}%
\pgfpathlineto{\pgfqpoint{1.247548in}{2.857182in}}%
\pgfpathlineto{\pgfqpoint{1.242289in}{2.874120in}}%
\pgfpathlineto{\pgfqpoint{1.237030in}{2.891284in}}%
\pgfpathlineto{\pgfqpoint{1.231771in}{2.908680in}}%
\pgfpathlineto{\pgfqpoint{1.226511in}{2.926313in}}%
\pgfpathlineto{\pgfqpoint{1.221252in}{2.944187in}}%
\pgfpathlineto{\pgfqpoint{1.215993in}{2.962307in}}%
\pgfpathlineto{\pgfqpoint{1.210734in}{2.980679in}}%
\pgfpathlineto{\pgfqpoint{1.205474in}{2.999307in}}%
\pgfpathlineto{\pgfqpoint{1.200215in}{3.018198in}}%
\pgfpathlineto{\pgfqpoint{1.194956in}{3.037356in}}%
\pgfpathlineto{\pgfqpoint{1.189697in}{3.056788in}}%
\pgfpathlineto{\pgfqpoint{1.184437in}{3.076499in}}%
\pgfpathlineto{\pgfqpoint{1.179178in}{3.096495in}}%
\pgfpathlineto{\pgfqpoint{1.173919in}{3.116783in}}%
\pgfpathlineto{\pgfqpoint{1.168660in}{3.137370in}}%
\pgfpathlineto{\pgfqpoint{1.163401in}{3.158260in}}%
\pgfpathlineto{\pgfqpoint{1.158141in}{3.179463in}}%
\pgfpathlineto{\pgfqpoint{1.152882in}{3.200984in}}%
\pgfpathlineto{\pgfqpoint{1.147623in}{3.222830in}}%
\pgfpathlineto{\pgfqpoint{1.142364in}{3.245010in}}%
\pgfpathlineto{\pgfqpoint{1.137104in}{3.267531in}}%
\pgfpathlineto{\pgfqpoint{1.131845in}{3.290400in}}%
\pgfpathlineto{\pgfqpoint{1.126586in}{3.313627in}}%
\pgfpathlineto{\pgfqpoint{1.121327in}{3.337219in}}%
\pgfpathlineto{\pgfqpoint{1.116068in}{3.361185in}}%
\pgfpathlineto{\pgfqpoint{1.110808in}{3.385534in}}%
\pgfpathlineto{\pgfqpoint{1.105549in}{3.410275in}}%
\pgfpathlineto{\pgfqpoint{1.100290in}{3.435418in}}%
\pgfpathlineto{\pgfqpoint{1.095031in}{3.460973in}}%
\pgfpathlineto{\pgfqpoint{1.089771in}{3.486950in}}%
\pgfpathlineto{\pgfqpoint{1.084512in}{3.513359in}}%
\pgfpathlineto{\pgfqpoint{1.079253in}{3.540212in}}%
\pgfpathlineto{\pgfqpoint{1.073994in}{3.567519in}}%
\pgfpathlineto{\pgfqpoint{1.068735in}{3.595293in}}%
\pgfpathlineto{\pgfqpoint{1.063475in}{3.623545in}}%
\pgfpathlineto{\pgfqpoint{1.058216in}{3.652288in}}%
\pgfpathlineto{\pgfqpoint{1.052957in}{3.681534in}}%
\pgfpathlineto{\pgfqpoint{1.047698in}{3.711298in}}%
\pgfpathlineto{\pgfqpoint{1.042438in}{3.741592in}}%
\pgfpathlineto{\pgfqpoint{1.037179in}{3.772432in}}%
\pgfpathlineto{\pgfqpoint{1.031920in}{3.803831in}}%
\pgfpathlineto{\pgfqpoint{1.026661in}{3.835806in}}%
\pgfpathlineto{\pgfqpoint{1.021401in}{3.868373in}}%
\pgfpathlineto{\pgfqpoint{1.016142in}{3.901547in}}%
\pgfpathlineto{\pgfqpoint{1.010883in}{3.935347in}}%
\pgfpathlineto{\pgfqpoint{1.005624in}{3.969789in}}%
\pgfpathlineto{\pgfqpoint{1.000365in}{4.004894in}}%
\pgfpathlineto{\pgfqpoint{0.995105in}{4.040679in}}%
\pgfpathlineto{\pgfqpoint{0.989846in}{4.077164in}}%
\pgfpathlineto{\pgfqpoint{0.984587in}{4.114371in}}%
\pgfpathlineto{\pgfqpoint{0.979328in}{4.152322in}}%
\pgfpathlineto{\pgfqpoint{0.974068in}{4.191038in}}%
\pgfpathlineto{\pgfqpoint{0.968809in}{4.230543in}}%
\pgfpathlineto{\pgfqpoint{0.882794in}{3.577850in}}%
\pgfpathclose%
\pgfusepath{stroke,fill}%
\end{pgfscope}%
\begin{pgfscope}%
\pgfpathrectangle{\pgfqpoint{0.882794in}{0.589583in}}{\pgfqpoint{6.200000in}{4.620000in}}%
\pgfusepath{clip}%
\pgfsetbuttcap%
\pgfsetroundjoin%
\definecolor{currentfill}{rgb}{1.000000,1.000000,1.000000}%
\pgfsetfillcolor{currentfill}%
\pgfsetlinewidth{1.003750pt}%
\definecolor{currentstroke}{rgb}{1.000000,1.000000,1.000000}%
\pgfsetstrokecolor{currentstroke}%
\pgfsetdash{}{0pt}%
\pgfsys@defobject{currentmarker}{\pgfqpoint{0.874193in}{0.314383in}}{\pgfqpoint{5.867281in}{3.512581in}}{%
\pgfpathmoveto{\pgfqpoint{0.874193in}{0.314383in}}%
\pgfpathlineto{\pgfqpoint{0.874193in}{3.512581in}}%
\pgfpathlineto{\pgfqpoint{0.878488in}{3.480318in}}%
\pgfpathlineto{\pgfqpoint{0.882783in}{3.448700in}}%
\pgfpathlineto{\pgfqpoint{0.887078in}{3.417707in}}%
\pgfpathlineto{\pgfqpoint{0.891373in}{3.387321in}}%
\pgfpathlineto{\pgfqpoint{0.895668in}{3.357524in}}%
\pgfpathlineto{\pgfqpoint{0.899963in}{3.328300in}}%
\pgfpathlineto{\pgfqpoint{0.904258in}{3.299632in}}%
\pgfpathlineto{\pgfqpoint{0.908553in}{3.271504in}}%
\pgfpathlineto{\pgfqpoint{0.912848in}{3.243901in}}%
\pgfpathlineto{\pgfqpoint{0.917143in}{3.216808in}}%
\pgfpathlineto{\pgfqpoint{0.921438in}{3.190212in}}%
\pgfpathlineto{\pgfqpoint{0.925733in}{3.164099in}}%
\pgfpathlineto{\pgfqpoint{0.930028in}{3.138456in}}%
\pgfpathlineto{\pgfqpoint{0.934323in}{3.113270in}}%
\pgfpathlineto{\pgfqpoint{0.938618in}{3.088530in}}%
\pgfpathlineto{\pgfqpoint{0.942913in}{3.064223in}}%
\pgfpathlineto{\pgfqpoint{0.947208in}{3.040339in}}%
\pgfpathlineto{\pgfqpoint{0.951503in}{3.016865in}}%
\pgfpathlineto{\pgfqpoint{0.955798in}{2.993793in}}%
\pgfpathlineto{\pgfqpoint{0.960094in}{2.971111in}}%
\pgfpathlineto{\pgfqpoint{0.964389in}{2.948810in}}%
\pgfpathlineto{\pgfqpoint{0.968684in}{2.926880in}}%
\pgfpathlineto{\pgfqpoint{0.972979in}{2.905313in}}%
\pgfpathlineto{\pgfqpoint{0.977274in}{2.884098in}}%
\pgfpathlineto{\pgfqpoint{0.981569in}{2.863228in}}%
\pgfpathlineto{\pgfqpoint{0.985864in}{2.842695in}}%
\pgfpathlineto{\pgfqpoint{0.990159in}{2.822489in}}%
\pgfpathlineto{\pgfqpoint{0.994454in}{2.802604in}}%
\pgfpathlineto{\pgfqpoint{0.998749in}{2.783032in}}%
\pgfpathlineto{\pgfqpoint{1.003044in}{2.763765in}}%
\pgfpathlineto{\pgfqpoint{1.007339in}{2.744797in}}%
\pgfpathlineto{\pgfqpoint{1.011634in}{2.726120in}}%
\pgfpathlineto{\pgfqpoint{1.015929in}{2.707728in}}%
\pgfpathlineto{\pgfqpoint{1.020224in}{2.689615in}}%
\pgfpathlineto{\pgfqpoint{1.024519in}{2.671774in}}%
\pgfpathlineto{\pgfqpoint{1.028814in}{2.654198in}}%
\pgfpathlineto{\pgfqpoint{1.033109in}{2.636883in}}%
\pgfpathlineto{\pgfqpoint{1.037404in}{2.619822in}}%
\pgfpathlineto{\pgfqpoint{1.041699in}{2.603010in}}%
\pgfpathlineto{\pgfqpoint{1.045994in}{2.586441in}}%
\pgfpathlineto{\pgfqpoint{1.050289in}{2.570111in}}%
\pgfpathlineto{\pgfqpoint{1.054584in}{2.554014in}}%
\pgfpathlineto{\pgfqpoint{1.058879in}{2.538144in}}%
\pgfpathlineto{\pgfqpoint{1.063174in}{2.522498in}}%
\pgfpathlineto{\pgfqpoint{1.067469in}{2.507071in}}%
\pgfpathlineto{\pgfqpoint{1.071764in}{2.491858in}}%
\pgfpathlineto{\pgfqpoint{1.076059in}{2.476854in}}%
\pgfpathlineto{\pgfqpoint{1.080354in}{2.462056in}}%
\pgfpathlineto{\pgfqpoint{1.084650in}{2.447459in}}%
\pgfpathlineto{\pgfqpoint{1.088945in}{2.433059in}}%
\pgfpathlineto{\pgfqpoint{1.093240in}{2.418852in}}%
\pgfpathlineto{\pgfqpoint{1.097535in}{2.404835in}}%
\pgfpathlineto{\pgfqpoint{1.101830in}{2.391002in}}%
\pgfpathlineto{\pgfqpoint{1.106125in}{2.377352in}}%
\pgfpathlineto{\pgfqpoint{1.110420in}{2.363880in}}%
\pgfpathlineto{\pgfqpoint{1.114715in}{2.350583in}}%
\pgfpathlineto{\pgfqpoint{1.119010in}{2.337457in}}%
\pgfpathlineto{\pgfqpoint{1.123305in}{2.324500in}}%
\pgfpathlineto{\pgfqpoint{1.127600in}{2.311707in}}%
\pgfpathlineto{\pgfqpoint{1.131895in}{2.299076in}}%
\pgfpathlineto{\pgfqpoint{1.136190in}{2.286604in}}%
\pgfpathlineto{\pgfqpoint{1.140485in}{2.274288in}}%
\pgfpathlineto{\pgfqpoint{1.144780in}{2.262124in}}%
\pgfpathlineto{\pgfqpoint{1.149075in}{2.250111in}}%
\pgfpathlineto{\pgfqpoint{1.153370in}{2.238245in}}%
\pgfpathlineto{\pgfqpoint{1.157665in}{2.226523in}}%
\pgfpathlineto{\pgfqpoint{1.161960in}{2.214944in}}%
\pgfpathlineto{\pgfqpoint{1.166255in}{2.203503in}}%
\pgfpathlineto{\pgfqpoint{1.170550in}{2.192200in}}%
\pgfpathlineto{\pgfqpoint{1.174845in}{2.181031in}}%
\pgfpathlineto{\pgfqpoint{1.179140in}{2.169995in}}%
\pgfpathlineto{\pgfqpoint{1.183435in}{2.159088in}}%
\pgfpathlineto{\pgfqpoint{1.187730in}{2.148308in}}%
\pgfpathlineto{\pgfqpoint{1.192025in}{2.137654in}}%
\pgfpathlineto{\pgfqpoint{1.196320in}{2.127123in}}%
\pgfpathlineto{\pgfqpoint{1.200615in}{2.116712in}}%
\pgfpathlineto{\pgfqpoint{1.204910in}{2.106421in}}%
\pgfpathlineto{\pgfqpoint{1.209206in}{2.096246in}}%
\pgfpathlineto{\pgfqpoint{1.213501in}{2.086187in}}%
\pgfpathlineto{\pgfqpoint{1.217796in}{2.076240in}}%
\pgfpathlineto{\pgfqpoint{1.222091in}{2.066405in}}%
\pgfpathlineto{\pgfqpoint{1.226386in}{2.056678in}}%
\pgfpathlineto{\pgfqpoint{1.230681in}{2.047059in}}%
\pgfpathlineto{\pgfqpoint{1.234976in}{2.037546in}}%
\pgfpathlineto{\pgfqpoint{1.239271in}{2.028136in}}%
\pgfpathlineto{\pgfqpoint{1.243566in}{2.018829in}}%
\pgfpathlineto{\pgfqpoint{1.247861in}{2.009622in}}%
\pgfpathlineto{\pgfqpoint{1.252156in}{2.000514in}}%
\pgfpathlineto{\pgfqpoint{1.256451in}{1.991504in}}%
\pgfpathlineto{\pgfqpoint{1.260746in}{1.982589in}}%
\pgfpathlineto{\pgfqpoint{1.265041in}{1.973769in}}%
\pgfpathlineto{\pgfqpoint{1.269336in}{1.965041in}}%
\pgfpathlineto{\pgfqpoint{1.273631in}{1.956405in}}%
\pgfpathlineto{\pgfqpoint{1.277926in}{1.947859in}}%
\pgfpathlineto{\pgfqpoint{1.282221in}{1.939401in}}%
\pgfpathlineto{\pgfqpoint{1.286516in}{1.931030in}}%
\pgfpathlineto{\pgfqpoint{1.290811in}{1.922745in}}%
\pgfpathlineto{\pgfqpoint{1.295106in}{1.914544in}}%
\pgfpathlineto{\pgfqpoint{1.299401in}{1.906427in}}%
\pgfpathlineto{\pgfqpoint{1.303696in}{1.898392in}}%
\pgfpathlineto{\pgfqpoint{1.307991in}{1.890437in}}%
\pgfpathlineto{\pgfqpoint{1.312286in}{1.882562in}}%
\pgfpathlineto{\pgfqpoint{1.316581in}{1.874765in}}%
\pgfpathlineto{\pgfqpoint{1.320876in}{1.867046in}}%
\pgfpathlineto{\pgfqpoint{1.325171in}{1.859402in}}%
\pgfpathlineto{\pgfqpoint{1.329466in}{1.851833in}}%
\pgfpathlineto{\pgfqpoint{1.333762in}{1.844338in}}%
\pgfpathlineto{\pgfqpoint{1.338057in}{1.836916in}}%
\pgfpathlineto{\pgfqpoint{1.342352in}{1.829565in}}%
\pgfpathlineto{\pgfqpoint{1.346647in}{1.822285in}}%
\pgfpathlineto{\pgfqpoint{1.350942in}{1.815075in}}%
\pgfpathlineto{\pgfqpoint{1.355237in}{1.807933in}}%
\pgfpathlineto{\pgfqpoint{1.359532in}{1.800859in}}%
\pgfpathlineto{\pgfqpoint{1.363827in}{1.793852in}}%
\pgfpathlineto{\pgfqpoint{1.368122in}{1.786910in}}%
\pgfpathlineto{\pgfqpoint{1.372417in}{1.780033in}}%
\pgfpathlineto{\pgfqpoint{1.376712in}{1.773221in}}%
\pgfpathlineto{\pgfqpoint{1.381007in}{1.766471in}}%
\pgfpathlineto{\pgfqpoint{1.385302in}{1.759783in}}%
\pgfpathlineto{\pgfqpoint{1.389597in}{1.753157in}}%
\pgfpathlineto{\pgfqpoint{1.393892in}{1.746591in}}%
\pgfpathlineto{\pgfqpoint{1.398187in}{1.740085in}}%
\pgfpathlineto{\pgfqpoint{1.402482in}{1.733637in}}%
\pgfpathlineto{\pgfqpoint{1.406777in}{1.727248in}}%
\pgfpathlineto{\pgfqpoint{1.411072in}{1.720916in}}%
\pgfpathlineto{\pgfqpoint{1.415367in}{1.714641in}}%
\pgfpathlineto{\pgfqpoint{1.419662in}{1.708421in}}%
\pgfpathlineto{\pgfqpoint{1.423957in}{1.702256in}}%
\pgfpathlineto{\pgfqpoint{1.428252in}{1.696146in}}%
\pgfpathlineto{\pgfqpoint{1.432547in}{1.690089in}}%
\pgfpathlineto{\pgfqpoint{1.436842in}{1.684085in}}%
\pgfpathlineto{\pgfqpoint{1.441137in}{1.678133in}}%
\pgfpathlineto{\pgfqpoint{1.445432in}{1.672233in}}%
\pgfpathlineto{\pgfqpoint{1.449727in}{1.666383in}}%
\pgfpathlineto{\pgfqpoint{1.454022in}{1.660584in}}%
\pgfpathlineto{\pgfqpoint{1.458318in}{1.654834in}}%
\pgfpathlineto{\pgfqpoint{1.462613in}{1.649133in}}%
\pgfpathlineto{\pgfqpoint{1.466908in}{1.643480in}}%
\pgfpathlineto{\pgfqpoint{1.471203in}{1.637875in}}%
\pgfpathlineto{\pgfqpoint{1.475498in}{1.632318in}}%
\pgfpathlineto{\pgfqpoint{1.479793in}{1.626806in}}%
\pgfpathlineto{\pgfqpoint{1.484088in}{1.621341in}}%
\pgfpathlineto{\pgfqpoint{1.488383in}{1.615921in}}%
\pgfpathlineto{\pgfqpoint{1.492678in}{1.610545in}}%
\pgfpathlineto{\pgfqpoint{1.496973in}{1.605214in}}%
\pgfpathlineto{\pgfqpoint{1.501268in}{1.599927in}}%
\pgfpathlineto{\pgfqpoint{1.505563in}{1.594682in}}%
\pgfpathlineto{\pgfqpoint{1.509858in}{1.589481in}}%
\pgfpathlineto{\pgfqpoint{1.514153in}{1.584321in}}%
\pgfpathlineto{\pgfqpoint{1.518448in}{1.579203in}}%
\pgfpathlineto{\pgfqpoint{1.522743in}{1.574126in}}%
\pgfpathlineto{\pgfqpoint{1.527038in}{1.569090in}}%
\pgfpathlineto{\pgfqpoint{1.531333in}{1.564094in}}%
\pgfpathlineto{\pgfqpoint{1.535628in}{1.559137in}}%
\pgfpathlineto{\pgfqpoint{1.539923in}{1.554219in}}%
\pgfpathlineto{\pgfqpoint{1.544218in}{1.549341in}}%
\pgfpathlineto{\pgfqpoint{1.548513in}{1.544500in}}%
\pgfpathlineto{\pgfqpoint{1.552808in}{1.539698in}}%
\pgfpathlineto{\pgfqpoint{1.557103in}{1.534932in}}%
\pgfpathlineto{\pgfqpoint{1.561398in}{1.530204in}}%
\pgfpathlineto{\pgfqpoint{1.565693in}{1.525512in}}%
\pgfpathlineto{\pgfqpoint{1.569988in}{1.520856in}}%
\pgfpathlineto{\pgfqpoint{1.574283in}{1.516236in}}%
\pgfpathlineto{\pgfqpoint{1.578578in}{1.511651in}}%
\pgfpathlineto{\pgfqpoint{1.582874in}{1.507101in}}%
\pgfpathlineto{\pgfqpoint{1.587169in}{1.502585in}}%
\pgfpathlineto{\pgfqpoint{1.591464in}{1.498104in}}%
\pgfpathlineto{\pgfqpoint{1.595759in}{1.493656in}}%
\pgfpathlineto{\pgfqpoint{1.600054in}{1.489241in}}%
\pgfpathlineto{\pgfqpoint{1.604349in}{1.484859in}}%
\pgfpathlineto{\pgfqpoint{1.608644in}{1.480510in}}%
\pgfpathlineto{\pgfqpoint{1.612939in}{1.476193in}}%
\pgfpathlineto{\pgfqpoint{1.617234in}{1.471908in}}%
\pgfpathlineto{\pgfqpoint{1.621529in}{1.467655in}}%
\pgfpathlineto{\pgfqpoint{1.625824in}{1.463432in}}%
\pgfpathlineto{\pgfqpoint{1.630119in}{1.459241in}}%
\pgfpathlineto{\pgfqpoint{1.634414in}{1.455080in}}%
\pgfpathlineto{\pgfqpoint{1.638709in}{1.450949in}}%
\pgfpathlineto{\pgfqpoint{1.643004in}{1.446847in}}%
\pgfpathlineto{\pgfqpoint{1.647299in}{1.442776in}}%
\pgfpathlineto{\pgfqpoint{1.651594in}{1.438733in}}%
\pgfpathlineto{\pgfqpoint{1.655889in}{1.434719in}}%
\pgfpathlineto{\pgfqpoint{1.660184in}{1.430734in}}%
\pgfpathlineto{\pgfqpoint{1.664479in}{1.426778in}}%
\pgfpathlineto{\pgfqpoint{1.668774in}{1.422849in}}%
\pgfpathlineto{\pgfqpoint{1.673069in}{1.418947in}}%
\pgfpathlineto{\pgfqpoint{1.677364in}{1.415073in}}%
\pgfpathlineto{\pgfqpoint{1.681659in}{1.411227in}}%
\pgfpathlineto{\pgfqpoint{1.685954in}{1.407407in}}%
\pgfpathlineto{\pgfqpoint{1.690249in}{1.403613in}}%
\pgfpathlineto{\pgfqpoint{1.694544in}{1.399846in}}%
\pgfpathlineto{\pgfqpoint{1.698839in}{1.396105in}}%
\pgfpathlineto{\pgfqpoint{1.703134in}{1.392389in}}%
\pgfpathlineto{\pgfqpoint{1.707430in}{1.388699in}}%
\pgfpathlineto{\pgfqpoint{1.711725in}{1.385034in}}%
\pgfpathlineto{\pgfqpoint{1.716020in}{1.381394in}}%
\pgfpathlineto{\pgfqpoint{1.720315in}{1.377778in}}%
\pgfpathlineto{\pgfqpoint{1.724610in}{1.374187in}}%
\pgfpathlineto{\pgfqpoint{1.728905in}{1.370621in}}%
\pgfpathlineto{\pgfqpoint{1.733200in}{1.367078in}}%
\pgfpathlineto{\pgfqpoint{1.737495in}{1.363559in}}%
\pgfpathlineto{\pgfqpoint{1.741790in}{1.360063in}}%
\pgfpathlineto{\pgfqpoint{1.746085in}{1.356590in}}%
\pgfpathlineto{\pgfqpoint{1.750380in}{1.353141in}}%
\pgfpathlineto{\pgfqpoint{1.754675in}{1.349714in}}%
\pgfpathlineto{\pgfqpoint{1.758970in}{1.346310in}}%
\pgfpathlineto{\pgfqpoint{1.763265in}{1.342928in}}%
\pgfpathlineto{\pgfqpoint{1.767560in}{1.339568in}}%
\pgfpathlineto{\pgfqpoint{1.771855in}{1.336230in}}%
\pgfpathlineto{\pgfqpoint{1.776150in}{1.332914in}}%
\pgfpathlineto{\pgfqpoint{1.780445in}{1.329619in}}%
\pgfpathlineto{\pgfqpoint{1.784740in}{1.326345in}}%
\pgfpathlineto{\pgfqpoint{1.789035in}{1.323093in}}%
\pgfpathlineto{\pgfqpoint{1.793330in}{1.319861in}}%
\pgfpathlineto{\pgfqpoint{1.797625in}{1.316650in}}%
\pgfpathlineto{\pgfqpoint{1.801920in}{1.313460in}}%
\pgfpathlineto{\pgfqpoint{1.806215in}{1.310289in}}%
\pgfpathlineto{\pgfqpoint{1.810510in}{1.307139in}}%
\pgfpathlineto{\pgfqpoint{1.814805in}{1.304008in}}%
\pgfpathlineto{\pgfqpoint{1.819100in}{1.300898in}}%
\pgfpathlineto{\pgfqpoint{1.823395in}{1.297806in}}%
\pgfpathlineto{\pgfqpoint{1.827690in}{1.294734in}}%
\pgfpathlineto{\pgfqpoint{1.831986in}{1.291682in}}%
\pgfpathlineto{\pgfqpoint{1.836281in}{1.288648in}}%
\pgfpathlineto{\pgfqpoint{1.840576in}{1.285633in}}%
\pgfpathlineto{\pgfqpoint{1.844871in}{1.282636in}}%
\pgfpathlineto{\pgfqpoint{1.849166in}{1.279658in}}%
\pgfpathlineto{\pgfqpoint{1.853461in}{1.276698in}}%
\pgfpathlineto{\pgfqpoint{1.857756in}{1.273757in}}%
\pgfpathlineto{\pgfqpoint{1.862051in}{1.270833in}}%
\pgfpathlineto{\pgfqpoint{1.866346in}{1.267927in}}%
\pgfpathlineto{\pgfqpoint{1.870641in}{1.265039in}}%
\pgfpathlineto{\pgfqpoint{1.874936in}{1.262168in}}%
\pgfpathlineto{\pgfqpoint{1.879231in}{1.259314in}}%
\pgfpathlineto{\pgfqpoint{1.883526in}{1.256477in}}%
\pgfpathlineto{\pgfqpoint{1.887821in}{1.253658in}}%
\pgfpathlineto{\pgfqpoint{1.892116in}{1.250855in}}%
\pgfpathlineto{\pgfqpoint{1.896411in}{1.248069in}}%
\pgfpathlineto{\pgfqpoint{1.900706in}{1.245300in}}%
\pgfpathlineto{\pgfqpoint{1.905001in}{1.242547in}}%
\pgfpathlineto{\pgfqpoint{1.909296in}{1.239810in}}%
\pgfpathlineto{\pgfqpoint{1.913591in}{1.237089in}}%
\pgfpathlineto{\pgfqpoint{1.917886in}{1.234384in}}%
\pgfpathlineto{\pgfqpoint{1.922181in}{1.231695in}}%
\pgfpathlineto{\pgfqpoint{1.926476in}{1.229022in}}%
\pgfpathlineto{\pgfqpoint{1.930771in}{1.226364in}}%
\pgfpathlineto{\pgfqpoint{1.935066in}{1.223721in}}%
\pgfpathlineto{\pgfqpoint{1.939361in}{1.221094in}}%
\pgfpathlineto{\pgfqpoint{1.943656in}{1.218482in}}%
\pgfpathlineto{\pgfqpoint{1.947951in}{1.215885in}}%
\pgfpathlineto{\pgfqpoint{1.952246in}{1.213303in}}%
\pgfpathlineto{\pgfqpoint{1.956542in}{1.210736in}}%
\pgfpathlineto{\pgfqpoint{1.960837in}{1.208183in}}%
\pgfpathlineto{\pgfqpoint{1.965132in}{1.205645in}}%
\pgfpathlineto{\pgfqpoint{1.969427in}{1.203121in}}%
\pgfpathlineto{\pgfqpoint{1.973722in}{1.200611in}}%
\pgfpathlineto{\pgfqpoint{1.978017in}{1.198116in}}%
\pgfpathlineto{\pgfqpoint{1.982312in}{1.195634in}}%
\pgfpathlineto{\pgfqpoint{1.986607in}{1.193167in}}%
\pgfpathlineto{\pgfqpoint{1.990902in}{1.190713in}}%
\pgfpathlineto{\pgfqpoint{1.995197in}{1.188273in}}%
\pgfpathlineto{\pgfqpoint{1.999492in}{1.185846in}}%
\pgfpathlineto{\pgfqpoint{2.003787in}{1.183433in}}%
\pgfpathlineto{\pgfqpoint{2.008082in}{1.181033in}}%
\pgfpathlineto{\pgfqpoint{2.012377in}{1.178646in}}%
\pgfpathlineto{\pgfqpoint{2.016672in}{1.176273in}}%
\pgfpathlineto{\pgfqpoint{2.020967in}{1.173912in}}%
\pgfpathlineto{\pgfqpoint{2.025262in}{1.171565in}}%
\pgfpathlineto{\pgfqpoint{2.029557in}{1.169230in}}%
\pgfpathlineto{\pgfqpoint{2.033852in}{1.166908in}}%
\pgfpathlineto{\pgfqpoint{2.038147in}{1.164598in}}%
\pgfpathlineto{\pgfqpoint{2.042442in}{1.162301in}}%
\pgfpathlineto{\pgfqpoint{2.046737in}{1.160017in}}%
\pgfpathlineto{\pgfqpoint{2.051032in}{1.157744in}}%
\pgfpathlineto{\pgfqpoint{2.055327in}{1.155484in}}%
\pgfpathlineto{\pgfqpoint{2.059622in}{1.153236in}}%
\pgfpathlineto{\pgfqpoint{2.063917in}{1.151000in}}%
\pgfpathlineto{\pgfqpoint{2.068212in}{1.148775in}}%
\pgfpathlineto{\pgfqpoint{2.072507in}{1.146563in}}%
\pgfpathlineto{\pgfqpoint{2.076802in}{1.144362in}}%
\pgfpathlineto{\pgfqpoint{2.081098in}{1.142173in}}%
\pgfpathlineto{\pgfqpoint{2.085393in}{1.139995in}}%
\pgfpathlineto{\pgfqpoint{2.089688in}{1.137829in}}%
\pgfpathlineto{\pgfqpoint{2.093983in}{1.135674in}}%
\pgfpathlineto{\pgfqpoint{2.098278in}{1.133531in}}%
\pgfpathlineto{\pgfqpoint{2.102573in}{1.131398in}}%
\pgfpathlineto{\pgfqpoint{2.106868in}{1.129277in}}%
\pgfpathlineto{\pgfqpoint{2.111163in}{1.127166in}}%
\pgfpathlineto{\pgfqpoint{2.115458in}{1.125067in}}%
\pgfpathlineto{\pgfqpoint{2.119753in}{1.122978in}}%
\pgfpathlineto{\pgfqpoint{2.124048in}{1.120900in}}%
\pgfpathlineto{\pgfqpoint{2.128343in}{1.118833in}}%
\pgfpathlineto{\pgfqpoint{2.132638in}{1.116776in}}%
\pgfpathlineto{\pgfqpoint{2.136933in}{1.114730in}}%
\pgfpathlineto{\pgfqpoint{2.141228in}{1.112694in}}%
\pgfpathlineto{\pgfqpoint{2.145523in}{1.110669in}}%
\pgfpathlineto{\pgfqpoint{2.149818in}{1.108653in}}%
\pgfpathlineto{\pgfqpoint{2.154113in}{1.106648in}}%
\pgfpathlineto{\pgfqpoint{2.158408in}{1.104653in}}%
\pgfpathlineto{\pgfqpoint{2.162703in}{1.102668in}}%
\pgfpathlineto{\pgfqpoint{2.166998in}{1.100693in}}%
\pgfpathlineto{\pgfqpoint{2.171293in}{1.098728in}}%
\pgfpathlineto{\pgfqpoint{2.175588in}{1.096773in}}%
\pgfpathlineto{\pgfqpoint{2.179883in}{1.094827in}}%
\pgfpathlineto{\pgfqpoint{2.184178in}{1.092891in}}%
\pgfpathlineto{\pgfqpoint{2.188473in}{1.090965in}}%
\pgfpathlineto{\pgfqpoint{2.192768in}{1.089048in}}%
\pgfpathlineto{\pgfqpoint{2.197063in}{1.087141in}}%
\pgfpathlineto{\pgfqpoint{2.201358in}{1.085242in}}%
\pgfpathlineto{\pgfqpoint{2.205654in}{1.083354in}}%
\pgfpathlineto{\pgfqpoint{2.209949in}{1.081474in}}%
\pgfpathlineto{\pgfqpoint{2.214244in}{1.079604in}}%
\pgfpathlineto{\pgfqpoint{2.218539in}{1.077743in}}%
\pgfpathlineto{\pgfqpoint{2.222834in}{1.075890in}}%
\pgfpathlineto{\pgfqpoint{2.227129in}{1.074047in}}%
\pgfpathlineto{\pgfqpoint{2.231424in}{1.072213in}}%
\pgfpathlineto{\pgfqpoint{2.235719in}{1.070387in}}%
\pgfpathlineto{\pgfqpoint{2.240014in}{1.068570in}}%
\pgfpathlineto{\pgfqpoint{2.244309in}{1.066762in}}%
\pgfpathlineto{\pgfqpoint{2.248604in}{1.064963in}}%
\pgfpathlineto{\pgfqpoint{2.252899in}{1.063172in}}%
\pgfpathlineto{\pgfqpoint{2.257194in}{1.061390in}}%
\pgfpathlineto{\pgfqpoint{2.261489in}{1.059616in}}%
\pgfpathlineto{\pgfqpoint{2.265784in}{1.057851in}}%
\pgfpathlineto{\pgfqpoint{2.270079in}{1.056094in}}%
\pgfpathlineto{\pgfqpoint{2.274374in}{1.054345in}}%
\pgfpathlineto{\pgfqpoint{2.278669in}{1.052604in}}%
\pgfpathlineto{\pgfqpoint{2.282964in}{1.050872in}}%
\pgfpathlineto{\pgfqpoint{2.287259in}{1.049148in}}%
\pgfpathlineto{\pgfqpoint{2.291554in}{1.047431in}}%
\pgfpathlineto{\pgfqpoint{2.295849in}{1.045723in}}%
\pgfpathlineto{\pgfqpoint{2.300144in}{1.044023in}}%
\pgfpathlineto{\pgfqpoint{2.304439in}{1.042331in}}%
\pgfpathlineto{\pgfqpoint{2.308734in}{1.040646in}}%
\pgfpathlineto{\pgfqpoint{2.313029in}{1.038969in}}%
\pgfpathlineto{\pgfqpoint{2.317324in}{1.037300in}}%
\pgfpathlineto{\pgfqpoint{2.321619in}{1.035639in}}%
\pgfpathlineto{\pgfqpoint{2.325914in}{1.033985in}}%
\pgfpathlineto{\pgfqpoint{2.330210in}{1.032339in}}%
\pgfpathlineto{\pgfqpoint{2.334505in}{1.030700in}}%
\pgfpathlineto{\pgfqpoint{2.338800in}{1.029069in}}%
\pgfpathlineto{\pgfqpoint{2.343095in}{1.027445in}}%
\pgfpathlineto{\pgfqpoint{2.347390in}{1.025829in}}%
\pgfpathlineto{\pgfqpoint{2.351685in}{1.024220in}}%
\pgfpathlineto{\pgfqpoint{2.355980in}{1.022618in}}%
\pgfpathlineto{\pgfqpoint{2.360275in}{1.021023in}}%
\pgfpathlineto{\pgfqpoint{2.364570in}{1.019436in}}%
\pgfpathlineto{\pgfqpoint{2.368865in}{1.017855in}}%
\pgfpathlineto{\pgfqpoint{2.373160in}{1.016282in}}%
\pgfpathlineto{\pgfqpoint{2.377455in}{1.014716in}}%
\pgfpathlineto{\pgfqpoint{2.381750in}{1.013156in}}%
\pgfpathlineto{\pgfqpoint{2.386045in}{1.011604in}}%
\pgfpathlineto{\pgfqpoint{2.390340in}{1.010058in}}%
\pgfpathlineto{\pgfqpoint{2.394635in}{1.008520in}}%
\pgfpathlineto{\pgfqpoint{2.398930in}{1.006988in}}%
\pgfpathlineto{\pgfqpoint{2.403225in}{1.005463in}}%
\pgfpathlineto{\pgfqpoint{2.407520in}{1.003944in}}%
\pgfpathlineto{\pgfqpoint{2.411815in}{1.002433in}}%
\pgfpathlineto{\pgfqpoint{2.416110in}{1.000927in}}%
\pgfpathlineto{\pgfqpoint{2.420405in}{0.999429in}}%
\pgfpathlineto{\pgfqpoint{2.424700in}{0.997937in}}%
\pgfpathlineto{\pgfqpoint{2.428995in}{0.996451in}}%
\pgfpathlineto{\pgfqpoint{2.433290in}{0.994972in}}%
\pgfpathlineto{\pgfqpoint{2.437585in}{0.993499in}}%
\pgfpathlineto{\pgfqpoint{2.441880in}{0.992033in}}%
\pgfpathlineto{\pgfqpoint{2.446175in}{0.990573in}}%
\pgfpathlineto{\pgfqpoint{2.450470in}{0.989119in}}%
\pgfpathlineto{\pgfqpoint{2.454766in}{0.987672in}}%
\pgfpathlineto{\pgfqpoint{2.459061in}{0.986230in}}%
\pgfpathlineto{\pgfqpoint{2.463356in}{0.984795in}}%
\pgfpathlineto{\pgfqpoint{2.467651in}{0.983366in}}%
\pgfpathlineto{\pgfqpoint{2.471946in}{0.981943in}}%
\pgfpathlineto{\pgfqpoint{2.476241in}{0.980526in}}%
\pgfpathlineto{\pgfqpoint{2.480536in}{0.979115in}}%
\pgfpathlineto{\pgfqpoint{2.484831in}{0.977710in}}%
\pgfpathlineto{\pgfqpoint{2.489126in}{0.976311in}}%
\pgfpathlineto{\pgfqpoint{2.493421in}{0.974918in}}%
\pgfpathlineto{\pgfqpoint{2.497716in}{0.973531in}}%
\pgfpathlineto{\pgfqpoint{2.502011in}{0.972149in}}%
\pgfpathlineto{\pgfqpoint{2.506306in}{0.970774in}}%
\pgfpathlineto{\pgfqpoint{2.510601in}{0.969404in}}%
\pgfpathlineto{\pgfqpoint{2.514896in}{0.968039in}}%
\pgfpathlineto{\pgfqpoint{2.519191in}{0.966681in}}%
\pgfpathlineto{\pgfqpoint{2.523486in}{0.965328in}}%
\pgfpathlineto{\pgfqpoint{2.527781in}{0.963981in}}%
\pgfpathlineto{\pgfqpoint{2.532076in}{0.962639in}}%
\pgfpathlineto{\pgfqpoint{2.536371in}{0.961303in}}%
\pgfpathlineto{\pgfqpoint{2.540666in}{0.959972in}}%
\pgfpathlineto{\pgfqpoint{2.544961in}{0.958647in}}%
\pgfpathlineto{\pgfqpoint{2.549256in}{0.957327in}}%
\pgfpathlineto{\pgfqpoint{2.553551in}{0.956012in}}%
\pgfpathlineto{\pgfqpoint{2.557846in}{0.954703in}}%
\pgfpathlineto{\pgfqpoint{2.562141in}{0.953399in}}%
\pgfpathlineto{\pgfqpoint{2.566436in}{0.952101in}}%
\pgfpathlineto{\pgfqpoint{2.570731in}{0.950808in}}%
\pgfpathlineto{\pgfqpoint{2.575026in}{0.949520in}}%
\pgfpathlineto{\pgfqpoint{2.579322in}{0.948237in}}%
\pgfpathlineto{\pgfqpoint{2.583617in}{0.946959in}}%
\pgfpathlineto{\pgfqpoint{2.587912in}{0.945687in}}%
\pgfpathlineto{\pgfqpoint{2.592207in}{0.944420in}}%
\pgfpathlineto{\pgfqpoint{2.596502in}{0.943157in}}%
\pgfpathlineto{\pgfqpoint{2.600797in}{0.941900in}}%
\pgfpathlineto{\pgfqpoint{2.605092in}{0.940648in}}%
\pgfpathlineto{\pgfqpoint{2.609387in}{0.939401in}}%
\pgfpathlineto{\pgfqpoint{2.613682in}{0.938158in}}%
\pgfpathlineto{\pgfqpoint{2.617977in}{0.936921in}}%
\pgfpathlineto{\pgfqpoint{2.622272in}{0.935689in}}%
\pgfpathlineto{\pgfqpoint{2.626567in}{0.934461in}}%
\pgfpathlineto{\pgfqpoint{2.630862in}{0.933238in}}%
\pgfpathlineto{\pgfqpoint{2.635157in}{0.932020in}}%
\pgfpathlineto{\pgfqpoint{2.639452in}{0.930807in}}%
\pgfpathlineto{\pgfqpoint{2.643747in}{0.929599in}}%
\pgfpathlineto{\pgfqpoint{2.648042in}{0.928395in}}%
\pgfpathlineto{\pgfqpoint{2.652337in}{0.927196in}}%
\pgfpathlineto{\pgfqpoint{2.656632in}{0.926002in}}%
\pgfpathlineto{\pgfqpoint{2.660927in}{0.924812in}}%
\pgfpathlineto{\pgfqpoint{2.665222in}{0.923627in}}%
\pgfpathlineto{\pgfqpoint{2.669517in}{0.922447in}}%
\pgfpathlineto{\pgfqpoint{2.673812in}{0.921271in}}%
\pgfpathlineto{\pgfqpoint{2.678107in}{0.920100in}}%
\pgfpathlineto{\pgfqpoint{2.682402in}{0.918933in}}%
\pgfpathlineto{\pgfqpoint{2.686697in}{0.917771in}}%
\pgfpathlineto{\pgfqpoint{2.690992in}{0.916613in}}%
\pgfpathlineto{\pgfqpoint{2.695287in}{0.915460in}}%
\pgfpathlineto{\pgfqpoint{2.699582in}{0.914311in}}%
\pgfpathlineto{\pgfqpoint{2.703878in}{0.913166in}}%
\pgfpathlineto{\pgfqpoint{2.708173in}{0.912026in}}%
\pgfpathlineto{\pgfqpoint{2.712468in}{0.910890in}}%
\pgfpathlineto{\pgfqpoint{2.716763in}{0.909758in}}%
\pgfpathlineto{\pgfqpoint{2.721058in}{0.908631in}}%
\pgfpathlineto{\pgfqpoint{2.725353in}{0.907508in}}%
\pgfpathlineto{\pgfqpoint{2.729648in}{0.906389in}}%
\pgfpathlineto{\pgfqpoint{2.733943in}{0.905274in}}%
\pgfpathlineto{\pgfqpoint{2.738238in}{0.904164in}}%
\pgfpathlineto{\pgfqpoint{2.742533in}{0.903058in}}%
\pgfpathlineto{\pgfqpoint{2.746828in}{0.901956in}}%
\pgfpathlineto{\pgfqpoint{2.751123in}{0.900858in}}%
\pgfpathlineto{\pgfqpoint{2.755418in}{0.899764in}}%
\pgfpathlineto{\pgfqpoint{2.759713in}{0.898674in}}%
\pgfpathlineto{\pgfqpoint{2.764008in}{0.897588in}}%
\pgfpathlineto{\pgfqpoint{2.768303in}{0.896506in}}%
\pgfpathlineto{\pgfqpoint{2.772598in}{0.895429in}}%
\pgfpathlineto{\pgfqpoint{2.776893in}{0.894355in}}%
\pgfpathlineto{\pgfqpoint{2.781188in}{0.893285in}}%
\pgfpathlineto{\pgfqpoint{2.785483in}{0.892219in}}%
\pgfpathlineto{\pgfqpoint{2.789778in}{0.891157in}}%
\pgfpathlineto{\pgfqpoint{2.794073in}{0.890099in}}%
\pgfpathlineto{\pgfqpoint{2.798368in}{0.889045in}}%
\pgfpathlineto{\pgfqpoint{2.802663in}{0.887995in}}%
\pgfpathlineto{\pgfqpoint{2.806958in}{0.886948in}}%
\pgfpathlineto{\pgfqpoint{2.811253in}{0.885905in}}%
\pgfpathlineto{\pgfqpoint{2.815548in}{0.884867in}}%
\pgfpathlineto{\pgfqpoint{2.819843in}{0.883831in}}%
\pgfpathlineto{\pgfqpoint{2.824138in}{0.882800in}}%
\pgfpathlineto{\pgfqpoint{2.828434in}{0.881772in}}%
\pgfpathlineto{\pgfqpoint{2.832729in}{0.880749in}}%
\pgfpathlineto{\pgfqpoint{2.837024in}{0.879728in}}%
\pgfpathlineto{\pgfqpoint{2.841319in}{0.878712in}}%
\pgfpathlineto{\pgfqpoint{2.845614in}{0.877699in}}%
\pgfpathlineto{\pgfqpoint{2.849909in}{0.876690in}}%
\pgfpathlineto{\pgfqpoint{2.854204in}{0.875684in}}%
\pgfpathlineto{\pgfqpoint{2.858499in}{0.874682in}}%
\pgfpathlineto{\pgfqpoint{2.862794in}{0.873683in}}%
\pgfpathlineto{\pgfqpoint{2.867089in}{0.872688in}}%
\pgfpathlineto{\pgfqpoint{2.871384in}{0.871697in}}%
\pgfpathlineto{\pgfqpoint{2.875679in}{0.870709in}}%
\pgfpathlineto{\pgfqpoint{2.879974in}{0.869725in}}%
\pgfpathlineto{\pgfqpoint{2.884269in}{0.868744in}}%
\pgfpathlineto{\pgfqpoint{2.888564in}{0.867766in}}%
\pgfpathlineto{\pgfqpoint{2.892859in}{0.866792in}}%
\pgfpathlineto{\pgfqpoint{2.897154in}{0.865821in}}%
\pgfpathlineto{\pgfqpoint{2.901449in}{0.864854in}}%
\pgfpathlineto{\pgfqpoint{2.905744in}{0.863890in}}%
\pgfpathlineto{\pgfqpoint{2.910039in}{0.862930in}}%
\pgfpathlineto{\pgfqpoint{2.914334in}{0.861973in}}%
\pgfpathlineto{\pgfqpoint{2.918629in}{0.861019in}}%
\pgfpathlineto{\pgfqpoint{2.922924in}{0.860069in}}%
\pgfpathlineto{\pgfqpoint{2.927219in}{0.859121in}}%
\pgfpathlineto{\pgfqpoint{2.931514in}{0.858178in}}%
\pgfpathlineto{\pgfqpoint{2.935809in}{0.857237in}}%
\pgfpathlineto{\pgfqpoint{2.940104in}{0.856300in}}%
\pgfpathlineto{\pgfqpoint{2.944399in}{0.855365in}}%
\pgfpathlineto{\pgfqpoint{2.948694in}{0.854435in}}%
\pgfpathlineto{\pgfqpoint{2.952990in}{0.853507in}}%
\pgfpathlineto{\pgfqpoint{2.957285in}{0.852582in}}%
\pgfpathlineto{\pgfqpoint{2.961580in}{0.851661in}}%
\pgfpathlineto{\pgfqpoint{2.965875in}{0.850743in}}%
\pgfpathlineto{\pgfqpoint{2.970170in}{0.849828in}}%
\pgfpathlineto{\pgfqpoint{2.974465in}{0.848916in}}%
\pgfpathlineto{\pgfqpoint{2.978760in}{0.848007in}}%
\pgfpathlineto{\pgfqpoint{2.983055in}{0.847101in}}%
\pgfpathlineto{\pgfqpoint{2.987350in}{0.846198in}}%
\pgfpathlineto{\pgfqpoint{2.991645in}{0.845299in}}%
\pgfpathlineto{\pgfqpoint{2.995940in}{0.844402in}}%
\pgfpathlineto{\pgfqpoint{3.000235in}{0.843508in}}%
\pgfpathlineto{\pgfqpoint{3.004530in}{0.842618in}}%
\pgfpathlineto{\pgfqpoint{3.008825in}{0.841730in}}%
\pgfpathlineto{\pgfqpoint{3.013120in}{0.840846in}}%
\pgfpathlineto{\pgfqpoint{3.017415in}{0.839964in}}%
\pgfpathlineto{\pgfqpoint{3.021710in}{0.839085in}}%
\pgfpathlineto{\pgfqpoint{3.026005in}{0.838210in}}%
\pgfpathlineto{\pgfqpoint{3.030300in}{0.837337in}}%
\pgfpathlineto{\pgfqpoint{3.034595in}{0.836467in}}%
\pgfpathlineto{\pgfqpoint{3.038890in}{0.835600in}}%
\pgfpathlineto{\pgfqpoint{3.043185in}{0.834735in}}%
\pgfpathlineto{\pgfqpoint{3.047480in}{0.833874in}}%
\pgfpathlineto{\pgfqpoint{3.051775in}{0.833016in}}%
\pgfpathlineto{\pgfqpoint{3.056070in}{0.832160in}}%
\pgfpathlineto{\pgfqpoint{3.060365in}{0.831307in}}%
\pgfpathlineto{\pgfqpoint{3.064660in}{0.830457in}}%
\pgfpathlineto{\pgfqpoint{3.068955in}{0.829610in}}%
\pgfpathlineto{\pgfqpoint{3.073250in}{0.828765in}}%
\pgfpathlineto{\pgfqpoint{3.077546in}{0.827924in}}%
\pgfpathlineto{\pgfqpoint{3.081841in}{0.827085in}}%
\pgfpathlineto{\pgfqpoint{3.086136in}{0.826249in}}%
\pgfpathlineto{\pgfqpoint{3.090431in}{0.825415in}}%
\pgfpathlineto{\pgfqpoint{3.094726in}{0.824584in}}%
\pgfpathlineto{\pgfqpoint{3.099021in}{0.823756in}}%
\pgfpathlineto{\pgfqpoint{3.103316in}{0.822931in}}%
\pgfpathlineto{\pgfqpoint{3.107611in}{0.822108in}}%
\pgfpathlineto{\pgfqpoint{3.111906in}{0.821288in}}%
\pgfpathlineto{\pgfqpoint{3.116201in}{0.820471in}}%
\pgfpathlineto{\pgfqpoint{3.120496in}{0.819656in}}%
\pgfpathlineto{\pgfqpoint{3.124791in}{0.818844in}}%
\pgfpathlineto{\pgfqpoint{3.129086in}{0.818034in}}%
\pgfpathlineto{\pgfqpoint{3.133381in}{0.817227in}}%
\pgfpathlineto{\pgfqpoint{3.137676in}{0.816423in}}%
\pgfpathlineto{\pgfqpoint{3.141971in}{0.815621in}}%
\pgfpathlineto{\pgfqpoint{3.146266in}{0.814822in}}%
\pgfpathlineto{\pgfqpoint{3.150561in}{0.814025in}}%
\pgfpathlineto{\pgfqpoint{3.154856in}{0.813231in}}%
\pgfpathlineto{\pgfqpoint{3.159151in}{0.812439in}}%
\pgfpathlineto{\pgfqpoint{3.163446in}{0.811650in}}%
\pgfpathlineto{\pgfqpoint{3.167741in}{0.810863in}}%
\pgfpathlineto{\pgfqpoint{3.172036in}{0.810079in}}%
\pgfpathlineto{\pgfqpoint{3.176331in}{0.809298in}}%
\pgfpathlineto{\pgfqpoint{3.180626in}{0.808518in}}%
\pgfpathlineto{\pgfqpoint{3.184921in}{0.807742in}}%
\pgfpathlineto{\pgfqpoint{3.189216in}{0.806967in}}%
\pgfpathlineto{\pgfqpoint{3.193511in}{0.806195in}}%
\pgfpathlineto{\pgfqpoint{3.197806in}{0.805426in}}%
\pgfpathlineto{\pgfqpoint{3.202102in}{0.804659in}}%
\pgfpathlineto{\pgfqpoint{3.206397in}{0.803894in}}%
\pgfpathlineto{\pgfqpoint{3.210692in}{0.803132in}}%
\pgfpathlineto{\pgfqpoint{3.214987in}{0.802372in}}%
\pgfpathlineto{\pgfqpoint{3.219282in}{0.801614in}}%
\pgfpathlineto{\pgfqpoint{3.223577in}{0.800859in}}%
\pgfpathlineto{\pgfqpoint{3.227872in}{0.800106in}}%
\pgfpathlineto{\pgfqpoint{3.232167in}{0.799355in}}%
\pgfpathlineto{\pgfqpoint{3.236462in}{0.798607in}}%
\pgfpathlineto{\pgfqpoint{3.240757in}{0.797861in}}%
\pgfpathlineto{\pgfqpoint{3.245052in}{0.797118in}}%
\pgfpathlineto{\pgfqpoint{3.249347in}{0.796376in}}%
\pgfpathlineto{\pgfqpoint{3.253642in}{0.795637in}}%
\pgfpathlineto{\pgfqpoint{3.257937in}{0.794900in}}%
\pgfpathlineto{\pgfqpoint{3.262232in}{0.794166in}}%
\pgfpathlineto{\pgfqpoint{3.266527in}{0.793433in}}%
\pgfpathlineto{\pgfqpoint{3.270822in}{0.792703in}}%
\pgfpathlineto{\pgfqpoint{3.275117in}{0.791975in}}%
\pgfpathlineto{\pgfqpoint{3.279412in}{0.791250in}}%
\pgfpathlineto{\pgfqpoint{3.283707in}{0.790526in}}%
\pgfpathlineto{\pgfqpoint{3.288002in}{0.789805in}}%
\pgfpathlineto{\pgfqpoint{3.292297in}{0.789086in}}%
\pgfpathlineto{\pgfqpoint{3.296592in}{0.788369in}}%
\pgfpathlineto{\pgfqpoint{3.300887in}{0.787654in}}%
\pgfpathlineto{\pgfqpoint{3.305182in}{0.786941in}}%
\pgfpathlineto{\pgfqpoint{3.309477in}{0.786231in}}%
\pgfpathlineto{\pgfqpoint{3.313772in}{0.785523in}}%
\pgfpathlineto{\pgfqpoint{3.318067in}{0.784816in}}%
\pgfpathlineto{\pgfqpoint{3.322362in}{0.784112in}}%
\pgfpathlineto{\pgfqpoint{3.326658in}{0.783410in}}%
\pgfpathlineto{\pgfqpoint{3.330953in}{0.782710in}}%
\pgfpathlineto{\pgfqpoint{3.335248in}{0.782013in}}%
\pgfpathlineto{\pgfqpoint{3.339543in}{0.781317in}}%
\pgfpathlineto{\pgfqpoint{3.343838in}{0.780623in}}%
\pgfpathlineto{\pgfqpoint{3.348133in}{0.779932in}}%
\pgfpathlineto{\pgfqpoint{3.352428in}{0.779242in}}%
\pgfpathlineto{\pgfqpoint{3.356723in}{0.778554in}}%
\pgfpathlineto{\pgfqpoint{3.361018in}{0.777869in}}%
\pgfpathlineto{\pgfqpoint{3.365313in}{0.777185in}}%
\pgfpathlineto{\pgfqpoint{3.369608in}{0.776504in}}%
\pgfpathlineto{\pgfqpoint{3.373903in}{0.775825in}}%
\pgfpathlineto{\pgfqpoint{3.378198in}{0.775147in}}%
\pgfpathlineto{\pgfqpoint{3.382493in}{0.774472in}}%
\pgfpathlineto{\pgfqpoint{3.386788in}{0.773798in}}%
\pgfpathlineto{\pgfqpoint{3.391083in}{0.773127in}}%
\pgfpathlineto{\pgfqpoint{3.395378in}{0.772457in}}%
\pgfpathlineto{\pgfqpoint{3.399673in}{0.771789in}}%
\pgfpathlineto{\pgfqpoint{3.403968in}{0.771124in}}%
\pgfpathlineto{\pgfqpoint{3.408263in}{0.770460in}}%
\pgfpathlineto{\pgfqpoint{3.412558in}{0.769798in}}%
\pgfpathlineto{\pgfqpoint{3.416853in}{0.769138in}}%
\pgfpathlineto{\pgfqpoint{3.421148in}{0.768480in}}%
\pgfpathlineto{\pgfqpoint{3.425443in}{0.767824in}}%
\pgfpathlineto{\pgfqpoint{3.429738in}{0.767170in}}%
\pgfpathlineto{\pgfqpoint{3.434033in}{0.766518in}}%
\pgfpathlineto{\pgfqpoint{3.438328in}{0.765867in}}%
\pgfpathlineto{\pgfqpoint{3.442623in}{0.765219in}}%
\pgfpathlineto{\pgfqpoint{3.446919in}{0.764572in}}%
\pgfpathlineto{\pgfqpoint{3.451214in}{0.763927in}}%
\pgfpathlineto{\pgfqpoint{3.455509in}{0.763284in}}%
\pgfpathlineto{\pgfqpoint{3.459804in}{0.762643in}}%
\pgfpathlineto{\pgfqpoint{3.464099in}{0.762004in}}%
\pgfpathlineto{\pgfqpoint{3.468394in}{0.761366in}}%
\pgfpathlineto{\pgfqpoint{3.472689in}{0.760730in}}%
\pgfpathlineto{\pgfqpoint{3.476984in}{0.760096in}}%
\pgfpathlineto{\pgfqpoint{3.481279in}{0.759464in}}%
\pgfpathlineto{\pgfqpoint{3.485574in}{0.758834in}}%
\pgfpathlineto{\pgfqpoint{3.489869in}{0.758206in}}%
\pgfpathlineto{\pgfqpoint{3.494164in}{0.757579in}}%
\pgfpathlineto{\pgfqpoint{3.498459in}{0.756954in}}%
\pgfpathlineto{\pgfqpoint{3.502754in}{0.756331in}}%
\pgfpathlineto{\pgfqpoint{3.507049in}{0.755709in}}%
\pgfpathlineto{\pgfqpoint{3.511344in}{0.755089in}}%
\pgfpathlineto{\pgfqpoint{3.515639in}{0.754471in}}%
\pgfpathlineto{\pgfqpoint{3.519934in}{0.753855in}}%
\pgfpathlineto{\pgfqpoint{3.524229in}{0.753241in}}%
\pgfpathlineto{\pgfqpoint{3.528524in}{0.752628in}}%
\pgfpathlineto{\pgfqpoint{3.532819in}{0.752017in}}%
\pgfpathlineto{\pgfqpoint{3.537114in}{0.751407in}}%
\pgfpathlineto{\pgfqpoint{3.541409in}{0.750800in}}%
\pgfpathlineto{\pgfqpoint{3.545704in}{0.750194in}}%
\pgfpathlineto{\pgfqpoint{3.549999in}{0.749589in}}%
\pgfpathlineto{\pgfqpoint{3.554294in}{0.748987in}}%
\pgfpathlineto{\pgfqpoint{3.558589in}{0.748386in}}%
\pgfpathlineto{\pgfqpoint{3.562884in}{0.747786in}}%
\pgfpathlineto{\pgfqpoint{3.567179in}{0.747189in}}%
\pgfpathlineto{\pgfqpoint{3.571475in}{0.746592in}}%
\pgfpathlineto{\pgfqpoint{3.575770in}{0.745998in}}%
\pgfpathlineto{\pgfqpoint{3.580065in}{0.745405in}}%
\pgfpathlineto{\pgfqpoint{3.584360in}{0.744814in}}%
\pgfpathlineto{\pgfqpoint{3.588655in}{0.744225in}}%
\pgfpathlineto{\pgfqpoint{3.592950in}{0.743637in}}%
\pgfpathlineto{\pgfqpoint{3.597245in}{0.743050in}}%
\pgfpathlineto{\pgfqpoint{3.601540in}{0.742466in}}%
\pgfpathlineto{\pgfqpoint{3.605835in}{0.741883in}}%
\pgfpathlineto{\pgfqpoint{3.610130in}{0.741301in}}%
\pgfpathlineto{\pgfqpoint{3.614425in}{0.740721in}}%
\pgfpathlineto{\pgfqpoint{3.618720in}{0.740143in}}%
\pgfpathlineto{\pgfqpoint{3.623015in}{0.739566in}}%
\pgfpathlineto{\pgfqpoint{3.627310in}{0.738991in}}%
\pgfpathlineto{\pgfqpoint{3.631605in}{0.738417in}}%
\pgfpathlineto{\pgfqpoint{3.635900in}{0.737845in}}%
\pgfpathlineto{\pgfqpoint{3.640195in}{0.737274in}}%
\pgfpathlineto{\pgfqpoint{3.644490in}{0.736705in}}%
\pgfpathlineto{\pgfqpoint{3.648785in}{0.736138in}}%
\pgfpathlineto{\pgfqpoint{3.653080in}{0.735572in}}%
\pgfpathlineto{\pgfqpoint{3.657375in}{0.735007in}}%
\pgfpathlineto{\pgfqpoint{3.661670in}{0.734444in}}%
\pgfpathlineto{\pgfqpoint{3.665965in}{0.733883in}}%
\pgfpathlineto{\pgfqpoint{3.670260in}{0.733323in}}%
\pgfpathlineto{\pgfqpoint{3.674555in}{0.732764in}}%
\pgfpathlineto{\pgfqpoint{3.678850in}{0.732207in}}%
\pgfpathlineto{\pgfqpoint{3.683145in}{0.731652in}}%
\pgfpathlineto{\pgfqpoint{3.687440in}{0.731098in}}%
\pgfpathlineto{\pgfqpoint{3.691735in}{0.730545in}}%
\pgfpathlineto{\pgfqpoint{3.696031in}{0.729994in}}%
\pgfpathlineto{\pgfqpoint{3.700326in}{0.729444in}}%
\pgfpathlineto{\pgfqpoint{3.704621in}{0.728896in}}%
\pgfpathlineto{\pgfqpoint{3.708916in}{0.728349in}}%
\pgfpathlineto{\pgfqpoint{3.713211in}{0.727804in}}%
\pgfpathlineto{\pgfqpoint{3.717506in}{0.727260in}}%
\pgfpathlineto{\pgfqpoint{3.721801in}{0.726718in}}%
\pgfpathlineto{\pgfqpoint{3.726096in}{0.726177in}}%
\pgfpathlineto{\pgfqpoint{3.730391in}{0.725637in}}%
\pgfpathlineto{\pgfqpoint{3.734686in}{0.725099in}}%
\pgfpathlineto{\pgfqpoint{3.738981in}{0.724562in}}%
\pgfpathlineto{\pgfqpoint{3.743276in}{0.724027in}}%
\pgfpathlineto{\pgfqpoint{3.747571in}{0.723493in}}%
\pgfpathlineto{\pgfqpoint{3.751866in}{0.722960in}}%
\pgfpathlineto{\pgfqpoint{3.756161in}{0.722429in}}%
\pgfpathlineto{\pgfqpoint{3.760456in}{0.721899in}}%
\pgfpathlineto{\pgfqpoint{3.764751in}{0.721371in}}%
\pgfpathlineto{\pgfqpoint{3.769046in}{0.720843in}}%
\pgfpathlineto{\pgfqpoint{3.773341in}{0.720318in}}%
\pgfpathlineto{\pgfqpoint{3.777636in}{0.719793in}}%
\pgfpathlineto{\pgfqpoint{3.781931in}{0.719270in}}%
\pgfpathlineto{\pgfqpoint{3.786226in}{0.718749in}}%
\pgfpathlineto{\pgfqpoint{3.790521in}{0.718228in}}%
\pgfpathlineto{\pgfqpoint{3.794816in}{0.717709in}}%
\pgfpathlineto{\pgfqpoint{3.799111in}{0.717192in}}%
\pgfpathlineto{\pgfqpoint{3.803406in}{0.716675in}}%
\pgfpathlineto{\pgfqpoint{3.807701in}{0.716160in}}%
\pgfpathlineto{\pgfqpoint{3.811996in}{0.715647in}}%
\pgfpathlineto{\pgfqpoint{3.816291in}{0.715134in}}%
\pgfpathlineto{\pgfqpoint{3.820587in}{0.714623in}}%
\pgfpathlineto{\pgfqpoint{3.824882in}{0.714113in}}%
\pgfpathlineto{\pgfqpoint{3.829177in}{0.713605in}}%
\pgfpathlineto{\pgfqpoint{3.833472in}{0.713098in}}%
\pgfpathlineto{\pgfqpoint{3.837767in}{0.712592in}}%
\pgfpathlineto{\pgfqpoint{3.842062in}{0.712087in}}%
\pgfpathlineto{\pgfqpoint{3.846357in}{0.711584in}}%
\pgfpathlineto{\pgfqpoint{3.850652in}{0.711082in}}%
\pgfpathlineto{\pgfqpoint{3.854947in}{0.710581in}}%
\pgfpathlineto{\pgfqpoint{3.859242in}{0.710081in}}%
\pgfpathlineto{\pgfqpoint{3.863537in}{0.709583in}}%
\pgfpathlineto{\pgfqpoint{3.867832in}{0.709086in}}%
\pgfpathlineto{\pgfqpoint{3.872127in}{0.708590in}}%
\pgfpathlineto{\pgfqpoint{3.876422in}{0.708096in}}%
\pgfpathlineto{\pgfqpoint{3.880717in}{0.707603in}}%
\pgfpathlineto{\pgfqpoint{3.885012in}{0.707110in}}%
\pgfpathlineto{\pgfqpoint{3.889307in}{0.706620in}}%
\pgfpathlineto{\pgfqpoint{3.893602in}{0.706130in}}%
\pgfpathlineto{\pgfqpoint{3.897897in}{0.705642in}}%
\pgfpathlineto{\pgfqpoint{3.902192in}{0.705154in}}%
\pgfpathlineto{\pgfqpoint{3.906487in}{0.704669in}}%
\pgfpathlineto{\pgfqpoint{3.910782in}{0.704184in}}%
\pgfpathlineto{\pgfqpoint{3.915077in}{0.703700in}}%
\pgfpathlineto{\pgfqpoint{3.919372in}{0.703218in}}%
\pgfpathlineto{\pgfqpoint{3.923667in}{0.702737in}}%
\pgfpathlineto{\pgfqpoint{3.927962in}{0.702257in}}%
\pgfpathlineto{\pgfqpoint{3.932257in}{0.701778in}}%
\pgfpathlineto{\pgfqpoint{3.936552in}{0.701300in}}%
\pgfpathlineto{\pgfqpoint{3.940847in}{0.700824in}}%
\pgfpathlineto{\pgfqpoint{3.945143in}{0.700349in}}%
\pgfpathlineto{\pgfqpoint{3.949438in}{0.699875in}}%
\pgfpathlineto{\pgfqpoint{3.953733in}{0.699402in}}%
\pgfpathlineto{\pgfqpoint{3.958028in}{0.698930in}}%
\pgfpathlineto{\pgfqpoint{3.962323in}{0.698459in}}%
\pgfpathlineto{\pgfqpoint{3.966618in}{0.697990in}}%
\pgfpathlineto{\pgfqpoint{3.970913in}{0.697522in}}%
\pgfpathlineto{\pgfqpoint{3.975208in}{0.697054in}}%
\pgfpathlineto{\pgfqpoint{3.979503in}{0.696588in}}%
\pgfpathlineto{\pgfqpoint{3.983798in}{0.696123in}}%
\pgfpathlineto{\pgfqpoint{3.988093in}{0.695660in}}%
\pgfpathlineto{\pgfqpoint{3.992388in}{0.695197in}}%
\pgfpathlineto{\pgfqpoint{3.996683in}{0.694736in}}%
\pgfpathlineto{\pgfqpoint{4.000978in}{0.694275in}}%
\pgfpathlineto{\pgfqpoint{4.005273in}{0.693816in}}%
\pgfpathlineto{\pgfqpoint{4.009568in}{0.693358in}}%
\pgfpathlineto{\pgfqpoint{4.013863in}{0.692901in}}%
\pgfpathlineto{\pgfqpoint{4.018158in}{0.692445in}}%
\pgfpathlineto{\pgfqpoint{4.022453in}{0.691990in}}%
\pgfpathlineto{\pgfqpoint{4.026748in}{0.691536in}}%
\pgfpathlineto{\pgfqpoint{4.031043in}{0.691083in}}%
\pgfpathlineto{\pgfqpoint{4.035338in}{0.690632in}}%
\pgfpathlineto{\pgfqpoint{4.039633in}{0.690181in}}%
\pgfpathlineto{\pgfqpoint{4.043928in}{0.689732in}}%
\pgfpathlineto{\pgfqpoint{4.048223in}{0.689283in}}%
\pgfpathlineto{\pgfqpoint{4.052518in}{0.688836in}}%
\pgfpathlineto{\pgfqpoint{4.056813in}{0.688390in}}%
\pgfpathlineto{\pgfqpoint{4.061108in}{0.687945in}}%
\pgfpathlineto{\pgfqpoint{4.065403in}{0.687500in}}%
\pgfpathlineto{\pgfqpoint{4.069699in}{0.687057in}}%
\pgfpathlineto{\pgfqpoint{4.073994in}{0.686615in}}%
\pgfpathlineto{\pgfqpoint{4.078289in}{0.686174in}}%
\pgfpathlineto{\pgfqpoint{4.082584in}{0.685735in}}%
\pgfpathlineto{\pgfqpoint{4.086879in}{0.685296in}}%
\pgfpathlineto{\pgfqpoint{4.091174in}{0.684858in}}%
\pgfpathlineto{\pgfqpoint{4.095469in}{0.684421in}}%
\pgfpathlineto{\pgfqpoint{4.099764in}{0.683985in}}%
\pgfpathlineto{\pgfqpoint{4.104059in}{0.683550in}}%
\pgfpathlineto{\pgfqpoint{4.108354in}{0.683117in}}%
\pgfpathlineto{\pgfqpoint{4.112649in}{0.682684in}}%
\pgfpathlineto{\pgfqpoint{4.116944in}{0.682252in}}%
\pgfpathlineto{\pgfqpoint{4.121239in}{0.681822in}}%
\pgfpathlineto{\pgfqpoint{4.125534in}{0.681392in}}%
\pgfpathlineto{\pgfqpoint{4.129829in}{0.680963in}}%
\pgfpathlineto{\pgfqpoint{4.134124in}{0.680536in}}%
\pgfpathlineto{\pgfqpoint{4.138419in}{0.680109in}}%
\pgfpathlineto{\pgfqpoint{4.142714in}{0.679683in}}%
\pgfpathlineto{\pgfqpoint{4.147009in}{0.679258in}}%
\pgfpathlineto{\pgfqpoint{4.151304in}{0.678835in}}%
\pgfpathlineto{\pgfqpoint{4.155599in}{0.678412in}}%
\pgfpathlineto{\pgfqpoint{4.159894in}{0.677990in}}%
\pgfpathlineto{\pgfqpoint{4.164189in}{0.677569in}}%
\pgfpathlineto{\pgfqpoint{4.168484in}{0.677150in}}%
\pgfpathlineto{\pgfqpoint{4.172779in}{0.676731in}}%
\pgfpathlineto{\pgfqpoint{4.177074in}{0.676313in}}%
\pgfpathlineto{\pgfqpoint{4.181369in}{0.675896in}}%
\pgfpathlineto{\pgfqpoint{4.185664in}{0.675480in}}%
\pgfpathlineto{\pgfqpoint{4.189959in}{0.675065in}}%
\pgfpathlineto{\pgfqpoint{4.194255in}{0.674651in}}%
\pgfpathlineto{\pgfqpoint{4.198550in}{0.674238in}}%
\pgfpathlineto{\pgfqpoint{4.202845in}{0.673826in}}%
\pgfpathlineto{\pgfqpoint{4.207140in}{0.673415in}}%
\pgfpathlineto{\pgfqpoint{4.211435in}{0.673004in}}%
\pgfpathlineto{\pgfqpoint{4.215730in}{0.672595in}}%
\pgfpathlineto{\pgfqpoint{4.220025in}{0.672187in}}%
\pgfpathlineto{\pgfqpoint{4.224320in}{0.671779in}}%
\pgfpathlineto{\pgfqpoint{4.228615in}{0.671373in}}%
\pgfpathlineto{\pgfqpoint{4.232910in}{0.670967in}}%
\pgfpathlineto{\pgfqpoint{4.237205in}{0.670562in}}%
\pgfpathlineto{\pgfqpoint{4.241500in}{0.670159in}}%
\pgfpathlineto{\pgfqpoint{4.245795in}{0.669756in}}%
\pgfpathlineto{\pgfqpoint{4.250090in}{0.669354in}}%
\pgfpathlineto{\pgfqpoint{4.254385in}{0.668953in}}%
\pgfpathlineto{\pgfqpoint{4.258680in}{0.668553in}}%
\pgfpathlineto{\pgfqpoint{4.262975in}{0.668153in}}%
\pgfpathlineto{\pgfqpoint{4.267270in}{0.667755in}}%
\pgfpathlineto{\pgfqpoint{4.271565in}{0.667358in}}%
\pgfpathlineto{\pgfqpoint{4.275860in}{0.666961in}}%
\pgfpathlineto{\pgfqpoint{4.280155in}{0.666565in}}%
\pgfpathlineto{\pgfqpoint{4.284450in}{0.666171in}}%
\pgfpathlineto{\pgfqpoint{4.288745in}{0.665777in}}%
\pgfpathlineto{\pgfqpoint{4.293040in}{0.665384in}}%
\pgfpathlineto{\pgfqpoint{4.297335in}{0.664992in}}%
\pgfpathlineto{\pgfqpoint{4.301630in}{0.664600in}}%
\pgfpathlineto{\pgfqpoint{4.305925in}{0.664210in}}%
\pgfpathlineto{\pgfqpoint{4.310220in}{0.663821in}}%
\pgfpathlineto{\pgfqpoint{4.314515in}{0.663432in}}%
\pgfpathlineto{\pgfqpoint{4.318811in}{0.663044in}}%
\pgfpathlineto{\pgfqpoint{4.323106in}{0.662657in}}%
\pgfpathlineto{\pgfqpoint{4.327401in}{0.662271in}}%
\pgfpathlineto{\pgfqpoint{4.331696in}{0.661886in}}%
\pgfpathlineto{\pgfqpoint{4.335991in}{0.661502in}}%
\pgfpathlineto{\pgfqpoint{4.340286in}{0.661118in}}%
\pgfpathlineto{\pgfqpoint{4.344581in}{0.660735in}}%
\pgfpathlineto{\pgfqpoint{4.348876in}{0.660354in}}%
\pgfpathlineto{\pgfqpoint{4.353171in}{0.659973in}}%
\pgfpathlineto{\pgfqpoint{4.357466in}{0.659593in}}%
\pgfpathlineto{\pgfqpoint{4.361761in}{0.659213in}}%
\pgfpathlineto{\pgfqpoint{4.366056in}{0.658835in}}%
\pgfpathlineto{\pgfqpoint{4.370351in}{0.658457in}}%
\pgfpathlineto{\pgfqpoint{4.374646in}{0.658080in}}%
\pgfpathlineto{\pgfqpoint{4.378941in}{0.657704in}}%
\pgfpathlineto{\pgfqpoint{4.383236in}{0.657329in}}%
\pgfpathlineto{\pgfqpoint{4.387531in}{0.656955in}}%
\pgfpathlineto{\pgfqpoint{4.391826in}{0.656581in}}%
\pgfpathlineto{\pgfqpoint{4.396121in}{0.656209in}}%
\pgfpathlineto{\pgfqpoint{4.400416in}{0.655837in}}%
\pgfpathlineto{\pgfqpoint{4.404711in}{0.655466in}}%
\pgfpathlineto{\pgfqpoint{4.409006in}{0.655095in}}%
\pgfpathlineto{\pgfqpoint{4.413301in}{0.654726in}}%
\pgfpathlineto{\pgfqpoint{4.417596in}{0.654357in}}%
\pgfpathlineto{\pgfqpoint{4.421891in}{0.653989in}}%
\pgfpathlineto{\pgfqpoint{4.426186in}{0.653622in}}%
\pgfpathlineto{\pgfqpoint{4.430481in}{0.653256in}}%
\pgfpathlineto{\pgfqpoint{4.434776in}{0.652890in}}%
\pgfpathlineto{\pgfqpoint{4.439071in}{0.652526in}}%
\pgfpathlineto{\pgfqpoint{4.443367in}{0.652162in}}%
\pgfpathlineto{\pgfqpoint{4.447662in}{0.651799in}}%
\pgfpathlineto{\pgfqpoint{4.451957in}{0.651436in}}%
\pgfpathlineto{\pgfqpoint{4.456252in}{0.651075in}}%
\pgfpathlineto{\pgfqpoint{4.460547in}{0.650714in}}%
\pgfpathlineto{\pgfqpoint{4.464842in}{0.650354in}}%
\pgfpathlineto{\pgfqpoint{4.469137in}{0.649994in}}%
\pgfpathlineto{\pgfqpoint{4.473432in}{0.649636in}}%
\pgfpathlineto{\pgfqpoint{4.477727in}{0.649278in}}%
\pgfpathlineto{\pgfqpoint{4.482022in}{0.648921in}}%
\pgfpathlineto{\pgfqpoint{4.486317in}{0.648565in}}%
\pgfpathlineto{\pgfqpoint{4.490612in}{0.648210in}}%
\pgfpathlineto{\pgfqpoint{4.494907in}{0.647855in}}%
\pgfpathlineto{\pgfqpoint{4.499202in}{0.647501in}}%
\pgfpathlineto{\pgfqpoint{4.503497in}{0.647148in}}%
\pgfpathlineto{\pgfqpoint{4.507792in}{0.646795in}}%
\pgfpathlineto{\pgfqpoint{4.512087in}{0.646444in}}%
\pgfpathlineto{\pgfqpoint{4.516382in}{0.646093in}}%
\pgfpathlineto{\pgfqpoint{4.520677in}{0.645742in}}%
\pgfpathlineto{\pgfqpoint{4.524972in}{0.645393in}}%
\pgfpathlineto{\pgfqpoint{4.529267in}{0.645044in}}%
\pgfpathlineto{\pgfqpoint{4.533562in}{0.644696in}}%
\pgfpathlineto{\pgfqpoint{4.537857in}{0.644349in}}%
\pgfpathlineto{\pgfqpoint{4.542152in}{0.644002in}}%
\pgfpathlineto{\pgfqpoint{4.546447in}{0.643656in}}%
\pgfpathlineto{\pgfqpoint{4.550742in}{0.643311in}}%
\pgfpathlineto{\pgfqpoint{4.555037in}{0.642967in}}%
\pgfpathlineto{\pgfqpoint{4.559332in}{0.642623in}}%
\pgfpathlineto{\pgfqpoint{4.563627in}{0.642280in}}%
\pgfpathlineto{\pgfqpoint{4.567923in}{0.641938in}}%
\pgfpathlineto{\pgfqpoint{4.572218in}{0.641597in}}%
\pgfpathlineto{\pgfqpoint{4.576513in}{0.641256in}}%
\pgfpathlineto{\pgfqpoint{4.580808in}{0.640916in}}%
\pgfpathlineto{\pgfqpoint{4.585103in}{0.640576in}}%
\pgfpathlineto{\pgfqpoint{4.589398in}{0.640238in}}%
\pgfpathlineto{\pgfqpoint{4.593693in}{0.639900in}}%
\pgfpathlineto{\pgfqpoint{4.597988in}{0.639562in}}%
\pgfpathlineto{\pgfqpoint{4.602283in}{0.639226in}}%
\pgfpathlineto{\pgfqpoint{4.606578in}{0.638890in}}%
\pgfpathlineto{\pgfqpoint{4.610873in}{0.638555in}}%
\pgfpathlineto{\pgfqpoint{4.615168in}{0.638220in}}%
\pgfpathlineto{\pgfqpoint{4.619463in}{0.637886in}}%
\pgfpathlineto{\pgfqpoint{4.623758in}{0.637553in}}%
\pgfpathlineto{\pgfqpoint{4.628053in}{0.637221in}}%
\pgfpathlineto{\pgfqpoint{4.632348in}{0.636889in}}%
\pgfpathlineto{\pgfqpoint{4.636643in}{0.636558in}}%
\pgfpathlineto{\pgfqpoint{4.640938in}{0.636228in}}%
\pgfpathlineto{\pgfqpoint{4.645233in}{0.635898in}}%
\pgfpathlineto{\pgfqpoint{4.649528in}{0.635569in}}%
\pgfpathlineto{\pgfqpoint{4.653823in}{0.635241in}}%
\pgfpathlineto{\pgfqpoint{4.658118in}{0.634913in}}%
\pgfpathlineto{\pgfqpoint{4.662413in}{0.634586in}}%
\pgfpathlineto{\pgfqpoint{4.666708in}{0.634259in}}%
\pgfpathlineto{\pgfqpoint{4.671003in}{0.633934in}}%
\pgfpathlineto{\pgfqpoint{4.675298in}{0.633609in}}%
\pgfpathlineto{\pgfqpoint{4.679593in}{0.633284in}}%
\pgfpathlineto{\pgfqpoint{4.683888in}{0.632961in}}%
\pgfpathlineto{\pgfqpoint{4.688183in}{0.632638in}}%
\pgfpathlineto{\pgfqpoint{4.692479in}{0.632315in}}%
\pgfpathlineto{\pgfqpoint{4.696774in}{0.631993in}}%
\pgfpathlineto{\pgfqpoint{4.701069in}{0.631672in}}%
\pgfpathlineto{\pgfqpoint{4.705364in}{0.631352in}}%
\pgfpathlineto{\pgfqpoint{4.709659in}{0.631032in}}%
\pgfpathlineto{\pgfqpoint{4.713954in}{0.630713in}}%
\pgfpathlineto{\pgfqpoint{4.718249in}{0.630394in}}%
\pgfpathlineto{\pgfqpoint{4.722544in}{0.630077in}}%
\pgfpathlineto{\pgfqpoint{4.726839in}{0.629759in}}%
\pgfpathlineto{\pgfqpoint{4.731134in}{0.629443in}}%
\pgfpathlineto{\pgfqpoint{4.735429in}{0.629127in}}%
\pgfpathlineto{\pgfqpoint{4.739724in}{0.628811in}}%
\pgfpathlineto{\pgfqpoint{4.744019in}{0.628497in}}%
\pgfpathlineto{\pgfqpoint{4.748314in}{0.628183in}}%
\pgfpathlineto{\pgfqpoint{4.752609in}{0.627869in}}%
\pgfpathlineto{\pgfqpoint{4.756904in}{0.627556in}}%
\pgfpathlineto{\pgfqpoint{4.761199in}{0.627244in}}%
\pgfpathlineto{\pgfqpoint{4.765494in}{0.626933in}}%
\pgfpathlineto{\pgfqpoint{4.769789in}{0.626622in}}%
\pgfpathlineto{\pgfqpoint{4.774084in}{0.626311in}}%
\pgfpathlineto{\pgfqpoint{4.778379in}{0.626002in}}%
\pgfpathlineto{\pgfqpoint{4.782674in}{0.625692in}}%
\pgfpathlineto{\pgfqpoint{4.786969in}{0.625384in}}%
\pgfpathlineto{\pgfqpoint{4.791264in}{0.625076in}}%
\pgfpathlineto{\pgfqpoint{4.795559in}{0.624769in}}%
\pgfpathlineto{\pgfqpoint{4.799854in}{0.624462in}}%
\pgfpathlineto{\pgfqpoint{4.804149in}{0.624156in}}%
\pgfpathlineto{\pgfqpoint{4.808444in}{0.623851in}}%
\pgfpathlineto{\pgfqpoint{4.812739in}{0.623546in}}%
\pgfpathlineto{\pgfqpoint{4.817035in}{0.623242in}}%
\pgfpathlineto{\pgfqpoint{4.821330in}{0.622938in}}%
\pgfpathlineto{\pgfqpoint{4.825625in}{0.622635in}}%
\pgfpathlineto{\pgfqpoint{4.829920in}{0.622332in}}%
\pgfpathlineto{\pgfqpoint{4.834215in}{0.622030in}}%
\pgfpathlineto{\pgfqpoint{4.838510in}{0.621729in}}%
\pgfpathlineto{\pgfqpoint{4.842805in}{0.621428in}}%
\pgfpathlineto{\pgfqpoint{4.847100in}{0.621128in}}%
\pgfpathlineto{\pgfqpoint{4.851395in}{0.620829in}}%
\pgfpathlineto{\pgfqpoint{4.855690in}{0.620530in}}%
\pgfpathlineto{\pgfqpoint{4.859985in}{0.620232in}}%
\pgfpathlineto{\pgfqpoint{4.864280in}{0.619934in}}%
\pgfpathlineto{\pgfqpoint{4.868575in}{0.619637in}}%
\pgfpathlineto{\pgfqpoint{4.872870in}{0.619340in}}%
\pgfpathlineto{\pgfqpoint{4.877165in}{0.619044in}}%
\pgfpathlineto{\pgfqpoint{4.881460in}{0.618749in}}%
\pgfpathlineto{\pgfqpoint{4.885755in}{0.618454in}}%
\pgfpathlineto{\pgfqpoint{4.890050in}{0.618159in}}%
\pgfpathlineto{\pgfqpoint{4.894345in}{0.617866in}}%
\pgfpathlineto{\pgfqpoint{4.898640in}{0.617572in}}%
\pgfpathlineto{\pgfqpoint{4.902935in}{0.617280in}}%
\pgfpathlineto{\pgfqpoint{4.907230in}{0.616988in}}%
\pgfpathlineto{\pgfqpoint{4.911525in}{0.616696in}}%
\pgfpathlineto{\pgfqpoint{4.915820in}{0.616405in}}%
\pgfpathlineto{\pgfqpoint{4.920115in}{0.616115in}}%
\pgfpathlineto{\pgfqpoint{4.924410in}{0.615825in}}%
\pgfpathlineto{\pgfqpoint{4.928705in}{0.615536in}}%
\pgfpathlineto{\pgfqpoint{4.933000in}{0.615247in}}%
\pgfpathlineto{\pgfqpoint{4.937295in}{0.614959in}}%
\pgfpathlineto{\pgfqpoint{4.941591in}{0.614671in}}%
\pgfpathlineto{\pgfqpoint{4.945886in}{0.614384in}}%
\pgfpathlineto{\pgfqpoint{4.950181in}{0.614098in}}%
\pgfpathlineto{\pgfqpoint{4.954476in}{0.613812in}}%
\pgfpathlineto{\pgfqpoint{4.958771in}{0.613527in}}%
\pgfpathlineto{\pgfqpoint{4.963066in}{0.613242in}}%
\pgfpathlineto{\pgfqpoint{4.967361in}{0.612957in}}%
\pgfpathlineto{\pgfqpoint{4.971656in}{0.612674in}}%
\pgfpathlineto{\pgfqpoint{4.975951in}{0.612390in}}%
\pgfpathlineto{\pgfqpoint{4.980246in}{0.612108in}}%
\pgfpathlineto{\pgfqpoint{4.984541in}{0.611825in}}%
\pgfpathlineto{\pgfqpoint{4.988836in}{0.611544in}}%
\pgfpathlineto{\pgfqpoint{4.993131in}{0.611263in}}%
\pgfpathlineto{\pgfqpoint{4.997426in}{0.610982in}}%
\pgfpathlineto{\pgfqpoint{5.001721in}{0.610702in}}%
\pgfpathlineto{\pgfqpoint{5.006016in}{0.610423in}}%
\pgfpathlineto{\pgfqpoint{5.010311in}{0.610144in}}%
\pgfpathlineto{\pgfqpoint{5.014606in}{0.609865in}}%
\pgfpathlineto{\pgfqpoint{5.018901in}{0.609587in}}%
\pgfpathlineto{\pgfqpoint{5.023196in}{0.609310in}}%
\pgfpathlineto{\pgfqpoint{5.027491in}{0.609033in}}%
\pgfpathlineto{\pgfqpoint{5.031786in}{0.608757in}}%
\pgfpathlineto{\pgfqpoint{5.036081in}{0.608481in}}%
\pgfpathlineto{\pgfqpoint{5.040376in}{0.608205in}}%
\pgfpathlineto{\pgfqpoint{5.044671in}{0.607930in}}%
\pgfpathlineto{\pgfqpoint{5.048966in}{0.607656in}}%
\pgfpathlineto{\pgfqpoint{5.053261in}{0.607382in}}%
\pgfpathlineto{\pgfqpoint{5.057556in}{0.607109in}}%
\pgfpathlineto{\pgfqpoint{5.061851in}{0.606836in}}%
\pgfpathlineto{\pgfqpoint{5.066147in}{0.606564in}}%
\pgfpathlineto{\pgfqpoint{5.070442in}{0.606292in}}%
\pgfpathlineto{\pgfqpoint{5.074737in}{0.606021in}}%
\pgfpathlineto{\pgfqpoint{5.079032in}{0.605750in}}%
\pgfpathlineto{\pgfqpoint{5.083327in}{0.605480in}}%
\pgfpathlineto{\pgfqpoint{5.087622in}{0.605210in}}%
\pgfpathlineto{\pgfqpoint{5.091917in}{0.604941in}}%
\pgfpathlineto{\pgfqpoint{5.096212in}{0.604672in}}%
\pgfpathlineto{\pgfqpoint{5.100507in}{0.604404in}}%
\pgfpathlineto{\pgfqpoint{5.104802in}{0.604136in}}%
\pgfpathlineto{\pgfqpoint{5.109097in}{0.603869in}}%
\pgfpathlineto{\pgfqpoint{5.113392in}{0.603602in}}%
\pgfpathlineto{\pgfqpoint{5.117687in}{0.603336in}}%
\pgfpathlineto{\pgfqpoint{5.121982in}{0.603070in}}%
\pgfpathlineto{\pgfqpoint{5.126277in}{0.602805in}}%
\pgfpathlineto{\pgfqpoint{5.130572in}{0.602540in}}%
\pgfpathlineto{\pgfqpoint{5.134867in}{0.602276in}}%
\pgfpathlineto{\pgfqpoint{5.139162in}{0.602012in}}%
\pgfpathlineto{\pgfqpoint{5.143457in}{0.601748in}}%
\pgfpathlineto{\pgfqpoint{5.147752in}{0.601486in}}%
\pgfpathlineto{\pgfqpoint{5.152047in}{0.601223in}}%
\pgfpathlineto{\pgfqpoint{5.156342in}{0.600961in}}%
\pgfpathlineto{\pgfqpoint{5.160637in}{0.600700in}}%
\pgfpathlineto{\pgfqpoint{5.164932in}{0.600439in}}%
\pgfpathlineto{\pgfqpoint{5.164932in}{0.600439in}}%
\pgfpathlineto{\pgfqpoint{5.172027in}{0.600017in}}%
\pgfpathlineto{\pgfqpoint{5.179121in}{0.599611in}}%
\pgfpathlineto{\pgfqpoint{5.186216in}{0.599220in}}%
\pgfpathlineto{\pgfqpoint{5.193310in}{0.598843in}}%
\pgfpathlineto{\pgfqpoint{5.200404in}{0.598479in}}%
\pgfpathlineto{\pgfqpoint{5.207499in}{0.598127in}}%
\pgfpathlineto{\pgfqpoint{5.214593in}{0.597787in}}%
\pgfpathlineto{\pgfqpoint{5.221688in}{0.597458in}}%
\pgfpathlineto{\pgfqpoint{5.228782in}{0.597139in}}%
\pgfpathlineto{\pgfqpoint{5.235877in}{0.596829in}}%
\pgfpathlineto{\pgfqpoint{5.242971in}{0.596528in}}%
\pgfpathlineto{\pgfqpoint{5.250065in}{0.596236in}}%
\pgfpathlineto{\pgfqpoint{5.257160in}{0.595952in}}%
\pgfpathlineto{\pgfqpoint{5.264254in}{0.595675in}}%
\pgfpathlineto{\pgfqpoint{5.271349in}{0.595406in}}%
\pgfpathlineto{\pgfqpoint{5.278443in}{0.595144in}}%
\pgfpathlineto{\pgfqpoint{5.285538in}{0.594889in}}%
\pgfpathlineto{\pgfqpoint{5.292632in}{0.594640in}}%
\pgfpathlineto{\pgfqpoint{5.299726in}{0.594397in}}%
\pgfpathlineto{\pgfqpoint{5.306821in}{0.594159in}}%
\pgfpathlineto{\pgfqpoint{5.313915in}{0.593928in}}%
\pgfpathlineto{\pgfqpoint{5.321010in}{0.593701in}}%
\pgfpathlineto{\pgfqpoint{5.328104in}{0.593480in}}%
\pgfpathlineto{\pgfqpoint{5.335199in}{0.593263in}}%
\pgfpathlineto{\pgfqpoint{5.342293in}{0.593052in}}%
\pgfpathlineto{\pgfqpoint{5.349387in}{0.592844in}}%
\pgfpathlineto{\pgfqpoint{5.356482in}{0.592642in}}%
\pgfpathlineto{\pgfqpoint{5.363576in}{0.592443in}}%
\pgfpathlineto{\pgfqpoint{5.370671in}{0.592248in}}%
\pgfpathlineto{\pgfqpoint{5.377765in}{0.592057in}}%
\pgfpathlineto{\pgfqpoint{5.384860in}{0.591870in}}%
\pgfpathlineto{\pgfqpoint{5.391954in}{0.591687in}}%
\pgfpathlineto{\pgfqpoint{5.399048in}{0.591507in}}%
\pgfpathlineto{\pgfqpoint{5.406143in}{0.591330in}}%
\pgfpathlineto{\pgfqpoint{5.413237in}{0.591157in}}%
\pgfpathlineto{\pgfqpoint{5.420332in}{0.590987in}}%
\pgfpathlineto{\pgfqpoint{5.427426in}{0.590820in}}%
\pgfpathlineto{\pgfqpoint{5.434521in}{0.590656in}}%
\pgfpathlineto{\pgfqpoint{5.441615in}{0.590495in}}%
\pgfpathlineto{\pgfqpoint{5.448709in}{0.590336in}}%
\pgfpathlineto{\pgfqpoint{5.455804in}{0.590181in}}%
\pgfpathlineto{\pgfqpoint{5.462898in}{0.590028in}}%
\pgfpathlineto{\pgfqpoint{5.469993in}{0.589877in}}%
\pgfpathlineto{\pgfqpoint{5.477087in}{0.589729in}}%
\pgfpathlineto{\pgfqpoint{5.484182in}{0.589583in}}%
\pgfpathlineto{\pgfqpoint{5.491276in}{0.589440in}}%
\pgfpathlineto{\pgfqpoint{5.498370in}{0.589299in}}%
\pgfpathlineto{\pgfqpoint{5.505465in}{0.589160in}}%
\pgfpathlineto{\pgfqpoint{5.512559in}{0.589024in}}%
\pgfpathlineto{\pgfqpoint{5.519654in}{0.588889in}}%
\pgfpathlineto{\pgfqpoint{5.526748in}{0.588757in}}%
\pgfpathlineto{\pgfqpoint{5.533843in}{0.588626in}}%
\pgfpathlineto{\pgfqpoint{5.540937in}{0.588497in}}%
\pgfpathlineto{\pgfqpoint{5.548031in}{0.588371in}}%
\pgfpathlineto{\pgfqpoint{5.555126in}{0.588246in}}%
\pgfpathlineto{\pgfqpoint{5.562220in}{0.588123in}}%
\pgfpathlineto{\pgfqpoint{5.569315in}{0.588002in}}%
\pgfpathlineto{\pgfqpoint{5.576409in}{0.587882in}}%
\pgfpathlineto{\pgfqpoint{5.583504in}{0.587764in}}%
\pgfpathlineto{\pgfqpoint{5.590598in}{0.587648in}}%
\pgfpathlineto{\pgfqpoint{5.597692in}{0.587533in}}%
\pgfpathlineto{\pgfqpoint{5.604787in}{0.587420in}}%
\pgfpathlineto{\pgfqpoint{5.611881in}{0.587308in}}%
\pgfpathlineto{\pgfqpoint{5.618976in}{0.587198in}}%
\pgfpathlineto{\pgfqpoint{5.626070in}{0.587090in}}%
\pgfpathlineto{\pgfqpoint{5.633165in}{0.586982in}}%
\pgfpathlineto{\pgfqpoint{5.640259in}{0.586876in}}%
\pgfpathlineto{\pgfqpoint{5.647353in}{0.586772in}}%
\pgfpathlineto{\pgfqpoint{5.654448in}{0.586669in}}%
\pgfpathlineto{\pgfqpoint{5.661542in}{0.586567in}}%
\pgfpathlineto{\pgfqpoint{5.668637in}{0.586466in}}%
\pgfpathlineto{\pgfqpoint{5.675731in}{0.586367in}}%
\pgfpathlineto{\pgfqpoint{5.682826in}{0.586269in}}%
\pgfpathlineto{\pgfqpoint{5.689920in}{0.586172in}}%
\pgfpathlineto{\pgfqpoint{5.697014in}{0.586076in}}%
\pgfpathlineto{\pgfqpoint{5.704109in}{0.585982in}}%
\pgfpathlineto{\pgfqpoint{5.711203in}{0.585888in}}%
\pgfpathlineto{\pgfqpoint{5.718298in}{0.585796in}}%
\pgfpathlineto{\pgfqpoint{5.725392in}{0.585705in}}%
\pgfpathlineto{\pgfqpoint{5.732487in}{0.585615in}}%
\pgfpathlineto{\pgfqpoint{5.739581in}{0.585525in}}%
\pgfpathlineto{\pgfqpoint{5.746675in}{0.585437in}}%
\pgfpathlineto{\pgfqpoint{5.753770in}{0.585350in}}%
\pgfpathlineto{\pgfqpoint{5.760864in}{0.585264in}}%
\pgfpathlineto{\pgfqpoint{5.767959in}{0.585179in}}%
\pgfpathlineto{\pgfqpoint{5.775053in}{0.585095in}}%
\pgfpathlineto{\pgfqpoint{5.782147in}{0.585011in}}%
\pgfpathlineto{\pgfqpoint{5.789242in}{0.584929in}}%
\pgfpathlineto{\pgfqpoint{5.796336in}{0.584848in}}%
\pgfpathlineto{\pgfqpoint{5.803431in}{0.584767in}}%
\pgfpathlineto{\pgfqpoint{5.810525in}{0.584687in}}%
\pgfpathlineto{\pgfqpoint{5.817620in}{0.584608in}}%
\pgfpathlineto{\pgfqpoint{5.824714in}{0.584530in}}%
\pgfpathlineto{\pgfqpoint{5.831808in}{0.584453in}}%
\pgfpathlineto{\pgfqpoint{5.838903in}{0.584377in}}%
\pgfpathlineto{\pgfqpoint{5.845997in}{0.584301in}}%
\pgfpathlineto{\pgfqpoint{5.853092in}{0.584226in}}%
\pgfpathlineto{\pgfqpoint{5.860186in}{0.584152in}}%
\pgfpathlineto{\pgfqpoint{5.867281in}{0.584079in}}%
\pgfpathlineto{\pgfqpoint{5.867281in}{0.314383in}}%
\pgfpathlineto{\pgfqpoint{5.867281in}{0.314383in}}%
\pgfpathlineto{\pgfqpoint{5.860186in}{0.314383in}}%
\pgfpathlineto{\pgfqpoint{5.853092in}{0.314383in}}%
\pgfpathlineto{\pgfqpoint{5.845997in}{0.314383in}}%
\pgfpathlineto{\pgfqpoint{5.838903in}{0.314383in}}%
\pgfpathlineto{\pgfqpoint{5.831808in}{0.314383in}}%
\pgfpathlineto{\pgfqpoint{5.824714in}{0.314383in}}%
\pgfpathlineto{\pgfqpoint{5.817620in}{0.314383in}}%
\pgfpathlineto{\pgfqpoint{5.810525in}{0.314383in}}%
\pgfpathlineto{\pgfqpoint{5.803431in}{0.314383in}}%
\pgfpathlineto{\pgfqpoint{5.796336in}{0.314383in}}%
\pgfpathlineto{\pgfqpoint{5.789242in}{0.314383in}}%
\pgfpathlineto{\pgfqpoint{5.782147in}{0.314383in}}%
\pgfpathlineto{\pgfqpoint{5.775053in}{0.314383in}}%
\pgfpathlineto{\pgfqpoint{5.767959in}{0.314383in}}%
\pgfpathlineto{\pgfqpoint{5.760864in}{0.314383in}}%
\pgfpathlineto{\pgfqpoint{5.753770in}{0.314383in}}%
\pgfpathlineto{\pgfqpoint{5.746675in}{0.314383in}}%
\pgfpathlineto{\pgfqpoint{5.739581in}{0.314383in}}%
\pgfpathlineto{\pgfqpoint{5.732487in}{0.314383in}}%
\pgfpathlineto{\pgfqpoint{5.725392in}{0.314383in}}%
\pgfpathlineto{\pgfqpoint{5.718298in}{0.314383in}}%
\pgfpathlineto{\pgfqpoint{5.711203in}{0.314383in}}%
\pgfpathlineto{\pgfqpoint{5.704109in}{0.314383in}}%
\pgfpathlineto{\pgfqpoint{5.697014in}{0.314383in}}%
\pgfpathlineto{\pgfqpoint{5.689920in}{0.314383in}}%
\pgfpathlineto{\pgfqpoint{5.682826in}{0.314383in}}%
\pgfpathlineto{\pgfqpoint{5.675731in}{0.314383in}}%
\pgfpathlineto{\pgfqpoint{5.668637in}{0.314383in}}%
\pgfpathlineto{\pgfqpoint{5.661542in}{0.314383in}}%
\pgfpathlineto{\pgfqpoint{5.654448in}{0.314383in}}%
\pgfpathlineto{\pgfqpoint{5.647353in}{0.314383in}}%
\pgfpathlineto{\pgfqpoint{5.640259in}{0.314383in}}%
\pgfpathlineto{\pgfqpoint{5.633165in}{0.314383in}}%
\pgfpathlineto{\pgfqpoint{5.626070in}{0.314383in}}%
\pgfpathlineto{\pgfqpoint{5.618976in}{0.314383in}}%
\pgfpathlineto{\pgfqpoint{5.611881in}{0.314383in}}%
\pgfpathlineto{\pgfqpoint{5.604787in}{0.314383in}}%
\pgfpathlineto{\pgfqpoint{5.597692in}{0.314383in}}%
\pgfpathlineto{\pgfqpoint{5.590598in}{0.314383in}}%
\pgfpathlineto{\pgfqpoint{5.583504in}{0.314383in}}%
\pgfpathlineto{\pgfqpoint{5.576409in}{0.314383in}}%
\pgfpathlineto{\pgfqpoint{5.569315in}{0.314383in}}%
\pgfpathlineto{\pgfqpoint{5.562220in}{0.314383in}}%
\pgfpathlineto{\pgfqpoint{5.555126in}{0.314383in}}%
\pgfpathlineto{\pgfqpoint{5.548031in}{0.314383in}}%
\pgfpathlineto{\pgfqpoint{5.540937in}{0.314383in}}%
\pgfpathlineto{\pgfqpoint{5.533843in}{0.314383in}}%
\pgfpathlineto{\pgfqpoint{5.526748in}{0.314383in}}%
\pgfpathlineto{\pgfqpoint{5.519654in}{0.314383in}}%
\pgfpathlineto{\pgfqpoint{5.512559in}{0.314383in}}%
\pgfpathlineto{\pgfqpoint{5.505465in}{0.314383in}}%
\pgfpathlineto{\pgfqpoint{5.498370in}{0.314383in}}%
\pgfpathlineto{\pgfqpoint{5.491276in}{0.314383in}}%
\pgfpathlineto{\pgfqpoint{5.484182in}{0.314383in}}%
\pgfpathlineto{\pgfqpoint{5.477087in}{0.314383in}}%
\pgfpathlineto{\pgfqpoint{5.469993in}{0.314383in}}%
\pgfpathlineto{\pgfqpoint{5.462898in}{0.314383in}}%
\pgfpathlineto{\pgfqpoint{5.455804in}{0.314383in}}%
\pgfpathlineto{\pgfqpoint{5.448709in}{0.314383in}}%
\pgfpathlineto{\pgfqpoint{5.441615in}{0.314383in}}%
\pgfpathlineto{\pgfqpoint{5.434521in}{0.314383in}}%
\pgfpathlineto{\pgfqpoint{5.427426in}{0.314383in}}%
\pgfpathlineto{\pgfqpoint{5.420332in}{0.314383in}}%
\pgfpathlineto{\pgfqpoint{5.413237in}{0.314383in}}%
\pgfpathlineto{\pgfqpoint{5.406143in}{0.314383in}}%
\pgfpathlineto{\pgfqpoint{5.399048in}{0.314383in}}%
\pgfpathlineto{\pgfqpoint{5.391954in}{0.314383in}}%
\pgfpathlineto{\pgfqpoint{5.384860in}{0.314383in}}%
\pgfpathlineto{\pgfqpoint{5.377765in}{0.314383in}}%
\pgfpathlineto{\pgfqpoint{5.370671in}{0.314383in}}%
\pgfpathlineto{\pgfqpoint{5.363576in}{0.314383in}}%
\pgfpathlineto{\pgfqpoint{5.356482in}{0.314383in}}%
\pgfpathlineto{\pgfqpoint{5.349387in}{0.314383in}}%
\pgfpathlineto{\pgfqpoint{5.342293in}{0.314383in}}%
\pgfpathlineto{\pgfqpoint{5.335199in}{0.314383in}}%
\pgfpathlineto{\pgfqpoint{5.328104in}{0.314383in}}%
\pgfpathlineto{\pgfqpoint{5.321010in}{0.314383in}}%
\pgfpathlineto{\pgfqpoint{5.313915in}{0.314383in}}%
\pgfpathlineto{\pgfqpoint{5.306821in}{0.314383in}}%
\pgfpathlineto{\pgfqpoint{5.299726in}{0.314383in}}%
\pgfpathlineto{\pgfqpoint{5.292632in}{0.314383in}}%
\pgfpathlineto{\pgfqpoint{5.285538in}{0.314383in}}%
\pgfpathlineto{\pgfqpoint{5.278443in}{0.314383in}}%
\pgfpathlineto{\pgfqpoint{5.271349in}{0.314383in}}%
\pgfpathlineto{\pgfqpoint{5.264254in}{0.314383in}}%
\pgfpathlineto{\pgfqpoint{5.257160in}{0.314383in}}%
\pgfpathlineto{\pgfqpoint{5.250065in}{0.314383in}}%
\pgfpathlineto{\pgfqpoint{5.242971in}{0.314383in}}%
\pgfpathlineto{\pgfqpoint{5.235877in}{0.314383in}}%
\pgfpathlineto{\pgfqpoint{5.228782in}{0.314383in}}%
\pgfpathlineto{\pgfqpoint{5.221688in}{0.314383in}}%
\pgfpathlineto{\pgfqpoint{5.214593in}{0.314383in}}%
\pgfpathlineto{\pgfqpoint{5.207499in}{0.314383in}}%
\pgfpathlineto{\pgfqpoint{5.200404in}{0.314383in}}%
\pgfpathlineto{\pgfqpoint{5.193310in}{0.314383in}}%
\pgfpathlineto{\pgfqpoint{5.186216in}{0.314383in}}%
\pgfpathlineto{\pgfqpoint{5.179121in}{0.314383in}}%
\pgfpathlineto{\pgfqpoint{5.172027in}{0.314383in}}%
\pgfpathlineto{\pgfqpoint{5.164932in}{0.314383in}}%
\pgfpathlineto{\pgfqpoint{5.164932in}{0.314383in}}%
\pgfpathlineto{\pgfqpoint{5.160637in}{0.314383in}}%
\pgfpathlineto{\pgfqpoint{5.156342in}{0.314383in}}%
\pgfpathlineto{\pgfqpoint{5.152047in}{0.314383in}}%
\pgfpathlineto{\pgfqpoint{5.147752in}{0.314383in}}%
\pgfpathlineto{\pgfqpoint{5.143457in}{0.314383in}}%
\pgfpathlineto{\pgfqpoint{5.139162in}{0.314383in}}%
\pgfpathlineto{\pgfqpoint{5.134867in}{0.314383in}}%
\pgfpathlineto{\pgfqpoint{5.130572in}{0.314383in}}%
\pgfpathlineto{\pgfqpoint{5.126277in}{0.314383in}}%
\pgfpathlineto{\pgfqpoint{5.121982in}{0.314383in}}%
\pgfpathlineto{\pgfqpoint{5.117687in}{0.314383in}}%
\pgfpathlineto{\pgfqpoint{5.113392in}{0.314383in}}%
\pgfpathlineto{\pgfqpoint{5.109097in}{0.314383in}}%
\pgfpathlineto{\pgfqpoint{5.104802in}{0.314383in}}%
\pgfpathlineto{\pgfqpoint{5.100507in}{0.314383in}}%
\pgfpathlineto{\pgfqpoint{5.096212in}{0.314383in}}%
\pgfpathlineto{\pgfqpoint{5.091917in}{0.314383in}}%
\pgfpathlineto{\pgfqpoint{5.087622in}{0.314383in}}%
\pgfpathlineto{\pgfqpoint{5.083327in}{0.314383in}}%
\pgfpathlineto{\pgfqpoint{5.079032in}{0.314383in}}%
\pgfpathlineto{\pgfqpoint{5.074737in}{0.314383in}}%
\pgfpathlineto{\pgfqpoint{5.070442in}{0.314383in}}%
\pgfpathlineto{\pgfqpoint{5.066147in}{0.314383in}}%
\pgfpathlineto{\pgfqpoint{5.061851in}{0.314383in}}%
\pgfpathlineto{\pgfqpoint{5.057556in}{0.314383in}}%
\pgfpathlineto{\pgfqpoint{5.053261in}{0.314383in}}%
\pgfpathlineto{\pgfqpoint{5.048966in}{0.314383in}}%
\pgfpathlineto{\pgfqpoint{5.044671in}{0.314383in}}%
\pgfpathlineto{\pgfqpoint{5.040376in}{0.314383in}}%
\pgfpathlineto{\pgfqpoint{5.036081in}{0.314383in}}%
\pgfpathlineto{\pgfqpoint{5.031786in}{0.314383in}}%
\pgfpathlineto{\pgfqpoint{5.027491in}{0.314383in}}%
\pgfpathlineto{\pgfqpoint{5.023196in}{0.314383in}}%
\pgfpathlineto{\pgfqpoint{5.018901in}{0.314383in}}%
\pgfpathlineto{\pgfqpoint{5.014606in}{0.314383in}}%
\pgfpathlineto{\pgfqpoint{5.010311in}{0.314383in}}%
\pgfpathlineto{\pgfqpoint{5.006016in}{0.314383in}}%
\pgfpathlineto{\pgfqpoint{5.001721in}{0.314383in}}%
\pgfpathlineto{\pgfqpoint{4.997426in}{0.314383in}}%
\pgfpathlineto{\pgfqpoint{4.993131in}{0.314383in}}%
\pgfpathlineto{\pgfqpoint{4.988836in}{0.314383in}}%
\pgfpathlineto{\pgfqpoint{4.984541in}{0.314383in}}%
\pgfpathlineto{\pgfqpoint{4.980246in}{0.314383in}}%
\pgfpathlineto{\pgfqpoint{4.975951in}{0.314383in}}%
\pgfpathlineto{\pgfqpoint{4.971656in}{0.314383in}}%
\pgfpathlineto{\pgfqpoint{4.967361in}{0.314383in}}%
\pgfpathlineto{\pgfqpoint{4.963066in}{0.314383in}}%
\pgfpathlineto{\pgfqpoint{4.958771in}{0.314383in}}%
\pgfpathlineto{\pgfqpoint{4.954476in}{0.314383in}}%
\pgfpathlineto{\pgfqpoint{4.950181in}{0.314383in}}%
\pgfpathlineto{\pgfqpoint{4.945886in}{0.314383in}}%
\pgfpathlineto{\pgfqpoint{4.941591in}{0.314383in}}%
\pgfpathlineto{\pgfqpoint{4.937295in}{0.314383in}}%
\pgfpathlineto{\pgfqpoint{4.933000in}{0.314383in}}%
\pgfpathlineto{\pgfqpoint{4.928705in}{0.314383in}}%
\pgfpathlineto{\pgfqpoint{4.924410in}{0.314383in}}%
\pgfpathlineto{\pgfqpoint{4.920115in}{0.314383in}}%
\pgfpathlineto{\pgfqpoint{4.915820in}{0.314383in}}%
\pgfpathlineto{\pgfqpoint{4.911525in}{0.314383in}}%
\pgfpathlineto{\pgfqpoint{4.907230in}{0.314383in}}%
\pgfpathlineto{\pgfqpoint{4.902935in}{0.314383in}}%
\pgfpathlineto{\pgfqpoint{4.898640in}{0.314383in}}%
\pgfpathlineto{\pgfqpoint{4.894345in}{0.314383in}}%
\pgfpathlineto{\pgfqpoint{4.890050in}{0.314383in}}%
\pgfpathlineto{\pgfqpoint{4.885755in}{0.314383in}}%
\pgfpathlineto{\pgfqpoint{4.881460in}{0.314383in}}%
\pgfpathlineto{\pgfqpoint{4.877165in}{0.314383in}}%
\pgfpathlineto{\pgfqpoint{4.872870in}{0.314383in}}%
\pgfpathlineto{\pgfqpoint{4.868575in}{0.314383in}}%
\pgfpathlineto{\pgfqpoint{4.864280in}{0.314383in}}%
\pgfpathlineto{\pgfqpoint{4.859985in}{0.314383in}}%
\pgfpathlineto{\pgfqpoint{4.855690in}{0.314383in}}%
\pgfpathlineto{\pgfqpoint{4.851395in}{0.314383in}}%
\pgfpathlineto{\pgfqpoint{4.847100in}{0.314383in}}%
\pgfpathlineto{\pgfqpoint{4.842805in}{0.314383in}}%
\pgfpathlineto{\pgfqpoint{4.838510in}{0.314383in}}%
\pgfpathlineto{\pgfqpoint{4.834215in}{0.314383in}}%
\pgfpathlineto{\pgfqpoint{4.829920in}{0.314383in}}%
\pgfpathlineto{\pgfqpoint{4.825625in}{0.314383in}}%
\pgfpathlineto{\pgfqpoint{4.821330in}{0.314383in}}%
\pgfpathlineto{\pgfqpoint{4.817035in}{0.314383in}}%
\pgfpathlineto{\pgfqpoint{4.812739in}{0.314383in}}%
\pgfpathlineto{\pgfqpoint{4.808444in}{0.314383in}}%
\pgfpathlineto{\pgfqpoint{4.804149in}{0.314383in}}%
\pgfpathlineto{\pgfqpoint{4.799854in}{0.314383in}}%
\pgfpathlineto{\pgfqpoint{4.795559in}{0.314383in}}%
\pgfpathlineto{\pgfqpoint{4.791264in}{0.314383in}}%
\pgfpathlineto{\pgfqpoint{4.786969in}{0.314383in}}%
\pgfpathlineto{\pgfqpoint{4.782674in}{0.314383in}}%
\pgfpathlineto{\pgfqpoint{4.778379in}{0.314383in}}%
\pgfpathlineto{\pgfqpoint{4.774084in}{0.314383in}}%
\pgfpathlineto{\pgfqpoint{4.769789in}{0.314383in}}%
\pgfpathlineto{\pgfqpoint{4.765494in}{0.314383in}}%
\pgfpathlineto{\pgfqpoint{4.761199in}{0.314383in}}%
\pgfpathlineto{\pgfqpoint{4.756904in}{0.314383in}}%
\pgfpathlineto{\pgfqpoint{4.752609in}{0.314383in}}%
\pgfpathlineto{\pgfqpoint{4.748314in}{0.314383in}}%
\pgfpathlineto{\pgfqpoint{4.744019in}{0.314383in}}%
\pgfpathlineto{\pgfqpoint{4.739724in}{0.314383in}}%
\pgfpathlineto{\pgfqpoint{4.735429in}{0.314383in}}%
\pgfpathlineto{\pgfqpoint{4.731134in}{0.314383in}}%
\pgfpathlineto{\pgfqpoint{4.726839in}{0.314383in}}%
\pgfpathlineto{\pgfqpoint{4.722544in}{0.314383in}}%
\pgfpathlineto{\pgfqpoint{4.718249in}{0.314383in}}%
\pgfpathlineto{\pgfqpoint{4.713954in}{0.314383in}}%
\pgfpathlineto{\pgfqpoint{4.709659in}{0.314383in}}%
\pgfpathlineto{\pgfqpoint{4.705364in}{0.314383in}}%
\pgfpathlineto{\pgfqpoint{4.701069in}{0.314383in}}%
\pgfpathlineto{\pgfqpoint{4.696774in}{0.314383in}}%
\pgfpathlineto{\pgfqpoint{4.692479in}{0.314383in}}%
\pgfpathlineto{\pgfqpoint{4.688183in}{0.314383in}}%
\pgfpathlineto{\pgfqpoint{4.683888in}{0.314383in}}%
\pgfpathlineto{\pgfqpoint{4.679593in}{0.314383in}}%
\pgfpathlineto{\pgfqpoint{4.675298in}{0.314383in}}%
\pgfpathlineto{\pgfqpoint{4.671003in}{0.314383in}}%
\pgfpathlineto{\pgfqpoint{4.666708in}{0.314383in}}%
\pgfpathlineto{\pgfqpoint{4.662413in}{0.314383in}}%
\pgfpathlineto{\pgfqpoint{4.658118in}{0.314383in}}%
\pgfpathlineto{\pgfqpoint{4.653823in}{0.314383in}}%
\pgfpathlineto{\pgfqpoint{4.649528in}{0.314383in}}%
\pgfpathlineto{\pgfqpoint{4.645233in}{0.314383in}}%
\pgfpathlineto{\pgfqpoint{4.640938in}{0.314383in}}%
\pgfpathlineto{\pgfqpoint{4.636643in}{0.314383in}}%
\pgfpathlineto{\pgfqpoint{4.632348in}{0.314383in}}%
\pgfpathlineto{\pgfqpoint{4.628053in}{0.314383in}}%
\pgfpathlineto{\pgfqpoint{4.623758in}{0.314383in}}%
\pgfpathlineto{\pgfqpoint{4.619463in}{0.314383in}}%
\pgfpathlineto{\pgfqpoint{4.615168in}{0.314383in}}%
\pgfpathlineto{\pgfqpoint{4.610873in}{0.314383in}}%
\pgfpathlineto{\pgfqpoint{4.606578in}{0.314383in}}%
\pgfpathlineto{\pgfqpoint{4.602283in}{0.314383in}}%
\pgfpathlineto{\pgfqpoint{4.597988in}{0.314383in}}%
\pgfpathlineto{\pgfqpoint{4.593693in}{0.314383in}}%
\pgfpathlineto{\pgfqpoint{4.589398in}{0.314383in}}%
\pgfpathlineto{\pgfqpoint{4.585103in}{0.314383in}}%
\pgfpathlineto{\pgfqpoint{4.580808in}{0.314383in}}%
\pgfpathlineto{\pgfqpoint{4.576513in}{0.314383in}}%
\pgfpathlineto{\pgfqpoint{4.572218in}{0.314383in}}%
\pgfpathlineto{\pgfqpoint{4.567923in}{0.314383in}}%
\pgfpathlineto{\pgfqpoint{4.563627in}{0.314383in}}%
\pgfpathlineto{\pgfqpoint{4.559332in}{0.314383in}}%
\pgfpathlineto{\pgfqpoint{4.555037in}{0.314383in}}%
\pgfpathlineto{\pgfqpoint{4.550742in}{0.314383in}}%
\pgfpathlineto{\pgfqpoint{4.546447in}{0.314383in}}%
\pgfpathlineto{\pgfqpoint{4.542152in}{0.314383in}}%
\pgfpathlineto{\pgfqpoint{4.537857in}{0.314383in}}%
\pgfpathlineto{\pgfqpoint{4.533562in}{0.314383in}}%
\pgfpathlineto{\pgfqpoint{4.529267in}{0.314383in}}%
\pgfpathlineto{\pgfqpoint{4.524972in}{0.314383in}}%
\pgfpathlineto{\pgfqpoint{4.520677in}{0.314383in}}%
\pgfpathlineto{\pgfqpoint{4.516382in}{0.314383in}}%
\pgfpathlineto{\pgfqpoint{4.512087in}{0.314383in}}%
\pgfpathlineto{\pgfqpoint{4.507792in}{0.314383in}}%
\pgfpathlineto{\pgfqpoint{4.503497in}{0.314383in}}%
\pgfpathlineto{\pgfqpoint{4.499202in}{0.314383in}}%
\pgfpathlineto{\pgfqpoint{4.494907in}{0.314383in}}%
\pgfpathlineto{\pgfqpoint{4.490612in}{0.314383in}}%
\pgfpathlineto{\pgfqpoint{4.486317in}{0.314383in}}%
\pgfpathlineto{\pgfqpoint{4.482022in}{0.314383in}}%
\pgfpathlineto{\pgfqpoint{4.477727in}{0.314383in}}%
\pgfpathlineto{\pgfqpoint{4.473432in}{0.314383in}}%
\pgfpathlineto{\pgfqpoint{4.469137in}{0.314383in}}%
\pgfpathlineto{\pgfqpoint{4.464842in}{0.314383in}}%
\pgfpathlineto{\pgfqpoint{4.460547in}{0.314383in}}%
\pgfpathlineto{\pgfqpoint{4.456252in}{0.314383in}}%
\pgfpathlineto{\pgfqpoint{4.451957in}{0.314383in}}%
\pgfpathlineto{\pgfqpoint{4.447662in}{0.314383in}}%
\pgfpathlineto{\pgfqpoint{4.443367in}{0.314383in}}%
\pgfpathlineto{\pgfqpoint{4.439071in}{0.314383in}}%
\pgfpathlineto{\pgfqpoint{4.434776in}{0.314383in}}%
\pgfpathlineto{\pgfqpoint{4.430481in}{0.314383in}}%
\pgfpathlineto{\pgfqpoint{4.426186in}{0.314383in}}%
\pgfpathlineto{\pgfqpoint{4.421891in}{0.314383in}}%
\pgfpathlineto{\pgfqpoint{4.417596in}{0.314383in}}%
\pgfpathlineto{\pgfqpoint{4.413301in}{0.314383in}}%
\pgfpathlineto{\pgfqpoint{4.409006in}{0.314383in}}%
\pgfpathlineto{\pgfqpoint{4.404711in}{0.314383in}}%
\pgfpathlineto{\pgfqpoint{4.400416in}{0.314383in}}%
\pgfpathlineto{\pgfqpoint{4.396121in}{0.314383in}}%
\pgfpathlineto{\pgfqpoint{4.391826in}{0.314383in}}%
\pgfpathlineto{\pgfqpoint{4.387531in}{0.314383in}}%
\pgfpathlineto{\pgfqpoint{4.383236in}{0.314383in}}%
\pgfpathlineto{\pgfqpoint{4.378941in}{0.314383in}}%
\pgfpathlineto{\pgfqpoint{4.374646in}{0.314383in}}%
\pgfpathlineto{\pgfqpoint{4.370351in}{0.314383in}}%
\pgfpathlineto{\pgfqpoint{4.366056in}{0.314383in}}%
\pgfpathlineto{\pgfqpoint{4.361761in}{0.314383in}}%
\pgfpathlineto{\pgfqpoint{4.357466in}{0.314383in}}%
\pgfpathlineto{\pgfqpoint{4.353171in}{0.314383in}}%
\pgfpathlineto{\pgfqpoint{4.348876in}{0.314383in}}%
\pgfpathlineto{\pgfqpoint{4.344581in}{0.314383in}}%
\pgfpathlineto{\pgfqpoint{4.340286in}{0.314383in}}%
\pgfpathlineto{\pgfqpoint{4.335991in}{0.314383in}}%
\pgfpathlineto{\pgfqpoint{4.331696in}{0.314383in}}%
\pgfpathlineto{\pgfqpoint{4.327401in}{0.314383in}}%
\pgfpathlineto{\pgfqpoint{4.323106in}{0.314383in}}%
\pgfpathlineto{\pgfqpoint{4.318811in}{0.314383in}}%
\pgfpathlineto{\pgfqpoint{4.314515in}{0.314383in}}%
\pgfpathlineto{\pgfqpoint{4.310220in}{0.314383in}}%
\pgfpathlineto{\pgfqpoint{4.305925in}{0.314383in}}%
\pgfpathlineto{\pgfqpoint{4.301630in}{0.314383in}}%
\pgfpathlineto{\pgfqpoint{4.297335in}{0.314383in}}%
\pgfpathlineto{\pgfqpoint{4.293040in}{0.314383in}}%
\pgfpathlineto{\pgfqpoint{4.288745in}{0.314383in}}%
\pgfpathlineto{\pgfqpoint{4.284450in}{0.314383in}}%
\pgfpathlineto{\pgfqpoint{4.280155in}{0.314383in}}%
\pgfpathlineto{\pgfqpoint{4.275860in}{0.314383in}}%
\pgfpathlineto{\pgfqpoint{4.271565in}{0.314383in}}%
\pgfpathlineto{\pgfqpoint{4.267270in}{0.314383in}}%
\pgfpathlineto{\pgfqpoint{4.262975in}{0.314383in}}%
\pgfpathlineto{\pgfqpoint{4.258680in}{0.314383in}}%
\pgfpathlineto{\pgfqpoint{4.254385in}{0.314383in}}%
\pgfpathlineto{\pgfqpoint{4.250090in}{0.314383in}}%
\pgfpathlineto{\pgfqpoint{4.245795in}{0.314383in}}%
\pgfpathlineto{\pgfqpoint{4.241500in}{0.314383in}}%
\pgfpathlineto{\pgfqpoint{4.237205in}{0.314383in}}%
\pgfpathlineto{\pgfqpoint{4.232910in}{0.314383in}}%
\pgfpathlineto{\pgfqpoint{4.228615in}{0.314383in}}%
\pgfpathlineto{\pgfqpoint{4.224320in}{0.314383in}}%
\pgfpathlineto{\pgfqpoint{4.220025in}{0.314383in}}%
\pgfpathlineto{\pgfqpoint{4.215730in}{0.314383in}}%
\pgfpathlineto{\pgfqpoint{4.211435in}{0.314383in}}%
\pgfpathlineto{\pgfqpoint{4.207140in}{0.314383in}}%
\pgfpathlineto{\pgfqpoint{4.202845in}{0.314383in}}%
\pgfpathlineto{\pgfqpoint{4.198550in}{0.314383in}}%
\pgfpathlineto{\pgfqpoint{4.194255in}{0.314383in}}%
\pgfpathlineto{\pgfqpoint{4.189959in}{0.314383in}}%
\pgfpathlineto{\pgfqpoint{4.185664in}{0.314383in}}%
\pgfpathlineto{\pgfqpoint{4.181369in}{0.314383in}}%
\pgfpathlineto{\pgfqpoint{4.177074in}{0.314383in}}%
\pgfpathlineto{\pgfqpoint{4.172779in}{0.314383in}}%
\pgfpathlineto{\pgfqpoint{4.168484in}{0.314383in}}%
\pgfpathlineto{\pgfqpoint{4.164189in}{0.314383in}}%
\pgfpathlineto{\pgfqpoint{4.159894in}{0.314383in}}%
\pgfpathlineto{\pgfqpoint{4.155599in}{0.314383in}}%
\pgfpathlineto{\pgfqpoint{4.151304in}{0.314383in}}%
\pgfpathlineto{\pgfqpoint{4.147009in}{0.314383in}}%
\pgfpathlineto{\pgfqpoint{4.142714in}{0.314383in}}%
\pgfpathlineto{\pgfqpoint{4.138419in}{0.314383in}}%
\pgfpathlineto{\pgfqpoint{4.134124in}{0.314383in}}%
\pgfpathlineto{\pgfqpoint{4.129829in}{0.314383in}}%
\pgfpathlineto{\pgfqpoint{4.125534in}{0.314383in}}%
\pgfpathlineto{\pgfqpoint{4.121239in}{0.314383in}}%
\pgfpathlineto{\pgfqpoint{4.116944in}{0.314383in}}%
\pgfpathlineto{\pgfqpoint{4.112649in}{0.314383in}}%
\pgfpathlineto{\pgfqpoint{4.108354in}{0.314383in}}%
\pgfpathlineto{\pgfqpoint{4.104059in}{0.314383in}}%
\pgfpathlineto{\pgfqpoint{4.099764in}{0.314383in}}%
\pgfpathlineto{\pgfqpoint{4.095469in}{0.314383in}}%
\pgfpathlineto{\pgfqpoint{4.091174in}{0.314383in}}%
\pgfpathlineto{\pgfqpoint{4.086879in}{0.314383in}}%
\pgfpathlineto{\pgfqpoint{4.082584in}{0.314383in}}%
\pgfpathlineto{\pgfqpoint{4.078289in}{0.314383in}}%
\pgfpathlineto{\pgfqpoint{4.073994in}{0.314383in}}%
\pgfpathlineto{\pgfqpoint{4.069699in}{0.314383in}}%
\pgfpathlineto{\pgfqpoint{4.065403in}{0.314383in}}%
\pgfpathlineto{\pgfqpoint{4.061108in}{0.314383in}}%
\pgfpathlineto{\pgfqpoint{4.056813in}{0.314383in}}%
\pgfpathlineto{\pgfqpoint{4.052518in}{0.314383in}}%
\pgfpathlineto{\pgfqpoint{4.048223in}{0.314383in}}%
\pgfpathlineto{\pgfqpoint{4.043928in}{0.314383in}}%
\pgfpathlineto{\pgfqpoint{4.039633in}{0.314383in}}%
\pgfpathlineto{\pgfqpoint{4.035338in}{0.314383in}}%
\pgfpathlineto{\pgfqpoint{4.031043in}{0.314383in}}%
\pgfpathlineto{\pgfqpoint{4.026748in}{0.314383in}}%
\pgfpathlineto{\pgfqpoint{4.022453in}{0.314383in}}%
\pgfpathlineto{\pgfqpoint{4.018158in}{0.314383in}}%
\pgfpathlineto{\pgfqpoint{4.013863in}{0.314383in}}%
\pgfpathlineto{\pgfqpoint{4.009568in}{0.314383in}}%
\pgfpathlineto{\pgfqpoint{4.005273in}{0.314383in}}%
\pgfpathlineto{\pgfqpoint{4.000978in}{0.314383in}}%
\pgfpathlineto{\pgfqpoint{3.996683in}{0.314383in}}%
\pgfpathlineto{\pgfqpoint{3.992388in}{0.314383in}}%
\pgfpathlineto{\pgfqpoint{3.988093in}{0.314383in}}%
\pgfpathlineto{\pgfqpoint{3.983798in}{0.314383in}}%
\pgfpathlineto{\pgfqpoint{3.979503in}{0.314383in}}%
\pgfpathlineto{\pgfqpoint{3.975208in}{0.314383in}}%
\pgfpathlineto{\pgfqpoint{3.970913in}{0.314383in}}%
\pgfpathlineto{\pgfqpoint{3.966618in}{0.314383in}}%
\pgfpathlineto{\pgfqpoint{3.962323in}{0.314383in}}%
\pgfpathlineto{\pgfqpoint{3.958028in}{0.314383in}}%
\pgfpathlineto{\pgfqpoint{3.953733in}{0.314383in}}%
\pgfpathlineto{\pgfqpoint{3.949438in}{0.314383in}}%
\pgfpathlineto{\pgfqpoint{3.945143in}{0.314383in}}%
\pgfpathlineto{\pgfqpoint{3.940847in}{0.314383in}}%
\pgfpathlineto{\pgfqpoint{3.936552in}{0.314383in}}%
\pgfpathlineto{\pgfqpoint{3.932257in}{0.314383in}}%
\pgfpathlineto{\pgfqpoint{3.927962in}{0.314383in}}%
\pgfpathlineto{\pgfqpoint{3.923667in}{0.314383in}}%
\pgfpathlineto{\pgfqpoint{3.919372in}{0.314383in}}%
\pgfpathlineto{\pgfqpoint{3.915077in}{0.314383in}}%
\pgfpathlineto{\pgfqpoint{3.910782in}{0.314383in}}%
\pgfpathlineto{\pgfqpoint{3.906487in}{0.314383in}}%
\pgfpathlineto{\pgfqpoint{3.902192in}{0.314383in}}%
\pgfpathlineto{\pgfqpoint{3.897897in}{0.314383in}}%
\pgfpathlineto{\pgfqpoint{3.893602in}{0.314383in}}%
\pgfpathlineto{\pgfqpoint{3.889307in}{0.314383in}}%
\pgfpathlineto{\pgfqpoint{3.885012in}{0.314383in}}%
\pgfpathlineto{\pgfqpoint{3.880717in}{0.314383in}}%
\pgfpathlineto{\pgfqpoint{3.876422in}{0.314383in}}%
\pgfpathlineto{\pgfqpoint{3.872127in}{0.314383in}}%
\pgfpathlineto{\pgfqpoint{3.867832in}{0.314383in}}%
\pgfpathlineto{\pgfqpoint{3.863537in}{0.314383in}}%
\pgfpathlineto{\pgfqpoint{3.859242in}{0.314383in}}%
\pgfpathlineto{\pgfqpoint{3.854947in}{0.314383in}}%
\pgfpathlineto{\pgfqpoint{3.850652in}{0.314383in}}%
\pgfpathlineto{\pgfqpoint{3.846357in}{0.314383in}}%
\pgfpathlineto{\pgfqpoint{3.842062in}{0.314383in}}%
\pgfpathlineto{\pgfqpoint{3.837767in}{0.314383in}}%
\pgfpathlineto{\pgfqpoint{3.833472in}{0.314383in}}%
\pgfpathlineto{\pgfqpoint{3.829177in}{0.314383in}}%
\pgfpathlineto{\pgfqpoint{3.824882in}{0.314383in}}%
\pgfpathlineto{\pgfqpoint{3.820587in}{0.314383in}}%
\pgfpathlineto{\pgfqpoint{3.816291in}{0.314383in}}%
\pgfpathlineto{\pgfqpoint{3.811996in}{0.314383in}}%
\pgfpathlineto{\pgfqpoint{3.807701in}{0.314383in}}%
\pgfpathlineto{\pgfqpoint{3.803406in}{0.314383in}}%
\pgfpathlineto{\pgfqpoint{3.799111in}{0.314383in}}%
\pgfpathlineto{\pgfqpoint{3.794816in}{0.314383in}}%
\pgfpathlineto{\pgfqpoint{3.790521in}{0.314383in}}%
\pgfpathlineto{\pgfqpoint{3.786226in}{0.314383in}}%
\pgfpathlineto{\pgfqpoint{3.781931in}{0.314383in}}%
\pgfpathlineto{\pgfqpoint{3.777636in}{0.314383in}}%
\pgfpathlineto{\pgfqpoint{3.773341in}{0.314383in}}%
\pgfpathlineto{\pgfqpoint{3.769046in}{0.314383in}}%
\pgfpathlineto{\pgfqpoint{3.764751in}{0.314383in}}%
\pgfpathlineto{\pgfqpoint{3.760456in}{0.314383in}}%
\pgfpathlineto{\pgfqpoint{3.756161in}{0.314383in}}%
\pgfpathlineto{\pgfqpoint{3.751866in}{0.314383in}}%
\pgfpathlineto{\pgfqpoint{3.747571in}{0.314383in}}%
\pgfpathlineto{\pgfqpoint{3.743276in}{0.314383in}}%
\pgfpathlineto{\pgfqpoint{3.738981in}{0.314383in}}%
\pgfpathlineto{\pgfqpoint{3.734686in}{0.314383in}}%
\pgfpathlineto{\pgfqpoint{3.730391in}{0.314383in}}%
\pgfpathlineto{\pgfqpoint{3.726096in}{0.314383in}}%
\pgfpathlineto{\pgfqpoint{3.721801in}{0.314383in}}%
\pgfpathlineto{\pgfqpoint{3.717506in}{0.314383in}}%
\pgfpathlineto{\pgfqpoint{3.713211in}{0.314383in}}%
\pgfpathlineto{\pgfqpoint{3.708916in}{0.314383in}}%
\pgfpathlineto{\pgfqpoint{3.704621in}{0.314383in}}%
\pgfpathlineto{\pgfqpoint{3.700326in}{0.314383in}}%
\pgfpathlineto{\pgfqpoint{3.696031in}{0.314383in}}%
\pgfpathlineto{\pgfqpoint{3.691735in}{0.314383in}}%
\pgfpathlineto{\pgfqpoint{3.687440in}{0.314383in}}%
\pgfpathlineto{\pgfqpoint{3.683145in}{0.314383in}}%
\pgfpathlineto{\pgfqpoint{3.678850in}{0.314383in}}%
\pgfpathlineto{\pgfqpoint{3.674555in}{0.314383in}}%
\pgfpathlineto{\pgfqpoint{3.670260in}{0.314383in}}%
\pgfpathlineto{\pgfqpoint{3.665965in}{0.314383in}}%
\pgfpathlineto{\pgfqpoint{3.661670in}{0.314383in}}%
\pgfpathlineto{\pgfqpoint{3.657375in}{0.314383in}}%
\pgfpathlineto{\pgfqpoint{3.653080in}{0.314383in}}%
\pgfpathlineto{\pgfqpoint{3.648785in}{0.314383in}}%
\pgfpathlineto{\pgfqpoint{3.644490in}{0.314383in}}%
\pgfpathlineto{\pgfqpoint{3.640195in}{0.314383in}}%
\pgfpathlineto{\pgfqpoint{3.635900in}{0.314383in}}%
\pgfpathlineto{\pgfqpoint{3.631605in}{0.314383in}}%
\pgfpathlineto{\pgfqpoint{3.627310in}{0.314383in}}%
\pgfpathlineto{\pgfqpoint{3.623015in}{0.314383in}}%
\pgfpathlineto{\pgfqpoint{3.618720in}{0.314383in}}%
\pgfpathlineto{\pgfqpoint{3.614425in}{0.314383in}}%
\pgfpathlineto{\pgfqpoint{3.610130in}{0.314383in}}%
\pgfpathlineto{\pgfqpoint{3.605835in}{0.314383in}}%
\pgfpathlineto{\pgfqpoint{3.601540in}{0.314383in}}%
\pgfpathlineto{\pgfqpoint{3.597245in}{0.314383in}}%
\pgfpathlineto{\pgfqpoint{3.592950in}{0.314383in}}%
\pgfpathlineto{\pgfqpoint{3.588655in}{0.314383in}}%
\pgfpathlineto{\pgfqpoint{3.584360in}{0.314383in}}%
\pgfpathlineto{\pgfqpoint{3.580065in}{0.314383in}}%
\pgfpathlineto{\pgfqpoint{3.575770in}{0.314383in}}%
\pgfpathlineto{\pgfqpoint{3.571475in}{0.314383in}}%
\pgfpathlineto{\pgfqpoint{3.567179in}{0.314383in}}%
\pgfpathlineto{\pgfqpoint{3.562884in}{0.314383in}}%
\pgfpathlineto{\pgfqpoint{3.558589in}{0.314383in}}%
\pgfpathlineto{\pgfqpoint{3.554294in}{0.314383in}}%
\pgfpathlineto{\pgfqpoint{3.549999in}{0.314383in}}%
\pgfpathlineto{\pgfqpoint{3.545704in}{0.314383in}}%
\pgfpathlineto{\pgfqpoint{3.541409in}{0.314383in}}%
\pgfpathlineto{\pgfqpoint{3.537114in}{0.314383in}}%
\pgfpathlineto{\pgfqpoint{3.532819in}{0.314383in}}%
\pgfpathlineto{\pgfqpoint{3.528524in}{0.314383in}}%
\pgfpathlineto{\pgfqpoint{3.524229in}{0.314383in}}%
\pgfpathlineto{\pgfqpoint{3.519934in}{0.314383in}}%
\pgfpathlineto{\pgfqpoint{3.515639in}{0.314383in}}%
\pgfpathlineto{\pgfqpoint{3.511344in}{0.314383in}}%
\pgfpathlineto{\pgfqpoint{3.507049in}{0.314383in}}%
\pgfpathlineto{\pgfqpoint{3.502754in}{0.314383in}}%
\pgfpathlineto{\pgfqpoint{3.498459in}{0.314383in}}%
\pgfpathlineto{\pgfqpoint{3.494164in}{0.314383in}}%
\pgfpathlineto{\pgfqpoint{3.489869in}{0.314383in}}%
\pgfpathlineto{\pgfqpoint{3.485574in}{0.314383in}}%
\pgfpathlineto{\pgfqpoint{3.481279in}{0.314383in}}%
\pgfpathlineto{\pgfqpoint{3.476984in}{0.314383in}}%
\pgfpathlineto{\pgfqpoint{3.472689in}{0.314383in}}%
\pgfpathlineto{\pgfqpoint{3.468394in}{0.314383in}}%
\pgfpathlineto{\pgfqpoint{3.464099in}{0.314383in}}%
\pgfpathlineto{\pgfqpoint{3.459804in}{0.314383in}}%
\pgfpathlineto{\pgfqpoint{3.455509in}{0.314383in}}%
\pgfpathlineto{\pgfqpoint{3.451214in}{0.314383in}}%
\pgfpathlineto{\pgfqpoint{3.446919in}{0.314383in}}%
\pgfpathlineto{\pgfqpoint{3.442623in}{0.314383in}}%
\pgfpathlineto{\pgfqpoint{3.438328in}{0.314383in}}%
\pgfpathlineto{\pgfqpoint{3.434033in}{0.314383in}}%
\pgfpathlineto{\pgfqpoint{3.429738in}{0.314383in}}%
\pgfpathlineto{\pgfqpoint{3.425443in}{0.314383in}}%
\pgfpathlineto{\pgfqpoint{3.421148in}{0.314383in}}%
\pgfpathlineto{\pgfqpoint{3.416853in}{0.314383in}}%
\pgfpathlineto{\pgfqpoint{3.412558in}{0.314383in}}%
\pgfpathlineto{\pgfqpoint{3.408263in}{0.314383in}}%
\pgfpathlineto{\pgfqpoint{3.403968in}{0.314383in}}%
\pgfpathlineto{\pgfqpoint{3.399673in}{0.314383in}}%
\pgfpathlineto{\pgfqpoint{3.395378in}{0.314383in}}%
\pgfpathlineto{\pgfqpoint{3.391083in}{0.314383in}}%
\pgfpathlineto{\pgfqpoint{3.386788in}{0.314383in}}%
\pgfpathlineto{\pgfqpoint{3.382493in}{0.314383in}}%
\pgfpathlineto{\pgfqpoint{3.378198in}{0.314383in}}%
\pgfpathlineto{\pgfqpoint{3.373903in}{0.314383in}}%
\pgfpathlineto{\pgfqpoint{3.369608in}{0.314383in}}%
\pgfpathlineto{\pgfqpoint{3.365313in}{0.314383in}}%
\pgfpathlineto{\pgfqpoint{3.361018in}{0.314383in}}%
\pgfpathlineto{\pgfqpoint{3.356723in}{0.314383in}}%
\pgfpathlineto{\pgfqpoint{3.352428in}{0.314383in}}%
\pgfpathlineto{\pgfqpoint{3.348133in}{0.314383in}}%
\pgfpathlineto{\pgfqpoint{3.343838in}{0.314383in}}%
\pgfpathlineto{\pgfqpoint{3.339543in}{0.314383in}}%
\pgfpathlineto{\pgfqpoint{3.335248in}{0.314383in}}%
\pgfpathlineto{\pgfqpoint{3.330953in}{0.314383in}}%
\pgfpathlineto{\pgfqpoint{3.326658in}{0.314383in}}%
\pgfpathlineto{\pgfqpoint{3.322362in}{0.314383in}}%
\pgfpathlineto{\pgfqpoint{3.318067in}{0.314383in}}%
\pgfpathlineto{\pgfqpoint{3.313772in}{0.314383in}}%
\pgfpathlineto{\pgfqpoint{3.309477in}{0.314383in}}%
\pgfpathlineto{\pgfqpoint{3.305182in}{0.314383in}}%
\pgfpathlineto{\pgfqpoint{3.300887in}{0.314383in}}%
\pgfpathlineto{\pgfqpoint{3.296592in}{0.314383in}}%
\pgfpathlineto{\pgfqpoint{3.292297in}{0.314383in}}%
\pgfpathlineto{\pgfqpoint{3.288002in}{0.314383in}}%
\pgfpathlineto{\pgfqpoint{3.283707in}{0.314383in}}%
\pgfpathlineto{\pgfqpoint{3.279412in}{0.314383in}}%
\pgfpathlineto{\pgfqpoint{3.275117in}{0.314383in}}%
\pgfpathlineto{\pgfqpoint{3.270822in}{0.314383in}}%
\pgfpathlineto{\pgfqpoint{3.266527in}{0.314383in}}%
\pgfpathlineto{\pgfqpoint{3.262232in}{0.314383in}}%
\pgfpathlineto{\pgfqpoint{3.257937in}{0.314383in}}%
\pgfpathlineto{\pgfqpoint{3.253642in}{0.314383in}}%
\pgfpathlineto{\pgfqpoint{3.249347in}{0.314383in}}%
\pgfpathlineto{\pgfqpoint{3.245052in}{0.314383in}}%
\pgfpathlineto{\pgfqpoint{3.240757in}{0.314383in}}%
\pgfpathlineto{\pgfqpoint{3.236462in}{0.314383in}}%
\pgfpathlineto{\pgfqpoint{3.232167in}{0.314383in}}%
\pgfpathlineto{\pgfqpoint{3.227872in}{0.314383in}}%
\pgfpathlineto{\pgfqpoint{3.223577in}{0.314383in}}%
\pgfpathlineto{\pgfqpoint{3.219282in}{0.314383in}}%
\pgfpathlineto{\pgfqpoint{3.214987in}{0.314383in}}%
\pgfpathlineto{\pgfqpoint{3.210692in}{0.314383in}}%
\pgfpathlineto{\pgfqpoint{3.206397in}{0.314383in}}%
\pgfpathlineto{\pgfqpoint{3.202102in}{0.314383in}}%
\pgfpathlineto{\pgfqpoint{3.197806in}{0.314383in}}%
\pgfpathlineto{\pgfqpoint{3.193511in}{0.314383in}}%
\pgfpathlineto{\pgfqpoint{3.189216in}{0.314383in}}%
\pgfpathlineto{\pgfqpoint{3.184921in}{0.314383in}}%
\pgfpathlineto{\pgfqpoint{3.180626in}{0.314383in}}%
\pgfpathlineto{\pgfqpoint{3.176331in}{0.314383in}}%
\pgfpathlineto{\pgfqpoint{3.172036in}{0.314383in}}%
\pgfpathlineto{\pgfqpoint{3.167741in}{0.314383in}}%
\pgfpathlineto{\pgfqpoint{3.163446in}{0.314383in}}%
\pgfpathlineto{\pgfqpoint{3.159151in}{0.314383in}}%
\pgfpathlineto{\pgfqpoint{3.154856in}{0.314383in}}%
\pgfpathlineto{\pgfqpoint{3.150561in}{0.314383in}}%
\pgfpathlineto{\pgfqpoint{3.146266in}{0.314383in}}%
\pgfpathlineto{\pgfqpoint{3.141971in}{0.314383in}}%
\pgfpathlineto{\pgfqpoint{3.137676in}{0.314383in}}%
\pgfpathlineto{\pgfqpoint{3.133381in}{0.314383in}}%
\pgfpathlineto{\pgfqpoint{3.129086in}{0.314383in}}%
\pgfpathlineto{\pgfqpoint{3.124791in}{0.314383in}}%
\pgfpathlineto{\pgfqpoint{3.120496in}{0.314383in}}%
\pgfpathlineto{\pgfqpoint{3.116201in}{0.314383in}}%
\pgfpathlineto{\pgfqpoint{3.111906in}{0.314383in}}%
\pgfpathlineto{\pgfqpoint{3.107611in}{0.314383in}}%
\pgfpathlineto{\pgfqpoint{3.103316in}{0.314383in}}%
\pgfpathlineto{\pgfqpoint{3.099021in}{0.314383in}}%
\pgfpathlineto{\pgfqpoint{3.094726in}{0.314383in}}%
\pgfpathlineto{\pgfqpoint{3.090431in}{0.314383in}}%
\pgfpathlineto{\pgfqpoint{3.086136in}{0.314383in}}%
\pgfpathlineto{\pgfqpoint{3.081841in}{0.314383in}}%
\pgfpathlineto{\pgfqpoint{3.077546in}{0.314383in}}%
\pgfpathlineto{\pgfqpoint{3.073250in}{0.314383in}}%
\pgfpathlineto{\pgfqpoint{3.068955in}{0.314383in}}%
\pgfpathlineto{\pgfqpoint{3.064660in}{0.314383in}}%
\pgfpathlineto{\pgfqpoint{3.060365in}{0.314383in}}%
\pgfpathlineto{\pgfqpoint{3.056070in}{0.314383in}}%
\pgfpathlineto{\pgfqpoint{3.051775in}{0.314383in}}%
\pgfpathlineto{\pgfqpoint{3.047480in}{0.314383in}}%
\pgfpathlineto{\pgfqpoint{3.043185in}{0.314383in}}%
\pgfpathlineto{\pgfqpoint{3.038890in}{0.314383in}}%
\pgfpathlineto{\pgfqpoint{3.034595in}{0.314383in}}%
\pgfpathlineto{\pgfqpoint{3.030300in}{0.314383in}}%
\pgfpathlineto{\pgfqpoint{3.026005in}{0.314383in}}%
\pgfpathlineto{\pgfqpoint{3.021710in}{0.314383in}}%
\pgfpathlineto{\pgfqpoint{3.017415in}{0.314383in}}%
\pgfpathlineto{\pgfqpoint{3.013120in}{0.314383in}}%
\pgfpathlineto{\pgfqpoint{3.008825in}{0.314383in}}%
\pgfpathlineto{\pgfqpoint{3.004530in}{0.314383in}}%
\pgfpathlineto{\pgfqpoint{3.000235in}{0.314383in}}%
\pgfpathlineto{\pgfqpoint{2.995940in}{0.314383in}}%
\pgfpathlineto{\pgfqpoint{2.991645in}{0.314383in}}%
\pgfpathlineto{\pgfqpoint{2.987350in}{0.314383in}}%
\pgfpathlineto{\pgfqpoint{2.983055in}{0.314383in}}%
\pgfpathlineto{\pgfqpoint{2.978760in}{0.314383in}}%
\pgfpathlineto{\pgfqpoint{2.974465in}{0.314383in}}%
\pgfpathlineto{\pgfqpoint{2.970170in}{0.314383in}}%
\pgfpathlineto{\pgfqpoint{2.965875in}{0.314383in}}%
\pgfpathlineto{\pgfqpoint{2.961580in}{0.314383in}}%
\pgfpathlineto{\pgfqpoint{2.957285in}{0.314383in}}%
\pgfpathlineto{\pgfqpoint{2.952990in}{0.314383in}}%
\pgfpathlineto{\pgfqpoint{2.948694in}{0.314383in}}%
\pgfpathlineto{\pgfqpoint{2.944399in}{0.314383in}}%
\pgfpathlineto{\pgfqpoint{2.940104in}{0.314383in}}%
\pgfpathlineto{\pgfqpoint{2.935809in}{0.314383in}}%
\pgfpathlineto{\pgfqpoint{2.931514in}{0.314383in}}%
\pgfpathlineto{\pgfqpoint{2.927219in}{0.314383in}}%
\pgfpathlineto{\pgfqpoint{2.922924in}{0.314383in}}%
\pgfpathlineto{\pgfqpoint{2.918629in}{0.314383in}}%
\pgfpathlineto{\pgfqpoint{2.914334in}{0.314383in}}%
\pgfpathlineto{\pgfqpoint{2.910039in}{0.314383in}}%
\pgfpathlineto{\pgfqpoint{2.905744in}{0.314383in}}%
\pgfpathlineto{\pgfqpoint{2.901449in}{0.314383in}}%
\pgfpathlineto{\pgfqpoint{2.897154in}{0.314383in}}%
\pgfpathlineto{\pgfqpoint{2.892859in}{0.314383in}}%
\pgfpathlineto{\pgfqpoint{2.888564in}{0.314383in}}%
\pgfpathlineto{\pgfqpoint{2.884269in}{0.314383in}}%
\pgfpathlineto{\pgfqpoint{2.879974in}{0.314383in}}%
\pgfpathlineto{\pgfqpoint{2.875679in}{0.314383in}}%
\pgfpathlineto{\pgfqpoint{2.871384in}{0.314383in}}%
\pgfpathlineto{\pgfqpoint{2.867089in}{0.314383in}}%
\pgfpathlineto{\pgfqpoint{2.862794in}{0.314383in}}%
\pgfpathlineto{\pgfqpoint{2.858499in}{0.314383in}}%
\pgfpathlineto{\pgfqpoint{2.854204in}{0.314383in}}%
\pgfpathlineto{\pgfqpoint{2.849909in}{0.314383in}}%
\pgfpathlineto{\pgfqpoint{2.845614in}{0.314383in}}%
\pgfpathlineto{\pgfqpoint{2.841319in}{0.314383in}}%
\pgfpathlineto{\pgfqpoint{2.837024in}{0.314383in}}%
\pgfpathlineto{\pgfqpoint{2.832729in}{0.314383in}}%
\pgfpathlineto{\pgfqpoint{2.828434in}{0.314383in}}%
\pgfpathlineto{\pgfqpoint{2.824138in}{0.314383in}}%
\pgfpathlineto{\pgfqpoint{2.819843in}{0.314383in}}%
\pgfpathlineto{\pgfqpoint{2.815548in}{0.314383in}}%
\pgfpathlineto{\pgfqpoint{2.811253in}{0.314383in}}%
\pgfpathlineto{\pgfqpoint{2.806958in}{0.314383in}}%
\pgfpathlineto{\pgfqpoint{2.802663in}{0.314383in}}%
\pgfpathlineto{\pgfqpoint{2.798368in}{0.314383in}}%
\pgfpathlineto{\pgfqpoint{2.794073in}{0.314383in}}%
\pgfpathlineto{\pgfqpoint{2.789778in}{0.314383in}}%
\pgfpathlineto{\pgfqpoint{2.785483in}{0.314383in}}%
\pgfpathlineto{\pgfqpoint{2.781188in}{0.314383in}}%
\pgfpathlineto{\pgfqpoint{2.776893in}{0.314383in}}%
\pgfpathlineto{\pgfqpoint{2.772598in}{0.314383in}}%
\pgfpathlineto{\pgfqpoint{2.768303in}{0.314383in}}%
\pgfpathlineto{\pgfqpoint{2.764008in}{0.314383in}}%
\pgfpathlineto{\pgfqpoint{2.759713in}{0.314383in}}%
\pgfpathlineto{\pgfqpoint{2.755418in}{0.314383in}}%
\pgfpathlineto{\pgfqpoint{2.751123in}{0.314383in}}%
\pgfpathlineto{\pgfqpoint{2.746828in}{0.314383in}}%
\pgfpathlineto{\pgfqpoint{2.742533in}{0.314383in}}%
\pgfpathlineto{\pgfqpoint{2.738238in}{0.314383in}}%
\pgfpathlineto{\pgfqpoint{2.733943in}{0.314383in}}%
\pgfpathlineto{\pgfqpoint{2.729648in}{0.314383in}}%
\pgfpathlineto{\pgfqpoint{2.725353in}{0.314383in}}%
\pgfpathlineto{\pgfqpoint{2.721058in}{0.314383in}}%
\pgfpathlineto{\pgfqpoint{2.716763in}{0.314383in}}%
\pgfpathlineto{\pgfqpoint{2.712468in}{0.314383in}}%
\pgfpathlineto{\pgfqpoint{2.708173in}{0.314383in}}%
\pgfpathlineto{\pgfqpoint{2.703878in}{0.314383in}}%
\pgfpathlineto{\pgfqpoint{2.699582in}{0.314383in}}%
\pgfpathlineto{\pgfqpoint{2.695287in}{0.314383in}}%
\pgfpathlineto{\pgfqpoint{2.690992in}{0.314383in}}%
\pgfpathlineto{\pgfqpoint{2.686697in}{0.314383in}}%
\pgfpathlineto{\pgfqpoint{2.682402in}{0.314383in}}%
\pgfpathlineto{\pgfqpoint{2.678107in}{0.314383in}}%
\pgfpathlineto{\pgfqpoint{2.673812in}{0.314383in}}%
\pgfpathlineto{\pgfqpoint{2.669517in}{0.314383in}}%
\pgfpathlineto{\pgfqpoint{2.665222in}{0.314383in}}%
\pgfpathlineto{\pgfqpoint{2.660927in}{0.314383in}}%
\pgfpathlineto{\pgfqpoint{2.656632in}{0.314383in}}%
\pgfpathlineto{\pgfqpoint{2.652337in}{0.314383in}}%
\pgfpathlineto{\pgfqpoint{2.648042in}{0.314383in}}%
\pgfpathlineto{\pgfqpoint{2.643747in}{0.314383in}}%
\pgfpathlineto{\pgfqpoint{2.639452in}{0.314383in}}%
\pgfpathlineto{\pgfqpoint{2.635157in}{0.314383in}}%
\pgfpathlineto{\pgfqpoint{2.630862in}{0.314383in}}%
\pgfpathlineto{\pgfqpoint{2.626567in}{0.314383in}}%
\pgfpathlineto{\pgfqpoint{2.622272in}{0.314383in}}%
\pgfpathlineto{\pgfqpoint{2.617977in}{0.314383in}}%
\pgfpathlineto{\pgfqpoint{2.613682in}{0.314383in}}%
\pgfpathlineto{\pgfqpoint{2.609387in}{0.314383in}}%
\pgfpathlineto{\pgfqpoint{2.605092in}{0.314383in}}%
\pgfpathlineto{\pgfqpoint{2.600797in}{0.314383in}}%
\pgfpathlineto{\pgfqpoint{2.596502in}{0.314383in}}%
\pgfpathlineto{\pgfqpoint{2.592207in}{0.314383in}}%
\pgfpathlineto{\pgfqpoint{2.587912in}{0.314383in}}%
\pgfpathlineto{\pgfqpoint{2.583617in}{0.314383in}}%
\pgfpathlineto{\pgfqpoint{2.579322in}{0.314383in}}%
\pgfpathlineto{\pgfqpoint{2.575026in}{0.314383in}}%
\pgfpathlineto{\pgfqpoint{2.570731in}{0.314383in}}%
\pgfpathlineto{\pgfqpoint{2.566436in}{0.314383in}}%
\pgfpathlineto{\pgfqpoint{2.562141in}{0.314383in}}%
\pgfpathlineto{\pgfqpoint{2.557846in}{0.314383in}}%
\pgfpathlineto{\pgfqpoint{2.553551in}{0.314383in}}%
\pgfpathlineto{\pgfqpoint{2.549256in}{0.314383in}}%
\pgfpathlineto{\pgfqpoint{2.544961in}{0.314383in}}%
\pgfpathlineto{\pgfqpoint{2.540666in}{0.314383in}}%
\pgfpathlineto{\pgfqpoint{2.536371in}{0.314383in}}%
\pgfpathlineto{\pgfqpoint{2.532076in}{0.314383in}}%
\pgfpathlineto{\pgfqpoint{2.527781in}{0.314383in}}%
\pgfpathlineto{\pgfqpoint{2.523486in}{0.314383in}}%
\pgfpathlineto{\pgfqpoint{2.519191in}{0.314383in}}%
\pgfpathlineto{\pgfqpoint{2.514896in}{0.314383in}}%
\pgfpathlineto{\pgfqpoint{2.510601in}{0.314383in}}%
\pgfpathlineto{\pgfqpoint{2.506306in}{0.314383in}}%
\pgfpathlineto{\pgfqpoint{2.502011in}{0.314383in}}%
\pgfpathlineto{\pgfqpoint{2.497716in}{0.314383in}}%
\pgfpathlineto{\pgfqpoint{2.493421in}{0.314383in}}%
\pgfpathlineto{\pgfqpoint{2.489126in}{0.314383in}}%
\pgfpathlineto{\pgfqpoint{2.484831in}{0.314383in}}%
\pgfpathlineto{\pgfqpoint{2.480536in}{0.314383in}}%
\pgfpathlineto{\pgfqpoint{2.476241in}{0.314383in}}%
\pgfpathlineto{\pgfqpoint{2.471946in}{0.314383in}}%
\pgfpathlineto{\pgfqpoint{2.467651in}{0.314383in}}%
\pgfpathlineto{\pgfqpoint{2.463356in}{0.314383in}}%
\pgfpathlineto{\pgfqpoint{2.459061in}{0.314383in}}%
\pgfpathlineto{\pgfqpoint{2.454766in}{0.314383in}}%
\pgfpathlineto{\pgfqpoint{2.450470in}{0.314383in}}%
\pgfpathlineto{\pgfqpoint{2.446175in}{0.314383in}}%
\pgfpathlineto{\pgfqpoint{2.441880in}{0.314383in}}%
\pgfpathlineto{\pgfqpoint{2.437585in}{0.314383in}}%
\pgfpathlineto{\pgfqpoint{2.433290in}{0.314383in}}%
\pgfpathlineto{\pgfqpoint{2.428995in}{0.314383in}}%
\pgfpathlineto{\pgfqpoint{2.424700in}{0.314383in}}%
\pgfpathlineto{\pgfqpoint{2.420405in}{0.314383in}}%
\pgfpathlineto{\pgfqpoint{2.416110in}{0.314383in}}%
\pgfpathlineto{\pgfqpoint{2.411815in}{0.314383in}}%
\pgfpathlineto{\pgfqpoint{2.407520in}{0.314383in}}%
\pgfpathlineto{\pgfqpoint{2.403225in}{0.314383in}}%
\pgfpathlineto{\pgfqpoint{2.398930in}{0.314383in}}%
\pgfpathlineto{\pgfqpoint{2.394635in}{0.314383in}}%
\pgfpathlineto{\pgfqpoint{2.390340in}{0.314383in}}%
\pgfpathlineto{\pgfqpoint{2.386045in}{0.314383in}}%
\pgfpathlineto{\pgfqpoint{2.381750in}{0.314383in}}%
\pgfpathlineto{\pgfqpoint{2.377455in}{0.314383in}}%
\pgfpathlineto{\pgfqpoint{2.373160in}{0.314383in}}%
\pgfpathlineto{\pgfqpoint{2.368865in}{0.314383in}}%
\pgfpathlineto{\pgfqpoint{2.364570in}{0.314383in}}%
\pgfpathlineto{\pgfqpoint{2.360275in}{0.314383in}}%
\pgfpathlineto{\pgfqpoint{2.355980in}{0.314383in}}%
\pgfpathlineto{\pgfqpoint{2.351685in}{0.314383in}}%
\pgfpathlineto{\pgfqpoint{2.347390in}{0.314383in}}%
\pgfpathlineto{\pgfqpoint{2.343095in}{0.314383in}}%
\pgfpathlineto{\pgfqpoint{2.338800in}{0.314383in}}%
\pgfpathlineto{\pgfqpoint{2.334505in}{0.314383in}}%
\pgfpathlineto{\pgfqpoint{2.330210in}{0.314383in}}%
\pgfpathlineto{\pgfqpoint{2.325914in}{0.314383in}}%
\pgfpathlineto{\pgfqpoint{2.321619in}{0.314383in}}%
\pgfpathlineto{\pgfqpoint{2.317324in}{0.314383in}}%
\pgfpathlineto{\pgfqpoint{2.313029in}{0.314383in}}%
\pgfpathlineto{\pgfqpoint{2.308734in}{0.314383in}}%
\pgfpathlineto{\pgfqpoint{2.304439in}{0.314383in}}%
\pgfpathlineto{\pgfqpoint{2.300144in}{0.314383in}}%
\pgfpathlineto{\pgfqpoint{2.295849in}{0.314383in}}%
\pgfpathlineto{\pgfqpoint{2.291554in}{0.314383in}}%
\pgfpathlineto{\pgfqpoint{2.287259in}{0.314383in}}%
\pgfpathlineto{\pgfqpoint{2.282964in}{0.314383in}}%
\pgfpathlineto{\pgfqpoint{2.278669in}{0.314383in}}%
\pgfpathlineto{\pgfqpoint{2.274374in}{0.314383in}}%
\pgfpathlineto{\pgfqpoint{2.270079in}{0.314383in}}%
\pgfpathlineto{\pgfqpoint{2.265784in}{0.314383in}}%
\pgfpathlineto{\pgfqpoint{2.261489in}{0.314383in}}%
\pgfpathlineto{\pgfqpoint{2.257194in}{0.314383in}}%
\pgfpathlineto{\pgfqpoint{2.252899in}{0.314383in}}%
\pgfpathlineto{\pgfqpoint{2.248604in}{0.314383in}}%
\pgfpathlineto{\pgfqpoint{2.244309in}{0.314383in}}%
\pgfpathlineto{\pgfqpoint{2.240014in}{0.314383in}}%
\pgfpathlineto{\pgfqpoint{2.235719in}{0.314383in}}%
\pgfpathlineto{\pgfqpoint{2.231424in}{0.314383in}}%
\pgfpathlineto{\pgfqpoint{2.227129in}{0.314383in}}%
\pgfpathlineto{\pgfqpoint{2.222834in}{0.314383in}}%
\pgfpathlineto{\pgfqpoint{2.218539in}{0.314383in}}%
\pgfpathlineto{\pgfqpoint{2.214244in}{0.314383in}}%
\pgfpathlineto{\pgfqpoint{2.209949in}{0.314383in}}%
\pgfpathlineto{\pgfqpoint{2.205654in}{0.314383in}}%
\pgfpathlineto{\pgfqpoint{2.201358in}{0.314383in}}%
\pgfpathlineto{\pgfqpoint{2.197063in}{0.314383in}}%
\pgfpathlineto{\pgfqpoint{2.192768in}{0.314383in}}%
\pgfpathlineto{\pgfqpoint{2.188473in}{0.314383in}}%
\pgfpathlineto{\pgfqpoint{2.184178in}{0.314383in}}%
\pgfpathlineto{\pgfqpoint{2.179883in}{0.314383in}}%
\pgfpathlineto{\pgfqpoint{2.175588in}{0.314383in}}%
\pgfpathlineto{\pgfqpoint{2.171293in}{0.314383in}}%
\pgfpathlineto{\pgfqpoint{2.166998in}{0.314383in}}%
\pgfpathlineto{\pgfqpoint{2.162703in}{0.314383in}}%
\pgfpathlineto{\pgfqpoint{2.158408in}{0.314383in}}%
\pgfpathlineto{\pgfqpoint{2.154113in}{0.314383in}}%
\pgfpathlineto{\pgfqpoint{2.149818in}{0.314383in}}%
\pgfpathlineto{\pgfqpoint{2.145523in}{0.314383in}}%
\pgfpathlineto{\pgfqpoint{2.141228in}{0.314383in}}%
\pgfpathlineto{\pgfqpoint{2.136933in}{0.314383in}}%
\pgfpathlineto{\pgfqpoint{2.132638in}{0.314383in}}%
\pgfpathlineto{\pgfqpoint{2.128343in}{0.314383in}}%
\pgfpathlineto{\pgfqpoint{2.124048in}{0.314383in}}%
\pgfpathlineto{\pgfqpoint{2.119753in}{0.314383in}}%
\pgfpathlineto{\pgfqpoint{2.115458in}{0.314383in}}%
\pgfpathlineto{\pgfqpoint{2.111163in}{0.314383in}}%
\pgfpathlineto{\pgfqpoint{2.106868in}{0.314383in}}%
\pgfpathlineto{\pgfqpoint{2.102573in}{0.314383in}}%
\pgfpathlineto{\pgfqpoint{2.098278in}{0.314383in}}%
\pgfpathlineto{\pgfqpoint{2.093983in}{0.314383in}}%
\pgfpathlineto{\pgfqpoint{2.089688in}{0.314383in}}%
\pgfpathlineto{\pgfqpoint{2.085393in}{0.314383in}}%
\pgfpathlineto{\pgfqpoint{2.081098in}{0.314383in}}%
\pgfpathlineto{\pgfqpoint{2.076802in}{0.314383in}}%
\pgfpathlineto{\pgfqpoint{2.072507in}{0.314383in}}%
\pgfpathlineto{\pgfqpoint{2.068212in}{0.314383in}}%
\pgfpathlineto{\pgfqpoint{2.063917in}{0.314383in}}%
\pgfpathlineto{\pgfqpoint{2.059622in}{0.314383in}}%
\pgfpathlineto{\pgfqpoint{2.055327in}{0.314383in}}%
\pgfpathlineto{\pgfqpoint{2.051032in}{0.314383in}}%
\pgfpathlineto{\pgfqpoint{2.046737in}{0.314383in}}%
\pgfpathlineto{\pgfqpoint{2.042442in}{0.314383in}}%
\pgfpathlineto{\pgfqpoint{2.038147in}{0.314383in}}%
\pgfpathlineto{\pgfqpoint{2.033852in}{0.314383in}}%
\pgfpathlineto{\pgfqpoint{2.029557in}{0.314383in}}%
\pgfpathlineto{\pgfqpoint{2.025262in}{0.314383in}}%
\pgfpathlineto{\pgfqpoint{2.020967in}{0.314383in}}%
\pgfpathlineto{\pgfqpoint{2.016672in}{0.314383in}}%
\pgfpathlineto{\pgfqpoint{2.012377in}{0.314383in}}%
\pgfpathlineto{\pgfqpoint{2.008082in}{0.314383in}}%
\pgfpathlineto{\pgfqpoint{2.003787in}{0.314383in}}%
\pgfpathlineto{\pgfqpoint{1.999492in}{0.314383in}}%
\pgfpathlineto{\pgfqpoint{1.995197in}{0.314383in}}%
\pgfpathlineto{\pgfqpoint{1.990902in}{0.314383in}}%
\pgfpathlineto{\pgfqpoint{1.986607in}{0.314383in}}%
\pgfpathlineto{\pgfqpoint{1.982312in}{0.314383in}}%
\pgfpathlineto{\pgfqpoint{1.978017in}{0.314383in}}%
\pgfpathlineto{\pgfqpoint{1.973722in}{0.314383in}}%
\pgfpathlineto{\pgfqpoint{1.969427in}{0.314383in}}%
\pgfpathlineto{\pgfqpoint{1.965132in}{0.314383in}}%
\pgfpathlineto{\pgfqpoint{1.960837in}{0.314383in}}%
\pgfpathlineto{\pgfqpoint{1.956542in}{0.314383in}}%
\pgfpathlineto{\pgfqpoint{1.952246in}{0.314383in}}%
\pgfpathlineto{\pgfqpoint{1.947951in}{0.314383in}}%
\pgfpathlineto{\pgfqpoint{1.943656in}{0.314383in}}%
\pgfpathlineto{\pgfqpoint{1.939361in}{0.314383in}}%
\pgfpathlineto{\pgfqpoint{1.935066in}{0.314383in}}%
\pgfpathlineto{\pgfqpoint{1.930771in}{0.314383in}}%
\pgfpathlineto{\pgfqpoint{1.926476in}{0.314383in}}%
\pgfpathlineto{\pgfqpoint{1.922181in}{0.314383in}}%
\pgfpathlineto{\pgfqpoint{1.917886in}{0.314383in}}%
\pgfpathlineto{\pgfqpoint{1.913591in}{0.314383in}}%
\pgfpathlineto{\pgfqpoint{1.909296in}{0.314383in}}%
\pgfpathlineto{\pgfqpoint{1.905001in}{0.314383in}}%
\pgfpathlineto{\pgfqpoint{1.900706in}{0.314383in}}%
\pgfpathlineto{\pgfqpoint{1.896411in}{0.314383in}}%
\pgfpathlineto{\pgfqpoint{1.892116in}{0.314383in}}%
\pgfpathlineto{\pgfqpoint{1.887821in}{0.314383in}}%
\pgfpathlineto{\pgfqpoint{1.883526in}{0.314383in}}%
\pgfpathlineto{\pgfqpoint{1.879231in}{0.314383in}}%
\pgfpathlineto{\pgfqpoint{1.874936in}{0.314383in}}%
\pgfpathlineto{\pgfqpoint{1.870641in}{0.314383in}}%
\pgfpathlineto{\pgfqpoint{1.866346in}{0.314383in}}%
\pgfpathlineto{\pgfqpoint{1.862051in}{0.314383in}}%
\pgfpathlineto{\pgfqpoint{1.857756in}{0.314383in}}%
\pgfpathlineto{\pgfqpoint{1.853461in}{0.314383in}}%
\pgfpathlineto{\pgfqpoint{1.849166in}{0.314383in}}%
\pgfpathlineto{\pgfqpoint{1.844871in}{0.314383in}}%
\pgfpathlineto{\pgfqpoint{1.840576in}{0.314383in}}%
\pgfpathlineto{\pgfqpoint{1.836281in}{0.314383in}}%
\pgfpathlineto{\pgfqpoint{1.831986in}{0.314383in}}%
\pgfpathlineto{\pgfqpoint{1.827690in}{0.314383in}}%
\pgfpathlineto{\pgfqpoint{1.823395in}{0.314383in}}%
\pgfpathlineto{\pgfqpoint{1.819100in}{0.314383in}}%
\pgfpathlineto{\pgfqpoint{1.814805in}{0.314383in}}%
\pgfpathlineto{\pgfqpoint{1.810510in}{0.314383in}}%
\pgfpathlineto{\pgfqpoint{1.806215in}{0.314383in}}%
\pgfpathlineto{\pgfqpoint{1.801920in}{0.314383in}}%
\pgfpathlineto{\pgfqpoint{1.797625in}{0.314383in}}%
\pgfpathlineto{\pgfqpoint{1.793330in}{0.314383in}}%
\pgfpathlineto{\pgfqpoint{1.789035in}{0.314383in}}%
\pgfpathlineto{\pgfqpoint{1.784740in}{0.314383in}}%
\pgfpathlineto{\pgfqpoint{1.780445in}{0.314383in}}%
\pgfpathlineto{\pgfqpoint{1.776150in}{0.314383in}}%
\pgfpathlineto{\pgfqpoint{1.771855in}{0.314383in}}%
\pgfpathlineto{\pgfqpoint{1.767560in}{0.314383in}}%
\pgfpathlineto{\pgfqpoint{1.763265in}{0.314383in}}%
\pgfpathlineto{\pgfqpoint{1.758970in}{0.314383in}}%
\pgfpathlineto{\pgfqpoint{1.754675in}{0.314383in}}%
\pgfpathlineto{\pgfqpoint{1.750380in}{0.314383in}}%
\pgfpathlineto{\pgfqpoint{1.746085in}{0.314383in}}%
\pgfpathlineto{\pgfqpoint{1.741790in}{0.314383in}}%
\pgfpathlineto{\pgfqpoint{1.737495in}{0.314383in}}%
\pgfpathlineto{\pgfqpoint{1.733200in}{0.314383in}}%
\pgfpathlineto{\pgfqpoint{1.728905in}{0.314383in}}%
\pgfpathlineto{\pgfqpoint{1.724610in}{0.314383in}}%
\pgfpathlineto{\pgfqpoint{1.720315in}{0.314383in}}%
\pgfpathlineto{\pgfqpoint{1.716020in}{0.314383in}}%
\pgfpathlineto{\pgfqpoint{1.711725in}{0.314383in}}%
\pgfpathlineto{\pgfqpoint{1.707430in}{0.314383in}}%
\pgfpathlineto{\pgfqpoint{1.703134in}{0.314383in}}%
\pgfpathlineto{\pgfqpoint{1.698839in}{0.314383in}}%
\pgfpathlineto{\pgfqpoint{1.694544in}{0.314383in}}%
\pgfpathlineto{\pgfqpoint{1.690249in}{0.314383in}}%
\pgfpathlineto{\pgfqpoint{1.685954in}{0.314383in}}%
\pgfpathlineto{\pgfqpoint{1.681659in}{0.314383in}}%
\pgfpathlineto{\pgfqpoint{1.677364in}{0.314383in}}%
\pgfpathlineto{\pgfqpoint{1.673069in}{0.314383in}}%
\pgfpathlineto{\pgfqpoint{1.668774in}{0.314383in}}%
\pgfpathlineto{\pgfqpoint{1.664479in}{0.314383in}}%
\pgfpathlineto{\pgfqpoint{1.660184in}{0.314383in}}%
\pgfpathlineto{\pgfqpoint{1.655889in}{0.314383in}}%
\pgfpathlineto{\pgfqpoint{1.651594in}{0.314383in}}%
\pgfpathlineto{\pgfqpoint{1.647299in}{0.314383in}}%
\pgfpathlineto{\pgfqpoint{1.643004in}{0.314383in}}%
\pgfpathlineto{\pgfqpoint{1.638709in}{0.314383in}}%
\pgfpathlineto{\pgfqpoint{1.634414in}{0.314383in}}%
\pgfpathlineto{\pgfqpoint{1.630119in}{0.314383in}}%
\pgfpathlineto{\pgfqpoint{1.625824in}{0.314383in}}%
\pgfpathlineto{\pgfqpoint{1.621529in}{0.314383in}}%
\pgfpathlineto{\pgfqpoint{1.617234in}{0.314383in}}%
\pgfpathlineto{\pgfqpoint{1.612939in}{0.314383in}}%
\pgfpathlineto{\pgfqpoint{1.608644in}{0.314383in}}%
\pgfpathlineto{\pgfqpoint{1.604349in}{0.314383in}}%
\pgfpathlineto{\pgfqpoint{1.600054in}{0.314383in}}%
\pgfpathlineto{\pgfqpoint{1.595759in}{0.314383in}}%
\pgfpathlineto{\pgfqpoint{1.591464in}{0.314383in}}%
\pgfpathlineto{\pgfqpoint{1.587169in}{0.314383in}}%
\pgfpathlineto{\pgfqpoint{1.582874in}{0.314383in}}%
\pgfpathlineto{\pgfqpoint{1.578578in}{0.314383in}}%
\pgfpathlineto{\pgfqpoint{1.574283in}{0.314383in}}%
\pgfpathlineto{\pgfqpoint{1.569988in}{0.314383in}}%
\pgfpathlineto{\pgfqpoint{1.565693in}{0.314383in}}%
\pgfpathlineto{\pgfqpoint{1.561398in}{0.314383in}}%
\pgfpathlineto{\pgfqpoint{1.557103in}{0.314383in}}%
\pgfpathlineto{\pgfqpoint{1.552808in}{0.314383in}}%
\pgfpathlineto{\pgfqpoint{1.548513in}{0.314383in}}%
\pgfpathlineto{\pgfqpoint{1.544218in}{0.314383in}}%
\pgfpathlineto{\pgfqpoint{1.539923in}{0.314383in}}%
\pgfpathlineto{\pgfqpoint{1.535628in}{0.314383in}}%
\pgfpathlineto{\pgfqpoint{1.531333in}{0.314383in}}%
\pgfpathlineto{\pgfqpoint{1.527038in}{0.314383in}}%
\pgfpathlineto{\pgfqpoint{1.522743in}{0.314383in}}%
\pgfpathlineto{\pgfqpoint{1.518448in}{0.314383in}}%
\pgfpathlineto{\pgfqpoint{1.514153in}{0.314383in}}%
\pgfpathlineto{\pgfqpoint{1.509858in}{0.314383in}}%
\pgfpathlineto{\pgfqpoint{1.505563in}{0.314383in}}%
\pgfpathlineto{\pgfqpoint{1.501268in}{0.314383in}}%
\pgfpathlineto{\pgfqpoint{1.496973in}{0.314383in}}%
\pgfpathlineto{\pgfqpoint{1.492678in}{0.314383in}}%
\pgfpathlineto{\pgfqpoint{1.488383in}{0.314383in}}%
\pgfpathlineto{\pgfqpoint{1.484088in}{0.314383in}}%
\pgfpathlineto{\pgfqpoint{1.479793in}{0.314383in}}%
\pgfpathlineto{\pgfqpoint{1.475498in}{0.314383in}}%
\pgfpathlineto{\pgfqpoint{1.471203in}{0.314383in}}%
\pgfpathlineto{\pgfqpoint{1.466908in}{0.314383in}}%
\pgfpathlineto{\pgfqpoint{1.462613in}{0.314383in}}%
\pgfpathlineto{\pgfqpoint{1.458318in}{0.314383in}}%
\pgfpathlineto{\pgfqpoint{1.454022in}{0.314383in}}%
\pgfpathlineto{\pgfqpoint{1.449727in}{0.314383in}}%
\pgfpathlineto{\pgfqpoint{1.445432in}{0.314383in}}%
\pgfpathlineto{\pgfqpoint{1.441137in}{0.314383in}}%
\pgfpathlineto{\pgfqpoint{1.436842in}{0.314383in}}%
\pgfpathlineto{\pgfqpoint{1.432547in}{0.314383in}}%
\pgfpathlineto{\pgfqpoint{1.428252in}{0.314383in}}%
\pgfpathlineto{\pgfqpoint{1.423957in}{0.314383in}}%
\pgfpathlineto{\pgfqpoint{1.419662in}{0.314383in}}%
\pgfpathlineto{\pgfqpoint{1.415367in}{0.314383in}}%
\pgfpathlineto{\pgfqpoint{1.411072in}{0.314383in}}%
\pgfpathlineto{\pgfqpoint{1.406777in}{0.314383in}}%
\pgfpathlineto{\pgfqpoint{1.402482in}{0.314383in}}%
\pgfpathlineto{\pgfqpoint{1.398187in}{0.314383in}}%
\pgfpathlineto{\pgfqpoint{1.393892in}{0.314383in}}%
\pgfpathlineto{\pgfqpoint{1.389597in}{0.314383in}}%
\pgfpathlineto{\pgfqpoint{1.385302in}{0.314383in}}%
\pgfpathlineto{\pgfqpoint{1.381007in}{0.314383in}}%
\pgfpathlineto{\pgfqpoint{1.376712in}{0.314383in}}%
\pgfpathlineto{\pgfqpoint{1.372417in}{0.314383in}}%
\pgfpathlineto{\pgfqpoint{1.368122in}{0.314383in}}%
\pgfpathlineto{\pgfqpoint{1.363827in}{0.314383in}}%
\pgfpathlineto{\pgfqpoint{1.359532in}{0.314383in}}%
\pgfpathlineto{\pgfqpoint{1.355237in}{0.314383in}}%
\pgfpathlineto{\pgfqpoint{1.350942in}{0.314383in}}%
\pgfpathlineto{\pgfqpoint{1.346647in}{0.314383in}}%
\pgfpathlineto{\pgfqpoint{1.342352in}{0.314383in}}%
\pgfpathlineto{\pgfqpoint{1.338057in}{0.314383in}}%
\pgfpathlineto{\pgfqpoint{1.333762in}{0.314383in}}%
\pgfpathlineto{\pgfqpoint{1.329466in}{0.314383in}}%
\pgfpathlineto{\pgfqpoint{1.325171in}{0.314383in}}%
\pgfpathlineto{\pgfqpoint{1.320876in}{0.314383in}}%
\pgfpathlineto{\pgfqpoint{1.316581in}{0.314383in}}%
\pgfpathlineto{\pgfqpoint{1.312286in}{0.314383in}}%
\pgfpathlineto{\pgfqpoint{1.307991in}{0.314383in}}%
\pgfpathlineto{\pgfqpoint{1.303696in}{0.314383in}}%
\pgfpathlineto{\pgfqpoint{1.299401in}{0.314383in}}%
\pgfpathlineto{\pgfqpoint{1.295106in}{0.314383in}}%
\pgfpathlineto{\pgfqpoint{1.290811in}{0.314383in}}%
\pgfpathlineto{\pgfqpoint{1.286516in}{0.314383in}}%
\pgfpathlineto{\pgfqpoint{1.282221in}{0.314383in}}%
\pgfpathlineto{\pgfqpoint{1.277926in}{0.314383in}}%
\pgfpathlineto{\pgfqpoint{1.273631in}{0.314383in}}%
\pgfpathlineto{\pgfqpoint{1.269336in}{0.314383in}}%
\pgfpathlineto{\pgfqpoint{1.265041in}{0.314383in}}%
\pgfpathlineto{\pgfqpoint{1.260746in}{0.314383in}}%
\pgfpathlineto{\pgfqpoint{1.256451in}{0.314383in}}%
\pgfpathlineto{\pgfqpoint{1.252156in}{0.314383in}}%
\pgfpathlineto{\pgfqpoint{1.247861in}{0.314383in}}%
\pgfpathlineto{\pgfqpoint{1.243566in}{0.314383in}}%
\pgfpathlineto{\pgfqpoint{1.239271in}{0.314383in}}%
\pgfpathlineto{\pgfqpoint{1.234976in}{0.314383in}}%
\pgfpathlineto{\pgfqpoint{1.230681in}{0.314383in}}%
\pgfpathlineto{\pgfqpoint{1.226386in}{0.314383in}}%
\pgfpathlineto{\pgfqpoint{1.222091in}{0.314383in}}%
\pgfpathlineto{\pgfqpoint{1.217796in}{0.314383in}}%
\pgfpathlineto{\pgfqpoint{1.213501in}{0.314383in}}%
\pgfpathlineto{\pgfqpoint{1.209206in}{0.314383in}}%
\pgfpathlineto{\pgfqpoint{1.204910in}{0.314383in}}%
\pgfpathlineto{\pgfqpoint{1.200615in}{0.314383in}}%
\pgfpathlineto{\pgfqpoint{1.196320in}{0.314383in}}%
\pgfpathlineto{\pgfqpoint{1.192025in}{0.314383in}}%
\pgfpathlineto{\pgfqpoint{1.187730in}{0.314383in}}%
\pgfpathlineto{\pgfqpoint{1.183435in}{0.314383in}}%
\pgfpathlineto{\pgfqpoint{1.179140in}{0.314383in}}%
\pgfpathlineto{\pgfqpoint{1.174845in}{0.314383in}}%
\pgfpathlineto{\pgfqpoint{1.170550in}{0.314383in}}%
\pgfpathlineto{\pgfqpoint{1.166255in}{0.314383in}}%
\pgfpathlineto{\pgfqpoint{1.161960in}{0.314383in}}%
\pgfpathlineto{\pgfqpoint{1.157665in}{0.314383in}}%
\pgfpathlineto{\pgfqpoint{1.153370in}{0.314383in}}%
\pgfpathlineto{\pgfqpoint{1.149075in}{0.314383in}}%
\pgfpathlineto{\pgfqpoint{1.144780in}{0.314383in}}%
\pgfpathlineto{\pgfqpoint{1.140485in}{0.314383in}}%
\pgfpathlineto{\pgfqpoint{1.136190in}{0.314383in}}%
\pgfpathlineto{\pgfqpoint{1.131895in}{0.314383in}}%
\pgfpathlineto{\pgfqpoint{1.127600in}{0.314383in}}%
\pgfpathlineto{\pgfqpoint{1.123305in}{0.314383in}}%
\pgfpathlineto{\pgfqpoint{1.119010in}{0.314383in}}%
\pgfpathlineto{\pgfqpoint{1.114715in}{0.314383in}}%
\pgfpathlineto{\pgfqpoint{1.110420in}{0.314383in}}%
\pgfpathlineto{\pgfqpoint{1.106125in}{0.314383in}}%
\pgfpathlineto{\pgfqpoint{1.101830in}{0.314383in}}%
\pgfpathlineto{\pgfqpoint{1.097535in}{0.314383in}}%
\pgfpathlineto{\pgfqpoint{1.093240in}{0.314383in}}%
\pgfpathlineto{\pgfqpoint{1.088945in}{0.314383in}}%
\pgfpathlineto{\pgfqpoint{1.084650in}{0.314383in}}%
\pgfpathlineto{\pgfqpoint{1.080354in}{0.314383in}}%
\pgfpathlineto{\pgfqpoint{1.076059in}{0.314383in}}%
\pgfpathlineto{\pgfqpoint{1.071764in}{0.314383in}}%
\pgfpathlineto{\pgfqpoint{1.067469in}{0.314383in}}%
\pgfpathlineto{\pgfqpoint{1.063174in}{0.314383in}}%
\pgfpathlineto{\pgfqpoint{1.058879in}{0.314383in}}%
\pgfpathlineto{\pgfqpoint{1.054584in}{0.314383in}}%
\pgfpathlineto{\pgfqpoint{1.050289in}{0.314383in}}%
\pgfpathlineto{\pgfqpoint{1.045994in}{0.314383in}}%
\pgfpathlineto{\pgfqpoint{1.041699in}{0.314383in}}%
\pgfpathlineto{\pgfqpoint{1.037404in}{0.314383in}}%
\pgfpathlineto{\pgfqpoint{1.033109in}{0.314383in}}%
\pgfpathlineto{\pgfqpoint{1.028814in}{0.314383in}}%
\pgfpathlineto{\pgfqpoint{1.024519in}{0.314383in}}%
\pgfpathlineto{\pgfqpoint{1.020224in}{0.314383in}}%
\pgfpathlineto{\pgfqpoint{1.015929in}{0.314383in}}%
\pgfpathlineto{\pgfqpoint{1.011634in}{0.314383in}}%
\pgfpathlineto{\pgfqpoint{1.007339in}{0.314383in}}%
\pgfpathlineto{\pgfqpoint{1.003044in}{0.314383in}}%
\pgfpathlineto{\pgfqpoint{0.998749in}{0.314383in}}%
\pgfpathlineto{\pgfqpoint{0.994454in}{0.314383in}}%
\pgfpathlineto{\pgfqpoint{0.990159in}{0.314383in}}%
\pgfpathlineto{\pgfqpoint{0.985864in}{0.314383in}}%
\pgfpathlineto{\pgfqpoint{0.981569in}{0.314383in}}%
\pgfpathlineto{\pgfqpoint{0.977274in}{0.314383in}}%
\pgfpathlineto{\pgfqpoint{0.972979in}{0.314383in}}%
\pgfpathlineto{\pgfqpoint{0.968684in}{0.314383in}}%
\pgfpathlineto{\pgfqpoint{0.964389in}{0.314383in}}%
\pgfpathlineto{\pgfqpoint{0.960094in}{0.314383in}}%
\pgfpathlineto{\pgfqpoint{0.955798in}{0.314383in}}%
\pgfpathlineto{\pgfqpoint{0.951503in}{0.314383in}}%
\pgfpathlineto{\pgfqpoint{0.947208in}{0.314383in}}%
\pgfpathlineto{\pgfqpoint{0.942913in}{0.314383in}}%
\pgfpathlineto{\pgfqpoint{0.938618in}{0.314383in}}%
\pgfpathlineto{\pgfqpoint{0.934323in}{0.314383in}}%
\pgfpathlineto{\pgfqpoint{0.930028in}{0.314383in}}%
\pgfpathlineto{\pgfqpoint{0.925733in}{0.314383in}}%
\pgfpathlineto{\pgfqpoint{0.921438in}{0.314383in}}%
\pgfpathlineto{\pgfqpoint{0.917143in}{0.314383in}}%
\pgfpathlineto{\pgfqpoint{0.912848in}{0.314383in}}%
\pgfpathlineto{\pgfqpoint{0.908553in}{0.314383in}}%
\pgfpathlineto{\pgfqpoint{0.904258in}{0.314383in}}%
\pgfpathlineto{\pgfqpoint{0.899963in}{0.314383in}}%
\pgfpathlineto{\pgfqpoint{0.895668in}{0.314383in}}%
\pgfpathlineto{\pgfqpoint{0.891373in}{0.314383in}}%
\pgfpathlineto{\pgfqpoint{0.887078in}{0.314383in}}%
\pgfpathlineto{\pgfqpoint{0.882783in}{0.314383in}}%
\pgfpathlineto{\pgfqpoint{0.878488in}{0.314383in}}%
\pgfpathlineto{\pgfqpoint{0.874193in}{0.314383in}}%
\pgfpathlineto{\pgfqpoint{0.874193in}{0.314383in}}%
\pgfpathclose%
\pgfusepath{stroke,fill}%
}%
\begin{pgfscope}%
\pgfsys@transformshift{0.000000in}{0.000000in}%
\pgfsys@useobject{currentmarker}{}%
\end{pgfscope}%
\end{pgfscope}%
\begin{pgfscope}%
\pgfpathrectangle{\pgfqpoint{0.882794in}{0.589583in}}{\pgfqpoint{6.200000in}{4.620000in}}%
\pgfusepath{clip}%
\pgfsetbuttcap%
\pgfsetmiterjoin%
\definecolor{currentfill}{rgb}{0.121569,0.466667,0.705882}%
\pgfsetfillcolor{currentfill}%
\pgfsetfillopacity{0.200000}%
\pgfsetlinewidth{0.000000pt}%
\definecolor{currentstroke}{rgb}{0.000000,0.000000,0.000000}%
\pgfsetstrokecolor{currentstroke}%
\pgfsetstrokeopacity{0.200000}%
\pgfsetdash{}{0pt}%
\pgfpathmoveto{\pgfqpoint{0.882794in}{3.500183in}}%
\pgfpathlineto{\pgfqpoint{0.882794in}{5.209583in}}%
\pgfpathlineto{\pgfqpoint{0.968809in}{5.209583in}}%
\pgfpathlineto{\pgfqpoint{0.968809in}{3.500183in}}%
\pgfpathlineto{\pgfqpoint{0.882794in}{3.500183in}}%
\pgfpathclose%
\pgfusepath{fill}%
\end{pgfscope}%
\begin{pgfscope}%
\pgfsetbuttcap%
\pgfsetmiterjoin%
\definecolor{currentfill}{rgb}{0.121569,0.466667,0.705882}%
\pgfsetfillcolor{currentfill}%
\pgfsetfillopacity{0.200000}%
\pgfsetlinewidth{0.000000pt}%
\definecolor{currentstroke}{rgb}{0.000000,0.000000,0.000000}%
\pgfsetstrokecolor{currentstroke}%
\pgfsetstrokeopacity{0.200000}%
\pgfsetdash{}{0pt}%
\pgfpathrectangle{\pgfqpoint{0.882794in}{0.589583in}}{\pgfqpoint{6.200000in}{4.620000in}}%
\pgfusepath{clip}%
\pgfpathmoveto{\pgfqpoint{0.882794in}{3.500183in}}%
\pgfpathlineto{\pgfqpoint{0.882794in}{5.209583in}}%
\pgfpathlineto{\pgfqpoint{0.968809in}{5.209583in}}%
\pgfpathlineto{\pgfqpoint{0.968809in}{3.500183in}}%
\pgfpathlineto{\pgfqpoint{0.882794in}{3.500183in}}%
\pgfpathclose%
\pgfusepath{clip}%
\pgfsys@defobject{currentpattern}{\pgfqpoint{0in}{0in}}{\pgfqpoint{1in}{1in}}{%
\begin{pgfscope}%
\pgfpathrectangle{\pgfqpoint{0in}{0in}}{\pgfqpoint{1in}{1in}}%
\pgfusepath{clip}%
\pgfpathmoveto{\pgfqpoint{-0.500000in}{0.500000in}}%
\pgfpathlineto{\pgfqpoint{0.500000in}{1.500000in}}%
\pgfpathmoveto{\pgfqpoint{-0.416667in}{0.416667in}}%
\pgfpathlineto{\pgfqpoint{0.583333in}{1.416667in}}%
\pgfpathmoveto{\pgfqpoint{-0.333333in}{0.333333in}}%
\pgfpathlineto{\pgfqpoint{0.666667in}{1.333333in}}%
\pgfpathmoveto{\pgfqpoint{-0.250000in}{0.250000in}}%
\pgfpathlineto{\pgfqpoint{0.750000in}{1.250000in}}%
\pgfpathmoveto{\pgfqpoint{-0.166667in}{0.166667in}}%
\pgfpathlineto{\pgfqpoint{0.833333in}{1.166667in}}%
\pgfpathmoveto{\pgfqpoint{-0.083333in}{0.083333in}}%
\pgfpathlineto{\pgfqpoint{0.916667in}{1.083333in}}%
\pgfpathmoveto{\pgfqpoint{0.000000in}{0.000000in}}%
\pgfpathlineto{\pgfqpoint{1.000000in}{1.000000in}}%
\pgfpathmoveto{\pgfqpoint{0.083333in}{-0.083333in}}%
\pgfpathlineto{\pgfqpoint{1.083333in}{0.916667in}}%
\pgfpathmoveto{\pgfqpoint{0.166667in}{-0.166667in}}%
\pgfpathlineto{\pgfqpoint{1.166667in}{0.833333in}}%
\pgfpathmoveto{\pgfqpoint{0.250000in}{-0.250000in}}%
\pgfpathlineto{\pgfqpoint{1.250000in}{0.750000in}}%
\pgfpathmoveto{\pgfqpoint{0.333333in}{-0.333333in}}%
\pgfpathlineto{\pgfqpoint{1.333333in}{0.666667in}}%
\pgfpathmoveto{\pgfqpoint{0.416667in}{-0.416667in}}%
\pgfpathlineto{\pgfqpoint{1.416667in}{0.583333in}}%
\pgfpathmoveto{\pgfqpoint{0.500000in}{-0.500000in}}%
\pgfpathlineto{\pgfqpoint{1.500000in}{0.500000in}}%
\pgfusepath{stroke}%
\end{pgfscope}%
}%
\pgfsys@transformshift{0.882794in}{3.500183in}%
\pgfsys@useobject{currentpattern}{}%
\pgfsys@transformshift{1in}{0in}%
\pgfsys@transformshift{-1in}{0in}%
\pgfsys@transformshift{0in}{1in}%
\pgfsys@useobject{currentpattern}{}%
\pgfsys@transformshift{1in}{0in}%
\pgfsys@transformshift{-1in}{0in}%
\pgfsys@transformshift{0in}{1in}%
\end{pgfscope}%
\begin{pgfscope}%
\pgfpathrectangle{\pgfqpoint{0.882794in}{0.589583in}}{\pgfqpoint{6.200000in}{4.620000in}}%
\pgfusepath{clip}%
\pgfsetbuttcap%
\pgfsetroundjoin%
\definecolor{currentfill}{rgb}{0.750000,0.750000,0.000000}%
\pgfsetfillcolor{currentfill}%
\pgfsetlinewidth{1.003750pt}%
\definecolor{currentstroke}{rgb}{0.750000,0.750000,0.000000}%
\pgfsetstrokecolor{currentstroke}%
\pgfsetdash{}{0pt}%
\pgfsys@defobject{currentmarker}{\pgfqpoint{-0.093403in}{-0.079453in}}{\pgfqpoint{0.093403in}{0.098209in}}{%
\pgfpathmoveto{\pgfqpoint{0.000000in}{0.098209in}}%
\pgfpathlineto{\pgfqpoint{-0.022049in}{0.030348in}}%
\pgfpathlineto{\pgfqpoint{-0.093403in}{0.030348in}}%
\pgfpathlineto{\pgfqpoint{-0.035677in}{-0.011592in}}%
\pgfpathlineto{\pgfqpoint{-0.057726in}{-0.079453in}}%
\pgfpathlineto{\pgfqpoint{-0.000000in}{-0.037513in}}%
\pgfpathlineto{\pgfqpoint{0.057726in}{-0.079453in}}%
\pgfpathlineto{\pgfqpoint{0.035677in}{-0.011592in}}%
\pgfpathlineto{\pgfqpoint{0.093403in}{0.030348in}}%
\pgfpathlineto{\pgfqpoint{0.022049in}{0.030348in}}%
\pgfpathlineto{\pgfqpoint{0.000000in}{0.098209in}}%
\pgfpathclose%
\pgfusepath{stroke,fill}%
}%
\begin{pgfscope}%
\pgfsys@transformshift{0.943999in}{3.591336in}%
\pgfsys@useobject{currentmarker}{}%
\end{pgfscope}%
\end{pgfscope}%
\begin{pgfscope}%
\pgfpathrectangle{\pgfqpoint{0.882794in}{0.589583in}}{\pgfqpoint{6.200000in}{4.620000in}}%
\pgfusepath{clip}%
\pgfsetrectcap%
\pgfsetroundjoin%
\pgfsetlinewidth{0.803000pt}%
\definecolor{currentstroke}{rgb}{0.690196,0.690196,0.690196}%
\pgfsetstrokecolor{currentstroke}%
\pgfsetstrokeopacity{0.300000}%
\pgfsetdash{}{0pt}%
\pgfpathmoveto{\pgfqpoint{1.527906in}{0.589583in}}%
\pgfpathlineto{\pgfqpoint{1.527906in}{5.209583in}}%
\pgfusepath{stroke}%
\end{pgfscope}%
\begin{pgfscope}%
\pgfsetbuttcap%
\pgfsetroundjoin%
\definecolor{currentfill}{rgb}{0.000000,0.000000,0.000000}%
\pgfsetfillcolor{currentfill}%
\pgfsetlinewidth{0.803000pt}%
\definecolor{currentstroke}{rgb}{0.000000,0.000000,0.000000}%
\pgfsetstrokecolor{currentstroke}%
\pgfsetdash{}{0pt}%
\pgfsys@defobject{currentmarker}{\pgfqpoint{0.000000in}{-0.048611in}}{\pgfqpoint{0.000000in}{0.000000in}}{%
\pgfpathmoveto{\pgfqpoint{0.000000in}{0.000000in}}%
\pgfpathlineto{\pgfqpoint{0.000000in}{-0.048611in}}%
\pgfusepath{stroke,fill}%
}%
\begin{pgfscope}%
\pgfsys@transformshift{1.527906in}{0.589583in}%
\pgfsys@useobject{currentmarker}{}%
\end{pgfscope}%
\end{pgfscope}%
\begin{pgfscope}%
\definecolor{textcolor}{rgb}{0.000000,0.000000,0.000000}%
\pgfsetstrokecolor{textcolor}%
\pgfsetfillcolor{textcolor}%
\pgftext[x=1.527906in,y=0.492361in,,top]{\color{textcolor}\rmfamily\fontsize{10.000000}{12.000000}\selectfont \(\displaystyle {0.01}\)}%
\end{pgfscope}%
\begin{pgfscope}%
\pgfpathrectangle{\pgfqpoint{0.882794in}{0.589583in}}{\pgfqpoint{6.200000in}{4.620000in}}%
\pgfusepath{clip}%
\pgfsetrectcap%
\pgfsetroundjoin%
\pgfsetlinewidth{0.803000pt}%
\definecolor{currentstroke}{rgb}{0.690196,0.690196,0.690196}%
\pgfsetstrokecolor{currentstroke}%
\pgfsetstrokeopacity{0.300000}%
\pgfsetdash{}{0pt}%
\pgfpathmoveto{\pgfqpoint{2.603093in}{0.589583in}}%
\pgfpathlineto{\pgfqpoint{2.603093in}{5.209583in}}%
\pgfusepath{stroke}%
\end{pgfscope}%
\begin{pgfscope}%
\pgfsetbuttcap%
\pgfsetroundjoin%
\definecolor{currentfill}{rgb}{0.000000,0.000000,0.000000}%
\pgfsetfillcolor{currentfill}%
\pgfsetlinewidth{0.803000pt}%
\definecolor{currentstroke}{rgb}{0.000000,0.000000,0.000000}%
\pgfsetstrokecolor{currentstroke}%
\pgfsetdash{}{0pt}%
\pgfsys@defobject{currentmarker}{\pgfqpoint{0.000000in}{-0.048611in}}{\pgfqpoint{0.000000in}{0.000000in}}{%
\pgfpathmoveto{\pgfqpoint{0.000000in}{0.000000in}}%
\pgfpathlineto{\pgfqpoint{0.000000in}{-0.048611in}}%
\pgfusepath{stroke,fill}%
}%
\begin{pgfscope}%
\pgfsys@transformshift{2.603093in}{0.589583in}%
\pgfsys@useobject{currentmarker}{}%
\end{pgfscope}%
\end{pgfscope}%
\begin{pgfscope}%
\definecolor{textcolor}{rgb}{0.000000,0.000000,0.000000}%
\pgfsetstrokecolor{textcolor}%
\pgfsetfillcolor{textcolor}%
\pgftext[x=2.603093in,y=0.492361in,,top]{\color{textcolor}\rmfamily\fontsize{10.000000}{12.000000}\selectfont \(\displaystyle {0.02}\)}%
\end{pgfscope}%
\begin{pgfscope}%
\pgfpathrectangle{\pgfqpoint{0.882794in}{0.589583in}}{\pgfqpoint{6.200000in}{4.620000in}}%
\pgfusepath{clip}%
\pgfsetrectcap%
\pgfsetroundjoin%
\pgfsetlinewidth{0.803000pt}%
\definecolor{currentstroke}{rgb}{0.690196,0.690196,0.690196}%
\pgfsetstrokecolor{currentstroke}%
\pgfsetstrokeopacity{0.300000}%
\pgfsetdash{}{0pt}%
\pgfpathmoveto{\pgfqpoint{3.678279in}{0.589583in}}%
\pgfpathlineto{\pgfqpoint{3.678279in}{5.209583in}}%
\pgfusepath{stroke}%
\end{pgfscope}%
\begin{pgfscope}%
\pgfsetbuttcap%
\pgfsetroundjoin%
\definecolor{currentfill}{rgb}{0.000000,0.000000,0.000000}%
\pgfsetfillcolor{currentfill}%
\pgfsetlinewidth{0.803000pt}%
\definecolor{currentstroke}{rgb}{0.000000,0.000000,0.000000}%
\pgfsetstrokecolor{currentstroke}%
\pgfsetdash{}{0pt}%
\pgfsys@defobject{currentmarker}{\pgfqpoint{0.000000in}{-0.048611in}}{\pgfqpoint{0.000000in}{0.000000in}}{%
\pgfpathmoveto{\pgfqpoint{0.000000in}{0.000000in}}%
\pgfpathlineto{\pgfqpoint{0.000000in}{-0.048611in}}%
\pgfusepath{stroke,fill}%
}%
\begin{pgfscope}%
\pgfsys@transformshift{3.678279in}{0.589583in}%
\pgfsys@useobject{currentmarker}{}%
\end{pgfscope}%
\end{pgfscope}%
\begin{pgfscope}%
\definecolor{textcolor}{rgb}{0.000000,0.000000,0.000000}%
\pgfsetstrokecolor{textcolor}%
\pgfsetfillcolor{textcolor}%
\pgftext[x=3.678279in,y=0.492361in,,top]{\color{textcolor}\rmfamily\fontsize{10.000000}{12.000000}\selectfont \(\displaystyle {0.03}\)}%
\end{pgfscope}%
\begin{pgfscope}%
\pgfpathrectangle{\pgfqpoint{0.882794in}{0.589583in}}{\pgfqpoint{6.200000in}{4.620000in}}%
\pgfusepath{clip}%
\pgfsetrectcap%
\pgfsetroundjoin%
\pgfsetlinewidth{0.803000pt}%
\definecolor{currentstroke}{rgb}{0.690196,0.690196,0.690196}%
\pgfsetstrokecolor{currentstroke}%
\pgfsetstrokeopacity{0.300000}%
\pgfsetdash{}{0pt}%
\pgfpathmoveto{\pgfqpoint{4.753466in}{0.589583in}}%
\pgfpathlineto{\pgfqpoint{4.753466in}{5.209583in}}%
\pgfusepath{stroke}%
\end{pgfscope}%
\begin{pgfscope}%
\pgfsetbuttcap%
\pgfsetroundjoin%
\definecolor{currentfill}{rgb}{0.000000,0.000000,0.000000}%
\pgfsetfillcolor{currentfill}%
\pgfsetlinewidth{0.803000pt}%
\definecolor{currentstroke}{rgb}{0.000000,0.000000,0.000000}%
\pgfsetstrokecolor{currentstroke}%
\pgfsetdash{}{0pt}%
\pgfsys@defobject{currentmarker}{\pgfqpoint{0.000000in}{-0.048611in}}{\pgfqpoint{0.000000in}{0.000000in}}{%
\pgfpathmoveto{\pgfqpoint{0.000000in}{0.000000in}}%
\pgfpathlineto{\pgfqpoint{0.000000in}{-0.048611in}}%
\pgfusepath{stroke,fill}%
}%
\begin{pgfscope}%
\pgfsys@transformshift{4.753466in}{0.589583in}%
\pgfsys@useobject{currentmarker}{}%
\end{pgfscope}%
\end{pgfscope}%
\begin{pgfscope}%
\definecolor{textcolor}{rgb}{0.000000,0.000000,0.000000}%
\pgfsetstrokecolor{textcolor}%
\pgfsetfillcolor{textcolor}%
\pgftext[x=4.753466in,y=0.492361in,,top]{\color{textcolor}\rmfamily\fontsize{10.000000}{12.000000}\selectfont \(\displaystyle {0.04}\)}%
\end{pgfscope}%
\begin{pgfscope}%
\pgfpathrectangle{\pgfqpoint{0.882794in}{0.589583in}}{\pgfqpoint{6.200000in}{4.620000in}}%
\pgfusepath{clip}%
\pgfsetrectcap%
\pgfsetroundjoin%
\pgfsetlinewidth{0.803000pt}%
\definecolor{currentstroke}{rgb}{0.690196,0.690196,0.690196}%
\pgfsetstrokecolor{currentstroke}%
\pgfsetstrokeopacity{0.300000}%
\pgfsetdash{}{0pt}%
\pgfpathmoveto{\pgfqpoint{5.828652in}{0.589583in}}%
\pgfpathlineto{\pgfqpoint{5.828652in}{5.209583in}}%
\pgfusepath{stroke}%
\end{pgfscope}%
\begin{pgfscope}%
\pgfsetbuttcap%
\pgfsetroundjoin%
\definecolor{currentfill}{rgb}{0.000000,0.000000,0.000000}%
\pgfsetfillcolor{currentfill}%
\pgfsetlinewidth{0.803000pt}%
\definecolor{currentstroke}{rgb}{0.000000,0.000000,0.000000}%
\pgfsetstrokecolor{currentstroke}%
\pgfsetdash{}{0pt}%
\pgfsys@defobject{currentmarker}{\pgfqpoint{0.000000in}{-0.048611in}}{\pgfqpoint{0.000000in}{0.000000in}}{%
\pgfpathmoveto{\pgfqpoint{0.000000in}{0.000000in}}%
\pgfpathlineto{\pgfqpoint{0.000000in}{-0.048611in}}%
\pgfusepath{stroke,fill}%
}%
\begin{pgfscope}%
\pgfsys@transformshift{5.828652in}{0.589583in}%
\pgfsys@useobject{currentmarker}{}%
\end{pgfscope}%
\end{pgfscope}%
\begin{pgfscope}%
\definecolor{textcolor}{rgb}{0.000000,0.000000,0.000000}%
\pgfsetstrokecolor{textcolor}%
\pgfsetfillcolor{textcolor}%
\pgftext[x=5.828652in,y=0.492361in,,top]{\color{textcolor}\rmfamily\fontsize{10.000000}{12.000000}\selectfont \(\displaystyle {0.05}\)}%
\end{pgfscope}%
\begin{pgfscope}%
\pgfpathrectangle{\pgfqpoint{0.882794in}{0.589583in}}{\pgfqpoint{6.200000in}{4.620000in}}%
\pgfusepath{clip}%
\pgfsetrectcap%
\pgfsetroundjoin%
\pgfsetlinewidth{0.803000pt}%
\definecolor{currentstroke}{rgb}{0.690196,0.690196,0.690196}%
\pgfsetstrokecolor{currentstroke}%
\pgfsetstrokeopacity{0.300000}%
\pgfsetdash{}{0pt}%
\pgfpathmoveto{\pgfqpoint{6.903839in}{0.589583in}}%
\pgfpathlineto{\pgfqpoint{6.903839in}{5.209583in}}%
\pgfusepath{stroke}%
\end{pgfscope}%
\begin{pgfscope}%
\pgfsetbuttcap%
\pgfsetroundjoin%
\definecolor{currentfill}{rgb}{0.000000,0.000000,0.000000}%
\pgfsetfillcolor{currentfill}%
\pgfsetlinewidth{0.803000pt}%
\definecolor{currentstroke}{rgb}{0.000000,0.000000,0.000000}%
\pgfsetstrokecolor{currentstroke}%
\pgfsetdash{}{0pt}%
\pgfsys@defobject{currentmarker}{\pgfqpoint{0.000000in}{-0.048611in}}{\pgfqpoint{0.000000in}{0.000000in}}{%
\pgfpathmoveto{\pgfqpoint{0.000000in}{0.000000in}}%
\pgfpathlineto{\pgfqpoint{0.000000in}{-0.048611in}}%
\pgfusepath{stroke,fill}%
}%
\begin{pgfscope}%
\pgfsys@transformshift{6.903839in}{0.589583in}%
\pgfsys@useobject{currentmarker}{}%
\end{pgfscope}%
\end{pgfscope}%
\begin{pgfscope}%
\definecolor{textcolor}{rgb}{0.000000,0.000000,0.000000}%
\pgfsetstrokecolor{textcolor}%
\pgfsetfillcolor{textcolor}%
\pgftext[x=6.903839in,y=0.492361in,,top]{\color{textcolor}\rmfamily\fontsize{10.000000}{12.000000}\selectfont \(\displaystyle {0.06}\)}%
\end{pgfscope}%
\begin{pgfscope}%
\definecolor{textcolor}{rgb}{0.000000,0.000000,0.000000}%
\pgfsetstrokecolor{textcolor}%
\pgfsetfillcolor{textcolor}%
\pgftext[x=3.982794in,y=0.313349in,,top]{\color{textcolor}\rmfamily\fontsize{16.000000}{19.200000}\selectfont f1}%
\end{pgfscope}%
\begin{pgfscope}%
\pgfpathrectangle{\pgfqpoint{0.882794in}{0.589583in}}{\pgfqpoint{6.200000in}{4.620000in}}%
\pgfusepath{clip}%
\pgfsetrectcap%
\pgfsetroundjoin%
\pgfsetlinewidth{0.803000pt}%
\definecolor{currentstroke}{rgb}{0.690196,0.690196,0.690196}%
\pgfsetstrokecolor{currentstroke}%
\pgfsetstrokeopacity{0.300000}%
\pgfsetdash{}{0pt}%
\pgfpathmoveto{\pgfqpoint{0.882794in}{0.967077in}}%
\pgfpathlineto{\pgfqpoint{7.082794in}{0.967077in}}%
\pgfusepath{stroke}%
\end{pgfscope}%
\begin{pgfscope}%
\pgfsetbuttcap%
\pgfsetroundjoin%
\definecolor{currentfill}{rgb}{0.000000,0.000000,0.000000}%
\pgfsetfillcolor{currentfill}%
\pgfsetlinewidth{0.803000pt}%
\definecolor{currentstroke}{rgb}{0.000000,0.000000,0.000000}%
\pgfsetstrokecolor{currentstroke}%
\pgfsetdash{}{0pt}%
\pgfsys@defobject{currentmarker}{\pgfqpoint{-0.048611in}{0.000000in}}{\pgfqpoint{-0.000000in}{0.000000in}}{%
\pgfpathmoveto{\pgfqpoint{-0.000000in}{0.000000in}}%
\pgfpathlineto{\pgfqpoint{-0.048611in}{0.000000in}}%
\pgfusepath{stroke,fill}%
}%
\begin{pgfscope}%
\pgfsys@transformshift{0.882794in}{0.967077in}%
\pgfsys@useobject{currentmarker}{}%
\end{pgfscope}%
\end{pgfscope}%
\begin{pgfscope}%
\definecolor{textcolor}{rgb}{0.000000,0.000000,0.000000}%
\pgfsetstrokecolor{textcolor}%
\pgfsetfillcolor{textcolor}%
\pgftext[x=0.438349in, y=0.918851in, left, base]{\color{textcolor}\rmfamily\fontsize{10.000000}{12.000000}\selectfont \(\displaystyle {20000}\)}%
\end{pgfscope}%
\begin{pgfscope}%
\pgfpathrectangle{\pgfqpoint{0.882794in}{0.589583in}}{\pgfqpoint{6.200000in}{4.620000in}}%
\pgfusepath{clip}%
\pgfsetrectcap%
\pgfsetroundjoin%
\pgfsetlinewidth{0.803000pt}%
\definecolor{currentstroke}{rgb}{0.690196,0.690196,0.690196}%
\pgfsetstrokecolor{currentstroke}%
\pgfsetstrokeopacity{0.300000}%
\pgfsetdash{}{0pt}%
\pgfpathmoveto{\pgfqpoint{0.882794in}{1.619770in}}%
\pgfpathlineto{\pgfqpoint{7.082794in}{1.619770in}}%
\pgfusepath{stroke}%
\end{pgfscope}%
\begin{pgfscope}%
\pgfsetbuttcap%
\pgfsetroundjoin%
\definecolor{currentfill}{rgb}{0.000000,0.000000,0.000000}%
\pgfsetfillcolor{currentfill}%
\pgfsetlinewidth{0.803000pt}%
\definecolor{currentstroke}{rgb}{0.000000,0.000000,0.000000}%
\pgfsetstrokecolor{currentstroke}%
\pgfsetdash{}{0pt}%
\pgfsys@defobject{currentmarker}{\pgfqpoint{-0.048611in}{0.000000in}}{\pgfqpoint{-0.000000in}{0.000000in}}{%
\pgfpathmoveto{\pgfqpoint{-0.000000in}{0.000000in}}%
\pgfpathlineto{\pgfqpoint{-0.048611in}{0.000000in}}%
\pgfusepath{stroke,fill}%
}%
\begin{pgfscope}%
\pgfsys@transformshift{0.882794in}{1.619770in}%
\pgfsys@useobject{currentmarker}{}%
\end{pgfscope}%
\end{pgfscope}%
\begin{pgfscope}%
\definecolor{textcolor}{rgb}{0.000000,0.000000,0.000000}%
\pgfsetstrokecolor{textcolor}%
\pgfsetfillcolor{textcolor}%
\pgftext[x=0.438349in, y=1.571545in, left, base]{\color{textcolor}\rmfamily\fontsize{10.000000}{12.000000}\selectfont \(\displaystyle {40000}\)}%
\end{pgfscope}%
\begin{pgfscope}%
\pgfpathrectangle{\pgfqpoint{0.882794in}{0.589583in}}{\pgfqpoint{6.200000in}{4.620000in}}%
\pgfusepath{clip}%
\pgfsetrectcap%
\pgfsetroundjoin%
\pgfsetlinewidth{0.803000pt}%
\definecolor{currentstroke}{rgb}{0.690196,0.690196,0.690196}%
\pgfsetstrokecolor{currentstroke}%
\pgfsetstrokeopacity{0.300000}%
\pgfsetdash{}{0pt}%
\pgfpathmoveto{\pgfqpoint{0.882794in}{2.272463in}}%
\pgfpathlineto{\pgfqpoint{7.082794in}{2.272463in}}%
\pgfusepath{stroke}%
\end{pgfscope}%
\begin{pgfscope}%
\pgfsetbuttcap%
\pgfsetroundjoin%
\definecolor{currentfill}{rgb}{0.000000,0.000000,0.000000}%
\pgfsetfillcolor{currentfill}%
\pgfsetlinewidth{0.803000pt}%
\definecolor{currentstroke}{rgb}{0.000000,0.000000,0.000000}%
\pgfsetstrokecolor{currentstroke}%
\pgfsetdash{}{0pt}%
\pgfsys@defobject{currentmarker}{\pgfqpoint{-0.048611in}{0.000000in}}{\pgfqpoint{-0.000000in}{0.000000in}}{%
\pgfpathmoveto{\pgfqpoint{-0.000000in}{0.000000in}}%
\pgfpathlineto{\pgfqpoint{-0.048611in}{0.000000in}}%
\pgfusepath{stroke,fill}%
}%
\begin{pgfscope}%
\pgfsys@transformshift{0.882794in}{2.272463in}%
\pgfsys@useobject{currentmarker}{}%
\end{pgfscope}%
\end{pgfscope}%
\begin{pgfscope}%
\definecolor{textcolor}{rgb}{0.000000,0.000000,0.000000}%
\pgfsetstrokecolor{textcolor}%
\pgfsetfillcolor{textcolor}%
\pgftext[x=0.438349in, y=2.224238in, left, base]{\color{textcolor}\rmfamily\fontsize{10.000000}{12.000000}\selectfont \(\displaystyle {60000}\)}%
\end{pgfscope}%
\begin{pgfscope}%
\pgfpathrectangle{\pgfqpoint{0.882794in}{0.589583in}}{\pgfqpoint{6.200000in}{4.620000in}}%
\pgfusepath{clip}%
\pgfsetrectcap%
\pgfsetroundjoin%
\pgfsetlinewidth{0.803000pt}%
\definecolor{currentstroke}{rgb}{0.690196,0.690196,0.690196}%
\pgfsetstrokecolor{currentstroke}%
\pgfsetstrokeopacity{0.300000}%
\pgfsetdash{}{0pt}%
\pgfpathmoveto{\pgfqpoint{0.882794in}{2.925157in}}%
\pgfpathlineto{\pgfqpoint{7.082794in}{2.925157in}}%
\pgfusepath{stroke}%
\end{pgfscope}%
\begin{pgfscope}%
\pgfsetbuttcap%
\pgfsetroundjoin%
\definecolor{currentfill}{rgb}{0.000000,0.000000,0.000000}%
\pgfsetfillcolor{currentfill}%
\pgfsetlinewidth{0.803000pt}%
\definecolor{currentstroke}{rgb}{0.000000,0.000000,0.000000}%
\pgfsetstrokecolor{currentstroke}%
\pgfsetdash{}{0pt}%
\pgfsys@defobject{currentmarker}{\pgfqpoint{-0.048611in}{0.000000in}}{\pgfqpoint{-0.000000in}{0.000000in}}{%
\pgfpathmoveto{\pgfqpoint{-0.000000in}{0.000000in}}%
\pgfpathlineto{\pgfqpoint{-0.048611in}{0.000000in}}%
\pgfusepath{stroke,fill}%
}%
\begin{pgfscope}%
\pgfsys@transformshift{0.882794in}{2.925157in}%
\pgfsys@useobject{currentmarker}{}%
\end{pgfscope}%
\end{pgfscope}%
\begin{pgfscope}%
\definecolor{textcolor}{rgb}{0.000000,0.000000,0.000000}%
\pgfsetstrokecolor{textcolor}%
\pgfsetfillcolor{textcolor}%
\pgftext[x=0.438349in, y=2.876931in, left, base]{\color{textcolor}\rmfamily\fontsize{10.000000}{12.000000}\selectfont \(\displaystyle {80000}\)}%
\end{pgfscope}%
\begin{pgfscope}%
\pgfpathrectangle{\pgfqpoint{0.882794in}{0.589583in}}{\pgfqpoint{6.200000in}{4.620000in}}%
\pgfusepath{clip}%
\pgfsetrectcap%
\pgfsetroundjoin%
\pgfsetlinewidth{0.803000pt}%
\definecolor{currentstroke}{rgb}{0.690196,0.690196,0.690196}%
\pgfsetstrokecolor{currentstroke}%
\pgfsetstrokeopacity{0.300000}%
\pgfsetdash{}{0pt}%
\pgfpathmoveto{\pgfqpoint{0.882794in}{3.577850in}}%
\pgfpathlineto{\pgfqpoint{7.082794in}{3.577850in}}%
\pgfusepath{stroke}%
\end{pgfscope}%
\begin{pgfscope}%
\pgfsetbuttcap%
\pgfsetroundjoin%
\definecolor{currentfill}{rgb}{0.000000,0.000000,0.000000}%
\pgfsetfillcolor{currentfill}%
\pgfsetlinewidth{0.803000pt}%
\definecolor{currentstroke}{rgb}{0.000000,0.000000,0.000000}%
\pgfsetstrokecolor{currentstroke}%
\pgfsetdash{}{0pt}%
\pgfsys@defobject{currentmarker}{\pgfqpoint{-0.048611in}{0.000000in}}{\pgfqpoint{-0.000000in}{0.000000in}}{%
\pgfpathmoveto{\pgfqpoint{-0.000000in}{0.000000in}}%
\pgfpathlineto{\pgfqpoint{-0.048611in}{0.000000in}}%
\pgfusepath{stroke,fill}%
}%
\begin{pgfscope}%
\pgfsys@transformshift{0.882794in}{3.577850in}%
\pgfsys@useobject{currentmarker}{}%
\end{pgfscope}%
\end{pgfscope}%
\begin{pgfscope}%
\definecolor{textcolor}{rgb}{0.000000,0.000000,0.000000}%
\pgfsetstrokecolor{textcolor}%
\pgfsetfillcolor{textcolor}%
\pgftext[x=0.368904in, y=3.529625in, left, base]{\color{textcolor}\rmfamily\fontsize{10.000000}{12.000000}\selectfont \(\displaystyle {100000}\)}%
\end{pgfscope}%
\begin{pgfscope}%
\pgfpathrectangle{\pgfqpoint{0.882794in}{0.589583in}}{\pgfqpoint{6.200000in}{4.620000in}}%
\pgfusepath{clip}%
\pgfsetrectcap%
\pgfsetroundjoin%
\pgfsetlinewidth{0.803000pt}%
\definecolor{currentstroke}{rgb}{0.690196,0.690196,0.690196}%
\pgfsetstrokecolor{currentstroke}%
\pgfsetstrokeopacity{0.300000}%
\pgfsetdash{}{0pt}%
\pgfpathmoveto{\pgfqpoint{0.882794in}{4.230543in}}%
\pgfpathlineto{\pgfqpoint{7.082794in}{4.230543in}}%
\pgfusepath{stroke}%
\end{pgfscope}%
\begin{pgfscope}%
\pgfsetbuttcap%
\pgfsetroundjoin%
\definecolor{currentfill}{rgb}{0.000000,0.000000,0.000000}%
\pgfsetfillcolor{currentfill}%
\pgfsetlinewidth{0.803000pt}%
\definecolor{currentstroke}{rgb}{0.000000,0.000000,0.000000}%
\pgfsetstrokecolor{currentstroke}%
\pgfsetdash{}{0pt}%
\pgfsys@defobject{currentmarker}{\pgfqpoint{-0.048611in}{0.000000in}}{\pgfqpoint{-0.000000in}{0.000000in}}{%
\pgfpathmoveto{\pgfqpoint{-0.000000in}{0.000000in}}%
\pgfpathlineto{\pgfqpoint{-0.048611in}{0.000000in}}%
\pgfusepath{stroke,fill}%
}%
\begin{pgfscope}%
\pgfsys@transformshift{0.882794in}{4.230543in}%
\pgfsys@useobject{currentmarker}{}%
\end{pgfscope}%
\end{pgfscope}%
\begin{pgfscope}%
\definecolor{textcolor}{rgb}{0.000000,0.000000,0.000000}%
\pgfsetstrokecolor{textcolor}%
\pgfsetfillcolor{textcolor}%
\pgftext[x=0.368904in, y=4.182318in, left, base]{\color{textcolor}\rmfamily\fontsize{10.000000}{12.000000}\selectfont \(\displaystyle {120000}\)}%
\end{pgfscope}%
\begin{pgfscope}%
\pgfpathrectangle{\pgfqpoint{0.882794in}{0.589583in}}{\pgfqpoint{6.200000in}{4.620000in}}%
\pgfusepath{clip}%
\pgfsetrectcap%
\pgfsetroundjoin%
\pgfsetlinewidth{0.803000pt}%
\definecolor{currentstroke}{rgb}{0.690196,0.690196,0.690196}%
\pgfsetstrokecolor{currentstroke}%
\pgfsetstrokeopacity{0.300000}%
\pgfsetdash{}{0pt}%
\pgfpathmoveto{\pgfqpoint{0.882794in}{4.883236in}}%
\pgfpathlineto{\pgfqpoint{7.082794in}{4.883236in}}%
\pgfusepath{stroke}%
\end{pgfscope}%
\begin{pgfscope}%
\pgfsetbuttcap%
\pgfsetroundjoin%
\definecolor{currentfill}{rgb}{0.000000,0.000000,0.000000}%
\pgfsetfillcolor{currentfill}%
\pgfsetlinewidth{0.803000pt}%
\definecolor{currentstroke}{rgb}{0.000000,0.000000,0.000000}%
\pgfsetstrokecolor{currentstroke}%
\pgfsetdash{}{0pt}%
\pgfsys@defobject{currentmarker}{\pgfqpoint{-0.048611in}{0.000000in}}{\pgfqpoint{-0.000000in}{0.000000in}}{%
\pgfpathmoveto{\pgfqpoint{-0.000000in}{0.000000in}}%
\pgfpathlineto{\pgfqpoint{-0.048611in}{0.000000in}}%
\pgfusepath{stroke,fill}%
}%
\begin{pgfscope}%
\pgfsys@transformshift{0.882794in}{4.883236in}%
\pgfsys@useobject{currentmarker}{}%
\end{pgfscope}%
\end{pgfscope}%
\begin{pgfscope}%
\definecolor{textcolor}{rgb}{0.000000,0.000000,0.000000}%
\pgfsetstrokecolor{textcolor}%
\pgfsetfillcolor{textcolor}%
\pgftext[x=0.368904in, y=4.835011in, left, base]{\color{textcolor}\rmfamily\fontsize{10.000000}{12.000000}\selectfont \(\displaystyle {140000}\)}%
\end{pgfscope}%
\begin{pgfscope}%
\definecolor{textcolor}{rgb}{0.000000,0.000000,0.000000}%
\pgfsetstrokecolor{textcolor}%
\pgfsetfillcolor{textcolor}%
\pgftext[x=0.313349in,y=2.899583in,,bottom,rotate=90.000000]{\color{textcolor}\rmfamily\fontsize{16.000000}{19.200000}\selectfont f2}%
\end{pgfscope}%
\begin{pgfscope}%
\pgfpathrectangle{\pgfqpoint{0.882794in}{0.589583in}}{\pgfqpoint{6.200000in}{4.620000in}}%
\pgfusepath{clip}%
\pgfsetrectcap%
\pgfsetroundjoin%
\pgfsetlinewidth{3.011250pt}%
\definecolor{currentstroke}{rgb}{0.000000,0.000000,0.000000}%
\pgfsetstrokecolor{currentstroke}%
\pgfsetdash{}{0pt}%
\pgfpathmoveto{\pgfqpoint{0.882794in}{3.577850in}}%
\pgfpathlineto{\pgfqpoint{0.891560in}{3.512665in}}%
\pgfpathlineto{\pgfqpoint{0.900325in}{3.450034in}}%
\pgfpathlineto{\pgfqpoint{0.909090in}{3.389809in}}%
\pgfpathlineto{\pgfqpoint{0.917856in}{3.331853in}}%
\pgfpathlineto{\pgfqpoint{0.926621in}{3.276041in}}%
\pgfpathlineto{\pgfqpoint{0.935387in}{3.222257in}}%
\pgfpathlineto{\pgfqpoint{0.944152in}{3.170391in}}%
\pgfpathlineto{\pgfqpoint{0.952917in}{3.120342in}}%
\pgfpathlineto{\pgfqpoint{0.961683in}{3.072018in}}%
\pgfpathlineto{\pgfqpoint{0.970448in}{3.025330in}}%
\pgfpathlineto{\pgfqpoint{0.979213in}{2.980197in}}%
\pgfpathlineto{\pgfqpoint{0.987979in}{2.936541in}}%
\pgfpathlineto{\pgfqpoint{0.996744in}{2.894293in}}%
\pgfpathlineto{\pgfqpoint{1.005510in}{2.853384in}}%
\pgfpathlineto{\pgfqpoint{1.014275in}{2.813753in}}%
\pgfpathlineto{\pgfqpoint{1.023040in}{2.775340in}}%
\pgfpathlineto{\pgfqpoint{1.031806in}{2.738089in}}%
\pgfpathlineto{\pgfqpoint{1.040571in}{2.701949in}}%
\pgfpathlineto{\pgfqpoint{1.049336in}{2.666872in}}%
\pgfpathlineto{\pgfqpoint{1.062485in}{2.616146in}}%
\pgfpathlineto{\pgfqpoint{1.075633in}{2.567562in}}%
\pgfpathlineto{\pgfqpoint{1.088781in}{2.520986in}}%
\pgfpathlineto{\pgfqpoint{1.101929in}{2.476297in}}%
\pgfpathlineto{\pgfqpoint{1.115077in}{2.433382in}}%
\pgfpathlineto{\pgfqpoint{1.128225in}{2.392138in}}%
\pgfpathlineto{\pgfqpoint{1.141373in}{2.352469in}}%
\pgfpathlineto{\pgfqpoint{1.154521in}{2.314286in}}%
\pgfpathlineto{\pgfqpoint{1.167669in}{2.277507in}}%
\pgfpathlineto{\pgfqpoint{1.180817in}{2.242057in}}%
\pgfpathlineto{\pgfqpoint{1.193965in}{2.207864in}}%
\pgfpathlineto{\pgfqpoint{1.207113in}{2.174863in}}%
\pgfpathlineto{\pgfqpoint{1.220261in}{2.142993in}}%
\pgfpathlineto{\pgfqpoint{1.233409in}{2.112196in}}%
\pgfpathlineto{\pgfqpoint{1.246557in}{2.082420in}}%
\pgfpathlineto{\pgfqpoint{1.259705in}{2.053614in}}%
\pgfpathlineto{\pgfqpoint{1.272854in}{2.025731in}}%
\pgfpathlineto{\pgfqpoint{1.286002in}{1.998728in}}%
\pgfpathlineto{\pgfqpoint{1.299150in}{1.972564in}}%
\pgfpathlineto{\pgfqpoint{1.312298in}{1.947201in}}%
\pgfpathlineto{\pgfqpoint{1.325446in}{1.922602in}}%
\pgfpathlineto{\pgfqpoint{1.338594in}{1.898733in}}%
\pgfpathlineto{\pgfqpoint{1.351742in}{1.875562in}}%
\pgfpathlineto{\pgfqpoint{1.364890in}{1.853059in}}%
\pgfpathlineto{\pgfqpoint{1.378038in}{1.831195in}}%
\pgfpathlineto{\pgfqpoint{1.391186in}{1.809945in}}%
\pgfpathlineto{\pgfqpoint{1.404334in}{1.789281in}}%
\pgfpathlineto{\pgfqpoint{1.417482in}{1.769181in}}%
\pgfpathlineto{\pgfqpoint{1.430630in}{1.749621in}}%
\pgfpathlineto{\pgfqpoint{1.443778in}{1.730580in}}%
\pgfpathlineto{\pgfqpoint{1.456926in}{1.712038in}}%
\pgfpathlineto{\pgfqpoint{1.474457in}{1.688057in}}%
\pgfpathlineto{\pgfqpoint{1.491988in}{1.664886in}}%
\pgfpathlineto{\pgfqpoint{1.509519in}{1.642483in}}%
\pgfpathlineto{\pgfqpoint{1.527049in}{1.620811in}}%
\pgfpathlineto{\pgfqpoint{1.544580in}{1.599835in}}%
\pgfpathlineto{\pgfqpoint{1.562111in}{1.579522in}}%
\pgfpathlineto{\pgfqpoint{1.579642in}{1.559841in}}%
\pgfpathlineto{\pgfqpoint{1.597172in}{1.540763in}}%
\pgfpathlineto{\pgfqpoint{1.614703in}{1.522261in}}%
\pgfpathlineto{\pgfqpoint{1.632234in}{1.504309in}}%
\pgfpathlineto{\pgfqpoint{1.649765in}{1.486882in}}%
\pgfpathlineto{\pgfqpoint{1.667296in}{1.469959in}}%
\pgfpathlineto{\pgfqpoint{1.684826in}{1.453517in}}%
\pgfpathlineto{\pgfqpoint{1.702357in}{1.437537in}}%
\pgfpathlineto{\pgfqpoint{1.719888in}{1.421998in}}%
\pgfpathlineto{\pgfqpoint{1.737419in}{1.406884in}}%
\pgfpathlineto{\pgfqpoint{1.754949in}{1.392176in}}%
\pgfpathlineto{\pgfqpoint{1.776863in}{1.374340in}}%
\pgfpathlineto{\pgfqpoint{1.798776in}{1.357084in}}%
\pgfpathlineto{\pgfqpoint{1.820690in}{1.340381in}}%
\pgfpathlineto{\pgfqpoint{1.842603in}{1.324205in}}%
\pgfpathlineto{\pgfqpoint{1.864516in}{1.308531in}}%
\pgfpathlineto{\pgfqpoint{1.886430in}{1.293336in}}%
\pgfpathlineto{\pgfqpoint{1.908343in}{1.278598in}}%
\pgfpathlineto{\pgfqpoint{1.930257in}{1.264298in}}%
\pgfpathlineto{\pgfqpoint{1.952170in}{1.250416in}}%
\pgfpathlineto{\pgfqpoint{1.974084in}{1.236933in}}%
\pgfpathlineto{\pgfqpoint{1.995997in}{1.223834in}}%
\pgfpathlineto{\pgfqpoint{2.017911in}{1.211101in}}%
\pgfpathlineto{\pgfqpoint{2.044207in}{1.196284in}}%
\pgfpathlineto{\pgfqpoint{2.070503in}{1.181950in}}%
\pgfpathlineto{\pgfqpoint{2.096799in}{1.168073in}}%
\pgfpathlineto{\pgfqpoint{2.123095in}{1.154634in}}%
\pgfpathlineto{\pgfqpoint{2.149391in}{1.141611in}}%
\pgfpathlineto{\pgfqpoint{2.175687in}{1.128986in}}%
\pgfpathlineto{\pgfqpoint{2.201983in}{1.116740in}}%
\pgfpathlineto{\pgfqpoint{2.228280in}{1.104857in}}%
\pgfpathlineto{\pgfqpoint{2.258958in}{1.091431in}}%
\pgfpathlineto{\pgfqpoint{2.289637in}{1.078454in}}%
\pgfpathlineto{\pgfqpoint{2.320316in}{1.065902in}}%
\pgfpathlineto{\pgfqpoint{2.350995in}{1.053757in}}%
\pgfpathlineto{\pgfqpoint{2.381674in}{1.041997in}}%
\pgfpathlineto{\pgfqpoint{2.412353in}{1.030606in}}%
\pgfpathlineto{\pgfqpoint{2.447414in}{1.018017in}}%
\pgfpathlineto{\pgfqpoint{2.482476in}{1.005863in}}%
\pgfpathlineto{\pgfqpoint{2.517537in}{0.994121in}}%
\pgfpathlineto{\pgfqpoint{2.552599in}{0.982771in}}%
\pgfpathlineto{\pgfqpoint{2.587660in}{0.971795in}}%
\pgfpathlineto{\pgfqpoint{2.622722in}{0.961173in}}%
\pgfpathlineto{\pgfqpoint{2.662166in}{0.949626in}}%
\pgfpathlineto{\pgfqpoint{2.701610in}{0.938484in}}%
\pgfpathlineto{\pgfqpoint{2.741054in}{0.927726in}}%
\pgfpathlineto{\pgfqpoint{2.780498in}{0.917333in}}%
\pgfpathlineto{\pgfqpoint{2.824325in}{0.906191in}}%
\pgfpathlineto{\pgfqpoint{2.868152in}{0.895453in}}%
\pgfpathlineto{\pgfqpoint{2.911979in}{0.885098in}}%
\pgfpathlineto{\pgfqpoint{2.955806in}{0.875105in}}%
\pgfpathlineto{\pgfqpoint{3.004015in}{0.864509in}}%
\pgfpathlineto{\pgfqpoint{3.052225in}{0.854307in}}%
\pgfpathlineto{\pgfqpoint{3.100435in}{0.844476in}}%
\pgfpathlineto{\pgfqpoint{3.153027in}{0.834152in}}%
\pgfpathlineto{\pgfqpoint{3.205619in}{0.824222in}}%
\pgfpathlineto{\pgfqpoint{3.258211in}{0.814664in}}%
\pgfpathlineto{\pgfqpoint{3.315186in}{0.804707in}}%
\pgfpathlineto{\pgfqpoint{3.372161in}{0.795138in}}%
\pgfpathlineto{\pgfqpoint{3.429136in}{0.785935in}}%
\pgfpathlineto{\pgfqpoint{3.490494in}{0.776411in}}%
\pgfpathlineto{\pgfqpoint{3.551851in}{0.767263in}}%
\pgfpathlineto{\pgfqpoint{3.617592in}{0.757856in}}%
\pgfpathlineto{\pgfqpoint{3.683332in}{0.748832in}}%
\pgfpathlineto{\pgfqpoint{3.749072in}{0.740167in}}%
\pgfpathlineto{\pgfqpoint{3.819195in}{0.731298in}}%
\pgfpathlineto{\pgfqpoint{3.889318in}{0.722791in}}%
\pgfpathlineto{\pgfqpoint{3.963824in}{0.714125in}}%
\pgfpathlineto{\pgfqpoint{4.042713in}{0.705341in}}%
\pgfpathlineto{\pgfqpoint{4.121601in}{0.696934in}}%
\pgfpathlineto{\pgfqpoint{4.204872in}{0.688444in}}%
\pgfpathlineto{\pgfqpoint{4.288143in}{0.680323in}}%
\pgfpathlineto{\pgfqpoint{4.375797in}{0.672147in}}%
\pgfpathlineto{\pgfqpoint{4.467833in}{0.663946in}}%
\pgfpathlineto{\pgfqpoint{4.559870in}{0.656113in}}%
\pgfpathlineto{\pgfqpoint{4.656289in}{0.648274in}}%
\pgfpathlineto{\pgfqpoint{4.757091in}{0.640455in}}%
\pgfpathlineto{\pgfqpoint{4.862275in}{0.632677in}}%
\pgfpathlineto{\pgfqpoint{4.971842in}{0.624960in}}%
\pgfpathlineto{\pgfqpoint{5.085792in}{0.617321in}}%
\pgfpathlineto{\pgfqpoint{5.204125in}{0.609777in}}%
\pgfpathlineto{\pgfqpoint{5.282818in}{0.605033in}}%
\pgfpathlineto{\pgfqpoint{5.326253in}{0.602909in}}%
\pgfpathlineto{\pgfqpoint{5.376927in}{0.600874in}}%
\pgfpathlineto{\pgfqpoint{5.434841in}{0.598955in}}%
\pgfpathlineto{\pgfqpoint{5.499994in}{0.597163in}}%
\pgfpathlineto{\pgfqpoint{5.579625in}{0.595348in}}%
\pgfpathlineto{\pgfqpoint{5.673735in}{0.593586in}}%
\pgfpathlineto{\pgfqpoint{5.789562in}{0.591818in}}%
\pgfpathlineto{\pgfqpoint{5.927107in}{0.590123in}}%
\pgfpathlineto{\pgfqpoint{5.977782in}{0.589583in}}%
\pgfpathlineto{\pgfqpoint{5.977782in}{0.589583in}}%
\pgfusepath{stroke}%
\end{pgfscope}%
\begin{pgfscope}%
\pgfpathrectangle{\pgfqpoint{0.882794in}{0.589583in}}{\pgfqpoint{6.200000in}{4.620000in}}%
\pgfusepath{clip}%
\pgfsetrectcap%
\pgfsetroundjoin%
\pgfsetlinewidth{1.003750pt}%
\definecolor{currentstroke}{rgb}{0.000000,0.000000,0.000000}%
\pgfsetstrokecolor{currentstroke}%
\pgfsetstrokeopacity{0.100000}%
\pgfsetdash{}{0pt}%
\pgfpathmoveto{\pgfqpoint{0.968809in}{4.230543in}}%
\pgfpathlineto{\pgfqpoint{0.979328in}{4.152322in}}%
\pgfpathlineto{\pgfqpoint{0.989846in}{4.077164in}}%
\pgfpathlineto{\pgfqpoint{1.000365in}{4.004894in}}%
\pgfpathlineto{\pgfqpoint{1.010883in}{3.935347in}}%
\pgfpathlineto{\pgfqpoint{1.021401in}{3.868373in}}%
\pgfpathlineto{\pgfqpoint{1.031920in}{3.803831in}}%
\pgfpathlineto{\pgfqpoint{1.042438in}{3.741592in}}%
\pgfpathlineto{\pgfqpoint{1.052957in}{3.681534in}}%
\pgfpathlineto{\pgfqpoint{1.063475in}{3.623545in}}%
\pgfpathlineto{\pgfqpoint{1.073994in}{3.567519in}}%
\pgfpathlineto{\pgfqpoint{1.084512in}{3.513359in}}%
\pgfpathlineto{\pgfqpoint{1.095031in}{3.460973in}}%
\pgfpathlineto{\pgfqpoint{1.105549in}{3.410275in}}%
\pgfpathlineto{\pgfqpoint{1.116068in}{3.361185in}}%
\pgfpathlineto{\pgfqpoint{1.126586in}{3.313627in}}%
\pgfpathlineto{\pgfqpoint{1.137104in}{3.267531in}}%
\pgfpathlineto{\pgfqpoint{1.147623in}{3.222830in}}%
\pgfpathlineto{\pgfqpoint{1.158141in}{3.179463in}}%
\pgfpathlineto{\pgfqpoint{1.168660in}{3.137370in}}%
\pgfpathlineto{\pgfqpoint{1.179178in}{3.096495in}}%
\pgfpathlineto{\pgfqpoint{1.189697in}{3.056788in}}%
\pgfpathlineto{\pgfqpoint{1.200215in}{3.018198in}}%
\pgfpathlineto{\pgfqpoint{1.210734in}{2.980679in}}%
\pgfpathlineto{\pgfqpoint{1.221252in}{2.944187in}}%
\pgfpathlineto{\pgfqpoint{1.231771in}{2.908680in}}%
\pgfpathlineto{\pgfqpoint{1.242289in}{2.874120in}}%
\pgfpathlineto{\pgfqpoint{1.252807in}{2.840468in}}%
\pgfpathlineto{\pgfqpoint{1.263326in}{2.807689in}}%
\pgfpathlineto{\pgfqpoint{1.273844in}{2.775750in}}%
\pgfpathlineto{\pgfqpoint{1.284363in}{2.744619in}}%
\pgfpathlineto{\pgfqpoint{1.300140in}{2.699372in}}%
\pgfpathlineto{\pgfqpoint{1.315918in}{2.655779in}}%
\pgfpathlineto{\pgfqpoint{1.331696in}{2.613751in}}%
\pgfpathlineto{\pgfqpoint{1.347473in}{2.573205in}}%
\pgfpathlineto{\pgfqpoint{1.363251in}{2.534064in}}%
\pgfpathlineto{\pgfqpoint{1.379029in}{2.496256in}}%
\pgfpathlineto{\pgfqpoint{1.394807in}{2.459715in}}%
\pgfpathlineto{\pgfqpoint{1.410584in}{2.424378in}}%
\pgfpathlineto{\pgfqpoint{1.426362in}{2.390186in}}%
\pgfpathlineto{\pgfqpoint{1.442140in}{2.357084in}}%
\pgfpathlineto{\pgfqpoint{1.457917in}{2.325022in}}%
\pgfpathlineto{\pgfqpoint{1.473695in}{2.293951in}}%
\pgfpathlineto{\pgfqpoint{1.489473in}{2.263825in}}%
\pgfpathlineto{\pgfqpoint{1.505250in}{2.234602in}}%
\pgfpathlineto{\pgfqpoint{1.521028in}{2.206243in}}%
\pgfpathlineto{\pgfqpoint{1.536806in}{2.178709in}}%
\pgfpathlineto{\pgfqpoint{1.552583in}{2.151965in}}%
\pgfpathlineto{\pgfqpoint{1.568361in}{2.125977in}}%
\pgfpathlineto{\pgfqpoint{1.584139in}{2.100715in}}%
\pgfpathlineto{\pgfqpoint{1.599916in}{2.076147in}}%
\pgfpathlineto{\pgfqpoint{1.615694in}{2.052246in}}%
\pgfpathlineto{\pgfqpoint{1.631472in}{2.028984in}}%
\pgfpathlineto{\pgfqpoint{1.647249in}{2.006337in}}%
\pgfpathlineto{\pgfqpoint{1.663027in}{1.984281in}}%
\pgfpathlineto{\pgfqpoint{1.678805in}{1.962792in}}%
\pgfpathlineto{\pgfqpoint{1.694582in}{1.941849in}}%
\pgfpathlineto{\pgfqpoint{1.710360in}{1.921432in}}%
\pgfpathlineto{\pgfqpoint{1.726138in}{1.901520in}}%
\pgfpathlineto{\pgfqpoint{1.741915in}{1.882096in}}%
\pgfpathlineto{\pgfqpoint{1.757693in}{1.863142in}}%
\pgfpathlineto{\pgfqpoint{1.773471in}{1.844641in}}%
\pgfpathlineto{\pgfqpoint{1.789248in}{1.826576in}}%
\pgfpathlineto{\pgfqpoint{1.810285in}{1.803143in}}%
\pgfpathlineto{\pgfqpoint{1.831322in}{1.780425in}}%
\pgfpathlineto{\pgfqpoint{1.852359in}{1.758390in}}%
\pgfpathlineto{\pgfqpoint{1.873396in}{1.737008in}}%
\pgfpathlineto{\pgfqpoint{1.894433in}{1.716250in}}%
\pgfpathlineto{\pgfqpoint{1.915470in}{1.696088in}}%
\pgfpathlineto{\pgfqpoint{1.936507in}{1.676499in}}%
\pgfpathlineto{\pgfqpoint{1.957544in}{1.657457in}}%
\pgfpathlineto{\pgfqpoint{1.978581in}{1.638940in}}%
\pgfpathlineto{\pgfqpoint{1.999618in}{1.620927in}}%
\pgfpathlineto{\pgfqpoint{2.020654in}{1.603397in}}%
\pgfpathlineto{\pgfqpoint{2.041691in}{1.586331in}}%
\pgfpathlineto{\pgfqpoint{2.062728in}{1.569712in}}%
\pgfpathlineto{\pgfqpoint{2.083765in}{1.553521in}}%
\pgfpathlineto{\pgfqpoint{2.104802in}{1.537742in}}%
\pgfpathlineto{\pgfqpoint{2.125839in}{1.522360in}}%
\pgfpathlineto{\pgfqpoint{2.146876in}{1.507360in}}%
\pgfpathlineto{\pgfqpoint{2.173172in}{1.489126in}}%
\pgfpathlineto{\pgfqpoint{2.199468in}{1.471441in}}%
\pgfpathlineto{\pgfqpoint{2.225764in}{1.454281in}}%
\pgfpathlineto{\pgfqpoint{2.252060in}{1.437622in}}%
\pgfpathlineto{\pgfqpoint{2.278356in}{1.421443in}}%
\pgfpathlineto{\pgfqpoint{2.304653in}{1.405724in}}%
\pgfpathlineto{\pgfqpoint{2.330949in}{1.390444in}}%
\pgfpathlineto{\pgfqpoint{2.357245in}{1.375587in}}%
\pgfpathlineto{\pgfqpoint{2.383541in}{1.361134in}}%
\pgfpathlineto{\pgfqpoint{2.409837in}{1.347070in}}%
\pgfpathlineto{\pgfqpoint{2.436133in}{1.333379in}}%
\pgfpathlineto{\pgfqpoint{2.462429in}{1.320046in}}%
\pgfpathlineto{\pgfqpoint{2.493985in}{1.304499in}}%
\pgfpathlineto{\pgfqpoint{2.525540in}{1.289426in}}%
\pgfpathlineto{\pgfqpoint{2.557095in}{1.274806in}}%
\pgfpathlineto{\pgfqpoint{2.588651in}{1.260617in}}%
\pgfpathlineto{\pgfqpoint{2.620206in}{1.246841in}}%
\pgfpathlineto{\pgfqpoint{2.651762in}{1.233461in}}%
\pgfpathlineto{\pgfqpoint{2.683317in}{1.220459in}}%
\pgfpathlineto{\pgfqpoint{2.714872in}{1.207820in}}%
\pgfpathlineto{\pgfqpoint{2.746428in}{1.195528in}}%
\pgfpathlineto{\pgfqpoint{2.783242in}{1.181609in}}%
\pgfpathlineto{\pgfqpoint{2.820057in}{1.168123in}}%
\pgfpathlineto{\pgfqpoint{2.856871in}{1.155050in}}%
\pgfpathlineto{\pgfqpoint{2.893686in}{1.142371in}}%
\pgfpathlineto{\pgfqpoint{2.930501in}{1.130069in}}%
\pgfpathlineto{\pgfqpoint{2.967315in}{1.118127in}}%
\pgfpathlineto{\pgfqpoint{3.009389in}{1.104900in}}%
\pgfpathlineto{\pgfqpoint{3.051463in}{1.092101in}}%
\pgfpathlineto{\pgfqpoint{3.093537in}{1.079711in}}%
\pgfpathlineto{\pgfqpoint{3.135610in}{1.067709in}}%
\pgfpathlineto{\pgfqpoint{3.177684in}{1.056077in}}%
\pgfpathlineto{\pgfqpoint{3.219758in}{1.044799in}}%
\pgfpathlineto{\pgfqpoint{3.267091in}{1.032515in}}%
\pgfpathlineto{\pgfqpoint{3.314424in}{1.020637in}}%
\pgfpathlineto{\pgfqpoint{3.361757in}{1.009146in}}%
\pgfpathlineto{\pgfqpoint{3.409090in}{0.998022in}}%
\pgfpathlineto{\pgfqpoint{3.456423in}{0.987249in}}%
\pgfpathlineto{\pgfqpoint{3.509015in}{0.975671in}}%
\pgfpathlineto{\pgfqpoint{3.561608in}{0.964484in}}%
\pgfpathlineto{\pgfqpoint{3.614200in}{0.953669in}}%
\pgfpathlineto{\pgfqpoint{3.666792in}{0.943208in}}%
\pgfpathlineto{\pgfqpoint{3.724644in}{0.932090in}}%
\pgfpathlineto{\pgfqpoint{3.782495in}{0.921358in}}%
\pgfpathlineto{\pgfqpoint{3.840347in}{0.910992in}}%
\pgfpathlineto{\pgfqpoint{3.903457in}{0.900081in}}%
\pgfpathlineto{\pgfqpoint{3.966568in}{0.889562in}}%
\pgfpathlineto{\pgfqpoint{4.029679in}{0.879413in}}%
\pgfpathlineto{\pgfqpoint{4.098049in}{0.868816in}}%
\pgfpathlineto{\pgfqpoint{4.166419in}{0.858609in}}%
\pgfpathlineto{\pgfqpoint{4.234789in}{0.848771in}}%
\pgfpathlineto{\pgfqpoint{4.308418in}{0.838566in}}%
\pgfpathlineto{\pgfqpoint{4.382047in}{0.828743in}}%
\pgfpathlineto{\pgfqpoint{4.455676in}{0.819282in}}%
\pgfpathlineto{\pgfqpoint{4.534564in}{0.809524in}}%
\pgfpathlineto{\pgfqpoint{4.613453in}{0.800136in}}%
\pgfpathlineto{\pgfqpoint{4.697600in}{0.790507in}}%
\pgfpathlineto{\pgfqpoint{4.781748in}{0.781252in}}%
\pgfpathlineto{\pgfqpoint{4.871155in}{0.771805in}}%
\pgfpathlineto{\pgfqpoint{4.960562in}{0.762733in}}%
\pgfpathlineto{\pgfqpoint{5.055228in}{0.753511in}}%
\pgfpathlineto{\pgfqpoint{5.149894in}{0.744661in}}%
\pgfpathlineto{\pgfqpoint{5.249819in}{0.735698in}}%
\pgfpathlineto{\pgfqpoint{5.349744in}{0.727101in}}%
\pgfpathlineto{\pgfqpoint{5.454929in}{0.718423in}}%
\pgfpathlineto{\pgfqpoint{5.565373in}{0.709695in}}%
\pgfpathlineto{\pgfqpoint{5.675816in}{0.701336in}}%
\pgfpathlineto{\pgfqpoint{5.791519in}{0.692950in}}%
\pgfpathlineto{\pgfqpoint{5.912482in}{0.684562in}}%
\pgfpathlineto{\pgfqpoint{6.038703in}{0.676198in}}%
\pgfpathlineto{\pgfqpoint{6.164924in}{0.668203in}}%
\pgfpathlineto{\pgfqpoint{6.248837in}{0.663163in}}%
\pgfpathlineto{\pgfqpoint{6.292272in}{0.661005in}}%
\pgfpathlineto{\pgfqpoint{6.344395in}{0.658823in}}%
\pgfpathlineto{\pgfqpoint{6.405204in}{0.656682in}}%
\pgfpathlineto{\pgfqpoint{6.474701in}{0.654626in}}%
\pgfpathlineto{\pgfqpoint{6.552884in}{0.652676in}}%
\pgfpathlineto{\pgfqpoint{6.648442in}{0.650677in}}%
\pgfpathlineto{\pgfqpoint{6.761373in}{0.648714in}}%
\pgfpathlineto{\pgfqpoint{6.891679in}{0.646838in}}%
\pgfpathlineto{\pgfqpoint{7.048046in}{0.644988in}}%
\pgfpathlineto{\pgfqpoint{7.082794in}{0.644623in}}%
\pgfpathlineto{\pgfqpoint{7.082794in}{0.644623in}}%
\pgfusepath{stroke}%
\end{pgfscope}%
\begin{pgfscope}%
\pgfsetrectcap%
\pgfsetmiterjoin%
\pgfsetlinewidth{0.803000pt}%
\definecolor{currentstroke}{rgb}{0.000000,0.000000,0.000000}%
\pgfsetstrokecolor{currentstroke}%
\pgfsetdash{}{0pt}%
\pgfpathmoveto{\pgfqpoint{0.882794in}{0.589583in}}%
\pgfpathlineto{\pgfqpoint{0.882794in}{5.209583in}}%
\pgfusepath{stroke}%
\end{pgfscope}%
\begin{pgfscope}%
\pgfsetrectcap%
\pgfsetmiterjoin%
\pgfsetlinewidth{0.803000pt}%
\definecolor{currentstroke}{rgb}{0.000000,0.000000,0.000000}%
\pgfsetstrokecolor{currentstroke}%
\pgfsetdash{}{0pt}%
\pgfpathmoveto{\pgfqpoint{7.082794in}{0.589583in}}%
\pgfpathlineto{\pgfqpoint{7.082794in}{5.209583in}}%
\pgfusepath{stroke}%
\end{pgfscope}%
\begin{pgfscope}%
\pgfsetrectcap%
\pgfsetmiterjoin%
\pgfsetlinewidth{0.803000pt}%
\definecolor{currentstroke}{rgb}{0.000000,0.000000,0.000000}%
\pgfsetstrokecolor{currentstroke}%
\pgfsetdash{}{0pt}%
\pgfpathmoveto{\pgfqpoint{0.882794in}{0.589583in}}%
\pgfpathlineto{\pgfqpoint{7.082794in}{0.589583in}}%
\pgfusepath{stroke}%
\end{pgfscope}%
\begin{pgfscope}%
\pgfsetrectcap%
\pgfsetmiterjoin%
\pgfsetlinewidth{0.803000pt}%
\definecolor{currentstroke}{rgb}{0.000000,0.000000,0.000000}%
\pgfsetstrokecolor{currentstroke}%
\pgfsetdash{}{0pt}%
\pgfpathmoveto{\pgfqpoint{0.882794in}{5.209583in}}%
\pgfpathlineto{\pgfqpoint{7.082794in}{5.209583in}}%
\pgfusepath{stroke}%
\end{pgfscope}%
\begin{pgfscope}%
\definecolor{textcolor}{rgb}{0.000000,0.000000,0.000000}%
\pgfsetstrokecolor{textcolor}%
\pgfsetfillcolor{textcolor}%
\pgftext[x=3.982794in,y=5.292916in,,base]{\color{textcolor}\rmfamily\fontsize{20.000000}{24.000000}\selectfont Objective Space}%
\end{pgfscope}%
\begin{pgfscope}%
\pgfsetbuttcap%
\pgfsetmiterjoin%
\definecolor{currentfill}{rgb}{0.300000,0.300000,0.300000}%
\pgfsetfillcolor{currentfill}%
\pgfsetfillopacity{0.500000}%
\pgfsetlinewidth{1.003750pt}%
\definecolor{currentstroke}{rgb}{0.300000,0.300000,0.300000}%
\pgfsetstrokecolor{currentstroke}%
\pgfsetstrokeopacity{0.500000}%
\pgfsetdash{}{0pt}%
\pgfpathmoveto{\pgfqpoint{3.187459in}{2.953313in}}%
\pgfpathlineto{\pgfqpoint{6.916128in}{2.953313in}}%
\pgfpathquadraticcurveto{\pgfqpoint{6.971683in}{2.953313in}}{\pgfqpoint{6.971683in}{3.008868in}}%
\pgfpathlineto{\pgfqpoint{6.971683in}{4.987361in}}%
\pgfpathquadraticcurveto{\pgfqpoint{6.971683in}{5.042916in}}{\pgfqpoint{6.916128in}{5.042916in}}%
\pgfpathlineto{\pgfqpoint{3.187459in}{5.042916in}}%
\pgfpathquadraticcurveto{\pgfqpoint{3.131904in}{5.042916in}}{\pgfqpoint{3.131904in}{4.987361in}}%
\pgfpathlineto{\pgfqpoint{3.131904in}{3.008868in}}%
\pgfpathquadraticcurveto{\pgfqpoint{3.131904in}{2.953313in}}{\pgfqpoint{3.187459in}{2.953313in}}%
\pgfpathlineto{\pgfqpoint{3.187459in}{2.953313in}}%
\pgfpathclose%
\pgfusepath{stroke,fill}%
\end{pgfscope}%
\begin{pgfscope}%
\pgfsetbuttcap%
\pgfsetmiterjoin%
\definecolor{currentfill}{rgb}{1.000000,1.000000,1.000000}%
\pgfsetfillcolor{currentfill}%
\pgfsetlinewidth{1.003750pt}%
\definecolor{currentstroke}{rgb}{0.800000,0.800000,0.800000}%
\pgfsetstrokecolor{currentstroke}%
\pgfsetdash{}{0pt}%
\pgfpathmoveto{\pgfqpoint{3.159682in}{2.981091in}}%
\pgfpathlineto{\pgfqpoint{6.888350in}{2.981091in}}%
\pgfpathquadraticcurveto{\pgfqpoint{6.943905in}{2.981091in}}{\pgfqpoint{6.943905in}{3.036646in}}%
\pgfpathlineto{\pgfqpoint{6.943905in}{5.015139in}}%
\pgfpathquadraticcurveto{\pgfqpoint{6.943905in}{5.070694in}}{\pgfqpoint{6.888350in}{5.070694in}}%
\pgfpathlineto{\pgfqpoint{3.159682in}{5.070694in}}%
\pgfpathquadraticcurveto{\pgfqpoint{3.104126in}{5.070694in}}{\pgfqpoint{3.104126in}{5.015139in}}%
\pgfpathlineto{\pgfqpoint{3.104126in}{3.036646in}}%
\pgfpathquadraticcurveto{\pgfqpoint{3.104126in}{2.981091in}}{\pgfqpoint{3.159682in}{2.981091in}}%
\pgfpathlineto{\pgfqpoint{3.159682in}{2.981091in}}%
\pgfpathclose%
\pgfusepath{stroke,fill}%
\end{pgfscope}%
\begin{pgfscope}%
\pgfsetrectcap%
\pgfsetroundjoin%
\pgfsetlinewidth{3.011250pt}%
\definecolor{currentstroke}{rgb}{0.000000,0.000000,0.000000}%
\pgfsetstrokecolor{currentstroke}%
\pgfsetdash{}{0pt}%
\pgfpathmoveto{\pgfqpoint{3.215237in}{4.856767in}}%
\pgfpathlineto{\pgfqpoint{3.493015in}{4.856767in}}%
\pgfpathlineto{\pgfqpoint{3.770793in}{4.856767in}}%
\pgfusepath{stroke}%
\end{pgfscope}%
\begin{pgfscope}%
\definecolor{textcolor}{rgb}{0.000000,0.000000,0.000000}%
\pgfsetstrokecolor{textcolor}%
\pgfsetfillcolor{textcolor}%
\pgftext[x=3.993015in,y=4.759545in,left,base]{\color{textcolor}\rmfamily\fontsize{20.000000}{24.000000}\selectfont Pareto Front}%
\end{pgfscope}%
\begin{pgfscope}%
\pgfsetbuttcap%
\pgfsetroundjoin%
\definecolor{currentfill}{rgb}{0.121569,0.466667,0.705882}%
\pgfsetfillcolor{currentfill}%
\pgfsetlinewidth{1.003750pt}%
\definecolor{currentstroke}{rgb}{0.121569,0.466667,0.705882}%
\pgfsetstrokecolor{currentstroke}%
\pgfsetdash{}{0pt}%
\pgfsys@defobject{currentmarker}{\pgfqpoint{-0.012028in}{-0.012028in}}{\pgfqpoint{0.012028in}{0.012028in}}{%
\pgfpathmoveto{\pgfqpoint{0.000000in}{-0.012028in}}%
\pgfpathcurveto{\pgfqpoint{0.003190in}{-0.012028in}}{\pgfqpoint{0.006250in}{-0.010761in}}{\pgfqpoint{0.008505in}{-0.008505in}}%
\pgfpathcurveto{\pgfqpoint{0.010761in}{-0.006250in}}{\pgfqpoint{0.012028in}{-0.003190in}}{\pgfqpoint{0.012028in}{0.000000in}}%
\pgfpathcurveto{\pgfqpoint{0.012028in}{0.003190in}}{\pgfqpoint{0.010761in}{0.006250in}}{\pgfqpoint{0.008505in}{0.008505in}}%
\pgfpathcurveto{\pgfqpoint{0.006250in}{0.010761in}}{\pgfqpoint{0.003190in}{0.012028in}}{\pgfqpoint{0.000000in}{0.012028in}}%
\pgfpathcurveto{\pgfqpoint{-0.003190in}{0.012028in}}{\pgfqpoint{-0.006250in}{0.010761in}}{\pgfqpoint{-0.008505in}{0.008505in}}%
\pgfpathcurveto{\pgfqpoint{-0.010761in}{0.006250in}}{\pgfqpoint{-0.012028in}{0.003190in}}{\pgfqpoint{-0.012028in}{0.000000in}}%
\pgfpathcurveto{\pgfqpoint{-0.012028in}{-0.003190in}}{\pgfqpoint{-0.010761in}{-0.006250in}}{\pgfqpoint{-0.008505in}{-0.008505in}}%
\pgfpathcurveto{\pgfqpoint{-0.006250in}{-0.010761in}}{\pgfqpoint{-0.003190in}{-0.012028in}}{\pgfqpoint{0.000000in}{-0.012028in}}%
\pgfpathlineto{\pgfqpoint{0.000000in}{-0.012028in}}%
\pgfpathclose%
\pgfusepath{stroke,fill}%
}%
\begin{pgfscope}%
\pgfsys@transformshift{3.493015in}{4.437505in}%
\pgfsys@useobject{currentmarker}{}%
\end{pgfscope}%
\end{pgfscope}%
\begin{pgfscope}%
\definecolor{textcolor}{rgb}{0.000000,0.000000,0.000000}%
\pgfsetstrokecolor{textcolor}%
\pgfsetfillcolor{textcolor}%
\pgftext[x=3.993015in,y=4.364588in,left,base]{\color{textcolor}\rmfamily\fontsize{20.000000}{24.000000}\selectfont Tested points}%
\end{pgfscope}%
\begin{pgfscope}%
\pgfsetbuttcap%
\pgfsetroundjoin%
\definecolor{currentfill}{rgb}{0.839216,0.152941,0.156863}%
\pgfsetfillcolor{currentfill}%
\pgfsetlinewidth{1.003750pt}%
\definecolor{currentstroke}{rgb}{0.839216,0.152941,0.156863}%
\pgfsetstrokecolor{currentstroke}%
\pgfsetdash{}{0pt}%
\pgfsys@defobject{currentmarker}{\pgfqpoint{-0.031056in}{-0.031056in}}{\pgfqpoint{0.031056in}{0.031056in}}{%
\pgfpathmoveto{\pgfqpoint{0.000000in}{-0.031056in}}%
\pgfpathcurveto{\pgfqpoint{0.008236in}{-0.031056in}}{\pgfqpoint{0.016136in}{-0.027784in}}{\pgfqpoint{0.021960in}{-0.021960in}}%
\pgfpathcurveto{\pgfqpoint{0.027784in}{-0.016136in}}{\pgfqpoint{0.031056in}{-0.008236in}}{\pgfqpoint{0.031056in}{0.000000in}}%
\pgfpathcurveto{\pgfqpoint{0.031056in}{0.008236in}}{\pgfqpoint{0.027784in}{0.016136in}}{\pgfqpoint{0.021960in}{0.021960in}}%
\pgfpathcurveto{\pgfqpoint{0.016136in}{0.027784in}}{\pgfqpoint{0.008236in}{0.031056in}}{\pgfqpoint{0.000000in}{0.031056in}}%
\pgfpathcurveto{\pgfqpoint{-0.008236in}{0.031056in}}{\pgfqpoint{-0.016136in}{0.027784in}}{\pgfqpoint{-0.021960in}{0.021960in}}%
\pgfpathcurveto{\pgfqpoint{-0.027784in}{0.016136in}}{\pgfqpoint{-0.031056in}{0.008236in}}{\pgfqpoint{-0.031056in}{0.000000in}}%
\pgfpathcurveto{\pgfqpoint{-0.031056in}{-0.008236in}}{\pgfqpoint{-0.027784in}{-0.016136in}}{\pgfqpoint{-0.021960in}{-0.021960in}}%
\pgfpathcurveto{\pgfqpoint{-0.016136in}{-0.027784in}}{\pgfqpoint{-0.008236in}{-0.031056in}}{\pgfqpoint{0.000000in}{-0.031056in}}%
\pgfpathlineto{\pgfqpoint{0.000000in}{-0.031056in}}%
\pgfpathclose%
\pgfusepath{stroke,fill}%
}%
\begin{pgfscope}%
\pgfsys@transformshift{3.493015in}{4.042548in}%
\pgfsys@useobject{currentmarker}{}%
\end{pgfscope}%
\end{pgfscope}%
\begin{pgfscope}%
\definecolor{textcolor}{rgb}{0.000000,0.000000,0.000000}%
\pgfsetstrokecolor{textcolor}%
\pgfsetfillcolor{textcolor}%
\pgftext[x=3.993015in,y=3.969632in,left,base]{\color{textcolor}\rmfamily\fontsize{20.000000}{24.000000}\selectfont Alternative solutions}%
\end{pgfscope}%
\begin{pgfscope}%
\pgfsetbuttcap%
\pgfsetmiterjoin%
\definecolor{currentfill}{rgb}{0.121569,0.466667,0.705882}%
\pgfsetfillcolor{currentfill}%
\pgfsetfillopacity{0.200000}%
\pgfsetlinewidth{0.000000pt}%
\definecolor{currentstroke}{rgb}{0.000000,0.000000,0.000000}%
\pgfsetstrokecolor{currentstroke}%
\pgfsetstrokeopacity{0.200000}%
\pgfsetdash{}{0pt}%
\pgfpathmoveto{\pgfqpoint{3.215237in}{3.558931in}}%
\pgfpathlineto{\pgfqpoint{3.770793in}{3.558931in}}%
\pgfpathlineto{\pgfqpoint{3.770793in}{3.753376in}}%
\pgfpathlineto{\pgfqpoint{3.215237in}{3.753376in}}%
\pgfpathlineto{\pgfqpoint{3.215237in}{3.558931in}}%
\pgfpathclose%
\pgfusepath{fill}%
\end{pgfscope}%
\begin{pgfscope}%
\pgfsetbuttcap%
\pgfsetmiterjoin%
\definecolor{currentfill}{rgb}{0.121569,0.466667,0.705882}%
\pgfsetfillcolor{currentfill}%
\pgfsetfillopacity{0.200000}%
\pgfsetlinewidth{0.000000pt}%
\definecolor{currentstroke}{rgb}{0.000000,0.000000,0.000000}%
\pgfsetstrokecolor{currentstroke}%
\pgfsetstrokeopacity{0.200000}%
\pgfsetdash{}{0pt}%
\pgfpathmoveto{\pgfqpoint{3.215237in}{3.558931in}}%
\pgfpathlineto{\pgfqpoint{3.770793in}{3.558931in}}%
\pgfpathlineto{\pgfqpoint{3.770793in}{3.753376in}}%
\pgfpathlineto{\pgfqpoint{3.215237in}{3.753376in}}%
\pgfpathlineto{\pgfqpoint{3.215237in}{3.558931in}}%
\pgfpathclose%
\pgfusepath{clip}%
\pgfsys@defobject{currentpattern}{\pgfqpoint{0in}{0in}}{\pgfqpoint{1in}{1in}}{%
\begin{pgfscope}%
\pgfpathrectangle{\pgfqpoint{0in}{0in}}{\pgfqpoint{1in}{1in}}%
\pgfusepath{clip}%
\pgfpathmoveto{\pgfqpoint{-0.500000in}{0.500000in}}%
\pgfpathlineto{\pgfqpoint{0.500000in}{1.500000in}}%
\pgfpathmoveto{\pgfqpoint{-0.416667in}{0.416667in}}%
\pgfpathlineto{\pgfqpoint{0.583333in}{1.416667in}}%
\pgfpathmoveto{\pgfqpoint{-0.333333in}{0.333333in}}%
\pgfpathlineto{\pgfqpoint{0.666667in}{1.333333in}}%
\pgfpathmoveto{\pgfqpoint{-0.250000in}{0.250000in}}%
\pgfpathlineto{\pgfqpoint{0.750000in}{1.250000in}}%
\pgfpathmoveto{\pgfqpoint{-0.166667in}{0.166667in}}%
\pgfpathlineto{\pgfqpoint{0.833333in}{1.166667in}}%
\pgfpathmoveto{\pgfqpoint{-0.083333in}{0.083333in}}%
\pgfpathlineto{\pgfqpoint{0.916667in}{1.083333in}}%
\pgfpathmoveto{\pgfqpoint{0.000000in}{0.000000in}}%
\pgfpathlineto{\pgfqpoint{1.000000in}{1.000000in}}%
\pgfpathmoveto{\pgfqpoint{0.083333in}{-0.083333in}}%
\pgfpathlineto{\pgfqpoint{1.083333in}{0.916667in}}%
\pgfpathmoveto{\pgfqpoint{0.166667in}{-0.166667in}}%
\pgfpathlineto{\pgfqpoint{1.166667in}{0.833333in}}%
\pgfpathmoveto{\pgfqpoint{0.250000in}{-0.250000in}}%
\pgfpathlineto{\pgfqpoint{1.250000in}{0.750000in}}%
\pgfpathmoveto{\pgfqpoint{0.333333in}{-0.333333in}}%
\pgfpathlineto{\pgfqpoint{1.333333in}{0.666667in}}%
\pgfpathmoveto{\pgfqpoint{0.416667in}{-0.416667in}}%
\pgfpathlineto{\pgfqpoint{1.416667in}{0.583333in}}%
\pgfpathmoveto{\pgfqpoint{0.500000in}{-0.500000in}}%
\pgfpathlineto{\pgfqpoint{1.500000in}{0.500000in}}%
\pgfusepath{stroke}%
\end{pgfscope}%
}%
\pgfsys@transformshift{3.215237in}{3.558931in}%
\pgfsys@useobject{currentpattern}{}%
\pgfsys@transformshift{1in}{0in}%
\pgfsys@transformshift{-1in}{0in}%
\pgfsys@transformshift{0in}{1in}%
\end{pgfscope}%
\begin{pgfscope}%
\definecolor{textcolor}{rgb}{0.000000,0.000000,0.000000}%
\pgfsetstrokecolor{textcolor}%
\pgfsetfillcolor{textcolor}%
\pgftext[x=3.993015in,y=3.558931in,left,base]{\color{textcolor}\rmfamily\fontsize{20.000000}{24.000000}\selectfont MGA Search Space (F1)}%
\end{pgfscope}%
\begin{pgfscope}%
\pgfsetbuttcap%
\pgfsetroundjoin%
\definecolor{currentfill}{rgb}{0.750000,0.750000,0.000000}%
\pgfsetfillcolor{currentfill}%
\pgfsetlinewidth{1.003750pt}%
\definecolor{currentstroke}{rgb}{0.750000,0.750000,0.000000}%
\pgfsetstrokecolor{currentstroke}%
\pgfsetdash{}{0pt}%
\pgfsys@defobject{currentmarker}{\pgfqpoint{-0.093403in}{-0.079453in}}{\pgfqpoint{0.093403in}{0.098209in}}{%
\pgfpathmoveto{\pgfqpoint{0.000000in}{0.098209in}}%
\pgfpathlineto{\pgfqpoint{-0.022049in}{0.030348in}}%
\pgfpathlineto{\pgfqpoint{-0.093403in}{0.030348in}}%
\pgfpathlineto{\pgfqpoint{-0.035677in}{-0.011592in}}%
\pgfpathlineto{\pgfqpoint{-0.057726in}{-0.079453in}}%
\pgfpathlineto{\pgfqpoint{-0.000000in}{-0.037513in}}%
\pgfpathlineto{\pgfqpoint{0.057726in}{-0.079453in}}%
\pgfpathlineto{\pgfqpoint{0.035677in}{-0.011592in}}%
\pgfpathlineto{\pgfqpoint{0.093403in}{0.030348in}}%
\pgfpathlineto{\pgfqpoint{0.022049in}{0.030348in}}%
\pgfpathlineto{\pgfqpoint{0.000000in}{0.098209in}}%
\pgfpathclose%
\pgfusepath{stroke,fill}%
}%
\begin{pgfscope}%
\pgfsys@transformshift{3.493015in}{3.221148in}%
\pgfsys@useobject{currentmarker}{}%
\end{pgfscope}%
\end{pgfscope}%
\begin{pgfscope}%
\definecolor{textcolor}{rgb}{0.000000,0.000000,0.000000}%
\pgfsetstrokecolor{textcolor}%
\pgfsetfillcolor{textcolor}%
\pgftext[x=3.993015in,y=3.148231in,left,base]{\color{textcolor}\rmfamily\fontsize{20.000000}{24.000000}\selectfont Intersecting Point}%
\end{pgfscope}%
\begin{pgfscope}%
\definecolor{textcolor}{rgb}{0.000000,0.000000,0.000000}%
\pgfsetstrokecolor{textcolor}%
\pgfsetfillcolor{textcolor}%
\pgftext[x=3.882794in,y=5.809583in,,top]{\color{textcolor}\rmfamily\fontsize{24.000000}{28.800000}\selectfont Multi-objective MGA}%
\end{pgfscope}%
\end{pgfpicture}%
\makeatother%
\endgroup%
}
            \caption{Alternative solutions identified in the near optimal space.}
            \label{fig:nd-alt-points}
        \end{figure}
    \end{columns}

\end{frame}