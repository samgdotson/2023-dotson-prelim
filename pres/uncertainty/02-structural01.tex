\begin{frame}
    \frametitle{Structural Uncertainty}

    \begin{columns}
        \column[t]{4cm}
        \begin{figure}
            \centering
            \resizebox{\columnwidth}{!}{
            \begin{tikzpicture}[nodes={text depth=0.25ex,text height=1.25ex distance=1.7cm}]
                    \tikzstyle{every node}=[font=\small]
                    \tikzstyle{vertex} = [circle, draw=black, fill=illiniblue]
                    \tikzstyle{hidden} = [draw=none]
                    \tikzstyle{edge} = [<->, very thick]
                    
                    % \node[vertex](v1) at (0,5) {\textbf{Normative}};
                    \node[vertex](v2) at (4,0) {\textbf{Structural}};

        
            \end{tikzpicture}
            }
            % \caption{Parametric Uncertainty}
            % \label{fig:triarchic-uncertainty}
        \end{figure}

        \column[t]{6cm}
        \begin{definition}[Structural Uncertainty]
            [R]efers to the imperfect and incomplete nature of the equations describing the system \cite{decarolis_using_2011}.
        \end{definition}
        
        This type of uncertainty will \textit{always} persist.
    \end{columns}

\end{frame}

\begin{frame}
    \frametitle{Examples of Structural Uncertainty}

    \begin{itemize}
        \item Objective functions (most typical)\pause
        \item Spatiotemporal resolution\pause
        \item Physics fidelity\pause
        \item Solution method
    \end{itemize}

\end{frame}

\begin{frame}
    \frametitle{Addressing Structural Uncertainty}

    

\end{frame}