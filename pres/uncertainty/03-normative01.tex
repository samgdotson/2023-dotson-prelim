\begin{frame}
    \frametitle{Normative Uncertainty}

    % Stating your assumptions is a necessary but insufficient condition for addressing normative uncertainty.

    \begin{columns}
        \column[t]{4cm}
        \begin{figure}
            \centering
            \resizebox{\columnwidth}{!}{
            \begin{tikzpicture}[nodes={text depth=0.25ex,text height=1.25ex distance=1.7cm}]
                    \tikzstyle{every node}=[font=\small]
                    \tikzstyle{vertex} = [circle, draw=black, fill=illiniblue]
                    \tikzstyle{hidden} = [draw=none]
                    \tikzstyle{edge} = [<->, very thick]
                    
                    \node[vertex](v1) at (0,5) {\textbf{Normative}};

        
            \end{tikzpicture}
            }
            % \caption{Parametric Uncertainty}
            % \label{fig:triarchic-uncertainty}
        \end{figure}

        \column[t]{6cm}
        \begin{block}{Normative Uncertainty}
            Arises from the plurality of morally defensible, but incompatible, choices;
            and a plurality of moral theories justifying those choices 
            \cite{taebi_governing_2020,van_uffelen_revisiting_2024}.
        \end{block}
        
    \end{columns}

\end{frame}

\begin{frame}
    \frametitle{Consequences of unacknowledged normative uncertainties}

    \begin{enumerate}
        \item Implicit normative premises cannot be debated,
        \item Precludes alternative formulations of \textit{justice},
        \item Raises doubts about legitimacy of conclusions.
    \end{enumerate}

\end{frame}


\begin{frame}
    \frametitle{Addressing Normative Uncertainty}

    % Related to the \boldorange{human dimension} of modeling systems \cite{pfenninger_energy_2014}. \\~\\ \pause

    % Why is this hard to include?\\~\\ \pause
    
    There are no formal methods to address normative uncertainty...\pause \textit{in engineering.}

\end{frame}