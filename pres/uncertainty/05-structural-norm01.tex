\begin{frame}
    \frametitle{Choosing among alternatives}
    \begin{columns}
        \column[t]{6cm}
        
        \textit{Generating prescriptive conclusions is the primary reason to
        model energy systems \cite{decarolis_using_2011}.}\\~\\

        % If the solution to structural uncertainty was identifying alternative,
        % ``sub-optimal'' solutions, then the prescriptive stage means deciding
        % among these diverse alternatives.\\~\\ 

        \begin{block}{Arrow's Impossibility Theorem}
            It is impossible to construct a utility function that maps
            individual preferences onto a global preference order without
            imposition or dictating \cite{kasprzyk_many_2013,
            franssen_arrows_2005,arrow_difficulty_1950}.
        \end{block}

        \column[t]{4cm}
        \begin{figure}
            \centering
            \resizebox{\columnwidth}{!}{
            \begin{tikzpicture}[nodes={text depth=0.25ex,text height=1.25ex distance=1.7cm}]
                    \tikzstyle{every node}=[font=\small] \tikzstyle{vertex} =
                    [circle, draw=black, fill=illiniblue] \tikzstyle{hidden} =
                    [draw=none] \tikzstyle{edge} = [<->, very thick]
                    
                    \node[vertex](v1) at (0,5) {\textbf{Normative}};
                    \node[vertex](v2) at (4,0) {\textbf{Structural}};

                    \draw[edge] (v1) -- (v2);

                    \node[hidden](h2) at (0.75, 5) {};
        
                    % % hidden nodes for v2
                    \node[hidden](h3) at (4, 0.75) {};

                    \draw[draw=none] (h2) -- (h3) node[anchor=mid, midway,
                    sloped]{\textbf{Prescriptive}};
    
            \end{tikzpicture}
            }
            % \caption{Parametric Uncertainty} \label{fig:triarchic-uncertainty}
        \end{figure}
        
    \end{columns}

\end{frame}

\begin{frame}
    \frametitle{Consequences of Arrow's Theorem}

    \begin{enumerate}
        \item There is no one-size-fits-all method for public engagement or
        decision-making.
        \item The methods of engagement must ``open up'' debate rather than
        ``close it down'' \cite{wilsdon_see-through_2004,dryzek_deliberative_2013}. 
        \item Ideals of justice and ``just outcomes'' can never be adequately
        captured by an aggregated ``metric'' --- this would imply a utility
        function that could map individual preferences to a collective
        preference.
    \end{enumerate}

\end{frame}

\begin{frame}<0>
    \frametitle{Potential Pitfalls}

    \begin{enumerate}
        \item Reproducing errors of ``public understanding of science'' and the
        ``deficit model'' \cite{wynne_misunderstood_1992,wynne_public_2006}.
    \end{enumerate}

\end{frame}

% \begin{frame} \frametitle{Purpose of Multiobjective Methods}

%     \textit{The second purpose of multiobjective methods is to help
%     participants in the planning process define and articulate their values,
%     apply them rationally and consistently, and document the resuits. The
%     object is to inspire confidence in the soundness of the decision without
%     being unnecessarily difficult. Multiobjective methods used in this manner
%     can also help negotiation, by quantifying and communicating the priorities
%     held by different interests \cite{hobbs_optimization_1995}.} \\~\\

%     Although the usefulness of these methods were recognized long ago, the
%     application of these methods was stunted by computational tools and data
%     visualization capabilities.\\~\\

%     Prior articulation methods vs interactive methods. 
    %  Expanding on this 
%         idea, multiobjective optimization ``help people to understand the
%         problem better, explore their feelings, form a coherent, defensible set
%         of values, and understand the implications of those values for the
%         decision'' \cite{hobbs_optimization_1995}. 

% \end{frame}

% \begin{frame} \frametitle{Connecting to Energy Justice}

%     The three-legged description of energy justice, recognition, procedure,
%     and distribution, describes the necessary ingredients for a ``just''
%     outcome. However, there are many articulations for each of these three
%     types of justice. This type of plurality makes it challenging to define a
%     universally ``just'' outcome.

%     The way forward is through deliberation. This is why Arrow's theorem runs
%     between normative and structural uncertainties. Choosing a set of
%     objectives to model is ultimately a normative choice. Even if this choice
%     were clear, the result remains normatively uncertain because modelers and
%     stakeholders must then choose among alternatives identified by the model.
%     Thus, these two are in dialogue with each other.

% \end{frame}