\begin{frame}
    \frametitle{Prescriptive: Structural-Normative}
    \begin{columns}
        \column[t]{6cm}
        
        Generating prescriptive conclusions is the primary reason to model energy systems.\\~\\

        If the solution to structural uncertainty was identifying alternative, ``sub-optimal'' 
        solutions, then the prescriptive stage means deciding among these diverse alternatives.

        \begin{theorem}[Arrow's Impossibility Theorem]
            It is impossible to construct a utility function that maps individual preferences 
            onto a global preference order without imposition or dictating \cite{kasprzyk_many_2013,
            franssen_arrows_2005,arrow_difficulty_1950}.
        \end{theorem}

        \column[t]{4cm}
        \begin{figure}
            \centering
            \resizebox{\columnwidth}{!}{
            \begin{tikzpicture}[nodes={text depth=0.25ex,text height=1.25ex distance=1.7cm}]
                    \tikzstyle{every node}=[font=\small]
                    \tikzstyle{vertex} = [circle, draw=black, fill=illiniblue]
                    \tikzstyle{hidden} = [draw=none]
                    \tikzstyle{edge} = [<->, very thick]
                    
                    \node[vertex](v1) at (0,5) {\textbf{Normative}};
                    \node[vertex](v2) at (4,0) {\textbf{Structural}};

                    \draw[edge] (v1) -- (v2);

                    \node[hidden](h2) at (0.75, 5) {};
        
                    % % hidden nodes for v2
                    \node[hidden](h3) at (4, 0.75) {};

                    \draw[draw=none] (h2) -- (h3) node[anchor=mid, midway, sloped]{\textbf{Prescriptive}};
        
        
            \end{tikzpicture}
            }
            % \caption{Parametric Uncertainty}
            % \label{fig:triarchic-uncertainty}
        \end{figure}
        
    \end{columns}

\end{frame}