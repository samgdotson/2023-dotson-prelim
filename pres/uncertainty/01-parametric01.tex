\begin{frame}
    \frametitle{Parametric Uncertainty}
    \begin{columns}
        \column[t]{4cm}
        \begin{figure}
            \centering
            \resizebox{\columnwidth}{!}{
            \begin{tikzpicture}[nodes={text depth=0.25ex,text height=1.25ex distance=1.7cm}]
                    \tikzstyle{every node}=[font=\small] \tikzstyle{vertex} =
                    [circle, draw=black, fill=illiniblue] \tikzstyle{hidden} =
                    [draw=none] \tikzstyle{edge} = [<->, very thick]

                    \node[vertex](v3) at (-4,0) {\textbf{Parametric}};
        
            \end{tikzpicture}
            }
            % \caption{Parametric Uncertainty} \label{fig:triarchic-uncertainty}
        \end{figure}

        \column[t]{6cm}
        \begin{block}{Parametric Uncertainty}
            Related to uncertainty in model inputs (empirical values). The most
            commonly addressed type of uncertainty in science and engineering
            \cite{yue_review_2018,decarolis_using_2011,morgan_uncertainty_1990}.    
        \end{block}
        
        % May be classified as either \boldorange{aleatory} or
        % \boldorange{epistemic}
        % \cite{pfenninger_energy_2014,kiureghian_aleatory_2009}. 
    \end{columns}

\end{frame}

\begin{frame}
    \frametitle{Examples of Parametric Uncertainty}

    \begin{columns}
        \column[t]{5cm}
        \begin{figure}
            \centering
            \resizebox{\columnwidth}{!}{
                \input{../docs/figures/multiple-distributions.pgf}
            }
            \caption{Possible distributions of several parameters.}
            \label{fig:multi-distributions}
        \end{figure}

        \column[t]{5cm}

        \begin{itemize}
            \item \textcolor{black}{Rates (e.g., interest, learning, growth),}
            \item \textcolor{black}{costs (e.g., fuel, capital, O\&M),}
            \item \textcolor{black}{aggregated energy demand,}
            \item \textcolor{black}{spent fuel burnup
            \cite{feng_sensitivity_2020},}
            \item \textcolor{black}{nuclear cross-section data
            \cite{eades_influence_2016,radaideh_combining_2019},}
            \item \textcolor{black}{likelihood and magnitude of consequences
            (i.e., probabilistic risk assessment).}
        \end{itemize}

        % Note: \textbf{\textcolor{illiniblue}{aleatory}} and
        % \textbf{\textcolor{illiniorange}{epistemic}}
    \end{columns}

\end{frame}

% \begin{frame}
%     \frametitle{Consequences of not addressing parametric uncertainty}

%     A majority of ESOM articles use \textit{scenario analysis} to weakly ddress
%     parametric uncertainty \cite{yue_review_2018}. \\~\\

%     Leading to: 
%     \begin{enumerate}
%         \item Overconfidence in results
%         \item Cognitive myopia
%         \item Implicit normative biases
%     \end{enumerate}


% \end{frame}

% \begin{frame}
%     \frametitle{Addressing parametric uncertainty}

%     \begin{columns}
%         \column[t]{5cm}
%         \begin{figure}
%             \centering
%             \resizebox{\columnwidth}{!}{
%                 %% Creator: Matplotlib, PGF backend
%%
%% To include the figure in your LaTeX document, write
%%   \input{<filename>.pgf}
%%
%% Make sure the required packages are loaded in your preamble
%%   \usepackage{pgf}
%%
%% Also ensure that all the required font packages are loaded; for instance,
%% the lmodern package is sometimes necessary when using math font.
%%   \usepackage{lmodern}
%%
%% Figures using additional raster images can only be included by \input if
%% they are in the same directory as the main LaTeX file. For loading figures
%% from other directories you can use the `import` package
%%   \usepackage{import}
%%
%% and then include the figures with
%%   \import{<path to file>}{<filename>.pgf}
%%
%% Matplotlib used the following preamble
%%   
%%   \makeatletter\@ifpackageloaded{underscore}{}{\usepackage[strings]{underscore}}\makeatother
%%
\begingroup%
\makeatletter%
\begin{pgfpicture}%
\pgfpathrectangle{\pgfpointorigin}{\pgfqpoint{7.492359in}{5.900571in}}%
\pgfusepath{use as bounding box, clip}%
\begin{pgfscope}%
\pgfsetbuttcap%
\pgfsetmiterjoin%
\definecolor{currentfill}{rgb}{0.827451,0.827451,0.827451}%
\pgfsetfillcolor{currentfill}%
\pgfsetlinewidth{0.000000pt}%
\definecolor{currentstroke}{rgb}{0.000000,0.000000,0.000000}%
\pgfsetstrokecolor{currentstroke}%
\pgfsetdash{}{0pt}%
\pgfpathmoveto{\pgfqpoint{0.000000in}{0.000000in}}%
\pgfpathlineto{\pgfqpoint{7.492359in}{0.000000in}}%
\pgfpathlineto{\pgfqpoint{7.492359in}{5.900571in}}%
\pgfpathlineto{\pgfqpoint{0.000000in}{5.900571in}}%
\pgfpathlineto{\pgfqpoint{0.000000in}{0.000000in}}%
\pgfpathclose%
\pgfusepath{fill}%
\end{pgfscope}%
\begin{pgfscope}%
\pgfsetbuttcap%
\pgfsetmiterjoin%
\definecolor{currentfill}{rgb}{1.000000,1.000000,1.000000}%
\pgfsetfillcolor{currentfill}%
\pgfsetlinewidth{0.000000pt}%
\definecolor{currentstroke}{rgb}{0.000000,0.000000,0.000000}%
\pgfsetstrokecolor{currentstroke}%
\pgfsetstrokeopacity{0.000000}%
\pgfsetdash{}{0pt}%
\pgfpathmoveto{\pgfqpoint{0.417359in}{0.410571in}}%
\pgfpathlineto{\pgfqpoint{7.392359in}{0.410571in}}%
\pgfpathlineto{\pgfqpoint{7.392359in}{5.800571in}}%
\pgfpathlineto{\pgfqpoint{0.417359in}{5.800571in}}%
\pgfpathlineto{\pgfqpoint{0.417359in}{0.410571in}}%
\pgfpathclose%
\pgfusepath{fill}%
\end{pgfscope}%
\begin{pgfscope}%
\pgfpathrectangle{\pgfqpoint{0.417359in}{0.410571in}}{\pgfqpoint{6.975000in}{5.390000in}}%
\pgfusepath{clip}%
\pgfsetbuttcap%
\pgfsetroundjoin%
\definecolor{currentfill}{rgb}{0.501961,0.501961,0.501961}%
\pgfsetfillcolor{currentfill}%
\pgfsetfillopacity{0.400000}%
\pgfsetlinewidth{1.003750pt}%
\definecolor{currentstroke}{rgb}{0.501961,0.501961,0.501961}%
\pgfsetstrokecolor{currentstroke}%
\pgfsetstrokeopacity{0.400000}%
\pgfsetdash{}{0pt}%
\pgfsys@defobject{currentmarker}{\pgfqpoint{0.734404in}{1.296968in}}{\pgfqpoint{7.075313in}{5.160653in}}{%
\pgfpathmoveto{\pgfqpoint{0.734404in}{2.579761in}}%
\pgfpathlineto{\pgfqpoint{0.734404in}{1.296968in}}%
\pgfpathlineto{\pgfqpoint{1.368495in}{2.035436in}}%
\pgfpathlineto{\pgfqpoint{2.002586in}{2.669991in}}%
\pgfpathlineto{\pgfqpoint{2.636677in}{3.190642in}}%
\pgfpathlineto{\pgfqpoint{3.270768in}{3.592433in}}%
\pgfpathlineto{\pgfqpoint{3.904859in}{3.888659in}}%
\pgfpathlineto{\pgfqpoint{4.538950in}{4.102111in}}%
\pgfpathlineto{\pgfqpoint{5.173040in}{4.248955in}}%
\pgfpathlineto{\pgfqpoint{5.807131in}{4.337330in}}%
\pgfpathlineto{\pgfqpoint{6.441222in}{4.370818in}}%
\pgfpathlineto{\pgfqpoint{7.075313in}{4.350541in}}%
\pgfpathlineto{\pgfqpoint{7.075313in}{5.129627in}}%
\pgfpathlineto{\pgfqpoint{7.075313in}{5.129627in}}%
\pgfpathlineto{\pgfqpoint{6.441222in}{5.160653in}}%
\pgfpathlineto{\pgfqpoint{5.807131in}{5.109522in}}%
\pgfpathlineto{\pgfqpoint{5.173040in}{4.977358in}}%
\pgfpathlineto{\pgfqpoint{4.538950in}{4.767740in}}%
\pgfpathlineto{\pgfqpoint{3.904859in}{4.488809in}}%
\pgfpathlineto{\pgfqpoint{3.270768in}{4.156730in}}%
\pgfpathlineto{\pgfqpoint{2.636677in}{3.794295in}}%
\pgfpathlineto{\pgfqpoint{2.002586in}{3.414798in}}%
\pgfpathlineto{\pgfqpoint{1.368495in}{3.013284in}}%
\pgfpathlineto{\pgfqpoint{0.734404in}{2.579761in}}%
\pgfpathlineto{\pgfqpoint{0.734404in}{2.579761in}}%
\pgfpathclose%
\pgfusepath{stroke,fill}%
}%
\begin{pgfscope}%
\pgfsys@transformshift{0.000000in}{0.000000in}%
\pgfsys@useobject{currentmarker}{}%
\end{pgfscope}%
\end{pgfscope}%
\begin{pgfscope}%
\pgfpathrectangle{\pgfqpoint{0.417359in}{0.410571in}}{\pgfqpoint{6.975000in}{5.390000in}}%
\pgfusepath{clip}%
\pgfsetbuttcap%
\pgfsetroundjoin%
\definecolor{currentfill}{rgb}{0.501961,0.501961,0.501961}%
\pgfsetfillcolor{currentfill}%
\pgfsetfillopacity{0.200000}%
\pgfsetlinewidth{1.003750pt}%
\definecolor{currentstroke}{rgb}{0.501961,0.501961,0.501961}%
\pgfsetstrokecolor{currentstroke}%
\pgfsetstrokeopacity{0.200000}%
\pgfsetdash{}{0pt}%
\pgfsys@defobject{currentmarker}{\pgfqpoint{0.734404in}{0.655571in}}{\pgfqpoint{7.075313in}{5.555571in}}{%
\pgfpathmoveto{\pgfqpoint{0.734404in}{3.221158in}}%
\pgfpathlineto{\pgfqpoint{0.734404in}{0.655571in}}%
\pgfpathlineto{\pgfqpoint{1.368495in}{1.546512in}}%
\pgfpathlineto{\pgfqpoint{2.002586in}{2.297588in}}%
\pgfpathlineto{\pgfqpoint{2.636677in}{2.888815in}}%
\pgfpathlineto{\pgfqpoint{3.270768in}{3.310285in}}%
\pgfpathlineto{\pgfqpoint{3.904859in}{3.588584in}}%
\pgfpathlineto{\pgfqpoint{4.538950in}{3.769297in}}%
\pgfpathlineto{\pgfqpoint{5.173040in}{3.884753in}}%
\pgfpathlineto{\pgfqpoint{5.807131in}{3.951234in}}%
\pgfpathlineto{\pgfqpoint{6.441222in}{3.975900in}}%
\pgfpathlineto{\pgfqpoint{7.075313in}{3.960998in}}%
\pgfpathlineto{\pgfqpoint{7.075313in}{5.519170in}}%
\pgfpathlineto{\pgfqpoint{7.075313in}{5.519170in}}%
\pgfpathlineto{\pgfqpoint{6.441222in}{5.555571in}}%
\pgfpathlineto{\pgfqpoint{5.807131in}{5.495618in}}%
\pgfpathlineto{\pgfqpoint{5.173040in}{5.341559in}}%
\pgfpathlineto{\pgfqpoint{4.538950in}{5.100554in}}%
\pgfpathlineto{\pgfqpoint{3.904859in}{4.788883in}}%
\pgfpathlineto{\pgfqpoint{3.270768in}{4.438878in}}%
\pgfpathlineto{\pgfqpoint{2.636677in}{4.096121in}}%
\pgfpathlineto{\pgfqpoint{2.002586in}{3.787201in}}%
\pgfpathlineto{\pgfqpoint{1.368495in}{3.502208in}}%
\pgfpathlineto{\pgfqpoint{0.734404in}{3.221158in}}%
\pgfpathlineto{\pgfqpoint{0.734404in}{3.221158in}}%
\pgfpathclose%
\pgfusepath{stroke,fill}%
}%
\begin{pgfscope}%
\pgfsys@transformshift{0.000000in}{0.000000in}%
\pgfsys@useobject{currentmarker}{}%
\end{pgfscope}%
\end{pgfscope}%
\begin{pgfscope}%
\pgfsetbuttcap%
\pgfsetroundjoin%
\definecolor{currentfill}{rgb}{0.000000,0.000000,0.000000}%
\pgfsetfillcolor{currentfill}%
\pgfsetlinewidth{0.803000pt}%
\definecolor{currentstroke}{rgb}{0.000000,0.000000,0.000000}%
\pgfsetstrokecolor{currentstroke}%
\pgfsetdash{}{0pt}%
\pgfsys@defobject{currentmarker}{\pgfqpoint{0.000000in}{-0.048611in}}{\pgfqpoint{0.000000in}{0.000000in}}{%
\pgfpathmoveto{\pgfqpoint{0.000000in}{0.000000in}}%
\pgfpathlineto{\pgfqpoint{0.000000in}{-0.048611in}}%
\pgfusepath{stroke,fill}%
}%
\begin{pgfscope}%
\pgfsys@transformshift{0.734404in}{0.410571in}%
\pgfsys@useobject{currentmarker}{}%
\end{pgfscope}%
\end{pgfscope}%
\begin{pgfscope}%
\definecolor{textcolor}{rgb}{0.000000,0.000000,0.000000}%
\pgfsetstrokecolor{textcolor}%
\pgfsetfillcolor{textcolor}%
\pgftext[x=0.734404in,y=0.313349in,,top]{\color{textcolor}\rmfamily\fontsize{16.000000}{19.200000}\selectfont \(\displaystyle {0}\)}%
\end{pgfscope}%
\begin{pgfscope}%
\pgfsetbuttcap%
\pgfsetroundjoin%
\definecolor{currentfill}{rgb}{0.000000,0.000000,0.000000}%
\pgfsetfillcolor{currentfill}%
\pgfsetlinewidth{0.803000pt}%
\definecolor{currentstroke}{rgb}{0.000000,0.000000,0.000000}%
\pgfsetstrokecolor{currentstroke}%
\pgfsetdash{}{0pt}%
\pgfsys@defobject{currentmarker}{\pgfqpoint{0.000000in}{-0.048611in}}{\pgfqpoint{0.000000in}{0.000000in}}{%
\pgfpathmoveto{\pgfqpoint{0.000000in}{0.000000in}}%
\pgfpathlineto{\pgfqpoint{0.000000in}{-0.048611in}}%
\pgfusepath{stroke,fill}%
}%
\begin{pgfscope}%
\pgfsys@transformshift{2.002586in}{0.410571in}%
\pgfsys@useobject{currentmarker}{}%
\end{pgfscope}%
\end{pgfscope}%
\begin{pgfscope}%
\definecolor{textcolor}{rgb}{0.000000,0.000000,0.000000}%
\pgfsetstrokecolor{textcolor}%
\pgfsetfillcolor{textcolor}%
\pgftext[x=2.002586in,y=0.313349in,,top]{\color{textcolor}\rmfamily\fontsize{16.000000}{19.200000}\selectfont \(\displaystyle {2}\)}%
\end{pgfscope}%
\begin{pgfscope}%
\pgfsetbuttcap%
\pgfsetroundjoin%
\definecolor{currentfill}{rgb}{0.000000,0.000000,0.000000}%
\pgfsetfillcolor{currentfill}%
\pgfsetlinewidth{0.803000pt}%
\definecolor{currentstroke}{rgb}{0.000000,0.000000,0.000000}%
\pgfsetstrokecolor{currentstroke}%
\pgfsetdash{}{0pt}%
\pgfsys@defobject{currentmarker}{\pgfqpoint{0.000000in}{-0.048611in}}{\pgfqpoint{0.000000in}{0.000000in}}{%
\pgfpathmoveto{\pgfqpoint{0.000000in}{0.000000in}}%
\pgfpathlineto{\pgfqpoint{0.000000in}{-0.048611in}}%
\pgfusepath{stroke,fill}%
}%
\begin{pgfscope}%
\pgfsys@transformshift{3.270768in}{0.410571in}%
\pgfsys@useobject{currentmarker}{}%
\end{pgfscope}%
\end{pgfscope}%
\begin{pgfscope}%
\definecolor{textcolor}{rgb}{0.000000,0.000000,0.000000}%
\pgfsetstrokecolor{textcolor}%
\pgfsetfillcolor{textcolor}%
\pgftext[x=3.270768in,y=0.313349in,,top]{\color{textcolor}\rmfamily\fontsize{16.000000}{19.200000}\selectfont \(\displaystyle {4}\)}%
\end{pgfscope}%
\begin{pgfscope}%
\pgfsetbuttcap%
\pgfsetroundjoin%
\definecolor{currentfill}{rgb}{0.000000,0.000000,0.000000}%
\pgfsetfillcolor{currentfill}%
\pgfsetlinewidth{0.803000pt}%
\definecolor{currentstroke}{rgb}{0.000000,0.000000,0.000000}%
\pgfsetstrokecolor{currentstroke}%
\pgfsetdash{}{0pt}%
\pgfsys@defobject{currentmarker}{\pgfqpoint{0.000000in}{-0.048611in}}{\pgfqpoint{0.000000in}{0.000000in}}{%
\pgfpathmoveto{\pgfqpoint{0.000000in}{0.000000in}}%
\pgfpathlineto{\pgfqpoint{0.000000in}{-0.048611in}}%
\pgfusepath{stroke,fill}%
}%
\begin{pgfscope}%
\pgfsys@transformshift{4.538950in}{0.410571in}%
\pgfsys@useobject{currentmarker}{}%
\end{pgfscope}%
\end{pgfscope}%
\begin{pgfscope}%
\definecolor{textcolor}{rgb}{0.000000,0.000000,0.000000}%
\pgfsetstrokecolor{textcolor}%
\pgfsetfillcolor{textcolor}%
\pgftext[x=4.538950in,y=0.313349in,,top]{\color{textcolor}\rmfamily\fontsize{16.000000}{19.200000}\selectfont \(\displaystyle {6}\)}%
\end{pgfscope}%
\begin{pgfscope}%
\pgfsetbuttcap%
\pgfsetroundjoin%
\definecolor{currentfill}{rgb}{0.000000,0.000000,0.000000}%
\pgfsetfillcolor{currentfill}%
\pgfsetlinewidth{0.803000pt}%
\definecolor{currentstroke}{rgb}{0.000000,0.000000,0.000000}%
\pgfsetstrokecolor{currentstroke}%
\pgfsetdash{}{0pt}%
\pgfsys@defobject{currentmarker}{\pgfqpoint{0.000000in}{-0.048611in}}{\pgfqpoint{0.000000in}{0.000000in}}{%
\pgfpathmoveto{\pgfqpoint{0.000000in}{0.000000in}}%
\pgfpathlineto{\pgfqpoint{0.000000in}{-0.048611in}}%
\pgfusepath{stroke,fill}%
}%
\begin{pgfscope}%
\pgfsys@transformshift{5.807131in}{0.410571in}%
\pgfsys@useobject{currentmarker}{}%
\end{pgfscope}%
\end{pgfscope}%
\begin{pgfscope}%
\definecolor{textcolor}{rgb}{0.000000,0.000000,0.000000}%
\pgfsetstrokecolor{textcolor}%
\pgfsetfillcolor{textcolor}%
\pgftext[x=5.807131in,y=0.313349in,,top]{\color{textcolor}\rmfamily\fontsize{16.000000}{19.200000}\selectfont \(\displaystyle {8}\)}%
\end{pgfscope}%
\begin{pgfscope}%
\pgfsetbuttcap%
\pgfsetroundjoin%
\definecolor{currentfill}{rgb}{0.000000,0.000000,0.000000}%
\pgfsetfillcolor{currentfill}%
\pgfsetlinewidth{0.803000pt}%
\definecolor{currentstroke}{rgb}{0.000000,0.000000,0.000000}%
\pgfsetstrokecolor{currentstroke}%
\pgfsetdash{}{0pt}%
\pgfsys@defobject{currentmarker}{\pgfqpoint{0.000000in}{-0.048611in}}{\pgfqpoint{0.000000in}{0.000000in}}{%
\pgfpathmoveto{\pgfqpoint{0.000000in}{0.000000in}}%
\pgfpathlineto{\pgfqpoint{0.000000in}{-0.048611in}}%
\pgfusepath{stroke,fill}%
}%
\begin{pgfscope}%
\pgfsys@transformshift{7.075313in}{0.410571in}%
\pgfsys@useobject{currentmarker}{}%
\end{pgfscope}%
\end{pgfscope}%
\begin{pgfscope}%
\definecolor{textcolor}{rgb}{0.000000,0.000000,0.000000}%
\pgfsetstrokecolor{textcolor}%
\pgfsetfillcolor{textcolor}%
\pgftext[x=7.075313in,y=0.313349in,,top]{\color{textcolor}\rmfamily\fontsize{16.000000}{19.200000}\selectfont \(\displaystyle {10}\)}%
\end{pgfscope}%
\begin{pgfscope}%
\pgfsetbuttcap%
\pgfsetroundjoin%
\definecolor{currentfill}{rgb}{0.000000,0.000000,0.000000}%
\pgfsetfillcolor{currentfill}%
\pgfsetlinewidth{0.803000pt}%
\definecolor{currentstroke}{rgb}{0.000000,0.000000,0.000000}%
\pgfsetstrokecolor{currentstroke}%
\pgfsetdash{}{0pt}%
\pgfsys@defobject{currentmarker}{\pgfqpoint{-0.048611in}{0.000000in}}{\pgfqpoint{-0.000000in}{0.000000in}}{%
\pgfpathmoveto{\pgfqpoint{-0.000000in}{0.000000in}}%
\pgfpathlineto{\pgfqpoint{-0.048611in}{0.000000in}}%
\pgfusepath{stroke,fill}%
}%
\begin{pgfscope}%
\pgfsys@transformshift{0.417359in}{1.072419in}%
\pgfsys@useobject{currentmarker}{}%
\end{pgfscope}%
\end{pgfscope}%
\begin{pgfscope}%
\definecolor{textcolor}{rgb}{0.000000,0.000000,0.000000}%
\pgfsetstrokecolor{textcolor}%
\pgfsetfillcolor{textcolor}%
\pgftext[x=0.210068in, y=0.989085in, left, base]{\color{textcolor}\rmfamily\fontsize{16.000000}{19.200000}\selectfont \(\displaystyle {4}\)}%
\end{pgfscope}%
\begin{pgfscope}%
\pgfsetbuttcap%
\pgfsetroundjoin%
\definecolor{currentfill}{rgb}{0.000000,0.000000,0.000000}%
\pgfsetfillcolor{currentfill}%
\pgfsetlinewidth{0.803000pt}%
\definecolor{currentstroke}{rgb}{0.000000,0.000000,0.000000}%
\pgfsetstrokecolor{currentstroke}%
\pgfsetdash{}{0pt}%
\pgfsys@defobject{currentmarker}{\pgfqpoint{-0.048611in}{0.000000in}}{\pgfqpoint{-0.000000in}{0.000000in}}{%
\pgfpathmoveto{\pgfqpoint{-0.000000in}{0.000000in}}%
\pgfpathlineto{\pgfqpoint{-0.048611in}{0.000000in}}%
\pgfusepath{stroke,fill}%
}%
\begin{pgfscope}%
\pgfsys@transformshift{0.417359in}{1.864679in}%
\pgfsys@useobject{currentmarker}{}%
\end{pgfscope}%
\end{pgfscope}%
\begin{pgfscope}%
\definecolor{textcolor}{rgb}{0.000000,0.000000,0.000000}%
\pgfsetstrokecolor{textcolor}%
\pgfsetfillcolor{textcolor}%
\pgftext[x=0.210068in, y=1.781345in, left, base]{\color{textcolor}\rmfamily\fontsize{16.000000}{19.200000}\selectfont \(\displaystyle {6}\)}%
\end{pgfscope}%
\begin{pgfscope}%
\pgfsetbuttcap%
\pgfsetroundjoin%
\definecolor{currentfill}{rgb}{0.000000,0.000000,0.000000}%
\pgfsetfillcolor{currentfill}%
\pgfsetlinewidth{0.803000pt}%
\definecolor{currentstroke}{rgb}{0.000000,0.000000,0.000000}%
\pgfsetstrokecolor{currentstroke}%
\pgfsetdash{}{0pt}%
\pgfsys@defobject{currentmarker}{\pgfqpoint{-0.048611in}{0.000000in}}{\pgfqpoint{-0.000000in}{0.000000in}}{%
\pgfpathmoveto{\pgfqpoint{-0.000000in}{0.000000in}}%
\pgfpathlineto{\pgfqpoint{-0.048611in}{0.000000in}}%
\pgfusepath{stroke,fill}%
}%
\begin{pgfscope}%
\pgfsys@transformshift{0.417359in}{2.656939in}%
\pgfsys@useobject{currentmarker}{}%
\end{pgfscope}%
\end{pgfscope}%
\begin{pgfscope}%
\definecolor{textcolor}{rgb}{0.000000,0.000000,0.000000}%
\pgfsetstrokecolor{textcolor}%
\pgfsetfillcolor{textcolor}%
\pgftext[x=0.210068in, y=2.573605in, left, base]{\color{textcolor}\rmfamily\fontsize{16.000000}{19.200000}\selectfont \(\displaystyle {8}\)}%
\end{pgfscope}%
\begin{pgfscope}%
\pgfsetbuttcap%
\pgfsetroundjoin%
\definecolor{currentfill}{rgb}{0.000000,0.000000,0.000000}%
\pgfsetfillcolor{currentfill}%
\pgfsetlinewidth{0.803000pt}%
\definecolor{currentstroke}{rgb}{0.000000,0.000000,0.000000}%
\pgfsetstrokecolor{currentstroke}%
\pgfsetdash{}{0pt}%
\pgfsys@defobject{currentmarker}{\pgfqpoint{-0.048611in}{0.000000in}}{\pgfqpoint{-0.000000in}{0.000000in}}{%
\pgfpathmoveto{\pgfqpoint{-0.000000in}{0.000000in}}%
\pgfpathlineto{\pgfqpoint{-0.048611in}{0.000000in}}%
\pgfusepath{stroke,fill}%
}%
\begin{pgfscope}%
\pgfsys@transformshift{0.417359in}{3.449199in}%
\pgfsys@useobject{currentmarker}{}%
\end{pgfscope}%
\end{pgfscope}%
\begin{pgfscope}%
\definecolor{textcolor}{rgb}{0.000000,0.000000,0.000000}%
\pgfsetstrokecolor{textcolor}%
\pgfsetfillcolor{textcolor}%
\pgftext[x=0.100000in, y=3.365865in, left, base]{\color{textcolor}\rmfamily\fontsize{16.000000}{19.200000}\selectfont \(\displaystyle {10}\)}%
\end{pgfscope}%
\begin{pgfscope}%
\pgfsetbuttcap%
\pgfsetroundjoin%
\definecolor{currentfill}{rgb}{0.000000,0.000000,0.000000}%
\pgfsetfillcolor{currentfill}%
\pgfsetlinewidth{0.803000pt}%
\definecolor{currentstroke}{rgb}{0.000000,0.000000,0.000000}%
\pgfsetstrokecolor{currentstroke}%
\pgfsetdash{}{0pt}%
\pgfsys@defobject{currentmarker}{\pgfqpoint{-0.048611in}{0.000000in}}{\pgfqpoint{-0.000000in}{0.000000in}}{%
\pgfpathmoveto{\pgfqpoint{-0.000000in}{0.000000in}}%
\pgfpathlineto{\pgfqpoint{-0.048611in}{0.000000in}}%
\pgfusepath{stroke,fill}%
}%
\begin{pgfscope}%
\pgfsys@transformshift{0.417359in}{4.241459in}%
\pgfsys@useobject{currentmarker}{}%
\end{pgfscope}%
\end{pgfscope}%
\begin{pgfscope}%
\definecolor{textcolor}{rgb}{0.000000,0.000000,0.000000}%
\pgfsetstrokecolor{textcolor}%
\pgfsetfillcolor{textcolor}%
\pgftext[x=0.100000in, y=4.158125in, left, base]{\color{textcolor}\rmfamily\fontsize{16.000000}{19.200000}\selectfont \(\displaystyle {12}\)}%
\end{pgfscope}%
\begin{pgfscope}%
\pgfsetbuttcap%
\pgfsetroundjoin%
\definecolor{currentfill}{rgb}{0.000000,0.000000,0.000000}%
\pgfsetfillcolor{currentfill}%
\pgfsetlinewidth{0.803000pt}%
\definecolor{currentstroke}{rgb}{0.000000,0.000000,0.000000}%
\pgfsetstrokecolor{currentstroke}%
\pgfsetdash{}{0pt}%
\pgfsys@defobject{currentmarker}{\pgfqpoint{-0.048611in}{0.000000in}}{\pgfqpoint{-0.000000in}{0.000000in}}{%
\pgfpathmoveto{\pgfqpoint{-0.000000in}{0.000000in}}%
\pgfpathlineto{\pgfqpoint{-0.048611in}{0.000000in}}%
\pgfusepath{stroke,fill}%
}%
\begin{pgfscope}%
\pgfsys@transformshift{0.417359in}{5.033719in}%
\pgfsys@useobject{currentmarker}{}%
\end{pgfscope}%
\end{pgfscope}%
\begin{pgfscope}%
\definecolor{textcolor}{rgb}{0.000000,0.000000,0.000000}%
\pgfsetstrokecolor{textcolor}%
\pgfsetfillcolor{textcolor}%
\pgftext[x=0.100000in, y=4.950385in, left, base]{\color{textcolor}\rmfamily\fontsize{16.000000}{19.200000}\selectfont \(\displaystyle {14}\)}%
\end{pgfscope}%
\begin{pgfscope}%
\pgfpathrectangle{\pgfqpoint{0.417359in}{0.410571in}}{\pgfqpoint{6.975000in}{5.390000in}}%
\pgfusepath{clip}%
\pgfsetrectcap%
\pgfsetroundjoin%
\pgfsetlinewidth{1.505625pt}%
\definecolor{currentstroke}{rgb}{0.121569,0.466667,0.705882}%
\pgfsetstrokecolor{currentstroke}%
\pgfsetdash{}{0pt}%
\pgfpathmoveto{\pgfqpoint{0.734404in}{1.938364in}}%
\pgfpathlineto{\pgfqpoint{1.368495in}{2.524360in}}%
\pgfpathlineto{\pgfqpoint{2.002586in}{3.042394in}}%
\pgfpathlineto{\pgfqpoint{2.636677in}{3.492468in}}%
\pgfpathlineto{\pgfqpoint{3.270768in}{3.874581in}}%
\pgfpathlineto{\pgfqpoint{3.904859in}{4.188734in}}%
\pgfpathlineto{\pgfqpoint{4.538950in}{4.434925in}}%
\pgfpathlineto{\pgfqpoint{5.173040in}{4.613156in}}%
\pgfpathlineto{\pgfqpoint{5.807131in}{4.723426in}}%
\pgfpathlineto{\pgfqpoint{6.441222in}{4.765735in}}%
\pgfpathlineto{\pgfqpoint{7.075313in}{4.740084in}}%
\pgfusepath{stroke}%
\end{pgfscope}%
\begin{pgfscope}%
\pgfsetrectcap%
\pgfsetmiterjoin%
\pgfsetlinewidth{0.803000pt}%
\definecolor{currentstroke}{rgb}{0.000000,0.000000,0.000000}%
\pgfsetstrokecolor{currentstroke}%
\pgfsetdash{}{0pt}%
\pgfpathmoveto{\pgfqpoint{0.417359in}{0.410571in}}%
\pgfpathlineto{\pgfqpoint{0.417359in}{5.800571in}}%
\pgfusepath{stroke}%
\end{pgfscope}%
\begin{pgfscope}%
\pgfsetrectcap%
\pgfsetmiterjoin%
\pgfsetlinewidth{0.803000pt}%
\definecolor{currentstroke}{rgb}{0.000000,0.000000,0.000000}%
\pgfsetstrokecolor{currentstroke}%
\pgfsetdash{}{0pt}%
\pgfpathmoveto{\pgfqpoint{7.392359in}{0.410571in}}%
\pgfpathlineto{\pgfqpoint{7.392359in}{5.800571in}}%
\pgfusepath{stroke}%
\end{pgfscope}%
\begin{pgfscope}%
\pgfsetrectcap%
\pgfsetmiterjoin%
\pgfsetlinewidth{0.803000pt}%
\definecolor{currentstroke}{rgb}{0.000000,0.000000,0.000000}%
\pgfsetstrokecolor{currentstroke}%
\pgfsetdash{}{0pt}%
\pgfpathmoveto{\pgfqpoint{0.417359in}{0.410571in}}%
\pgfpathlineto{\pgfqpoint{7.392359in}{0.410571in}}%
\pgfusepath{stroke}%
\end{pgfscope}%
\begin{pgfscope}%
\pgfsetrectcap%
\pgfsetmiterjoin%
\pgfsetlinewidth{0.803000pt}%
\definecolor{currentstroke}{rgb}{0.000000,0.000000,0.000000}%
\pgfsetstrokecolor{currentstroke}%
\pgfsetdash{}{0pt}%
\pgfpathmoveto{\pgfqpoint{0.417359in}{5.800571in}}%
\pgfpathlineto{\pgfqpoint{7.392359in}{5.800571in}}%
\pgfusepath{stroke}%
\end{pgfscope}%
\end{pgfpicture}%
\makeatother%
\endgroup%

%             }
%             \caption{Systematically addressing parametric uncertainty produces confidence intervals.}
%             \label{fig:confidence-intervals}
%         \end{figure}

%         \column[t]{5cm}
%         \boldblue{Idea}: Rerun a simulation until you reach a large enough
%         sample size to do statistics.\\~\\
%         Formal methods to address parametric uncertainty$^*$:
%         \begin{itemize}
%             \item ``Monte Carlo'' (i.e., statistical sampling)
%             \item Sensitivity analysis (specific or global)
%             \item Stochastic optimization
%         \end{itemize}


%         % \small{$^*$These methods are appropriate for \boldorange{aleatory}
%         % uncertainties.}
%     \end{columns}

% \end{frame}
