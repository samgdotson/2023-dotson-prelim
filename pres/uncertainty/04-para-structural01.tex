\begin{frame}
    \frametitle{Descriptive: Parametric-Structural}

    This is where the ``research question'' lives. 

    \begin{itemize}
        \item What is being modeled (i.e. what are the in/dependent variables)?
        \item How are time series represented? (e.g., weather / demand data)? % normative if motivated by an unstated normative premise.
        \item Which technologies are included in the simulation? % normative, especially if certain technologies are excluded without explanation.
        \item What is the spatiotemporal scale/resolution of the model? % likely normative if not considering intergenerational justice.
    \end{itemize}


    \begin{figure}
        \centering
        \resizebox{0.60\columnwidth}{!}{
        \begin{tikzpicture}[nodes={text depth=0.25ex,text height=1.25ex distance=1.7cm}]
                \tikzstyle{every node}=[font=\small]
                \tikzstyle{vertex} = [circle, draw=black, fill=illiniblue]
                \tikzstyle{hidden} = [draw=none]
                \tikzstyle{edge} = [<->, very thick]
                
                % \node[vertex](v1) at (0,5) {\textbf{Normative}};
                \node[vertex](v2) at (4,0) {\textbf{Structural}};
                \node[vertex](v3) at (-4,0) {\textbf{Parametric}};
    
                % \draw[edge] (v1) -- (v2);
                \draw[edge] (v2) -- (v3);

                \node[hidden](h4) at (4, -0.7) {};
    
                % % hidden nodes for v3
                \node[hidden](h5) at (-4, -0.7) {};
                % \node[hidden](h6) at (-4, 0.75) {};
    
                \draw[draw=none] (h4) -- (h5) node[anchor=mid, midway, sloped]{\textbf{Descriptive}};
    
    
        \end{tikzpicture}
        }
        % \caption{Parametric Uncertainty}
        % \label{fig:triarchic-uncertainty}
    \end{figure}

\end{frame}