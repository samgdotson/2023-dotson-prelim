% \begin{frame}
%     \frametitle{Gap \#3: Overcoming Arrow's Theorem}

%     \begin{enumerate}
%         \item Deciding among alternative solutions is challenging without a normative premise.
%         \item Without direct consultation of stakeholders, it's impossible know how they would understand tradeoffs.
%         \item Capturing the ``human dimension'' requires incorporating formal methods from social science: case studies,
%         interviews, focus groups, surveys, etc. The ESOM literature struggles to do this.
%     \end{enumerate}

% \end{frame}
\begin{frame}
    \frametitle{Gap 2: Normative Uncertainty \& Deliberative Processes}
    \begin{block}{Technical Gap}
        \begin{enumerate}
            \item Deciding among alternative solutions is challenging without a normative premise.
            \item Without direct consultation of stakeholders, it's impossible know how they would understand tradeoffs.
            \item Capturing the ``human dimension'' requires incorporating formal methods from social science: case studies,
            interviews, focus groups, surveys, etc. The ESOM literature struggles to do this \cite{pfenninger_energy_2014}.
        \end{enumerate}
    \end{block}
    \begin{block}{Proposed Work Component II: Integrative theory of uncertainties}
        Further develop the unifying theory of model development through the lens of 
        addressing triple uncertainties.
    \end{block}
    \begin{block}{Proposed Work Component III: Case study of Champaign-Urbana}
        Case study of energy planning processes in the Champaign-Urbana region to validate
        the usefulness of \texttt{Osier} and test the salience of various uncertainties.
    \end{block}
    
\end{frame}