\chapter{Proposed Work}


\begin{enumerate}
    \item How will I expand the theoretical basis of this work? Or, rather, what
    theoretical areas will my dissertation explore and expand?
    \begin{itemize}
        \item Provide a detailed outline of the nuclear energy debate.
        \item Give a case study on the ways traditional decision making
        processes in nuclear energy (e.g., Yucca Mountain) led to poor outcomes,
        and show positive examples of inclusive processes (i.e., consent-based
        siting) producing superior outcomes.
        \item Provide a detailed outline for the challenges associated with
        solar and wind.
        \item How does the literature answer the question of ``what drives
        opposition to new energy projects?'' (It's not \ac{nimybism}).
        \item Social movement theory and democratic processes
        \item \textcolor{red}{Constructing a ``normative premise'' that situates
        and clearly articulates the usefulness of this work.}
    \end{itemize}
    \item What technical features have yet to be implemented into \ac{osier}?
    \begin{itemize}
        \item Issue: The current code is unacceptably slow for many-objective
        problems. The four objective model took 26.5 days to run.
        \textbf{Solution}: More work needs to be done on investigating the
        parallelizability of \texttt{CPLEX}. Alternatively, rather than using a
        \ac{milp} model to operate dispatch, using hiearchical model (i.e., a
        ``rules'' based model) to dispatch energy could enable multiple
        processes, reducing the computational cost of the problem. Additionally,
        this type of model is conceptually simpler than \ac{milp} and would make
        \ac{osier} more accessible.
        \item Issue: Data transparency and quality assurance. \textbf{Solution}:
        Quality data is essential to generating trustworthy results. The
        \ac{atb} produced by \ac{nrel} is considered the gold standard for cost
        projections for electricity generating technologies. \ac{osier} will
        directly integrate data from the \ac{atb}.
        \item \textcolor{red}{Note: You should emphasize the novelty of your MGA
        algorithm during your prelim.}
    \end{itemize}
    \item What will I do to ``validate'' framework I created (\ac{osier})?
    \begin{itemize}
        \item One use of \ac{osier} is to develop an energy future for a given
        community. 
        \item I propose validating \ac{osier}'s usefulness with a counterfactual
        case study of the energy visioning processes of three municipalities:
        Urbana, Champaign, and the \ac{uiuc}.
        \item The questions of interest are:
        \begin{itemize}
            \item Would these municipalities use this tool?
            \item What are the energy priorities of these municipalities?
            \item How would employing this tool differ from existing visioning
            strategies or processes?
            \item What are the existing planning practices?
            \item Should there be more collective planning among
        \end{itemize}
        \item The results of this analysis could take the form of:
        \begin{itemize}
            \item ``These are the stated energy goals of this municipality.''
            \item ``This is how the results from \ac{osier} would change the
            results.'' (Results in this case includes the ``optimal'' solutions
            for an energy future \textit{and} the ways the process itself would
            change.)
        \end{itemize}
        \item The methodology for this study includes:
        \begin{itemize}
            \item Interviewing public facing figures at each municipality
            involved in the respective planning processes. These subjects could
            include:
            \begin{itemize}
                \item director of energy planning and public engagement
                \item organizations such as \ac{isee}, SECS, and SSC. 
            \end{itemize}
            \item Reviewing published documents related to their energy visions.
            Such as \ac{uiuc}'s \ac{icap}.
        \end{itemize}
        
    \end{itemize}
\end{enumerate}