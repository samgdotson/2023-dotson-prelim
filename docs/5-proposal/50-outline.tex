\chapter{Proposed Work}


\begin{enumerate}
    \item How will I expand the theoretical basis of this work? Or, rather, what
    theoretical areas will my dissertation explore and expand?
    \begin{itemize}
        \item Provide a detailed outline of the nuclear energy debate.
        \item Give a case study on the ways traditional decision making processes in 
        nuclear energy (e.g., Yucca Mountain) led to poor outcomes, and show positive
        examples of inclusive processes (i.e., consent-based siting) producing superior
        outcomes.
        \item Provide a detailed outline for the challenges associated with solar and wind.
        \item How does the literature answer the question of ``what drives opposition to new 
        energy projects?'' (It's not \ac{nimybism}).
        \item Social movement theory and democratic processes
    \end{itemize}
    \item What technical features have yet to be implemented into \ac{osier}?
    \begin{itemize}
        \item Issue: The current code is unacceptably slow for many-objective
        problems. The four objective model took 26.5 days to run. \textbf{Solution}:
        More work needs to be done on investigating the parallelizability of \texttt{CPLEX}.
        Alternatively, rather than using a \ac{milp} model to operate dispatch, using
        hiearchical model (i.e., a ``rules'' based model) to dispatch energy could enable
        multiple processes, reducing the computational cost of the problem. Additionally,
        this type of model is conceptually simpler than \ac{milp} and would make \ac{osier}
        more accessible.
        \item Issue: Data transparency and quality assurance. \textbf{Solution}: Quality data
        is essential to generating trustworthy results. The \ac{atb} produced by \ac{nrel} is 
        considered the gold standard for cost projections for electricity generating technologies.
        \ac{osier} will directly integrate data from the \ac{atb}.
        \item \textcolor{red}{Note: You should emphasize the novelty of your MGA algorithm during your prelim.}
    \end{itemize}
    \item What will I do to ``validate'' framework I created (\ac{osier})?
\end{enumerate}