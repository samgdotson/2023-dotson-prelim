\chapter{Proposed Work}

The literature review in Chapter \ref{chapter:lit-review} characterized the
wicked problem of climate change, identified the current gaps in \ac{esom}
methods, and motivated the need to incorporate ideas from a energy justice and
other non-engineering discplines, in order to fully apprehend the challenge.
Chapter \ref{chapter:methods} detailed the development of \ac{osier}, a novel
\ac{esom} framework designed to incorporate conceptions of energy justice. This
chapter outlines the future work to deepen the theoretical foundation of this
thesis, improve \ac{osier}'s functionality, and validate \ac{osier} as a useful
framework for enhancing decision-making processes for more just outcomes.

\section{Expanding the theoretical basis of this work}

This section elaborates on how I will expand the theoretical basis of this
thesis.

How will I expand the theoretical basis of this work? Or, rather, what
theoretical areas will my dissertation explore and expand?
\begin{enumerate}
    \item Provide a detailed outline of the nuclear energy debate. What are the
    normative assumptions at play on both sides?
    \item Give a case study on the ways traditional decision making processes in
    nuclear energy (e.g., Yucca Mountain) led to poor outcomes, and show
    positive examples of inclusive processes (i.e., consent-based siting)
    producing superior outcomes.
    \item Provide a detailed outline for the challenges associated with solar
    and wind.
    \item How does the literature answer the question of ``what drives
    opposition to new energy projects?'' (It's not \ac{nimbyism}).
    \item Social movement theory and democratic processes
    \item Address ``normative uncertainty'' \cite{van_uffelen_revisiting_2024}
    and how it relates to ``structural uncertainty''
    \cite{decarolis_using_2011}.
    \item \textcolor{red}{Constructing a ``normative premise'' that situates and
    clearly articulates the usefulness of this work. I.e., ``define what
    `justice,' or `just outcomes' means for this work''}
\end{enumerate}


\subsection{What drives opposition to new energy projects?}
Observing the dissonance between the awareness of anthropogenic climate change
and policy actions to mitigate the effects of climate change is one of the key
motivators for this work. Further, in instances where action is being taken ---
such as the construction of renewable energy projects following government
subsidies, for instance --- what drives public opposition? I will elucidate this
question by incorporating literature from social movement theory
\cite{mcadam_social_2017,mcadam_putting_2012} into this thesis. Importantly, the literature shows
that \ac{nimbyism} is not the primary driver of public opposition to energy
projects \cite{konisky_proximity_2021}, rather, support for these energy
projects is more strongly conditioned on genuine public participation in the
decision-making process \cite{summers_influencing_2020,ottinger_procedural_2014,
walker_procedural_2017,barragan-contreras_procedural_2022,gonyo_resident_2021}.


\section{Improvements to \ac{osier}}

What technical features have yet to be implemented into \ac{osier}?
\begin{enumerate}
    \item Issue: The current code is unacceptably slow for many-objective
    problems. The four objective model took 26.5 days to run. \textbf{Solution}:
    More work needs to be done on investigating the parallelizability of
    \texttt{CPLEX}. Alternatively, rather than using a \ac{milp} model to
    operate dispatch, using hiearchical model (i.e., a ``rules'' based model) to
    dispatch energy could enable multiple processes, reducing the computational
    cost of the problem. Additionally, this type of model is conceptually
    simpler than \ac{milp} and would make \ac{osier} more accessible.
    \item Issue: Data transparency and quality assurance. \textbf{Solution}:
    Quality data is essential to generating trustworthy results. The \ac{atb}
    produced by \ac{nrel} is considered the gold standard for cost projections
    for electricity generating technologies. \ac{osier} will directly integrate
    data from the \ac{atb}.
    \item \textcolor{red}{Note: You should emphasize the novelty of your MGA
    algorithm during your prelim.}
    \item Issue: The \ac{mga} algorithm could yet be improved by developing a selection
    strategy that more accurately captures the spirit of \ac{mga} by identifying 
    \textit{maximally different solutions in the design space.} \textbf{Solution}:
    This could be accomplished through an algorithm known as greedy permutation, or 
    farthest-first-search, from computational geometry that is frequently used in 
    topological data analysis \cite{cavanna_geometric_2015,eppstein_approximate_2020}.
\end{enumerate}

\section{Validating \ac{osier}}

\ac{osier}'s primary purpose is to translate policy preferences of the
lay-public into actionable energy visions for a given municipality. The idea is
that if decision-makers used a tool like \ac{osier} to support their decisions
and incorporate ideas from their constituents --- ideas that may be distinct
from the preconceptions of decision-makers themselves --- then stronger actions
toward addressing climate change may be taken with more just outcomes. The last,
and arguably most important, component of this thesis is to validate
\ac{osier}'s usefulness in this regard. I propose validating \ac{osier}'s
usefulness with a case study of the energy visioning processes of three
municipalities: Urbana, Champaign, and the \ac{uiuc}.

\subsection{Reviewing the energy visions for each municipality}

Reviewing published documents related to their energy visions. Such as
\ac{uiuc}'s \ac{icap}
\cite{institute_for_sustainability_energy_and_environment_illinois_2020}.

Compare the stated energy visions for each community with the literature
\cite{elmallah_frontlining_2022}.

In addition to energy \textit{visions}, I may evaluate the decision making
process for specific energy projects. Such as the microreactor project at
\ac{uiuc} or the recent solar farm being constructed on Market Street in
Champaign. How do these projects fit with the stated energy visions for their
respective communities? How involved was the lay-public in making these
decisions? Who were the stakeholders?

\subsection{Develop the hypothetical \ac{osier} procedure}

Before interviewing decision-makers and asking if \ac{osier} would be useful to
them I need to formulate a hypothetical decision-making process that includes
\ac{osier}.

\subsection{Determining the right set of interviewees}

In order to gauge the \ac{osier}'s usability, I will interview public facing
figures from each municipality involved in the energy or community planning
processes for their respective communities.


\begin{table}[ht!]
    \centering
    \caption{Potential interviewees to evaluate the usefulness of \ac{osier}.}
    \begin{tabular}[pos]{lllllll}
    Name & Title & Affiliation & \multicolumn{4}{c}{Note} \\
    \toprule
    Bruce A. Knight & Planning \& Development Director & City of Champaign & \multicolumn{4}{l}{Directs the Champaign Planning department}\\
    Lacey Rains Lowe & Senior Planner for Advanced Planning & City of Champaign & \multicolumn{4}{l}{Participated in creating the Champaign sustainability plan}\\
    Jeremy Guest && \ac{uiuc} &&&&\\
    Maddhu Khanna && \ac{uiuc} &&&&\\
    Luis Rodir\'iguez && \ac{uiuc} &&&&\\
    Kevin Garcia & Principal Planner & City of Urbana &&&&\\
    &&&&&&\\
    &&&&&&\\
    &&&&&&\\
    &&&&&&\\
    \bottomrule
\end{tabular}
    \label{table:subjects}
\end{table}


\subsection{Conducting the Interviews}

This section outlines the proposed interview questions and 

\textit{Interviews will be conducted and analyzed with the awareness that the
questions I ask, my choice of wording, and my demeanor when asking, all have an
affect on the answers generated by interviewees.}

\subsubsection{Questions about current planning processes}
The following questions are aimed at interviewees responsible for guiding the
planning process in each community. For example, they may be urban planners or
in charge of public engagement. These questions are not appropriate for an
external party that may be involved in the execution of an specific energy goal
but are nonetheless excluded from the initial decision making process.

\begin{enumerate}
    \item How would you describe your community's energy vision or priorities?
    \item How did your community develop these priorities (or this vision)?
    \item Does your community have current best practices for making planning
    decisions?
    \item How does your community use modeling software to support its vision,
    if at all?
    \item What are the pain points you experience in developing these visions?
    \item What is the role of expert testimony/consultation/input in creating an
    energy vision?
    \item How do you percieve the dialogue between community members and its
    decision-makers?
    \item Does it seem like preferences or concerns from the community are
    incorporated? 
    \item Do members of your community understand why a particular decision was
    reached?
    \item How does the energy visioning process in your community consider the
    impact on its neighbors?
    \item Should there be more collective planning among the three communities?
\end{enumerate}

\subsubsection{Questions and discussion about modeling tools and \ac{osier}} The
previous set of questions set a foundation for energy planning from the
perspective of a decision-maker or planner. 

\begin{enumerate}
    \item (After presenting results from this model in two ways) Which
    presentation do you think would best facilitate dialogue between the
    lay-public and decision-makers?
    \item What objectives do you think would be important to model in designing
    an energy vision for your community?
    \item Would your municipality use this tool? 
    \item If not this specific tool, is there a tool that exists/doesn't exist
    that is/would be useful?
    \item What changes would need to be made to \ac{osier} for it to be useful
    to you?
    \item How would employing this tool differ from existing visioning
    strategies or processes?
\end{enumerate}

\textit{\textcolor{red}{Insert a table or list of questions and describe what
information will be gleaned from each.}}

The interviews will be recorded and I will use the responses from interviewees
to conduct a thematic analysis
\cite{braun_toward_2023,maguire_doing_2017,scharp_what_2019}.

What will I do to ``validate'' framework I created (\ac{osier})?
\begin{enumerate}
    \item The questions of interest are (\textcolor{red}{for each question, what
    information would I gain by asking it?}):

    \item The results of this analysis could take the form of:
    \begin{itemize}
        \item ``These are the stated energy goals of this municipality.''
        \item ``This is how the results from \ac{osier} would change the
        results.'' (Results in this case includes the ``optimal'' solutions for
        an energy future \textit{and} the ways the process itself would change.)
    \end{itemize}
    \item The methodology for this study includes:
    \begin{itemize}
        \item Interviewing public facing figures at each municipality involved
        in the respective planning processes. These subjects could include:
        \begin{itemize}
            \item director of energy planning and public engagement
            \item organizations such as \ac{isee}, SECS, and SSC. 
            \item members of the \ac{uiuc} microreactor project (e.g., Prof.
            Caleb Brooks, Prof. Tomasz Kozlowski)
        \end{itemize}
        \item Analyzing the human-subjects data (interviews) with thematic
        analysis
    \end{itemize}
    \item What is outside the scope of this study? Although the intention behind
    \ac{osier} is to help translate policy preferences of the lay-public into
    actionable energy visions for a given municipality, interviewing and
    surveying members of the public is out-of-scope for this project. Further,
    it goes without saying that developing an effective framework for the
    purposes of procedural justice that determining the needs of a group before
    developing the code in earnest.
    \item Also outside the scope of this project is determining which form of
    the results is most accessible to lay-audiences. Ideally with a much larger
    number of participants I could do a/b testing to tease out an answer to this
    question.
\end{enumerate}