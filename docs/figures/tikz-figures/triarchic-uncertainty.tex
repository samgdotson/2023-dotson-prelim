\begin{figure}[ht]
    \centering
    \begin{tikzpicture}[nodes={text depth=0.25ex,text height=1.25ex distance=1.7cm}]
            \tikzstyle{every node}=[font=\small]
            \tikzstyle{vertex} = [circle, draw=black, fill=illiniblue]
            \tikzstyle{hidden} = [draw=none]
            \tikzstyle{edge} = [<->, very thick]
            
            \node[vertex](v1) at (0,5) {\textbf{Normative}};
            \node[vertex](v2) at (4,0) {\textbf{Structural}};
            \node[vertex](v3) at (-4,0) {\textbf{Parametric}};

            \draw[edge] (v1) -- (v2);
            \draw[edge] (v2) -- (v3);
            \draw[edge] (v1) -- (v3);

            % hidden nodes for v1
            \node[hidden](h1) at (-0.75, 5) {};
            \node[hidden](h2) at (0.75, 5) {};

            % hidden nodes for v2
            \node[hidden](h3) at (4, 0.75) {};
            \node[hidden](h4) at (4, -0.7) {};

            % hidden nodes for v3
            \node[hidden](h5) at (-4, -0.7) {};
            \node[hidden](h6) at (-4, 0.75) {};

            \draw[draw=none] (h4) -- (h5) node[anchor=mid, midway, sloped]{\textbf{Descriptive}};
            \draw[draw=none] (h6) -- (h1) node[anchor=mid, midway, sloped]{\textbf{Pre-descriptive}};
            \draw[draw=none] (h2) -- (h3) node[anchor=mid, midway, sloped]{\textbf{Prescriptive}};


            % objectivity scale
            \node[hidden](u1) at (6,5) {\textbf{Subjective}};
            \node[hidden](u2) at (6,0) {\textbf{Objective}};
            \draw[edge] (u1) -- (u2);


    \end{tikzpicture}
    \caption{Types of uncertainties and how their relationships inform modeling practice.}
    \label{fig:triarchic-uncertainty}
\end{figure}