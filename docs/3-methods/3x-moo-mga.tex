\section{\acs{mga} with \acl{moo}}
\label{section:mga-moo}

This thesis applies some ideas from \ac{mga} to the analysis of the sub-optimal
space from a \acl{moo} problem. Due to their iterative process, \acp{ga}
naturally generate many samples in a problem's feasible space. However, this
does not lead to a ``limited set'' of solutions but rather a potentially
infinite set. Some literature developed \acp{ga} that directly use \ac{mga} in
the iterative process
\cite{zechman_evolutionary_2004,zechman_evolutionary_2013}. However, existing
Python libraries such as \ac{pymoo} and \ac{deap} do not implement these
methods, and the challenge is not an inability to sample the sub-optimal space,
but rather to provide a comprehensible subset of solutions. The algorithm I
developed in this thesis to search the near-feasible space is the following:

\begin{enumerate}
    \item Identify a slack value (e.g., 10\% or 0.1). 
    \item Create a ``near-feasible front'' where the coordinates of each point
    are multiplied by unity plus the slack value.
    \item Every individual is checked if all of its coordinates are
    \begin{itemize}
        \item below all of the coordinates for at least one point on the
        near-feasible front and
        \item above all of the coordinates for at least one point on the Pareto
        front.
    \end{itemize}  
    \item Lastly, a subset of points may be randomly sampled from the interior
    points for analysis.
\end{enumerate}
\noindent
Figure \ref{fig:nd-mga} and Figure \ref{fig:3d-mga} demonstrate this algorithm
with 10 percent slack for a 2-D and 3-D Pareto front, respectively. Figure
\ref{fig:nd-mga} shows clearly that only points within the near-optimal space
(gray) are considered. Illustrating this behavior in three dimensions (and
above) is considerably more difficult. The 3-D interior points should be covered
by both surfaces, obstructing their view. Figure \ref{fig:3d-mga} shows that
this is the case in three panels. First, a top view of an opaque Pareto front
(green) where no interior points can be observed. Second, the same view with a
translucent Pareto front, revealing interior points and the near-optimal front
(blue). Finally, the view from underneath the near-optimal front once again
obscures the interior points, except for two near the edges of the sub-optimal
space. The tested points are omitted for clarity.
 

\begin{figure}[h]
  \centering
  \resizebox{0.6\columnwidth}{!}{%% Creator: Matplotlib, PGF backend
%%
%% To include the figure in your LaTeX document, write
%%   \input{<filename>.pgf}
%%
%% Make sure the required packages are loaded in your preamble
%%   \usepackage{pgf}
%%
%% Also ensure that all the required font packages are loaded; for instance,
%% the lmodern package is sometimes necessary when using math font.
%%   \usepackage{lmodern}
%%
%% Figures using additional raster images can only be included by \input if
%% they are in the same directory as the main LaTeX file. For loading figures
%% from other directories you can use the `import` package
%%   \usepackage{import}
%%
%% and then include the figures with
%%   \import{<path to file>}{<filename>.pgf}
%%
%% Matplotlib used the following preamble
%%
\begingroup%
\makeatletter%
\begin{pgfpicture}%
\pgfpathrectangle{\pgfpointorigin}{\pgfqpoint{7.147223in}{5.232237in}}%
\pgfusepath{use as bounding box, clip}%
\begin{pgfscope}%
\pgfsetbuttcap%
\pgfsetmiterjoin%
\definecolor{currentfill}{rgb}{1.000000,1.000000,1.000000}%
\pgfsetfillcolor{currentfill}%
\pgfsetlinewidth{0.000000pt}%
\definecolor{currentstroke}{rgb}{0.000000,0.000000,0.000000}%
\pgfsetstrokecolor{currentstroke}%
\pgfsetdash{}{0pt}%
\pgfpathmoveto{\pgfqpoint{0.000000in}{0.000000in}}%
\pgfpathlineto{\pgfqpoint{7.147223in}{0.000000in}}%
\pgfpathlineto{\pgfqpoint{7.147223in}{5.232237in}}%
\pgfpathlineto{\pgfqpoint{0.000000in}{5.232237in}}%
\pgfpathlineto{\pgfqpoint{0.000000in}{0.000000in}}%
\pgfpathclose%
\pgfusepath{fill}%
\end{pgfscope}%
\begin{pgfscope}%
\pgfsetbuttcap%
\pgfsetmiterjoin%
\definecolor{currentfill}{rgb}{1.000000,1.000000,1.000000}%
\pgfsetfillcolor{currentfill}%
\pgfsetlinewidth{0.000000pt}%
\definecolor{currentstroke}{rgb}{0.000000,0.000000,0.000000}%
\pgfsetstrokecolor{currentstroke}%
\pgfsetstrokeopacity{0.000000}%
\pgfsetdash{}{0pt}%
\pgfpathmoveto{\pgfqpoint{0.847223in}{0.554012in}}%
\pgfpathlineto{\pgfqpoint{7.047223in}{0.554012in}}%
\pgfpathlineto{\pgfqpoint{7.047223in}{5.084012in}}%
\pgfpathlineto{\pgfqpoint{0.847223in}{5.084012in}}%
\pgfpathlineto{\pgfqpoint{0.847223in}{0.554012in}}%
\pgfpathclose%
\pgfusepath{fill}%
\end{pgfscope}%
\begin{pgfscope}%
\pgfpathrectangle{\pgfqpoint{0.847223in}{0.554012in}}{\pgfqpoint{6.200000in}{4.530000in}}%
\pgfusepath{clip}%
\pgfsetbuttcap%
\pgfsetroundjoin%
\definecolor{currentfill}{rgb}{0.121569,0.466667,0.705882}%
\pgfsetfillcolor{currentfill}%
\pgfsetlinewidth{1.003750pt}%
\definecolor{currentstroke}{rgb}{0.121569,0.466667,0.705882}%
\pgfsetstrokecolor{currentstroke}%
\pgfsetdash{}{0pt}%
\pgfsys@defobject{currentmarker}{\pgfqpoint{-0.012028in}{-0.012028in}}{\pgfqpoint{0.012028in}{0.012028in}}{%
\pgfpathmoveto{\pgfqpoint{0.000000in}{-0.012028in}}%
\pgfpathcurveto{\pgfqpoint{0.003190in}{-0.012028in}}{\pgfqpoint{0.006250in}{-0.010761in}}{\pgfqpoint{0.008505in}{-0.008505in}}%
\pgfpathcurveto{\pgfqpoint{0.010761in}{-0.006250in}}{\pgfqpoint{0.012028in}{-0.003190in}}{\pgfqpoint{0.012028in}{0.000000in}}%
\pgfpathcurveto{\pgfqpoint{0.012028in}{0.003190in}}{\pgfqpoint{0.010761in}{0.006250in}}{\pgfqpoint{0.008505in}{0.008505in}}%
\pgfpathcurveto{\pgfqpoint{0.006250in}{0.010761in}}{\pgfqpoint{0.003190in}{0.012028in}}{\pgfqpoint{0.000000in}{0.012028in}}%
\pgfpathcurveto{\pgfqpoint{-0.003190in}{0.012028in}}{\pgfqpoint{-0.006250in}{0.010761in}}{\pgfqpoint{-0.008505in}{0.008505in}}%
\pgfpathcurveto{\pgfqpoint{-0.010761in}{0.006250in}}{\pgfqpoint{-0.012028in}{0.003190in}}{\pgfqpoint{-0.012028in}{0.000000in}}%
\pgfpathcurveto{\pgfqpoint{-0.012028in}{-0.003190in}}{\pgfqpoint{-0.010761in}{-0.006250in}}{\pgfqpoint{-0.008505in}{-0.008505in}}%
\pgfpathcurveto{\pgfqpoint{-0.006250in}{-0.010761in}}{\pgfqpoint{-0.003190in}{-0.012028in}}{\pgfqpoint{0.000000in}{-0.012028in}}%
\pgfpathlineto{\pgfqpoint{0.000000in}{-0.012028in}}%
\pgfpathclose%
\pgfusepath{stroke,fill}%
}%
\begin{pgfscope}%
\pgfsys@transformshift{1.994402in}{3.983936in}%
\pgfsys@useobject{currentmarker}{}%
\end{pgfscope}%
\begin{pgfscope}%
\pgfsys@transformshift{4.141954in}{4.993470in}%
\pgfsys@useobject{currentmarker}{}%
\end{pgfscope}%
\begin{pgfscope}%
\pgfsys@transformshift{3.444935in}{3.234972in}%
\pgfsys@useobject{currentmarker}{}%
\end{pgfscope}%
\begin{pgfscope}%
\pgfsys@transformshift{3.553386in}{3.607254in}%
\pgfsys@useobject{currentmarker}{}%
\end{pgfscope}%
\begin{pgfscope}%
\pgfsys@transformshift{4.712017in}{0.221972in}%
\pgfsys@useobject{currentmarker}{}%
\end{pgfscope}%
\begin{pgfscope}%
\pgfsys@transformshift{3.506592in}{3.942426in}%
\pgfsys@useobject{currentmarker}{}%
\end{pgfscope}%
\begin{pgfscope}%
\pgfsys@transformshift{0.981386in}{2.417629in}%
\pgfsys@useobject{currentmarker}{}%
\end{pgfscope}%
\begin{pgfscope}%
\pgfsys@transformshift{3.788473in}{5.014319in}%
\pgfsys@useobject{currentmarker}{}%
\end{pgfscope}%
\begin{pgfscope}%
\pgfsys@transformshift{3.087477in}{3.649939in}%
\pgfsys@useobject{currentmarker}{}%
\end{pgfscope}%
\begin{pgfscope}%
\pgfsys@transformshift{0.845559in}{1.276456in}%
\pgfsys@useobject{currentmarker}{}%
\end{pgfscope}%
\begin{pgfscope}%
\pgfsys@transformshift{0.736929in}{4.310278in}%
\pgfsys@useobject{currentmarker}{}%
\end{pgfscope}%
\begin{pgfscope}%
\pgfsys@transformshift{5.510489in}{3.434807in}%
\pgfsys@useobject{currentmarker}{}%
\end{pgfscope}%
\begin{pgfscope}%
\pgfsys@transformshift{0.701767in}{3.608719in}%
\pgfsys@useobject{currentmarker}{}%
\end{pgfscope}%
\begin{pgfscope}%
\pgfsys@transformshift{3.199407in}{3.247165in}%
\pgfsys@useobject{currentmarker}{}%
\end{pgfscope}%
\begin{pgfscope}%
\pgfsys@transformshift{1.299871in}{3.891702in}%
\pgfsys@useobject{currentmarker}{}%
\end{pgfscope}%
\begin{pgfscope}%
\pgfsys@transformshift{5.260327in}{0.178658in}%
\pgfsys@useobject{currentmarker}{}%
\end{pgfscope}%
\begin{pgfscope}%
\pgfsys@transformshift{2.811975in}{4.001300in}%
\pgfsys@useobject{currentmarker}{}%
\end{pgfscope}%
\begin{pgfscope}%
\pgfsys@transformshift{4.494047in}{1.346880in}%
\pgfsys@useobject{currentmarker}{}%
\end{pgfscope}%
\begin{pgfscope}%
\pgfsys@transformshift{3.658638in}{0.469982in}%
\pgfsys@useobject{currentmarker}{}%
\end{pgfscope}%
\begin{pgfscope}%
\pgfsys@transformshift{1.404963in}{0.875078in}%
\pgfsys@useobject{currentmarker}{}%
\end{pgfscope}%
\begin{pgfscope}%
\pgfsys@transformshift{4.437689in}{0.799299in}%
\pgfsys@useobject{currentmarker}{}%
\end{pgfscope}%
\begin{pgfscope}%
\pgfsys@transformshift{3.959007in}{3.450524in}%
\pgfsys@useobject{currentmarker}{}%
\end{pgfscope}%
\begin{pgfscope}%
\pgfsys@transformshift{0.378027in}{1.995991in}%
\pgfsys@useobject{currentmarker}{}%
\end{pgfscope}%
\begin{pgfscope}%
\pgfsys@transformshift{4.134675in}{3.922345in}%
\pgfsys@useobject{currentmarker}{}%
\end{pgfscope}%
\begin{pgfscope}%
\pgfsys@transformshift{1.654689in}{3.651862in}%
\pgfsys@useobject{currentmarker}{}%
\end{pgfscope}%
\begin{pgfscope}%
\pgfsys@transformshift{4.513807in}{3.244646in}%
\pgfsys@useobject{currentmarker}{}%
\end{pgfscope}%
\begin{pgfscope}%
\pgfsys@transformshift{0.822064in}{4.276519in}%
\pgfsys@useobject{currentmarker}{}%
\end{pgfscope}%
\begin{pgfscope}%
\pgfsys@transformshift{0.582469in}{3.809454in}%
\pgfsys@useobject{currentmarker}{}%
\end{pgfscope}%
\begin{pgfscope}%
\pgfsys@transformshift{3.164857in}{1.334163in}%
\pgfsys@useobject{currentmarker}{}%
\end{pgfscope}%
\begin{pgfscope}%
\pgfsys@transformshift{3.359380in}{1.265954in}%
\pgfsys@useobject{currentmarker}{}%
\end{pgfscope}%
\begin{pgfscope}%
\pgfsys@transformshift{2.863214in}{1.073054in}%
\pgfsys@useobject{currentmarker}{}%
\end{pgfscope}%
\begin{pgfscope}%
\pgfsys@transformshift{3.247435in}{4.203115in}%
\pgfsys@useobject{currentmarker}{}%
\end{pgfscope}%
\begin{pgfscope}%
\pgfsys@transformshift{0.963603in}{0.476987in}%
\pgfsys@useobject{currentmarker}{}%
\end{pgfscope}%
\begin{pgfscope}%
\pgfsys@transformshift{3.142940in}{3.788356in}%
\pgfsys@useobject{currentmarker}{}%
\end{pgfscope}%
\begin{pgfscope}%
\pgfsys@transformshift{1.833452in}{0.965613in}%
\pgfsys@useobject{currentmarker}{}%
\end{pgfscope}%
\begin{pgfscope}%
\pgfsys@transformshift{3.093676in}{3.651131in}%
\pgfsys@useobject{currentmarker}{}%
\end{pgfscope}%
\begin{pgfscope}%
\pgfsys@transformshift{1.063313in}{2.196109in}%
\pgfsys@useobject{currentmarker}{}%
\end{pgfscope}%
\begin{pgfscope}%
\pgfsys@transformshift{5.142785in}{1.825322in}%
\pgfsys@useobject{currentmarker}{}%
\end{pgfscope}%
\begin{pgfscope}%
\pgfsys@transformshift{2.553809in}{0.806625in}%
\pgfsys@useobject{currentmarker}{}%
\end{pgfscope}%
\begin{pgfscope}%
\pgfsys@transformshift{2.083687in}{4.455079in}%
\pgfsys@useobject{currentmarker}{}%
\end{pgfscope}%
\begin{pgfscope}%
\pgfsys@transformshift{5.078702in}{2.927374in}%
\pgfsys@useobject{currentmarker}{}%
\end{pgfscope}%
\begin{pgfscope}%
\pgfsys@transformshift{5.499905in}{2.400626in}%
\pgfsys@useobject{currentmarker}{}%
\end{pgfscope}%
\begin{pgfscope}%
\pgfsys@transformshift{2.204742in}{4.681714in}%
\pgfsys@useobject{currentmarker}{}%
\end{pgfscope}%
\begin{pgfscope}%
\pgfsys@transformshift{5.756043in}{3.189752in}%
\pgfsys@useobject{currentmarker}{}%
\end{pgfscope}%
\begin{pgfscope}%
\pgfsys@transformshift{4.195870in}{0.328748in}%
\pgfsys@useobject{currentmarker}{}%
\end{pgfscope}%
\begin{pgfscope}%
\pgfsys@transformshift{3.492233in}{4.655597in}%
\pgfsys@useobject{currentmarker}{}%
\end{pgfscope}%
\begin{pgfscope}%
\pgfsys@transformshift{4.472004in}{2.266696in}%
\pgfsys@useobject{currentmarker}{}%
\end{pgfscope}%
\begin{pgfscope}%
\pgfsys@transformshift{1.819479in}{4.694956in}%
\pgfsys@useobject{currentmarker}{}%
\end{pgfscope}%
\begin{pgfscope}%
\pgfsys@transformshift{1.431887in}{2.422736in}%
\pgfsys@useobject{currentmarker}{}%
\end{pgfscope}%
\begin{pgfscope}%
\pgfsys@transformshift{5.595178in}{0.774424in}%
\pgfsys@useobject{currentmarker}{}%
\end{pgfscope}%
\begin{pgfscope}%
\pgfsys@transformshift{0.682181in}{2.854575in}%
\pgfsys@useobject{currentmarker}{}%
\end{pgfscope}%
\begin{pgfscope}%
\pgfsys@transformshift{2.784227in}{2.208437in}%
\pgfsys@useobject{currentmarker}{}%
\end{pgfscope}%
\begin{pgfscope}%
\pgfsys@transformshift{1.052437in}{4.841904in}%
\pgfsys@useobject{currentmarker}{}%
\end{pgfscope}%
\begin{pgfscope}%
\pgfsys@transformshift{4.766003in}{4.224732in}%
\pgfsys@useobject{currentmarker}{}%
\end{pgfscope}%
\begin{pgfscope}%
\pgfsys@transformshift{4.220565in}{2.703849in}%
\pgfsys@useobject{currentmarker}{}%
\end{pgfscope}%
\begin{pgfscope}%
\pgfsys@transformshift{3.581062in}{0.213553in}%
\pgfsys@useobject{currentmarker}{}%
\end{pgfscope}%
\begin{pgfscope}%
\pgfsys@transformshift{3.391985in}{4.033559in}%
\pgfsys@useobject{currentmarker}{}%
\end{pgfscope}%
\begin{pgfscope}%
\pgfsys@transformshift{3.445019in}{1.452903in}%
\pgfsys@useobject{currentmarker}{}%
\end{pgfscope}%
\begin{pgfscope}%
\pgfsys@transformshift{2.271775in}{2.731402in}%
\pgfsys@useobject{currentmarker}{}%
\end{pgfscope}%
\begin{pgfscope}%
\pgfsys@transformshift{0.959832in}{4.161604in}%
\pgfsys@useobject{currentmarker}{}%
\end{pgfscope}%
\begin{pgfscope}%
\pgfsys@transformshift{3.607059in}{1.973021in}%
\pgfsys@useobject{currentmarker}{}%
\end{pgfscope}%
\begin{pgfscope}%
\pgfsys@transformshift{0.342970in}{0.396341in}%
\pgfsys@useobject{currentmarker}{}%
\end{pgfscope}%
\begin{pgfscope}%
\pgfsys@transformshift{1.560614in}{3.836315in}%
\pgfsys@useobject{currentmarker}{}%
\end{pgfscope}%
\begin{pgfscope}%
\pgfsys@transformshift{1.252475in}{1.974454in}%
\pgfsys@useobject{currentmarker}{}%
\end{pgfscope}%
\begin{pgfscope}%
\pgfsys@transformshift{0.833197in}{3.979089in}%
\pgfsys@useobject{currentmarker}{}%
\end{pgfscope}%
\begin{pgfscope}%
\pgfsys@transformshift{2.995900in}{4.829129in}%
\pgfsys@useobject{currentmarker}{}%
\end{pgfscope}%
\begin{pgfscope}%
\pgfsys@transformshift{5.237041in}{0.372835in}%
\pgfsys@useobject{currentmarker}{}%
\end{pgfscope}%
\begin{pgfscope}%
\pgfsys@transformshift{5.256594in}{4.211288in}%
\pgfsys@useobject{currentmarker}{}%
\end{pgfscope}%
\begin{pgfscope}%
\pgfsys@transformshift{1.922414in}{5.068003in}%
\pgfsys@useobject{currentmarker}{}%
\end{pgfscope}%
\begin{pgfscope}%
\pgfsys@transformshift{5.665249in}{3.572977in}%
\pgfsys@useobject{currentmarker}{}%
\end{pgfscope}%
\begin{pgfscope}%
\pgfsys@transformshift{2.420610in}{5.009358in}%
\pgfsys@useobject{currentmarker}{}%
\end{pgfscope}%
\begin{pgfscope}%
\pgfsys@transformshift{3.231745in}{3.341030in}%
\pgfsys@useobject{currentmarker}{}%
\end{pgfscope}%
\begin{pgfscope}%
\pgfsys@transformshift{3.766545in}{0.328839in}%
\pgfsys@useobject{currentmarker}{}%
\end{pgfscope}%
\begin{pgfscope}%
\pgfsys@transformshift{0.988201in}{2.672742in}%
\pgfsys@useobject{currentmarker}{}%
\end{pgfscope}%
\begin{pgfscope}%
\pgfsys@transformshift{0.689358in}{1.213996in}%
\pgfsys@useobject{currentmarker}{}%
\end{pgfscope}%
\begin{pgfscope}%
\pgfsys@transformshift{5.255593in}{1.205984in}%
\pgfsys@useobject{currentmarker}{}%
\end{pgfscope}%
\begin{pgfscope}%
\pgfsys@transformshift{5.266134in}{3.315420in}%
\pgfsys@useobject{currentmarker}{}%
\end{pgfscope}%
\begin{pgfscope}%
\pgfsys@transformshift{2.562638in}{4.463779in}%
\pgfsys@useobject{currentmarker}{}%
\end{pgfscope}%
\begin{pgfscope}%
\pgfsys@transformshift{4.090257in}{3.687882in}%
\pgfsys@useobject{currentmarker}{}%
\end{pgfscope}%
\begin{pgfscope}%
\pgfsys@transformshift{4.922743in}{1.359111in}%
\pgfsys@useobject{currentmarker}{}%
\end{pgfscope}%
\begin{pgfscope}%
\pgfsys@transformshift{1.769265in}{3.234943in}%
\pgfsys@useobject{currentmarker}{}%
\end{pgfscope}%
\begin{pgfscope}%
\pgfsys@transformshift{0.940521in}{1.111567in}%
\pgfsys@useobject{currentmarker}{}%
\end{pgfscope}%
\begin{pgfscope}%
\pgfsys@transformshift{1.733063in}{1.890834in}%
\pgfsys@useobject{currentmarker}{}%
\end{pgfscope}%
\begin{pgfscope}%
\pgfsys@transformshift{5.005293in}{4.918521in}%
\pgfsys@useobject{currentmarker}{}%
\end{pgfscope}%
\begin{pgfscope}%
\pgfsys@transformshift{0.835870in}{0.501650in}%
\pgfsys@useobject{currentmarker}{}%
\end{pgfscope}%
\begin{pgfscope}%
\pgfsys@transformshift{0.777192in}{3.327676in}%
\pgfsys@useobject{currentmarker}{}%
\end{pgfscope}%
\begin{pgfscope}%
\pgfsys@transformshift{1.270875in}{2.161366in}%
\pgfsys@useobject{currentmarker}{}%
\end{pgfscope}%
\begin{pgfscope}%
\pgfsys@transformshift{2.279035in}{0.994295in}%
\pgfsys@useobject{currentmarker}{}%
\end{pgfscope}%
\begin{pgfscope}%
\pgfsys@transformshift{4.994615in}{2.887450in}%
\pgfsys@useobject{currentmarker}{}%
\end{pgfscope}%
\begin{pgfscope}%
\pgfsys@transformshift{0.516738in}{3.884248in}%
\pgfsys@useobject{currentmarker}{}%
\end{pgfscope}%
\begin{pgfscope}%
\pgfsys@transformshift{5.757662in}{2.690581in}%
\pgfsys@useobject{currentmarker}{}%
\end{pgfscope}%
\begin{pgfscope}%
\pgfsys@transformshift{3.490675in}{0.363181in}%
\pgfsys@useobject{currentmarker}{}%
\end{pgfscope}%
\begin{pgfscope}%
\pgfsys@transformshift{0.508732in}{2.817718in}%
\pgfsys@useobject{currentmarker}{}%
\end{pgfscope}%
\begin{pgfscope}%
\pgfsys@transformshift{4.216840in}{1.613686in}%
\pgfsys@useobject{currentmarker}{}%
\end{pgfscope}%
\begin{pgfscope}%
\pgfsys@transformshift{5.439437in}{3.184713in}%
\pgfsys@useobject{currentmarker}{}%
\end{pgfscope}%
\begin{pgfscope}%
\pgfsys@transformshift{4.574836in}{1.786348in}%
\pgfsys@useobject{currentmarker}{}%
\end{pgfscope}%
\begin{pgfscope}%
\pgfsys@transformshift{5.601084in}{4.796727in}%
\pgfsys@useobject{currentmarker}{}%
\end{pgfscope}%
\begin{pgfscope}%
\pgfsys@transformshift{2.684045in}{3.366741in}%
\pgfsys@useobject{currentmarker}{}%
\end{pgfscope}%
\begin{pgfscope}%
\pgfsys@transformshift{1.181544in}{4.549943in}%
\pgfsys@useobject{currentmarker}{}%
\end{pgfscope}%
\begin{pgfscope}%
\pgfsys@transformshift{0.430288in}{2.962898in}%
\pgfsys@useobject{currentmarker}{}%
\end{pgfscope}%
\begin{pgfscope}%
\pgfsys@transformshift{2.295119in}{2.017709in}%
\pgfsys@useobject{currentmarker}{}%
\end{pgfscope}%
\begin{pgfscope}%
\pgfsys@transformshift{0.486778in}{1.904291in}%
\pgfsys@useobject{currentmarker}{}%
\end{pgfscope}%
\begin{pgfscope}%
\pgfsys@transformshift{3.280774in}{0.879420in}%
\pgfsys@useobject{currentmarker}{}%
\end{pgfscope}%
\begin{pgfscope}%
\pgfsys@transformshift{4.827063in}{4.460710in}%
\pgfsys@useobject{currentmarker}{}%
\end{pgfscope}%
\begin{pgfscope}%
\pgfsys@transformshift{4.537303in}{0.911877in}%
\pgfsys@useobject{currentmarker}{}%
\end{pgfscope}%
\begin{pgfscope}%
\pgfsys@transformshift{1.678590in}{4.153680in}%
\pgfsys@useobject{currentmarker}{}%
\end{pgfscope}%
\begin{pgfscope}%
\pgfsys@transformshift{1.535371in}{1.473720in}%
\pgfsys@useobject{currentmarker}{}%
\end{pgfscope}%
\begin{pgfscope}%
\pgfsys@transformshift{4.598972in}{1.254213in}%
\pgfsys@useobject{currentmarker}{}%
\end{pgfscope}%
\begin{pgfscope}%
\pgfsys@transformshift{4.967369in}{3.587269in}%
\pgfsys@useobject{currentmarker}{}%
\end{pgfscope}%
\begin{pgfscope}%
\pgfsys@transformshift{3.139255in}{2.230914in}%
\pgfsys@useobject{currentmarker}{}%
\end{pgfscope}%
\begin{pgfscope}%
\pgfsys@transformshift{2.096418in}{4.261704in}%
\pgfsys@useobject{currentmarker}{}%
\end{pgfscope}%
\begin{pgfscope}%
\pgfsys@transformshift{5.337430in}{0.458243in}%
\pgfsys@useobject{currentmarker}{}%
\end{pgfscope}%
\begin{pgfscope}%
\pgfsys@transformshift{5.058780in}{1.422782in}%
\pgfsys@useobject{currentmarker}{}%
\end{pgfscope}%
\begin{pgfscope}%
\pgfsys@transformshift{3.180492in}{2.673979in}%
\pgfsys@useobject{currentmarker}{}%
\end{pgfscope}%
\begin{pgfscope}%
\pgfsys@transformshift{1.506158in}{0.922604in}%
\pgfsys@useobject{currentmarker}{}%
\end{pgfscope}%
\begin{pgfscope}%
\pgfsys@transformshift{3.934849in}{3.955088in}%
\pgfsys@useobject{currentmarker}{}%
\end{pgfscope}%
\begin{pgfscope}%
\pgfsys@transformshift{1.899463in}{4.838816in}%
\pgfsys@useobject{currentmarker}{}%
\end{pgfscope}%
\begin{pgfscope}%
\pgfsys@transformshift{4.529531in}{4.886612in}%
\pgfsys@useobject{currentmarker}{}%
\end{pgfscope}%
\begin{pgfscope}%
\pgfsys@transformshift{2.965905in}{0.230671in}%
\pgfsys@useobject{currentmarker}{}%
\end{pgfscope}%
\begin{pgfscope}%
\pgfsys@transformshift{4.365607in}{4.486250in}%
\pgfsys@useobject{currentmarker}{}%
\end{pgfscope}%
\begin{pgfscope}%
\pgfsys@transformshift{0.899915in}{4.273981in}%
\pgfsys@useobject{currentmarker}{}%
\end{pgfscope}%
\begin{pgfscope}%
\pgfsys@transformshift{3.337315in}{3.796022in}%
\pgfsys@useobject{currentmarker}{}%
\end{pgfscope}%
\begin{pgfscope}%
\pgfsys@transformshift{2.321792in}{1.791466in}%
\pgfsys@useobject{currentmarker}{}%
\end{pgfscope}%
\begin{pgfscope}%
\pgfsys@transformshift{1.592502in}{4.137885in}%
\pgfsys@useobject{currentmarker}{}%
\end{pgfscope}%
\begin{pgfscope}%
\pgfsys@transformshift{4.822162in}{0.973223in}%
\pgfsys@useobject{currentmarker}{}%
\end{pgfscope}%
\begin{pgfscope}%
\pgfsys@transformshift{1.344910in}{3.371682in}%
\pgfsys@useobject{currentmarker}{}%
\end{pgfscope}%
\begin{pgfscope}%
\pgfsys@transformshift{1.800283in}{0.527482in}%
\pgfsys@useobject{currentmarker}{}%
\end{pgfscope}%
\begin{pgfscope}%
\pgfsys@transformshift{0.604948in}{1.636016in}%
\pgfsys@useobject{currentmarker}{}%
\end{pgfscope}%
\begin{pgfscope}%
\pgfsys@transformshift{1.306958in}{2.641278in}%
\pgfsys@useobject{currentmarker}{}%
\end{pgfscope}%
\begin{pgfscope}%
\pgfsys@transformshift{1.966696in}{3.768792in}%
\pgfsys@useobject{currentmarker}{}%
\end{pgfscope}%
\begin{pgfscope}%
\pgfsys@transformshift{0.912984in}{2.100774in}%
\pgfsys@useobject{currentmarker}{}%
\end{pgfscope}%
\begin{pgfscope}%
\pgfsys@transformshift{1.786590in}{4.865654in}%
\pgfsys@useobject{currentmarker}{}%
\end{pgfscope}%
\begin{pgfscope}%
\pgfsys@transformshift{3.457999in}{0.681048in}%
\pgfsys@useobject{currentmarker}{}%
\end{pgfscope}%
\begin{pgfscope}%
\pgfsys@transformshift{0.683797in}{1.218207in}%
\pgfsys@useobject{currentmarker}{}%
\end{pgfscope}%
\begin{pgfscope}%
\pgfsys@transformshift{3.988174in}{2.243133in}%
\pgfsys@useobject{currentmarker}{}%
\end{pgfscope}%
\begin{pgfscope}%
\pgfsys@transformshift{0.714370in}{0.834946in}%
\pgfsys@useobject{currentmarker}{}%
\end{pgfscope}%
\begin{pgfscope}%
\pgfsys@transformshift{4.144691in}{0.576344in}%
\pgfsys@useobject{currentmarker}{}%
\end{pgfscope}%
\begin{pgfscope}%
\pgfsys@transformshift{1.467566in}{1.331694in}%
\pgfsys@useobject{currentmarker}{}%
\end{pgfscope}%
\begin{pgfscope}%
\pgfsys@transformshift{3.749476in}{1.006115in}%
\pgfsys@useobject{currentmarker}{}%
\end{pgfscope}%
\begin{pgfscope}%
\pgfsys@transformshift{0.778632in}{0.710265in}%
\pgfsys@useobject{currentmarker}{}%
\end{pgfscope}%
\begin{pgfscope}%
\pgfsys@transformshift{3.489152in}{4.182894in}%
\pgfsys@useobject{currentmarker}{}%
\end{pgfscope}%
\begin{pgfscope}%
\pgfsys@transformshift{5.599434in}{2.579456in}%
\pgfsys@useobject{currentmarker}{}%
\end{pgfscope}%
\begin{pgfscope}%
\pgfsys@transformshift{3.591369in}{0.526869in}%
\pgfsys@useobject{currentmarker}{}%
\end{pgfscope}%
\begin{pgfscope}%
\pgfsys@transformshift{4.065469in}{4.587568in}%
\pgfsys@useobject{currentmarker}{}%
\end{pgfscope}%
\begin{pgfscope}%
\pgfsys@transformshift{1.445449in}{2.102566in}%
\pgfsys@useobject{currentmarker}{}%
\end{pgfscope}%
\begin{pgfscope}%
\pgfsys@transformshift{3.338818in}{1.620805in}%
\pgfsys@useobject{currentmarker}{}%
\end{pgfscope}%
\begin{pgfscope}%
\pgfsys@transformshift{4.997281in}{1.815574in}%
\pgfsys@useobject{currentmarker}{}%
\end{pgfscope}%
\begin{pgfscope}%
\pgfsys@transformshift{0.624196in}{4.014713in}%
\pgfsys@useobject{currentmarker}{}%
\end{pgfscope}%
\begin{pgfscope}%
\pgfsys@transformshift{4.861323in}{0.396283in}%
\pgfsys@useobject{currentmarker}{}%
\end{pgfscope}%
\begin{pgfscope}%
\pgfsys@transformshift{3.249324in}{2.310939in}%
\pgfsys@useobject{currentmarker}{}%
\end{pgfscope}%
\begin{pgfscope}%
\pgfsys@transformshift{4.187468in}{3.720404in}%
\pgfsys@useobject{currentmarker}{}%
\end{pgfscope}%
\begin{pgfscope}%
\pgfsys@transformshift{5.710550in}{1.232120in}%
\pgfsys@useobject{currentmarker}{}%
\end{pgfscope}%
\begin{pgfscope}%
\pgfsys@transformshift{0.497072in}{4.104987in}%
\pgfsys@useobject{currentmarker}{}%
\end{pgfscope}%
\begin{pgfscope}%
\pgfsys@transformshift{1.739835in}{1.729515in}%
\pgfsys@useobject{currentmarker}{}%
\end{pgfscope}%
\begin{pgfscope}%
\pgfsys@transformshift{3.312259in}{0.174824in}%
\pgfsys@useobject{currentmarker}{}%
\end{pgfscope}%
\begin{pgfscope}%
\pgfsys@transformshift{4.149136in}{5.065733in}%
\pgfsys@useobject{currentmarker}{}%
\end{pgfscope}%
\begin{pgfscope}%
\pgfsys@transformshift{5.245344in}{0.223306in}%
\pgfsys@useobject{currentmarker}{}%
\end{pgfscope}%
\begin{pgfscope}%
\pgfsys@transformshift{2.278647in}{3.768171in}%
\pgfsys@useobject{currentmarker}{}%
\end{pgfscope}%
\begin{pgfscope}%
\pgfsys@transformshift{3.195565in}{2.345762in}%
\pgfsys@useobject{currentmarker}{}%
\end{pgfscope}%
\begin{pgfscope}%
\pgfsys@transformshift{2.169002in}{1.088637in}%
\pgfsys@useobject{currentmarker}{}%
\end{pgfscope}%
\begin{pgfscope}%
\pgfsys@transformshift{4.649326in}{1.752670in}%
\pgfsys@useobject{currentmarker}{}%
\end{pgfscope}%
\begin{pgfscope}%
\pgfsys@transformshift{5.170849in}{0.587740in}%
\pgfsys@useobject{currentmarker}{}%
\end{pgfscope}%
\begin{pgfscope}%
\pgfsys@transformshift{3.625487in}{2.988193in}%
\pgfsys@useobject{currentmarker}{}%
\end{pgfscope}%
\begin{pgfscope}%
\pgfsys@transformshift{2.340555in}{0.288611in}%
\pgfsys@useobject{currentmarker}{}%
\end{pgfscope}%
\begin{pgfscope}%
\pgfsys@transformshift{2.568739in}{2.672978in}%
\pgfsys@useobject{currentmarker}{}%
\end{pgfscope}%
\begin{pgfscope}%
\pgfsys@transformshift{0.397133in}{3.052877in}%
\pgfsys@useobject{currentmarker}{}%
\end{pgfscope}%
\begin{pgfscope}%
\pgfsys@transformshift{0.376190in}{2.210011in}%
\pgfsys@useobject{currentmarker}{}%
\end{pgfscope}%
\begin{pgfscope}%
\pgfsys@transformshift{4.182117in}{2.490880in}%
\pgfsys@useobject{currentmarker}{}%
\end{pgfscope}%
\begin{pgfscope}%
\pgfsys@transformshift{3.420369in}{0.289179in}%
\pgfsys@useobject{currentmarker}{}%
\end{pgfscope}%
\begin{pgfscope}%
\pgfsys@transformshift{4.320044in}{0.243874in}%
\pgfsys@useobject{currentmarker}{}%
\end{pgfscope}%
\begin{pgfscope}%
\pgfsys@transformshift{0.922713in}{2.119462in}%
\pgfsys@useobject{currentmarker}{}%
\end{pgfscope}%
\begin{pgfscope}%
\pgfsys@transformshift{5.490228in}{3.909380in}%
\pgfsys@useobject{currentmarker}{}%
\end{pgfscope}%
\begin{pgfscope}%
\pgfsys@transformshift{4.793523in}{1.032000in}%
\pgfsys@useobject{currentmarker}{}%
\end{pgfscope}%
\begin{pgfscope}%
\pgfsys@transformshift{1.110987in}{0.532623in}%
\pgfsys@useobject{currentmarker}{}%
\end{pgfscope}%
\begin{pgfscope}%
\pgfsys@transformshift{1.623194in}{3.658474in}%
\pgfsys@useobject{currentmarker}{}%
\end{pgfscope}%
\begin{pgfscope}%
\pgfsys@transformshift{0.845031in}{1.183640in}%
\pgfsys@useobject{currentmarker}{}%
\end{pgfscope}%
\begin{pgfscope}%
\pgfsys@transformshift{4.226408in}{0.794713in}%
\pgfsys@useobject{currentmarker}{}%
\end{pgfscope}%
\begin{pgfscope}%
\pgfsys@transformshift{3.893177in}{4.638725in}%
\pgfsys@useobject{currentmarker}{}%
\end{pgfscope}%
\begin{pgfscope}%
\pgfsys@transformshift{1.824932in}{2.514327in}%
\pgfsys@useobject{currentmarker}{}%
\end{pgfscope}%
\begin{pgfscope}%
\pgfsys@transformshift{2.652791in}{2.042860in}%
\pgfsys@useobject{currentmarker}{}%
\end{pgfscope}%
\begin{pgfscope}%
\pgfsys@transformshift{1.870397in}{3.120960in}%
\pgfsys@useobject{currentmarker}{}%
\end{pgfscope}%
\begin{pgfscope}%
\pgfsys@transformshift{2.041223in}{1.863448in}%
\pgfsys@useobject{currentmarker}{}%
\end{pgfscope}%
\begin{pgfscope}%
\pgfsys@transformshift{4.033501in}{4.495096in}%
\pgfsys@useobject{currentmarker}{}%
\end{pgfscope}%
\begin{pgfscope}%
\pgfsys@transformshift{1.711424in}{2.854093in}%
\pgfsys@useobject{currentmarker}{}%
\end{pgfscope}%
\begin{pgfscope}%
\pgfsys@transformshift{2.470038in}{0.400152in}%
\pgfsys@useobject{currentmarker}{}%
\end{pgfscope}%
\begin{pgfscope}%
\pgfsys@transformshift{2.485285in}{0.415211in}%
\pgfsys@useobject{currentmarker}{}%
\end{pgfscope}%
\begin{pgfscope}%
\pgfsys@transformshift{4.792930in}{1.763094in}%
\pgfsys@useobject{currentmarker}{}%
\end{pgfscope}%
\begin{pgfscope}%
\pgfsys@transformshift{1.186500in}{5.076878in}%
\pgfsys@useobject{currentmarker}{}%
\end{pgfscope}%
\begin{pgfscope}%
\pgfsys@transformshift{2.071632in}{4.116659in}%
\pgfsys@useobject{currentmarker}{}%
\end{pgfscope}%
\begin{pgfscope}%
\pgfsys@transformshift{3.811698in}{3.935223in}%
\pgfsys@useobject{currentmarker}{}%
\end{pgfscope}%
\begin{pgfscope}%
\pgfsys@transformshift{4.138323in}{3.804570in}%
\pgfsys@useobject{currentmarker}{}%
\end{pgfscope}%
\begin{pgfscope}%
\pgfsys@transformshift{0.337338in}{3.926583in}%
\pgfsys@useobject{currentmarker}{}%
\end{pgfscope}%
\begin{pgfscope}%
\pgfsys@transformshift{5.048878in}{4.964232in}%
\pgfsys@useobject{currentmarker}{}%
\end{pgfscope}%
\begin{pgfscope}%
\pgfsys@transformshift{5.644216in}{1.930689in}%
\pgfsys@useobject{currentmarker}{}%
\end{pgfscope}%
\begin{pgfscope}%
\pgfsys@transformshift{5.637421in}{3.971029in}%
\pgfsys@useobject{currentmarker}{}%
\end{pgfscope}%
\begin{pgfscope}%
\pgfsys@transformshift{4.174858in}{1.366030in}%
\pgfsys@useobject{currentmarker}{}%
\end{pgfscope}%
\begin{pgfscope}%
\pgfsys@transformshift{5.318895in}{0.883001in}%
\pgfsys@useobject{currentmarker}{}%
\end{pgfscope}%
\begin{pgfscope}%
\pgfsys@transformshift{2.459655in}{4.243069in}%
\pgfsys@useobject{currentmarker}{}%
\end{pgfscope}%
\begin{pgfscope}%
\pgfsys@transformshift{4.521322in}{0.352919in}%
\pgfsys@useobject{currentmarker}{}%
\end{pgfscope}%
\begin{pgfscope}%
\pgfsys@transformshift{3.005185in}{1.982249in}%
\pgfsys@useobject{currentmarker}{}%
\end{pgfscope}%
\begin{pgfscope}%
\pgfsys@transformshift{3.191441in}{3.509443in}%
\pgfsys@useobject{currentmarker}{}%
\end{pgfscope}%
\begin{pgfscope}%
\pgfsys@transformshift{1.865433in}{2.428833in}%
\pgfsys@useobject{currentmarker}{}%
\end{pgfscope}%
\begin{pgfscope}%
\pgfsys@transformshift{5.698149in}{1.492921in}%
\pgfsys@useobject{currentmarker}{}%
\end{pgfscope}%
\begin{pgfscope}%
\pgfsys@transformshift{4.449642in}{1.585057in}%
\pgfsys@useobject{currentmarker}{}%
\end{pgfscope}%
\begin{pgfscope}%
\pgfsys@transformshift{2.112220in}{1.588398in}%
\pgfsys@useobject{currentmarker}{}%
\end{pgfscope}%
\begin{pgfscope}%
\pgfsys@transformshift{5.530978in}{4.023259in}%
\pgfsys@useobject{currentmarker}{}%
\end{pgfscope}%
\begin{pgfscope}%
\pgfsys@transformshift{1.459180in}{1.824106in}%
\pgfsys@useobject{currentmarker}{}%
\end{pgfscope}%
\begin{pgfscope}%
\pgfsys@transformshift{1.130025in}{3.744712in}%
\pgfsys@useobject{currentmarker}{}%
\end{pgfscope}%
\begin{pgfscope}%
\pgfsys@transformshift{1.063397in}{1.256972in}%
\pgfsys@useobject{currentmarker}{}%
\end{pgfscope}%
\begin{pgfscope}%
\pgfsys@transformshift{0.765803in}{1.999930in}%
\pgfsys@useobject{currentmarker}{}%
\end{pgfscope}%
\begin{pgfscope}%
\pgfsys@transformshift{1.982752in}{0.190044in}%
\pgfsys@useobject{currentmarker}{}%
\end{pgfscope}%
\begin{pgfscope}%
\pgfsys@transformshift{1.436192in}{0.834782in}%
\pgfsys@useobject{currentmarker}{}%
\end{pgfscope}%
\begin{pgfscope}%
\pgfsys@transformshift{3.624944in}{3.490342in}%
\pgfsys@useobject{currentmarker}{}%
\end{pgfscope}%
\begin{pgfscope}%
\pgfsys@transformshift{4.822887in}{2.695772in}%
\pgfsys@useobject{currentmarker}{}%
\end{pgfscope}%
\begin{pgfscope}%
\pgfsys@transformshift{3.715763in}{2.241427in}%
\pgfsys@useobject{currentmarker}{}%
\end{pgfscope}%
\begin{pgfscope}%
\pgfsys@transformshift{3.832365in}{2.963587in}%
\pgfsys@useobject{currentmarker}{}%
\end{pgfscope}%
\begin{pgfscope}%
\pgfsys@transformshift{3.664513in}{4.206442in}%
\pgfsys@useobject{currentmarker}{}%
\end{pgfscope}%
\begin{pgfscope}%
\pgfsys@transformshift{4.057457in}{1.734632in}%
\pgfsys@useobject{currentmarker}{}%
\end{pgfscope}%
\begin{pgfscope}%
\pgfsys@transformshift{5.382513in}{4.634916in}%
\pgfsys@useobject{currentmarker}{}%
\end{pgfscope}%
\begin{pgfscope}%
\pgfsys@transformshift{1.352496in}{4.858882in}%
\pgfsys@useobject{currentmarker}{}%
\end{pgfscope}%
\begin{pgfscope}%
\pgfsys@transformshift{4.978733in}{4.601877in}%
\pgfsys@useobject{currentmarker}{}%
\end{pgfscope}%
\begin{pgfscope}%
\pgfsys@transformshift{5.536722in}{2.751823in}%
\pgfsys@useobject{currentmarker}{}%
\end{pgfscope}%
\begin{pgfscope}%
\pgfsys@transformshift{5.102399in}{4.652584in}%
\pgfsys@useobject{currentmarker}{}%
\end{pgfscope}%
\begin{pgfscope}%
\pgfsys@transformshift{4.171483in}{3.397051in}%
\pgfsys@useobject{currentmarker}{}%
\end{pgfscope}%
\begin{pgfscope}%
\pgfsys@transformshift{1.527430in}{2.627794in}%
\pgfsys@useobject{currentmarker}{}%
\end{pgfscope}%
\begin{pgfscope}%
\pgfsys@transformshift{3.980606in}{3.653451in}%
\pgfsys@useobject{currentmarker}{}%
\end{pgfscope}%
\begin{pgfscope}%
\pgfsys@transformshift{5.723139in}{4.957031in}%
\pgfsys@useobject{currentmarker}{}%
\end{pgfscope}%
\begin{pgfscope}%
\pgfsys@transformshift{5.603190in}{0.177311in}%
\pgfsys@useobject{currentmarker}{}%
\end{pgfscope}%
\begin{pgfscope}%
\pgfsys@transformshift{4.671984in}{3.838777in}%
\pgfsys@useobject{currentmarker}{}%
\end{pgfscope}%
\begin{pgfscope}%
\pgfsys@transformshift{2.941910in}{2.992327in}%
\pgfsys@useobject{currentmarker}{}%
\end{pgfscope}%
\begin{pgfscope}%
\pgfsys@transformshift{4.646743in}{3.758532in}%
\pgfsys@useobject{currentmarker}{}%
\end{pgfscope}%
\begin{pgfscope}%
\pgfsys@transformshift{5.674709in}{4.677537in}%
\pgfsys@useobject{currentmarker}{}%
\end{pgfscope}%
\begin{pgfscope}%
\pgfsys@transformshift{5.264029in}{4.930712in}%
\pgfsys@useobject{currentmarker}{}%
\end{pgfscope}%
\begin{pgfscope}%
\pgfsys@transformshift{5.682420in}{1.115657in}%
\pgfsys@useobject{currentmarker}{}%
\end{pgfscope}%
\begin{pgfscope}%
\pgfsys@transformshift{4.479154in}{1.890909in}%
\pgfsys@useobject{currentmarker}{}%
\end{pgfscope}%
\begin{pgfscope}%
\pgfsys@transformshift{1.665501in}{4.943143in}%
\pgfsys@useobject{currentmarker}{}%
\end{pgfscope}%
\begin{pgfscope}%
\pgfsys@transformshift{0.838179in}{0.626121in}%
\pgfsys@useobject{currentmarker}{}%
\end{pgfscope}%
\begin{pgfscope}%
\pgfsys@transformshift{4.217658in}{2.034262in}%
\pgfsys@useobject{currentmarker}{}%
\end{pgfscope}%
\begin{pgfscope}%
\pgfsys@transformshift{0.474642in}{1.951992in}%
\pgfsys@useobject{currentmarker}{}%
\end{pgfscope}%
\begin{pgfscope}%
\pgfsys@transformshift{2.659706in}{0.363939in}%
\pgfsys@useobject{currentmarker}{}%
\end{pgfscope}%
\begin{pgfscope}%
\pgfsys@transformshift{5.480316in}{2.063159in}%
\pgfsys@useobject{currentmarker}{}%
\end{pgfscope}%
\begin{pgfscope}%
\pgfsys@transformshift{2.292914in}{2.379796in}%
\pgfsys@useobject{currentmarker}{}%
\end{pgfscope}%
\begin{pgfscope}%
\pgfsys@transformshift{5.377539in}{1.291750in}%
\pgfsys@useobject{currentmarker}{}%
\end{pgfscope}%
\begin{pgfscope}%
\pgfsys@transformshift{2.219160in}{4.920660in}%
\pgfsys@useobject{currentmarker}{}%
\end{pgfscope}%
\begin{pgfscope}%
\pgfsys@transformshift{3.896608in}{0.250045in}%
\pgfsys@useobject{currentmarker}{}%
\end{pgfscope}%
\begin{pgfscope}%
\pgfsys@transformshift{3.985389in}{2.234543in}%
\pgfsys@useobject{currentmarker}{}%
\end{pgfscope}%
\begin{pgfscope}%
\pgfsys@transformshift{2.915728in}{4.576839in}%
\pgfsys@useobject{currentmarker}{}%
\end{pgfscope}%
\begin{pgfscope}%
\pgfsys@transformshift{2.374038in}{3.623051in}%
\pgfsys@useobject{currentmarker}{}%
\end{pgfscope}%
\begin{pgfscope}%
\pgfsys@transformshift{0.555545in}{2.487059in}%
\pgfsys@useobject{currentmarker}{}%
\end{pgfscope}%
\begin{pgfscope}%
\pgfsys@transformshift{3.479980in}{0.701077in}%
\pgfsys@useobject{currentmarker}{}%
\end{pgfscope}%
\begin{pgfscope}%
\pgfsys@transformshift{0.718597in}{2.254542in}%
\pgfsys@useobject{currentmarker}{}%
\end{pgfscope}%
\begin{pgfscope}%
\pgfsys@transformshift{4.845226in}{3.088104in}%
\pgfsys@useobject{currentmarker}{}%
\end{pgfscope}%
\begin{pgfscope}%
\pgfsys@transformshift{4.374784in}{0.593946in}%
\pgfsys@useobject{currentmarker}{}%
\end{pgfscope}%
\begin{pgfscope}%
\pgfsys@transformshift{4.427805in}{1.377162in}%
\pgfsys@useobject{currentmarker}{}%
\end{pgfscope}%
\begin{pgfscope}%
\pgfsys@transformshift{2.143101in}{1.234730in}%
\pgfsys@useobject{currentmarker}{}%
\end{pgfscope}%
\begin{pgfscope}%
\pgfsys@transformshift{4.105260in}{1.239407in}%
\pgfsys@useobject{currentmarker}{}%
\end{pgfscope}%
\begin{pgfscope}%
\pgfsys@transformshift{1.044949in}{0.375428in}%
\pgfsys@useobject{currentmarker}{}%
\end{pgfscope}%
\begin{pgfscope}%
\pgfsys@transformshift{4.382267in}{3.870892in}%
\pgfsys@useobject{currentmarker}{}%
\end{pgfscope}%
\begin{pgfscope}%
\pgfsys@transformshift{1.909531in}{3.346497in}%
\pgfsys@useobject{currentmarker}{}%
\end{pgfscope}%
\begin{pgfscope}%
\pgfsys@transformshift{5.097287in}{0.176641in}%
\pgfsys@useobject{currentmarker}{}%
\end{pgfscope}%
\begin{pgfscope}%
\pgfsys@transformshift{4.447669in}{3.365410in}%
\pgfsys@useobject{currentmarker}{}%
\end{pgfscope}%
\begin{pgfscope}%
\pgfsys@transformshift{1.651515in}{4.182598in}%
\pgfsys@useobject{currentmarker}{}%
\end{pgfscope}%
\begin{pgfscope}%
\pgfsys@transformshift{1.461668in}{2.945276in}%
\pgfsys@useobject{currentmarker}{}%
\end{pgfscope}%
\begin{pgfscope}%
\pgfsys@transformshift{0.849656in}{0.556549in}%
\pgfsys@useobject{currentmarker}{}%
\end{pgfscope}%
\begin{pgfscope}%
\pgfsys@transformshift{4.588532in}{4.377821in}%
\pgfsys@useobject{currentmarker}{}%
\end{pgfscope}%
\begin{pgfscope}%
\pgfsys@transformshift{2.780100in}{4.321527in}%
\pgfsys@useobject{currentmarker}{}%
\end{pgfscope}%
\begin{pgfscope}%
\pgfsys@transformshift{4.647894in}{5.029234in}%
\pgfsys@useobject{currentmarker}{}%
\end{pgfscope}%
\begin{pgfscope}%
\pgfsys@transformshift{2.232649in}{4.751288in}%
\pgfsys@useobject{currentmarker}{}%
\end{pgfscope}%
\begin{pgfscope}%
\pgfsys@transformshift{4.284858in}{2.661514in}%
\pgfsys@useobject{currentmarker}{}%
\end{pgfscope}%
\begin{pgfscope}%
\pgfsys@transformshift{1.466371in}{2.498429in}%
\pgfsys@useobject{currentmarker}{}%
\end{pgfscope}%
\begin{pgfscope}%
\pgfsys@transformshift{3.143498in}{0.739408in}%
\pgfsys@useobject{currentmarker}{}%
\end{pgfscope}%
\begin{pgfscope}%
\pgfsys@transformshift{0.550980in}{0.865881in}%
\pgfsys@useobject{currentmarker}{}%
\end{pgfscope}%
\begin{pgfscope}%
\pgfsys@transformshift{5.291086in}{3.091382in}%
\pgfsys@useobject{currentmarker}{}%
\end{pgfscope}%
\begin{pgfscope}%
\pgfsys@transformshift{2.998946in}{0.486886in}%
\pgfsys@useobject{currentmarker}{}%
\end{pgfscope}%
\begin{pgfscope}%
\pgfsys@transformshift{3.171260in}{0.853262in}%
\pgfsys@useobject{currentmarker}{}%
\end{pgfscope}%
\begin{pgfscope}%
\pgfsys@transformshift{1.958903in}{1.465736in}%
\pgfsys@useobject{currentmarker}{}%
\end{pgfscope}%
\begin{pgfscope}%
\pgfsys@transformshift{4.698649in}{4.231140in}%
\pgfsys@useobject{currentmarker}{}%
\end{pgfscope}%
\begin{pgfscope}%
\pgfsys@transformshift{3.354107in}{0.931903in}%
\pgfsys@useobject{currentmarker}{}%
\end{pgfscope}%
\begin{pgfscope}%
\pgfsys@transformshift{0.552441in}{3.232527in}%
\pgfsys@useobject{currentmarker}{}%
\end{pgfscope}%
\begin{pgfscope}%
\pgfsys@transformshift{4.930463in}{1.517514in}%
\pgfsys@useobject{currentmarker}{}%
\end{pgfscope}%
\begin{pgfscope}%
\pgfsys@transformshift{0.338611in}{2.898368in}%
\pgfsys@useobject{currentmarker}{}%
\end{pgfscope}%
\begin{pgfscope}%
\pgfsys@transformshift{4.731579in}{0.637848in}%
\pgfsys@useobject{currentmarker}{}%
\end{pgfscope}%
\begin{pgfscope}%
\pgfsys@transformshift{0.408396in}{1.528482in}%
\pgfsys@useobject{currentmarker}{}%
\end{pgfscope}%
\begin{pgfscope}%
\pgfsys@transformshift{0.880690in}{2.775868in}%
\pgfsys@useobject{currentmarker}{}%
\end{pgfscope}%
\begin{pgfscope}%
\pgfsys@transformshift{2.011104in}{4.713271in}%
\pgfsys@useobject{currentmarker}{}%
\end{pgfscope}%
\begin{pgfscope}%
\pgfsys@transformshift{4.074944in}{0.279288in}%
\pgfsys@useobject{currentmarker}{}%
\end{pgfscope}%
\begin{pgfscope}%
\pgfsys@transformshift{4.749025in}{4.382515in}%
\pgfsys@useobject{currentmarker}{}%
\end{pgfscope}%
\begin{pgfscope}%
\pgfsys@transformshift{1.529107in}{4.295077in}%
\pgfsys@useobject{currentmarker}{}%
\end{pgfscope}%
\begin{pgfscope}%
\pgfsys@transformshift{0.666767in}{2.806926in}%
\pgfsys@useobject{currentmarker}{}%
\end{pgfscope}%
\begin{pgfscope}%
\pgfsys@transformshift{1.095370in}{0.487429in}%
\pgfsys@useobject{currentmarker}{}%
\end{pgfscope}%
\begin{pgfscope}%
\pgfsys@transformshift{1.224671in}{3.138011in}%
\pgfsys@useobject{currentmarker}{}%
\end{pgfscope}%
\begin{pgfscope}%
\pgfsys@transformshift{0.857370in}{2.130654in}%
\pgfsys@useobject{currentmarker}{}%
\end{pgfscope}%
\begin{pgfscope}%
\pgfsys@transformshift{3.613475in}{3.242825in}%
\pgfsys@useobject{currentmarker}{}%
\end{pgfscope}%
\begin{pgfscope}%
\pgfsys@transformshift{4.248697in}{3.026985in}%
\pgfsys@useobject{currentmarker}{}%
\end{pgfscope}%
\begin{pgfscope}%
\pgfsys@transformshift{1.237245in}{2.448096in}%
\pgfsys@useobject{currentmarker}{}%
\end{pgfscope}%
\begin{pgfscope}%
\pgfsys@transformshift{0.513782in}{2.909340in}%
\pgfsys@useobject{currentmarker}{}%
\end{pgfscope}%
\begin{pgfscope}%
\pgfsys@transformshift{2.497183in}{4.660685in}%
\pgfsys@useobject{currentmarker}{}%
\end{pgfscope}%
\begin{pgfscope}%
\pgfsys@transformshift{3.072665in}{3.204889in}%
\pgfsys@useobject{currentmarker}{}%
\end{pgfscope}%
\begin{pgfscope}%
\pgfsys@transformshift{0.763713in}{3.739120in}%
\pgfsys@useobject{currentmarker}{}%
\end{pgfscope}%
\begin{pgfscope}%
\pgfsys@transformshift{2.788800in}{1.950812in}%
\pgfsys@useobject{currentmarker}{}%
\end{pgfscope}%
\begin{pgfscope}%
\pgfsys@transformshift{4.260104in}{4.161804in}%
\pgfsys@useobject{currentmarker}{}%
\end{pgfscope}%
\begin{pgfscope}%
\pgfsys@transformshift{2.708446in}{2.522219in}%
\pgfsys@useobject{currentmarker}{}%
\end{pgfscope}%
\begin{pgfscope}%
\pgfsys@transformshift{0.578478in}{0.799740in}%
\pgfsys@useobject{currentmarker}{}%
\end{pgfscope}%
\begin{pgfscope}%
\pgfsys@transformshift{2.035744in}{1.610077in}%
\pgfsys@useobject{currentmarker}{}%
\end{pgfscope}%
\begin{pgfscope}%
\pgfsys@transformshift{0.567119in}{3.682112in}%
\pgfsys@useobject{currentmarker}{}%
\end{pgfscope}%
\begin{pgfscope}%
\pgfsys@transformshift{4.905514in}{1.906328in}%
\pgfsys@useobject{currentmarker}{}%
\end{pgfscope}%
\begin{pgfscope}%
\pgfsys@transformshift{1.995250in}{4.084081in}%
\pgfsys@useobject{currentmarker}{}%
\end{pgfscope}%
\begin{pgfscope}%
\pgfsys@transformshift{5.386004in}{0.810281in}%
\pgfsys@useobject{currentmarker}{}%
\end{pgfscope}%
\begin{pgfscope}%
\pgfsys@transformshift{4.379686in}{3.019894in}%
\pgfsys@useobject{currentmarker}{}%
\end{pgfscope}%
\begin{pgfscope}%
\pgfsys@transformshift{2.033524in}{0.376327in}%
\pgfsys@useobject{currentmarker}{}%
\end{pgfscope}%
\begin{pgfscope}%
\pgfsys@transformshift{5.695729in}{3.628497in}%
\pgfsys@useobject{currentmarker}{}%
\end{pgfscope}%
\begin{pgfscope}%
\pgfsys@transformshift{1.267916in}{4.723073in}%
\pgfsys@useobject{currentmarker}{}%
\end{pgfscope}%
\begin{pgfscope}%
\pgfsys@transformshift{1.924295in}{2.292818in}%
\pgfsys@useobject{currentmarker}{}%
\end{pgfscope}%
\begin{pgfscope}%
\pgfsys@transformshift{4.514874in}{4.118856in}%
\pgfsys@useobject{currentmarker}{}%
\end{pgfscope}%
\begin{pgfscope}%
\pgfsys@transformshift{5.746011in}{4.844030in}%
\pgfsys@useobject{currentmarker}{}%
\end{pgfscope}%
\begin{pgfscope}%
\pgfsys@transformshift{2.824187in}{3.181728in}%
\pgfsys@useobject{currentmarker}{}%
\end{pgfscope}%
\begin{pgfscope}%
\pgfsys@transformshift{5.094042in}{1.649708in}%
\pgfsys@useobject{currentmarker}{}%
\end{pgfscope}%
\begin{pgfscope}%
\pgfsys@transformshift{1.229852in}{1.559049in}%
\pgfsys@useobject{currentmarker}{}%
\end{pgfscope}%
\begin{pgfscope}%
\pgfsys@transformshift{5.136836in}{0.709070in}%
\pgfsys@useobject{currentmarker}{}%
\end{pgfscope}%
\begin{pgfscope}%
\pgfsys@transformshift{3.912971in}{1.454986in}%
\pgfsys@useobject{currentmarker}{}%
\end{pgfscope}%
\begin{pgfscope}%
\pgfsys@transformshift{2.892125in}{3.989023in}%
\pgfsys@useobject{currentmarker}{}%
\end{pgfscope}%
\begin{pgfscope}%
\pgfsys@transformshift{2.488947in}{5.082039in}%
\pgfsys@useobject{currentmarker}{}%
\end{pgfscope}%
\begin{pgfscope}%
\pgfsys@transformshift{0.488573in}{3.107312in}%
\pgfsys@useobject{currentmarker}{}%
\end{pgfscope}%
\begin{pgfscope}%
\pgfsys@transformshift{3.998515in}{4.423583in}%
\pgfsys@useobject{currentmarker}{}%
\end{pgfscope}%
\begin{pgfscope}%
\pgfsys@transformshift{3.526756in}{1.574241in}%
\pgfsys@useobject{currentmarker}{}%
\end{pgfscope}%
\begin{pgfscope}%
\pgfsys@transformshift{2.412378in}{2.431441in}%
\pgfsys@useobject{currentmarker}{}%
\end{pgfscope}%
\begin{pgfscope}%
\pgfsys@transformshift{5.399312in}{0.680949in}%
\pgfsys@useobject{currentmarker}{}%
\end{pgfscope}%
\begin{pgfscope}%
\pgfsys@transformshift{2.892156in}{4.964160in}%
\pgfsys@useobject{currentmarker}{}%
\end{pgfscope}%
\begin{pgfscope}%
\pgfsys@transformshift{2.284572in}{1.929250in}%
\pgfsys@useobject{currentmarker}{}%
\end{pgfscope}%
\begin{pgfscope}%
\pgfsys@transformshift{3.221952in}{3.645213in}%
\pgfsys@useobject{currentmarker}{}%
\end{pgfscope}%
\begin{pgfscope}%
\pgfsys@transformshift{2.307092in}{4.786328in}%
\pgfsys@useobject{currentmarker}{}%
\end{pgfscope}%
\begin{pgfscope}%
\pgfsys@transformshift{3.819404in}{2.629439in}%
\pgfsys@useobject{currentmarker}{}%
\end{pgfscope}%
\begin{pgfscope}%
\pgfsys@transformshift{2.533169in}{3.271107in}%
\pgfsys@useobject{currentmarker}{}%
\end{pgfscope}%
\begin{pgfscope}%
\pgfsys@transformshift{1.933699in}{0.489835in}%
\pgfsys@useobject{currentmarker}{}%
\end{pgfscope}%
\begin{pgfscope}%
\pgfsys@transformshift{3.524391in}{4.588287in}%
\pgfsys@useobject{currentmarker}{}%
\end{pgfscope}%
\begin{pgfscope}%
\pgfsys@transformshift{1.807765in}{4.579286in}%
\pgfsys@useobject{currentmarker}{}%
\end{pgfscope}%
\begin{pgfscope}%
\pgfsys@transformshift{2.720238in}{0.456688in}%
\pgfsys@useobject{currentmarker}{}%
\end{pgfscope}%
\begin{pgfscope}%
\pgfsys@transformshift{1.614309in}{0.238819in}%
\pgfsys@useobject{currentmarker}{}%
\end{pgfscope}%
\begin{pgfscope}%
\pgfsys@transformshift{1.005957in}{5.018094in}%
\pgfsys@useobject{currentmarker}{}%
\end{pgfscope}%
\begin{pgfscope}%
\pgfsys@transformshift{3.555020in}{0.823018in}%
\pgfsys@useobject{currentmarker}{}%
\end{pgfscope}%
\begin{pgfscope}%
\pgfsys@transformshift{3.858371in}{4.164447in}%
\pgfsys@useobject{currentmarker}{}%
\end{pgfscope}%
\begin{pgfscope}%
\pgfsys@transformshift{4.129398in}{1.388931in}%
\pgfsys@useobject{currentmarker}{}%
\end{pgfscope}%
\begin{pgfscope}%
\pgfsys@transformshift{4.195701in}{4.726257in}%
\pgfsys@useobject{currentmarker}{}%
\end{pgfscope}%
\begin{pgfscope}%
\pgfsys@transformshift{1.161365in}{3.301302in}%
\pgfsys@useobject{currentmarker}{}%
\end{pgfscope}%
\begin{pgfscope}%
\pgfsys@transformshift{1.423098in}{0.219420in}%
\pgfsys@useobject{currentmarker}{}%
\end{pgfscope}%
\begin{pgfscope}%
\pgfsys@transformshift{3.273436in}{3.131854in}%
\pgfsys@useobject{currentmarker}{}%
\end{pgfscope}%
\begin{pgfscope}%
\pgfsys@transformshift{5.088026in}{0.359695in}%
\pgfsys@useobject{currentmarker}{}%
\end{pgfscope}%
\begin{pgfscope}%
\pgfsys@transformshift{0.857903in}{0.280054in}%
\pgfsys@useobject{currentmarker}{}%
\end{pgfscope}%
\begin{pgfscope}%
\pgfsys@transformshift{4.749579in}{1.785422in}%
\pgfsys@useobject{currentmarker}{}%
\end{pgfscope}%
\begin{pgfscope}%
\pgfsys@transformshift{5.171721in}{1.358709in}%
\pgfsys@useobject{currentmarker}{}%
\end{pgfscope}%
\begin{pgfscope}%
\pgfsys@transformshift{4.111727in}{2.782716in}%
\pgfsys@useobject{currentmarker}{}%
\end{pgfscope}%
\begin{pgfscope}%
\pgfsys@transformshift{0.415054in}{1.209962in}%
\pgfsys@useobject{currentmarker}{}%
\end{pgfscope}%
\begin{pgfscope}%
\pgfsys@transformshift{5.141549in}{1.285380in}%
\pgfsys@useobject{currentmarker}{}%
\end{pgfscope}%
\begin{pgfscope}%
\pgfsys@transformshift{2.860025in}{1.547955in}%
\pgfsys@useobject{currentmarker}{}%
\end{pgfscope}%
\begin{pgfscope}%
\pgfsys@transformshift{4.072569in}{2.711564in}%
\pgfsys@useobject{currentmarker}{}%
\end{pgfscope}%
\begin{pgfscope}%
\pgfsys@transformshift{0.929413in}{3.540510in}%
\pgfsys@useobject{currentmarker}{}%
\end{pgfscope}%
\begin{pgfscope}%
\pgfsys@transformshift{1.672423in}{4.580394in}%
\pgfsys@useobject{currentmarker}{}%
\end{pgfscope}%
\begin{pgfscope}%
\pgfsys@transformshift{1.504960in}{0.164944in}%
\pgfsys@useobject{currentmarker}{}%
\end{pgfscope}%
\begin{pgfscope}%
\pgfsys@transformshift{5.012359in}{0.239438in}%
\pgfsys@useobject{currentmarker}{}%
\end{pgfscope}%
\begin{pgfscope}%
\pgfsys@transformshift{4.994900in}{0.419387in}%
\pgfsys@useobject{currentmarker}{}%
\end{pgfscope}%
\begin{pgfscope}%
\pgfsys@transformshift{3.799732in}{2.915254in}%
\pgfsys@useobject{currentmarker}{}%
\end{pgfscope}%
\begin{pgfscope}%
\pgfsys@transformshift{0.502991in}{3.955770in}%
\pgfsys@useobject{currentmarker}{}%
\end{pgfscope}%
\begin{pgfscope}%
\pgfsys@transformshift{3.813994in}{3.187473in}%
\pgfsys@useobject{currentmarker}{}%
\end{pgfscope}%
\begin{pgfscope}%
\pgfsys@transformshift{4.701532in}{2.199113in}%
\pgfsys@useobject{currentmarker}{}%
\end{pgfscope}%
\begin{pgfscope}%
\pgfsys@transformshift{1.243402in}{4.099571in}%
\pgfsys@useobject{currentmarker}{}%
\end{pgfscope}%
\begin{pgfscope}%
\pgfsys@transformshift{1.645063in}{1.343797in}%
\pgfsys@useobject{currentmarker}{}%
\end{pgfscope}%
\begin{pgfscope}%
\pgfsys@transformshift{1.442982in}{0.836346in}%
\pgfsys@useobject{currentmarker}{}%
\end{pgfscope}%
\begin{pgfscope}%
\pgfsys@transformshift{4.257042in}{4.448359in}%
\pgfsys@useobject{currentmarker}{}%
\end{pgfscope}%
\begin{pgfscope}%
\pgfsys@transformshift{3.728180in}{3.987121in}%
\pgfsys@useobject{currentmarker}{}%
\end{pgfscope}%
\begin{pgfscope}%
\pgfsys@transformshift{5.207983in}{3.500001in}%
\pgfsys@useobject{currentmarker}{}%
\end{pgfscope}%
\begin{pgfscope}%
\pgfsys@transformshift{5.452460in}{0.292382in}%
\pgfsys@useobject{currentmarker}{}%
\end{pgfscope}%
\begin{pgfscope}%
\pgfsys@transformshift{1.658587in}{2.358372in}%
\pgfsys@useobject{currentmarker}{}%
\end{pgfscope}%
\begin{pgfscope}%
\pgfsys@transformshift{2.534700in}{1.489663in}%
\pgfsys@useobject{currentmarker}{}%
\end{pgfscope}%
\begin{pgfscope}%
\pgfsys@transformshift{5.237987in}{3.105524in}%
\pgfsys@useobject{currentmarker}{}%
\end{pgfscope}%
\begin{pgfscope}%
\pgfsys@transformshift{1.135474in}{3.314816in}%
\pgfsys@useobject{currentmarker}{}%
\end{pgfscope}%
\begin{pgfscope}%
\pgfsys@transformshift{2.045850in}{3.106810in}%
\pgfsys@useobject{currentmarker}{}%
\end{pgfscope}%
\begin{pgfscope}%
\pgfsys@transformshift{2.740183in}{0.201660in}%
\pgfsys@useobject{currentmarker}{}%
\end{pgfscope}%
\begin{pgfscope}%
\pgfsys@transformshift{4.383681in}{3.634776in}%
\pgfsys@useobject{currentmarker}{}%
\end{pgfscope}%
\begin{pgfscope}%
\pgfsys@transformshift{0.572970in}{1.004630in}%
\pgfsys@useobject{currentmarker}{}%
\end{pgfscope}%
\begin{pgfscope}%
\pgfsys@transformshift{4.296025in}{4.425933in}%
\pgfsys@useobject{currentmarker}{}%
\end{pgfscope}%
\begin{pgfscope}%
\pgfsys@transformshift{2.946690in}{4.437610in}%
\pgfsys@useobject{currentmarker}{}%
\end{pgfscope}%
\begin{pgfscope}%
\pgfsys@transformshift{1.116639in}{3.375054in}%
\pgfsys@useobject{currentmarker}{}%
\end{pgfscope}%
\begin{pgfscope}%
\pgfsys@transformshift{3.329716in}{1.694266in}%
\pgfsys@useobject{currentmarker}{}%
\end{pgfscope}%
\begin{pgfscope}%
\pgfsys@transformshift{4.431445in}{0.274682in}%
\pgfsys@useobject{currentmarker}{}%
\end{pgfscope}%
\begin{pgfscope}%
\pgfsys@transformshift{1.135623in}{1.387158in}%
\pgfsys@useobject{currentmarker}{}%
\end{pgfscope}%
\begin{pgfscope}%
\pgfsys@transformshift{3.572181in}{1.210712in}%
\pgfsys@useobject{currentmarker}{}%
\end{pgfscope}%
\begin{pgfscope}%
\pgfsys@transformshift{3.335175in}{3.647711in}%
\pgfsys@useobject{currentmarker}{}%
\end{pgfscope}%
\begin{pgfscope}%
\pgfsys@transformshift{1.692887in}{4.933991in}%
\pgfsys@useobject{currentmarker}{}%
\end{pgfscope}%
\begin{pgfscope}%
\pgfsys@transformshift{0.617486in}{0.204596in}%
\pgfsys@useobject{currentmarker}{}%
\end{pgfscope}%
\begin{pgfscope}%
\pgfsys@transformshift{3.116150in}{3.394098in}%
\pgfsys@useobject{currentmarker}{}%
\end{pgfscope}%
\begin{pgfscope}%
\pgfsys@transformshift{4.751193in}{0.904823in}%
\pgfsys@useobject{currentmarker}{}%
\end{pgfscope}%
\begin{pgfscope}%
\pgfsys@transformshift{5.633484in}{4.424260in}%
\pgfsys@useobject{currentmarker}{}%
\end{pgfscope}%
\begin{pgfscope}%
\pgfsys@transformshift{1.002364in}{2.635660in}%
\pgfsys@useobject{currentmarker}{}%
\end{pgfscope}%
\begin{pgfscope}%
\pgfsys@transformshift{1.187107in}{3.058955in}%
\pgfsys@useobject{currentmarker}{}%
\end{pgfscope}%
\begin{pgfscope}%
\pgfsys@transformshift{5.272514in}{3.451341in}%
\pgfsys@useobject{currentmarker}{}%
\end{pgfscope}%
\begin{pgfscope}%
\pgfsys@transformshift{1.931106in}{3.765483in}%
\pgfsys@useobject{currentmarker}{}%
\end{pgfscope}%
\begin{pgfscope}%
\pgfsys@transformshift{0.783807in}{0.400754in}%
\pgfsys@useobject{currentmarker}{}%
\end{pgfscope}%
\begin{pgfscope}%
\pgfsys@transformshift{4.937505in}{3.868115in}%
\pgfsys@useobject{currentmarker}{}%
\end{pgfscope}%
\begin{pgfscope}%
\pgfsys@transformshift{1.041329in}{4.526314in}%
\pgfsys@useobject{currentmarker}{}%
\end{pgfscope}%
\begin{pgfscope}%
\pgfsys@transformshift{1.336890in}{1.356943in}%
\pgfsys@useobject{currentmarker}{}%
\end{pgfscope}%
\begin{pgfscope}%
\pgfsys@transformshift{3.017815in}{2.410528in}%
\pgfsys@useobject{currentmarker}{}%
\end{pgfscope}%
\begin{pgfscope}%
\pgfsys@transformshift{1.525335in}{3.234142in}%
\pgfsys@useobject{currentmarker}{}%
\end{pgfscope}%
\begin{pgfscope}%
\pgfsys@transformshift{0.453860in}{2.284489in}%
\pgfsys@useobject{currentmarker}{}%
\end{pgfscope}%
\begin{pgfscope}%
\pgfsys@transformshift{3.269787in}{3.257458in}%
\pgfsys@useobject{currentmarker}{}%
\end{pgfscope}%
\begin{pgfscope}%
\pgfsys@transformshift{4.618651in}{2.303049in}%
\pgfsys@useobject{currentmarker}{}%
\end{pgfscope}%
\begin{pgfscope}%
\pgfsys@transformshift{2.089220in}{1.125920in}%
\pgfsys@useobject{currentmarker}{}%
\end{pgfscope}%
\begin{pgfscope}%
\pgfsys@transformshift{5.217341in}{2.895969in}%
\pgfsys@useobject{currentmarker}{}%
\end{pgfscope}%
\begin{pgfscope}%
\pgfsys@transformshift{2.088534in}{4.628722in}%
\pgfsys@useobject{currentmarker}{}%
\end{pgfscope}%
\begin{pgfscope}%
\pgfsys@transformshift{5.307196in}{3.747823in}%
\pgfsys@useobject{currentmarker}{}%
\end{pgfscope}%
\begin{pgfscope}%
\pgfsys@transformshift{4.307631in}{3.736590in}%
\pgfsys@useobject{currentmarker}{}%
\end{pgfscope}%
\begin{pgfscope}%
\pgfsys@transformshift{0.914807in}{0.173627in}%
\pgfsys@useobject{currentmarker}{}%
\end{pgfscope}%
\begin{pgfscope}%
\pgfsys@transformshift{4.406107in}{5.023580in}%
\pgfsys@useobject{currentmarker}{}%
\end{pgfscope}%
\begin{pgfscope}%
\pgfsys@transformshift{1.422307in}{4.583681in}%
\pgfsys@useobject{currentmarker}{}%
\end{pgfscope}%
\begin{pgfscope}%
\pgfsys@transformshift{5.692523in}{2.528287in}%
\pgfsys@useobject{currentmarker}{}%
\end{pgfscope}%
\begin{pgfscope}%
\pgfsys@transformshift{3.838500in}{4.727794in}%
\pgfsys@useobject{currentmarker}{}%
\end{pgfscope}%
\begin{pgfscope}%
\pgfsys@transformshift{4.338173in}{0.567388in}%
\pgfsys@useobject{currentmarker}{}%
\end{pgfscope}%
\begin{pgfscope}%
\pgfsys@transformshift{3.464009in}{1.131653in}%
\pgfsys@useobject{currentmarker}{}%
\end{pgfscope}%
\begin{pgfscope}%
\pgfsys@transformshift{4.447097in}{3.637854in}%
\pgfsys@useobject{currentmarker}{}%
\end{pgfscope}%
\begin{pgfscope}%
\pgfsys@transformshift{1.603328in}{0.753646in}%
\pgfsys@useobject{currentmarker}{}%
\end{pgfscope}%
\begin{pgfscope}%
\pgfsys@transformshift{1.078519in}{4.793450in}%
\pgfsys@useobject{currentmarker}{}%
\end{pgfscope}%
\begin{pgfscope}%
\pgfsys@transformshift{2.730874in}{3.742445in}%
\pgfsys@useobject{currentmarker}{}%
\end{pgfscope}%
\begin{pgfscope}%
\pgfsys@transformshift{4.159211in}{2.403385in}%
\pgfsys@useobject{currentmarker}{}%
\end{pgfscope}%
\begin{pgfscope}%
\pgfsys@transformshift{0.897629in}{5.008622in}%
\pgfsys@useobject{currentmarker}{}%
\end{pgfscope}%
\begin{pgfscope}%
\pgfsys@transformshift{4.793179in}{1.230394in}%
\pgfsys@useobject{currentmarker}{}%
\end{pgfscope}%
\begin{pgfscope}%
\pgfsys@transformshift{4.866825in}{3.791616in}%
\pgfsys@useobject{currentmarker}{}%
\end{pgfscope}%
\begin{pgfscope}%
\pgfsys@transformshift{4.201362in}{1.021187in}%
\pgfsys@useobject{currentmarker}{}%
\end{pgfscope}%
\begin{pgfscope}%
\pgfsys@transformshift{1.458017in}{4.162556in}%
\pgfsys@useobject{currentmarker}{}%
\end{pgfscope}%
\begin{pgfscope}%
\pgfsys@transformshift{5.451879in}{1.712120in}%
\pgfsys@useobject{currentmarker}{}%
\end{pgfscope}%
\begin{pgfscope}%
\pgfsys@transformshift{0.814144in}{1.857160in}%
\pgfsys@useobject{currentmarker}{}%
\end{pgfscope}%
\begin{pgfscope}%
\pgfsys@transformshift{3.672156in}{2.474009in}%
\pgfsys@useobject{currentmarker}{}%
\end{pgfscope}%
\begin{pgfscope}%
\pgfsys@transformshift{4.039346in}{4.508270in}%
\pgfsys@useobject{currentmarker}{}%
\end{pgfscope}%
\begin{pgfscope}%
\pgfsys@transformshift{2.453652in}{1.236225in}%
\pgfsys@useobject{currentmarker}{}%
\end{pgfscope}%
\begin{pgfscope}%
\pgfsys@transformshift{2.211993in}{4.616444in}%
\pgfsys@useobject{currentmarker}{}%
\end{pgfscope}%
\begin{pgfscope}%
\pgfsys@transformshift{1.086989in}{1.724608in}%
\pgfsys@useobject{currentmarker}{}%
\end{pgfscope}%
\begin{pgfscope}%
\pgfsys@transformshift{1.653051in}{0.872629in}%
\pgfsys@useobject{currentmarker}{}%
\end{pgfscope}%
\begin{pgfscope}%
\pgfsys@transformshift{0.411566in}{1.867350in}%
\pgfsys@useobject{currentmarker}{}%
\end{pgfscope}%
\begin{pgfscope}%
\pgfsys@transformshift{1.396061in}{2.966841in}%
\pgfsys@useobject{currentmarker}{}%
\end{pgfscope}%
\begin{pgfscope}%
\pgfsys@transformshift{1.967205in}{0.183898in}%
\pgfsys@useobject{currentmarker}{}%
\end{pgfscope}%
\begin{pgfscope}%
\pgfsys@transformshift{4.993233in}{3.909059in}%
\pgfsys@useobject{currentmarker}{}%
\end{pgfscope}%
\begin{pgfscope}%
\pgfsys@transformshift{2.603909in}{4.804440in}%
\pgfsys@useobject{currentmarker}{}%
\end{pgfscope}%
\begin{pgfscope}%
\pgfsys@transformshift{3.971125in}{2.657018in}%
\pgfsys@useobject{currentmarker}{}%
\end{pgfscope}%
\begin{pgfscope}%
\pgfsys@transformshift{4.430655in}{3.753835in}%
\pgfsys@useobject{currentmarker}{}%
\end{pgfscope}%
\begin{pgfscope}%
\pgfsys@transformshift{1.769349in}{3.520444in}%
\pgfsys@useobject{currentmarker}{}%
\end{pgfscope}%
\begin{pgfscope}%
\pgfsys@transformshift{5.023719in}{3.276002in}%
\pgfsys@useobject{currentmarker}{}%
\end{pgfscope}%
\begin{pgfscope}%
\pgfsys@transformshift{3.099848in}{5.010987in}%
\pgfsys@useobject{currentmarker}{}%
\end{pgfscope}%
\begin{pgfscope}%
\pgfsys@transformshift{5.247074in}{2.706760in}%
\pgfsys@useobject{currentmarker}{}%
\end{pgfscope}%
\begin{pgfscope}%
\pgfsys@transformshift{1.135783in}{0.966035in}%
\pgfsys@useobject{currentmarker}{}%
\end{pgfscope}%
\begin{pgfscope}%
\pgfsys@transformshift{5.405105in}{3.568331in}%
\pgfsys@useobject{currentmarker}{}%
\end{pgfscope}%
\begin{pgfscope}%
\pgfsys@transformshift{2.071345in}{4.200198in}%
\pgfsys@useobject{currentmarker}{}%
\end{pgfscope}%
\begin{pgfscope}%
\pgfsys@transformshift{0.601260in}{3.065110in}%
\pgfsys@useobject{currentmarker}{}%
\end{pgfscope}%
\begin{pgfscope}%
\pgfsys@transformshift{0.912272in}{3.224204in}%
\pgfsys@useobject{currentmarker}{}%
\end{pgfscope}%
\begin{pgfscope}%
\pgfsys@transformshift{2.328566in}{3.212541in}%
\pgfsys@useobject{currentmarker}{}%
\end{pgfscope}%
\begin{pgfscope}%
\pgfsys@transformshift{5.058366in}{3.197496in}%
\pgfsys@useobject{currentmarker}{}%
\end{pgfscope}%
\begin{pgfscope}%
\pgfsys@transformshift{1.318411in}{3.601595in}%
\pgfsys@useobject{currentmarker}{}%
\end{pgfscope}%
\begin{pgfscope}%
\pgfsys@transformshift{1.052497in}{3.838521in}%
\pgfsys@useobject{currentmarker}{}%
\end{pgfscope}%
\begin{pgfscope}%
\pgfsys@transformshift{1.430094in}{0.162025in}%
\pgfsys@useobject{currentmarker}{}%
\end{pgfscope}%
\begin{pgfscope}%
\pgfsys@transformshift{4.950999in}{0.485013in}%
\pgfsys@useobject{currentmarker}{}%
\end{pgfscope}%
\begin{pgfscope}%
\pgfsys@transformshift{1.521756in}{1.014880in}%
\pgfsys@useobject{currentmarker}{}%
\end{pgfscope}%
\begin{pgfscope}%
\pgfsys@transformshift{5.565718in}{0.350039in}%
\pgfsys@useobject{currentmarker}{}%
\end{pgfscope}%
\begin{pgfscope}%
\pgfsys@transformshift{2.879477in}{0.398417in}%
\pgfsys@useobject{currentmarker}{}%
\end{pgfscope}%
\begin{pgfscope}%
\pgfsys@transformshift{1.172291in}{1.997690in}%
\pgfsys@useobject{currentmarker}{}%
\end{pgfscope}%
\begin{pgfscope}%
\pgfsys@transformshift{0.849549in}{3.743189in}%
\pgfsys@useobject{currentmarker}{}%
\end{pgfscope}%
\begin{pgfscope}%
\pgfsys@transformshift{2.826877in}{1.757925in}%
\pgfsys@useobject{currentmarker}{}%
\end{pgfscope}%
\begin{pgfscope}%
\pgfsys@transformshift{0.901547in}{0.150291in}%
\pgfsys@useobject{currentmarker}{}%
\end{pgfscope}%
\begin{pgfscope}%
\pgfsys@transformshift{2.673230in}{0.345914in}%
\pgfsys@useobject{currentmarker}{}%
\end{pgfscope}%
\begin{pgfscope}%
\pgfsys@transformshift{4.404805in}{3.481289in}%
\pgfsys@useobject{currentmarker}{}%
\end{pgfscope}%
\begin{pgfscope}%
\pgfsys@transformshift{2.110028in}{1.551713in}%
\pgfsys@useobject{currentmarker}{}%
\end{pgfscope}%
\begin{pgfscope}%
\pgfsys@transformshift{2.935531in}{3.107708in}%
\pgfsys@useobject{currentmarker}{}%
\end{pgfscope}%
\begin{pgfscope}%
\pgfsys@transformshift{3.198466in}{2.466823in}%
\pgfsys@useobject{currentmarker}{}%
\end{pgfscope}%
\begin{pgfscope}%
\pgfsys@transformshift{1.290732in}{1.792606in}%
\pgfsys@useobject{currentmarker}{}%
\end{pgfscope}%
\begin{pgfscope}%
\pgfsys@transformshift{0.590195in}{2.616567in}%
\pgfsys@useobject{currentmarker}{}%
\end{pgfscope}%
\begin{pgfscope}%
\pgfsys@transformshift{0.377274in}{4.273117in}%
\pgfsys@useobject{currentmarker}{}%
\end{pgfscope}%
\begin{pgfscope}%
\pgfsys@transformshift{2.775859in}{0.763579in}%
\pgfsys@useobject{currentmarker}{}%
\end{pgfscope}%
\begin{pgfscope}%
\pgfsys@transformshift{2.240349in}{4.671362in}%
\pgfsys@useobject{currentmarker}{}%
\end{pgfscope}%
\begin{pgfscope}%
\pgfsys@transformshift{0.906564in}{3.062129in}%
\pgfsys@useobject{currentmarker}{}%
\end{pgfscope}%
\begin{pgfscope}%
\pgfsys@transformshift{2.117374in}{1.530968in}%
\pgfsys@useobject{currentmarker}{}%
\end{pgfscope}%
\begin{pgfscope}%
\pgfsys@transformshift{4.112741in}{1.577139in}%
\pgfsys@useobject{currentmarker}{}%
\end{pgfscope}%
\begin{pgfscope}%
\pgfsys@transformshift{2.622086in}{1.357717in}%
\pgfsys@useobject{currentmarker}{}%
\end{pgfscope}%
\begin{pgfscope}%
\pgfsys@transformshift{1.593777in}{1.803734in}%
\pgfsys@useobject{currentmarker}{}%
\end{pgfscope}%
\begin{pgfscope}%
\pgfsys@transformshift{1.055736in}{1.347303in}%
\pgfsys@useobject{currentmarker}{}%
\end{pgfscope}%
\begin{pgfscope}%
\pgfsys@transformshift{0.357861in}{1.276492in}%
\pgfsys@useobject{currentmarker}{}%
\end{pgfscope}%
\begin{pgfscope}%
\pgfsys@transformshift{1.027692in}{3.683413in}%
\pgfsys@useobject{currentmarker}{}%
\end{pgfscope}%
\begin{pgfscope}%
\pgfsys@transformshift{2.014391in}{1.948342in}%
\pgfsys@useobject{currentmarker}{}%
\end{pgfscope}%
\begin{pgfscope}%
\pgfsys@transformshift{4.653895in}{4.179617in}%
\pgfsys@useobject{currentmarker}{}%
\end{pgfscope}%
\begin{pgfscope}%
\pgfsys@transformshift{5.615053in}{4.514308in}%
\pgfsys@useobject{currentmarker}{}%
\end{pgfscope}%
\begin{pgfscope}%
\pgfsys@transformshift{3.064491in}{2.893977in}%
\pgfsys@useobject{currentmarker}{}%
\end{pgfscope}%
\begin{pgfscope}%
\pgfsys@transformshift{2.950130in}{2.840983in}%
\pgfsys@useobject{currentmarker}{}%
\end{pgfscope}%
\begin{pgfscope}%
\pgfsys@transformshift{1.884271in}{0.611466in}%
\pgfsys@useobject{currentmarker}{}%
\end{pgfscope}%
\begin{pgfscope}%
\pgfsys@transformshift{0.612619in}{3.658587in}%
\pgfsys@useobject{currentmarker}{}%
\end{pgfscope}%
\begin{pgfscope}%
\pgfsys@transformshift{1.610573in}{3.723871in}%
\pgfsys@useobject{currentmarker}{}%
\end{pgfscope}%
\begin{pgfscope}%
\pgfsys@transformshift{1.052120in}{2.952316in}%
\pgfsys@useobject{currentmarker}{}%
\end{pgfscope}%
\begin{pgfscope}%
\pgfsys@transformshift{5.471219in}{4.049739in}%
\pgfsys@useobject{currentmarker}{}%
\end{pgfscope}%
\begin{pgfscope}%
\pgfsys@transformshift{3.225456in}{1.133563in}%
\pgfsys@useobject{currentmarker}{}%
\end{pgfscope}%
\begin{pgfscope}%
\pgfsys@transformshift{1.016195in}{0.398547in}%
\pgfsys@useobject{currentmarker}{}%
\end{pgfscope}%
\begin{pgfscope}%
\pgfsys@transformshift{1.222968in}{0.794487in}%
\pgfsys@useobject{currentmarker}{}%
\end{pgfscope}%
\begin{pgfscope}%
\pgfsys@transformshift{2.881760in}{3.415060in}%
\pgfsys@useobject{currentmarker}{}%
\end{pgfscope}%
\begin{pgfscope}%
\pgfsys@transformshift{3.433474in}{0.743647in}%
\pgfsys@useobject{currentmarker}{}%
\end{pgfscope}%
\begin{pgfscope}%
\pgfsys@transformshift{0.686686in}{2.059103in}%
\pgfsys@useobject{currentmarker}{}%
\end{pgfscope}%
\begin{pgfscope}%
\pgfsys@transformshift{5.475632in}{0.997069in}%
\pgfsys@useobject{currentmarker}{}%
\end{pgfscope}%
\begin{pgfscope}%
\pgfsys@transformshift{0.626750in}{1.096951in}%
\pgfsys@useobject{currentmarker}{}%
\end{pgfscope}%
\begin{pgfscope}%
\pgfsys@transformshift{2.790704in}{1.632107in}%
\pgfsys@useobject{currentmarker}{}%
\end{pgfscope}%
\begin{pgfscope}%
\pgfsys@transformshift{4.076890in}{2.634363in}%
\pgfsys@useobject{currentmarker}{}%
\end{pgfscope}%
\begin{pgfscope}%
\pgfsys@transformshift{5.567022in}{4.149523in}%
\pgfsys@useobject{currentmarker}{}%
\end{pgfscope}%
\begin{pgfscope}%
\pgfsys@transformshift{1.318932in}{4.498705in}%
\pgfsys@useobject{currentmarker}{}%
\end{pgfscope}%
\begin{pgfscope}%
\pgfsys@transformshift{3.461611in}{2.752197in}%
\pgfsys@useobject{currentmarker}{}%
\end{pgfscope}%
\begin{pgfscope}%
\pgfsys@transformshift{3.058745in}{3.651221in}%
\pgfsys@useobject{currentmarker}{}%
\end{pgfscope}%
\begin{pgfscope}%
\pgfsys@transformshift{3.549377in}{3.712864in}%
\pgfsys@useobject{currentmarker}{}%
\end{pgfscope}%
\begin{pgfscope}%
\pgfsys@transformshift{1.054896in}{3.811241in}%
\pgfsys@useobject{currentmarker}{}%
\end{pgfscope}%
\begin{pgfscope}%
\pgfsys@transformshift{1.867523in}{2.739463in}%
\pgfsys@useobject{currentmarker}{}%
\end{pgfscope}%
\begin{pgfscope}%
\pgfsys@transformshift{2.416175in}{0.925461in}%
\pgfsys@useobject{currentmarker}{}%
\end{pgfscope}%
\begin{pgfscope}%
\pgfsys@transformshift{4.380308in}{4.814360in}%
\pgfsys@useobject{currentmarker}{}%
\end{pgfscope}%
\begin{pgfscope}%
\pgfsys@transformshift{1.031805in}{2.347916in}%
\pgfsys@useobject{currentmarker}{}%
\end{pgfscope}%
\begin{pgfscope}%
\pgfsys@transformshift{0.521285in}{1.452915in}%
\pgfsys@useobject{currentmarker}{}%
\end{pgfscope}%
\begin{pgfscope}%
\pgfsys@transformshift{0.658785in}{2.354943in}%
\pgfsys@useobject{currentmarker}{}%
\end{pgfscope}%
\end{pgfscope}%
\begin{pgfscope}%
\pgfpathrectangle{\pgfqpoint{0.847223in}{0.554012in}}{\pgfqpoint{6.200000in}{4.530000in}}%
\pgfusepath{clip}%
\pgfsetbuttcap%
\pgfsetroundjoin%
\definecolor{currentfill}{rgb}{0.839216,0.152941,0.156863}%
\pgfsetfillcolor{currentfill}%
\pgfsetlinewidth{1.003750pt}%
\definecolor{currentstroke}{rgb}{0.839216,0.152941,0.156863}%
\pgfsetstrokecolor{currentstroke}%
\pgfsetdash{}{0pt}%
\pgfsys@defobject{currentmarker}{\pgfqpoint{-0.031056in}{-0.031056in}}{\pgfqpoint{0.031056in}{0.031056in}}{%
\pgfpathmoveto{\pgfqpoint{0.000000in}{-0.031056in}}%
\pgfpathcurveto{\pgfqpoint{0.008236in}{-0.031056in}}{\pgfqpoint{0.016136in}{-0.027784in}}{\pgfqpoint{0.021960in}{-0.021960in}}%
\pgfpathcurveto{\pgfqpoint{0.027784in}{-0.016136in}}{\pgfqpoint{0.031056in}{-0.008236in}}{\pgfqpoint{0.031056in}{0.000000in}}%
\pgfpathcurveto{\pgfqpoint{0.031056in}{0.008236in}}{\pgfqpoint{0.027784in}{0.016136in}}{\pgfqpoint{0.021960in}{0.021960in}}%
\pgfpathcurveto{\pgfqpoint{0.016136in}{0.027784in}}{\pgfqpoint{0.008236in}{0.031056in}}{\pgfqpoint{0.000000in}{0.031056in}}%
\pgfpathcurveto{\pgfqpoint{-0.008236in}{0.031056in}}{\pgfqpoint{-0.016136in}{0.027784in}}{\pgfqpoint{-0.021960in}{0.021960in}}%
\pgfpathcurveto{\pgfqpoint{-0.027784in}{0.016136in}}{\pgfqpoint{-0.031056in}{0.008236in}}{\pgfqpoint{-0.031056in}{0.000000in}}%
\pgfpathcurveto{\pgfqpoint{-0.031056in}{-0.008236in}}{\pgfqpoint{-0.027784in}{-0.016136in}}{\pgfqpoint{-0.021960in}{-0.021960in}}%
\pgfpathcurveto{\pgfqpoint{-0.016136in}{-0.027784in}}{\pgfqpoint{-0.008236in}{-0.031056in}}{\pgfqpoint{0.000000in}{-0.031056in}}%
\pgfpathlineto{\pgfqpoint{0.000000in}{-0.031056in}}%
\pgfpathclose%
\pgfusepath{stroke,fill}%
}%
\begin{pgfscope}%
\pgfsys@transformshift{0.822064in}{4.276519in}%
\pgfsys@useobject{currentmarker}{}%
\end{pgfscope}%
\begin{pgfscope}%
\pgfsys@transformshift{2.863214in}{1.073054in}%
\pgfsys@useobject{currentmarker}{}%
\end{pgfscope}%
\begin{pgfscope}%
\pgfsys@transformshift{1.431887in}{2.422736in}%
\pgfsys@useobject{currentmarker}{}%
\end{pgfscope}%
\begin{pgfscope}%
\pgfsys@transformshift{0.959832in}{4.161604in}%
\pgfsys@useobject{currentmarker}{}%
\end{pgfscope}%
\begin{pgfscope}%
\pgfsys@transformshift{0.833197in}{3.979089in}%
\pgfsys@useobject{currentmarker}{}%
\end{pgfscope}%
\begin{pgfscope}%
\pgfsys@transformshift{1.733063in}{1.890834in}%
\pgfsys@useobject{currentmarker}{}%
\end{pgfscope}%
\begin{pgfscope}%
\pgfsys@transformshift{1.270875in}{2.161366in}%
\pgfsys@useobject{currentmarker}{}%
\end{pgfscope}%
\begin{pgfscope}%
\pgfsys@transformshift{3.280774in}{0.879420in}%
\pgfsys@useobject{currentmarker}{}%
\end{pgfscope}%
\begin{pgfscope}%
\pgfsys@transformshift{0.899915in}{4.273981in}%
\pgfsys@useobject{currentmarker}{}%
\end{pgfscope}%
\begin{pgfscope}%
\pgfsys@transformshift{1.306958in}{2.641278in}%
\pgfsys@useobject{currentmarker}{}%
\end{pgfscope}%
\begin{pgfscope}%
\pgfsys@transformshift{1.445449in}{2.102566in}%
\pgfsys@useobject{currentmarker}{}%
\end{pgfscope}%
\begin{pgfscope}%
\pgfsys@transformshift{1.739835in}{1.729515in}%
\pgfsys@useobject{currentmarker}{}%
\end{pgfscope}%
\begin{pgfscope}%
\pgfsys@transformshift{5.170849in}{0.587740in}%
\pgfsys@useobject{currentmarker}{}%
\end{pgfscope}%
\begin{pgfscope}%
\pgfsys@transformshift{4.226408in}{0.794713in}%
\pgfsys@useobject{currentmarker}{}%
\end{pgfscope}%
\begin{pgfscope}%
\pgfsys@transformshift{1.459180in}{1.824106in}%
\pgfsys@useobject{currentmarker}{}%
\end{pgfscope}%
\begin{pgfscope}%
\pgfsys@transformshift{4.374784in}{0.593946in}%
\pgfsys@useobject{currentmarker}{}%
\end{pgfscope}%
\begin{pgfscope}%
\pgfsys@transformshift{2.143101in}{1.234730in}%
\pgfsys@useobject{currentmarker}{}%
\end{pgfscope}%
\begin{pgfscope}%
\pgfsys@transformshift{3.171260in}{0.853262in}%
\pgfsys@useobject{currentmarker}{}%
\end{pgfscope}%
\begin{pgfscope}%
\pgfsys@transformshift{1.958903in}{1.465736in}%
\pgfsys@useobject{currentmarker}{}%
\end{pgfscope}%
\begin{pgfscope}%
\pgfsys@transformshift{3.354107in}{0.931903in}%
\pgfsys@useobject{currentmarker}{}%
\end{pgfscope}%
\begin{pgfscope}%
\pgfsys@transformshift{4.731579in}{0.637848in}%
\pgfsys@useobject{currentmarker}{}%
\end{pgfscope}%
\begin{pgfscope}%
\pgfsys@transformshift{1.237245in}{2.448096in}%
\pgfsys@useobject{currentmarker}{}%
\end{pgfscope}%
\begin{pgfscope}%
\pgfsys@transformshift{2.035744in}{1.610077in}%
\pgfsys@useobject{currentmarker}{}%
\end{pgfscope}%
\begin{pgfscope}%
\pgfsys@transformshift{3.555020in}{0.823018in}%
\pgfsys@useobject{currentmarker}{}%
\end{pgfscope}%
\begin{pgfscope}%
\pgfsys@transformshift{0.929413in}{3.540510in}%
\pgfsys@useobject{currentmarker}{}%
\end{pgfscope}%
\begin{pgfscope}%
\pgfsys@transformshift{1.116639in}{3.375054in}%
\pgfsys@useobject{currentmarker}{}%
\end{pgfscope}%
\begin{pgfscope}%
\pgfsys@transformshift{1.187107in}{3.058955in}%
\pgfsys@useobject{currentmarker}{}%
\end{pgfscope}%
\begin{pgfscope}%
\pgfsys@transformshift{2.453652in}{1.236225in}%
\pgfsys@useobject{currentmarker}{}%
\end{pgfscope}%
\begin{pgfscope}%
\pgfsys@transformshift{0.912272in}{3.224204in}%
\pgfsys@useobject{currentmarker}{}%
\end{pgfscope}%
\begin{pgfscope}%
\pgfsys@transformshift{0.849549in}{3.743189in}%
\pgfsys@useobject{currentmarker}{}%
\end{pgfscope}%
\begin{pgfscope}%
\pgfsys@transformshift{2.110028in}{1.551713in}%
\pgfsys@useobject{currentmarker}{}%
\end{pgfscope}%
\begin{pgfscope}%
\pgfsys@transformshift{2.117374in}{1.530968in}%
\pgfsys@useobject{currentmarker}{}%
\end{pgfscope}%
\begin{pgfscope}%
\pgfsys@transformshift{1.593777in}{1.803734in}%
\pgfsys@useobject{currentmarker}{}%
\end{pgfscope}%
\begin{pgfscope}%
\pgfsys@transformshift{1.027692in}{3.683413in}%
\pgfsys@useobject{currentmarker}{}%
\end{pgfscope}%
\begin{pgfscope}%
\pgfsys@transformshift{1.052120in}{2.952316in}%
\pgfsys@useobject{currentmarker}{}%
\end{pgfscope}%
\begin{pgfscope}%
\pgfsys@transformshift{3.433474in}{0.743647in}%
\pgfsys@useobject{currentmarker}{}%
\end{pgfscope}%
\end{pgfscope}%
\begin{pgfscope}%
\pgfpathrectangle{\pgfqpoint{0.847223in}{0.554012in}}{\pgfqpoint{6.200000in}{4.530000in}}%
\pgfusepath{clip}%
\pgfsetbuttcap%
\pgfsetmiterjoin%
\definecolor{currentfill}{rgb}{0.501961,0.501961,0.501961}%
\pgfsetfillcolor{currentfill}%
\pgfsetfillopacity{0.200000}%
\pgfsetlinewidth{1.003750pt}%
\definecolor{currentstroke}{rgb}{0.501961,0.501961,0.501961}%
\pgfsetstrokecolor{currentstroke}%
\pgfsetstrokeopacity{0.200000}%
\pgfsetdash{}{0pt}%
\pgfpathmoveto{\pgfqpoint{0.759998in}{4.259481in}}%
\pgfpathlineto{\pgfqpoint{0.764442in}{4.217893in}}%
\pgfpathlineto{\pgfqpoint{0.768887in}{4.177136in}}%
\pgfpathlineto{\pgfqpoint{0.773331in}{4.137184in}}%
\pgfpathlineto{\pgfqpoint{0.777775in}{4.098015in}}%
\pgfpathlineto{\pgfqpoint{0.782220in}{4.059605in}}%
\pgfpathlineto{\pgfqpoint{0.786664in}{4.021934in}}%
\pgfpathlineto{\pgfqpoint{0.791108in}{3.984978in}}%
\pgfpathlineto{\pgfqpoint{0.795553in}{3.948720in}}%
\pgfpathlineto{\pgfqpoint{0.799997in}{3.913138in}}%
\pgfpathlineto{\pgfqpoint{0.804441in}{3.878214in}}%
\pgfpathlineto{\pgfqpoint{0.808886in}{3.843931in}}%
\pgfpathlineto{\pgfqpoint{0.813330in}{3.810270in}}%
\pgfpathlineto{\pgfqpoint{0.817775in}{3.777214in}}%
\pgfpathlineto{\pgfqpoint{0.822219in}{3.744749in}}%
\pgfpathlineto{\pgfqpoint{0.826663in}{3.712857in}}%
\pgfpathlineto{\pgfqpoint{0.831108in}{3.681524in}}%
\pgfpathlineto{\pgfqpoint{0.835552in}{3.650736in}}%
\pgfpathlineto{\pgfqpoint{0.839996in}{3.620477in}}%
\pgfpathlineto{\pgfqpoint{0.844441in}{3.590736in}}%
\pgfpathlineto{\pgfqpoint{0.848885in}{3.561497in}}%
\pgfpathlineto{\pgfqpoint{0.853329in}{3.532750in}}%
\pgfpathlineto{\pgfqpoint{0.857774in}{3.504482in}}%
\pgfpathlineto{\pgfqpoint{0.862218in}{3.476680in}}%
\pgfpathlineto{\pgfqpoint{0.866662in}{3.449333in}}%
\pgfpathlineto{\pgfqpoint{0.871107in}{3.422431in}}%
\pgfpathlineto{\pgfqpoint{0.875551in}{3.395962in}}%
\pgfpathlineto{\pgfqpoint{0.879995in}{3.369916in}}%
\pgfpathlineto{\pgfqpoint{0.884440in}{3.344283in}}%
\pgfpathlineto{\pgfqpoint{0.888884in}{3.319053in}}%
\pgfpathlineto{\pgfqpoint{0.893328in}{3.294217in}}%
\pgfpathlineto{\pgfqpoint{0.897773in}{3.269766in}}%
\pgfpathlineto{\pgfqpoint{0.902217in}{3.245691in}}%
\pgfpathlineto{\pgfqpoint{0.906661in}{3.221983in}}%
\pgfpathlineto{\pgfqpoint{0.911106in}{3.198633in}}%
\pgfpathlineto{\pgfqpoint{0.915550in}{3.175635in}}%
\pgfpathlineto{\pgfqpoint{0.919994in}{3.152979in}}%
\pgfpathlineto{\pgfqpoint{0.924439in}{3.130659in}}%
\pgfpathlineto{\pgfqpoint{0.928883in}{3.108666in}}%
\pgfpathlineto{\pgfqpoint{0.933328in}{3.086995in}}%
\pgfpathlineto{\pgfqpoint{0.937772in}{3.065637in}}%
\pgfpathlineto{\pgfqpoint{0.942216in}{3.044586in}}%
\pgfpathlineto{\pgfqpoint{0.946661in}{3.023836in}}%
\pgfpathlineto{\pgfqpoint{0.951105in}{3.003379in}}%
\pgfpathlineto{\pgfqpoint{0.955549in}{2.983211in}}%
\pgfpathlineto{\pgfqpoint{0.959994in}{2.963324in}}%
\pgfpathlineto{\pgfqpoint{0.964438in}{2.943714in}}%
\pgfpathlineto{\pgfqpoint{0.968882in}{2.924373in}}%
\pgfpathlineto{\pgfqpoint{0.973327in}{2.905297in}}%
\pgfpathlineto{\pgfqpoint{0.977771in}{2.886481in}}%
\pgfpathlineto{\pgfqpoint{0.982215in}{2.867919in}}%
\pgfpathlineto{\pgfqpoint{0.986660in}{2.849605in}}%
\pgfpathlineto{\pgfqpoint{0.991104in}{2.831536in}}%
\pgfpathlineto{\pgfqpoint{0.995548in}{2.813705in}}%
\pgfpathlineto{\pgfqpoint{0.999993in}{2.796109in}}%
\pgfpathlineto{\pgfqpoint{1.004437in}{2.778743in}}%
\pgfpathlineto{\pgfqpoint{1.008881in}{2.761602in}}%
\pgfpathlineto{\pgfqpoint{1.013326in}{2.744683in}}%
\pgfpathlineto{\pgfqpoint{1.017770in}{2.727980in}}%
\pgfpathlineto{\pgfqpoint{1.022214in}{2.711489in}}%
\pgfpathlineto{\pgfqpoint{1.026659in}{2.695207in}}%
\pgfpathlineto{\pgfqpoint{1.031103in}{2.679130in}}%
\pgfpathlineto{\pgfqpoint{1.035548in}{2.663254in}}%
\pgfpathlineto{\pgfqpoint{1.039992in}{2.647574in}}%
\pgfpathlineto{\pgfqpoint{1.044436in}{2.632088in}}%
\pgfpathlineto{\pgfqpoint{1.048881in}{2.616792in}}%
\pgfpathlineto{\pgfqpoint{1.053325in}{2.601682in}}%
\pgfpathlineto{\pgfqpoint{1.057769in}{2.586756in}}%
\pgfpathlineto{\pgfqpoint{1.062214in}{2.572009in}}%
\pgfpathlineto{\pgfqpoint{1.066658in}{2.557438in}}%
\pgfpathlineto{\pgfqpoint{1.071102in}{2.543041in}}%
\pgfpathlineto{\pgfqpoint{1.075547in}{2.528814in}}%
\pgfpathlineto{\pgfqpoint{1.079991in}{2.514754in}}%
\pgfpathlineto{\pgfqpoint{1.084435in}{2.500859in}}%
\pgfpathlineto{\pgfqpoint{1.088880in}{2.487125in}}%
\pgfpathlineto{\pgfqpoint{1.093324in}{2.473549in}}%
\pgfpathlineto{\pgfqpoint{1.097768in}{2.460130in}}%
\pgfpathlineto{\pgfqpoint{1.102213in}{2.446864in}}%
\pgfpathlineto{\pgfqpoint{1.106657in}{2.433748in}}%
\pgfpathlineto{\pgfqpoint{1.111101in}{2.420781in}}%
\pgfpathlineto{\pgfqpoint{1.115546in}{2.407959in}}%
\pgfpathlineto{\pgfqpoint{1.119990in}{2.395281in}}%
\pgfpathlineto{\pgfqpoint{1.124434in}{2.382743in}}%
\pgfpathlineto{\pgfqpoint{1.128879in}{2.370343in}}%
\pgfpathlineto{\pgfqpoint{1.133323in}{2.358080in}}%
\pgfpathlineto{\pgfqpoint{1.137767in}{2.345951in}}%
\pgfpathlineto{\pgfqpoint{1.142212in}{2.333953in}}%
\pgfpathlineto{\pgfqpoint{1.146656in}{2.322085in}}%
\pgfpathlineto{\pgfqpoint{1.151101in}{2.310345in}}%
\pgfpathlineto{\pgfqpoint{1.155545in}{2.298730in}}%
\pgfpathlineto{\pgfqpoint{1.159989in}{2.287238in}}%
\pgfpathlineto{\pgfqpoint{1.164434in}{2.275868in}}%
\pgfpathlineto{\pgfqpoint{1.168878in}{2.264618in}}%
\pgfpathlineto{\pgfqpoint{1.173322in}{2.253485in}}%
\pgfpathlineto{\pgfqpoint{1.177767in}{2.242468in}}%
\pgfpathlineto{\pgfqpoint{1.182211in}{2.231566in}}%
\pgfpathlineto{\pgfqpoint{1.186655in}{2.220775in}}%
\pgfpathlineto{\pgfqpoint{1.191100in}{2.210096in}}%
\pgfpathlineto{\pgfqpoint{1.195544in}{2.199525in}}%
\pgfpathlineto{\pgfqpoint{1.199988in}{2.189061in}}%
\pgfpathlineto{\pgfqpoint{1.204433in}{2.178703in}}%
\pgfpathlineto{\pgfqpoint{1.208877in}{2.168449in}}%
\pgfpathlineto{\pgfqpoint{1.213321in}{2.158298in}}%
\pgfpathlineto{\pgfqpoint{1.217766in}{2.148247in}}%
\pgfpathlineto{\pgfqpoint{1.222210in}{2.138296in}}%
\pgfpathlineto{\pgfqpoint{1.226654in}{2.128443in}}%
\pgfpathlineto{\pgfqpoint{1.231099in}{2.118686in}}%
\pgfpathlineto{\pgfqpoint{1.235543in}{2.109025in}}%
\pgfpathlineto{\pgfqpoint{1.239987in}{2.099457in}}%
\pgfpathlineto{\pgfqpoint{1.244432in}{2.089982in}}%
\pgfpathlineto{\pgfqpoint{1.248876in}{2.080597in}}%
\pgfpathlineto{\pgfqpoint{1.253320in}{2.071303in}}%
\pgfpathlineto{\pgfqpoint{1.257765in}{2.062097in}}%
\pgfpathlineto{\pgfqpoint{1.262209in}{2.052978in}}%
\pgfpathlineto{\pgfqpoint{1.266654in}{2.043945in}}%
\pgfpathlineto{\pgfqpoint{1.271098in}{2.034997in}}%
\pgfpathlineto{\pgfqpoint{1.275542in}{2.026132in}}%
\pgfpathlineto{\pgfqpoint{1.279987in}{2.017350in}}%
\pgfpathlineto{\pgfqpoint{1.284431in}{2.008649in}}%
\pgfpathlineto{\pgfqpoint{1.288875in}{2.000029in}}%
\pgfpathlineto{\pgfqpoint{1.293320in}{1.991487in}}%
\pgfpathlineto{\pgfqpoint{1.297764in}{1.983023in}}%
\pgfpathlineto{\pgfqpoint{1.302208in}{1.974636in}}%
\pgfpathlineto{\pgfqpoint{1.306653in}{1.966326in}}%
\pgfpathlineto{\pgfqpoint{1.311097in}{1.958089in}}%
\pgfpathlineto{\pgfqpoint{1.315541in}{1.949927in}}%
\pgfpathlineto{\pgfqpoint{1.319986in}{1.941838in}}%
\pgfpathlineto{\pgfqpoint{1.324430in}{1.933820in}}%
\pgfpathlineto{\pgfqpoint{1.328874in}{1.925873in}}%
\pgfpathlineto{\pgfqpoint{1.333319in}{1.917997in}}%
\pgfpathlineto{\pgfqpoint{1.337763in}{1.910189in}}%
\pgfpathlineto{\pgfqpoint{1.342207in}{1.902449in}}%
\pgfpathlineto{\pgfqpoint{1.346652in}{1.894777in}}%
\pgfpathlineto{\pgfqpoint{1.351096in}{1.887171in}}%
\pgfpathlineto{\pgfqpoint{1.355540in}{1.879631in}}%
\pgfpathlineto{\pgfqpoint{1.359985in}{1.872155in}}%
\pgfpathlineto{\pgfqpoint{1.364429in}{1.864743in}}%
\pgfpathlineto{\pgfqpoint{1.368873in}{1.857395in}}%
\pgfpathlineto{\pgfqpoint{1.373318in}{1.850108in}}%
\pgfpathlineto{\pgfqpoint{1.377762in}{1.842883in}}%
\pgfpathlineto{\pgfqpoint{1.382207in}{1.835719in}}%
\pgfpathlineto{\pgfqpoint{1.386651in}{1.828614in}}%
\pgfpathlineto{\pgfqpoint{1.391095in}{1.821569in}}%
\pgfpathlineto{\pgfqpoint{1.395540in}{1.814582in}}%
\pgfpathlineto{\pgfqpoint{1.399984in}{1.807653in}}%
\pgfpathlineto{\pgfqpoint{1.404428in}{1.800781in}}%
\pgfpathlineto{\pgfqpoint{1.408873in}{1.793965in}}%
\pgfpathlineto{\pgfqpoint{1.413317in}{1.787205in}}%
\pgfpathlineto{\pgfqpoint{1.417761in}{1.780499in}}%
\pgfpathlineto{\pgfqpoint{1.422206in}{1.773848in}}%
\pgfpathlineto{\pgfqpoint{1.426650in}{1.767251in}}%
\pgfpathlineto{\pgfqpoint{1.431094in}{1.760707in}}%
\pgfpathlineto{\pgfqpoint{1.435539in}{1.754214in}}%
\pgfpathlineto{\pgfqpoint{1.439983in}{1.747774in}}%
\pgfpathlineto{\pgfqpoint{1.444427in}{1.741385in}}%
\pgfpathlineto{\pgfqpoint{1.448872in}{1.735046in}}%
\pgfpathlineto{\pgfqpoint{1.453316in}{1.728757in}}%
\pgfpathlineto{\pgfqpoint{1.457760in}{1.722517in}}%
\pgfpathlineto{\pgfqpoint{1.462205in}{1.716326in}}%
\pgfpathlineto{\pgfqpoint{1.466649in}{1.710183in}}%
\pgfpathlineto{\pgfqpoint{1.471093in}{1.704088in}}%
\pgfpathlineto{\pgfqpoint{1.475538in}{1.698040in}}%
\pgfpathlineto{\pgfqpoint{1.479982in}{1.692038in}}%
\pgfpathlineto{\pgfqpoint{1.484426in}{1.686083in}}%
\pgfpathlineto{\pgfqpoint{1.488871in}{1.680173in}}%
\pgfpathlineto{\pgfqpoint{1.493315in}{1.674307in}}%
\pgfpathlineto{\pgfqpoint{1.497760in}{1.668486in}}%
\pgfpathlineto{\pgfqpoint{1.502204in}{1.662709in}}%
\pgfpathlineto{\pgfqpoint{1.506648in}{1.656976in}}%
\pgfpathlineto{\pgfqpoint{1.511093in}{1.651285in}}%
\pgfpathlineto{\pgfqpoint{1.515537in}{1.645637in}}%
\pgfpathlineto{\pgfqpoint{1.519981in}{1.640031in}}%
\pgfpathlineto{\pgfqpoint{1.524426in}{1.634466in}}%
\pgfpathlineto{\pgfqpoint{1.528870in}{1.628942in}}%
\pgfpathlineto{\pgfqpoint{1.533314in}{1.623459in}}%
\pgfpathlineto{\pgfqpoint{1.537759in}{1.618016in}}%
\pgfpathlineto{\pgfqpoint{1.542203in}{1.612613in}}%
\pgfpathlineto{\pgfqpoint{1.546647in}{1.607249in}}%
\pgfpathlineto{\pgfqpoint{1.551092in}{1.601924in}}%
\pgfpathlineto{\pgfqpoint{1.555536in}{1.596637in}}%
\pgfpathlineto{\pgfqpoint{1.559980in}{1.591389in}}%
\pgfpathlineto{\pgfqpoint{1.564425in}{1.586178in}}%
\pgfpathlineto{\pgfqpoint{1.568869in}{1.581004in}}%
\pgfpathlineto{\pgfqpoint{1.573313in}{1.575867in}}%
\pgfpathlineto{\pgfqpoint{1.577758in}{1.570766in}}%
\pgfpathlineto{\pgfqpoint{1.582202in}{1.565702in}}%
\pgfpathlineto{\pgfqpoint{1.586646in}{1.560673in}}%
\pgfpathlineto{\pgfqpoint{1.591091in}{1.555679in}}%
\pgfpathlineto{\pgfqpoint{1.595535in}{1.550720in}}%
\pgfpathlineto{\pgfqpoint{1.599979in}{1.545796in}}%
\pgfpathlineto{\pgfqpoint{1.604424in}{1.540906in}}%
\pgfpathlineto{\pgfqpoint{1.608868in}{1.536050in}}%
\pgfpathlineto{\pgfqpoint{1.613313in}{1.531227in}}%
\pgfpathlineto{\pgfqpoint{1.617757in}{1.526437in}}%
\pgfpathlineto{\pgfqpoint{1.622201in}{1.521681in}}%
\pgfpathlineto{\pgfqpoint{1.626646in}{1.516956in}}%
\pgfpathlineto{\pgfqpoint{1.631090in}{1.512264in}}%
\pgfpathlineto{\pgfqpoint{1.635534in}{1.507604in}}%
\pgfpathlineto{\pgfqpoint{1.639979in}{1.502975in}}%
\pgfpathlineto{\pgfqpoint{1.644423in}{1.498377in}}%
\pgfpathlineto{\pgfqpoint{1.648867in}{1.493810in}}%
\pgfpathlineto{\pgfqpoint{1.653312in}{1.489273in}}%
\pgfpathlineto{\pgfqpoint{1.657756in}{1.484767in}}%
\pgfpathlineto{\pgfqpoint{1.662200in}{1.480291in}}%
\pgfpathlineto{\pgfqpoint{1.666645in}{1.475844in}}%
\pgfpathlineto{\pgfqpoint{1.671089in}{1.471427in}}%
\pgfpathlineto{\pgfqpoint{1.675533in}{1.467039in}}%
\pgfpathlineto{\pgfqpoint{1.679978in}{1.462679in}}%
\pgfpathlineto{\pgfqpoint{1.684422in}{1.458348in}}%
\pgfpathlineto{\pgfqpoint{1.688866in}{1.454045in}}%
\pgfpathlineto{\pgfqpoint{1.693311in}{1.449771in}}%
\pgfpathlineto{\pgfqpoint{1.697755in}{1.445523in}}%
\pgfpathlineto{\pgfqpoint{1.702199in}{1.441304in}}%
\pgfpathlineto{\pgfqpoint{1.706644in}{1.437111in}}%
\pgfpathlineto{\pgfqpoint{1.711088in}{1.432945in}}%
\pgfpathlineto{\pgfqpoint{1.715533in}{1.428806in}}%
\pgfpathlineto{\pgfqpoint{1.719977in}{1.424693in}}%
\pgfpathlineto{\pgfqpoint{1.724421in}{1.420606in}}%
\pgfpathlineto{\pgfqpoint{1.728866in}{1.416545in}}%
\pgfpathlineto{\pgfqpoint{1.733310in}{1.412510in}}%
\pgfpathlineto{\pgfqpoint{1.737754in}{1.408500in}}%
\pgfpathlineto{\pgfqpoint{1.742199in}{1.404515in}}%
\pgfpathlineto{\pgfqpoint{1.746643in}{1.400555in}}%
\pgfpathlineto{\pgfqpoint{1.751087in}{1.396620in}}%
\pgfpathlineto{\pgfqpoint{1.755532in}{1.392709in}}%
\pgfpathlineto{\pgfqpoint{1.759976in}{1.388823in}}%
\pgfpathlineto{\pgfqpoint{1.764420in}{1.384960in}}%
\pgfpathlineto{\pgfqpoint{1.768865in}{1.381121in}}%
\pgfpathlineto{\pgfqpoint{1.773309in}{1.377306in}}%
\pgfpathlineto{\pgfqpoint{1.777753in}{1.373514in}}%
\pgfpathlineto{\pgfqpoint{1.782198in}{1.369745in}}%
\pgfpathlineto{\pgfqpoint{1.786642in}{1.365999in}}%
\pgfpathlineto{\pgfqpoint{1.791086in}{1.362276in}}%
\pgfpathlineto{\pgfqpoint{1.795531in}{1.358575in}}%
\pgfpathlineto{\pgfqpoint{1.799975in}{1.354896in}}%
\pgfpathlineto{\pgfqpoint{1.804419in}{1.351240in}}%
\pgfpathlineto{\pgfqpoint{1.808864in}{1.347605in}}%
\pgfpathlineto{\pgfqpoint{1.813308in}{1.343993in}}%
\pgfpathlineto{\pgfqpoint{1.817752in}{1.340401in}}%
\pgfpathlineto{\pgfqpoint{1.822197in}{1.336831in}}%
\pgfpathlineto{\pgfqpoint{1.826641in}{1.333282in}}%
\pgfpathlineto{\pgfqpoint{1.831086in}{1.329754in}}%
\pgfpathlineto{\pgfqpoint{1.835530in}{1.326247in}}%
\pgfpathlineto{\pgfqpoint{1.839974in}{1.322760in}}%
\pgfpathlineto{\pgfqpoint{1.844419in}{1.319294in}}%
\pgfpathlineto{\pgfqpoint{1.848863in}{1.315848in}}%
\pgfpathlineto{\pgfqpoint{1.853307in}{1.312422in}}%
\pgfpathlineto{\pgfqpoint{1.857752in}{1.309016in}}%
\pgfpathlineto{\pgfqpoint{1.862196in}{1.305629in}}%
\pgfpathlineto{\pgfqpoint{1.866640in}{1.302262in}}%
\pgfpathlineto{\pgfqpoint{1.871085in}{1.298914in}}%
\pgfpathlineto{\pgfqpoint{1.875529in}{1.295586in}}%
\pgfpathlineto{\pgfqpoint{1.879973in}{1.292276in}}%
\pgfpathlineto{\pgfqpoint{1.884418in}{1.288986in}}%
\pgfpathlineto{\pgfqpoint{1.888862in}{1.285714in}}%
\pgfpathlineto{\pgfqpoint{1.893306in}{1.282460in}}%
\pgfpathlineto{\pgfqpoint{1.897751in}{1.279225in}}%
\pgfpathlineto{\pgfqpoint{1.902195in}{1.276008in}}%
\pgfpathlineto{\pgfqpoint{1.906639in}{1.272810in}}%
\pgfpathlineto{\pgfqpoint{1.911084in}{1.269629in}}%
\pgfpathlineto{\pgfqpoint{1.915528in}{1.266466in}}%
\pgfpathlineto{\pgfqpoint{1.919972in}{1.263320in}}%
\pgfpathlineto{\pgfqpoint{1.924417in}{1.260192in}}%
\pgfpathlineto{\pgfqpoint{1.928861in}{1.257081in}}%
\pgfpathlineto{\pgfqpoint{1.933305in}{1.253988in}}%
\pgfpathlineto{\pgfqpoint{1.937750in}{1.250911in}}%
\pgfpathlineto{\pgfqpoint{1.942194in}{1.247852in}}%
\pgfpathlineto{\pgfqpoint{1.946639in}{1.244809in}}%
\pgfpathlineto{\pgfqpoint{1.951083in}{1.241783in}}%
\pgfpathlineto{\pgfqpoint{1.955527in}{1.238773in}}%
\pgfpathlineto{\pgfqpoint{1.959972in}{1.235780in}}%
\pgfpathlineto{\pgfqpoint{1.964416in}{1.232803in}}%
\pgfpathlineto{\pgfqpoint{1.968860in}{1.229842in}}%
\pgfpathlineto{\pgfqpoint{1.973305in}{1.226897in}}%
\pgfpathlineto{\pgfqpoint{1.977749in}{1.223967in}}%
\pgfpathlineto{\pgfqpoint{1.982193in}{1.221054in}}%
\pgfpathlineto{\pgfqpoint{1.986638in}{1.218156in}}%
\pgfpathlineto{\pgfqpoint{1.991082in}{1.215273in}}%
\pgfpathlineto{\pgfqpoint{1.995526in}{1.212406in}}%
\pgfpathlineto{\pgfqpoint{1.999971in}{1.209554in}}%
\pgfpathlineto{\pgfqpoint{2.004415in}{1.206717in}}%
\pgfpathlineto{\pgfqpoint{2.008859in}{1.203895in}}%
\pgfpathlineto{\pgfqpoint{2.013304in}{1.201088in}}%
\pgfpathlineto{\pgfqpoint{2.017748in}{1.198296in}}%
\pgfpathlineto{\pgfqpoint{2.022192in}{1.195518in}}%
\pgfpathlineto{\pgfqpoint{2.026637in}{1.192755in}}%
\pgfpathlineto{\pgfqpoint{2.031081in}{1.190006in}}%
\pgfpathlineto{\pgfqpoint{2.035525in}{1.187271in}}%
\pgfpathlineto{\pgfqpoint{2.039970in}{1.184551in}}%
\pgfpathlineto{\pgfqpoint{2.044414in}{1.181844in}}%
\pgfpathlineto{\pgfqpoint{2.048858in}{1.179152in}}%
\pgfpathlineto{\pgfqpoint{2.053303in}{1.176473in}}%
\pgfpathlineto{\pgfqpoint{2.057747in}{1.173808in}}%
\pgfpathlineto{\pgfqpoint{2.062192in}{1.171157in}}%
\pgfpathlineto{\pgfqpoint{2.066636in}{1.168519in}}%
\pgfpathlineto{\pgfqpoint{2.071080in}{1.165895in}}%
\pgfpathlineto{\pgfqpoint{2.075525in}{1.163284in}}%
\pgfpathlineto{\pgfqpoint{2.079969in}{1.160686in}}%
\pgfpathlineto{\pgfqpoint{2.084413in}{1.158102in}}%
\pgfpathlineto{\pgfqpoint{2.088858in}{1.155530in}}%
\pgfpathlineto{\pgfqpoint{2.093302in}{1.152971in}}%
\pgfpathlineto{\pgfqpoint{2.097746in}{1.150425in}}%
\pgfpathlineto{\pgfqpoint{2.102191in}{1.147892in}}%
\pgfpathlineto{\pgfqpoint{2.106635in}{1.145372in}}%
\pgfpathlineto{\pgfqpoint{2.111079in}{1.142864in}}%
\pgfpathlineto{\pgfqpoint{2.115524in}{1.140368in}}%
\pgfpathlineto{\pgfqpoint{2.119968in}{1.137885in}}%
\pgfpathlineto{\pgfqpoint{2.124412in}{1.135414in}}%
\pgfpathlineto{\pgfqpoint{2.128857in}{1.132955in}}%
\pgfpathlineto{\pgfqpoint{2.133301in}{1.130509in}}%
\pgfpathlineto{\pgfqpoint{2.137745in}{1.128074in}}%
\pgfpathlineto{\pgfqpoint{2.142190in}{1.125651in}}%
\pgfpathlineto{\pgfqpoint{2.146634in}{1.123240in}}%
\pgfpathlineto{\pgfqpoint{2.151078in}{1.120841in}}%
\pgfpathlineto{\pgfqpoint{2.155523in}{1.118453in}}%
\pgfpathlineto{\pgfqpoint{2.159967in}{1.116077in}}%
\pgfpathlineto{\pgfqpoint{2.164411in}{1.113713in}}%
\pgfpathlineto{\pgfqpoint{2.168856in}{1.111359in}}%
\pgfpathlineto{\pgfqpoint{2.173300in}{1.109017in}}%
\pgfpathlineto{\pgfqpoint{2.177745in}{1.106687in}}%
\pgfpathlineto{\pgfqpoint{2.182189in}{1.104367in}}%
\pgfpathlineto{\pgfqpoint{2.186633in}{1.102059in}}%
\pgfpathlineto{\pgfqpoint{2.191078in}{1.099761in}}%
\pgfpathlineto{\pgfqpoint{2.195522in}{1.097475in}}%
\pgfpathlineto{\pgfqpoint{2.199966in}{1.095199in}}%
\pgfpathlineto{\pgfqpoint{2.204411in}{1.092934in}}%
\pgfpathlineto{\pgfqpoint{2.208855in}{1.090680in}}%
\pgfpathlineto{\pgfqpoint{2.213299in}{1.088436in}}%
\pgfpathlineto{\pgfqpoint{2.217744in}{1.086203in}}%
\pgfpathlineto{\pgfqpoint{2.222188in}{1.083980in}}%
\pgfpathlineto{\pgfqpoint{2.226632in}{1.081768in}}%
\pgfpathlineto{\pgfqpoint{2.231077in}{1.079566in}}%
\pgfpathlineto{\pgfqpoint{2.235521in}{1.077374in}}%
\pgfpathlineto{\pgfqpoint{2.239965in}{1.075193in}}%
\pgfpathlineto{\pgfqpoint{2.244410in}{1.073021in}}%
\pgfpathlineto{\pgfqpoint{2.248854in}{1.070860in}}%
\pgfpathlineto{\pgfqpoint{2.253298in}{1.068708in}}%
\pgfpathlineto{\pgfqpoint{2.257743in}{1.066567in}}%
\pgfpathlineto{\pgfqpoint{2.262187in}{1.064435in}}%
\pgfpathlineto{\pgfqpoint{2.266631in}{1.062313in}}%
\pgfpathlineto{\pgfqpoint{2.271076in}{1.060201in}}%
\pgfpathlineto{\pgfqpoint{2.275520in}{1.058098in}}%
\pgfpathlineto{\pgfqpoint{2.279964in}{1.056005in}}%
\pgfpathlineto{\pgfqpoint{2.284409in}{1.053921in}}%
\pgfpathlineto{\pgfqpoint{2.288853in}{1.051847in}}%
\pgfpathlineto{\pgfqpoint{2.293298in}{1.049782in}}%
\pgfpathlineto{\pgfqpoint{2.297742in}{1.047726in}}%
\pgfpathlineto{\pgfqpoint{2.302186in}{1.045680in}}%
\pgfpathlineto{\pgfqpoint{2.306631in}{1.043643in}}%
\pgfpathlineto{\pgfqpoint{2.311075in}{1.041615in}}%
\pgfpathlineto{\pgfqpoint{2.315519in}{1.039596in}}%
\pgfpathlineto{\pgfqpoint{2.319964in}{1.037586in}}%
\pgfpathlineto{\pgfqpoint{2.324408in}{1.035584in}}%
\pgfpathlineto{\pgfqpoint{2.328852in}{1.033592in}}%
\pgfpathlineto{\pgfqpoint{2.333297in}{1.031609in}}%
\pgfpathlineto{\pgfqpoint{2.337741in}{1.029634in}}%
\pgfpathlineto{\pgfqpoint{2.342185in}{1.027668in}}%
\pgfpathlineto{\pgfqpoint{2.346630in}{1.025711in}}%
\pgfpathlineto{\pgfqpoint{2.351074in}{1.023762in}}%
\pgfpathlineto{\pgfqpoint{2.355518in}{1.021822in}}%
\pgfpathlineto{\pgfqpoint{2.359963in}{1.019890in}}%
\pgfpathlineto{\pgfqpoint{2.364407in}{1.017967in}}%
\pgfpathlineto{\pgfqpoint{2.368851in}{1.016052in}}%
\pgfpathlineto{\pgfqpoint{2.373296in}{1.014145in}}%
\pgfpathlineto{\pgfqpoint{2.377740in}{1.012247in}}%
\pgfpathlineto{\pgfqpoint{2.382184in}{1.010357in}}%
\pgfpathlineto{\pgfqpoint{2.386629in}{1.008474in}}%
\pgfpathlineto{\pgfqpoint{2.391073in}{1.006600in}}%
\pgfpathlineto{\pgfqpoint{2.395517in}{1.004735in}}%
\pgfpathlineto{\pgfqpoint{2.399962in}{1.002877in}}%
\pgfpathlineto{\pgfqpoint{2.404406in}{1.001027in}}%
\pgfpathlineto{\pgfqpoint{2.408851in}{0.999184in}}%
\pgfpathlineto{\pgfqpoint{2.413295in}{0.997350in}}%
\pgfpathlineto{\pgfqpoint{2.417739in}{0.995524in}}%
\pgfpathlineto{\pgfqpoint{2.422184in}{0.993705in}}%
\pgfpathlineto{\pgfqpoint{2.426628in}{0.991894in}}%
\pgfpathlineto{\pgfqpoint{2.431072in}{0.990090in}}%
\pgfpathlineto{\pgfqpoint{2.435517in}{0.988294in}}%
\pgfpathlineto{\pgfqpoint{2.439961in}{0.986506in}}%
\pgfpathlineto{\pgfqpoint{2.444405in}{0.984725in}}%
\pgfpathlineto{\pgfqpoint{2.448850in}{0.982952in}}%
\pgfpathlineto{\pgfqpoint{2.453294in}{0.981186in}}%
\pgfpathlineto{\pgfqpoint{2.457738in}{0.979427in}}%
\pgfpathlineto{\pgfqpoint{2.462183in}{0.977676in}}%
\pgfpathlineto{\pgfqpoint{2.466627in}{0.975932in}}%
\pgfpathlineto{\pgfqpoint{2.471071in}{0.974195in}}%
\pgfpathlineto{\pgfqpoint{2.475516in}{0.972466in}}%
\pgfpathlineto{\pgfqpoint{2.479960in}{0.970743in}}%
\pgfpathlineto{\pgfqpoint{2.484404in}{0.969028in}}%
\pgfpathlineto{\pgfqpoint{2.488849in}{0.967319in}}%
\pgfpathlineto{\pgfqpoint{2.493293in}{0.965618in}}%
\pgfpathlineto{\pgfqpoint{2.497737in}{0.963924in}}%
\pgfpathlineto{\pgfqpoint{2.502182in}{0.962236in}}%
\pgfpathlineto{\pgfqpoint{2.506626in}{0.960556in}}%
\pgfpathlineto{\pgfqpoint{2.511071in}{0.958882in}}%
\pgfpathlineto{\pgfqpoint{2.515515in}{0.957215in}}%
\pgfpathlineto{\pgfqpoint{2.519959in}{0.955554in}}%
\pgfpathlineto{\pgfqpoint{2.524404in}{0.953901in}}%
\pgfpathlineto{\pgfqpoint{2.528848in}{0.952254in}}%
\pgfpathlineto{\pgfqpoint{2.533292in}{0.950614in}}%
\pgfpathlineto{\pgfqpoint{2.537737in}{0.948980in}}%
\pgfpathlineto{\pgfqpoint{2.542181in}{0.947353in}}%
\pgfpathlineto{\pgfqpoint{2.546625in}{0.945732in}}%
\pgfpathlineto{\pgfqpoint{2.551070in}{0.944118in}}%
\pgfpathlineto{\pgfqpoint{2.555514in}{0.942510in}}%
\pgfpathlineto{\pgfqpoint{2.559958in}{0.940909in}}%
\pgfpathlineto{\pgfqpoint{2.564403in}{0.939314in}}%
\pgfpathlineto{\pgfqpoint{2.568847in}{0.937725in}}%
\pgfpathlineto{\pgfqpoint{2.573291in}{0.936143in}}%
\pgfpathlineto{\pgfqpoint{2.577736in}{0.934567in}}%
\pgfpathlineto{\pgfqpoint{2.582180in}{0.932997in}}%
\pgfpathlineto{\pgfqpoint{2.586624in}{0.931433in}}%
\pgfpathlineto{\pgfqpoint{2.591069in}{0.929875in}}%
\pgfpathlineto{\pgfqpoint{2.595513in}{0.928324in}}%
\pgfpathlineto{\pgfqpoint{2.599957in}{0.926778in}}%
\pgfpathlineto{\pgfqpoint{2.604402in}{0.925239in}}%
\pgfpathlineto{\pgfqpoint{2.608846in}{0.923705in}}%
\pgfpathlineto{\pgfqpoint{2.613290in}{0.922178in}}%
\pgfpathlineto{\pgfqpoint{2.617735in}{0.920656in}}%
\pgfpathlineto{\pgfqpoint{2.622179in}{0.919141in}}%
\pgfpathlineto{\pgfqpoint{2.626624in}{0.917631in}}%
\pgfpathlineto{\pgfqpoint{2.631068in}{0.916127in}}%
\pgfpathlineto{\pgfqpoint{2.635512in}{0.914628in}}%
\pgfpathlineto{\pgfqpoint{2.639957in}{0.913136in}}%
\pgfpathlineto{\pgfqpoint{2.644401in}{0.911649in}}%
\pgfpathlineto{\pgfqpoint{2.648845in}{0.910168in}}%
\pgfpathlineto{\pgfqpoint{2.653290in}{0.908692in}}%
\pgfpathlineto{\pgfqpoint{2.657734in}{0.907223in}}%
\pgfpathlineto{\pgfqpoint{2.662178in}{0.905758in}}%
\pgfpathlineto{\pgfqpoint{2.666623in}{0.904300in}}%
\pgfpathlineto{\pgfqpoint{2.671067in}{0.902846in}}%
\pgfpathlineto{\pgfqpoint{2.675511in}{0.901399in}}%
\pgfpathlineto{\pgfqpoint{2.679956in}{0.899957in}}%
\pgfpathlineto{\pgfqpoint{2.684400in}{0.898520in}}%
\pgfpathlineto{\pgfqpoint{2.688844in}{0.897088in}}%
\pgfpathlineto{\pgfqpoint{2.693289in}{0.895662in}}%
\pgfpathlineto{\pgfqpoint{2.697733in}{0.894242in}}%
\pgfpathlineto{\pgfqpoint{2.702177in}{0.892826in}}%
\pgfpathlineto{\pgfqpoint{2.706622in}{0.891416in}}%
\pgfpathlineto{\pgfqpoint{2.711066in}{0.890011in}}%
\pgfpathlineto{\pgfqpoint{2.715510in}{0.888612in}}%
\pgfpathlineto{\pgfqpoint{2.719955in}{0.887217in}}%
\pgfpathlineto{\pgfqpoint{2.724399in}{0.885828in}}%
\pgfpathlineto{\pgfqpoint{2.728843in}{0.884444in}}%
\pgfpathlineto{\pgfqpoint{2.733288in}{0.883065in}}%
\pgfpathlineto{\pgfqpoint{2.737732in}{0.881691in}}%
\pgfpathlineto{\pgfqpoint{2.742177in}{0.880322in}}%
\pgfpathlineto{\pgfqpoint{2.746621in}{0.878958in}}%
\pgfpathlineto{\pgfqpoint{2.751065in}{0.877599in}}%
\pgfpathlineto{\pgfqpoint{2.755510in}{0.876245in}}%
\pgfpathlineto{\pgfqpoint{2.759954in}{0.874896in}}%
\pgfpathlineto{\pgfqpoint{2.764398in}{0.873552in}}%
\pgfpathlineto{\pgfqpoint{2.768843in}{0.872213in}}%
\pgfpathlineto{\pgfqpoint{2.773287in}{0.870879in}}%
\pgfpathlineto{\pgfqpoint{2.777731in}{0.869549in}}%
\pgfpathlineto{\pgfqpoint{2.782176in}{0.868225in}}%
\pgfpathlineto{\pgfqpoint{2.786620in}{0.866905in}}%
\pgfpathlineto{\pgfqpoint{2.791064in}{0.865589in}}%
\pgfpathlineto{\pgfqpoint{2.795509in}{0.864279in}}%
\pgfpathlineto{\pgfqpoint{2.799953in}{0.862973in}}%
\pgfpathlineto{\pgfqpoint{2.804397in}{0.861672in}}%
\pgfpathlineto{\pgfqpoint{2.808842in}{0.860376in}}%
\pgfpathlineto{\pgfqpoint{2.813286in}{0.859084in}}%
\pgfpathlineto{\pgfqpoint{2.817730in}{0.857797in}}%
\pgfpathlineto{\pgfqpoint{2.822175in}{0.856515in}}%
\pgfpathlineto{\pgfqpoint{2.826619in}{0.855236in}}%
\pgfpathlineto{\pgfqpoint{2.831063in}{0.853963in}}%
\pgfpathlineto{\pgfqpoint{2.835508in}{0.852694in}}%
\pgfpathlineto{\pgfqpoint{2.839952in}{0.851430in}}%
\pgfpathlineto{\pgfqpoint{2.844396in}{0.850170in}}%
\pgfpathlineto{\pgfqpoint{2.848841in}{0.848914in}}%
\pgfpathlineto{\pgfqpoint{2.853285in}{0.847663in}}%
\pgfpathlineto{\pgfqpoint{2.857730in}{0.846416in}}%
\pgfpathlineto{\pgfqpoint{2.862174in}{0.845174in}}%
\pgfpathlineto{\pgfqpoint{2.866618in}{0.843935in}}%
\pgfpathlineto{\pgfqpoint{2.871063in}{0.842702in}}%
\pgfpathlineto{\pgfqpoint{2.875507in}{0.841472in}}%
\pgfpathlineto{\pgfqpoint{2.879951in}{0.840247in}}%
\pgfpathlineto{\pgfqpoint{2.884396in}{0.839026in}}%
\pgfpathlineto{\pgfqpoint{2.888840in}{0.837809in}}%
\pgfpathlineto{\pgfqpoint{2.893284in}{0.836597in}}%
\pgfpathlineto{\pgfqpoint{2.897729in}{0.835389in}}%
\pgfpathlineto{\pgfqpoint{2.902173in}{0.834184in}}%
\pgfpathlineto{\pgfqpoint{2.906617in}{0.832984in}}%
\pgfpathlineto{\pgfqpoint{2.911062in}{0.831789in}}%
\pgfpathlineto{\pgfqpoint{2.915506in}{0.830597in}}%
\pgfpathlineto{\pgfqpoint{2.919950in}{0.829409in}}%
\pgfpathlineto{\pgfqpoint{2.924395in}{0.828225in}}%
\pgfpathlineto{\pgfqpoint{2.928839in}{0.827046in}}%
\pgfpathlineto{\pgfqpoint{2.933283in}{0.825870in}}%
\pgfpathlineto{\pgfqpoint{2.937728in}{0.824699in}}%
\pgfpathlineto{\pgfqpoint{2.942172in}{0.823531in}}%
\pgfpathlineto{\pgfqpoint{2.946616in}{0.822367in}}%
\pgfpathlineto{\pgfqpoint{2.951061in}{0.821208in}}%
\pgfpathlineto{\pgfqpoint{2.955505in}{0.820052in}}%
\pgfpathlineto{\pgfqpoint{2.959949in}{0.818900in}}%
\pgfpathlineto{\pgfqpoint{2.964394in}{0.817752in}}%
\pgfpathlineto{\pgfqpoint{2.968838in}{0.816608in}}%
\pgfpathlineto{\pgfqpoint{2.973283in}{0.815468in}}%
\pgfpathlineto{\pgfqpoint{2.977727in}{0.814331in}}%
\pgfpathlineto{\pgfqpoint{2.982171in}{0.813198in}}%
\pgfpathlineto{\pgfqpoint{2.986616in}{0.812069in}}%
\pgfpathlineto{\pgfqpoint{2.991060in}{0.810944in}}%
\pgfpathlineto{\pgfqpoint{2.995504in}{0.809823in}}%
\pgfpathlineto{\pgfqpoint{2.999949in}{0.808705in}}%
\pgfpathlineto{\pgfqpoint{3.004393in}{0.807591in}}%
\pgfpathlineto{\pgfqpoint{3.008837in}{0.806481in}}%
\pgfpathlineto{\pgfqpoint{3.013282in}{0.805374in}}%
\pgfpathlineto{\pgfqpoint{3.017726in}{0.804271in}}%
\pgfpathlineto{\pgfqpoint{3.022170in}{0.803172in}}%
\pgfpathlineto{\pgfqpoint{3.026615in}{0.802076in}}%
\pgfpathlineto{\pgfqpoint{3.031059in}{0.800984in}}%
\pgfpathlineto{\pgfqpoint{3.035503in}{0.799896in}}%
\pgfpathlineto{\pgfqpoint{3.039948in}{0.798811in}}%
\pgfpathlineto{\pgfqpoint{3.044392in}{0.797729in}}%
\pgfpathlineto{\pgfqpoint{3.048836in}{0.796651in}}%
\pgfpathlineto{\pgfqpoint{3.053281in}{0.795577in}}%
\pgfpathlineto{\pgfqpoint{3.057725in}{0.794506in}}%
\pgfpathlineto{\pgfqpoint{3.062169in}{0.793439in}}%
\pgfpathlineto{\pgfqpoint{3.066614in}{0.792375in}}%
\pgfpathlineto{\pgfqpoint{3.071058in}{0.791314in}}%
\pgfpathlineto{\pgfqpoint{3.075502in}{0.790257in}}%
\pgfpathlineto{\pgfqpoint{3.079947in}{0.789203in}}%
\pgfpathlineto{\pgfqpoint{3.084391in}{0.788153in}}%
\pgfpathlineto{\pgfqpoint{3.088836in}{0.787106in}}%
\pgfpathlineto{\pgfqpoint{3.093280in}{0.786062in}}%
\pgfpathlineto{\pgfqpoint{3.097724in}{0.785022in}}%
\pgfpathlineto{\pgfqpoint{3.102169in}{0.783985in}}%
\pgfpathlineto{\pgfqpoint{3.106613in}{0.782952in}}%
\pgfpathlineto{\pgfqpoint{3.111057in}{0.781921in}}%
\pgfpathlineto{\pgfqpoint{3.115502in}{0.780894in}}%
\pgfpathlineto{\pgfqpoint{3.119946in}{0.779871in}}%
\pgfpathlineto{\pgfqpoint{3.124390in}{0.778850in}}%
\pgfpathlineto{\pgfqpoint{3.128835in}{0.777833in}}%
\pgfpathlineto{\pgfqpoint{3.133279in}{0.776819in}}%
\pgfpathlineto{\pgfqpoint{3.137723in}{0.775808in}}%
\pgfpathlineto{\pgfqpoint{3.142168in}{0.774800in}}%
\pgfpathlineto{\pgfqpoint{3.146612in}{0.773796in}}%
\pgfpathlineto{\pgfqpoint{3.151056in}{0.772795in}}%
\pgfpathlineto{\pgfqpoint{3.155501in}{0.771796in}}%
\pgfpathlineto{\pgfqpoint{3.159945in}{0.770801in}}%
\pgfpathlineto{\pgfqpoint{3.164389in}{0.769809in}}%
\pgfpathlineto{\pgfqpoint{3.168834in}{0.768821in}}%
\pgfpathlineto{\pgfqpoint{3.173278in}{0.767835in}}%
\pgfpathlineto{\pgfqpoint{3.177722in}{0.766852in}}%
\pgfpathlineto{\pgfqpoint{3.182167in}{0.765873in}}%
\pgfpathlineto{\pgfqpoint{3.186611in}{0.764896in}}%
\pgfpathlineto{\pgfqpoint{3.191056in}{0.763922in}}%
\pgfpathlineto{\pgfqpoint{3.195500in}{0.762952in}}%
\pgfpathlineto{\pgfqpoint{3.199944in}{0.761984in}}%
\pgfpathlineto{\pgfqpoint{3.204389in}{0.761020in}}%
\pgfpathlineto{\pgfqpoint{3.208833in}{0.760058in}}%
\pgfpathlineto{\pgfqpoint{3.213277in}{0.759100in}}%
\pgfpathlineto{\pgfqpoint{3.217722in}{0.758144in}}%
\pgfpathlineto{\pgfqpoint{3.222166in}{0.757191in}}%
\pgfpathlineto{\pgfqpoint{3.226610in}{0.756241in}}%
\pgfpathlineto{\pgfqpoint{3.231055in}{0.755294in}}%
\pgfpathlineto{\pgfqpoint{3.235499in}{0.754350in}}%
\pgfpathlineto{\pgfqpoint{3.239943in}{0.753409in}}%
\pgfpathlineto{\pgfqpoint{3.244388in}{0.752471in}}%
\pgfpathlineto{\pgfqpoint{3.248832in}{0.751536in}}%
\pgfpathlineto{\pgfqpoint{3.253276in}{0.750603in}}%
\pgfpathlineto{\pgfqpoint{3.257721in}{0.749673in}}%
\pgfpathlineto{\pgfqpoint{3.262165in}{0.748746in}}%
\pgfpathlineto{\pgfqpoint{3.266609in}{0.747822in}}%
\pgfpathlineto{\pgfqpoint{3.271054in}{0.746901in}}%
\pgfpathlineto{\pgfqpoint{3.275498in}{0.745982in}}%
\pgfpathlineto{\pgfqpoint{3.279942in}{0.745066in}}%
\pgfpathlineto{\pgfqpoint{3.284387in}{0.744153in}}%
\pgfpathlineto{\pgfqpoint{3.288831in}{0.743243in}}%
\pgfpathlineto{\pgfqpoint{3.293275in}{0.742335in}}%
\pgfpathlineto{\pgfqpoint{3.297720in}{0.741430in}}%
\pgfpathlineto{\pgfqpoint{3.302164in}{0.740528in}}%
\pgfpathlineto{\pgfqpoint{3.306609in}{0.739628in}}%
\pgfpathlineto{\pgfqpoint{3.311053in}{0.738732in}}%
\pgfpathlineto{\pgfqpoint{3.315497in}{0.737837in}}%
\pgfpathlineto{\pgfqpoint{3.319942in}{0.736946in}}%
\pgfpathlineto{\pgfqpoint{3.324386in}{0.736057in}}%
\pgfpathlineto{\pgfqpoint{3.328830in}{0.735171in}}%
\pgfpathlineto{\pgfqpoint{3.333275in}{0.734287in}}%
\pgfpathlineto{\pgfqpoint{3.337719in}{0.733406in}}%
\pgfpathlineto{\pgfqpoint{3.342163in}{0.732528in}}%
\pgfpathlineto{\pgfqpoint{3.346608in}{0.731652in}}%
\pgfpathlineto{\pgfqpoint{3.351052in}{0.730778in}}%
\pgfpathlineto{\pgfqpoint{3.355496in}{0.729908in}}%
\pgfpathlineto{\pgfqpoint{3.359941in}{0.729040in}}%
\pgfpathlineto{\pgfqpoint{3.364385in}{0.728174in}}%
\pgfpathlineto{\pgfqpoint{3.368829in}{0.727311in}}%
\pgfpathlineto{\pgfqpoint{3.373274in}{0.726450in}}%
\pgfpathlineto{\pgfqpoint{3.377718in}{0.725592in}}%
\pgfpathlineto{\pgfqpoint{3.382162in}{0.724737in}}%
\pgfpathlineto{\pgfqpoint{3.386607in}{0.723883in}}%
\pgfpathlineto{\pgfqpoint{3.391051in}{0.723033in}}%
\pgfpathlineto{\pgfqpoint{3.395495in}{0.722185in}}%
\pgfpathlineto{\pgfqpoint{3.399940in}{0.721339in}}%
\pgfpathlineto{\pgfqpoint{3.404384in}{0.720496in}}%
\pgfpathlineto{\pgfqpoint{3.408828in}{0.719655in}}%
\pgfpathlineto{\pgfqpoint{3.413273in}{0.718816in}}%
\pgfpathlineto{\pgfqpoint{3.417717in}{0.717980in}}%
\pgfpathlineto{\pgfqpoint{3.422162in}{0.717147in}}%
\pgfpathlineto{\pgfqpoint{3.426606in}{0.716315in}}%
\pgfpathlineto{\pgfqpoint{3.431050in}{0.715486in}}%
\pgfpathlineto{\pgfqpoint{3.435495in}{0.714660in}}%
\pgfpathlineto{\pgfqpoint{3.439939in}{0.713836in}}%
\pgfpathlineto{\pgfqpoint{3.444383in}{0.713014in}}%
\pgfpathlineto{\pgfqpoint{3.448828in}{0.712195in}}%
\pgfpathlineto{\pgfqpoint{3.453272in}{0.711377in}}%
\pgfpathlineto{\pgfqpoint{3.457716in}{0.710563in}}%
\pgfpathlineto{\pgfqpoint{3.462161in}{0.709750in}}%
\pgfpathlineto{\pgfqpoint{3.466605in}{0.708940in}}%
\pgfpathlineto{\pgfqpoint{3.471049in}{0.708132in}}%
\pgfpathlineto{\pgfqpoint{3.475494in}{0.707326in}}%
\pgfpathlineto{\pgfqpoint{3.479938in}{0.706523in}}%
\pgfpathlineto{\pgfqpoint{3.484382in}{0.705722in}}%
\pgfpathlineto{\pgfqpoint{3.488827in}{0.704923in}}%
\pgfpathlineto{\pgfqpoint{3.493271in}{0.704126in}}%
\pgfpathlineto{\pgfqpoint{3.497715in}{0.703332in}}%
\pgfpathlineto{\pgfqpoint{3.502160in}{0.702540in}}%
\pgfpathlineto{\pgfqpoint{3.506604in}{0.701750in}}%
\pgfpathlineto{\pgfqpoint{3.511048in}{0.700962in}}%
\pgfpathlineto{\pgfqpoint{3.515493in}{0.700177in}}%
\pgfpathlineto{\pgfqpoint{3.519937in}{0.699393in}}%
\pgfpathlineto{\pgfqpoint{3.524381in}{0.698612in}}%
\pgfpathlineto{\pgfqpoint{3.528826in}{0.697833in}}%
\pgfpathlineto{\pgfqpoint{3.533270in}{0.697056in}}%
\pgfpathlineto{\pgfqpoint{3.537715in}{0.696281in}}%
\pgfpathlineto{\pgfqpoint{3.542159in}{0.695509in}}%
\pgfpathlineto{\pgfqpoint{3.546603in}{0.694738in}}%
\pgfpathlineto{\pgfqpoint{3.551048in}{0.693970in}}%
\pgfpathlineto{\pgfqpoint{3.555492in}{0.693204in}}%
\pgfpathlineto{\pgfqpoint{3.559936in}{0.692440in}}%
\pgfpathlineto{\pgfqpoint{3.564381in}{0.691678in}}%
\pgfpathlineto{\pgfqpoint{3.568825in}{0.690918in}}%
\pgfpathlineto{\pgfqpoint{3.573269in}{0.690160in}}%
\pgfpathlineto{\pgfqpoint{3.577714in}{0.689404in}}%
\pgfpathlineto{\pgfqpoint{3.582158in}{0.688650in}}%
\pgfpathlineto{\pgfqpoint{3.586602in}{0.687899in}}%
\pgfpathlineto{\pgfqpoint{3.591047in}{0.687149in}}%
\pgfpathlineto{\pgfqpoint{3.595491in}{0.686402in}}%
\pgfpathlineto{\pgfqpoint{3.599935in}{0.685656in}}%
\pgfpathlineto{\pgfqpoint{3.604380in}{0.684912in}}%
\pgfpathlineto{\pgfqpoint{3.608824in}{0.684171in}}%
\pgfpathlineto{\pgfqpoint{3.613268in}{0.683431in}}%
\pgfpathlineto{\pgfqpoint{3.617713in}{0.682694in}}%
\pgfpathlineto{\pgfqpoint{3.622157in}{0.681958in}}%
\pgfpathlineto{\pgfqpoint{3.626601in}{0.681225in}}%
\pgfpathlineto{\pgfqpoint{3.631046in}{0.680493in}}%
\pgfpathlineto{\pgfqpoint{3.635490in}{0.679764in}}%
\pgfpathlineto{\pgfqpoint{3.639934in}{0.679036in}}%
\pgfpathlineto{\pgfqpoint{3.644379in}{0.678310in}}%
\pgfpathlineto{\pgfqpoint{3.648823in}{0.677586in}}%
\pgfpathlineto{\pgfqpoint{3.653268in}{0.676865in}}%
\pgfpathlineto{\pgfqpoint{3.657712in}{0.676145in}}%
\pgfpathlineto{\pgfqpoint{3.662156in}{0.675427in}}%
\pgfpathlineto{\pgfqpoint{3.666601in}{0.674711in}}%
\pgfpathlineto{\pgfqpoint{3.671045in}{0.673996in}}%
\pgfpathlineto{\pgfqpoint{3.675489in}{0.673284in}}%
\pgfpathlineto{\pgfqpoint{3.679934in}{0.672574in}}%
\pgfpathlineto{\pgfqpoint{3.684378in}{0.671865in}}%
\pgfpathlineto{\pgfqpoint{3.688822in}{0.671158in}}%
\pgfpathlineto{\pgfqpoint{3.693267in}{0.670454in}}%
\pgfpathlineto{\pgfqpoint{3.697711in}{0.669751in}}%
\pgfpathlineto{\pgfqpoint{3.702155in}{0.669050in}}%
\pgfpathlineto{\pgfqpoint{3.706600in}{0.668350in}}%
\pgfpathlineto{\pgfqpoint{3.711044in}{0.667653in}}%
\pgfpathlineto{\pgfqpoint{3.715488in}{0.666957in}}%
\pgfpathlineto{\pgfqpoint{3.719933in}{0.666264in}}%
\pgfpathlineto{\pgfqpoint{3.724377in}{0.665572in}}%
\pgfpathlineto{\pgfqpoint{3.728821in}{0.664881in}}%
\pgfpathlineto{\pgfqpoint{3.733266in}{0.664193in}}%
\pgfpathlineto{\pgfqpoint{3.737710in}{0.663507in}}%
\pgfpathlineto{\pgfqpoint{3.742154in}{0.662822in}}%
\pgfpathlineto{\pgfqpoint{3.746599in}{0.662139in}}%
\pgfpathlineto{\pgfqpoint{3.751043in}{0.661458in}}%
\pgfpathlineto{\pgfqpoint{3.755487in}{0.660778in}}%
\pgfpathlineto{\pgfqpoint{3.759932in}{0.660100in}}%
\pgfpathlineto{\pgfqpoint{3.764376in}{0.659424in}}%
\pgfpathlineto{\pgfqpoint{3.768821in}{0.658750in}}%
\pgfpathlineto{\pgfqpoint{3.773265in}{0.658078in}}%
\pgfpathlineto{\pgfqpoint{3.777709in}{0.657407in}}%
\pgfpathlineto{\pgfqpoint{3.782154in}{0.656738in}}%
\pgfpathlineto{\pgfqpoint{3.786598in}{0.656071in}}%
\pgfpathlineto{\pgfqpoint{3.791042in}{0.655405in}}%
\pgfpathlineto{\pgfqpoint{3.795487in}{0.654741in}}%
\pgfpathlineto{\pgfqpoint{3.799931in}{0.654079in}}%
\pgfpathlineto{\pgfqpoint{3.804375in}{0.653419in}}%
\pgfpathlineto{\pgfqpoint{3.808820in}{0.652760in}}%
\pgfpathlineto{\pgfqpoint{3.813264in}{0.652103in}}%
\pgfpathlineto{\pgfqpoint{3.817708in}{0.651447in}}%
\pgfpathlineto{\pgfqpoint{3.822153in}{0.650793in}}%
\pgfpathlineto{\pgfqpoint{3.826597in}{0.650141in}}%
\pgfpathlineto{\pgfqpoint{3.831041in}{0.649491in}}%
\pgfpathlineto{\pgfqpoint{3.835486in}{0.648842in}}%
\pgfpathlineto{\pgfqpoint{3.839930in}{0.648195in}}%
\pgfpathlineto{\pgfqpoint{3.844374in}{0.647549in}}%
\pgfpathlineto{\pgfqpoint{3.848819in}{0.646905in}}%
\pgfpathlineto{\pgfqpoint{3.853263in}{0.646263in}}%
\pgfpathlineto{\pgfqpoint{3.857707in}{0.645622in}}%
\pgfpathlineto{\pgfqpoint{3.862152in}{0.644983in}}%
\pgfpathlineto{\pgfqpoint{3.866596in}{0.644346in}}%
\pgfpathlineto{\pgfqpoint{3.871040in}{0.643710in}}%
\pgfpathlineto{\pgfqpoint{3.875485in}{0.643076in}}%
\pgfpathlineto{\pgfqpoint{3.879929in}{0.642443in}}%
\pgfpathlineto{\pgfqpoint{3.884374in}{0.641812in}}%
\pgfpathlineto{\pgfqpoint{3.888818in}{0.641182in}}%
\pgfpathlineto{\pgfqpoint{3.893262in}{0.640554in}}%
\pgfpathlineto{\pgfqpoint{3.897707in}{0.639928in}}%
\pgfpathlineto{\pgfqpoint{3.902151in}{0.639303in}}%
\pgfpathlineto{\pgfqpoint{3.906595in}{0.638680in}}%
\pgfpathlineto{\pgfqpoint{3.911040in}{0.638058in}}%
\pgfpathlineto{\pgfqpoint{3.915484in}{0.637438in}}%
\pgfpathlineto{\pgfqpoint{3.919928in}{0.636819in}}%
\pgfpathlineto{\pgfqpoint{3.924373in}{0.636202in}}%
\pgfpathlineto{\pgfqpoint{3.928817in}{0.635586in}}%
\pgfpathlineto{\pgfqpoint{3.933261in}{0.634972in}}%
\pgfpathlineto{\pgfqpoint{3.937706in}{0.634359in}}%
\pgfpathlineto{\pgfqpoint{3.942150in}{0.633748in}}%
\pgfpathlineto{\pgfqpoint{3.946594in}{0.633139in}}%
\pgfpathlineto{\pgfqpoint{3.951039in}{0.632530in}}%
\pgfpathlineto{\pgfqpoint{3.955483in}{0.631924in}}%
\pgfpathlineto{\pgfqpoint{3.959927in}{0.631319in}}%
\pgfpathlineto{\pgfqpoint{3.964372in}{0.630715in}}%
\pgfpathlineto{\pgfqpoint{3.968816in}{0.630113in}}%
\pgfpathlineto{\pgfqpoint{3.973260in}{0.629512in}}%
\pgfpathlineto{\pgfqpoint{3.977705in}{0.628913in}}%
\pgfpathlineto{\pgfqpoint{3.982149in}{0.628315in}}%
\pgfpathlineto{\pgfqpoint{3.986594in}{0.627719in}}%
\pgfpathlineto{\pgfqpoint{3.991038in}{0.627124in}}%
\pgfpathlineto{\pgfqpoint{3.995482in}{0.626530in}}%
\pgfpathlineto{\pgfqpoint{3.999927in}{0.625938in}}%
\pgfpathlineto{\pgfqpoint{4.004371in}{0.625347in}}%
\pgfpathlineto{\pgfqpoint{4.008815in}{0.624758in}}%
\pgfpathlineto{\pgfqpoint{4.013260in}{0.624171in}}%
\pgfpathlineto{\pgfqpoint{4.017704in}{0.623584in}}%
\pgfpathlineto{\pgfqpoint{4.022148in}{0.622999in}}%
\pgfpathlineto{\pgfqpoint{4.026593in}{0.622416in}}%
\pgfpathlineto{\pgfqpoint{4.031037in}{0.621834in}}%
\pgfpathlineto{\pgfqpoint{4.035481in}{0.621253in}}%
\pgfpathlineto{\pgfqpoint{4.039926in}{0.620673in}}%
\pgfpathlineto{\pgfqpoint{4.044370in}{0.620095in}}%
\pgfpathlineto{\pgfqpoint{4.048814in}{0.619519in}}%
\pgfpathlineto{\pgfqpoint{4.053259in}{0.618944in}}%
\pgfpathlineto{\pgfqpoint{4.057703in}{0.618370in}}%
\pgfpathlineto{\pgfqpoint{4.062147in}{0.617797in}}%
\pgfpathlineto{\pgfqpoint{4.066592in}{0.617226in}}%
\pgfpathlineto{\pgfqpoint{4.071036in}{0.616656in}}%
\pgfpathlineto{\pgfqpoint{4.075480in}{0.616088in}}%
\pgfpathlineto{\pgfqpoint{4.079925in}{0.615521in}}%
\pgfpathlineto{\pgfqpoint{4.084369in}{0.614955in}}%
\pgfpathlineto{\pgfqpoint{4.088813in}{0.614391in}}%
\pgfpathlineto{\pgfqpoint{4.093258in}{0.613828in}}%
\pgfpathlineto{\pgfqpoint{4.097702in}{0.613266in}}%
\pgfpathlineto{\pgfqpoint{4.102147in}{0.612705in}}%
\pgfpathlineto{\pgfqpoint{4.106591in}{0.612146in}}%
\pgfpathlineto{\pgfqpoint{4.111035in}{0.611588in}}%
\pgfpathlineto{\pgfqpoint{4.115480in}{0.611032in}}%
\pgfpathlineto{\pgfqpoint{4.119924in}{0.610477in}}%
\pgfpathlineto{\pgfqpoint{4.124368in}{0.609923in}}%
\pgfpathlineto{\pgfqpoint{4.128813in}{0.609370in}}%
\pgfpathlineto{\pgfqpoint{4.133257in}{0.608819in}}%
\pgfpathlineto{\pgfqpoint{4.137701in}{0.608269in}}%
\pgfpathlineto{\pgfqpoint{4.142146in}{0.607720in}}%
\pgfpathlineto{\pgfqpoint{4.146590in}{0.607173in}}%
\pgfpathlineto{\pgfqpoint{4.151034in}{0.606627in}}%
\pgfpathlineto{\pgfqpoint{4.155479in}{0.606082in}}%
\pgfpathlineto{\pgfqpoint{4.159923in}{0.605538in}}%
\pgfpathlineto{\pgfqpoint{4.164367in}{0.604996in}}%
\pgfpathlineto{\pgfqpoint{4.168812in}{0.604454in}}%
\pgfpathlineto{\pgfqpoint{4.173256in}{0.603915in}}%
\pgfpathlineto{\pgfqpoint{4.177700in}{0.603376in}}%
\pgfpathlineto{\pgfqpoint{4.182145in}{0.602838in}}%
\pgfpathlineto{\pgfqpoint{4.186589in}{0.602302in}}%
\pgfpathlineto{\pgfqpoint{4.191033in}{0.601767in}}%
\pgfpathlineto{\pgfqpoint{4.195478in}{0.601234in}}%
\pgfpathlineto{\pgfqpoint{4.199922in}{0.600701in}}%
\pgfpathlineto{\pgfqpoint{4.204366in}{0.600170in}}%
\pgfpathlineto{\pgfqpoint{4.208811in}{0.599640in}}%
\pgfpathlineto{\pgfqpoint{4.213255in}{0.599111in}}%
\pgfpathlineto{\pgfqpoint{4.217700in}{0.598583in}}%
\pgfpathlineto{\pgfqpoint{4.222144in}{0.598057in}}%
\pgfpathlineto{\pgfqpoint{4.226588in}{0.597532in}}%
\pgfpathlineto{\pgfqpoint{4.231033in}{0.597008in}}%
\pgfpathlineto{\pgfqpoint{4.235477in}{0.596485in}}%
\pgfpathlineto{\pgfqpoint{4.239921in}{0.595963in}}%
\pgfpathlineto{\pgfqpoint{4.244366in}{0.595443in}}%
\pgfpathlineto{\pgfqpoint{4.248810in}{0.594923in}}%
\pgfpathlineto{\pgfqpoint{4.253254in}{0.594405in}}%
\pgfpathlineto{\pgfqpoint{4.257699in}{0.593888in}}%
\pgfpathlineto{\pgfqpoint{4.262143in}{0.593372in}}%
\pgfpathlineto{\pgfqpoint{4.266587in}{0.592858in}}%
\pgfpathlineto{\pgfqpoint{4.271032in}{0.592344in}}%
\pgfpathlineto{\pgfqpoint{4.275476in}{0.591832in}}%
\pgfpathlineto{\pgfqpoint{4.279920in}{0.591321in}}%
\pgfpathlineto{\pgfqpoint{4.284365in}{0.590811in}}%
\pgfpathlineto{\pgfqpoint{4.288809in}{0.590302in}}%
\pgfpathlineto{\pgfqpoint{4.293253in}{0.589794in}}%
\pgfpathlineto{\pgfqpoint{4.297698in}{0.589288in}}%
\pgfpathlineto{\pgfqpoint{4.302142in}{0.588782in}}%
\pgfpathlineto{\pgfqpoint{4.306586in}{0.588278in}}%
\pgfpathlineto{\pgfqpoint{4.311031in}{0.587775in}}%
\pgfpathlineto{\pgfqpoint{4.315475in}{0.587272in}}%
\pgfpathlineto{\pgfqpoint{4.319919in}{0.586771in}}%
\pgfpathlineto{\pgfqpoint{4.324364in}{0.586272in}}%
\pgfpathlineto{\pgfqpoint{4.328808in}{0.585773in}}%
\pgfpathlineto{\pgfqpoint{4.333253in}{0.585275in}}%
\pgfpathlineto{\pgfqpoint{4.337697in}{0.584779in}}%
\pgfpathlineto{\pgfqpoint{4.342141in}{0.584283in}}%
\pgfpathlineto{\pgfqpoint{4.346586in}{0.583789in}}%
\pgfpathlineto{\pgfqpoint{4.351030in}{0.583296in}}%
\pgfpathlineto{\pgfqpoint{4.355474in}{0.582803in}}%
\pgfpathlineto{\pgfqpoint{4.359919in}{0.582312in}}%
\pgfpathlineto{\pgfqpoint{4.364363in}{0.581822in}}%
\pgfpathlineto{\pgfqpoint{4.368807in}{0.581333in}}%
\pgfpathlineto{\pgfqpoint{4.373252in}{0.580846in}}%
\pgfpathlineto{\pgfqpoint{4.377696in}{0.580359in}}%
\pgfpathlineto{\pgfqpoint{4.382140in}{0.579873in}}%
\pgfpathlineto{\pgfqpoint{4.386585in}{0.579388in}}%
\pgfpathlineto{\pgfqpoint{4.391029in}{0.578905in}}%
\pgfpathlineto{\pgfqpoint{4.395473in}{0.578422in}}%
\pgfpathlineto{\pgfqpoint{4.399918in}{0.577941in}}%
\pgfpathlineto{\pgfqpoint{4.404362in}{0.577460in}}%
\pgfpathlineto{\pgfqpoint{4.408806in}{0.576981in}}%
\pgfpathlineto{\pgfqpoint{4.413251in}{0.576503in}}%
\pgfpathlineto{\pgfqpoint{4.417695in}{0.576025in}}%
\pgfpathlineto{\pgfqpoint{4.422139in}{0.575549in}}%
\pgfpathlineto{\pgfqpoint{4.426584in}{0.575074in}}%
\pgfpathlineto{\pgfqpoint{4.431028in}{0.574599in}}%
\pgfpathlineto{\pgfqpoint{4.435472in}{0.574126in}}%
\pgfpathlineto{\pgfqpoint{4.439917in}{0.573654in}}%
\pgfpathlineto{\pgfqpoint{4.444361in}{0.573183in}}%
\pgfpathlineto{\pgfqpoint{4.448806in}{0.572713in}}%
\pgfpathlineto{\pgfqpoint{4.453250in}{0.572244in}}%
\pgfpathlineto{\pgfqpoint{4.457694in}{0.571776in}}%
\pgfpathlineto{\pgfqpoint{4.462139in}{0.571308in}}%
\pgfpathlineto{\pgfqpoint{4.466583in}{0.570842in}}%
\pgfpathlineto{\pgfqpoint{4.471027in}{0.570377in}}%
\pgfpathlineto{\pgfqpoint{4.475472in}{0.569913in}}%
\pgfpathlineto{\pgfqpoint{4.479916in}{0.569450in}}%
\pgfpathlineto{\pgfqpoint{4.484360in}{0.568988in}}%
\pgfpathlineto{\pgfqpoint{4.488805in}{0.568527in}}%
\pgfpathlineto{\pgfqpoint{4.493249in}{0.568066in}}%
\pgfpathlineto{\pgfqpoint{4.497693in}{0.567607in}}%
\pgfpathlineto{\pgfqpoint{4.502138in}{0.567149in}}%
\pgfpathlineto{\pgfqpoint{4.506582in}{0.566692in}}%
\pgfpathlineto{\pgfqpoint{4.511026in}{0.566236in}}%
\pgfpathlineto{\pgfqpoint{4.515471in}{0.565780in}}%
\pgfpathlineto{\pgfqpoint{4.519915in}{0.565326in}}%
\pgfpathlineto{\pgfqpoint{4.524359in}{0.564873in}}%
\pgfpathlineto{\pgfqpoint{4.528804in}{0.564420in}}%
\pgfpathlineto{\pgfqpoint{4.533248in}{0.563969in}}%
\pgfpathlineto{\pgfqpoint{4.537692in}{0.563518in}}%
\pgfpathlineto{\pgfqpoint{4.542137in}{0.563069in}}%
\pgfpathlineto{\pgfqpoint{4.546581in}{0.562620in}}%
\pgfpathlineto{\pgfqpoint{4.551025in}{0.562172in}}%
\pgfpathlineto{\pgfqpoint{4.555470in}{0.561726in}}%
\pgfpathlineto{\pgfqpoint{4.559914in}{0.561280in}}%
\pgfpathlineto{\pgfqpoint{4.564359in}{0.560835in}}%
\pgfpathlineto{\pgfqpoint{4.568803in}{0.560391in}}%
\pgfpathlineto{\pgfqpoint{4.573247in}{0.559948in}}%
\pgfpathlineto{\pgfqpoint{4.577692in}{0.559506in}}%
\pgfpathlineto{\pgfqpoint{4.582136in}{0.559065in}}%
\pgfpathlineto{\pgfqpoint{4.586580in}{0.558625in}}%
\pgfpathlineto{\pgfqpoint{4.591025in}{0.558185in}}%
\pgfpathlineto{\pgfqpoint{4.595469in}{0.557747in}}%
\pgfpathlineto{\pgfqpoint{4.599913in}{0.557309in}}%
\pgfpathlineto{\pgfqpoint{4.604358in}{0.556873in}}%
\pgfpathlineto{\pgfqpoint{4.608802in}{0.556437in}}%
\pgfpathlineto{\pgfqpoint{4.613246in}{0.556002in}}%
\pgfpathlineto{\pgfqpoint{4.617691in}{0.555568in}}%
\pgfpathlineto{\pgfqpoint{4.622135in}{0.555136in}}%
\pgfpathlineto{\pgfqpoint{4.626579in}{0.554703in}}%
\pgfpathlineto{\pgfqpoint{4.631024in}{0.554272in}}%
\pgfpathlineto{\pgfqpoint{4.635468in}{0.553842in}}%
\pgfpathlineto{\pgfqpoint{4.639912in}{0.553413in}}%
\pgfpathlineto{\pgfqpoint{4.644357in}{0.552984in}}%
\pgfpathlineto{\pgfqpoint{4.648801in}{0.552556in}}%
\pgfpathlineto{\pgfqpoint{4.653245in}{0.552130in}}%
\pgfpathlineto{\pgfqpoint{4.657690in}{0.551704in}}%
\pgfpathlineto{\pgfqpoint{4.662134in}{0.551279in}}%
\pgfpathlineto{\pgfqpoint{4.666579in}{0.550855in}}%
\pgfpathlineto{\pgfqpoint{4.671023in}{0.550431in}}%
\pgfpathlineto{\pgfqpoint{4.675467in}{0.550009in}}%
\pgfpathlineto{\pgfqpoint{4.679912in}{0.549587in}}%
\pgfpathlineto{\pgfqpoint{4.684356in}{0.549167in}}%
\pgfpathlineto{\pgfqpoint{4.688800in}{0.548747in}}%
\pgfpathlineto{\pgfqpoint{4.693245in}{0.548328in}}%
\pgfpathlineto{\pgfqpoint{4.697689in}{0.547910in}}%
\pgfpathlineto{\pgfqpoint{4.702133in}{0.547492in}}%
\pgfpathlineto{\pgfqpoint{4.706578in}{0.547076in}}%
\pgfpathlineto{\pgfqpoint{4.711022in}{0.546660in}}%
\pgfpathlineto{\pgfqpoint{4.715466in}{0.546246in}}%
\pgfpathlineto{\pgfqpoint{4.719911in}{0.545832in}}%
\pgfpathlineto{\pgfqpoint{4.724355in}{0.545419in}}%
\pgfpathlineto{\pgfqpoint{4.728799in}{0.545006in}}%
\pgfpathlineto{\pgfqpoint{4.733244in}{0.544595in}}%
\pgfpathlineto{\pgfqpoint{4.737688in}{0.544184in}}%
\pgfpathlineto{\pgfqpoint{4.742132in}{0.543775in}}%
\pgfpathlineto{\pgfqpoint{4.746577in}{0.543366in}}%
\pgfpathlineto{\pgfqpoint{4.751021in}{0.542957in}}%
\pgfpathlineto{\pgfqpoint{4.755465in}{0.542550in}}%
\pgfpathlineto{\pgfqpoint{4.759910in}{0.542144in}}%
\pgfpathlineto{\pgfqpoint{4.764354in}{0.541738in}}%
\pgfpathlineto{\pgfqpoint{4.768798in}{0.541333in}}%
\pgfpathlineto{\pgfqpoint{4.773243in}{0.540929in}}%
\pgfpathlineto{\pgfqpoint{4.777687in}{0.540526in}}%
\pgfpathlineto{\pgfqpoint{4.782132in}{0.540123in}}%
\pgfpathlineto{\pgfqpoint{4.786576in}{0.539722in}}%
\pgfpathlineto{\pgfqpoint{4.791020in}{0.539321in}}%
\pgfpathlineto{\pgfqpoint{4.795465in}{0.538921in}}%
\pgfpathlineto{\pgfqpoint{4.799909in}{0.538522in}}%
\pgfpathlineto{\pgfqpoint{4.804353in}{0.538123in}}%
\pgfpathlineto{\pgfqpoint{4.808798in}{0.537726in}}%
\pgfpathlineto{\pgfqpoint{4.813242in}{0.537329in}}%
\pgfpathlineto{\pgfqpoint{4.817686in}{0.536933in}}%
\pgfpathlineto{\pgfqpoint{4.822131in}{0.536537in}}%
\pgfpathlineto{\pgfqpoint{4.826575in}{0.536143in}}%
\pgfpathlineto{\pgfqpoint{4.831019in}{0.535749in}}%
\pgfpathlineto{\pgfqpoint{4.835464in}{0.535356in}}%
\pgfpathlineto{\pgfqpoint{4.839908in}{0.534964in}}%
\pgfpathlineto{\pgfqpoint{4.844352in}{0.534572in}}%
\pgfpathlineto{\pgfqpoint{4.848797in}{0.534182in}}%
\pgfpathlineto{\pgfqpoint{4.853241in}{0.533792in}}%
\pgfpathlineto{\pgfqpoint{4.857685in}{0.533403in}}%
\pgfpathlineto{\pgfqpoint{4.862130in}{0.533014in}}%
\pgfpathlineto{\pgfqpoint{4.866574in}{0.532627in}}%
\pgfpathlineto{\pgfqpoint{4.871018in}{0.532240in}}%
\pgfpathlineto{\pgfqpoint{4.875463in}{0.531854in}}%
\pgfpathlineto{\pgfqpoint{4.879907in}{0.531468in}}%
\pgfpathlineto{\pgfqpoint{4.884351in}{0.531084in}}%
\pgfpathlineto{\pgfqpoint{4.888796in}{0.530700in}}%
\pgfpathlineto{\pgfqpoint{4.893240in}{0.530317in}}%
\pgfpathlineto{\pgfqpoint{4.897685in}{0.529935in}}%
\pgfpathlineto{\pgfqpoint{4.902129in}{0.529553in}}%
\pgfpathlineto{\pgfqpoint{4.906573in}{0.529172in}}%
\pgfpathlineto{\pgfqpoint{4.911018in}{0.528792in}}%
\pgfpathlineto{\pgfqpoint{4.915462in}{0.528413in}}%
\pgfpathlineto{\pgfqpoint{4.919906in}{0.528034in}}%
\pgfpathlineto{\pgfqpoint{4.924351in}{0.527656in}}%
\pgfpathlineto{\pgfqpoint{4.928795in}{0.527279in}}%
\pgfpathlineto{\pgfqpoint{4.933239in}{0.526902in}}%
\pgfpathlineto{\pgfqpoint{4.937684in}{0.526527in}}%
\pgfpathlineto{\pgfqpoint{4.942128in}{0.526152in}}%
\pgfpathlineto{\pgfqpoint{4.946572in}{0.525777in}}%
\pgfpathlineto{\pgfqpoint{4.951017in}{0.525404in}}%
\pgfpathlineto{\pgfqpoint{4.955461in}{0.525031in}}%
\pgfpathlineto{\pgfqpoint{4.959905in}{0.524659in}}%
\pgfpathlineto{\pgfqpoint{4.964350in}{0.524287in}}%
\pgfpathlineto{\pgfqpoint{4.968794in}{0.523917in}}%
\pgfpathlineto{\pgfqpoint{4.973238in}{0.523546in}}%
\pgfpathlineto{\pgfqpoint{4.977683in}{0.523177in}}%
\pgfpathlineto{\pgfqpoint{4.982127in}{0.522809in}}%
\pgfpathlineto{\pgfqpoint{4.986571in}{0.522441in}}%
\pgfpathlineto{\pgfqpoint{4.991016in}{0.522073in}}%
\pgfpathlineto{\pgfqpoint{4.995460in}{0.521707in}}%
\pgfpathlineto{\pgfqpoint{4.999904in}{0.521341in}}%
\pgfpathlineto{\pgfqpoint{5.004349in}{0.520976in}}%
\pgfpathlineto{\pgfqpoint{5.008793in}{0.520612in}}%
\pgfpathlineto{\pgfqpoint{5.013238in}{0.520248in}}%
\pgfpathlineto{\pgfqpoint{5.017682in}{0.519885in}}%
\pgfpathlineto{\pgfqpoint{5.022126in}{0.519522in}}%
\pgfpathlineto{\pgfqpoint{5.026571in}{0.519161in}}%
\pgfpathlineto{\pgfqpoint{5.031015in}{0.518800in}}%
\pgfpathlineto{\pgfqpoint{5.035459in}{0.518440in}}%
\pgfpathlineto{\pgfqpoint{5.039904in}{0.518080in}}%
\pgfpathlineto{\pgfqpoint{5.044348in}{0.517721in}}%
\pgfpathlineto{\pgfqpoint{5.048792in}{0.517363in}}%
\pgfpathlineto{\pgfqpoint{5.053237in}{0.517005in}}%
\pgfpathlineto{\pgfqpoint{5.057681in}{0.516648in}}%
\pgfpathlineto{\pgfqpoint{5.062125in}{0.516292in}}%
\pgfpathlineto{\pgfqpoint{5.066570in}{0.515936in}}%
\pgfpathlineto{\pgfqpoint{5.071014in}{0.515581in}}%
\pgfpathlineto{\pgfqpoint{5.075458in}{0.515227in}}%
\pgfpathlineto{\pgfqpoint{5.079903in}{0.514874in}}%
\pgfpathlineto{\pgfqpoint{5.084347in}{0.514521in}}%
\pgfpathlineto{\pgfqpoint{5.088791in}{0.514168in}}%
\pgfpathlineto{\pgfqpoint{5.093236in}{0.513817in}}%
\pgfpathlineto{\pgfqpoint{5.097680in}{0.513466in}}%
\pgfpathlineto{\pgfqpoint{5.102124in}{0.513115in}}%
\pgfpathlineto{\pgfqpoint{5.106569in}{0.512766in}}%
\pgfpathlineto{\pgfqpoint{5.111013in}{0.512417in}}%
\pgfpathlineto{\pgfqpoint{5.115457in}{0.512068in}}%
\pgfpathlineto{\pgfqpoint{5.119902in}{0.511721in}}%
\pgfpathlineto{\pgfqpoint{5.124346in}{0.511374in}}%
\pgfpathlineto{\pgfqpoint{5.128791in}{0.511027in}}%
\pgfpathlineto{\pgfqpoint{5.133235in}{0.510681in}}%
\pgfpathlineto{\pgfqpoint{5.137679in}{0.510336in}}%
\pgfpathlineto{\pgfqpoint{5.142124in}{0.509992in}}%
\pgfpathlineto{\pgfqpoint{5.146568in}{0.509648in}}%
\pgfpathlineto{\pgfqpoint{5.151012in}{0.509304in}}%
\pgfpathlineto{\pgfqpoint{5.155457in}{0.508962in}}%
\pgfpathlineto{\pgfqpoint{5.159901in}{0.508620in}}%
\pgfpathlineto{\pgfqpoint{5.164345in}{0.508278in}}%
\pgfpathlineto{\pgfqpoint{5.168790in}{0.507938in}}%
\pgfpathlineto{\pgfqpoint{5.173234in}{0.507598in}}%
\pgfpathlineto{\pgfqpoint{5.177678in}{0.507258in}}%
\pgfpathlineto{\pgfqpoint{5.182123in}{0.506919in}}%
\pgfpathlineto{\pgfqpoint{5.186567in}{0.506581in}}%
\pgfpathlineto{\pgfqpoint{5.191011in}{0.506243in}}%
\pgfpathlineto{\pgfqpoint{5.195456in}{0.505906in}}%
\pgfpathlineto{\pgfqpoint{5.199900in}{0.505570in}}%
\pgfpathlineto{\pgfqpoint{5.199900in}{0.505570in}}%
\pgfpathlineto{\pgfqpoint{5.207241in}{0.505026in}}%
\pgfpathlineto{\pgfqpoint{5.214582in}{0.504503in}}%
\pgfpathlineto{\pgfqpoint{5.221923in}{0.503999in}}%
\pgfpathlineto{\pgfqpoint{5.229264in}{0.503513in}}%
\pgfpathlineto{\pgfqpoint{5.236605in}{0.503044in}}%
\pgfpathlineto{\pgfqpoint{5.243946in}{0.502590in}}%
\pgfpathlineto{\pgfqpoint{5.251287in}{0.502152in}}%
\pgfpathlineto{\pgfqpoint{5.258629in}{0.501727in}}%
\pgfpathlineto{\pgfqpoint{5.265970in}{0.501316in}}%
\pgfpathlineto{\pgfqpoint{5.273311in}{0.500916in}}%
\pgfpathlineto{\pgfqpoint{5.280652in}{0.500529in}}%
\pgfpathlineto{\pgfqpoint{5.287993in}{0.500152in}}%
\pgfpathlineto{\pgfqpoint{5.295334in}{0.499786in}}%
\pgfpathlineto{\pgfqpoint{5.302675in}{0.499430in}}%
\pgfpathlineto{\pgfqpoint{5.310016in}{0.499083in}}%
\pgfpathlineto{\pgfqpoint{5.317357in}{0.498745in}}%
\pgfpathlineto{\pgfqpoint{5.324698in}{0.498416in}}%
\pgfpathlineto{\pgfqpoint{5.332039in}{0.498095in}}%
\pgfpathlineto{\pgfqpoint{5.339380in}{0.497781in}}%
\pgfpathlineto{\pgfqpoint{5.346721in}{0.497475in}}%
\pgfpathlineto{\pgfqpoint{5.354062in}{0.497177in}}%
\pgfpathlineto{\pgfqpoint{5.361403in}{0.496885in}}%
\pgfpathlineto{\pgfqpoint{5.368744in}{0.496599in}}%
\pgfpathlineto{\pgfqpoint{5.376085in}{0.496320in}}%
\pgfpathlineto{\pgfqpoint{5.383426in}{0.496047in}}%
\pgfpathlineto{\pgfqpoint{5.390768in}{0.495780in}}%
\pgfpathlineto{\pgfqpoint{5.398109in}{0.495519in}}%
\pgfpathlineto{\pgfqpoint{5.405450in}{0.495263in}}%
\pgfpathlineto{\pgfqpoint{5.412791in}{0.495012in}}%
\pgfpathlineto{\pgfqpoint{5.420132in}{0.494766in}}%
\pgfpathlineto{\pgfqpoint{5.427473in}{0.494525in}}%
\pgfpathlineto{\pgfqpoint{5.434814in}{0.494288in}}%
\pgfpathlineto{\pgfqpoint{5.442155in}{0.494056in}}%
\pgfpathlineto{\pgfqpoint{5.449496in}{0.493829in}}%
\pgfpathlineto{\pgfqpoint{5.456837in}{0.493605in}}%
\pgfpathlineto{\pgfqpoint{5.464178in}{0.493386in}}%
\pgfpathlineto{\pgfqpoint{5.471519in}{0.493171in}}%
\pgfpathlineto{\pgfqpoint{5.478860in}{0.492959in}}%
\pgfpathlineto{\pgfqpoint{5.486201in}{0.492751in}}%
\pgfpathlineto{\pgfqpoint{5.493542in}{0.492547in}}%
\pgfpathlineto{\pgfqpoint{5.500883in}{0.492347in}}%
\pgfpathlineto{\pgfqpoint{5.508224in}{0.492149in}}%
\pgfpathlineto{\pgfqpoint{5.515566in}{0.491955in}}%
\pgfpathlineto{\pgfqpoint{5.522907in}{0.491764in}}%
\pgfpathlineto{\pgfqpoint{5.530248in}{0.491577in}}%
\pgfpathlineto{\pgfqpoint{5.537589in}{0.491392in}}%
\pgfpathlineto{\pgfqpoint{5.544930in}{0.491210in}}%
\pgfpathlineto{\pgfqpoint{5.552271in}{0.491031in}}%
\pgfpathlineto{\pgfqpoint{5.559612in}{0.490855in}}%
\pgfpathlineto{\pgfqpoint{5.566953in}{0.490682in}}%
\pgfpathlineto{\pgfqpoint{5.574294in}{0.490511in}}%
\pgfpathlineto{\pgfqpoint{5.581635in}{0.490343in}}%
\pgfpathlineto{\pgfqpoint{5.588976in}{0.490177in}}%
\pgfpathlineto{\pgfqpoint{5.596317in}{0.490013in}}%
\pgfpathlineto{\pgfqpoint{5.603658in}{0.489853in}}%
\pgfpathlineto{\pgfqpoint{5.610999in}{0.489694in}}%
\pgfpathlineto{\pgfqpoint{5.618340in}{0.489538in}}%
\pgfpathlineto{\pgfqpoint{5.625681in}{0.489383in}}%
\pgfpathlineto{\pgfqpoint{5.633022in}{0.489231in}}%
\pgfpathlineto{\pgfqpoint{5.640363in}{0.489082in}}%
\pgfpathlineto{\pgfqpoint{5.647705in}{0.488934in}}%
\pgfpathlineto{\pgfqpoint{5.655046in}{0.488788in}}%
\pgfpathlineto{\pgfqpoint{5.662387in}{0.488644in}}%
\pgfpathlineto{\pgfqpoint{5.669728in}{0.488502in}}%
\pgfpathlineto{\pgfqpoint{5.677069in}{0.488362in}}%
\pgfpathlineto{\pgfqpoint{5.684410in}{0.488224in}}%
\pgfpathlineto{\pgfqpoint{5.691751in}{0.488087in}}%
\pgfpathlineto{\pgfqpoint{5.699092in}{0.487953in}}%
\pgfpathlineto{\pgfqpoint{5.706433in}{0.487820in}}%
\pgfpathlineto{\pgfqpoint{5.713774in}{0.487688in}}%
\pgfpathlineto{\pgfqpoint{5.721115in}{0.487559in}}%
\pgfpathlineto{\pgfqpoint{5.728456in}{0.487431in}}%
\pgfpathlineto{\pgfqpoint{5.735797in}{0.487304in}}%
\pgfpathlineto{\pgfqpoint{5.743138in}{0.487179in}}%
\pgfpathlineto{\pgfqpoint{5.750479in}{0.487056in}}%
\pgfpathlineto{\pgfqpoint{5.757820in}{0.486934in}}%
\pgfpathlineto{\pgfqpoint{5.765161in}{0.486814in}}%
\pgfpathlineto{\pgfqpoint{5.772503in}{0.486695in}}%
\pgfpathlineto{\pgfqpoint{5.779844in}{0.486577in}}%
\pgfpathlineto{\pgfqpoint{5.787185in}{0.486461in}}%
\pgfpathlineto{\pgfqpoint{5.794526in}{0.486346in}}%
\pgfpathlineto{\pgfqpoint{5.801867in}{0.486232in}}%
\pgfpathlineto{\pgfqpoint{5.809208in}{0.486120in}}%
\pgfpathlineto{\pgfqpoint{5.816549in}{0.486009in}}%
\pgfpathlineto{\pgfqpoint{5.823890in}{0.485899in}}%
\pgfpathlineto{\pgfqpoint{5.831231in}{0.485791in}}%
\pgfpathlineto{\pgfqpoint{5.838572in}{0.485683in}}%
\pgfpathlineto{\pgfqpoint{5.845913in}{0.485577in}}%
\pgfpathlineto{\pgfqpoint{5.853254in}{0.485472in}}%
\pgfpathlineto{\pgfqpoint{5.860595in}{0.485368in}}%
\pgfpathlineto{\pgfqpoint{5.867936in}{0.485265in}}%
\pgfpathlineto{\pgfqpoint{5.875277in}{0.485164in}}%
\pgfpathlineto{\pgfqpoint{5.882618in}{0.485063in}}%
\pgfpathlineto{\pgfqpoint{5.889959in}{0.484964in}}%
\pgfpathlineto{\pgfqpoint{5.897300in}{0.484865in}}%
\pgfpathlineto{\pgfqpoint{5.904642in}{0.484768in}}%
\pgfpathlineto{\pgfqpoint{5.911983in}{0.484671in}}%
\pgfpathlineto{\pgfqpoint{5.919324in}{0.484576in}}%
\pgfpathlineto{\pgfqpoint{5.926665in}{0.484481in}}%
\pgfpathlineto{\pgfqpoint{7.047223in}{0.554012in}}%
\pgfpathlineto{\pgfqpoint{7.038414in}{0.554125in}}%
\pgfpathlineto{\pgfqpoint{7.029605in}{0.554240in}}%
\pgfpathlineto{\pgfqpoint{7.020795in}{0.554355in}}%
\pgfpathlineto{\pgfqpoint{7.011986in}{0.554472in}}%
\pgfpathlineto{\pgfqpoint{7.003177in}{0.554591in}}%
\pgfpathlineto{\pgfqpoint{6.994367in}{0.554710in}}%
\pgfpathlineto{\pgfqpoint{6.985558in}{0.554831in}}%
\pgfpathlineto{\pgfqpoint{6.976749in}{0.554953in}}%
\pgfpathlineto{\pgfqpoint{6.967940in}{0.555076in}}%
\pgfpathlineto{\pgfqpoint{6.959130in}{0.555201in}}%
\pgfpathlineto{\pgfqpoint{6.950321in}{0.555327in}}%
\pgfpathlineto{\pgfqpoint{6.941512in}{0.555454in}}%
\pgfpathlineto{\pgfqpoint{6.932703in}{0.555583in}}%
\pgfpathlineto{\pgfqpoint{6.923893in}{0.555713in}}%
\pgfpathlineto{\pgfqpoint{6.915084in}{0.555845in}}%
\pgfpathlineto{\pgfqpoint{6.906275in}{0.555978in}}%
\pgfpathlineto{\pgfqpoint{6.897465in}{0.556113in}}%
\pgfpathlineto{\pgfqpoint{6.888656in}{0.556249in}}%
\pgfpathlineto{\pgfqpoint{6.879847in}{0.556387in}}%
\pgfpathlineto{\pgfqpoint{6.871038in}{0.556527in}}%
\pgfpathlineto{\pgfqpoint{6.862228in}{0.556668in}}%
\pgfpathlineto{\pgfqpoint{6.853419in}{0.556811in}}%
\pgfpathlineto{\pgfqpoint{6.844610in}{0.556955in}}%
\pgfpathlineto{\pgfqpoint{6.835801in}{0.557101in}}%
\pgfpathlineto{\pgfqpoint{6.826991in}{0.557249in}}%
\pgfpathlineto{\pgfqpoint{6.818182in}{0.557399in}}%
\pgfpathlineto{\pgfqpoint{6.809373in}{0.557551in}}%
\pgfpathlineto{\pgfqpoint{6.800564in}{0.557705in}}%
\pgfpathlineto{\pgfqpoint{6.791754in}{0.557860in}}%
\pgfpathlineto{\pgfqpoint{6.782945in}{0.558018in}}%
\pgfpathlineto{\pgfqpoint{6.774136in}{0.558177in}}%
\pgfpathlineto{\pgfqpoint{6.765326in}{0.558339in}}%
\pgfpathlineto{\pgfqpoint{6.756517in}{0.558503in}}%
\pgfpathlineto{\pgfqpoint{6.747708in}{0.558669in}}%
\pgfpathlineto{\pgfqpoint{6.738899in}{0.558837in}}%
\pgfpathlineto{\pgfqpoint{6.730089in}{0.559007in}}%
\pgfpathlineto{\pgfqpoint{6.721280in}{0.559180in}}%
\pgfpathlineto{\pgfqpoint{6.712471in}{0.559355in}}%
\pgfpathlineto{\pgfqpoint{6.703662in}{0.559532in}}%
\pgfpathlineto{\pgfqpoint{6.694852in}{0.559712in}}%
\pgfpathlineto{\pgfqpoint{6.686043in}{0.559894in}}%
\pgfpathlineto{\pgfqpoint{6.677234in}{0.560079in}}%
\pgfpathlineto{\pgfqpoint{6.668425in}{0.560267in}}%
\pgfpathlineto{\pgfqpoint{6.659615in}{0.560457in}}%
\pgfpathlineto{\pgfqpoint{6.650806in}{0.560650in}}%
\pgfpathlineto{\pgfqpoint{6.641997in}{0.560846in}}%
\pgfpathlineto{\pgfqpoint{6.633187in}{0.561045in}}%
\pgfpathlineto{\pgfqpoint{6.624378in}{0.561247in}}%
\pgfpathlineto{\pgfqpoint{6.615569in}{0.561452in}}%
\pgfpathlineto{\pgfqpoint{6.606760in}{0.561660in}}%
\pgfpathlineto{\pgfqpoint{6.597950in}{0.561872in}}%
\pgfpathlineto{\pgfqpoint{6.589141in}{0.562086in}}%
\pgfpathlineto{\pgfqpoint{6.580332in}{0.562305in}}%
\pgfpathlineto{\pgfqpoint{6.571523in}{0.562526in}}%
\pgfpathlineto{\pgfqpoint{6.562713in}{0.562751in}}%
\pgfpathlineto{\pgfqpoint{6.553904in}{0.562980in}}%
\pgfpathlineto{\pgfqpoint{6.545095in}{0.563213in}}%
\pgfpathlineto{\pgfqpoint{6.536285in}{0.563450in}}%
\pgfpathlineto{\pgfqpoint{6.527476in}{0.563691in}}%
\pgfpathlineto{\pgfqpoint{6.518667in}{0.563936in}}%
\pgfpathlineto{\pgfqpoint{6.509858in}{0.564185in}}%
\pgfpathlineto{\pgfqpoint{6.501048in}{0.564439in}}%
\pgfpathlineto{\pgfqpoint{6.492239in}{0.564698in}}%
\pgfpathlineto{\pgfqpoint{6.483430in}{0.564961in}}%
\pgfpathlineto{\pgfqpoint{6.474621in}{0.565229in}}%
\pgfpathlineto{\pgfqpoint{6.465811in}{0.565502in}}%
\pgfpathlineto{\pgfqpoint{6.457002in}{0.565780in}}%
\pgfpathlineto{\pgfqpoint{6.448193in}{0.566064in}}%
\pgfpathlineto{\pgfqpoint{6.439384in}{0.566353in}}%
\pgfpathlineto{\pgfqpoint{6.430574in}{0.566648in}}%
\pgfpathlineto{\pgfqpoint{6.421765in}{0.566949in}}%
\pgfpathlineto{\pgfqpoint{6.412956in}{0.567257in}}%
\pgfpathlineto{\pgfqpoint{6.404146in}{0.567571in}}%
\pgfpathlineto{\pgfqpoint{6.395337in}{0.567891in}}%
\pgfpathlineto{\pgfqpoint{6.386528in}{0.568219in}}%
\pgfpathlineto{\pgfqpoint{6.377719in}{0.568554in}}%
\pgfpathlineto{\pgfqpoint{6.368909in}{0.568896in}}%
\pgfpathlineto{\pgfqpoint{6.360100in}{0.569246in}}%
\pgfpathlineto{\pgfqpoint{6.351291in}{0.569605in}}%
\pgfpathlineto{\pgfqpoint{6.342482in}{0.569972in}}%
\pgfpathlineto{\pgfqpoint{6.333672in}{0.570348in}}%
\pgfpathlineto{\pgfqpoint{6.324863in}{0.570733in}}%
\pgfpathlineto{\pgfqpoint{6.316054in}{0.571128in}}%
\pgfpathlineto{\pgfqpoint{6.307245in}{0.571534in}}%
\pgfpathlineto{\pgfqpoint{6.298435in}{0.571950in}}%
\pgfpathlineto{\pgfqpoint{6.289626in}{0.572377in}}%
\pgfpathlineto{\pgfqpoint{6.280817in}{0.572817in}}%
\pgfpathlineto{\pgfqpoint{6.272007in}{0.573269in}}%
\pgfpathlineto{\pgfqpoint{6.263198in}{0.573734in}}%
\pgfpathlineto{\pgfqpoint{6.254389in}{0.574213in}}%
\pgfpathlineto{\pgfqpoint{6.245580in}{0.574707in}}%
\pgfpathlineto{\pgfqpoint{6.236770in}{0.575217in}}%
\pgfpathlineto{\pgfqpoint{6.227961in}{0.575743in}}%
\pgfpathlineto{\pgfqpoint{6.219152in}{0.576287in}}%
\pgfpathlineto{\pgfqpoint{6.210343in}{0.576850in}}%
\pgfpathlineto{\pgfqpoint{6.201533in}{0.577433in}}%
\pgfpathlineto{\pgfqpoint{6.192724in}{0.578037in}}%
\pgfpathlineto{\pgfqpoint{6.183915in}{0.578665in}}%
\pgfpathlineto{\pgfqpoint{6.175105in}{0.579318in}}%
\pgfpathlineto{\pgfqpoint{6.175105in}{0.579318in}}%
\pgfpathlineto{\pgfqpoint{6.169772in}{0.579722in}}%
\pgfpathlineto{\pgfqpoint{6.164439in}{0.580126in}}%
\pgfpathlineto{\pgfqpoint{6.159106in}{0.580531in}}%
\pgfpathlineto{\pgfqpoint{6.153773in}{0.580937in}}%
\pgfpathlineto{\pgfqpoint{6.148439in}{0.581344in}}%
\pgfpathlineto{\pgfqpoint{6.143106in}{0.581751in}}%
\pgfpathlineto{\pgfqpoint{6.137773in}{0.582160in}}%
\pgfpathlineto{\pgfqpoint{6.132440in}{0.582568in}}%
\pgfpathlineto{\pgfqpoint{6.127107in}{0.582978in}}%
\pgfpathlineto{\pgfqpoint{6.121773in}{0.583388in}}%
\pgfpathlineto{\pgfqpoint{6.116440in}{0.583800in}}%
\pgfpathlineto{\pgfqpoint{6.111107in}{0.584212in}}%
\pgfpathlineto{\pgfqpoint{6.105774in}{0.584624in}}%
\pgfpathlineto{\pgfqpoint{6.100440in}{0.585038in}}%
\pgfpathlineto{\pgfqpoint{6.095107in}{0.585452in}}%
\pgfpathlineto{\pgfqpoint{6.089774in}{0.585867in}}%
\pgfpathlineto{\pgfqpoint{6.084441in}{0.586282in}}%
\pgfpathlineto{\pgfqpoint{6.079108in}{0.586699in}}%
\pgfpathlineto{\pgfqpoint{6.073774in}{0.587116in}}%
\pgfpathlineto{\pgfqpoint{6.068441in}{0.587534in}}%
\pgfpathlineto{\pgfqpoint{6.063108in}{0.587953in}}%
\pgfpathlineto{\pgfqpoint{6.057775in}{0.588373in}}%
\pgfpathlineto{\pgfqpoint{6.052442in}{0.588793in}}%
\pgfpathlineto{\pgfqpoint{6.047108in}{0.589214in}}%
\pgfpathlineto{\pgfqpoint{6.041775in}{0.589636in}}%
\pgfpathlineto{\pgfqpoint{6.036442in}{0.590059in}}%
\pgfpathlineto{\pgfqpoint{6.031109in}{0.590482in}}%
\pgfpathlineto{\pgfqpoint{6.025775in}{0.590907in}}%
\pgfpathlineto{\pgfqpoint{6.020442in}{0.591332in}}%
\pgfpathlineto{\pgfqpoint{6.015109in}{0.591758in}}%
\pgfpathlineto{\pgfqpoint{6.009776in}{0.592185in}}%
\pgfpathlineto{\pgfqpoint{6.004443in}{0.592612in}}%
\pgfpathlineto{\pgfqpoint{5.999109in}{0.593040in}}%
\pgfpathlineto{\pgfqpoint{5.993776in}{0.593469in}}%
\pgfpathlineto{\pgfqpoint{5.988443in}{0.593899in}}%
\pgfpathlineto{\pgfqpoint{5.983110in}{0.594330in}}%
\pgfpathlineto{\pgfqpoint{5.977777in}{0.594762in}}%
\pgfpathlineto{\pgfqpoint{5.972443in}{0.595194in}}%
\pgfpathlineto{\pgfqpoint{5.967110in}{0.595627in}}%
\pgfpathlineto{\pgfqpoint{5.961777in}{0.596061in}}%
\pgfpathlineto{\pgfqpoint{5.956444in}{0.596496in}}%
\pgfpathlineto{\pgfqpoint{5.951110in}{0.596932in}}%
\pgfpathlineto{\pgfqpoint{5.945777in}{0.597368in}}%
\pgfpathlineto{\pgfqpoint{5.940444in}{0.597806in}}%
\pgfpathlineto{\pgfqpoint{5.935111in}{0.598244in}}%
\pgfpathlineto{\pgfqpoint{5.929778in}{0.598683in}}%
\pgfpathlineto{\pgfqpoint{5.924444in}{0.599122in}}%
\pgfpathlineto{\pgfqpoint{5.919111in}{0.599563in}}%
\pgfpathlineto{\pgfqpoint{5.913778in}{0.600005in}}%
\pgfpathlineto{\pgfqpoint{5.908445in}{0.600447in}}%
\pgfpathlineto{\pgfqpoint{5.903112in}{0.600890in}}%
\pgfpathlineto{\pgfqpoint{5.897778in}{0.601334in}}%
\pgfpathlineto{\pgfqpoint{5.892445in}{0.601779in}}%
\pgfpathlineto{\pgfqpoint{5.887112in}{0.602225in}}%
\pgfpathlineto{\pgfqpoint{5.881779in}{0.602671in}}%
\pgfpathlineto{\pgfqpoint{5.876445in}{0.603119in}}%
\pgfpathlineto{\pgfqpoint{5.871112in}{0.603567in}}%
\pgfpathlineto{\pgfqpoint{5.865779in}{0.604016in}}%
\pgfpathlineto{\pgfqpoint{5.860446in}{0.604466in}}%
\pgfpathlineto{\pgfqpoint{5.855113in}{0.604917in}}%
\pgfpathlineto{\pgfqpoint{5.849779in}{0.605369in}}%
\pgfpathlineto{\pgfqpoint{5.844446in}{0.605821in}}%
\pgfpathlineto{\pgfqpoint{5.839113in}{0.606275in}}%
\pgfpathlineto{\pgfqpoint{5.833780in}{0.606729in}}%
\pgfpathlineto{\pgfqpoint{5.828446in}{0.607185in}}%
\pgfpathlineto{\pgfqpoint{5.823113in}{0.607641in}}%
\pgfpathlineto{\pgfqpoint{5.817780in}{0.608098in}}%
\pgfpathlineto{\pgfqpoint{5.812447in}{0.608556in}}%
\pgfpathlineto{\pgfqpoint{5.807114in}{0.609015in}}%
\pgfpathlineto{\pgfqpoint{5.801780in}{0.609474in}}%
\pgfpathlineto{\pgfqpoint{5.796447in}{0.609935in}}%
\pgfpathlineto{\pgfqpoint{5.791114in}{0.610396in}}%
\pgfpathlineto{\pgfqpoint{5.785781in}{0.610859in}}%
\pgfpathlineto{\pgfqpoint{5.780448in}{0.611322in}}%
\pgfpathlineto{\pgfqpoint{5.775114in}{0.611786in}}%
\pgfpathlineto{\pgfqpoint{5.769781in}{0.612252in}}%
\pgfpathlineto{\pgfqpoint{5.764448in}{0.612718in}}%
\pgfpathlineto{\pgfqpoint{5.759115in}{0.613185in}}%
\pgfpathlineto{\pgfqpoint{5.753781in}{0.613652in}}%
\pgfpathlineto{\pgfqpoint{5.748448in}{0.614121in}}%
\pgfpathlineto{\pgfqpoint{5.743115in}{0.614591in}}%
\pgfpathlineto{\pgfqpoint{5.737782in}{0.615062in}}%
\pgfpathlineto{\pgfqpoint{5.732449in}{0.615533in}}%
\pgfpathlineto{\pgfqpoint{5.727115in}{0.616006in}}%
\pgfpathlineto{\pgfqpoint{5.721782in}{0.616479in}}%
\pgfpathlineto{\pgfqpoint{5.716449in}{0.616953in}}%
\pgfpathlineto{\pgfqpoint{5.711116in}{0.617429in}}%
\pgfpathlineto{\pgfqpoint{5.705783in}{0.617905in}}%
\pgfpathlineto{\pgfqpoint{5.700449in}{0.618382in}}%
\pgfpathlineto{\pgfqpoint{5.695116in}{0.618860in}}%
\pgfpathlineto{\pgfqpoint{5.689783in}{0.619339in}}%
\pgfpathlineto{\pgfqpoint{5.684450in}{0.619819in}}%
\pgfpathlineto{\pgfqpoint{5.679116in}{0.620300in}}%
\pgfpathlineto{\pgfqpoint{5.673783in}{0.620782in}}%
\pgfpathlineto{\pgfqpoint{5.668450in}{0.621265in}}%
\pgfpathlineto{\pgfqpoint{5.663117in}{0.621749in}}%
\pgfpathlineto{\pgfqpoint{5.657784in}{0.622234in}}%
\pgfpathlineto{\pgfqpoint{5.652450in}{0.622720in}}%
\pgfpathlineto{\pgfqpoint{5.647117in}{0.623207in}}%
\pgfpathlineto{\pgfqpoint{5.641784in}{0.623695in}}%
\pgfpathlineto{\pgfqpoint{5.636451in}{0.624183in}}%
\pgfpathlineto{\pgfqpoint{5.631118in}{0.624673in}}%
\pgfpathlineto{\pgfqpoint{5.625784in}{0.625164in}}%
\pgfpathlineto{\pgfqpoint{5.620451in}{0.625655in}}%
\pgfpathlineto{\pgfqpoint{5.615118in}{0.626148in}}%
\pgfpathlineto{\pgfqpoint{5.609785in}{0.626642in}}%
\pgfpathlineto{\pgfqpoint{5.604451in}{0.627137in}}%
\pgfpathlineto{\pgfqpoint{5.599118in}{0.627632in}}%
\pgfpathlineto{\pgfqpoint{5.593785in}{0.628129in}}%
\pgfpathlineto{\pgfqpoint{5.588452in}{0.628627in}}%
\pgfpathlineto{\pgfqpoint{5.583119in}{0.629125in}}%
\pgfpathlineto{\pgfqpoint{5.577785in}{0.629625in}}%
\pgfpathlineto{\pgfqpoint{5.572452in}{0.630126in}}%
\pgfpathlineto{\pgfqpoint{5.567119in}{0.630628in}}%
\pgfpathlineto{\pgfqpoint{5.561786in}{0.631130in}}%
\pgfpathlineto{\pgfqpoint{5.556452in}{0.631634in}}%
\pgfpathlineto{\pgfqpoint{5.551119in}{0.632139in}}%
\pgfpathlineto{\pgfqpoint{5.545786in}{0.632645in}}%
\pgfpathlineto{\pgfqpoint{5.540453in}{0.633152in}}%
\pgfpathlineto{\pgfqpoint{5.535120in}{0.633660in}}%
\pgfpathlineto{\pgfqpoint{5.529786in}{0.634169in}}%
\pgfpathlineto{\pgfqpoint{5.524453in}{0.634679in}}%
\pgfpathlineto{\pgfqpoint{5.519120in}{0.635190in}}%
\pgfpathlineto{\pgfqpoint{5.513787in}{0.635702in}}%
\pgfpathlineto{\pgfqpoint{5.508454in}{0.636215in}}%
\pgfpathlineto{\pgfqpoint{5.503120in}{0.636729in}}%
\pgfpathlineto{\pgfqpoint{5.497787in}{0.637245in}}%
\pgfpathlineto{\pgfqpoint{5.492454in}{0.637761in}}%
\pgfpathlineto{\pgfqpoint{5.487121in}{0.638278in}}%
\pgfpathlineto{\pgfqpoint{5.481787in}{0.638797in}}%
\pgfpathlineto{\pgfqpoint{5.476454in}{0.639316in}}%
\pgfpathlineto{\pgfqpoint{5.471121in}{0.639837in}}%
\pgfpathlineto{\pgfqpoint{5.465788in}{0.640359in}}%
\pgfpathlineto{\pgfqpoint{5.460455in}{0.640882in}}%
\pgfpathlineto{\pgfqpoint{5.455121in}{0.641406in}}%
\pgfpathlineto{\pgfqpoint{5.449788in}{0.641930in}}%
\pgfpathlineto{\pgfqpoint{5.444455in}{0.642457in}}%
\pgfpathlineto{\pgfqpoint{5.439122in}{0.642984in}}%
\pgfpathlineto{\pgfqpoint{5.433789in}{0.643512in}}%
\pgfpathlineto{\pgfqpoint{5.428455in}{0.644041in}}%
\pgfpathlineto{\pgfqpoint{5.423122in}{0.644572in}}%
\pgfpathlineto{\pgfqpoint{5.417789in}{0.645103in}}%
\pgfpathlineto{\pgfqpoint{5.412456in}{0.645636in}}%
\pgfpathlineto{\pgfqpoint{5.407122in}{0.646170in}}%
\pgfpathlineto{\pgfqpoint{5.401789in}{0.646705in}}%
\pgfpathlineto{\pgfqpoint{5.396456in}{0.647241in}}%
\pgfpathlineto{\pgfqpoint{5.391123in}{0.647778in}}%
\pgfpathlineto{\pgfqpoint{5.385790in}{0.648317in}}%
\pgfpathlineto{\pgfqpoint{5.380456in}{0.648856in}}%
\pgfpathlineto{\pgfqpoint{5.375123in}{0.649397in}}%
\pgfpathlineto{\pgfqpoint{5.369790in}{0.649938in}}%
\pgfpathlineto{\pgfqpoint{5.364457in}{0.650481in}}%
\pgfpathlineto{\pgfqpoint{5.359124in}{0.651025in}}%
\pgfpathlineto{\pgfqpoint{5.353790in}{0.651571in}}%
\pgfpathlineto{\pgfqpoint{5.348457in}{0.652117in}}%
\pgfpathlineto{\pgfqpoint{5.343124in}{0.652664in}}%
\pgfpathlineto{\pgfqpoint{5.337791in}{0.653213in}}%
\pgfpathlineto{\pgfqpoint{5.332457in}{0.653763in}}%
\pgfpathlineto{\pgfqpoint{5.327124in}{0.654314in}}%
\pgfpathlineto{\pgfqpoint{5.321791in}{0.654866in}}%
\pgfpathlineto{\pgfqpoint{5.316458in}{0.655420in}}%
\pgfpathlineto{\pgfqpoint{5.311125in}{0.655974in}}%
\pgfpathlineto{\pgfqpoint{5.305791in}{0.656530in}}%
\pgfpathlineto{\pgfqpoint{5.300458in}{0.657087in}}%
\pgfpathlineto{\pgfqpoint{5.295125in}{0.657645in}}%
\pgfpathlineto{\pgfqpoint{5.289792in}{0.658204in}}%
\pgfpathlineto{\pgfqpoint{5.284458in}{0.658765in}}%
\pgfpathlineto{\pgfqpoint{5.279125in}{0.659327in}}%
\pgfpathlineto{\pgfqpoint{5.273792in}{0.659890in}}%
\pgfpathlineto{\pgfqpoint{5.268459in}{0.660454in}}%
\pgfpathlineto{\pgfqpoint{5.263126in}{0.661019in}}%
\pgfpathlineto{\pgfqpoint{5.257792in}{0.661586in}}%
\pgfpathlineto{\pgfqpoint{5.252459in}{0.662154in}}%
\pgfpathlineto{\pgfqpoint{5.247126in}{0.662723in}}%
\pgfpathlineto{\pgfqpoint{5.241793in}{0.663293in}}%
\pgfpathlineto{\pgfqpoint{5.236460in}{0.663864in}}%
\pgfpathlineto{\pgfqpoint{5.231126in}{0.664437in}}%
\pgfpathlineto{\pgfqpoint{5.225793in}{0.665011in}}%
\pgfpathlineto{\pgfqpoint{5.220460in}{0.665587in}}%
\pgfpathlineto{\pgfqpoint{5.215127in}{0.666163in}}%
\pgfpathlineto{\pgfqpoint{5.209793in}{0.666741in}}%
\pgfpathlineto{\pgfqpoint{5.204460in}{0.667320in}}%
\pgfpathlineto{\pgfqpoint{5.199127in}{0.667900in}}%
\pgfpathlineto{\pgfqpoint{5.193794in}{0.668482in}}%
\pgfpathlineto{\pgfqpoint{5.188461in}{0.669065in}}%
\pgfpathlineto{\pgfqpoint{5.183127in}{0.669649in}}%
\pgfpathlineto{\pgfqpoint{5.177794in}{0.670234in}}%
\pgfpathlineto{\pgfqpoint{5.172461in}{0.670821in}}%
\pgfpathlineto{\pgfqpoint{5.167128in}{0.671409in}}%
\pgfpathlineto{\pgfqpoint{5.161795in}{0.671998in}}%
\pgfpathlineto{\pgfqpoint{5.156461in}{0.672589in}}%
\pgfpathlineto{\pgfqpoint{5.151128in}{0.673181in}}%
\pgfpathlineto{\pgfqpoint{5.145795in}{0.673774in}}%
\pgfpathlineto{\pgfqpoint{5.140462in}{0.674369in}}%
\pgfpathlineto{\pgfqpoint{5.135128in}{0.674965in}}%
\pgfpathlineto{\pgfqpoint{5.129795in}{0.675562in}}%
\pgfpathlineto{\pgfqpoint{5.124462in}{0.676160in}}%
\pgfpathlineto{\pgfqpoint{5.119129in}{0.676760in}}%
\pgfpathlineto{\pgfqpoint{5.113796in}{0.677361in}}%
\pgfpathlineto{\pgfqpoint{5.108462in}{0.677964in}}%
\pgfpathlineto{\pgfqpoint{5.103129in}{0.678568in}}%
\pgfpathlineto{\pgfqpoint{5.097796in}{0.679173in}}%
\pgfpathlineto{\pgfqpoint{5.092463in}{0.679779in}}%
\pgfpathlineto{\pgfqpoint{5.087130in}{0.680387in}}%
\pgfpathlineto{\pgfqpoint{5.081796in}{0.680997in}}%
\pgfpathlineto{\pgfqpoint{5.076463in}{0.681607in}}%
\pgfpathlineto{\pgfqpoint{5.071130in}{0.682219in}}%
\pgfpathlineto{\pgfqpoint{5.065797in}{0.682833in}}%
\pgfpathlineto{\pgfqpoint{5.060463in}{0.683447in}}%
\pgfpathlineto{\pgfqpoint{5.055130in}{0.684064in}}%
\pgfpathlineto{\pgfqpoint{5.049797in}{0.684681in}}%
\pgfpathlineto{\pgfqpoint{5.044464in}{0.685300in}}%
\pgfpathlineto{\pgfqpoint{5.039131in}{0.685920in}}%
\pgfpathlineto{\pgfqpoint{5.033797in}{0.686542in}}%
\pgfpathlineto{\pgfqpoint{5.028464in}{0.687165in}}%
\pgfpathlineto{\pgfqpoint{5.023131in}{0.687790in}}%
\pgfpathlineto{\pgfqpoint{5.017798in}{0.688416in}}%
\pgfpathlineto{\pgfqpoint{5.012464in}{0.689043in}}%
\pgfpathlineto{\pgfqpoint{5.007131in}{0.689672in}}%
\pgfpathlineto{\pgfqpoint{5.001798in}{0.690303in}}%
\pgfpathlineto{\pgfqpoint{4.996465in}{0.690934in}}%
\pgfpathlineto{\pgfqpoint{4.991132in}{0.691567in}}%
\pgfpathlineto{\pgfqpoint{4.985798in}{0.692202in}}%
\pgfpathlineto{\pgfqpoint{4.980465in}{0.692838in}}%
\pgfpathlineto{\pgfqpoint{4.975132in}{0.693476in}}%
\pgfpathlineto{\pgfqpoint{4.969799in}{0.694115in}}%
\pgfpathlineto{\pgfqpoint{4.964466in}{0.694755in}}%
\pgfpathlineto{\pgfqpoint{4.959132in}{0.695397in}}%
\pgfpathlineto{\pgfqpoint{4.953799in}{0.696040in}}%
\pgfpathlineto{\pgfqpoint{4.948466in}{0.696685in}}%
\pgfpathlineto{\pgfqpoint{4.943133in}{0.697332in}}%
\pgfpathlineto{\pgfqpoint{4.937799in}{0.697980in}}%
\pgfpathlineto{\pgfqpoint{4.932466in}{0.698629in}}%
\pgfpathlineto{\pgfqpoint{4.927133in}{0.699280in}}%
\pgfpathlineto{\pgfqpoint{4.921800in}{0.699932in}}%
\pgfpathlineto{\pgfqpoint{4.916467in}{0.700586in}}%
\pgfpathlineto{\pgfqpoint{4.911133in}{0.701242in}}%
\pgfpathlineto{\pgfqpoint{4.905800in}{0.701899in}}%
\pgfpathlineto{\pgfqpoint{4.900467in}{0.702557in}}%
\pgfpathlineto{\pgfqpoint{4.895134in}{0.703217in}}%
\pgfpathlineto{\pgfqpoint{4.889801in}{0.703879in}}%
\pgfpathlineto{\pgfqpoint{4.884467in}{0.704542in}}%
\pgfpathlineto{\pgfqpoint{4.879134in}{0.705206in}}%
\pgfpathlineto{\pgfqpoint{4.873801in}{0.705873in}}%
\pgfpathlineto{\pgfqpoint{4.868468in}{0.706540in}}%
\pgfpathlineto{\pgfqpoint{4.863134in}{0.707210in}}%
\pgfpathlineto{\pgfqpoint{4.857801in}{0.707881in}}%
\pgfpathlineto{\pgfqpoint{4.852468in}{0.708553in}}%
\pgfpathlineto{\pgfqpoint{4.847135in}{0.709227in}}%
\pgfpathlineto{\pgfqpoint{4.841802in}{0.709903in}}%
\pgfpathlineto{\pgfqpoint{4.836468in}{0.710580in}}%
\pgfpathlineto{\pgfqpoint{4.831135in}{0.711259in}}%
\pgfpathlineto{\pgfqpoint{4.825802in}{0.711940in}}%
\pgfpathlineto{\pgfqpoint{4.820469in}{0.712622in}}%
\pgfpathlineto{\pgfqpoint{4.815136in}{0.713306in}}%
\pgfpathlineto{\pgfqpoint{4.809802in}{0.713991in}}%
\pgfpathlineto{\pgfqpoint{4.804469in}{0.714678in}}%
\pgfpathlineto{\pgfqpoint{4.799136in}{0.715367in}}%
\pgfpathlineto{\pgfqpoint{4.793803in}{0.716057in}}%
\pgfpathlineto{\pgfqpoint{4.788469in}{0.716749in}}%
\pgfpathlineto{\pgfqpoint{4.783136in}{0.717442in}}%
\pgfpathlineto{\pgfqpoint{4.777803in}{0.718138in}}%
\pgfpathlineto{\pgfqpoint{4.772470in}{0.718834in}}%
\pgfpathlineto{\pgfqpoint{4.767137in}{0.719533in}}%
\pgfpathlineto{\pgfqpoint{4.761803in}{0.720233in}}%
\pgfpathlineto{\pgfqpoint{4.756470in}{0.720935in}}%
\pgfpathlineto{\pgfqpoint{4.751137in}{0.721639in}}%
\pgfpathlineto{\pgfqpoint{4.745804in}{0.722344in}}%
\pgfpathlineto{\pgfqpoint{4.740470in}{0.723051in}}%
\pgfpathlineto{\pgfqpoint{4.735137in}{0.723760in}}%
\pgfpathlineto{\pgfqpoint{4.729804in}{0.724470in}}%
\pgfpathlineto{\pgfqpoint{4.724471in}{0.725183in}}%
\pgfpathlineto{\pgfqpoint{4.719138in}{0.725897in}}%
\pgfpathlineto{\pgfqpoint{4.713804in}{0.726612in}}%
\pgfpathlineto{\pgfqpoint{4.708471in}{0.727330in}}%
\pgfpathlineto{\pgfqpoint{4.703138in}{0.728049in}}%
\pgfpathlineto{\pgfqpoint{4.697805in}{0.728770in}}%
\pgfpathlineto{\pgfqpoint{4.692472in}{0.729492in}}%
\pgfpathlineto{\pgfqpoint{4.687138in}{0.730217in}}%
\pgfpathlineto{\pgfqpoint{4.681805in}{0.730943in}}%
\pgfpathlineto{\pgfqpoint{4.676472in}{0.731671in}}%
\pgfpathlineto{\pgfqpoint{4.671139in}{0.732401in}}%
\pgfpathlineto{\pgfqpoint{4.665805in}{0.733132in}}%
\pgfpathlineto{\pgfqpoint{4.660472in}{0.733865in}}%
\pgfpathlineto{\pgfqpoint{4.655139in}{0.734601in}}%
\pgfpathlineto{\pgfqpoint{4.649806in}{0.735338in}}%
\pgfpathlineto{\pgfqpoint{4.644473in}{0.736076in}}%
\pgfpathlineto{\pgfqpoint{4.639139in}{0.736817in}}%
\pgfpathlineto{\pgfqpoint{4.633806in}{0.737559in}}%
\pgfpathlineto{\pgfqpoint{4.628473in}{0.738304in}}%
\pgfpathlineto{\pgfqpoint{4.623140in}{0.739050in}}%
\pgfpathlineto{\pgfqpoint{4.617807in}{0.739798in}}%
\pgfpathlineto{\pgfqpoint{4.612473in}{0.740548in}}%
\pgfpathlineto{\pgfqpoint{4.607140in}{0.741299in}}%
\pgfpathlineto{\pgfqpoint{4.601807in}{0.742053in}}%
\pgfpathlineto{\pgfqpoint{4.596474in}{0.742808in}}%
\pgfpathlineto{\pgfqpoint{4.591140in}{0.743566in}}%
\pgfpathlineto{\pgfqpoint{4.585807in}{0.744325in}}%
\pgfpathlineto{\pgfqpoint{4.580474in}{0.745086in}}%
\pgfpathlineto{\pgfqpoint{4.575141in}{0.745849in}}%
\pgfpathlineto{\pgfqpoint{4.569808in}{0.746614in}}%
\pgfpathlineto{\pgfqpoint{4.564474in}{0.747381in}}%
\pgfpathlineto{\pgfqpoint{4.559141in}{0.748150in}}%
\pgfpathlineto{\pgfqpoint{4.553808in}{0.748921in}}%
\pgfpathlineto{\pgfqpoint{4.548475in}{0.749693in}}%
\pgfpathlineto{\pgfqpoint{4.543142in}{0.750468in}}%
\pgfpathlineto{\pgfqpoint{4.537808in}{0.751245in}}%
\pgfpathlineto{\pgfqpoint{4.532475in}{0.752023in}}%
\pgfpathlineto{\pgfqpoint{4.527142in}{0.752804in}}%
\pgfpathlineto{\pgfqpoint{4.521809in}{0.753586in}}%
\pgfpathlineto{\pgfqpoint{4.516475in}{0.754371in}}%
\pgfpathlineto{\pgfqpoint{4.511142in}{0.755157in}}%
\pgfpathlineto{\pgfqpoint{4.505809in}{0.755946in}}%
\pgfpathlineto{\pgfqpoint{4.500476in}{0.756737in}}%
\pgfpathlineto{\pgfqpoint{4.495143in}{0.757529in}}%
\pgfpathlineto{\pgfqpoint{4.489809in}{0.758324in}}%
\pgfpathlineto{\pgfqpoint{4.484476in}{0.759120in}}%
\pgfpathlineto{\pgfqpoint{4.479143in}{0.759919in}}%
\pgfpathlineto{\pgfqpoint{4.473810in}{0.760720in}}%
\pgfpathlineto{\pgfqpoint{4.468476in}{0.761523in}}%
\pgfpathlineto{\pgfqpoint{4.463143in}{0.762328in}}%
\pgfpathlineto{\pgfqpoint{4.457810in}{0.763135in}}%
\pgfpathlineto{\pgfqpoint{4.452477in}{0.763944in}}%
\pgfpathlineto{\pgfqpoint{4.447144in}{0.764755in}}%
\pgfpathlineto{\pgfqpoint{4.441810in}{0.765568in}}%
\pgfpathlineto{\pgfqpoint{4.436477in}{0.766383in}}%
\pgfpathlineto{\pgfqpoint{4.431144in}{0.767201in}}%
\pgfpathlineto{\pgfqpoint{4.425811in}{0.768020in}}%
\pgfpathlineto{\pgfqpoint{4.420478in}{0.768842in}}%
\pgfpathlineto{\pgfqpoint{4.415144in}{0.769666in}}%
\pgfpathlineto{\pgfqpoint{4.409811in}{0.770492in}}%
\pgfpathlineto{\pgfqpoint{4.404478in}{0.771320in}}%
\pgfpathlineto{\pgfqpoint{4.399145in}{0.772151in}}%
\pgfpathlineto{\pgfqpoint{4.393811in}{0.772983in}}%
\pgfpathlineto{\pgfqpoint{4.388478in}{0.773818in}}%
\pgfpathlineto{\pgfqpoint{4.383145in}{0.774655in}}%
\pgfpathlineto{\pgfqpoint{4.377812in}{0.775494in}}%
\pgfpathlineto{\pgfqpoint{4.372479in}{0.776335in}}%
\pgfpathlineto{\pgfqpoint{4.367145in}{0.777179in}}%
\pgfpathlineto{\pgfqpoint{4.361812in}{0.778024in}}%
\pgfpathlineto{\pgfqpoint{4.356479in}{0.778872in}}%
\pgfpathlineto{\pgfqpoint{4.351146in}{0.779723in}}%
\pgfpathlineto{\pgfqpoint{4.345813in}{0.780575in}}%
\pgfpathlineto{\pgfqpoint{4.340479in}{0.781430in}}%
\pgfpathlineto{\pgfqpoint{4.335146in}{0.782287in}}%
\pgfpathlineto{\pgfqpoint{4.329813in}{0.783146in}}%
\pgfpathlineto{\pgfqpoint{4.324480in}{0.784008in}}%
\pgfpathlineto{\pgfqpoint{4.319146in}{0.784872in}}%
\pgfpathlineto{\pgfqpoint{4.313813in}{0.785738in}}%
\pgfpathlineto{\pgfqpoint{4.308480in}{0.786606in}}%
\pgfpathlineto{\pgfqpoint{4.303147in}{0.787477in}}%
\pgfpathlineto{\pgfqpoint{4.297814in}{0.788351in}}%
\pgfpathlineto{\pgfqpoint{4.292480in}{0.789226in}}%
\pgfpathlineto{\pgfqpoint{4.287147in}{0.790104in}}%
\pgfpathlineto{\pgfqpoint{4.281814in}{0.790984in}}%
\pgfpathlineto{\pgfqpoint{4.276481in}{0.791867in}}%
\pgfpathlineto{\pgfqpoint{4.271148in}{0.792752in}}%
\pgfpathlineto{\pgfqpoint{4.265814in}{0.793639in}}%
\pgfpathlineto{\pgfqpoint{4.260481in}{0.794529in}}%
\pgfpathlineto{\pgfqpoint{4.255148in}{0.795421in}}%
\pgfpathlineto{\pgfqpoint{4.249815in}{0.796316in}}%
\pgfpathlineto{\pgfqpoint{4.244481in}{0.797213in}}%
\pgfpathlineto{\pgfqpoint{4.239148in}{0.798113in}}%
\pgfpathlineto{\pgfqpoint{4.233815in}{0.799015in}}%
\pgfpathlineto{\pgfqpoint{4.228482in}{0.799919in}}%
\pgfpathlineto{\pgfqpoint{4.223149in}{0.800826in}}%
\pgfpathlineto{\pgfqpoint{4.217815in}{0.801736in}}%
\pgfpathlineto{\pgfqpoint{4.212482in}{0.802647in}}%
\pgfpathlineto{\pgfqpoint{4.207149in}{0.803562in}}%
\pgfpathlineto{\pgfqpoint{4.201816in}{0.804479in}}%
\pgfpathlineto{\pgfqpoint{4.196482in}{0.805398in}}%
\pgfpathlineto{\pgfqpoint{4.191149in}{0.806320in}}%
\pgfpathlineto{\pgfqpoint{4.185816in}{0.807245in}}%
\pgfpathlineto{\pgfqpoint{4.180483in}{0.808172in}}%
\pgfpathlineto{\pgfqpoint{4.175150in}{0.809102in}}%
\pgfpathlineto{\pgfqpoint{4.169816in}{0.810034in}}%
\pgfpathlineto{\pgfqpoint{4.164483in}{0.810969in}}%
\pgfpathlineto{\pgfqpoint{4.159150in}{0.811906in}}%
\pgfpathlineto{\pgfqpoint{4.153817in}{0.812846in}}%
\pgfpathlineto{\pgfqpoint{4.148484in}{0.813789in}}%
\pgfpathlineto{\pgfqpoint{4.143150in}{0.814734in}}%
\pgfpathlineto{\pgfqpoint{4.137817in}{0.815682in}}%
\pgfpathlineto{\pgfqpoint{4.132484in}{0.816633in}}%
\pgfpathlineto{\pgfqpoint{4.127151in}{0.817586in}}%
\pgfpathlineto{\pgfqpoint{4.121817in}{0.818542in}}%
\pgfpathlineto{\pgfqpoint{4.116484in}{0.819501in}}%
\pgfpathlineto{\pgfqpoint{4.111151in}{0.820462in}}%
\pgfpathlineto{\pgfqpoint{4.105818in}{0.821426in}}%
\pgfpathlineto{\pgfqpoint{4.100485in}{0.822393in}}%
\pgfpathlineto{\pgfqpoint{4.095151in}{0.823362in}}%
\pgfpathlineto{\pgfqpoint{4.089818in}{0.824334in}}%
\pgfpathlineto{\pgfqpoint{4.084485in}{0.825309in}}%
\pgfpathlineto{\pgfqpoint{4.079152in}{0.826287in}}%
\pgfpathlineto{\pgfqpoint{4.073819in}{0.827268in}}%
\pgfpathlineto{\pgfqpoint{4.068485in}{0.828251in}}%
\pgfpathlineto{\pgfqpoint{4.063152in}{0.829237in}}%
\pgfpathlineto{\pgfqpoint{4.057819in}{0.830226in}}%
\pgfpathlineto{\pgfqpoint{4.052486in}{0.831218in}}%
\pgfpathlineto{\pgfqpoint{4.047152in}{0.832213in}}%
\pgfpathlineto{\pgfqpoint{4.041819in}{0.833210in}}%
\pgfpathlineto{\pgfqpoint{4.036486in}{0.834211in}}%
\pgfpathlineto{\pgfqpoint{4.031153in}{0.835214in}}%
\pgfpathlineto{\pgfqpoint{4.025820in}{0.836220in}}%
\pgfpathlineto{\pgfqpoint{4.020486in}{0.837229in}}%
\pgfpathlineto{\pgfqpoint{4.015153in}{0.838241in}}%
\pgfpathlineto{\pgfqpoint{4.009820in}{0.839256in}}%
\pgfpathlineto{\pgfqpoint{4.004487in}{0.840274in}}%
\pgfpathlineto{\pgfqpoint{3.999154in}{0.841294in}}%
\pgfpathlineto{\pgfqpoint{3.993820in}{0.842318in}}%
\pgfpathlineto{\pgfqpoint{3.988487in}{0.843345in}}%
\pgfpathlineto{\pgfqpoint{3.983154in}{0.844375in}}%
\pgfpathlineto{\pgfqpoint{3.977821in}{0.845407in}}%
\pgfpathlineto{\pgfqpoint{3.972487in}{0.846443in}}%
\pgfpathlineto{\pgfqpoint{3.967154in}{0.847482in}}%
\pgfpathlineto{\pgfqpoint{3.961821in}{0.848524in}}%
\pgfpathlineto{\pgfqpoint{3.956488in}{0.849568in}}%
\pgfpathlineto{\pgfqpoint{3.951155in}{0.850616in}}%
\pgfpathlineto{\pgfqpoint{3.945821in}{0.851667in}}%
\pgfpathlineto{\pgfqpoint{3.940488in}{0.852722in}}%
\pgfpathlineto{\pgfqpoint{3.935155in}{0.853779in}}%
\pgfpathlineto{\pgfqpoint{3.929822in}{0.854839in}}%
\pgfpathlineto{\pgfqpoint{3.924488in}{0.855903in}}%
\pgfpathlineto{\pgfqpoint{3.919155in}{0.856969in}}%
\pgfpathlineto{\pgfqpoint{3.913822in}{0.858039in}}%
\pgfpathlineto{\pgfqpoint{3.908489in}{0.859112in}}%
\pgfpathlineto{\pgfqpoint{3.903156in}{0.860188in}}%
\pgfpathlineto{\pgfqpoint{3.897822in}{0.861268in}}%
\pgfpathlineto{\pgfqpoint{3.892489in}{0.862351in}}%
\pgfpathlineto{\pgfqpoint{3.887156in}{0.863436in}}%
\pgfpathlineto{\pgfqpoint{3.881823in}{0.864526in}}%
\pgfpathlineto{\pgfqpoint{3.876490in}{0.865618in}}%
\pgfpathlineto{\pgfqpoint{3.871156in}{0.866714in}}%
\pgfpathlineto{\pgfqpoint{3.865823in}{0.867813in}}%
\pgfpathlineto{\pgfqpoint{3.860490in}{0.868915in}}%
\pgfpathlineto{\pgfqpoint{3.855157in}{0.870021in}}%
\pgfpathlineto{\pgfqpoint{3.849823in}{0.871130in}}%
\pgfpathlineto{\pgfqpoint{3.844490in}{0.872242in}}%
\pgfpathlineto{\pgfqpoint{3.839157in}{0.873358in}}%
\pgfpathlineto{\pgfqpoint{3.833824in}{0.874477in}}%
\pgfpathlineto{\pgfqpoint{3.828491in}{0.875599in}}%
\pgfpathlineto{\pgfqpoint{3.823157in}{0.876725in}}%
\pgfpathlineto{\pgfqpoint{3.817824in}{0.877855in}}%
\pgfpathlineto{\pgfqpoint{3.812491in}{0.878988in}}%
\pgfpathlineto{\pgfqpoint{3.807158in}{0.880124in}}%
\pgfpathlineto{\pgfqpoint{3.801825in}{0.881264in}}%
\pgfpathlineto{\pgfqpoint{3.796491in}{0.882407in}}%
\pgfpathlineto{\pgfqpoint{3.791158in}{0.883554in}}%
\pgfpathlineto{\pgfqpoint{3.785825in}{0.884704in}}%
\pgfpathlineto{\pgfqpoint{3.780492in}{0.885858in}}%
\pgfpathlineto{\pgfqpoint{3.775158in}{0.887016in}}%
\pgfpathlineto{\pgfqpoint{3.769825in}{0.888177in}}%
\pgfpathlineto{\pgfqpoint{3.764492in}{0.889341in}}%
\pgfpathlineto{\pgfqpoint{3.759159in}{0.890509in}}%
\pgfpathlineto{\pgfqpoint{3.753826in}{0.891681in}}%
\pgfpathlineto{\pgfqpoint{3.748492in}{0.892857in}}%
\pgfpathlineto{\pgfqpoint{3.743159in}{0.894036in}}%
\pgfpathlineto{\pgfqpoint{3.737826in}{0.895219in}}%
\pgfpathlineto{\pgfqpoint{3.732493in}{0.896406in}}%
\pgfpathlineto{\pgfqpoint{3.727160in}{0.897596in}}%
\pgfpathlineto{\pgfqpoint{3.721826in}{0.898790in}}%
\pgfpathlineto{\pgfqpoint{3.716493in}{0.899988in}}%
\pgfpathlineto{\pgfqpoint{3.711160in}{0.901189in}}%
\pgfpathlineto{\pgfqpoint{3.705827in}{0.902395in}}%
\pgfpathlineto{\pgfqpoint{3.700493in}{0.903604in}}%
\pgfpathlineto{\pgfqpoint{3.695160in}{0.904817in}}%
\pgfpathlineto{\pgfqpoint{3.689827in}{0.906034in}}%
\pgfpathlineto{\pgfqpoint{3.684494in}{0.907254in}}%
\pgfpathlineto{\pgfqpoint{3.679161in}{0.908479in}}%
\pgfpathlineto{\pgfqpoint{3.673827in}{0.909708in}}%
\pgfpathlineto{\pgfqpoint{3.668494in}{0.910940in}}%
\pgfpathlineto{\pgfqpoint{3.663161in}{0.912176in}}%
\pgfpathlineto{\pgfqpoint{3.657828in}{0.913417in}}%
\pgfpathlineto{\pgfqpoint{3.652494in}{0.914661in}}%
\pgfpathlineto{\pgfqpoint{3.647161in}{0.915909in}}%
\pgfpathlineto{\pgfqpoint{3.641828in}{0.917161in}}%
\pgfpathlineto{\pgfqpoint{3.636495in}{0.918418in}}%
\pgfpathlineto{\pgfqpoint{3.631162in}{0.919678in}}%
\pgfpathlineto{\pgfqpoint{3.625828in}{0.920943in}}%
\pgfpathlineto{\pgfqpoint{3.620495in}{0.922211in}}%
\pgfpathlineto{\pgfqpoint{3.615162in}{0.923484in}}%
\pgfpathlineto{\pgfqpoint{3.609829in}{0.924760in}}%
\pgfpathlineto{\pgfqpoint{3.604496in}{0.926041in}}%
\pgfpathlineto{\pgfqpoint{3.599162in}{0.927326in}}%
\pgfpathlineto{\pgfqpoint{3.593829in}{0.928616in}}%
\pgfpathlineto{\pgfqpoint{3.588496in}{0.929909in}}%
\pgfpathlineto{\pgfqpoint{3.583163in}{0.931207in}}%
\pgfpathlineto{\pgfqpoint{3.577829in}{0.932509in}}%
\pgfpathlineto{\pgfqpoint{3.572496in}{0.933815in}}%
\pgfpathlineto{\pgfqpoint{3.567163in}{0.935126in}}%
\pgfpathlineto{\pgfqpoint{3.561830in}{0.936441in}}%
\pgfpathlineto{\pgfqpoint{3.556497in}{0.937760in}}%
\pgfpathlineto{\pgfqpoint{3.551163in}{0.939083in}}%
\pgfpathlineto{\pgfqpoint{3.545830in}{0.940411in}}%
\pgfpathlineto{\pgfqpoint{3.540497in}{0.941744in}}%
\pgfpathlineto{\pgfqpoint{3.535164in}{0.943081in}}%
\pgfpathlineto{\pgfqpoint{3.529831in}{0.944422in}}%
\pgfpathlineto{\pgfqpoint{3.524497in}{0.945767in}}%
\pgfpathlineto{\pgfqpoint{3.519164in}{0.947118in}}%
\pgfpathlineto{\pgfqpoint{3.513831in}{0.948472in}}%
\pgfpathlineto{\pgfqpoint{3.508498in}{0.949832in}}%
\pgfpathlineto{\pgfqpoint{3.503164in}{0.951195in}}%
\pgfpathlineto{\pgfqpoint{3.497831in}{0.952564in}}%
\pgfpathlineto{\pgfqpoint{3.492498in}{0.953937in}}%
\pgfpathlineto{\pgfqpoint{3.487165in}{0.955314in}}%
\pgfpathlineto{\pgfqpoint{3.481832in}{0.956697in}}%
\pgfpathlineto{\pgfqpoint{3.476498in}{0.958084in}}%
\pgfpathlineto{\pgfqpoint{3.471165in}{0.959475in}}%
\pgfpathlineto{\pgfqpoint{3.465832in}{0.960872in}}%
\pgfpathlineto{\pgfqpoint{3.460499in}{0.962273in}}%
\pgfpathlineto{\pgfqpoint{3.455166in}{0.963679in}}%
\pgfpathlineto{\pgfqpoint{3.449832in}{0.965089in}}%
\pgfpathlineto{\pgfqpoint{3.444499in}{0.966505in}}%
\pgfpathlineto{\pgfqpoint{3.439166in}{0.967925in}}%
\pgfpathlineto{\pgfqpoint{3.433833in}{0.969350in}}%
\pgfpathlineto{\pgfqpoint{3.428499in}{0.970781in}}%
\pgfpathlineto{\pgfqpoint{3.423166in}{0.972216in}}%
\pgfpathlineto{\pgfqpoint{3.417833in}{0.973656in}}%
\pgfpathlineto{\pgfqpoint{3.412500in}{0.975101in}}%
\pgfpathlineto{\pgfqpoint{3.407167in}{0.976551in}}%
\pgfpathlineto{\pgfqpoint{3.401833in}{0.978006in}}%
\pgfpathlineto{\pgfqpoint{3.396500in}{0.979466in}}%
\pgfpathlineto{\pgfqpoint{3.391167in}{0.980931in}}%
\pgfpathlineto{\pgfqpoint{3.385834in}{0.982401in}}%
\pgfpathlineto{\pgfqpoint{3.380500in}{0.983876in}}%
\pgfpathlineto{\pgfqpoint{3.375167in}{0.985357in}}%
\pgfpathlineto{\pgfqpoint{3.369834in}{0.986843in}}%
\pgfpathlineto{\pgfqpoint{3.364501in}{0.988333in}}%
\pgfpathlineto{\pgfqpoint{3.359168in}{0.989830in}}%
\pgfpathlineto{\pgfqpoint{3.353834in}{0.991331in}}%
\pgfpathlineto{\pgfqpoint{3.348501in}{0.992838in}}%
\pgfpathlineto{\pgfqpoint{3.343168in}{0.994350in}}%
\pgfpathlineto{\pgfqpoint{3.337835in}{0.995867in}}%
\pgfpathlineto{\pgfqpoint{3.332502in}{0.997390in}}%
\pgfpathlineto{\pgfqpoint{3.327168in}{0.998918in}}%
\pgfpathlineto{\pgfqpoint{3.321835in}{1.000452in}}%
\pgfpathlineto{\pgfqpoint{3.316502in}{1.001991in}}%
\pgfpathlineto{\pgfqpoint{3.311169in}{1.003535in}}%
\pgfpathlineto{\pgfqpoint{3.305835in}{1.005085in}}%
\pgfpathlineto{\pgfqpoint{3.300502in}{1.006641in}}%
\pgfpathlineto{\pgfqpoint{3.295169in}{1.008202in}}%
\pgfpathlineto{\pgfqpoint{3.289836in}{1.009769in}}%
\pgfpathlineto{\pgfqpoint{3.284503in}{1.011342in}}%
\pgfpathlineto{\pgfqpoint{3.279169in}{1.012920in}}%
\pgfpathlineto{\pgfqpoint{3.273836in}{1.014504in}}%
\pgfpathlineto{\pgfqpoint{3.268503in}{1.016093in}}%
\pgfpathlineto{\pgfqpoint{3.263170in}{1.017689in}}%
\pgfpathlineto{\pgfqpoint{3.257837in}{1.019290in}}%
\pgfpathlineto{\pgfqpoint{3.252503in}{1.020897in}}%
\pgfpathlineto{\pgfqpoint{3.247170in}{1.022510in}}%
\pgfpathlineto{\pgfqpoint{3.241837in}{1.024129in}}%
\pgfpathlineto{\pgfqpoint{3.236504in}{1.025753in}}%
\pgfpathlineto{\pgfqpoint{3.231170in}{1.027384in}}%
\pgfpathlineto{\pgfqpoint{3.225837in}{1.029021in}}%
\pgfpathlineto{\pgfqpoint{3.220504in}{1.030663in}}%
\pgfpathlineto{\pgfqpoint{3.215171in}{1.032312in}}%
\pgfpathlineto{\pgfqpoint{3.209838in}{1.033967in}}%
\pgfpathlineto{\pgfqpoint{3.204504in}{1.035628in}}%
\pgfpathlineto{\pgfqpoint{3.199171in}{1.037295in}}%
\pgfpathlineto{\pgfqpoint{3.193838in}{1.038968in}}%
\pgfpathlineto{\pgfqpoint{3.188505in}{1.040648in}}%
\pgfpathlineto{\pgfqpoint{3.183172in}{1.042334in}}%
\pgfpathlineto{\pgfqpoint{3.177838in}{1.044026in}}%
\pgfpathlineto{\pgfqpoint{3.172505in}{1.045724in}}%
\pgfpathlineto{\pgfqpoint{3.167172in}{1.047429in}}%
\pgfpathlineto{\pgfqpoint{3.161839in}{1.049140in}}%
\pgfpathlineto{\pgfqpoint{3.156505in}{1.050858in}}%
\pgfpathlineto{\pgfqpoint{3.151172in}{1.052582in}}%
\pgfpathlineto{\pgfqpoint{3.145839in}{1.054313in}}%
\pgfpathlineto{\pgfqpoint{3.140506in}{1.056050in}}%
\pgfpathlineto{\pgfqpoint{3.135173in}{1.057794in}}%
\pgfpathlineto{\pgfqpoint{3.129839in}{1.059544in}}%
\pgfpathlineto{\pgfqpoint{3.124506in}{1.061301in}}%
\pgfpathlineto{\pgfqpoint{3.119173in}{1.063065in}}%
\pgfpathlineto{\pgfqpoint{3.113840in}{1.064836in}}%
\pgfpathlineto{\pgfqpoint{3.108506in}{1.066613in}}%
\pgfpathlineto{\pgfqpoint{3.103173in}{1.068397in}}%
\pgfpathlineto{\pgfqpoint{3.097840in}{1.070188in}}%
\pgfpathlineto{\pgfqpoint{3.092507in}{1.071986in}}%
\pgfpathlineto{\pgfqpoint{3.087174in}{1.073791in}}%
\pgfpathlineto{\pgfqpoint{3.081840in}{1.075603in}}%
\pgfpathlineto{\pgfqpoint{3.076507in}{1.077422in}}%
\pgfpathlineto{\pgfqpoint{3.071174in}{1.079248in}}%
\pgfpathlineto{\pgfqpoint{3.065841in}{1.081081in}}%
\pgfpathlineto{\pgfqpoint{3.060508in}{1.082921in}}%
\pgfpathlineto{\pgfqpoint{3.055174in}{1.084768in}}%
\pgfpathlineto{\pgfqpoint{3.049841in}{1.086623in}}%
\pgfpathlineto{\pgfqpoint{3.044508in}{1.088485in}}%
\pgfpathlineto{\pgfqpoint{3.039175in}{1.090354in}}%
\pgfpathlineto{\pgfqpoint{3.033841in}{1.092231in}}%
\pgfpathlineto{\pgfqpoint{3.028508in}{1.094115in}}%
\pgfpathlineto{\pgfqpoint{3.023175in}{1.096006in}}%
\pgfpathlineto{\pgfqpoint{3.017842in}{1.097905in}}%
\pgfpathlineto{\pgfqpoint{3.012509in}{1.099811in}}%
\pgfpathlineto{\pgfqpoint{3.007175in}{1.101725in}}%
\pgfpathlineto{\pgfqpoint{3.001842in}{1.103647in}}%
\pgfpathlineto{\pgfqpoint{2.996509in}{1.105576in}}%
\pgfpathlineto{\pgfqpoint{2.991176in}{1.107513in}}%
\pgfpathlineto{\pgfqpoint{2.985843in}{1.109458in}}%
\pgfpathlineto{\pgfqpoint{2.980509in}{1.111410in}}%
\pgfpathlineto{\pgfqpoint{2.975176in}{1.113371in}}%
\pgfpathlineto{\pgfqpoint{2.969843in}{1.115339in}}%
\pgfpathlineto{\pgfqpoint{2.964510in}{1.117315in}}%
\pgfpathlineto{\pgfqpoint{2.959176in}{1.119300in}}%
\pgfpathlineto{\pgfqpoint{2.953843in}{1.121292in}}%
\pgfpathlineto{\pgfqpoint{2.948510in}{1.123292in}}%
\pgfpathlineto{\pgfqpoint{2.943177in}{1.125301in}}%
\pgfpathlineto{\pgfqpoint{2.937844in}{1.127318in}}%
\pgfpathlineto{\pgfqpoint{2.932510in}{1.129343in}}%
\pgfpathlineto{\pgfqpoint{2.927177in}{1.131376in}}%
\pgfpathlineto{\pgfqpoint{2.921844in}{1.133418in}}%
\pgfpathlineto{\pgfqpoint{2.916511in}{1.135468in}}%
\pgfpathlineto{\pgfqpoint{2.911178in}{1.137526in}}%
\pgfpathlineto{\pgfqpoint{2.905844in}{1.139593in}}%
\pgfpathlineto{\pgfqpoint{2.900511in}{1.141669in}}%
\pgfpathlineto{\pgfqpoint{2.895178in}{1.143753in}}%
\pgfpathlineto{\pgfqpoint{2.889845in}{1.145846in}}%
\pgfpathlineto{\pgfqpoint{2.884511in}{1.147947in}}%
\pgfpathlineto{\pgfqpoint{2.879178in}{1.150058in}}%
\pgfpathlineto{\pgfqpoint{2.873845in}{1.152177in}}%
\pgfpathlineto{\pgfqpoint{2.868512in}{1.154305in}}%
\pgfpathlineto{\pgfqpoint{2.863179in}{1.156442in}}%
\pgfpathlineto{\pgfqpoint{2.857845in}{1.158588in}}%
\pgfpathlineto{\pgfqpoint{2.852512in}{1.160743in}}%
\pgfpathlineto{\pgfqpoint{2.847179in}{1.162907in}}%
\pgfpathlineto{\pgfqpoint{2.841846in}{1.165080in}}%
\pgfpathlineto{\pgfqpoint{2.836512in}{1.167263in}}%
\pgfpathlineto{\pgfqpoint{2.831179in}{1.169454in}}%
\pgfpathlineto{\pgfqpoint{2.825846in}{1.171656in}}%
\pgfpathlineto{\pgfqpoint{2.820513in}{1.173866in}}%
\pgfpathlineto{\pgfqpoint{2.815180in}{1.176086in}}%
\pgfpathlineto{\pgfqpoint{2.809846in}{1.178316in}}%
\pgfpathlineto{\pgfqpoint{2.804513in}{1.180555in}}%
\pgfpathlineto{\pgfqpoint{2.799180in}{1.182804in}}%
\pgfpathlineto{\pgfqpoint{2.793847in}{1.185062in}}%
\pgfpathlineto{\pgfqpoint{2.788514in}{1.187330in}}%
\pgfpathlineto{\pgfqpoint{2.783180in}{1.189608in}}%
\pgfpathlineto{\pgfqpoint{2.777847in}{1.191896in}}%
\pgfpathlineto{\pgfqpoint{2.772514in}{1.194194in}}%
\pgfpathlineto{\pgfqpoint{2.767181in}{1.196502in}}%
\pgfpathlineto{\pgfqpoint{2.761847in}{1.198821in}}%
\pgfpathlineto{\pgfqpoint{2.756514in}{1.201149in}}%
\pgfpathlineto{\pgfqpoint{2.751181in}{1.203487in}}%
\pgfpathlineto{\pgfqpoint{2.745848in}{1.205836in}}%
\pgfpathlineto{\pgfqpoint{2.740515in}{1.208195in}}%
\pgfpathlineto{\pgfqpoint{2.735181in}{1.210565in}}%
\pgfpathlineto{\pgfqpoint{2.729848in}{1.212945in}}%
\pgfpathlineto{\pgfqpoint{2.724515in}{1.215336in}}%
\pgfpathlineto{\pgfqpoint{2.719182in}{1.217737in}}%
\pgfpathlineto{\pgfqpoint{2.713849in}{1.220149in}}%
\pgfpathlineto{\pgfqpoint{2.708515in}{1.222572in}}%
\pgfpathlineto{\pgfqpoint{2.703182in}{1.225005in}}%
\pgfpathlineto{\pgfqpoint{2.697849in}{1.227450in}}%
\pgfpathlineto{\pgfqpoint{2.692516in}{1.229906in}}%
\pgfpathlineto{\pgfqpoint{2.687182in}{1.232372in}}%
\pgfpathlineto{\pgfqpoint{2.681849in}{1.234850in}}%
\pgfpathlineto{\pgfqpoint{2.676516in}{1.237339in}}%
\pgfpathlineto{\pgfqpoint{2.671183in}{1.239840in}}%
\pgfpathlineto{\pgfqpoint{2.665850in}{1.242352in}}%
\pgfpathlineto{\pgfqpoint{2.660516in}{1.244875in}}%
\pgfpathlineto{\pgfqpoint{2.655183in}{1.247410in}}%
\pgfpathlineto{\pgfqpoint{2.649850in}{1.249956in}}%
\pgfpathlineto{\pgfqpoint{2.644517in}{1.252514in}}%
\pgfpathlineto{\pgfqpoint{2.639184in}{1.255084in}}%
\pgfpathlineto{\pgfqpoint{2.633850in}{1.257666in}}%
\pgfpathlineto{\pgfqpoint{2.628517in}{1.260260in}}%
\pgfpathlineto{\pgfqpoint{2.623184in}{1.262866in}}%
\pgfpathlineto{\pgfqpoint{2.617851in}{1.265484in}}%
\pgfpathlineto{\pgfqpoint{2.612517in}{1.268114in}}%
\pgfpathlineto{\pgfqpoint{2.607184in}{1.270756in}}%
\pgfpathlineto{\pgfqpoint{2.601851in}{1.273411in}}%
\pgfpathlineto{\pgfqpoint{2.596518in}{1.276078in}}%
\pgfpathlineto{\pgfqpoint{2.591185in}{1.278758in}}%
\pgfpathlineto{\pgfqpoint{2.585851in}{1.281450in}}%
\pgfpathlineto{\pgfqpoint{2.580518in}{1.284155in}}%
\pgfpathlineto{\pgfqpoint{2.575185in}{1.286873in}}%
\pgfpathlineto{\pgfqpoint{2.569852in}{1.289604in}}%
\pgfpathlineto{\pgfqpoint{2.564518in}{1.292348in}}%
\pgfpathlineto{\pgfqpoint{2.559185in}{1.295105in}}%
\pgfpathlineto{\pgfqpoint{2.553852in}{1.297875in}}%
\pgfpathlineto{\pgfqpoint{2.548519in}{1.300658in}}%
\pgfpathlineto{\pgfqpoint{2.543186in}{1.303455in}}%
\pgfpathlineto{\pgfqpoint{2.537852in}{1.306265in}}%
\pgfpathlineto{\pgfqpoint{2.532519in}{1.309089in}}%
\pgfpathlineto{\pgfqpoint{2.527186in}{1.311927in}}%
\pgfpathlineto{\pgfqpoint{2.521853in}{1.314778in}}%
\pgfpathlineto{\pgfqpoint{2.516520in}{1.317643in}}%
\pgfpathlineto{\pgfqpoint{2.511186in}{1.320522in}}%
\pgfpathlineto{\pgfqpoint{2.505853in}{1.323416in}}%
\pgfpathlineto{\pgfqpoint{2.500520in}{1.326323in}}%
\pgfpathlineto{\pgfqpoint{2.495187in}{1.329245in}}%
\pgfpathlineto{\pgfqpoint{2.489853in}{1.332181in}}%
\pgfpathlineto{\pgfqpoint{2.484520in}{1.335131in}}%
\pgfpathlineto{\pgfqpoint{2.479187in}{1.338096in}}%
\pgfpathlineto{\pgfqpoint{2.473854in}{1.341076in}}%
\pgfpathlineto{\pgfqpoint{2.468521in}{1.344071in}}%
\pgfpathlineto{\pgfqpoint{2.463187in}{1.347080in}}%
\pgfpathlineto{\pgfqpoint{2.457854in}{1.350105in}}%
\pgfpathlineto{\pgfqpoint{2.452521in}{1.353145in}}%
\pgfpathlineto{\pgfqpoint{2.447188in}{1.356200in}}%
\pgfpathlineto{\pgfqpoint{2.441855in}{1.359270in}}%
\pgfpathlineto{\pgfqpoint{2.436521in}{1.362356in}}%
\pgfpathlineto{\pgfqpoint{2.431188in}{1.365458in}}%
\pgfpathlineto{\pgfqpoint{2.425855in}{1.368575in}}%
\pgfpathlineto{\pgfqpoint{2.420522in}{1.371708in}}%
\pgfpathlineto{\pgfqpoint{2.415188in}{1.374858in}}%
\pgfpathlineto{\pgfqpoint{2.409855in}{1.378023in}}%
\pgfpathlineto{\pgfqpoint{2.404522in}{1.381204in}}%
\pgfpathlineto{\pgfqpoint{2.399189in}{1.384402in}}%
\pgfpathlineto{\pgfqpoint{2.393856in}{1.387616in}}%
\pgfpathlineto{\pgfqpoint{2.388522in}{1.390847in}}%
\pgfpathlineto{\pgfqpoint{2.383189in}{1.394095in}}%
\pgfpathlineto{\pgfqpoint{2.377856in}{1.397360in}}%
\pgfpathlineto{\pgfqpoint{2.372523in}{1.400641in}}%
\pgfpathlineto{\pgfqpoint{2.367190in}{1.403940in}}%
\pgfpathlineto{\pgfqpoint{2.361856in}{1.407256in}}%
\pgfpathlineto{\pgfqpoint{2.356523in}{1.410589in}}%
\pgfpathlineto{\pgfqpoint{2.351190in}{1.413940in}}%
\pgfpathlineto{\pgfqpoint{2.345857in}{1.417308in}}%
\pgfpathlineto{\pgfqpoint{2.340523in}{1.420695in}}%
\pgfpathlineto{\pgfqpoint{2.335190in}{1.424099in}}%
\pgfpathlineto{\pgfqpoint{2.329857in}{1.427521in}}%
\pgfpathlineto{\pgfqpoint{2.324524in}{1.430962in}}%
\pgfpathlineto{\pgfqpoint{2.319191in}{1.434421in}}%
\pgfpathlineto{\pgfqpoint{2.313857in}{1.437899in}}%
\pgfpathlineto{\pgfqpoint{2.308524in}{1.441395in}}%
\pgfpathlineto{\pgfqpoint{2.303191in}{1.444910in}}%
\pgfpathlineto{\pgfqpoint{2.297858in}{1.448444in}}%
\pgfpathlineto{\pgfqpoint{2.292524in}{1.451997in}}%
\pgfpathlineto{\pgfqpoint{2.287191in}{1.455570in}}%
\pgfpathlineto{\pgfqpoint{2.281858in}{1.459162in}}%
\pgfpathlineto{\pgfqpoint{2.276525in}{1.462774in}}%
\pgfpathlineto{\pgfqpoint{2.271192in}{1.466405in}}%
\pgfpathlineto{\pgfqpoint{2.265858in}{1.470056in}}%
\pgfpathlineto{\pgfqpoint{2.260525in}{1.473728in}}%
\pgfpathlineto{\pgfqpoint{2.255192in}{1.477420in}}%
\pgfpathlineto{\pgfqpoint{2.249859in}{1.481132in}}%
\pgfpathlineto{\pgfqpoint{2.244526in}{1.484865in}}%
\pgfpathlineto{\pgfqpoint{2.239192in}{1.488618in}}%
\pgfpathlineto{\pgfqpoint{2.233859in}{1.492393in}}%
\pgfpathlineto{\pgfqpoint{2.228526in}{1.496189in}}%
\pgfpathlineto{\pgfqpoint{2.223193in}{1.500006in}}%
\pgfpathlineto{\pgfqpoint{2.217859in}{1.503844in}}%
\pgfpathlineto{\pgfqpoint{2.212526in}{1.507704in}}%
\pgfpathlineto{\pgfqpoint{2.207193in}{1.511586in}}%
\pgfpathlineto{\pgfqpoint{2.201860in}{1.515491in}}%
\pgfpathlineto{\pgfqpoint{2.196527in}{1.519417in}}%
\pgfpathlineto{\pgfqpoint{2.191193in}{1.523366in}}%
\pgfpathlineto{\pgfqpoint{2.185860in}{1.527337in}}%
\pgfpathlineto{\pgfqpoint{2.180527in}{1.531331in}}%
\pgfpathlineto{\pgfqpoint{2.175194in}{1.535349in}}%
\pgfpathlineto{\pgfqpoint{2.169861in}{1.539389in}}%
\pgfpathlineto{\pgfqpoint{2.164527in}{1.543453in}}%
\pgfpathlineto{\pgfqpoint{2.159194in}{1.547540in}}%
\pgfpathlineto{\pgfqpoint{2.153861in}{1.551652in}}%
\pgfpathlineto{\pgfqpoint{2.148528in}{1.555787in}}%
\pgfpathlineto{\pgfqpoint{2.143194in}{1.559947in}}%
\pgfpathlineto{\pgfqpoint{2.137861in}{1.564131in}}%
\pgfpathlineto{\pgfqpoint{2.132528in}{1.568339in}}%
\pgfpathlineto{\pgfqpoint{2.127195in}{1.572573in}}%
\pgfpathlineto{\pgfqpoint{2.121862in}{1.576832in}}%
\pgfpathlineto{\pgfqpoint{2.116528in}{1.581116in}}%
\pgfpathlineto{\pgfqpoint{2.111195in}{1.585425in}}%
\pgfpathlineto{\pgfqpoint{2.105862in}{1.589761in}}%
\pgfpathlineto{\pgfqpoint{2.100529in}{1.594122in}}%
\pgfpathlineto{\pgfqpoint{2.095196in}{1.598510in}}%
\pgfpathlineto{\pgfqpoint{2.089862in}{1.602924in}}%
\pgfpathlineto{\pgfqpoint{2.084529in}{1.607365in}}%
\pgfpathlineto{\pgfqpoint{2.079196in}{1.611833in}}%
\pgfpathlineto{\pgfqpoint{2.073863in}{1.616328in}}%
\pgfpathlineto{\pgfqpoint{2.068529in}{1.620851in}}%
\pgfpathlineto{\pgfqpoint{2.063196in}{1.625401in}}%
\pgfpathlineto{\pgfqpoint{2.057863in}{1.629980in}}%
\pgfpathlineto{\pgfqpoint{2.052530in}{1.634586in}}%
\pgfpathlineto{\pgfqpoint{2.047197in}{1.639221in}}%
\pgfpathlineto{\pgfqpoint{2.041863in}{1.643885in}}%
\pgfpathlineto{\pgfqpoint{2.036530in}{1.648578in}}%
\pgfpathlineto{\pgfqpoint{2.031197in}{1.653301in}}%
\pgfpathlineto{\pgfqpoint{2.025864in}{1.658053in}}%
\pgfpathlineto{\pgfqpoint{2.020531in}{1.662834in}}%
\pgfpathlineto{\pgfqpoint{2.015197in}{1.667646in}}%
\pgfpathlineto{\pgfqpoint{2.009864in}{1.672489in}}%
\pgfpathlineto{\pgfqpoint{2.004531in}{1.677362in}}%
\pgfpathlineto{\pgfqpoint{1.999198in}{1.682266in}}%
\pgfpathlineto{\pgfqpoint{1.993864in}{1.687201in}}%
\pgfpathlineto{\pgfqpoint{1.988531in}{1.692168in}}%
\pgfpathlineto{\pgfqpoint{1.983198in}{1.697167in}}%
\pgfpathlineto{\pgfqpoint{1.977865in}{1.702199in}}%
\pgfpathlineto{\pgfqpoint{1.972532in}{1.707262in}}%
\pgfpathlineto{\pgfqpoint{1.967198in}{1.712359in}}%
\pgfpathlineto{\pgfqpoint{1.961865in}{1.717489in}}%
\pgfpathlineto{\pgfqpoint{1.956532in}{1.722652in}}%
\pgfpathlineto{\pgfqpoint{1.951199in}{1.727849in}}%
\pgfpathlineto{\pgfqpoint{1.945865in}{1.733081in}}%
\pgfpathlineto{\pgfqpoint{1.940532in}{1.738347in}}%
\pgfpathlineto{\pgfqpoint{1.935199in}{1.743647in}}%
\pgfpathlineto{\pgfqpoint{1.929866in}{1.748983in}}%
\pgfpathlineto{\pgfqpoint{1.924533in}{1.754355in}}%
\pgfpathlineto{\pgfqpoint{1.919199in}{1.759762in}}%
\pgfpathlineto{\pgfqpoint{1.913866in}{1.765206in}}%
\pgfpathlineto{\pgfqpoint{1.908533in}{1.770686in}}%
\pgfpathlineto{\pgfqpoint{1.903200in}{1.776204in}}%
\pgfpathlineto{\pgfqpoint{1.897867in}{1.781758in}}%
\pgfpathlineto{\pgfqpoint{1.892533in}{1.787351in}}%
\pgfpathlineto{\pgfqpoint{1.887200in}{1.792982in}}%
\pgfpathlineto{\pgfqpoint{1.881867in}{1.798651in}}%
\pgfpathlineto{\pgfqpoint{1.876534in}{1.804359in}}%
\pgfpathlineto{\pgfqpoint{1.871200in}{1.810107in}}%
\pgfpathlineto{\pgfqpoint{1.865867in}{1.815894in}}%
\pgfpathlineto{\pgfqpoint{1.860534in}{1.821721in}}%
\pgfpathlineto{\pgfqpoint{1.855201in}{1.827589in}}%
\pgfpathlineto{\pgfqpoint{1.849868in}{1.833498in}}%
\pgfpathlineto{\pgfqpoint{1.844534in}{1.839449in}}%
\pgfpathlineto{\pgfqpoint{1.839201in}{1.845441in}}%
\pgfpathlineto{\pgfqpoint{1.833868in}{1.851476in}}%
\pgfpathlineto{\pgfqpoint{1.828535in}{1.857554in}}%
\pgfpathlineto{\pgfqpoint{1.823202in}{1.863674in}}%
\pgfpathlineto{\pgfqpoint{1.817868in}{1.869839in}}%
\pgfpathlineto{\pgfqpoint{1.812535in}{1.876047in}}%
\pgfpathlineto{\pgfqpoint{1.807202in}{1.882301in}}%
\pgfpathlineto{\pgfqpoint{1.801869in}{1.888599in}}%
\pgfpathlineto{\pgfqpoint{1.796535in}{1.894943in}}%
\pgfpathlineto{\pgfqpoint{1.791202in}{1.901333in}}%
\pgfpathlineto{\pgfqpoint{1.785869in}{1.907770in}}%
\pgfpathlineto{\pgfqpoint{1.780536in}{1.914254in}}%
\pgfpathlineto{\pgfqpoint{1.775203in}{1.920785in}}%
\pgfpathlineto{\pgfqpoint{1.769869in}{1.927365in}}%
\pgfpathlineto{\pgfqpoint{1.764536in}{1.933993in}}%
\pgfpathlineto{\pgfqpoint{1.759203in}{1.940671in}}%
\pgfpathlineto{\pgfqpoint{1.753870in}{1.947398in}}%
\pgfpathlineto{\pgfqpoint{1.748537in}{1.954176in}}%
\pgfpathlineto{\pgfqpoint{1.743203in}{1.961005in}}%
\pgfpathlineto{\pgfqpoint{1.737870in}{1.967885in}}%
\pgfpathlineto{\pgfqpoint{1.732537in}{1.974818in}}%
\pgfpathlineto{\pgfqpoint{1.727204in}{1.981803in}}%
\pgfpathlineto{\pgfqpoint{1.721870in}{1.988841in}}%
\pgfpathlineto{\pgfqpoint{1.716537in}{1.995934in}}%
\pgfpathlineto{\pgfqpoint{1.711204in}{2.003080in}}%
\pgfpathlineto{\pgfqpoint{1.705871in}{2.010282in}}%
\pgfpathlineto{\pgfqpoint{1.700538in}{2.017540in}}%
\pgfpathlineto{\pgfqpoint{1.695204in}{2.024854in}}%
\pgfpathlineto{\pgfqpoint{1.689871in}{2.032226in}}%
\pgfpathlineto{\pgfqpoint{1.684538in}{2.039655in}}%
\pgfpathlineto{\pgfqpoint{1.679205in}{2.047142in}}%
\pgfpathlineto{\pgfqpoint{1.673871in}{2.054689in}}%
\pgfpathlineto{\pgfqpoint{1.668538in}{2.062296in}}%
\pgfpathlineto{\pgfqpoint{1.663205in}{2.069963in}}%
\pgfpathlineto{\pgfqpoint{1.657872in}{2.077692in}}%
\pgfpathlineto{\pgfqpoint{1.652539in}{2.085482in}}%
\pgfpathlineto{\pgfqpoint{1.647205in}{2.093335in}}%
\pgfpathlineto{\pgfqpoint{1.641872in}{2.101252in}}%
\pgfpathlineto{\pgfqpoint{1.636539in}{2.109234in}}%
\pgfpathlineto{\pgfqpoint{1.631206in}{2.117280in}}%
\pgfpathlineto{\pgfqpoint{1.625873in}{2.125392in}}%
\pgfpathlineto{\pgfqpoint{1.620539in}{2.133571in}}%
\pgfpathlineto{\pgfqpoint{1.615206in}{2.141818in}}%
\pgfpathlineto{\pgfqpoint{1.609873in}{2.150133in}}%
\pgfpathlineto{\pgfqpoint{1.604540in}{2.158517in}}%
\pgfpathlineto{\pgfqpoint{1.599206in}{2.166971in}}%
\pgfpathlineto{\pgfqpoint{1.593873in}{2.175497in}}%
\pgfpathlineto{\pgfqpoint{1.588540in}{2.184094in}}%
\pgfpathlineto{\pgfqpoint{1.583207in}{2.192764in}}%
\pgfpathlineto{\pgfqpoint{1.577874in}{2.201508in}}%
\pgfpathlineto{\pgfqpoint{1.572540in}{2.210326in}}%
\pgfpathlineto{\pgfqpoint{1.567207in}{2.219220in}}%
\pgfpathlineto{\pgfqpoint{1.561874in}{2.228191in}}%
\pgfpathlineto{\pgfqpoint{1.556541in}{2.237240in}}%
\pgfpathlineto{\pgfqpoint{1.551208in}{2.246367in}}%
\pgfpathlineto{\pgfqpoint{1.545874in}{2.255573in}}%
\pgfpathlineto{\pgfqpoint{1.540541in}{2.264861in}}%
\pgfpathlineto{\pgfqpoint{1.535208in}{2.274230in}}%
\pgfpathlineto{\pgfqpoint{1.529875in}{2.283682in}}%
\pgfpathlineto{\pgfqpoint{1.524541in}{2.293218in}}%
\pgfpathlineto{\pgfqpoint{1.519208in}{2.302839in}}%
\pgfpathlineto{\pgfqpoint{1.513875in}{2.312547in}}%
\pgfpathlineto{\pgfqpoint{1.508542in}{2.322342in}}%
\pgfpathlineto{\pgfqpoint{1.503209in}{2.332225in}}%
\pgfpathlineto{\pgfqpoint{1.497875in}{2.342198in}}%
\pgfpathlineto{\pgfqpoint{1.492542in}{2.352262in}}%
\pgfpathlineto{\pgfqpoint{1.487209in}{2.362419in}}%
\pgfpathlineto{\pgfqpoint{1.481876in}{2.372669in}}%
\pgfpathlineto{\pgfqpoint{1.476543in}{2.383014in}}%
\pgfpathlineto{\pgfqpoint{1.471209in}{2.393455in}}%
\pgfpathlineto{\pgfqpoint{1.465876in}{2.403993in}}%
\pgfpathlineto{\pgfqpoint{1.460543in}{2.414631in}}%
\pgfpathlineto{\pgfqpoint{1.455210in}{2.425368in}}%
\pgfpathlineto{\pgfqpoint{1.449876in}{2.436208in}}%
\pgfpathlineto{\pgfqpoint{1.444543in}{2.447150in}}%
\pgfpathlineto{\pgfqpoint{1.439210in}{2.458198in}}%
\pgfpathlineto{\pgfqpoint{1.433877in}{2.469351in}}%
\pgfpathlineto{\pgfqpoint{1.428544in}{2.480612in}}%
\pgfpathlineto{\pgfqpoint{1.423210in}{2.491983in}}%
\pgfpathlineto{\pgfqpoint{1.417877in}{2.503464in}}%
\pgfpathlineto{\pgfqpoint{1.412544in}{2.515058in}}%
\pgfpathlineto{\pgfqpoint{1.407211in}{2.526766in}}%
\pgfpathlineto{\pgfqpoint{1.401877in}{2.538589in}}%
\pgfpathlineto{\pgfqpoint{1.396544in}{2.550531in}}%
\pgfpathlineto{\pgfqpoint{1.391211in}{2.562591in}}%
\pgfpathlineto{\pgfqpoint{1.385878in}{2.574773in}}%
\pgfpathlineto{\pgfqpoint{1.380545in}{2.587078in}}%
\pgfpathlineto{\pgfqpoint{1.375211in}{2.599507in}}%
\pgfpathlineto{\pgfqpoint{1.369878in}{2.612064in}}%
\pgfpathlineto{\pgfqpoint{1.364545in}{2.624749in}}%
\pgfpathlineto{\pgfqpoint{1.359212in}{2.637565in}}%
\pgfpathlineto{\pgfqpoint{1.353879in}{2.650513in}}%
\pgfpathlineto{\pgfqpoint{1.348545in}{2.663596in}}%
\pgfpathlineto{\pgfqpoint{1.343212in}{2.676816in}}%
\pgfpathlineto{\pgfqpoint{1.337879in}{2.690176in}}%
\pgfpathlineto{\pgfqpoint{1.332546in}{2.703676in}}%
\pgfpathlineto{\pgfqpoint{1.327212in}{2.717320in}}%
\pgfpathlineto{\pgfqpoint{1.321879in}{2.731110in}}%
\pgfpathlineto{\pgfqpoint{1.316546in}{2.745048in}}%
\pgfpathlineto{\pgfqpoint{1.311213in}{2.759136in}}%
\pgfpathlineto{\pgfqpoint{1.305880in}{2.773378in}}%
\pgfpathlineto{\pgfqpoint{1.300546in}{2.787775in}}%
\pgfpathlineto{\pgfqpoint{1.295213in}{2.802330in}}%
\pgfpathlineto{\pgfqpoint{1.289880in}{2.817046in}}%
\pgfpathlineto{\pgfqpoint{1.284547in}{2.831926in}}%
\pgfpathlineto{\pgfqpoint{1.279214in}{2.846971in}}%
\pgfpathlineto{\pgfqpoint{1.273880in}{2.862185in}}%
\pgfpathlineto{\pgfqpoint{1.268547in}{2.877572in}}%
\pgfpathlineto{\pgfqpoint{1.263214in}{2.893132in}}%
\pgfpathlineto{\pgfqpoint{1.257881in}{2.908871in}}%
\pgfpathlineto{\pgfqpoint{1.252547in}{2.924790in}}%
\pgfpathlineto{\pgfqpoint{1.247214in}{2.940894in}}%
\pgfpathlineto{\pgfqpoint{1.241881in}{2.957184in}}%
\pgfpathlineto{\pgfqpoint{1.236548in}{2.973665in}}%
\pgfpathlineto{\pgfqpoint{1.231215in}{2.990339in}}%
\pgfpathlineto{\pgfqpoint{1.225881in}{3.007211in}}%
\pgfpathlineto{\pgfqpoint{1.220548in}{3.024283in}}%
\pgfpathlineto{\pgfqpoint{1.215215in}{3.041560in}}%
\pgfpathlineto{\pgfqpoint{1.209882in}{3.059045in}}%
\pgfpathlineto{\pgfqpoint{1.204549in}{3.076741in}}%
\pgfpathlineto{\pgfqpoint{1.199215in}{3.094653in}}%
\pgfpathlineto{\pgfqpoint{1.193882in}{3.112785in}}%
\pgfpathlineto{\pgfqpoint{1.188549in}{3.131140in}}%
\pgfpathlineto{\pgfqpoint{1.183216in}{3.149723in}}%
\pgfpathlineto{\pgfqpoint{1.177882in}{3.168538in}}%
\pgfpathlineto{\pgfqpoint{1.172549in}{3.187590in}}%
\pgfpathlineto{\pgfqpoint{1.167216in}{3.206883in}}%
\pgfpathlineto{\pgfqpoint{1.161883in}{3.226421in}}%
\pgfpathlineto{\pgfqpoint{1.156550in}{3.246210in}}%
\pgfpathlineto{\pgfqpoint{1.151216in}{3.266253in}}%
\pgfpathlineto{\pgfqpoint{1.145883in}{3.286557in}}%
\pgfpathlineto{\pgfqpoint{1.140550in}{3.307126in}}%
\pgfpathlineto{\pgfqpoint{1.135217in}{3.327965in}}%
\pgfpathlineto{\pgfqpoint{1.129883in}{3.349081in}}%
\pgfpathlineto{\pgfqpoint{1.124550in}{3.370477in}}%
\pgfpathlineto{\pgfqpoint{1.119217in}{3.392160in}}%
\pgfpathlineto{\pgfqpoint{1.113884in}{3.414137in}}%
\pgfpathlineto{\pgfqpoint{1.108551in}{3.436411in}}%
\pgfpathlineto{\pgfqpoint{1.103217in}{3.458991in}}%
\pgfpathlineto{\pgfqpoint{1.097884in}{3.481882in}}%
\pgfpathlineto{\pgfqpoint{1.092551in}{3.505091in}}%
\pgfpathlineto{\pgfqpoint{1.087218in}{3.528623in}}%
\pgfpathlineto{\pgfqpoint{1.081885in}{3.552487in}}%
\pgfpathlineto{\pgfqpoint{1.076551in}{3.576690in}}%
\pgfpathlineto{\pgfqpoint{1.071218in}{3.601237in}}%
\pgfpathlineto{\pgfqpoint{1.065885in}{3.626138in}}%
\pgfpathlineto{\pgfqpoint{1.060552in}{3.651399in}}%
\pgfpathlineto{\pgfqpoint{1.055218in}{3.677028in}}%
\pgfpathlineto{\pgfqpoint{1.049885in}{3.703034in}}%
\pgfpathlineto{\pgfqpoint{1.044552in}{3.729425in}}%
\pgfpathlineto{\pgfqpoint{1.039219in}{3.756209in}}%
\pgfpathlineto{\pgfqpoint{1.033886in}{3.783396in}}%
\pgfpathlineto{\pgfqpoint{1.028552in}{3.810994in}}%
\pgfpathlineto{\pgfqpoint{1.023219in}{3.839013in}}%
\pgfpathlineto{\pgfqpoint{1.017886in}{3.867463in}}%
\pgfpathlineto{\pgfqpoint{1.012553in}{3.896354in}}%
\pgfpathlineto{\pgfqpoint{1.007220in}{3.925695in}}%
\pgfpathlineto{\pgfqpoint{1.001886in}{3.955498in}}%
\pgfpathlineto{\pgfqpoint{0.996553in}{3.985774in}}%
\pgfpathlineto{\pgfqpoint{0.991220in}{4.016533in}}%
\pgfpathlineto{\pgfqpoint{0.985887in}{4.047788in}}%
\pgfpathlineto{\pgfqpoint{0.980553in}{4.079551in}}%
\pgfpathlineto{\pgfqpoint{0.975220in}{4.111834in}}%
\pgfpathlineto{\pgfqpoint{0.969887in}{4.144650in}}%
\pgfpathlineto{\pgfqpoint{0.964554in}{4.178012in}}%
\pgfpathlineto{\pgfqpoint{0.959221in}{4.211935in}}%
\pgfpathlineto{\pgfqpoint{0.953887in}{4.246431in}}%
\pgfpathlineto{\pgfqpoint{0.948554in}{4.281517in}}%
\pgfpathlineto{\pgfqpoint{0.943221in}{4.317207in}}%
\pgfpathlineto{\pgfqpoint{0.937888in}{4.353517in}}%
\pgfpathlineto{\pgfqpoint{0.932555in}{4.390463in}}%
\pgfpathlineto{\pgfqpoint{0.927221in}{4.428063in}}%
\pgfpathlineto{\pgfqpoint{0.921888in}{4.466333in}}%
\pgfpathlineto{\pgfqpoint{0.916555in}{4.505291in}}%
\pgfpathlineto{\pgfqpoint{0.911222in}{4.544958in}}%
\pgfpathlineto{\pgfqpoint{0.905888in}{4.585351in}}%
\pgfpathlineto{\pgfqpoint{0.900555in}{4.626491in}}%
\pgfpathlineto{\pgfqpoint{0.895222in}{4.668400in}}%
\pgfpathlineto{\pgfqpoint{0.889889in}{4.711098in}}%
\pgfpathlineto{\pgfqpoint{0.884556in}{4.754608in}}%
\pgfpathlineto{\pgfqpoint{0.879222in}{4.798955in}}%
\pgfpathlineto{\pgfqpoint{0.873889in}{4.844161in}}%
\pgfpathlineto{\pgfqpoint{0.868556in}{4.890252in}}%
\pgfpathlineto{\pgfqpoint{0.863223in}{4.937255in}}%
\pgfpathlineto{\pgfqpoint{0.857889in}{4.985197in}}%
\pgfpathlineto{\pgfqpoint{0.852556in}{5.034106in}}%
\pgfpathlineto{\pgfqpoint{0.847223in}{5.084012in}}%
\pgfpathlineto{\pgfqpoint{0.759998in}{4.259481in}}%
\pgfpathclose%
\pgfusepath{stroke,fill}%
\end{pgfscope}%
\begin{pgfscope}%
\pgfpathrectangle{\pgfqpoint{0.847223in}{0.554012in}}{\pgfqpoint{6.200000in}{4.530000in}}%
\pgfusepath{clip}%
\pgfsetbuttcap%
\pgfsetroundjoin%
\definecolor{currentfill}{rgb}{1.000000,1.000000,1.000000}%
\pgfsetfillcolor{currentfill}%
\pgfsetlinewidth{1.003750pt}%
\definecolor{currentstroke}{rgb}{1.000000,1.000000,1.000000}%
\pgfsetstrokecolor{currentstroke}%
\pgfsetdash{}{0pt}%
\pgfsys@defobject{currentmarker}{\pgfqpoint{0.751276in}{0.136829in}}{\pgfqpoint{5.814609in}{4.177028in}}{%
\pgfpathmoveto{\pgfqpoint{0.751276in}{0.136829in}}%
\pgfpathlineto{\pgfqpoint{0.751276in}{4.177028in}}%
\pgfpathlineto{\pgfqpoint{0.755631in}{4.136272in}}%
\pgfpathlineto{\pgfqpoint{0.759986in}{4.096329in}}%
\pgfpathlineto{\pgfqpoint{0.764342in}{4.057177in}}%
\pgfpathlineto{\pgfqpoint{0.768697in}{4.018791in}}%
\pgfpathlineto{\pgfqpoint{0.773053in}{3.981150in}}%
\pgfpathlineto{\pgfqpoint{0.777408in}{3.944232in}}%
\pgfpathlineto{\pgfqpoint{0.781764in}{3.908015in}}%
\pgfpathlineto{\pgfqpoint{0.786119in}{3.872482in}}%
\pgfpathlineto{\pgfqpoint{0.790475in}{3.837612in}}%
\pgfpathlineto{\pgfqpoint{0.794830in}{3.803387in}}%
\pgfpathlineto{\pgfqpoint{0.799186in}{3.769789in}}%
\pgfpathlineto{\pgfqpoint{0.803541in}{3.736801in}}%
\pgfpathlineto{\pgfqpoint{0.807897in}{3.704407in}}%
\pgfpathlineto{\pgfqpoint{0.812252in}{3.672590in}}%
\pgfpathlineto{\pgfqpoint{0.816607in}{3.641336in}}%
\pgfpathlineto{\pgfqpoint{0.820963in}{3.610630in}}%
\pgfpathlineto{\pgfqpoint{0.825318in}{3.580457in}}%
\pgfpathlineto{\pgfqpoint{0.829674in}{3.550804in}}%
\pgfpathlineto{\pgfqpoint{0.834029in}{3.521657in}}%
\pgfpathlineto{\pgfqpoint{0.838385in}{3.493004in}}%
\pgfpathlineto{\pgfqpoint{0.842740in}{3.464832in}}%
\pgfpathlineto{\pgfqpoint{0.847096in}{3.437129in}}%
\pgfpathlineto{\pgfqpoint{0.851451in}{3.409883in}}%
\pgfpathlineto{\pgfqpoint{0.855807in}{3.383083in}}%
\pgfpathlineto{\pgfqpoint{0.860162in}{3.356719in}}%
\pgfpathlineto{\pgfqpoint{0.864517in}{3.330779in}}%
\pgfpathlineto{\pgfqpoint{0.868873in}{3.305254in}}%
\pgfpathlineto{\pgfqpoint{0.873228in}{3.280134in}}%
\pgfpathlineto{\pgfqpoint{0.877584in}{3.255409in}}%
\pgfpathlineto{\pgfqpoint{0.881939in}{3.231070in}}%
\pgfpathlineto{\pgfqpoint{0.886295in}{3.207107in}}%
\pgfpathlineto{\pgfqpoint{0.890650in}{3.183514in}}%
\pgfpathlineto{\pgfqpoint{0.895006in}{3.160280in}}%
\pgfpathlineto{\pgfqpoint{0.899361in}{3.137397in}}%
\pgfpathlineto{\pgfqpoint{0.903717in}{3.114859in}}%
\pgfpathlineto{\pgfqpoint{0.908072in}{3.092656in}}%
\pgfpathlineto{\pgfqpoint{0.912428in}{3.070782in}}%
\pgfpathlineto{\pgfqpoint{0.916783in}{3.049230in}}%
\pgfpathlineto{\pgfqpoint{0.921138in}{3.027991in}}%
\pgfpathlineto{\pgfqpoint{0.925494in}{3.007061in}}%
\pgfpathlineto{\pgfqpoint{0.929849in}{2.986431in}}%
\pgfpathlineto{\pgfqpoint{0.934205in}{2.966096in}}%
\pgfpathlineto{\pgfqpoint{0.938560in}{2.946048in}}%
\pgfpathlineto{\pgfqpoint{0.942916in}{2.926283in}}%
\pgfpathlineto{\pgfqpoint{0.947271in}{2.906794in}}%
\pgfpathlineto{\pgfqpoint{0.951627in}{2.887576in}}%
\pgfpathlineto{\pgfqpoint{0.955982in}{2.868622in}}%
\pgfpathlineto{\pgfqpoint{0.960338in}{2.849928in}}%
\pgfpathlineto{\pgfqpoint{0.964693in}{2.831488in}}%
\pgfpathlineto{\pgfqpoint{0.969049in}{2.813297in}}%
\pgfpathlineto{\pgfqpoint{0.973404in}{2.795350in}}%
\pgfpathlineto{\pgfqpoint{0.977759in}{2.777642in}}%
\pgfpathlineto{\pgfqpoint{0.982115in}{2.760168in}}%
\pgfpathlineto{\pgfqpoint{0.986470in}{2.742924in}}%
\pgfpathlineto{\pgfqpoint{0.990826in}{2.725905in}}%
\pgfpathlineto{\pgfqpoint{0.995181in}{2.709107in}}%
\pgfpathlineto{\pgfqpoint{0.999537in}{2.692526in}}%
\pgfpathlineto{\pgfqpoint{1.003892in}{2.676157in}}%
\pgfpathlineto{\pgfqpoint{1.008248in}{2.659996in}}%
\pgfpathlineto{\pgfqpoint{1.012603in}{2.644040in}}%
\pgfpathlineto{\pgfqpoint{1.016959in}{2.628284in}}%
\pgfpathlineto{\pgfqpoint{1.021314in}{2.612725in}}%
\pgfpathlineto{\pgfqpoint{1.025669in}{2.597359in}}%
\pgfpathlineto{\pgfqpoint{1.030025in}{2.582183in}}%
\pgfpathlineto{\pgfqpoint{1.034380in}{2.567193in}}%
\pgfpathlineto{\pgfqpoint{1.038736in}{2.552385in}}%
\pgfpathlineto{\pgfqpoint{1.043091in}{2.537757in}}%
\pgfpathlineto{\pgfqpoint{1.047447in}{2.523305in}}%
\pgfpathlineto{\pgfqpoint{1.051802in}{2.509026in}}%
\pgfpathlineto{\pgfqpoint{1.056158in}{2.494917in}}%
\pgfpathlineto{\pgfqpoint{1.060513in}{2.480974in}}%
\pgfpathlineto{\pgfqpoint{1.064869in}{2.467196in}}%
\pgfpathlineto{\pgfqpoint{1.069224in}{2.453578in}}%
\pgfpathlineto{\pgfqpoint{1.073580in}{2.440119in}}%
\pgfpathlineto{\pgfqpoint{1.077935in}{2.426815in}}%
\pgfpathlineto{\pgfqpoint{1.082290in}{2.413664in}}%
\pgfpathlineto{\pgfqpoint{1.086646in}{2.400663in}}%
\pgfpathlineto{\pgfqpoint{1.091001in}{2.387810in}}%
\pgfpathlineto{\pgfqpoint{1.095357in}{2.375102in}}%
\pgfpathlineto{\pgfqpoint{1.099712in}{2.362537in}}%
\pgfpathlineto{\pgfqpoint{1.104068in}{2.350112in}}%
\pgfpathlineto{\pgfqpoint{1.108423in}{2.337824in}}%
\pgfpathlineto{\pgfqpoint{1.112779in}{2.325673in}}%
\pgfpathlineto{\pgfqpoint{1.117134in}{2.313655in}}%
\pgfpathlineto{\pgfqpoint{1.121490in}{2.301768in}}%
\pgfpathlineto{\pgfqpoint{1.125845in}{2.290011in}}%
\pgfpathlineto{\pgfqpoint{1.130201in}{2.278380in}}%
\pgfpathlineto{\pgfqpoint{1.134556in}{2.266874in}}%
\pgfpathlineto{\pgfqpoint{1.138911in}{2.255492in}}%
\pgfpathlineto{\pgfqpoint{1.143267in}{2.244230in}}%
\pgfpathlineto{\pgfqpoint{1.147622in}{2.233087in}}%
\pgfpathlineto{\pgfqpoint{1.151978in}{2.222062in}}%
\pgfpathlineto{\pgfqpoint{1.156333in}{2.211152in}}%
\pgfpathlineto{\pgfqpoint{1.160689in}{2.200356in}}%
\pgfpathlineto{\pgfqpoint{1.165044in}{2.189671in}}%
\pgfpathlineto{\pgfqpoint{1.169400in}{2.179096in}}%
\pgfpathlineto{\pgfqpoint{1.173755in}{2.168630in}}%
\pgfpathlineto{\pgfqpoint{1.178111in}{2.158271in}}%
\pgfpathlineto{\pgfqpoint{1.182466in}{2.148016in}}%
\pgfpathlineto{\pgfqpoint{1.186821in}{2.137866in}}%
\pgfpathlineto{\pgfqpoint{1.191177in}{2.127817in}}%
\pgfpathlineto{\pgfqpoint{1.195532in}{2.117868in}}%
\pgfpathlineto{\pgfqpoint{1.199888in}{2.108019in}}%
\pgfpathlineto{\pgfqpoint{1.204243in}{2.098267in}}%
\pgfpathlineto{\pgfqpoint{1.208599in}{2.088611in}}%
\pgfpathlineto{\pgfqpoint{1.212954in}{2.079049in}}%
\pgfpathlineto{\pgfqpoint{1.217310in}{2.069581in}}%
\pgfpathlineto{\pgfqpoint{1.221665in}{2.060205in}}%
\pgfpathlineto{\pgfqpoint{1.226021in}{2.050919in}}%
\pgfpathlineto{\pgfqpoint{1.230376in}{2.041722in}}%
\pgfpathlineto{\pgfqpoint{1.234732in}{2.032613in}}%
\pgfpathlineto{\pgfqpoint{1.239087in}{2.023591in}}%
\pgfpathlineto{\pgfqpoint{1.243442in}{2.014655in}}%
\pgfpathlineto{\pgfqpoint{1.247798in}{2.005803in}}%
\pgfpathlineto{\pgfqpoint{1.252153in}{1.997034in}}%
\pgfpathlineto{\pgfqpoint{1.256509in}{1.988346in}}%
\pgfpathlineto{\pgfqpoint{1.260864in}{1.979740in}}%
\pgfpathlineto{\pgfqpoint{1.265220in}{1.971213in}}%
\pgfpathlineto{\pgfqpoint{1.269575in}{1.962765in}}%
\pgfpathlineto{\pgfqpoint{1.273931in}{1.954394in}}%
\pgfpathlineto{\pgfqpoint{1.278286in}{1.946099in}}%
\pgfpathlineto{\pgfqpoint{1.282642in}{1.937880in}}%
\pgfpathlineto{\pgfqpoint{1.286997in}{1.929736in}}%
\pgfpathlineto{\pgfqpoint{1.291352in}{1.921664in}}%
\pgfpathlineto{\pgfqpoint{1.295708in}{1.913665in}}%
\pgfpathlineto{\pgfqpoint{1.300063in}{1.905737in}}%
\pgfpathlineto{\pgfqpoint{1.304419in}{1.897880in}}%
\pgfpathlineto{\pgfqpoint{1.308774in}{1.890092in}}%
\pgfpathlineto{\pgfqpoint{1.313130in}{1.882373in}}%
\pgfpathlineto{\pgfqpoint{1.317485in}{1.874722in}}%
\pgfpathlineto{\pgfqpoint{1.321841in}{1.867137in}}%
\pgfpathlineto{\pgfqpoint{1.326196in}{1.859618in}}%
\pgfpathlineto{\pgfqpoint{1.330552in}{1.852164in}}%
\pgfpathlineto{\pgfqpoint{1.334907in}{1.844775in}}%
\pgfpathlineto{\pgfqpoint{1.339263in}{1.837449in}}%
\pgfpathlineto{\pgfqpoint{1.343618in}{1.830185in}}%
\pgfpathlineto{\pgfqpoint{1.347973in}{1.822983in}}%
\pgfpathlineto{\pgfqpoint{1.352329in}{1.815843in}}%
\pgfpathlineto{\pgfqpoint{1.356684in}{1.808762in}}%
\pgfpathlineto{\pgfqpoint{1.361040in}{1.801741in}}%
\pgfpathlineto{\pgfqpoint{1.365395in}{1.794779in}}%
\pgfpathlineto{\pgfqpoint{1.369751in}{1.787874in}}%
\pgfpathlineto{\pgfqpoint{1.374106in}{1.781027in}}%
\pgfpathlineto{\pgfqpoint{1.378462in}{1.774236in}}%
\pgfpathlineto{\pgfqpoint{1.382817in}{1.767502in}}%
\pgfpathlineto{\pgfqpoint{1.387173in}{1.760822in}}%
\pgfpathlineto{\pgfqpoint{1.391528in}{1.754197in}}%
\pgfpathlineto{\pgfqpoint{1.395884in}{1.747626in}}%
\pgfpathlineto{\pgfqpoint{1.400239in}{1.741108in}}%
\pgfpathlineto{\pgfqpoint{1.404594in}{1.734643in}}%
\pgfpathlineto{\pgfqpoint{1.408950in}{1.728229in}}%
\pgfpathlineto{\pgfqpoint{1.413305in}{1.721867in}}%
\pgfpathlineto{\pgfqpoint{1.417661in}{1.715555in}}%
\pgfpathlineto{\pgfqpoint{1.422016in}{1.709294in}}%
\pgfpathlineto{\pgfqpoint{1.426372in}{1.703081in}}%
\pgfpathlineto{\pgfqpoint{1.430727in}{1.696918in}}%
\pgfpathlineto{\pgfqpoint{1.435083in}{1.690803in}}%
\pgfpathlineto{\pgfqpoint{1.439438in}{1.684736in}}%
\pgfpathlineto{\pgfqpoint{1.443794in}{1.678716in}}%
\pgfpathlineto{\pgfqpoint{1.448149in}{1.672743in}}%
\pgfpathlineto{\pgfqpoint{1.452504in}{1.666816in}}%
\pgfpathlineto{\pgfqpoint{1.456860in}{1.660934in}}%
\pgfpathlineto{\pgfqpoint{1.461215in}{1.655098in}}%
\pgfpathlineto{\pgfqpoint{1.465571in}{1.649306in}}%
\pgfpathlineto{\pgfqpoint{1.469926in}{1.643558in}}%
\pgfpathlineto{\pgfqpoint{1.474282in}{1.637853in}}%
\pgfpathlineto{\pgfqpoint{1.478637in}{1.632192in}}%
\pgfpathlineto{\pgfqpoint{1.482993in}{1.626573in}}%
\pgfpathlineto{\pgfqpoint{1.487348in}{1.620996in}}%
\pgfpathlineto{\pgfqpoint{1.491704in}{1.615461in}}%
\pgfpathlineto{\pgfqpoint{1.496059in}{1.609967in}}%
\pgfpathlineto{\pgfqpoint{1.500415in}{1.604513in}}%
\pgfpathlineto{\pgfqpoint{1.504770in}{1.599100in}}%
\pgfpathlineto{\pgfqpoint{1.509125in}{1.593727in}}%
\pgfpathlineto{\pgfqpoint{1.513481in}{1.588392in}}%
\pgfpathlineto{\pgfqpoint{1.517836in}{1.583097in}}%
\pgfpathlineto{\pgfqpoint{1.522192in}{1.577841in}}%
\pgfpathlineto{\pgfqpoint{1.526547in}{1.572622in}}%
\pgfpathlineto{\pgfqpoint{1.530903in}{1.567441in}}%
\pgfpathlineto{\pgfqpoint{1.535258in}{1.562297in}}%
\pgfpathlineto{\pgfqpoint{1.539614in}{1.557191in}}%
\pgfpathlineto{\pgfqpoint{1.543969in}{1.552120in}}%
\pgfpathlineto{\pgfqpoint{1.548325in}{1.547086in}}%
\pgfpathlineto{\pgfqpoint{1.552680in}{1.542087in}}%
\pgfpathlineto{\pgfqpoint{1.557036in}{1.537124in}}%
\pgfpathlineto{\pgfqpoint{1.561391in}{1.532196in}}%
\pgfpathlineto{\pgfqpoint{1.565746in}{1.527302in}}%
\pgfpathlineto{\pgfqpoint{1.570102in}{1.522442in}}%
\pgfpathlineto{\pgfqpoint{1.574457in}{1.517617in}}%
\pgfpathlineto{\pgfqpoint{1.578813in}{1.512824in}}%
\pgfpathlineto{\pgfqpoint{1.583168in}{1.508065in}}%
\pgfpathlineto{\pgfqpoint{1.587524in}{1.503339in}}%
\pgfpathlineto{\pgfqpoint{1.591879in}{1.498645in}}%
\pgfpathlineto{\pgfqpoint{1.596235in}{1.493984in}}%
\pgfpathlineto{\pgfqpoint{1.600590in}{1.489354in}}%
\pgfpathlineto{\pgfqpoint{1.604946in}{1.484755in}}%
\pgfpathlineto{\pgfqpoint{1.609301in}{1.480188in}}%
\pgfpathlineto{\pgfqpoint{1.613656in}{1.475652in}}%
\pgfpathlineto{\pgfqpoint{1.618012in}{1.471146in}}%
\pgfpathlineto{\pgfqpoint{1.622367in}{1.466670in}}%
\pgfpathlineto{\pgfqpoint{1.626723in}{1.462224in}}%
\pgfpathlineto{\pgfqpoint{1.631078in}{1.457808in}}%
\pgfpathlineto{\pgfqpoint{1.635434in}{1.453422in}}%
\pgfpathlineto{\pgfqpoint{1.639789in}{1.449064in}}%
\pgfpathlineto{\pgfqpoint{1.644145in}{1.444735in}}%
\pgfpathlineto{\pgfqpoint{1.648500in}{1.440435in}}%
\pgfpathlineto{\pgfqpoint{1.652856in}{1.436162in}}%
\pgfpathlineto{\pgfqpoint{1.657211in}{1.431918in}}%
\pgfpathlineto{\pgfqpoint{1.661567in}{1.427701in}}%
\pgfpathlineto{\pgfqpoint{1.665922in}{1.423512in}}%
\pgfpathlineto{\pgfqpoint{1.670277in}{1.419350in}}%
\pgfpathlineto{\pgfqpoint{1.674633in}{1.415214in}}%
\pgfpathlineto{\pgfqpoint{1.678988in}{1.411105in}}%
\pgfpathlineto{\pgfqpoint{1.683344in}{1.407023in}}%
\pgfpathlineto{\pgfqpoint{1.687699in}{1.402966in}}%
\pgfpathlineto{\pgfqpoint{1.692055in}{1.398936in}}%
\pgfpathlineto{\pgfqpoint{1.696410in}{1.394931in}}%
\pgfpathlineto{\pgfqpoint{1.700766in}{1.390951in}}%
\pgfpathlineto{\pgfqpoint{1.705121in}{1.386996in}}%
\pgfpathlineto{\pgfqpoint{1.709477in}{1.383067in}}%
\pgfpathlineto{\pgfqpoint{1.713832in}{1.379162in}}%
\pgfpathlineto{\pgfqpoint{1.718188in}{1.375281in}}%
\pgfpathlineto{\pgfqpoint{1.722543in}{1.371424in}}%
\pgfpathlineto{\pgfqpoint{1.726898in}{1.367592in}}%
\pgfpathlineto{\pgfqpoint{1.731254in}{1.363783in}}%
\pgfpathlineto{\pgfqpoint{1.735609in}{1.359997in}}%
\pgfpathlineto{\pgfqpoint{1.739965in}{1.356235in}}%
\pgfpathlineto{\pgfqpoint{1.744320in}{1.352496in}}%
\pgfpathlineto{\pgfqpoint{1.748676in}{1.348780in}}%
\pgfpathlineto{\pgfqpoint{1.753031in}{1.345087in}}%
\pgfpathlineto{\pgfqpoint{1.757387in}{1.341416in}}%
\pgfpathlineto{\pgfqpoint{1.761742in}{1.337767in}}%
\pgfpathlineto{\pgfqpoint{1.766098in}{1.334140in}}%
\pgfpathlineto{\pgfqpoint{1.770453in}{1.330535in}}%
\pgfpathlineto{\pgfqpoint{1.774808in}{1.326952in}}%
\pgfpathlineto{\pgfqpoint{1.779164in}{1.323390in}}%
\pgfpathlineto{\pgfqpoint{1.783519in}{1.319849in}}%
\pgfpathlineto{\pgfqpoint{1.787875in}{1.316330in}}%
\pgfpathlineto{\pgfqpoint{1.792230in}{1.312831in}}%
\pgfpathlineto{\pgfqpoint{1.796586in}{1.309353in}}%
\pgfpathlineto{\pgfqpoint{1.800941in}{1.305896in}}%
\pgfpathlineto{\pgfqpoint{1.805297in}{1.302459in}}%
\pgfpathlineto{\pgfqpoint{1.809652in}{1.299042in}}%
\pgfpathlineto{\pgfqpoint{1.814008in}{1.295645in}}%
\pgfpathlineto{\pgfqpoint{1.818363in}{1.292267in}}%
\pgfpathlineto{\pgfqpoint{1.822719in}{1.288910in}}%
\pgfpathlineto{\pgfqpoint{1.827074in}{1.285572in}}%
\pgfpathlineto{\pgfqpoint{1.831429in}{1.282253in}}%
\pgfpathlineto{\pgfqpoint{1.835785in}{1.278953in}}%
\pgfpathlineto{\pgfqpoint{1.840140in}{1.275673in}}%
\pgfpathlineto{\pgfqpoint{1.844496in}{1.272411in}}%
\pgfpathlineto{\pgfqpoint{1.848851in}{1.269167in}}%
\pgfpathlineto{\pgfqpoint{1.853207in}{1.265942in}}%
\pgfpathlineto{\pgfqpoint{1.857562in}{1.262736in}}%
\pgfpathlineto{\pgfqpoint{1.861918in}{1.259548in}}%
\pgfpathlineto{\pgfqpoint{1.866273in}{1.256377in}}%
\pgfpathlineto{\pgfqpoint{1.870629in}{1.253225in}}%
\pgfpathlineto{\pgfqpoint{1.874984in}{1.250090in}}%
\pgfpathlineto{\pgfqpoint{1.879340in}{1.246973in}}%
\pgfpathlineto{\pgfqpoint{1.883695in}{1.243873in}}%
\pgfpathlineto{\pgfqpoint{1.888050in}{1.240790in}}%
\pgfpathlineto{\pgfqpoint{1.892406in}{1.237725in}}%
\pgfpathlineto{\pgfqpoint{1.896761in}{1.234676in}}%
\pgfpathlineto{\pgfqpoint{1.901117in}{1.231645in}}%
\pgfpathlineto{\pgfqpoint{1.905472in}{1.228630in}}%
\pgfpathlineto{\pgfqpoint{1.909828in}{1.225631in}}%
\pgfpathlineto{\pgfqpoint{1.914183in}{1.222649in}}%
\pgfpathlineto{\pgfqpoint{1.918539in}{1.219684in}}%
\pgfpathlineto{\pgfqpoint{1.922894in}{1.216734in}}%
\pgfpathlineto{\pgfqpoint{1.927250in}{1.213801in}}%
\pgfpathlineto{\pgfqpoint{1.931605in}{1.210883in}}%
\pgfpathlineto{\pgfqpoint{1.935960in}{1.207981in}}%
\pgfpathlineto{\pgfqpoint{1.940316in}{1.205095in}}%
\pgfpathlineto{\pgfqpoint{1.944671in}{1.202224in}}%
\pgfpathlineto{\pgfqpoint{1.949027in}{1.199369in}}%
\pgfpathlineto{\pgfqpoint{1.953382in}{1.196529in}}%
\pgfpathlineto{\pgfqpoint{1.957738in}{1.193704in}}%
\pgfpathlineto{\pgfqpoint{1.962093in}{1.190894in}}%
\pgfpathlineto{\pgfqpoint{1.966449in}{1.188099in}}%
\pgfpathlineto{\pgfqpoint{1.970804in}{1.185319in}}%
\pgfpathlineto{\pgfqpoint{1.975160in}{1.182554in}}%
\pgfpathlineto{\pgfqpoint{1.979515in}{1.179803in}}%
\pgfpathlineto{\pgfqpoint{1.983871in}{1.177066in}}%
\pgfpathlineto{\pgfqpoint{1.988226in}{1.174344in}}%
\pgfpathlineto{\pgfqpoint{1.992581in}{1.171636in}}%
\pgfpathlineto{\pgfqpoint{1.996937in}{1.168942in}}%
\pgfpathlineto{\pgfqpoint{2.001292in}{1.166262in}}%
\pgfpathlineto{\pgfqpoint{2.005648in}{1.163596in}}%
\pgfpathlineto{\pgfqpoint{2.010003in}{1.160944in}}%
\pgfpathlineto{\pgfqpoint{2.014359in}{1.158305in}}%
\pgfpathlineto{\pgfqpoint{2.018714in}{1.155680in}}%
\pgfpathlineto{\pgfqpoint{2.023070in}{1.153069in}}%
\pgfpathlineto{\pgfqpoint{2.027425in}{1.150471in}}%
\pgfpathlineto{\pgfqpoint{2.031781in}{1.147886in}}%
\pgfpathlineto{\pgfqpoint{2.036136in}{1.145314in}}%
\pgfpathlineto{\pgfqpoint{2.040492in}{1.142755in}}%
\pgfpathlineto{\pgfqpoint{2.044847in}{1.140209in}}%
\pgfpathlineto{\pgfqpoint{2.049202in}{1.137676in}}%
\pgfpathlineto{\pgfqpoint{2.053558in}{1.135156in}}%
\pgfpathlineto{\pgfqpoint{2.057913in}{1.132649in}}%
\pgfpathlineto{\pgfqpoint{2.062269in}{1.130154in}}%
\pgfpathlineto{\pgfqpoint{2.066624in}{1.127671in}}%
\pgfpathlineto{\pgfqpoint{2.070980in}{1.125201in}}%
\pgfpathlineto{\pgfqpoint{2.075335in}{1.122743in}}%
\pgfpathlineto{\pgfqpoint{2.079691in}{1.120297in}}%
\pgfpathlineto{\pgfqpoint{2.084046in}{1.117864in}}%
\pgfpathlineto{\pgfqpoint{2.088402in}{1.115442in}}%
\pgfpathlineto{\pgfqpoint{2.092757in}{1.113033in}}%
\pgfpathlineto{\pgfqpoint{2.097112in}{1.110635in}}%
\pgfpathlineto{\pgfqpoint{2.101468in}{1.108249in}}%
\pgfpathlineto{\pgfqpoint{2.105823in}{1.105875in}}%
\pgfpathlineto{\pgfqpoint{2.110179in}{1.103512in}}%
\pgfpathlineto{\pgfqpoint{2.114534in}{1.101161in}}%
\pgfpathlineto{\pgfqpoint{2.118890in}{1.098821in}}%
\pgfpathlineto{\pgfqpoint{2.123245in}{1.096492in}}%
\pgfpathlineto{\pgfqpoint{2.127601in}{1.094175in}}%
\pgfpathlineto{\pgfqpoint{2.131956in}{1.091869in}}%
\pgfpathlineto{\pgfqpoint{2.136312in}{1.089574in}}%
\pgfpathlineto{\pgfqpoint{2.140667in}{1.087290in}}%
\pgfpathlineto{\pgfqpoint{2.145023in}{1.085017in}}%
\pgfpathlineto{\pgfqpoint{2.149378in}{1.082754in}}%
\pgfpathlineto{\pgfqpoint{2.153733in}{1.080503in}}%
\pgfpathlineto{\pgfqpoint{2.158089in}{1.078262in}}%
\pgfpathlineto{\pgfqpoint{2.162444in}{1.076032in}}%
\pgfpathlineto{\pgfqpoint{2.166800in}{1.073812in}}%
\pgfpathlineto{\pgfqpoint{2.171155in}{1.071603in}}%
\pgfpathlineto{\pgfqpoint{2.175511in}{1.069404in}}%
\pgfpathlineto{\pgfqpoint{2.179866in}{1.067216in}}%
\pgfpathlineto{\pgfqpoint{2.184222in}{1.065037in}}%
\pgfpathlineto{\pgfqpoint{2.188577in}{1.062869in}}%
\pgfpathlineto{\pgfqpoint{2.192933in}{1.060711in}}%
\pgfpathlineto{\pgfqpoint{2.197288in}{1.058563in}}%
\pgfpathlineto{\pgfqpoint{2.201644in}{1.056426in}}%
\pgfpathlineto{\pgfqpoint{2.205999in}{1.054297in}}%
\pgfpathlineto{\pgfqpoint{2.210354in}{1.052179in}}%
\pgfpathlineto{\pgfqpoint{2.214710in}{1.050071in}}%
\pgfpathlineto{\pgfqpoint{2.219065in}{1.047972in}}%
\pgfpathlineto{\pgfqpoint{2.223421in}{1.045883in}}%
\pgfpathlineto{\pgfqpoint{2.227776in}{1.043803in}}%
\pgfpathlineto{\pgfqpoint{2.232132in}{1.041733in}}%
\pgfpathlineto{\pgfqpoint{2.236487in}{1.039672in}}%
\pgfpathlineto{\pgfqpoint{2.240843in}{1.037621in}}%
\pgfpathlineto{\pgfqpoint{2.245198in}{1.035579in}}%
\pgfpathlineto{\pgfqpoint{2.249554in}{1.033546in}}%
\pgfpathlineto{\pgfqpoint{2.253909in}{1.031523in}}%
\pgfpathlineto{\pgfqpoint{2.258264in}{1.029508in}}%
\pgfpathlineto{\pgfqpoint{2.262620in}{1.027503in}}%
\pgfpathlineto{\pgfqpoint{2.266975in}{1.025506in}}%
\pgfpathlineto{\pgfqpoint{2.271331in}{1.023519in}}%
\pgfpathlineto{\pgfqpoint{2.275686in}{1.021540in}}%
\pgfpathlineto{\pgfqpoint{2.280042in}{1.019570in}}%
\pgfpathlineto{\pgfqpoint{2.284397in}{1.017609in}}%
\pgfpathlineto{\pgfqpoint{2.288753in}{1.015657in}}%
\pgfpathlineto{\pgfqpoint{2.293108in}{1.013713in}}%
\pgfpathlineto{\pgfqpoint{2.297464in}{1.011778in}}%
\pgfpathlineto{\pgfqpoint{2.301819in}{1.009851in}}%
\pgfpathlineto{\pgfqpoint{2.306175in}{1.007933in}}%
\pgfpathlineto{\pgfqpoint{2.310530in}{1.006023in}}%
\pgfpathlineto{\pgfqpoint{2.314885in}{1.004122in}}%
\pgfpathlineto{\pgfqpoint{2.319241in}{1.002229in}}%
\pgfpathlineto{\pgfqpoint{2.323596in}{1.000344in}}%
\pgfpathlineto{\pgfqpoint{2.327952in}{0.998467in}}%
\pgfpathlineto{\pgfqpoint{2.332307in}{0.996599in}}%
\pgfpathlineto{\pgfqpoint{2.336663in}{0.994738in}}%
\pgfpathlineto{\pgfqpoint{2.341018in}{0.992886in}}%
\pgfpathlineto{\pgfqpoint{2.345374in}{0.991042in}}%
\pgfpathlineto{\pgfqpoint{2.349729in}{0.989205in}}%
\pgfpathlineto{\pgfqpoint{2.354085in}{0.987376in}}%
\pgfpathlineto{\pgfqpoint{2.358440in}{0.985556in}}%
\pgfpathlineto{\pgfqpoint{2.362796in}{0.983743in}}%
\pgfpathlineto{\pgfqpoint{2.367151in}{0.981937in}}%
\pgfpathlineto{\pgfqpoint{2.371506in}{0.980140in}}%
\pgfpathlineto{\pgfqpoint{2.375862in}{0.978350in}}%
\pgfpathlineto{\pgfqpoint{2.380217in}{0.976567in}}%
\pgfpathlineto{\pgfqpoint{2.384573in}{0.974792in}}%
\pgfpathlineto{\pgfqpoint{2.388928in}{0.973025in}}%
\pgfpathlineto{\pgfqpoint{2.393284in}{0.971265in}}%
\pgfpathlineto{\pgfqpoint{2.397639in}{0.969513in}}%
\pgfpathlineto{\pgfqpoint{2.401995in}{0.967767in}}%
\pgfpathlineto{\pgfqpoint{2.406350in}{0.966030in}}%
\pgfpathlineto{\pgfqpoint{2.410706in}{0.964299in}}%
\pgfpathlineto{\pgfqpoint{2.415061in}{0.962575in}}%
\pgfpathlineto{\pgfqpoint{2.419416in}{0.960859in}}%
\pgfpathlineto{\pgfqpoint{2.423772in}{0.959150in}}%
\pgfpathlineto{\pgfqpoint{2.428127in}{0.957448in}}%
\pgfpathlineto{\pgfqpoint{2.432483in}{0.955753in}}%
\pgfpathlineto{\pgfqpoint{2.436838in}{0.954065in}}%
\pgfpathlineto{\pgfqpoint{2.441194in}{0.952384in}}%
\pgfpathlineto{\pgfqpoint{2.445549in}{0.950710in}}%
\pgfpathlineto{\pgfqpoint{2.449905in}{0.949042in}}%
\pgfpathlineto{\pgfqpoint{2.454260in}{0.947382in}}%
\pgfpathlineto{\pgfqpoint{2.458616in}{0.945728in}}%
\pgfpathlineto{\pgfqpoint{2.462971in}{0.944081in}}%
\pgfpathlineto{\pgfqpoint{2.467327in}{0.942441in}}%
\pgfpathlineto{\pgfqpoint{2.471682in}{0.940807in}}%
\pgfpathlineto{\pgfqpoint{2.476037in}{0.939180in}}%
\pgfpathlineto{\pgfqpoint{2.480393in}{0.937559in}}%
\pgfpathlineto{\pgfqpoint{2.484748in}{0.935945in}}%
\pgfpathlineto{\pgfqpoint{2.489104in}{0.934338in}}%
\pgfpathlineto{\pgfqpoint{2.493459in}{0.932737in}}%
\pgfpathlineto{\pgfqpoint{2.497815in}{0.931142in}}%
\pgfpathlineto{\pgfqpoint{2.502170in}{0.929554in}}%
\pgfpathlineto{\pgfqpoint{2.506526in}{0.927972in}}%
\pgfpathlineto{\pgfqpoint{2.510881in}{0.926397in}}%
\pgfpathlineto{\pgfqpoint{2.515237in}{0.924827in}}%
\pgfpathlineto{\pgfqpoint{2.519592in}{0.923264in}}%
\pgfpathlineto{\pgfqpoint{2.523948in}{0.921707in}}%
\pgfpathlineto{\pgfqpoint{2.528303in}{0.920157in}}%
\pgfpathlineto{\pgfqpoint{2.532658in}{0.918612in}}%
\pgfpathlineto{\pgfqpoint{2.537014in}{0.917074in}}%
\pgfpathlineto{\pgfqpoint{2.541369in}{0.915541in}}%
\pgfpathlineto{\pgfqpoint{2.545725in}{0.914015in}}%
\pgfpathlineto{\pgfqpoint{2.550080in}{0.912494in}}%
\pgfpathlineto{\pgfqpoint{2.554436in}{0.910979in}}%
\pgfpathlineto{\pgfqpoint{2.558791in}{0.909471in}}%
\pgfpathlineto{\pgfqpoint{2.563147in}{0.907968in}}%
\pgfpathlineto{\pgfqpoint{2.567502in}{0.906471in}}%
\pgfpathlineto{\pgfqpoint{2.571858in}{0.904980in}}%
\pgfpathlineto{\pgfqpoint{2.576213in}{0.903494in}}%
\pgfpathlineto{\pgfqpoint{2.580568in}{0.902015in}}%
\pgfpathlineto{\pgfqpoint{2.584924in}{0.900541in}}%
\pgfpathlineto{\pgfqpoint{2.589279in}{0.899072in}}%
\pgfpathlineto{\pgfqpoint{2.593635in}{0.897610in}}%
\pgfpathlineto{\pgfqpoint{2.597990in}{0.896153in}}%
\pgfpathlineto{\pgfqpoint{2.602346in}{0.894701in}}%
\pgfpathlineto{\pgfqpoint{2.606701in}{0.893255in}}%
\pgfpathlineto{\pgfqpoint{2.611057in}{0.891815in}}%
\pgfpathlineto{\pgfqpoint{2.615412in}{0.890380in}}%
\pgfpathlineto{\pgfqpoint{2.619768in}{0.888950in}}%
\pgfpathlineto{\pgfqpoint{2.624123in}{0.887526in}}%
\pgfpathlineto{\pgfqpoint{2.628479in}{0.886107in}}%
\pgfpathlineto{\pgfqpoint{2.632834in}{0.884694in}}%
\pgfpathlineto{\pgfqpoint{2.637189in}{0.883286in}}%
\pgfpathlineto{\pgfqpoint{2.641545in}{0.881883in}}%
\pgfpathlineto{\pgfqpoint{2.645900in}{0.880486in}}%
\pgfpathlineto{\pgfqpoint{2.650256in}{0.879093in}}%
\pgfpathlineto{\pgfqpoint{2.654611in}{0.877706in}}%
\pgfpathlineto{\pgfqpoint{2.658967in}{0.876324in}}%
\pgfpathlineto{\pgfqpoint{2.663322in}{0.874948in}}%
\pgfpathlineto{\pgfqpoint{2.667678in}{0.873576in}}%
\pgfpathlineto{\pgfqpoint{2.672033in}{0.872209in}}%
\pgfpathlineto{\pgfqpoint{2.676389in}{0.870848in}}%
\pgfpathlineto{\pgfqpoint{2.680744in}{0.869492in}}%
\pgfpathlineto{\pgfqpoint{2.685100in}{0.868140in}}%
\pgfpathlineto{\pgfqpoint{2.689455in}{0.866794in}}%
\pgfpathlineto{\pgfqpoint{2.693810in}{0.865452in}}%
\pgfpathlineto{\pgfqpoint{2.698166in}{0.864115in}}%
\pgfpathlineto{\pgfqpoint{2.702521in}{0.862784in}}%
\pgfpathlineto{\pgfqpoint{2.706877in}{0.861457in}}%
\pgfpathlineto{\pgfqpoint{2.711232in}{0.860135in}}%
\pgfpathlineto{\pgfqpoint{2.715588in}{0.858818in}}%
\pgfpathlineto{\pgfqpoint{2.719943in}{0.857505in}}%
\pgfpathlineto{\pgfqpoint{2.724299in}{0.856198in}}%
\pgfpathlineto{\pgfqpoint{2.728654in}{0.854895in}}%
\pgfpathlineto{\pgfqpoint{2.733010in}{0.853597in}}%
\pgfpathlineto{\pgfqpoint{2.737365in}{0.852303in}}%
\pgfpathlineto{\pgfqpoint{2.741720in}{0.851014in}}%
\pgfpathlineto{\pgfqpoint{2.746076in}{0.849730in}}%
\pgfpathlineto{\pgfqpoint{2.750431in}{0.848450in}}%
\pgfpathlineto{\pgfqpoint{2.754787in}{0.847175in}}%
\pgfpathlineto{\pgfqpoint{2.759142in}{0.845905in}}%
\pgfpathlineto{\pgfqpoint{2.763498in}{0.844639in}}%
\pgfpathlineto{\pgfqpoint{2.767853in}{0.843378in}}%
\pgfpathlineto{\pgfqpoint{2.772209in}{0.842121in}}%
\pgfpathlineto{\pgfqpoint{2.776564in}{0.840868in}}%
\pgfpathlineto{\pgfqpoint{2.780920in}{0.839620in}}%
\pgfpathlineto{\pgfqpoint{2.785275in}{0.838377in}}%
\pgfpathlineto{\pgfqpoint{2.789631in}{0.837138in}}%
\pgfpathlineto{\pgfqpoint{2.793986in}{0.835903in}}%
\pgfpathlineto{\pgfqpoint{2.798341in}{0.834672in}}%
\pgfpathlineto{\pgfqpoint{2.802697in}{0.833446in}}%
\pgfpathlineto{\pgfqpoint{2.807052in}{0.832224in}}%
\pgfpathlineto{\pgfqpoint{2.811408in}{0.831007in}}%
\pgfpathlineto{\pgfqpoint{2.815763in}{0.829793in}}%
\pgfpathlineto{\pgfqpoint{2.820119in}{0.828584in}}%
\pgfpathlineto{\pgfqpoint{2.824474in}{0.827379in}}%
\pgfpathlineto{\pgfqpoint{2.828830in}{0.826179in}}%
\pgfpathlineto{\pgfqpoint{2.833185in}{0.824982in}}%
\pgfpathlineto{\pgfqpoint{2.837541in}{0.823790in}}%
\pgfpathlineto{\pgfqpoint{2.841896in}{0.822602in}}%
\pgfpathlineto{\pgfqpoint{2.846252in}{0.821417in}}%
\pgfpathlineto{\pgfqpoint{2.850607in}{0.820237in}}%
\pgfpathlineto{\pgfqpoint{2.854962in}{0.819061in}}%
\pgfpathlineto{\pgfqpoint{2.859318in}{0.817889in}}%
\pgfpathlineto{\pgfqpoint{2.863673in}{0.816721in}}%
\pgfpathlineto{\pgfqpoint{2.868029in}{0.815557in}}%
\pgfpathlineto{\pgfqpoint{2.872384in}{0.814398in}}%
\pgfpathlineto{\pgfqpoint{2.876740in}{0.813242in}}%
\pgfpathlineto{\pgfqpoint{2.881095in}{0.812089in}}%
\pgfpathlineto{\pgfqpoint{2.885451in}{0.810941in}}%
\pgfpathlineto{\pgfqpoint{2.889806in}{0.809797in}}%
\pgfpathlineto{\pgfqpoint{2.894162in}{0.808657in}}%
\pgfpathlineto{\pgfqpoint{2.898517in}{0.807520in}}%
\pgfpathlineto{\pgfqpoint{2.902872in}{0.806388in}}%
\pgfpathlineto{\pgfqpoint{2.907228in}{0.805259in}}%
\pgfpathlineto{\pgfqpoint{2.911583in}{0.804134in}}%
\pgfpathlineto{\pgfqpoint{2.915939in}{0.803012in}}%
\pgfpathlineto{\pgfqpoint{2.920294in}{0.801895in}}%
\pgfpathlineto{\pgfqpoint{2.924650in}{0.800781in}}%
\pgfpathlineto{\pgfqpoint{2.929005in}{0.799671in}}%
\pgfpathlineto{\pgfqpoint{2.933361in}{0.798565in}}%
\pgfpathlineto{\pgfqpoint{2.937716in}{0.797462in}}%
\pgfpathlineto{\pgfqpoint{2.942072in}{0.796363in}}%
\pgfpathlineto{\pgfqpoint{2.946427in}{0.795268in}}%
\pgfpathlineto{\pgfqpoint{2.950783in}{0.794176in}}%
\pgfpathlineto{\pgfqpoint{2.955138in}{0.793088in}}%
\pgfpathlineto{\pgfqpoint{2.959493in}{0.792003in}}%
\pgfpathlineto{\pgfqpoint{2.963849in}{0.790922in}}%
\pgfpathlineto{\pgfqpoint{2.968204in}{0.789845in}}%
\pgfpathlineto{\pgfqpoint{2.972560in}{0.788771in}}%
\pgfpathlineto{\pgfqpoint{2.976915in}{0.787701in}}%
\pgfpathlineto{\pgfqpoint{2.981271in}{0.786634in}}%
\pgfpathlineto{\pgfqpoint{2.985626in}{0.785571in}}%
\pgfpathlineto{\pgfqpoint{2.989982in}{0.784511in}}%
\pgfpathlineto{\pgfqpoint{2.994337in}{0.783455in}}%
\pgfpathlineto{\pgfqpoint{2.998693in}{0.782402in}}%
\pgfpathlineto{\pgfqpoint{3.003048in}{0.781352in}}%
\pgfpathlineto{\pgfqpoint{3.007404in}{0.780306in}}%
\pgfpathlineto{\pgfqpoint{3.011759in}{0.779264in}}%
\pgfpathlineto{\pgfqpoint{3.016114in}{0.778224in}}%
\pgfpathlineto{\pgfqpoint{3.020470in}{0.777188in}}%
\pgfpathlineto{\pgfqpoint{3.024825in}{0.776156in}}%
\pgfpathlineto{\pgfqpoint{3.029181in}{0.775126in}}%
\pgfpathlineto{\pgfqpoint{3.033536in}{0.774100in}}%
\pgfpathlineto{\pgfqpoint{3.037892in}{0.773078in}}%
\pgfpathlineto{\pgfqpoint{3.042247in}{0.772058in}}%
\pgfpathlineto{\pgfqpoint{3.046603in}{0.771042in}}%
\pgfpathlineto{\pgfqpoint{3.050958in}{0.770029in}}%
\pgfpathlineto{\pgfqpoint{3.055314in}{0.769020in}}%
\pgfpathlineto{\pgfqpoint{3.059669in}{0.768013in}}%
\pgfpathlineto{\pgfqpoint{3.064024in}{0.767010in}}%
\pgfpathlineto{\pgfqpoint{3.068380in}{0.766010in}}%
\pgfpathlineto{\pgfqpoint{3.072735in}{0.765013in}}%
\pgfpathlineto{\pgfqpoint{3.077091in}{0.764019in}}%
\pgfpathlineto{\pgfqpoint{3.081446in}{0.763028in}}%
\pgfpathlineto{\pgfqpoint{3.085802in}{0.762041in}}%
\pgfpathlineto{\pgfqpoint{3.090157in}{0.761057in}}%
\pgfpathlineto{\pgfqpoint{3.094513in}{0.760075in}}%
\pgfpathlineto{\pgfqpoint{3.098868in}{0.759097in}}%
\pgfpathlineto{\pgfqpoint{3.103224in}{0.758122in}}%
\pgfpathlineto{\pgfqpoint{3.107579in}{0.757150in}}%
\pgfpathlineto{\pgfqpoint{3.111935in}{0.756181in}}%
\pgfpathlineto{\pgfqpoint{3.116290in}{0.755215in}}%
\pgfpathlineto{\pgfqpoint{3.120645in}{0.754252in}}%
\pgfpathlineto{\pgfqpoint{3.125001in}{0.753292in}}%
\pgfpathlineto{\pgfqpoint{3.129356in}{0.752335in}}%
\pgfpathlineto{\pgfqpoint{3.133712in}{0.751381in}}%
\pgfpathlineto{\pgfqpoint{3.138067in}{0.750429in}}%
\pgfpathlineto{\pgfqpoint{3.142423in}{0.749481in}}%
\pgfpathlineto{\pgfqpoint{3.146778in}{0.748536in}}%
\pgfpathlineto{\pgfqpoint{3.151134in}{0.747594in}}%
\pgfpathlineto{\pgfqpoint{3.155489in}{0.746654in}}%
\pgfpathlineto{\pgfqpoint{3.159845in}{0.745718in}}%
\pgfpathlineto{\pgfqpoint{3.164200in}{0.744784in}}%
\pgfpathlineto{\pgfqpoint{3.168556in}{0.743853in}}%
\pgfpathlineto{\pgfqpoint{3.172911in}{0.742925in}}%
\pgfpathlineto{\pgfqpoint{3.177266in}{0.742000in}}%
\pgfpathlineto{\pgfqpoint{3.181622in}{0.741078in}}%
\pgfpathlineto{\pgfqpoint{3.185977in}{0.740158in}}%
\pgfpathlineto{\pgfqpoint{3.190333in}{0.739241in}}%
\pgfpathlineto{\pgfqpoint{3.194688in}{0.738327in}}%
\pgfpathlineto{\pgfqpoint{3.199044in}{0.737416in}}%
\pgfpathlineto{\pgfqpoint{3.203399in}{0.736508in}}%
\pgfpathlineto{\pgfqpoint{3.207755in}{0.735602in}}%
\pgfpathlineto{\pgfqpoint{3.212110in}{0.734699in}}%
\pgfpathlineto{\pgfqpoint{3.216466in}{0.733799in}}%
\pgfpathlineto{\pgfqpoint{3.220821in}{0.732901in}}%
\pgfpathlineto{\pgfqpoint{3.225176in}{0.732007in}}%
\pgfpathlineto{\pgfqpoint{3.229532in}{0.731114in}}%
\pgfpathlineto{\pgfqpoint{3.233887in}{0.730225in}}%
\pgfpathlineto{\pgfqpoint{3.238243in}{0.729338in}}%
\pgfpathlineto{\pgfqpoint{3.242598in}{0.728454in}}%
\pgfpathlineto{\pgfqpoint{3.246954in}{0.727572in}}%
\pgfpathlineto{\pgfqpoint{3.251309in}{0.726694in}}%
\pgfpathlineto{\pgfqpoint{3.255665in}{0.725817in}}%
\pgfpathlineto{\pgfqpoint{3.260020in}{0.724944in}}%
\pgfpathlineto{\pgfqpoint{3.264376in}{0.724072in}}%
\pgfpathlineto{\pgfqpoint{3.268731in}{0.723204in}}%
\pgfpathlineto{\pgfqpoint{3.273087in}{0.722338in}}%
\pgfpathlineto{\pgfqpoint{3.277442in}{0.721475in}}%
\pgfpathlineto{\pgfqpoint{3.281797in}{0.720614in}}%
\pgfpathlineto{\pgfqpoint{3.286153in}{0.719755in}}%
\pgfpathlineto{\pgfqpoint{3.290508in}{0.718899in}}%
\pgfpathlineto{\pgfqpoint{3.294864in}{0.718046in}}%
\pgfpathlineto{\pgfqpoint{3.299219in}{0.717195in}}%
\pgfpathlineto{\pgfqpoint{3.303575in}{0.716347in}}%
\pgfpathlineto{\pgfqpoint{3.307930in}{0.715501in}}%
\pgfpathlineto{\pgfqpoint{3.312286in}{0.714658in}}%
\pgfpathlineto{\pgfqpoint{3.316641in}{0.713817in}}%
\pgfpathlineto{\pgfqpoint{3.320997in}{0.712978in}}%
\pgfpathlineto{\pgfqpoint{3.325352in}{0.712142in}}%
\pgfpathlineto{\pgfqpoint{3.329708in}{0.711309in}}%
\pgfpathlineto{\pgfqpoint{3.334063in}{0.710477in}}%
\pgfpathlineto{\pgfqpoint{3.338418in}{0.709649in}}%
\pgfpathlineto{\pgfqpoint{3.342774in}{0.708822in}}%
\pgfpathlineto{\pgfqpoint{3.347129in}{0.707998in}}%
\pgfpathlineto{\pgfqpoint{3.351485in}{0.707177in}}%
\pgfpathlineto{\pgfqpoint{3.355840in}{0.706357in}}%
\pgfpathlineto{\pgfqpoint{3.360196in}{0.705540in}}%
\pgfpathlineto{\pgfqpoint{3.364551in}{0.704726in}}%
\pgfpathlineto{\pgfqpoint{3.368907in}{0.703913in}}%
\pgfpathlineto{\pgfqpoint{3.373262in}{0.703103in}}%
\pgfpathlineto{\pgfqpoint{3.377618in}{0.702296in}}%
\pgfpathlineto{\pgfqpoint{3.381973in}{0.701490in}}%
\pgfpathlineto{\pgfqpoint{3.386328in}{0.700687in}}%
\pgfpathlineto{\pgfqpoint{3.390684in}{0.699886in}}%
\pgfpathlineto{\pgfqpoint{3.395039in}{0.699088in}}%
\pgfpathlineto{\pgfqpoint{3.399395in}{0.698292in}}%
\pgfpathlineto{\pgfqpoint{3.403750in}{0.697498in}}%
\pgfpathlineto{\pgfqpoint{3.408106in}{0.696706in}}%
\pgfpathlineto{\pgfqpoint{3.412461in}{0.695916in}}%
\pgfpathlineto{\pgfqpoint{3.416817in}{0.695129in}}%
\pgfpathlineto{\pgfqpoint{3.421172in}{0.694344in}}%
\pgfpathlineto{\pgfqpoint{3.425528in}{0.693561in}}%
\pgfpathlineto{\pgfqpoint{3.429883in}{0.692780in}}%
\pgfpathlineto{\pgfqpoint{3.434239in}{0.692002in}}%
\pgfpathlineto{\pgfqpoint{3.438594in}{0.691226in}}%
\pgfpathlineto{\pgfqpoint{3.442949in}{0.690451in}}%
\pgfpathlineto{\pgfqpoint{3.447305in}{0.689679in}}%
\pgfpathlineto{\pgfqpoint{3.451660in}{0.688910in}}%
\pgfpathlineto{\pgfqpoint{3.456016in}{0.688142in}}%
\pgfpathlineto{\pgfqpoint{3.460371in}{0.687376in}}%
\pgfpathlineto{\pgfqpoint{3.464727in}{0.686613in}}%
\pgfpathlineto{\pgfqpoint{3.469082in}{0.685852in}}%
\pgfpathlineto{\pgfqpoint{3.473438in}{0.685092in}}%
\pgfpathlineto{\pgfqpoint{3.477793in}{0.684335in}}%
\pgfpathlineto{\pgfqpoint{3.482149in}{0.683580in}}%
\pgfpathlineto{\pgfqpoint{3.486504in}{0.682827in}}%
\pgfpathlineto{\pgfqpoint{3.490860in}{0.682076in}}%
\pgfpathlineto{\pgfqpoint{3.495215in}{0.681327in}}%
\pgfpathlineto{\pgfqpoint{3.499570in}{0.680581in}}%
\pgfpathlineto{\pgfqpoint{3.503926in}{0.679836in}}%
\pgfpathlineto{\pgfqpoint{3.508281in}{0.679093in}}%
\pgfpathlineto{\pgfqpoint{3.512637in}{0.678353in}}%
\pgfpathlineto{\pgfqpoint{3.516992in}{0.677614in}}%
\pgfpathlineto{\pgfqpoint{3.521348in}{0.676877in}}%
\pgfpathlineto{\pgfqpoint{3.525703in}{0.676143in}}%
\pgfpathlineto{\pgfqpoint{3.530059in}{0.675410in}}%
\pgfpathlineto{\pgfqpoint{3.534414in}{0.674679in}}%
\pgfpathlineto{\pgfqpoint{3.538770in}{0.673951in}}%
\pgfpathlineto{\pgfqpoint{3.543125in}{0.673224in}}%
\pgfpathlineto{\pgfqpoint{3.547480in}{0.672499in}}%
\pgfpathlineto{\pgfqpoint{3.551836in}{0.671777in}}%
\pgfpathlineto{\pgfqpoint{3.556191in}{0.671056in}}%
\pgfpathlineto{\pgfqpoint{3.560547in}{0.670337in}}%
\pgfpathlineto{\pgfqpoint{3.564902in}{0.669620in}}%
\pgfpathlineto{\pgfqpoint{3.569258in}{0.668905in}}%
\pgfpathlineto{\pgfqpoint{3.573613in}{0.668192in}}%
\pgfpathlineto{\pgfqpoint{3.577969in}{0.667481in}}%
\pgfpathlineto{\pgfqpoint{3.582324in}{0.666771in}}%
\pgfpathlineto{\pgfqpoint{3.586680in}{0.666064in}}%
\pgfpathlineto{\pgfqpoint{3.591035in}{0.665358in}}%
\pgfpathlineto{\pgfqpoint{3.595391in}{0.664655in}}%
\pgfpathlineto{\pgfqpoint{3.599746in}{0.663953in}}%
\pgfpathlineto{\pgfqpoint{3.604101in}{0.663253in}}%
\pgfpathlineto{\pgfqpoint{3.608457in}{0.662555in}}%
\pgfpathlineto{\pgfqpoint{3.612812in}{0.661859in}}%
\pgfpathlineto{\pgfqpoint{3.617168in}{0.661164in}}%
\pgfpathlineto{\pgfqpoint{3.621523in}{0.660472in}}%
\pgfpathlineto{\pgfqpoint{3.625879in}{0.659781in}}%
\pgfpathlineto{\pgfqpoint{3.630234in}{0.659092in}}%
\pgfpathlineto{\pgfqpoint{3.634590in}{0.658405in}}%
\pgfpathlineto{\pgfqpoint{3.638945in}{0.657720in}}%
\pgfpathlineto{\pgfqpoint{3.643301in}{0.657036in}}%
\pgfpathlineto{\pgfqpoint{3.647656in}{0.656355in}}%
\pgfpathlineto{\pgfqpoint{3.652012in}{0.655675in}}%
\pgfpathlineto{\pgfqpoint{3.656367in}{0.654997in}}%
\pgfpathlineto{\pgfqpoint{3.660722in}{0.654320in}}%
\pgfpathlineto{\pgfqpoint{3.665078in}{0.653646in}}%
\pgfpathlineto{\pgfqpoint{3.669433in}{0.652973in}}%
\pgfpathlineto{\pgfqpoint{3.673789in}{0.652302in}}%
\pgfpathlineto{\pgfqpoint{3.678144in}{0.651633in}}%
\pgfpathlineto{\pgfqpoint{3.682500in}{0.650965in}}%
\pgfpathlineto{\pgfqpoint{3.686855in}{0.650299in}}%
\pgfpathlineto{\pgfqpoint{3.691211in}{0.649635in}}%
\pgfpathlineto{\pgfqpoint{3.695566in}{0.648973in}}%
\pgfpathlineto{\pgfqpoint{3.699922in}{0.648312in}}%
\pgfpathlineto{\pgfqpoint{3.704277in}{0.647653in}}%
\pgfpathlineto{\pgfqpoint{3.708632in}{0.646995in}}%
\pgfpathlineto{\pgfqpoint{3.712988in}{0.646340in}}%
\pgfpathlineto{\pgfqpoint{3.717343in}{0.645686in}}%
\pgfpathlineto{\pgfqpoint{3.721699in}{0.645034in}}%
\pgfpathlineto{\pgfqpoint{3.726054in}{0.644383in}}%
\pgfpathlineto{\pgfqpoint{3.730410in}{0.643734in}}%
\pgfpathlineto{\pgfqpoint{3.734765in}{0.643087in}}%
\pgfpathlineto{\pgfqpoint{3.739121in}{0.642441in}}%
\pgfpathlineto{\pgfqpoint{3.743476in}{0.641797in}}%
\pgfpathlineto{\pgfqpoint{3.747832in}{0.641155in}}%
\pgfpathlineto{\pgfqpoint{3.752187in}{0.640514in}}%
\pgfpathlineto{\pgfqpoint{3.756543in}{0.639875in}}%
\pgfpathlineto{\pgfqpoint{3.760898in}{0.639238in}}%
\pgfpathlineto{\pgfqpoint{3.765253in}{0.638602in}}%
\pgfpathlineto{\pgfqpoint{3.769609in}{0.637967in}}%
\pgfpathlineto{\pgfqpoint{3.773964in}{0.637335in}}%
\pgfpathlineto{\pgfqpoint{3.778320in}{0.636704in}}%
\pgfpathlineto{\pgfqpoint{3.782675in}{0.636074in}}%
\pgfpathlineto{\pgfqpoint{3.787031in}{0.635446in}}%
\pgfpathlineto{\pgfqpoint{3.791386in}{0.634820in}}%
\pgfpathlineto{\pgfqpoint{3.795742in}{0.634195in}}%
\pgfpathlineto{\pgfqpoint{3.800097in}{0.633572in}}%
\pgfpathlineto{\pgfqpoint{3.804453in}{0.632951in}}%
\pgfpathlineto{\pgfqpoint{3.808808in}{0.632331in}}%
\pgfpathlineto{\pgfqpoint{3.813164in}{0.631712in}}%
\pgfpathlineto{\pgfqpoint{3.817519in}{0.631095in}}%
\pgfpathlineto{\pgfqpoint{3.821874in}{0.630480in}}%
\pgfpathlineto{\pgfqpoint{3.826230in}{0.629866in}}%
\pgfpathlineto{\pgfqpoint{3.830585in}{0.629253in}}%
\pgfpathlineto{\pgfqpoint{3.834941in}{0.628643in}}%
\pgfpathlineto{\pgfqpoint{3.839296in}{0.628033in}}%
\pgfpathlineto{\pgfqpoint{3.843652in}{0.627425in}}%
\pgfpathlineto{\pgfqpoint{3.848007in}{0.626819in}}%
\pgfpathlineto{\pgfqpoint{3.852363in}{0.626214in}}%
\pgfpathlineto{\pgfqpoint{3.856718in}{0.625611in}}%
\pgfpathlineto{\pgfqpoint{3.861074in}{0.625009in}}%
\pgfpathlineto{\pgfqpoint{3.865429in}{0.624409in}}%
\pgfpathlineto{\pgfqpoint{3.869784in}{0.623810in}}%
\pgfpathlineto{\pgfqpoint{3.874140in}{0.623212in}}%
\pgfpathlineto{\pgfqpoint{3.878495in}{0.622616in}}%
\pgfpathlineto{\pgfqpoint{3.882851in}{0.622022in}}%
\pgfpathlineto{\pgfqpoint{3.887206in}{0.621429in}}%
\pgfpathlineto{\pgfqpoint{3.891562in}{0.620837in}}%
\pgfpathlineto{\pgfqpoint{3.895917in}{0.620247in}}%
\pgfpathlineto{\pgfqpoint{3.900273in}{0.619658in}}%
\pgfpathlineto{\pgfqpoint{3.904628in}{0.619071in}}%
\pgfpathlineto{\pgfqpoint{3.908984in}{0.618485in}}%
\pgfpathlineto{\pgfqpoint{3.913339in}{0.617901in}}%
\pgfpathlineto{\pgfqpoint{3.917695in}{0.617318in}}%
\pgfpathlineto{\pgfqpoint{3.922050in}{0.616736in}}%
\pgfpathlineto{\pgfqpoint{3.926405in}{0.616156in}}%
\pgfpathlineto{\pgfqpoint{3.930761in}{0.615577in}}%
\pgfpathlineto{\pgfqpoint{3.935116in}{0.615000in}}%
\pgfpathlineto{\pgfqpoint{3.939472in}{0.614424in}}%
\pgfpathlineto{\pgfqpoint{3.943827in}{0.613849in}}%
\pgfpathlineto{\pgfqpoint{3.948183in}{0.613276in}}%
\pgfpathlineto{\pgfqpoint{3.952538in}{0.612704in}}%
\pgfpathlineto{\pgfqpoint{3.956894in}{0.612133in}}%
\pgfpathlineto{\pgfqpoint{3.961249in}{0.611564in}}%
\pgfpathlineto{\pgfqpoint{3.965605in}{0.610996in}}%
\pgfpathlineto{\pgfqpoint{3.969960in}{0.610430in}}%
\pgfpathlineto{\pgfqpoint{3.974316in}{0.609865in}}%
\pgfpathlineto{\pgfqpoint{3.978671in}{0.609301in}}%
\pgfpathlineto{\pgfqpoint{3.983026in}{0.608739in}}%
\pgfpathlineto{\pgfqpoint{3.987382in}{0.608178in}}%
\pgfpathlineto{\pgfqpoint{3.991737in}{0.607618in}}%
\pgfpathlineto{\pgfqpoint{3.996093in}{0.607060in}}%
\pgfpathlineto{\pgfqpoint{4.000448in}{0.606503in}}%
\pgfpathlineto{\pgfqpoint{4.004804in}{0.605947in}}%
\pgfpathlineto{\pgfqpoint{4.009159in}{0.605393in}}%
\pgfpathlineto{\pgfqpoint{4.013515in}{0.604839in}}%
\pgfpathlineto{\pgfqpoint{4.017870in}{0.604288in}}%
\pgfpathlineto{\pgfqpoint{4.022226in}{0.603737in}}%
\pgfpathlineto{\pgfqpoint{4.026581in}{0.603188in}}%
\pgfpathlineto{\pgfqpoint{4.030936in}{0.602640in}}%
\pgfpathlineto{\pgfqpoint{4.035292in}{0.602093in}}%
\pgfpathlineto{\pgfqpoint{4.039647in}{0.601548in}}%
\pgfpathlineto{\pgfqpoint{4.044003in}{0.601004in}}%
\pgfpathlineto{\pgfqpoint{4.048358in}{0.600461in}}%
\pgfpathlineto{\pgfqpoint{4.052714in}{0.599919in}}%
\pgfpathlineto{\pgfqpoint{4.057069in}{0.599379in}}%
\pgfpathlineto{\pgfqpoint{4.061425in}{0.598840in}}%
\pgfpathlineto{\pgfqpoint{4.065780in}{0.598302in}}%
\pgfpathlineto{\pgfqpoint{4.070136in}{0.597766in}}%
\pgfpathlineto{\pgfqpoint{4.074491in}{0.597231in}}%
\pgfpathlineto{\pgfqpoint{4.078847in}{0.596697in}}%
\pgfpathlineto{\pgfqpoint{4.083202in}{0.596164in}}%
\pgfpathlineto{\pgfqpoint{4.087557in}{0.595632in}}%
\pgfpathlineto{\pgfqpoint{4.091913in}{0.595102in}}%
\pgfpathlineto{\pgfqpoint{4.096268in}{0.594573in}}%
\pgfpathlineto{\pgfqpoint{4.100624in}{0.594045in}}%
\pgfpathlineto{\pgfqpoint{4.104979in}{0.593518in}}%
\pgfpathlineto{\pgfqpoint{4.109335in}{0.592993in}}%
\pgfpathlineto{\pgfqpoint{4.113690in}{0.592469in}}%
\pgfpathlineto{\pgfqpoint{4.118046in}{0.591946in}}%
\pgfpathlineto{\pgfqpoint{4.122401in}{0.591424in}}%
\pgfpathlineto{\pgfqpoint{4.126757in}{0.590903in}}%
\pgfpathlineto{\pgfqpoint{4.131112in}{0.590384in}}%
\pgfpathlineto{\pgfqpoint{4.135468in}{0.589865in}}%
\pgfpathlineto{\pgfqpoint{4.139823in}{0.589348in}}%
\pgfpathlineto{\pgfqpoint{4.144178in}{0.588832in}}%
\pgfpathlineto{\pgfqpoint{4.148534in}{0.588318in}}%
\pgfpathlineto{\pgfqpoint{4.152889in}{0.587804in}}%
\pgfpathlineto{\pgfqpoint{4.157245in}{0.587292in}}%
\pgfpathlineto{\pgfqpoint{4.161600in}{0.586780in}}%
\pgfpathlineto{\pgfqpoint{4.165956in}{0.586270in}}%
\pgfpathlineto{\pgfqpoint{4.170311in}{0.585761in}}%
\pgfpathlineto{\pgfqpoint{4.174667in}{0.585254in}}%
\pgfpathlineto{\pgfqpoint{4.179022in}{0.584747in}}%
\pgfpathlineto{\pgfqpoint{4.183378in}{0.584242in}}%
\pgfpathlineto{\pgfqpoint{4.187733in}{0.583737in}}%
\pgfpathlineto{\pgfqpoint{4.192088in}{0.583234in}}%
\pgfpathlineto{\pgfqpoint{4.196444in}{0.582732in}}%
\pgfpathlineto{\pgfqpoint{4.200799in}{0.582231in}}%
\pgfpathlineto{\pgfqpoint{4.205155in}{0.581731in}}%
\pgfpathlineto{\pgfqpoint{4.209510in}{0.581232in}}%
\pgfpathlineto{\pgfqpoint{4.213866in}{0.580735in}}%
\pgfpathlineto{\pgfqpoint{4.218221in}{0.580238in}}%
\pgfpathlineto{\pgfqpoint{4.222577in}{0.579743in}}%
\pgfpathlineto{\pgfqpoint{4.226932in}{0.579249in}}%
\pgfpathlineto{\pgfqpoint{4.231288in}{0.578756in}}%
\pgfpathlineto{\pgfqpoint{4.235643in}{0.578264in}}%
\pgfpathlineto{\pgfqpoint{4.239999in}{0.577773in}}%
\pgfpathlineto{\pgfqpoint{4.244354in}{0.577283in}}%
\pgfpathlineto{\pgfqpoint{4.248709in}{0.576794in}}%
\pgfpathlineto{\pgfqpoint{4.253065in}{0.576306in}}%
\pgfpathlineto{\pgfqpoint{4.257420in}{0.575820in}}%
\pgfpathlineto{\pgfqpoint{4.261776in}{0.575334in}}%
\pgfpathlineto{\pgfqpoint{4.266131in}{0.574850in}}%
\pgfpathlineto{\pgfqpoint{4.270487in}{0.574366in}}%
\pgfpathlineto{\pgfqpoint{4.274842in}{0.573884in}}%
\pgfpathlineto{\pgfqpoint{4.279198in}{0.573403in}}%
\pgfpathlineto{\pgfqpoint{4.283553in}{0.572922in}}%
\pgfpathlineto{\pgfqpoint{4.287909in}{0.572443in}}%
\pgfpathlineto{\pgfqpoint{4.292264in}{0.571965in}}%
\pgfpathlineto{\pgfqpoint{4.296620in}{0.571488in}}%
\pgfpathlineto{\pgfqpoint{4.300975in}{0.571012in}}%
\pgfpathlineto{\pgfqpoint{4.305330in}{0.570537in}}%
\pgfpathlineto{\pgfqpoint{4.309686in}{0.570063in}}%
\pgfpathlineto{\pgfqpoint{4.314041in}{0.569590in}}%
\pgfpathlineto{\pgfqpoint{4.318397in}{0.569118in}}%
\pgfpathlineto{\pgfqpoint{4.322752in}{0.568648in}}%
\pgfpathlineto{\pgfqpoint{4.327108in}{0.568178in}}%
\pgfpathlineto{\pgfqpoint{4.331463in}{0.567709in}}%
\pgfpathlineto{\pgfqpoint{4.335819in}{0.567241in}}%
\pgfpathlineto{\pgfqpoint{4.340174in}{0.566775in}}%
\pgfpathlineto{\pgfqpoint{4.344530in}{0.566309in}}%
\pgfpathlineto{\pgfqpoint{4.348885in}{0.565844in}}%
\pgfpathlineto{\pgfqpoint{4.353240in}{0.565380in}}%
\pgfpathlineto{\pgfqpoint{4.357596in}{0.564918in}}%
\pgfpathlineto{\pgfqpoint{4.361951in}{0.564456in}}%
\pgfpathlineto{\pgfqpoint{4.366307in}{0.563995in}}%
\pgfpathlineto{\pgfqpoint{4.370662in}{0.563535in}}%
\pgfpathlineto{\pgfqpoint{4.375018in}{0.563077in}}%
\pgfpathlineto{\pgfqpoint{4.379373in}{0.562619in}}%
\pgfpathlineto{\pgfqpoint{4.383729in}{0.562162in}}%
\pgfpathlineto{\pgfqpoint{4.388084in}{0.561706in}}%
\pgfpathlineto{\pgfqpoint{4.392440in}{0.561251in}}%
\pgfpathlineto{\pgfqpoint{4.396795in}{0.560797in}}%
\pgfpathlineto{\pgfqpoint{4.401151in}{0.560345in}}%
\pgfpathlineto{\pgfqpoint{4.405506in}{0.559893in}}%
\pgfpathlineto{\pgfqpoint{4.409861in}{0.559442in}}%
\pgfpathlineto{\pgfqpoint{4.414217in}{0.558992in}}%
\pgfpathlineto{\pgfqpoint{4.418572in}{0.558543in}}%
\pgfpathlineto{\pgfqpoint{4.422928in}{0.558095in}}%
\pgfpathlineto{\pgfqpoint{4.427283in}{0.557647in}}%
\pgfpathlineto{\pgfqpoint{4.431639in}{0.557201in}}%
\pgfpathlineto{\pgfqpoint{4.435994in}{0.556756in}}%
\pgfpathlineto{\pgfqpoint{4.440350in}{0.556312in}}%
\pgfpathlineto{\pgfqpoint{4.444705in}{0.555868in}}%
\pgfpathlineto{\pgfqpoint{4.449061in}{0.555426in}}%
\pgfpathlineto{\pgfqpoint{4.453416in}{0.554984in}}%
\pgfpathlineto{\pgfqpoint{4.457772in}{0.554544in}}%
\pgfpathlineto{\pgfqpoint{4.462127in}{0.554104in}}%
\pgfpathlineto{\pgfqpoint{4.466482in}{0.553665in}}%
\pgfpathlineto{\pgfqpoint{4.470838in}{0.553228in}}%
\pgfpathlineto{\pgfqpoint{4.475193in}{0.552791in}}%
\pgfpathlineto{\pgfqpoint{4.479549in}{0.552355in}}%
\pgfpathlineto{\pgfqpoint{4.483904in}{0.551920in}}%
\pgfpathlineto{\pgfqpoint{4.488260in}{0.551486in}}%
\pgfpathlineto{\pgfqpoint{4.492615in}{0.551052in}}%
\pgfpathlineto{\pgfqpoint{4.496971in}{0.550620in}}%
\pgfpathlineto{\pgfqpoint{4.501326in}{0.550189in}}%
\pgfpathlineto{\pgfqpoint{4.505682in}{0.549758in}}%
\pgfpathlineto{\pgfqpoint{4.510037in}{0.549329in}}%
\pgfpathlineto{\pgfqpoint{4.514392in}{0.548900in}}%
\pgfpathlineto{\pgfqpoint{4.518748in}{0.548472in}}%
\pgfpathlineto{\pgfqpoint{4.523103in}{0.548045in}}%
\pgfpathlineto{\pgfqpoint{4.527459in}{0.547619in}}%
\pgfpathlineto{\pgfqpoint{4.531814in}{0.547194in}}%
\pgfpathlineto{\pgfqpoint{4.536170in}{0.546769in}}%
\pgfpathlineto{\pgfqpoint{4.540525in}{0.546346in}}%
\pgfpathlineto{\pgfqpoint{4.544881in}{0.545923in}}%
\pgfpathlineto{\pgfqpoint{4.549236in}{0.545502in}}%
\pgfpathlineto{\pgfqpoint{4.553592in}{0.545081in}}%
\pgfpathlineto{\pgfqpoint{4.557947in}{0.544661in}}%
\pgfpathlineto{\pgfqpoint{4.562303in}{0.544242in}}%
\pgfpathlineto{\pgfqpoint{4.566658in}{0.543824in}}%
\pgfpathlineto{\pgfqpoint{4.571013in}{0.543406in}}%
\pgfpathlineto{\pgfqpoint{4.575369in}{0.542990in}}%
\pgfpathlineto{\pgfqpoint{4.579724in}{0.542574in}}%
\pgfpathlineto{\pgfqpoint{4.584080in}{0.542159in}}%
\pgfpathlineto{\pgfqpoint{4.588435in}{0.541745in}}%
\pgfpathlineto{\pgfqpoint{4.592791in}{0.541332in}}%
\pgfpathlineto{\pgfqpoint{4.597146in}{0.540920in}}%
\pgfpathlineto{\pgfqpoint{4.601502in}{0.540508in}}%
\pgfpathlineto{\pgfqpoint{4.605857in}{0.540098in}}%
\pgfpathlineto{\pgfqpoint{4.610213in}{0.539688in}}%
\pgfpathlineto{\pgfqpoint{4.614568in}{0.539279in}}%
\pgfpathlineto{\pgfqpoint{4.618924in}{0.538871in}}%
\pgfpathlineto{\pgfqpoint{4.623279in}{0.538464in}}%
\pgfpathlineto{\pgfqpoint{4.627634in}{0.538057in}}%
\pgfpathlineto{\pgfqpoint{4.631990in}{0.537652in}}%
\pgfpathlineto{\pgfqpoint{4.636345in}{0.537247in}}%
\pgfpathlineto{\pgfqpoint{4.640701in}{0.536843in}}%
\pgfpathlineto{\pgfqpoint{4.645056in}{0.536440in}}%
\pgfpathlineto{\pgfqpoint{4.649412in}{0.536037in}}%
\pgfpathlineto{\pgfqpoint{4.653767in}{0.535636in}}%
\pgfpathlineto{\pgfqpoint{4.658123in}{0.535235in}}%
\pgfpathlineto{\pgfqpoint{4.662478in}{0.534835in}}%
\pgfpathlineto{\pgfqpoint{4.666834in}{0.534436in}}%
\pgfpathlineto{\pgfqpoint{4.671189in}{0.534037in}}%
\pgfpathlineto{\pgfqpoint{4.675544in}{0.533640in}}%
\pgfpathlineto{\pgfqpoint{4.679900in}{0.533243in}}%
\pgfpathlineto{\pgfqpoint{4.684255in}{0.532847in}}%
\pgfpathlineto{\pgfqpoint{4.688611in}{0.532452in}}%
\pgfpathlineto{\pgfqpoint{4.692966in}{0.532058in}}%
\pgfpathlineto{\pgfqpoint{4.697322in}{0.531664in}}%
\pgfpathlineto{\pgfqpoint{4.701677in}{0.531271in}}%
\pgfpathlineto{\pgfqpoint{4.706033in}{0.530879in}}%
\pgfpathlineto{\pgfqpoint{4.710388in}{0.530488in}}%
\pgfpathlineto{\pgfqpoint{4.714744in}{0.530097in}}%
\pgfpathlineto{\pgfqpoint{4.719099in}{0.529708in}}%
\pgfpathlineto{\pgfqpoint{4.723455in}{0.529319in}}%
\pgfpathlineto{\pgfqpoint{4.727810in}{0.528931in}}%
\pgfpathlineto{\pgfqpoint{4.732165in}{0.528543in}}%
\pgfpathlineto{\pgfqpoint{4.736521in}{0.528157in}}%
\pgfpathlineto{\pgfqpoint{4.740876in}{0.527771in}}%
\pgfpathlineto{\pgfqpoint{4.745232in}{0.527386in}}%
\pgfpathlineto{\pgfqpoint{4.749587in}{0.527001in}}%
\pgfpathlineto{\pgfqpoint{4.753943in}{0.526618in}}%
\pgfpathlineto{\pgfqpoint{4.758298in}{0.526235in}}%
\pgfpathlineto{\pgfqpoint{4.762654in}{0.525853in}}%
\pgfpathlineto{\pgfqpoint{4.767009in}{0.525471in}}%
\pgfpathlineto{\pgfqpoint{4.771365in}{0.525091in}}%
\pgfpathlineto{\pgfqpoint{4.775720in}{0.524711in}}%
\pgfpathlineto{\pgfqpoint{4.780076in}{0.524332in}}%
\pgfpathlineto{\pgfqpoint{4.784431in}{0.523953in}}%
\pgfpathlineto{\pgfqpoint{4.788786in}{0.523576in}}%
\pgfpathlineto{\pgfqpoint{4.793142in}{0.523199in}}%
\pgfpathlineto{\pgfqpoint{4.797497in}{0.522823in}}%
\pgfpathlineto{\pgfqpoint{4.801853in}{0.522447in}}%
\pgfpathlineto{\pgfqpoint{4.806208in}{0.522072in}}%
\pgfpathlineto{\pgfqpoint{4.810564in}{0.521699in}}%
\pgfpathlineto{\pgfqpoint{4.814919in}{0.521325in}}%
\pgfpathlineto{\pgfqpoint{4.819275in}{0.520953in}}%
\pgfpathlineto{\pgfqpoint{4.823630in}{0.520581in}}%
\pgfpathlineto{\pgfqpoint{4.827986in}{0.520210in}}%
\pgfpathlineto{\pgfqpoint{4.832341in}{0.519839in}}%
\pgfpathlineto{\pgfqpoint{4.836696in}{0.519470in}}%
\pgfpathlineto{\pgfqpoint{4.841052in}{0.519101in}}%
\pgfpathlineto{\pgfqpoint{4.845407in}{0.518733in}}%
\pgfpathlineto{\pgfqpoint{4.849763in}{0.518365in}}%
\pgfpathlineto{\pgfqpoint{4.854118in}{0.517998in}}%
\pgfpathlineto{\pgfqpoint{4.858474in}{0.517632in}}%
\pgfpathlineto{\pgfqpoint{4.862829in}{0.517267in}}%
\pgfpathlineto{\pgfqpoint{4.867185in}{0.516902in}}%
\pgfpathlineto{\pgfqpoint{4.871540in}{0.516538in}}%
\pgfpathlineto{\pgfqpoint{4.875896in}{0.516175in}}%
\pgfpathlineto{\pgfqpoint{4.880251in}{0.515812in}}%
\pgfpathlineto{\pgfqpoint{4.884607in}{0.515450in}}%
\pgfpathlineto{\pgfqpoint{4.888962in}{0.515089in}}%
\pgfpathlineto{\pgfqpoint{4.893317in}{0.514728in}}%
\pgfpathlineto{\pgfqpoint{4.897673in}{0.514369in}}%
\pgfpathlineto{\pgfqpoint{4.902028in}{0.514009in}}%
\pgfpathlineto{\pgfqpoint{4.906384in}{0.513651in}}%
\pgfpathlineto{\pgfqpoint{4.910739in}{0.513293in}}%
\pgfpathlineto{\pgfqpoint{4.915095in}{0.512936in}}%
\pgfpathlineto{\pgfqpoint{4.919450in}{0.512580in}}%
\pgfpathlineto{\pgfqpoint{4.923806in}{0.512224in}}%
\pgfpathlineto{\pgfqpoint{4.928161in}{0.511869in}}%
\pgfpathlineto{\pgfqpoint{4.932517in}{0.511514in}}%
\pgfpathlineto{\pgfqpoint{4.936872in}{0.511160in}}%
\pgfpathlineto{\pgfqpoint{4.941228in}{0.510807in}}%
\pgfpathlineto{\pgfqpoint{4.945583in}{0.510455in}}%
\pgfpathlineto{\pgfqpoint{4.949938in}{0.510103in}}%
\pgfpathlineto{\pgfqpoint{4.954294in}{0.509752in}}%
\pgfpathlineto{\pgfqpoint{4.958649in}{0.509402in}}%
\pgfpathlineto{\pgfqpoint{4.963005in}{0.509052in}}%
\pgfpathlineto{\pgfqpoint{4.967360in}{0.508703in}}%
\pgfpathlineto{\pgfqpoint{4.971716in}{0.508354in}}%
\pgfpathlineto{\pgfqpoint{4.976071in}{0.508006in}}%
\pgfpathlineto{\pgfqpoint{4.980427in}{0.507659in}}%
\pgfpathlineto{\pgfqpoint{4.984782in}{0.507313in}}%
\pgfpathlineto{\pgfqpoint{4.989138in}{0.506967in}}%
\pgfpathlineto{\pgfqpoint{4.993493in}{0.506621in}}%
\pgfpathlineto{\pgfqpoint{4.997848in}{0.506277in}}%
\pgfpathlineto{\pgfqpoint{5.002204in}{0.505933in}}%
\pgfpathlineto{\pgfqpoint{5.006559in}{0.505590in}}%
\pgfpathlineto{\pgfqpoint{5.010915in}{0.505247in}}%
\pgfpathlineto{\pgfqpoint{5.015270in}{0.504905in}}%
\pgfpathlineto{\pgfqpoint{5.019626in}{0.504564in}}%
\pgfpathlineto{\pgfqpoint{5.023981in}{0.504223in}}%
\pgfpathlineto{\pgfqpoint{5.028337in}{0.503883in}}%
\pgfpathlineto{\pgfqpoint{5.032692in}{0.503543in}}%
\pgfpathlineto{\pgfqpoint{5.037048in}{0.503204in}}%
\pgfpathlineto{\pgfqpoint{5.041403in}{0.502866in}}%
\pgfpathlineto{\pgfqpoint{5.045759in}{0.502528in}}%
\pgfpathlineto{\pgfqpoint{5.050114in}{0.502191in}}%
\pgfpathlineto{\pgfqpoint{5.054469in}{0.501855in}}%
\pgfpathlineto{\pgfqpoint{5.058825in}{0.501519in}}%
\pgfpathlineto{\pgfqpoint{5.063180in}{0.501184in}}%
\pgfpathlineto{\pgfqpoint{5.067536in}{0.500849in}}%
\pgfpathlineto{\pgfqpoint{5.071891in}{0.500516in}}%
\pgfpathlineto{\pgfqpoint{5.076247in}{0.500182in}}%
\pgfpathlineto{\pgfqpoint{5.080602in}{0.499850in}}%
\pgfpathlineto{\pgfqpoint{5.084958in}{0.499517in}}%
\pgfpathlineto{\pgfqpoint{5.089313in}{0.499186in}}%
\pgfpathlineto{\pgfqpoint{5.093669in}{0.498855in}}%
\pgfpathlineto{\pgfqpoint{5.098024in}{0.498525in}}%
\pgfpathlineto{\pgfqpoint{5.102380in}{0.498195in}}%
\pgfpathlineto{\pgfqpoint{5.102380in}{0.498195in}}%
\pgfpathlineto{\pgfqpoint{5.109574in}{0.497662in}}%
\pgfpathlineto{\pgfqpoint{5.116768in}{0.497149in}}%
\pgfpathlineto{\pgfqpoint{5.123962in}{0.496655in}}%
\pgfpathlineto{\pgfqpoint{5.131156in}{0.496179in}}%
\pgfpathlineto{\pgfqpoint{5.138351in}{0.495719in}}%
\pgfpathlineto{\pgfqpoint{5.145545in}{0.495275in}}%
\pgfpathlineto{\pgfqpoint{5.152739in}{0.494845in}}%
\pgfpathlineto{\pgfqpoint{5.159933in}{0.494429in}}%
\pgfpathlineto{\pgfqpoint{5.167128in}{0.494026in}}%
\pgfpathlineto{\pgfqpoint{5.174322in}{0.493635in}}%
\pgfpathlineto{\pgfqpoint{5.181516in}{0.493255in}}%
\pgfpathlineto{\pgfqpoint{5.188710in}{0.492886in}}%
\pgfpathlineto{\pgfqpoint{5.195905in}{0.492527in}}%
\pgfpathlineto{\pgfqpoint{5.203099in}{0.492178in}}%
\pgfpathlineto{\pgfqpoint{5.210293in}{0.491838in}}%
\pgfpathlineto{\pgfqpoint{5.217487in}{0.491507in}}%
\pgfpathlineto{\pgfqpoint{5.224682in}{0.491184in}}%
\pgfpathlineto{\pgfqpoint{5.231876in}{0.490869in}}%
\pgfpathlineto{\pgfqpoint{5.239070in}{0.490562in}}%
\pgfpathlineto{\pgfqpoint{5.246264in}{0.490262in}}%
\pgfpathlineto{\pgfqpoint{5.253458in}{0.489970in}}%
\pgfpathlineto{\pgfqpoint{5.260653in}{0.489684in}}%
\pgfpathlineto{\pgfqpoint{5.267847in}{0.489404in}}%
\pgfpathlineto{\pgfqpoint{5.275041in}{0.489131in}}%
\pgfpathlineto{\pgfqpoint{5.282235in}{0.488863in}}%
\pgfpathlineto{\pgfqpoint{5.289430in}{0.488601in}}%
\pgfpathlineto{\pgfqpoint{5.296624in}{0.488345in}}%
\pgfpathlineto{\pgfqpoint{5.303818in}{0.488094in}}%
\pgfpathlineto{\pgfqpoint{5.311012in}{0.487848in}}%
\pgfpathlineto{\pgfqpoint{5.318207in}{0.487607in}}%
\pgfpathlineto{\pgfqpoint{5.325401in}{0.487371in}}%
\pgfpathlineto{\pgfqpoint{5.332595in}{0.487139in}}%
\pgfpathlineto{\pgfqpoint{5.339789in}{0.486912in}}%
\pgfpathlineto{\pgfqpoint{5.346984in}{0.486689in}}%
\pgfpathlineto{\pgfqpoint{5.354178in}{0.486470in}}%
\pgfpathlineto{\pgfqpoint{5.361372in}{0.486255in}}%
\pgfpathlineto{\pgfqpoint{5.368566in}{0.486044in}}%
\pgfpathlineto{\pgfqpoint{5.375760in}{0.485837in}}%
\pgfpathlineto{\pgfqpoint{5.382955in}{0.485633in}}%
\pgfpathlineto{\pgfqpoint{5.390149in}{0.485433in}}%
\pgfpathlineto{\pgfqpoint{5.397343in}{0.485236in}}%
\pgfpathlineto{\pgfqpoint{5.404537in}{0.485043in}}%
\pgfpathlineto{\pgfqpoint{5.411732in}{0.484853in}}%
\pgfpathlineto{\pgfqpoint{5.418926in}{0.484666in}}%
\pgfpathlineto{\pgfqpoint{5.426120in}{0.484482in}}%
\pgfpathlineto{\pgfqpoint{5.433314in}{0.484301in}}%
\pgfpathlineto{\pgfqpoint{5.440509in}{0.484122in}}%
\pgfpathlineto{\pgfqpoint{5.447703in}{0.483947in}}%
\pgfpathlineto{\pgfqpoint{5.454897in}{0.483775in}}%
\pgfpathlineto{\pgfqpoint{5.462091in}{0.483605in}}%
\pgfpathlineto{\pgfqpoint{5.469286in}{0.483437in}}%
\pgfpathlineto{\pgfqpoint{5.476480in}{0.483272in}}%
\pgfpathlineto{\pgfqpoint{5.483674in}{0.483110in}}%
\pgfpathlineto{\pgfqpoint{5.490868in}{0.482950in}}%
\pgfpathlineto{\pgfqpoint{5.498062in}{0.482792in}}%
\pgfpathlineto{\pgfqpoint{5.505257in}{0.482637in}}%
\pgfpathlineto{\pgfqpoint{5.512451in}{0.482483in}}%
\pgfpathlineto{\pgfqpoint{5.519645in}{0.482332in}}%
\pgfpathlineto{\pgfqpoint{5.526839in}{0.482183in}}%
\pgfpathlineto{\pgfqpoint{5.534034in}{0.482036in}}%
\pgfpathlineto{\pgfqpoint{5.541228in}{0.481892in}}%
\pgfpathlineto{\pgfqpoint{5.548422in}{0.481749in}}%
\pgfpathlineto{\pgfqpoint{5.555616in}{0.481608in}}%
\pgfpathlineto{\pgfqpoint{5.562811in}{0.481469in}}%
\pgfpathlineto{\pgfqpoint{5.570005in}{0.481331in}}%
\pgfpathlineto{\pgfqpoint{5.577199in}{0.481196in}}%
\pgfpathlineto{\pgfqpoint{5.584393in}{0.481062in}}%
\pgfpathlineto{\pgfqpoint{5.591588in}{0.480930in}}%
\pgfpathlineto{\pgfqpoint{5.598782in}{0.480800in}}%
\pgfpathlineto{\pgfqpoint{5.605976in}{0.480671in}}%
\pgfpathlineto{\pgfqpoint{5.613170in}{0.480544in}}%
\pgfpathlineto{\pgfqpoint{5.620364in}{0.480419in}}%
\pgfpathlineto{\pgfqpoint{5.627559in}{0.480295in}}%
\pgfpathlineto{\pgfqpoint{5.634753in}{0.480172in}}%
\pgfpathlineto{\pgfqpoint{5.641947in}{0.480051in}}%
\pgfpathlineto{\pgfqpoint{5.649141in}{0.479932in}}%
\pgfpathlineto{\pgfqpoint{5.656336in}{0.479814in}}%
\pgfpathlineto{\pgfqpoint{5.663530in}{0.479697in}}%
\pgfpathlineto{\pgfqpoint{5.670724in}{0.479582in}}%
\pgfpathlineto{\pgfqpoint{5.677918in}{0.479468in}}%
\pgfpathlineto{\pgfqpoint{5.685113in}{0.479355in}}%
\pgfpathlineto{\pgfqpoint{5.692307in}{0.479244in}}%
\pgfpathlineto{\pgfqpoint{5.699501in}{0.479134in}}%
\pgfpathlineto{\pgfqpoint{5.706695in}{0.479025in}}%
\pgfpathlineto{\pgfqpoint{5.713890in}{0.478918in}}%
\pgfpathlineto{\pgfqpoint{5.721084in}{0.478811in}}%
\pgfpathlineto{\pgfqpoint{5.728278in}{0.478706in}}%
\pgfpathlineto{\pgfqpoint{5.735472in}{0.478602in}}%
\pgfpathlineto{\pgfqpoint{5.742667in}{0.478499in}}%
\pgfpathlineto{\pgfqpoint{5.749861in}{0.478397in}}%
\pgfpathlineto{\pgfqpoint{5.757055in}{0.478297in}}%
\pgfpathlineto{\pgfqpoint{5.764249in}{0.478197in}}%
\pgfpathlineto{\pgfqpoint{5.771443in}{0.478098in}}%
\pgfpathlineto{\pgfqpoint{5.778638in}{0.478001in}}%
\pgfpathlineto{\pgfqpoint{5.785832in}{0.477904in}}%
\pgfpathlineto{\pgfqpoint{5.793026in}{0.477809in}}%
\pgfpathlineto{\pgfqpoint{5.800220in}{0.477714in}}%
\pgfpathlineto{\pgfqpoint{5.807415in}{0.477621in}}%
\pgfpathlineto{\pgfqpoint{5.814609in}{0.477528in}}%
\pgfpathlineto{\pgfqpoint{5.814609in}{0.136829in}}%
\pgfpathlineto{\pgfqpoint{5.814609in}{0.136829in}}%
\pgfpathlineto{\pgfqpoint{5.807415in}{0.136829in}}%
\pgfpathlineto{\pgfqpoint{5.800220in}{0.136829in}}%
\pgfpathlineto{\pgfqpoint{5.793026in}{0.136829in}}%
\pgfpathlineto{\pgfqpoint{5.785832in}{0.136829in}}%
\pgfpathlineto{\pgfqpoint{5.778638in}{0.136829in}}%
\pgfpathlineto{\pgfqpoint{5.771443in}{0.136829in}}%
\pgfpathlineto{\pgfqpoint{5.764249in}{0.136829in}}%
\pgfpathlineto{\pgfqpoint{5.757055in}{0.136829in}}%
\pgfpathlineto{\pgfqpoint{5.749861in}{0.136829in}}%
\pgfpathlineto{\pgfqpoint{5.742667in}{0.136829in}}%
\pgfpathlineto{\pgfqpoint{5.735472in}{0.136829in}}%
\pgfpathlineto{\pgfqpoint{5.728278in}{0.136829in}}%
\pgfpathlineto{\pgfqpoint{5.721084in}{0.136829in}}%
\pgfpathlineto{\pgfqpoint{5.713890in}{0.136829in}}%
\pgfpathlineto{\pgfqpoint{5.706695in}{0.136829in}}%
\pgfpathlineto{\pgfqpoint{5.699501in}{0.136829in}}%
\pgfpathlineto{\pgfqpoint{5.692307in}{0.136829in}}%
\pgfpathlineto{\pgfqpoint{5.685113in}{0.136829in}}%
\pgfpathlineto{\pgfqpoint{5.677918in}{0.136829in}}%
\pgfpathlineto{\pgfqpoint{5.670724in}{0.136829in}}%
\pgfpathlineto{\pgfqpoint{5.663530in}{0.136829in}}%
\pgfpathlineto{\pgfqpoint{5.656336in}{0.136829in}}%
\pgfpathlineto{\pgfqpoint{5.649141in}{0.136829in}}%
\pgfpathlineto{\pgfqpoint{5.641947in}{0.136829in}}%
\pgfpathlineto{\pgfqpoint{5.634753in}{0.136829in}}%
\pgfpathlineto{\pgfqpoint{5.627559in}{0.136829in}}%
\pgfpathlineto{\pgfqpoint{5.620364in}{0.136829in}}%
\pgfpathlineto{\pgfqpoint{5.613170in}{0.136829in}}%
\pgfpathlineto{\pgfqpoint{5.605976in}{0.136829in}}%
\pgfpathlineto{\pgfqpoint{5.598782in}{0.136829in}}%
\pgfpathlineto{\pgfqpoint{5.591588in}{0.136829in}}%
\pgfpathlineto{\pgfqpoint{5.584393in}{0.136829in}}%
\pgfpathlineto{\pgfqpoint{5.577199in}{0.136829in}}%
\pgfpathlineto{\pgfqpoint{5.570005in}{0.136829in}}%
\pgfpathlineto{\pgfqpoint{5.562811in}{0.136829in}}%
\pgfpathlineto{\pgfqpoint{5.555616in}{0.136829in}}%
\pgfpathlineto{\pgfqpoint{5.548422in}{0.136829in}}%
\pgfpathlineto{\pgfqpoint{5.541228in}{0.136829in}}%
\pgfpathlineto{\pgfqpoint{5.534034in}{0.136829in}}%
\pgfpathlineto{\pgfqpoint{5.526839in}{0.136829in}}%
\pgfpathlineto{\pgfqpoint{5.519645in}{0.136829in}}%
\pgfpathlineto{\pgfqpoint{5.512451in}{0.136829in}}%
\pgfpathlineto{\pgfqpoint{5.505257in}{0.136829in}}%
\pgfpathlineto{\pgfqpoint{5.498062in}{0.136829in}}%
\pgfpathlineto{\pgfqpoint{5.490868in}{0.136829in}}%
\pgfpathlineto{\pgfqpoint{5.483674in}{0.136829in}}%
\pgfpathlineto{\pgfqpoint{5.476480in}{0.136829in}}%
\pgfpathlineto{\pgfqpoint{5.469286in}{0.136829in}}%
\pgfpathlineto{\pgfqpoint{5.462091in}{0.136829in}}%
\pgfpathlineto{\pgfqpoint{5.454897in}{0.136829in}}%
\pgfpathlineto{\pgfqpoint{5.447703in}{0.136829in}}%
\pgfpathlineto{\pgfqpoint{5.440509in}{0.136829in}}%
\pgfpathlineto{\pgfqpoint{5.433314in}{0.136829in}}%
\pgfpathlineto{\pgfqpoint{5.426120in}{0.136829in}}%
\pgfpathlineto{\pgfqpoint{5.418926in}{0.136829in}}%
\pgfpathlineto{\pgfqpoint{5.411732in}{0.136829in}}%
\pgfpathlineto{\pgfqpoint{5.404537in}{0.136829in}}%
\pgfpathlineto{\pgfqpoint{5.397343in}{0.136829in}}%
\pgfpathlineto{\pgfqpoint{5.390149in}{0.136829in}}%
\pgfpathlineto{\pgfqpoint{5.382955in}{0.136829in}}%
\pgfpathlineto{\pgfqpoint{5.375760in}{0.136829in}}%
\pgfpathlineto{\pgfqpoint{5.368566in}{0.136829in}}%
\pgfpathlineto{\pgfqpoint{5.361372in}{0.136829in}}%
\pgfpathlineto{\pgfqpoint{5.354178in}{0.136829in}}%
\pgfpathlineto{\pgfqpoint{5.346984in}{0.136829in}}%
\pgfpathlineto{\pgfqpoint{5.339789in}{0.136829in}}%
\pgfpathlineto{\pgfqpoint{5.332595in}{0.136829in}}%
\pgfpathlineto{\pgfqpoint{5.325401in}{0.136829in}}%
\pgfpathlineto{\pgfqpoint{5.318207in}{0.136829in}}%
\pgfpathlineto{\pgfqpoint{5.311012in}{0.136829in}}%
\pgfpathlineto{\pgfqpoint{5.303818in}{0.136829in}}%
\pgfpathlineto{\pgfqpoint{5.296624in}{0.136829in}}%
\pgfpathlineto{\pgfqpoint{5.289430in}{0.136829in}}%
\pgfpathlineto{\pgfqpoint{5.282235in}{0.136829in}}%
\pgfpathlineto{\pgfqpoint{5.275041in}{0.136829in}}%
\pgfpathlineto{\pgfqpoint{5.267847in}{0.136829in}}%
\pgfpathlineto{\pgfqpoint{5.260653in}{0.136829in}}%
\pgfpathlineto{\pgfqpoint{5.253458in}{0.136829in}}%
\pgfpathlineto{\pgfqpoint{5.246264in}{0.136829in}}%
\pgfpathlineto{\pgfqpoint{5.239070in}{0.136829in}}%
\pgfpathlineto{\pgfqpoint{5.231876in}{0.136829in}}%
\pgfpathlineto{\pgfqpoint{5.224682in}{0.136829in}}%
\pgfpathlineto{\pgfqpoint{5.217487in}{0.136829in}}%
\pgfpathlineto{\pgfqpoint{5.210293in}{0.136829in}}%
\pgfpathlineto{\pgfqpoint{5.203099in}{0.136829in}}%
\pgfpathlineto{\pgfqpoint{5.195905in}{0.136829in}}%
\pgfpathlineto{\pgfqpoint{5.188710in}{0.136829in}}%
\pgfpathlineto{\pgfqpoint{5.181516in}{0.136829in}}%
\pgfpathlineto{\pgfqpoint{5.174322in}{0.136829in}}%
\pgfpathlineto{\pgfqpoint{5.167128in}{0.136829in}}%
\pgfpathlineto{\pgfqpoint{5.159933in}{0.136829in}}%
\pgfpathlineto{\pgfqpoint{5.152739in}{0.136829in}}%
\pgfpathlineto{\pgfqpoint{5.145545in}{0.136829in}}%
\pgfpathlineto{\pgfqpoint{5.138351in}{0.136829in}}%
\pgfpathlineto{\pgfqpoint{5.131156in}{0.136829in}}%
\pgfpathlineto{\pgfqpoint{5.123962in}{0.136829in}}%
\pgfpathlineto{\pgfqpoint{5.116768in}{0.136829in}}%
\pgfpathlineto{\pgfqpoint{5.109574in}{0.136829in}}%
\pgfpathlineto{\pgfqpoint{5.102380in}{0.136829in}}%
\pgfpathlineto{\pgfqpoint{5.102380in}{0.136829in}}%
\pgfpathlineto{\pgfqpoint{5.098024in}{0.136829in}}%
\pgfpathlineto{\pgfqpoint{5.093669in}{0.136829in}}%
\pgfpathlineto{\pgfqpoint{5.089313in}{0.136829in}}%
\pgfpathlineto{\pgfqpoint{5.084958in}{0.136829in}}%
\pgfpathlineto{\pgfqpoint{5.080602in}{0.136829in}}%
\pgfpathlineto{\pgfqpoint{5.076247in}{0.136829in}}%
\pgfpathlineto{\pgfqpoint{5.071891in}{0.136829in}}%
\pgfpathlineto{\pgfqpoint{5.067536in}{0.136829in}}%
\pgfpathlineto{\pgfqpoint{5.063180in}{0.136829in}}%
\pgfpathlineto{\pgfqpoint{5.058825in}{0.136829in}}%
\pgfpathlineto{\pgfqpoint{5.054469in}{0.136829in}}%
\pgfpathlineto{\pgfqpoint{5.050114in}{0.136829in}}%
\pgfpathlineto{\pgfqpoint{5.045759in}{0.136829in}}%
\pgfpathlineto{\pgfqpoint{5.041403in}{0.136829in}}%
\pgfpathlineto{\pgfqpoint{5.037048in}{0.136829in}}%
\pgfpathlineto{\pgfqpoint{5.032692in}{0.136829in}}%
\pgfpathlineto{\pgfqpoint{5.028337in}{0.136829in}}%
\pgfpathlineto{\pgfqpoint{5.023981in}{0.136829in}}%
\pgfpathlineto{\pgfqpoint{5.019626in}{0.136829in}}%
\pgfpathlineto{\pgfqpoint{5.015270in}{0.136829in}}%
\pgfpathlineto{\pgfqpoint{5.010915in}{0.136829in}}%
\pgfpathlineto{\pgfqpoint{5.006559in}{0.136829in}}%
\pgfpathlineto{\pgfqpoint{5.002204in}{0.136829in}}%
\pgfpathlineto{\pgfqpoint{4.997848in}{0.136829in}}%
\pgfpathlineto{\pgfqpoint{4.993493in}{0.136829in}}%
\pgfpathlineto{\pgfqpoint{4.989138in}{0.136829in}}%
\pgfpathlineto{\pgfqpoint{4.984782in}{0.136829in}}%
\pgfpathlineto{\pgfqpoint{4.980427in}{0.136829in}}%
\pgfpathlineto{\pgfqpoint{4.976071in}{0.136829in}}%
\pgfpathlineto{\pgfqpoint{4.971716in}{0.136829in}}%
\pgfpathlineto{\pgfqpoint{4.967360in}{0.136829in}}%
\pgfpathlineto{\pgfqpoint{4.963005in}{0.136829in}}%
\pgfpathlineto{\pgfqpoint{4.958649in}{0.136829in}}%
\pgfpathlineto{\pgfqpoint{4.954294in}{0.136829in}}%
\pgfpathlineto{\pgfqpoint{4.949938in}{0.136829in}}%
\pgfpathlineto{\pgfqpoint{4.945583in}{0.136829in}}%
\pgfpathlineto{\pgfqpoint{4.941228in}{0.136829in}}%
\pgfpathlineto{\pgfqpoint{4.936872in}{0.136829in}}%
\pgfpathlineto{\pgfqpoint{4.932517in}{0.136829in}}%
\pgfpathlineto{\pgfqpoint{4.928161in}{0.136829in}}%
\pgfpathlineto{\pgfqpoint{4.923806in}{0.136829in}}%
\pgfpathlineto{\pgfqpoint{4.919450in}{0.136829in}}%
\pgfpathlineto{\pgfqpoint{4.915095in}{0.136829in}}%
\pgfpathlineto{\pgfqpoint{4.910739in}{0.136829in}}%
\pgfpathlineto{\pgfqpoint{4.906384in}{0.136829in}}%
\pgfpathlineto{\pgfqpoint{4.902028in}{0.136829in}}%
\pgfpathlineto{\pgfqpoint{4.897673in}{0.136829in}}%
\pgfpathlineto{\pgfqpoint{4.893317in}{0.136829in}}%
\pgfpathlineto{\pgfqpoint{4.888962in}{0.136829in}}%
\pgfpathlineto{\pgfqpoint{4.884607in}{0.136829in}}%
\pgfpathlineto{\pgfqpoint{4.880251in}{0.136829in}}%
\pgfpathlineto{\pgfqpoint{4.875896in}{0.136829in}}%
\pgfpathlineto{\pgfqpoint{4.871540in}{0.136829in}}%
\pgfpathlineto{\pgfqpoint{4.867185in}{0.136829in}}%
\pgfpathlineto{\pgfqpoint{4.862829in}{0.136829in}}%
\pgfpathlineto{\pgfqpoint{4.858474in}{0.136829in}}%
\pgfpathlineto{\pgfqpoint{4.854118in}{0.136829in}}%
\pgfpathlineto{\pgfqpoint{4.849763in}{0.136829in}}%
\pgfpathlineto{\pgfqpoint{4.845407in}{0.136829in}}%
\pgfpathlineto{\pgfqpoint{4.841052in}{0.136829in}}%
\pgfpathlineto{\pgfqpoint{4.836696in}{0.136829in}}%
\pgfpathlineto{\pgfqpoint{4.832341in}{0.136829in}}%
\pgfpathlineto{\pgfqpoint{4.827986in}{0.136829in}}%
\pgfpathlineto{\pgfqpoint{4.823630in}{0.136829in}}%
\pgfpathlineto{\pgfqpoint{4.819275in}{0.136829in}}%
\pgfpathlineto{\pgfqpoint{4.814919in}{0.136829in}}%
\pgfpathlineto{\pgfqpoint{4.810564in}{0.136829in}}%
\pgfpathlineto{\pgfqpoint{4.806208in}{0.136829in}}%
\pgfpathlineto{\pgfqpoint{4.801853in}{0.136829in}}%
\pgfpathlineto{\pgfqpoint{4.797497in}{0.136829in}}%
\pgfpathlineto{\pgfqpoint{4.793142in}{0.136829in}}%
\pgfpathlineto{\pgfqpoint{4.788786in}{0.136829in}}%
\pgfpathlineto{\pgfqpoint{4.784431in}{0.136829in}}%
\pgfpathlineto{\pgfqpoint{4.780076in}{0.136829in}}%
\pgfpathlineto{\pgfqpoint{4.775720in}{0.136829in}}%
\pgfpathlineto{\pgfqpoint{4.771365in}{0.136829in}}%
\pgfpathlineto{\pgfqpoint{4.767009in}{0.136829in}}%
\pgfpathlineto{\pgfqpoint{4.762654in}{0.136829in}}%
\pgfpathlineto{\pgfqpoint{4.758298in}{0.136829in}}%
\pgfpathlineto{\pgfqpoint{4.753943in}{0.136829in}}%
\pgfpathlineto{\pgfqpoint{4.749587in}{0.136829in}}%
\pgfpathlineto{\pgfqpoint{4.745232in}{0.136829in}}%
\pgfpathlineto{\pgfqpoint{4.740876in}{0.136829in}}%
\pgfpathlineto{\pgfqpoint{4.736521in}{0.136829in}}%
\pgfpathlineto{\pgfqpoint{4.732165in}{0.136829in}}%
\pgfpathlineto{\pgfqpoint{4.727810in}{0.136829in}}%
\pgfpathlineto{\pgfqpoint{4.723455in}{0.136829in}}%
\pgfpathlineto{\pgfqpoint{4.719099in}{0.136829in}}%
\pgfpathlineto{\pgfqpoint{4.714744in}{0.136829in}}%
\pgfpathlineto{\pgfqpoint{4.710388in}{0.136829in}}%
\pgfpathlineto{\pgfqpoint{4.706033in}{0.136829in}}%
\pgfpathlineto{\pgfqpoint{4.701677in}{0.136829in}}%
\pgfpathlineto{\pgfqpoint{4.697322in}{0.136829in}}%
\pgfpathlineto{\pgfqpoint{4.692966in}{0.136829in}}%
\pgfpathlineto{\pgfqpoint{4.688611in}{0.136829in}}%
\pgfpathlineto{\pgfqpoint{4.684255in}{0.136829in}}%
\pgfpathlineto{\pgfqpoint{4.679900in}{0.136829in}}%
\pgfpathlineto{\pgfqpoint{4.675544in}{0.136829in}}%
\pgfpathlineto{\pgfqpoint{4.671189in}{0.136829in}}%
\pgfpathlineto{\pgfqpoint{4.666834in}{0.136829in}}%
\pgfpathlineto{\pgfqpoint{4.662478in}{0.136829in}}%
\pgfpathlineto{\pgfqpoint{4.658123in}{0.136829in}}%
\pgfpathlineto{\pgfqpoint{4.653767in}{0.136829in}}%
\pgfpathlineto{\pgfqpoint{4.649412in}{0.136829in}}%
\pgfpathlineto{\pgfqpoint{4.645056in}{0.136829in}}%
\pgfpathlineto{\pgfqpoint{4.640701in}{0.136829in}}%
\pgfpathlineto{\pgfqpoint{4.636345in}{0.136829in}}%
\pgfpathlineto{\pgfqpoint{4.631990in}{0.136829in}}%
\pgfpathlineto{\pgfqpoint{4.627634in}{0.136829in}}%
\pgfpathlineto{\pgfqpoint{4.623279in}{0.136829in}}%
\pgfpathlineto{\pgfqpoint{4.618924in}{0.136829in}}%
\pgfpathlineto{\pgfqpoint{4.614568in}{0.136829in}}%
\pgfpathlineto{\pgfqpoint{4.610213in}{0.136829in}}%
\pgfpathlineto{\pgfqpoint{4.605857in}{0.136829in}}%
\pgfpathlineto{\pgfqpoint{4.601502in}{0.136829in}}%
\pgfpathlineto{\pgfqpoint{4.597146in}{0.136829in}}%
\pgfpathlineto{\pgfqpoint{4.592791in}{0.136829in}}%
\pgfpathlineto{\pgfqpoint{4.588435in}{0.136829in}}%
\pgfpathlineto{\pgfqpoint{4.584080in}{0.136829in}}%
\pgfpathlineto{\pgfqpoint{4.579724in}{0.136829in}}%
\pgfpathlineto{\pgfqpoint{4.575369in}{0.136829in}}%
\pgfpathlineto{\pgfqpoint{4.571013in}{0.136829in}}%
\pgfpathlineto{\pgfqpoint{4.566658in}{0.136829in}}%
\pgfpathlineto{\pgfqpoint{4.562303in}{0.136829in}}%
\pgfpathlineto{\pgfqpoint{4.557947in}{0.136829in}}%
\pgfpathlineto{\pgfqpoint{4.553592in}{0.136829in}}%
\pgfpathlineto{\pgfqpoint{4.549236in}{0.136829in}}%
\pgfpathlineto{\pgfqpoint{4.544881in}{0.136829in}}%
\pgfpathlineto{\pgfqpoint{4.540525in}{0.136829in}}%
\pgfpathlineto{\pgfqpoint{4.536170in}{0.136829in}}%
\pgfpathlineto{\pgfqpoint{4.531814in}{0.136829in}}%
\pgfpathlineto{\pgfqpoint{4.527459in}{0.136829in}}%
\pgfpathlineto{\pgfqpoint{4.523103in}{0.136829in}}%
\pgfpathlineto{\pgfqpoint{4.518748in}{0.136829in}}%
\pgfpathlineto{\pgfqpoint{4.514392in}{0.136829in}}%
\pgfpathlineto{\pgfqpoint{4.510037in}{0.136829in}}%
\pgfpathlineto{\pgfqpoint{4.505682in}{0.136829in}}%
\pgfpathlineto{\pgfqpoint{4.501326in}{0.136829in}}%
\pgfpathlineto{\pgfqpoint{4.496971in}{0.136829in}}%
\pgfpathlineto{\pgfqpoint{4.492615in}{0.136829in}}%
\pgfpathlineto{\pgfqpoint{4.488260in}{0.136829in}}%
\pgfpathlineto{\pgfqpoint{4.483904in}{0.136829in}}%
\pgfpathlineto{\pgfqpoint{4.479549in}{0.136829in}}%
\pgfpathlineto{\pgfqpoint{4.475193in}{0.136829in}}%
\pgfpathlineto{\pgfqpoint{4.470838in}{0.136829in}}%
\pgfpathlineto{\pgfqpoint{4.466482in}{0.136829in}}%
\pgfpathlineto{\pgfqpoint{4.462127in}{0.136829in}}%
\pgfpathlineto{\pgfqpoint{4.457772in}{0.136829in}}%
\pgfpathlineto{\pgfqpoint{4.453416in}{0.136829in}}%
\pgfpathlineto{\pgfqpoint{4.449061in}{0.136829in}}%
\pgfpathlineto{\pgfqpoint{4.444705in}{0.136829in}}%
\pgfpathlineto{\pgfqpoint{4.440350in}{0.136829in}}%
\pgfpathlineto{\pgfqpoint{4.435994in}{0.136829in}}%
\pgfpathlineto{\pgfqpoint{4.431639in}{0.136829in}}%
\pgfpathlineto{\pgfqpoint{4.427283in}{0.136829in}}%
\pgfpathlineto{\pgfqpoint{4.422928in}{0.136829in}}%
\pgfpathlineto{\pgfqpoint{4.418572in}{0.136829in}}%
\pgfpathlineto{\pgfqpoint{4.414217in}{0.136829in}}%
\pgfpathlineto{\pgfqpoint{4.409861in}{0.136829in}}%
\pgfpathlineto{\pgfqpoint{4.405506in}{0.136829in}}%
\pgfpathlineto{\pgfqpoint{4.401151in}{0.136829in}}%
\pgfpathlineto{\pgfqpoint{4.396795in}{0.136829in}}%
\pgfpathlineto{\pgfqpoint{4.392440in}{0.136829in}}%
\pgfpathlineto{\pgfqpoint{4.388084in}{0.136829in}}%
\pgfpathlineto{\pgfqpoint{4.383729in}{0.136829in}}%
\pgfpathlineto{\pgfqpoint{4.379373in}{0.136829in}}%
\pgfpathlineto{\pgfqpoint{4.375018in}{0.136829in}}%
\pgfpathlineto{\pgfqpoint{4.370662in}{0.136829in}}%
\pgfpathlineto{\pgfqpoint{4.366307in}{0.136829in}}%
\pgfpathlineto{\pgfqpoint{4.361951in}{0.136829in}}%
\pgfpathlineto{\pgfqpoint{4.357596in}{0.136829in}}%
\pgfpathlineto{\pgfqpoint{4.353240in}{0.136829in}}%
\pgfpathlineto{\pgfqpoint{4.348885in}{0.136829in}}%
\pgfpathlineto{\pgfqpoint{4.344530in}{0.136829in}}%
\pgfpathlineto{\pgfqpoint{4.340174in}{0.136829in}}%
\pgfpathlineto{\pgfqpoint{4.335819in}{0.136829in}}%
\pgfpathlineto{\pgfqpoint{4.331463in}{0.136829in}}%
\pgfpathlineto{\pgfqpoint{4.327108in}{0.136829in}}%
\pgfpathlineto{\pgfqpoint{4.322752in}{0.136829in}}%
\pgfpathlineto{\pgfqpoint{4.318397in}{0.136829in}}%
\pgfpathlineto{\pgfqpoint{4.314041in}{0.136829in}}%
\pgfpathlineto{\pgfqpoint{4.309686in}{0.136829in}}%
\pgfpathlineto{\pgfqpoint{4.305330in}{0.136829in}}%
\pgfpathlineto{\pgfqpoint{4.300975in}{0.136829in}}%
\pgfpathlineto{\pgfqpoint{4.296620in}{0.136829in}}%
\pgfpathlineto{\pgfqpoint{4.292264in}{0.136829in}}%
\pgfpathlineto{\pgfqpoint{4.287909in}{0.136829in}}%
\pgfpathlineto{\pgfqpoint{4.283553in}{0.136829in}}%
\pgfpathlineto{\pgfqpoint{4.279198in}{0.136829in}}%
\pgfpathlineto{\pgfqpoint{4.274842in}{0.136829in}}%
\pgfpathlineto{\pgfqpoint{4.270487in}{0.136829in}}%
\pgfpathlineto{\pgfqpoint{4.266131in}{0.136829in}}%
\pgfpathlineto{\pgfqpoint{4.261776in}{0.136829in}}%
\pgfpathlineto{\pgfqpoint{4.257420in}{0.136829in}}%
\pgfpathlineto{\pgfqpoint{4.253065in}{0.136829in}}%
\pgfpathlineto{\pgfqpoint{4.248709in}{0.136829in}}%
\pgfpathlineto{\pgfqpoint{4.244354in}{0.136829in}}%
\pgfpathlineto{\pgfqpoint{4.239999in}{0.136829in}}%
\pgfpathlineto{\pgfqpoint{4.235643in}{0.136829in}}%
\pgfpathlineto{\pgfqpoint{4.231288in}{0.136829in}}%
\pgfpathlineto{\pgfqpoint{4.226932in}{0.136829in}}%
\pgfpathlineto{\pgfqpoint{4.222577in}{0.136829in}}%
\pgfpathlineto{\pgfqpoint{4.218221in}{0.136829in}}%
\pgfpathlineto{\pgfqpoint{4.213866in}{0.136829in}}%
\pgfpathlineto{\pgfqpoint{4.209510in}{0.136829in}}%
\pgfpathlineto{\pgfqpoint{4.205155in}{0.136829in}}%
\pgfpathlineto{\pgfqpoint{4.200799in}{0.136829in}}%
\pgfpathlineto{\pgfqpoint{4.196444in}{0.136829in}}%
\pgfpathlineto{\pgfqpoint{4.192088in}{0.136829in}}%
\pgfpathlineto{\pgfqpoint{4.187733in}{0.136829in}}%
\pgfpathlineto{\pgfqpoint{4.183378in}{0.136829in}}%
\pgfpathlineto{\pgfqpoint{4.179022in}{0.136829in}}%
\pgfpathlineto{\pgfqpoint{4.174667in}{0.136829in}}%
\pgfpathlineto{\pgfqpoint{4.170311in}{0.136829in}}%
\pgfpathlineto{\pgfqpoint{4.165956in}{0.136829in}}%
\pgfpathlineto{\pgfqpoint{4.161600in}{0.136829in}}%
\pgfpathlineto{\pgfqpoint{4.157245in}{0.136829in}}%
\pgfpathlineto{\pgfqpoint{4.152889in}{0.136829in}}%
\pgfpathlineto{\pgfqpoint{4.148534in}{0.136829in}}%
\pgfpathlineto{\pgfqpoint{4.144178in}{0.136829in}}%
\pgfpathlineto{\pgfqpoint{4.139823in}{0.136829in}}%
\pgfpathlineto{\pgfqpoint{4.135468in}{0.136829in}}%
\pgfpathlineto{\pgfqpoint{4.131112in}{0.136829in}}%
\pgfpathlineto{\pgfqpoint{4.126757in}{0.136829in}}%
\pgfpathlineto{\pgfqpoint{4.122401in}{0.136829in}}%
\pgfpathlineto{\pgfqpoint{4.118046in}{0.136829in}}%
\pgfpathlineto{\pgfqpoint{4.113690in}{0.136829in}}%
\pgfpathlineto{\pgfqpoint{4.109335in}{0.136829in}}%
\pgfpathlineto{\pgfqpoint{4.104979in}{0.136829in}}%
\pgfpathlineto{\pgfqpoint{4.100624in}{0.136829in}}%
\pgfpathlineto{\pgfqpoint{4.096268in}{0.136829in}}%
\pgfpathlineto{\pgfqpoint{4.091913in}{0.136829in}}%
\pgfpathlineto{\pgfqpoint{4.087557in}{0.136829in}}%
\pgfpathlineto{\pgfqpoint{4.083202in}{0.136829in}}%
\pgfpathlineto{\pgfqpoint{4.078847in}{0.136829in}}%
\pgfpathlineto{\pgfqpoint{4.074491in}{0.136829in}}%
\pgfpathlineto{\pgfqpoint{4.070136in}{0.136829in}}%
\pgfpathlineto{\pgfqpoint{4.065780in}{0.136829in}}%
\pgfpathlineto{\pgfqpoint{4.061425in}{0.136829in}}%
\pgfpathlineto{\pgfqpoint{4.057069in}{0.136829in}}%
\pgfpathlineto{\pgfqpoint{4.052714in}{0.136829in}}%
\pgfpathlineto{\pgfqpoint{4.048358in}{0.136829in}}%
\pgfpathlineto{\pgfqpoint{4.044003in}{0.136829in}}%
\pgfpathlineto{\pgfqpoint{4.039647in}{0.136829in}}%
\pgfpathlineto{\pgfqpoint{4.035292in}{0.136829in}}%
\pgfpathlineto{\pgfqpoint{4.030936in}{0.136829in}}%
\pgfpathlineto{\pgfqpoint{4.026581in}{0.136829in}}%
\pgfpathlineto{\pgfqpoint{4.022226in}{0.136829in}}%
\pgfpathlineto{\pgfqpoint{4.017870in}{0.136829in}}%
\pgfpathlineto{\pgfqpoint{4.013515in}{0.136829in}}%
\pgfpathlineto{\pgfqpoint{4.009159in}{0.136829in}}%
\pgfpathlineto{\pgfqpoint{4.004804in}{0.136829in}}%
\pgfpathlineto{\pgfqpoint{4.000448in}{0.136829in}}%
\pgfpathlineto{\pgfqpoint{3.996093in}{0.136829in}}%
\pgfpathlineto{\pgfqpoint{3.991737in}{0.136829in}}%
\pgfpathlineto{\pgfqpoint{3.987382in}{0.136829in}}%
\pgfpathlineto{\pgfqpoint{3.983026in}{0.136829in}}%
\pgfpathlineto{\pgfqpoint{3.978671in}{0.136829in}}%
\pgfpathlineto{\pgfqpoint{3.974316in}{0.136829in}}%
\pgfpathlineto{\pgfqpoint{3.969960in}{0.136829in}}%
\pgfpathlineto{\pgfqpoint{3.965605in}{0.136829in}}%
\pgfpathlineto{\pgfqpoint{3.961249in}{0.136829in}}%
\pgfpathlineto{\pgfqpoint{3.956894in}{0.136829in}}%
\pgfpathlineto{\pgfqpoint{3.952538in}{0.136829in}}%
\pgfpathlineto{\pgfqpoint{3.948183in}{0.136829in}}%
\pgfpathlineto{\pgfqpoint{3.943827in}{0.136829in}}%
\pgfpathlineto{\pgfqpoint{3.939472in}{0.136829in}}%
\pgfpathlineto{\pgfqpoint{3.935116in}{0.136829in}}%
\pgfpathlineto{\pgfqpoint{3.930761in}{0.136829in}}%
\pgfpathlineto{\pgfqpoint{3.926405in}{0.136829in}}%
\pgfpathlineto{\pgfqpoint{3.922050in}{0.136829in}}%
\pgfpathlineto{\pgfqpoint{3.917695in}{0.136829in}}%
\pgfpathlineto{\pgfqpoint{3.913339in}{0.136829in}}%
\pgfpathlineto{\pgfqpoint{3.908984in}{0.136829in}}%
\pgfpathlineto{\pgfqpoint{3.904628in}{0.136829in}}%
\pgfpathlineto{\pgfqpoint{3.900273in}{0.136829in}}%
\pgfpathlineto{\pgfqpoint{3.895917in}{0.136829in}}%
\pgfpathlineto{\pgfqpoint{3.891562in}{0.136829in}}%
\pgfpathlineto{\pgfqpoint{3.887206in}{0.136829in}}%
\pgfpathlineto{\pgfqpoint{3.882851in}{0.136829in}}%
\pgfpathlineto{\pgfqpoint{3.878495in}{0.136829in}}%
\pgfpathlineto{\pgfqpoint{3.874140in}{0.136829in}}%
\pgfpathlineto{\pgfqpoint{3.869784in}{0.136829in}}%
\pgfpathlineto{\pgfqpoint{3.865429in}{0.136829in}}%
\pgfpathlineto{\pgfqpoint{3.861074in}{0.136829in}}%
\pgfpathlineto{\pgfqpoint{3.856718in}{0.136829in}}%
\pgfpathlineto{\pgfqpoint{3.852363in}{0.136829in}}%
\pgfpathlineto{\pgfqpoint{3.848007in}{0.136829in}}%
\pgfpathlineto{\pgfqpoint{3.843652in}{0.136829in}}%
\pgfpathlineto{\pgfqpoint{3.839296in}{0.136829in}}%
\pgfpathlineto{\pgfqpoint{3.834941in}{0.136829in}}%
\pgfpathlineto{\pgfqpoint{3.830585in}{0.136829in}}%
\pgfpathlineto{\pgfqpoint{3.826230in}{0.136829in}}%
\pgfpathlineto{\pgfqpoint{3.821874in}{0.136829in}}%
\pgfpathlineto{\pgfqpoint{3.817519in}{0.136829in}}%
\pgfpathlineto{\pgfqpoint{3.813164in}{0.136829in}}%
\pgfpathlineto{\pgfqpoint{3.808808in}{0.136829in}}%
\pgfpathlineto{\pgfqpoint{3.804453in}{0.136829in}}%
\pgfpathlineto{\pgfqpoint{3.800097in}{0.136829in}}%
\pgfpathlineto{\pgfqpoint{3.795742in}{0.136829in}}%
\pgfpathlineto{\pgfqpoint{3.791386in}{0.136829in}}%
\pgfpathlineto{\pgfqpoint{3.787031in}{0.136829in}}%
\pgfpathlineto{\pgfqpoint{3.782675in}{0.136829in}}%
\pgfpathlineto{\pgfqpoint{3.778320in}{0.136829in}}%
\pgfpathlineto{\pgfqpoint{3.773964in}{0.136829in}}%
\pgfpathlineto{\pgfqpoint{3.769609in}{0.136829in}}%
\pgfpathlineto{\pgfqpoint{3.765253in}{0.136829in}}%
\pgfpathlineto{\pgfqpoint{3.760898in}{0.136829in}}%
\pgfpathlineto{\pgfqpoint{3.756543in}{0.136829in}}%
\pgfpathlineto{\pgfqpoint{3.752187in}{0.136829in}}%
\pgfpathlineto{\pgfqpoint{3.747832in}{0.136829in}}%
\pgfpathlineto{\pgfqpoint{3.743476in}{0.136829in}}%
\pgfpathlineto{\pgfqpoint{3.739121in}{0.136829in}}%
\pgfpathlineto{\pgfqpoint{3.734765in}{0.136829in}}%
\pgfpathlineto{\pgfqpoint{3.730410in}{0.136829in}}%
\pgfpathlineto{\pgfqpoint{3.726054in}{0.136829in}}%
\pgfpathlineto{\pgfqpoint{3.721699in}{0.136829in}}%
\pgfpathlineto{\pgfqpoint{3.717343in}{0.136829in}}%
\pgfpathlineto{\pgfqpoint{3.712988in}{0.136829in}}%
\pgfpathlineto{\pgfqpoint{3.708632in}{0.136829in}}%
\pgfpathlineto{\pgfqpoint{3.704277in}{0.136829in}}%
\pgfpathlineto{\pgfqpoint{3.699922in}{0.136829in}}%
\pgfpathlineto{\pgfqpoint{3.695566in}{0.136829in}}%
\pgfpathlineto{\pgfqpoint{3.691211in}{0.136829in}}%
\pgfpathlineto{\pgfqpoint{3.686855in}{0.136829in}}%
\pgfpathlineto{\pgfqpoint{3.682500in}{0.136829in}}%
\pgfpathlineto{\pgfqpoint{3.678144in}{0.136829in}}%
\pgfpathlineto{\pgfqpoint{3.673789in}{0.136829in}}%
\pgfpathlineto{\pgfqpoint{3.669433in}{0.136829in}}%
\pgfpathlineto{\pgfqpoint{3.665078in}{0.136829in}}%
\pgfpathlineto{\pgfqpoint{3.660722in}{0.136829in}}%
\pgfpathlineto{\pgfqpoint{3.656367in}{0.136829in}}%
\pgfpathlineto{\pgfqpoint{3.652012in}{0.136829in}}%
\pgfpathlineto{\pgfqpoint{3.647656in}{0.136829in}}%
\pgfpathlineto{\pgfqpoint{3.643301in}{0.136829in}}%
\pgfpathlineto{\pgfqpoint{3.638945in}{0.136829in}}%
\pgfpathlineto{\pgfqpoint{3.634590in}{0.136829in}}%
\pgfpathlineto{\pgfqpoint{3.630234in}{0.136829in}}%
\pgfpathlineto{\pgfqpoint{3.625879in}{0.136829in}}%
\pgfpathlineto{\pgfqpoint{3.621523in}{0.136829in}}%
\pgfpathlineto{\pgfqpoint{3.617168in}{0.136829in}}%
\pgfpathlineto{\pgfqpoint{3.612812in}{0.136829in}}%
\pgfpathlineto{\pgfqpoint{3.608457in}{0.136829in}}%
\pgfpathlineto{\pgfqpoint{3.604101in}{0.136829in}}%
\pgfpathlineto{\pgfqpoint{3.599746in}{0.136829in}}%
\pgfpathlineto{\pgfqpoint{3.595391in}{0.136829in}}%
\pgfpathlineto{\pgfqpoint{3.591035in}{0.136829in}}%
\pgfpathlineto{\pgfqpoint{3.586680in}{0.136829in}}%
\pgfpathlineto{\pgfqpoint{3.582324in}{0.136829in}}%
\pgfpathlineto{\pgfqpoint{3.577969in}{0.136829in}}%
\pgfpathlineto{\pgfqpoint{3.573613in}{0.136829in}}%
\pgfpathlineto{\pgfqpoint{3.569258in}{0.136829in}}%
\pgfpathlineto{\pgfqpoint{3.564902in}{0.136829in}}%
\pgfpathlineto{\pgfqpoint{3.560547in}{0.136829in}}%
\pgfpathlineto{\pgfqpoint{3.556191in}{0.136829in}}%
\pgfpathlineto{\pgfqpoint{3.551836in}{0.136829in}}%
\pgfpathlineto{\pgfqpoint{3.547480in}{0.136829in}}%
\pgfpathlineto{\pgfqpoint{3.543125in}{0.136829in}}%
\pgfpathlineto{\pgfqpoint{3.538770in}{0.136829in}}%
\pgfpathlineto{\pgfqpoint{3.534414in}{0.136829in}}%
\pgfpathlineto{\pgfqpoint{3.530059in}{0.136829in}}%
\pgfpathlineto{\pgfqpoint{3.525703in}{0.136829in}}%
\pgfpathlineto{\pgfqpoint{3.521348in}{0.136829in}}%
\pgfpathlineto{\pgfqpoint{3.516992in}{0.136829in}}%
\pgfpathlineto{\pgfqpoint{3.512637in}{0.136829in}}%
\pgfpathlineto{\pgfqpoint{3.508281in}{0.136829in}}%
\pgfpathlineto{\pgfqpoint{3.503926in}{0.136829in}}%
\pgfpathlineto{\pgfqpoint{3.499570in}{0.136829in}}%
\pgfpathlineto{\pgfqpoint{3.495215in}{0.136829in}}%
\pgfpathlineto{\pgfqpoint{3.490860in}{0.136829in}}%
\pgfpathlineto{\pgfqpoint{3.486504in}{0.136829in}}%
\pgfpathlineto{\pgfqpoint{3.482149in}{0.136829in}}%
\pgfpathlineto{\pgfqpoint{3.477793in}{0.136829in}}%
\pgfpathlineto{\pgfqpoint{3.473438in}{0.136829in}}%
\pgfpathlineto{\pgfqpoint{3.469082in}{0.136829in}}%
\pgfpathlineto{\pgfqpoint{3.464727in}{0.136829in}}%
\pgfpathlineto{\pgfqpoint{3.460371in}{0.136829in}}%
\pgfpathlineto{\pgfqpoint{3.456016in}{0.136829in}}%
\pgfpathlineto{\pgfqpoint{3.451660in}{0.136829in}}%
\pgfpathlineto{\pgfqpoint{3.447305in}{0.136829in}}%
\pgfpathlineto{\pgfqpoint{3.442949in}{0.136829in}}%
\pgfpathlineto{\pgfqpoint{3.438594in}{0.136829in}}%
\pgfpathlineto{\pgfqpoint{3.434239in}{0.136829in}}%
\pgfpathlineto{\pgfqpoint{3.429883in}{0.136829in}}%
\pgfpathlineto{\pgfqpoint{3.425528in}{0.136829in}}%
\pgfpathlineto{\pgfqpoint{3.421172in}{0.136829in}}%
\pgfpathlineto{\pgfqpoint{3.416817in}{0.136829in}}%
\pgfpathlineto{\pgfqpoint{3.412461in}{0.136829in}}%
\pgfpathlineto{\pgfqpoint{3.408106in}{0.136829in}}%
\pgfpathlineto{\pgfqpoint{3.403750in}{0.136829in}}%
\pgfpathlineto{\pgfqpoint{3.399395in}{0.136829in}}%
\pgfpathlineto{\pgfqpoint{3.395039in}{0.136829in}}%
\pgfpathlineto{\pgfqpoint{3.390684in}{0.136829in}}%
\pgfpathlineto{\pgfqpoint{3.386328in}{0.136829in}}%
\pgfpathlineto{\pgfqpoint{3.381973in}{0.136829in}}%
\pgfpathlineto{\pgfqpoint{3.377618in}{0.136829in}}%
\pgfpathlineto{\pgfqpoint{3.373262in}{0.136829in}}%
\pgfpathlineto{\pgfqpoint{3.368907in}{0.136829in}}%
\pgfpathlineto{\pgfqpoint{3.364551in}{0.136829in}}%
\pgfpathlineto{\pgfqpoint{3.360196in}{0.136829in}}%
\pgfpathlineto{\pgfqpoint{3.355840in}{0.136829in}}%
\pgfpathlineto{\pgfqpoint{3.351485in}{0.136829in}}%
\pgfpathlineto{\pgfqpoint{3.347129in}{0.136829in}}%
\pgfpathlineto{\pgfqpoint{3.342774in}{0.136829in}}%
\pgfpathlineto{\pgfqpoint{3.338418in}{0.136829in}}%
\pgfpathlineto{\pgfqpoint{3.334063in}{0.136829in}}%
\pgfpathlineto{\pgfqpoint{3.329708in}{0.136829in}}%
\pgfpathlineto{\pgfqpoint{3.325352in}{0.136829in}}%
\pgfpathlineto{\pgfqpoint{3.320997in}{0.136829in}}%
\pgfpathlineto{\pgfqpoint{3.316641in}{0.136829in}}%
\pgfpathlineto{\pgfqpoint{3.312286in}{0.136829in}}%
\pgfpathlineto{\pgfqpoint{3.307930in}{0.136829in}}%
\pgfpathlineto{\pgfqpoint{3.303575in}{0.136829in}}%
\pgfpathlineto{\pgfqpoint{3.299219in}{0.136829in}}%
\pgfpathlineto{\pgfqpoint{3.294864in}{0.136829in}}%
\pgfpathlineto{\pgfqpoint{3.290508in}{0.136829in}}%
\pgfpathlineto{\pgfqpoint{3.286153in}{0.136829in}}%
\pgfpathlineto{\pgfqpoint{3.281797in}{0.136829in}}%
\pgfpathlineto{\pgfqpoint{3.277442in}{0.136829in}}%
\pgfpathlineto{\pgfqpoint{3.273087in}{0.136829in}}%
\pgfpathlineto{\pgfqpoint{3.268731in}{0.136829in}}%
\pgfpathlineto{\pgfqpoint{3.264376in}{0.136829in}}%
\pgfpathlineto{\pgfqpoint{3.260020in}{0.136829in}}%
\pgfpathlineto{\pgfqpoint{3.255665in}{0.136829in}}%
\pgfpathlineto{\pgfqpoint{3.251309in}{0.136829in}}%
\pgfpathlineto{\pgfqpoint{3.246954in}{0.136829in}}%
\pgfpathlineto{\pgfqpoint{3.242598in}{0.136829in}}%
\pgfpathlineto{\pgfqpoint{3.238243in}{0.136829in}}%
\pgfpathlineto{\pgfqpoint{3.233887in}{0.136829in}}%
\pgfpathlineto{\pgfqpoint{3.229532in}{0.136829in}}%
\pgfpathlineto{\pgfqpoint{3.225176in}{0.136829in}}%
\pgfpathlineto{\pgfqpoint{3.220821in}{0.136829in}}%
\pgfpathlineto{\pgfqpoint{3.216466in}{0.136829in}}%
\pgfpathlineto{\pgfqpoint{3.212110in}{0.136829in}}%
\pgfpathlineto{\pgfqpoint{3.207755in}{0.136829in}}%
\pgfpathlineto{\pgfqpoint{3.203399in}{0.136829in}}%
\pgfpathlineto{\pgfqpoint{3.199044in}{0.136829in}}%
\pgfpathlineto{\pgfqpoint{3.194688in}{0.136829in}}%
\pgfpathlineto{\pgfqpoint{3.190333in}{0.136829in}}%
\pgfpathlineto{\pgfqpoint{3.185977in}{0.136829in}}%
\pgfpathlineto{\pgfqpoint{3.181622in}{0.136829in}}%
\pgfpathlineto{\pgfqpoint{3.177266in}{0.136829in}}%
\pgfpathlineto{\pgfqpoint{3.172911in}{0.136829in}}%
\pgfpathlineto{\pgfqpoint{3.168556in}{0.136829in}}%
\pgfpathlineto{\pgfqpoint{3.164200in}{0.136829in}}%
\pgfpathlineto{\pgfqpoint{3.159845in}{0.136829in}}%
\pgfpathlineto{\pgfqpoint{3.155489in}{0.136829in}}%
\pgfpathlineto{\pgfqpoint{3.151134in}{0.136829in}}%
\pgfpathlineto{\pgfqpoint{3.146778in}{0.136829in}}%
\pgfpathlineto{\pgfqpoint{3.142423in}{0.136829in}}%
\pgfpathlineto{\pgfqpoint{3.138067in}{0.136829in}}%
\pgfpathlineto{\pgfqpoint{3.133712in}{0.136829in}}%
\pgfpathlineto{\pgfqpoint{3.129356in}{0.136829in}}%
\pgfpathlineto{\pgfqpoint{3.125001in}{0.136829in}}%
\pgfpathlineto{\pgfqpoint{3.120645in}{0.136829in}}%
\pgfpathlineto{\pgfqpoint{3.116290in}{0.136829in}}%
\pgfpathlineto{\pgfqpoint{3.111935in}{0.136829in}}%
\pgfpathlineto{\pgfqpoint{3.107579in}{0.136829in}}%
\pgfpathlineto{\pgfqpoint{3.103224in}{0.136829in}}%
\pgfpathlineto{\pgfqpoint{3.098868in}{0.136829in}}%
\pgfpathlineto{\pgfqpoint{3.094513in}{0.136829in}}%
\pgfpathlineto{\pgfqpoint{3.090157in}{0.136829in}}%
\pgfpathlineto{\pgfqpoint{3.085802in}{0.136829in}}%
\pgfpathlineto{\pgfqpoint{3.081446in}{0.136829in}}%
\pgfpathlineto{\pgfqpoint{3.077091in}{0.136829in}}%
\pgfpathlineto{\pgfqpoint{3.072735in}{0.136829in}}%
\pgfpathlineto{\pgfqpoint{3.068380in}{0.136829in}}%
\pgfpathlineto{\pgfqpoint{3.064024in}{0.136829in}}%
\pgfpathlineto{\pgfqpoint{3.059669in}{0.136829in}}%
\pgfpathlineto{\pgfqpoint{3.055314in}{0.136829in}}%
\pgfpathlineto{\pgfqpoint{3.050958in}{0.136829in}}%
\pgfpathlineto{\pgfqpoint{3.046603in}{0.136829in}}%
\pgfpathlineto{\pgfqpoint{3.042247in}{0.136829in}}%
\pgfpathlineto{\pgfqpoint{3.037892in}{0.136829in}}%
\pgfpathlineto{\pgfqpoint{3.033536in}{0.136829in}}%
\pgfpathlineto{\pgfqpoint{3.029181in}{0.136829in}}%
\pgfpathlineto{\pgfqpoint{3.024825in}{0.136829in}}%
\pgfpathlineto{\pgfqpoint{3.020470in}{0.136829in}}%
\pgfpathlineto{\pgfqpoint{3.016114in}{0.136829in}}%
\pgfpathlineto{\pgfqpoint{3.011759in}{0.136829in}}%
\pgfpathlineto{\pgfqpoint{3.007404in}{0.136829in}}%
\pgfpathlineto{\pgfqpoint{3.003048in}{0.136829in}}%
\pgfpathlineto{\pgfqpoint{2.998693in}{0.136829in}}%
\pgfpathlineto{\pgfqpoint{2.994337in}{0.136829in}}%
\pgfpathlineto{\pgfqpoint{2.989982in}{0.136829in}}%
\pgfpathlineto{\pgfqpoint{2.985626in}{0.136829in}}%
\pgfpathlineto{\pgfqpoint{2.981271in}{0.136829in}}%
\pgfpathlineto{\pgfqpoint{2.976915in}{0.136829in}}%
\pgfpathlineto{\pgfqpoint{2.972560in}{0.136829in}}%
\pgfpathlineto{\pgfqpoint{2.968204in}{0.136829in}}%
\pgfpathlineto{\pgfqpoint{2.963849in}{0.136829in}}%
\pgfpathlineto{\pgfqpoint{2.959493in}{0.136829in}}%
\pgfpathlineto{\pgfqpoint{2.955138in}{0.136829in}}%
\pgfpathlineto{\pgfqpoint{2.950783in}{0.136829in}}%
\pgfpathlineto{\pgfqpoint{2.946427in}{0.136829in}}%
\pgfpathlineto{\pgfqpoint{2.942072in}{0.136829in}}%
\pgfpathlineto{\pgfqpoint{2.937716in}{0.136829in}}%
\pgfpathlineto{\pgfqpoint{2.933361in}{0.136829in}}%
\pgfpathlineto{\pgfqpoint{2.929005in}{0.136829in}}%
\pgfpathlineto{\pgfqpoint{2.924650in}{0.136829in}}%
\pgfpathlineto{\pgfqpoint{2.920294in}{0.136829in}}%
\pgfpathlineto{\pgfqpoint{2.915939in}{0.136829in}}%
\pgfpathlineto{\pgfqpoint{2.911583in}{0.136829in}}%
\pgfpathlineto{\pgfqpoint{2.907228in}{0.136829in}}%
\pgfpathlineto{\pgfqpoint{2.902872in}{0.136829in}}%
\pgfpathlineto{\pgfqpoint{2.898517in}{0.136829in}}%
\pgfpathlineto{\pgfqpoint{2.894162in}{0.136829in}}%
\pgfpathlineto{\pgfqpoint{2.889806in}{0.136829in}}%
\pgfpathlineto{\pgfqpoint{2.885451in}{0.136829in}}%
\pgfpathlineto{\pgfqpoint{2.881095in}{0.136829in}}%
\pgfpathlineto{\pgfqpoint{2.876740in}{0.136829in}}%
\pgfpathlineto{\pgfqpoint{2.872384in}{0.136829in}}%
\pgfpathlineto{\pgfqpoint{2.868029in}{0.136829in}}%
\pgfpathlineto{\pgfqpoint{2.863673in}{0.136829in}}%
\pgfpathlineto{\pgfqpoint{2.859318in}{0.136829in}}%
\pgfpathlineto{\pgfqpoint{2.854962in}{0.136829in}}%
\pgfpathlineto{\pgfqpoint{2.850607in}{0.136829in}}%
\pgfpathlineto{\pgfqpoint{2.846252in}{0.136829in}}%
\pgfpathlineto{\pgfqpoint{2.841896in}{0.136829in}}%
\pgfpathlineto{\pgfqpoint{2.837541in}{0.136829in}}%
\pgfpathlineto{\pgfqpoint{2.833185in}{0.136829in}}%
\pgfpathlineto{\pgfqpoint{2.828830in}{0.136829in}}%
\pgfpathlineto{\pgfqpoint{2.824474in}{0.136829in}}%
\pgfpathlineto{\pgfqpoint{2.820119in}{0.136829in}}%
\pgfpathlineto{\pgfqpoint{2.815763in}{0.136829in}}%
\pgfpathlineto{\pgfqpoint{2.811408in}{0.136829in}}%
\pgfpathlineto{\pgfqpoint{2.807052in}{0.136829in}}%
\pgfpathlineto{\pgfqpoint{2.802697in}{0.136829in}}%
\pgfpathlineto{\pgfqpoint{2.798341in}{0.136829in}}%
\pgfpathlineto{\pgfqpoint{2.793986in}{0.136829in}}%
\pgfpathlineto{\pgfqpoint{2.789631in}{0.136829in}}%
\pgfpathlineto{\pgfqpoint{2.785275in}{0.136829in}}%
\pgfpathlineto{\pgfqpoint{2.780920in}{0.136829in}}%
\pgfpathlineto{\pgfqpoint{2.776564in}{0.136829in}}%
\pgfpathlineto{\pgfqpoint{2.772209in}{0.136829in}}%
\pgfpathlineto{\pgfqpoint{2.767853in}{0.136829in}}%
\pgfpathlineto{\pgfqpoint{2.763498in}{0.136829in}}%
\pgfpathlineto{\pgfqpoint{2.759142in}{0.136829in}}%
\pgfpathlineto{\pgfqpoint{2.754787in}{0.136829in}}%
\pgfpathlineto{\pgfqpoint{2.750431in}{0.136829in}}%
\pgfpathlineto{\pgfqpoint{2.746076in}{0.136829in}}%
\pgfpathlineto{\pgfqpoint{2.741720in}{0.136829in}}%
\pgfpathlineto{\pgfqpoint{2.737365in}{0.136829in}}%
\pgfpathlineto{\pgfqpoint{2.733010in}{0.136829in}}%
\pgfpathlineto{\pgfqpoint{2.728654in}{0.136829in}}%
\pgfpathlineto{\pgfqpoint{2.724299in}{0.136829in}}%
\pgfpathlineto{\pgfqpoint{2.719943in}{0.136829in}}%
\pgfpathlineto{\pgfqpoint{2.715588in}{0.136829in}}%
\pgfpathlineto{\pgfqpoint{2.711232in}{0.136829in}}%
\pgfpathlineto{\pgfqpoint{2.706877in}{0.136829in}}%
\pgfpathlineto{\pgfqpoint{2.702521in}{0.136829in}}%
\pgfpathlineto{\pgfqpoint{2.698166in}{0.136829in}}%
\pgfpathlineto{\pgfqpoint{2.693810in}{0.136829in}}%
\pgfpathlineto{\pgfqpoint{2.689455in}{0.136829in}}%
\pgfpathlineto{\pgfqpoint{2.685100in}{0.136829in}}%
\pgfpathlineto{\pgfqpoint{2.680744in}{0.136829in}}%
\pgfpathlineto{\pgfqpoint{2.676389in}{0.136829in}}%
\pgfpathlineto{\pgfqpoint{2.672033in}{0.136829in}}%
\pgfpathlineto{\pgfqpoint{2.667678in}{0.136829in}}%
\pgfpathlineto{\pgfqpoint{2.663322in}{0.136829in}}%
\pgfpathlineto{\pgfqpoint{2.658967in}{0.136829in}}%
\pgfpathlineto{\pgfqpoint{2.654611in}{0.136829in}}%
\pgfpathlineto{\pgfqpoint{2.650256in}{0.136829in}}%
\pgfpathlineto{\pgfqpoint{2.645900in}{0.136829in}}%
\pgfpathlineto{\pgfqpoint{2.641545in}{0.136829in}}%
\pgfpathlineto{\pgfqpoint{2.637189in}{0.136829in}}%
\pgfpathlineto{\pgfqpoint{2.632834in}{0.136829in}}%
\pgfpathlineto{\pgfqpoint{2.628479in}{0.136829in}}%
\pgfpathlineto{\pgfqpoint{2.624123in}{0.136829in}}%
\pgfpathlineto{\pgfqpoint{2.619768in}{0.136829in}}%
\pgfpathlineto{\pgfqpoint{2.615412in}{0.136829in}}%
\pgfpathlineto{\pgfqpoint{2.611057in}{0.136829in}}%
\pgfpathlineto{\pgfqpoint{2.606701in}{0.136829in}}%
\pgfpathlineto{\pgfqpoint{2.602346in}{0.136829in}}%
\pgfpathlineto{\pgfqpoint{2.597990in}{0.136829in}}%
\pgfpathlineto{\pgfqpoint{2.593635in}{0.136829in}}%
\pgfpathlineto{\pgfqpoint{2.589279in}{0.136829in}}%
\pgfpathlineto{\pgfqpoint{2.584924in}{0.136829in}}%
\pgfpathlineto{\pgfqpoint{2.580568in}{0.136829in}}%
\pgfpathlineto{\pgfqpoint{2.576213in}{0.136829in}}%
\pgfpathlineto{\pgfqpoint{2.571858in}{0.136829in}}%
\pgfpathlineto{\pgfqpoint{2.567502in}{0.136829in}}%
\pgfpathlineto{\pgfqpoint{2.563147in}{0.136829in}}%
\pgfpathlineto{\pgfqpoint{2.558791in}{0.136829in}}%
\pgfpathlineto{\pgfqpoint{2.554436in}{0.136829in}}%
\pgfpathlineto{\pgfqpoint{2.550080in}{0.136829in}}%
\pgfpathlineto{\pgfqpoint{2.545725in}{0.136829in}}%
\pgfpathlineto{\pgfqpoint{2.541369in}{0.136829in}}%
\pgfpathlineto{\pgfqpoint{2.537014in}{0.136829in}}%
\pgfpathlineto{\pgfqpoint{2.532658in}{0.136829in}}%
\pgfpathlineto{\pgfqpoint{2.528303in}{0.136829in}}%
\pgfpathlineto{\pgfqpoint{2.523948in}{0.136829in}}%
\pgfpathlineto{\pgfqpoint{2.519592in}{0.136829in}}%
\pgfpathlineto{\pgfqpoint{2.515237in}{0.136829in}}%
\pgfpathlineto{\pgfqpoint{2.510881in}{0.136829in}}%
\pgfpathlineto{\pgfqpoint{2.506526in}{0.136829in}}%
\pgfpathlineto{\pgfqpoint{2.502170in}{0.136829in}}%
\pgfpathlineto{\pgfqpoint{2.497815in}{0.136829in}}%
\pgfpathlineto{\pgfqpoint{2.493459in}{0.136829in}}%
\pgfpathlineto{\pgfqpoint{2.489104in}{0.136829in}}%
\pgfpathlineto{\pgfqpoint{2.484748in}{0.136829in}}%
\pgfpathlineto{\pgfqpoint{2.480393in}{0.136829in}}%
\pgfpathlineto{\pgfqpoint{2.476037in}{0.136829in}}%
\pgfpathlineto{\pgfqpoint{2.471682in}{0.136829in}}%
\pgfpathlineto{\pgfqpoint{2.467327in}{0.136829in}}%
\pgfpathlineto{\pgfqpoint{2.462971in}{0.136829in}}%
\pgfpathlineto{\pgfqpoint{2.458616in}{0.136829in}}%
\pgfpathlineto{\pgfqpoint{2.454260in}{0.136829in}}%
\pgfpathlineto{\pgfqpoint{2.449905in}{0.136829in}}%
\pgfpathlineto{\pgfqpoint{2.445549in}{0.136829in}}%
\pgfpathlineto{\pgfqpoint{2.441194in}{0.136829in}}%
\pgfpathlineto{\pgfqpoint{2.436838in}{0.136829in}}%
\pgfpathlineto{\pgfqpoint{2.432483in}{0.136829in}}%
\pgfpathlineto{\pgfqpoint{2.428127in}{0.136829in}}%
\pgfpathlineto{\pgfqpoint{2.423772in}{0.136829in}}%
\pgfpathlineto{\pgfqpoint{2.419416in}{0.136829in}}%
\pgfpathlineto{\pgfqpoint{2.415061in}{0.136829in}}%
\pgfpathlineto{\pgfqpoint{2.410706in}{0.136829in}}%
\pgfpathlineto{\pgfqpoint{2.406350in}{0.136829in}}%
\pgfpathlineto{\pgfqpoint{2.401995in}{0.136829in}}%
\pgfpathlineto{\pgfqpoint{2.397639in}{0.136829in}}%
\pgfpathlineto{\pgfqpoint{2.393284in}{0.136829in}}%
\pgfpathlineto{\pgfqpoint{2.388928in}{0.136829in}}%
\pgfpathlineto{\pgfqpoint{2.384573in}{0.136829in}}%
\pgfpathlineto{\pgfqpoint{2.380217in}{0.136829in}}%
\pgfpathlineto{\pgfqpoint{2.375862in}{0.136829in}}%
\pgfpathlineto{\pgfqpoint{2.371506in}{0.136829in}}%
\pgfpathlineto{\pgfqpoint{2.367151in}{0.136829in}}%
\pgfpathlineto{\pgfqpoint{2.362796in}{0.136829in}}%
\pgfpathlineto{\pgfqpoint{2.358440in}{0.136829in}}%
\pgfpathlineto{\pgfqpoint{2.354085in}{0.136829in}}%
\pgfpathlineto{\pgfqpoint{2.349729in}{0.136829in}}%
\pgfpathlineto{\pgfqpoint{2.345374in}{0.136829in}}%
\pgfpathlineto{\pgfqpoint{2.341018in}{0.136829in}}%
\pgfpathlineto{\pgfqpoint{2.336663in}{0.136829in}}%
\pgfpathlineto{\pgfqpoint{2.332307in}{0.136829in}}%
\pgfpathlineto{\pgfqpoint{2.327952in}{0.136829in}}%
\pgfpathlineto{\pgfqpoint{2.323596in}{0.136829in}}%
\pgfpathlineto{\pgfqpoint{2.319241in}{0.136829in}}%
\pgfpathlineto{\pgfqpoint{2.314885in}{0.136829in}}%
\pgfpathlineto{\pgfqpoint{2.310530in}{0.136829in}}%
\pgfpathlineto{\pgfqpoint{2.306175in}{0.136829in}}%
\pgfpathlineto{\pgfqpoint{2.301819in}{0.136829in}}%
\pgfpathlineto{\pgfqpoint{2.297464in}{0.136829in}}%
\pgfpathlineto{\pgfqpoint{2.293108in}{0.136829in}}%
\pgfpathlineto{\pgfqpoint{2.288753in}{0.136829in}}%
\pgfpathlineto{\pgfqpoint{2.284397in}{0.136829in}}%
\pgfpathlineto{\pgfqpoint{2.280042in}{0.136829in}}%
\pgfpathlineto{\pgfqpoint{2.275686in}{0.136829in}}%
\pgfpathlineto{\pgfqpoint{2.271331in}{0.136829in}}%
\pgfpathlineto{\pgfqpoint{2.266975in}{0.136829in}}%
\pgfpathlineto{\pgfqpoint{2.262620in}{0.136829in}}%
\pgfpathlineto{\pgfqpoint{2.258264in}{0.136829in}}%
\pgfpathlineto{\pgfqpoint{2.253909in}{0.136829in}}%
\pgfpathlineto{\pgfqpoint{2.249554in}{0.136829in}}%
\pgfpathlineto{\pgfqpoint{2.245198in}{0.136829in}}%
\pgfpathlineto{\pgfqpoint{2.240843in}{0.136829in}}%
\pgfpathlineto{\pgfqpoint{2.236487in}{0.136829in}}%
\pgfpathlineto{\pgfqpoint{2.232132in}{0.136829in}}%
\pgfpathlineto{\pgfqpoint{2.227776in}{0.136829in}}%
\pgfpathlineto{\pgfqpoint{2.223421in}{0.136829in}}%
\pgfpathlineto{\pgfqpoint{2.219065in}{0.136829in}}%
\pgfpathlineto{\pgfqpoint{2.214710in}{0.136829in}}%
\pgfpathlineto{\pgfqpoint{2.210354in}{0.136829in}}%
\pgfpathlineto{\pgfqpoint{2.205999in}{0.136829in}}%
\pgfpathlineto{\pgfqpoint{2.201644in}{0.136829in}}%
\pgfpathlineto{\pgfqpoint{2.197288in}{0.136829in}}%
\pgfpathlineto{\pgfqpoint{2.192933in}{0.136829in}}%
\pgfpathlineto{\pgfqpoint{2.188577in}{0.136829in}}%
\pgfpathlineto{\pgfqpoint{2.184222in}{0.136829in}}%
\pgfpathlineto{\pgfqpoint{2.179866in}{0.136829in}}%
\pgfpathlineto{\pgfqpoint{2.175511in}{0.136829in}}%
\pgfpathlineto{\pgfqpoint{2.171155in}{0.136829in}}%
\pgfpathlineto{\pgfqpoint{2.166800in}{0.136829in}}%
\pgfpathlineto{\pgfqpoint{2.162444in}{0.136829in}}%
\pgfpathlineto{\pgfqpoint{2.158089in}{0.136829in}}%
\pgfpathlineto{\pgfqpoint{2.153733in}{0.136829in}}%
\pgfpathlineto{\pgfqpoint{2.149378in}{0.136829in}}%
\pgfpathlineto{\pgfqpoint{2.145023in}{0.136829in}}%
\pgfpathlineto{\pgfqpoint{2.140667in}{0.136829in}}%
\pgfpathlineto{\pgfqpoint{2.136312in}{0.136829in}}%
\pgfpathlineto{\pgfqpoint{2.131956in}{0.136829in}}%
\pgfpathlineto{\pgfqpoint{2.127601in}{0.136829in}}%
\pgfpathlineto{\pgfqpoint{2.123245in}{0.136829in}}%
\pgfpathlineto{\pgfqpoint{2.118890in}{0.136829in}}%
\pgfpathlineto{\pgfqpoint{2.114534in}{0.136829in}}%
\pgfpathlineto{\pgfqpoint{2.110179in}{0.136829in}}%
\pgfpathlineto{\pgfqpoint{2.105823in}{0.136829in}}%
\pgfpathlineto{\pgfqpoint{2.101468in}{0.136829in}}%
\pgfpathlineto{\pgfqpoint{2.097112in}{0.136829in}}%
\pgfpathlineto{\pgfqpoint{2.092757in}{0.136829in}}%
\pgfpathlineto{\pgfqpoint{2.088402in}{0.136829in}}%
\pgfpathlineto{\pgfqpoint{2.084046in}{0.136829in}}%
\pgfpathlineto{\pgfqpoint{2.079691in}{0.136829in}}%
\pgfpathlineto{\pgfqpoint{2.075335in}{0.136829in}}%
\pgfpathlineto{\pgfqpoint{2.070980in}{0.136829in}}%
\pgfpathlineto{\pgfqpoint{2.066624in}{0.136829in}}%
\pgfpathlineto{\pgfqpoint{2.062269in}{0.136829in}}%
\pgfpathlineto{\pgfqpoint{2.057913in}{0.136829in}}%
\pgfpathlineto{\pgfqpoint{2.053558in}{0.136829in}}%
\pgfpathlineto{\pgfqpoint{2.049202in}{0.136829in}}%
\pgfpathlineto{\pgfqpoint{2.044847in}{0.136829in}}%
\pgfpathlineto{\pgfqpoint{2.040492in}{0.136829in}}%
\pgfpathlineto{\pgfqpoint{2.036136in}{0.136829in}}%
\pgfpathlineto{\pgfqpoint{2.031781in}{0.136829in}}%
\pgfpathlineto{\pgfqpoint{2.027425in}{0.136829in}}%
\pgfpathlineto{\pgfqpoint{2.023070in}{0.136829in}}%
\pgfpathlineto{\pgfqpoint{2.018714in}{0.136829in}}%
\pgfpathlineto{\pgfqpoint{2.014359in}{0.136829in}}%
\pgfpathlineto{\pgfqpoint{2.010003in}{0.136829in}}%
\pgfpathlineto{\pgfqpoint{2.005648in}{0.136829in}}%
\pgfpathlineto{\pgfqpoint{2.001292in}{0.136829in}}%
\pgfpathlineto{\pgfqpoint{1.996937in}{0.136829in}}%
\pgfpathlineto{\pgfqpoint{1.992581in}{0.136829in}}%
\pgfpathlineto{\pgfqpoint{1.988226in}{0.136829in}}%
\pgfpathlineto{\pgfqpoint{1.983871in}{0.136829in}}%
\pgfpathlineto{\pgfqpoint{1.979515in}{0.136829in}}%
\pgfpathlineto{\pgfqpoint{1.975160in}{0.136829in}}%
\pgfpathlineto{\pgfqpoint{1.970804in}{0.136829in}}%
\pgfpathlineto{\pgfqpoint{1.966449in}{0.136829in}}%
\pgfpathlineto{\pgfqpoint{1.962093in}{0.136829in}}%
\pgfpathlineto{\pgfqpoint{1.957738in}{0.136829in}}%
\pgfpathlineto{\pgfqpoint{1.953382in}{0.136829in}}%
\pgfpathlineto{\pgfqpoint{1.949027in}{0.136829in}}%
\pgfpathlineto{\pgfqpoint{1.944671in}{0.136829in}}%
\pgfpathlineto{\pgfqpoint{1.940316in}{0.136829in}}%
\pgfpathlineto{\pgfqpoint{1.935960in}{0.136829in}}%
\pgfpathlineto{\pgfqpoint{1.931605in}{0.136829in}}%
\pgfpathlineto{\pgfqpoint{1.927250in}{0.136829in}}%
\pgfpathlineto{\pgfqpoint{1.922894in}{0.136829in}}%
\pgfpathlineto{\pgfqpoint{1.918539in}{0.136829in}}%
\pgfpathlineto{\pgfqpoint{1.914183in}{0.136829in}}%
\pgfpathlineto{\pgfqpoint{1.909828in}{0.136829in}}%
\pgfpathlineto{\pgfqpoint{1.905472in}{0.136829in}}%
\pgfpathlineto{\pgfqpoint{1.901117in}{0.136829in}}%
\pgfpathlineto{\pgfqpoint{1.896761in}{0.136829in}}%
\pgfpathlineto{\pgfqpoint{1.892406in}{0.136829in}}%
\pgfpathlineto{\pgfqpoint{1.888050in}{0.136829in}}%
\pgfpathlineto{\pgfqpoint{1.883695in}{0.136829in}}%
\pgfpathlineto{\pgfqpoint{1.879340in}{0.136829in}}%
\pgfpathlineto{\pgfqpoint{1.874984in}{0.136829in}}%
\pgfpathlineto{\pgfqpoint{1.870629in}{0.136829in}}%
\pgfpathlineto{\pgfqpoint{1.866273in}{0.136829in}}%
\pgfpathlineto{\pgfqpoint{1.861918in}{0.136829in}}%
\pgfpathlineto{\pgfqpoint{1.857562in}{0.136829in}}%
\pgfpathlineto{\pgfqpoint{1.853207in}{0.136829in}}%
\pgfpathlineto{\pgfqpoint{1.848851in}{0.136829in}}%
\pgfpathlineto{\pgfqpoint{1.844496in}{0.136829in}}%
\pgfpathlineto{\pgfqpoint{1.840140in}{0.136829in}}%
\pgfpathlineto{\pgfqpoint{1.835785in}{0.136829in}}%
\pgfpathlineto{\pgfqpoint{1.831429in}{0.136829in}}%
\pgfpathlineto{\pgfqpoint{1.827074in}{0.136829in}}%
\pgfpathlineto{\pgfqpoint{1.822719in}{0.136829in}}%
\pgfpathlineto{\pgfqpoint{1.818363in}{0.136829in}}%
\pgfpathlineto{\pgfqpoint{1.814008in}{0.136829in}}%
\pgfpathlineto{\pgfqpoint{1.809652in}{0.136829in}}%
\pgfpathlineto{\pgfqpoint{1.805297in}{0.136829in}}%
\pgfpathlineto{\pgfqpoint{1.800941in}{0.136829in}}%
\pgfpathlineto{\pgfqpoint{1.796586in}{0.136829in}}%
\pgfpathlineto{\pgfqpoint{1.792230in}{0.136829in}}%
\pgfpathlineto{\pgfqpoint{1.787875in}{0.136829in}}%
\pgfpathlineto{\pgfqpoint{1.783519in}{0.136829in}}%
\pgfpathlineto{\pgfqpoint{1.779164in}{0.136829in}}%
\pgfpathlineto{\pgfqpoint{1.774808in}{0.136829in}}%
\pgfpathlineto{\pgfqpoint{1.770453in}{0.136829in}}%
\pgfpathlineto{\pgfqpoint{1.766098in}{0.136829in}}%
\pgfpathlineto{\pgfqpoint{1.761742in}{0.136829in}}%
\pgfpathlineto{\pgfqpoint{1.757387in}{0.136829in}}%
\pgfpathlineto{\pgfqpoint{1.753031in}{0.136829in}}%
\pgfpathlineto{\pgfqpoint{1.748676in}{0.136829in}}%
\pgfpathlineto{\pgfqpoint{1.744320in}{0.136829in}}%
\pgfpathlineto{\pgfqpoint{1.739965in}{0.136829in}}%
\pgfpathlineto{\pgfqpoint{1.735609in}{0.136829in}}%
\pgfpathlineto{\pgfqpoint{1.731254in}{0.136829in}}%
\pgfpathlineto{\pgfqpoint{1.726898in}{0.136829in}}%
\pgfpathlineto{\pgfqpoint{1.722543in}{0.136829in}}%
\pgfpathlineto{\pgfqpoint{1.718188in}{0.136829in}}%
\pgfpathlineto{\pgfqpoint{1.713832in}{0.136829in}}%
\pgfpathlineto{\pgfqpoint{1.709477in}{0.136829in}}%
\pgfpathlineto{\pgfqpoint{1.705121in}{0.136829in}}%
\pgfpathlineto{\pgfqpoint{1.700766in}{0.136829in}}%
\pgfpathlineto{\pgfqpoint{1.696410in}{0.136829in}}%
\pgfpathlineto{\pgfqpoint{1.692055in}{0.136829in}}%
\pgfpathlineto{\pgfqpoint{1.687699in}{0.136829in}}%
\pgfpathlineto{\pgfqpoint{1.683344in}{0.136829in}}%
\pgfpathlineto{\pgfqpoint{1.678988in}{0.136829in}}%
\pgfpathlineto{\pgfqpoint{1.674633in}{0.136829in}}%
\pgfpathlineto{\pgfqpoint{1.670277in}{0.136829in}}%
\pgfpathlineto{\pgfqpoint{1.665922in}{0.136829in}}%
\pgfpathlineto{\pgfqpoint{1.661567in}{0.136829in}}%
\pgfpathlineto{\pgfqpoint{1.657211in}{0.136829in}}%
\pgfpathlineto{\pgfqpoint{1.652856in}{0.136829in}}%
\pgfpathlineto{\pgfqpoint{1.648500in}{0.136829in}}%
\pgfpathlineto{\pgfqpoint{1.644145in}{0.136829in}}%
\pgfpathlineto{\pgfqpoint{1.639789in}{0.136829in}}%
\pgfpathlineto{\pgfqpoint{1.635434in}{0.136829in}}%
\pgfpathlineto{\pgfqpoint{1.631078in}{0.136829in}}%
\pgfpathlineto{\pgfqpoint{1.626723in}{0.136829in}}%
\pgfpathlineto{\pgfqpoint{1.622367in}{0.136829in}}%
\pgfpathlineto{\pgfqpoint{1.618012in}{0.136829in}}%
\pgfpathlineto{\pgfqpoint{1.613656in}{0.136829in}}%
\pgfpathlineto{\pgfqpoint{1.609301in}{0.136829in}}%
\pgfpathlineto{\pgfqpoint{1.604946in}{0.136829in}}%
\pgfpathlineto{\pgfqpoint{1.600590in}{0.136829in}}%
\pgfpathlineto{\pgfqpoint{1.596235in}{0.136829in}}%
\pgfpathlineto{\pgfqpoint{1.591879in}{0.136829in}}%
\pgfpathlineto{\pgfqpoint{1.587524in}{0.136829in}}%
\pgfpathlineto{\pgfqpoint{1.583168in}{0.136829in}}%
\pgfpathlineto{\pgfqpoint{1.578813in}{0.136829in}}%
\pgfpathlineto{\pgfqpoint{1.574457in}{0.136829in}}%
\pgfpathlineto{\pgfqpoint{1.570102in}{0.136829in}}%
\pgfpathlineto{\pgfqpoint{1.565746in}{0.136829in}}%
\pgfpathlineto{\pgfqpoint{1.561391in}{0.136829in}}%
\pgfpathlineto{\pgfqpoint{1.557036in}{0.136829in}}%
\pgfpathlineto{\pgfqpoint{1.552680in}{0.136829in}}%
\pgfpathlineto{\pgfqpoint{1.548325in}{0.136829in}}%
\pgfpathlineto{\pgfqpoint{1.543969in}{0.136829in}}%
\pgfpathlineto{\pgfqpoint{1.539614in}{0.136829in}}%
\pgfpathlineto{\pgfqpoint{1.535258in}{0.136829in}}%
\pgfpathlineto{\pgfqpoint{1.530903in}{0.136829in}}%
\pgfpathlineto{\pgfqpoint{1.526547in}{0.136829in}}%
\pgfpathlineto{\pgfqpoint{1.522192in}{0.136829in}}%
\pgfpathlineto{\pgfqpoint{1.517836in}{0.136829in}}%
\pgfpathlineto{\pgfqpoint{1.513481in}{0.136829in}}%
\pgfpathlineto{\pgfqpoint{1.509125in}{0.136829in}}%
\pgfpathlineto{\pgfqpoint{1.504770in}{0.136829in}}%
\pgfpathlineto{\pgfqpoint{1.500415in}{0.136829in}}%
\pgfpathlineto{\pgfqpoint{1.496059in}{0.136829in}}%
\pgfpathlineto{\pgfqpoint{1.491704in}{0.136829in}}%
\pgfpathlineto{\pgfqpoint{1.487348in}{0.136829in}}%
\pgfpathlineto{\pgfqpoint{1.482993in}{0.136829in}}%
\pgfpathlineto{\pgfqpoint{1.478637in}{0.136829in}}%
\pgfpathlineto{\pgfqpoint{1.474282in}{0.136829in}}%
\pgfpathlineto{\pgfqpoint{1.469926in}{0.136829in}}%
\pgfpathlineto{\pgfqpoint{1.465571in}{0.136829in}}%
\pgfpathlineto{\pgfqpoint{1.461215in}{0.136829in}}%
\pgfpathlineto{\pgfqpoint{1.456860in}{0.136829in}}%
\pgfpathlineto{\pgfqpoint{1.452504in}{0.136829in}}%
\pgfpathlineto{\pgfqpoint{1.448149in}{0.136829in}}%
\pgfpathlineto{\pgfqpoint{1.443794in}{0.136829in}}%
\pgfpathlineto{\pgfqpoint{1.439438in}{0.136829in}}%
\pgfpathlineto{\pgfqpoint{1.435083in}{0.136829in}}%
\pgfpathlineto{\pgfqpoint{1.430727in}{0.136829in}}%
\pgfpathlineto{\pgfqpoint{1.426372in}{0.136829in}}%
\pgfpathlineto{\pgfqpoint{1.422016in}{0.136829in}}%
\pgfpathlineto{\pgfqpoint{1.417661in}{0.136829in}}%
\pgfpathlineto{\pgfqpoint{1.413305in}{0.136829in}}%
\pgfpathlineto{\pgfqpoint{1.408950in}{0.136829in}}%
\pgfpathlineto{\pgfqpoint{1.404594in}{0.136829in}}%
\pgfpathlineto{\pgfqpoint{1.400239in}{0.136829in}}%
\pgfpathlineto{\pgfqpoint{1.395884in}{0.136829in}}%
\pgfpathlineto{\pgfqpoint{1.391528in}{0.136829in}}%
\pgfpathlineto{\pgfqpoint{1.387173in}{0.136829in}}%
\pgfpathlineto{\pgfqpoint{1.382817in}{0.136829in}}%
\pgfpathlineto{\pgfqpoint{1.378462in}{0.136829in}}%
\pgfpathlineto{\pgfqpoint{1.374106in}{0.136829in}}%
\pgfpathlineto{\pgfqpoint{1.369751in}{0.136829in}}%
\pgfpathlineto{\pgfqpoint{1.365395in}{0.136829in}}%
\pgfpathlineto{\pgfqpoint{1.361040in}{0.136829in}}%
\pgfpathlineto{\pgfqpoint{1.356684in}{0.136829in}}%
\pgfpathlineto{\pgfqpoint{1.352329in}{0.136829in}}%
\pgfpathlineto{\pgfqpoint{1.347973in}{0.136829in}}%
\pgfpathlineto{\pgfqpoint{1.343618in}{0.136829in}}%
\pgfpathlineto{\pgfqpoint{1.339263in}{0.136829in}}%
\pgfpathlineto{\pgfqpoint{1.334907in}{0.136829in}}%
\pgfpathlineto{\pgfqpoint{1.330552in}{0.136829in}}%
\pgfpathlineto{\pgfqpoint{1.326196in}{0.136829in}}%
\pgfpathlineto{\pgfqpoint{1.321841in}{0.136829in}}%
\pgfpathlineto{\pgfqpoint{1.317485in}{0.136829in}}%
\pgfpathlineto{\pgfqpoint{1.313130in}{0.136829in}}%
\pgfpathlineto{\pgfqpoint{1.308774in}{0.136829in}}%
\pgfpathlineto{\pgfqpoint{1.304419in}{0.136829in}}%
\pgfpathlineto{\pgfqpoint{1.300063in}{0.136829in}}%
\pgfpathlineto{\pgfqpoint{1.295708in}{0.136829in}}%
\pgfpathlineto{\pgfqpoint{1.291352in}{0.136829in}}%
\pgfpathlineto{\pgfqpoint{1.286997in}{0.136829in}}%
\pgfpathlineto{\pgfqpoint{1.282642in}{0.136829in}}%
\pgfpathlineto{\pgfqpoint{1.278286in}{0.136829in}}%
\pgfpathlineto{\pgfqpoint{1.273931in}{0.136829in}}%
\pgfpathlineto{\pgfqpoint{1.269575in}{0.136829in}}%
\pgfpathlineto{\pgfqpoint{1.265220in}{0.136829in}}%
\pgfpathlineto{\pgfqpoint{1.260864in}{0.136829in}}%
\pgfpathlineto{\pgfqpoint{1.256509in}{0.136829in}}%
\pgfpathlineto{\pgfqpoint{1.252153in}{0.136829in}}%
\pgfpathlineto{\pgfqpoint{1.247798in}{0.136829in}}%
\pgfpathlineto{\pgfqpoint{1.243442in}{0.136829in}}%
\pgfpathlineto{\pgfqpoint{1.239087in}{0.136829in}}%
\pgfpathlineto{\pgfqpoint{1.234732in}{0.136829in}}%
\pgfpathlineto{\pgfqpoint{1.230376in}{0.136829in}}%
\pgfpathlineto{\pgfqpoint{1.226021in}{0.136829in}}%
\pgfpathlineto{\pgfqpoint{1.221665in}{0.136829in}}%
\pgfpathlineto{\pgfqpoint{1.217310in}{0.136829in}}%
\pgfpathlineto{\pgfqpoint{1.212954in}{0.136829in}}%
\pgfpathlineto{\pgfqpoint{1.208599in}{0.136829in}}%
\pgfpathlineto{\pgfqpoint{1.204243in}{0.136829in}}%
\pgfpathlineto{\pgfqpoint{1.199888in}{0.136829in}}%
\pgfpathlineto{\pgfqpoint{1.195532in}{0.136829in}}%
\pgfpathlineto{\pgfqpoint{1.191177in}{0.136829in}}%
\pgfpathlineto{\pgfqpoint{1.186821in}{0.136829in}}%
\pgfpathlineto{\pgfqpoint{1.182466in}{0.136829in}}%
\pgfpathlineto{\pgfqpoint{1.178111in}{0.136829in}}%
\pgfpathlineto{\pgfqpoint{1.173755in}{0.136829in}}%
\pgfpathlineto{\pgfqpoint{1.169400in}{0.136829in}}%
\pgfpathlineto{\pgfqpoint{1.165044in}{0.136829in}}%
\pgfpathlineto{\pgfqpoint{1.160689in}{0.136829in}}%
\pgfpathlineto{\pgfqpoint{1.156333in}{0.136829in}}%
\pgfpathlineto{\pgfqpoint{1.151978in}{0.136829in}}%
\pgfpathlineto{\pgfqpoint{1.147622in}{0.136829in}}%
\pgfpathlineto{\pgfqpoint{1.143267in}{0.136829in}}%
\pgfpathlineto{\pgfqpoint{1.138911in}{0.136829in}}%
\pgfpathlineto{\pgfqpoint{1.134556in}{0.136829in}}%
\pgfpathlineto{\pgfqpoint{1.130201in}{0.136829in}}%
\pgfpathlineto{\pgfqpoint{1.125845in}{0.136829in}}%
\pgfpathlineto{\pgfqpoint{1.121490in}{0.136829in}}%
\pgfpathlineto{\pgfqpoint{1.117134in}{0.136829in}}%
\pgfpathlineto{\pgfqpoint{1.112779in}{0.136829in}}%
\pgfpathlineto{\pgfqpoint{1.108423in}{0.136829in}}%
\pgfpathlineto{\pgfqpoint{1.104068in}{0.136829in}}%
\pgfpathlineto{\pgfqpoint{1.099712in}{0.136829in}}%
\pgfpathlineto{\pgfqpoint{1.095357in}{0.136829in}}%
\pgfpathlineto{\pgfqpoint{1.091001in}{0.136829in}}%
\pgfpathlineto{\pgfqpoint{1.086646in}{0.136829in}}%
\pgfpathlineto{\pgfqpoint{1.082290in}{0.136829in}}%
\pgfpathlineto{\pgfqpoint{1.077935in}{0.136829in}}%
\pgfpathlineto{\pgfqpoint{1.073580in}{0.136829in}}%
\pgfpathlineto{\pgfqpoint{1.069224in}{0.136829in}}%
\pgfpathlineto{\pgfqpoint{1.064869in}{0.136829in}}%
\pgfpathlineto{\pgfqpoint{1.060513in}{0.136829in}}%
\pgfpathlineto{\pgfqpoint{1.056158in}{0.136829in}}%
\pgfpathlineto{\pgfqpoint{1.051802in}{0.136829in}}%
\pgfpathlineto{\pgfqpoint{1.047447in}{0.136829in}}%
\pgfpathlineto{\pgfqpoint{1.043091in}{0.136829in}}%
\pgfpathlineto{\pgfqpoint{1.038736in}{0.136829in}}%
\pgfpathlineto{\pgfqpoint{1.034380in}{0.136829in}}%
\pgfpathlineto{\pgfqpoint{1.030025in}{0.136829in}}%
\pgfpathlineto{\pgfqpoint{1.025669in}{0.136829in}}%
\pgfpathlineto{\pgfqpoint{1.021314in}{0.136829in}}%
\pgfpathlineto{\pgfqpoint{1.016959in}{0.136829in}}%
\pgfpathlineto{\pgfqpoint{1.012603in}{0.136829in}}%
\pgfpathlineto{\pgfqpoint{1.008248in}{0.136829in}}%
\pgfpathlineto{\pgfqpoint{1.003892in}{0.136829in}}%
\pgfpathlineto{\pgfqpoint{0.999537in}{0.136829in}}%
\pgfpathlineto{\pgfqpoint{0.995181in}{0.136829in}}%
\pgfpathlineto{\pgfqpoint{0.990826in}{0.136829in}}%
\pgfpathlineto{\pgfqpoint{0.986470in}{0.136829in}}%
\pgfpathlineto{\pgfqpoint{0.982115in}{0.136829in}}%
\pgfpathlineto{\pgfqpoint{0.977759in}{0.136829in}}%
\pgfpathlineto{\pgfqpoint{0.973404in}{0.136829in}}%
\pgfpathlineto{\pgfqpoint{0.969049in}{0.136829in}}%
\pgfpathlineto{\pgfqpoint{0.964693in}{0.136829in}}%
\pgfpathlineto{\pgfqpoint{0.960338in}{0.136829in}}%
\pgfpathlineto{\pgfqpoint{0.955982in}{0.136829in}}%
\pgfpathlineto{\pgfqpoint{0.951627in}{0.136829in}}%
\pgfpathlineto{\pgfqpoint{0.947271in}{0.136829in}}%
\pgfpathlineto{\pgfqpoint{0.942916in}{0.136829in}}%
\pgfpathlineto{\pgfqpoint{0.938560in}{0.136829in}}%
\pgfpathlineto{\pgfqpoint{0.934205in}{0.136829in}}%
\pgfpathlineto{\pgfqpoint{0.929849in}{0.136829in}}%
\pgfpathlineto{\pgfqpoint{0.925494in}{0.136829in}}%
\pgfpathlineto{\pgfqpoint{0.921138in}{0.136829in}}%
\pgfpathlineto{\pgfqpoint{0.916783in}{0.136829in}}%
\pgfpathlineto{\pgfqpoint{0.912428in}{0.136829in}}%
\pgfpathlineto{\pgfqpoint{0.908072in}{0.136829in}}%
\pgfpathlineto{\pgfqpoint{0.903717in}{0.136829in}}%
\pgfpathlineto{\pgfqpoint{0.899361in}{0.136829in}}%
\pgfpathlineto{\pgfqpoint{0.895006in}{0.136829in}}%
\pgfpathlineto{\pgfqpoint{0.890650in}{0.136829in}}%
\pgfpathlineto{\pgfqpoint{0.886295in}{0.136829in}}%
\pgfpathlineto{\pgfqpoint{0.881939in}{0.136829in}}%
\pgfpathlineto{\pgfqpoint{0.877584in}{0.136829in}}%
\pgfpathlineto{\pgfqpoint{0.873228in}{0.136829in}}%
\pgfpathlineto{\pgfqpoint{0.868873in}{0.136829in}}%
\pgfpathlineto{\pgfqpoint{0.864517in}{0.136829in}}%
\pgfpathlineto{\pgfqpoint{0.860162in}{0.136829in}}%
\pgfpathlineto{\pgfqpoint{0.855807in}{0.136829in}}%
\pgfpathlineto{\pgfqpoint{0.851451in}{0.136829in}}%
\pgfpathlineto{\pgfqpoint{0.847096in}{0.136829in}}%
\pgfpathlineto{\pgfqpoint{0.842740in}{0.136829in}}%
\pgfpathlineto{\pgfqpoint{0.838385in}{0.136829in}}%
\pgfpathlineto{\pgfqpoint{0.834029in}{0.136829in}}%
\pgfpathlineto{\pgfqpoint{0.829674in}{0.136829in}}%
\pgfpathlineto{\pgfqpoint{0.825318in}{0.136829in}}%
\pgfpathlineto{\pgfqpoint{0.820963in}{0.136829in}}%
\pgfpathlineto{\pgfqpoint{0.816607in}{0.136829in}}%
\pgfpathlineto{\pgfqpoint{0.812252in}{0.136829in}}%
\pgfpathlineto{\pgfqpoint{0.807897in}{0.136829in}}%
\pgfpathlineto{\pgfqpoint{0.803541in}{0.136829in}}%
\pgfpathlineto{\pgfqpoint{0.799186in}{0.136829in}}%
\pgfpathlineto{\pgfqpoint{0.794830in}{0.136829in}}%
\pgfpathlineto{\pgfqpoint{0.790475in}{0.136829in}}%
\pgfpathlineto{\pgfqpoint{0.786119in}{0.136829in}}%
\pgfpathlineto{\pgfqpoint{0.781764in}{0.136829in}}%
\pgfpathlineto{\pgfqpoint{0.777408in}{0.136829in}}%
\pgfpathlineto{\pgfqpoint{0.773053in}{0.136829in}}%
\pgfpathlineto{\pgfqpoint{0.768697in}{0.136829in}}%
\pgfpathlineto{\pgfqpoint{0.764342in}{0.136829in}}%
\pgfpathlineto{\pgfqpoint{0.759986in}{0.136829in}}%
\pgfpathlineto{\pgfqpoint{0.755631in}{0.136829in}}%
\pgfpathlineto{\pgfqpoint{0.751276in}{0.136829in}}%
\pgfpathlineto{\pgfqpoint{0.751276in}{0.136829in}}%
\pgfpathclose%
\pgfusepath{stroke,fill}%
}%
\begin{pgfscope}%
\pgfsys@transformshift{0.000000in}{0.000000in}%
\pgfsys@useobject{currentmarker}{}%
\end{pgfscope}%
\end{pgfscope}%
\begin{pgfscope}%
\pgfsetbuttcap%
\pgfsetroundjoin%
\definecolor{currentfill}{rgb}{0.000000,0.000000,0.000000}%
\pgfsetfillcolor{currentfill}%
\pgfsetlinewidth{0.803000pt}%
\definecolor{currentstroke}{rgb}{0.000000,0.000000,0.000000}%
\pgfsetstrokecolor{currentstroke}%
\pgfsetdash{}{0pt}%
\pgfsys@defobject{currentmarker}{\pgfqpoint{0.000000in}{-0.048611in}}{\pgfqpoint{0.000000in}{0.000000in}}{%
\pgfpathmoveto{\pgfqpoint{0.000000in}{0.000000in}}%
\pgfpathlineto{\pgfqpoint{0.000000in}{-0.048611in}}%
\pgfusepath{stroke,fill}%
}%
\begin{pgfscope}%
\pgfsys@transformshift{1.414186in}{0.554012in}%
\pgfsys@useobject{currentmarker}{}%
\end{pgfscope}%
\end{pgfscope}%
\begin{pgfscope}%
\definecolor{textcolor}{rgb}{0.000000,0.000000,0.000000}%
\pgfsetstrokecolor{textcolor}%
\pgfsetfillcolor{textcolor}%
\pgftext[x=1.414186in,y=0.456790in,,top]{\color{textcolor}\rmfamily\fontsize{10.000000}{12.000000}\selectfont \(\displaystyle {0.01}\)}%
\end{pgfscope}%
\begin{pgfscope}%
\pgfsetbuttcap%
\pgfsetroundjoin%
\definecolor{currentfill}{rgb}{0.000000,0.000000,0.000000}%
\pgfsetfillcolor{currentfill}%
\pgfsetlinewidth{0.803000pt}%
\definecolor{currentstroke}{rgb}{0.000000,0.000000,0.000000}%
\pgfsetstrokecolor{currentstroke}%
\pgfsetdash{}{0pt}%
\pgfsys@defobject{currentmarker}{\pgfqpoint{0.000000in}{-0.048611in}}{\pgfqpoint{0.000000in}{0.000000in}}{%
\pgfpathmoveto{\pgfqpoint{0.000000in}{0.000000in}}%
\pgfpathlineto{\pgfqpoint{0.000000in}{-0.048611in}}%
\pgfusepath{stroke,fill}%
}%
\begin{pgfscope}%
\pgfsys@transformshift{2.504499in}{0.554012in}%
\pgfsys@useobject{currentmarker}{}%
\end{pgfscope}%
\end{pgfscope}%
\begin{pgfscope}%
\definecolor{textcolor}{rgb}{0.000000,0.000000,0.000000}%
\pgfsetstrokecolor{textcolor}%
\pgfsetfillcolor{textcolor}%
\pgftext[x=2.504499in,y=0.456790in,,top]{\color{textcolor}\rmfamily\fontsize{10.000000}{12.000000}\selectfont \(\displaystyle {0.02}\)}%
\end{pgfscope}%
\begin{pgfscope}%
\pgfsetbuttcap%
\pgfsetroundjoin%
\definecolor{currentfill}{rgb}{0.000000,0.000000,0.000000}%
\pgfsetfillcolor{currentfill}%
\pgfsetlinewidth{0.803000pt}%
\definecolor{currentstroke}{rgb}{0.000000,0.000000,0.000000}%
\pgfsetstrokecolor{currentstroke}%
\pgfsetdash{}{0pt}%
\pgfsys@defobject{currentmarker}{\pgfqpoint{0.000000in}{-0.048611in}}{\pgfqpoint{0.000000in}{0.000000in}}{%
\pgfpathmoveto{\pgfqpoint{0.000000in}{0.000000in}}%
\pgfpathlineto{\pgfqpoint{0.000000in}{-0.048611in}}%
\pgfusepath{stroke,fill}%
}%
\begin{pgfscope}%
\pgfsys@transformshift{3.594811in}{0.554012in}%
\pgfsys@useobject{currentmarker}{}%
\end{pgfscope}%
\end{pgfscope}%
\begin{pgfscope}%
\definecolor{textcolor}{rgb}{0.000000,0.000000,0.000000}%
\pgfsetstrokecolor{textcolor}%
\pgfsetfillcolor{textcolor}%
\pgftext[x=3.594811in,y=0.456790in,,top]{\color{textcolor}\rmfamily\fontsize{10.000000}{12.000000}\selectfont \(\displaystyle {0.03}\)}%
\end{pgfscope}%
\begin{pgfscope}%
\pgfsetbuttcap%
\pgfsetroundjoin%
\definecolor{currentfill}{rgb}{0.000000,0.000000,0.000000}%
\pgfsetfillcolor{currentfill}%
\pgfsetlinewidth{0.803000pt}%
\definecolor{currentstroke}{rgb}{0.000000,0.000000,0.000000}%
\pgfsetstrokecolor{currentstroke}%
\pgfsetdash{}{0pt}%
\pgfsys@defobject{currentmarker}{\pgfqpoint{0.000000in}{-0.048611in}}{\pgfqpoint{0.000000in}{0.000000in}}{%
\pgfpathmoveto{\pgfqpoint{0.000000in}{0.000000in}}%
\pgfpathlineto{\pgfqpoint{0.000000in}{-0.048611in}}%
\pgfusepath{stroke,fill}%
}%
\begin{pgfscope}%
\pgfsys@transformshift{4.685124in}{0.554012in}%
\pgfsys@useobject{currentmarker}{}%
\end{pgfscope}%
\end{pgfscope}%
\begin{pgfscope}%
\definecolor{textcolor}{rgb}{0.000000,0.000000,0.000000}%
\pgfsetstrokecolor{textcolor}%
\pgfsetfillcolor{textcolor}%
\pgftext[x=4.685124in,y=0.456790in,,top]{\color{textcolor}\rmfamily\fontsize{10.000000}{12.000000}\selectfont \(\displaystyle {0.04}\)}%
\end{pgfscope}%
\begin{pgfscope}%
\pgfsetbuttcap%
\pgfsetroundjoin%
\definecolor{currentfill}{rgb}{0.000000,0.000000,0.000000}%
\pgfsetfillcolor{currentfill}%
\pgfsetlinewidth{0.803000pt}%
\definecolor{currentstroke}{rgb}{0.000000,0.000000,0.000000}%
\pgfsetstrokecolor{currentstroke}%
\pgfsetdash{}{0pt}%
\pgfsys@defobject{currentmarker}{\pgfqpoint{0.000000in}{-0.048611in}}{\pgfqpoint{0.000000in}{0.000000in}}{%
\pgfpathmoveto{\pgfqpoint{0.000000in}{0.000000in}}%
\pgfpathlineto{\pgfqpoint{0.000000in}{-0.048611in}}%
\pgfusepath{stroke,fill}%
}%
\begin{pgfscope}%
\pgfsys@transformshift{5.775437in}{0.554012in}%
\pgfsys@useobject{currentmarker}{}%
\end{pgfscope}%
\end{pgfscope}%
\begin{pgfscope}%
\definecolor{textcolor}{rgb}{0.000000,0.000000,0.000000}%
\pgfsetstrokecolor{textcolor}%
\pgfsetfillcolor{textcolor}%
\pgftext[x=5.775437in,y=0.456790in,,top]{\color{textcolor}\rmfamily\fontsize{10.000000}{12.000000}\selectfont \(\displaystyle {0.05}\)}%
\end{pgfscope}%
\begin{pgfscope}%
\pgfsetbuttcap%
\pgfsetroundjoin%
\definecolor{currentfill}{rgb}{0.000000,0.000000,0.000000}%
\pgfsetfillcolor{currentfill}%
\pgfsetlinewidth{0.803000pt}%
\definecolor{currentstroke}{rgb}{0.000000,0.000000,0.000000}%
\pgfsetstrokecolor{currentstroke}%
\pgfsetdash{}{0pt}%
\pgfsys@defobject{currentmarker}{\pgfqpoint{0.000000in}{-0.048611in}}{\pgfqpoint{0.000000in}{0.000000in}}{%
\pgfpathmoveto{\pgfqpoint{0.000000in}{0.000000in}}%
\pgfpathlineto{\pgfqpoint{0.000000in}{-0.048611in}}%
\pgfusepath{stroke,fill}%
}%
\begin{pgfscope}%
\pgfsys@transformshift{6.865750in}{0.554012in}%
\pgfsys@useobject{currentmarker}{}%
\end{pgfscope}%
\end{pgfscope}%
\begin{pgfscope}%
\definecolor{textcolor}{rgb}{0.000000,0.000000,0.000000}%
\pgfsetstrokecolor{textcolor}%
\pgfsetfillcolor{textcolor}%
\pgftext[x=6.865750in,y=0.456790in,,top]{\color{textcolor}\rmfamily\fontsize{10.000000}{12.000000}\selectfont \(\displaystyle {0.06}\)}%
\end{pgfscope}%
\begin{pgfscope}%
\definecolor{textcolor}{rgb}{0.000000,0.000000,0.000000}%
\pgfsetstrokecolor{textcolor}%
\pgfsetfillcolor{textcolor}%
\pgftext[x=3.947223in,y=0.277777in,,top]{\color{textcolor}\rmfamily\fontsize{14.000000}{16.800000}\selectfont f1}%
\end{pgfscope}%
\begin{pgfscope}%
\pgfsetbuttcap%
\pgfsetroundjoin%
\definecolor{currentfill}{rgb}{0.000000,0.000000,0.000000}%
\pgfsetfillcolor{currentfill}%
\pgfsetlinewidth{0.803000pt}%
\definecolor{currentstroke}{rgb}{0.000000,0.000000,0.000000}%
\pgfsetstrokecolor{currentstroke}%
\pgfsetdash{}{0pt}%
\pgfsys@defobject{currentmarker}{\pgfqpoint{-0.048611in}{0.000000in}}{\pgfqpoint{-0.000000in}{0.000000in}}{%
\pgfpathmoveto{\pgfqpoint{-0.000000in}{0.000000in}}%
\pgfpathlineto{\pgfqpoint{-0.048611in}{0.000000in}}%
\pgfusepath{stroke,fill}%
}%
\begin{pgfscope}%
\pgfsys@transformshift{0.847223in}{0.961359in}%
\pgfsys@useobject{currentmarker}{}%
\end{pgfscope}%
\end{pgfscope}%
\begin{pgfscope}%
\definecolor{textcolor}{rgb}{0.000000,0.000000,0.000000}%
\pgfsetstrokecolor{textcolor}%
\pgfsetfillcolor{textcolor}%
\pgftext[x=0.402777in, y=0.913134in, left, base]{\color{textcolor}\rmfamily\fontsize{10.000000}{12.000000}\selectfont \(\displaystyle {20000}\)}%
\end{pgfscope}%
\begin{pgfscope}%
\pgfsetbuttcap%
\pgfsetroundjoin%
\definecolor{currentfill}{rgb}{0.000000,0.000000,0.000000}%
\pgfsetfillcolor{currentfill}%
\pgfsetlinewidth{0.803000pt}%
\definecolor{currentstroke}{rgb}{0.000000,0.000000,0.000000}%
\pgfsetstrokecolor{currentstroke}%
\pgfsetdash{}{0pt}%
\pgfsys@defobject{currentmarker}{\pgfqpoint{-0.048611in}{0.000000in}}{\pgfqpoint{-0.000000in}{0.000000in}}{%
\pgfpathmoveto{\pgfqpoint{-0.000000in}{0.000000in}}%
\pgfpathlineto{\pgfqpoint{-0.048611in}{0.000000in}}%
\pgfusepath{stroke,fill}%
}%
\begin{pgfscope}%
\pgfsys@transformshift{0.847223in}{1.785890in}%
\pgfsys@useobject{currentmarker}{}%
\end{pgfscope}%
\end{pgfscope}%
\begin{pgfscope}%
\definecolor{textcolor}{rgb}{0.000000,0.000000,0.000000}%
\pgfsetstrokecolor{textcolor}%
\pgfsetfillcolor{textcolor}%
\pgftext[x=0.402777in, y=1.737664in, left, base]{\color{textcolor}\rmfamily\fontsize{10.000000}{12.000000}\selectfont \(\displaystyle {40000}\)}%
\end{pgfscope}%
\begin{pgfscope}%
\pgfsetbuttcap%
\pgfsetroundjoin%
\definecolor{currentfill}{rgb}{0.000000,0.000000,0.000000}%
\pgfsetfillcolor{currentfill}%
\pgfsetlinewidth{0.803000pt}%
\definecolor{currentstroke}{rgb}{0.000000,0.000000,0.000000}%
\pgfsetstrokecolor{currentstroke}%
\pgfsetdash{}{0pt}%
\pgfsys@defobject{currentmarker}{\pgfqpoint{-0.048611in}{0.000000in}}{\pgfqpoint{-0.000000in}{0.000000in}}{%
\pgfpathmoveto{\pgfqpoint{-0.000000in}{0.000000in}}%
\pgfpathlineto{\pgfqpoint{-0.048611in}{0.000000in}}%
\pgfusepath{stroke,fill}%
}%
\begin{pgfscope}%
\pgfsys@transformshift{0.847223in}{2.610420in}%
\pgfsys@useobject{currentmarker}{}%
\end{pgfscope}%
\end{pgfscope}%
\begin{pgfscope}%
\definecolor{textcolor}{rgb}{0.000000,0.000000,0.000000}%
\pgfsetstrokecolor{textcolor}%
\pgfsetfillcolor{textcolor}%
\pgftext[x=0.402777in, y=2.562195in, left, base]{\color{textcolor}\rmfamily\fontsize{10.000000}{12.000000}\selectfont \(\displaystyle {60000}\)}%
\end{pgfscope}%
\begin{pgfscope}%
\pgfsetbuttcap%
\pgfsetroundjoin%
\definecolor{currentfill}{rgb}{0.000000,0.000000,0.000000}%
\pgfsetfillcolor{currentfill}%
\pgfsetlinewidth{0.803000pt}%
\definecolor{currentstroke}{rgb}{0.000000,0.000000,0.000000}%
\pgfsetstrokecolor{currentstroke}%
\pgfsetdash{}{0pt}%
\pgfsys@defobject{currentmarker}{\pgfqpoint{-0.048611in}{0.000000in}}{\pgfqpoint{-0.000000in}{0.000000in}}{%
\pgfpathmoveto{\pgfqpoint{-0.000000in}{0.000000in}}%
\pgfpathlineto{\pgfqpoint{-0.048611in}{0.000000in}}%
\pgfusepath{stroke,fill}%
}%
\begin{pgfscope}%
\pgfsys@transformshift{0.847223in}{3.434951in}%
\pgfsys@useobject{currentmarker}{}%
\end{pgfscope}%
\end{pgfscope}%
\begin{pgfscope}%
\definecolor{textcolor}{rgb}{0.000000,0.000000,0.000000}%
\pgfsetstrokecolor{textcolor}%
\pgfsetfillcolor{textcolor}%
\pgftext[x=0.402777in, y=3.386725in, left, base]{\color{textcolor}\rmfamily\fontsize{10.000000}{12.000000}\selectfont \(\displaystyle {80000}\)}%
\end{pgfscope}%
\begin{pgfscope}%
\pgfsetbuttcap%
\pgfsetroundjoin%
\definecolor{currentfill}{rgb}{0.000000,0.000000,0.000000}%
\pgfsetfillcolor{currentfill}%
\pgfsetlinewidth{0.803000pt}%
\definecolor{currentstroke}{rgb}{0.000000,0.000000,0.000000}%
\pgfsetstrokecolor{currentstroke}%
\pgfsetdash{}{0pt}%
\pgfsys@defobject{currentmarker}{\pgfqpoint{-0.048611in}{0.000000in}}{\pgfqpoint{-0.000000in}{0.000000in}}{%
\pgfpathmoveto{\pgfqpoint{-0.000000in}{0.000000in}}%
\pgfpathlineto{\pgfqpoint{-0.048611in}{0.000000in}}%
\pgfusepath{stroke,fill}%
}%
\begin{pgfscope}%
\pgfsys@transformshift{0.847223in}{4.259481in}%
\pgfsys@useobject{currentmarker}{}%
\end{pgfscope}%
\end{pgfscope}%
\begin{pgfscope}%
\definecolor{textcolor}{rgb}{0.000000,0.000000,0.000000}%
\pgfsetstrokecolor{textcolor}%
\pgfsetfillcolor{textcolor}%
\pgftext[x=0.333333in, y=4.211256in, left, base]{\color{textcolor}\rmfamily\fontsize{10.000000}{12.000000}\selectfont \(\displaystyle {100000}\)}%
\end{pgfscope}%
\begin{pgfscope}%
\pgfsetbuttcap%
\pgfsetroundjoin%
\definecolor{currentfill}{rgb}{0.000000,0.000000,0.000000}%
\pgfsetfillcolor{currentfill}%
\pgfsetlinewidth{0.803000pt}%
\definecolor{currentstroke}{rgb}{0.000000,0.000000,0.000000}%
\pgfsetstrokecolor{currentstroke}%
\pgfsetdash{}{0pt}%
\pgfsys@defobject{currentmarker}{\pgfqpoint{-0.048611in}{0.000000in}}{\pgfqpoint{-0.000000in}{0.000000in}}{%
\pgfpathmoveto{\pgfqpoint{-0.000000in}{0.000000in}}%
\pgfpathlineto{\pgfqpoint{-0.048611in}{0.000000in}}%
\pgfusepath{stroke,fill}%
}%
\begin{pgfscope}%
\pgfsys@transformshift{0.847223in}{5.084012in}%
\pgfsys@useobject{currentmarker}{}%
\end{pgfscope}%
\end{pgfscope}%
\begin{pgfscope}%
\definecolor{textcolor}{rgb}{0.000000,0.000000,0.000000}%
\pgfsetstrokecolor{textcolor}%
\pgfsetfillcolor{textcolor}%
\pgftext[x=0.333333in, y=5.035787in, left, base]{\color{textcolor}\rmfamily\fontsize{10.000000}{12.000000}\selectfont \(\displaystyle {120000}\)}%
\end{pgfscope}%
\begin{pgfscope}%
\definecolor{textcolor}{rgb}{0.000000,0.000000,0.000000}%
\pgfsetstrokecolor{textcolor}%
\pgfsetfillcolor{textcolor}%
\pgftext[x=0.277777in,y=2.819012in,,bottom,rotate=90.000000]{\color{textcolor}\rmfamily\fontsize{14.000000}{16.800000}\selectfont f2}%
\end{pgfscope}%
\begin{pgfscope}%
\pgfpathrectangle{\pgfqpoint{0.847223in}{0.554012in}}{\pgfqpoint{6.200000in}{4.530000in}}%
\pgfusepath{clip}%
\pgfsetrectcap%
\pgfsetroundjoin%
\pgfsetlinewidth{3.011250pt}%
\definecolor{currentstroke}{rgb}{0.000000,0.000000,0.000000}%
\pgfsetstrokecolor{currentstroke}%
\pgfsetdash{}{0pt}%
\pgfpathmoveto{\pgfqpoint{0.845556in}{3.583395in}}%
\pgfpathlineto{\pgfqpoint{0.853329in}{3.532750in}}%
\pgfpathlineto{\pgfqpoint{0.862218in}{3.476680in}}%
\pgfpathlineto{\pgfqpoint{0.871107in}{3.422431in}}%
\pgfpathlineto{\pgfqpoint{0.879995in}{3.369916in}}%
\pgfpathlineto{\pgfqpoint{0.888884in}{3.319053in}}%
\pgfpathlineto{\pgfqpoint{0.897773in}{3.269766in}}%
\pgfpathlineto{\pgfqpoint{0.906661in}{3.221983in}}%
\pgfpathlineto{\pgfqpoint{0.915550in}{3.175635in}}%
\pgfpathlineto{\pgfqpoint{0.924439in}{3.130659in}}%
\pgfpathlineto{\pgfqpoint{0.933328in}{3.086995in}}%
\pgfpathlineto{\pgfqpoint{0.942216in}{3.044586in}}%
\pgfpathlineto{\pgfqpoint{0.955549in}{2.983211in}}%
\pgfpathlineto{\pgfqpoint{0.968882in}{2.924373in}}%
\pgfpathlineto{\pgfqpoint{0.982215in}{2.867919in}}%
\pgfpathlineto{\pgfqpoint{0.995548in}{2.813705in}}%
\pgfpathlineto{\pgfqpoint{1.008881in}{2.761602in}}%
\pgfpathlineto{\pgfqpoint{1.022214in}{2.711489in}}%
\pgfpathlineto{\pgfqpoint{1.035548in}{2.663254in}}%
\pgfpathlineto{\pgfqpoint{1.048881in}{2.616792in}}%
\pgfpathlineto{\pgfqpoint{1.062214in}{2.572009in}}%
\pgfpathlineto{\pgfqpoint{1.075547in}{2.528814in}}%
\pgfpathlineto{\pgfqpoint{1.088880in}{2.487125in}}%
\pgfpathlineto{\pgfqpoint{1.102213in}{2.446864in}}%
\pgfpathlineto{\pgfqpoint{1.115546in}{2.407959in}}%
\pgfpathlineto{\pgfqpoint{1.128879in}{2.370343in}}%
\pgfpathlineto{\pgfqpoint{1.142212in}{2.333953in}}%
\pgfpathlineto{\pgfqpoint{1.155545in}{2.298730in}}%
\pgfpathlineto{\pgfqpoint{1.168878in}{2.264618in}}%
\pgfpathlineto{\pgfqpoint{1.182211in}{2.231566in}}%
\pgfpathlineto{\pgfqpoint{1.195544in}{2.199525in}}%
\pgfpathlineto{\pgfqpoint{1.208877in}{2.168449in}}%
\pgfpathlineto{\pgfqpoint{1.222210in}{2.138296in}}%
\pgfpathlineto{\pgfqpoint{1.235543in}{2.109025in}}%
\pgfpathlineto{\pgfqpoint{1.248876in}{2.080597in}}%
\pgfpathlineto{\pgfqpoint{1.262209in}{2.052978in}}%
\pgfpathlineto{\pgfqpoint{1.275542in}{2.026132in}}%
\pgfpathlineto{\pgfqpoint{1.288875in}{2.000029in}}%
\pgfpathlineto{\pgfqpoint{1.302208in}{1.974636in}}%
\pgfpathlineto{\pgfqpoint{1.315541in}{1.949927in}}%
\pgfpathlineto{\pgfqpoint{1.328874in}{1.925873in}}%
\pgfpathlineto{\pgfqpoint{1.342207in}{1.902449in}}%
\pgfpathlineto{\pgfqpoint{1.355540in}{1.879631in}}%
\pgfpathlineto{\pgfqpoint{1.368873in}{1.857395in}}%
\pgfpathlineto{\pgfqpoint{1.382207in}{1.835719in}}%
\pgfpathlineto{\pgfqpoint{1.395540in}{1.814582in}}%
\pgfpathlineto{\pgfqpoint{1.413317in}{1.787205in}}%
\pgfpathlineto{\pgfqpoint{1.431094in}{1.760707in}}%
\pgfpathlineto{\pgfqpoint{1.448872in}{1.735046in}}%
\pgfpathlineto{\pgfqpoint{1.466649in}{1.710183in}}%
\pgfpathlineto{\pgfqpoint{1.484426in}{1.686083in}}%
\pgfpathlineto{\pgfqpoint{1.502204in}{1.662709in}}%
\pgfpathlineto{\pgfqpoint{1.519981in}{1.640031in}}%
\pgfpathlineto{\pgfqpoint{1.537759in}{1.618016in}}%
\pgfpathlineto{\pgfqpoint{1.555536in}{1.596637in}}%
\pgfpathlineto{\pgfqpoint{1.573313in}{1.575867in}}%
\pgfpathlineto{\pgfqpoint{1.591091in}{1.555679in}}%
\pgfpathlineto{\pgfqpoint{1.608868in}{1.536050in}}%
\pgfpathlineto{\pgfqpoint{1.626646in}{1.516956in}}%
\pgfpathlineto{\pgfqpoint{1.644423in}{1.498377in}}%
\pgfpathlineto{\pgfqpoint{1.662200in}{1.480291in}}%
\pgfpathlineto{\pgfqpoint{1.679978in}{1.462679in}}%
\pgfpathlineto{\pgfqpoint{1.697755in}{1.445523in}}%
\pgfpathlineto{\pgfqpoint{1.715533in}{1.428806in}}%
\pgfpathlineto{\pgfqpoint{1.733310in}{1.412510in}}%
\pgfpathlineto{\pgfqpoint{1.755532in}{1.392709in}}%
\pgfpathlineto{\pgfqpoint{1.777753in}{1.373514in}}%
\pgfpathlineto{\pgfqpoint{1.799975in}{1.354896in}}%
\pgfpathlineto{\pgfqpoint{1.822197in}{1.336831in}}%
\pgfpathlineto{\pgfqpoint{1.844419in}{1.319294in}}%
\pgfpathlineto{\pgfqpoint{1.866640in}{1.302262in}}%
\pgfpathlineto{\pgfqpoint{1.888862in}{1.285714in}}%
\pgfpathlineto{\pgfqpoint{1.911084in}{1.269629in}}%
\pgfpathlineto{\pgfqpoint{1.933305in}{1.253988in}}%
\pgfpathlineto{\pgfqpoint{1.955527in}{1.238773in}}%
\pgfpathlineto{\pgfqpoint{1.977749in}{1.223967in}}%
\pgfpathlineto{\pgfqpoint{1.999971in}{1.209554in}}%
\pgfpathlineto{\pgfqpoint{2.022192in}{1.195518in}}%
\pgfpathlineto{\pgfqpoint{2.048858in}{1.179152in}}%
\pgfpathlineto{\pgfqpoint{2.075525in}{1.163284in}}%
\pgfpathlineto{\pgfqpoint{2.102191in}{1.147892in}}%
\pgfpathlineto{\pgfqpoint{2.128857in}{1.132955in}}%
\pgfpathlineto{\pgfqpoint{2.155523in}{1.118453in}}%
\pgfpathlineto{\pgfqpoint{2.182189in}{1.104367in}}%
\pgfpathlineto{\pgfqpoint{2.208855in}{1.090680in}}%
\pgfpathlineto{\pgfqpoint{2.235521in}{1.077374in}}%
\pgfpathlineto{\pgfqpoint{2.262187in}{1.064435in}}%
\pgfpathlineto{\pgfqpoint{2.293298in}{1.049782in}}%
\pgfpathlineto{\pgfqpoint{2.324408in}{1.035584in}}%
\pgfpathlineto{\pgfqpoint{2.355518in}{1.021822in}}%
\pgfpathlineto{\pgfqpoint{2.386629in}{1.008474in}}%
\pgfpathlineto{\pgfqpoint{2.417739in}{0.995524in}}%
\pgfpathlineto{\pgfqpoint{2.448850in}{0.982952in}}%
\pgfpathlineto{\pgfqpoint{2.479960in}{0.970743in}}%
\pgfpathlineto{\pgfqpoint{2.515515in}{0.957215in}}%
\pgfpathlineto{\pgfqpoint{2.551070in}{0.944118in}}%
\pgfpathlineto{\pgfqpoint{2.586624in}{0.931433in}}%
\pgfpathlineto{\pgfqpoint{2.622179in}{0.919141in}}%
\pgfpathlineto{\pgfqpoint{2.657734in}{0.907223in}}%
\pgfpathlineto{\pgfqpoint{2.693289in}{0.895662in}}%
\pgfpathlineto{\pgfqpoint{2.733288in}{0.883065in}}%
\pgfpathlineto{\pgfqpoint{2.773287in}{0.870879in}}%
\pgfpathlineto{\pgfqpoint{2.813286in}{0.859084in}}%
\pgfpathlineto{\pgfqpoint{2.853285in}{0.847663in}}%
\pgfpathlineto{\pgfqpoint{2.893284in}{0.836597in}}%
\pgfpathlineto{\pgfqpoint{2.937728in}{0.824699in}}%
\pgfpathlineto{\pgfqpoint{2.982171in}{0.813198in}}%
\pgfpathlineto{\pgfqpoint{3.026615in}{0.802076in}}%
\pgfpathlineto{\pgfqpoint{3.071058in}{0.791314in}}%
\pgfpathlineto{\pgfqpoint{3.119946in}{0.779871in}}%
\pgfpathlineto{\pgfqpoint{3.168834in}{0.768821in}}%
\pgfpathlineto{\pgfqpoint{3.217722in}{0.758144in}}%
\pgfpathlineto{\pgfqpoint{3.266609in}{0.747822in}}%
\pgfpathlineto{\pgfqpoint{3.319942in}{0.736946in}}%
\pgfpathlineto{\pgfqpoint{3.373274in}{0.726450in}}%
\pgfpathlineto{\pgfqpoint{3.426606in}{0.716315in}}%
\pgfpathlineto{\pgfqpoint{3.484382in}{0.705722in}}%
\pgfpathlineto{\pgfqpoint{3.542159in}{0.695509in}}%
\pgfpathlineto{\pgfqpoint{3.599935in}{0.685656in}}%
\pgfpathlineto{\pgfqpoint{3.662156in}{0.675427in}}%
\pgfpathlineto{\pgfqpoint{3.724377in}{0.665572in}}%
\pgfpathlineto{\pgfqpoint{3.791042in}{0.655405in}}%
\pgfpathlineto{\pgfqpoint{3.857707in}{0.645622in}}%
\pgfpathlineto{\pgfqpoint{3.924373in}{0.636202in}}%
\pgfpathlineto{\pgfqpoint{3.995482in}{0.626530in}}%
\pgfpathlineto{\pgfqpoint{4.066592in}{0.617226in}}%
\pgfpathlineto{\pgfqpoint{4.142146in}{0.607720in}}%
\pgfpathlineto{\pgfqpoint{4.217700in}{0.598583in}}%
\pgfpathlineto{\pgfqpoint{4.297698in}{0.589288in}}%
\pgfpathlineto{\pgfqpoint{4.377696in}{0.580359in}}%
\pgfpathlineto{\pgfqpoint{4.462139in}{0.571308in}}%
\pgfpathlineto{\pgfqpoint{4.546581in}{0.562620in}}%
\pgfpathlineto{\pgfqpoint{4.635468in}{0.553842in}}%
\pgfpathlineto{\pgfqpoint{4.651001in}{0.552345in}}%
\pgfpathlineto{\pgfqpoint{4.651001in}{0.552345in}}%
\pgfusepath{stroke}%
\end{pgfscope}%
\begin{pgfscope}%
\pgfpathrectangle{\pgfqpoint{0.847223in}{0.554012in}}{\pgfqpoint{6.200000in}{4.530000in}}%
\pgfusepath{clip}%
\pgfsetrectcap%
\pgfsetroundjoin%
\pgfsetlinewidth{1.003750pt}%
\definecolor{currentstroke}{rgb}{0.000000,0.000000,0.000000}%
\pgfsetstrokecolor{currentstroke}%
\pgfsetstrokeopacity{0.200000}%
\pgfsetdash{}{0pt}%
\pgfpathmoveto{\pgfqpoint{0.847223in}{5.084012in}}%
\pgfpathlineto{\pgfqpoint{0.857889in}{4.985197in}}%
\pgfpathlineto{\pgfqpoint{0.868556in}{4.890252in}}%
\pgfpathlineto{\pgfqpoint{0.879222in}{4.798955in}}%
\pgfpathlineto{\pgfqpoint{0.889889in}{4.711098in}}%
\pgfpathlineto{\pgfqpoint{0.900555in}{4.626491in}}%
\pgfpathlineto{\pgfqpoint{0.911222in}{4.544958in}}%
\pgfpathlineto{\pgfqpoint{0.921888in}{4.466333in}}%
\pgfpathlineto{\pgfqpoint{0.932555in}{4.390463in}}%
\pgfpathlineto{\pgfqpoint{0.943221in}{4.317207in}}%
\pgfpathlineto{\pgfqpoint{0.953887in}{4.246431in}}%
\pgfpathlineto{\pgfqpoint{0.964554in}{4.178012in}}%
\pgfpathlineto{\pgfqpoint{0.975220in}{4.111834in}}%
\pgfpathlineto{\pgfqpoint{0.985887in}{4.047788in}}%
\pgfpathlineto{\pgfqpoint{0.996553in}{3.985774in}}%
\pgfpathlineto{\pgfqpoint{1.007220in}{3.925695in}}%
\pgfpathlineto{\pgfqpoint{1.017886in}{3.867463in}}%
\pgfpathlineto{\pgfqpoint{1.028552in}{3.810994in}}%
\pgfpathlineto{\pgfqpoint{1.039219in}{3.756209in}}%
\pgfpathlineto{\pgfqpoint{1.049885in}{3.703034in}}%
\pgfpathlineto{\pgfqpoint{1.060552in}{3.651399in}}%
\pgfpathlineto{\pgfqpoint{1.071218in}{3.601237in}}%
\pgfpathlineto{\pgfqpoint{1.081885in}{3.552487in}}%
\pgfpathlineto{\pgfqpoint{1.092551in}{3.505091in}}%
\pgfpathlineto{\pgfqpoint{1.103217in}{3.458991in}}%
\pgfpathlineto{\pgfqpoint{1.113884in}{3.414137in}}%
\pgfpathlineto{\pgfqpoint{1.124550in}{3.370477in}}%
\pgfpathlineto{\pgfqpoint{1.135217in}{3.327965in}}%
\pgfpathlineto{\pgfqpoint{1.145883in}{3.286557in}}%
\pgfpathlineto{\pgfqpoint{1.156550in}{3.246210in}}%
\pgfpathlineto{\pgfqpoint{1.167216in}{3.206883in}}%
\pgfpathlineto{\pgfqpoint{1.177882in}{3.168538in}}%
\pgfpathlineto{\pgfqpoint{1.188549in}{3.131140in}}%
\pgfpathlineto{\pgfqpoint{1.199215in}{3.094653in}}%
\pgfpathlineto{\pgfqpoint{1.215215in}{3.041560in}}%
\pgfpathlineto{\pgfqpoint{1.231215in}{2.990339in}}%
\pgfpathlineto{\pgfqpoint{1.247214in}{2.940894in}}%
\pgfpathlineto{\pgfqpoint{1.263214in}{2.893132in}}%
\pgfpathlineto{\pgfqpoint{1.279214in}{2.846971in}}%
\pgfpathlineto{\pgfqpoint{1.295213in}{2.802330in}}%
\pgfpathlineto{\pgfqpoint{1.311213in}{2.759136in}}%
\pgfpathlineto{\pgfqpoint{1.327212in}{2.717320in}}%
\pgfpathlineto{\pgfqpoint{1.343212in}{2.676816in}}%
\pgfpathlineto{\pgfqpoint{1.359212in}{2.637565in}}%
\pgfpathlineto{\pgfqpoint{1.375211in}{2.599507in}}%
\pgfpathlineto{\pgfqpoint{1.391211in}{2.562591in}}%
\pgfpathlineto{\pgfqpoint{1.407211in}{2.526766in}}%
\pgfpathlineto{\pgfqpoint{1.423210in}{2.491983in}}%
\pgfpathlineto{\pgfqpoint{1.439210in}{2.458198in}}%
\pgfpathlineto{\pgfqpoint{1.455210in}{2.425368in}}%
\pgfpathlineto{\pgfqpoint{1.471209in}{2.393455in}}%
\pgfpathlineto{\pgfqpoint{1.487209in}{2.362419in}}%
\pgfpathlineto{\pgfqpoint{1.503209in}{2.332225in}}%
\pgfpathlineto{\pgfqpoint{1.519208in}{2.302839in}}%
\pgfpathlineto{\pgfqpoint{1.535208in}{2.274230in}}%
\pgfpathlineto{\pgfqpoint{1.551208in}{2.246367in}}%
\pgfpathlineto{\pgfqpoint{1.567207in}{2.219220in}}%
\pgfpathlineto{\pgfqpoint{1.583207in}{2.192764in}}%
\pgfpathlineto{\pgfqpoint{1.599206in}{2.166971in}}%
\pgfpathlineto{\pgfqpoint{1.615206in}{2.141818in}}%
\pgfpathlineto{\pgfqpoint{1.631206in}{2.117280in}}%
\pgfpathlineto{\pgfqpoint{1.647205in}{2.093335in}}%
\pgfpathlineto{\pgfqpoint{1.663205in}{2.069963in}}%
\pgfpathlineto{\pgfqpoint{1.679205in}{2.047142in}}%
\pgfpathlineto{\pgfqpoint{1.695204in}{2.024854in}}%
\pgfpathlineto{\pgfqpoint{1.711204in}{2.003080in}}%
\pgfpathlineto{\pgfqpoint{1.727204in}{1.981803in}}%
\pgfpathlineto{\pgfqpoint{1.743203in}{1.961005in}}%
\pgfpathlineto{\pgfqpoint{1.759203in}{1.940671in}}%
\pgfpathlineto{\pgfqpoint{1.780536in}{1.914254in}}%
\pgfpathlineto{\pgfqpoint{1.801869in}{1.888599in}}%
\pgfpathlineto{\pgfqpoint{1.823202in}{1.863674in}}%
\pgfpathlineto{\pgfqpoint{1.844534in}{1.839449in}}%
\pgfpathlineto{\pgfqpoint{1.865867in}{1.815894in}}%
\pgfpathlineto{\pgfqpoint{1.887200in}{1.792982in}}%
\pgfpathlineto{\pgfqpoint{1.908533in}{1.770686in}}%
\pgfpathlineto{\pgfqpoint{1.929866in}{1.748983in}}%
\pgfpathlineto{\pgfqpoint{1.951199in}{1.727849in}}%
\pgfpathlineto{\pgfqpoint{1.972532in}{1.707262in}}%
\pgfpathlineto{\pgfqpoint{1.993864in}{1.687201in}}%
\pgfpathlineto{\pgfqpoint{2.015197in}{1.667646in}}%
\pgfpathlineto{\pgfqpoint{2.036530in}{1.648578in}}%
\pgfpathlineto{\pgfqpoint{2.057863in}{1.629980in}}%
\pgfpathlineto{\pgfqpoint{2.079196in}{1.611833in}}%
\pgfpathlineto{\pgfqpoint{2.100529in}{1.594122in}}%
\pgfpathlineto{\pgfqpoint{2.121862in}{1.576832in}}%
\pgfpathlineto{\pgfqpoint{2.143194in}{1.559947in}}%
\pgfpathlineto{\pgfqpoint{2.164527in}{1.543453in}}%
\pgfpathlineto{\pgfqpoint{2.191193in}{1.523366in}}%
\pgfpathlineto{\pgfqpoint{2.217859in}{1.503844in}}%
\pgfpathlineto{\pgfqpoint{2.244526in}{1.484865in}}%
\pgfpathlineto{\pgfqpoint{2.271192in}{1.466405in}}%
\pgfpathlineto{\pgfqpoint{2.297858in}{1.448444in}}%
\pgfpathlineto{\pgfqpoint{2.324524in}{1.430962in}}%
\pgfpathlineto{\pgfqpoint{2.351190in}{1.413940in}}%
\pgfpathlineto{\pgfqpoint{2.377856in}{1.397360in}}%
\pgfpathlineto{\pgfqpoint{2.404522in}{1.381204in}}%
\pgfpathlineto{\pgfqpoint{2.431188in}{1.365458in}}%
\pgfpathlineto{\pgfqpoint{2.457854in}{1.350105in}}%
\pgfpathlineto{\pgfqpoint{2.484520in}{1.335131in}}%
\pgfpathlineto{\pgfqpoint{2.511186in}{1.320522in}}%
\pgfpathlineto{\pgfqpoint{2.543186in}{1.303455in}}%
\pgfpathlineto{\pgfqpoint{2.575185in}{1.286873in}}%
\pgfpathlineto{\pgfqpoint{2.607184in}{1.270756in}}%
\pgfpathlineto{\pgfqpoint{2.639184in}{1.255084in}}%
\pgfpathlineto{\pgfqpoint{2.671183in}{1.239840in}}%
\pgfpathlineto{\pgfqpoint{2.703182in}{1.225005in}}%
\pgfpathlineto{\pgfqpoint{2.735181in}{1.210565in}}%
\pgfpathlineto{\pgfqpoint{2.767181in}{1.196502in}}%
\pgfpathlineto{\pgfqpoint{2.799180in}{1.182804in}}%
\pgfpathlineto{\pgfqpoint{2.836512in}{1.167263in}}%
\pgfpathlineto{\pgfqpoint{2.873845in}{1.152177in}}%
\pgfpathlineto{\pgfqpoint{2.911178in}{1.137526in}}%
\pgfpathlineto{\pgfqpoint{2.948510in}{1.123292in}}%
\pgfpathlineto{\pgfqpoint{2.985843in}{1.109458in}}%
\pgfpathlineto{\pgfqpoint{3.023175in}{1.096006in}}%
\pgfpathlineto{\pgfqpoint{3.060508in}{1.082921in}}%
\pgfpathlineto{\pgfqpoint{3.097840in}{1.070188in}}%
\pgfpathlineto{\pgfqpoint{3.140506in}{1.056050in}}%
\pgfpathlineto{\pgfqpoint{3.183172in}{1.042334in}}%
\pgfpathlineto{\pgfqpoint{3.225837in}{1.029021in}}%
\pgfpathlineto{\pgfqpoint{3.268503in}{1.016093in}}%
\pgfpathlineto{\pgfqpoint{3.311169in}{1.003535in}}%
\pgfpathlineto{\pgfqpoint{3.353834in}{0.991331in}}%
\pgfpathlineto{\pgfqpoint{3.401833in}{0.978006in}}%
\pgfpathlineto{\pgfqpoint{3.449832in}{0.965089in}}%
\pgfpathlineto{\pgfqpoint{3.497831in}{0.952564in}}%
\pgfpathlineto{\pgfqpoint{3.545830in}{0.940411in}}%
\pgfpathlineto{\pgfqpoint{3.593829in}{0.928616in}}%
\pgfpathlineto{\pgfqpoint{3.647161in}{0.915909in}}%
\pgfpathlineto{\pgfqpoint{3.700493in}{0.903604in}}%
\pgfpathlineto{\pgfqpoint{3.753826in}{0.891681in}}%
\pgfpathlineto{\pgfqpoint{3.807158in}{0.880124in}}%
\pgfpathlineto{\pgfqpoint{3.865823in}{0.867813in}}%
\pgfpathlineto{\pgfqpoint{3.924488in}{0.855903in}}%
\pgfpathlineto{\pgfqpoint{3.983154in}{0.844375in}}%
\pgfpathlineto{\pgfqpoint{4.041819in}{0.833210in}}%
\pgfpathlineto{\pgfqpoint{4.105818in}{0.821426in}}%
\pgfpathlineto{\pgfqpoint{4.169816in}{0.810034in}}%
\pgfpathlineto{\pgfqpoint{4.233815in}{0.799015in}}%
\pgfpathlineto{\pgfqpoint{4.297814in}{0.788351in}}%
\pgfpathlineto{\pgfqpoint{4.367145in}{0.777179in}}%
\pgfpathlineto{\pgfqpoint{4.436477in}{0.766383in}}%
\pgfpathlineto{\pgfqpoint{4.505809in}{0.755946in}}%
\pgfpathlineto{\pgfqpoint{4.580474in}{0.745086in}}%
\pgfpathlineto{\pgfqpoint{4.655139in}{0.734601in}}%
\pgfpathlineto{\pgfqpoint{4.735137in}{0.723760in}}%
\pgfpathlineto{\pgfqpoint{4.815136in}{0.713306in}}%
\pgfpathlineto{\pgfqpoint{4.895134in}{0.703217in}}%
\pgfpathlineto{\pgfqpoint{4.980465in}{0.692838in}}%
\pgfpathlineto{\pgfqpoint{5.065797in}{0.682833in}}%
\pgfpathlineto{\pgfqpoint{5.156461in}{0.672589in}}%
\pgfpathlineto{\pgfqpoint{5.247126in}{0.662723in}}%
\pgfpathlineto{\pgfqpoint{5.343124in}{0.652664in}}%
\pgfpathlineto{\pgfqpoint{5.439122in}{0.642984in}}%
\pgfpathlineto{\pgfqpoint{5.540453in}{0.633152in}}%
\pgfpathlineto{\pgfqpoint{5.641784in}{0.623695in}}%
\pgfpathlineto{\pgfqpoint{5.748448in}{0.614121in}}%
\pgfpathlineto{\pgfqpoint{5.855113in}{0.604917in}}%
\pgfpathlineto{\pgfqpoint{5.967110in}{0.595627in}}%
\pgfpathlineto{\pgfqpoint{6.084441in}{0.586282in}}%
\pgfpathlineto{\pgfqpoint{6.183915in}{0.578665in}}%
\pgfpathlineto{\pgfqpoint{6.219152in}{0.576287in}}%
\pgfpathlineto{\pgfqpoint{6.263198in}{0.573734in}}%
\pgfpathlineto{\pgfqpoint{6.316054in}{0.571128in}}%
\pgfpathlineto{\pgfqpoint{6.368909in}{0.568896in}}%
\pgfpathlineto{\pgfqpoint{6.430574in}{0.566648in}}%
\pgfpathlineto{\pgfqpoint{6.501048in}{0.564439in}}%
\pgfpathlineto{\pgfqpoint{6.580332in}{0.562305in}}%
\pgfpathlineto{\pgfqpoint{6.677234in}{0.560079in}}%
\pgfpathlineto{\pgfqpoint{6.782945in}{0.558018in}}%
\pgfpathlineto{\pgfqpoint{6.906275in}{0.555978in}}%
\pgfpathlineto{\pgfqpoint{7.047223in}{0.554012in}}%
\pgfpathlineto{\pgfqpoint{7.047223in}{0.554012in}}%
\pgfusepath{stroke}%
\end{pgfscope}%
\begin{pgfscope}%
\pgfsetrectcap%
\pgfsetmiterjoin%
\pgfsetlinewidth{0.803000pt}%
\definecolor{currentstroke}{rgb}{0.000000,0.000000,0.000000}%
\pgfsetstrokecolor{currentstroke}%
\pgfsetdash{}{0pt}%
\pgfpathmoveto{\pgfqpoint{0.847223in}{0.554012in}}%
\pgfpathlineto{\pgfqpoint{0.847223in}{5.084012in}}%
\pgfusepath{stroke}%
\end{pgfscope}%
\begin{pgfscope}%
\pgfsetrectcap%
\pgfsetmiterjoin%
\pgfsetlinewidth{0.803000pt}%
\definecolor{currentstroke}{rgb}{0.000000,0.000000,0.000000}%
\pgfsetstrokecolor{currentstroke}%
\pgfsetdash{}{0pt}%
\pgfpathmoveto{\pgfqpoint{7.047223in}{0.554012in}}%
\pgfpathlineto{\pgfqpoint{7.047223in}{5.084012in}}%
\pgfusepath{stroke}%
\end{pgfscope}%
\begin{pgfscope}%
\pgfsetrectcap%
\pgfsetmiterjoin%
\pgfsetlinewidth{0.803000pt}%
\definecolor{currentstroke}{rgb}{0.000000,0.000000,0.000000}%
\pgfsetstrokecolor{currentstroke}%
\pgfsetdash{}{0pt}%
\pgfpathmoveto{\pgfqpoint{0.847223in}{0.554012in}}%
\pgfpathlineto{\pgfqpoint{7.047223in}{0.554012in}}%
\pgfusepath{stroke}%
\end{pgfscope}%
\begin{pgfscope}%
\pgfsetrectcap%
\pgfsetmiterjoin%
\pgfsetlinewidth{0.803000pt}%
\definecolor{currentstroke}{rgb}{0.000000,0.000000,0.000000}%
\pgfsetstrokecolor{currentstroke}%
\pgfsetdash{}{0pt}%
\pgfpathmoveto{\pgfqpoint{0.847223in}{5.084012in}}%
\pgfpathlineto{\pgfqpoint{7.047223in}{5.084012in}}%
\pgfusepath{stroke}%
\end{pgfscope}%
\begin{pgfscope}%
\pgfsetbuttcap%
\pgfsetmiterjoin%
\definecolor{currentfill}{rgb}{0.300000,0.300000,0.300000}%
\pgfsetfillcolor{currentfill}%
\pgfsetfillopacity{0.500000}%
\pgfsetlinewidth{1.003750pt}%
\definecolor{currentstroke}{rgb}{0.300000,0.300000,0.300000}%
\pgfsetstrokecolor{currentstroke}%
\pgfsetstrokeopacity{0.500000}%
\pgfsetdash{}{0pt}%
\pgfpathmoveto{\pgfqpoint{4.535764in}{4.075680in}}%
\pgfpathlineto{\pgfqpoint{6.938890in}{4.075680in}}%
\pgfpathquadraticcurveto{\pgfqpoint{6.977779in}{4.075680in}}{\pgfqpoint{6.977779in}{4.114569in}}%
\pgfpathlineto{\pgfqpoint{6.977779in}{4.920123in}}%
\pgfpathquadraticcurveto{\pgfqpoint{6.977779in}{4.959012in}}{\pgfqpoint{6.938890in}{4.959012in}}%
\pgfpathlineto{\pgfqpoint{4.535764in}{4.959012in}}%
\pgfpathquadraticcurveto{\pgfqpoint{4.496875in}{4.959012in}}{\pgfqpoint{4.496875in}{4.920123in}}%
\pgfpathlineto{\pgfqpoint{4.496875in}{4.114569in}}%
\pgfpathquadraticcurveto{\pgfqpoint{4.496875in}{4.075680in}}{\pgfqpoint{4.535764in}{4.075680in}}%
\pgfpathlineto{\pgfqpoint{4.535764in}{4.075680in}}%
\pgfpathclose%
\pgfusepath{stroke,fill}%
\end{pgfscope}%
\begin{pgfscope}%
\pgfsetbuttcap%
\pgfsetmiterjoin%
\definecolor{currentfill}{rgb}{1.000000,1.000000,1.000000}%
\pgfsetfillcolor{currentfill}%
\pgfsetlinewidth{1.003750pt}%
\definecolor{currentstroke}{rgb}{0.800000,0.800000,0.800000}%
\pgfsetstrokecolor{currentstroke}%
\pgfsetdash{}{0pt}%
\pgfpathmoveto{\pgfqpoint{4.507986in}{4.103458in}}%
\pgfpathlineto{\pgfqpoint{6.911112in}{4.103458in}}%
\pgfpathquadraticcurveto{\pgfqpoint{6.950001in}{4.103458in}}{\pgfqpoint{6.950001in}{4.142346in}}%
\pgfpathlineto{\pgfqpoint{6.950001in}{4.947901in}}%
\pgfpathquadraticcurveto{\pgfqpoint{6.950001in}{4.986790in}}{\pgfqpoint{6.911112in}{4.986790in}}%
\pgfpathlineto{\pgfqpoint{4.507986in}{4.986790in}}%
\pgfpathquadraticcurveto{\pgfqpoint{4.469097in}{4.986790in}}{\pgfqpoint{4.469097in}{4.947901in}}%
\pgfpathlineto{\pgfqpoint{4.469097in}{4.142346in}}%
\pgfpathquadraticcurveto{\pgfqpoint{4.469097in}{4.103458in}}{\pgfqpoint{4.507986in}{4.103458in}}%
\pgfpathlineto{\pgfqpoint{4.507986in}{4.103458in}}%
\pgfpathclose%
\pgfusepath{stroke,fill}%
\end{pgfscope}%
\begin{pgfscope}%
\pgfsetrectcap%
\pgfsetroundjoin%
\pgfsetlinewidth{3.011250pt}%
\definecolor{currentstroke}{rgb}{0.000000,0.000000,0.000000}%
\pgfsetstrokecolor{currentstroke}%
\pgfsetdash{}{0pt}%
\pgfpathmoveto{\pgfqpoint{4.546875in}{4.838179in}}%
\pgfpathlineto{\pgfqpoint{4.741320in}{4.838179in}}%
\pgfpathlineto{\pgfqpoint{4.935764in}{4.838179in}}%
\pgfusepath{stroke}%
\end{pgfscope}%
\begin{pgfscope}%
\definecolor{textcolor}{rgb}{0.000000,0.000000,0.000000}%
\pgfsetstrokecolor{textcolor}%
\pgfsetfillcolor{textcolor}%
\pgftext[x=5.091320in,y=4.770123in,left,base]{\color{textcolor}\rmfamily\fontsize{14.000000}{16.800000}\selectfont Pareto Front}%
\end{pgfscope}%
\begin{pgfscope}%
\pgfsetbuttcap%
\pgfsetroundjoin%
\definecolor{currentfill}{rgb}{0.121569,0.466667,0.705882}%
\pgfsetfillcolor{currentfill}%
\pgfsetlinewidth{1.003750pt}%
\definecolor{currentstroke}{rgb}{0.121569,0.466667,0.705882}%
\pgfsetstrokecolor{currentstroke}%
\pgfsetdash{}{0pt}%
\pgfsys@defobject{currentmarker}{\pgfqpoint{-0.012028in}{-0.012028in}}{\pgfqpoint{0.012028in}{0.012028in}}{%
\pgfpathmoveto{\pgfqpoint{0.000000in}{-0.012028in}}%
\pgfpathcurveto{\pgfqpoint{0.003190in}{-0.012028in}}{\pgfqpoint{0.006250in}{-0.010761in}}{\pgfqpoint{0.008505in}{-0.008505in}}%
\pgfpathcurveto{\pgfqpoint{0.010761in}{-0.006250in}}{\pgfqpoint{0.012028in}{-0.003190in}}{\pgfqpoint{0.012028in}{0.000000in}}%
\pgfpathcurveto{\pgfqpoint{0.012028in}{0.003190in}}{\pgfqpoint{0.010761in}{0.006250in}}{\pgfqpoint{0.008505in}{0.008505in}}%
\pgfpathcurveto{\pgfqpoint{0.006250in}{0.010761in}}{\pgfqpoint{0.003190in}{0.012028in}}{\pgfqpoint{0.000000in}{0.012028in}}%
\pgfpathcurveto{\pgfqpoint{-0.003190in}{0.012028in}}{\pgfqpoint{-0.006250in}{0.010761in}}{\pgfqpoint{-0.008505in}{0.008505in}}%
\pgfpathcurveto{\pgfqpoint{-0.010761in}{0.006250in}}{\pgfqpoint{-0.012028in}{0.003190in}}{\pgfqpoint{-0.012028in}{0.000000in}}%
\pgfpathcurveto{\pgfqpoint{-0.012028in}{-0.003190in}}{\pgfqpoint{-0.010761in}{-0.006250in}}{\pgfqpoint{-0.008505in}{-0.008505in}}%
\pgfpathcurveto{\pgfqpoint{-0.006250in}{-0.010761in}}{\pgfqpoint{-0.003190in}{-0.012028in}}{\pgfqpoint{0.000000in}{-0.012028in}}%
\pgfpathlineto{\pgfqpoint{0.000000in}{-0.012028in}}%
\pgfpathclose%
\pgfusepath{stroke,fill}%
}%
\begin{pgfscope}%
\pgfsys@transformshift{4.741320in}{4.546165in}%
\pgfsys@useobject{currentmarker}{}%
\end{pgfscope}%
\end{pgfscope}%
\begin{pgfscope}%
\definecolor{textcolor}{rgb}{0.000000,0.000000,0.000000}%
\pgfsetstrokecolor{textcolor}%
\pgfsetfillcolor{textcolor}%
\pgftext[x=5.091320in,y=4.495124in,left,base]{\color{textcolor}\rmfamily\fontsize{14.000000}{16.800000}\selectfont Tested points}%
\end{pgfscope}%
\begin{pgfscope}%
\pgfsetbuttcap%
\pgfsetroundjoin%
\definecolor{currentfill}{rgb}{0.839216,0.152941,0.156863}%
\pgfsetfillcolor{currentfill}%
\pgfsetlinewidth{1.003750pt}%
\definecolor{currentstroke}{rgb}{0.839216,0.152941,0.156863}%
\pgfsetstrokecolor{currentstroke}%
\pgfsetdash{}{0pt}%
\pgfsys@defobject{currentmarker}{\pgfqpoint{-0.031056in}{-0.031056in}}{\pgfqpoint{0.031056in}{0.031056in}}{%
\pgfpathmoveto{\pgfqpoint{0.000000in}{-0.031056in}}%
\pgfpathcurveto{\pgfqpoint{0.008236in}{-0.031056in}}{\pgfqpoint{0.016136in}{-0.027784in}}{\pgfqpoint{0.021960in}{-0.021960in}}%
\pgfpathcurveto{\pgfqpoint{0.027784in}{-0.016136in}}{\pgfqpoint{0.031056in}{-0.008236in}}{\pgfqpoint{0.031056in}{0.000000in}}%
\pgfpathcurveto{\pgfqpoint{0.031056in}{0.008236in}}{\pgfqpoint{0.027784in}{0.016136in}}{\pgfqpoint{0.021960in}{0.021960in}}%
\pgfpathcurveto{\pgfqpoint{0.016136in}{0.027784in}}{\pgfqpoint{0.008236in}{0.031056in}}{\pgfqpoint{0.000000in}{0.031056in}}%
\pgfpathcurveto{\pgfqpoint{-0.008236in}{0.031056in}}{\pgfqpoint{-0.016136in}{0.027784in}}{\pgfqpoint{-0.021960in}{0.021960in}}%
\pgfpathcurveto{\pgfqpoint{-0.027784in}{0.016136in}}{\pgfqpoint{-0.031056in}{0.008236in}}{\pgfqpoint{-0.031056in}{0.000000in}}%
\pgfpathcurveto{\pgfqpoint{-0.031056in}{-0.008236in}}{\pgfqpoint{-0.027784in}{-0.016136in}}{\pgfqpoint{-0.021960in}{-0.021960in}}%
\pgfpathcurveto{\pgfqpoint{-0.016136in}{-0.027784in}}{\pgfqpoint{-0.008236in}{-0.031056in}}{\pgfqpoint{0.000000in}{-0.031056in}}%
\pgfpathlineto{\pgfqpoint{0.000000in}{-0.031056in}}%
\pgfpathclose%
\pgfusepath{stroke,fill}%
}%
\begin{pgfscope}%
\pgfsys@transformshift{4.741320in}{4.271166in}%
\pgfsys@useobject{currentmarker}{}%
\end{pgfscope}%
\end{pgfscope}%
\begin{pgfscope}%
\definecolor{textcolor}{rgb}{0.000000,0.000000,0.000000}%
\pgfsetstrokecolor{textcolor}%
\pgfsetfillcolor{textcolor}%
\pgftext[x=5.091320in,y=4.220124in,left,base]{\color{textcolor}\rmfamily\fontsize{14.000000}{16.800000}\selectfont Alternative solutions}%
\end{pgfscope}%
\end{pgfpicture}%
\makeatother%
\endgroup%
}
  \caption{All of the alternative points inside the near-feasible space selected
  using the algorithm described in Section \ref{section:mga-moo}.}
  \label{fig:nd-mga}
\end{figure}

\begin{figure}[H]
  \centering
  \resizebox{1\columnwidth}{!}{%% Creator: Matplotlib, PGF backend
%%
%% To include the figure in your LaTeX document, write
%%   \input{<filename>.pgf}
%%
%% Make sure the required packages are loaded in your preamble
%%   \usepackage{pgf}
%%
%% Also ensure that all the required font packages are loaded; for instance,
%% the lmodern package is sometimes necessary when using math font.
%%   \usepackage{lmodern}
%%
%% Figures using additional raster images can only be included by \input if
%% they are in the same directory as the main LaTeX file. For loading figures
%% from other directories you can use the `import` package
%%   \usepackage{import}
%%
%% and then include the figures with
%%   \import{<path to file>}{<filename>.pgf}
%%
%% Matplotlib used the following preamble
%%
\begingroup%
\makeatletter%
\begin{pgfpicture}%
\pgfpathrectangle{\pgfpointorigin}{\pgfqpoint{12.508921in}{4.305549in}}%
\pgfusepath{use as bounding box, clip}%
\begin{pgfscope}%
\pgfsetbuttcap%
\pgfsetmiterjoin%
\definecolor{currentfill}{rgb}{1.000000,1.000000,1.000000}%
\pgfsetfillcolor{currentfill}%
\pgfsetlinewidth{0.000000pt}%
\definecolor{currentstroke}{rgb}{0.000000,0.000000,0.000000}%
\pgfsetstrokecolor{currentstroke}%
\pgfsetdash{}{0pt}%
\pgfpathmoveto{\pgfqpoint{0.000000in}{0.000000in}}%
\pgfpathlineto{\pgfqpoint{12.508921in}{0.000000in}}%
\pgfpathlineto{\pgfqpoint{12.508921in}{4.305549in}}%
\pgfpathlineto{\pgfqpoint{0.000000in}{4.305549in}}%
\pgfpathlineto{\pgfqpoint{0.000000in}{0.000000in}}%
\pgfpathclose%
\pgfusepath{fill}%
\end{pgfscope}%
\begin{pgfscope}%
\pgfsetbuttcap%
\pgfsetmiterjoin%
\definecolor{currentfill}{rgb}{1.000000,1.000000,1.000000}%
\pgfsetfillcolor{currentfill}%
\pgfsetlinewidth{0.000000pt}%
\definecolor{currentstroke}{rgb}{0.000000,0.000000,0.000000}%
\pgfsetstrokecolor{currentstroke}%
\pgfsetstrokeopacity{0.000000}%
\pgfsetdash{}{0pt}%
\pgfpathmoveto{\pgfqpoint{4.658921in}{0.208778in}}%
\pgfpathlineto{\pgfqpoint{8.458921in}{0.208778in}}%
\pgfpathlineto{\pgfqpoint{8.458921in}{4.008778in}}%
\pgfpathlineto{\pgfqpoint{4.658921in}{4.008778in}}%
\pgfpathlineto{\pgfqpoint{4.658921in}{0.208778in}}%
\pgfpathclose%
\pgfusepath{fill}%
\end{pgfscope}%
\begin{pgfscope}%
\pgfsetbuttcap%
\pgfsetmiterjoin%
\definecolor{currentfill}{rgb}{0.950000,0.950000,0.950000}%
\pgfsetfillcolor{currentfill}%
\pgfsetfillopacity{0.500000}%
\pgfsetlinewidth{1.003750pt}%
\definecolor{currentstroke}{rgb}{0.950000,0.950000,0.950000}%
\pgfsetstrokecolor{currentstroke}%
\pgfsetstrokeopacity{0.500000}%
\pgfsetdash{}{0pt}%
\pgfpathmoveto{\pgfqpoint{6.610272in}{2.686431in}}%
\pgfpathlineto{\pgfqpoint{8.286450in}{1.531563in}}%
\pgfpathlineto{\pgfqpoint{8.395724in}{2.829674in}}%
\pgfpathlineto{\pgfqpoint{6.610272in}{3.980304in}}%
\pgfusepath{stroke,fill}%
\end{pgfscope}%
\begin{pgfscope}%
\pgfsetbuttcap%
\pgfsetmiterjoin%
\definecolor{currentfill}{rgb}{0.900000,0.900000,0.900000}%
\pgfsetfillcolor{currentfill}%
\pgfsetfillopacity{0.500000}%
\pgfsetlinewidth{1.003750pt}%
\definecolor{currentstroke}{rgb}{0.900000,0.900000,0.900000}%
\pgfsetstrokecolor{currentstroke}%
\pgfsetstrokeopacity{0.500000}%
\pgfsetdash{}{0pt}%
\pgfpathmoveto{\pgfqpoint{6.610272in}{2.686431in}}%
\pgfpathlineto{\pgfqpoint{4.934094in}{1.531563in}}%
\pgfpathlineto{\pgfqpoint{4.824820in}{2.829674in}}%
\pgfpathlineto{\pgfqpoint{6.610272in}{3.980304in}}%
\pgfusepath{stroke,fill}%
\end{pgfscope}%
\begin{pgfscope}%
\pgfsetbuttcap%
\pgfsetmiterjoin%
\definecolor{currentfill}{rgb}{0.925000,0.925000,0.925000}%
\pgfsetfillcolor{currentfill}%
\pgfsetfillopacity{0.500000}%
\pgfsetlinewidth{1.003750pt}%
\definecolor{currentstroke}{rgb}{0.925000,0.925000,0.925000}%
\pgfsetstrokecolor{currentstroke}%
\pgfsetstrokeopacity{0.500000}%
\pgfsetdash{}{0pt}%
\pgfpathmoveto{\pgfqpoint{6.610272in}{2.686431in}}%
\pgfpathlineto{\pgfqpoint{4.934094in}{1.531563in}}%
\pgfpathlineto{\pgfqpoint{6.610272in}{0.235257in}}%
\pgfpathlineto{\pgfqpoint{8.286450in}{1.531563in}}%
\pgfusepath{stroke,fill}%
\end{pgfscope}%
\begin{pgfscope}%
\pgfsetrectcap%
\pgfsetroundjoin%
\pgfsetlinewidth{0.803000pt}%
\definecolor{currentstroke}{rgb}{0.000000,0.000000,0.000000}%
\pgfsetstrokecolor{currentstroke}%
\pgfsetdash{}{0pt}%
\pgfpathmoveto{\pgfqpoint{8.286450in}{1.531563in}}%
\pgfpathlineto{\pgfqpoint{6.610272in}{0.235257in}}%
\pgfusepath{stroke}%
\end{pgfscope}%
\begin{pgfscope}%
\definecolor{textcolor}{rgb}{0.000000,0.000000,0.000000}%
\pgfsetstrokecolor{textcolor}%
\pgfsetfillcolor{textcolor}%
\pgftext[x=7.847764in,y=0.356972in,,]{\color{textcolor}\rmfamily\fontsize{14.000000}{16.800000}\selectfont f1}%
\end{pgfscope}%
\begin{pgfscope}%
\pgfsetbuttcap%
\pgfsetroundjoin%
\pgfsetlinewidth{0.803000pt}%
\definecolor{currentstroke}{rgb}{0.690196,0.690196,0.690196}%
\pgfsetstrokecolor{currentstroke}%
\pgfsetdash{}{0pt}%
\pgfpathmoveto{\pgfqpoint{8.185568in}{1.453544in}}%
\pgfpathlineto{\pgfqpoint{6.509074in}{2.616707in}}%
\pgfpathlineto{\pgfqpoint{6.502836in}{3.911067in}}%
\pgfusepath{stroke}%
\end{pgfscope}%
\begin{pgfscope}%
\pgfsetbuttcap%
\pgfsetroundjoin%
\pgfsetlinewidth{0.803000pt}%
\definecolor{currentstroke}{rgb}{0.690196,0.690196,0.690196}%
\pgfsetstrokecolor{currentstroke}%
\pgfsetdash{}{0pt}%
\pgfpathmoveto{\pgfqpoint{7.930751in}{1.256476in}}%
\pgfpathlineto{\pgfqpoint{6.253638in}{2.440714in}}%
\pgfpathlineto{\pgfqpoint{6.231449in}{3.736172in}}%
\pgfusepath{stroke}%
\end{pgfscope}%
\begin{pgfscope}%
\pgfsetbuttcap%
\pgfsetroundjoin%
\pgfsetlinewidth{0.803000pt}%
\definecolor{currentstroke}{rgb}{0.690196,0.690196,0.690196}%
\pgfsetstrokecolor{currentstroke}%
\pgfsetdash{}{0pt}%
\pgfpathmoveto{\pgfqpoint{7.671163in}{1.055718in}}%
\pgfpathlineto{\pgfqpoint{5.993686in}{2.261610in}}%
\pgfpathlineto{\pgfqpoint{5.954962in}{3.557991in}}%
\pgfusepath{stroke}%
\end{pgfscope}%
\begin{pgfscope}%
\pgfsetbuttcap%
\pgfsetroundjoin%
\pgfsetlinewidth{0.803000pt}%
\definecolor{currentstroke}{rgb}{0.690196,0.690196,0.690196}%
\pgfsetstrokecolor{currentstroke}%
\pgfsetdash{}{0pt}%
\pgfpathmoveto{\pgfqpoint{7.406668in}{0.851166in}}%
\pgfpathlineto{\pgfqpoint{5.729096in}{2.079311in}}%
\pgfpathlineto{\pgfqpoint{5.673231in}{3.376430in}}%
\pgfusepath{stroke}%
\end{pgfscope}%
\begin{pgfscope}%
\pgfsetbuttcap%
\pgfsetroundjoin%
\pgfsetlinewidth{0.803000pt}%
\definecolor{currentstroke}{rgb}{0.690196,0.690196,0.690196}%
\pgfsetstrokecolor{currentstroke}%
\pgfsetdash{}{0pt}%
\pgfpathmoveto{\pgfqpoint{7.137126in}{0.642710in}}%
\pgfpathlineto{\pgfqpoint{5.459744in}{1.893730in}}%
\pgfpathlineto{\pgfqpoint{5.386104in}{3.191392in}}%
\pgfusepath{stroke}%
\end{pgfscope}%
\begin{pgfscope}%
\pgfsetbuttcap%
\pgfsetroundjoin%
\pgfsetlinewidth{0.803000pt}%
\definecolor{currentstroke}{rgb}{0.690196,0.690196,0.690196}%
\pgfsetstrokecolor{currentstroke}%
\pgfsetdash{}{0pt}%
\pgfpathmoveto{\pgfqpoint{6.862392in}{0.430239in}}%
\pgfpathlineto{\pgfqpoint{5.185501in}{1.704779in}}%
\pgfpathlineto{\pgfqpoint{5.093425in}{3.002776in}}%
\pgfusepath{stroke}%
\end{pgfscope}%
\begin{pgfscope}%
\pgfsetrectcap%
\pgfsetroundjoin%
\pgfsetlinewidth{0.803000pt}%
\definecolor{currentstroke}{rgb}{0.000000,0.000000,0.000000}%
\pgfsetstrokecolor{currentstroke}%
\pgfsetdash{}{0pt}%
\pgfpathmoveto{\pgfqpoint{8.171386in}{1.463384in}}%
\pgfpathlineto{\pgfqpoint{8.213972in}{1.433838in}}%
\pgfusepath{stroke}%
\end{pgfscope}%
\begin{pgfscope}%
\definecolor{textcolor}{rgb}{0.000000,0.000000,0.000000}%
\pgfsetstrokecolor{textcolor}%
\pgfsetfillcolor{textcolor}%
\pgftext[x=8.320852in,y=1.255971in,,top]{\color{textcolor}\rmfamily\fontsize{10.000000}{12.000000}\selectfont \(\displaystyle {0.0}\)}%
\end{pgfscope}%
\begin{pgfscope}%
\pgfsetrectcap%
\pgfsetroundjoin%
\pgfsetlinewidth{0.803000pt}%
\definecolor{currentstroke}{rgb}{0.000000,0.000000,0.000000}%
\pgfsetstrokecolor{currentstroke}%
\pgfsetdash{}{0pt}%
\pgfpathmoveto{\pgfqpoint{7.916556in}{1.266499in}}%
\pgfpathlineto{\pgfqpoint{7.959180in}{1.236402in}}%
\pgfusepath{stroke}%
\end{pgfscope}%
\begin{pgfscope}%
\definecolor{textcolor}{rgb}{0.000000,0.000000,0.000000}%
\pgfsetstrokecolor{textcolor}%
\pgfsetfillcolor{textcolor}%
\pgftext[x=8.067362in,y=1.057127in,,top]{\color{textcolor}\rmfamily\fontsize{10.000000}{12.000000}\selectfont \(\displaystyle {0.2}\)}%
\end{pgfscope}%
\begin{pgfscope}%
\pgfsetrectcap%
\pgfsetroundjoin%
\pgfsetlinewidth{0.803000pt}%
\definecolor{currentstroke}{rgb}{0.000000,0.000000,0.000000}%
\pgfsetstrokecolor{currentstroke}%
\pgfsetdash{}{0pt}%
\pgfpathmoveto{\pgfqpoint{7.656958in}{1.065930in}}%
\pgfpathlineto{\pgfqpoint{7.699613in}{1.035266in}}%
\pgfusepath{stroke}%
\end{pgfscope}%
\begin{pgfscope}%
\definecolor{textcolor}{rgb}{0.000000,0.000000,0.000000}%
\pgfsetstrokecolor{textcolor}%
\pgfsetfillcolor{textcolor}%
\pgftext[x=7.809127in,y=0.854561in,,top]{\color{textcolor}\rmfamily\fontsize{10.000000}{12.000000}\selectfont \(\displaystyle {0.4}\)}%
\end{pgfscope}%
\begin{pgfscope}%
\pgfsetrectcap%
\pgfsetroundjoin%
\pgfsetlinewidth{0.803000pt}%
\definecolor{currentstroke}{rgb}{0.000000,0.000000,0.000000}%
\pgfsetstrokecolor{currentstroke}%
\pgfsetdash{}{0pt}%
\pgfpathmoveto{\pgfqpoint{7.392455in}{0.861571in}}%
\pgfpathlineto{\pgfqpoint{7.435135in}{0.830325in}}%
\pgfusepath{stroke}%
\end{pgfscope}%
\begin{pgfscope}%
\definecolor{textcolor}{rgb}{0.000000,0.000000,0.000000}%
\pgfsetstrokecolor{textcolor}%
\pgfsetfillcolor{textcolor}%
\pgftext[x=7.546013in,y=0.648168in,,top]{\color{textcolor}\rmfamily\fontsize{10.000000}{12.000000}\selectfont \(\displaystyle {0.6}\)}%
\end{pgfscope}%
\begin{pgfscope}%
\pgfsetrectcap%
\pgfsetroundjoin%
\pgfsetlinewidth{0.803000pt}%
\definecolor{currentstroke}{rgb}{0.000000,0.000000,0.000000}%
\pgfsetstrokecolor{currentstroke}%
\pgfsetdash{}{0pt}%
\pgfpathmoveto{\pgfqpoint{7.122907in}{0.653315in}}%
\pgfpathlineto{\pgfqpoint{7.165605in}{0.621470in}}%
\pgfusepath{stroke}%
\end{pgfscope}%
\begin{pgfscope}%
\definecolor{textcolor}{rgb}{0.000000,0.000000,0.000000}%
\pgfsetstrokecolor{textcolor}%
\pgfsetfillcolor{textcolor}%
\pgftext[x=7.277880in,y=0.437838in,,top]{\color{textcolor}\rmfamily\fontsize{10.000000}{12.000000}\selectfont \(\displaystyle {0.8}\)}%
\end{pgfscope}%
\begin{pgfscope}%
\pgfsetrectcap%
\pgfsetroundjoin%
\pgfsetlinewidth{0.803000pt}%
\definecolor{currentstroke}{rgb}{0.000000,0.000000,0.000000}%
\pgfsetstrokecolor{currentstroke}%
\pgfsetdash{}{0pt}%
\pgfpathmoveto{\pgfqpoint{6.848169in}{0.441049in}}%
\pgfpathlineto{\pgfqpoint{6.890878in}{0.408587in}}%
\pgfusepath{stroke}%
\end{pgfscope}%
\begin{pgfscope}%
\definecolor{textcolor}{rgb}{0.000000,0.000000,0.000000}%
\pgfsetstrokecolor{textcolor}%
\pgfsetfillcolor{textcolor}%
\pgftext[x=7.004582in,y=0.223457in,,top]{\color{textcolor}\rmfamily\fontsize{10.000000}{12.000000}\selectfont \(\displaystyle {1.0}\)}%
\end{pgfscope}%
\begin{pgfscope}%
\pgfsetrectcap%
\pgfsetroundjoin%
\pgfsetlinewidth{0.803000pt}%
\definecolor{currentstroke}{rgb}{0.000000,0.000000,0.000000}%
\pgfsetstrokecolor{currentstroke}%
\pgfsetdash{}{0pt}%
\pgfpathmoveto{\pgfqpoint{4.934094in}{1.531563in}}%
\pgfpathlineto{\pgfqpoint{6.610272in}{0.235257in}}%
\pgfusepath{stroke}%
\end{pgfscope}%
\begin{pgfscope}%
\definecolor{textcolor}{rgb}{0.000000,0.000000,0.000000}%
\pgfsetstrokecolor{textcolor}%
\pgfsetfillcolor{textcolor}%
\pgftext[x=5.372780in,y=0.356972in,,]{\color{textcolor}\rmfamily\fontsize{14.000000}{16.800000}\selectfont f2}%
\end{pgfscope}%
\begin{pgfscope}%
\pgfsetbuttcap%
\pgfsetroundjoin%
\pgfsetlinewidth{0.803000pt}%
\definecolor{currentstroke}{rgb}{0.690196,0.690196,0.690196}%
\pgfsetstrokecolor{currentstroke}%
\pgfsetdash{}{0pt}%
\pgfpathmoveto{\pgfqpoint{6.717708in}{3.911067in}}%
\pgfpathlineto{\pgfqpoint{6.711470in}{2.616707in}}%
\pgfpathlineto{\pgfqpoint{5.034976in}{1.453544in}}%
\pgfusepath{stroke}%
\end{pgfscope}%
\begin{pgfscope}%
\pgfsetbuttcap%
\pgfsetroundjoin%
\pgfsetlinewidth{0.803000pt}%
\definecolor{currentstroke}{rgb}{0.690196,0.690196,0.690196}%
\pgfsetstrokecolor{currentstroke}%
\pgfsetdash{}{0pt}%
\pgfpathmoveto{\pgfqpoint{6.989095in}{3.736172in}}%
\pgfpathlineto{\pgfqpoint{6.966906in}{2.440714in}}%
\pgfpathlineto{\pgfqpoint{5.289793in}{1.256476in}}%
\pgfusepath{stroke}%
\end{pgfscope}%
\begin{pgfscope}%
\pgfsetbuttcap%
\pgfsetroundjoin%
\pgfsetlinewidth{0.803000pt}%
\definecolor{currentstroke}{rgb}{0.690196,0.690196,0.690196}%
\pgfsetstrokecolor{currentstroke}%
\pgfsetdash{}{0pt}%
\pgfpathmoveto{\pgfqpoint{7.265582in}{3.557991in}}%
\pgfpathlineto{\pgfqpoint{7.226858in}{2.261610in}}%
\pgfpathlineto{\pgfqpoint{5.549381in}{1.055718in}}%
\pgfusepath{stroke}%
\end{pgfscope}%
\begin{pgfscope}%
\pgfsetbuttcap%
\pgfsetroundjoin%
\pgfsetlinewidth{0.803000pt}%
\definecolor{currentstroke}{rgb}{0.690196,0.690196,0.690196}%
\pgfsetstrokecolor{currentstroke}%
\pgfsetdash{}{0pt}%
\pgfpathmoveto{\pgfqpoint{7.547313in}{3.376430in}}%
\pgfpathlineto{\pgfqpoint{7.491448in}{2.079311in}}%
\pgfpathlineto{\pgfqpoint{5.813876in}{0.851166in}}%
\pgfusepath{stroke}%
\end{pgfscope}%
\begin{pgfscope}%
\pgfsetbuttcap%
\pgfsetroundjoin%
\pgfsetlinewidth{0.803000pt}%
\definecolor{currentstroke}{rgb}{0.690196,0.690196,0.690196}%
\pgfsetstrokecolor{currentstroke}%
\pgfsetdash{}{0pt}%
\pgfpathmoveto{\pgfqpoint{7.834440in}{3.191392in}}%
\pgfpathlineto{\pgfqpoint{7.760800in}{1.893730in}}%
\pgfpathlineto{\pgfqpoint{6.083418in}{0.642710in}}%
\pgfusepath{stroke}%
\end{pgfscope}%
\begin{pgfscope}%
\pgfsetbuttcap%
\pgfsetroundjoin%
\pgfsetlinewidth{0.803000pt}%
\definecolor{currentstroke}{rgb}{0.690196,0.690196,0.690196}%
\pgfsetstrokecolor{currentstroke}%
\pgfsetdash{}{0pt}%
\pgfpathmoveto{\pgfqpoint{8.127119in}{3.002776in}}%
\pgfpathlineto{\pgfqpoint{8.035043in}{1.704779in}}%
\pgfpathlineto{\pgfqpoint{6.358152in}{0.430239in}}%
\pgfusepath{stroke}%
\end{pgfscope}%
\begin{pgfscope}%
\pgfsetrectcap%
\pgfsetroundjoin%
\pgfsetlinewidth{0.803000pt}%
\definecolor{currentstroke}{rgb}{0.000000,0.000000,0.000000}%
\pgfsetstrokecolor{currentstroke}%
\pgfsetdash{}{0pt}%
\pgfpathmoveto{\pgfqpoint{5.049158in}{1.463384in}}%
\pgfpathlineto{\pgfqpoint{5.006572in}{1.433838in}}%
\pgfusepath{stroke}%
\end{pgfscope}%
\begin{pgfscope}%
\definecolor{textcolor}{rgb}{0.000000,0.000000,0.000000}%
\pgfsetstrokecolor{textcolor}%
\pgfsetfillcolor{textcolor}%
\pgftext[x=4.899692in,y=1.255971in,,top]{\color{textcolor}\rmfamily\fontsize{10.000000}{12.000000}\selectfont \(\displaystyle {0.0}\)}%
\end{pgfscope}%
\begin{pgfscope}%
\pgfsetrectcap%
\pgfsetroundjoin%
\pgfsetlinewidth{0.803000pt}%
\definecolor{currentstroke}{rgb}{0.000000,0.000000,0.000000}%
\pgfsetstrokecolor{currentstroke}%
\pgfsetdash{}{0pt}%
\pgfpathmoveto{\pgfqpoint{5.303988in}{1.266499in}}%
\pgfpathlineto{\pgfqpoint{5.261364in}{1.236402in}}%
\pgfusepath{stroke}%
\end{pgfscope}%
\begin{pgfscope}%
\definecolor{textcolor}{rgb}{0.000000,0.000000,0.000000}%
\pgfsetstrokecolor{textcolor}%
\pgfsetfillcolor{textcolor}%
\pgftext[x=5.153182in,y=1.057127in,,top]{\color{textcolor}\rmfamily\fontsize{10.000000}{12.000000}\selectfont \(\displaystyle {0.2}\)}%
\end{pgfscope}%
\begin{pgfscope}%
\pgfsetrectcap%
\pgfsetroundjoin%
\pgfsetlinewidth{0.803000pt}%
\definecolor{currentstroke}{rgb}{0.000000,0.000000,0.000000}%
\pgfsetstrokecolor{currentstroke}%
\pgfsetdash{}{0pt}%
\pgfpathmoveto{\pgfqpoint{5.563586in}{1.065930in}}%
\pgfpathlineto{\pgfqpoint{5.520931in}{1.035266in}}%
\pgfusepath{stroke}%
\end{pgfscope}%
\begin{pgfscope}%
\definecolor{textcolor}{rgb}{0.000000,0.000000,0.000000}%
\pgfsetstrokecolor{textcolor}%
\pgfsetfillcolor{textcolor}%
\pgftext[x=5.411417in,y=0.854561in,,top]{\color{textcolor}\rmfamily\fontsize{10.000000}{12.000000}\selectfont \(\displaystyle {0.4}\)}%
\end{pgfscope}%
\begin{pgfscope}%
\pgfsetrectcap%
\pgfsetroundjoin%
\pgfsetlinewidth{0.803000pt}%
\definecolor{currentstroke}{rgb}{0.000000,0.000000,0.000000}%
\pgfsetstrokecolor{currentstroke}%
\pgfsetdash{}{0pt}%
\pgfpathmoveto{\pgfqpoint{5.828089in}{0.861571in}}%
\pgfpathlineto{\pgfqpoint{5.785409in}{0.830325in}}%
\pgfusepath{stroke}%
\end{pgfscope}%
\begin{pgfscope}%
\definecolor{textcolor}{rgb}{0.000000,0.000000,0.000000}%
\pgfsetstrokecolor{textcolor}%
\pgfsetfillcolor{textcolor}%
\pgftext[x=5.674531in,y=0.648168in,,top]{\color{textcolor}\rmfamily\fontsize{10.000000}{12.000000}\selectfont \(\displaystyle {0.6}\)}%
\end{pgfscope}%
\begin{pgfscope}%
\pgfsetrectcap%
\pgfsetroundjoin%
\pgfsetlinewidth{0.803000pt}%
\definecolor{currentstroke}{rgb}{0.000000,0.000000,0.000000}%
\pgfsetstrokecolor{currentstroke}%
\pgfsetdash{}{0pt}%
\pgfpathmoveto{\pgfqpoint{6.097637in}{0.653315in}}%
\pgfpathlineto{\pgfqpoint{6.054939in}{0.621470in}}%
\pgfusepath{stroke}%
\end{pgfscope}%
\begin{pgfscope}%
\definecolor{textcolor}{rgb}{0.000000,0.000000,0.000000}%
\pgfsetstrokecolor{textcolor}%
\pgfsetfillcolor{textcolor}%
\pgftext[x=5.942664in,y=0.437838in,,top]{\color{textcolor}\rmfamily\fontsize{10.000000}{12.000000}\selectfont \(\displaystyle {0.8}\)}%
\end{pgfscope}%
\begin{pgfscope}%
\pgfsetrectcap%
\pgfsetroundjoin%
\pgfsetlinewidth{0.803000pt}%
\definecolor{currentstroke}{rgb}{0.000000,0.000000,0.000000}%
\pgfsetstrokecolor{currentstroke}%
\pgfsetdash{}{0pt}%
\pgfpathmoveto{\pgfqpoint{6.372375in}{0.441049in}}%
\pgfpathlineto{\pgfqpoint{6.329666in}{0.408587in}}%
\pgfusepath{stroke}%
\end{pgfscope}%
\begin{pgfscope}%
\definecolor{textcolor}{rgb}{0.000000,0.000000,0.000000}%
\pgfsetstrokecolor{textcolor}%
\pgfsetfillcolor{textcolor}%
\pgftext[x=6.215962in,y=0.223457in,,top]{\color{textcolor}\rmfamily\fontsize{10.000000}{12.000000}\selectfont \(\displaystyle {1.0}\)}%
\end{pgfscope}%
\begin{pgfscope}%
\pgfsetrectcap%
\pgfsetroundjoin%
\pgfsetlinewidth{0.803000pt}%
\definecolor{currentstroke}{rgb}{0.000000,0.000000,0.000000}%
\pgfsetstrokecolor{currentstroke}%
\pgfsetdash{}{0pt}%
\pgfpathmoveto{\pgfqpoint{4.934094in}{1.531563in}}%
\pgfpathlineto{\pgfqpoint{4.824820in}{2.829674in}}%
\pgfusepath{stroke}%
\end{pgfscope}%
\begin{pgfscope}%
\definecolor{textcolor}{rgb}{0.000000,0.000000,0.000000}%
\pgfsetstrokecolor{textcolor}%
\pgfsetfillcolor{textcolor}%
\pgftext[x=4.128876in,y=2.160130in,,]{\color{textcolor}\rmfamily\fontsize{14.000000}{16.800000}\selectfont f3}%
\end{pgfscope}%
\begin{pgfscope}%
\pgfsetbuttcap%
\pgfsetroundjoin%
\pgfsetlinewidth{0.803000pt}%
\definecolor{currentstroke}{rgb}{0.690196,0.690196,0.690196}%
\pgfsetstrokecolor{currentstroke}%
\pgfsetdash{}{0pt}%
\pgfpathmoveto{\pgfqpoint{4.927541in}{1.609418in}}%
\pgfpathlineto{\pgfqpoint{6.610272in}{2.764291in}}%
\pgfpathlineto{\pgfqpoint{8.293003in}{1.609418in}}%
\pgfusepath{stroke}%
\end{pgfscope}%
\begin{pgfscope}%
\pgfsetbuttcap%
\pgfsetroundjoin%
\pgfsetlinewidth{0.803000pt}%
\definecolor{currentstroke}{rgb}{0.690196,0.690196,0.690196}%
\pgfsetstrokecolor{currentstroke}%
\pgfsetdash{}{0pt}%
\pgfpathmoveto{\pgfqpoint{4.910973in}{1.806226in}}%
\pgfpathlineto{\pgfqpoint{6.610272in}{2.960964in}}%
\pgfpathlineto{\pgfqpoint{8.309571in}{1.806226in}}%
\pgfusepath{stroke}%
\end{pgfscope}%
\begin{pgfscope}%
\pgfsetbuttcap%
\pgfsetroundjoin%
\pgfsetlinewidth{0.803000pt}%
\definecolor{currentstroke}{rgb}{0.690196,0.690196,0.690196}%
\pgfsetstrokecolor{currentstroke}%
\pgfsetdash{}{0pt}%
\pgfpathmoveto{\pgfqpoint{4.894077in}{2.006949in}}%
\pgfpathlineto{\pgfqpoint{6.610272in}{3.161331in}}%
\pgfpathlineto{\pgfqpoint{8.326467in}{2.006949in}}%
\pgfusepath{stroke}%
\end{pgfscope}%
\begin{pgfscope}%
\pgfsetbuttcap%
\pgfsetroundjoin%
\pgfsetlinewidth{0.803000pt}%
\definecolor{currentstroke}{rgb}{0.690196,0.690196,0.690196}%
\pgfsetstrokecolor{currentstroke}%
\pgfsetdash{}{0pt}%
\pgfpathmoveto{\pgfqpoint{4.876841in}{2.211703in}}%
\pgfpathlineto{\pgfqpoint{6.610272in}{3.365496in}}%
\pgfpathlineto{\pgfqpoint{8.343703in}{2.211703in}}%
\pgfusepath{stroke}%
\end{pgfscope}%
\begin{pgfscope}%
\pgfsetbuttcap%
\pgfsetroundjoin%
\pgfsetlinewidth{0.803000pt}%
\definecolor{currentstroke}{rgb}{0.690196,0.690196,0.690196}%
\pgfsetstrokecolor{currentstroke}%
\pgfsetdash{}{0pt}%
\pgfpathmoveto{\pgfqpoint{4.859255in}{2.420612in}}%
\pgfpathlineto{\pgfqpoint{6.610272in}{3.573568in}}%
\pgfpathlineto{\pgfqpoint{8.361289in}{2.420612in}}%
\pgfusepath{stroke}%
\end{pgfscope}%
\begin{pgfscope}%
\pgfsetbuttcap%
\pgfsetroundjoin%
\pgfsetlinewidth{0.803000pt}%
\definecolor{currentstroke}{rgb}{0.690196,0.690196,0.690196}%
\pgfsetstrokecolor{currentstroke}%
\pgfsetdash{}{0pt}%
\pgfpathmoveto{\pgfqpoint{4.841309in}{2.633803in}}%
\pgfpathlineto{\pgfqpoint{6.610272in}{3.785660in}}%
\pgfpathlineto{\pgfqpoint{8.379235in}{2.633803in}}%
\pgfusepath{stroke}%
\end{pgfscope}%
\begin{pgfscope}%
\pgfsetrectcap%
\pgfsetroundjoin%
\pgfsetlinewidth{0.803000pt}%
\definecolor{currentstroke}{rgb}{0.000000,0.000000,0.000000}%
\pgfsetstrokecolor{currentstroke}%
\pgfsetdash{}{0pt}%
\pgfpathmoveto{\pgfqpoint{4.941776in}{1.619187in}}%
\pgfpathlineto{\pgfqpoint{4.899031in}{1.589851in}}%
\pgfusepath{stroke}%
\end{pgfscope}%
\begin{pgfscope}%
\definecolor{textcolor}{rgb}{0.000000,0.000000,0.000000}%
\pgfsetstrokecolor{textcolor}%
\pgfsetfillcolor{textcolor}%
\pgftext[x=4.655942in,y=1.609418in,,top]{\color{textcolor}\rmfamily\fontsize{10.000000}{12.000000}\selectfont \(\displaystyle {0.0}\)}%
\end{pgfscope}%
\begin{pgfscope}%
\pgfsetrectcap%
\pgfsetroundjoin%
\pgfsetlinewidth{0.803000pt}%
\definecolor{currentstroke}{rgb}{0.000000,0.000000,0.000000}%
\pgfsetstrokecolor{currentstroke}%
\pgfsetdash{}{0pt}%
\pgfpathmoveto{\pgfqpoint{4.925356in}{1.816000in}}%
\pgfpathlineto{\pgfqpoint{4.882167in}{1.786651in}}%
\pgfusepath{stroke}%
\end{pgfscope}%
\begin{pgfscope}%
\definecolor{textcolor}{rgb}{0.000000,0.000000,0.000000}%
\pgfsetstrokecolor{textcolor}%
\pgfsetfillcolor{textcolor}%
\pgftext[x=4.636701in,y=1.806226in,,top]{\color{textcolor}\rmfamily\fontsize{10.000000}{12.000000}\selectfont \(\displaystyle {0.2}\)}%
\end{pgfscope}%
\begin{pgfscope}%
\pgfsetrectcap%
\pgfsetroundjoin%
\pgfsetlinewidth{0.803000pt}%
\definecolor{currentstroke}{rgb}{0.000000,0.000000,0.000000}%
\pgfsetstrokecolor{currentstroke}%
\pgfsetdash{}{0pt}%
\pgfpathmoveto{\pgfqpoint{4.908611in}{2.016725in}}%
\pgfpathlineto{\pgfqpoint{4.864968in}{1.987369in}}%
\pgfusepath{stroke}%
\end{pgfscope}%
\begin{pgfscope}%
\definecolor{textcolor}{rgb}{0.000000,0.000000,0.000000}%
\pgfsetstrokecolor{textcolor}%
\pgfsetfillcolor{textcolor}%
\pgftext[x=4.617077in,y=2.006949in,,top]{\color{textcolor}\rmfamily\fontsize{10.000000}{12.000000}\selectfont \(\displaystyle {0.4}\)}%
\end{pgfscope}%
\begin{pgfscope}%
\pgfsetrectcap%
\pgfsetroundjoin%
\pgfsetlinewidth{0.803000pt}%
\definecolor{currentstroke}{rgb}{0.000000,0.000000,0.000000}%
\pgfsetstrokecolor{currentstroke}%
\pgfsetdash{}{0pt}%
\pgfpathmoveto{\pgfqpoint{4.891529in}{2.221479in}}%
\pgfpathlineto{\pgfqpoint{4.847423in}{2.192122in}}%
\pgfusepath{stroke}%
\end{pgfscope}%
\begin{pgfscope}%
\definecolor{textcolor}{rgb}{0.000000,0.000000,0.000000}%
\pgfsetstrokecolor{textcolor}%
\pgfsetfillcolor{textcolor}%
\pgftext[x=4.597059in,y=2.211703in,,top]{\color{textcolor}\rmfamily\fontsize{10.000000}{12.000000}\selectfont \(\displaystyle {0.6}\)}%
\end{pgfscope}%
\begin{pgfscope}%
\pgfsetrectcap%
\pgfsetroundjoin%
\pgfsetlinewidth{0.803000pt}%
\definecolor{currentstroke}{rgb}{0.000000,0.000000,0.000000}%
\pgfsetstrokecolor{currentstroke}%
\pgfsetdash{}{0pt}%
\pgfpathmoveto{\pgfqpoint{4.874100in}{2.430387in}}%
\pgfpathlineto{\pgfqpoint{4.829521in}{2.401034in}}%
\pgfusepath{stroke}%
\end{pgfscope}%
\begin{pgfscope}%
\definecolor{textcolor}{rgb}{0.000000,0.000000,0.000000}%
\pgfsetstrokecolor{textcolor}%
\pgfsetfillcolor{textcolor}%
\pgftext[x=4.576635in,y=2.420612in,,top]{\color{textcolor}\rmfamily\fontsize{10.000000}{12.000000}\selectfont \(\displaystyle {0.8}\)}%
\end{pgfscope}%
\begin{pgfscope}%
\pgfsetrectcap%
\pgfsetroundjoin%
\pgfsetlinewidth{0.803000pt}%
\definecolor{currentstroke}{rgb}{0.000000,0.000000,0.000000}%
\pgfsetstrokecolor{currentstroke}%
\pgfsetdash{}{0pt}%
\pgfpathmoveto{\pgfqpoint{4.856315in}{2.643574in}}%
\pgfpathlineto{\pgfqpoint{4.811252in}{2.614232in}}%
\pgfusepath{stroke}%
\end{pgfscope}%
\begin{pgfscope}%
\definecolor{textcolor}{rgb}{0.000000,0.000000,0.000000}%
\pgfsetstrokecolor{textcolor}%
\pgfsetfillcolor{textcolor}%
\pgftext[x=4.555792in,y=2.633803in,,top]{\color{textcolor}\rmfamily\fontsize{10.000000}{12.000000}\selectfont \(\displaystyle {1.0}\)}%
\end{pgfscope}%
\begin{pgfscope}%
\pgfpathrectangle{\pgfqpoint{4.658921in}{0.208778in}}{\pgfqpoint{3.800000in}{3.800000in}}%
\pgfusepath{clip}%
\pgfsetbuttcap%
\pgfsetroundjoin%
\definecolor{currentfill}{rgb}{0.050070,0.192203,0.290728}%
\pgfsetfillcolor{currentfill}%
\pgfsetlinewidth{0.000000pt}%
\definecolor{currentstroke}{rgb}{0.000000,0.000000,0.000000}%
\pgfsetstrokecolor{currentstroke}%
\pgfsetdash{}{0pt}%
\pgfpathmoveto{\pgfqpoint{5.402319in}{1.622564in}}%
\pgfpathlineto{\pgfqpoint{5.281189in}{1.798485in}}%
\pgfpathlineto{\pgfqpoint{5.281320in}{1.705908in}}%
\pgfpathlineto{\pgfqpoint{5.402319in}{1.622564in}}%
\pgfpathclose%
\pgfusepath{fill}%
\end{pgfscope}%
\begin{pgfscope}%
\pgfpathrectangle{\pgfqpoint{4.658921in}{0.208778in}}{\pgfqpoint{3.800000in}{3.800000in}}%
\pgfusepath{clip}%
\pgfsetbuttcap%
\pgfsetroundjoin%
\definecolor{currentfill}{rgb}{0.050070,0.192203,0.290728}%
\pgfsetfillcolor{currentfill}%
\pgfsetlinewidth{0.000000pt}%
\definecolor{currentstroke}{rgb}{0.000000,0.000000,0.000000}%
\pgfsetstrokecolor{currentstroke}%
\pgfsetdash{}{0pt}%
\pgfpathmoveto{\pgfqpoint{7.939355in}{1.798485in}}%
\pgfpathlineto{\pgfqpoint{7.818225in}{1.622564in}}%
\pgfpathlineto{\pgfqpoint{7.939224in}{1.705908in}}%
\pgfpathlineto{\pgfqpoint{7.939355in}{1.798485in}}%
\pgfpathclose%
\pgfusepath{fill}%
\end{pgfscope}%
\begin{pgfscope}%
\pgfpathrectangle{\pgfqpoint{4.658921in}{0.208778in}}{\pgfqpoint{3.800000in}{3.800000in}}%
\pgfusepath{clip}%
\pgfsetbuttcap%
\pgfsetroundjoin%
\definecolor{currentfill}{rgb}{0.090605,0.347808,0.526096}%
\pgfsetfillcolor{currentfill}%
\pgfsetlinewidth{0.000000pt}%
\definecolor{currentstroke}{rgb}{0.000000,0.000000,0.000000}%
\pgfsetstrokecolor{currentstroke}%
\pgfsetdash{}{0pt}%
\pgfpathmoveto{\pgfqpoint{6.489141in}{3.562584in}}%
\pgfpathlineto{\pgfqpoint{6.731403in}{3.562584in}}%
\pgfpathlineto{\pgfqpoint{6.610272in}{3.646458in}}%
\pgfpathlineto{\pgfqpoint{6.489141in}{3.562584in}}%
\pgfpathclose%
\pgfusepath{fill}%
\end{pgfscope}%
\begin{pgfscope}%
\pgfpathrectangle{\pgfqpoint{4.658921in}{0.208778in}}{\pgfqpoint{3.800000in}{3.800000in}}%
\pgfusepath{clip}%
\pgfsetbuttcap%
\pgfsetroundjoin%
\definecolor{currentfill}{rgb}{0.047548,0.182523,0.276086}%
\pgfsetfillcolor{currentfill}%
\pgfsetlinewidth{0.000000pt}%
\definecolor{currentstroke}{rgb}{0.000000,0.000000,0.000000}%
\pgfsetstrokecolor{currentstroke}%
\pgfsetdash{}{0pt}%
\pgfpathmoveto{\pgfqpoint{5.559260in}{1.534809in}}%
\pgfpathlineto{\pgfqpoint{5.415384in}{1.716873in}}%
\pgfpathlineto{\pgfqpoint{5.402319in}{1.622564in}}%
\pgfpathlineto{\pgfqpoint{5.559260in}{1.534809in}}%
\pgfpathclose%
\pgfusepath{fill}%
\end{pgfscope}%
\begin{pgfscope}%
\pgfpathrectangle{\pgfqpoint{4.658921in}{0.208778in}}{\pgfqpoint{3.800000in}{3.800000in}}%
\pgfusepath{clip}%
\pgfsetbuttcap%
\pgfsetroundjoin%
\definecolor{currentfill}{rgb}{0.047548,0.182523,0.276086}%
\pgfsetfillcolor{currentfill}%
\pgfsetlinewidth{0.000000pt}%
\definecolor{currentstroke}{rgb}{0.000000,0.000000,0.000000}%
\pgfsetstrokecolor{currentstroke}%
\pgfsetdash{}{0pt}%
\pgfpathmoveto{\pgfqpoint{7.818225in}{1.622564in}}%
\pgfpathlineto{\pgfqpoint{7.805160in}{1.716873in}}%
\pgfpathlineto{\pgfqpoint{7.661284in}{1.534809in}}%
\pgfpathlineto{\pgfqpoint{7.818225in}{1.622564in}}%
\pgfpathclose%
\pgfusepath{fill}%
\end{pgfscope}%
\begin{pgfscope}%
\pgfpathrectangle{\pgfqpoint{4.658921in}{0.208778in}}{\pgfqpoint{3.800000in}{3.800000in}}%
\pgfusepath{clip}%
\pgfsetbuttcap%
\pgfsetroundjoin%
\definecolor{currentfill}{rgb}{0.048960,0.187944,0.284285}%
\pgfsetfillcolor{currentfill}%
\pgfsetlinewidth{0.000000pt}%
\definecolor{currentstroke}{rgb}{0.000000,0.000000,0.000000}%
\pgfsetstrokecolor{currentstroke}%
\pgfsetdash{}{0pt}%
\pgfpathmoveto{\pgfqpoint{5.281189in}{1.798485in}}%
\pgfpathlineto{\pgfqpoint{5.402319in}{1.622564in}}%
\pgfpathlineto{\pgfqpoint{5.401224in}{2.252417in}}%
\pgfpathlineto{\pgfqpoint{5.281189in}{1.798485in}}%
\pgfpathclose%
\pgfusepath{fill}%
\end{pgfscope}%
\begin{pgfscope}%
\pgfpathrectangle{\pgfqpoint{4.658921in}{0.208778in}}{\pgfqpoint{3.800000in}{3.800000in}}%
\pgfusepath{clip}%
\pgfsetbuttcap%
\pgfsetroundjoin%
\definecolor{currentfill}{rgb}{0.048960,0.187944,0.284285}%
\pgfsetfillcolor{currentfill}%
\pgfsetlinewidth{0.000000pt}%
\definecolor{currentstroke}{rgb}{0.000000,0.000000,0.000000}%
\pgfsetstrokecolor{currentstroke}%
\pgfsetdash{}{0pt}%
\pgfpathmoveto{\pgfqpoint{7.939355in}{1.798485in}}%
\pgfpathlineto{\pgfqpoint{7.819320in}{2.252417in}}%
\pgfpathlineto{\pgfqpoint{7.818225in}{1.622564in}}%
\pgfpathlineto{\pgfqpoint{7.939355in}{1.798485in}}%
\pgfpathclose%
\pgfusepath{fill}%
\end{pgfscope}%
\begin{pgfscope}%
\pgfpathrectangle{\pgfqpoint{4.658921in}{0.208778in}}{\pgfqpoint{3.800000in}{3.800000in}}%
\pgfusepath{clip}%
\pgfsetbuttcap%
\pgfsetroundjoin%
\definecolor{currentfill}{rgb}{0.070254,0.269685,0.407928}%
\pgfsetfillcolor{currentfill}%
\pgfsetlinewidth{0.000000pt}%
\definecolor{currentstroke}{rgb}{0.000000,0.000000,0.000000}%
\pgfsetstrokecolor{currentstroke}%
\pgfsetdash{}{0pt}%
\pgfpathmoveto{\pgfqpoint{5.401224in}{2.252417in}}%
\pgfpathlineto{\pgfqpoint{5.402319in}{1.622564in}}%
\pgfpathlineto{\pgfqpoint{5.415384in}{1.716873in}}%
\pgfpathlineto{\pgfqpoint{5.401224in}{2.252417in}}%
\pgfpathclose%
\pgfusepath{fill}%
\end{pgfscope}%
\begin{pgfscope}%
\pgfpathrectangle{\pgfqpoint{4.658921in}{0.208778in}}{\pgfqpoint{3.800000in}{3.800000in}}%
\pgfusepath{clip}%
\pgfsetbuttcap%
\pgfsetroundjoin%
\definecolor{currentfill}{rgb}{0.070254,0.269685,0.407928}%
\pgfsetfillcolor{currentfill}%
\pgfsetlinewidth{0.000000pt}%
\definecolor{currentstroke}{rgb}{0.000000,0.000000,0.000000}%
\pgfsetstrokecolor{currentstroke}%
\pgfsetdash{}{0pt}%
\pgfpathmoveto{\pgfqpoint{7.805160in}{1.716873in}}%
\pgfpathlineto{\pgfqpoint{7.818225in}{1.622564in}}%
\pgfpathlineto{\pgfqpoint{7.819320in}{2.252417in}}%
\pgfpathlineto{\pgfqpoint{7.805160in}{1.716873in}}%
\pgfpathclose%
\pgfusepath{fill}%
\end{pgfscope}%
\begin{pgfscope}%
\pgfpathrectangle{\pgfqpoint{4.658921in}{0.208778in}}{\pgfqpoint{3.800000in}{3.800000in}}%
\pgfusepath{clip}%
\pgfsetbuttcap%
\pgfsetroundjoin%
\definecolor{currentfill}{rgb}{0.044978,0.172658,0.261163}%
\pgfsetfillcolor{currentfill}%
\pgfsetlinewidth{0.000000pt}%
\definecolor{currentstroke}{rgb}{0.000000,0.000000,0.000000}%
\pgfsetstrokecolor{currentstroke}%
\pgfsetdash{}{0pt}%
\pgfpathmoveto{\pgfqpoint{5.759458in}{1.448342in}}%
\pgfpathlineto{\pgfqpoint{5.593156in}{1.631177in}}%
\pgfpathlineto{\pgfqpoint{5.559260in}{1.534809in}}%
\pgfpathlineto{\pgfqpoint{5.759458in}{1.448342in}}%
\pgfpathclose%
\pgfusepath{fill}%
\end{pgfscope}%
\begin{pgfscope}%
\pgfpathrectangle{\pgfqpoint{4.658921in}{0.208778in}}{\pgfqpoint{3.800000in}{3.800000in}}%
\pgfusepath{clip}%
\pgfsetbuttcap%
\pgfsetroundjoin%
\definecolor{currentfill}{rgb}{0.044978,0.172658,0.261163}%
\pgfsetfillcolor{currentfill}%
\pgfsetlinewidth{0.000000pt}%
\definecolor{currentstroke}{rgb}{0.000000,0.000000,0.000000}%
\pgfsetstrokecolor{currentstroke}%
\pgfsetdash{}{0pt}%
\pgfpathmoveto{\pgfqpoint{7.661284in}{1.534809in}}%
\pgfpathlineto{\pgfqpoint{7.627388in}{1.631177in}}%
\pgfpathlineto{\pgfqpoint{7.461086in}{1.448342in}}%
\pgfpathlineto{\pgfqpoint{7.661284in}{1.534809in}}%
\pgfpathclose%
\pgfusepath{fill}%
\end{pgfscope}%
\begin{pgfscope}%
\pgfpathrectangle{\pgfqpoint{4.658921in}{0.208778in}}{\pgfqpoint{3.800000in}{3.800000in}}%
\pgfusepath{clip}%
\pgfsetbuttcap%
\pgfsetroundjoin%
\definecolor{currentfill}{rgb}{0.081954,0.314596,0.475860}%
\pgfsetfillcolor{currentfill}%
\pgfsetlinewidth{0.000000pt}%
\definecolor{currentstroke}{rgb}{0.000000,0.000000,0.000000}%
\pgfsetstrokecolor{currentstroke}%
\pgfsetdash{}{0pt}%
\pgfpathmoveto{\pgfqpoint{6.731403in}{3.562584in}}%
\pgfpathlineto{\pgfqpoint{6.489141in}{3.562584in}}%
\pgfpathlineto{\pgfqpoint{6.436911in}{2.956025in}}%
\pgfpathlineto{\pgfqpoint{6.731403in}{3.562584in}}%
\pgfpathclose%
\pgfusepath{fill}%
\end{pgfscope}%
\begin{pgfscope}%
\pgfpathrectangle{\pgfqpoint{4.658921in}{0.208778in}}{\pgfqpoint{3.800000in}{3.800000in}}%
\pgfusepath{clip}%
\pgfsetbuttcap%
\pgfsetroundjoin%
\definecolor{currentfill}{rgb}{0.047247,0.181368,0.274339}%
\pgfsetfillcolor{currentfill}%
\pgfsetlinewidth{0.000000pt}%
\definecolor{currentstroke}{rgb}{0.000000,0.000000,0.000000}%
\pgfsetstrokecolor{currentstroke}%
\pgfsetdash{}{0pt}%
\pgfpathmoveto{\pgfqpoint{5.415384in}{1.716873in}}%
\pgfpathlineto{\pgfqpoint{5.559260in}{1.534809in}}%
\pgfpathlineto{\pgfqpoint{5.605472in}{2.212212in}}%
\pgfpathlineto{\pgfqpoint{5.415384in}{1.716873in}}%
\pgfpathclose%
\pgfusepath{fill}%
\end{pgfscope}%
\begin{pgfscope}%
\pgfpathrectangle{\pgfqpoint{4.658921in}{0.208778in}}{\pgfqpoint{3.800000in}{3.800000in}}%
\pgfusepath{clip}%
\pgfsetbuttcap%
\pgfsetroundjoin%
\definecolor{currentfill}{rgb}{0.047247,0.181368,0.274339}%
\pgfsetfillcolor{currentfill}%
\pgfsetlinewidth{0.000000pt}%
\definecolor{currentstroke}{rgb}{0.000000,0.000000,0.000000}%
\pgfsetstrokecolor{currentstroke}%
\pgfsetdash{}{0pt}%
\pgfpathmoveto{\pgfqpoint{7.615072in}{2.212212in}}%
\pgfpathlineto{\pgfqpoint{7.661284in}{1.534809in}}%
\pgfpathlineto{\pgfqpoint{7.805160in}{1.716873in}}%
\pgfpathlineto{\pgfqpoint{7.615072in}{2.212212in}}%
\pgfpathclose%
\pgfusepath{fill}%
\end{pgfscope}%
\begin{pgfscope}%
\pgfpathrectangle{\pgfqpoint{4.658921in}{0.208778in}}{\pgfqpoint{3.800000in}{3.800000in}}%
\pgfusepath{clip}%
\pgfsetbuttcap%
\pgfsetroundjoin%
\definecolor{currentfill}{rgb}{0.067179,0.257880,0.390071}%
\pgfsetfillcolor{currentfill}%
\pgfsetlinewidth{0.000000pt}%
\definecolor{currentstroke}{rgb}{0.000000,0.000000,0.000000}%
\pgfsetstrokecolor{currentstroke}%
\pgfsetdash{}{0pt}%
\pgfpathmoveto{\pgfqpoint{5.605472in}{2.212212in}}%
\pgfpathlineto{\pgfqpoint{5.559260in}{1.534809in}}%
\pgfpathlineto{\pgfqpoint{5.593156in}{1.631177in}}%
\pgfpathlineto{\pgfqpoint{5.605472in}{2.212212in}}%
\pgfpathclose%
\pgfusepath{fill}%
\end{pgfscope}%
\begin{pgfscope}%
\pgfpathrectangle{\pgfqpoint{4.658921in}{0.208778in}}{\pgfqpoint{3.800000in}{3.800000in}}%
\pgfusepath{clip}%
\pgfsetbuttcap%
\pgfsetroundjoin%
\definecolor{currentfill}{rgb}{0.067179,0.257880,0.390071}%
\pgfsetfillcolor{currentfill}%
\pgfsetlinewidth{0.000000pt}%
\definecolor{currentstroke}{rgb}{0.000000,0.000000,0.000000}%
\pgfsetstrokecolor{currentstroke}%
\pgfsetdash{}{0pt}%
\pgfpathmoveto{\pgfqpoint{7.627388in}{1.631177in}}%
\pgfpathlineto{\pgfqpoint{7.661284in}{1.534809in}}%
\pgfpathlineto{\pgfqpoint{7.615072in}{2.212212in}}%
\pgfpathlineto{\pgfqpoint{7.627388in}{1.631177in}}%
\pgfpathclose%
\pgfusepath{fill}%
\end{pgfscope}%
\begin{pgfscope}%
\pgfpathrectangle{\pgfqpoint{4.658921in}{0.208778in}}{\pgfqpoint{3.800000in}{3.800000in}}%
\pgfusepath{clip}%
\pgfsetbuttcap%
\pgfsetroundjoin%
\definecolor{currentfill}{rgb}{0.042579,0.163449,0.247234}%
\pgfsetfillcolor{currentfill}%
\pgfsetlinewidth{0.000000pt}%
\definecolor{currentstroke}{rgb}{0.000000,0.000000,0.000000}%
\pgfsetstrokecolor{currentstroke}%
\pgfsetdash{}{0pt}%
\pgfpathmoveto{\pgfqpoint{5.759458in}{1.448342in}}%
\pgfpathlineto{\pgfqpoint{6.006510in}{1.372555in}}%
\pgfpathlineto{\pgfqpoint{5.823361in}{1.548607in}}%
\pgfpathlineto{\pgfqpoint{5.759458in}{1.448342in}}%
\pgfpathclose%
\pgfusepath{fill}%
\end{pgfscope}%
\begin{pgfscope}%
\pgfpathrectangle{\pgfqpoint{4.658921in}{0.208778in}}{\pgfqpoint{3.800000in}{3.800000in}}%
\pgfusepath{clip}%
\pgfsetbuttcap%
\pgfsetroundjoin%
\definecolor{currentfill}{rgb}{0.042579,0.163449,0.247234}%
\pgfsetfillcolor{currentfill}%
\pgfsetlinewidth{0.000000pt}%
\definecolor{currentstroke}{rgb}{0.000000,0.000000,0.000000}%
\pgfsetstrokecolor{currentstroke}%
\pgfsetdash{}{0pt}%
\pgfpathmoveto{\pgfqpoint{7.397183in}{1.548607in}}%
\pgfpathlineto{\pgfqpoint{7.214034in}{1.372555in}}%
\pgfpathlineto{\pgfqpoint{7.461086in}{1.448342in}}%
\pgfpathlineto{\pgfqpoint{7.397183in}{1.548607in}}%
\pgfpathclose%
\pgfusepath{fill}%
\end{pgfscope}%
\begin{pgfscope}%
\pgfpathrectangle{\pgfqpoint{4.658921in}{0.208778in}}{\pgfqpoint{3.800000in}{3.800000in}}%
\pgfusepath{clip}%
\pgfsetbuttcap%
\pgfsetroundjoin%
\definecolor{currentfill}{rgb}{0.052493,0.201505,0.304798}%
\pgfsetfillcolor{currentfill}%
\pgfsetlinewidth{0.000000pt}%
\definecolor{currentstroke}{rgb}{0.000000,0.000000,0.000000}%
\pgfsetstrokecolor{currentstroke}%
\pgfsetdash{}{0pt}%
\pgfpathmoveto{\pgfqpoint{5.401224in}{2.252417in}}%
\pgfpathlineto{\pgfqpoint{5.415384in}{1.716873in}}%
\pgfpathlineto{\pgfqpoint{5.605472in}{2.212212in}}%
\pgfpathlineto{\pgfqpoint{5.401224in}{2.252417in}}%
\pgfpathclose%
\pgfusepath{fill}%
\end{pgfscope}%
\begin{pgfscope}%
\pgfpathrectangle{\pgfqpoint{4.658921in}{0.208778in}}{\pgfqpoint{3.800000in}{3.800000in}}%
\pgfusepath{clip}%
\pgfsetbuttcap%
\pgfsetroundjoin%
\definecolor{currentfill}{rgb}{0.052493,0.201505,0.304798}%
\pgfsetfillcolor{currentfill}%
\pgfsetlinewidth{0.000000pt}%
\definecolor{currentstroke}{rgb}{0.000000,0.000000,0.000000}%
\pgfsetstrokecolor{currentstroke}%
\pgfsetdash{}{0pt}%
\pgfpathmoveto{\pgfqpoint{7.615072in}{2.212212in}}%
\pgfpathlineto{\pgfqpoint{7.805160in}{1.716873in}}%
\pgfpathlineto{\pgfqpoint{7.819320in}{2.252417in}}%
\pgfpathlineto{\pgfqpoint{7.615072in}{2.212212in}}%
\pgfpathclose%
\pgfusepath{fill}%
\end{pgfscope}%
\begin{pgfscope}%
\pgfpathrectangle{\pgfqpoint{4.658921in}{0.208778in}}{\pgfqpoint{3.800000in}{3.800000in}}%
\pgfusepath{clip}%
\pgfsetbuttcap%
\pgfsetroundjoin%
\definecolor{currentfill}{rgb}{0.082280,0.315849,0.477755}%
\pgfsetfillcolor{currentfill}%
\pgfsetlinewidth{0.000000pt}%
\definecolor{currentstroke}{rgb}{0.000000,0.000000,0.000000}%
\pgfsetstrokecolor{currentstroke}%
\pgfsetdash{}{0pt}%
\pgfpathmoveto{\pgfqpoint{7.215560in}{3.134503in}}%
\pgfpathlineto{\pgfqpoint{6.731403in}{3.562584in}}%
\pgfpathlineto{\pgfqpoint{6.783633in}{2.956025in}}%
\pgfpathlineto{\pgfqpoint{7.215560in}{3.134503in}}%
\pgfpathclose%
\pgfusepath{fill}%
\end{pgfscope}%
\begin{pgfscope}%
\pgfpathrectangle{\pgfqpoint{4.658921in}{0.208778in}}{\pgfqpoint{3.800000in}{3.800000in}}%
\pgfusepath{clip}%
\pgfsetbuttcap%
\pgfsetroundjoin%
\definecolor{currentfill}{rgb}{0.082280,0.315849,0.477755}%
\pgfsetfillcolor{currentfill}%
\pgfsetlinewidth{0.000000pt}%
\definecolor{currentstroke}{rgb}{0.000000,0.000000,0.000000}%
\pgfsetstrokecolor{currentstroke}%
\pgfsetdash{}{0pt}%
\pgfpathmoveto{\pgfqpoint{6.436911in}{2.956025in}}%
\pgfpathlineto{\pgfqpoint{6.489141in}{3.562584in}}%
\pgfpathlineto{\pgfqpoint{6.004984in}{3.134503in}}%
\pgfpathlineto{\pgfqpoint{6.436911in}{2.956025in}}%
\pgfpathclose%
\pgfusepath{fill}%
\end{pgfscope}%
\begin{pgfscope}%
\pgfpathrectangle{\pgfqpoint{4.658921in}{0.208778in}}{\pgfqpoint{3.800000in}{3.800000in}}%
\pgfusepath{clip}%
\pgfsetbuttcap%
\pgfsetroundjoin%
\definecolor{currentfill}{rgb}{0.045702,0.175435,0.265364}%
\pgfsetfillcolor{currentfill}%
\pgfsetlinewidth{0.000000pt}%
\definecolor{currentstroke}{rgb}{0.000000,0.000000,0.000000}%
\pgfsetstrokecolor{currentstroke}%
\pgfsetdash{}{0pt}%
\pgfpathmoveto{\pgfqpoint{5.593156in}{1.631177in}}%
\pgfpathlineto{\pgfqpoint{5.759458in}{1.448342in}}%
\pgfpathlineto{\pgfqpoint{5.878629in}{2.174201in}}%
\pgfpathlineto{\pgfqpoint{5.593156in}{1.631177in}}%
\pgfpathclose%
\pgfusepath{fill}%
\end{pgfscope}%
\begin{pgfscope}%
\pgfpathrectangle{\pgfqpoint{4.658921in}{0.208778in}}{\pgfqpoint{3.800000in}{3.800000in}}%
\pgfusepath{clip}%
\pgfsetbuttcap%
\pgfsetroundjoin%
\definecolor{currentfill}{rgb}{0.045702,0.175435,0.265364}%
\pgfsetfillcolor{currentfill}%
\pgfsetlinewidth{0.000000pt}%
\definecolor{currentstroke}{rgb}{0.000000,0.000000,0.000000}%
\pgfsetstrokecolor{currentstroke}%
\pgfsetdash{}{0pt}%
\pgfpathmoveto{\pgfqpoint{7.341915in}{2.174201in}}%
\pgfpathlineto{\pgfqpoint{7.461086in}{1.448342in}}%
\pgfpathlineto{\pgfqpoint{7.627388in}{1.631177in}}%
\pgfpathlineto{\pgfqpoint{7.341915in}{2.174201in}}%
\pgfpathclose%
\pgfusepath{fill}%
\end{pgfscope}%
\begin{pgfscope}%
\pgfpathrectangle{\pgfqpoint{4.658921in}{0.208778in}}{\pgfqpoint{3.800000in}{3.800000in}}%
\pgfusepath{clip}%
\pgfsetbuttcap%
\pgfsetroundjoin%
\definecolor{currentfill}{rgb}{0.040669,0.156116,0.236142}%
\pgfsetfillcolor{currentfill}%
\pgfsetlinewidth{0.000000pt}%
\definecolor{currentstroke}{rgb}{0.000000,0.000000,0.000000}%
\pgfsetstrokecolor{currentstroke}%
\pgfsetdash{}{0pt}%
\pgfpathmoveto{\pgfqpoint{6.108398in}{1.480874in}}%
\pgfpathlineto{\pgfqpoint{6.006510in}{1.372555in}}%
\pgfpathlineto{\pgfqpoint{6.295611in}{1.319560in}}%
\pgfpathlineto{\pgfqpoint{6.108398in}{1.480874in}}%
\pgfpathclose%
\pgfusepath{fill}%
\end{pgfscope}%
\begin{pgfscope}%
\pgfpathrectangle{\pgfqpoint{4.658921in}{0.208778in}}{\pgfqpoint{3.800000in}{3.800000in}}%
\pgfusepath{clip}%
\pgfsetbuttcap%
\pgfsetroundjoin%
\definecolor{currentfill}{rgb}{0.040669,0.156116,0.236142}%
\pgfsetfillcolor{currentfill}%
\pgfsetlinewidth{0.000000pt}%
\definecolor{currentstroke}{rgb}{0.000000,0.000000,0.000000}%
\pgfsetstrokecolor{currentstroke}%
\pgfsetdash{}{0pt}%
\pgfpathmoveto{\pgfqpoint{6.924933in}{1.319560in}}%
\pgfpathlineto{\pgfqpoint{7.214034in}{1.372555in}}%
\pgfpathlineto{\pgfqpoint{7.112146in}{1.480874in}}%
\pgfpathlineto{\pgfqpoint{6.924933in}{1.319560in}}%
\pgfpathclose%
\pgfusepath{fill}%
\end{pgfscope}%
\begin{pgfscope}%
\pgfpathrectangle{\pgfqpoint{4.658921in}{0.208778in}}{\pgfqpoint{3.800000in}{3.800000in}}%
\pgfusepath{clip}%
\pgfsetbuttcap%
\pgfsetroundjoin%
\definecolor{currentfill}{rgb}{0.063981,0.245604,0.371502}%
\pgfsetfillcolor{currentfill}%
\pgfsetlinewidth{0.000000pt}%
\definecolor{currentstroke}{rgb}{0.000000,0.000000,0.000000}%
\pgfsetstrokecolor{currentstroke}%
\pgfsetdash{}{0pt}%
\pgfpathmoveto{\pgfqpoint{5.878629in}{2.174201in}}%
\pgfpathlineto{\pgfqpoint{5.759458in}{1.448342in}}%
\pgfpathlineto{\pgfqpoint{5.823361in}{1.548607in}}%
\pgfpathlineto{\pgfqpoint{5.878629in}{2.174201in}}%
\pgfpathclose%
\pgfusepath{fill}%
\end{pgfscope}%
\begin{pgfscope}%
\pgfpathrectangle{\pgfqpoint{4.658921in}{0.208778in}}{\pgfqpoint{3.800000in}{3.800000in}}%
\pgfusepath{clip}%
\pgfsetbuttcap%
\pgfsetroundjoin%
\definecolor{currentfill}{rgb}{0.063981,0.245604,0.371502}%
\pgfsetfillcolor{currentfill}%
\pgfsetlinewidth{0.000000pt}%
\definecolor{currentstroke}{rgb}{0.000000,0.000000,0.000000}%
\pgfsetstrokecolor{currentstroke}%
\pgfsetdash{}{0pt}%
\pgfpathmoveto{\pgfqpoint{7.397183in}{1.548607in}}%
\pgfpathlineto{\pgfqpoint{7.461086in}{1.448342in}}%
\pgfpathlineto{\pgfqpoint{7.341915in}{2.174201in}}%
\pgfpathlineto{\pgfqpoint{7.397183in}{1.548607in}}%
\pgfpathclose%
\pgfusepath{fill}%
\end{pgfscope}%
\begin{pgfscope}%
\pgfpathrectangle{\pgfqpoint{4.658921in}{0.208778in}}{\pgfqpoint{3.800000in}{3.800000in}}%
\pgfusepath{clip}%
\pgfsetbuttcap%
\pgfsetroundjoin%
\definecolor{currentfill}{rgb}{0.060942,0.233938,0.353856}%
\pgfsetfillcolor{currentfill}%
\pgfsetlinewidth{0.000000pt}%
\definecolor{currentstroke}{rgb}{0.000000,0.000000,0.000000}%
\pgfsetstrokecolor{currentstroke}%
\pgfsetdash{}{0pt}%
\pgfpathmoveto{\pgfqpoint{5.605472in}{2.212212in}}%
\pgfpathlineto{\pgfqpoint{5.507147in}{2.466299in}}%
\pgfpathlineto{\pgfqpoint{5.401224in}{2.252417in}}%
\pgfpathlineto{\pgfqpoint{5.605472in}{2.212212in}}%
\pgfpathclose%
\pgfusepath{fill}%
\end{pgfscope}%
\begin{pgfscope}%
\pgfpathrectangle{\pgfqpoint{4.658921in}{0.208778in}}{\pgfqpoint{3.800000in}{3.800000in}}%
\pgfusepath{clip}%
\pgfsetbuttcap%
\pgfsetroundjoin%
\definecolor{currentfill}{rgb}{0.060942,0.233938,0.353856}%
\pgfsetfillcolor{currentfill}%
\pgfsetlinewidth{0.000000pt}%
\definecolor{currentstroke}{rgb}{0.000000,0.000000,0.000000}%
\pgfsetstrokecolor{currentstroke}%
\pgfsetdash{}{0pt}%
\pgfpathmoveto{\pgfqpoint{7.819320in}{2.252417in}}%
\pgfpathlineto{\pgfqpoint{7.713397in}{2.466299in}}%
\pgfpathlineto{\pgfqpoint{7.615072in}{2.212212in}}%
\pgfpathlineto{\pgfqpoint{7.819320in}{2.252417in}}%
\pgfpathclose%
\pgfusepath{fill}%
\end{pgfscope}%
\begin{pgfscope}%
\pgfpathrectangle{\pgfqpoint{4.658921in}{0.208778in}}{\pgfqpoint{3.800000in}{3.800000in}}%
\pgfusepath{clip}%
\pgfsetbuttcap%
\pgfsetroundjoin%
\definecolor{currentfill}{rgb}{0.081954,0.314596,0.475860}%
\pgfsetfillcolor{currentfill}%
\pgfsetlinewidth{0.000000pt}%
\definecolor{currentstroke}{rgb}{0.000000,0.000000,0.000000}%
\pgfsetstrokecolor{currentstroke}%
\pgfsetdash{}{0pt}%
\pgfpathmoveto{\pgfqpoint{6.436911in}{2.956025in}}%
\pgfpathlineto{\pgfqpoint{6.783633in}{2.956025in}}%
\pgfpathlineto{\pgfqpoint{6.731403in}{3.562584in}}%
\pgfpathlineto{\pgfqpoint{6.436911in}{2.956025in}}%
\pgfpathclose%
\pgfusepath{fill}%
\end{pgfscope}%
\begin{pgfscope}%
\pgfpathrectangle{\pgfqpoint{4.658921in}{0.208778in}}{\pgfqpoint{3.800000in}{3.800000in}}%
\pgfusepath{clip}%
\pgfsetbuttcap%
\pgfsetroundjoin%
\definecolor{currentfill}{rgb}{0.039595,0.151995,0.229908}%
\pgfsetfillcolor{currentfill}%
\pgfsetlinewidth{0.000000pt}%
\definecolor{currentstroke}{rgb}{0.000000,0.000000,0.000000}%
\pgfsetstrokecolor{currentstroke}%
\pgfsetdash{}{0pt}%
\pgfpathmoveto{\pgfqpoint{6.924933in}{1.319560in}}%
\pgfpathlineto{\pgfqpoint{6.783257in}{1.441955in}}%
\pgfpathlineto{\pgfqpoint{6.610272in}{1.300375in}}%
\pgfpathlineto{\pgfqpoint{6.924933in}{1.319560in}}%
\pgfpathclose%
\pgfusepath{fill}%
\end{pgfscope}%
\begin{pgfscope}%
\pgfpathrectangle{\pgfqpoint{4.658921in}{0.208778in}}{\pgfqpoint{3.800000in}{3.800000in}}%
\pgfusepath{clip}%
\pgfsetbuttcap%
\pgfsetroundjoin%
\definecolor{currentfill}{rgb}{0.039595,0.151995,0.229908}%
\pgfsetfillcolor{currentfill}%
\pgfsetlinewidth{0.000000pt}%
\definecolor{currentstroke}{rgb}{0.000000,0.000000,0.000000}%
\pgfsetstrokecolor{currentstroke}%
\pgfsetdash{}{0pt}%
\pgfpathmoveto{\pgfqpoint{6.610272in}{1.300375in}}%
\pgfpathlineto{\pgfqpoint{6.437287in}{1.441955in}}%
\pgfpathlineto{\pgfqpoint{6.295611in}{1.319560in}}%
\pgfpathlineto{\pgfqpoint{6.610272in}{1.300375in}}%
\pgfpathclose%
\pgfusepath{fill}%
\end{pgfscope}%
\begin{pgfscope}%
\pgfpathrectangle{\pgfqpoint{4.658921in}{0.208778in}}{\pgfqpoint{3.800000in}{3.800000in}}%
\pgfusepath{clip}%
\pgfsetbuttcap%
\pgfsetroundjoin%
\definecolor{currentfill}{rgb}{0.075436,0.289576,0.438014}%
\pgfsetfillcolor{currentfill}%
\pgfsetlinewidth{0.000000pt}%
\definecolor{currentstroke}{rgb}{0.000000,0.000000,0.000000}%
\pgfsetstrokecolor{currentstroke}%
\pgfsetdash{}{0pt}%
\pgfpathmoveto{\pgfqpoint{6.107344in}{2.945086in}}%
\pgfpathlineto{\pgfqpoint{6.004984in}{3.134503in}}%
\pgfpathlineto{\pgfqpoint{5.821807in}{2.926052in}}%
\pgfpathlineto{\pgfqpoint{6.107344in}{2.945086in}}%
\pgfpathclose%
\pgfusepath{fill}%
\end{pgfscope}%
\begin{pgfscope}%
\pgfpathrectangle{\pgfqpoint{4.658921in}{0.208778in}}{\pgfqpoint{3.800000in}{3.800000in}}%
\pgfusepath{clip}%
\pgfsetbuttcap%
\pgfsetroundjoin%
\definecolor{currentfill}{rgb}{0.075436,0.289576,0.438014}%
\pgfsetfillcolor{currentfill}%
\pgfsetlinewidth{0.000000pt}%
\definecolor{currentstroke}{rgb}{0.000000,0.000000,0.000000}%
\pgfsetstrokecolor{currentstroke}%
\pgfsetdash{}{0pt}%
\pgfpathmoveto{\pgfqpoint{7.398737in}{2.926052in}}%
\pgfpathlineto{\pgfqpoint{7.215560in}{3.134503in}}%
\pgfpathlineto{\pgfqpoint{7.113200in}{2.945086in}}%
\pgfpathlineto{\pgfqpoint{7.398737in}{2.926052in}}%
\pgfpathclose%
\pgfusepath{fill}%
\end{pgfscope}%
\begin{pgfscope}%
\pgfpathrectangle{\pgfqpoint{4.658921in}{0.208778in}}{\pgfqpoint{3.800000in}{3.800000in}}%
\pgfusepath{clip}%
\pgfsetbuttcap%
\pgfsetroundjoin%
\definecolor{currentfill}{rgb}{0.062760,0.240916,0.364410}%
\pgfsetfillcolor{currentfill}%
\pgfsetlinewidth{0.000000pt}%
\definecolor{currentstroke}{rgb}{0.000000,0.000000,0.000000}%
\pgfsetstrokecolor{currentstroke}%
\pgfsetdash{}{0pt}%
\pgfpathmoveto{\pgfqpoint{5.605472in}{2.212212in}}%
\pgfpathlineto{\pgfqpoint{5.821807in}{2.926052in}}%
\pgfpathlineto{\pgfqpoint{5.507147in}{2.466299in}}%
\pgfpathlineto{\pgfqpoint{5.605472in}{2.212212in}}%
\pgfpathclose%
\pgfusepath{fill}%
\end{pgfscope}%
\begin{pgfscope}%
\pgfpathrectangle{\pgfqpoint{4.658921in}{0.208778in}}{\pgfqpoint{3.800000in}{3.800000in}}%
\pgfusepath{clip}%
\pgfsetbuttcap%
\pgfsetroundjoin%
\definecolor{currentfill}{rgb}{0.062760,0.240916,0.364410}%
\pgfsetfillcolor{currentfill}%
\pgfsetlinewidth{0.000000pt}%
\definecolor{currentstroke}{rgb}{0.000000,0.000000,0.000000}%
\pgfsetstrokecolor{currentstroke}%
\pgfsetdash{}{0pt}%
\pgfpathmoveto{\pgfqpoint{7.713397in}{2.466299in}}%
\pgfpathlineto{\pgfqpoint{7.398737in}{2.926052in}}%
\pgfpathlineto{\pgfqpoint{7.615072in}{2.212212in}}%
\pgfpathlineto{\pgfqpoint{7.713397in}{2.466299in}}%
\pgfpathclose%
\pgfusepath{fill}%
\end{pgfscope}%
\begin{pgfscope}%
\pgfpathrectangle{\pgfqpoint{4.658921in}{0.208778in}}{\pgfqpoint{3.800000in}{3.800000in}}%
\pgfusepath{clip}%
\pgfsetbuttcap%
\pgfsetroundjoin%
\definecolor{currentfill}{rgb}{0.043508,0.167016,0.252629}%
\pgfsetfillcolor{currentfill}%
\pgfsetlinewidth{0.000000pt}%
\definecolor{currentstroke}{rgb}{0.000000,0.000000,0.000000}%
\pgfsetstrokecolor{currentstroke}%
\pgfsetdash{}{0pt}%
\pgfpathmoveto{\pgfqpoint{5.823361in}{1.548607in}}%
\pgfpathlineto{\pgfqpoint{6.006510in}{1.372555in}}%
\pgfpathlineto{\pgfqpoint{6.050897in}{1.893599in}}%
\pgfpathlineto{\pgfqpoint{5.823361in}{1.548607in}}%
\pgfpathclose%
\pgfusepath{fill}%
\end{pgfscope}%
\begin{pgfscope}%
\pgfpathrectangle{\pgfqpoint{4.658921in}{0.208778in}}{\pgfqpoint{3.800000in}{3.800000in}}%
\pgfusepath{clip}%
\pgfsetbuttcap%
\pgfsetroundjoin%
\definecolor{currentfill}{rgb}{0.043508,0.167016,0.252629}%
\pgfsetfillcolor{currentfill}%
\pgfsetlinewidth{0.000000pt}%
\definecolor{currentstroke}{rgb}{0.000000,0.000000,0.000000}%
\pgfsetstrokecolor{currentstroke}%
\pgfsetdash{}{0pt}%
\pgfpathmoveto{\pgfqpoint{7.169647in}{1.893599in}}%
\pgfpathlineto{\pgfqpoint{7.214034in}{1.372555in}}%
\pgfpathlineto{\pgfqpoint{7.397183in}{1.548607in}}%
\pgfpathlineto{\pgfqpoint{7.169647in}{1.893599in}}%
\pgfpathclose%
\pgfusepath{fill}%
\end{pgfscope}%
\begin{pgfscope}%
\pgfpathrectangle{\pgfqpoint{4.658921in}{0.208778in}}{\pgfqpoint{3.800000in}{3.800000in}}%
\pgfusepath{clip}%
\pgfsetbuttcap%
\pgfsetroundjoin%
\definecolor{currentfill}{rgb}{0.050011,0.191979,0.290388}%
\pgfsetfillcolor{currentfill}%
\pgfsetlinewidth{0.000000pt}%
\definecolor{currentstroke}{rgb}{0.000000,0.000000,0.000000}%
\pgfsetstrokecolor{currentstroke}%
\pgfsetdash{}{0pt}%
\pgfpathmoveto{\pgfqpoint{5.605472in}{2.212212in}}%
\pgfpathlineto{\pgfqpoint{5.593156in}{1.631177in}}%
\pgfpathlineto{\pgfqpoint{5.878629in}{2.174201in}}%
\pgfpathlineto{\pgfqpoint{5.605472in}{2.212212in}}%
\pgfpathclose%
\pgfusepath{fill}%
\end{pgfscope}%
\begin{pgfscope}%
\pgfpathrectangle{\pgfqpoint{4.658921in}{0.208778in}}{\pgfqpoint{3.800000in}{3.800000in}}%
\pgfusepath{clip}%
\pgfsetbuttcap%
\pgfsetroundjoin%
\definecolor{currentfill}{rgb}{0.050011,0.191979,0.290388}%
\pgfsetfillcolor{currentfill}%
\pgfsetlinewidth{0.000000pt}%
\definecolor{currentstroke}{rgb}{0.000000,0.000000,0.000000}%
\pgfsetstrokecolor{currentstroke}%
\pgfsetdash{}{0pt}%
\pgfpathmoveto{\pgfqpoint{7.341915in}{2.174201in}}%
\pgfpathlineto{\pgfqpoint{7.627388in}{1.631177in}}%
\pgfpathlineto{\pgfqpoint{7.615072in}{2.212212in}}%
\pgfpathlineto{\pgfqpoint{7.341915in}{2.174201in}}%
\pgfpathclose%
\pgfusepath{fill}%
\end{pgfscope}%
\begin{pgfscope}%
\pgfpathrectangle{\pgfqpoint{4.658921in}{0.208778in}}{\pgfqpoint{3.800000in}{3.800000in}}%
\pgfusepath{clip}%
\pgfsetbuttcap%
\pgfsetroundjoin%
\definecolor{currentfill}{rgb}{0.049941,0.191710,0.289982}%
\pgfsetfillcolor{currentfill}%
\pgfsetlinewidth{0.000000pt}%
\definecolor{currentstroke}{rgb}{0.000000,0.000000,0.000000}%
\pgfsetstrokecolor{currentstroke}%
\pgfsetdash{}{0pt}%
\pgfpathmoveto{\pgfqpoint{6.050897in}{1.893599in}}%
\pgfpathlineto{\pgfqpoint{6.006510in}{1.372555in}}%
\pgfpathlineto{\pgfqpoint{6.108398in}{1.480874in}}%
\pgfpathlineto{\pgfqpoint{6.050897in}{1.893599in}}%
\pgfpathclose%
\pgfusepath{fill}%
\end{pgfscope}%
\begin{pgfscope}%
\pgfpathrectangle{\pgfqpoint{4.658921in}{0.208778in}}{\pgfqpoint{3.800000in}{3.800000in}}%
\pgfusepath{clip}%
\pgfsetbuttcap%
\pgfsetroundjoin%
\definecolor{currentfill}{rgb}{0.049941,0.191710,0.289982}%
\pgfsetfillcolor{currentfill}%
\pgfsetlinewidth{0.000000pt}%
\definecolor{currentstroke}{rgb}{0.000000,0.000000,0.000000}%
\pgfsetstrokecolor{currentstroke}%
\pgfsetdash{}{0pt}%
\pgfpathmoveto{\pgfqpoint{7.112146in}{1.480874in}}%
\pgfpathlineto{\pgfqpoint{7.214034in}{1.372555in}}%
\pgfpathlineto{\pgfqpoint{7.169647in}{1.893599in}}%
\pgfpathlineto{\pgfqpoint{7.112146in}{1.480874in}}%
\pgfpathclose%
\pgfusepath{fill}%
\end{pgfscope}%
\begin{pgfscope}%
\pgfpathrectangle{\pgfqpoint{4.658921in}{0.208778in}}{\pgfqpoint{3.800000in}{3.800000in}}%
\pgfusepath{clip}%
\pgfsetbuttcap%
\pgfsetroundjoin%
\definecolor{currentfill}{rgb}{0.078663,0.301965,0.456754}%
\pgfsetfillcolor{currentfill}%
\pgfsetlinewidth{0.000000pt}%
\definecolor{currentstroke}{rgb}{0.000000,0.000000,0.000000}%
\pgfsetstrokecolor{currentstroke}%
\pgfsetdash{}{0pt}%
\pgfpathmoveto{\pgfqpoint{7.113200in}{2.945086in}}%
\pgfpathlineto{\pgfqpoint{7.215560in}{3.134503in}}%
\pgfpathlineto{\pgfqpoint{6.783633in}{2.956025in}}%
\pgfpathlineto{\pgfqpoint{7.113200in}{2.945086in}}%
\pgfpathclose%
\pgfusepath{fill}%
\end{pgfscope}%
\begin{pgfscope}%
\pgfpathrectangle{\pgfqpoint{4.658921in}{0.208778in}}{\pgfqpoint{3.800000in}{3.800000in}}%
\pgfusepath{clip}%
\pgfsetbuttcap%
\pgfsetroundjoin%
\definecolor{currentfill}{rgb}{0.078663,0.301965,0.456754}%
\pgfsetfillcolor{currentfill}%
\pgfsetlinewidth{0.000000pt}%
\definecolor{currentstroke}{rgb}{0.000000,0.000000,0.000000}%
\pgfsetstrokecolor{currentstroke}%
\pgfsetdash{}{0pt}%
\pgfpathmoveto{\pgfqpoint{6.436911in}{2.956025in}}%
\pgfpathlineto{\pgfqpoint{6.004984in}{3.134503in}}%
\pgfpathlineto{\pgfqpoint{6.107344in}{2.945086in}}%
\pgfpathlineto{\pgfqpoint{6.436911in}{2.956025in}}%
\pgfpathclose%
\pgfusepath{fill}%
\end{pgfscope}%
\begin{pgfscope}%
\pgfpathrectangle{\pgfqpoint{4.658921in}{0.208778in}}{\pgfqpoint{3.800000in}{3.800000in}}%
\pgfusepath{clip}%
\pgfsetbuttcap%
\pgfsetroundjoin%
\definecolor{currentfill}{rgb}{0.064954,0.249341,0.377155}%
\pgfsetfillcolor{currentfill}%
\pgfsetlinewidth{0.000000pt}%
\definecolor{currentstroke}{rgb}{0.000000,0.000000,0.000000}%
\pgfsetstrokecolor{currentstroke}%
\pgfsetdash{}{0pt}%
\pgfpathmoveto{\pgfqpoint{5.878629in}{2.174201in}}%
\pgfpathlineto{\pgfqpoint{5.821807in}{2.926052in}}%
\pgfpathlineto{\pgfqpoint{5.605472in}{2.212212in}}%
\pgfpathlineto{\pgfqpoint{5.878629in}{2.174201in}}%
\pgfpathclose%
\pgfusepath{fill}%
\end{pgfscope}%
\begin{pgfscope}%
\pgfpathrectangle{\pgfqpoint{4.658921in}{0.208778in}}{\pgfqpoint{3.800000in}{3.800000in}}%
\pgfusepath{clip}%
\pgfsetbuttcap%
\pgfsetroundjoin%
\definecolor{currentfill}{rgb}{0.064954,0.249341,0.377155}%
\pgfsetfillcolor{currentfill}%
\pgfsetlinewidth{0.000000pt}%
\definecolor{currentstroke}{rgb}{0.000000,0.000000,0.000000}%
\pgfsetstrokecolor{currentstroke}%
\pgfsetdash{}{0pt}%
\pgfpathmoveto{\pgfqpoint{7.615072in}{2.212212in}}%
\pgfpathlineto{\pgfqpoint{7.398737in}{2.926052in}}%
\pgfpathlineto{\pgfqpoint{7.341915in}{2.174201in}}%
\pgfpathlineto{\pgfqpoint{7.615072in}{2.212212in}}%
\pgfpathclose%
\pgfusepath{fill}%
\end{pgfscope}%
\begin{pgfscope}%
\pgfpathrectangle{\pgfqpoint{4.658921in}{0.208778in}}{\pgfqpoint{3.800000in}{3.800000in}}%
\pgfusepath{clip}%
\pgfsetbuttcap%
\pgfsetroundjoin%
\definecolor{currentfill}{rgb}{0.042669,0.163794,0.247755}%
\pgfsetfillcolor{currentfill}%
\pgfsetlinewidth{0.000000pt}%
\definecolor{currentstroke}{rgb}{0.000000,0.000000,0.000000}%
\pgfsetstrokecolor{currentstroke}%
\pgfsetdash{}{0pt}%
\pgfpathmoveto{\pgfqpoint{7.112146in}{1.480874in}}%
\pgfpathlineto{\pgfqpoint{6.804641in}{1.862594in}}%
\pgfpathlineto{\pgfqpoint{6.924933in}{1.319560in}}%
\pgfpathlineto{\pgfqpoint{7.112146in}{1.480874in}}%
\pgfpathclose%
\pgfusepath{fill}%
\end{pgfscope}%
\begin{pgfscope}%
\pgfpathrectangle{\pgfqpoint{4.658921in}{0.208778in}}{\pgfqpoint{3.800000in}{3.800000in}}%
\pgfusepath{clip}%
\pgfsetbuttcap%
\pgfsetroundjoin%
\definecolor{currentfill}{rgb}{0.042669,0.163794,0.247755}%
\pgfsetfillcolor{currentfill}%
\pgfsetlinewidth{0.000000pt}%
\definecolor{currentstroke}{rgb}{0.000000,0.000000,0.000000}%
\pgfsetstrokecolor{currentstroke}%
\pgfsetdash{}{0pt}%
\pgfpathmoveto{\pgfqpoint{6.295611in}{1.319560in}}%
\pgfpathlineto{\pgfqpoint{6.415903in}{1.862594in}}%
\pgfpathlineto{\pgfqpoint{6.108398in}{1.480874in}}%
\pgfpathlineto{\pgfqpoint{6.295611in}{1.319560in}}%
\pgfpathclose%
\pgfusepath{fill}%
\end{pgfscope}%
\begin{pgfscope}%
\pgfpathrectangle{\pgfqpoint{4.658921in}{0.208778in}}{\pgfqpoint{3.800000in}{3.800000in}}%
\pgfusepath{clip}%
\pgfsetbuttcap%
\pgfsetroundjoin%
\definecolor{currentfill}{rgb}{0.068541,0.263111,0.397982}%
\pgfsetfillcolor{currentfill}%
\pgfsetlinewidth{0.000000pt}%
\definecolor{currentstroke}{rgb}{0.000000,0.000000,0.000000}%
\pgfsetstrokecolor{currentstroke}%
\pgfsetdash{}{0pt}%
\pgfpathmoveto{\pgfqpoint{6.107344in}{2.945086in}}%
\pgfpathlineto{\pgfqpoint{5.821807in}{2.926052in}}%
\pgfpathlineto{\pgfqpoint{6.244171in}{2.707660in}}%
\pgfpathlineto{\pgfqpoint{6.107344in}{2.945086in}}%
\pgfpathclose%
\pgfusepath{fill}%
\end{pgfscope}%
\begin{pgfscope}%
\pgfpathrectangle{\pgfqpoint{4.658921in}{0.208778in}}{\pgfqpoint{3.800000in}{3.800000in}}%
\pgfusepath{clip}%
\pgfsetbuttcap%
\pgfsetroundjoin%
\definecolor{currentfill}{rgb}{0.068541,0.263111,0.397982}%
\pgfsetfillcolor{currentfill}%
\pgfsetlinewidth{0.000000pt}%
\definecolor{currentstroke}{rgb}{0.000000,0.000000,0.000000}%
\pgfsetstrokecolor{currentstroke}%
\pgfsetdash{}{0pt}%
\pgfpathmoveto{\pgfqpoint{6.976373in}{2.707660in}}%
\pgfpathlineto{\pgfqpoint{7.398737in}{2.926052in}}%
\pgfpathlineto{\pgfqpoint{7.113200in}{2.945086in}}%
\pgfpathlineto{\pgfqpoint{6.976373in}{2.707660in}}%
\pgfpathclose%
\pgfusepath{fill}%
\end{pgfscope}%
\begin{pgfscope}%
\pgfpathrectangle{\pgfqpoint{4.658921in}{0.208778in}}{\pgfqpoint{3.800000in}{3.800000in}}%
\pgfusepath{clip}%
\pgfsetbuttcap%
\pgfsetroundjoin%
\definecolor{currentfill}{rgb}{0.047555,0.182548,0.276123}%
\pgfsetfillcolor{currentfill}%
\pgfsetlinewidth{0.000000pt}%
\definecolor{currentstroke}{rgb}{0.000000,0.000000,0.000000}%
\pgfsetstrokecolor{currentstroke}%
\pgfsetdash{}{0pt}%
\pgfpathmoveto{\pgfqpoint{6.295611in}{1.319560in}}%
\pgfpathlineto{\pgfqpoint{6.437287in}{1.441955in}}%
\pgfpathlineto{\pgfqpoint{6.415903in}{1.862594in}}%
\pgfpathlineto{\pgfqpoint{6.295611in}{1.319560in}}%
\pgfpathclose%
\pgfusepath{fill}%
\end{pgfscope}%
\begin{pgfscope}%
\pgfpathrectangle{\pgfqpoint{4.658921in}{0.208778in}}{\pgfqpoint{3.800000in}{3.800000in}}%
\pgfusepath{clip}%
\pgfsetbuttcap%
\pgfsetroundjoin%
\definecolor{currentfill}{rgb}{0.047555,0.182548,0.276123}%
\pgfsetfillcolor{currentfill}%
\pgfsetlinewidth{0.000000pt}%
\definecolor{currentstroke}{rgb}{0.000000,0.000000,0.000000}%
\pgfsetstrokecolor{currentstroke}%
\pgfsetdash{}{0pt}%
\pgfpathmoveto{\pgfqpoint{6.804641in}{1.862594in}}%
\pgfpathlineto{\pgfqpoint{6.783257in}{1.441955in}}%
\pgfpathlineto{\pgfqpoint{6.924933in}{1.319560in}}%
\pgfpathlineto{\pgfqpoint{6.804641in}{1.862594in}}%
\pgfpathclose%
\pgfusepath{fill}%
\end{pgfscope}%
\begin{pgfscope}%
\pgfpathrectangle{\pgfqpoint{4.658921in}{0.208778in}}{\pgfqpoint{3.800000in}{3.800000in}}%
\pgfusepath{clip}%
\pgfsetbuttcap%
\pgfsetroundjoin%
\definecolor{currentfill}{rgb}{0.046101,0.176968,0.267683}%
\pgfsetfillcolor{currentfill}%
\pgfsetlinewidth{0.000000pt}%
\definecolor{currentstroke}{rgb}{0.000000,0.000000,0.000000}%
\pgfsetstrokecolor{currentstroke}%
\pgfsetdash{}{0pt}%
\pgfpathmoveto{\pgfqpoint{6.610272in}{1.300375in}}%
\pgfpathlineto{\pgfqpoint{6.415903in}{1.862594in}}%
\pgfpathlineto{\pgfqpoint{6.437287in}{1.441955in}}%
\pgfpathlineto{\pgfqpoint{6.610272in}{1.300375in}}%
\pgfpathclose%
\pgfusepath{fill}%
\end{pgfscope}%
\begin{pgfscope}%
\pgfpathrectangle{\pgfqpoint{4.658921in}{0.208778in}}{\pgfqpoint{3.800000in}{3.800000in}}%
\pgfusepath{clip}%
\pgfsetbuttcap%
\pgfsetroundjoin%
\definecolor{currentfill}{rgb}{0.046101,0.176968,0.267683}%
\pgfsetfillcolor{currentfill}%
\pgfsetlinewidth{0.000000pt}%
\definecolor{currentstroke}{rgb}{0.000000,0.000000,0.000000}%
\pgfsetstrokecolor{currentstroke}%
\pgfsetdash{}{0pt}%
\pgfpathmoveto{\pgfqpoint{6.783257in}{1.441955in}}%
\pgfpathlineto{\pgfqpoint{6.804641in}{1.862594in}}%
\pgfpathlineto{\pgfqpoint{6.610272in}{1.300375in}}%
\pgfpathlineto{\pgfqpoint{6.783257in}{1.441955in}}%
\pgfpathclose%
\pgfusepath{fill}%
\end{pgfscope}%
\begin{pgfscope}%
\pgfpathrectangle{\pgfqpoint{4.658921in}{0.208778in}}{\pgfqpoint{3.800000in}{3.800000in}}%
\pgfusepath{clip}%
\pgfsetbuttcap%
\pgfsetroundjoin%
\definecolor{currentfill}{rgb}{0.051850,0.199036,0.301063}%
\pgfsetfillcolor{currentfill}%
\pgfsetlinewidth{0.000000pt}%
\definecolor{currentstroke}{rgb}{0.000000,0.000000,0.000000}%
\pgfsetstrokecolor{currentstroke}%
\pgfsetdash{}{0pt}%
\pgfpathmoveto{\pgfqpoint{5.823361in}{1.548607in}}%
\pgfpathlineto{\pgfqpoint{6.050897in}{1.893599in}}%
\pgfpathlineto{\pgfqpoint{5.878629in}{2.174201in}}%
\pgfpathlineto{\pgfqpoint{5.823361in}{1.548607in}}%
\pgfpathclose%
\pgfusepath{fill}%
\end{pgfscope}%
\begin{pgfscope}%
\pgfpathrectangle{\pgfqpoint{4.658921in}{0.208778in}}{\pgfqpoint{3.800000in}{3.800000in}}%
\pgfusepath{clip}%
\pgfsetbuttcap%
\pgfsetroundjoin%
\definecolor{currentfill}{rgb}{0.051850,0.199036,0.301063}%
\pgfsetfillcolor{currentfill}%
\pgfsetlinewidth{0.000000pt}%
\definecolor{currentstroke}{rgb}{0.000000,0.000000,0.000000}%
\pgfsetstrokecolor{currentstroke}%
\pgfsetdash{}{0pt}%
\pgfpathmoveto{\pgfqpoint{7.341915in}{2.174201in}}%
\pgfpathlineto{\pgfqpoint{7.169647in}{1.893599in}}%
\pgfpathlineto{\pgfqpoint{7.397183in}{1.548607in}}%
\pgfpathlineto{\pgfqpoint{7.341915in}{2.174201in}}%
\pgfpathclose%
\pgfusepath{fill}%
\end{pgfscope}%
\begin{pgfscope}%
\pgfpathrectangle{\pgfqpoint{4.658921in}{0.208778in}}{\pgfqpoint{3.800000in}{3.800000in}}%
\pgfusepath{clip}%
\pgfsetbuttcap%
\pgfsetroundjoin%
\definecolor{currentfill}{rgb}{0.064759,0.248590,0.376018}%
\pgfsetfillcolor{currentfill}%
\pgfsetlinewidth{0.000000pt}%
\definecolor{currentstroke}{rgb}{0.000000,0.000000,0.000000}%
\pgfsetstrokecolor{currentstroke}%
\pgfsetdash{}{0pt}%
\pgfpathmoveto{\pgfqpoint{5.878629in}{2.174201in}}%
\pgfpathlineto{\pgfqpoint{6.244171in}{2.707660in}}%
\pgfpathlineto{\pgfqpoint{5.821807in}{2.926052in}}%
\pgfpathlineto{\pgfqpoint{5.878629in}{2.174201in}}%
\pgfpathclose%
\pgfusepath{fill}%
\end{pgfscope}%
\begin{pgfscope}%
\pgfpathrectangle{\pgfqpoint{4.658921in}{0.208778in}}{\pgfqpoint{3.800000in}{3.800000in}}%
\pgfusepath{clip}%
\pgfsetbuttcap%
\pgfsetroundjoin%
\definecolor{currentfill}{rgb}{0.064759,0.248590,0.376018}%
\pgfsetfillcolor{currentfill}%
\pgfsetlinewidth{0.000000pt}%
\definecolor{currentstroke}{rgb}{0.000000,0.000000,0.000000}%
\pgfsetstrokecolor{currentstroke}%
\pgfsetdash{}{0pt}%
\pgfpathmoveto{\pgfqpoint{7.398737in}{2.926052in}}%
\pgfpathlineto{\pgfqpoint{6.976373in}{2.707660in}}%
\pgfpathlineto{\pgfqpoint{7.341915in}{2.174201in}}%
\pgfpathlineto{\pgfqpoint{7.398737in}{2.926052in}}%
\pgfpathclose%
\pgfusepath{fill}%
\end{pgfscope}%
\begin{pgfscope}%
\pgfpathrectangle{\pgfqpoint{4.658921in}{0.208778in}}{\pgfqpoint{3.800000in}{3.800000in}}%
\pgfusepath{clip}%
\pgfsetbuttcap%
\pgfsetroundjoin%
\definecolor{currentfill}{rgb}{0.071694,0.275212,0.416288}%
\pgfsetfillcolor{currentfill}%
\pgfsetlinewidth{0.000000pt}%
\definecolor{currentstroke}{rgb}{0.000000,0.000000,0.000000}%
\pgfsetstrokecolor{currentstroke}%
\pgfsetdash{}{0pt}%
\pgfpathmoveto{\pgfqpoint{6.244171in}{2.707660in}}%
\pgfpathlineto{\pgfqpoint{6.436911in}{2.956025in}}%
\pgfpathlineto{\pgfqpoint{6.107344in}{2.945086in}}%
\pgfpathlineto{\pgfqpoint{6.244171in}{2.707660in}}%
\pgfpathclose%
\pgfusepath{fill}%
\end{pgfscope}%
\begin{pgfscope}%
\pgfpathrectangle{\pgfqpoint{4.658921in}{0.208778in}}{\pgfqpoint{3.800000in}{3.800000in}}%
\pgfusepath{clip}%
\pgfsetbuttcap%
\pgfsetroundjoin%
\definecolor{currentfill}{rgb}{0.071694,0.275212,0.416288}%
\pgfsetfillcolor{currentfill}%
\pgfsetlinewidth{0.000000pt}%
\definecolor{currentstroke}{rgb}{0.000000,0.000000,0.000000}%
\pgfsetstrokecolor{currentstroke}%
\pgfsetdash{}{0pt}%
\pgfpathmoveto{\pgfqpoint{7.113200in}{2.945086in}}%
\pgfpathlineto{\pgfqpoint{6.783633in}{2.956025in}}%
\pgfpathlineto{\pgfqpoint{6.976373in}{2.707660in}}%
\pgfpathlineto{\pgfqpoint{7.113200in}{2.945086in}}%
\pgfpathclose%
\pgfusepath{fill}%
\end{pgfscope}%
\begin{pgfscope}%
\pgfpathrectangle{\pgfqpoint{4.658921in}{0.208778in}}{\pgfqpoint{3.800000in}{3.800000in}}%
\pgfusepath{clip}%
\pgfsetbuttcap%
\pgfsetroundjoin%
\definecolor{currentfill}{rgb}{0.071636,0.274990,0.415951}%
\pgfsetfillcolor{currentfill}%
\pgfsetlinewidth{0.000000pt}%
\definecolor{currentstroke}{rgb}{0.000000,0.000000,0.000000}%
\pgfsetstrokecolor{currentstroke}%
\pgfsetdash{}{0pt}%
\pgfpathmoveto{\pgfqpoint{6.610272in}{2.709310in}}%
\pgfpathlineto{\pgfqpoint{6.783633in}{2.956025in}}%
\pgfpathlineto{\pgfqpoint{6.436911in}{2.956025in}}%
\pgfpathlineto{\pgfqpoint{6.610272in}{2.709310in}}%
\pgfpathclose%
\pgfusepath{fill}%
\end{pgfscope}%
\begin{pgfscope}%
\pgfpathrectangle{\pgfqpoint{4.658921in}{0.208778in}}{\pgfqpoint{3.800000in}{3.800000in}}%
\pgfusepath{clip}%
\pgfsetbuttcap%
\pgfsetroundjoin%
\definecolor{currentfill}{rgb}{0.045820,0.175891,0.266053}%
\pgfsetfillcolor{currentfill}%
\pgfsetlinewidth{0.000000pt}%
\definecolor{currentstroke}{rgb}{0.000000,0.000000,0.000000}%
\pgfsetstrokecolor{currentstroke}%
\pgfsetdash{}{0pt}%
\pgfpathmoveto{\pgfqpoint{6.108398in}{1.480874in}}%
\pgfpathlineto{\pgfqpoint{6.221375in}{2.145805in}}%
\pgfpathlineto{\pgfqpoint{6.050897in}{1.893599in}}%
\pgfpathlineto{\pgfqpoint{6.108398in}{1.480874in}}%
\pgfpathclose%
\pgfusepath{fill}%
\end{pgfscope}%
\begin{pgfscope}%
\pgfpathrectangle{\pgfqpoint{4.658921in}{0.208778in}}{\pgfqpoint{3.800000in}{3.800000in}}%
\pgfusepath{clip}%
\pgfsetbuttcap%
\pgfsetroundjoin%
\definecolor{currentfill}{rgb}{0.045820,0.175891,0.266053}%
\pgfsetfillcolor{currentfill}%
\pgfsetlinewidth{0.000000pt}%
\definecolor{currentstroke}{rgb}{0.000000,0.000000,0.000000}%
\pgfsetstrokecolor{currentstroke}%
\pgfsetdash{}{0pt}%
\pgfpathmoveto{\pgfqpoint{7.169647in}{1.893599in}}%
\pgfpathlineto{\pgfqpoint{6.999169in}{2.145805in}}%
\pgfpathlineto{\pgfqpoint{7.112146in}{1.480874in}}%
\pgfpathlineto{\pgfqpoint{7.169647in}{1.893599in}}%
\pgfpathclose%
\pgfusepath{fill}%
\end{pgfscope}%
\begin{pgfscope}%
\pgfpathrectangle{\pgfqpoint{4.658921in}{0.208778in}}{\pgfqpoint{3.800000in}{3.800000in}}%
\pgfusepath{clip}%
\pgfsetbuttcap%
\pgfsetroundjoin%
\definecolor{currentfill}{rgb}{0.046814,0.179706,0.271825}%
\pgfsetfillcolor{currentfill}%
\pgfsetlinewidth{0.000000pt}%
\definecolor{currentstroke}{rgb}{0.000000,0.000000,0.000000}%
\pgfsetstrokecolor{currentstroke}%
\pgfsetdash{}{0pt}%
\pgfpathmoveto{\pgfqpoint{6.415903in}{1.862594in}}%
\pgfpathlineto{\pgfqpoint{6.610272in}{1.300375in}}%
\pgfpathlineto{\pgfqpoint{6.610272in}{2.135109in}}%
\pgfpathlineto{\pgfqpoint{6.415903in}{1.862594in}}%
\pgfpathclose%
\pgfusepath{fill}%
\end{pgfscope}%
\begin{pgfscope}%
\pgfpathrectangle{\pgfqpoint{4.658921in}{0.208778in}}{\pgfqpoint{3.800000in}{3.800000in}}%
\pgfusepath{clip}%
\pgfsetbuttcap%
\pgfsetroundjoin%
\definecolor{currentfill}{rgb}{0.046814,0.179706,0.271825}%
\pgfsetfillcolor{currentfill}%
\pgfsetlinewidth{0.000000pt}%
\definecolor{currentstroke}{rgb}{0.000000,0.000000,0.000000}%
\pgfsetstrokecolor{currentstroke}%
\pgfsetdash{}{0pt}%
\pgfpathmoveto{\pgfqpoint{6.610272in}{2.135109in}}%
\pgfpathlineto{\pgfqpoint{6.610272in}{1.300375in}}%
\pgfpathlineto{\pgfqpoint{6.804641in}{1.862594in}}%
\pgfpathlineto{\pgfqpoint{6.610272in}{2.135109in}}%
\pgfpathclose%
\pgfusepath{fill}%
\end{pgfscope}%
\begin{pgfscope}%
\pgfpathrectangle{\pgfqpoint{4.658921in}{0.208778in}}{\pgfqpoint{3.800000in}{3.800000in}}%
\pgfusepath{clip}%
\pgfsetbuttcap%
\pgfsetroundjoin%
\definecolor{currentfill}{rgb}{0.069261,0.265872,0.402159}%
\pgfsetfillcolor{currentfill}%
\pgfsetlinewidth{0.000000pt}%
\definecolor{currentstroke}{rgb}{0.000000,0.000000,0.000000}%
\pgfsetstrokecolor{currentstroke}%
\pgfsetdash{}{0pt}%
\pgfpathmoveto{\pgfqpoint{6.610272in}{2.709310in}}%
\pgfpathlineto{\pgfqpoint{6.436911in}{2.956025in}}%
\pgfpathlineto{\pgfqpoint{6.244171in}{2.707660in}}%
\pgfpathlineto{\pgfqpoint{6.610272in}{2.709310in}}%
\pgfpathclose%
\pgfusepath{fill}%
\end{pgfscope}%
\begin{pgfscope}%
\pgfpathrectangle{\pgfqpoint{4.658921in}{0.208778in}}{\pgfqpoint{3.800000in}{3.800000in}}%
\pgfusepath{clip}%
\pgfsetbuttcap%
\pgfsetroundjoin%
\definecolor{currentfill}{rgb}{0.069261,0.265872,0.402159}%
\pgfsetfillcolor{currentfill}%
\pgfsetlinewidth{0.000000pt}%
\definecolor{currentstroke}{rgb}{0.000000,0.000000,0.000000}%
\pgfsetstrokecolor{currentstroke}%
\pgfsetdash{}{0pt}%
\pgfpathmoveto{\pgfqpoint{6.976373in}{2.707660in}}%
\pgfpathlineto{\pgfqpoint{6.783633in}{2.956025in}}%
\pgfpathlineto{\pgfqpoint{6.610272in}{2.709310in}}%
\pgfpathlineto{\pgfqpoint{6.976373in}{2.707660in}}%
\pgfpathclose%
\pgfusepath{fill}%
\end{pgfscope}%
\begin{pgfscope}%
\pgfpathrectangle{\pgfqpoint{4.658921in}{0.208778in}}{\pgfqpoint{3.800000in}{3.800000in}}%
\pgfusepath{clip}%
\pgfsetbuttcap%
\pgfsetroundjoin%
\definecolor{currentfill}{rgb}{0.049465,0.189883,0.287218}%
\pgfsetfillcolor{currentfill}%
\pgfsetlinewidth{0.000000pt}%
\definecolor{currentstroke}{rgb}{0.000000,0.000000,0.000000}%
\pgfsetstrokecolor{currentstroke}%
\pgfsetdash{}{0pt}%
\pgfpathmoveto{\pgfqpoint{6.108398in}{1.480874in}}%
\pgfpathlineto{\pgfqpoint{6.415903in}{1.862594in}}%
\pgfpathlineto{\pgfqpoint{6.221375in}{2.145805in}}%
\pgfpathlineto{\pgfqpoint{6.108398in}{1.480874in}}%
\pgfpathclose%
\pgfusepath{fill}%
\end{pgfscope}%
\begin{pgfscope}%
\pgfpathrectangle{\pgfqpoint{4.658921in}{0.208778in}}{\pgfqpoint{3.800000in}{3.800000in}}%
\pgfusepath{clip}%
\pgfsetbuttcap%
\pgfsetroundjoin%
\definecolor{currentfill}{rgb}{0.049465,0.189883,0.287218}%
\pgfsetfillcolor{currentfill}%
\pgfsetlinewidth{0.000000pt}%
\definecolor{currentstroke}{rgb}{0.000000,0.000000,0.000000}%
\pgfsetstrokecolor{currentstroke}%
\pgfsetdash{}{0pt}%
\pgfpathmoveto{\pgfqpoint{6.999169in}{2.145805in}}%
\pgfpathlineto{\pgfqpoint{6.804641in}{1.862594in}}%
\pgfpathlineto{\pgfqpoint{7.112146in}{1.480874in}}%
\pgfpathlineto{\pgfqpoint{6.999169in}{2.145805in}}%
\pgfpathclose%
\pgfusepath{fill}%
\end{pgfscope}%
\begin{pgfscope}%
\pgfpathrectangle{\pgfqpoint{4.658921in}{0.208778in}}{\pgfqpoint{3.800000in}{3.800000in}}%
\pgfusepath{clip}%
\pgfsetbuttcap%
\pgfsetroundjoin%
\definecolor{currentfill}{rgb}{0.061576,0.236373,0.357539}%
\pgfsetfillcolor{currentfill}%
\pgfsetlinewidth{0.000000pt}%
\definecolor{currentstroke}{rgb}{0.000000,0.000000,0.000000}%
\pgfsetstrokecolor{currentstroke}%
\pgfsetdash{}{0pt}%
\pgfpathmoveto{\pgfqpoint{6.221375in}{2.145805in}}%
\pgfpathlineto{\pgfqpoint{6.244171in}{2.707660in}}%
\pgfpathlineto{\pgfqpoint{5.878629in}{2.174201in}}%
\pgfpathlineto{\pgfqpoint{6.221375in}{2.145805in}}%
\pgfpathclose%
\pgfusepath{fill}%
\end{pgfscope}%
\begin{pgfscope}%
\pgfpathrectangle{\pgfqpoint{4.658921in}{0.208778in}}{\pgfqpoint{3.800000in}{3.800000in}}%
\pgfusepath{clip}%
\pgfsetbuttcap%
\pgfsetroundjoin%
\definecolor{currentfill}{rgb}{0.061576,0.236373,0.357539}%
\pgfsetfillcolor{currentfill}%
\pgfsetlinewidth{0.000000pt}%
\definecolor{currentstroke}{rgb}{0.000000,0.000000,0.000000}%
\pgfsetstrokecolor{currentstroke}%
\pgfsetdash{}{0pt}%
\pgfpathmoveto{\pgfqpoint{7.341915in}{2.174201in}}%
\pgfpathlineto{\pgfqpoint{6.976373in}{2.707660in}}%
\pgfpathlineto{\pgfqpoint{6.999169in}{2.145805in}}%
\pgfpathlineto{\pgfqpoint{7.341915in}{2.174201in}}%
\pgfpathclose%
\pgfusepath{fill}%
\end{pgfscope}%
\begin{pgfscope}%
\pgfpathrectangle{\pgfqpoint{4.658921in}{0.208778in}}{\pgfqpoint{3.800000in}{3.800000in}}%
\pgfusepath{clip}%
\pgfsetbuttcap%
\pgfsetroundjoin%
\definecolor{currentfill}{rgb}{0.053541,0.205528,0.310883}%
\pgfsetfillcolor{currentfill}%
\pgfsetlinewidth{0.000000pt}%
\definecolor{currentstroke}{rgb}{0.000000,0.000000,0.000000}%
\pgfsetstrokecolor{currentstroke}%
\pgfsetdash{}{0pt}%
\pgfpathmoveto{\pgfqpoint{5.878629in}{2.174201in}}%
\pgfpathlineto{\pgfqpoint{6.050897in}{1.893599in}}%
\pgfpathlineto{\pgfqpoint{6.221375in}{2.145805in}}%
\pgfpathlineto{\pgfqpoint{5.878629in}{2.174201in}}%
\pgfpathclose%
\pgfusepath{fill}%
\end{pgfscope}%
\begin{pgfscope}%
\pgfpathrectangle{\pgfqpoint{4.658921in}{0.208778in}}{\pgfqpoint{3.800000in}{3.800000in}}%
\pgfusepath{clip}%
\pgfsetbuttcap%
\pgfsetroundjoin%
\definecolor{currentfill}{rgb}{0.053541,0.205528,0.310883}%
\pgfsetfillcolor{currentfill}%
\pgfsetlinewidth{0.000000pt}%
\definecolor{currentstroke}{rgb}{0.000000,0.000000,0.000000}%
\pgfsetstrokecolor{currentstroke}%
\pgfsetdash{}{0pt}%
\pgfpathmoveto{\pgfqpoint{6.999169in}{2.145805in}}%
\pgfpathlineto{\pgfqpoint{7.169647in}{1.893599in}}%
\pgfpathlineto{\pgfqpoint{7.341915in}{2.174201in}}%
\pgfpathlineto{\pgfqpoint{6.999169in}{2.145805in}}%
\pgfpathclose%
\pgfusepath{fill}%
\end{pgfscope}%
\begin{pgfscope}%
\pgfpathrectangle{\pgfqpoint{4.658921in}{0.208778in}}{\pgfqpoint{3.800000in}{3.800000in}}%
\pgfusepath{clip}%
\pgfsetbuttcap%
\pgfsetroundjoin%
\definecolor{currentfill}{rgb}{0.060634,0.232757,0.352069}%
\pgfsetfillcolor{currentfill}%
\pgfsetlinewidth{0.000000pt}%
\definecolor{currentstroke}{rgb}{0.000000,0.000000,0.000000}%
\pgfsetstrokecolor{currentstroke}%
\pgfsetdash{}{0pt}%
\pgfpathmoveto{\pgfqpoint{6.244171in}{2.707660in}}%
\pgfpathlineto{\pgfqpoint{6.221375in}{2.145805in}}%
\pgfpathlineto{\pgfqpoint{6.610272in}{2.709310in}}%
\pgfpathlineto{\pgfqpoint{6.244171in}{2.707660in}}%
\pgfpathclose%
\pgfusepath{fill}%
\end{pgfscope}%
\begin{pgfscope}%
\pgfpathrectangle{\pgfqpoint{4.658921in}{0.208778in}}{\pgfqpoint{3.800000in}{3.800000in}}%
\pgfusepath{clip}%
\pgfsetbuttcap%
\pgfsetroundjoin%
\definecolor{currentfill}{rgb}{0.060634,0.232757,0.352069}%
\pgfsetfillcolor{currentfill}%
\pgfsetlinewidth{0.000000pt}%
\definecolor{currentstroke}{rgb}{0.000000,0.000000,0.000000}%
\pgfsetstrokecolor{currentstroke}%
\pgfsetdash{}{0pt}%
\pgfpathmoveto{\pgfqpoint{6.610272in}{2.709310in}}%
\pgfpathlineto{\pgfqpoint{6.999169in}{2.145805in}}%
\pgfpathlineto{\pgfqpoint{6.976373in}{2.707660in}}%
\pgfpathlineto{\pgfqpoint{6.610272in}{2.709310in}}%
\pgfpathclose%
\pgfusepath{fill}%
\end{pgfscope}%
\begin{pgfscope}%
\pgfpathrectangle{\pgfqpoint{4.658921in}{0.208778in}}{\pgfqpoint{3.800000in}{3.800000in}}%
\pgfusepath{clip}%
\pgfsetbuttcap%
\pgfsetroundjoin%
\definecolor{currentfill}{rgb}{0.060773,0.233289,0.352874}%
\pgfsetfillcolor{currentfill}%
\pgfsetlinewidth{0.000000pt}%
\definecolor{currentstroke}{rgb}{0.000000,0.000000,0.000000}%
\pgfsetstrokecolor{currentstroke}%
\pgfsetdash{}{0pt}%
\pgfpathmoveto{\pgfqpoint{6.610272in}{2.135109in}}%
\pgfpathlineto{\pgfqpoint{6.610272in}{2.709310in}}%
\pgfpathlineto{\pgfqpoint{6.221375in}{2.145805in}}%
\pgfpathlineto{\pgfqpoint{6.610272in}{2.135109in}}%
\pgfpathclose%
\pgfusepath{fill}%
\end{pgfscope}%
\begin{pgfscope}%
\pgfpathrectangle{\pgfqpoint{4.658921in}{0.208778in}}{\pgfqpoint{3.800000in}{3.800000in}}%
\pgfusepath{clip}%
\pgfsetbuttcap%
\pgfsetroundjoin%
\definecolor{currentfill}{rgb}{0.060773,0.233289,0.352874}%
\pgfsetfillcolor{currentfill}%
\pgfsetlinewidth{0.000000pt}%
\definecolor{currentstroke}{rgb}{0.000000,0.000000,0.000000}%
\pgfsetstrokecolor{currentstroke}%
\pgfsetdash{}{0pt}%
\pgfpathmoveto{\pgfqpoint{6.999169in}{2.145805in}}%
\pgfpathlineto{\pgfqpoint{6.610272in}{2.709310in}}%
\pgfpathlineto{\pgfqpoint{6.610272in}{2.135109in}}%
\pgfpathlineto{\pgfqpoint{6.999169in}{2.145805in}}%
\pgfpathclose%
\pgfusepath{fill}%
\end{pgfscope}%
\begin{pgfscope}%
\pgfpathrectangle{\pgfqpoint{4.658921in}{0.208778in}}{\pgfqpoint{3.800000in}{3.800000in}}%
\pgfusepath{clip}%
\pgfsetbuttcap%
\pgfsetroundjoin%
\definecolor{currentfill}{rgb}{0.052607,0.201942,0.305459}%
\pgfsetfillcolor{currentfill}%
\pgfsetlinewidth{0.000000pt}%
\definecolor{currentstroke}{rgb}{0.000000,0.000000,0.000000}%
\pgfsetstrokecolor{currentstroke}%
\pgfsetdash{}{0pt}%
\pgfpathmoveto{\pgfqpoint{6.221375in}{2.145805in}}%
\pgfpathlineto{\pgfqpoint{6.415903in}{1.862594in}}%
\pgfpathlineto{\pgfqpoint{6.610272in}{2.135109in}}%
\pgfpathlineto{\pgfqpoint{6.221375in}{2.145805in}}%
\pgfpathclose%
\pgfusepath{fill}%
\end{pgfscope}%
\begin{pgfscope}%
\pgfpathrectangle{\pgfqpoint{4.658921in}{0.208778in}}{\pgfqpoint{3.800000in}{3.800000in}}%
\pgfusepath{clip}%
\pgfsetbuttcap%
\pgfsetroundjoin%
\definecolor{currentfill}{rgb}{0.052607,0.201942,0.305459}%
\pgfsetfillcolor{currentfill}%
\pgfsetlinewidth{0.000000pt}%
\definecolor{currentstroke}{rgb}{0.000000,0.000000,0.000000}%
\pgfsetstrokecolor{currentstroke}%
\pgfsetdash{}{0pt}%
\pgfpathmoveto{\pgfqpoint{6.610272in}{2.135109in}}%
\pgfpathlineto{\pgfqpoint{6.804641in}{1.862594in}}%
\pgfpathlineto{\pgfqpoint{6.999169in}{2.145805in}}%
\pgfpathlineto{\pgfqpoint{6.610272in}{2.135109in}}%
\pgfpathclose%
\pgfusepath{fill}%
\end{pgfscope}%
\begin{pgfscope}%
\pgfpathrectangle{\pgfqpoint{4.658921in}{0.208778in}}{\pgfqpoint{3.800000in}{3.800000in}}%
\pgfusepath{clip}%
\pgfsetbuttcap%
\pgfsetroundjoin%
\definecolor{currentfill}{rgb}{0.839216,0.152941,0.156863}%
\pgfsetfillcolor{currentfill}%
\pgfsetfillopacity{0.300000}%
\pgfsetlinewidth{1.003750pt}%
\definecolor{currentstroke}{rgb}{0.839216,0.152941,0.156863}%
\pgfsetstrokecolor{currentstroke}%
\pgfsetstrokeopacity{0.300000}%
\pgfsetdash{}{0pt}%
\pgfpathmoveto{\pgfqpoint{5.785792in}{1.331622in}}%
\pgfpathcurveto{\pgfqpoint{5.795879in}{1.331622in}}{\pgfqpoint{5.805555in}{1.335630in}}{\pgfqpoint{5.812688in}{1.342762in}}%
\pgfpathcurveto{\pgfqpoint{5.819820in}{1.349895in}}{\pgfqpoint{5.823828in}{1.359571in}}{\pgfqpoint{5.823828in}{1.369658in}}%
\pgfpathcurveto{\pgfqpoint{5.823828in}{1.379745in}}{\pgfqpoint{5.819820in}{1.389421in}}{\pgfqpoint{5.812688in}{1.396554in}}%
\pgfpathcurveto{\pgfqpoint{5.805555in}{1.403687in}}{\pgfqpoint{5.795879in}{1.407694in}}{\pgfqpoint{5.785792in}{1.407694in}}%
\pgfpathcurveto{\pgfqpoint{5.775705in}{1.407694in}}{\pgfqpoint{5.766029in}{1.403687in}}{\pgfqpoint{5.758896in}{1.396554in}}%
\pgfpathcurveto{\pgfqpoint{5.751763in}{1.389421in}}{\pgfqpoint{5.747756in}{1.379745in}}{\pgfqpoint{5.747756in}{1.369658in}}%
\pgfpathcurveto{\pgfqpoint{5.747756in}{1.359571in}}{\pgfqpoint{5.751763in}{1.349895in}}{\pgfqpoint{5.758896in}{1.342762in}}%
\pgfpathcurveto{\pgfqpoint{5.766029in}{1.335630in}}{\pgfqpoint{5.775705in}{1.331622in}}{\pgfqpoint{5.785792in}{1.331622in}}%
\pgfpathlineto{\pgfqpoint{5.785792in}{1.331622in}}%
\pgfpathclose%
\pgfusepath{stroke,fill}%
\end{pgfscope}%
\begin{pgfscope}%
\pgfpathrectangle{\pgfqpoint{4.658921in}{0.208778in}}{\pgfqpoint{3.800000in}{3.800000in}}%
\pgfusepath{clip}%
\pgfsetbuttcap%
\pgfsetroundjoin%
\definecolor{currentfill}{rgb}{0.839216,0.152941,0.156863}%
\pgfsetfillcolor{currentfill}%
\pgfsetfillopacity{0.383610}%
\pgfsetlinewidth{1.003750pt}%
\definecolor{currentstroke}{rgb}{0.839216,0.152941,0.156863}%
\pgfsetstrokecolor{currentstroke}%
\pgfsetstrokeopacity{0.383610}%
\pgfsetdash{}{0pt}%
\pgfpathmoveto{\pgfqpoint{5.758197in}{1.421460in}}%
\pgfpathcurveto{\pgfqpoint{5.768284in}{1.421460in}}{\pgfqpoint{5.777960in}{1.425468in}}{\pgfqpoint{5.785093in}{1.432601in}}%
\pgfpathcurveto{\pgfqpoint{5.792225in}{1.439734in}}{\pgfqpoint{5.796233in}{1.449409in}}{\pgfqpoint{5.796233in}{1.459497in}}%
\pgfpathcurveto{\pgfqpoint{5.796233in}{1.469584in}}{\pgfqpoint{5.792225in}{1.479259in}}{\pgfqpoint{5.785093in}{1.486392in}}%
\pgfpathcurveto{\pgfqpoint{5.777960in}{1.493525in}}{\pgfqpoint{5.768284in}{1.497533in}}{\pgfqpoint{5.758197in}{1.497533in}}%
\pgfpathcurveto{\pgfqpoint{5.748110in}{1.497533in}}{\pgfqpoint{5.738434in}{1.493525in}}{\pgfqpoint{5.731301in}{1.486392in}}%
\pgfpathcurveto{\pgfqpoint{5.724168in}{1.479259in}}{\pgfqpoint{5.720161in}{1.469584in}}{\pgfqpoint{5.720161in}{1.459497in}}%
\pgfpathcurveto{\pgfqpoint{5.720161in}{1.449409in}}{\pgfqpoint{5.724168in}{1.439734in}}{\pgfqpoint{5.731301in}{1.432601in}}%
\pgfpathcurveto{\pgfqpoint{5.738434in}{1.425468in}}{\pgfqpoint{5.748110in}{1.421460in}}{\pgfqpoint{5.758197in}{1.421460in}}%
\pgfpathlineto{\pgfqpoint{5.758197in}{1.421460in}}%
\pgfpathclose%
\pgfusepath{stroke,fill}%
\end{pgfscope}%
\begin{pgfscope}%
\pgfpathrectangle{\pgfqpoint{4.658921in}{0.208778in}}{\pgfqpoint{3.800000in}{3.800000in}}%
\pgfusepath{clip}%
\pgfsetbuttcap%
\pgfsetroundjoin%
\definecolor{currentfill}{rgb}{0.839216,0.152941,0.156863}%
\pgfsetfillcolor{currentfill}%
\pgfsetfillopacity{0.457533}%
\pgfsetlinewidth{1.003750pt}%
\definecolor{currentstroke}{rgb}{0.839216,0.152941,0.156863}%
\pgfsetstrokecolor{currentstroke}%
\pgfsetstrokeopacity{0.457533}%
\pgfsetdash{}{0pt}%
\pgfpathmoveto{\pgfqpoint{5.974941in}{1.356079in}}%
\pgfpathcurveto{\pgfqpoint{5.985028in}{1.356079in}}{\pgfqpoint{5.994704in}{1.360087in}}{\pgfqpoint{6.001837in}{1.367220in}}%
\pgfpathcurveto{\pgfqpoint{6.008970in}{1.374353in}}{\pgfqpoint{6.012977in}{1.384028in}}{\pgfqpoint{6.012977in}{1.394116in}}%
\pgfpathcurveto{\pgfqpoint{6.012977in}{1.404203in}}{\pgfqpoint{6.008970in}{1.413878in}}{\pgfqpoint{6.001837in}{1.421011in}}%
\pgfpathcurveto{\pgfqpoint{5.994704in}{1.428144in}}{\pgfqpoint{5.985028in}{1.432152in}}{\pgfqpoint{5.974941in}{1.432152in}}%
\pgfpathcurveto{\pgfqpoint{5.964854in}{1.432152in}}{\pgfqpoint{5.955178in}{1.428144in}}{\pgfqpoint{5.948045in}{1.421011in}}%
\pgfpathcurveto{\pgfqpoint{5.940912in}{1.413878in}}{\pgfqpoint{5.936905in}{1.404203in}}{\pgfqpoint{5.936905in}{1.394116in}}%
\pgfpathcurveto{\pgfqpoint{5.936905in}{1.384028in}}{\pgfqpoint{5.940912in}{1.374353in}}{\pgfqpoint{5.948045in}{1.367220in}}%
\pgfpathcurveto{\pgfqpoint{5.955178in}{1.360087in}}{\pgfqpoint{5.964854in}{1.356079in}}{\pgfqpoint{5.974941in}{1.356079in}}%
\pgfpathlineto{\pgfqpoint{5.974941in}{1.356079in}}%
\pgfpathclose%
\pgfusepath{stroke,fill}%
\end{pgfscope}%
\begin{pgfscope}%
\pgfpathrectangle{\pgfqpoint{4.658921in}{0.208778in}}{\pgfqpoint{3.800000in}{3.800000in}}%
\pgfusepath{clip}%
\pgfsetbuttcap%
\pgfsetroundjoin%
\definecolor{currentfill}{rgb}{0.839216,0.152941,0.156863}%
\pgfsetfillcolor{currentfill}%
\pgfsetfillopacity{0.492303}%
\pgfsetlinewidth{1.003750pt}%
\definecolor{currentstroke}{rgb}{0.839216,0.152941,0.156863}%
\pgfsetstrokecolor{currentstroke}%
\pgfsetstrokeopacity{0.492303}%
\pgfsetdash{}{0pt}%
\pgfpathmoveto{\pgfqpoint{5.525764in}{2.118941in}}%
\pgfpathcurveto{\pgfqpoint{5.535851in}{2.118941in}}{\pgfqpoint{5.545527in}{2.122949in}}{\pgfqpoint{5.552659in}{2.130082in}}%
\pgfpathcurveto{\pgfqpoint{5.559792in}{2.137215in}}{\pgfqpoint{5.563800in}{2.146890in}}{\pgfqpoint{5.563800in}{2.156978in}}%
\pgfpathcurveto{\pgfqpoint{5.563800in}{2.167065in}}{\pgfqpoint{5.559792in}{2.176740in}}{\pgfqpoint{5.552659in}{2.183873in}}%
\pgfpathcurveto{\pgfqpoint{5.545527in}{2.191006in}}{\pgfqpoint{5.535851in}{2.195014in}}{\pgfqpoint{5.525764in}{2.195014in}}%
\pgfpathcurveto{\pgfqpoint{5.515676in}{2.195014in}}{\pgfqpoint{5.506001in}{2.191006in}}{\pgfqpoint{5.498868in}{2.183873in}}%
\pgfpathcurveto{\pgfqpoint{5.491735in}{2.176740in}}{\pgfqpoint{5.487727in}{2.167065in}}{\pgfqpoint{5.487727in}{2.156978in}}%
\pgfpathcurveto{\pgfqpoint{5.487727in}{2.146890in}}{\pgfqpoint{5.491735in}{2.137215in}}{\pgfqpoint{5.498868in}{2.130082in}}%
\pgfpathcurveto{\pgfqpoint{5.506001in}{2.122949in}}{\pgfqpoint{5.515676in}{2.118941in}}{\pgfqpoint{5.525764in}{2.118941in}}%
\pgfpathlineto{\pgfqpoint{5.525764in}{2.118941in}}%
\pgfpathclose%
\pgfusepath{stroke,fill}%
\end{pgfscope}%
\begin{pgfscope}%
\pgfpathrectangle{\pgfqpoint{4.658921in}{0.208778in}}{\pgfqpoint{3.800000in}{3.800000in}}%
\pgfusepath{clip}%
\pgfsetbuttcap%
\pgfsetroundjoin%
\definecolor{currentfill}{rgb}{0.839216,0.152941,0.156863}%
\pgfsetfillcolor{currentfill}%
\pgfsetfillopacity{0.498590}%
\pgfsetlinewidth{1.003750pt}%
\definecolor{currentstroke}{rgb}{0.839216,0.152941,0.156863}%
\pgfsetstrokecolor{currentstroke}%
\pgfsetstrokeopacity{0.498590}%
\pgfsetdash{}{0pt}%
\pgfpathmoveto{\pgfqpoint{7.814236in}{2.271734in}}%
\pgfpathcurveto{\pgfqpoint{7.824324in}{2.271734in}}{\pgfqpoint{7.833999in}{2.275742in}}{\pgfqpoint{7.841132in}{2.282875in}}%
\pgfpathcurveto{\pgfqpoint{7.848265in}{2.290008in}}{\pgfqpoint{7.852273in}{2.299683in}}{\pgfqpoint{7.852273in}{2.309771in}}%
\pgfpathcurveto{\pgfqpoint{7.852273in}{2.319858in}}{\pgfqpoint{7.848265in}{2.329533in}}{\pgfqpoint{7.841132in}{2.336666in}}%
\pgfpathcurveto{\pgfqpoint{7.833999in}{2.343799in}}{\pgfqpoint{7.824324in}{2.347807in}}{\pgfqpoint{7.814236in}{2.347807in}}%
\pgfpathcurveto{\pgfqpoint{7.804149in}{2.347807in}}{\pgfqpoint{7.794473in}{2.343799in}}{\pgfqpoint{7.787341in}{2.336666in}}%
\pgfpathcurveto{\pgfqpoint{7.780208in}{2.329533in}}{\pgfqpoint{7.776200in}{2.319858in}}{\pgfqpoint{7.776200in}{2.309771in}}%
\pgfpathcurveto{\pgfqpoint{7.776200in}{2.299683in}}{\pgfqpoint{7.780208in}{2.290008in}}{\pgfqpoint{7.787341in}{2.282875in}}%
\pgfpathcurveto{\pgfqpoint{7.794473in}{2.275742in}}{\pgfqpoint{7.804149in}{2.271734in}}{\pgfqpoint{7.814236in}{2.271734in}}%
\pgfpathlineto{\pgfqpoint{7.814236in}{2.271734in}}%
\pgfpathclose%
\pgfusepath{stroke,fill}%
\end{pgfscope}%
\begin{pgfscope}%
\pgfpathrectangle{\pgfqpoint{4.658921in}{0.208778in}}{\pgfqpoint{3.800000in}{3.800000in}}%
\pgfusepath{clip}%
\pgfsetbuttcap%
\pgfsetroundjoin%
\definecolor{currentfill}{rgb}{0.839216,0.152941,0.156863}%
\pgfsetfillcolor{currentfill}%
\pgfsetfillopacity{0.612876}%
\pgfsetlinewidth{1.003750pt}%
\definecolor{currentstroke}{rgb}{0.839216,0.152941,0.156863}%
\pgfsetstrokecolor{currentstroke}%
\pgfsetstrokeopacity{0.612876}%
\pgfsetdash{}{0pt}%
\pgfpathmoveto{\pgfqpoint{5.991957in}{1.379714in}}%
\pgfpathcurveto{\pgfqpoint{6.002044in}{1.379714in}}{\pgfqpoint{6.011720in}{1.383721in}}{\pgfqpoint{6.018853in}{1.390854in}}%
\pgfpathcurveto{\pgfqpoint{6.025985in}{1.397987in}}{\pgfqpoint{6.029993in}{1.407663in}}{\pgfqpoint{6.029993in}{1.417750in}}%
\pgfpathcurveto{\pgfqpoint{6.029993in}{1.427837in}}{\pgfqpoint{6.025985in}{1.437513in}}{\pgfqpoint{6.018853in}{1.444646in}}%
\pgfpathcurveto{\pgfqpoint{6.011720in}{1.451778in}}{\pgfqpoint{6.002044in}{1.455786in}}{\pgfqpoint{5.991957in}{1.455786in}}%
\pgfpathcurveto{\pgfqpoint{5.981870in}{1.455786in}}{\pgfqpoint{5.972194in}{1.451778in}}{\pgfqpoint{5.965061in}{1.444646in}}%
\pgfpathcurveto{\pgfqpoint{5.957928in}{1.437513in}}{\pgfqpoint{5.953921in}{1.427837in}}{\pgfqpoint{5.953921in}{1.417750in}}%
\pgfpathcurveto{\pgfqpoint{5.953921in}{1.407663in}}{\pgfqpoint{5.957928in}{1.397987in}}{\pgfqpoint{5.965061in}{1.390854in}}%
\pgfpathcurveto{\pgfqpoint{5.972194in}{1.383721in}}{\pgfqpoint{5.981870in}{1.379714in}}{\pgfqpoint{5.991957in}{1.379714in}}%
\pgfpathlineto{\pgfqpoint{5.991957in}{1.379714in}}%
\pgfpathclose%
\pgfusepath{stroke,fill}%
\end{pgfscope}%
\begin{pgfscope}%
\pgfpathrectangle{\pgfqpoint{4.658921in}{0.208778in}}{\pgfqpoint{3.800000in}{3.800000in}}%
\pgfusepath{clip}%
\pgfsetbuttcap%
\pgfsetroundjoin%
\definecolor{currentfill}{rgb}{0.839216,0.152941,0.156863}%
\pgfsetfillcolor{currentfill}%
\pgfsetfillopacity{0.625674}%
\pgfsetlinewidth{1.003750pt}%
\definecolor{currentstroke}{rgb}{0.839216,0.152941,0.156863}%
\pgfsetstrokecolor{currentstroke}%
\pgfsetstrokeopacity{0.625674}%
\pgfsetdash{}{0pt}%
\pgfpathmoveto{\pgfqpoint{7.212967in}{3.025545in}}%
\pgfpathcurveto{\pgfqpoint{7.223054in}{3.025545in}}{\pgfqpoint{7.232730in}{3.029552in}}{\pgfqpoint{7.239862in}{3.036685in}}%
\pgfpathcurveto{\pgfqpoint{7.246995in}{3.043818in}}{\pgfqpoint{7.251003in}{3.053494in}}{\pgfqpoint{7.251003in}{3.063581in}}%
\pgfpathcurveto{\pgfqpoint{7.251003in}{3.073668in}}{\pgfqpoint{7.246995in}{3.083344in}}{\pgfqpoint{7.239862in}{3.090477in}}%
\pgfpathcurveto{\pgfqpoint{7.232730in}{3.097609in}}{\pgfqpoint{7.223054in}{3.101617in}}{\pgfqpoint{7.212967in}{3.101617in}}%
\pgfpathcurveto{\pgfqpoint{7.202879in}{3.101617in}}{\pgfqpoint{7.193204in}{3.097609in}}{\pgfqpoint{7.186071in}{3.090477in}}%
\pgfpathcurveto{\pgfqpoint{7.178938in}{3.083344in}}{\pgfqpoint{7.174930in}{3.073668in}}{\pgfqpoint{7.174930in}{3.063581in}}%
\pgfpathcurveto{\pgfqpoint{7.174930in}{3.053494in}}{\pgfqpoint{7.178938in}{3.043818in}}{\pgfqpoint{7.186071in}{3.036685in}}%
\pgfpathcurveto{\pgfqpoint{7.193204in}{3.029552in}}{\pgfqpoint{7.202879in}{3.025545in}}{\pgfqpoint{7.212967in}{3.025545in}}%
\pgfpathlineto{\pgfqpoint{7.212967in}{3.025545in}}%
\pgfpathclose%
\pgfusepath{stroke,fill}%
\end{pgfscope}%
\begin{pgfscope}%
\pgfpathrectangle{\pgfqpoint{4.658921in}{0.208778in}}{\pgfqpoint{3.800000in}{3.800000in}}%
\pgfusepath{clip}%
\pgfsetbuttcap%
\pgfsetroundjoin%
\definecolor{currentfill}{rgb}{0.839216,0.152941,0.156863}%
\pgfsetfillcolor{currentfill}%
\pgfsetfillopacity{0.631635}%
\pgfsetlinewidth{1.003750pt}%
\definecolor{currentstroke}{rgb}{0.839216,0.152941,0.156863}%
\pgfsetstrokecolor{currentstroke}%
\pgfsetstrokeopacity{0.631635}%
\pgfsetdash{}{0pt}%
\pgfpathmoveto{\pgfqpoint{7.200605in}{2.942909in}}%
\pgfpathcurveto{\pgfqpoint{7.210692in}{2.942909in}}{\pgfqpoint{7.220368in}{2.946917in}}{\pgfqpoint{7.227501in}{2.954050in}}%
\pgfpathcurveto{\pgfqpoint{7.234634in}{2.961183in}}{\pgfqpoint{7.238641in}{2.970858in}}{\pgfqpoint{7.238641in}{2.980946in}}%
\pgfpathcurveto{\pgfqpoint{7.238641in}{2.991033in}}{\pgfqpoint{7.234634in}{3.000709in}}{\pgfqpoint{7.227501in}{3.007841in}}%
\pgfpathcurveto{\pgfqpoint{7.220368in}{3.014974in}}{\pgfqpoint{7.210692in}{3.018982in}}{\pgfqpoint{7.200605in}{3.018982in}}%
\pgfpathcurveto{\pgfqpoint{7.190518in}{3.018982in}}{\pgfqpoint{7.180842in}{3.014974in}}{\pgfqpoint{7.173709in}{3.007841in}}%
\pgfpathcurveto{\pgfqpoint{7.166577in}{3.000709in}}{\pgfqpoint{7.162569in}{2.991033in}}{\pgfqpoint{7.162569in}{2.980946in}}%
\pgfpathcurveto{\pgfqpoint{7.162569in}{2.970858in}}{\pgfqpoint{7.166577in}{2.961183in}}{\pgfqpoint{7.173709in}{2.954050in}}%
\pgfpathcurveto{\pgfqpoint{7.180842in}{2.946917in}}{\pgfqpoint{7.190518in}{2.942909in}}{\pgfqpoint{7.200605in}{2.942909in}}%
\pgfpathlineto{\pgfqpoint{7.200605in}{2.942909in}}%
\pgfpathclose%
\pgfusepath{stroke,fill}%
\end{pgfscope}%
\begin{pgfscope}%
\pgfpathrectangle{\pgfqpoint{4.658921in}{0.208778in}}{\pgfqpoint{3.800000in}{3.800000in}}%
\pgfusepath{clip}%
\pgfsetbuttcap%
\pgfsetroundjoin%
\definecolor{currentfill}{rgb}{0.839216,0.152941,0.156863}%
\pgfsetfillcolor{currentfill}%
\pgfsetfillopacity{0.634032}%
\pgfsetlinewidth{1.003750pt}%
\definecolor{currentstroke}{rgb}{0.839216,0.152941,0.156863}%
\pgfsetstrokecolor{currentstroke}%
\pgfsetstrokeopacity{0.634032}%
\pgfsetdash{}{0pt}%
\pgfpathmoveto{\pgfqpoint{7.577084in}{2.149100in}}%
\pgfpathcurveto{\pgfqpoint{7.587171in}{2.149100in}}{\pgfqpoint{7.596847in}{2.153108in}}{\pgfqpoint{7.603980in}{2.160241in}}%
\pgfpathcurveto{\pgfqpoint{7.611113in}{2.167374in}}{\pgfqpoint{7.615120in}{2.177049in}}{\pgfqpoint{7.615120in}{2.187137in}}%
\pgfpathcurveto{\pgfqpoint{7.615120in}{2.197224in}}{\pgfqpoint{7.611113in}{2.206900in}}{\pgfqpoint{7.603980in}{2.214032in}}%
\pgfpathcurveto{\pgfqpoint{7.596847in}{2.221165in}}{\pgfqpoint{7.587171in}{2.225173in}}{\pgfqpoint{7.577084in}{2.225173in}}%
\pgfpathcurveto{\pgfqpoint{7.566997in}{2.225173in}}{\pgfqpoint{7.557321in}{2.221165in}}{\pgfqpoint{7.550188in}{2.214032in}}%
\pgfpathcurveto{\pgfqpoint{7.543056in}{2.206900in}}{\pgfqpoint{7.539048in}{2.197224in}}{\pgfqpoint{7.539048in}{2.187137in}}%
\pgfpathcurveto{\pgfqpoint{7.539048in}{2.177049in}}{\pgfqpoint{7.543056in}{2.167374in}}{\pgfqpoint{7.550188in}{2.160241in}}%
\pgfpathcurveto{\pgfqpoint{7.557321in}{2.153108in}}{\pgfqpoint{7.566997in}{2.149100in}}{\pgfqpoint{7.577084in}{2.149100in}}%
\pgfpathlineto{\pgfqpoint{7.577084in}{2.149100in}}%
\pgfpathclose%
\pgfusepath{stroke,fill}%
\end{pgfscope}%
\begin{pgfscope}%
\pgfpathrectangle{\pgfqpoint{4.658921in}{0.208778in}}{\pgfqpoint{3.800000in}{3.800000in}}%
\pgfusepath{clip}%
\pgfsetbuttcap%
\pgfsetroundjoin%
\definecolor{currentfill}{rgb}{0.839216,0.152941,0.156863}%
\pgfsetfillcolor{currentfill}%
\pgfsetfillopacity{0.652064}%
\pgfsetlinewidth{1.003750pt}%
\definecolor{currentstroke}{rgb}{0.839216,0.152941,0.156863}%
\pgfsetstrokecolor{currentstroke}%
\pgfsetstrokeopacity{0.652064}%
\pgfsetdash{}{0pt}%
\pgfpathmoveto{\pgfqpoint{6.345222in}{1.341889in}}%
\pgfpathcurveto{\pgfqpoint{6.355309in}{1.341889in}}{\pgfqpoint{6.364985in}{1.345897in}}{\pgfqpoint{6.372118in}{1.353030in}}%
\pgfpathcurveto{\pgfqpoint{6.379250in}{1.360163in}}{\pgfqpoint{6.383258in}{1.369838in}}{\pgfqpoint{6.383258in}{1.379926in}}%
\pgfpathcurveto{\pgfqpoint{6.383258in}{1.390013in}}{\pgfqpoint{6.379250in}{1.399688in}}{\pgfqpoint{6.372118in}{1.406821in}}%
\pgfpathcurveto{\pgfqpoint{6.364985in}{1.413954in}}{\pgfqpoint{6.355309in}{1.417962in}}{\pgfqpoint{6.345222in}{1.417962in}}%
\pgfpathcurveto{\pgfqpoint{6.335134in}{1.417962in}}{\pgfqpoint{6.325459in}{1.413954in}}{\pgfqpoint{6.318326in}{1.406821in}}%
\pgfpathcurveto{\pgfqpoint{6.311193in}{1.399688in}}{\pgfqpoint{6.307186in}{1.390013in}}{\pgfqpoint{6.307186in}{1.379926in}}%
\pgfpathcurveto{\pgfqpoint{6.307186in}{1.369838in}}{\pgfqpoint{6.311193in}{1.360163in}}{\pgfqpoint{6.318326in}{1.353030in}}%
\pgfpathcurveto{\pgfqpoint{6.325459in}{1.345897in}}{\pgfqpoint{6.335134in}{1.341889in}}{\pgfqpoint{6.345222in}{1.341889in}}%
\pgfpathlineto{\pgfqpoint{6.345222in}{1.341889in}}%
\pgfpathclose%
\pgfusepath{stroke,fill}%
\end{pgfscope}%
\begin{pgfscope}%
\pgfpathrectangle{\pgfqpoint{4.658921in}{0.208778in}}{\pgfqpoint{3.800000in}{3.800000in}}%
\pgfusepath{clip}%
\pgfsetbuttcap%
\pgfsetroundjoin%
\definecolor{currentfill}{rgb}{0.839216,0.152941,0.156863}%
\pgfsetfillcolor{currentfill}%
\pgfsetfillopacity{0.738747}%
\pgfsetlinewidth{1.003750pt}%
\definecolor{currentstroke}{rgb}{0.839216,0.152941,0.156863}%
\pgfsetstrokecolor{currentstroke}%
\pgfsetstrokeopacity{0.738747}%
\pgfsetdash{}{0pt}%
\pgfpathmoveto{\pgfqpoint{5.497941in}{2.113471in}}%
\pgfpathcurveto{\pgfqpoint{5.508029in}{2.113471in}}{\pgfqpoint{5.517704in}{2.117479in}}{\pgfqpoint{5.524837in}{2.124612in}}%
\pgfpathcurveto{\pgfqpoint{5.531970in}{2.131744in}}{\pgfqpoint{5.535978in}{2.141420in}}{\pgfqpoint{5.535978in}{2.151507in}}%
\pgfpathcurveto{\pgfqpoint{5.535978in}{2.161595in}}{\pgfqpoint{5.531970in}{2.171270in}}{\pgfqpoint{5.524837in}{2.178403in}}%
\pgfpathcurveto{\pgfqpoint{5.517704in}{2.185536in}}{\pgfqpoint{5.508029in}{2.189544in}}{\pgfqpoint{5.497941in}{2.189544in}}%
\pgfpathcurveto{\pgfqpoint{5.487854in}{2.189544in}}{\pgfqpoint{5.478178in}{2.185536in}}{\pgfqpoint{5.471046in}{2.178403in}}%
\pgfpathcurveto{\pgfqpoint{5.463913in}{2.171270in}}{\pgfqpoint{5.459905in}{2.161595in}}{\pgfqpoint{5.459905in}{2.151507in}}%
\pgfpathcurveto{\pgfqpoint{5.459905in}{2.141420in}}{\pgfqpoint{5.463913in}{2.131744in}}{\pgfqpoint{5.471046in}{2.124612in}}%
\pgfpathcurveto{\pgfqpoint{5.478178in}{2.117479in}}{\pgfqpoint{5.487854in}{2.113471in}}{\pgfqpoint{5.497941in}{2.113471in}}%
\pgfpathlineto{\pgfqpoint{5.497941in}{2.113471in}}%
\pgfpathclose%
\pgfusepath{stroke,fill}%
\end{pgfscope}%
\begin{pgfscope}%
\pgfpathrectangle{\pgfqpoint{4.658921in}{0.208778in}}{\pgfqpoint{3.800000in}{3.800000in}}%
\pgfusepath{clip}%
\pgfsetbuttcap%
\pgfsetroundjoin%
\definecolor{currentfill}{rgb}{0.839216,0.152941,0.156863}%
\pgfsetfillcolor{currentfill}%
\pgfsetfillopacity{0.791813}%
\pgfsetlinewidth{1.003750pt}%
\definecolor{currentstroke}{rgb}{0.839216,0.152941,0.156863}%
\pgfsetstrokecolor{currentstroke}%
\pgfsetstrokeopacity{0.791813}%
\pgfsetdash{}{0pt}%
\pgfpathmoveto{\pgfqpoint{6.183031in}{2.817079in}}%
\pgfpathcurveto{\pgfqpoint{6.193119in}{2.817079in}}{\pgfqpoint{6.202794in}{2.821087in}}{\pgfqpoint{6.209927in}{2.828220in}}%
\pgfpathcurveto{\pgfqpoint{6.217060in}{2.835353in}}{\pgfqpoint{6.221068in}{2.845028in}}{\pgfqpoint{6.221068in}{2.855116in}}%
\pgfpathcurveto{\pgfqpoint{6.221068in}{2.865203in}}{\pgfqpoint{6.217060in}{2.874879in}}{\pgfqpoint{6.209927in}{2.882011in}}%
\pgfpathcurveto{\pgfqpoint{6.202794in}{2.889144in}}{\pgfqpoint{6.193119in}{2.893152in}}{\pgfqpoint{6.183031in}{2.893152in}}%
\pgfpathcurveto{\pgfqpoint{6.172944in}{2.893152in}}{\pgfqpoint{6.163268in}{2.889144in}}{\pgfqpoint{6.156136in}{2.882011in}}%
\pgfpathcurveto{\pgfqpoint{6.149003in}{2.874879in}}{\pgfqpoint{6.144995in}{2.865203in}}{\pgfqpoint{6.144995in}{2.855116in}}%
\pgfpathcurveto{\pgfqpoint{6.144995in}{2.845028in}}{\pgfqpoint{6.149003in}{2.835353in}}{\pgfqpoint{6.156136in}{2.828220in}}%
\pgfpathcurveto{\pgfqpoint{6.163268in}{2.821087in}}{\pgfqpoint{6.172944in}{2.817079in}}{\pgfqpoint{6.183031in}{2.817079in}}%
\pgfpathlineto{\pgfqpoint{6.183031in}{2.817079in}}%
\pgfpathclose%
\pgfusepath{stroke,fill}%
\end{pgfscope}%
\begin{pgfscope}%
\pgfpathrectangle{\pgfqpoint{4.658921in}{0.208778in}}{\pgfqpoint{3.800000in}{3.800000in}}%
\pgfusepath{clip}%
\pgfsetbuttcap%
\pgfsetroundjoin%
\definecolor{currentfill}{rgb}{0.839216,0.152941,0.156863}%
\pgfsetfillcolor{currentfill}%
\pgfsetfillopacity{0.864233}%
\pgfsetlinewidth{1.003750pt}%
\definecolor{currentstroke}{rgb}{0.839216,0.152941,0.156863}%
\pgfsetstrokecolor{currentstroke}%
\pgfsetstrokeopacity{0.864233}%
\pgfsetdash{}{0pt}%
\pgfpathmoveto{\pgfqpoint{7.108724in}{2.695780in}}%
\pgfpathcurveto{\pgfqpoint{7.118811in}{2.695780in}}{\pgfqpoint{7.128487in}{2.699788in}}{\pgfqpoint{7.135619in}{2.706921in}}%
\pgfpathcurveto{\pgfqpoint{7.142752in}{2.714054in}}{\pgfqpoint{7.146760in}{2.723729in}}{\pgfqpoint{7.146760in}{2.733816in}}%
\pgfpathcurveto{\pgfqpoint{7.146760in}{2.743904in}}{\pgfqpoint{7.142752in}{2.753579in}}{\pgfqpoint{7.135619in}{2.760712in}}%
\pgfpathcurveto{\pgfqpoint{7.128487in}{2.767845in}}{\pgfqpoint{7.118811in}{2.771853in}}{\pgfqpoint{7.108724in}{2.771853in}}%
\pgfpathcurveto{\pgfqpoint{7.098636in}{2.771853in}}{\pgfqpoint{7.088961in}{2.767845in}}{\pgfqpoint{7.081828in}{2.760712in}}%
\pgfpathcurveto{\pgfqpoint{7.074695in}{2.753579in}}{\pgfqpoint{7.070687in}{2.743904in}}{\pgfqpoint{7.070687in}{2.733816in}}%
\pgfpathcurveto{\pgfqpoint{7.070687in}{2.723729in}}{\pgfqpoint{7.074695in}{2.714054in}}{\pgfqpoint{7.081828in}{2.706921in}}%
\pgfpathcurveto{\pgfqpoint{7.088961in}{2.699788in}}{\pgfqpoint{7.098636in}{2.695780in}}{\pgfqpoint{7.108724in}{2.695780in}}%
\pgfpathlineto{\pgfqpoint{7.108724in}{2.695780in}}%
\pgfpathclose%
\pgfusepath{stroke,fill}%
\end{pgfscope}%
\begin{pgfscope}%
\pgfpathrectangle{\pgfqpoint{4.658921in}{0.208778in}}{\pgfqpoint{3.800000in}{3.800000in}}%
\pgfusepath{clip}%
\pgfsetbuttcap%
\pgfsetroundjoin%
\definecolor{currentfill}{rgb}{0.839216,0.152941,0.156863}%
\pgfsetfillcolor{currentfill}%
\pgfsetfillopacity{0.929084}%
\pgfsetlinewidth{1.003750pt}%
\definecolor{currentstroke}{rgb}{0.839216,0.152941,0.156863}%
\pgfsetstrokecolor{currentstroke}%
\pgfsetstrokeopacity{0.929084}%
\pgfsetdash{}{0pt}%
\pgfpathmoveto{\pgfqpoint{7.191263in}{1.809510in}}%
\pgfpathcurveto{\pgfqpoint{7.201350in}{1.809510in}}{\pgfqpoint{7.211026in}{1.813517in}}{\pgfqpoint{7.218159in}{1.820650in}}%
\pgfpathcurveto{\pgfqpoint{7.225291in}{1.827783in}}{\pgfqpoint{7.229299in}{1.837459in}}{\pgfqpoint{7.229299in}{1.847546in}}%
\pgfpathcurveto{\pgfqpoint{7.229299in}{1.857633in}}{\pgfqpoint{7.225291in}{1.867309in}}{\pgfqpoint{7.218159in}{1.874442in}}%
\pgfpathcurveto{\pgfqpoint{7.211026in}{1.881574in}}{\pgfqpoint{7.201350in}{1.885582in}}{\pgfqpoint{7.191263in}{1.885582in}}%
\pgfpathcurveto{\pgfqpoint{7.181175in}{1.885582in}}{\pgfqpoint{7.171500in}{1.881574in}}{\pgfqpoint{7.164367in}{1.874442in}}%
\pgfpathcurveto{\pgfqpoint{7.157234in}{1.867309in}}{\pgfqpoint{7.153227in}{1.857633in}}{\pgfqpoint{7.153227in}{1.847546in}}%
\pgfpathcurveto{\pgfqpoint{7.153227in}{1.837459in}}{\pgfqpoint{7.157234in}{1.827783in}}{\pgfqpoint{7.164367in}{1.820650in}}%
\pgfpathcurveto{\pgfqpoint{7.171500in}{1.813517in}}{\pgfqpoint{7.181175in}{1.809510in}}{\pgfqpoint{7.191263in}{1.809510in}}%
\pgfpathlineto{\pgfqpoint{7.191263in}{1.809510in}}%
\pgfpathclose%
\pgfusepath{stroke,fill}%
\end{pgfscope}%
\begin{pgfscope}%
\pgfpathrectangle{\pgfqpoint{4.658921in}{0.208778in}}{\pgfqpoint{3.800000in}{3.800000in}}%
\pgfusepath{clip}%
\pgfsetbuttcap%
\pgfsetroundjoin%
\definecolor{currentfill}{rgb}{0.839216,0.152941,0.156863}%
\pgfsetfillcolor{currentfill}%
\pgfsetlinewidth{1.003750pt}%
\definecolor{currentstroke}{rgb}{0.839216,0.152941,0.156863}%
\pgfsetstrokecolor{currentstroke}%
\pgfsetdash{}{0pt}%
\pgfpathmoveto{\pgfqpoint{7.024505in}{2.112077in}}%
\pgfpathcurveto{\pgfqpoint{7.034592in}{2.112077in}}{\pgfqpoint{7.044268in}{2.116085in}}{\pgfqpoint{7.051401in}{2.123218in}}%
\pgfpathcurveto{\pgfqpoint{7.058533in}{2.130350in}}{\pgfqpoint{7.062541in}{2.140026in}}{\pgfqpoint{7.062541in}{2.150113in}}%
\pgfpathcurveto{\pgfqpoint{7.062541in}{2.160201in}}{\pgfqpoint{7.058533in}{2.169876in}}{\pgfqpoint{7.051401in}{2.177009in}}%
\pgfpathcurveto{\pgfqpoint{7.044268in}{2.184142in}}{\pgfqpoint{7.034592in}{2.188150in}}{\pgfqpoint{7.024505in}{2.188150in}}%
\pgfpathcurveto{\pgfqpoint{7.014417in}{2.188150in}}{\pgfqpoint{7.004742in}{2.184142in}}{\pgfqpoint{6.997609in}{2.177009in}}%
\pgfpathcurveto{\pgfqpoint{6.990476in}{2.169876in}}{\pgfqpoint{6.986469in}{2.160201in}}{\pgfqpoint{6.986469in}{2.150113in}}%
\pgfpathcurveto{\pgfqpoint{6.986469in}{2.140026in}}{\pgfqpoint{6.990476in}{2.130350in}}{\pgfqpoint{6.997609in}{2.123218in}}%
\pgfpathcurveto{\pgfqpoint{7.004742in}{2.116085in}}{\pgfqpoint{7.014417in}{2.112077in}}{\pgfqpoint{7.024505in}{2.112077in}}%
\pgfpathlineto{\pgfqpoint{7.024505in}{2.112077in}}%
\pgfpathclose%
\pgfusepath{stroke,fill}%
\end{pgfscope}%
\begin{pgfscope}%
\pgfpathrectangle{\pgfqpoint{4.658921in}{0.208778in}}{\pgfqpoint{3.800000in}{3.800000in}}%
\pgfusepath{clip}%
\pgfsetbuttcap%
\pgfsetroundjoin%
\definecolor{currentfill}{rgb}{0.071067,0.258424,0.071067}%
\pgfsetfillcolor{currentfill}%
\pgfsetfillopacity{0.200000}%
\pgfsetlinewidth{0.000000pt}%
\definecolor{currentstroke}{rgb}{0.000000,0.000000,0.000000}%
\pgfsetstrokecolor{currentstroke}%
\pgfsetdash{}{0pt}%
\pgfpathmoveto{\pgfqpoint{5.274860in}{1.516856in}}%
\pgfpathlineto{\pgfqpoint{5.140909in}{1.711315in}}%
\pgfpathlineto{\pgfqpoint{5.141706in}{1.609418in}}%
\pgfpathlineto{\pgfqpoint{5.274860in}{1.516856in}}%
\pgfpathclose%
\pgfusepath{fill}%
\end{pgfscope}%
\begin{pgfscope}%
\pgfpathrectangle{\pgfqpoint{4.658921in}{0.208778in}}{\pgfqpoint{3.800000in}{3.800000in}}%
\pgfusepath{clip}%
\pgfsetbuttcap%
\pgfsetroundjoin%
\definecolor{currentfill}{rgb}{0.071067,0.258424,0.071067}%
\pgfsetfillcolor{currentfill}%
\pgfsetfillopacity{0.200000}%
\pgfsetlinewidth{0.000000pt}%
\definecolor{currentstroke}{rgb}{0.000000,0.000000,0.000000}%
\pgfsetstrokecolor{currentstroke}%
\pgfsetdash{}{0pt}%
\pgfpathmoveto{\pgfqpoint{8.079635in}{1.711315in}}%
\pgfpathlineto{\pgfqpoint{7.945684in}{1.516856in}}%
\pgfpathlineto{\pgfqpoint{8.078838in}{1.609418in}}%
\pgfpathlineto{\pgfqpoint{8.079635in}{1.711315in}}%
\pgfpathclose%
\pgfusepath{fill}%
\end{pgfscope}%
\begin{pgfscope}%
\pgfpathrectangle{\pgfqpoint{4.658921in}{0.208778in}}{\pgfqpoint{3.800000in}{3.800000in}}%
\pgfusepath{clip}%
\pgfsetbuttcap%
\pgfsetroundjoin%
\definecolor{currentfill}{rgb}{0.128601,0.467641,0.128601}%
\pgfsetfillcolor{currentfill}%
\pgfsetfillopacity{0.200000}%
\pgfsetlinewidth{0.000000pt}%
\definecolor{currentstroke}{rgb}{0.000000,0.000000,0.000000}%
\pgfsetstrokecolor{currentstroke}%
\pgfsetdash{}{0pt}%
\pgfpathmoveto{\pgfqpoint{6.476320in}{3.661884in}}%
\pgfpathlineto{\pgfqpoint{6.744224in}{3.661884in}}%
\pgfpathlineto{\pgfqpoint{6.610272in}{3.754163in}}%
\pgfpathlineto{\pgfqpoint{6.476320in}{3.661884in}}%
\pgfpathclose%
\pgfusepath{fill}%
\end{pgfscope}%
\begin{pgfscope}%
\pgfpathrectangle{\pgfqpoint{4.658921in}{0.208778in}}{\pgfqpoint{3.800000in}{3.800000in}}%
\pgfusepath{clip}%
\pgfsetbuttcap%
\pgfsetroundjoin%
\definecolor{currentfill}{rgb}{0.067488,0.245410,0.067488}%
\pgfsetfillcolor{currentfill}%
\pgfsetfillopacity{0.200000}%
\pgfsetlinewidth{0.000000pt}%
\definecolor{currentstroke}{rgb}{0.000000,0.000000,0.000000}%
\pgfsetstrokecolor{currentstroke}%
\pgfsetdash{}{0pt}%
\pgfpathmoveto{\pgfqpoint{5.447851in}{1.419312in}}%
\pgfpathlineto{\pgfqpoint{5.288610in}{1.620589in}}%
\pgfpathlineto{\pgfqpoint{5.274860in}{1.516856in}}%
\pgfpathlineto{\pgfqpoint{5.447851in}{1.419312in}}%
\pgfpathclose%
\pgfusepath{fill}%
\end{pgfscope}%
\begin{pgfscope}%
\pgfpathrectangle{\pgfqpoint{4.658921in}{0.208778in}}{\pgfqpoint{3.800000in}{3.800000in}}%
\pgfusepath{clip}%
\pgfsetbuttcap%
\pgfsetroundjoin%
\definecolor{currentfill}{rgb}{0.067488,0.245410,0.067488}%
\pgfsetfillcolor{currentfill}%
\pgfsetfillopacity{0.200000}%
\pgfsetlinewidth{0.000000pt}%
\definecolor{currentstroke}{rgb}{0.000000,0.000000,0.000000}%
\pgfsetstrokecolor{currentstroke}%
\pgfsetdash{}{0pt}%
\pgfpathmoveto{\pgfqpoint{7.945684in}{1.516856in}}%
\pgfpathlineto{\pgfqpoint{7.931934in}{1.620589in}}%
\pgfpathlineto{\pgfqpoint{7.772693in}{1.419312in}}%
\pgfpathlineto{\pgfqpoint{7.945684in}{1.516856in}}%
\pgfpathclose%
\pgfusepath{fill}%
\end{pgfscope}%
\begin{pgfscope}%
\pgfpathrectangle{\pgfqpoint{4.658921in}{0.208778in}}{\pgfqpoint{3.800000in}{3.800000in}}%
\pgfusepath{clip}%
\pgfsetbuttcap%
\pgfsetroundjoin%
\definecolor{currentfill}{rgb}{0.069492,0.252698,0.069492}%
\pgfsetfillcolor{currentfill}%
\pgfsetfillopacity{0.200000}%
\pgfsetlinewidth{0.000000pt}%
\definecolor{currentstroke}{rgb}{0.000000,0.000000,0.000000}%
\pgfsetstrokecolor{currentstroke}%
\pgfsetdash{}{0pt}%
\pgfpathmoveto{\pgfqpoint{5.140909in}{1.711315in}}%
\pgfpathlineto{\pgfqpoint{5.274860in}{1.516856in}}%
\pgfpathlineto{\pgfqpoint{5.271794in}{2.212598in}}%
\pgfpathlineto{\pgfqpoint{5.140909in}{1.711315in}}%
\pgfpathclose%
\pgfusepath{fill}%
\end{pgfscope}%
\begin{pgfscope}%
\pgfpathrectangle{\pgfqpoint{4.658921in}{0.208778in}}{\pgfqpoint{3.800000in}{3.800000in}}%
\pgfusepath{clip}%
\pgfsetbuttcap%
\pgfsetroundjoin%
\definecolor{currentfill}{rgb}{0.069492,0.252698,0.069492}%
\pgfsetfillcolor{currentfill}%
\pgfsetfillopacity{0.200000}%
\pgfsetlinewidth{0.000000pt}%
\definecolor{currentstroke}{rgb}{0.000000,0.000000,0.000000}%
\pgfsetstrokecolor{currentstroke}%
\pgfsetdash{}{0pt}%
\pgfpathmoveto{\pgfqpoint{8.079635in}{1.711315in}}%
\pgfpathlineto{\pgfqpoint{7.948750in}{2.212598in}}%
\pgfpathlineto{\pgfqpoint{7.945684in}{1.516856in}}%
\pgfpathlineto{\pgfqpoint{8.079635in}{1.711315in}}%
\pgfpathclose%
\pgfusepath{fill}%
\end{pgfscope}%
\begin{pgfscope}%
\pgfpathrectangle{\pgfqpoint{4.658921in}{0.208778in}}{\pgfqpoint{3.800000in}{3.800000in}}%
\pgfusepath{clip}%
\pgfsetbuttcap%
\pgfsetroundjoin%
\definecolor{currentfill}{rgb}{0.099716,0.362602,0.099716}%
\pgfsetfillcolor{currentfill}%
\pgfsetfillopacity{0.200000}%
\pgfsetlinewidth{0.000000pt}%
\definecolor{currentstroke}{rgb}{0.000000,0.000000,0.000000}%
\pgfsetstrokecolor{currentstroke}%
\pgfsetdash{}{0pt}%
\pgfpathmoveto{\pgfqpoint{5.271794in}{2.212598in}}%
\pgfpathlineto{\pgfqpoint{5.274860in}{1.516856in}}%
\pgfpathlineto{\pgfqpoint{5.288610in}{1.620589in}}%
\pgfpathlineto{\pgfqpoint{5.271794in}{2.212598in}}%
\pgfpathclose%
\pgfusepath{fill}%
\end{pgfscope}%
\begin{pgfscope}%
\pgfpathrectangle{\pgfqpoint{4.658921in}{0.208778in}}{\pgfqpoint{3.800000in}{3.800000in}}%
\pgfusepath{clip}%
\pgfsetbuttcap%
\pgfsetroundjoin%
\definecolor{currentfill}{rgb}{0.099716,0.362602,0.099716}%
\pgfsetfillcolor{currentfill}%
\pgfsetfillopacity{0.200000}%
\pgfsetlinewidth{0.000000pt}%
\definecolor{currentstroke}{rgb}{0.000000,0.000000,0.000000}%
\pgfsetstrokecolor{currentstroke}%
\pgfsetdash{}{0pt}%
\pgfpathmoveto{\pgfqpoint{7.931934in}{1.620589in}}%
\pgfpathlineto{\pgfqpoint{7.945684in}{1.516856in}}%
\pgfpathlineto{\pgfqpoint{7.948750in}{2.212598in}}%
\pgfpathlineto{\pgfqpoint{7.931934in}{1.620589in}}%
\pgfpathclose%
\pgfusepath{fill}%
\end{pgfscope}%
\begin{pgfscope}%
\pgfpathrectangle{\pgfqpoint{4.658921in}{0.208778in}}{\pgfqpoint{3.800000in}{3.800000in}}%
\pgfusepath{clip}%
\pgfsetbuttcap%
\pgfsetroundjoin%
\definecolor{currentfill}{rgb}{0.063840,0.232145,0.063840}%
\pgfsetfillcolor{currentfill}%
\pgfsetfillopacity{0.200000}%
\pgfsetlinewidth{0.000000pt}%
\definecolor{currentstroke}{rgb}{0.000000,0.000000,0.000000}%
\pgfsetstrokecolor{currentstroke}%
\pgfsetdash{}{0pt}%
\pgfpathmoveto{\pgfqpoint{5.668862in}{1.323116in}}%
\pgfpathlineto{\pgfqpoint{5.484653in}{1.525226in}}%
\pgfpathlineto{\pgfqpoint{5.447851in}{1.419312in}}%
\pgfpathlineto{\pgfqpoint{5.668862in}{1.323116in}}%
\pgfpathclose%
\pgfusepath{fill}%
\end{pgfscope}%
\begin{pgfscope}%
\pgfpathrectangle{\pgfqpoint{4.658921in}{0.208778in}}{\pgfqpoint{3.800000in}{3.800000in}}%
\pgfusepath{clip}%
\pgfsetbuttcap%
\pgfsetroundjoin%
\definecolor{currentfill}{rgb}{0.063840,0.232145,0.063840}%
\pgfsetfillcolor{currentfill}%
\pgfsetfillopacity{0.200000}%
\pgfsetlinewidth{0.000000pt}%
\definecolor{currentstroke}{rgb}{0.000000,0.000000,0.000000}%
\pgfsetstrokecolor{currentstroke}%
\pgfsetdash{}{0pt}%
\pgfpathmoveto{\pgfqpoint{7.772693in}{1.419312in}}%
\pgfpathlineto{\pgfqpoint{7.735891in}{1.525226in}}%
\pgfpathlineto{\pgfqpoint{7.551682in}{1.323116in}}%
\pgfpathlineto{\pgfqpoint{7.772693in}{1.419312in}}%
\pgfpathclose%
\pgfusepath{fill}%
\end{pgfscope}%
\begin{pgfscope}%
\pgfpathrectangle{\pgfqpoint{4.658921in}{0.208778in}}{\pgfqpoint{3.800000in}{3.800000in}}%
\pgfusepath{clip}%
\pgfsetbuttcap%
\pgfsetroundjoin%
\definecolor{currentfill}{rgb}{0.116321,0.422987,0.116321}%
\pgfsetfillcolor{currentfill}%
\pgfsetfillopacity{0.200000}%
\pgfsetlinewidth{0.000000pt}%
\definecolor{currentstroke}{rgb}{0.000000,0.000000,0.000000}%
\pgfsetstrokecolor{currentstroke}%
\pgfsetdash{}{0pt}%
\pgfpathmoveto{\pgfqpoint{6.744224in}{3.661884in}}%
\pgfpathlineto{\pgfqpoint{6.476320in}{3.661884in}}%
\pgfpathlineto{\pgfqpoint{6.418151in}{2.991915in}}%
\pgfpathlineto{\pgfqpoint{6.744224in}{3.661884in}}%
\pgfpathclose%
\pgfusepath{fill}%
\end{pgfscope}%
\begin{pgfscope}%
\pgfpathrectangle{\pgfqpoint{4.658921in}{0.208778in}}{\pgfqpoint{3.800000in}{3.800000in}}%
\pgfusepath{clip}%
\pgfsetbuttcap%
\pgfsetroundjoin%
\definecolor{currentfill}{rgb}{0.067061,0.243857,0.067061}%
\pgfsetfillcolor{currentfill}%
\pgfsetfillopacity{0.200000}%
\pgfsetlinewidth{0.000000pt}%
\definecolor{currentstroke}{rgb}{0.000000,0.000000,0.000000}%
\pgfsetstrokecolor{currentstroke}%
\pgfsetdash{}{0pt}%
\pgfpathmoveto{\pgfqpoint{5.288610in}{1.620589in}}%
\pgfpathlineto{\pgfqpoint{5.447851in}{1.419312in}}%
\pgfpathlineto{\pgfqpoint{5.497135in}{2.167772in}}%
\pgfpathlineto{\pgfqpoint{5.288610in}{1.620589in}}%
\pgfpathclose%
\pgfusepath{fill}%
\end{pgfscope}%
\begin{pgfscope}%
\pgfpathrectangle{\pgfqpoint{4.658921in}{0.208778in}}{\pgfqpoint{3.800000in}{3.800000in}}%
\pgfusepath{clip}%
\pgfsetbuttcap%
\pgfsetroundjoin%
\definecolor{currentfill}{rgb}{0.067061,0.243857,0.067061}%
\pgfsetfillcolor{currentfill}%
\pgfsetfillopacity{0.200000}%
\pgfsetlinewidth{0.000000pt}%
\definecolor{currentstroke}{rgb}{0.000000,0.000000,0.000000}%
\pgfsetstrokecolor{currentstroke}%
\pgfsetdash{}{0pt}%
\pgfpathmoveto{\pgfqpoint{7.723409in}{2.167772in}}%
\pgfpathlineto{\pgfqpoint{7.772693in}{1.419312in}}%
\pgfpathlineto{\pgfqpoint{7.931934in}{1.620589in}}%
\pgfpathlineto{\pgfqpoint{7.723409in}{2.167772in}}%
\pgfpathclose%
\pgfusepath{fill}%
\end{pgfscope}%
\begin{pgfscope}%
\pgfpathrectangle{\pgfqpoint{4.658921in}{0.208778in}}{\pgfqpoint{3.800000in}{3.800000in}}%
\pgfusepath{clip}%
\pgfsetbuttcap%
\pgfsetroundjoin%
\definecolor{currentfill}{rgb}{0.095351,0.346729,0.095351}%
\pgfsetfillcolor{currentfill}%
\pgfsetfillopacity{0.200000}%
\pgfsetlinewidth{0.000000pt}%
\definecolor{currentstroke}{rgb}{0.000000,0.000000,0.000000}%
\pgfsetstrokecolor{currentstroke}%
\pgfsetdash{}{0pt}%
\pgfpathmoveto{\pgfqpoint{5.497135in}{2.167772in}}%
\pgfpathlineto{\pgfqpoint{5.447851in}{1.419312in}}%
\pgfpathlineto{\pgfqpoint{5.484653in}{1.525226in}}%
\pgfpathlineto{\pgfqpoint{5.497135in}{2.167772in}}%
\pgfpathclose%
\pgfusepath{fill}%
\end{pgfscope}%
\begin{pgfscope}%
\pgfpathrectangle{\pgfqpoint{4.658921in}{0.208778in}}{\pgfqpoint{3.800000in}{3.800000in}}%
\pgfusepath{clip}%
\pgfsetbuttcap%
\pgfsetroundjoin%
\definecolor{currentfill}{rgb}{0.095351,0.346729,0.095351}%
\pgfsetfillcolor{currentfill}%
\pgfsetfillopacity{0.200000}%
\pgfsetlinewidth{0.000000pt}%
\definecolor{currentstroke}{rgb}{0.000000,0.000000,0.000000}%
\pgfsetstrokecolor{currentstroke}%
\pgfsetdash{}{0pt}%
\pgfpathmoveto{\pgfqpoint{7.735891in}{1.525226in}}%
\pgfpathlineto{\pgfqpoint{7.772693in}{1.419312in}}%
\pgfpathlineto{\pgfqpoint{7.723409in}{2.167772in}}%
\pgfpathlineto{\pgfqpoint{7.735891in}{1.525226in}}%
\pgfpathclose%
\pgfusepath{fill}%
\end{pgfscope}%
\begin{pgfscope}%
\pgfpathrectangle{\pgfqpoint{4.658921in}{0.208778in}}{\pgfqpoint{3.800000in}{3.800000in}}%
\pgfusepath{clip}%
\pgfsetbuttcap%
\pgfsetroundjoin%
\definecolor{currentfill}{rgb}{0.060435,0.219763,0.060435}%
\pgfsetfillcolor{currentfill}%
\pgfsetfillopacity{0.200000}%
\pgfsetlinewidth{0.000000pt}%
\definecolor{currentstroke}{rgb}{0.000000,0.000000,0.000000}%
\pgfsetstrokecolor{currentstroke}%
\pgfsetdash{}{0pt}%
\pgfpathmoveto{\pgfqpoint{5.668862in}{1.323116in}}%
\pgfpathlineto{\pgfqpoint{5.941967in}{1.238733in}}%
\pgfpathlineto{\pgfqpoint{5.738977in}{1.433246in}}%
\pgfpathlineto{\pgfqpoint{5.668862in}{1.323116in}}%
\pgfpathclose%
\pgfusepath{fill}%
\end{pgfscope}%
\begin{pgfscope}%
\pgfpathrectangle{\pgfqpoint{4.658921in}{0.208778in}}{\pgfqpoint{3.800000in}{3.800000in}}%
\pgfusepath{clip}%
\pgfsetbuttcap%
\pgfsetroundjoin%
\definecolor{currentfill}{rgb}{0.060435,0.219763,0.060435}%
\pgfsetfillcolor{currentfill}%
\pgfsetfillopacity{0.200000}%
\pgfsetlinewidth{0.000000pt}%
\definecolor{currentstroke}{rgb}{0.000000,0.000000,0.000000}%
\pgfsetstrokecolor{currentstroke}%
\pgfsetdash{}{0pt}%
\pgfpathmoveto{\pgfqpoint{7.481567in}{1.433246in}}%
\pgfpathlineto{\pgfqpoint{7.278577in}{1.238733in}}%
\pgfpathlineto{\pgfqpoint{7.551682in}{1.323116in}}%
\pgfpathlineto{\pgfqpoint{7.481567in}{1.433246in}}%
\pgfpathclose%
\pgfusepath{fill}%
\end{pgfscope}%
\begin{pgfscope}%
\pgfpathrectangle{\pgfqpoint{4.658921in}{0.208778in}}{\pgfqpoint{3.800000in}{3.800000in}}%
\pgfusepath{clip}%
\pgfsetbuttcap%
\pgfsetroundjoin%
\definecolor{currentfill}{rgb}{0.074506,0.270932,0.074506}%
\pgfsetfillcolor{currentfill}%
\pgfsetfillopacity{0.200000}%
\pgfsetlinewidth{0.000000pt}%
\definecolor{currentstroke}{rgb}{0.000000,0.000000,0.000000}%
\pgfsetstrokecolor{currentstroke}%
\pgfsetdash{}{0pt}%
\pgfpathmoveto{\pgfqpoint{5.271794in}{2.212598in}}%
\pgfpathlineto{\pgfqpoint{5.288610in}{1.620589in}}%
\pgfpathlineto{\pgfqpoint{5.497135in}{2.167772in}}%
\pgfpathlineto{\pgfqpoint{5.271794in}{2.212598in}}%
\pgfpathclose%
\pgfusepath{fill}%
\end{pgfscope}%
\begin{pgfscope}%
\pgfpathrectangle{\pgfqpoint{4.658921in}{0.208778in}}{\pgfqpoint{3.800000in}{3.800000in}}%
\pgfusepath{clip}%
\pgfsetbuttcap%
\pgfsetroundjoin%
\definecolor{currentfill}{rgb}{0.074506,0.270932,0.074506}%
\pgfsetfillcolor{currentfill}%
\pgfsetfillopacity{0.200000}%
\pgfsetlinewidth{0.000000pt}%
\definecolor{currentstroke}{rgb}{0.000000,0.000000,0.000000}%
\pgfsetstrokecolor{currentstroke}%
\pgfsetdash{}{0pt}%
\pgfpathmoveto{\pgfqpoint{7.723409in}{2.167772in}}%
\pgfpathlineto{\pgfqpoint{7.931934in}{1.620589in}}%
\pgfpathlineto{\pgfqpoint{7.948750in}{2.212598in}}%
\pgfpathlineto{\pgfqpoint{7.723409in}{2.167772in}}%
\pgfpathclose%
\pgfusepath{fill}%
\end{pgfscope}%
\begin{pgfscope}%
\pgfpathrectangle{\pgfqpoint{4.658921in}{0.208778in}}{\pgfqpoint{3.800000in}{3.800000in}}%
\pgfusepath{clip}%
\pgfsetbuttcap%
\pgfsetroundjoin%
\definecolor{currentfill}{rgb}{0.116785,0.424671,0.116785}%
\pgfsetfillcolor{currentfill}%
\pgfsetfillopacity{0.200000}%
\pgfsetlinewidth{0.000000pt}%
\definecolor{currentstroke}{rgb}{0.000000,0.000000,0.000000}%
\pgfsetstrokecolor{currentstroke}%
\pgfsetdash{}{0pt}%
\pgfpathmoveto{\pgfqpoint{7.280448in}{3.189189in}}%
\pgfpathlineto{\pgfqpoint{6.744224in}{3.661884in}}%
\pgfpathlineto{\pgfqpoint{6.802393in}{2.991915in}}%
\pgfpathlineto{\pgfqpoint{7.280448in}{3.189189in}}%
\pgfpathclose%
\pgfusepath{fill}%
\end{pgfscope}%
\begin{pgfscope}%
\pgfpathrectangle{\pgfqpoint{4.658921in}{0.208778in}}{\pgfqpoint{3.800000in}{3.800000in}}%
\pgfusepath{clip}%
\pgfsetbuttcap%
\pgfsetroundjoin%
\definecolor{currentfill}{rgb}{0.116785,0.424671,0.116785}%
\pgfsetfillcolor{currentfill}%
\pgfsetfillopacity{0.200000}%
\pgfsetlinewidth{0.000000pt}%
\definecolor{currentstroke}{rgb}{0.000000,0.000000,0.000000}%
\pgfsetstrokecolor{currentstroke}%
\pgfsetdash{}{0pt}%
\pgfpathmoveto{\pgfqpoint{6.418151in}{2.991915in}}%
\pgfpathlineto{\pgfqpoint{6.476320in}{3.661884in}}%
\pgfpathlineto{\pgfqpoint{5.940096in}{3.189189in}}%
\pgfpathlineto{\pgfqpoint{6.418151in}{2.991915in}}%
\pgfpathclose%
\pgfusepath{fill}%
\end{pgfscope}%
\begin{pgfscope}%
\pgfpathrectangle{\pgfqpoint{4.658921in}{0.208778in}}{\pgfqpoint{3.800000in}{3.800000in}}%
\pgfusepath{clip}%
\pgfsetbuttcap%
\pgfsetroundjoin%
\definecolor{currentfill}{rgb}{0.064867,0.235879,0.064867}%
\pgfsetfillcolor{currentfill}%
\pgfsetfillopacity{0.200000}%
\pgfsetlinewidth{0.000000pt}%
\definecolor{currentstroke}{rgb}{0.000000,0.000000,0.000000}%
\pgfsetstrokecolor{currentstroke}%
\pgfsetdash{}{0pt}%
\pgfpathmoveto{\pgfqpoint{5.484653in}{1.525226in}}%
\pgfpathlineto{\pgfqpoint{5.668862in}{1.323116in}}%
\pgfpathlineto{\pgfqpoint{5.799213in}{2.125336in}}%
\pgfpathlineto{\pgfqpoint{5.484653in}{1.525226in}}%
\pgfpathclose%
\pgfusepath{fill}%
\end{pgfscope}%
\begin{pgfscope}%
\pgfpathrectangle{\pgfqpoint{4.658921in}{0.208778in}}{\pgfqpoint{3.800000in}{3.800000in}}%
\pgfusepath{clip}%
\pgfsetbuttcap%
\pgfsetroundjoin%
\definecolor{currentfill}{rgb}{0.064867,0.235879,0.064867}%
\pgfsetfillcolor{currentfill}%
\pgfsetfillopacity{0.200000}%
\pgfsetlinewidth{0.000000pt}%
\definecolor{currentstroke}{rgb}{0.000000,0.000000,0.000000}%
\pgfsetstrokecolor{currentstroke}%
\pgfsetdash{}{0pt}%
\pgfpathmoveto{\pgfqpoint{7.421331in}{2.125336in}}%
\pgfpathlineto{\pgfqpoint{7.551682in}{1.323116in}}%
\pgfpathlineto{\pgfqpoint{7.735891in}{1.525226in}}%
\pgfpathlineto{\pgfqpoint{7.421331in}{2.125336in}}%
\pgfpathclose%
\pgfusepath{fill}%
\end{pgfscope}%
\begin{pgfscope}%
\pgfpathrectangle{\pgfqpoint{4.658921in}{0.208778in}}{\pgfqpoint{3.800000in}{3.800000in}}%
\pgfusepath{clip}%
\pgfsetbuttcap%
\pgfsetroundjoin%
\definecolor{currentfill}{rgb}{0.057724,0.209904,0.057724}%
\pgfsetfillcolor{currentfill}%
\pgfsetfillopacity{0.200000}%
\pgfsetlinewidth{0.000000pt}%
\definecolor{currentstroke}{rgb}{0.000000,0.000000,0.000000}%
\pgfsetstrokecolor{currentstroke}%
\pgfsetdash{}{0pt}%
\pgfpathmoveto{\pgfqpoint{6.054350in}{1.357725in}}%
\pgfpathlineto{\pgfqpoint{5.941967in}{1.238733in}}%
\pgfpathlineto{\pgfqpoint{6.261881in}{1.179689in}}%
\pgfpathlineto{\pgfqpoint{6.054350in}{1.357725in}}%
\pgfpathclose%
\pgfusepath{fill}%
\end{pgfscope}%
\begin{pgfscope}%
\pgfpathrectangle{\pgfqpoint{4.658921in}{0.208778in}}{\pgfqpoint{3.800000in}{3.800000in}}%
\pgfusepath{clip}%
\pgfsetbuttcap%
\pgfsetroundjoin%
\definecolor{currentfill}{rgb}{0.057724,0.209904,0.057724}%
\pgfsetfillcolor{currentfill}%
\pgfsetfillopacity{0.200000}%
\pgfsetlinewidth{0.000000pt}%
\definecolor{currentstroke}{rgb}{0.000000,0.000000,0.000000}%
\pgfsetstrokecolor{currentstroke}%
\pgfsetdash{}{0pt}%
\pgfpathmoveto{\pgfqpoint{6.958663in}{1.179689in}}%
\pgfpathlineto{\pgfqpoint{7.278577in}{1.238733in}}%
\pgfpathlineto{\pgfqpoint{7.166194in}{1.357725in}}%
\pgfpathlineto{\pgfqpoint{6.958663in}{1.179689in}}%
\pgfpathclose%
\pgfusepath{fill}%
\end{pgfscope}%
\begin{pgfscope}%
\pgfpathrectangle{\pgfqpoint{4.658921in}{0.208778in}}{\pgfqpoint{3.800000in}{3.800000in}}%
\pgfusepath{clip}%
\pgfsetbuttcap%
\pgfsetroundjoin%
\definecolor{currentfill}{rgb}{0.086498,0.314539,0.086498}%
\pgfsetfillcolor{currentfill}%
\pgfsetfillopacity{0.200000}%
\pgfsetlinewidth{0.000000pt}%
\definecolor{currentstroke}{rgb}{0.000000,0.000000,0.000000}%
\pgfsetstrokecolor{currentstroke}%
\pgfsetdash{}{0pt}%
\pgfpathmoveto{\pgfqpoint{7.948750in}{2.212598in}}%
\pgfpathlineto{\pgfqpoint{7.831863in}{2.449323in}}%
\pgfpathlineto{\pgfqpoint{7.723409in}{2.167772in}}%
\pgfpathlineto{\pgfqpoint{7.948750in}{2.212598in}}%
\pgfpathclose%
\pgfusepath{fill}%
\end{pgfscope}%
\begin{pgfscope}%
\pgfpathrectangle{\pgfqpoint{4.658921in}{0.208778in}}{\pgfqpoint{3.800000in}{3.800000in}}%
\pgfusepath{clip}%
\pgfsetbuttcap%
\pgfsetroundjoin%
\definecolor{currentfill}{rgb}{0.086498,0.314539,0.086498}%
\pgfsetfillcolor{currentfill}%
\pgfsetfillopacity{0.200000}%
\pgfsetlinewidth{0.000000pt}%
\definecolor{currentstroke}{rgb}{0.000000,0.000000,0.000000}%
\pgfsetstrokecolor{currentstroke}%
\pgfsetdash{}{0pt}%
\pgfpathmoveto{\pgfqpoint{5.497135in}{2.167772in}}%
\pgfpathlineto{\pgfqpoint{5.388681in}{2.449323in}}%
\pgfpathlineto{\pgfqpoint{5.271794in}{2.212598in}}%
\pgfpathlineto{\pgfqpoint{5.497135in}{2.167772in}}%
\pgfpathclose%
\pgfusepath{fill}%
\end{pgfscope}%
\begin{pgfscope}%
\pgfpathrectangle{\pgfqpoint{4.658921in}{0.208778in}}{\pgfqpoint{3.800000in}{3.800000in}}%
\pgfusepath{clip}%
\pgfsetbuttcap%
\pgfsetroundjoin%
\definecolor{currentfill}{rgb}{0.090812,0.330224,0.090812}%
\pgfsetfillcolor{currentfill}%
\pgfsetfillopacity{0.200000}%
\pgfsetlinewidth{0.000000pt}%
\definecolor{currentstroke}{rgb}{0.000000,0.000000,0.000000}%
\pgfsetstrokecolor{currentstroke}%
\pgfsetdash{}{0pt}%
\pgfpathmoveto{\pgfqpoint{5.799213in}{2.125336in}}%
\pgfpathlineto{\pgfqpoint{5.668862in}{1.323116in}}%
\pgfpathlineto{\pgfqpoint{5.738977in}{1.433246in}}%
\pgfpathlineto{\pgfqpoint{5.799213in}{2.125336in}}%
\pgfpathclose%
\pgfusepath{fill}%
\end{pgfscope}%
\begin{pgfscope}%
\pgfpathrectangle{\pgfqpoint{4.658921in}{0.208778in}}{\pgfqpoint{3.800000in}{3.800000in}}%
\pgfusepath{clip}%
\pgfsetbuttcap%
\pgfsetroundjoin%
\definecolor{currentfill}{rgb}{0.090812,0.330224,0.090812}%
\pgfsetfillcolor{currentfill}%
\pgfsetfillopacity{0.200000}%
\pgfsetlinewidth{0.000000pt}%
\definecolor{currentstroke}{rgb}{0.000000,0.000000,0.000000}%
\pgfsetstrokecolor{currentstroke}%
\pgfsetdash{}{0pt}%
\pgfpathmoveto{\pgfqpoint{7.481567in}{1.433246in}}%
\pgfpathlineto{\pgfqpoint{7.551682in}{1.323116in}}%
\pgfpathlineto{\pgfqpoint{7.421331in}{2.125336in}}%
\pgfpathlineto{\pgfqpoint{7.481567in}{1.433246in}}%
\pgfpathclose%
\pgfusepath{fill}%
\end{pgfscope}%
\begin{pgfscope}%
\pgfpathrectangle{\pgfqpoint{4.658921in}{0.208778in}}{\pgfqpoint{3.800000in}{3.800000in}}%
\pgfusepath{clip}%
\pgfsetbuttcap%
\pgfsetroundjoin%
\definecolor{currentfill}{rgb}{0.116321,0.422987,0.116321}%
\pgfsetfillcolor{currentfill}%
\pgfsetfillopacity{0.200000}%
\pgfsetlinewidth{0.000000pt}%
\definecolor{currentstroke}{rgb}{0.000000,0.000000,0.000000}%
\pgfsetstrokecolor{currentstroke}%
\pgfsetdash{}{0pt}%
\pgfpathmoveto{\pgfqpoint{6.418151in}{2.991915in}}%
\pgfpathlineto{\pgfqpoint{6.802393in}{2.991915in}}%
\pgfpathlineto{\pgfqpoint{6.744224in}{3.661884in}}%
\pgfpathlineto{\pgfqpoint{6.418151in}{2.991915in}}%
\pgfpathclose%
\pgfusepath{fill}%
\end{pgfscope}%
\begin{pgfscope}%
\pgfpathrectangle{\pgfqpoint{4.658921in}{0.208778in}}{\pgfqpoint{3.800000in}{3.800000in}}%
\pgfusepath{clip}%
\pgfsetbuttcap%
\pgfsetroundjoin%
\definecolor{currentfill}{rgb}{0.056200,0.204363,0.056200}%
\pgfsetfillcolor{currentfill}%
\pgfsetfillopacity{0.200000}%
\pgfsetlinewidth{0.000000pt}%
\definecolor{currentstroke}{rgb}{0.000000,0.000000,0.000000}%
\pgfsetstrokecolor{currentstroke}%
\pgfsetdash{}{0pt}%
\pgfpathmoveto{\pgfqpoint{6.610272in}{1.158306in}}%
\pgfpathlineto{\pgfqpoint{6.418612in}{1.314303in}}%
\pgfpathlineto{\pgfqpoint{6.261881in}{1.179689in}}%
\pgfpathlineto{\pgfqpoint{6.610272in}{1.158306in}}%
\pgfpathclose%
\pgfusepath{fill}%
\end{pgfscope}%
\begin{pgfscope}%
\pgfpathrectangle{\pgfqpoint{4.658921in}{0.208778in}}{\pgfqpoint{3.800000in}{3.800000in}}%
\pgfusepath{clip}%
\pgfsetbuttcap%
\pgfsetroundjoin%
\definecolor{currentfill}{rgb}{0.056200,0.204363,0.056200}%
\pgfsetfillcolor{currentfill}%
\pgfsetfillopacity{0.200000}%
\pgfsetlinewidth{0.000000pt}%
\definecolor{currentstroke}{rgb}{0.000000,0.000000,0.000000}%
\pgfsetstrokecolor{currentstroke}%
\pgfsetdash{}{0pt}%
\pgfpathmoveto{\pgfqpoint{6.958663in}{1.179689in}}%
\pgfpathlineto{\pgfqpoint{6.801932in}{1.314303in}}%
\pgfpathlineto{\pgfqpoint{6.610272in}{1.158306in}}%
\pgfpathlineto{\pgfqpoint{6.958663in}{1.179689in}}%
\pgfpathclose%
\pgfusepath{fill}%
\end{pgfscope}%
\begin{pgfscope}%
\pgfpathrectangle{\pgfqpoint{4.658921in}{0.208778in}}{\pgfqpoint{3.800000in}{3.800000in}}%
\pgfusepath{clip}%
\pgfsetbuttcap%
\pgfsetroundjoin%
\definecolor{currentfill}{rgb}{0.107070,0.389346,0.107070}%
\pgfsetfillcolor{currentfill}%
\pgfsetfillopacity{0.200000}%
\pgfsetlinewidth{0.000000pt}%
\definecolor{currentstroke}{rgb}{0.000000,0.000000,0.000000}%
\pgfsetstrokecolor{currentstroke}%
\pgfsetdash{}{0pt}%
\pgfpathmoveto{\pgfqpoint{7.483473in}{2.958473in}}%
\pgfpathlineto{\pgfqpoint{7.280448in}{3.189189in}}%
\pgfpathlineto{\pgfqpoint{7.167488in}{2.979704in}}%
\pgfpathlineto{\pgfqpoint{7.483473in}{2.958473in}}%
\pgfpathclose%
\pgfusepath{fill}%
\end{pgfscope}%
\begin{pgfscope}%
\pgfpathrectangle{\pgfqpoint{4.658921in}{0.208778in}}{\pgfqpoint{3.800000in}{3.800000in}}%
\pgfusepath{clip}%
\pgfsetbuttcap%
\pgfsetroundjoin%
\definecolor{currentfill}{rgb}{0.107070,0.389346,0.107070}%
\pgfsetfillcolor{currentfill}%
\pgfsetfillopacity{0.200000}%
\pgfsetlinewidth{0.000000pt}%
\definecolor{currentstroke}{rgb}{0.000000,0.000000,0.000000}%
\pgfsetstrokecolor{currentstroke}%
\pgfsetdash{}{0pt}%
\pgfpathmoveto{\pgfqpoint{6.053056in}{2.979704in}}%
\pgfpathlineto{\pgfqpoint{5.940096in}{3.189189in}}%
\pgfpathlineto{\pgfqpoint{5.737071in}{2.958473in}}%
\pgfpathlineto{\pgfqpoint{6.053056in}{2.979704in}}%
\pgfpathclose%
\pgfusepath{fill}%
\end{pgfscope}%
\begin{pgfscope}%
\pgfpathrectangle{\pgfqpoint{4.658921in}{0.208778in}}{\pgfqpoint{3.800000in}{3.800000in}}%
\pgfusepath{clip}%
\pgfsetbuttcap%
\pgfsetroundjoin%
\definecolor{currentfill}{rgb}{0.089078,0.323920,0.089078}%
\pgfsetfillcolor{currentfill}%
\pgfsetfillopacity{0.200000}%
\pgfsetlinewidth{0.000000pt}%
\definecolor{currentstroke}{rgb}{0.000000,0.000000,0.000000}%
\pgfsetstrokecolor{currentstroke}%
\pgfsetdash{}{0pt}%
\pgfpathmoveto{\pgfqpoint{5.497135in}{2.167772in}}%
\pgfpathlineto{\pgfqpoint{5.737071in}{2.958473in}}%
\pgfpathlineto{\pgfqpoint{5.388681in}{2.449323in}}%
\pgfpathlineto{\pgfqpoint{5.497135in}{2.167772in}}%
\pgfpathclose%
\pgfusepath{fill}%
\end{pgfscope}%
\begin{pgfscope}%
\pgfpathrectangle{\pgfqpoint{4.658921in}{0.208778in}}{\pgfqpoint{3.800000in}{3.800000in}}%
\pgfusepath{clip}%
\pgfsetbuttcap%
\pgfsetroundjoin%
\definecolor{currentfill}{rgb}{0.089078,0.323920,0.089078}%
\pgfsetfillcolor{currentfill}%
\pgfsetfillopacity{0.200000}%
\pgfsetlinewidth{0.000000pt}%
\definecolor{currentstroke}{rgb}{0.000000,0.000000,0.000000}%
\pgfsetstrokecolor{currentstroke}%
\pgfsetdash{}{0pt}%
\pgfpathmoveto{\pgfqpoint{7.831863in}{2.449323in}}%
\pgfpathlineto{\pgfqpoint{7.483473in}{2.958473in}}%
\pgfpathlineto{\pgfqpoint{7.723409in}{2.167772in}}%
\pgfpathlineto{\pgfqpoint{7.831863in}{2.449323in}}%
\pgfpathclose%
\pgfusepath{fill}%
\end{pgfscope}%
\begin{pgfscope}%
\pgfpathrectangle{\pgfqpoint{4.658921in}{0.208778in}}{\pgfqpoint{3.800000in}{3.800000in}}%
\pgfusepath{clip}%
\pgfsetbuttcap%
\pgfsetroundjoin%
\definecolor{currentfill}{rgb}{0.061754,0.224559,0.061754}%
\pgfsetfillcolor{currentfill}%
\pgfsetfillopacity{0.200000}%
\pgfsetlinewidth{0.000000pt}%
\definecolor{currentstroke}{rgb}{0.000000,0.000000,0.000000}%
\pgfsetstrokecolor{currentstroke}%
\pgfsetdash{}{0pt}%
\pgfpathmoveto{\pgfqpoint{5.738977in}{1.433246in}}%
\pgfpathlineto{\pgfqpoint{5.941967in}{1.238733in}}%
\pgfpathlineto{\pgfqpoint{5.990110in}{1.814186in}}%
\pgfpathlineto{\pgfqpoint{5.738977in}{1.433246in}}%
\pgfpathclose%
\pgfusepath{fill}%
\end{pgfscope}%
\begin{pgfscope}%
\pgfpathrectangle{\pgfqpoint{4.658921in}{0.208778in}}{\pgfqpoint{3.800000in}{3.800000in}}%
\pgfusepath{clip}%
\pgfsetbuttcap%
\pgfsetroundjoin%
\definecolor{currentfill}{rgb}{0.061754,0.224559,0.061754}%
\pgfsetfillcolor{currentfill}%
\pgfsetfillopacity{0.200000}%
\pgfsetlinewidth{0.000000pt}%
\definecolor{currentstroke}{rgb}{0.000000,0.000000,0.000000}%
\pgfsetstrokecolor{currentstroke}%
\pgfsetdash{}{0pt}%
\pgfpathmoveto{\pgfqpoint{7.230434in}{1.814186in}}%
\pgfpathlineto{\pgfqpoint{7.278577in}{1.238733in}}%
\pgfpathlineto{\pgfqpoint{7.481567in}{1.433246in}}%
\pgfpathlineto{\pgfqpoint{7.230434in}{1.814186in}}%
\pgfpathclose%
\pgfusepath{fill}%
\end{pgfscope}%
\begin{pgfscope}%
\pgfpathrectangle{\pgfqpoint{4.658921in}{0.208778in}}{\pgfqpoint{3.800000in}{3.800000in}}%
\pgfusepath{clip}%
\pgfsetbuttcap%
\pgfsetroundjoin%
\definecolor{currentfill}{rgb}{0.070984,0.258123,0.070984}%
\pgfsetfillcolor{currentfill}%
\pgfsetfillopacity{0.200000}%
\pgfsetlinewidth{0.000000pt}%
\definecolor{currentstroke}{rgb}{0.000000,0.000000,0.000000}%
\pgfsetstrokecolor{currentstroke}%
\pgfsetdash{}{0pt}%
\pgfpathmoveto{\pgfqpoint{5.497135in}{2.167772in}}%
\pgfpathlineto{\pgfqpoint{5.484653in}{1.525226in}}%
\pgfpathlineto{\pgfqpoint{5.799213in}{2.125336in}}%
\pgfpathlineto{\pgfqpoint{5.497135in}{2.167772in}}%
\pgfpathclose%
\pgfusepath{fill}%
\end{pgfscope}%
\begin{pgfscope}%
\pgfpathrectangle{\pgfqpoint{4.658921in}{0.208778in}}{\pgfqpoint{3.800000in}{3.800000in}}%
\pgfusepath{clip}%
\pgfsetbuttcap%
\pgfsetroundjoin%
\definecolor{currentfill}{rgb}{0.070984,0.258123,0.070984}%
\pgfsetfillcolor{currentfill}%
\pgfsetfillopacity{0.200000}%
\pgfsetlinewidth{0.000000pt}%
\definecolor{currentstroke}{rgb}{0.000000,0.000000,0.000000}%
\pgfsetstrokecolor{currentstroke}%
\pgfsetdash{}{0pt}%
\pgfpathmoveto{\pgfqpoint{7.421331in}{2.125336in}}%
\pgfpathlineto{\pgfqpoint{7.735891in}{1.525226in}}%
\pgfpathlineto{\pgfqpoint{7.723409in}{2.167772in}}%
\pgfpathlineto{\pgfqpoint{7.421331in}{2.125336in}}%
\pgfpathclose%
\pgfusepath{fill}%
\end{pgfscope}%
\begin{pgfscope}%
\pgfpathrectangle{\pgfqpoint{4.658921in}{0.208778in}}{\pgfqpoint{3.800000in}{3.800000in}}%
\pgfusepath{clip}%
\pgfsetbuttcap%
\pgfsetroundjoin%
\definecolor{currentfill}{rgb}{0.070885,0.257762,0.070885}%
\pgfsetfillcolor{currentfill}%
\pgfsetfillopacity{0.200000}%
\pgfsetlinewidth{0.000000pt}%
\definecolor{currentstroke}{rgb}{0.000000,0.000000,0.000000}%
\pgfsetstrokecolor{currentstroke}%
\pgfsetdash{}{0pt}%
\pgfpathmoveto{\pgfqpoint{5.990110in}{1.814186in}}%
\pgfpathlineto{\pgfqpoint{5.941967in}{1.238733in}}%
\pgfpathlineto{\pgfqpoint{6.054350in}{1.357725in}}%
\pgfpathlineto{\pgfqpoint{5.990110in}{1.814186in}}%
\pgfpathclose%
\pgfusepath{fill}%
\end{pgfscope}%
\begin{pgfscope}%
\pgfpathrectangle{\pgfqpoint{4.658921in}{0.208778in}}{\pgfqpoint{3.800000in}{3.800000in}}%
\pgfusepath{clip}%
\pgfsetbuttcap%
\pgfsetroundjoin%
\definecolor{currentfill}{rgb}{0.070885,0.257762,0.070885}%
\pgfsetfillcolor{currentfill}%
\pgfsetfillopacity{0.200000}%
\pgfsetlinewidth{0.000000pt}%
\definecolor{currentstroke}{rgb}{0.000000,0.000000,0.000000}%
\pgfsetstrokecolor{currentstroke}%
\pgfsetdash{}{0pt}%
\pgfpathmoveto{\pgfqpoint{7.166194in}{1.357725in}}%
\pgfpathlineto{\pgfqpoint{7.278577in}{1.238733in}}%
\pgfpathlineto{\pgfqpoint{7.230434in}{1.814186in}}%
\pgfpathlineto{\pgfqpoint{7.166194in}{1.357725in}}%
\pgfpathclose%
\pgfusepath{fill}%
\end{pgfscope}%
\begin{pgfscope}%
\pgfpathrectangle{\pgfqpoint{4.658921in}{0.208778in}}{\pgfqpoint{3.800000in}{3.800000in}}%
\pgfusepath{clip}%
\pgfsetbuttcap%
\pgfsetroundjoin%
\definecolor{currentfill}{rgb}{0.111651,0.406004,0.111651}%
\pgfsetfillcolor{currentfill}%
\pgfsetfillopacity{0.200000}%
\pgfsetlinewidth{0.000000pt}%
\definecolor{currentstroke}{rgb}{0.000000,0.000000,0.000000}%
\pgfsetstrokecolor{currentstroke}%
\pgfsetdash{}{0pt}%
\pgfpathmoveto{\pgfqpoint{7.167488in}{2.979704in}}%
\pgfpathlineto{\pgfqpoint{7.280448in}{3.189189in}}%
\pgfpathlineto{\pgfqpoint{6.802393in}{2.991915in}}%
\pgfpathlineto{\pgfqpoint{7.167488in}{2.979704in}}%
\pgfpathclose%
\pgfusepath{fill}%
\end{pgfscope}%
\begin{pgfscope}%
\pgfpathrectangle{\pgfqpoint{4.658921in}{0.208778in}}{\pgfqpoint{3.800000in}{3.800000in}}%
\pgfusepath{clip}%
\pgfsetbuttcap%
\pgfsetroundjoin%
\definecolor{currentfill}{rgb}{0.111651,0.406004,0.111651}%
\pgfsetfillcolor{currentfill}%
\pgfsetfillopacity{0.200000}%
\pgfsetlinewidth{0.000000pt}%
\definecolor{currentstroke}{rgb}{0.000000,0.000000,0.000000}%
\pgfsetstrokecolor{currentstroke}%
\pgfsetdash{}{0pt}%
\pgfpathmoveto{\pgfqpoint{6.418151in}{2.991915in}}%
\pgfpathlineto{\pgfqpoint{5.940096in}{3.189189in}}%
\pgfpathlineto{\pgfqpoint{6.053056in}{2.979704in}}%
\pgfpathlineto{\pgfqpoint{6.418151in}{2.991915in}}%
\pgfpathclose%
\pgfusepath{fill}%
\end{pgfscope}%
\begin{pgfscope}%
\pgfpathrectangle{\pgfqpoint{4.658921in}{0.208778in}}{\pgfqpoint{3.800000in}{3.800000in}}%
\pgfusepath{clip}%
\pgfsetbuttcap%
\pgfsetroundjoin%
\definecolor{currentfill}{rgb}{0.092193,0.335248,0.092193}%
\pgfsetfillcolor{currentfill}%
\pgfsetfillopacity{0.200000}%
\pgfsetlinewidth{0.000000pt}%
\definecolor{currentstroke}{rgb}{0.000000,0.000000,0.000000}%
\pgfsetstrokecolor{currentstroke}%
\pgfsetdash{}{0pt}%
\pgfpathmoveto{\pgfqpoint{5.799213in}{2.125336in}}%
\pgfpathlineto{\pgfqpoint{5.737071in}{2.958473in}}%
\pgfpathlineto{\pgfqpoint{5.497135in}{2.167772in}}%
\pgfpathlineto{\pgfqpoint{5.799213in}{2.125336in}}%
\pgfpathclose%
\pgfusepath{fill}%
\end{pgfscope}%
\begin{pgfscope}%
\pgfpathrectangle{\pgfqpoint{4.658921in}{0.208778in}}{\pgfqpoint{3.800000in}{3.800000in}}%
\pgfusepath{clip}%
\pgfsetbuttcap%
\pgfsetroundjoin%
\definecolor{currentfill}{rgb}{0.092193,0.335248,0.092193}%
\pgfsetfillcolor{currentfill}%
\pgfsetfillopacity{0.200000}%
\pgfsetlinewidth{0.000000pt}%
\definecolor{currentstroke}{rgb}{0.000000,0.000000,0.000000}%
\pgfsetstrokecolor{currentstroke}%
\pgfsetdash{}{0pt}%
\pgfpathmoveto{\pgfqpoint{7.723409in}{2.167772in}}%
\pgfpathlineto{\pgfqpoint{7.483473in}{2.958473in}}%
\pgfpathlineto{\pgfqpoint{7.421331in}{2.125336in}}%
\pgfpathlineto{\pgfqpoint{7.723409in}{2.167772in}}%
\pgfpathclose%
\pgfusepath{fill}%
\end{pgfscope}%
\begin{pgfscope}%
\pgfpathrectangle{\pgfqpoint{4.658921in}{0.208778in}}{\pgfqpoint{3.800000in}{3.800000in}}%
\pgfusepath{clip}%
\pgfsetbuttcap%
\pgfsetroundjoin%
\definecolor{currentfill}{rgb}{0.060562,0.220227,0.060562}%
\pgfsetfillcolor{currentfill}%
\pgfsetfillopacity{0.200000}%
\pgfsetlinewidth{0.000000pt}%
\definecolor{currentstroke}{rgb}{0.000000,0.000000,0.000000}%
\pgfsetstrokecolor{currentstroke}%
\pgfsetdash{}{0pt}%
\pgfpathmoveto{\pgfqpoint{6.261881in}{1.179689in}}%
\pgfpathlineto{\pgfqpoint{6.394709in}{1.779529in}}%
\pgfpathlineto{\pgfqpoint{6.054350in}{1.357725in}}%
\pgfpathlineto{\pgfqpoint{6.261881in}{1.179689in}}%
\pgfpathclose%
\pgfusepath{fill}%
\end{pgfscope}%
\begin{pgfscope}%
\pgfpathrectangle{\pgfqpoint{4.658921in}{0.208778in}}{\pgfqpoint{3.800000in}{3.800000in}}%
\pgfusepath{clip}%
\pgfsetbuttcap%
\pgfsetroundjoin%
\definecolor{currentfill}{rgb}{0.060562,0.220227,0.060562}%
\pgfsetfillcolor{currentfill}%
\pgfsetfillopacity{0.200000}%
\pgfsetlinewidth{0.000000pt}%
\definecolor{currentstroke}{rgb}{0.000000,0.000000,0.000000}%
\pgfsetstrokecolor{currentstroke}%
\pgfsetdash{}{0pt}%
\pgfpathmoveto{\pgfqpoint{7.166194in}{1.357725in}}%
\pgfpathlineto{\pgfqpoint{6.825835in}{1.779529in}}%
\pgfpathlineto{\pgfqpoint{6.958663in}{1.179689in}}%
\pgfpathlineto{\pgfqpoint{7.166194in}{1.357725in}}%
\pgfpathclose%
\pgfusepath{fill}%
\end{pgfscope}%
\begin{pgfscope}%
\pgfpathrectangle{\pgfqpoint{4.658921in}{0.208778in}}{\pgfqpoint{3.800000in}{3.800000in}}%
\pgfusepath{clip}%
\pgfsetbuttcap%
\pgfsetroundjoin%
\definecolor{currentfill}{rgb}{0.097285,0.353762,0.097285}%
\pgfsetfillcolor{currentfill}%
\pgfsetfillopacity{0.200000}%
\pgfsetlinewidth{0.000000pt}%
\definecolor{currentstroke}{rgb}{0.000000,0.000000,0.000000}%
\pgfsetstrokecolor{currentstroke}%
\pgfsetdash{}{0pt}%
\pgfpathmoveto{\pgfqpoint{6.053056in}{2.979704in}}%
\pgfpathlineto{\pgfqpoint{5.737071in}{2.958473in}}%
\pgfpathlineto{\pgfqpoint{6.204399in}{2.716708in}}%
\pgfpathlineto{\pgfqpoint{6.053056in}{2.979704in}}%
\pgfpathclose%
\pgfusepath{fill}%
\end{pgfscope}%
\begin{pgfscope}%
\pgfpathrectangle{\pgfqpoint{4.658921in}{0.208778in}}{\pgfqpoint{3.800000in}{3.800000in}}%
\pgfusepath{clip}%
\pgfsetbuttcap%
\pgfsetroundjoin%
\definecolor{currentfill}{rgb}{0.097285,0.353762,0.097285}%
\pgfsetfillcolor{currentfill}%
\pgfsetfillopacity{0.200000}%
\pgfsetlinewidth{0.000000pt}%
\definecolor{currentstroke}{rgb}{0.000000,0.000000,0.000000}%
\pgfsetstrokecolor{currentstroke}%
\pgfsetdash{}{0pt}%
\pgfpathmoveto{\pgfqpoint{7.016145in}{2.716708in}}%
\pgfpathlineto{\pgfqpoint{7.483473in}{2.958473in}}%
\pgfpathlineto{\pgfqpoint{7.167488in}{2.979704in}}%
\pgfpathlineto{\pgfqpoint{7.016145in}{2.716708in}}%
\pgfpathclose%
\pgfusepath{fill}%
\end{pgfscope}%
\begin{pgfscope}%
\pgfpathrectangle{\pgfqpoint{4.658921in}{0.208778in}}{\pgfqpoint{3.800000in}{3.800000in}}%
\pgfusepath{clip}%
\pgfsetbuttcap%
\pgfsetroundjoin%
\definecolor{currentfill}{rgb}{0.067497,0.245443,0.067497}%
\pgfsetfillcolor{currentfill}%
\pgfsetfillopacity{0.200000}%
\pgfsetlinewidth{0.000000pt}%
\definecolor{currentstroke}{rgb}{0.000000,0.000000,0.000000}%
\pgfsetstrokecolor{currentstroke}%
\pgfsetdash{}{0pt}%
\pgfpathmoveto{\pgfqpoint{6.825835in}{1.779529in}}%
\pgfpathlineto{\pgfqpoint{6.801932in}{1.314303in}}%
\pgfpathlineto{\pgfqpoint{6.958663in}{1.179689in}}%
\pgfpathlineto{\pgfqpoint{6.825835in}{1.779529in}}%
\pgfpathclose%
\pgfusepath{fill}%
\end{pgfscope}%
\begin{pgfscope}%
\pgfpathrectangle{\pgfqpoint{4.658921in}{0.208778in}}{\pgfqpoint{3.800000in}{3.800000in}}%
\pgfusepath{clip}%
\pgfsetbuttcap%
\pgfsetroundjoin%
\definecolor{currentfill}{rgb}{0.067497,0.245443,0.067497}%
\pgfsetfillcolor{currentfill}%
\pgfsetfillopacity{0.200000}%
\pgfsetlinewidth{0.000000pt}%
\definecolor{currentstroke}{rgb}{0.000000,0.000000,0.000000}%
\pgfsetstrokecolor{currentstroke}%
\pgfsetdash{}{0pt}%
\pgfpathmoveto{\pgfqpoint{6.261881in}{1.179689in}}%
\pgfpathlineto{\pgfqpoint{6.418612in}{1.314303in}}%
\pgfpathlineto{\pgfqpoint{6.394709in}{1.779529in}}%
\pgfpathlineto{\pgfqpoint{6.261881in}{1.179689in}}%
\pgfpathclose%
\pgfusepath{fill}%
\end{pgfscope}%
\begin{pgfscope}%
\pgfpathrectangle{\pgfqpoint{4.658921in}{0.208778in}}{\pgfqpoint{3.800000in}{3.800000in}}%
\pgfusepath{clip}%
\pgfsetbuttcap%
\pgfsetroundjoin%
\definecolor{currentfill}{rgb}{0.065434,0.237940,0.065434}%
\pgfsetfillcolor{currentfill}%
\pgfsetfillopacity{0.200000}%
\pgfsetlinewidth{0.000000pt}%
\definecolor{currentstroke}{rgb}{0.000000,0.000000,0.000000}%
\pgfsetstrokecolor{currentstroke}%
\pgfsetdash{}{0pt}%
\pgfpathmoveto{\pgfqpoint{6.801932in}{1.314303in}}%
\pgfpathlineto{\pgfqpoint{6.825835in}{1.779529in}}%
\pgfpathlineto{\pgfqpoint{6.610272in}{1.158306in}}%
\pgfpathlineto{\pgfqpoint{6.801932in}{1.314303in}}%
\pgfpathclose%
\pgfusepath{fill}%
\end{pgfscope}%
\begin{pgfscope}%
\pgfpathrectangle{\pgfqpoint{4.658921in}{0.208778in}}{\pgfqpoint{3.800000in}{3.800000in}}%
\pgfusepath{clip}%
\pgfsetbuttcap%
\pgfsetroundjoin%
\definecolor{currentfill}{rgb}{0.065434,0.237940,0.065434}%
\pgfsetfillcolor{currentfill}%
\pgfsetfillopacity{0.200000}%
\pgfsetlinewidth{0.000000pt}%
\definecolor{currentstroke}{rgb}{0.000000,0.000000,0.000000}%
\pgfsetstrokecolor{currentstroke}%
\pgfsetdash{}{0pt}%
\pgfpathmoveto{\pgfqpoint{6.610272in}{1.158306in}}%
\pgfpathlineto{\pgfqpoint{6.394709in}{1.779529in}}%
\pgfpathlineto{\pgfqpoint{6.418612in}{1.314303in}}%
\pgfpathlineto{\pgfqpoint{6.610272in}{1.158306in}}%
\pgfpathclose%
\pgfusepath{fill}%
\end{pgfscope}%
\begin{pgfscope}%
\pgfpathrectangle{\pgfqpoint{4.658921in}{0.208778in}}{\pgfqpoint{3.800000in}{3.800000in}}%
\pgfusepath{clip}%
\pgfsetbuttcap%
\pgfsetroundjoin%
\definecolor{currentfill}{rgb}{0.073593,0.267612,0.073593}%
\pgfsetfillcolor{currentfill}%
\pgfsetfillopacity{0.200000}%
\pgfsetlinewidth{0.000000pt}%
\definecolor{currentstroke}{rgb}{0.000000,0.000000,0.000000}%
\pgfsetstrokecolor{currentstroke}%
\pgfsetdash{}{0pt}%
\pgfpathmoveto{\pgfqpoint{5.738977in}{1.433246in}}%
\pgfpathlineto{\pgfqpoint{5.990110in}{1.814186in}}%
\pgfpathlineto{\pgfqpoint{5.799213in}{2.125336in}}%
\pgfpathlineto{\pgfqpoint{5.738977in}{1.433246in}}%
\pgfpathclose%
\pgfusepath{fill}%
\end{pgfscope}%
\begin{pgfscope}%
\pgfpathrectangle{\pgfqpoint{4.658921in}{0.208778in}}{\pgfqpoint{3.800000in}{3.800000in}}%
\pgfusepath{clip}%
\pgfsetbuttcap%
\pgfsetroundjoin%
\definecolor{currentfill}{rgb}{0.073593,0.267612,0.073593}%
\pgfsetfillcolor{currentfill}%
\pgfsetfillopacity{0.200000}%
\pgfsetlinewidth{0.000000pt}%
\definecolor{currentstroke}{rgb}{0.000000,0.000000,0.000000}%
\pgfsetstrokecolor{currentstroke}%
\pgfsetdash{}{0pt}%
\pgfpathmoveto{\pgfqpoint{7.421331in}{2.125336in}}%
\pgfpathlineto{\pgfqpoint{7.230434in}{1.814186in}}%
\pgfpathlineto{\pgfqpoint{7.481567in}{1.433246in}}%
\pgfpathlineto{\pgfqpoint{7.421331in}{2.125336in}}%
\pgfpathclose%
\pgfusepath{fill}%
\end{pgfscope}%
\begin{pgfscope}%
\pgfpathrectangle{\pgfqpoint{4.658921in}{0.208778in}}{\pgfqpoint{3.800000in}{3.800000in}}%
\pgfusepath{clip}%
\pgfsetbuttcap%
\pgfsetroundjoin%
\definecolor{currentfill}{rgb}{0.091915,0.334238,0.091915}%
\pgfsetfillcolor{currentfill}%
\pgfsetfillopacity{0.200000}%
\pgfsetlinewidth{0.000000pt}%
\definecolor{currentstroke}{rgb}{0.000000,0.000000,0.000000}%
\pgfsetstrokecolor{currentstroke}%
\pgfsetdash{}{0pt}%
\pgfpathmoveto{\pgfqpoint{5.799213in}{2.125336in}}%
\pgfpathlineto{\pgfqpoint{6.204399in}{2.716708in}}%
\pgfpathlineto{\pgfqpoint{5.737071in}{2.958473in}}%
\pgfpathlineto{\pgfqpoint{5.799213in}{2.125336in}}%
\pgfpathclose%
\pgfusepath{fill}%
\end{pgfscope}%
\begin{pgfscope}%
\pgfpathrectangle{\pgfqpoint{4.658921in}{0.208778in}}{\pgfqpoint{3.800000in}{3.800000in}}%
\pgfusepath{clip}%
\pgfsetbuttcap%
\pgfsetroundjoin%
\definecolor{currentfill}{rgb}{0.091915,0.334238,0.091915}%
\pgfsetfillcolor{currentfill}%
\pgfsetfillopacity{0.200000}%
\pgfsetlinewidth{0.000000pt}%
\definecolor{currentstroke}{rgb}{0.000000,0.000000,0.000000}%
\pgfsetstrokecolor{currentstroke}%
\pgfsetdash{}{0pt}%
\pgfpathmoveto{\pgfqpoint{7.483473in}{2.958473in}}%
\pgfpathlineto{\pgfqpoint{7.016145in}{2.716708in}}%
\pgfpathlineto{\pgfqpoint{7.421331in}{2.125336in}}%
\pgfpathlineto{\pgfqpoint{7.483473in}{2.958473in}}%
\pgfpathclose%
\pgfusepath{fill}%
\end{pgfscope}%
\begin{pgfscope}%
\pgfpathrectangle{\pgfqpoint{4.658921in}{0.208778in}}{\pgfqpoint{3.800000in}{3.800000in}}%
\pgfusepath{clip}%
\pgfsetbuttcap%
\pgfsetroundjoin%
\definecolor{currentfill}{rgb}{0.101759,0.370033,0.101759}%
\pgfsetfillcolor{currentfill}%
\pgfsetfillopacity{0.200000}%
\pgfsetlinewidth{0.000000pt}%
\definecolor{currentstroke}{rgb}{0.000000,0.000000,0.000000}%
\pgfsetstrokecolor{currentstroke}%
\pgfsetdash{}{0pt}%
\pgfpathmoveto{\pgfqpoint{6.204399in}{2.716708in}}%
\pgfpathlineto{\pgfqpoint{6.418151in}{2.991915in}}%
\pgfpathlineto{\pgfqpoint{6.053056in}{2.979704in}}%
\pgfpathlineto{\pgfqpoint{6.204399in}{2.716708in}}%
\pgfpathclose%
\pgfusepath{fill}%
\end{pgfscope}%
\begin{pgfscope}%
\pgfpathrectangle{\pgfqpoint{4.658921in}{0.208778in}}{\pgfqpoint{3.800000in}{3.800000in}}%
\pgfusepath{clip}%
\pgfsetbuttcap%
\pgfsetroundjoin%
\definecolor{currentfill}{rgb}{0.101759,0.370033,0.101759}%
\pgfsetfillcolor{currentfill}%
\pgfsetfillopacity{0.200000}%
\pgfsetlinewidth{0.000000pt}%
\definecolor{currentstroke}{rgb}{0.000000,0.000000,0.000000}%
\pgfsetstrokecolor{currentstroke}%
\pgfsetdash{}{0pt}%
\pgfpathmoveto{\pgfqpoint{7.167488in}{2.979704in}}%
\pgfpathlineto{\pgfqpoint{6.802393in}{2.991915in}}%
\pgfpathlineto{\pgfqpoint{7.016145in}{2.716708in}}%
\pgfpathlineto{\pgfqpoint{7.167488in}{2.979704in}}%
\pgfpathclose%
\pgfusepath{fill}%
\end{pgfscope}%
\begin{pgfscope}%
\pgfpathrectangle{\pgfqpoint{4.658921in}{0.208778in}}{\pgfqpoint{3.800000in}{3.800000in}}%
\pgfusepath{clip}%
\pgfsetbuttcap%
\pgfsetroundjoin%
\definecolor{currentfill}{rgb}{0.101677,0.369734,0.101677}%
\pgfsetfillcolor{currentfill}%
\pgfsetfillopacity{0.200000}%
\pgfsetlinewidth{0.000000pt}%
\definecolor{currentstroke}{rgb}{0.000000,0.000000,0.000000}%
\pgfsetstrokecolor{currentstroke}%
\pgfsetdash{}{0pt}%
\pgfpathmoveto{\pgfqpoint{6.610272in}{2.718551in}}%
\pgfpathlineto{\pgfqpoint{6.802393in}{2.991915in}}%
\pgfpathlineto{\pgfqpoint{6.418151in}{2.991915in}}%
\pgfpathlineto{\pgfqpoint{6.610272in}{2.718551in}}%
\pgfpathclose%
\pgfusepath{fill}%
\end{pgfscope}%
\begin{pgfscope}%
\pgfpathrectangle{\pgfqpoint{4.658921in}{0.208778in}}{\pgfqpoint{3.800000in}{3.800000in}}%
\pgfusepath{clip}%
\pgfsetbuttcap%
\pgfsetroundjoin%
\definecolor{currentfill}{rgb}{0.065035,0.236492,0.065035}%
\pgfsetfillcolor{currentfill}%
\pgfsetfillopacity{0.200000}%
\pgfsetlinewidth{0.000000pt}%
\definecolor{currentstroke}{rgb}{0.000000,0.000000,0.000000}%
\pgfsetstrokecolor{currentstroke}%
\pgfsetdash{}{0pt}%
\pgfpathmoveto{\pgfqpoint{6.054350in}{1.357725in}}%
\pgfpathlineto{\pgfqpoint{6.178951in}{2.093598in}}%
\pgfpathlineto{\pgfqpoint{5.990110in}{1.814186in}}%
\pgfpathlineto{\pgfqpoint{6.054350in}{1.357725in}}%
\pgfpathclose%
\pgfusepath{fill}%
\end{pgfscope}%
\begin{pgfscope}%
\pgfpathrectangle{\pgfqpoint{4.658921in}{0.208778in}}{\pgfqpoint{3.800000in}{3.800000in}}%
\pgfusepath{clip}%
\pgfsetbuttcap%
\pgfsetroundjoin%
\definecolor{currentfill}{rgb}{0.065035,0.236492,0.065035}%
\pgfsetfillcolor{currentfill}%
\pgfsetfillopacity{0.200000}%
\pgfsetlinewidth{0.000000pt}%
\definecolor{currentstroke}{rgb}{0.000000,0.000000,0.000000}%
\pgfsetstrokecolor{currentstroke}%
\pgfsetdash{}{0pt}%
\pgfpathmoveto{\pgfqpoint{7.230434in}{1.814186in}}%
\pgfpathlineto{\pgfqpoint{7.041593in}{2.093598in}}%
\pgfpathlineto{\pgfqpoint{7.166194in}{1.357725in}}%
\pgfpathlineto{\pgfqpoint{7.230434in}{1.814186in}}%
\pgfpathclose%
\pgfusepath{fill}%
\end{pgfscope}%
\begin{pgfscope}%
\pgfpathrectangle{\pgfqpoint{4.658921in}{0.208778in}}{\pgfqpoint{3.800000in}{3.800000in}}%
\pgfusepath{clip}%
\pgfsetbuttcap%
\pgfsetroundjoin%
\definecolor{currentfill}{rgb}{0.066446,0.241622,0.066446}%
\pgfsetfillcolor{currentfill}%
\pgfsetfillopacity{0.200000}%
\pgfsetlinewidth{0.000000pt}%
\definecolor{currentstroke}{rgb}{0.000000,0.000000,0.000000}%
\pgfsetstrokecolor{currentstroke}%
\pgfsetdash{}{0pt}%
\pgfpathmoveto{\pgfqpoint{6.394709in}{1.779529in}}%
\pgfpathlineto{\pgfqpoint{6.610272in}{1.158306in}}%
\pgfpathlineto{\pgfqpoint{6.610272in}{2.081635in}}%
\pgfpathlineto{\pgfqpoint{6.394709in}{1.779529in}}%
\pgfpathclose%
\pgfusepath{fill}%
\end{pgfscope}%
\begin{pgfscope}%
\pgfpathrectangle{\pgfqpoint{4.658921in}{0.208778in}}{\pgfqpoint{3.800000in}{3.800000in}}%
\pgfusepath{clip}%
\pgfsetbuttcap%
\pgfsetroundjoin%
\definecolor{currentfill}{rgb}{0.066446,0.241622,0.066446}%
\pgfsetfillcolor{currentfill}%
\pgfsetfillopacity{0.200000}%
\pgfsetlinewidth{0.000000pt}%
\definecolor{currentstroke}{rgb}{0.000000,0.000000,0.000000}%
\pgfsetstrokecolor{currentstroke}%
\pgfsetdash{}{0pt}%
\pgfpathmoveto{\pgfqpoint{6.610272in}{2.081635in}}%
\pgfpathlineto{\pgfqpoint{6.610272in}{1.158306in}}%
\pgfpathlineto{\pgfqpoint{6.825835in}{1.779529in}}%
\pgfpathlineto{\pgfqpoint{6.610272in}{2.081635in}}%
\pgfpathclose%
\pgfusepath{fill}%
\end{pgfscope}%
\begin{pgfscope}%
\pgfpathrectangle{\pgfqpoint{4.658921in}{0.208778in}}{\pgfqpoint{3.800000in}{3.800000in}}%
\pgfusepath{clip}%
\pgfsetbuttcap%
\pgfsetroundjoin%
\definecolor{currentfill}{rgb}{0.098306,0.357475,0.098306}%
\pgfsetfillcolor{currentfill}%
\pgfsetfillopacity{0.200000}%
\pgfsetlinewidth{0.000000pt}%
\definecolor{currentstroke}{rgb}{0.000000,0.000000,0.000000}%
\pgfsetstrokecolor{currentstroke}%
\pgfsetdash{}{0pt}%
\pgfpathmoveto{\pgfqpoint{6.610272in}{2.718551in}}%
\pgfpathlineto{\pgfqpoint{6.418151in}{2.991915in}}%
\pgfpathlineto{\pgfqpoint{6.204399in}{2.716708in}}%
\pgfpathlineto{\pgfqpoint{6.610272in}{2.718551in}}%
\pgfpathclose%
\pgfusepath{fill}%
\end{pgfscope}%
\begin{pgfscope}%
\pgfpathrectangle{\pgfqpoint{4.658921in}{0.208778in}}{\pgfqpoint{3.800000in}{3.800000in}}%
\pgfusepath{clip}%
\pgfsetbuttcap%
\pgfsetroundjoin%
\definecolor{currentfill}{rgb}{0.098306,0.357475,0.098306}%
\pgfsetfillcolor{currentfill}%
\pgfsetfillopacity{0.200000}%
\pgfsetlinewidth{0.000000pt}%
\definecolor{currentstroke}{rgb}{0.000000,0.000000,0.000000}%
\pgfsetstrokecolor{currentstroke}%
\pgfsetdash{}{0pt}%
\pgfpathmoveto{\pgfqpoint{7.016145in}{2.716708in}}%
\pgfpathlineto{\pgfqpoint{6.802393in}{2.991915in}}%
\pgfpathlineto{\pgfqpoint{6.610272in}{2.718551in}}%
\pgfpathlineto{\pgfqpoint{7.016145in}{2.716708in}}%
\pgfpathclose%
\pgfusepath{fill}%
\end{pgfscope}%
\begin{pgfscope}%
\pgfpathrectangle{\pgfqpoint{4.658921in}{0.208778in}}{\pgfqpoint{3.800000in}{3.800000in}}%
\pgfusepath{clip}%
\pgfsetbuttcap%
\pgfsetroundjoin%
\definecolor{currentfill}{rgb}{0.070209,0.255305,0.070209}%
\pgfsetfillcolor{currentfill}%
\pgfsetfillopacity{0.200000}%
\pgfsetlinewidth{0.000000pt}%
\definecolor{currentstroke}{rgb}{0.000000,0.000000,0.000000}%
\pgfsetstrokecolor{currentstroke}%
\pgfsetdash{}{0pt}%
\pgfpathmoveto{\pgfqpoint{6.054350in}{1.357725in}}%
\pgfpathlineto{\pgfqpoint{6.394709in}{1.779529in}}%
\pgfpathlineto{\pgfqpoint{6.178951in}{2.093598in}}%
\pgfpathlineto{\pgfqpoint{6.054350in}{1.357725in}}%
\pgfpathclose%
\pgfusepath{fill}%
\end{pgfscope}%
\begin{pgfscope}%
\pgfpathrectangle{\pgfqpoint{4.658921in}{0.208778in}}{\pgfqpoint{3.800000in}{3.800000in}}%
\pgfusepath{clip}%
\pgfsetbuttcap%
\pgfsetroundjoin%
\definecolor{currentfill}{rgb}{0.070209,0.255305,0.070209}%
\pgfsetfillcolor{currentfill}%
\pgfsetfillopacity{0.200000}%
\pgfsetlinewidth{0.000000pt}%
\definecolor{currentstroke}{rgb}{0.000000,0.000000,0.000000}%
\pgfsetstrokecolor{currentstroke}%
\pgfsetdash{}{0pt}%
\pgfpathmoveto{\pgfqpoint{7.041593in}{2.093598in}}%
\pgfpathlineto{\pgfqpoint{6.825835in}{1.779529in}}%
\pgfpathlineto{\pgfqpoint{7.166194in}{1.357725in}}%
\pgfpathlineto{\pgfqpoint{7.041593in}{2.093598in}}%
\pgfpathclose%
\pgfusepath{fill}%
\end{pgfscope}%
\begin{pgfscope}%
\pgfpathrectangle{\pgfqpoint{4.658921in}{0.208778in}}{\pgfqpoint{3.800000in}{3.800000in}}%
\pgfusepath{clip}%
\pgfsetbuttcap%
\pgfsetroundjoin%
\definecolor{currentfill}{rgb}{0.087398,0.317812,0.087398}%
\pgfsetfillcolor{currentfill}%
\pgfsetfillopacity{0.200000}%
\pgfsetlinewidth{0.000000pt}%
\definecolor{currentstroke}{rgb}{0.000000,0.000000,0.000000}%
\pgfsetstrokecolor{currentstroke}%
\pgfsetdash{}{0pt}%
\pgfpathmoveto{\pgfqpoint{6.178951in}{2.093598in}}%
\pgfpathlineto{\pgfqpoint{6.204399in}{2.716708in}}%
\pgfpathlineto{\pgfqpoint{5.799213in}{2.125336in}}%
\pgfpathlineto{\pgfqpoint{6.178951in}{2.093598in}}%
\pgfpathclose%
\pgfusepath{fill}%
\end{pgfscope}%
\begin{pgfscope}%
\pgfpathrectangle{\pgfqpoint{4.658921in}{0.208778in}}{\pgfqpoint{3.800000in}{3.800000in}}%
\pgfusepath{clip}%
\pgfsetbuttcap%
\pgfsetroundjoin%
\definecolor{currentfill}{rgb}{0.087398,0.317812,0.087398}%
\pgfsetfillcolor{currentfill}%
\pgfsetfillopacity{0.200000}%
\pgfsetlinewidth{0.000000pt}%
\definecolor{currentstroke}{rgb}{0.000000,0.000000,0.000000}%
\pgfsetstrokecolor{currentstroke}%
\pgfsetdash{}{0pt}%
\pgfpathmoveto{\pgfqpoint{7.421331in}{2.125336in}}%
\pgfpathlineto{\pgfqpoint{7.016145in}{2.716708in}}%
\pgfpathlineto{\pgfqpoint{7.041593in}{2.093598in}}%
\pgfpathlineto{\pgfqpoint{7.421331in}{2.125336in}}%
\pgfpathclose%
\pgfusepath{fill}%
\end{pgfscope}%
\begin{pgfscope}%
\pgfpathrectangle{\pgfqpoint{4.658921in}{0.208778in}}{\pgfqpoint{3.800000in}{3.800000in}}%
\pgfusepath{clip}%
\pgfsetbuttcap%
\pgfsetroundjoin%
\definecolor{currentfill}{rgb}{0.075994,0.276341,0.075994}%
\pgfsetfillcolor{currentfill}%
\pgfsetfillopacity{0.200000}%
\pgfsetlinewidth{0.000000pt}%
\definecolor{currentstroke}{rgb}{0.000000,0.000000,0.000000}%
\pgfsetstrokecolor{currentstroke}%
\pgfsetdash{}{0pt}%
\pgfpathmoveto{\pgfqpoint{5.799213in}{2.125336in}}%
\pgfpathlineto{\pgfqpoint{5.990110in}{1.814186in}}%
\pgfpathlineto{\pgfqpoint{6.178951in}{2.093598in}}%
\pgfpathlineto{\pgfqpoint{5.799213in}{2.125336in}}%
\pgfpathclose%
\pgfusepath{fill}%
\end{pgfscope}%
\begin{pgfscope}%
\pgfpathrectangle{\pgfqpoint{4.658921in}{0.208778in}}{\pgfqpoint{3.800000in}{3.800000in}}%
\pgfusepath{clip}%
\pgfsetbuttcap%
\pgfsetroundjoin%
\definecolor{currentfill}{rgb}{0.075994,0.276341,0.075994}%
\pgfsetfillcolor{currentfill}%
\pgfsetfillopacity{0.200000}%
\pgfsetlinewidth{0.000000pt}%
\definecolor{currentstroke}{rgb}{0.000000,0.000000,0.000000}%
\pgfsetstrokecolor{currentstroke}%
\pgfsetdash{}{0pt}%
\pgfpathmoveto{\pgfqpoint{7.041593in}{2.093598in}}%
\pgfpathlineto{\pgfqpoint{7.230434in}{1.814186in}}%
\pgfpathlineto{\pgfqpoint{7.421331in}{2.125336in}}%
\pgfpathlineto{\pgfqpoint{7.041593in}{2.093598in}}%
\pgfpathclose%
\pgfusepath{fill}%
\end{pgfscope}%
\begin{pgfscope}%
\pgfpathrectangle{\pgfqpoint{4.658921in}{0.208778in}}{\pgfqpoint{3.800000in}{3.800000in}}%
\pgfusepath{clip}%
\pgfsetbuttcap%
\pgfsetroundjoin%
\definecolor{currentfill}{rgb}{0.086061,0.312950,0.086061}%
\pgfsetfillcolor{currentfill}%
\pgfsetfillopacity{0.200000}%
\pgfsetlinewidth{0.000000pt}%
\definecolor{currentstroke}{rgb}{0.000000,0.000000,0.000000}%
\pgfsetstrokecolor{currentstroke}%
\pgfsetdash{}{0pt}%
\pgfpathmoveto{\pgfqpoint{6.204399in}{2.716708in}}%
\pgfpathlineto{\pgfqpoint{6.178951in}{2.093598in}}%
\pgfpathlineto{\pgfqpoint{6.610272in}{2.718551in}}%
\pgfpathlineto{\pgfqpoint{6.204399in}{2.716708in}}%
\pgfpathclose%
\pgfusepath{fill}%
\end{pgfscope}%
\begin{pgfscope}%
\pgfpathrectangle{\pgfqpoint{4.658921in}{0.208778in}}{\pgfqpoint{3.800000in}{3.800000in}}%
\pgfusepath{clip}%
\pgfsetbuttcap%
\pgfsetroundjoin%
\definecolor{currentfill}{rgb}{0.086061,0.312950,0.086061}%
\pgfsetfillcolor{currentfill}%
\pgfsetfillopacity{0.200000}%
\pgfsetlinewidth{0.000000pt}%
\definecolor{currentstroke}{rgb}{0.000000,0.000000,0.000000}%
\pgfsetstrokecolor{currentstroke}%
\pgfsetdash{}{0pt}%
\pgfpathmoveto{\pgfqpoint{6.610272in}{2.718551in}}%
\pgfpathlineto{\pgfqpoint{7.041593in}{2.093598in}}%
\pgfpathlineto{\pgfqpoint{7.016145in}{2.716708in}}%
\pgfpathlineto{\pgfqpoint{6.610272in}{2.718551in}}%
\pgfpathclose%
\pgfusepath{fill}%
\end{pgfscope}%
\begin{pgfscope}%
\pgfpathrectangle{\pgfqpoint{4.658921in}{0.208778in}}{\pgfqpoint{3.800000in}{3.800000in}}%
\pgfusepath{clip}%
\pgfsetbuttcap%
\pgfsetroundjoin%
\definecolor{currentfill}{rgb}{0.086258,0.313666,0.086258}%
\pgfsetfillcolor{currentfill}%
\pgfsetfillopacity{0.200000}%
\pgfsetlinewidth{0.000000pt}%
\definecolor{currentstroke}{rgb}{0.000000,0.000000,0.000000}%
\pgfsetstrokecolor{currentstroke}%
\pgfsetdash{}{0pt}%
\pgfpathmoveto{\pgfqpoint{6.610272in}{2.081635in}}%
\pgfpathlineto{\pgfqpoint{6.610272in}{2.718551in}}%
\pgfpathlineto{\pgfqpoint{6.178951in}{2.093598in}}%
\pgfpathlineto{\pgfqpoint{6.610272in}{2.081635in}}%
\pgfpathclose%
\pgfusepath{fill}%
\end{pgfscope}%
\begin{pgfscope}%
\pgfpathrectangle{\pgfqpoint{4.658921in}{0.208778in}}{\pgfqpoint{3.800000in}{3.800000in}}%
\pgfusepath{clip}%
\pgfsetbuttcap%
\pgfsetroundjoin%
\definecolor{currentfill}{rgb}{0.086258,0.313666,0.086258}%
\pgfsetfillcolor{currentfill}%
\pgfsetfillopacity{0.200000}%
\pgfsetlinewidth{0.000000pt}%
\definecolor{currentstroke}{rgb}{0.000000,0.000000,0.000000}%
\pgfsetstrokecolor{currentstroke}%
\pgfsetdash{}{0pt}%
\pgfpathmoveto{\pgfqpoint{7.041593in}{2.093598in}}%
\pgfpathlineto{\pgfqpoint{6.610272in}{2.718551in}}%
\pgfpathlineto{\pgfqpoint{6.610272in}{2.081635in}}%
\pgfpathlineto{\pgfqpoint{7.041593in}{2.093598in}}%
\pgfpathclose%
\pgfusepath{fill}%
\end{pgfscope}%
\begin{pgfscope}%
\pgfpathrectangle{\pgfqpoint{4.658921in}{0.208778in}}{\pgfqpoint{3.800000in}{3.800000in}}%
\pgfusepath{clip}%
\pgfsetbuttcap%
\pgfsetroundjoin%
\definecolor{currentfill}{rgb}{0.074668,0.271519,0.074668}%
\pgfsetfillcolor{currentfill}%
\pgfsetfillopacity{0.200000}%
\pgfsetlinewidth{0.000000pt}%
\definecolor{currentstroke}{rgb}{0.000000,0.000000,0.000000}%
\pgfsetstrokecolor{currentstroke}%
\pgfsetdash{}{0pt}%
\pgfpathmoveto{\pgfqpoint{6.610272in}{2.081635in}}%
\pgfpathlineto{\pgfqpoint{6.825835in}{1.779529in}}%
\pgfpathlineto{\pgfqpoint{7.041593in}{2.093598in}}%
\pgfpathlineto{\pgfqpoint{6.610272in}{2.081635in}}%
\pgfpathclose%
\pgfusepath{fill}%
\end{pgfscope}%
\begin{pgfscope}%
\pgfpathrectangle{\pgfqpoint{4.658921in}{0.208778in}}{\pgfqpoint{3.800000in}{3.800000in}}%
\pgfusepath{clip}%
\pgfsetbuttcap%
\pgfsetroundjoin%
\definecolor{currentfill}{rgb}{0.074668,0.271519,0.074668}%
\pgfsetfillcolor{currentfill}%
\pgfsetfillopacity{0.200000}%
\pgfsetlinewidth{0.000000pt}%
\definecolor{currentstroke}{rgb}{0.000000,0.000000,0.000000}%
\pgfsetstrokecolor{currentstroke}%
\pgfsetdash{}{0pt}%
\pgfpathmoveto{\pgfqpoint{6.178951in}{2.093598in}}%
\pgfpathlineto{\pgfqpoint{6.394709in}{1.779529in}}%
\pgfpathlineto{\pgfqpoint{6.610272in}{2.081635in}}%
\pgfpathlineto{\pgfqpoint{6.178951in}{2.093598in}}%
\pgfpathclose%
\pgfusepath{fill}%
\end{pgfscope}%
\begin{pgfscope}%
\pgfsetbuttcap%
\pgfsetmiterjoin%
\definecolor{currentfill}{rgb}{1.000000,1.000000,1.000000}%
\pgfsetfillcolor{currentfill}%
\pgfsetlinewidth{0.000000pt}%
\definecolor{currentstroke}{rgb}{0.000000,0.000000,0.000000}%
\pgfsetstrokecolor{currentstroke}%
\pgfsetstrokeopacity{0.000000}%
\pgfsetdash{}{0pt}%
\pgfpathmoveto{\pgfqpoint{8.608921in}{0.208778in}}%
\pgfpathlineto{\pgfqpoint{12.408921in}{0.208778in}}%
\pgfpathlineto{\pgfqpoint{12.408921in}{4.008778in}}%
\pgfpathlineto{\pgfqpoint{8.608921in}{4.008778in}}%
\pgfpathlineto{\pgfqpoint{8.608921in}{0.208778in}}%
\pgfpathclose%
\pgfusepath{fill}%
\end{pgfscope}%
\begin{pgfscope}%
\pgfsetbuttcap%
\pgfsetmiterjoin%
\definecolor{currentfill}{rgb}{0.950000,0.950000,0.950000}%
\pgfsetfillcolor{currentfill}%
\pgfsetfillopacity{0.500000}%
\pgfsetlinewidth{1.003750pt}%
\definecolor{currentstroke}{rgb}{0.950000,0.950000,0.950000}%
\pgfsetstrokecolor{currentstroke}%
\pgfsetstrokeopacity{0.500000}%
\pgfsetdash{}{0pt}%
\pgfpathmoveto{\pgfqpoint{12.345724in}{1.490585in}}%
\pgfpathlineto{\pgfqpoint{10.560272in}{0.339956in}}%
\pgfpathlineto{\pgfqpoint{10.560272in}{1.633829in}}%
\pgfpathlineto{\pgfqpoint{12.236450in}{2.788696in}}%
\pgfusepath{stroke,fill}%
\end{pgfscope}%
\begin{pgfscope}%
\pgfsetbuttcap%
\pgfsetmiterjoin%
\definecolor{currentfill}{rgb}{0.900000,0.900000,0.900000}%
\pgfsetfillcolor{currentfill}%
\pgfsetfillopacity{0.500000}%
\pgfsetlinewidth{1.003750pt}%
\definecolor{currentstroke}{rgb}{0.900000,0.900000,0.900000}%
\pgfsetstrokecolor{currentstroke}%
\pgfsetstrokeopacity{0.500000}%
\pgfsetdash{}{0pt}%
\pgfpathmoveto{\pgfqpoint{8.774820in}{1.490585in}}%
\pgfpathlineto{\pgfqpoint{10.560272in}{0.339956in}}%
\pgfpathlineto{\pgfqpoint{10.560272in}{1.633829in}}%
\pgfpathlineto{\pgfqpoint{8.884094in}{2.788696in}}%
\pgfusepath{stroke,fill}%
\end{pgfscope}%
\begin{pgfscope}%
\pgfsetbuttcap%
\pgfsetmiterjoin%
\definecolor{currentfill}{rgb}{0.925000,0.925000,0.925000}%
\pgfsetfillcolor{currentfill}%
\pgfsetfillopacity{0.500000}%
\pgfsetlinewidth{1.003750pt}%
\definecolor{currentstroke}{rgb}{0.925000,0.925000,0.925000}%
\pgfsetstrokecolor{currentstroke}%
\pgfsetstrokeopacity{0.500000}%
\pgfsetdash{}{0pt}%
\pgfpathmoveto{\pgfqpoint{10.560272in}{4.085003in}}%
\pgfpathlineto{\pgfqpoint{12.236450in}{2.788696in}}%
\pgfpathlineto{\pgfqpoint{10.560272in}{1.633829in}}%
\pgfpathlineto{\pgfqpoint{8.884094in}{2.788696in}}%
\pgfusepath{stroke,fill}%
\end{pgfscope}%
\begin{pgfscope}%
\pgfsetrectcap%
\pgfsetroundjoin%
\pgfsetlinewidth{0.803000pt}%
\definecolor{currentstroke}{rgb}{0.000000,0.000000,0.000000}%
\pgfsetstrokecolor{currentstroke}%
\pgfsetdash{}{0pt}%
\pgfpathmoveto{\pgfqpoint{12.236450in}{2.788696in}}%
\pgfpathlineto{\pgfqpoint{10.560272in}{4.085003in}}%
\pgfusepath{stroke}%
\end{pgfscope}%
\begin{pgfscope}%
\definecolor{textcolor}{rgb}{0.000000,0.000000,0.000000}%
\pgfsetstrokecolor{textcolor}%
\pgfsetfillcolor{textcolor}%
\pgftext[x=11.797764in,y=3.963287in,,]{\color{textcolor}\rmfamily\fontsize{14.000000}{16.800000}\selectfont f1}%
\end{pgfscope}%
\begin{pgfscope}%
\pgfsetbuttcap%
\pgfsetroundjoin%
\pgfsetlinewidth{0.803000pt}%
\definecolor{currentstroke}{rgb}{0.690196,0.690196,0.690196}%
\pgfsetstrokecolor{currentstroke}%
\pgfsetdash{}{0pt}%
\pgfpathmoveto{\pgfqpoint{10.673030in}{3.997799in}}%
\pgfpathlineto{\pgfqpoint{8.996502in}{2.711249in}}%
\pgfpathlineto{\pgfqpoint{8.894954in}{1.413166in}}%
\pgfusepath{stroke}%
\end{pgfscope}%
\begin{pgfscope}%
\pgfsetbuttcap%
\pgfsetroundjoin%
\pgfsetlinewidth{0.803000pt}%
\definecolor{currentstroke}{rgb}{0.690196,0.690196,0.690196}%
\pgfsetstrokecolor{currentstroke}%
\pgfsetdash{}{0pt}%
\pgfpathmoveto{\pgfqpoint{10.950417in}{3.783276in}}%
\pgfpathlineto{\pgfqpoint{9.273242in}{2.520578in}}%
\pgfpathlineto{\pgfqpoint{9.190469in}{1.222722in}}%
\pgfusepath{stroke}%
\end{pgfscope}%
\begin{pgfscope}%
\pgfsetbuttcap%
\pgfsetroundjoin%
\pgfsetlinewidth{0.803000pt}%
\definecolor{currentstroke}{rgb}{0.690196,0.690196,0.690196}%
\pgfsetstrokecolor{currentstroke}%
\pgfsetdash{}{0pt}%
\pgfpathmoveto{\pgfqpoint{11.222537in}{3.572827in}}%
\pgfpathlineto{\pgfqpoint{9.545024in}{2.333324in}}%
\pgfpathlineto{\pgfqpoint{9.480351in}{1.035908in}}%
\pgfusepath{stroke}%
\end{pgfscope}%
\begin{pgfscope}%
\pgfsetbuttcap%
\pgfsetroundjoin%
\pgfsetlinewidth{0.803000pt}%
\definecolor{currentstroke}{rgb}{0.690196,0.690196,0.690196}%
\pgfsetstrokecolor{currentstroke}%
\pgfsetdash{}{0pt}%
\pgfpathmoveto{\pgfqpoint{11.489537in}{3.366336in}}%
\pgfpathlineto{\pgfqpoint{9.811978in}{2.149395in}}%
\pgfpathlineto{\pgfqpoint{9.764761in}{0.852620in}}%
\pgfusepath{stroke}%
\end{pgfscope}%
\begin{pgfscope}%
\pgfsetbuttcap%
\pgfsetroundjoin%
\pgfsetlinewidth{0.803000pt}%
\definecolor{currentstroke}{rgb}{0.690196,0.690196,0.690196}%
\pgfsetstrokecolor{currentstroke}%
\pgfsetdash{}{0pt}%
\pgfpathmoveto{\pgfqpoint{11.751561in}{3.163695in}}%
\pgfpathlineto{\pgfqpoint{10.074234in}{1.968704in}}%
\pgfpathlineto{\pgfqpoint{10.043852in}{0.672761in}}%
\pgfusepath{stroke}%
\end{pgfscope}%
\begin{pgfscope}%
\pgfsetbuttcap%
\pgfsetroundjoin%
\pgfsetlinewidth{0.803000pt}%
\definecolor{currentstroke}{rgb}{0.690196,0.690196,0.690196}%
\pgfsetstrokecolor{currentstroke}%
\pgfsetdash{}{0pt}%
\pgfpathmoveto{\pgfqpoint{12.008748in}{2.964794in}}%
\pgfpathlineto{\pgfqpoint{10.331913in}{1.791165in}}%
\pgfpathlineto{\pgfqpoint{10.317771in}{0.496235in}}%
\pgfusepath{stroke}%
\end{pgfscope}%
\begin{pgfscope}%
\pgfsetrectcap%
\pgfsetroundjoin%
\pgfsetlinewidth{0.803000pt}%
\definecolor{currentstroke}{rgb}{0.000000,0.000000,0.000000}%
\pgfsetstrokecolor{currentstroke}%
\pgfsetdash{}{0pt}%
\pgfpathmoveto{\pgfqpoint{10.658807in}{3.986884in}}%
\pgfpathlineto{\pgfqpoint{10.701519in}{4.019660in}}%
\pgfusepath{stroke}%
\end{pgfscope}%
\begin{pgfscope}%
\definecolor{textcolor}{rgb}{0.000000,0.000000,0.000000}%
\pgfsetstrokecolor{textcolor}%
\pgfsetfillcolor{textcolor}%
\pgftext[x=10.815950in,y=4.205549in,,top]{\color{textcolor}\rmfamily\fontsize{10.000000}{12.000000}\selectfont \(\displaystyle {0.0}\)}%
\end{pgfscope}%
\begin{pgfscope}%
\pgfsetrectcap%
\pgfsetroundjoin%
\pgfsetlinewidth{0.803000pt}%
\definecolor{currentstroke}{rgb}{0.000000,0.000000,0.000000}%
\pgfsetstrokecolor{currentstroke}%
\pgfsetdash{}{0pt}%
\pgfpathmoveto{\pgfqpoint{10.936196in}{3.772569in}}%
\pgfpathlineto{\pgfqpoint{10.978901in}{3.804720in}}%
\pgfusepath{stroke}%
\end{pgfscope}%
\begin{pgfscope}%
\definecolor{textcolor}{rgb}{0.000000,0.000000,0.000000}%
\pgfsetstrokecolor{textcolor}%
\pgfsetfillcolor{textcolor}%
\pgftext[x=11.091886in,y=3.989099in,,top]{\color{textcolor}\rmfamily\fontsize{10.000000}{12.000000}\selectfont \(\displaystyle {0.2}\)}%
\end{pgfscope}%
\begin{pgfscope}%
\pgfsetrectcap%
\pgfsetroundjoin%
\pgfsetlinewidth{0.803000pt}%
\definecolor{currentstroke}{rgb}{0.000000,0.000000,0.000000}%
\pgfsetstrokecolor{currentstroke}%
\pgfsetdash{}{0pt}%
\pgfpathmoveto{\pgfqpoint{11.208320in}{3.562322in}}%
\pgfpathlineto{\pgfqpoint{11.251010in}{3.593866in}}%
\pgfusepath{stroke}%
\end{pgfscope}%
\begin{pgfscope}%
\definecolor{textcolor}{rgb}{0.000000,0.000000,0.000000}%
\pgfsetstrokecolor{textcolor}%
\pgfsetfillcolor{textcolor}%
\pgftext[x=11.362583in,y=3.776758in,,top]{\color{textcolor}\rmfamily\fontsize{10.000000}{12.000000}\selectfont \(\displaystyle {0.4}\)}%
\end{pgfscope}%
\begin{pgfscope}%
\pgfsetrectcap%
\pgfsetroundjoin%
\pgfsetlinewidth{0.803000pt}%
\definecolor{currentstroke}{rgb}{0.000000,0.000000,0.000000}%
\pgfsetstrokecolor{currentstroke}%
\pgfsetdash{}{0pt}%
\pgfpathmoveto{\pgfqpoint{11.475328in}{3.356029in}}%
\pgfpathlineto{\pgfqpoint{11.517996in}{3.386981in}}%
\pgfusepath{stroke}%
\end{pgfscope}%
\begin{pgfscope}%
\definecolor{textcolor}{rgb}{0.000000,0.000000,0.000000}%
\pgfsetstrokecolor{textcolor}%
\pgfsetfillcolor{textcolor}%
\pgftext[x=11.628189in,y=3.568410in,,top]{\color{textcolor}\rmfamily\fontsize{10.000000}{12.000000}\selectfont \(\displaystyle {0.6}\)}%
\end{pgfscope}%
\begin{pgfscope}%
\pgfsetrectcap%
\pgfsetroundjoin%
\pgfsetlinewidth{0.803000pt}%
\definecolor{currentstroke}{rgb}{0.000000,0.000000,0.000000}%
\pgfsetstrokecolor{currentstroke}%
\pgfsetdash{}{0pt}%
\pgfpathmoveto{\pgfqpoint{11.737361in}{3.153578in}}%
\pgfpathlineto{\pgfqpoint{11.780001in}{3.183956in}}%
\pgfusepath{stroke}%
\end{pgfscope}%
\begin{pgfscope}%
\definecolor{textcolor}{rgb}{0.000000,0.000000,0.000000}%
\pgfsetstrokecolor{textcolor}%
\pgfsetfillcolor{textcolor}%
\pgftext[x=11.888846in,y=3.363944in,,top]{\color{textcolor}\rmfamily\fontsize{10.000000}{12.000000}\selectfont \(\displaystyle {0.8}\)}%
\end{pgfscope}%
\begin{pgfscope}%
\pgfsetrectcap%
\pgfsetroundjoin%
\pgfsetlinewidth{0.803000pt}%
\definecolor{currentstroke}{rgb}{0.000000,0.000000,0.000000}%
\pgfsetstrokecolor{currentstroke}%
\pgfsetdash{}{0pt}%
\pgfpathmoveto{\pgfqpoint{11.994559in}{2.954864in}}%
\pgfpathlineto{\pgfqpoint{12.037164in}{2.984683in}}%
\pgfusepath{stroke}%
\end{pgfscope}%
\begin{pgfscope}%
\definecolor{textcolor}{rgb}{0.000000,0.000000,0.000000}%
\pgfsetstrokecolor{textcolor}%
\pgfsetfillcolor{textcolor}%
\pgftext[x=12.144692in,y=3.163252in,,top]{\color{textcolor}\rmfamily\fontsize{10.000000}{12.000000}\selectfont \(\displaystyle {1.0}\)}%
\end{pgfscope}%
\begin{pgfscope}%
\pgfsetrectcap%
\pgfsetroundjoin%
\pgfsetlinewidth{0.803000pt}%
\definecolor{currentstroke}{rgb}{0.000000,0.000000,0.000000}%
\pgfsetstrokecolor{currentstroke}%
\pgfsetdash{}{0pt}%
\pgfpathmoveto{\pgfqpoint{8.884094in}{2.788696in}}%
\pgfpathlineto{\pgfqpoint{10.560272in}{4.085003in}}%
\pgfusepath{stroke}%
\end{pgfscope}%
\begin{pgfscope}%
\definecolor{textcolor}{rgb}{0.000000,0.000000,0.000000}%
\pgfsetstrokecolor{textcolor}%
\pgfsetfillcolor{textcolor}%
\pgftext[x=9.322780in,y=3.963287in,,]{\color{textcolor}\rmfamily\fontsize{14.000000}{16.800000}\selectfont f2}%
\end{pgfscope}%
\begin{pgfscope}%
\pgfsetbuttcap%
\pgfsetroundjoin%
\pgfsetlinewidth{0.803000pt}%
\definecolor{currentstroke}{rgb}{0.690196,0.690196,0.690196}%
\pgfsetstrokecolor{currentstroke}%
\pgfsetdash{}{0pt}%
\pgfpathmoveto{\pgfqpoint{12.225590in}{1.413166in}}%
\pgfpathlineto{\pgfqpoint{12.124042in}{2.711249in}}%
\pgfpathlineto{\pgfqpoint{10.447514in}{3.997799in}}%
\pgfusepath{stroke}%
\end{pgfscope}%
\begin{pgfscope}%
\pgfsetbuttcap%
\pgfsetroundjoin%
\pgfsetlinewidth{0.803000pt}%
\definecolor{currentstroke}{rgb}{0.690196,0.690196,0.690196}%
\pgfsetstrokecolor{currentstroke}%
\pgfsetdash{}{0pt}%
\pgfpathmoveto{\pgfqpoint{11.930075in}{1.222722in}}%
\pgfpathlineto{\pgfqpoint{11.847302in}{2.520578in}}%
\pgfpathlineto{\pgfqpoint{10.170127in}{3.783276in}}%
\pgfusepath{stroke}%
\end{pgfscope}%
\begin{pgfscope}%
\pgfsetbuttcap%
\pgfsetroundjoin%
\pgfsetlinewidth{0.803000pt}%
\definecolor{currentstroke}{rgb}{0.690196,0.690196,0.690196}%
\pgfsetstrokecolor{currentstroke}%
\pgfsetdash{}{0pt}%
\pgfpathmoveto{\pgfqpoint{11.640193in}{1.035908in}}%
\pgfpathlineto{\pgfqpoint{11.575520in}{2.333324in}}%
\pgfpathlineto{\pgfqpoint{9.898007in}{3.572827in}}%
\pgfusepath{stroke}%
\end{pgfscope}%
\begin{pgfscope}%
\pgfsetbuttcap%
\pgfsetroundjoin%
\pgfsetlinewidth{0.803000pt}%
\definecolor{currentstroke}{rgb}{0.690196,0.690196,0.690196}%
\pgfsetstrokecolor{currentstroke}%
\pgfsetdash{}{0pt}%
\pgfpathmoveto{\pgfqpoint{11.355783in}{0.852620in}}%
\pgfpathlineto{\pgfqpoint{11.308566in}{2.149395in}}%
\pgfpathlineto{\pgfqpoint{9.631007in}{3.366336in}}%
\pgfusepath{stroke}%
\end{pgfscope}%
\begin{pgfscope}%
\pgfsetbuttcap%
\pgfsetroundjoin%
\pgfsetlinewidth{0.803000pt}%
\definecolor{currentstroke}{rgb}{0.690196,0.690196,0.690196}%
\pgfsetstrokecolor{currentstroke}%
\pgfsetdash{}{0pt}%
\pgfpathmoveto{\pgfqpoint{11.076692in}{0.672761in}}%
\pgfpathlineto{\pgfqpoint{11.046310in}{1.968704in}}%
\pgfpathlineto{\pgfqpoint{9.368983in}{3.163695in}}%
\pgfusepath{stroke}%
\end{pgfscope}%
\begin{pgfscope}%
\pgfsetbuttcap%
\pgfsetroundjoin%
\pgfsetlinewidth{0.803000pt}%
\definecolor{currentstroke}{rgb}{0.690196,0.690196,0.690196}%
\pgfsetstrokecolor{currentstroke}%
\pgfsetdash{}{0pt}%
\pgfpathmoveto{\pgfqpoint{10.802773in}{0.496235in}}%
\pgfpathlineto{\pgfqpoint{10.788631in}{1.791165in}}%
\pgfpathlineto{\pgfqpoint{9.111796in}{2.964794in}}%
\pgfusepath{stroke}%
\end{pgfscope}%
\begin{pgfscope}%
\pgfsetrectcap%
\pgfsetroundjoin%
\pgfsetlinewidth{0.803000pt}%
\definecolor{currentstroke}{rgb}{0.000000,0.000000,0.000000}%
\pgfsetstrokecolor{currentstroke}%
\pgfsetdash{}{0pt}%
\pgfpathmoveto{\pgfqpoint{10.461737in}{3.986884in}}%
\pgfpathlineto{\pgfqpoint{10.419025in}{4.019660in}}%
\pgfusepath{stroke}%
\end{pgfscope}%
\begin{pgfscope}%
\definecolor{textcolor}{rgb}{0.000000,0.000000,0.000000}%
\pgfsetstrokecolor{textcolor}%
\pgfsetfillcolor{textcolor}%
\pgftext[x=10.304594in,y=4.205549in,,top]{\color{textcolor}\rmfamily\fontsize{10.000000}{12.000000}\selectfont \(\displaystyle {0.0}\)}%
\end{pgfscope}%
\begin{pgfscope}%
\pgfsetrectcap%
\pgfsetroundjoin%
\pgfsetlinewidth{0.803000pt}%
\definecolor{currentstroke}{rgb}{0.000000,0.000000,0.000000}%
\pgfsetstrokecolor{currentstroke}%
\pgfsetdash{}{0pt}%
\pgfpathmoveto{\pgfqpoint{10.184348in}{3.772569in}}%
\pgfpathlineto{\pgfqpoint{10.141643in}{3.804720in}}%
\pgfusepath{stroke}%
\end{pgfscope}%
\begin{pgfscope}%
\definecolor{textcolor}{rgb}{0.000000,0.000000,0.000000}%
\pgfsetstrokecolor{textcolor}%
\pgfsetfillcolor{textcolor}%
\pgftext[x=10.028658in,y=3.989099in,,top]{\color{textcolor}\rmfamily\fontsize{10.000000}{12.000000}\selectfont \(\displaystyle {0.2}\)}%
\end{pgfscope}%
\begin{pgfscope}%
\pgfsetrectcap%
\pgfsetroundjoin%
\pgfsetlinewidth{0.803000pt}%
\definecolor{currentstroke}{rgb}{0.000000,0.000000,0.000000}%
\pgfsetstrokecolor{currentstroke}%
\pgfsetdash{}{0pt}%
\pgfpathmoveto{\pgfqpoint{9.912224in}{3.562322in}}%
\pgfpathlineto{\pgfqpoint{9.869534in}{3.593866in}}%
\pgfusepath{stroke}%
\end{pgfscope}%
\begin{pgfscope}%
\definecolor{textcolor}{rgb}{0.000000,0.000000,0.000000}%
\pgfsetstrokecolor{textcolor}%
\pgfsetfillcolor{textcolor}%
\pgftext[x=9.757961in,y=3.776758in,,top]{\color{textcolor}\rmfamily\fontsize{10.000000}{12.000000}\selectfont \(\displaystyle {0.4}\)}%
\end{pgfscope}%
\begin{pgfscope}%
\pgfsetrectcap%
\pgfsetroundjoin%
\pgfsetlinewidth{0.803000pt}%
\definecolor{currentstroke}{rgb}{0.000000,0.000000,0.000000}%
\pgfsetstrokecolor{currentstroke}%
\pgfsetdash{}{0pt}%
\pgfpathmoveto{\pgfqpoint{9.645216in}{3.356029in}}%
\pgfpathlineto{\pgfqpoint{9.602548in}{3.386981in}}%
\pgfusepath{stroke}%
\end{pgfscope}%
\begin{pgfscope}%
\definecolor{textcolor}{rgb}{0.000000,0.000000,0.000000}%
\pgfsetstrokecolor{textcolor}%
\pgfsetfillcolor{textcolor}%
\pgftext[x=9.492355in,y=3.568410in,,top]{\color{textcolor}\rmfamily\fontsize{10.000000}{12.000000}\selectfont \(\displaystyle {0.6}\)}%
\end{pgfscope}%
\begin{pgfscope}%
\pgfsetrectcap%
\pgfsetroundjoin%
\pgfsetlinewidth{0.803000pt}%
\definecolor{currentstroke}{rgb}{0.000000,0.000000,0.000000}%
\pgfsetstrokecolor{currentstroke}%
\pgfsetdash{}{0pt}%
\pgfpathmoveto{\pgfqpoint{9.383183in}{3.153578in}}%
\pgfpathlineto{\pgfqpoint{9.340543in}{3.183956in}}%
\pgfusepath{stroke}%
\end{pgfscope}%
\begin{pgfscope}%
\definecolor{textcolor}{rgb}{0.000000,0.000000,0.000000}%
\pgfsetstrokecolor{textcolor}%
\pgfsetfillcolor{textcolor}%
\pgftext[x=9.231698in,y=3.363944in,,top]{\color{textcolor}\rmfamily\fontsize{10.000000}{12.000000}\selectfont \(\displaystyle {0.8}\)}%
\end{pgfscope}%
\begin{pgfscope}%
\pgfsetrectcap%
\pgfsetroundjoin%
\pgfsetlinewidth{0.803000pt}%
\definecolor{currentstroke}{rgb}{0.000000,0.000000,0.000000}%
\pgfsetstrokecolor{currentstroke}%
\pgfsetdash{}{0pt}%
\pgfpathmoveto{\pgfqpoint{9.125985in}{2.954864in}}%
\pgfpathlineto{\pgfqpoint{9.083380in}{2.984683in}}%
\pgfusepath{stroke}%
\end{pgfscope}%
\begin{pgfscope}%
\definecolor{textcolor}{rgb}{0.000000,0.000000,0.000000}%
\pgfsetstrokecolor{textcolor}%
\pgfsetfillcolor{textcolor}%
\pgftext[x=8.975852in,y=3.163252in,,top]{\color{textcolor}\rmfamily\fontsize{10.000000}{12.000000}\selectfont \(\displaystyle {1.0}\)}%
\end{pgfscope}%
\begin{pgfscope}%
\pgfsetrectcap%
\pgfsetroundjoin%
\pgfsetlinewidth{0.803000pt}%
\definecolor{currentstroke}{rgb}{0.000000,0.000000,0.000000}%
\pgfsetstrokecolor{currentstroke}%
\pgfsetdash{}{0pt}%
\pgfpathmoveto{\pgfqpoint{8.884094in}{2.788696in}}%
\pgfpathlineto{\pgfqpoint{8.774820in}{1.490585in}}%
\pgfusepath{stroke}%
\end{pgfscope}%
\begin{pgfscope}%
\definecolor{textcolor}{rgb}{0.000000,0.000000,0.000000}%
\pgfsetstrokecolor{textcolor}%
\pgfsetfillcolor{textcolor}%
\pgftext[x=8.078876in,y=2.160130in,,]{\color{textcolor}\rmfamily\fontsize{14.000000}{16.800000}\selectfont f3}%
\end{pgfscope}%
\begin{pgfscope}%
\pgfsetbuttcap%
\pgfsetroundjoin%
\pgfsetlinewidth{0.803000pt}%
\definecolor{currentstroke}{rgb}{0.690196,0.690196,0.690196}%
\pgfsetstrokecolor{currentstroke}%
\pgfsetdash{}{0pt}%
\pgfpathmoveto{\pgfqpoint{8.782197in}{1.578214in}}%
\pgfpathlineto{\pgfqpoint{10.560272in}{0.427010in}}%
\pgfpathlineto{\pgfqpoint{12.338347in}{1.578214in}}%
\pgfusepath{stroke}%
\end{pgfscope}%
\begin{pgfscope}%
\pgfsetbuttcap%
\pgfsetroundjoin%
\pgfsetlinewidth{0.803000pt}%
\definecolor{currentstroke}{rgb}{0.690196,0.690196,0.690196}%
\pgfsetstrokecolor{currentstroke}%
\pgfsetdash{}{0pt}%
\pgfpathmoveto{\pgfqpoint{8.800327in}{1.793596in}}%
\pgfpathlineto{\pgfqpoint{10.560272in}{0.641156in}}%
\pgfpathlineto{\pgfqpoint{12.320217in}{1.793596in}}%
\pgfusepath{stroke}%
\end{pgfscope}%
\begin{pgfscope}%
\pgfsetbuttcap%
\pgfsetroundjoin%
\pgfsetlinewidth{0.803000pt}%
\definecolor{currentstroke}{rgb}{0.690196,0.690196,0.690196}%
\pgfsetstrokecolor{currentstroke}%
\pgfsetdash{}{0pt}%
\pgfpathmoveto{\pgfqpoint{8.818092in}{2.004629in}}%
\pgfpathlineto{\pgfqpoint{10.560272in}{0.851223in}}%
\pgfpathlineto{\pgfqpoint{12.302452in}{2.004629in}}%
\pgfusepath{stroke}%
\end{pgfscope}%
\begin{pgfscope}%
\pgfsetbuttcap%
\pgfsetroundjoin%
\pgfsetlinewidth{0.803000pt}%
\definecolor{currentstroke}{rgb}{0.690196,0.690196,0.690196}%
\pgfsetstrokecolor{currentstroke}%
\pgfsetdash{}{0pt}%
\pgfpathmoveto{\pgfqpoint{8.835502in}{2.211445in}}%
\pgfpathlineto{\pgfqpoint{10.560272in}{1.057328in}}%
\pgfpathlineto{\pgfqpoint{12.285042in}{2.211445in}}%
\pgfusepath{stroke}%
\end{pgfscope}%
\begin{pgfscope}%
\pgfsetbuttcap%
\pgfsetroundjoin%
\pgfsetlinewidth{0.803000pt}%
\definecolor{currentstroke}{rgb}{0.690196,0.690196,0.690196}%
\pgfsetstrokecolor{currentstroke}%
\pgfsetdash{}{0pt}%
\pgfpathmoveto{\pgfqpoint{8.852567in}{2.414169in}}%
\pgfpathlineto{\pgfqpoint{10.560272in}{1.259580in}}%
\pgfpathlineto{\pgfqpoint{12.267977in}{2.414169in}}%
\pgfusepath{stroke}%
\end{pgfscope}%
\begin{pgfscope}%
\pgfsetbuttcap%
\pgfsetroundjoin%
\pgfsetlinewidth{0.803000pt}%
\definecolor{currentstroke}{rgb}{0.690196,0.690196,0.690196}%
\pgfsetstrokecolor{currentstroke}%
\pgfsetdash{}{0pt}%
\pgfpathmoveto{\pgfqpoint{8.869298in}{2.612920in}}%
\pgfpathlineto{\pgfqpoint{10.560272in}{1.458087in}}%
\pgfpathlineto{\pgfqpoint{12.251246in}{2.612920in}}%
\pgfusepath{stroke}%
\end{pgfscope}%
\begin{pgfscope}%
\pgfsetrectcap%
\pgfsetroundjoin%
\pgfsetlinewidth{0.803000pt}%
\definecolor{currentstroke}{rgb}{0.000000,0.000000,0.000000}%
\pgfsetstrokecolor{currentstroke}%
\pgfsetdash{}{0pt}%
\pgfpathmoveto{\pgfqpoint{8.797285in}{1.568446in}}%
\pgfpathlineto{\pgfqpoint{8.751977in}{1.597780in}}%
\pgfusepath{stroke}%
\end{pgfscope}%
\begin{pgfscope}%
\definecolor{textcolor}{rgb}{0.000000,0.000000,0.000000}%
\pgfsetstrokecolor{textcolor}%
\pgfsetfillcolor{textcolor}%
\pgftext[x=8.495209in,y=1.578214in,,top]{\color{textcolor}\rmfamily\fontsize{10.000000}{12.000000}\selectfont \(\displaystyle {0.0}\)}%
\end{pgfscope}%
\begin{pgfscope}%
\pgfsetrectcap%
\pgfsetroundjoin%
\pgfsetlinewidth{0.803000pt}%
\definecolor{currentstroke}{rgb}{0.000000,0.000000,0.000000}%
\pgfsetstrokecolor{currentstroke}%
\pgfsetdash{}{0pt}%
\pgfpathmoveto{\pgfqpoint{8.815253in}{1.783823in}}%
\pgfpathlineto{\pgfqpoint{8.770433in}{1.813171in}}%
\pgfusepath{stroke}%
\end{pgfscope}%
\begin{pgfscope}%
\definecolor{textcolor}{rgb}{0.000000,0.000000,0.000000}%
\pgfsetstrokecolor{textcolor}%
\pgfsetfillcolor{textcolor}%
\pgftext[x=8.516266in,y=1.793596in,,top]{\color{textcolor}\rmfamily\fontsize{10.000000}{12.000000}\selectfont \(\displaystyle {0.2}\)}%
\end{pgfscope}%
\begin{pgfscope}%
\pgfsetrectcap%
\pgfsetroundjoin%
\pgfsetlinewidth{0.803000pt}%
\definecolor{currentstroke}{rgb}{0.000000,0.000000,0.000000}%
\pgfsetstrokecolor{currentstroke}%
\pgfsetdash{}{0pt}%
\pgfpathmoveto{\pgfqpoint{8.832858in}{1.994853in}}%
\pgfpathlineto{\pgfqpoint{8.788517in}{2.024210in}}%
\pgfusepath{stroke}%
\end{pgfscope}%
\begin{pgfscope}%
\definecolor{textcolor}{rgb}{0.000000,0.000000,0.000000}%
\pgfsetstrokecolor{textcolor}%
\pgfsetfillcolor{textcolor}%
\pgftext[x=8.536898in,y=2.004629in,,top]{\color{textcolor}\rmfamily\fontsize{10.000000}{12.000000}\selectfont \(\displaystyle {0.4}\)}%
\end{pgfscope}%
\begin{pgfscope}%
\pgfsetrectcap%
\pgfsetroundjoin%
\pgfsetlinewidth{0.803000pt}%
\definecolor{currentstroke}{rgb}{0.000000,0.000000,0.000000}%
\pgfsetstrokecolor{currentstroke}%
\pgfsetdash{}{0pt}%
\pgfpathmoveto{\pgfqpoint{8.850112in}{2.201669in}}%
\pgfpathlineto{\pgfqpoint{8.806239in}{2.231026in}}%
\pgfusepath{stroke}%
\end{pgfscope}%
\begin{pgfscope}%
\definecolor{textcolor}{rgb}{0.000000,0.000000,0.000000}%
\pgfsetstrokecolor{textcolor}%
\pgfsetfillcolor{textcolor}%
\pgftext[x=8.557118in,y=2.211445in,,top]{\color{textcolor}\rmfamily\fontsize{10.000000}{12.000000}\selectfont \(\displaystyle {0.6}\)}%
\end{pgfscope}%
\begin{pgfscope}%
\pgfsetrectcap%
\pgfsetroundjoin%
\pgfsetlinewidth{0.803000pt}%
\definecolor{currentstroke}{rgb}{0.000000,0.000000,0.000000}%
\pgfsetstrokecolor{currentstroke}%
\pgfsetdash{}{0pt}%
\pgfpathmoveto{\pgfqpoint{8.867025in}{2.404393in}}%
\pgfpathlineto{\pgfqpoint{8.823610in}{2.433746in}}%
\pgfusepath{stroke}%
\end{pgfscope}%
\begin{pgfscope}%
\definecolor{textcolor}{rgb}{0.000000,0.000000,0.000000}%
\pgfsetstrokecolor{textcolor}%
\pgfsetfillcolor{textcolor}%
\pgftext[x=8.576937in,y=2.414169in,,top]{\color{textcolor}\rmfamily\fontsize{10.000000}{12.000000}\selectfont \(\displaystyle {0.8}\)}%
\end{pgfscope}%
\begin{pgfscope}%
\pgfsetrectcap%
\pgfsetroundjoin%
\pgfsetlinewidth{0.803000pt}%
\definecolor{currentstroke}{rgb}{0.000000,0.000000,0.000000}%
\pgfsetstrokecolor{currentstroke}%
\pgfsetdash{}{0pt}%
\pgfpathmoveto{\pgfqpoint{8.883606in}{2.603148in}}%
\pgfpathlineto{\pgfqpoint{8.840640in}{2.632491in}}%
\pgfusepath{stroke}%
\end{pgfscope}%
\begin{pgfscope}%
\definecolor{textcolor}{rgb}{0.000000,0.000000,0.000000}%
\pgfsetstrokecolor{textcolor}%
\pgfsetfillcolor{textcolor}%
\pgftext[x=8.596368in,y=2.612920in,,top]{\color{textcolor}\rmfamily\fontsize{10.000000}{12.000000}\selectfont \(\displaystyle {1.0}\)}%
\end{pgfscope}%
\begin{pgfscope}%
\pgfpathrectangle{\pgfqpoint{8.608921in}{0.208778in}}{\pgfqpoint{3.800000in}{3.800000in}}%
\pgfusepath{clip}%
\pgfsetbuttcap%
\pgfsetroundjoin%
\definecolor{currentfill}{rgb}{0.839216,0.152941,0.156863}%
\pgfsetfillcolor{currentfill}%
\pgfsetfillopacity{0.300000}%
\pgfsetlinewidth{1.003750pt}%
\definecolor{currentstroke}{rgb}{0.839216,0.152941,0.156863}%
\pgfsetstrokecolor{currentstroke}%
\pgfsetstrokeopacity{0.300000}%
\pgfsetdash{}{0pt}%
\pgfpathmoveto{\pgfqpoint{10.149581in}{2.112163in}}%
\pgfpathcurveto{\pgfqpoint{10.159668in}{2.112163in}}{\pgfqpoint{10.169344in}{2.116170in}}{\pgfqpoint{10.176476in}{2.123303in}}%
\pgfpathcurveto{\pgfqpoint{10.183609in}{2.130436in}}{\pgfqpoint{10.187617in}{2.140112in}}{\pgfqpoint{10.187617in}{2.150199in}}%
\pgfpathcurveto{\pgfqpoint{10.187617in}{2.160286in}}{\pgfqpoint{10.183609in}{2.169962in}}{\pgfqpoint{10.176476in}{2.177095in}}%
\pgfpathcurveto{\pgfqpoint{10.169344in}{2.184227in}}{\pgfqpoint{10.159668in}{2.188235in}}{\pgfqpoint{10.149581in}{2.188235in}}%
\pgfpathcurveto{\pgfqpoint{10.139493in}{2.188235in}}{\pgfqpoint{10.129818in}{2.184227in}}{\pgfqpoint{10.122685in}{2.177095in}}%
\pgfpathcurveto{\pgfqpoint{10.115552in}{2.169962in}}{\pgfqpoint{10.111544in}{2.160286in}}{\pgfqpoint{10.111544in}{2.150199in}}%
\pgfpathcurveto{\pgfqpoint{10.111544in}{2.140112in}}{\pgfqpoint{10.115552in}{2.130436in}}{\pgfqpoint{10.122685in}{2.123303in}}%
\pgfpathcurveto{\pgfqpoint{10.129818in}{2.116170in}}{\pgfqpoint{10.139493in}{2.112163in}}{\pgfqpoint{10.149581in}{2.112163in}}%
\pgfpathlineto{\pgfqpoint{10.149581in}{2.112163in}}%
\pgfpathclose%
\pgfusepath{stroke,fill}%
\end{pgfscope}%
\begin{pgfscope}%
\pgfpathrectangle{\pgfqpoint{8.608921in}{0.208778in}}{\pgfqpoint{3.800000in}{3.800000in}}%
\pgfusepath{clip}%
\pgfsetbuttcap%
\pgfsetroundjoin%
\definecolor{currentfill}{rgb}{0.839216,0.152941,0.156863}%
\pgfsetfillcolor{currentfill}%
\pgfsetfillopacity{0.368519}%
\pgfsetlinewidth{1.003750pt}%
\definecolor{currentstroke}{rgb}{0.839216,0.152941,0.156863}%
\pgfsetstrokecolor{currentstroke}%
\pgfsetstrokeopacity{0.368519}%
\pgfsetdash{}{0pt}%
\pgfpathmoveto{\pgfqpoint{9.982053in}{1.811001in}}%
\pgfpathcurveto{\pgfqpoint{9.992141in}{1.811001in}}{\pgfqpoint{10.001816in}{1.815009in}}{\pgfqpoint{10.008949in}{1.822142in}}%
\pgfpathcurveto{\pgfqpoint{10.016082in}{1.829274in}}{\pgfqpoint{10.020090in}{1.838950in}}{\pgfqpoint{10.020090in}{1.849037in}}%
\pgfpathcurveto{\pgfqpoint{10.020090in}{1.859125in}}{\pgfqpoint{10.016082in}{1.868800in}}{\pgfqpoint{10.008949in}{1.875933in}}%
\pgfpathcurveto{\pgfqpoint{10.001816in}{1.883066in}}{\pgfqpoint{9.992141in}{1.887074in}}{\pgfqpoint{9.982053in}{1.887074in}}%
\pgfpathcurveto{\pgfqpoint{9.971966in}{1.887074in}}{\pgfqpoint{9.962290in}{1.883066in}}{\pgfqpoint{9.955158in}{1.875933in}}%
\pgfpathcurveto{\pgfqpoint{9.948025in}{1.868800in}}{\pgfqpoint{9.944017in}{1.859125in}}{\pgfqpoint{9.944017in}{1.849037in}}%
\pgfpathcurveto{\pgfqpoint{9.944017in}{1.838950in}}{\pgfqpoint{9.948025in}{1.829274in}}{\pgfqpoint{9.955158in}{1.822142in}}%
\pgfpathcurveto{\pgfqpoint{9.962290in}{1.815009in}}{\pgfqpoint{9.971966in}{1.811001in}}{\pgfqpoint{9.982053in}{1.811001in}}%
\pgfpathlineto{\pgfqpoint{9.982053in}{1.811001in}}%
\pgfpathclose%
\pgfusepath{stroke,fill}%
\end{pgfscope}%
\begin{pgfscope}%
\pgfpathrectangle{\pgfqpoint{8.608921in}{0.208778in}}{\pgfqpoint{3.800000in}{3.800000in}}%
\pgfusepath{clip}%
\pgfsetbuttcap%
\pgfsetroundjoin%
\definecolor{currentfill}{rgb}{0.839216,0.152941,0.156863}%
\pgfsetfillcolor{currentfill}%
\pgfsetfillopacity{0.431635}%
\pgfsetlinewidth{1.003750pt}%
\definecolor{currentstroke}{rgb}{0.839216,0.152941,0.156863}%
\pgfsetstrokecolor{currentstroke}%
\pgfsetstrokeopacity{0.431635}%
\pgfsetdash{}{0pt}%
\pgfpathmoveto{\pgfqpoint{10.062464in}{2.695040in}}%
\pgfpathcurveto{\pgfqpoint{10.072551in}{2.695040in}}{\pgfqpoint{10.082227in}{2.699047in}}{\pgfqpoint{10.089359in}{2.706180in}}%
\pgfpathcurveto{\pgfqpoint{10.096492in}{2.713313in}}{\pgfqpoint{10.100500in}{2.722989in}}{\pgfqpoint{10.100500in}{2.733076in}}%
\pgfpathcurveto{\pgfqpoint{10.100500in}{2.743163in}}{\pgfqpoint{10.096492in}{2.752839in}}{\pgfqpoint{10.089359in}{2.759972in}}%
\pgfpathcurveto{\pgfqpoint{10.082227in}{2.767105in}}{\pgfqpoint{10.072551in}{2.771112in}}{\pgfqpoint{10.062464in}{2.771112in}}%
\pgfpathcurveto{\pgfqpoint{10.052376in}{2.771112in}}{\pgfqpoint{10.042701in}{2.767105in}}{\pgfqpoint{10.035568in}{2.759972in}}%
\pgfpathcurveto{\pgfqpoint{10.028435in}{2.752839in}}{\pgfqpoint{10.024427in}{2.743163in}}{\pgfqpoint{10.024427in}{2.733076in}}%
\pgfpathcurveto{\pgfqpoint{10.024427in}{2.722989in}}{\pgfqpoint{10.028435in}{2.713313in}}{\pgfqpoint{10.035568in}{2.706180in}}%
\pgfpathcurveto{\pgfqpoint{10.042701in}{2.699047in}}{\pgfqpoint{10.052376in}{2.695040in}}{\pgfqpoint{10.062464in}{2.695040in}}%
\pgfpathlineto{\pgfqpoint{10.062464in}{2.695040in}}%
\pgfpathclose%
\pgfusepath{stroke,fill}%
\end{pgfscope}%
\begin{pgfscope}%
\pgfpathrectangle{\pgfqpoint{8.608921in}{0.208778in}}{\pgfqpoint{3.800000in}{3.800000in}}%
\pgfusepath{clip}%
\pgfsetbuttcap%
\pgfsetroundjoin%
\definecolor{currentfill}{rgb}{0.839216,0.152941,0.156863}%
\pgfsetfillcolor{currentfill}%
\pgfsetfillopacity{0.502644}%
\pgfsetlinewidth{1.003750pt}%
\definecolor{currentstroke}{rgb}{0.839216,0.152941,0.156863}%
\pgfsetstrokecolor{currentstroke}%
\pgfsetstrokeopacity{0.502644}%
\pgfsetdash{}{0pt}%
\pgfpathmoveto{\pgfqpoint{10.988634in}{2.818904in}}%
\pgfpathcurveto{\pgfqpoint{10.998721in}{2.818904in}}{\pgfqpoint{11.008397in}{2.822911in}}{\pgfqpoint{11.015530in}{2.830044in}}%
\pgfpathcurveto{\pgfqpoint{11.022663in}{2.837177in}}{\pgfqpoint{11.026670in}{2.846853in}}{\pgfqpoint{11.026670in}{2.856940in}}%
\pgfpathcurveto{\pgfqpoint{11.026670in}{2.867027in}}{\pgfqpoint{11.022663in}{2.876703in}}{\pgfqpoint{11.015530in}{2.883836in}}%
\pgfpathcurveto{\pgfqpoint{11.008397in}{2.890969in}}{\pgfqpoint{10.998721in}{2.894976in}}{\pgfqpoint{10.988634in}{2.894976in}}%
\pgfpathcurveto{\pgfqpoint{10.978547in}{2.894976in}}{\pgfqpoint{10.968871in}{2.890969in}}{\pgfqpoint{10.961738in}{2.883836in}}%
\pgfpathcurveto{\pgfqpoint{10.954606in}{2.876703in}}{\pgfqpoint{10.950598in}{2.867027in}}{\pgfqpoint{10.950598in}{2.856940in}}%
\pgfpathcurveto{\pgfqpoint{10.950598in}{2.846853in}}{\pgfqpoint{10.954606in}{2.837177in}}{\pgfqpoint{10.961738in}{2.830044in}}%
\pgfpathcurveto{\pgfqpoint{10.968871in}{2.822911in}}{\pgfqpoint{10.978547in}{2.818904in}}{\pgfqpoint{10.988634in}{2.818904in}}%
\pgfpathlineto{\pgfqpoint{10.988634in}{2.818904in}}%
\pgfpathclose%
\pgfusepath{stroke,fill}%
\end{pgfscope}%
\begin{pgfscope}%
\pgfpathrectangle{\pgfqpoint{8.608921in}{0.208778in}}{\pgfqpoint{3.800000in}{3.800000in}}%
\pgfusepath{clip}%
\pgfsetbuttcap%
\pgfsetroundjoin%
\definecolor{currentfill}{rgb}{0.839216,0.152941,0.156863}%
\pgfsetfillcolor{currentfill}%
\pgfsetfillopacity{0.555030}%
\pgfsetlinewidth{1.003750pt}%
\definecolor{currentstroke}{rgb}{0.839216,0.152941,0.156863}%
\pgfsetstrokecolor{currentstroke}%
\pgfsetstrokeopacity{0.555030}%
\pgfsetdash{}{0pt}%
\pgfpathmoveto{\pgfqpoint{11.678736in}{2.113424in}}%
\pgfpathcurveto{\pgfqpoint{11.688823in}{2.113424in}}{\pgfqpoint{11.698499in}{2.117431in}}{\pgfqpoint{11.705632in}{2.124564in}}%
\pgfpathcurveto{\pgfqpoint{11.712764in}{2.131697in}}{\pgfqpoint{11.716772in}{2.141372in}}{\pgfqpoint{11.716772in}{2.151460in}}%
\pgfpathcurveto{\pgfqpoint{11.716772in}{2.161547in}}{\pgfqpoint{11.712764in}{2.171223in}}{\pgfqpoint{11.705632in}{2.178356in}}%
\pgfpathcurveto{\pgfqpoint{11.698499in}{2.185488in}}{\pgfqpoint{11.688823in}{2.189496in}}{\pgfqpoint{11.678736in}{2.189496in}}%
\pgfpathcurveto{\pgfqpoint{11.668648in}{2.189496in}}{\pgfqpoint{11.658973in}{2.185488in}}{\pgfqpoint{11.651840in}{2.178356in}}%
\pgfpathcurveto{\pgfqpoint{11.644707in}{2.171223in}}{\pgfqpoint{11.640699in}{2.161547in}}{\pgfqpoint{11.640699in}{2.151460in}}%
\pgfpathcurveto{\pgfqpoint{11.640699in}{2.141372in}}{\pgfqpoint{11.644707in}{2.131697in}}{\pgfqpoint{11.651840in}{2.124564in}}%
\pgfpathcurveto{\pgfqpoint{11.658973in}{2.117431in}}{\pgfqpoint{11.668648in}{2.113424in}}{\pgfqpoint{11.678736in}{2.113424in}}%
\pgfpathlineto{\pgfqpoint{11.678736in}{2.113424in}}%
\pgfpathclose%
\pgfusepath{stroke,fill}%
\end{pgfscope}%
\begin{pgfscope}%
\pgfpathrectangle{\pgfqpoint{8.608921in}{0.208778in}}{\pgfqpoint{3.800000in}{3.800000in}}%
\pgfusepath{clip}%
\pgfsetbuttcap%
\pgfsetroundjoin%
\definecolor{currentfill}{rgb}{0.839216,0.152941,0.156863}%
\pgfsetfillcolor{currentfill}%
\pgfsetfillopacity{0.641254}%
\pgfsetlinewidth{1.003750pt}%
\definecolor{currentstroke}{rgb}{0.839216,0.152941,0.156863}%
\pgfsetstrokecolor{currentstroke}%
\pgfsetstrokeopacity{0.641254}%
\pgfsetdash{}{0pt}%
\pgfpathmoveto{\pgfqpoint{10.828044in}{1.333878in}}%
\pgfpathcurveto{\pgfqpoint{10.838131in}{1.333878in}}{\pgfqpoint{10.847807in}{1.337886in}}{\pgfqpoint{10.854940in}{1.345018in}}%
\pgfpathcurveto{\pgfqpoint{10.862072in}{1.352151in}}{\pgfqpoint{10.866080in}{1.361827in}}{\pgfqpoint{10.866080in}{1.371914in}}%
\pgfpathcurveto{\pgfqpoint{10.866080in}{1.382001in}}{\pgfqpoint{10.862072in}{1.391677in}}{\pgfqpoint{10.854940in}{1.398810in}}%
\pgfpathcurveto{\pgfqpoint{10.847807in}{1.405943in}}{\pgfqpoint{10.838131in}{1.409950in}}{\pgfqpoint{10.828044in}{1.409950in}}%
\pgfpathcurveto{\pgfqpoint{10.817957in}{1.409950in}}{\pgfqpoint{10.808281in}{1.405943in}}{\pgfqpoint{10.801148in}{1.398810in}}%
\pgfpathcurveto{\pgfqpoint{10.794015in}{1.391677in}}{\pgfqpoint{10.790008in}{1.382001in}}{\pgfqpoint{10.790008in}{1.371914in}}%
\pgfpathcurveto{\pgfqpoint{10.790008in}{1.361827in}}{\pgfqpoint{10.794015in}{1.352151in}}{\pgfqpoint{10.801148in}{1.345018in}}%
\pgfpathcurveto{\pgfqpoint{10.808281in}{1.337886in}}{\pgfqpoint{10.817957in}{1.333878in}}{\pgfqpoint{10.828044in}{1.333878in}}%
\pgfpathlineto{\pgfqpoint{10.828044in}{1.333878in}}%
\pgfpathclose%
\pgfusepath{stroke,fill}%
\end{pgfscope}%
\begin{pgfscope}%
\pgfpathrectangle{\pgfqpoint{8.608921in}{0.208778in}}{\pgfqpoint{3.800000in}{3.800000in}}%
\pgfusepath{clip}%
\pgfsetbuttcap%
\pgfsetroundjoin%
\definecolor{currentfill}{rgb}{0.839216,0.152941,0.156863}%
\pgfsetfillcolor{currentfill}%
\pgfsetfillopacity{0.659293}%
\pgfsetlinewidth{1.003750pt}%
\definecolor{currentstroke}{rgb}{0.839216,0.152941,0.156863}%
\pgfsetstrokecolor{currentstroke}%
\pgfsetstrokeopacity{0.659293}%
\pgfsetdash{}{0pt}%
\pgfpathmoveto{\pgfqpoint{9.582571in}{2.149405in}}%
\pgfpathcurveto{\pgfqpoint{9.592658in}{2.149405in}}{\pgfqpoint{9.602333in}{2.153412in}}{\pgfqpoint{9.609466in}{2.160545in}}%
\pgfpathcurveto{\pgfqpoint{9.616599in}{2.167678in}}{\pgfqpoint{9.620607in}{2.177353in}}{\pgfqpoint{9.620607in}{2.187441in}}%
\pgfpathcurveto{\pgfqpoint{9.620607in}{2.197528in}}{\pgfqpoint{9.616599in}{2.207204in}}{\pgfqpoint{9.609466in}{2.214337in}}%
\pgfpathcurveto{\pgfqpoint{9.602333in}{2.221469in}}{\pgfqpoint{9.592658in}{2.225477in}}{\pgfqpoint{9.582571in}{2.225477in}}%
\pgfpathcurveto{\pgfqpoint{9.572483in}{2.225477in}}{\pgfqpoint{9.562808in}{2.221469in}}{\pgfqpoint{9.555675in}{2.214337in}}%
\pgfpathcurveto{\pgfqpoint{9.548542in}{2.207204in}}{\pgfqpoint{9.544534in}{2.197528in}}{\pgfqpoint{9.544534in}{2.187441in}}%
\pgfpathcurveto{\pgfqpoint{9.544534in}{2.177353in}}{\pgfqpoint{9.548542in}{2.167678in}}{\pgfqpoint{9.555675in}{2.160545in}}%
\pgfpathcurveto{\pgfqpoint{9.562808in}{2.153412in}}{\pgfqpoint{9.572483in}{2.149405in}}{\pgfqpoint{9.582571in}{2.149405in}}%
\pgfpathlineto{\pgfqpoint{9.582571in}{2.149405in}}%
\pgfpathclose%
\pgfusepath{stroke,fill}%
\end{pgfscope}%
\begin{pgfscope}%
\pgfpathrectangle{\pgfqpoint{8.608921in}{0.208778in}}{\pgfqpoint{3.800000in}{3.800000in}}%
\pgfusepath{clip}%
\pgfsetbuttcap%
\pgfsetroundjoin%
\definecolor{currentfill}{rgb}{0.839216,0.152941,0.156863}%
\pgfsetfillcolor{currentfill}%
\pgfsetfillopacity{0.661693}%
\pgfsetlinewidth{1.003750pt}%
\definecolor{currentstroke}{rgb}{0.839216,0.152941,0.156863}%
\pgfsetstrokecolor{currentstroke}%
\pgfsetstrokeopacity{0.661693}%
\pgfsetdash{}{0pt}%
\pgfpathmoveto{\pgfqpoint{9.963212in}{2.952263in}}%
\pgfpathcurveto{\pgfqpoint{9.973299in}{2.952263in}}{\pgfqpoint{9.982975in}{2.956271in}}{\pgfqpoint{9.990107in}{2.963404in}}%
\pgfpathcurveto{\pgfqpoint{9.997240in}{2.970536in}}{\pgfqpoint{10.001248in}{2.980212in}}{\pgfqpoint{10.001248in}{2.990299in}}%
\pgfpathcurveto{\pgfqpoint{10.001248in}{3.000387in}}{\pgfqpoint{9.997240in}{3.010062in}}{\pgfqpoint{9.990107in}{3.017195in}}%
\pgfpathcurveto{\pgfqpoint{9.982975in}{3.024328in}}{\pgfqpoint{9.973299in}{3.028336in}}{\pgfqpoint{9.963212in}{3.028336in}}%
\pgfpathcurveto{\pgfqpoint{9.953124in}{3.028336in}}{\pgfqpoint{9.943449in}{3.024328in}}{\pgfqpoint{9.936316in}{3.017195in}}%
\pgfpathcurveto{\pgfqpoint{9.929183in}{3.010062in}}{\pgfqpoint{9.925175in}{3.000387in}}{\pgfqpoint{9.925175in}{2.990299in}}%
\pgfpathcurveto{\pgfqpoint{9.925175in}{2.980212in}}{\pgfqpoint{9.929183in}{2.970536in}}{\pgfqpoint{9.936316in}{2.963404in}}%
\pgfpathcurveto{\pgfqpoint{9.943449in}{2.956271in}}{\pgfqpoint{9.953124in}{2.952263in}}{\pgfqpoint{9.963212in}{2.952263in}}%
\pgfpathlineto{\pgfqpoint{9.963212in}{2.952263in}}%
\pgfpathclose%
\pgfusepath{stroke,fill}%
\end{pgfscope}%
\begin{pgfscope}%
\pgfpathrectangle{\pgfqpoint{8.608921in}{0.208778in}}{\pgfqpoint{3.800000in}{3.800000in}}%
\pgfusepath{clip}%
\pgfsetbuttcap%
\pgfsetroundjoin%
\definecolor{currentfill}{rgb}{0.839216,0.152941,0.156863}%
\pgfsetfillcolor{currentfill}%
\pgfsetfillopacity{0.667667}%
\pgfsetlinewidth{1.003750pt}%
\definecolor{currentstroke}{rgb}{0.839216,0.152941,0.156863}%
\pgfsetstrokecolor{currentstroke}%
\pgfsetstrokeopacity{0.667667}%
\pgfsetdash{}{0pt}%
\pgfpathmoveto{\pgfqpoint{9.950511in}{3.036138in}}%
\pgfpathcurveto{\pgfqpoint{9.960598in}{3.036138in}}{\pgfqpoint{9.970274in}{3.040145in}}{\pgfqpoint{9.977406in}{3.047278in}}%
\pgfpathcurveto{\pgfqpoint{9.984539in}{3.054411in}}{\pgfqpoint{9.988547in}{3.064086in}}{\pgfqpoint{9.988547in}{3.074174in}}%
\pgfpathcurveto{\pgfqpoint{9.988547in}{3.084261in}}{\pgfqpoint{9.984539in}{3.093937in}}{\pgfqpoint{9.977406in}{3.101070in}}%
\pgfpathcurveto{\pgfqpoint{9.970274in}{3.108202in}}{\pgfqpoint{9.960598in}{3.112210in}}{\pgfqpoint{9.950511in}{3.112210in}}%
\pgfpathcurveto{\pgfqpoint{9.940423in}{3.112210in}}{\pgfqpoint{9.930748in}{3.108202in}}{\pgfqpoint{9.923615in}{3.101070in}}%
\pgfpathcurveto{\pgfqpoint{9.916482in}{3.093937in}}{\pgfqpoint{9.912474in}{3.084261in}}{\pgfqpoint{9.912474in}{3.074174in}}%
\pgfpathcurveto{\pgfqpoint{9.912474in}{3.064086in}}{\pgfqpoint{9.916482in}{3.054411in}}{\pgfqpoint{9.923615in}{3.047278in}}%
\pgfpathcurveto{\pgfqpoint{9.930748in}{3.040145in}}{\pgfqpoint{9.940423in}{3.036138in}}{\pgfqpoint{9.950511in}{3.036138in}}%
\pgfpathlineto{\pgfqpoint{9.950511in}{3.036138in}}%
\pgfpathclose%
\pgfusepath{stroke,fill}%
\end{pgfscope}%
\begin{pgfscope}%
\pgfpathrectangle{\pgfqpoint{8.608921in}{0.208778in}}{\pgfqpoint{3.800000in}{3.800000in}}%
\pgfusepath{clip}%
\pgfsetbuttcap%
\pgfsetroundjoin%
\definecolor{currentfill}{rgb}{0.839216,0.152941,0.156863}%
\pgfsetfillcolor{currentfill}%
\pgfsetfillopacity{0.680503}%
\pgfsetlinewidth{1.003750pt}%
\definecolor{currentstroke}{rgb}{0.839216,0.152941,0.156863}%
\pgfsetstrokecolor{currentstroke}%
\pgfsetstrokeopacity{0.680503}%
\pgfsetdash{}{0pt}%
\pgfpathmoveto{\pgfqpoint{11.186275in}{1.370484in}}%
\pgfpathcurveto{\pgfqpoint{11.196362in}{1.370484in}}{\pgfqpoint{11.206037in}{1.374491in}}{\pgfqpoint{11.213170in}{1.381624in}}%
\pgfpathcurveto{\pgfqpoint{11.220303in}{1.388757in}}{\pgfqpoint{11.224311in}{1.398433in}}{\pgfqpoint{11.224311in}{1.408520in}}%
\pgfpathcurveto{\pgfqpoint{11.224311in}{1.418607in}}{\pgfqpoint{11.220303in}{1.428283in}}{\pgfqpoint{11.213170in}{1.435416in}}%
\pgfpathcurveto{\pgfqpoint{11.206037in}{1.442549in}}{\pgfqpoint{11.196362in}{1.446556in}}{\pgfqpoint{11.186275in}{1.446556in}}%
\pgfpathcurveto{\pgfqpoint{11.176187in}{1.446556in}}{\pgfqpoint{11.166512in}{1.442549in}}{\pgfqpoint{11.159379in}{1.435416in}}%
\pgfpathcurveto{\pgfqpoint{11.152246in}{1.428283in}}{\pgfqpoint{11.148238in}{1.418607in}}{\pgfqpoint{11.148238in}{1.408520in}}%
\pgfpathcurveto{\pgfqpoint{11.148238in}{1.398433in}}{\pgfqpoint{11.152246in}{1.388757in}}{\pgfqpoint{11.159379in}{1.381624in}}%
\pgfpathcurveto{\pgfqpoint{11.166512in}{1.374491in}}{\pgfqpoint{11.176187in}{1.370484in}}{\pgfqpoint{11.186275in}{1.370484in}}%
\pgfpathlineto{\pgfqpoint{11.186275in}{1.370484in}}%
\pgfpathclose%
\pgfusepath{stroke,fill}%
\end{pgfscope}%
\begin{pgfscope}%
\pgfpathrectangle{\pgfqpoint{8.608921in}{0.208778in}}{\pgfqpoint{3.800000in}{3.800000in}}%
\pgfusepath{clip}%
\pgfsetbuttcap%
\pgfsetroundjoin%
\definecolor{currentfill}{rgb}{0.839216,0.152941,0.156863}%
\pgfsetfillcolor{currentfill}%
\pgfsetfillopacity{0.795933}%
\pgfsetlinewidth{1.003750pt}%
\definecolor{currentstroke}{rgb}{0.839216,0.152941,0.156863}%
\pgfsetstrokecolor{currentstroke}%
\pgfsetstrokeopacity{0.795933}%
\pgfsetdash{}{0pt}%
\pgfpathmoveto{\pgfqpoint{9.333676in}{2.274547in}}%
\pgfpathcurveto{\pgfqpoint{9.343763in}{2.274547in}}{\pgfqpoint{9.353438in}{2.278555in}}{\pgfqpoint{9.360571in}{2.285688in}}%
\pgfpathcurveto{\pgfqpoint{9.367704in}{2.292821in}}{\pgfqpoint{9.371712in}{2.302496in}}{\pgfqpoint{9.371712in}{2.312584in}}%
\pgfpathcurveto{\pgfqpoint{9.371712in}{2.322671in}}{\pgfqpoint{9.367704in}{2.332346in}}{\pgfqpoint{9.360571in}{2.339479in}}%
\pgfpathcurveto{\pgfqpoint{9.353438in}{2.346612in}}{\pgfqpoint{9.343763in}{2.350620in}}{\pgfqpoint{9.333676in}{2.350620in}}%
\pgfpathcurveto{\pgfqpoint{9.323588in}{2.350620in}}{\pgfqpoint{9.313913in}{2.346612in}}{\pgfqpoint{9.306780in}{2.339479in}}%
\pgfpathcurveto{\pgfqpoint{9.299647in}{2.332346in}}{\pgfqpoint{9.295639in}{2.322671in}}{\pgfqpoint{9.295639in}{2.312584in}}%
\pgfpathcurveto{\pgfqpoint{9.295639in}{2.302496in}}{\pgfqpoint{9.299647in}{2.292821in}}{\pgfqpoint{9.306780in}{2.285688in}}%
\pgfpathcurveto{\pgfqpoint{9.313913in}{2.278555in}}{\pgfqpoint{9.323588in}{2.274547in}}{\pgfqpoint{9.333676in}{2.274547in}}%
\pgfpathlineto{\pgfqpoint{9.333676in}{2.274547in}}%
\pgfpathclose%
\pgfusepath{stroke,fill}%
\end{pgfscope}%
\begin{pgfscope}%
\pgfpathrectangle{\pgfqpoint{8.608921in}{0.208778in}}{\pgfqpoint{3.800000in}{3.800000in}}%
\pgfusepath{clip}%
\pgfsetbuttcap%
\pgfsetroundjoin%
\definecolor{currentfill}{rgb}{0.839216,0.152941,0.156863}%
\pgfsetfillcolor{currentfill}%
\pgfsetfillopacity{0.802324}%
\pgfsetlinewidth{1.003750pt}%
\definecolor{currentstroke}{rgb}{0.839216,0.152941,0.156863}%
\pgfsetstrokecolor{currentstroke}%
\pgfsetstrokeopacity{0.802324}%
\pgfsetdash{}{0pt}%
\pgfpathmoveto{\pgfqpoint{11.665549in}{2.118881in}}%
\pgfpathcurveto{\pgfqpoint{11.675636in}{2.118881in}}{\pgfqpoint{11.685312in}{2.122889in}}{\pgfqpoint{11.692445in}{2.130021in}}%
\pgfpathcurveto{\pgfqpoint{11.699578in}{2.137154in}}{\pgfqpoint{11.703585in}{2.146830in}}{\pgfqpoint{11.703585in}{2.156917in}}%
\pgfpathcurveto{\pgfqpoint{11.703585in}{2.167005in}}{\pgfqpoint{11.699578in}{2.176680in}}{\pgfqpoint{11.692445in}{2.183813in}}%
\pgfpathcurveto{\pgfqpoint{11.685312in}{2.190946in}}{\pgfqpoint{11.675636in}{2.194953in}}{\pgfqpoint{11.665549in}{2.194953in}}%
\pgfpathcurveto{\pgfqpoint{11.655462in}{2.194953in}}{\pgfqpoint{11.645786in}{2.190946in}}{\pgfqpoint{11.638653in}{2.183813in}}%
\pgfpathcurveto{\pgfqpoint{11.631521in}{2.176680in}}{\pgfqpoint{11.627513in}{2.167005in}}{\pgfqpoint{11.627513in}{2.156917in}}%
\pgfpathcurveto{\pgfqpoint{11.627513in}{2.146830in}}{\pgfqpoint{11.631521in}{2.137154in}}{\pgfqpoint{11.638653in}{2.130021in}}%
\pgfpathcurveto{\pgfqpoint{11.645786in}{2.122889in}}{\pgfqpoint{11.655462in}{2.118881in}}{\pgfqpoint{11.665549in}{2.118881in}}%
\pgfpathlineto{\pgfqpoint{11.665549in}{2.118881in}}%
\pgfpathclose%
\pgfusepath{stroke,fill}%
\end{pgfscope}%
\begin{pgfscope}%
\pgfpathrectangle{\pgfqpoint{8.608921in}{0.208778in}}{\pgfqpoint{3.800000in}{3.800000in}}%
\pgfusepath{clip}%
\pgfsetbuttcap%
\pgfsetroundjoin%
\definecolor{currentfill}{rgb}{0.839216,0.152941,0.156863}%
\pgfsetfillcolor{currentfill}%
\pgfsetfillopacity{0.837755}%
\pgfsetlinewidth{1.003750pt}%
\definecolor{currentstroke}{rgb}{0.839216,0.152941,0.156863}%
\pgfsetstrokecolor{currentstroke}%
\pgfsetstrokeopacity{0.837755}%
\pgfsetdash{}{0pt}%
\pgfpathmoveto{\pgfqpoint{11.209011in}{1.339913in}}%
\pgfpathcurveto{\pgfqpoint{11.219099in}{1.339913in}}{\pgfqpoint{11.228774in}{1.343921in}}{\pgfqpoint{11.235907in}{1.351054in}}%
\pgfpathcurveto{\pgfqpoint{11.243040in}{1.358186in}}{\pgfqpoint{11.247047in}{1.367862in}}{\pgfqpoint{11.247047in}{1.377949in}}%
\pgfpathcurveto{\pgfqpoint{11.247047in}{1.388037in}}{\pgfqpoint{11.243040in}{1.397712in}}{\pgfqpoint{11.235907in}{1.404845in}}%
\pgfpathcurveto{\pgfqpoint{11.228774in}{1.411978in}}{\pgfqpoint{11.219099in}{1.415986in}}{\pgfqpoint{11.209011in}{1.415986in}}%
\pgfpathcurveto{\pgfqpoint{11.198924in}{1.415986in}}{\pgfqpoint{11.189248in}{1.411978in}}{\pgfqpoint{11.182115in}{1.404845in}}%
\pgfpathcurveto{\pgfqpoint{11.174983in}{1.397712in}}{\pgfqpoint{11.170975in}{1.388037in}}{\pgfqpoint{11.170975in}{1.377949in}}%
\pgfpathcurveto{\pgfqpoint{11.170975in}{1.367862in}}{\pgfqpoint{11.174983in}{1.358186in}}{\pgfqpoint{11.182115in}{1.351054in}}%
\pgfpathcurveto{\pgfqpoint{11.189248in}{1.343921in}}{\pgfqpoint{11.198924in}{1.339913in}}{\pgfqpoint{11.209011in}{1.339913in}}%
\pgfpathlineto{\pgfqpoint{11.209011in}{1.339913in}}%
\pgfpathclose%
\pgfusepath{stroke,fill}%
\end{pgfscope}%
\begin{pgfscope}%
\pgfpathrectangle{\pgfqpoint{8.608921in}{0.208778in}}{\pgfqpoint{3.800000in}{3.800000in}}%
\pgfusepath{clip}%
\pgfsetbuttcap%
\pgfsetroundjoin%
\definecolor{currentfill}{rgb}{0.839216,0.152941,0.156863}%
\pgfsetfillcolor{currentfill}%
\pgfsetfillopacity{0.913537}%
\pgfsetlinewidth{1.003750pt}%
\definecolor{currentstroke}{rgb}{0.839216,0.152941,0.156863}%
\pgfsetstrokecolor{currentstroke}%
\pgfsetstrokeopacity{0.913537}%
\pgfsetdash{}{0pt}%
\pgfpathmoveto{\pgfqpoint{11.433890in}{1.403746in}}%
\pgfpathcurveto{\pgfqpoint{11.443978in}{1.403746in}}{\pgfqpoint{11.453653in}{1.407754in}}{\pgfqpoint{11.460786in}{1.414887in}}%
\pgfpathcurveto{\pgfqpoint{11.467919in}{1.422019in}}{\pgfqpoint{11.471927in}{1.431695in}}{\pgfqpoint{11.471927in}{1.441782in}}%
\pgfpathcurveto{\pgfqpoint{11.471927in}{1.451870in}}{\pgfqpoint{11.467919in}{1.461545in}}{\pgfqpoint{11.460786in}{1.468678in}}%
\pgfpathcurveto{\pgfqpoint{11.453653in}{1.475811in}}{\pgfqpoint{11.443978in}{1.479819in}}{\pgfqpoint{11.433890in}{1.479819in}}%
\pgfpathcurveto{\pgfqpoint{11.423803in}{1.479819in}}{\pgfqpoint{11.414127in}{1.475811in}}{\pgfqpoint{11.406995in}{1.468678in}}%
\pgfpathcurveto{\pgfqpoint{11.399862in}{1.461545in}}{\pgfqpoint{11.395854in}{1.451870in}}{\pgfqpoint{11.395854in}{1.441782in}}%
\pgfpathcurveto{\pgfqpoint{11.395854in}{1.431695in}}{\pgfqpoint{11.399862in}{1.422019in}}{\pgfqpoint{11.406995in}{1.414887in}}%
\pgfpathcurveto{\pgfqpoint{11.414127in}{1.407754in}}{\pgfqpoint{11.423803in}{1.403746in}}{\pgfqpoint{11.433890in}{1.403746in}}%
\pgfpathlineto{\pgfqpoint{11.433890in}{1.403746in}}%
\pgfpathclose%
\pgfusepath{stroke,fill}%
\end{pgfscope}%
\begin{pgfscope}%
\pgfpathrectangle{\pgfqpoint{8.608921in}{0.208778in}}{\pgfqpoint{3.800000in}{3.800000in}}%
\pgfusepath{clip}%
\pgfsetbuttcap%
\pgfsetroundjoin%
\definecolor{currentfill}{rgb}{0.839216,0.152941,0.156863}%
\pgfsetfillcolor{currentfill}%
\pgfsetlinewidth{1.003750pt}%
\definecolor{currentstroke}{rgb}{0.839216,0.152941,0.156863}%
\pgfsetstrokecolor{currentstroke}%
\pgfsetdash{}{0pt}%
\pgfpathmoveto{\pgfqpoint{11.409529in}{1.307867in}}%
\pgfpathcurveto{\pgfqpoint{11.419616in}{1.307867in}}{\pgfqpoint{11.429292in}{1.311875in}}{\pgfqpoint{11.436424in}{1.319008in}}%
\pgfpathcurveto{\pgfqpoint{11.443557in}{1.326141in}}{\pgfqpoint{11.447565in}{1.335816in}}{\pgfqpoint{11.447565in}{1.345904in}}%
\pgfpathcurveto{\pgfqpoint{11.447565in}{1.355991in}}{\pgfqpoint{11.443557in}{1.365666in}}{\pgfqpoint{11.436424in}{1.372799in}}%
\pgfpathcurveto{\pgfqpoint{11.429292in}{1.379932in}}{\pgfqpoint{11.419616in}{1.383940in}}{\pgfqpoint{11.409529in}{1.383940in}}%
\pgfpathcurveto{\pgfqpoint{11.399441in}{1.383940in}}{\pgfqpoint{11.389766in}{1.379932in}}{\pgfqpoint{11.382633in}{1.372799in}}%
\pgfpathcurveto{\pgfqpoint{11.375500in}{1.365666in}}{\pgfqpoint{11.371492in}{1.355991in}}{\pgfqpoint{11.371492in}{1.345904in}}%
\pgfpathcurveto{\pgfqpoint{11.371492in}{1.335816in}}{\pgfqpoint{11.375500in}{1.326141in}}{\pgfqpoint{11.382633in}{1.319008in}}%
\pgfpathcurveto{\pgfqpoint{11.389766in}{1.311875in}}{\pgfqpoint{11.399441in}{1.307867in}}{\pgfqpoint{11.409529in}{1.307867in}}%
\pgfpathlineto{\pgfqpoint{11.409529in}{1.307867in}}%
\pgfpathclose%
\pgfusepath{stroke,fill}%
\end{pgfscope}%
\begin{pgfscope}%
\pgfpathrectangle{\pgfqpoint{8.608921in}{0.208778in}}{\pgfqpoint{3.800000in}{3.800000in}}%
\pgfusepath{clip}%
\pgfsetbuttcap%
\pgfsetroundjoin%
\definecolor{currentfill}{rgb}{0.074668,0.271519,0.074668}%
\pgfsetfillcolor{currentfill}%
\pgfsetfillopacity{0.200000}%
\pgfsetlinewidth{0.000000pt}%
\definecolor{currentstroke}{rgb}{0.000000,0.000000,0.000000}%
\pgfsetstrokecolor{currentstroke}%
\pgfsetdash{}{0pt}%
\pgfpathmoveto{\pgfqpoint{10.982677in}{2.094973in}}%
\pgfpathlineto{\pgfqpoint{10.771568in}{1.787064in}}%
\pgfpathlineto{\pgfqpoint{10.560272in}{2.083542in}}%
\pgfpathlineto{\pgfqpoint{10.982677in}{2.094973in}}%
\pgfpathclose%
\pgfusepath{fill}%
\end{pgfscope}%
\begin{pgfscope}%
\pgfpathrectangle{\pgfqpoint{8.608921in}{0.208778in}}{\pgfqpoint{3.800000in}{3.800000in}}%
\pgfusepath{clip}%
\pgfsetbuttcap%
\pgfsetroundjoin%
\definecolor{currentfill}{rgb}{0.074668,0.271519,0.074668}%
\pgfsetfillcolor{currentfill}%
\pgfsetfillopacity{0.200000}%
\pgfsetlinewidth{0.000000pt}%
\definecolor{currentstroke}{rgb}{0.000000,0.000000,0.000000}%
\pgfsetstrokecolor{currentstroke}%
\pgfsetdash{}{0pt}%
\pgfpathmoveto{\pgfqpoint{10.560272in}{2.083542in}}%
\pgfpathlineto{\pgfqpoint{10.348976in}{1.787064in}}%
\pgfpathlineto{\pgfqpoint{10.137867in}{2.094973in}}%
\pgfpathlineto{\pgfqpoint{10.560272in}{2.083542in}}%
\pgfpathclose%
\pgfusepath{fill}%
\end{pgfscope}%
\begin{pgfscope}%
\pgfpathrectangle{\pgfqpoint{8.608921in}{0.208778in}}{\pgfqpoint{3.800000in}{3.800000in}}%
\pgfusepath{clip}%
\pgfsetbuttcap%
\pgfsetroundjoin%
\definecolor{currentfill}{rgb}{0.086258,0.313666,0.086258}%
\pgfsetfillcolor{currentfill}%
\pgfsetfillopacity{0.200000}%
\pgfsetlinewidth{0.000000pt}%
\definecolor{currentstroke}{rgb}{0.000000,0.000000,0.000000}%
\pgfsetstrokecolor{currentstroke}%
\pgfsetdash{}{0pt}%
\pgfpathmoveto{\pgfqpoint{10.560272in}{2.083542in}}%
\pgfpathlineto{\pgfqpoint{10.560272in}{2.709791in}}%
\pgfpathlineto{\pgfqpoint{10.982677in}{2.094973in}}%
\pgfpathlineto{\pgfqpoint{10.560272in}{2.083542in}}%
\pgfpathclose%
\pgfusepath{fill}%
\end{pgfscope}%
\begin{pgfscope}%
\pgfpathrectangle{\pgfqpoint{8.608921in}{0.208778in}}{\pgfqpoint{3.800000in}{3.800000in}}%
\pgfusepath{clip}%
\pgfsetbuttcap%
\pgfsetroundjoin%
\definecolor{currentfill}{rgb}{0.086258,0.313666,0.086258}%
\pgfsetfillcolor{currentfill}%
\pgfsetfillopacity{0.200000}%
\pgfsetlinewidth{0.000000pt}%
\definecolor{currentstroke}{rgb}{0.000000,0.000000,0.000000}%
\pgfsetstrokecolor{currentstroke}%
\pgfsetdash{}{0pt}%
\pgfpathmoveto{\pgfqpoint{10.137867in}{2.094973in}}%
\pgfpathlineto{\pgfqpoint{10.560272in}{2.709791in}}%
\pgfpathlineto{\pgfqpoint{10.560272in}{2.083542in}}%
\pgfpathlineto{\pgfqpoint{10.137867in}{2.094973in}}%
\pgfpathclose%
\pgfusepath{fill}%
\end{pgfscope}%
\begin{pgfscope}%
\pgfpathrectangle{\pgfqpoint{8.608921in}{0.208778in}}{\pgfqpoint{3.800000in}{3.800000in}}%
\pgfusepath{clip}%
\pgfsetbuttcap%
\pgfsetroundjoin%
\definecolor{currentfill}{rgb}{0.086061,0.312950,0.086061}%
\pgfsetfillcolor{currentfill}%
\pgfsetfillopacity{0.200000}%
\pgfsetlinewidth{0.000000pt}%
\definecolor{currentstroke}{rgb}{0.000000,0.000000,0.000000}%
\pgfsetstrokecolor{currentstroke}%
\pgfsetdash{}{0pt}%
\pgfpathmoveto{\pgfqpoint{10.961038in}{2.709704in}}%
\pgfpathlineto{\pgfqpoint{10.982677in}{2.094973in}}%
\pgfpathlineto{\pgfqpoint{10.560272in}{2.709791in}}%
\pgfpathlineto{\pgfqpoint{10.961038in}{2.709704in}}%
\pgfpathclose%
\pgfusepath{fill}%
\end{pgfscope}%
\begin{pgfscope}%
\pgfpathrectangle{\pgfqpoint{8.608921in}{0.208778in}}{\pgfqpoint{3.800000in}{3.800000in}}%
\pgfusepath{clip}%
\pgfsetbuttcap%
\pgfsetroundjoin%
\definecolor{currentfill}{rgb}{0.086061,0.312950,0.086061}%
\pgfsetfillcolor{currentfill}%
\pgfsetfillopacity{0.200000}%
\pgfsetlinewidth{0.000000pt}%
\definecolor{currentstroke}{rgb}{0.000000,0.000000,0.000000}%
\pgfsetstrokecolor{currentstroke}%
\pgfsetdash{}{0pt}%
\pgfpathmoveto{\pgfqpoint{10.560272in}{2.709791in}}%
\pgfpathlineto{\pgfqpoint{10.137867in}{2.094973in}}%
\pgfpathlineto{\pgfqpoint{10.159506in}{2.709704in}}%
\pgfpathlineto{\pgfqpoint{10.560272in}{2.709791in}}%
\pgfpathclose%
\pgfusepath{fill}%
\end{pgfscope}%
\begin{pgfscope}%
\pgfpathrectangle{\pgfqpoint{8.608921in}{0.208778in}}{\pgfqpoint{3.800000in}{3.800000in}}%
\pgfusepath{clip}%
\pgfsetbuttcap%
\pgfsetroundjoin%
\definecolor{currentfill}{rgb}{0.075994,0.276341,0.075994}%
\pgfsetfillcolor{currentfill}%
\pgfsetfillopacity{0.200000}%
\pgfsetlinewidth{0.000000pt}%
\definecolor{currentstroke}{rgb}{0.000000,0.000000,0.000000}%
\pgfsetstrokecolor{currentstroke}%
\pgfsetdash{}{0pt}%
\pgfpathmoveto{\pgfqpoint{10.137867in}{2.094973in}}%
\pgfpathlineto{\pgfqpoint{9.948267in}{1.818736in}}%
\pgfpathlineto{\pgfqpoint{9.758068in}{2.125716in}}%
\pgfpathlineto{\pgfqpoint{10.137867in}{2.094973in}}%
\pgfpathclose%
\pgfusepath{fill}%
\end{pgfscope}%
\begin{pgfscope}%
\pgfpathrectangle{\pgfqpoint{8.608921in}{0.208778in}}{\pgfqpoint{3.800000in}{3.800000in}}%
\pgfusepath{clip}%
\pgfsetbuttcap%
\pgfsetroundjoin%
\definecolor{currentfill}{rgb}{0.075994,0.276341,0.075994}%
\pgfsetfillcolor{currentfill}%
\pgfsetfillopacity{0.200000}%
\pgfsetlinewidth{0.000000pt}%
\definecolor{currentstroke}{rgb}{0.000000,0.000000,0.000000}%
\pgfsetstrokecolor{currentstroke}%
\pgfsetdash{}{0pt}%
\pgfpathmoveto{\pgfqpoint{11.362476in}{2.125716in}}%
\pgfpathlineto{\pgfqpoint{11.172277in}{1.818736in}}%
\pgfpathlineto{\pgfqpoint{10.982677in}{2.094973in}}%
\pgfpathlineto{\pgfqpoint{11.362476in}{2.125716in}}%
\pgfpathclose%
\pgfusepath{fill}%
\end{pgfscope}%
\begin{pgfscope}%
\pgfpathrectangle{\pgfqpoint{8.608921in}{0.208778in}}{\pgfqpoint{3.800000in}{3.800000in}}%
\pgfusepath{clip}%
\pgfsetbuttcap%
\pgfsetroundjoin%
\definecolor{currentfill}{rgb}{0.087398,0.317812,0.087398}%
\pgfsetfillcolor{currentfill}%
\pgfsetfillopacity{0.200000}%
\pgfsetlinewidth{0.000000pt}%
\definecolor{currentstroke}{rgb}{0.000000,0.000000,0.000000}%
\pgfsetstrokecolor{currentstroke}%
\pgfsetdash{}{0pt}%
\pgfpathmoveto{\pgfqpoint{10.982677in}{2.094973in}}%
\pgfpathlineto{\pgfqpoint{10.961038in}{2.709704in}}%
\pgfpathlineto{\pgfqpoint{11.362476in}{2.125716in}}%
\pgfpathlineto{\pgfqpoint{10.982677in}{2.094973in}}%
\pgfpathclose%
\pgfusepath{fill}%
\end{pgfscope}%
\begin{pgfscope}%
\pgfpathrectangle{\pgfqpoint{8.608921in}{0.208778in}}{\pgfqpoint{3.800000in}{3.800000in}}%
\pgfusepath{clip}%
\pgfsetbuttcap%
\pgfsetroundjoin%
\definecolor{currentfill}{rgb}{0.087398,0.317812,0.087398}%
\pgfsetfillcolor{currentfill}%
\pgfsetfillopacity{0.200000}%
\pgfsetlinewidth{0.000000pt}%
\definecolor{currentstroke}{rgb}{0.000000,0.000000,0.000000}%
\pgfsetstrokecolor{currentstroke}%
\pgfsetdash{}{0pt}%
\pgfpathmoveto{\pgfqpoint{9.758068in}{2.125716in}}%
\pgfpathlineto{\pgfqpoint{10.159506in}{2.709704in}}%
\pgfpathlineto{\pgfqpoint{10.137867in}{2.094973in}}%
\pgfpathlineto{\pgfqpoint{9.758068in}{2.125716in}}%
\pgfpathclose%
\pgfusepath{fill}%
\end{pgfscope}%
\begin{pgfscope}%
\pgfpathrectangle{\pgfqpoint{8.608921in}{0.208778in}}{\pgfqpoint{3.800000in}{3.800000in}}%
\pgfusepath{clip}%
\pgfsetbuttcap%
\pgfsetroundjoin%
\definecolor{currentfill}{rgb}{0.070209,0.255305,0.070209}%
\pgfsetfillcolor{currentfill}%
\pgfsetfillopacity{0.200000}%
\pgfsetlinewidth{0.000000pt}%
\definecolor{currentstroke}{rgb}{0.000000,0.000000,0.000000}%
\pgfsetstrokecolor{currentstroke}%
\pgfsetdash{}{0pt}%
\pgfpathmoveto{\pgfqpoint{11.118844in}{1.353901in}}%
\pgfpathlineto{\pgfqpoint{10.771568in}{1.787064in}}%
\pgfpathlineto{\pgfqpoint{10.982677in}{2.094973in}}%
\pgfpathlineto{\pgfqpoint{11.118844in}{1.353901in}}%
\pgfpathclose%
\pgfusepath{fill}%
\end{pgfscope}%
\begin{pgfscope}%
\pgfpathrectangle{\pgfqpoint{8.608921in}{0.208778in}}{\pgfqpoint{3.800000in}{3.800000in}}%
\pgfusepath{clip}%
\pgfsetbuttcap%
\pgfsetroundjoin%
\definecolor{currentfill}{rgb}{0.070209,0.255305,0.070209}%
\pgfsetfillcolor{currentfill}%
\pgfsetfillopacity{0.200000}%
\pgfsetlinewidth{0.000000pt}%
\definecolor{currentstroke}{rgb}{0.000000,0.000000,0.000000}%
\pgfsetstrokecolor{currentstroke}%
\pgfsetdash{}{0pt}%
\pgfpathmoveto{\pgfqpoint{10.137867in}{2.094973in}}%
\pgfpathlineto{\pgfqpoint{10.348976in}{1.787064in}}%
\pgfpathlineto{\pgfqpoint{10.001700in}{1.353901in}}%
\pgfpathlineto{\pgfqpoint{10.137867in}{2.094973in}}%
\pgfpathclose%
\pgfusepath{fill}%
\end{pgfscope}%
\begin{pgfscope}%
\pgfpathrectangle{\pgfqpoint{8.608921in}{0.208778in}}{\pgfqpoint{3.800000in}{3.800000in}}%
\pgfusepath{clip}%
\pgfsetbuttcap%
\pgfsetroundjoin%
\definecolor{currentfill}{rgb}{0.098306,0.357475,0.098306}%
\pgfsetfillcolor{currentfill}%
\pgfsetfillopacity{0.200000}%
\pgfsetlinewidth{0.000000pt}%
\definecolor{currentstroke}{rgb}{0.000000,0.000000,0.000000}%
\pgfsetstrokecolor{currentstroke}%
\pgfsetdash{}{0pt}%
\pgfpathmoveto{\pgfqpoint{10.560272in}{2.709791in}}%
\pgfpathlineto{\pgfqpoint{10.751446in}{2.987811in}}%
\pgfpathlineto{\pgfqpoint{10.961038in}{2.709704in}}%
\pgfpathlineto{\pgfqpoint{10.560272in}{2.709791in}}%
\pgfpathclose%
\pgfusepath{fill}%
\end{pgfscope}%
\begin{pgfscope}%
\pgfpathrectangle{\pgfqpoint{8.608921in}{0.208778in}}{\pgfqpoint{3.800000in}{3.800000in}}%
\pgfusepath{clip}%
\pgfsetbuttcap%
\pgfsetroundjoin%
\definecolor{currentfill}{rgb}{0.098306,0.357475,0.098306}%
\pgfsetfillcolor{currentfill}%
\pgfsetfillopacity{0.200000}%
\pgfsetlinewidth{0.000000pt}%
\definecolor{currentstroke}{rgb}{0.000000,0.000000,0.000000}%
\pgfsetstrokecolor{currentstroke}%
\pgfsetdash{}{0pt}%
\pgfpathmoveto{\pgfqpoint{10.159506in}{2.709704in}}%
\pgfpathlineto{\pgfqpoint{10.369098in}{2.987811in}}%
\pgfpathlineto{\pgfqpoint{10.560272in}{2.709791in}}%
\pgfpathlineto{\pgfqpoint{10.159506in}{2.709704in}}%
\pgfpathclose%
\pgfusepath{fill}%
\end{pgfscope}%
\begin{pgfscope}%
\pgfpathrectangle{\pgfqpoint{8.608921in}{0.208778in}}{\pgfqpoint{3.800000in}{3.800000in}}%
\pgfusepath{clip}%
\pgfsetbuttcap%
\pgfsetroundjoin%
\definecolor{currentfill}{rgb}{0.066446,0.241622,0.066446}%
\pgfsetfillcolor{currentfill}%
\pgfsetfillopacity{0.200000}%
\pgfsetlinewidth{0.000000pt}%
\definecolor{currentstroke}{rgb}{0.000000,0.000000,0.000000}%
\pgfsetstrokecolor{currentstroke}%
\pgfsetdash{}{0pt}%
\pgfpathmoveto{\pgfqpoint{10.771568in}{1.787064in}}%
\pgfpathlineto{\pgfqpoint{10.560272in}{1.146147in}}%
\pgfpathlineto{\pgfqpoint{10.560272in}{2.083542in}}%
\pgfpathlineto{\pgfqpoint{10.771568in}{1.787064in}}%
\pgfpathclose%
\pgfusepath{fill}%
\end{pgfscope}%
\begin{pgfscope}%
\pgfpathrectangle{\pgfqpoint{8.608921in}{0.208778in}}{\pgfqpoint{3.800000in}{3.800000in}}%
\pgfusepath{clip}%
\pgfsetbuttcap%
\pgfsetroundjoin%
\definecolor{currentfill}{rgb}{0.066446,0.241622,0.066446}%
\pgfsetfillcolor{currentfill}%
\pgfsetfillopacity{0.200000}%
\pgfsetlinewidth{0.000000pt}%
\definecolor{currentstroke}{rgb}{0.000000,0.000000,0.000000}%
\pgfsetstrokecolor{currentstroke}%
\pgfsetdash{}{0pt}%
\pgfpathmoveto{\pgfqpoint{10.560272in}{2.083542in}}%
\pgfpathlineto{\pgfqpoint{10.560272in}{1.146147in}}%
\pgfpathlineto{\pgfqpoint{10.348976in}{1.787064in}}%
\pgfpathlineto{\pgfqpoint{10.560272in}{2.083542in}}%
\pgfpathclose%
\pgfusepath{fill}%
\end{pgfscope}%
\begin{pgfscope}%
\pgfpathrectangle{\pgfqpoint{8.608921in}{0.208778in}}{\pgfqpoint{3.800000in}{3.800000in}}%
\pgfusepath{clip}%
\pgfsetbuttcap%
\pgfsetroundjoin%
\definecolor{currentfill}{rgb}{0.065035,0.236492,0.065035}%
\pgfsetfillcolor{currentfill}%
\pgfsetfillopacity{0.200000}%
\pgfsetlinewidth{0.000000pt}%
\definecolor{currentstroke}{rgb}{0.000000,0.000000,0.000000}%
\pgfsetstrokecolor{currentstroke}%
\pgfsetdash{}{0pt}%
\pgfpathmoveto{\pgfqpoint{11.118844in}{1.353901in}}%
\pgfpathlineto{\pgfqpoint{10.982677in}{2.094973in}}%
\pgfpathlineto{\pgfqpoint{11.172277in}{1.818736in}}%
\pgfpathlineto{\pgfqpoint{11.118844in}{1.353901in}}%
\pgfpathclose%
\pgfusepath{fill}%
\end{pgfscope}%
\begin{pgfscope}%
\pgfpathrectangle{\pgfqpoint{8.608921in}{0.208778in}}{\pgfqpoint{3.800000in}{3.800000in}}%
\pgfusepath{clip}%
\pgfsetbuttcap%
\pgfsetroundjoin%
\definecolor{currentfill}{rgb}{0.065035,0.236492,0.065035}%
\pgfsetfillcolor{currentfill}%
\pgfsetfillopacity{0.200000}%
\pgfsetlinewidth{0.000000pt}%
\definecolor{currentstroke}{rgb}{0.000000,0.000000,0.000000}%
\pgfsetstrokecolor{currentstroke}%
\pgfsetdash{}{0pt}%
\pgfpathmoveto{\pgfqpoint{9.948267in}{1.818736in}}%
\pgfpathlineto{\pgfqpoint{10.137867in}{2.094973in}}%
\pgfpathlineto{\pgfqpoint{10.001700in}{1.353901in}}%
\pgfpathlineto{\pgfqpoint{9.948267in}{1.818736in}}%
\pgfpathclose%
\pgfusepath{fill}%
\end{pgfscope}%
\begin{pgfscope}%
\pgfpathrectangle{\pgfqpoint{8.608921in}{0.208778in}}{\pgfqpoint{3.800000in}{3.800000in}}%
\pgfusepath{clip}%
\pgfsetbuttcap%
\pgfsetroundjoin%
\definecolor{currentfill}{rgb}{0.101677,0.369734,0.101677}%
\pgfsetfillcolor{currentfill}%
\pgfsetfillopacity{0.200000}%
\pgfsetlinewidth{0.000000pt}%
\definecolor{currentstroke}{rgb}{0.000000,0.000000,0.000000}%
\pgfsetstrokecolor{currentstroke}%
\pgfsetdash{}{0pt}%
\pgfpathmoveto{\pgfqpoint{10.560272in}{2.709791in}}%
\pgfpathlineto{\pgfqpoint{10.369098in}{2.987811in}}%
\pgfpathlineto{\pgfqpoint{10.751446in}{2.987811in}}%
\pgfpathlineto{\pgfqpoint{10.560272in}{2.709791in}}%
\pgfpathclose%
\pgfusepath{fill}%
\end{pgfscope}%
\begin{pgfscope}%
\pgfpathrectangle{\pgfqpoint{8.608921in}{0.208778in}}{\pgfqpoint{3.800000in}{3.800000in}}%
\pgfusepath{clip}%
\pgfsetbuttcap%
\pgfsetroundjoin%
\definecolor{currentfill}{rgb}{0.101759,0.370033,0.101759}%
\pgfsetfillcolor{currentfill}%
\pgfsetfillopacity{0.200000}%
\pgfsetlinewidth{0.000000pt}%
\definecolor{currentstroke}{rgb}{0.000000,0.000000,0.000000}%
\pgfsetstrokecolor{currentstroke}%
\pgfsetdash{}{0pt}%
\pgfpathmoveto{\pgfqpoint{10.961038in}{2.709704in}}%
\pgfpathlineto{\pgfqpoint{10.751446in}{2.987811in}}%
\pgfpathlineto{\pgfqpoint{11.117544in}{2.979787in}}%
\pgfpathlineto{\pgfqpoint{10.961038in}{2.709704in}}%
\pgfpathclose%
\pgfusepath{fill}%
\end{pgfscope}%
\begin{pgfscope}%
\pgfpathrectangle{\pgfqpoint{8.608921in}{0.208778in}}{\pgfqpoint{3.800000in}{3.800000in}}%
\pgfusepath{clip}%
\pgfsetbuttcap%
\pgfsetroundjoin%
\definecolor{currentfill}{rgb}{0.101759,0.370033,0.101759}%
\pgfsetfillcolor{currentfill}%
\pgfsetfillopacity{0.200000}%
\pgfsetlinewidth{0.000000pt}%
\definecolor{currentstroke}{rgb}{0.000000,0.000000,0.000000}%
\pgfsetstrokecolor{currentstroke}%
\pgfsetdash{}{0pt}%
\pgfpathmoveto{\pgfqpoint{10.003000in}{2.979787in}}%
\pgfpathlineto{\pgfqpoint{10.369098in}{2.987811in}}%
\pgfpathlineto{\pgfqpoint{10.159506in}{2.709704in}}%
\pgfpathlineto{\pgfqpoint{10.003000in}{2.979787in}}%
\pgfpathclose%
\pgfusepath{fill}%
\end{pgfscope}%
\begin{pgfscope}%
\pgfpathrectangle{\pgfqpoint{8.608921in}{0.208778in}}{\pgfqpoint{3.800000in}{3.800000in}}%
\pgfusepath{clip}%
\pgfsetbuttcap%
\pgfsetroundjoin%
\definecolor{currentfill}{rgb}{0.091915,0.334238,0.091915}%
\pgfsetfillcolor{currentfill}%
\pgfsetfillopacity{0.200000}%
\pgfsetlinewidth{0.000000pt}%
\definecolor{currentstroke}{rgb}{0.000000,0.000000,0.000000}%
\pgfsetstrokecolor{currentstroke}%
\pgfsetdash{}{0pt}%
\pgfpathmoveto{\pgfqpoint{11.362476in}{2.125716in}}%
\pgfpathlineto{\pgfqpoint{10.961038in}{2.709704in}}%
\pgfpathlineto{\pgfqpoint{11.441310in}{2.965638in}}%
\pgfpathlineto{\pgfqpoint{11.362476in}{2.125716in}}%
\pgfpathclose%
\pgfusepath{fill}%
\end{pgfscope}%
\begin{pgfscope}%
\pgfpathrectangle{\pgfqpoint{8.608921in}{0.208778in}}{\pgfqpoint{3.800000in}{3.800000in}}%
\pgfusepath{clip}%
\pgfsetbuttcap%
\pgfsetroundjoin%
\definecolor{currentfill}{rgb}{0.091915,0.334238,0.091915}%
\pgfsetfillcolor{currentfill}%
\pgfsetfillopacity{0.200000}%
\pgfsetlinewidth{0.000000pt}%
\definecolor{currentstroke}{rgb}{0.000000,0.000000,0.000000}%
\pgfsetstrokecolor{currentstroke}%
\pgfsetdash{}{0pt}%
\pgfpathmoveto{\pgfqpoint{9.679234in}{2.965638in}}%
\pgfpathlineto{\pgfqpoint{10.159506in}{2.709704in}}%
\pgfpathlineto{\pgfqpoint{9.758068in}{2.125716in}}%
\pgfpathlineto{\pgfqpoint{9.679234in}{2.965638in}}%
\pgfpathclose%
\pgfusepath{fill}%
\end{pgfscope}%
\begin{pgfscope}%
\pgfpathrectangle{\pgfqpoint{8.608921in}{0.208778in}}{\pgfqpoint{3.800000in}{3.800000in}}%
\pgfusepath{clip}%
\pgfsetbuttcap%
\pgfsetroundjoin%
\definecolor{currentfill}{rgb}{0.073593,0.267612,0.073593}%
\pgfsetfillcolor{currentfill}%
\pgfsetfillopacity{0.200000}%
\pgfsetlinewidth{0.000000pt}%
\definecolor{currentstroke}{rgb}{0.000000,0.000000,0.000000}%
\pgfsetstrokecolor{currentstroke}%
\pgfsetdash{}{0pt}%
\pgfpathmoveto{\pgfqpoint{11.443260in}{1.423491in}}%
\pgfpathlineto{\pgfqpoint{11.172277in}{1.818736in}}%
\pgfpathlineto{\pgfqpoint{11.362476in}{2.125716in}}%
\pgfpathlineto{\pgfqpoint{11.443260in}{1.423491in}}%
\pgfpathclose%
\pgfusepath{fill}%
\end{pgfscope}%
\begin{pgfscope}%
\pgfpathrectangle{\pgfqpoint{8.608921in}{0.208778in}}{\pgfqpoint{3.800000in}{3.800000in}}%
\pgfusepath{clip}%
\pgfsetbuttcap%
\pgfsetroundjoin%
\definecolor{currentfill}{rgb}{0.073593,0.267612,0.073593}%
\pgfsetfillcolor{currentfill}%
\pgfsetfillopacity{0.200000}%
\pgfsetlinewidth{0.000000pt}%
\definecolor{currentstroke}{rgb}{0.000000,0.000000,0.000000}%
\pgfsetstrokecolor{currentstroke}%
\pgfsetdash{}{0pt}%
\pgfpathmoveto{\pgfqpoint{9.758068in}{2.125716in}}%
\pgfpathlineto{\pgfqpoint{9.948267in}{1.818736in}}%
\pgfpathlineto{\pgfqpoint{9.677284in}{1.423491in}}%
\pgfpathlineto{\pgfqpoint{9.758068in}{2.125716in}}%
\pgfpathclose%
\pgfusepath{fill}%
\end{pgfscope}%
\begin{pgfscope}%
\pgfpathrectangle{\pgfqpoint{8.608921in}{0.208778in}}{\pgfqpoint{3.800000in}{3.800000in}}%
\pgfusepath{clip}%
\pgfsetbuttcap%
\pgfsetroundjoin%
\definecolor{currentfill}{rgb}{0.065434,0.237940,0.065434}%
\pgfsetfillcolor{currentfill}%
\pgfsetfillopacity{0.200000}%
\pgfsetlinewidth{0.000000pt}%
\definecolor{currentstroke}{rgb}{0.000000,0.000000,0.000000}%
\pgfsetstrokecolor{currentstroke}%
\pgfsetdash{}{0pt}%
\pgfpathmoveto{\pgfqpoint{10.560272in}{1.146147in}}%
\pgfpathlineto{\pgfqpoint{10.771568in}{1.787064in}}%
\pgfpathlineto{\pgfqpoint{10.751905in}{1.314424in}}%
\pgfpathlineto{\pgfqpoint{10.560272in}{1.146147in}}%
\pgfpathclose%
\pgfusepath{fill}%
\end{pgfscope}%
\begin{pgfscope}%
\pgfpathrectangle{\pgfqpoint{8.608921in}{0.208778in}}{\pgfqpoint{3.800000in}{3.800000in}}%
\pgfusepath{clip}%
\pgfsetbuttcap%
\pgfsetroundjoin%
\definecolor{currentfill}{rgb}{0.065434,0.237940,0.065434}%
\pgfsetfillcolor{currentfill}%
\pgfsetfillopacity{0.200000}%
\pgfsetlinewidth{0.000000pt}%
\definecolor{currentstroke}{rgb}{0.000000,0.000000,0.000000}%
\pgfsetstrokecolor{currentstroke}%
\pgfsetdash{}{0pt}%
\pgfpathmoveto{\pgfqpoint{10.368639in}{1.314424in}}%
\pgfpathlineto{\pgfqpoint{10.348976in}{1.787064in}}%
\pgfpathlineto{\pgfqpoint{10.560272in}{1.146147in}}%
\pgfpathlineto{\pgfqpoint{10.368639in}{1.314424in}}%
\pgfpathclose%
\pgfusepath{fill}%
\end{pgfscope}%
\begin{pgfscope}%
\pgfpathrectangle{\pgfqpoint{8.608921in}{0.208778in}}{\pgfqpoint{3.800000in}{3.800000in}}%
\pgfusepath{clip}%
\pgfsetbuttcap%
\pgfsetroundjoin%
\definecolor{currentfill}{rgb}{0.067497,0.245443,0.067497}%
\pgfsetfillcolor{currentfill}%
\pgfsetfillopacity{0.200000}%
\pgfsetlinewidth{0.000000pt}%
\definecolor{currentstroke}{rgb}{0.000000,0.000000,0.000000}%
\pgfsetstrokecolor{currentstroke}%
\pgfsetdash{}{0pt}%
\pgfpathmoveto{\pgfqpoint{10.913597in}{1.165803in}}%
\pgfpathlineto{\pgfqpoint{10.751905in}{1.314424in}}%
\pgfpathlineto{\pgfqpoint{10.771568in}{1.787064in}}%
\pgfpathlineto{\pgfqpoint{10.913597in}{1.165803in}}%
\pgfpathclose%
\pgfusepath{fill}%
\end{pgfscope}%
\begin{pgfscope}%
\pgfpathrectangle{\pgfqpoint{8.608921in}{0.208778in}}{\pgfqpoint{3.800000in}{3.800000in}}%
\pgfusepath{clip}%
\pgfsetbuttcap%
\pgfsetroundjoin%
\definecolor{currentfill}{rgb}{0.067497,0.245443,0.067497}%
\pgfsetfillcolor{currentfill}%
\pgfsetfillopacity{0.200000}%
\pgfsetlinewidth{0.000000pt}%
\definecolor{currentstroke}{rgb}{0.000000,0.000000,0.000000}%
\pgfsetstrokecolor{currentstroke}%
\pgfsetdash{}{0pt}%
\pgfpathmoveto{\pgfqpoint{10.348976in}{1.787064in}}%
\pgfpathlineto{\pgfqpoint{10.368639in}{1.314424in}}%
\pgfpathlineto{\pgfqpoint{10.206947in}{1.165803in}}%
\pgfpathlineto{\pgfqpoint{10.348976in}{1.787064in}}%
\pgfpathclose%
\pgfusepath{fill}%
\end{pgfscope}%
\begin{pgfscope}%
\pgfpathrectangle{\pgfqpoint{8.608921in}{0.208778in}}{\pgfqpoint{3.800000in}{3.800000in}}%
\pgfusepath{clip}%
\pgfsetbuttcap%
\pgfsetroundjoin%
\definecolor{currentfill}{rgb}{0.097285,0.353762,0.097285}%
\pgfsetfillcolor{currentfill}%
\pgfsetfillopacity{0.200000}%
\pgfsetlinewidth{0.000000pt}%
\definecolor{currentstroke}{rgb}{0.000000,0.000000,0.000000}%
\pgfsetstrokecolor{currentstroke}%
\pgfsetdash{}{0pt}%
\pgfpathmoveto{\pgfqpoint{11.117544in}{2.979787in}}%
\pgfpathlineto{\pgfqpoint{11.441310in}{2.965638in}}%
\pgfpathlineto{\pgfqpoint{10.961038in}{2.709704in}}%
\pgfpathlineto{\pgfqpoint{11.117544in}{2.979787in}}%
\pgfpathclose%
\pgfusepath{fill}%
\end{pgfscope}%
\begin{pgfscope}%
\pgfpathrectangle{\pgfqpoint{8.608921in}{0.208778in}}{\pgfqpoint{3.800000in}{3.800000in}}%
\pgfusepath{clip}%
\pgfsetbuttcap%
\pgfsetroundjoin%
\definecolor{currentfill}{rgb}{0.097285,0.353762,0.097285}%
\pgfsetfillcolor{currentfill}%
\pgfsetfillopacity{0.200000}%
\pgfsetlinewidth{0.000000pt}%
\definecolor{currentstroke}{rgb}{0.000000,0.000000,0.000000}%
\pgfsetstrokecolor{currentstroke}%
\pgfsetdash{}{0pt}%
\pgfpathmoveto{\pgfqpoint{10.159506in}{2.709704in}}%
\pgfpathlineto{\pgfqpoint{9.679234in}{2.965638in}}%
\pgfpathlineto{\pgfqpoint{10.003000in}{2.979787in}}%
\pgfpathlineto{\pgfqpoint{10.159506in}{2.709704in}}%
\pgfpathclose%
\pgfusepath{fill}%
\end{pgfscope}%
\begin{pgfscope}%
\pgfpathrectangle{\pgfqpoint{8.608921in}{0.208778in}}{\pgfqpoint{3.800000in}{3.800000in}}%
\pgfusepath{clip}%
\pgfsetbuttcap%
\pgfsetroundjoin%
\definecolor{currentfill}{rgb}{0.060562,0.220227,0.060562}%
\pgfsetfillcolor{currentfill}%
\pgfsetfillopacity{0.200000}%
\pgfsetlinewidth{0.000000pt}%
\definecolor{currentstroke}{rgb}{0.000000,0.000000,0.000000}%
\pgfsetstrokecolor{currentstroke}%
\pgfsetdash{}{0pt}%
\pgfpathmoveto{\pgfqpoint{10.913597in}{1.165803in}}%
\pgfpathlineto{\pgfqpoint{10.771568in}{1.787064in}}%
\pgfpathlineto{\pgfqpoint{11.118844in}{1.353901in}}%
\pgfpathlineto{\pgfqpoint{10.913597in}{1.165803in}}%
\pgfpathclose%
\pgfusepath{fill}%
\end{pgfscope}%
\begin{pgfscope}%
\pgfpathrectangle{\pgfqpoint{8.608921in}{0.208778in}}{\pgfqpoint{3.800000in}{3.800000in}}%
\pgfusepath{clip}%
\pgfsetbuttcap%
\pgfsetroundjoin%
\definecolor{currentfill}{rgb}{0.060562,0.220227,0.060562}%
\pgfsetfillcolor{currentfill}%
\pgfsetfillopacity{0.200000}%
\pgfsetlinewidth{0.000000pt}%
\definecolor{currentstroke}{rgb}{0.000000,0.000000,0.000000}%
\pgfsetstrokecolor{currentstroke}%
\pgfsetdash{}{0pt}%
\pgfpathmoveto{\pgfqpoint{10.001700in}{1.353901in}}%
\pgfpathlineto{\pgfqpoint{10.348976in}{1.787064in}}%
\pgfpathlineto{\pgfqpoint{10.206947in}{1.165803in}}%
\pgfpathlineto{\pgfqpoint{10.001700in}{1.353901in}}%
\pgfpathclose%
\pgfusepath{fill}%
\end{pgfscope}%
\begin{pgfscope}%
\pgfpathrectangle{\pgfqpoint{8.608921in}{0.208778in}}{\pgfqpoint{3.800000in}{3.800000in}}%
\pgfusepath{clip}%
\pgfsetbuttcap%
\pgfsetroundjoin%
\definecolor{currentfill}{rgb}{0.092193,0.335248,0.092193}%
\pgfsetfillcolor{currentfill}%
\pgfsetfillopacity{0.200000}%
\pgfsetlinewidth{0.000000pt}%
\definecolor{currentstroke}{rgb}{0.000000,0.000000,0.000000}%
\pgfsetstrokecolor{currentstroke}%
\pgfsetdash{}{0pt}%
\pgfpathmoveto{\pgfqpoint{9.444430in}{2.167791in}}%
\pgfpathlineto{\pgfqpoint{9.679234in}{2.965638in}}%
\pgfpathlineto{\pgfqpoint{9.758068in}{2.125716in}}%
\pgfpathlineto{\pgfqpoint{9.444430in}{2.167791in}}%
\pgfpathclose%
\pgfusepath{fill}%
\end{pgfscope}%
\begin{pgfscope}%
\pgfpathrectangle{\pgfqpoint{8.608921in}{0.208778in}}{\pgfqpoint{3.800000in}{3.800000in}}%
\pgfusepath{clip}%
\pgfsetbuttcap%
\pgfsetroundjoin%
\definecolor{currentfill}{rgb}{0.092193,0.335248,0.092193}%
\pgfsetfillcolor{currentfill}%
\pgfsetfillopacity{0.200000}%
\pgfsetlinewidth{0.000000pt}%
\definecolor{currentstroke}{rgb}{0.000000,0.000000,0.000000}%
\pgfsetstrokecolor{currentstroke}%
\pgfsetdash{}{0pt}%
\pgfpathmoveto{\pgfqpoint{11.362476in}{2.125716in}}%
\pgfpathlineto{\pgfqpoint{11.441310in}{2.965638in}}%
\pgfpathlineto{\pgfqpoint{11.676114in}{2.167791in}}%
\pgfpathlineto{\pgfqpoint{11.362476in}{2.125716in}}%
\pgfpathclose%
\pgfusepath{fill}%
\end{pgfscope}%
\begin{pgfscope}%
\pgfpathrectangle{\pgfqpoint{8.608921in}{0.208778in}}{\pgfqpoint{3.800000in}{3.800000in}}%
\pgfusepath{clip}%
\pgfsetbuttcap%
\pgfsetroundjoin%
\definecolor{currentfill}{rgb}{0.111651,0.406004,0.111651}%
\pgfsetfillcolor{currentfill}%
\pgfsetfillopacity{0.200000}%
\pgfsetlinewidth{0.000000pt}%
\definecolor{currentstroke}{rgb}{0.000000,0.000000,0.000000}%
\pgfsetstrokecolor{currentstroke}%
\pgfsetdash{}{0pt}%
\pgfpathmoveto{\pgfqpoint{10.003000in}{2.979787in}}%
\pgfpathlineto{\pgfqpoint{9.880670in}{3.203663in}}%
\pgfpathlineto{\pgfqpoint{10.369098in}{2.987811in}}%
\pgfpathlineto{\pgfqpoint{10.003000in}{2.979787in}}%
\pgfpathclose%
\pgfusepath{fill}%
\end{pgfscope}%
\begin{pgfscope}%
\pgfpathrectangle{\pgfqpoint{8.608921in}{0.208778in}}{\pgfqpoint{3.800000in}{3.800000in}}%
\pgfusepath{clip}%
\pgfsetbuttcap%
\pgfsetroundjoin%
\definecolor{currentfill}{rgb}{0.111651,0.406004,0.111651}%
\pgfsetfillcolor{currentfill}%
\pgfsetfillopacity{0.200000}%
\pgfsetlinewidth{0.000000pt}%
\definecolor{currentstroke}{rgb}{0.000000,0.000000,0.000000}%
\pgfsetstrokecolor{currentstroke}%
\pgfsetdash{}{0pt}%
\pgfpathmoveto{\pgfqpoint{10.751446in}{2.987811in}}%
\pgfpathlineto{\pgfqpoint{11.239874in}{3.203663in}}%
\pgfpathlineto{\pgfqpoint{11.117544in}{2.979787in}}%
\pgfpathlineto{\pgfqpoint{10.751446in}{2.987811in}}%
\pgfpathclose%
\pgfusepath{fill}%
\end{pgfscope}%
\begin{pgfscope}%
\pgfpathrectangle{\pgfqpoint{8.608921in}{0.208778in}}{\pgfqpoint{3.800000in}{3.800000in}}%
\pgfusepath{clip}%
\pgfsetbuttcap%
\pgfsetroundjoin%
\definecolor{currentfill}{rgb}{0.070885,0.257762,0.070885}%
\pgfsetfillcolor{currentfill}%
\pgfsetfillopacity{0.200000}%
\pgfsetlinewidth{0.000000pt}%
\definecolor{currentstroke}{rgb}{0.000000,0.000000,0.000000}%
\pgfsetstrokecolor{currentstroke}%
\pgfsetdash{}{0pt}%
\pgfpathmoveto{\pgfqpoint{11.172277in}{1.818736in}}%
\pgfpathlineto{\pgfqpoint{11.241808in}{1.220491in}}%
\pgfpathlineto{\pgfqpoint{11.118844in}{1.353901in}}%
\pgfpathlineto{\pgfqpoint{11.172277in}{1.818736in}}%
\pgfpathclose%
\pgfusepath{fill}%
\end{pgfscope}%
\begin{pgfscope}%
\pgfpathrectangle{\pgfqpoint{8.608921in}{0.208778in}}{\pgfqpoint{3.800000in}{3.800000in}}%
\pgfusepath{clip}%
\pgfsetbuttcap%
\pgfsetroundjoin%
\definecolor{currentfill}{rgb}{0.070885,0.257762,0.070885}%
\pgfsetfillcolor{currentfill}%
\pgfsetfillopacity{0.200000}%
\pgfsetlinewidth{0.000000pt}%
\definecolor{currentstroke}{rgb}{0.000000,0.000000,0.000000}%
\pgfsetstrokecolor{currentstroke}%
\pgfsetdash{}{0pt}%
\pgfpathmoveto{\pgfqpoint{10.001700in}{1.353901in}}%
\pgfpathlineto{\pgfqpoint{9.878736in}{1.220491in}}%
\pgfpathlineto{\pgfqpoint{9.948267in}{1.818736in}}%
\pgfpathlineto{\pgfqpoint{10.001700in}{1.353901in}}%
\pgfpathclose%
\pgfusepath{fill}%
\end{pgfscope}%
\begin{pgfscope}%
\pgfpathrectangle{\pgfqpoint{8.608921in}{0.208778in}}{\pgfqpoint{3.800000in}{3.800000in}}%
\pgfusepath{clip}%
\pgfsetbuttcap%
\pgfsetroundjoin%
\definecolor{currentfill}{rgb}{0.070984,0.258123,0.070984}%
\pgfsetfillcolor{currentfill}%
\pgfsetfillopacity{0.200000}%
\pgfsetlinewidth{0.000000pt}%
\definecolor{currentstroke}{rgb}{0.000000,0.000000,0.000000}%
\pgfsetstrokecolor{currentstroke}%
\pgfsetdash{}{0pt}%
\pgfpathmoveto{\pgfqpoint{11.676114in}{2.167791in}}%
\pgfpathlineto{\pgfqpoint{11.713089in}{1.509884in}}%
\pgfpathlineto{\pgfqpoint{11.362476in}{2.125716in}}%
\pgfpathlineto{\pgfqpoint{11.676114in}{2.167791in}}%
\pgfpathclose%
\pgfusepath{fill}%
\end{pgfscope}%
\begin{pgfscope}%
\pgfpathrectangle{\pgfqpoint{8.608921in}{0.208778in}}{\pgfqpoint{3.800000in}{3.800000in}}%
\pgfusepath{clip}%
\pgfsetbuttcap%
\pgfsetroundjoin%
\definecolor{currentfill}{rgb}{0.070984,0.258123,0.070984}%
\pgfsetfillcolor{currentfill}%
\pgfsetfillopacity{0.200000}%
\pgfsetlinewidth{0.000000pt}%
\definecolor{currentstroke}{rgb}{0.000000,0.000000,0.000000}%
\pgfsetstrokecolor{currentstroke}%
\pgfsetdash{}{0pt}%
\pgfpathmoveto{\pgfqpoint{9.758068in}{2.125716in}}%
\pgfpathlineto{\pgfqpoint{9.407455in}{1.509884in}}%
\pgfpathlineto{\pgfqpoint{9.444430in}{2.167791in}}%
\pgfpathlineto{\pgfqpoint{9.758068in}{2.125716in}}%
\pgfpathclose%
\pgfusepath{fill}%
\end{pgfscope}%
\begin{pgfscope}%
\pgfpathrectangle{\pgfqpoint{8.608921in}{0.208778in}}{\pgfqpoint{3.800000in}{3.800000in}}%
\pgfusepath{clip}%
\pgfsetbuttcap%
\pgfsetroundjoin%
\definecolor{currentfill}{rgb}{0.061754,0.224559,0.061754}%
\pgfsetfillcolor{currentfill}%
\pgfsetfillopacity{0.200000}%
\pgfsetlinewidth{0.000000pt}%
\definecolor{currentstroke}{rgb}{0.000000,0.000000,0.000000}%
\pgfsetstrokecolor{currentstroke}%
\pgfsetdash{}{0pt}%
\pgfpathmoveto{\pgfqpoint{11.443260in}{1.423491in}}%
\pgfpathlineto{\pgfqpoint{11.241808in}{1.220491in}}%
\pgfpathlineto{\pgfqpoint{11.172277in}{1.818736in}}%
\pgfpathlineto{\pgfqpoint{11.443260in}{1.423491in}}%
\pgfpathclose%
\pgfusepath{fill}%
\end{pgfscope}%
\begin{pgfscope}%
\pgfpathrectangle{\pgfqpoint{8.608921in}{0.208778in}}{\pgfqpoint{3.800000in}{3.800000in}}%
\pgfusepath{clip}%
\pgfsetbuttcap%
\pgfsetroundjoin%
\definecolor{currentfill}{rgb}{0.061754,0.224559,0.061754}%
\pgfsetfillcolor{currentfill}%
\pgfsetfillopacity{0.200000}%
\pgfsetlinewidth{0.000000pt}%
\definecolor{currentstroke}{rgb}{0.000000,0.000000,0.000000}%
\pgfsetstrokecolor{currentstroke}%
\pgfsetdash{}{0pt}%
\pgfpathmoveto{\pgfqpoint{9.948267in}{1.818736in}}%
\pgfpathlineto{\pgfqpoint{9.878736in}{1.220491in}}%
\pgfpathlineto{\pgfqpoint{9.677284in}{1.423491in}}%
\pgfpathlineto{\pgfqpoint{9.948267in}{1.818736in}}%
\pgfpathclose%
\pgfusepath{fill}%
\end{pgfscope}%
\begin{pgfscope}%
\pgfpathrectangle{\pgfqpoint{8.608921in}{0.208778in}}{\pgfqpoint{3.800000in}{3.800000in}}%
\pgfusepath{clip}%
\pgfsetbuttcap%
\pgfsetroundjoin%
\definecolor{currentfill}{rgb}{0.089078,0.323920,0.089078}%
\pgfsetfillcolor{currentfill}%
\pgfsetfillopacity{0.200000}%
\pgfsetlinewidth{0.000000pt}%
\definecolor{currentstroke}{rgb}{0.000000,0.000000,0.000000}%
\pgfsetstrokecolor{currentstroke}%
\pgfsetdash{}{0pt}%
\pgfpathmoveto{\pgfqpoint{11.676114in}{2.167791in}}%
\pgfpathlineto{\pgfqpoint{11.441310in}{2.965638in}}%
\pgfpathlineto{\pgfqpoint{11.794634in}{2.452346in}}%
\pgfpathlineto{\pgfqpoint{11.676114in}{2.167791in}}%
\pgfpathclose%
\pgfusepath{fill}%
\end{pgfscope}%
\begin{pgfscope}%
\pgfpathrectangle{\pgfqpoint{8.608921in}{0.208778in}}{\pgfqpoint{3.800000in}{3.800000in}}%
\pgfusepath{clip}%
\pgfsetbuttcap%
\pgfsetroundjoin%
\definecolor{currentfill}{rgb}{0.089078,0.323920,0.089078}%
\pgfsetfillcolor{currentfill}%
\pgfsetfillopacity{0.200000}%
\pgfsetlinewidth{0.000000pt}%
\definecolor{currentstroke}{rgb}{0.000000,0.000000,0.000000}%
\pgfsetstrokecolor{currentstroke}%
\pgfsetdash{}{0pt}%
\pgfpathmoveto{\pgfqpoint{9.325910in}{2.452346in}}%
\pgfpathlineto{\pgfqpoint{9.679234in}{2.965638in}}%
\pgfpathlineto{\pgfqpoint{9.444430in}{2.167791in}}%
\pgfpathlineto{\pgfqpoint{9.325910in}{2.452346in}}%
\pgfpathclose%
\pgfusepath{fill}%
\end{pgfscope}%
\begin{pgfscope}%
\pgfpathrectangle{\pgfqpoint{8.608921in}{0.208778in}}{\pgfqpoint{3.800000in}{3.800000in}}%
\pgfusepath{clip}%
\pgfsetbuttcap%
\pgfsetroundjoin%
\definecolor{currentfill}{rgb}{0.107070,0.389346,0.107070}%
\pgfsetfillcolor{currentfill}%
\pgfsetfillopacity{0.200000}%
\pgfsetlinewidth{0.000000pt}%
\definecolor{currentstroke}{rgb}{0.000000,0.000000,0.000000}%
\pgfsetstrokecolor{currentstroke}%
\pgfsetdash{}{0pt}%
\pgfpathmoveto{\pgfqpoint{9.679234in}{2.965638in}}%
\pgfpathlineto{\pgfqpoint{9.880670in}{3.203663in}}%
\pgfpathlineto{\pgfqpoint{10.003000in}{2.979787in}}%
\pgfpathlineto{\pgfqpoint{9.679234in}{2.965638in}}%
\pgfpathclose%
\pgfusepath{fill}%
\end{pgfscope}%
\begin{pgfscope}%
\pgfpathrectangle{\pgfqpoint{8.608921in}{0.208778in}}{\pgfqpoint{3.800000in}{3.800000in}}%
\pgfusepath{clip}%
\pgfsetbuttcap%
\pgfsetroundjoin%
\definecolor{currentfill}{rgb}{0.107070,0.389346,0.107070}%
\pgfsetfillcolor{currentfill}%
\pgfsetfillopacity{0.200000}%
\pgfsetlinewidth{0.000000pt}%
\definecolor{currentstroke}{rgb}{0.000000,0.000000,0.000000}%
\pgfsetstrokecolor{currentstroke}%
\pgfsetdash{}{0pt}%
\pgfpathmoveto{\pgfqpoint{11.117544in}{2.979787in}}%
\pgfpathlineto{\pgfqpoint{11.239874in}{3.203663in}}%
\pgfpathlineto{\pgfqpoint{11.441310in}{2.965638in}}%
\pgfpathlineto{\pgfqpoint{11.117544in}{2.979787in}}%
\pgfpathclose%
\pgfusepath{fill}%
\end{pgfscope}%
\begin{pgfscope}%
\pgfpathrectangle{\pgfqpoint{8.608921in}{0.208778in}}{\pgfqpoint{3.800000in}{3.800000in}}%
\pgfusepath{clip}%
\pgfsetbuttcap%
\pgfsetroundjoin%
\definecolor{currentfill}{rgb}{0.056200,0.204363,0.056200}%
\pgfsetfillcolor{currentfill}%
\pgfsetfillopacity{0.200000}%
\pgfsetlinewidth{0.000000pt}%
\definecolor{currentstroke}{rgb}{0.000000,0.000000,0.000000}%
\pgfsetstrokecolor{currentstroke}%
\pgfsetdash{}{0pt}%
\pgfpathmoveto{\pgfqpoint{10.560272in}{1.146147in}}%
\pgfpathlineto{\pgfqpoint{10.751905in}{1.314424in}}%
\pgfpathlineto{\pgfqpoint{10.913597in}{1.165803in}}%
\pgfpathlineto{\pgfqpoint{10.560272in}{1.146147in}}%
\pgfpathclose%
\pgfusepath{fill}%
\end{pgfscope}%
\begin{pgfscope}%
\pgfpathrectangle{\pgfqpoint{8.608921in}{0.208778in}}{\pgfqpoint{3.800000in}{3.800000in}}%
\pgfusepath{clip}%
\pgfsetbuttcap%
\pgfsetroundjoin%
\definecolor{currentfill}{rgb}{0.056200,0.204363,0.056200}%
\pgfsetfillcolor{currentfill}%
\pgfsetfillopacity{0.200000}%
\pgfsetlinewidth{0.000000pt}%
\definecolor{currentstroke}{rgb}{0.000000,0.000000,0.000000}%
\pgfsetstrokecolor{currentstroke}%
\pgfsetdash{}{0pt}%
\pgfpathmoveto{\pgfqpoint{10.206947in}{1.165803in}}%
\pgfpathlineto{\pgfqpoint{10.368639in}{1.314424in}}%
\pgfpathlineto{\pgfqpoint{10.560272in}{1.146147in}}%
\pgfpathlineto{\pgfqpoint{10.206947in}{1.165803in}}%
\pgfpathclose%
\pgfusepath{fill}%
\end{pgfscope}%
\begin{pgfscope}%
\pgfpathrectangle{\pgfqpoint{8.608921in}{0.208778in}}{\pgfqpoint{3.800000in}{3.800000in}}%
\pgfusepath{clip}%
\pgfsetbuttcap%
\pgfsetroundjoin%
\definecolor{currentfill}{rgb}{0.086498,0.314539,0.086498}%
\pgfsetfillcolor{currentfill}%
\pgfsetfillopacity{0.200000}%
\pgfsetlinewidth{0.000000pt}%
\definecolor{currentstroke}{rgb}{0.000000,0.000000,0.000000}%
\pgfsetstrokecolor{currentstroke}%
\pgfsetdash{}{0pt}%
\pgfpathmoveto{\pgfqpoint{11.676114in}{2.167791in}}%
\pgfpathlineto{\pgfqpoint{11.794634in}{2.452346in}}%
\pgfpathlineto{\pgfqpoint{11.921408in}{2.213486in}}%
\pgfpathlineto{\pgfqpoint{11.676114in}{2.167791in}}%
\pgfpathclose%
\pgfusepath{fill}%
\end{pgfscope}%
\begin{pgfscope}%
\pgfpathrectangle{\pgfqpoint{8.608921in}{0.208778in}}{\pgfqpoint{3.800000in}{3.800000in}}%
\pgfusepath{clip}%
\pgfsetbuttcap%
\pgfsetroundjoin%
\definecolor{currentfill}{rgb}{0.086498,0.314539,0.086498}%
\pgfsetfillcolor{currentfill}%
\pgfsetfillopacity{0.200000}%
\pgfsetlinewidth{0.000000pt}%
\definecolor{currentstroke}{rgb}{0.000000,0.000000,0.000000}%
\pgfsetstrokecolor{currentstroke}%
\pgfsetdash{}{0pt}%
\pgfpathmoveto{\pgfqpoint{9.199136in}{2.213486in}}%
\pgfpathlineto{\pgfqpoint{9.325910in}{2.452346in}}%
\pgfpathlineto{\pgfqpoint{9.444430in}{2.167791in}}%
\pgfpathlineto{\pgfqpoint{9.199136in}{2.213486in}}%
\pgfpathclose%
\pgfusepath{fill}%
\end{pgfscope}%
\begin{pgfscope}%
\pgfpathrectangle{\pgfqpoint{8.608921in}{0.208778in}}{\pgfqpoint{3.800000in}{3.800000in}}%
\pgfusepath{clip}%
\pgfsetbuttcap%
\pgfsetroundjoin%
\definecolor{currentfill}{rgb}{0.090812,0.330224,0.090812}%
\pgfsetfillcolor{currentfill}%
\pgfsetfillopacity{0.200000}%
\pgfsetlinewidth{0.000000pt}%
\definecolor{currentstroke}{rgb}{0.000000,0.000000,0.000000}%
\pgfsetstrokecolor{currentstroke}%
\pgfsetdash{}{0pt}%
\pgfpathmoveto{\pgfqpoint{11.362476in}{2.125716in}}%
\pgfpathlineto{\pgfqpoint{11.528006in}{1.299711in}}%
\pgfpathlineto{\pgfqpoint{11.443260in}{1.423491in}}%
\pgfpathlineto{\pgfqpoint{11.362476in}{2.125716in}}%
\pgfpathclose%
\pgfusepath{fill}%
\end{pgfscope}%
\begin{pgfscope}%
\pgfpathrectangle{\pgfqpoint{8.608921in}{0.208778in}}{\pgfqpoint{3.800000in}{3.800000in}}%
\pgfusepath{clip}%
\pgfsetbuttcap%
\pgfsetroundjoin%
\definecolor{currentfill}{rgb}{0.090812,0.330224,0.090812}%
\pgfsetfillcolor{currentfill}%
\pgfsetfillopacity{0.200000}%
\pgfsetlinewidth{0.000000pt}%
\definecolor{currentstroke}{rgb}{0.000000,0.000000,0.000000}%
\pgfsetstrokecolor{currentstroke}%
\pgfsetdash{}{0pt}%
\pgfpathmoveto{\pgfqpoint{9.677284in}{1.423491in}}%
\pgfpathlineto{\pgfqpoint{9.592538in}{1.299711in}}%
\pgfpathlineto{\pgfqpoint{9.758068in}{2.125716in}}%
\pgfpathlineto{\pgfqpoint{9.677284in}{1.423491in}}%
\pgfpathclose%
\pgfusepath{fill}%
\end{pgfscope}%
\begin{pgfscope}%
\pgfpathrectangle{\pgfqpoint{8.608921in}{0.208778in}}{\pgfqpoint{3.800000in}{3.800000in}}%
\pgfusepath{clip}%
\pgfsetbuttcap%
\pgfsetroundjoin%
\definecolor{currentfill}{rgb}{0.116321,0.422987,0.116321}%
\pgfsetfillcolor{currentfill}%
\pgfsetfillopacity{0.200000}%
\pgfsetlinewidth{0.000000pt}%
\definecolor{currentstroke}{rgb}{0.000000,0.000000,0.000000}%
\pgfsetstrokecolor{currentstroke}%
\pgfsetdash{}{0pt}%
\pgfpathmoveto{\pgfqpoint{10.751446in}{2.987811in}}%
\pgfpathlineto{\pgfqpoint{10.369098in}{2.987811in}}%
\pgfpathlineto{\pgfqpoint{10.420874in}{3.722935in}}%
\pgfpathlineto{\pgfqpoint{10.751446in}{2.987811in}}%
\pgfpathclose%
\pgfusepath{fill}%
\end{pgfscope}%
\begin{pgfscope}%
\pgfpathrectangle{\pgfqpoint{8.608921in}{0.208778in}}{\pgfqpoint{3.800000in}{3.800000in}}%
\pgfusepath{clip}%
\pgfsetbuttcap%
\pgfsetroundjoin%
\definecolor{currentfill}{rgb}{0.057724,0.209904,0.057724}%
\pgfsetfillcolor{currentfill}%
\pgfsetfillopacity{0.200000}%
\pgfsetlinewidth{0.000000pt}%
\definecolor{currentstroke}{rgb}{0.000000,0.000000,0.000000}%
\pgfsetstrokecolor{currentstroke}%
\pgfsetdash{}{0pt}%
\pgfpathmoveto{\pgfqpoint{11.118844in}{1.353901in}}%
\pgfpathlineto{\pgfqpoint{11.241808in}{1.220491in}}%
\pgfpathlineto{\pgfqpoint{10.913597in}{1.165803in}}%
\pgfpathlineto{\pgfqpoint{11.118844in}{1.353901in}}%
\pgfpathclose%
\pgfusepath{fill}%
\end{pgfscope}%
\begin{pgfscope}%
\pgfpathrectangle{\pgfqpoint{8.608921in}{0.208778in}}{\pgfqpoint{3.800000in}{3.800000in}}%
\pgfusepath{clip}%
\pgfsetbuttcap%
\pgfsetroundjoin%
\definecolor{currentfill}{rgb}{0.057724,0.209904,0.057724}%
\pgfsetfillcolor{currentfill}%
\pgfsetfillopacity{0.200000}%
\pgfsetlinewidth{0.000000pt}%
\definecolor{currentstroke}{rgb}{0.000000,0.000000,0.000000}%
\pgfsetstrokecolor{currentstroke}%
\pgfsetdash{}{0pt}%
\pgfpathmoveto{\pgfqpoint{10.206947in}{1.165803in}}%
\pgfpathlineto{\pgfqpoint{9.878736in}{1.220491in}}%
\pgfpathlineto{\pgfqpoint{10.001700in}{1.353901in}}%
\pgfpathlineto{\pgfqpoint{10.206947in}{1.165803in}}%
\pgfpathclose%
\pgfusepath{fill}%
\end{pgfscope}%
\begin{pgfscope}%
\pgfpathrectangle{\pgfqpoint{8.608921in}{0.208778in}}{\pgfqpoint{3.800000in}{3.800000in}}%
\pgfusepath{clip}%
\pgfsetbuttcap%
\pgfsetroundjoin%
\definecolor{currentfill}{rgb}{0.064867,0.235879,0.064867}%
\pgfsetfillcolor{currentfill}%
\pgfsetfillopacity{0.200000}%
\pgfsetlinewidth{0.000000pt}%
\definecolor{currentstroke}{rgb}{0.000000,0.000000,0.000000}%
\pgfsetstrokecolor{currentstroke}%
\pgfsetdash{}{0pt}%
\pgfpathmoveto{\pgfqpoint{11.713089in}{1.509884in}}%
\pgfpathlineto{\pgfqpoint{11.528006in}{1.299711in}}%
\pgfpathlineto{\pgfqpoint{11.362476in}{2.125716in}}%
\pgfpathlineto{\pgfqpoint{11.713089in}{1.509884in}}%
\pgfpathclose%
\pgfusepath{fill}%
\end{pgfscope}%
\begin{pgfscope}%
\pgfpathrectangle{\pgfqpoint{8.608921in}{0.208778in}}{\pgfqpoint{3.800000in}{3.800000in}}%
\pgfusepath{clip}%
\pgfsetbuttcap%
\pgfsetroundjoin%
\definecolor{currentfill}{rgb}{0.064867,0.235879,0.064867}%
\pgfsetfillcolor{currentfill}%
\pgfsetfillopacity{0.200000}%
\pgfsetlinewidth{0.000000pt}%
\definecolor{currentstroke}{rgb}{0.000000,0.000000,0.000000}%
\pgfsetstrokecolor{currentstroke}%
\pgfsetdash{}{0pt}%
\pgfpathmoveto{\pgfqpoint{9.758068in}{2.125716in}}%
\pgfpathlineto{\pgfqpoint{9.592538in}{1.299711in}}%
\pgfpathlineto{\pgfqpoint{9.407455in}{1.509884in}}%
\pgfpathlineto{\pgfqpoint{9.758068in}{2.125716in}}%
\pgfpathclose%
\pgfusepath{fill}%
\end{pgfscope}%
\begin{pgfscope}%
\pgfpathrectangle{\pgfqpoint{8.608921in}{0.208778in}}{\pgfqpoint{3.800000in}{3.800000in}}%
\pgfusepath{clip}%
\pgfsetbuttcap%
\pgfsetroundjoin%
\definecolor{currentfill}{rgb}{0.116785,0.424671,0.116785}%
\pgfsetfillcolor{currentfill}%
\pgfsetfillopacity{0.200000}%
\pgfsetlinewidth{0.000000pt}%
\definecolor{currentstroke}{rgb}{0.000000,0.000000,0.000000}%
\pgfsetstrokecolor{currentstroke}%
\pgfsetdash{}{0pt}%
\pgfpathmoveto{\pgfqpoint{9.880670in}{3.203663in}}%
\pgfpathlineto{\pgfqpoint{10.420874in}{3.722935in}}%
\pgfpathlineto{\pgfqpoint{10.369098in}{2.987811in}}%
\pgfpathlineto{\pgfqpoint{9.880670in}{3.203663in}}%
\pgfpathclose%
\pgfusepath{fill}%
\end{pgfscope}%
\begin{pgfscope}%
\pgfpathrectangle{\pgfqpoint{8.608921in}{0.208778in}}{\pgfqpoint{3.800000in}{3.800000in}}%
\pgfusepath{clip}%
\pgfsetbuttcap%
\pgfsetroundjoin%
\definecolor{currentfill}{rgb}{0.116785,0.424671,0.116785}%
\pgfsetfillcolor{currentfill}%
\pgfsetfillopacity{0.200000}%
\pgfsetlinewidth{0.000000pt}%
\definecolor{currentstroke}{rgb}{0.000000,0.000000,0.000000}%
\pgfsetstrokecolor{currentstroke}%
\pgfsetdash{}{0pt}%
\pgfpathmoveto{\pgfqpoint{10.751446in}{2.987811in}}%
\pgfpathlineto{\pgfqpoint{10.699670in}{3.722935in}}%
\pgfpathlineto{\pgfqpoint{11.239874in}{3.203663in}}%
\pgfpathlineto{\pgfqpoint{10.751446in}{2.987811in}}%
\pgfpathclose%
\pgfusepath{fill}%
\end{pgfscope}%
\begin{pgfscope}%
\pgfpathrectangle{\pgfqpoint{8.608921in}{0.208778in}}{\pgfqpoint{3.800000in}{3.800000in}}%
\pgfusepath{clip}%
\pgfsetbuttcap%
\pgfsetroundjoin%
\definecolor{currentfill}{rgb}{0.074506,0.270932,0.074506}%
\pgfsetfillcolor{currentfill}%
\pgfsetfillopacity{0.200000}%
\pgfsetlinewidth{0.000000pt}%
\definecolor{currentstroke}{rgb}{0.000000,0.000000,0.000000}%
\pgfsetstrokecolor{currentstroke}%
\pgfsetdash{}{0pt}%
\pgfpathmoveto{\pgfqpoint{11.921408in}{2.213486in}}%
\pgfpathlineto{\pgfqpoint{11.928912in}{1.601411in}}%
\pgfpathlineto{\pgfqpoint{11.676114in}{2.167791in}}%
\pgfpathlineto{\pgfqpoint{11.921408in}{2.213486in}}%
\pgfpathclose%
\pgfusepath{fill}%
\end{pgfscope}%
\begin{pgfscope}%
\pgfpathrectangle{\pgfqpoint{8.608921in}{0.208778in}}{\pgfqpoint{3.800000in}{3.800000in}}%
\pgfusepath{clip}%
\pgfsetbuttcap%
\pgfsetroundjoin%
\definecolor{currentfill}{rgb}{0.074506,0.270932,0.074506}%
\pgfsetfillcolor{currentfill}%
\pgfsetfillopacity{0.200000}%
\pgfsetlinewidth{0.000000pt}%
\definecolor{currentstroke}{rgb}{0.000000,0.000000,0.000000}%
\pgfsetstrokecolor{currentstroke}%
\pgfsetdash{}{0pt}%
\pgfpathmoveto{\pgfqpoint{9.444430in}{2.167791in}}%
\pgfpathlineto{\pgfqpoint{9.191632in}{1.601411in}}%
\pgfpathlineto{\pgfqpoint{9.199136in}{2.213486in}}%
\pgfpathlineto{\pgfqpoint{9.444430in}{2.167791in}}%
\pgfpathclose%
\pgfusepath{fill}%
\end{pgfscope}%
\begin{pgfscope}%
\pgfpathrectangle{\pgfqpoint{8.608921in}{0.208778in}}{\pgfqpoint{3.800000in}{3.800000in}}%
\pgfusepath{clip}%
\pgfsetbuttcap%
\pgfsetroundjoin%
\definecolor{currentfill}{rgb}{0.060435,0.219763,0.060435}%
\pgfsetfillcolor{currentfill}%
\pgfsetfillopacity{0.200000}%
\pgfsetlinewidth{0.000000pt}%
\definecolor{currentstroke}{rgb}{0.000000,0.000000,0.000000}%
\pgfsetstrokecolor{currentstroke}%
\pgfsetdash{}{0pt}%
\pgfpathmoveto{\pgfqpoint{11.528006in}{1.299711in}}%
\pgfpathlineto{\pgfqpoint{11.241808in}{1.220491in}}%
\pgfpathlineto{\pgfqpoint{11.443260in}{1.423491in}}%
\pgfpathlineto{\pgfqpoint{11.528006in}{1.299711in}}%
\pgfpathclose%
\pgfusepath{fill}%
\end{pgfscope}%
\begin{pgfscope}%
\pgfpathrectangle{\pgfqpoint{8.608921in}{0.208778in}}{\pgfqpoint{3.800000in}{3.800000in}}%
\pgfusepath{clip}%
\pgfsetbuttcap%
\pgfsetroundjoin%
\definecolor{currentfill}{rgb}{0.060435,0.219763,0.060435}%
\pgfsetfillcolor{currentfill}%
\pgfsetfillopacity{0.200000}%
\pgfsetlinewidth{0.000000pt}%
\definecolor{currentstroke}{rgb}{0.000000,0.000000,0.000000}%
\pgfsetstrokecolor{currentstroke}%
\pgfsetdash{}{0pt}%
\pgfpathmoveto{\pgfqpoint{9.677284in}{1.423491in}}%
\pgfpathlineto{\pgfqpoint{9.878736in}{1.220491in}}%
\pgfpathlineto{\pgfqpoint{9.592538in}{1.299711in}}%
\pgfpathlineto{\pgfqpoint{9.677284in}{1.423491in}}%
\pgfpathclose%
\pgfusepath{fill}%
\end{pgfscope}%
\begin{pgfscope}%
\pgfpathrectangle{\pgfqpoint{8.608921in}{0.208778in}}{\pgfqpoint{3.800000in}{3.800000in}}%
\pgfusepath{clip}%
\pgfsetbuttcap%
\pgfsetroundjoin%
\definecolor{currentfill}{rgb}{0.095351,0.346729,0.095351}%
\pgfsetfillcolor{currentfill}%
\pgfsetfillopacity{0.200000}%
\pgfsetlinewidth{0.000000pt}%
\definecolor{currentstroke}{rgb}{0.000000,0.000000,0.000000}%
\pgfsetstrokecolor{currentstroke}%
\pgfsetdash{}{0pt}%
\pgfpathmoveto{\pgfqpoint{11.676114in}{2.167791in}}%
\pgfpathlineto{\pgfqpoint{11.766201in}{1.391584in}}%
\pgfpathlineto{\pgfqpoint{11.713089in}{1.509884in}}%
\pgfpathlineto{\pgfqpoint{11.676114in}{2.167791in}}%
\pgfpathclose%
\pgfusepath{fill}%
\end{pgfscope}%
\begin{pgfscope}%
\pgfpathrectangle{\pgfqpoint{8.608921in}{0.208778in}}{\pgfqpoint{3.800000in}{3.800000in}}%
\pgfusepath{clip}%
\pgfsetbuttcap%
\pgfsetroundjoin%
\definecolor{currentfill}{rgb}{0.095351,0.346729,0.095351}%
\pgfsetfillcolor{currentfill}%
\pgfsetfillopacity{0.200000}%
\pgfsetlinewidth{0.000000pt}%
\definecolor{currentstroke}{rgb}{0.000000,0.000000,0.000000}%
\pgfsetstrokecolor{currentstroke}%
\pgfsetdash{}{0pt}%
\pgfpathmoveto{\pgfqpoint{9.407455in}{1.509884in}}%
\pgfpathlineto{\pgfqpoint{9.354343in}{1.391584in}}%
\pgfpathlineto{\pgfqpoint{9.444430in}{2.167791in}}%
\pgfpathlineto{\pgfqpoint{9.407455in}{1.509884in}}%
\pgfpathclose%
\pgfusepath{fill}%
\end{pgfscope}%
\begin{pgfscope}%
\pgfpathrectangle{\pgfqpoint{8.608921in}{0.208778in}}{\pgfqpoint{3.800000in}{3.800000in}}%
\pgfusepath{clip}%
\pgfsetbuttcap%
\pgfsetroundjoin%
\definecolor{currentfill}{rgb}{0.067061,0.243857,0.067061}%
\pgfsetfillcolor{currentfill}%
\pgfsetfillopacity{0.200000}%
\pgfsetlinewidth{0.000000pt}%
\definecolor{currentstroke}{rgb}{0.000000,0.000000,0.000000}%
\pgfsetstrokecolor{currentstroke}%
\pgfsetdash{}{0pt}%
\pgfpathmoveto{\pgfqpoint{11.928912in}{1.601411in}}%
\pgfpathlineto{\pgfqpoint{11.766201in}{1.391584in}}%
\pgfpathlineto{\pgfqpoint{11.676114in}{2.167791in}}%
\pgfpathlineto{\pgfqpoint{11.928912in}{1.601411in}}%
\pgfpathclose%
\pgfusepath{fill}%
\end{pgfscope}%
\begin{pgfscope}%
\pgfpathrectangle{\pgfqpoint{8.608921in}{0.208778in}}{\pgfqpoint{3.800000in}{3.800000in}}%
\pgfusepath{clip}%
\pgfsetbuttcap%
\pgfsetroundjoin%
\definecolor{currentfill}{rgb}{0.067061,0.243857,0.067061}%
\pgfsetfillcolor{currentfill}%
\pgfsetfillopacity{0.200000}%
\pgfsetlinewidth{0.000000pt}%
\definecolor{currentstroke}{rgb}{0.000000,0.000000,0.000000}%
\pgfsetstrokecolor{currentstroke}%
\pgfsetdash{}{0pt}%
\pgfpathmoveto{\pgfqpoint{9.444430in}{2.167791in}}%
\pgfpathlineto{\pgfqpoint{9.354343in}{1.391584in}}%
\pgfpathlineto{\pgfqpoint{9.191632in}{1.601411in}}%
\pgfpathlineto{\pgfqpoint{9.444430in}{2.167791in}}%
\pgfpathclose%
\pgfusepath{fill}%
\end{pgfscope}%
\begin{pgfscope}%
\pgfpathrectangle{\pgfqpoint{8.608921in}{0.208778in}}{\pgfqpoint{3.800000in}{3.800000in}}%
\pgfusepath{clip}%
\pgfsetbuttcap%
\pgfsetroundjoin%
\definecolor{currentfill}{rgb}{0.116321,0.422987,0.116321}%
\pgfsetfillcolor{currentfill}%
\pgfsetfillopacity{0.200000}%
\pgfsetlinewidth{0.000000pt}%
\definecolor{currentstroke}{rgb}{0.000000,0.000000,0.000000}%
\pgfsetstrokecolor{currentstroke}%
\pgfsetdash{}{0pt}%
\pgfpathmoveto{\pgfqpoint{10.420874in}{3.722935in}}%
\pgfpathlineto{\pgfqpoint{10.699670in}{3.722935in}}%
\pgfpathlineto{\pgfqpoint{10.751446in}{2.987811in}}%
\pgfpathlineto{\pgfqpoint{10.420874in}{3.722935in}}%
\pgfpathclose%
\pgfusepath{fill}%
\end{pgfscope}%
\begin{pgfscope}%
\pgfpathrectangle{\pgfqpoint{8.608921in}{0.208778in}}{\pgfqpoint{3.800000in}{3.800000in}}%
\pgfusepath{clip}%
\pgfsetbuttcap%
\pgfsetroundjoin%
\definecolor{currentfill}{rgb}{0.063840,0.232145,0.063840}%
\pgfsetfillcolor{currentfill}%
\pgfsetfillopacity{0.200000}%
\pgfsetlinewidth{0.000000pt}%
\definecolor{currentstroke}{rgb}{0.000000,0.000000,0.000000}%
\pgfsetstrokecolor{currentstroke}%
\pgfsetdash{}{0pt}%
\pgfpathmoveto{\pgfqpoint{11.528006in}{1.299711in}}%
\pgfpathlineto{\pgfqpoint{11.713089in}{1.509884in}}%
\pgfpathlineto{\pgfqpoint{11.766201in}{1.391584in}}%
\pgfpathlineto{\pgfqpoint{11.528006in}{1.299711in}}%
\pgfpathclose%
\pgfusepath{fill}%
\end{pgfscope}%
\begin{pgfscope}%
\pgfpathrectangle{\pgfqpoint{8.608921in}{0.208778in}}{\pgfqpoint{3.800000in}{3.800000in}}%
\pgfusepath{clip}%
\pgfsetbuttcap%
\pgfsetroundjoin%
\definecolor{currentfill}{rgb}{0.063840,0.232145,0.063840}%
\pgfsetfillcolor{currentfill}%
\pgfsetfillopacity{0.200000}%
\pgfsetlinewidth{0.000000pt}%
\definecolor{currentstroke}{rgb}{0.000000,0.000000,0.000000}%
\pgfsetstrokecolor{currentstroke}%
\pgfsetdash{}{0pt}%
\pgfpathmoveto{\pgfqpoint{9.354343in}{1.391584in}}%
\pgfpathlineto{\pgfqpoint{9.407455in}{1.509884in}}%
\pgfpathlineto{\pgfqpoint{9.592538in}{1.299711in}}%
\pgfpathlineto{\pgfqpoint{9.354343in}{1.391584in}}%
\pgfpathclose%
\pgfusepath{fill}%
\end{pgfscope}%
\begin{pgfscope}%
\pgfpathrectangle{\pgfqpoint{8.608921in}{0.208778in}}{\pgfqpoint{3.800000in}{3.800000in}}%
\pgfusepath{clip}%
\pgfsetbuttcap%
\pgfsetroundjoin%
\definecolor{currentfill}{rgb}{0.099716,0.362602,0.099716}%
\pgfsetfillcolor{currentfill}%
\pgfsetfillopacity{0.200000}%
\pgfsetlinewidth{0.000000pt}%
\definecolor{currentstroke}{rgb}{0.000000,0.000000,0.000000}%
\pgfsetstrokecolor{currentstroke}%
\pgfsetdash{}{0pt}%
\pgfpathmoveto{\pgfqpoint{11.921408in}{2.213486in}}%
\pgfpathlineto{\pgfqpoint{11.958726in}{1.486489in}}%
\pgfpathlineto{\pgfqpoint{11.928912in}{1.601411in}}%
\pgfpathlineto{\pgfqpoint{11.921408in}{2.213486in}}%
\pgfpathclose%
\pgfusepath{fill}%
\end{pgfscope}%
\begin{pgfscope}%
\pgfpathrectangle{\pgfqpoint{8.608921in}{0.208778in}}{\pgfqpoint{3.800000in}{3.800000in}}%
\pgfusepath{clip}%
\pgfsetbuttcap%
\pgfsetroundjoin%
\definecolor{currentfill}{rgb}{0.099716,0.362602,0.099716}%
\pgfsetfillcolor{currentfill}%
\pgfsetfillopacity{0.200000}%
\pgfsetlinewidth{0.000000pt}%
\definecolor{currentstroke}{rgb}{0.000000,0.000000,0.000000}%
\pgfsetstrokecolor{currentstroke}%
\pgfsetdash{}{0pt}%
\pgfpathmoveto{\pgfqpoint{9.191632in}{1.601411in}}%
\pgfpathlineto{\pgfqpoint{9.161818in}{1.486489in}}%
\pgfpathlineto{\pgfqpoint{9.199136in}{2.213486in}}%
\pgfpathlineto{\pgfqpoint{9.191632in}{1.601411in}}%
\pgfpathclose%
\pgfusepath{fill}%
\end{pgfscope}%
\begin{pgfscope}%
\pgfpathrectangle{\pgfqpoint{8.608921in}{0.208778in}}{\pgfqpoint{3.800000in}{3.800000in}}%
\pgfusepath{clip}%
\pgfsetbuttcap%
\pgfsetroundjoin%
\definecolor{currentfill}{rgb}{0.069492,0.252698,0.069492}%
\pgfsetfillcolor{currentfill}%
\pgfsetfillopacity{0.200000}%
\pgfsetlinewidth{0.000000pt}%
\definecolor{currentstroke}{rgb}{0.000000,0.000000,0.000000}%
\pgfsetstrokecolor{currentstroke}%
\pgfsetdash{}{0pt}%
\pgfpathmoveto{\pgfqpoint{12.098122in}{1.690395in}}%
\pgfpathlineto{\pgfqpoint{11.958726in}{1.486489in}}%
\pgfpathlineto{\pgfqpoint{11.921408in}{2.213486in}}%
\pgfpathlineto{\pgfqpoint{12.098122in}{1.690395in}}%
\pgfpathclose%
\pgfusepath{fill}%
\end{pgfscope}%
\begin{pgfscope}%
\pgfpathrectangle{\pgfqpoint{8.608921in}{0.208778in}}{\pgfqpoint{3.800000in}{3.800000in}}%
\pgfusepath{clip}%
\pgfsetbuttcap%
\pgfsetroundjoin%
\definecolor{currentfill}{rgb}{0.069492,0.252698,0.069492}%
\pgfsetfillcolor{currentfill}%
\pgfsetfillopacity{0.200000}%
\pgfsetlinewidth{0.000000pt}%
\definecolor{currentstroke}{rgb}{0.000000,0.000000,0.000000}%
\pgfsetstrokecolor{currentstroke}%
\pgfsetdash{}{0pt}%
\pgfpathmoveto{\pgfqpoint{9.022422in}{1.690395in}}%
\pgfpathlineto{\pgfqpoint{9.199136in}{2.213486in}}%
\pgfpathlineto{\pgfqpoint{9.161818in}{1.486489in}}%
\pgfpathlineto{\pgfqpoint{9.022422in}{1.690395in}}%
\pgfpathclose%
\pgfusepath{fill}%
\end{pgfscope}%
\begin{pgfscope}%
\pgfpathrectangle{\pgfqpoint{8.608921in}{0.208778in}}{\pgfqpoint{3.800000in}{3.800000in}}%
\pgfusepath{clip}%
\pgfsetbuttcap%
\pgfsetroundjoin%
\definecolor{currentfill}{rgb}{0.067488,0.245410,0.067488}%
\pgfsetfillcolor{currentfill}%
\pgfsetfillopacity{0.200000}%
\pgfsetlinewidth{0.000000pt}%
\definecolor{currentstroke}{rgb}{0.000000,0.000000,0.000000}%
\pgfsetstrokecolor{currentstroke}%
\pgfsetdash{}{0pt}%
\pgfpathmoveto{\pgfqpoint{11.766201in}{1.391584in}}%
\pgfpathlineto{\pgfqpoint{11.928912in}{1.601411in}}%
\pgfpathlineto{\pgfqpoint{11.958726in}{1.486489in}}%
\pgfpathlineto{\pgfqpoint{11.766201in}{1.391584in}}%
\pgfpathclose%
\pgfusepath{fill}%
\end{pgfscope}%
\begin{pgfscope}%
\pgfpathrectangle{\pgfqpoint{8.608921in}{0.208778in}}{\pgfqpoint{3.800000in}{3.800000in}}%
\pgfusepath{clip}%
\pgfsetbuttcap%
\pgfsetroundjoin%
\definecolor{currentfill}{rgb}{0.067488,0.245410,0.067488}%
\pgfsetfillcolor{currentfill}%
\pgfsetfillopacity{0.200000}%
\pgfsetlinewidth{0.000000pt}%
\definecolor{currentstroke}{rgb}{0.000000,0.000000,0.000000}%
\pgfsetstrokecolor{currentstroke}%
\pgfsetdash{}{0pt}%
\pgfpathmoveto{\pgfqpoint{9.161818in}{1.486489in}}%
\pgfpathlineto{\pgfqpoint{9.191632in}{1.601411in}}%
\pgfpathlineto{\pgfqpoint{9.354343in}{1.391584in}}%
\pgfpathlineto{\pgfqpoint{9.161818in}{1.486489in}}%
\pgfpathclose%
\pgfusepath{fill}%
\end{pgfscope}%
\begin{pgfscope}%
\pgfpathrectangle{\pgfqpoint{8.608921in}{0.208778in}}{\pgfqpoint{3.800000in}{3.800000in}}%
\pgfusepath{clip}%
\pgfsetbuttcap%
\pgfsetroundjoin%
\definecolor{currentfill}{rgb}{0.128601,0.467641,0.128601}%
\pgfsetfillcolor{currentfill}%
\pgfsetfillopacity{0.200000}%
\pgfsetlinewidth{0.000000pt}%
\definecolor{currentstroke}{rgb}{0.000000,0.000000,0.000000}%
\pgfsetstrokecolor{currentstroke}%
\pgfsetdash{}{0pt}%
\pgfpathmoveto{\pgfqpoint{10.699670in}{3.722935in}}%
\pgfpathlineto{\pgfqpoint{10.420874in}{3.722935in}}%
\pgfpathlineto{\pgfqpoint{10.560272in}{3.833892in}}%
\pgfpathlineto{\pgfqpoint{10.699670in}{3.722935in}}%
\pgfpathclose%
\pgfusepath{fill}%
\end{pgfscope}%
\begin{pgfscope}%
\pgfpathrectangle{\pgfqpoint{8.608921in}{0.208778in}}{\pgfqpoint{3.800000in}{3.800000in}}%
\pgfusepath{clip}%
\pgfsetbuttcap%
\pgfsetroundjoin%
\definecolor{currentfill}{rgb}{0.071067,0.258424,0.071067}%
\pgfsetfillcolor{currentfill}%
\pgfsetfillopacity{0.200000}%
\pgfsetlinewidth{0.000000pt}%
\definecolor{currentstroke}{rgb}{0.000000,0.000000,0.000000}%
\pgfsetstrokecolor{currentstroke}%
\pgfsetdash{}{0pt}%
\pgfpathmoveto{\pgfqpoint{11.958726in}{1.486489in}}%
\pgfpathlineto{\pgfqpoint{12.098122in}{1.690395in}}%
\pgfpathlineto{\pgfqpoint{12.112047in}{1.578214in}}%
\pgfpathlineto{\pgfqpoint{11.958726in}{1.486489in}}%
\pgfpathclose%
\pgfusepath{fill}%
\end{pgfscope}%
\begin{pgfscope}%
\pgfpathrectangle{\pgfqpoint{8.608921in}{0.208778in}}{\pgfqpoint{3.800000in}{3.800000in}}%
\pgfusepath{clip}%
\pgfsetbuttcap%
\pgfsetroundjoin%
\definecolor{currentfill}{rgb}{0.071067,0.258424,0.071067}%
\pgfsetfillcolor{currentfill}%
\pgfsetfillopacity{0.200000}%
\pgfsetlinewidth{0.000000pt}%
\definecolor{currentstroke}{rgb}{0.000000,0.000000,0.000000}%
\pgfsetstrokecolor{currentstroke}%
\pgfsetdash{}{0pt}%
\pgfpathmoveto{\pgfqpoint{9.022422in}{1.690395in}}%
\pgfpathlineto{\pgfqpoint{9.161818in}{1.486489in}}%
\pgfpathlineto{\pgfqpoint{9.008497in}{1.578214in}}%
\pgfpathlineto{\pgfqpoint{9.022422in}{1.690395in}}%
\pgfpathclose%
\pgfusepath{fill}%
\end{pgfscope}%
\begin{pgfscope}%
\pgfpathrectangle{\pgfqpoint{8.608921in}{0.208778in}}{\pgfqpoint{3.800000in}{3.800000in}}%
\pgfusepath{clip}%
\pgfsetbuttcap%
\pgfsetroundjoin%
\definecolor{currentfill}{rgb}{0.052607,0.201942,0.305459}%
\pgfsetfillcolor{currentfill}%
\pgfsetlinewidth{0.000000pt}%
\definecolor{currentstroke}{rgb}{0.000000,0.000000,0.000000}%
\pgfsetstrokecolor{currentstroke}%
\pgfsetdash{}{0pt}%
\pgfpathmoveto{\pgfqpoint{10.947409in}{2.145870in}}%
\pgfpathlineto{\pgfqpoint{10.753919in}{1.863700in}}%
\pgfpathlineto{\pgfqpoint{10.560272in}{2.135306in}}%
\pgfpathlineto{\pgfqpoint{10.947409in}{2.145870in}}%
\pgfpathclose%
\pgfusepath{fill}%
\end{pgfscope}%
\begin{pgfscope}%
\pgfpathrectangle{\pgfqpoint{8.608921in}{0.208778in}}{\pgfqpoint{3.800000in}{3.800000in}}%
\pgfusepath{clip}%
\pgfsetbuttcap%
\pgfsetroundjoin%
\definecolor{currentfill}{rgb}{0.052607,0.201942,0.305459}%
\pgfsetfillcolor{currentfill}%
\pgfsetlinewidth{0.000000pt}%
\definecolor{currentstroke}{rgb}{0.000000,0.000000,0.000000}%
\pgfsetstrokecolor{currentstroke}%
\pgfsetdash{}{0pt}%
\pgfpathmoveto{\pgfqpoint{10.560272in}{2.135306in}}%
\pgfpathlineto{\pgfqpoint{10.366625in}{1.863700in}}%
\pgfpathlineto{\pgfqpoint{10.173135in}{2.145870in}}%
\pgfpathlineto{\pgfqpoint{10.560272in}{2.135306in}}%
\pgfpathclose%
\pgfusepath{fill}%
\end{pgfscope}%
\begin{pgfscope}%
\pgfpathrectangle{\pgfqpoint{8.608921in}{0.208778in}}{\pgfqpoint{3.800000in}{3.800000in}}%
\pgfusepath{clip}%
\pgfsetbuttcap%
\pgfsetroundjoin%
\definecolor{currentfill}{rgb}{0.060773,0.233289,0.352874}%
\pgfsetfillcolor{currentfill}%
\pgfsetlinewidth{0.000000pt}%
\definecolor{currentstroke}{rgb}{0.000000,0.000000,0.000000}%
\pgfsetstrokecolor{currentstroke}%
\pgfsetdash{}{0pt}%
\pgfpathmoveto{\pgfqpoint{10.560272in}{2.135306in}}%
\pgfpathlineto{\pgfqpoint{10.560272in}{2.709353in}}%
\pgfpathlineto{\pgfqpoint{10.947409in}{2.145870in}}%
\pgfpathlineto{\pgfqpoint{10.560272in}{2.135306in}}%
\pgfpathclose%
\pgfusepath{fill}%
\end{pgfscope}%
\begin{pgfscope}%
\pgfpathrectangle{\pgfqpoint{8.608921in}{0.208778in}}{\pgfqpoint{3.800000in}{3.800000in}}%
\pgfusepath{clip}%
\pgfsetbuttcap%
\pgfsetroundjoin%
\definecolor{currentfill}{rgb}{0.060773,0.233289,0.352874}%
\pgfsetfillcolor{currentfill}%
\pgfsetlinewidth{0.000000pt}%
\definecolor{currentstroke}{rgb}{0.000000,0.000000,0.000000}%
\pgfsetstrokecolor{currentstroke}%
\pgfsetdash{}{0pt}%
\pgfpathmoveto{\pgfqpoint{10.173135in}{2.145870in}}%
\pgfpathlineto{\pgfqpoint{10.560272in}{2.709353in}}%
\pgfpathlineto{\pgfqpoint{10.560272in}{2.135306in}}%
\pgfpathlineto{\pgfqpoint{10.173135in}{2.145870in}}%
\pgfpathclose%
\pgfusepath{fill}%
\end{pgfscope}%
\begin{pgfscope}%
\pgfpathrectangle{\pgfqpoint{8.608921in}{0.208778in}}{\pgfqpoint{3.800000in}{3.800000in}}%
\pgfusepath{clip}%
\pgfsetbuttcap%
\pgfsetroundjoin%
\definecolor{currentfill}{rgb}{0.060634,0.232757,0.352069}%
\pgfsetfillcolor{currentfill}%
\pgfsetlinewidth{0.000000pt}%
\definecolor{currentstroke}{rgb}{0.000000,0.000000,0.000000}%
\pgfsetstrokecolor{currentstroke}%
\pgfsetdash{}{0pt}%
\pgfpathmoveto{\pgfqpoint{10.927452in}{2.709274in}}%
\pgfpathlineto{\pgfqpoint{10.947409in}{2.145870in}}%
\pgfpathlineto{\pgfqpoint{10.560272in}{2.709353in}}%
\pgfpathlineto{\pgfqpoint{10.927452in}{2.709274in}}%
\pgfpathclose%
\pgfusepath{fill}%
\end{pgfscope}%
\begin{pgfscope}%
\pgfpathrectangle{\pgfqpoint{8.608921in}{0.208778in}}{\pgfqpoint{3.800000in}{3.800000in}}%
\pgfusepath{clip}%
\pgfsetbuttcap%
\pgfsetroundjoin%
\definecolor{currentfill}{rgb}{0.060634,0.232757,0.352069}%
\pgfsetfillcolor{currentfill}%
\pgfsetlinewidth{0.000000pt}%
\definecolor{currentstroke}{rgb}{0.000000,0.000000,0.000000}%
\pgfsetstrokecolor{currentstroke}%
\pgfsetdash{}{0pt}%
\pgfpathmoveto{\pgfqpoint{10.560272in}{2.709353in}}%
\pgfpathlineto{\pgfqpoint{10.173135in}{2.145870in}}%
\pgfpathlineto{\pgfqpoint{10.193092in}{2.709274in}}%
\pgfpathlineto{\pgfqpoint{10.560272in}{2.709353in}}%
\pgfpathclose%
\pgfusepath{fill}%
\end{pgfscope}%
\begin{pgfscope}%
\pgfpathrectangle{\pgfqpoint{8.608921in}{0.208778in}}{\pgfqpoint{3.800000in}{3.800000in}}%
\pgfusepath{clip}%
\pgfsetbuttcap%
\pgfsetroundjoin%
\definecolor{currentfill}{rgb}{0.053541,0.205528,0.310883}%
\pgfsetfillcolor{currentfill}%
\pgfsetlinewidth{0.000000pt}%
\definecolor{currentstroke}{rgb}{0.000000,0.000000,0.000000}%
\pgfsetstrokecolor{currentstroke}%
\pgfsetdash{}{0pt}%
\pgfpathmoveto{\pgfqpoint{11.295195in}{2.174264in}}%
\pgfpathlineto{\pgfqpoint{11.121001in}{1.892954in}}%
\pgfpathlineto{\pgfqpoint{10.947409in}{2.145870in}}%
\pgfpathlineto{\pgfqpoint{11.295195in}{2.174264in}}%
\pgfpathclose%
\pgfusepath{fill}%
\end{pgfscope}%
\begin{pgfscope}%
\pgfpathrectangle{\pgfqpoint{8.608921in}{0.208778in}}{\pgfqpoint{3.800000in}{3.800000in}}%
\pgfusepath{clip}%
\pgfsetbuttcap%
\pgfsetroundjoin%
\definecolor{currentfill}{rgb}{0.053541,0.205528,0.310883}%
\pgfsetfillcolor{currentfill}%
\pgfsetlinewidth{0.000000pt}%
\definecolor{currentstroke}{rgb}{0.000000,0.000000,0.000000}%
\pgfsetstrokecolor{currentstroke}%
\pgfsetdash{}{0pt}%
\pgfpathmoveto{\pgfqpoint{10.173135in}{2.145870in}}%
\pgfpathlineto{\pgfqpoint{9.999543in}{1.892954in}}%
\pgfpathlineto{\pgfqpoint{9.825349in}{2.174264in}}%
\pgfpathlineto{\pgfqpoint{10.173135in}{2.145870in}}%
\pgfpathclose%
\pgfusepath{fill}%
\end{pgfscope}%
\begin{pgfscope}%
\pgfpathrectangle{\pgfqpoint{8.608921in}{0.208778in}}{\pgfqpoint{3.800000in}{3.800000in}}%
\pgfusepath{clip}%
\pgfsetbuttcap%
\pgfsetroundjoin%
\definecolor{currentfill}{rgb}{0.061576,0.236373,0.357539}%
\pgfsetfillcolor{currentfill}%
\pgfsetlinewidth{0.000000pt}%
\definecolor{currentstroke}{rgb}{0.000000,0.000000,0.000000}%
\pgfsetstrokecolor{currentstroke}%
\pgfsetdash{}{0pt}%
\pgfpathmoveto{\pgfqpoint{10.947409in}{2.145870in}}%
\pgfpathlineto{\pgfqpoint{10.927452in}{2.709274in}}%
\pgfpathlineto{\pgfqpoint{11.295195in}{2.174264in}}%
\pgfpathlineto{\pgfqpoint{10.947409in}{2.145870in}}%
\pgfpathclose%
\pgfusepath{fill}%
\end{pgfscope}%
\begin{pgfscope}%
\pgfpathrectangle{\pgfqpoint{8.608921in}{0.208778in}}{\pgfqpoint{3.800000in}{3.800000in}}%
\pgfusepath{clip}%
\pgfsetbuttcap%
\pgfsetroundjoin%
\definecolor{currentfill}{rgb}{0.061576,0.236373,0.357539}%
\pgfsetfillcolor{currentfill}%
\pgfsetlinewidth{0.000000pt}%
\definecolor{currentstroke}{rgb}{0.000000,0.000000,0.000000}%
\pgfsetstrokecolor{currentstroke}%
\pgfsetdash{}{0pt}%
\pgfpathmoveto{\pgfqpoint{9.825349in}{2.174264in}}%
\pgfpathlineto{\pgfqpoint{10.193092in}{2.709274in}}%
\pgfpathlineto{\pgfqpoint{10.173135in}{2.145870in}}%
\pgfpathlineto{\pgfqpoint{9.825349in}{2.174264in}}%
\pgfpathclose%
\pgfusepath{fill}%
\end{pgfscope}%
\begin{pgfscope}%
\pgfpathrectangle{\pgfqpoint{8.608921in}{0.208778in}}{\pgfqpoint{3.800000in}{3.800000in}}%
\pgfusepath{clip}%
\pgfsetbuttcap%
\pgfsetroundjoin%
\definecolor{currentfill}{rgb}{0.049465,0.189883,0.287218}%
\pgfsetfillcolor{currentfill}%
\pgfsetlinewidth{0.000000pt}%
\definecolor{currentstroke}{rgb}{0.000000,0.000000,0.000000}%
\pgfsetstrokecolor{currentstroke}%
\pgfsetdash{}{0pt}%
\pgfpathmoveto{\pgfqpoint{10.173135in}{2.145870in}}%
\pgfpathlineto{\pgfqpoint{10.366625in}{1.863700in}}%
\pgfpathlineto{\pgfqpoint{10.048885in}{1.467999in}}%
\pgfpathlineto{\pgfqpoint{10.173135in}{2.145870in}}%
\pgfpathclose%
\pgfusepath{fill}%
\end{pgfscope}%
\begin{pgfscope}%
\pgfpathrectangle{\pgfqpoint{8.608921in}{0.208778in}}{\pgfqpoint{3.800000in}{3.800000in}}%
\pgfusepath{clip}%
\pgfsetbuttcap%
\pgfsetroundjoin%
\definecolor{currentfill}{rgb}{0.049465,0.189883,0.287218}%
\pgfsetfillcolor{currentfill}%
\pgfsetlinewidth{0.000000pt}%
\definecolor{currentstroke}{rgb}{0.000000,0.000000,0.000000}%
\pgfsetstrokecolor{currentstroke}%
\pgfsetdash{}{0pt}%
\pgfpathmoveto{\pgfqpoint{11.071659in}{1.467999in}}%
\pgfpathlineto{\pgfqpoint{10.753919in}{1.863700in}}%
\pgfpathlineto{\pgfqpoint{10.947409in}{2.145870in}}%
\pgfpathlineto{\pgfqpoint{11.071659in}{1.467999in}}%
\pgfpathclose%
\pgfusepath{fill}%
\end{pgfscope}%
\begin{pgfscope}%
\pgfpathrectangle{\pgfqpoint{8.608921in}{0.208778in}}{\pgfqpoint{3.800000in}{3.800000in}}%
\pgfusepath{clip}%
\pgfsetbuttcap%
\pgfsetroundjoin%
\definecolor{currentfill}{rgb}{0.069261,0.265872,0.402159}%
\pgfsetfillcolor{currentfill}%
\pgfsetlinewidth{0.000000pt}%
\definecolor{currentstroke}{rgb}{0.000000,0.000000,0.000000}%
\pgfsetstrokecolor{currentstroke}%
\pgfsetdash{}{0pt}%
\pgfpathmoveto{\pgfqpoint{10.560272in}{2.709353in}}%
\pgfpathlineto{\pgfqpoint{10.735368in}{2.963990in}}%
\pgfpathlineto{\pgfqpoint{10.927452in}{2.709274in}}%
\pgfpathlineto{\pgfqpoint{10.560272in}{2.709353in}}%
\pgfpathclose%
\pgfusepath{fill}%
\end{pgfscope}%
\begin{pgfscope}%
\pgfpathrectangle{\pgfqpoint{8.608921in}{0.208778in}}{\pgfqpoint{3.800000in}{3.800000in}}%
\pgfusepath{clip}%
\pgfsetbuttcap%
\pgfsetroundjoin%
\definecolor{currentfill}{rgb}{0.069261,0.265872,0.402159}%
\pgfsetfillcolor{currentfill}%
\pgfsetlinewidth{0.000000pt}%
\definecolor{currentstroke}{rgb}{0.000000,0.000000,0.000000}%
\pgfsetstrokecolor{currentstroke}%
\pgfsetdash{}{0pt}%
\pgfpathmoveto{\pgfqpoint{10.193092in}{2.709274in}}%
\pgfpathlineto{\pgfqpoint{10.385176in}{2.963990in}}%
\pgfpathlineto{\pgfqpoint{10.560272in}{2.709353in}}%
\pgfpathlineto{\pgfqpoint{10.193092in}{2.709274in}}%
\pgfpathclose%
\pgfusepath{fill}%
\end{pgfscope}%
\begin{pgfscope}%
\pgfpathrectangle{\pgfqpoint{8.608921in}{0.208778in}}{\pgfqpoint{3.800000in}{3.800000in}}%
\pgfusepath{clip}%
\pgfsetbuttcap%
\pgfsetroundjoin%
\definecolor{currentfill}{rgb}{0.046814,0.179706,0.271825}%
\pgfsetfillcolor{currentfill}%
\pgfsetlinewidth{0.000000pt}%
\definecolor{currentstroke}{rgb}{0.000000,0.000000,0.000000}%
\pgfsetstrokecolor{currentstroke}%
\pgfsetdash{}{0pt}%
\pgfpathmoveto{\pgfqpoint{10.753919in}{1.863700in}}%
\pgfpathlineto{\pgfqpoint{10.560272in}{1.278237in}}%
\pgfpathlineto{\pgfqpoint{10.560272in}{2.135306in}}%
\pgfpathlineto{\pgfqpoint{10.753919in}{1.863700in}}%
\pgfpathclose%
\pgfusepath{fill}%
\end{pgfscope}%
\begin{pgfscope}%
\pgfpathrectangle{\pgfqpoint{8.608921in}{0.208778in}}{\pgfqpoint{3.800000in}{3.800000in}}%
\pgfusepath{clip}%
\pgfsetbuttcap%
\pgfsetroundjoin%
\definecolor{currentfill}{rgb}{0.046814,0.179706,0.271825}%
\pgfsetfillcolor{currentfill}%
\pgfsetlinewidth{0.000000pt}%
\definecolor{currentstroke}{rgb}{0.000000,0.000000,0.000000}%
\pgfsetstrokecolor{currentstroke}%
\pgfsetdash{}{0pt}%
\pgfpathmoveto{\pgfqpoint{10.560272in}{2.135306in}}%
\pgfpathlineto{\pgfqpoint{10.560272in}{1.278237in}}%
\pgfpathlineto{\pgfqpoint{10.366625in}{1.863700in}}%
\pgfpathlineto{\pgfqpoint{10.560272in}{2.135306in}}%
\pgfpathclose%
\pgfusepath{fill}%
\end{pgfscope}%
\begin{pgfscope}%
\pgfpathrectangle{\pgfqpoint{8.608921in}{0.208778in}}{\pgfqpoint{3.800000in}{3.800000in}}%
\pgfusepath{clip}%
\pgfsetbuttcap%
\pgfsetroundjoin%
\definecolor{currentfill}{rgb}{0.045820,0.175891,0.266053}%
\pgfsetfillcolor{currentfill}%
\pgfsetlinewidth{0.000000pt}%
\definecolor{currentstroke}{rgb}{0.000000,0.000000,0.000000}%
\pgfsetstrokecolor{currentstroke}%
\pgfsetdash{}{0pt}%
\pgfpathmoveto{\pgfqpoint{11.071659in}{1.467999in}}%
\pgfpathlineto{\pgfqpoint{10.947409in}{2.145870in}}%
\pgfpathlineto{\pgfqpoint{11.121001in}{1.892954in}}%
\pgfpathlineto{\pgfqpoint{11.071659in}{1.467999in}}%
\pgfpathclose%
\pgfusepath{fill}%
\end{pgfscope}%
\begin{pgfscope}%
\pgfpathrectangle{\pgfqpoint{8.608921in}{0.208778in}}{\pgfqpoint{3.800000in}{3.800000in}}%
\pgfusepath{clip}%
\pgfsetbuttcap%
\pgfsetroundjoin%
\definecolor{currentfill}{rgb}{0.045820,0.175891,0.266053}%
\pgfsetfillcolor{currentfill}%
\pgfsetlinewidth{0.000000pt}%
\definecolor{currentstroke}{rgb}{0.000000,0.000000,0.000000}%
\pgfsetstrokecolor{currentstroke}%
\pgfsetdash{}{0pt}%
\pgfpathmoveto{\pgfqpoint{9.999543in}{1.892954in}}%
\pgfpathlineto{\pgfqpoint{10.173135in}{2.145870in}}%
\pgfpathlineto{\pgfqpoint{10.048885in}{1.467999in}}%
\pgfpathlineto{\pgfqpoint{9.999543in}{1.892954in}}%
\pgfpathclose%
\pgfusepath{fill}%
\end{pgfscope}%
\begin{pgfscope}%
\pgfpathrectangle{\pgfqpoint{8.608921in}{0.208778in}}{\pgfqpoint{3.800000in}{3.800000in}}%
\pgfusepath{clip}%
\pgfsetbuttcap%
\pgfsetroundjoin%
\definecolor{currentfill}{rgb}{0.071636,0.274990,0.415951}%
\pgfsetfillcolor{currentfill}%
\pgfsetlinewidth{0.000000pt}%
\definecolor{currentstroke}{rgb}{0.000000,0.000000,0.000000}%
\pgfsetstrokecolor{currentstroke}%
\pgfsetdash{}{0pt}%
\pgfpathmoveto{\pgfqpoint{10.560272in}{2.709353in}}%
\pgfpathlineto{\pgfqpoint{10.385176in}{2.963990in}}%
\pgfpathlineto{\pgfqpoint{10.735368in}{2.963990in}}%
\pgfpathlineto{\pgfqpoint{10.560272in}{2.709353in}}%
\pgfpathclose%
\pgfusepath{fill}%
\end{pgfscope}%
\begin{pgfscope}%
\pgfpathrectangle{\pgfqpoint{8.608921in}{0.208778in}}{\pgfqpoint{3.800000in}{3.800000in}}%
\pgfusepath{clip}%
\pgfsetbuttcap%
\pgfsetroundjoin%
\definecolor{currentfill}{rgb}{0.071694,0.275212,0.416288}%
\pgfsetfillcolor{currentfill}%
\pgfsetlinewidth{0.000000pt}%
\definecolor{currentstroke}{rgb}{0.000000,0.000000,0.000000}%
\pgfsetstrokecolor{currentstroke}%
\pgfsetdash{}{0pt}%
\pgfpathmoveto{\pgfqpoint{10.927452in}{2.709274in}}%
\pgfpathlineto{\pgfqpoint{10.735368in}{2.963990in}}%
\pgfpathlineto{\pgfqpoint{11.070569in}{2.956587in}}%
\pgfpathlineto{\pgfqpoint{10.927452in}{2.709274in}}%
\pgfpathclose%
\pgfusepath{fill}%
\end{pgfscope}%
\begin{pgfscope}%
\pgfpathrectangle{\pgfqpoint{8.608921in}{0.208778in}}{\pgfqpoint{3.800000in}{3.800000in}}%
\pgfusepath{clip}%
\pgfsetbuttcap%
\pgfsetroundjoin%
\definecolor{currentfill}{rgb}{0.071694,0.275212,0.416288}%
\pgfsetfillcolor{currentfill}%
\pgfsetlinewidth{0.000000pt}%
\definecolor{currentstroke}{rgb}{0.000000,0.000000,0.000000}%
\pgfsetstrokecolor{currentstroke}%
\pgfsetdash{}{0pt}%
\pgfpathmoveto{\pgfqpoint{10.049975in}{2.956587in}}%
\pgfpathlineto{\pgfqpoint{10.385176in}{2.963990in}}%
\pgfpathlineto{\pgfqpoint{10.193092in}{2.709274in}}%
\pgfpathlineto{\pgfqpoint{10.049975in}{2.956587in}}%
\pgfpathclose%
\pgfusepath{fill}%
\end{pgfscope}%
\begin{pgfscope}%
\pgfpathrectangle{\pgfqpoint{8.608921in}{0.208778in}}{\pgfqpoint{3.800000in}{3.800000in}}%
\pgfusepath{clip}%
\pgfsetbuttcap%
\pgfsetroundjoin%
\definecolor{currentfill}{rgb}{0.064759,0.248590,0.376018}%
\pgfsetfillcolor{currentfill}%
\pgfsetlinewidth{0.000000pt}%
\definecolor{currentstroke}{rgb}{0.000000,0.000000,0.000000}%
\pgfsetstrokecolor{currentstroke}%
\pgfsetdash{}{0pt}%
\pgfpathmoveto{\pgfqpoint{11.295195in}{2.174264in}}%
\pgfpathlineto{\pgfqpoint{10.927452in}{2.709274in}}%
\pgfpathlineto{\pgfqpoint{11.366741in}{2.943542in}}%
\pgfpathlineto{\pgfqpoint{11.295195in}{2.174264in}}%
\pgfpathclose%
\pgfusepath{fill}%
\end{pgfscope}%
\begin{pgfscope}%
\pgfpathrectangle{\pgfqpoint{8.608921in}{0.208778in}}{\pgfqpoint{3.800000in}{3.800000in}}%
\pgfusepath{clip}%
\pgfsetbuttcap%
\pgfsetroundjoin%
\definecolor{currentfill}{rgb}{0.064759,0.248590,0.376018}%
\pgfsetfillcolor{currentfill}%
\pgfsetlinewidth{0.000000pt}%
\definecolor{currentstroke}{rgb}{0.000000,0.000000,0.000000}%
\pgfsetstrokecolor{currentstroke}%
\pgfsetdash{}{0pt}%
\pgfpathmoveto{\pgfqpoint{9.753803in}{2.943542in}}%
\pgfpathlineto{\pgfqpoint{10.193092in}{2.709274in}}%
\pgfpathlineto{\pgfqpoint{9.825349in}{2.174264in}}%
\pgfpathlineto{\pgfqpoint{9.753803in}{2.943542in}}%
\pgfpathclose%
\pgfusepath{fill}%
\end{pgfscope}%
\begin{pgfscope}%
\pgfpathrectangle{\pgfqpoint{8.608921in}{0.208778in}}{\pgfqpoint{3.800000in}{3.800000in}}%
\pgfusepath{clip}%
\pgfsetbuttcap%
\pgfsetroundjoin%
\definecolor{currentfill}{rgb}{0.051850,0.199036,0.301063}%
\pgfsetfillcolor{currentfill}%
\pgfsetlinewidth{0.000000pt}%
\definecolor{currentstroke}{rgb}{0.000000,0.000000,0.000000}%
\pgfsetstrokecolor{currentstroke}%
\pgfsetdash{}{0pt}%
\pgfpathmoveto{\pgfqpoint{11.368375in}{1.532138in}}%
\pgfpathlineto{\pgfqpoint{11.121001in}{1.892954in}}%
\pgfpathlineto{\pgfqpoint{11.295195in}{2.174264in}}%
\pgfpathlineto{\pgfqpoint{11.368375in}{1.532138in}}%
\pgfpathclose%
\pgfusepath{fill}%
\end{pgfscope}%
\begin{pgfscope}%
\pgfpathrectangle{\pgfqpoint{8.608921in}{0.208778in}}{\pgfqpoint{3.800000in}{3.800000in}}%
\pgfusepath{clip}%
\pgfsetbuttcap%
\pgfsetroundjoin%
\definecolor{currentfill}{rgb}{0.051850,0.199036,0.301063}%
\pgfsetfillcolor{currentfill}%
\pgfsetlinewidth{0.000000pt}%
\definecolor{currentstroke}{rgb}{0.000000,0.000000,0.000000}%
\pgfsetstrokecolor{currentstroke}%
\pgfsetdash{}{0pt}%
\pgfpathmoveto{\pgfqpoint{9.825349in}{2.174264in}}%
\pgfpathlineto{\pgfqpoint{9.999543in}{1.892954in}}%
\pgfpathlineto{\pgfqpoint{9.752169in}{1.532138in}}%
\pgfpathlineto{\pgfqpoint{9.825349in}{2.174264in}}%
\pgfpathclose%
\pgfusepath{fill}%
\end{pgfscope}%
\begin{pgfscope}%
\pgfpathrectangle{\pgfqpoint{8.608921in}{0.208778in}}{\pgfqpoint{3.800000in}{3.800000in}}%
\pgfusepath{clip}%
\pgfsetbuttcap%
\pgfsetroundjoin%
\definecolor{currentfill}{rgb}{0.046101,0.176968,0.267683}%
\pgfsetfillcolor{currentfill}%
\pgfsetlinewidth{0.000000pt}%
\definecolor{currentstroke}{rgb}{0.000000,0.000000,0.000000}%
\pgfsetstrokecolor{currentstroke}%
\pgfsetdash{}{0pt}%
\pgfpathmoveto{\pgfqpoint{10.560272in}{1.278237in}}%
\pgfpathlineto{\pgfqpoint{10.753919in}{1.863700in}}%
\pgfpathlineto{\pgfqpoint{10.735753in}{1.431593in}}%
\pgfpathlineto{\pgfqpoint{10.560272in}{1.278237in}}%
\pgfpathclose%
\pgfusepath{fill}%
\end{pgfscope}%
\begin{pgfscope}%
\pgfpathrectangle{\pgfqpoint{8.608921in}{0.208778in}}{\pgfqpoint{3.800000in}{3.800000in}}%
\pgfusepath{clip}%
\pgfsetbuttcap%
\pgfsetroundjoin%
\definecolor{currentfill}{rgb}{0.046101,0.176968,0.267683}%
\pgfsetfillcolor{currentfill}%
\pgfsetlinewidth{0.000000pt}%
\definecolor{currentstroke}{rgb}{0.000000,0.000000,0.000000}%
\pgfsetstrokecolor{currentstroke}%
\pgfsetdash{}{0pt}%
\pgfpathmoveto{\pgfqpoint{10.384791in}{1.431593in}}%
\pgfpathlineto{\pgfqpoint{10.366625in}{1.863700in}}%
\pgfpathlineto{\pgfqpoint{10.560272in}{1.278237in}}%
\pgfpathlineto{\pgfqpoint{10.384791in}{1.431593in}}%
\pgfpathclose%
\pgfusepath{fill}%
\end{pgfscope}%
\begin{pgfscope}%
\pgfpathrectangle{\pgfqpoint{8.608921in}{0.208778in}}{\pgfqpoint{3.800000in}{3.800000in}}%
\pgfusepath{clip}%
\pgfsetbuttcap%
\pgfsetroundjoin%
\definecolor{currentfill}{rgb}{0.047555,0.182548,0.276123}%
\pgfsetfillcolor{currentfill}%
\pgfsetlinewidth{0.000000pt}%
\definecolor{currentstroke}{rgb}{0.000000,0.000000,0.000000}%
\pgfsetstrokecolor{currentstroke}%
\pgfsetdash{}{0pt}%
\pgfpathmoveto{\pgfqpoint{10.883622in}{1.296348in}}%
\pgfpathlineto{\pgfqpoint{10.735753in}{1.431593in}}%
\pgfpathlineto{\pgfqpoint{10.753919in}{1.863700in}}%
\pgfpathlineto{\pgfqpoint{10.883622in}{1.296348in}}%
\pgfpathclose%
\pgfusepath{fill}%
\end{pgfscope}%
\begin{pgfscope}%
\pgfpathrectangle{\pgfqpoint{8.608921in}{0.208778in}}{\pgfqpoint{3.800000in}{3.800000in}}%
\pgfusepath{clip}%
\pgfsetbuttcap%
\pgfsetroundjoin%
\definecolor{currentfill}{rgb}{0.047555,0.182548,0.276123}%
\pgfsetfillcolor{currentfill}%
\pgfsetlinewidth{0.000000pt}%
\definecolor{currentstroke}{rgb}{0.000000,0.000000,0.000000}%
\pgfsetstrokecolor{currentstroke}%
\pgfsetdash{}{0pt}%
\pgfpathmoveto{\pgfqpoint{10.366625in}{1.863700in}}%
\pgfpathlineto{\pgfqpoint{10.384791in}{1.431593in}}%
\pgfpathlineto{\pgfqpoint{10.236922in}{1.296348in}}%
\pgfpathlineto{\pgfqpoint{10.366625in}{1.863700in}}%
\pgfpathclose%
\pgfusepath{fill}%
\end{pgfscope}%
\begin{pgfscope}%
\pgfpathrectangle{\pgfqpoint{8.608921in}{0.208778in}}{\pgfqpoint{3.800000in}{3.800000in}}%
\pgfusepath{clip}%
\pgfsetbuttcap%
\pgfsetroundjoin%
\definecolor{currentfill}{rgb}{0.068541,0.263111,0.397982}%
\pgfsetfillcolor{currentfill}%
\pgfsetlinewidth{0.000000pt}%
\definecolor{currentstroke}{rgb}{0.000000,0.000000,0.000000}%
\pgfsetstrokecolor{currentstroke}%
\pgfsetdash{}{0pt}%
\pgfpathmoveto{\pgfqpoint{10.193092in}{2.709274in}}%
\pgfpathlineto{\pgfqpoint{9.753803in}{2.943542in}}%
\pgfpathlineto{\pgfqpoint{10.049975in}{2.956587in}}%
\pgfpathlineto{\pgfqpoint{10.193092in}{2.709274in}}%
\pgfpathclose%
\pgfusepath{fill}%
\end{pgfscope}%
\begin{pgfscope}%
\pgfpathrectangle{\pgfqpoint{8.608921in}{0.208778in}}{\pgfqpoint{3.800000in}{3.800000in}}%
\pgfusepath{clip}%
\pgfsetbuttcap%
\pgfsetroundjoin%
\definecolor{currentfill}{rgb}{0.068541,0.263111,0.397982}%
\pgfsetfillcolor{currentfill}%
\pgfsetlinewidth{0.000000pt}%
\definecolor{currentstroke}{rgb}{0.000000,0.000000,0.000000}%
\pgfsetstrokecolor{currentstroke}%
\pgfsetdash{}{0pt}%
\pgfpathmoveto{\pgfqpoint{11.070569in}{2.956587in}}%
\pgfpathlineto{\pgfqpoint{11.366741in}{2.943542in}}%
\pgfpathlineto{\pgfqpoint{10.927452in}{2.709274in}}%
\pgfpathlineto{\pgfqpoint{11.070569in}{2.956587in}}%
\pgfpathclose%
\pgfusepath{fill}%
\end{pgfscope}%
\begin{pgfscope}%
\pgfpathrectangle{\pgfqpoint{8.608921in}{0.208778in}}{\pgfqpoint{3.800000in}{3.800000in}}%
\pgfusepath{clip}%
\pgfsetbuttcap%
\pgfsetroundjoin%
\definecolor{currentfill}{rgb}{0.042669,0.163794,0.247755}%
\pgfsetfillcolor{currentfill}%
\pgfsetlinewidth{0.000000pt}%
\definecolor{currentstroke}{rgb}{0.000000,0.000000,0.000000}%
\pgfsetstrokecolor{currentstroke}%
\pgfsetdash{}{0pt}%
\pgfpathmoveto{\pgfqpoint{10.883622in}{1.296348in}}%
\pgfpathlineto{\pgfqpoint{10.753919in}{1.863700in}}%
\pgfpathlineto{\pgfqpoint{11.071659in}{1.467999in}}%
\pgfpathlineto{\pgfqpoint{10.883622in}{1.296348in}}%
\pgfpathclose%
\pgfusepath{fill}%
\end{pgfscope}%
\begin{pgfscope}%
\pgfpathrectangle{\pgfqpoint{8.608921in}{0.208778in}}{\pgfqpoint{3.800000in}{3.800000in}}%
\pgfusepath{clip}%
\pgfsetbuttcap%
\pgfsetroundjoin%
\definecolor{currentfill}{rgb}{0.042669,0.163794,0.247755}%
\pgfsetfillcolor{currentfill}%
\pgfsetlinewidth{0.000000pt}%
\definecolor{currentstroke}{rgb}{0.000000,0.000000,0.000000}%
\pgfsetstrokecolor{currentstroke}%
\pgfsetdash{}{0pt}%
\pgfpathmoveto{\pgfqpoint{10.048885in}{1.467999in}}%
\pgfpathlineto{\pgfqpoint{10.366625in}{1.863700in}}%
\pgfpathlineto{\pgfqpoint{10.236922in}{1.296348in}}%
\pgfpathlineto{\pgfqpoint{10.048885in}{1.467999in}}%
\pgfpathclose%
\pgfusepath{fill}%
\end{pgfscope}%
\begin{pgfscope}%
\pgfpathrectangle{\pgfqpoint{8.608921in}{0.208778in}}{\pgfqpoint{3.800000in}{3.800000in}}%
\pgfusepath{clip}%
\pgfsetbuttcap%
\pgfsetroundjoin%
\definecolor{currentfill}{rgb}{0.064954,0.249341,0.377155}%
\pgfsetfillcolor{currentfill}%
\pgfsetlinewidth{0.000000pt}%
\definecolor{currentstroke}{rgb}{0.000000,0.000000,0.000000}%
\pgfsetstrokecolor{currentstroke}%
\pgfsetdash{}{0pt}%
\pgfpathmoveto{\pgfqpoint{11.295195in}{2.174264in}}%
\pgfpathlineto{\pgfqpoint{11.366741in}{2.943542in}}%
\pgfpathlineto{\pgfqpoint{11.581954in}{2.213087in}}%
\pgfpathlineto{\pgfqpoint{11.295195in}{2.174264in}}%
\pgfpathclose%
\pgfusepath{fill}%
\end{pgfscope}%
\begin{pgfscope}%
\pgfpathrectangle{\pgfqpoint{8.608921in}{0.208778in}}{\pgfqpoint{3.800000in}{3.800000in}}%
\pgfusepath{clip}%
\pgfsetbuttcap%
\pgfsetroundjoin%
\definecolor{currentfill}{rgb}{0.064954,0.249341,0.377155}%
\pgfsetfillcolor{currentfill}%
\pgfsetlinewidth{0.000000pt}%
\definecolor{currentstroke}{rgb}{0.000000,0.000000,0.000000}%
\pgfsetstrokecolor{currentstroke}%
\pgfsetdash{}{0pt}%
\pgfpathmoveto{\pgfqpoint{9.538590in}{2.213087in}}%
\pgfpathlineto{\pgfqpoint{9.753803in}{2.943542in}}%
\pgfpathlineto{\pgfqpoint{9.825349in}{2.174264in}}%
\pgfpathlineto{\pgfqpoint{9.538590in}{2.213087in}}%
\pgfpathclose%
\pgfusepath{fill}%
\end{pgfscope}%
\begin{pgfscope}%
\pgfpathrectangle{\pgfqpoint{8.608921in}{0.208778in}}{\pgfqpoint{3.800000in}{3.800000in}}%
\pgfusepath{clip}%
\pgfsetbuttcap%
\pgfsetroundjoin%
\definecolor{currentfill}{rgb}{0.078663,0.301965,0.456754}%
\pgfsetfillcolor{currentfill}%
\pgfsetlinewidth{0.000000pt}%
\definecolor{currentstroke}{rgb}{0.000000,0.000000,0.000000}%
\pgfsetstrokecolor{currentstroke}%
\pgfsetdash{}{0pt}%
\pgfpathmoveto{\pgfqpoint{10.735368in}{2.963990in}}%
\pgfpathlineto{\pgfqpoint{11.182221in}{3.161323in}}%
\pgfpathlineto{\pgfqpoint{11.070569in}{2.956587in}}%
\pgfpathlineto{\pgfqpoint{10.735368in}{2.963990in}}%
\pgfpathclose%
\pgfusepath{fill}%
\end{pgfscope}%
\begin{pgfscope}%
\pgfpathrectangle{\pgfqpoint{8.608921in}{0.208778in}}{\pgfqpoint{3.800000in}{3.800000in}}%
\pgfusepath{clip}%
\pgfsetbuttcap%
\pgfsetroundjoin%
\definecolor{currentfill}{rgb}{0.078663,0.301965,0.456754}%
\pgfsetfillcolor{currentfill}%
\pgfsetlinewidth{0.000000pt}%
\definecolor{currentstroke}{rgb}{0.000000,0.000000,0.000000}%
\pgfsetstrokecolor{currentstroke}%
\pgfsetdash{}{0pt}%
\pgfpathmoveto{\pgfqpoint{10.049975in}{2.956587in}}%
\pgfpathlineto{\pgfqpoint{9.938323in}{3.161323in}}%
\pgfpathlineto{\pgfqpoint{10.385176in}{2.963990in}}%
\pgfpathlineto{\pgfqpoint{10.049975in}{2.956587in}}%
\pgfpathclose%
\pgfusepath{fill}%
\end{pgfscope}%
\begin{pgfscope}%
\pgfpathrectangle{\pgfqpoint{8.608921in}{0.208778in}}{\pgfqpoint{3.800000in}{3.800000in}}%
\pgfusepath{clip}%
\pgfsetbuttcap%
\pgfsetroundjoin%
\definecolor{currentfill}{rgb}{0.049941,0.191710,0.289982}%
\pgfsetfillcolor{currentfill}%
\pgfsetlinewidth{0.000000pt}%
\definecolor{currentstroke}{rgb}{0.000000,0.000000,0.000000}%
\pgfsetstrokecolor{currentstroke}%
\pgfsetdash{}{0pt}%
\pgfpathmoveto{\pgfqpoint{11.121001in}{1.892954in}}%
\pgfpathlineto{\pgfqpoint{11.183841in}{1.346718in}}%
\pgfpathlineto{\pgfqpoint{11.071659in}{1.467999in}}%
\pgfpathlineto{\pgfqpoint{11.121001in}{1.892954in}}%
\pgfpathclose%
\pgfusepath{fill}%
\end{pgfscope}%
\begin{pgfscope}%
\pgfpathrectangle{\pgfqpoint{8.608921in}{0.208778in}}{\pgfqpoint{3.800000in}{3.800000in}}%
\pgfusepath{clip}%
\pgfsetbuttcap%
\pgfsetroundjoin%
\definecolor{currentfill}{rgb}{0.049941,0.191710,0.289982}%
\pgfsetfillcolor{currentfill}%
\pgfsetlinewidth{0.000000pt}%
\definecolor{currentstroke}{rgb}{0.000000,0.000000,0.000000}%
\pgfsetstrokecolor{currentstroke}%
\pgfsetdash{}{0pt}%
\pgfpathmoveto{\pgfqpoint{10.048885in}{1.467999in}}%
\pgfpathlineto{\pgfqpoint{9.936703in}{1.346718in}}%
\pgfpathlineto{\pgfqpoint{9.999543in}{1.892954in}}%
\pgfpathlineto{\pgfqpoint{10.048885in}{1.467999in}}%
\pgfpathclose%
\pgfusepath{fill}%
\end{pgfscope}%
\begin{pgfscope}%
\pgfpathrectangle{\pgfqpoint{8.608921in}{0.208778in}}{\pgfqpoint{3.800000in}{3.800000in}}%
\pgfusepath{clip}%
\pgfsetbuttcap%
\pgfsetroundjoin%
\definecolor{currentfill}{rgb}{0.050011,0.191979,0.290388}%
\pgfsetfillcolor{currentfill}%
\pgfsetlinewidth{0.000000pt}%
\definecolor{currentstroke}{rgb}{0.000000,0.000000,0.000000}%
\pgfsetstrokecolor{currentstroke}%
\pgfsetdash{}{0pt}%
\pgfpathmoveto{\pgfqpoint{11.581954in}{2.213087in}}%
\pgfpathlineto{\pgfqpoint{11.614843in}{1.611699in}}%
\pgfpathlineto{\pgfqpoint{11.295195in}{2.174264in}}%
\pgfpathlineto{\pgfqpoint{11.581954in}{2.213087in}}%
\pgfpathclose%
\pgfusepath{fill}%
\end{pgfscope}%
\begin{pgfscope}%
\pgfpathrectangle{\pgfqpoint{8.608921in}{0.208778in}}{\pgfqpoint{3.800000in}{3.800000in}}%
\pgfusepath{clip}%
\pgfsetbuttcap%
\pgfsetroundjoin%
\definecolor{currentfill}{rgb}{0.050011,0.191979,0.290388}%
\pgfsetfillcolor{currentfill}%
\pgfsetlinewidth{0.000000pt}%
\definecolor{currentstroke}{rgb}{0.000000,0.000000,0.000000}%
\pgfsetstrokecolor{currentstroke}%
\pgfsetdash{}{0pt}%
\pgfpathmoveto{\pgfqpoint{9.825349in}{2.174264in}}%
\pgfpathlineto{\pgfqpoint{9.505701in}{1.611699in}}%
\pgfpathlineto{\pgfqpoint{9.538590in}{2.213087in}}%
\pgfpathlineto{\pgfqpoint{9.825349in}{2.174264in}}%
\pgfpathclose%
\pgfusepath{fill}%
\end{pgfscope}%
\begin{pgfscope}%
\pgfpathrectangle{\pgfqpoint{8.608921in}{0.208778in}}{\pgfqpoint{3.800000in}{3.800000in}}%
\pgfusepath{clip}%
\pgfsetbuttcap%
\pgfsetroundjoin%
\definecolor{currentfill}{rgb}{0.043508,0.167016,0.252629}%
\pgfsetfillcolor{currentfill}%
\pgfsetlinewidth{0.000000pt}%
\definecolor{currentstroke}{rgb}{0.000000,0.000000,0.000000}%
\pgfsetstrokecolor{currentstroke}%
\pgfsetdash{}{0pt}%
\pgfpathmoveto{\pgfqpoint{11.368375in}{1.532138in}}%
\pgfpathlineto{\pgfqpoint{11.183841in}{1.346718in}}%
\pgfpathlineto{\pgfqpoint{11.121001in}{1.892954in}}%
\pgfpathlineto{\pgfqpoint{11.368375in}{1.532138in}}%
\pgfpathclose%
\pgfusepath{fill}%
\end{pgfscope}%
\begin{pgfscope}%
\pgfpathrectangle{\pgfqpoint{8.608921in}{0.208778in}}{\pgfqpoint{3.800000in}{3.800000in}}%
\pgfusepath{clip}%
\pgfsetbuttcap%
\pgfsetroundjoin%
\definecolor{currentfill}{rgb}{0.043508,0.167016,0.252629}%
\pgfsetfillcolor{currentfill}%
\pgfsetlinewidth{0.000000pt}%
\definecolor{currentstroke}{rgb}{0.000000,0.000000,0.000000}%
\pgfsetstrokecolor{currentstroke}%
\pgfsetdash{}{0pt}%
\pgfpathmoveto{\pgfqpoint{9.999543in}{1.892954in}}%
\pgfpathlineto{\pgfqpoint{9.936703in}{1.346718in}}%
\pgfpathlineto{\pgfqpoint{9.752169in}{1.532138in}}%
\pgfpathlineto{\pgfqpoint{9.999543in}{1.892954in}}%
\pgfpathclose%
\pgfusepath{fill}%
\end{pgfscope}%
\begin{pgfscope}%
\pgfpathrectangle{\pgfqpoint{8.608921in}{0.208778in}}{\pgfqpoint{3.800000in}{3.800000in}}%
\pgfusepath{clip}%
\pgfsetbuttcap%
\pgfsetroundjoin%
\definecolor{currentfill}{rgb}{0.062760,0.240916,0.364410}%
\pgfsetfillcolor{currentfill}%
\pgfsetlinewidth{0.000000pt}%
\definecolor{currentstroke}{rgb}{0.000000,0.000000,0.000000}%
\pgfsetstrokecolor{currentstroke}%
\pgfsetdash{}{0pt}%
\pgfpathmoveto{\pgfqpoint{11.581954in}{2.213087in}}%
\pgfpathlineto{\pgfqpoint{11.366741in}{2.943542in}}%
\pgfpathlineto{\pgfqpoint{11.690090in}{2.473708in}}%
\pgfpathlineto{\pgfqpoint{11.581954in}{2.213087in}}%
\pgfpathclose%
\pgfusepath{fill}%
\end{pgfscope}%
\begin{pgfscope}%
\pgfpathrectangle{\pgfqpoint{8.608921in}{0.208778in}}{\pgfqpoint{3.800000in}{3.800000in}}%
\pgfusepath{clip}%
\pgfsetbuttcap%
\pgfsetroundjoin%
\definecolor{currentfill}{rgb}{0.062760,0.240916,0.364410}%
\pgfsetfillcolor{currentfill}%
\pgfsetlinewidth{0.000000pt}%
\definecolor{currentstroke}{rgb}{0.000000,0.000000,0.000000}%
\pgfsetstrokecolor{currentstroke}%
\pgfsetdash{}{0pt}%
\pgfpathmoveto{\pgfqpoint{9.430454in}{2.473708in}}%
\pgfpathlineto{\pgfqpoint{9.753803in}{2.943542in}}%
\pgfpathlineto{\pgfqpoint{9.538590in}{2.213087in}}%
\pgfpathlineto{\pgfqpoint{9.430454in}{2.473708in}}%
\pgfpathclose%
\pgfusepath{fill}%
\end{pgfscope}%
\begin{pgfscope}%
\pgfpathrectangle{\pgfqpoint{8.608921in}{0.208778in}}{\pgfqpoint{3.800000in}{3.800000in}}%
\pgfusepath{clip}%
\pgfsetbuttcap%
\pgfsetroundjoin%
\definecolor{currentfill}{rgb}{0.075436,0.289576,0.438014}%
\pgfsetfillcolor{currentfill}%
\pgfsetlinewidth{0.000000pt}%
\definecolor{currentstroke}{rgb}{0.000000,0.000000,0.000000}%
\pgfsetstrokecolor{currentstroke}%
\pgfsetdash{}{0pt}%
\pgfpathmoveto{\pgfqpoint{11.070569in}{2.956587in}}%
\pgfpathlineto{\pgfqpoint{11.182221in}{3.161323in}}%
\pgfpathlineto{\pgfqpoint{11.366741in}{2.943542in}}%
\pgfpathlineto{\pgfqpoint{11.070569in}{2.956587in}}%
\pgfpathclose%
\pgfusepath{fill}%
\end{pgfscope}%
\begin{pgfscope}%
\pgfpathrectangle{\pgfqpoint{8.608921in}{0.208778in}}{\pgfqpoint{3.800000in}{3.800000in}}%
\pgfusepath{clip}%
\pgfsetbuttcap%
\pgfsetroundjoin%
\definecolor{currentfill}{rgb}{0.075436,0.289576,0.438014}%
\pgfsetfillcolor{currentfill}%
\pgfsetlinewidth{0.000000pt}%
\definecolor{currentstroke}{rgb}{0.000000,0.000000,0.000000}%
\pgfsetstrokecolor{currentstroke}%
\pgfsetdash{}{0pt}%
\pgfpathmoveto{\pgfqpoint{9.753803in}{2.943542in}}%
\pgfpathlineto{\pgfqpoint{9.938323in}{3.161323in}}%
\pgfpathlineto{\pgfqpoint{10.049975in}{2.956587in}}%
\pgfpathlineto{\pgfqpoint{9.753803in}{2.943542in}}%
\pgfpathclose%
\pgfusepath{fill}%
\end{pgfscope}%
\begin{pgfscope}%
\pgfpathrectangle{\pgfqpoint{8.608921in}{0.208778in}}{\pgfqpoint{3.800000in}{3.800000in}}%
\pgfusepath{clip}%
\pgfsetbuttcap%
\pgfsetroundjoin%
\definecolor{currentfill}{rgb}{0.039595,0.151995,0.229908}%
\pgfsetfillcolor{currentfill}%
\pgfsetlinewidth{0.000000pt}%
\definecolor{currentstroke}{rgb}{0.000000,0.000000,0.000000}%
\pgfsetstrokecolor{currentstroke}%
\pgfsetdash{}{0pt}%
\pgfpathmoveto{\pgfqpoint{10.560272in}{1.278237in}}%
\pgfpathlineto{\pgfqpoint{10.735753in}{1.431593in}}%
\pgfpathlineto{\pgfqpoint{10.883622in}{1.296348in}}%
\pgfpathlineto{\pgfqpoint{10.560272in}{1.278237in}}%
\pgfpathclose%
\pgfusepath{fill}%
\end{pgfscope}%
\begin{pgfscope}%
\pgfpathrectangle{\pgfqpoint{8.608921in}{0.208778in}}{\pgfqpoint{3.800000in}{3.800000in}}%
\pgfusepath{clip}%
\pgfsetbuttcap%
\pgfsetroundjoin%
\definecolor{currentfill}{rgb}{0.039595,0.151995,0.229908}%
\pgfsetfillcolor{currentfill}%
\pgfsetlinewidth{0.000000pt}%
\definecolor{currentstroke}{rgb}{0.000000,0.000000,0.000000}%
\pgfsetstrokecolor{currentstroke}%
\pgfsetdash{}{0pt}%
\pgfpathmoveto{\pgfqpoint{10.236922in}{1.296348in}}%
\pgfpathlineto{\pgfqpoint{10.384791in}{1.431593in}}%
\pgfpathlineto{\pgfqpoint{10.560272in}{1.278237in}}%
\pgfpathlineto{\pgfqpoint{10.236922in}{1.296348in}}%
\pgfpathclose%
\pgfusepath{fill}%
\end{pgfscope}%
\begin{pgfscope}%
\pgfpathrectangle{\pgfqpoint{8.608921in}{0.208778in}}{\pgfqpoint{3.800000in}{3.800000in}}%
\pgfusepath{clip}%
\pgfsetbuttcap%
\pgfsetroundjoin%
\definecolor{currentfill}{rgb}{0.060942,0.233938,0.353856}%
\pgfsetfillcolor{currentfill}%
\pgfsetlinewidth{0.000000pt}%
\definecolor{currentstroke}{rgb}{0.000000,0.000000,0.000000}%
\pgfsetstrokecolor{currentstroke}%
\pgfsetdash{}{0pt}%
\pgfpathmoveto{\pgfqpoint{11.581954in}{2.213087in}}%
\pgfpathlineto{\pgfqpoint{11.690090in}{2.473708in}}%
\pgfpathlineto{\pgfqpoint{11.805787in}{2.255201in}}%
\pgfpathlineto{\pgfqpoint{11.581954in}{2.213087in}}%
\pgfpathclose%
\pgfusepath{fill}%
\end{pgfscope}%
\begin{pgfscope}%
\pgfpathrectangle{\pgfqpoint{8.608921in}{0.208778in}}{\pgfqpoint{3.800000in}{3.800000in}}%
\pgfusepath{clip}%
\pgfsetbuttcap%
\pgfsetroundjoin%
\definecolor{currentfill}{rgb}{0.060942,0.233938,0.353856}%
\pgfsetfillcolor{currentfill}%
\pgfsetlinewidth{0.000000pt}%
\definecolor{currentstroke}{rgb}{0.000000,0.000000,0.000000}%
\pgfsetstrokecolor{currentstroke}%
\pgfsetdash{}{0pt}%
\pgfpathmoveto{\pgfqpoint{9.314757in}{2.255201in}}%
\pgfpathlineto{\pgfqpoint{9.430454in}{2.473708in}}%
\pgfpathlineto{\pgfqpoint{9.538590in}{2.213087in}}%
\pgfpathlineto{\pgfqpoint{9.314757in}{2.255201in}}%
\pgfpathclose%
\pgfusepath{fill}%
\end{pgfscope}%
\begin{pgfscope}%
\pgfpathrectangle{\pgfqpoint{8.608921in}{0.208778in}}{\pgfqpoint{3.800000in}{3.800000in}}%
\pgfusepath{clip}%
\pgfsetbuttcap%
\pgfsetroundjoin%
\definecolor{currentfill}{rgb}{0.063981,0.245604,0.371502}%
\pgfsetfillcolor{currentfill}%
\pgfsetlinewidth{0.000000pt}%
\definecolor{currentstroke}{rgb}{0.000000,0.000000,0.000000}%
\pgfsetstrokecolor{currentstroke}%
\pgfsetdash{}{0pt}%
\pgfpathmoveto{\pgfqpoint{11.295195in}{2.174264in}}%
\pgfpathlineto{\pgfqpoint{11.445391in}{1.419643in}}%
\pgfpathlineto{\pgfqpoint{11.368375in}{1.532138in}}%
\pgfpathlineto{\pgfqpoint{11.295195in}{2.174264in}}%
\pgfpathclose%
\pgfusepath{fill}%
\end{pgfscope}%
\begin{pgfscope}%
\pgfpathrectangle{\pgfqpoint{8.608921in}{0.208778in}}{\pgfqpoint{3.800000in}{3.800000in}}%
\pgfusepath{clip}%
\pgfsetbuttcap%
\pgfsetroundjoin%
\definecolor{currentfill}{rgb}{0.063981,0.245604,0.371502}%
\pgfsetfillcolor{currentfill}%
\pgfsetlinewidth{0.000000pt}%
\definecolor{currentstroke}{rgb}{0.000000,0.000000,0.000000}%
\pgfsetstrokecolor{currentstroke}%
\pgfsetdash{}{0pt}%
\pgfpathmoveto{\pgfqpoint{9.752169in}{1.532138in}}%
\pgfpathlineto{\pgfqpoint{9.675153in}{1.419643in}}%
\pgfpathlineto{\pgfqpoint{9.825349in}{2.174264in}}%
\pgfpathlineto{\pgfqpoint{9.752169in}{1.532138in}}%
\pgfpathclose%
\pgfusepath{fill}%
\end{pgfscope}%
\begin{pgfscope}%
\pgfpathrectangle{\pgfqpoint{8.608921in}{0.208778in}}{\pgfqpoint{3.800000in}{3.800000in}}%
\pgfusepath{clip}%
\pgfsetbuttcap%
\pgfsetroundjoin%
\definecolor{currentfill}{rgb}{0.081954,0.314596,0.475860}%
\pgfsetfillcolor{currentfill}%
\pgfsetlinewidth{0.000000pt}%
\definecolor{currentstroke}{rgb}{0.000000,0.000000,0.000000}%
\pgfsetstrokecolor{currentstroke}%
\pgfsetdash{}{0pt}%
\pgfpathmoveto{\pgfqpoint{10.735368in}{2.963990in}}%
\pgfpathlineto{\pgfqpoint{10.385176in}{2.963990in}}%
\pgfpathlineto{\pgfqpoint{10.432843in}{3.635501in}}%
\pgfpathlineto{\pgfqpoint{10.735368in}{2.963990in}}%
\pgfpathclose%
\pgfusepath{fill}%
\end{pgfscope}%
\begin{pgfscope}%
\pgfpathrectangle{\pgfqpoint{8.608921in}{0.208778in}}{\pgfqpoint{3.800000in}{3.800000in}}%
\pgfusepath{clip}%
\pgfsetbuttcap%
\pgfsetroundjoin%
\definecolor{currentfill}{rgb}{0.040669,0.156116,0.236142}%
\pgfsetfillcolor{currentfill}%
\pgfsetlinewidth{0.000000pt}%
\definecolor{currentstroke}{rgb}{0.000000,0.000000,0.000000}%
\pgfsetstrokecolor{currentstroke}%
\pgfsetdash{}{0pt}%
\pgfpathmoveto{\pgfqpoint{11.071659in}{1.467999in}}%
\pgfpathlineto{\pgfqpoint{11.183841in}{1.346718in}}%
\pgfpathlineto{\pgfqpoint{10.883622in}{1.296348in}}%
\pgfpathlineto{\pgfqpoint{11.071659in}{1.467999in}}%
\pgfpathclose%
\pgfusepath{fill}%
\end{pgfscope}%
\begin{pgfscope}%
\pgfpathrectangle{\pgfqpoint{8.608921in}{0.208778in}}{\pgfqpoint{3.800000in}{3.800000in}}%
\pgfusepath{clip}%
\pgfsetbuttcap%
\pgfsetroundjoin%
\definecolor{currentfill}{rgb}{0.040669,0.156116,0.236142}%
\pgfsetfillcolor{currentfill}%
\pgfsetlinewidth{0.000000pt}%
\definecolor{currentstroke}{rgb}{0.000000,0.000000,0.000000}%
\pgfsetstrokecolor{currentstroke}%
\pgfsetdash{}{0pt}%
\pgfpathmoveto{\pgfqpoint{10.236922in}{1.296348in}}%
\pgfpathlineto{\pgfqpoint{9.936703in}{1.346718in}}%
\pgfpathlineto{\pgfqpoint{10.048885in}{1.467999in}}%
\pgfpathlineto{\pgfqpoint{10.236922in}{1.296348in}}%
\pgfpathclose%
\pgfusepath{fill}%
\end{pgfscope}%
\begin{pgfscope}%
\pgfpathrectangle{\pgfqpoint{8.608921in}{0.208778in}}{\pgfqpoint{3.800000in}{3.800000in}}%
\pgfusepath{clip}%
\pgfsetbuttcap%
\pgfsetroundjoin%
\definecolor{currentfill}{rgb}{0.045702,0.175435,0.265364}%
\pgfsetfillcolor{currentfill}%
\pgfsetlinewidth{0.000000pt}%
\definecolor{currentstroke}{rgb}{0.000000,0.000000,0.000000}%
\pgfsetstrokecolor{currentstroke}%
\pgfsetdash{}{0pt}%
\pgfpathmoveto{\pgfqpoint{11.614843in}{1.611699in}}%
\pgfpathlineto{\pgfqpoint{11.445391in}{1.419643in}}%
\pgfpathlineto{\pgfqpoint{11.295195in}{2.174264in}}%
\pgfpathlineto{\pgfqpoint{11.614843in}{1.611699in}}%
\pgfpathclose%
\pgfusepath{fill}%
\end{pgfscope}%
\begin{pgfscope}%
\pgfpathrectangle{\pgfqpoint{8.608921in}{0.208778in}}{\pgfqpoint{3.800000in}{3.800000in}}%
\pgfusepath{clip}%
\pgfsetbuttcap%
\pgfsetroundjoin%
\definecolor{currentfill}{rgb}{0.045702,0.175435,0.265364}%
\pgfsetfillcolor{currentfill}%
\pgfsetlinewidth{0.000000pt}%
\definecolor{currentstroke}{rgb}{0.000000,0.000000,0.000000}%
\pgfsetstrokecolor{currentstroke}%
\pgfsetdash{}{0pt}%
\pgfpathmoveto{\pgfqpoint{9.825349in}{2.174264in}}%
\pgfpathlineto{\pgfqpoint{9.675153in}{1.419643in}}%
\pgfpathlineto{\pgfqpoint{9.505701in}{1.611699in}}%
\pgfpathlineto{\pgfqpoint{9.825349in}{2.174264in}}%
\pgfpathclose%
\pgfusepath{fill}%
\end{pgfscope}%
\begin{pgfscope}%
\pgfpathrectangle{\pgfqpoint{8.608921in}{0.208778in}}{\pgfqpoint{3.800000in}{3.800000in}}%
\pgfusepath{clip}%
\pgfsetbuttcap%
\pgfsetroundjoin%
\definecolor{currentfill}{rgb}{0.082280,0.315849,0.477755}%
\pgfsetfillcolor{currentfill}%
\pgfsetlinewidth{0.000000pt}%
\definecolor{currentstroke}{rgb}{0.000000,0.000000,0.000000}%
\pgfsetstrokecolor{currentstroke}%
\pgfsetdash{}{0pt}%
\pgfpathmoveto{\pgfqpoint{10.735368in}{2.963990in}}%
\pgfpathlineto{\pgfqpoint{10.687701in}{3.635501in}}%
\pgfpathlineto{\pgfqpoint{11.182221in}{3.161323in}}%
\pgfpathlineto{\pgfqpoint{10.735368in}{2.963990in}}%
\pgfpathclose%
\pgfusepath{fill}%
\end{pgfscope}%
\begin{pgfscope}%
\pgfpathrectangle{\pgfqpoint{8.608921in}{0.208778in}}{\pgfqpoint{3.800000in}{3.800000in}}%
\pgfusepath{clip}%
\pgfsetbuttcap%
\pgfsetroundjoin%
\definecolor{currentfill}{rgb}{0.082280,0.315849,0.477755}%
\pgfsetfillcolor{currentfill}%
\pgfsetlinewidth{0.000000pt}%
\definecolor{currentstroke}{rgb}{0.000000,0.000000,0.000000}%
\pgfsetstrokecolor{currentstroke}%
\pgfsetdash{}{0pt}%
\pgfpathmoveto{\pgfqpoint{9.938323in}{3.161323in}}%
\pgfpathlineto{\pgfqpoint{10.432843in}{3.635501in}}%
\pgfpathlineto{\pgfqpoint{10.385176in}{2.963990in}}%
\pgfpathlineto{\pgfqpoint{9.938323in}{3.161323in}}%
\pgfpathclose%
\pgfusepath{fill}%
\end{pgfscope}%
\begin{pgfscope}%
\pgfpathrectangle{\pgfqpoint{8.608921in}{0.208778in}}{\pgfqpoint{3.800000in}{3.800000in}}%
\pgfusepath{clip}%
\pgfsetbuttcap%
\pgfsetroundjoin%
\definecolor{currentfill}{rgb}{0.052493,0.201505,0.304798}%
\pgfsetfillcolor{currentfill}%
\pgfsetlinewidth{0.000000pt}%
\definecolor{currentstroke}{rgb}{0.000000,0.000000,0.000000}%
\pgfsetstrokecolor{currentstroke}%
\pgfsetdash{}{0pt}%
\pgfpathmoveto{\pgfqpoint{11.805787in}{2.255201in}}%
\pgfpathlineto{\pgfqpoint{11.811673in}{1.695909in}}%
\pgfpathlineto{\pgfqpoint{11.581954in}{2.213087in}}%
\pgfpathlineto{\pgfqpoint{11.805787in}{2.255201in}}%
\pgfpathclose%
\pgfusepath{fill}%
\end{pgfscope}%
\begin{pgfscope}%
\pgfpathrectangle{\pgfqpoint{8.608921in}{0.208778in}}{\pgfqpoint{3.800000in}{3.800000in}}%
\pgfusepath{clip}%
\pgfsetbuttcap%
\pgfsetroundjoin%
\definecolor{currentfill}{rgb}{0.052493,0.201505,0.304798}%
\pgfsetfillcolor{currentfill}%
\pgfsetlinewidth{0.000000pt}%
\definecolor{currentstroke}{rgb}{0.000000,0.000000,0.000000}%
\pgfsetstrokecolor{currentstroke}%
\pgfsetdash{}{0pt}%
\pgfpathmoveto{\pgfqpoint{9.538590in}{2.213087in}}%
\pgfpathlineto{\pgfqpoint{9.308871in}{1.695909in}}%
\pgfpathlineto{\pgfqpoint{9.314757in}{2.255201in}}%
\pgfpathlineto{\pgfqpoint{9.538590in}{2.213087in}}%
\pgfpathclose%
\pgfusepath{fill}%
\end{pgfscope}%
\begin{pgfscope}%
\pgfpathrectangle{\pgfqpoint{8.608921in}{0.208778in}}{\pgfqpoint{3.800000in}{3.800000in}}%
\pgfusepath{clip}%
\pgfsetbuttcap%
\pgfsetroundjoin%
\definecolor{currentfill}{rgb}{0.042579,0.163449,0.247234}%
\pgfsetfillcolor{currentfill}%
\pgfsetlinewidth{0.000000pt}%
\definecolor{currentstroke}{rgb}{0.000000,0.000000,0.000000}%
\pgfsetstrokecolor{currentstroke}%
\pgfsetdash{}{0pt}%
\pgfpathmoveto{\pgfqpoint{11.445391in}{1.419643in}}%
\pgfpathlineto{\pgfqpoint{11.183841in}{1.346718in}}%
\pgfpathlineto{\pgfqpoint{11.368375in}{1.532138in}}%
\pgfpathlineto{\pgfqpoint{11.445391in}{1.419643in}}%
\pgfpathclose%
\pgfusepath{fill}%
\end{pgfscope}%
\begin{pgfscope}%
\pgfpathrectangle{\pgfqpoint{8.608921in}{0.208778in}}{\pgfqpoint{3.800000in}{3.800000in}}%
\pgfusepath{clip}%
\pgfsetbuttcap%
\pgfsetroundjoin%
\definecolor{currentfill}{rgb}{0.042579,0.163449,0.247234}%
\pgfsetfillcolor{currentfill}%
\pgfsetlinewidth{0.000000pt}%
\definecolor{currentstroke}{rgb}{0.000000,0.000000,0.000000}%
\pgfsetstrokecolor{currentstroke}%
\pgfsetdash{}{0pt}%
\pgfpathmoveto{\pgfqpoint{9.752169in}{1.532138in}}%
\pgfpathlineto{\pgfqpoint{9.936703in}{1.346718in}}%
\pgfpathlineto{\pgfqpoint{9.675153in}{1.419643in}}%
\pgfpathlineto{\pgfqpoint{9.752169in}{1.532138in}}%
\pgfpathclose%
\pgfusepath{fill}%
\end{pgfscope}%
\begin{pgfscope}%
\pgfpathrectangle{\pgfqpoint{8.608921in}{0.208778in}}{\pgfqpoint{3.800000in}{3.800000in}}%
\pgfusepath{clip}%
\pgfsetbuttcap%
\pgfsetroundjoin%
\definecolor{currentfill}{rgb}{0.067179,0.257880,0.390071}%
\pgfsetfillcolor{currentfill}%
\pgfsetlinewidth{0.000000pt}%
\definecolor{currentstroke}{rgb}{0.000000,0.000000,0.000000}%
\pgfsetstrokecolor{currentstroke}%
\pgfsetdash{}{0pt}%
\pgfpathmoveto{\pgfqpoint{11.581954in}{2.213087in}}%
\pgfpathlineto{\pgfqpoint{11.662813in}{1.504151in}}%
\pgfpathlineto{\pgfqpoint{11.614843in}{1.611699in}}%
\pgfpathlineto{\pgfqpoint{11.581954in}{2.213087in}}%
\pgfpathclose%
\pgfusepath{fill}%
\end{pgfscope}%
\begin{pgfscope}%
\pgfpathrectangle{\pgfqpoint{8.608921in}{0.208778in}}{\pgfqpoint{3.800000in}{3.800000in}}%
\pgfusepath{clip}%
\pgfsetbuttcap%
\pgfsetroundjoin%
\definecolor{currentfill}{rgb}{0.067179,0.257880,0.390071}%
\pgfsetfillcolor{currentfill}%
\pgfsetlinewidth{0.000000pt}%
\definecolor{currentstroke}{rgb}{0.000000,0.000000,0.000000}%
\pgfsetstrokecolor{currentstroke}%
\pgfsetdash{}{0pt}%
\pgfpathmoveto{\pgfqpoint{9.505701in}{1.611699in}}%
\pgfpathlineto{\pgfqpoint{9.457731in}{1.504151in}}%
\pgfpathlineto{\pgfqpoint{9.538590in}{2.213087in}}%
\pgfpathlineto{\pgfqpoint{9.505701in}{1.611699in}}%
\pgfpathclose%
\pgfusepath{fill}%
\end{pgfscope}%
\begin{pgfscope}%
\pgfpathrectangle{\pgfqpoint{8.608921in}{0.208778in}}{\pgfqpoint{3.800000in}{3.800000in}}%
\pgfusepath{clip}%
\pgfsetbuttcap%
\pgfsetroundjoin%
\definecolor{currentfill}{rgb}{0.047247,0.181368,0.274339}%
\pgfsetfillcolor{currentfill}%
\pgfsetlinewidth{0.000000pt}%
\definecolor{currentstroke}{rgb}{0.000000,0.000000,0.000000}%
\pgfsetstrokecolor{currentstroke}%
\pgfsetdash{}{0pt}%
\pgfpathmoveto{\pgfqpoint{11.811673in}{1.695909in}}%
\pgfpathlineto{\pgfqpoint{11.662813in}{1.504151in}}%
\pgfpathlineto{\pgfqpoint{11.581954in}{2.213087in}}%
\pgfpathlineto{\pgfqpoint{11.811673in}{1.695909in}}%
\pgfpathclose%
\pgfusepath{fill}%
\end{pgfscope}%
\begin{pgfscope}%
\pgfpathrectangle{\pgfqpoint{8.608921in}{0.208778in}}{\pgfqpoint{3.800000in}{3.800000in}}%
\pgfusepath{clip}%
\pgfsetbuttcap%
\pgfsetroundjoin%
\definecolor{currentfill}{rgb}{0.047247,0.181368,0.274339}%
\pgfsetfillcolor{currentfill}%
\pgfsetlinewidth{0.000000pt}%
\definecolor{currentstroke}{rgb}{0.000000,0.000000,0.000000}%
\pgfsetstrokecolor{currentstroke}%
\pgfsetdash{}{0pt}%
\pgfpathmoveto{\pgfqpoint{9.538590in}{2.213087in}}%
\pgfpathlineto{\pgfqpoint{9.457731in}{1.504151in}}%
\pgfpathlineto{\pgfqpoint{9.308871in}{1.695909in}}%
\pgfpathlineto{\pgfqpoint{9.538590in}{2.213087in}}%
\pgfpathclose%
\pgfusepath{fill}%
\end{pgfscope}%
\begin{pgfscope}%
\pgfpathrectangle{\pgfqpoint{8.608921in}{0.208778in}}{\pgfqpoint{3.800000in}{3.800000in}}%
\pgfusepath{clip}%
\pgfsetbuttcap%
\pgfsetroundjoin%
\definecolor{currentfill}{rgb}{0.081954,0.314596,0.475860}%
\pgfsetfillcolor{currentfill}%
\pgfsetlinewidth{0.000000pt}%
\definecolor{currentstroke}{rgb}{0.000000,0.000000,0.000000}%
\pgfsetstrokecolor{currentstroke}%
\pgfsetdash{}{0pt}%
\pgfpathmoveto{\pgfqpoint{10.432843in}{3.635501in}}%
\pgfpathlineto{\pgfqpoint{10.687701in}{3.635501in}}%
\pgfpathlineto{\pgfqpoint{10.735368in}{2.963990in}}%
\pgfpathlineto{\pgfqpoint{10.432843in}{3.635501in}}%
\pgfpathclose%
\pgfusepath{fill}%
\end{pgfscope}%
\begin{pgfscope}%
\pgfpathrectangle{\pgfqpoint{8.608921in}{0.208778in}}{\pgfqpoint{3.800000in}{3.800000in}}%
\pgfusepath{clip}%
\pgfsetbuttcap%
\pgfsetroundjoin%
\definecolor{currentfill}{rgb}{0.044978,0.172658,0.261163}%
\pgfsetfillcolor{currentfill}%
\pgfsetlinewidth{0.000000pt}%
\definecolor{currentstroke}{rgb}{0.000000,0.000000,0.000000}%
\pgfsetstrokecolor{currentstroke}%
\pgfsetdash{}{0pt}%
\pgfpathmoveto{\pgfqpoint{11.445391in}{1.419643in}}%
\pgfpathlineto{\pgfqpoint{11.614843in}{1.611699in}}%
\pgfpathlineto{\pgfqpoint{11.662813in}{1.504151in}}%
\pgfpathlineto{\pgfqpoint{11.445391in}{1.419643in}}%
\pgfpathclose%
\pgfusepath{fill}%
\end{pgfscope}%
\begin{pgfscope}%
\pgfpathrectangle{\pgfqpoint{8.608921in}{0.208778in}}{\pgfqpoint{3.800000in}{3.800000in}}%
\pgfusepath{clip}%
\pgfsetbuttcap%
\pgfsetroundjoin%
\definecolor{currentfill}{rgb}{0.044978,0.172658,0.261163}%
\pgfsetfillcolor{currentfill}%
\pgfsetlinewidth{0.000000pt}%
\definecolor{currentstroke}{rgb}{0.000000,0.000000,0.000000}%
\pgfsetstrokecolor{currentstroke}%
\pgfsetdash{}{0pt}%
\pgfpathmoveto{\pgfqpoint{9.457731in}{1.504151in}}%
\pgfpathlineto{\pgfqpoint{9.505701in}{1.611699in}}%
\pgfpathlineto{\pgfqpoint{9.675153in}{1.419643in}}%
\pgfpathlineto{\pgfqpoint{9.457731in}{1.504151in}}%
\pgfpathclose%
\pgfusepath{fill}%
\end{pgfscope}%
\begin{pgfscope}%
\pgfpathrectangle{\pgfqpoint{8.608921in}{0.208778in}}{\pgfqpoint{3.800000in}{3.800000in}}%
\pgfusepath{clip}%
\pgfsetbuttcap%
\pgfsetroundjoin%
\definecolor{currentfill}{rgb}{0.070254,0.269685,0.407928}%
\pgfsetfillcolor{currentfill}%
\pgfsetlinewidth{0.000000pt}%
\definecolor{currentstroke}{rgb}{0.000000,0.000000,0.000000}%
\pgfsetstrokecolor{currentstroke}%
\pgfsetdash{}{0pt}%
\pgfpathmoveto{\pgfqpoint{11.805787in}{2.255201in}}%
\pgfpathlineto{\pgfqpoint{11.838306in}{1.591377in}}%
\pgfpathlineto{\pgfqpoint{11.811673in}{1.695909in}}%
\pgfpathlineto{\pgfqpoint{11.805787in}{2.255201in}}%
\pgfpathclose%
\pgfusepath{fill}%
\end{pgfscope}%
\begin{pgfscope}%
\pgfpathrectangle{\pgfqpoint{8.608921in}{0.208778in}}{\pgfqpoint{3.800000in}{3.800000in}}%
\pgfusepath{clip}%
\pgfsetbuttcap%
\pgfsetroundjoin%
\definecolor{currentfill}{rgb}{0.070254,0.269685,0.407928}%
\pgfsetfillcolor{currentfill}%
\pgfsetlinewidth{0.000000pt}%
\definecolor{currentstroke}{rgb}{0.000000,0.000000,0.000000}%
\pgfsetstrokecolor{currentstroke}%
\pgfsetdash{}{0pt}%
\pgfpathmoveto{\pgfqpoint{9.308871in}{1.695909in}}%
\pgfpathlineto{\pgfqpoint{9.282238in}{1.591377in}}%
\pgfpathlineto{\pgfqpoint{9.314757in}{2.255201in}}%
\pgfpathlineto{\pgfqpoint{9.308871in}{1.695909in}}%
\pgfpathclose%
\pgfusepath{fill}%
\end{pgfscope}%
\begin{pgfscope}%
\pgfpathrectangle{\pgfqpoint{8.608921in}{0.208778in}}{\pgfqpoint{3.800000in}{3.800000in}}%
\pgfusepath{clip}%
\pgfsetbuttcap%
\pgfsetroundjoin%
\definecolor{currentfill}{rgb}{0.048960,0.187944,0.284285}%
\pgfsetfillcolor{currentfill}%
\pgfsetlinewidth{0.000000pt}%
\definecolor{currentstroke}{rgb}{0.000000,0.000000,0.000000}%
\pgfsetstrokecolor{currentstroke}%
\pgfsetdash{}{0pt}%
\pgfpathmoveto{\pgfqpoint{11.965734in}{1.777702in}}%
\pgfpathlineto{\pgfqpoint{11.838306in}{1.591377in}}%
\pgfpathlineto{\pgfqpoint{11.805787in}{2.255201in}}%
\pgfpathlineto{\pgfqpoint{11.965734in}{1.777702in}}%
\pgfpathclose%
\pgfusepath{fill}%
\end{pgfscope}%
\begin{pgfscope}%
\pgfpathrectangle{\pgfqpoint{8.608921in}{0.208778in}}{\pgfqpoint{3.800000in}{3.800000in}}%
\pgfusepath{clip}%
\pgfsetbuttcap%
\pgfsetroundjoin%
\definecolor{currentfill}{rgb}{0.048960,0.187944,0.284285}%
\pgfsetfillcolor{currentfill}%
\pgfsetlinewidth{0.000000pt}%
\definecolor{currentstroke}{rgb}{0.000000,0.000000,0.000000}%
\pgfsetstrokecolor{currentstroke}%
\pgfsetdash{}{0pt}%
\pgfpathmoveto{\pgfqpoint{9.154810in}{1.777702in}}%
\pgfpathlineto{\pgfqpoint{9.314757in}{2.255201in}}%
\pgfpathlineto{\pgfqpoint{9.282238in}{1.591377in}}%
\pgfpathlineto{\pgfqpoint{9.154810in}{1.777702in}}%
\pgfpathclose%
\pgfusepath{fill}%
\end{pgfscope}%
\begin{pgfscope}%
\pgfpathrectangle{\pgfqpoint{8.608921in}{0.208778in}}{\pgfqpoint{3.800000in}{3.800000in}}%
\pgfusepath{clip}%
\pgfsetbuttcap%
\pgfsetroundjoin%
\definecolor{currentfill}{rgb}{0.047548,0.182523,0.276086}%
\pgfsetfillcolor{currentfill}%
\pgfsetlinewidth{0.000000pt}%
\definecolor{currentstroke}{rgb}{0.000000,0.000000,0.000000}%
\pgfsetstrokecolor{currentstroke}%
\pgfsetdash{}{0pt}%
\pgfpathmoveto{\pgfqpoint{11.662813in}{1.504151in}}%
\pgfpathlineto{\pgfqpoint{11.811673in}{1.695909in}}%
\pgfpathlineto{\pgfqpoint{11.838306in}{1.591377in}}%
\pgfpathlineto{\pgfqpoint{11.662813in}{1.504151in}}%
\pgfpathclose%
\pgfusepath{fill}%
\end{pgfscope}%
\begin{pgfscope}%
\pgfpathrectangle{\pgfqpoint{8.608921in}{0.208778in}}{\pgfqpoint{3.800000in}{3.800000in}}%
\pgfusepath{clip}%
\pgfsetbuttcap%
\pgfsetroundjoin%
\definecolor{currentfill}{rgb}{0.047548,0.182523,0.276086}%
\pgfsetfillcolor{currentfill}%
\pgfsetlinewidth{0.000000pt}%
\definecolor{currentstroke}{rgb}{0.000000,0.000000,0.000000}%
\pgfsetstrokecolor{currentstroke}%
\pgfsetdash{}{0pt}%
\pgfpathmoveto{\pgfqpoint{9.282238in}{1.591377in}}%
\pgfpathlineto{\pgfqpoint{9.308871in}{1.695909in}}%
\pgfpathlineto{\pgfqpoint{9.457731in}{1.504151in}}%
\pgfpathlineto{\pgfqpoint{9.282238in}{1.591377in}}%
\pgfpathclose%
\pgfusepath{fill}%
\end{pgfscope}%
\begin{pgfscope}%
\pgfpathrectangle{\pgfqpoint{8.608921in}{0.208778in}}{\pgfqpoint{3.800000in}{3.800000in}}%
\pgfusepath{clip}%
\pgfsetbuttcap%
\pgfsetroundjoin%
\definecolor{currentfill}{rgb}{0.090605,0.347808,0.526096}%
\pgfsetfillcolor{currentfill}%
\pgfsetlinewidth{0.000000pt}%
\definecolor{currentstroke}{rgb}{0.000000,0.000000,0.000000}%
\pgfsetstrokecolor{currentstroke}%
\pgfsetdash{}{0pt}%
\pgfpathmoveto{\pgfqpoint{10.687701in}{3.635501in}}%
\pgfpathlineto{\pgfqpoint{10.432843in}{3.635501in}}%
\pgfpathlineto{\pgfqpoint{10.560272in}{3.736525in}}%
\pgfpathlineto{\pgfqpoint{10.687701in}{3.635501in}}%
\pgfpathclose%
\pgfusepath{fill}%
\end{pgfscope}%
\begin{pgfscope}%
\pgfpathrectangle{\pgfqpoint{8.608921in}{0.208778in}}{\pgfqpoint{3.800000in}{3.800000in}}%
\pgfusepath{clip}%
\pgfsetbuttcap%
\pgfsetroundjoin%
\definecolor{currentfill}{rgb}{0.050070,0.192203,0.290728}%
\pgfsetfillcolor{currentfill}%
\pgfsetlinewidth{0.000000pt}%
\definecolor{currentstroke}{rgb}{0.000000,0.000000,0.000000}%
\pgfsetstrokecolor{currentstroke}%
\pgfsetdash{}{0pt}%
\pgfpathmoveto{\pgfqpoint{11.838306in}{1.591377in}}%
\pgfpathlineto{\pgfqpoint{11.965734in}{1.777702in}}%
\pgfpathlineto{\pgfqpoint{11.977861in}{1.675613in}}%
\pgfpathlineto{\pgfqpoint{11.838306in}{1.591377in}}%
\pgfpathclose%
\pgfusepath{fill}%
\end{pgfscope}%
\begin{pgfscope}%
\pgfpathrectangle{\pgfqpoint{8.608921in}{0.208778in}}{\pgfqpoint{3.800000in}{3.800000in}}%
\pgfusepath{clip}%
\pgfsetbuttcap%
\pgfsetroundjoin%
\definecolor{currentfill}{rgb}{0.050070,0.192203,0.290728}%
\pgfsetfillcolor{currentfill}%
\pgfsetlinewidth{0.000000pt}%
\definecolor{currentstroke}{rgb}{0.000000,0.000000,0.000000}%
\pgfsetstrokecolor{currentstroke}%
\pgfsetdash{}{0pt}%
\pgfpathmoveto{\pgfqpoint{9.154810in}{1.777702in}}%
\pgfpathlineto{\pgfqpoint{9.282238in}{1.591377in}}%
\pgfpathlineto{\pgfqpoint{9.142683in}{1.675613in}}%
\pgfpathlineto{\pgfqpoint{9.154810in}{1.777702in}}%
\pgfpathclose%
\pgfusepath{fill}%
\end{pgfscope}%
\begin{pgfscope}%
\pgfsetbuttcap%
\pgfsetmiterjoin%
\definecolor{currentfill}{rgb}{1.000000,1.000000,1.000000}%
\pgfsetfillcolor{currentfill}%
\pgfsetlinewidth{0.000000pt}%
\definecolor{currentstroke}{rgb}{0.000000,0.000000,0.000000}%
\pgfsetstrokecolor{currentstroke}%
\pgfsetstrokeopacity{0.000000}%
\pgfsetdash{}{0pt}%
\pgfpathmoveto{\pgfqpoint{0.708921in}{0.208778in}}%
\pgfpathlineto{\pgfqpoint{4.508921in}{0.208778in}}%
\pgfpathlineto{\pgfqpoint{4.508921in}{4.008778in}}%
\pgfpathlineto{\pgfqpoint{0.708921in}{4.008778in}}%
\pgfpathlineto{\pgfqpoint{0.708921in}{0.208778in}}%
\pgfpathclose%
\pgfusepath{fill}%
\end{pgfscope}%
\begin{pgfscope}%
\pgfsetbuttcap%
\pgfsetmiterjoin%
\definecolor{currentfill}{rgb}{0.950000,0.950000,0.950000}%
\pgfsetfillcolor{currentfill}%
\pgfsetfillopacity{0.500000}%
\pgfsetlinewidth{1.003750pt}%
\definecolor{currentstroke}{rgb}{0.950000,0.950000,0.950000}%
\pgfsetstrokecolor{currentstroke}%
\pgfsetstrokeopacity{0.500000}%
\pgfsetdash{}{0pt}%
\pgfpathmoveto{\pgfqpoint{2.660272in}{2.686431in}}%
\pgfpathlineto{\pgfqpoint{4.336450in}{1.531563in}}%
\pgfpathlineto{\pgfqpoint{4.445724in}{2.829674in}}%
\pgfpathlineto{\pgfqpoint{2.660272in}{3.980304in}}%
\pgfusepath{stroke,fill}%
\end{pgfscope}%
\begin{pgfscope}%
\pgfsetbuttcap%
\pgfsetmiterjoin%
\definecolor{currentfill}{rgb}{0.900000,0.900000,0.900000}%
\pgfsetfillcolor{currentfill}%
\pgfsetfillopacity{0.500000}%
\pgfsetlinewidth{1.003750pt}%
\definecolor{currentstroke}{rgb}{0.900000,0.900000,0.900000}%
\pgfsetstrokecolor{currentstroke}%
\pgfsetstrokeopacity{0.500000}%
\pgfsetdash{}{0pt}%
\pgfpathmoveto{\pgfqpoint{2.660272in}{2.686431in}}%
\pgfpathlineto{\pgfqpoint{0.984094in}{1.531563in}}%
\pgfpathlineto{\pgfqpoint{0.874820in}{2.829674in}}%
\pgfpathlineto{\pgfqpoint{2.660272in}{3.980304in}}%
\pgfusepath{stroke,fill}%
\end{pgfscope}%
\begin{pgfscope}%
\pgfsetbuttcap%
\pgfsetmiterjoin%
\definecolor{currentfill}{rgb}{0.925000,0.925000,0.925000}%
\pgfsetfillcolor{currentfill}%
\pgfsetfillopacity{0.500000}%
\pgfsetlinewidth{1.003750pt}%
\definecolor{currentstroke}{rgb}{0.925000,0.925000,0.925000}%
\pgfsetstrokecolor{currentstroke}%
\pgfsetstrokeopacity{0.500000}%
\pgfsetdash{}{0pt}%
\pgfpathmoveto{\pgfqpoint{2.660272in}{2.686431in}}%
\pgfpathlineto{\pgfqpoint{0.984094in}{1.531563in}}%
\pgfpathlineto{\pgfqpoint{2.660272in}{0.235257in}}%
\pgfpathlineto{\pgfqpoint{4.336450in}{1.531563in}}%
\pgfusepath{stroke,fill}%
\end{pgfscope}%
\begin{pgfscope}%
\pgfsetrectcap%
\pgfsetroundjoin%
\pgfsetlinewidth{0.803000pt}%
\definecolor{currentstroke}{rgb}{0.000000,0.000000,0.000000}%
\pgfsetstrokecolor{currentstroke}%
\pgfsetdash{}{0pt}%
\pgfpathmoveto{\pgfqpoint{4.336450in}{1.531563in}}%
\pgfpathlineto{\pgfqpoint{2.660272in}{0.235257in}}%
\pgfusepath{stroke}%
\end{pgfscope}%
\begin{pgfscope}%
\definecolor{textcolor}{rgb}{0.000000,0.000000,0.000000}%
\pgfsetstrokecolor{textcolor}%
\pgfsetfillcolor{textcolor}%
\pgftext[x=3.897764in,y=0.356972in,,]{\color{textcolor}\rmfamily\fontsize{14.000000}{16.800000}\selectfont f1}%
\end{pgfscope}%
\begin{pgfscope}%
\pgfsetbuttcap%
\pgfsetroundjoin%
\pgfsetlinewidth{0.803000pt}%
\definecolor{currentstroke}{rgb}{0.690196,0.690196,0.690196}%
\pgfsetstrokecolor{currentstroke}%
\pgfsetdash{}{0pt}%
\pgfpathmoveto{\pgfqpoint{4.235568in}{1.453544in}}%
\pgfpathlineto{\pgfqpoint{2.559074in}{2.616707in}}%
\pgfpathlineto{\pgfqpoint{2.552836in}{3.911067in}}%
\pgfusepath{stroke}%
\end{pgfscope}%
\begin{pgfscope}%
\pgfsetbuttcap%
\pgfsetroundjoin%
\pgfsetlinewidth{0.803000pt}%
\definecolor{currentstroke}{rgb}{0.690196,0.690196,0.690196}%
\pgfsetstrokecolor{currentstroke}%
\pgfsetdash{}{0pt}%
\pgfpathmoveto{\pgfqpoint{3.980751in}{1.256476in}}%
\pgfpathlineto{\pgfqpoint{2.303638in}{2.440714in}}%
\pgfpathlineto{\pgfqpoint{2.281449in}{3.736172in}}%
\pgfusepath{stroke}%
\end{pgfscope}%
\begin{pgfscope}%
\pgfsetbuttcap%
\pgfsetroundjoin%
\pgfsetlinewidth{0.803000pt}%
\definecolor{currentstroke}{rgb}{0.690196,0.690196,0.690196}%
\pgfsetstrokecolor{currentstroke}%
\pgfsetdash{}{0pt}%
\pgfpathmoveto{\pgfqpoint{3.721163in}{1.055718in}}%
\pgfpathlineto{\pgfqpoint{2.043686in}{2.261610in}}%
\pgfpathlineto{\pgfqpoint{2.004962in}{3.557991in}}%
\pgfusepath{stroke}%
\end{pgfscope}%
\begin{pgfscope}%
\pgfsetbuttcap%
\pgfsetroundjoin%
\pgfsetlinewidth{0.803000pt}%
\definecolor{currentstroke}{rgb}{0.690196,0.690196,0.690196}%
\pgfsetstrokecolor{currentstroke}%
\pgfsetdash{}{0pt}%
\pgfpathmoveto{\pgfqpoint{3.456668in}{0.851166in}}%
\pgfpathlineto{\pgfqpoint{1.779096in}{2.079311in}}%
\pgfpathlineto{\pgfqpoint{1.723231in}{3.376430in}}%
\pgfusepath{stroke}%
\end{pgfscope}%
\begin{pgfscope}%
\pgfsetbuttcap%
\pgfsetroundjoin%
\pgfsetlinewidth{0.803000pt}%
\definecolor{currentstroke}{rgb}{0.690196,0.690196,0.690196}%
\pgfsetstrokecolor{currentstroke}%
\pgfsetdash{}{0pt}%
\pgfpathmoveto{\pgfqpoint{3.187126in}{0.642710in}}%
\pgfpathlineto{\pgfqpoint{1.509744in}{1.893730in}}%
\pgfpathlineto{\pgfqpoint{1.436104in}{3.191392in}}%
\pgfusepath{stroke}%
\end{pgfscope}%
\begin{pgfscope}%
\pgfsetbuttcap%
\pgfsetroundjoin%
\pgfsetlinewidth{0.803000pt}%
\definecolor{currentstroke}{rgb}{0.690196,0.690196,0.690196}%
\pgfsetstrokecolor{currentstroke}%
\pgfsetdash{}{0pt}%
\pgfpathmoveto{\pgfqpoint{2.912392in}{0.430239in}}%
\pgfpathlineto{\pgfqpoint{1.235501in}{1.704779in}}%
\pgfpathlineto{\pgfqpoint{1.143425in}{3.002776in}}%
\pgfusepath{stroke}%
\end{pgfscope}%
\begin{pgfscope}%
\pgfsetrectcap%
\pgfsetroundjoin%
\pgfsetlinewidth{0.803000pt}%
\definecolor{currentstroke}{rgb}{0.000000,0.000000,0.000000}%
\pgfsetstrokecolor{currentstroke}%
\pgfsetdash{}{0pt}%
\pgfpathmoveto{\pgfqpoint{4.221386in}{1.463384in}}%
\pgfpathlineto{\pgfqpoint{4.263972in}{1.433838in}}%
\pgfusepath{stroke}%
\end{pgfscope}%
\begin{pgfscope}%
\definecolor{textcolor}{rgb}{0.000000,0.000000,0.000000}%
\pgfsetstrokecolor{textcolor}%
\pgfsetfillcolor{textcolor}%
\pgftext[x=4.370852in,y=1.255971in,,top]{\color{textcolor}\rmfamily\fontsize{10.000000}{12.000000}\selectfont \(\displaystyle {0.0}\)}%
\end{pgfscope}%
\begin{pgfscope}%
\pgfsetrectcap%
\pgfsetroundjoin%
\pgfsetlinewidth{0.803000pt}%
\definecolor{currentstroke}{rgb}{0.000000,0.000000,0.000000}%
\pgfsetstrokecolor{currentstroke}%
\pgfsetdash{}{0pt}%
\pgfpathmoveto{\pgfqpoint{3.966556in}{1.266499in}}%
\pgfpathlineto{\pgfqpoint{4.009180in}{1.236402in}}%
\pgfusepath{stroke}%
\end{pgfscope}%
\begin{pgfscope}%
\definecolor{textcolor}{rgb}{0.000000,0.000000,0.000000}%
\pgfsetstrokecolor{textcolor}%
\pgfsetfillcolor{textcolor}%
\pgftext[x=4.117362in,y=1.057127in,,top]{\color{textcolor}\rmfamily\fontsize{10.000000}{12.000000}\selectfont \(\displaystyle {0.2}\)}%
\end{pgfscope}%
\begin{pgfscope}%
\pgfsetrectcap%
\pgfsetroundjoin%
\pgfsetlinewidth{0.803000pt}%
\definecolor{currentstroke}{rgb}{0.000000,0.000000,0.000000}%
\pgfsetstrokecolor{currentstroke}%
\pgfsetdash{}{0pt}%
\pgfpathmoveto{\pgfqpoint{3.706958in}{1.065930in}}%
\pgfpathlineto{\pgfqpoint{3.749613in}{1.035266in}}%
\pgfusepath{stroke}%
\end{pgfscope}%
\begin{pgfscope}%
\definecolor{textcolor}{rgb}{0.000000,0.000000,0.000000}%
\pgfsetstrokecolor{textcolor}%
\pgfsetfillcolor{textcolor}%
\pgftext[x=3.859127in,y=0.854561in,,top]{\color{textcolor}\rmfamily\fontsize{10.000000}{12.000000}\selectfont \(\displaystyle {0.4}\)}%
\end{pgfscope}%
\begin{pgfscope}%
\pgfsetrectcap%
\pgfsetroundjoin%
\pgfsetlinewidth{0.803000pt}%
\definecolor{currentstroke}{rgb}{0.000000,0.000000,0.000000}%
\pgfsetstrokecolor{currentstroke}%
\pgfsetdash{}{0pt}%
\pgfpathmoveto{\pgfqpoint{3.442455in}{0.861571in}}%
\pgfpathlineto{\pgfqpoint{3.485135in}{0.830325in}}%
\pgfusepath{stroke}%
\end{pgfscope}%
\begin{pgfscope}%
\definecolor{textcolor}{rgb}{0.000000,0.000000,0.000000}%
\pgfsetstrokecolor{textcolor}%
\pgfsetfillcolor{textcolor}%
\pgftext[x=3.596013in,y=0.648168in,,top]{\color{textcolor}\rmfamily\fontsize{10.000000}{12.000000}\selectfont \(\displaystyle {0.6}\)}%
\end{pgfscope}%
\begin{pgfscope}%
\pgfsetrectcap%
\pgfsetroundjoin%
\pgfsetlinewidth{0.803000pt}%
\definecolor{currentstroke}{rgb}{0.000000,0.000000,0.000000}%
\pgfsetstrokecolor{currentstroke}%
\pgfsetdash{}{0pt}%
\pgfpathmoveto{\pgfqpoint{3.172907in}{0.653315in}}%
\pgfpathlineto{\pgfqpoint{3.215605in}{0.621470in}}%
\pgfusepath{stroke}%
\end{pgfscope}%
\begin{pgfscope}%
\definecolor{textcolor}{rgb}{0.000000,0.000000,0.000000}%
\pgfsetstrokecolor{textcolor}%
\pgfsetfillcolor{textcolor}%
\pgftext[x=3.327880in,y=0.437838in,,top]{\color{textcolor}\rmfamily\fontsize{10.000000}{12.000000}\selectfont \(\displaystyle {0.8}\)}%
\end{pgfscope}%
\begin{pgfscope}%
\pgfsetrectcap%
\pgfsetroundjoin%
\pgfsetlinewidth{0.803000pt}%
\definecolor{currentstroke}{rgb}{0.000000,0.000000,0.000000}%
\pgfsetstrokecolor{currentstroke}%
\pgfsetdash{}{0pt}%
\pgfpathmoveto{\pgfqpoint{2.898169in}{0.441049in}}%
\pgfpathlineto{\pgfqpoint{2.940878in}{0.408587in}}%
\pgfusepath{stroke}%
\end{pgfscope}%
\begin{pgfscope}%
\definecolor{textcolor}{rgb}{0.000000,0.000000,0.000000}%
\pgfsetstrokecolor{textcolor}%
\pgfsetfillcolor{textcolor}%
\pgftext[x=3.054582in,y=0.223457in,,top]{\color{textcolor}\rmfamily\fontsize{10.000000}{12.000000}\selectfont \(\displaystyle {1.0}\)}%
\end{pgfscope}%
\begin{pgfscope}%
\pgfsetrectcap%
\pgfsetroundjoin%
\pgfsetlinewidth{0.803000pt}%
\definecolor{currentstroke}{rgb}{0.000000,0.000000,0.000000}%
\pgfsetstrokecolor{currentstroke}%
\pgfsetdash{}{0pt}%
\pgfpathmoveto{\pgfqpoint{0.984094in}{1.531563in}}%
\pgfpathlineto{\pgfqpoint{2.660272in}{0.235257in}}%
\pgfusepath{stroke}%
\end{pgfscope}%
\begin{pgfscope}%
\definecolor{textcolor}{rgb}{0.000000,0.000000,0.000000}%
\pgfsetstrokecolor{textcolor}%
\pgfsetfillcolor{textcolor}%
\pgftext[x=1.422780in,y=0.356972in,,]{\color{textcolor}\rmfamily\fontsize{14.000000}{16.800000}\selectfont f2}%
\end{pgfscope}%
\begin{pgfscope}%
\pgfsetbuttcap%
\pgfsetroundjoin%
\pgfsetlinewidth{0.803000pt}%
\definecolor{currentstroke}{rgb}{0.690196,0.690196,0.690196}%
\pgfsetstrokecolor{currentstroke}%
\pgfsetdash{}{0pt}%
\pgfpathmoveto{\pgfqpoint{2.767708in}{3.911067in}}%
\pgfpathlineto{\pgfqpoint{2.761470in}{2.616707in}}%
\pgfpathlineto{\pgfqpoint{1.084976in}{1.453544in}}%
\pgfusepath{stroke}%
\end{pgfscope}%
\begin{pgfscope}%
\pgfsetbuttcap%
\pgfsetroundjoin%
\pgfsetlinewidth{0.803000pt}%
\definecolor{currentstroke}{rgb}{0.690196,0.690196,0.690196}%
\pgfsetstrokecolor{currentstroke}%
\pgfsetdash{}{0pt}%
\pgfpathmoveto{\pgfqpoint{3.039095in}{3.736172in}}%
\pgfpathlineto{\pgfqpoint{3.016906in}{2.440714in}}%
\pgfpathlineto{\pgfqpoint{1.339793in}{1.256476in}}%
\pgfusepath{stroke}%
\end{pgfscope}%
\begin{pgfscope}%
\pgfsetbuttcap%
\pgfsetroundjoin%
\pgfsetlinewidth{0.803000pt}%
\definecolor{currentstroke}{rgb}{0.690196,0.690196,0.690196}%
\pgfsetstrokecolor{currentstroke}%
\pgfsetdash{}{0pt}%
\pgfpathmoveto{\pgfqpoint{3.315582in}{3.557991in}}%
\pgfpathlineto{\pgfqpoint{3.276858in}{2.261610in}}%
\pgfpathlineto{\pgfqpoint{1.599381in}{1.055718in}}%
\pgfusepath{stroke}%
\end{pgfscope}%
\begin{pgfscope}%
\pgfsetbuttcap%
\pgfsetroundjoin%
\pgfsetlinewidth{0.803000pt}%
\definecolor{currentstroke}{rgb}{0.690196,0.690196,0.690196}%
\pgfsetstrokecolor{currentstroke}%
\pgfsetdash{}{0pt}%
\pgfpathmoveto{\pgfqpoint{3.597313in}{3.376430in}}%
\pgfpathlineto{\pgfqpoint{3.541448in}{2.079311in}}%
\pgfpathlineto{\pgfqpoint{1.863876in}{0.851166in}}%
\pgfusepath{stroke}%
\end{pgfscope}%
\begin{pgfscope}%
\pgfsetbuttcap%
\pgfsetroundjoin%
\pgfsetlinewidth{0.803000pt}%
\definecolor{currentstroke}{rgb}{0.690196,0.690196,0.690196}%
\pgfsetstrokecolor{currentstroke}%
\pgfsetdash{}{0pt}%
\pgfpathmoveto{\pgfqpoint{3.884440in}{3.191392in}}%
\pgfpathlineto{\pgfqpoint{3.810800in}{1.893730in}}%
\pgfpathlineto{\pgfqpoint{2.133418in}{0.642710in}}%
\pgfusepath{stroke}%
\end{pgfscope}%
\begin{pgfscope}%
\pgfsetbuttcap%
\pgfsetroundjoin%
\pgfsetlinewidth{0.803000pt}%
\definecolor{currentstroke}{rgb}{0.690196,0.690196,0.690196}%
\pgfsetstrokecolor{currentstroke}%
\pgfsetdash{}{0pt}%
\pgfpathmoveto{\pgfqpoint{4.177119in}{3.002776in}}%
\pgfpathlineto{\pgfqpoint{4.085043in}{1.704779in}}%
\pgfpathlineto{\pgfqpoint{2.408152in}{0.430239in}}%
\pgfusepath{stroke}%
\end{pgfscope}%
\begin{pgfscope}%
\pgfsetrectcap%
\pgfsetroundjoin%
\pgfsetlinewidth{0.803000pt}%
\definecolor{currentstroke}{rgb}{0.000000,0.000000,0.000000}%
\pgfsetstrokecolor{currentstroke}%
\pgfsetdash{}{0pt}%
\pgfpathmoveto{\pgfqpoint{1.099158in}{1.463384in}}%
\pgfpathlineto{\pgfqpoint{1.056572in}{1.433838in}}%
\pgfusepath{stroke}%
\end{pgfscope}%
\begin{pgfscope}%
\definecolor{textcolor}{rgb}{0.000000,0.000000,0.000000}%
\pgfsetstrokecolor{textcolor}%
\pgfsetfillcolor{textcolor}%
\pgftext[x=0.949692in,y=1.255971in,,top]{\color{textcolor}\rmfamily\fontsize{10.000000}{12.000000}\selectfont \(\displaystyle {0.0}\)}%
\end{pgfscope}%
\begin{pgfscope}%
\pgfsetrectcap%
\pgfsetroundjoin%
\pgfsetlinewidth{0.803000pt}%
\definecolor{currentstroke}{rgb}{0.000000,0.000000,0.000000}%
\pgfsetstrokecolor{currentstroke}%
\pgfsetdash{}{0pt}%
\pgfpathmoveto{\pgfqpoint{1.353988in}{1.266499in}}%
\pgfpathlineto{\pgfqpoint{1.311364in}{1.236402in}}%
\pgfusepath{stroke}%
\end{pgfscope}%
\begin{pgfscope}%
\definecolor{textcolor}{rgb}{0.000000,0.000000,0.000000}%
\pgfsetstrokecolor{textcolor}%
\pgfsetfillcolor{textcolor}%
\pgftext[x=1.203182in,y=1.057127in,,top]{\color{textcolor}\rmfamily\fontsize{10.000000}{12.000000}\selectfont \(\displaystyle {0.2}\)}%
\end{pgfscope}%
\begin{pgfscope}%
\pgfsetrectcap%
\pgfsetroundjoin%
\pgfsetlinewidth{0.803000pt}%
\definecolor{currentstroke}{rgb}{0.000000,0.000000,0.000000}%
\pgfsetstrokecolor{currentstroke}%
\pgfsetdash{}{0pt}%
\pgfpathmoveto{\pgfqpoint{1.613586in}{1.065930in}}%
\pgfpathlineto{\pgfqpoint{1.570931in}{1.035266in}}%
\pgfusepath{stroke}%
\end{pgfscope}%
\begin{pgfscope}%
\definecolor{textcolor}{rgb}{0.000000,0.000000,0.000000}%
\pgfsetstrokecolor{textcolor}%
\pgfsetfillcolor{textcolor}%
\pgftext[x=1.461417in,y=0.854561in,,top]{\color{textcolor}\rmfamily\fontsize{10.000000}{12.000000}\selectfont \(\displaystyle {0.4}\)}%
\end{pgfscope}%
\begin{pgfscope}%
\pgfsetrectcap%
\pgfsetroundjoin%
\pgfsetlinewidth{0.803000pt}%
\definecolor{currentstroke}{rgb}{0.000000,0.000000,0.000000}%
\pgfsetstrokecolor{currentstroke}%
\pgfsetdash{}{0pt}%
\pgfpathmoveto{\pgfqpoint{1.878089in}{0.861571in}}%
\pgfpathlineto{\pgfqpoint{1.835409in}{0.830325in}}%
\pgfusepath{stroke}%
\end{pgfscope}%
\begin{pgfscope}%
\definecolor{textcolor}{rgb}{0.000000,0.000000,0.000000}%
\pgfsetstrokecolor{textcolor}%
\pgfsetfillcolor{textcolor}%
\pgftext[x=1.724531in,y=0.648168in,,top]{\color{textcolor}\rmfamily\fontsize{10.000000}{12.000000}\selectfont \(\displaystyle {0.6}\)}%
\end{pgfscope}%
\begin{pgfscope}%
\pgfsetrectcap%
\pgfsetroundjoin%
\pgfsetlinewidth{0.803000pt}%
\definecolor{currentstroke}{rgb}{0.000000,0.000000,0.000000}%
\pgfsetstrokecolor{currentstroke}%
\pgfsetdash{}{0pt}%
\pgfpathmoveto{\pgfqpoint{2.147637in}{0.653315in}}%
\pgfpathlineto{\pgfqpoint{2.104939in}{0.621470in}}%
\pgfusepath{stroke}%
\end{pgfscope}%
\begin{pgfscope}%
\definecolor{textcolor}{rgb}{0.000000,0.000000,0.000000}%
\pgfsetstrokecolor{textcolor}%
\pgfsetfillcolor{textcolor}%
\pgftext[x=1.992664in,y=0.437838in,,top]{\color{textcolor}\rmfamily\fontsize{10.000000}{12.000000}\selectfont \(\displaystyle {0.8}\)}%
\end{pgfscope}%
\begin{pgfscope}%
\pgfsetrectcap%
\pgfsetroundjoin%
\pgfsetlinewidth{0.803000pt}%
\definecolor{currentstroke}{rgb}{0.000000,0.000000,0.000000}%
\pgfsetstrokecolor{currentstroke}%
\pgfsetdash{}{0pt}%
\pgfpathmoveto{\pgfqpoint{2.422375in}{0.441049in}}%
\pgfpathlineto{\pgfqpoint{2.379666in}{0.408587in}}%
\pgfusepath{stroke}%
\end{pgfscope}%
\begin{pgfscope}%
\definecolor{textcolor}{rgb}{0.000000,0.000000,0.000000}%
\pgfsetstrokecolor{textcolor}%
\pgfsetfillcolor{textcolor}%
\pgftext[x=2.265962in,y=0.223457in,,top]{\color{textcolor}\rmfamily\fontsize{10.000000}{12.000000}\selectfont \(\displaystyle {1.0}\)}%
\end{pgfscope}%
\begin{pgfscope}%
\pgfsetrectcap%
\pgfsetroundjoin%
\pgfsetlinewidth{0.803000pt}%
\definecolor{currentstroke}{rgb}{0.000000,0.000000,0.000000}%
\pgfsetstrokecolor{currentstroke}%
\pgfsetdash{}{0pt}%
\pgfpathmoveto{\pgfqpoint{0.984094in}{1.531563in}}%
\pgfpathlineto{\pgfqpoint{0.874820in}{2.829674in}}%
\pgfusepath{stroke}%
\end{pgfscope}%
\begin{pgfscope}%
\definecolor{textcolor}{rgb}{0.000000,0.000000,0.000000}%
\pgfsetstrokecolor{textcolor}%
\pgfsetfillcolor{textcolor}%
\pgftext[x=0.178876in,y=2.160130in,,]{\color{textcolor}\rmfamily\fontsize{14.000000}{16.800000}\selectfont f3}%
\end{pgfscope}%
\begin{pgfscope}%
\pgfsetbuttcap%
\pgfsetroundjoin%
\pgfsetlinewidth{0.803000pt}%
\definecolor{currentstroke}{rgb}{0.690196,0.690196,0.690196}%
\pgfsetstrokecolor{currentstroke}%
\pgfsetdash{}{0pt}%
\pgfpathmoveto{\pgfqpoint{0.977541in}{1.609418in}}%
\pgfpathlineto{\pgfqpoint{2.660272in}{2.764291in}}%
\pgfpathlineto{\pgfqpoint{4.343003in}{1.609418in}}%
\pgfusepath{stroke}%
\end{pgfscope}%
\begin{pgfscope}%
\pgfsetbuttcap%
\pgfsetroundjoin%
\pgfsetlinewidth{0.803000pt}%
\definecolor{currentstroke}{rgb}{0.690196,0.690196,0.690196}%
\pgfsetstrokecolor{currentstroke}%
\pgfsetdash{}{0pt}%
\pgfpathmoveto{\pgfqpoint{0.960973in}{1.806226in}}%
\pgfpathlineto{\pgfqpoint{2.660272in}{2.960964in}}%
\pgfpathlineto{\pgfqpoint{4.359571in}{1.806226in}}%
\pgfusepath{stroke}%
\end{pgfscope}%
\begin{pgfscope}%
\pgfsetbuttcap%
\pgfsetroundjoin%
\pgfsetlinewidth{0.803000pt}%
\definecolor{currentstroke}{rgb}{0.690196,0.690196,0.690196}%
\pgfsetstrokecolor{currentstroke}%
\pgfsetdash{}{0pt}%
\pgfpathmoveto{\pgfqpoint{0.944077in}{2.006949in}}%
\pgfpathlineto{\pgfqpoint{2.660272in}{3.161331in}}%
\pgfpathlineto{\pgfqpoint{4.376467in}{2.006949in}}%
\pgfusepath{stroke}%
\end{pgfscope}%
\begin{pgfscope}%
\pgfsetbuttcap%
\pgfsetroundjoin%
\pgfsetlinewidth{0.803000pt}%
\definecolor{currentstroke}{rgb}{0.690196,0.690196,0.690196}%
\pgfsetstrokecolor{currentstroke}%
\pgfsetdash{}{0pt}%
\pgfpathmoveto{\pgfqpoint{0.926841in}{2.211703in}}%
\pgfpathlineto{\pgfqpoint{2.660272in}{3.365496in}}%
\pgfpathlineto{\pgfqpoint{4.393703in}{2.211703in}}%
\pgfusepath{stroke}%
\end{pgfscope}%
\begin{pgfscope}%
\pgfsetbuttcap%
\pgfsetroundjoin%
\pgfsetlinewidth{0.803000pt}%
\definecolor{currentstroke}{rgb}{0.690196,0.690196,0.690196}%
\pgfsetstrokecolor{currentstroke}%
\pgfsetdash{}{0pt}%
\pgfpathmoveto{\pgfqpoint{0.909255in}{2.420612in}}%
\pgfpathlineto{\pgfqpoint{2.660272in}{3.573568in}}%
\pgfpathlineto{\pgfqpoint{4.411289in}{2.420612in}}%
\pgfusepath{stroke}%
\end{pgfscope}%
\begin{pgfscope}%
\pgfsetbuttcap%
\pgfsetroundjoin%
\pgfsetlinewidth{0.803000pt}%
\definecolor{currentstroke}{rgb}{0.690196,0.690196,0.690196}%
\pgfsetstrokecolor{currentstroke}%
\pgfsetdash{}{0pt}%
\pgfpathmoveto{\pgfqpoint{0.891309in}{2.633803in}}%
\pgfpathlineto{\pgfqpoint{2.660272in}{3.785660in}}%
\pgfpathlineto{\pgfqpoint{4.429235in}{2.633803in}}%
\pgfusepath{stroke}%
\end{pgfscope}%
\begin{pgfscope}%
\pgfsetrectcap%
\pgfsetroundjoin%
\pgfsetlinewidth{0.803000pt}%
\definecolor{currentstroke}{rgb}{0.000000,0.000000,0.000000}%
\pgfsetstrokecolor{currentstroke}%
\pgfsetdash{}{0pt}%
\pgfpathmoveto{\pgfqpoint{0.991776in}{1.619187in}}%
\pgfpathlineto{\pgfqpoint{0.949031in}{1.589851in}}%
\pgfusepath{stroke}%
\end{pgfscope}%
\begin{pgfscope}%
\definecolor{textcolor}{rgb}{0.000000,0.000000,0.000000}%
\pgfsetstrokecolor{textcolor}%
\pgfsetfillcolor{textcolor}%
\pgftext[x=0.705942in,y=1.609418in,,top]{\color{textcolor}\rmfamily\fontsize{10.000000}{12.000000}\selectfont \(\displaystyle {0.0}\)}%
\end{pgfscope}%
\begin{pgfscope}%
\pgfsetrectcap%
\pgfsetroundjoin%
\pgfsetlinewidth{0.803000pt}%
\definecolor{currentstroke}{rgb}{0.000000,0.000000,0.000000}%
\pgfsetstrokecolor{currentstroke}%
\pgfsetdash{}{0pt}%
\pgfpathmoveto{\pgfqpoint{0.975356in}{1.816000in}}%
\pgfpathlineto{\pgfqpoint{0.932167in}{1.786651in}}%
\pgfusepath{stroke}%
\end{pgfscope}%
\begin{pgfscope}%
\definecolor{textcolor}{rgb}{0.000000,0.000000,0.000000}%
\pgfsetstrokecolor{textcolor}%
\pgfsetfillcolor{textcolor}%
\pgftext[x=0.686701in,y=1.806226in,,top]{\color{textcolor}\rmfamily\fontsize{10.000000}{12.000000}\selectfont \(\displaystyle {0.2}\)}%
\end{pgfscope}%
\begin{pgfscope}%
\pgfsetrectcap%
\pgfsetroundjoin%
\pgfsetlinewidth{0.803000pt}%
\definecolor{currentstroke}{rgb}{0.000000,0.000000,0.000000}%
\pgfsetstrokecolor{currentstroke}%
\pgfsetdash{}{0pt}%
\pgfpathmoveto{\pgfqpoint{0.958611in}{2.016725in}}%
\pgfpathlineto{\pgfqpoint{0.914968in}{1.987369in}}%
\pgfusepath{stroke}%
\end{pgfscope}%
\begin{pgfscope}%
\definecolor{textcolor}{rgb}{0.000000,0.000000,0.000000}%
\pgfsetstrokecolor{textcolor}%
\pgfsetfillcolor{textcolor}%
\pgftext[x=0.667077in,y=2.006949in,,top]{\color{textcolor}\rmfamily\fontsize{10.000000}{12.000000}\selectfont \(\displaystyle {0.4}\)}%
\end{pgfscope}%
\begin{pgfscope}%
\pgfsetrectcap%
\pgfsetroundjoin%
\pgfsetlinewidth{0.803000pt}%
\definecolor{currentstroke}{rgb}{0.000000,0.000000,0.000000}%
\pgfsetstrokecolor{currentstroke}%
\pgfsetdash{}{0pt}%
\pgfpathmoveto{\pgfqpoint{0.941529in}{2.221479in}}%
\pgfpathlineto{\pgfqpoint{0.897423in}{2.192122in}}%
\pgfusepath{stroke}%
\end{pgfscope}%
\begin{pgfscope}%
\definecolor{textcolor}{rgb}{0.000000,0.000000,0.000000}%
\pgfsetstrokecolor{textcolor}%
\pgfsetfillcolor{textcolor}%
\pgftext[x=0.647059in,y=2.211703in,,top]{\color{textcolor}\rmfamily\fontsize{10.000000}{12.000000}\selectfont \(\displaystyle {0.6}\)}%
\end{pgfscope}%
\begin{pgfscope}%
\pgfsetrectcap%
\pgfsetroundjoin%
\pgfsetlinewidth{0.803000pt}%
\definecolor{currentstroke}{rgb}{0.000000,0.000000,0.000000}%
\pgfsetstrokecolor{currentstroke}%
\pgfsetdash{}{0pt}%
\pgfpathmoveto{\pgfqpoint{0.924100in}{2.430387in}}%
\pgfpathlineto{\pgfqpoint{0.879521in}{2.401034in}}%
\pgfusepath{stroke}%
\end{pgfscope}%
\begin{pgfscope}%
\definecolor{textcolor}{rgb}{0.000000,0.000000,0.000000}%
\pgfsetstrokecolor{textcolor}%
\pgfsetfillcolor{textcolor}%
\pgftext[x=0.626635in,y=2.420612in,,top]{\color{textcolor}\rmfamily\fontsize{10.000000}{12.000000}\selectfont \(\displaystyle {0.8}\)}%
\end{pgfscope}%
\begin{pgfscope}%
\pgfsetrectcap%
\pgfsetroundjoin%
\pgfsetlinewidth{0.803000pt}%
\definecolor{currentstroke}{rgb}{0.000000,0.000000,0.000000}%
\pgfsetstrokecolor{currentstroke}%
\pgfsetdash{}{0pt}%
\pgfpathmoveto{\pgfqpoint{0.906315in}{2.643574in}}%
\pgfpathlineto{\pgfqpoint{0.861252in}{2.614232in}}%
\pgfusepath{stroke}%
\end{pgfscope}%
\begin{pgfscope}%
\definecolor{textcolor}{rgb}{0.000000,0.000000,0.000000}%
\pgfsetstrokecolor{textcolor}%
\pgfsetfillcolor{textcolor}%
\pgftext[x=0.605792in,y=2.633803in,,top]{\color{textcolor}\rmfamily\fontsize{10.000000}{12.000000}\selectfont \(\displaystyle {1.0}\)}%
\end{pgfscope}%
\begin{pgfscope}%
\pgfpathrectangle{\pgfqpoint{0.708921in}{0.208778in}}{\pgfqpoint{3.800000in}{3.800000in}}%
\pgfusepath{clip}%
\pgfsetbuttcap%
\pgfsetroundjoin%
\definecolor{currentfill}{rgb}{0.050070,0.192203,0.290728}%
\pgfsetfillcolor{currentfill}%
\pgfsetlinewidth{0.000000pt}%
\definecolor{currentstroke}{rgb}{0.000000,0.000000,0.000000}%
\pgfsetstrokecolor{currentstroke}%
\pgfsetdash{}{0pt}%
\pgfpathmoveto{\pgfqpoint{1.452319in}{1.622564in}}%
\pgfpathlineto{\pgfqpoint{1.331189in}{1.798485in}}%
\pgfpathlineto{\pgfqpoint{1.331320in}{1.705908in}}%
\pgfpathlineto{\pgfqpoint{1.452319in}{1.622564in}}%
\pgfpathclose%
\pgfusepath{fill}%
\end{pgfscope}%
\begin{pgfscope}%
\pgfpathrectangle{\pgfqpoint{0.708921in}{0.208778in}}{\pgfqpoint{3.800000in}{3.800000in}}%
\pgfusepath{clip}%
\pgfsetbuttcap%
\pgfsetroundjoin%
\definecolor{currentfill}{rgb}{0.050070,0.192203,0.290728}%
\pgfsetfillcolor{currentfill}%
\pgfsetlinewidth{0.000000pt}%
\definecolor{currentstroke}{rgb}{0.000000,0.000000,0.000000}%
\pgfsetstrokecolor{currentstroke}%
\pgfsetdash{}{0pt}%
\pgfpathmoveto{\pgfqpoint{3.989355in}{1.798485in}}%
\pgfpathlineto{\pgfqpoint{3.868225in}{1.622564in}}%
\pgfpathlineto{\pgfqpoint{3.989224in}{1.705908in}}%
\pgfpathlineto{\pgfqpoint{3.989355in}{1.798485in}}%
\pgfpathclose%
\pgfusepath{fill}%
\end{pgfscope}%
\begin{pgfscope}%
\pgfpathrectangle{\pgfqpoint{0.708921in}{0.208778in}}{\pgfqpoint{3.800000in}{3.800000in}}%
\pgfusepath{clip}%
\pgfsetbuttcap%
\pgfsetroundjoin%
\definecolor{currentfill}{rgb}{0.090605,0.347808,0.526096}%
\pgfsetfillcolor{currentfill}%
\pgfsetlinewidth{0.000000pt}%
\definecolor{currentstroke}{rgb}{0.000000,0.000000,0.000000}%
\pgfsetstrokecolor{currentstroke}%
\pgfsetdash{}{0pt}%
\pgfpathmoveto{\pgfqpoint{2.539141in}{3.562584in}}%
\pgfpathlineto{\pgfqpoint{2.781403in}{3.562584in}}%
\pgfpathlineto{\pgfqpoint{2.660272in}{3.646458in}}%
\pgfpathlineto{\pgfqpoint{2.539141in}{3.562584in}}%
\pgfpathclose%
\pgfusepath{fill}%
\end{pgfscope}%
\begin{pgfscope}%
\pgfpathrectangle{\pgfqpoint{0.708921in}{0.208778in}}{\pgfqpoint{3.800000in}{3.800000in}}%
\pgfusepath{clip}%
\pgfsetbuttcap%
\pgfsetroundjoin%
\definecolor{currentfill}{rgb}{0.047548,0.182523,0.276086}%
\pgfsetfillcolor{currentfill}%
\pgfsetlinewidth{0.000000pt}%
\definecolor{currentstroke}{rgb}{0.000000,0.000000,0.000000}%
\pgfsetstrokecolor{currentstroke}%
\pgfsetdash{}{0pt}%
\pgfpathmoveto{\pgfqpoint{1.609260in}{1.534809in}}%
\pgfpathlineto{\pgfqpoint{1.465384in}{1.716873in}}%
\pgfpathlineto{\pgfqpoint{1.452319in}{1.622564in}}%
\pgfpathlineto{\pgfqpoint{1.609260in}{1.534809in}}%
\pgfpathclose%
\pgfusepath{fill}%
\end{pgfscope}%
\begin{pgfscope}%
\pgfpathrectangle{\pgfqpoint{0.708921in}{0.208778in}}{\pgfqpoint{3.800000in}{3.800000in}}%
\pgfusepath{clip}%
\pgfsetbuttcap%
\pgfsetroundjoin%
\definecolor{currentfill}{rgb}{0.047548,0.182523,0.276086}%
\pgfsetfillcolor{currentfill}%
\pgfsetlinewidth{0.000000pt}%
\definecolor{currentstroke}{rgb}{0.000000,0.000000,0.000000}%
\pgfsetstrokecolor{currentstroke}%
\pgfsetdash{}{0pt}%
\pgfpathmoveto{\pgfqpoint{3.868225in}{1.622564in}}%
\pgfpathlineto{\pgfqpoint{3.855160in}{1.716873in}}%
\pgfpathlineto{\pgfqpoint{3.711284in}{1.534809in}}%
\pgfpathlineto{\pgfqpoint{3.868225in}{1.622564in}}%
\pgfpathclose%
\pgfusepath{fill}%
\end{pgfscope}%
\begin{pgfscope}%
\pgfpathrectangle{\pgfqpoint{0.708921in}{0.208778in}}{\pgfqpoint{3.800000in}{3.800000in}}%
\pgfusepath{clip}%
\pgfsetbuttcap%
\pgfsetroundjoin%
\definecolor{currentfill}{rgb}{0.048960,0.187944,0.284285}%
\pgfsetfillcolor{currentfill}%
\pgfsetlinewidth{0.000000pt}%
\definecolor{currentstroke}{rgb}{0.000000,0.000000,0.000000}%
\pgfsetstrokecolor{currentstroke}%
\pgfsetdash{}{0pt}%
\pgfpathmoveto{\pgfqpoint{1.331189in}{1.798485in}}%
\pgfpathlineto{\pgfqpoint{1.452319in}{1.622564in}}%
\pgfpathlineto{\pgfqpoint{1.451224in}{2.252417in}}%
\pgfpathlineto{\pgfqpoint{1.331189in}{1.798485in}}%
\pgfpathclose%
\pgfusepath{fill}%
\end{pgfscope}%
\begin{pgfscope}%
\pgfpathrectangle{\pgfqpoint{0.708921in}{0.208778in}}{\pgfqpoint{3.800000in}{3.800000in}}%
\pgfusepath{clip}%
\pgfsetbuttcap%
\pgfsetroundjoin%
\definecolor{currentfill}{rgb}{0.048960,0.187944,0.284285}%
\pgfsetfillcolor{currentfill}%
\pgfsetlinewidth{0.000000pt}%
\definecolor{currentstroke}{rgb}{0.000000,0.000000,0.000000}%
\pgfsetstrokecolor{currentstroke}%
\pgfsetdash{}{0pt}%
\pgfpathmoveto{\pgfqpoint{3.989355in}{1.798485in}}%
\pgfpathlineto{\pgfqpoint{3.869320in}{2.252417in}}%
\pgfpathlineto{\pgfqpoint{3.868225in}{1.622564in}}%
\pgfpathlineto{\pgfqpoint{3.989355in}{1.798485in}}%
\pgfpathclose%
\pgfusepath{fill}%
\end{pgfscope}%
\begin{pgfscope}%
\pgfpathrectangle{\pgfqpoint{0.708921in}{0.208778in}}{\pgfqpoint{3.800000in}{3.800000in}}%
\pgfusepath{clip}%
\pgfsetbuttcap%
\pgfsetroundjoin%
\definecolor{currentfill}{rgb}{0.070254,0.269685,0.407928}%
\pgfsetfillcolor{currentfill}%
\pgfsetlinewidth{0.000000pt}%
\definecolor{currentstroke}{rgb}{0.000000,0.000000,0.000000}%
\pgfsetstrokecolor{currentstroke}%
\pgfsetdash{}{0pt}%
\pgfpathmoveto{\pgfqpoint{1.451224in}{2.252417in}}%
\pgfpathlineto{\pgfqpoint{1.452319in}{1.622564in}}%
\pgfpathlineto{\pgfqpoint{1.465384in}{1.716873in}}%
\pgfpathlineto{\pgfqpoint{1.451224in}{2.252417in}}%
\pgfpathclose%
\pgfusepath{fill}%
\end{pgfscope}%
\begin{pgfscope}%
\pgfpathrectangle{\pgfqpoint{0.708921in}{0.208778in}}{\pgfqpoint{3.800000in}{3.800000in}}%
\pgfusepath{clip}%
\pgfsetbuttcap%
\pgfsetroundjoin%
\definecolor{currentfill}{rgb}{0.070254,0.269685,0.407928}%
\pgfsetfillcolor{currentfill}%
\pgfsetlinewidth{0.000000pt}%
\definecolor{currentstroke}{rgb}{0.000000,0.000000,0.000000}%
\pgfsetstrokecolor{currentstroke}%
\pgfsetdash{}{0pt}%
\pgfpathmoveto{\pgfqpoint{3.855160in}{1.716873in}}%
\pgfpathlineto{\pgfqpoint{3.868225in}{1.622564in}}%
\pgfpathlineto{\pgfqpoint{3.869320in}{2.252417in}}%
\pgfpathlineto{\pgfqpoint{3.855160in}{1.716873in}}%
\pgfpathclose%
\pgfusepath{fill}%
\end{pgfscope}%
\begin{pgfscope}%
\pgfpathrectangle{\pgfqpoint{0.708921in}{0.208778in}}{\pgfqpoint{3.800000in}{3.800000in}}%
\pgfusepath{clip}%
\pgfsetbuttcap%
\pgfsetroundjoin%
\definecolor{currentfill}{rgb}{0.044978,0.172658,0.261163}%
\pgfsetfillcolor{currentfill}%
\pgfsetlinewidth{0.000000pt}%
\definecolor{currentstroke}{rgb}{0.000000,0.000000,0.000000}%
\pgfsetstrokecolor{currentstroke}%
\pgfsetdash{}{0pt}%
\pgfpathmoveto{\pgfqpoint{1.809458in}{1.448342in}}%
\pgfpathlineto{\pgfqpoint{1.643156in}{1.631177in}}%
\pgfpathlineto{\pgfqpoint{1.609260in}{1.534809in}}%
\pgfpathlineto{\pgfqpoint{1.809458in}{1.448342in}}%
\pgfpathclose%
\pgfusepath{fill}%
\end{pgfscope}%
\begin{pgfscope}%
\pgfpathrectangle{\pgfqpoint{0.708921in}{0.208778in}}{\pgfqpoint{3.800000in}{3.800000in}}%
\pgfusepath{clip}%
\pgfsetbuttcap%
\pgfsetroundjoin%
\definecolor{currentfill}{rgb}{0.044978,0.172658,0.261163}%
\pgfsetfillcolor{currentfill}%
\pgfsetlinewidth{0.000000pt}%
\definecolor{currentstroke}{rgb}{0.000000,0.000000,0.000000}%
\pgfsetstrokecolor{currentstroke}%
\pgfsetdash{}{0pt}%
\pgfpathmoveto{\pgfqpoint{3.711284in}{1.534809in}}%
\pgfpathlineto{\pgfqpoint{3.677388in}{1.631177in}}%
\pgfpathlineto{\pgfqpoint{3.511086in}{1.448342in}}%
\pgfpathlineto{\pgfqpoint{3.711284in}{1.534809in}}%
\pgfpathclose%
\pgfusepath{fill}%
\end{pgfscope}%
\begin{pgfscope}%
\pgfpathrectangle{\pgfqpoint{0.708921in}{0.208778in}}{\pgfqpoint{3.800000in}{3.800000in}}%
\pgfusepath{clip}%
\pgfsetbuttcap%
\pgfsetroundjoin%
\definecolor{currentfill}{rgb}{0.081954,0.314596,0.475860}%
\pgfsetfillcolor{currentfill}%
\pgfsetlinewidth{0.000000pt}%
\definecolor{currentstroke}{rgb}{0.000000,0.000000,0.000000}%
\pgfsetstrokecolor{currentstroke}%
\pgfsetdash{}{0pt}%
\pgfpathmoveto{\pgfqpoint{2.781403in}{3.562584in}}%
\pgfpathlineto{\pgfqpoint{2.539141in}{3.562584in}}%
\pgfpathlineto{\pgfqpoint{2.486911in}{2.956025in}}%
\pgfpathlineto{\pgfqpoint{2.781403in}{3.562584in}}%
\pgfpathclose%
\pgfusepath{fill}%
\end{pgfscope}%
\begin{pgfscope}%
\pgfpathrectangle{\pgfqpoint{0.708921in}{0.208778in}}{\pgfqpoint{3.800000in}{3.800000in}}%
\pgfusepath{clip}%
\pgfsetbuttcap%
\pgfsetroundjoin%
\definecolor{currentfill}{rgb}{0.047247,0.181368,0.274339}%
\pgfsetfillcolor{currentfill}%
\pgfsetlinewidth{0.000000pt}%
\definecolor{currentstroke}{rgb}{0.000000,0.000000,0.000000}%
\pgfsetstrokecolor{currentstroke}%
\pgfsetdash{}{0pt}%
\pgfpathmoveto{\pgfqpoint{1.465384in}{1.716873in}}%
\pgfpathlineto{\pgfqpoint{1.609260in}{1.534809in}}%
\pgfpathlineto{\pgfqpoint{1.655472in}{2.212212in}}%
\pgfpathlineto{\pgfqpoint{1.465384in}{1.716873in}}%
\pgfpathclose%
\pgfusepath{fill}%
\end{pgfscope}%
\begin{pgfscope}%
\pgfpathrectangle{\pgfqpoint{0.708921in}{0.208778in}}{\pgfqpoint{3.800000in}{3.800000in}}%
\pgfusepath{clip}%
\pgfsetbuttcap%
\pgfsetroundjoin%
\definecolor{currentfill}{rgb}{0.047247,0.181368,0.274339}%
\pgfsetfillcolor{currentfill}%
\pgfsetlinewidth{0.000000pt}%
\definecolor{currentstroke}{rgb}{0.000000,0.000000,0.000000}%
\pgfsetstrokecolor{currentstroke}%
\pgfsetdash{}{0pt}%
\pgfpathmoveto{\pgfqpoint{3.665072in}{2.212212in}}%
\pgfpathlineto{\pgfqpoint{3.711284in}{1.534809in}}%
\pgfpathlineto{\pgfqpoint{3.855160in}{1.716873in}}%
\pgfpathlineto{\pgfqpoint{3.665072in}{2.212212in}}%
\pgfpathclose%
\pgfusepath{fill}%
\end{pgfscope}%
\begin{pgfscope}%
\pgfpathrectangle{\pgfqpoint{0.708921in}{0.208778in}}{\pgfqpoint{3.800000in}{3.800000in}}%
\pgfusepath{clip}%
\pgfsetbuttcap%
\pgfsetroundjoin%
\definecolor{currentfill}{rgb}{0.067179,0.257880,0.390071}%
\pgfsetfillcolor{currentfill}%
\pgfsetlinewidth{0.000000pt}%
\definecolor{currentstroke}{rgb}{0.000000,0.000000,0.000000}%
\pgfsetstrokecolor{currentstroke}%
\pgfsetdash{}{0pt}%
\pgfpathmoveto{\pgfqpoint{1.655472in}{2.212212in}}%
\pgfpathlineto{\pgfqpoint{1.609260in}{1.534809in}}%
\pgfpathlineto{\pgfqpoint{1.643156in}{1.631177in}}%
\pgfpathlineto{\pgfqpoint{1.655472in}{2.212212in}}%
\pgfpathclose%
\pgfusepath{fill}%
\end{pgfscope}%
\begin{pgfscope}%
\pgfpathrectangle{\pgfqpoint{0.708921in}{0.208778in}}{\pgfqpoint{3.800000in}{3.800000in}}%
\pgfusepath{clip}%
\pgfsetbuttcap%
\pgfsetroundjoin%
\definecolor{currentfill}{rgb}{0.067179,0.257880,0.390071}%
\pgfsetfillcolor{currentfill}%
\pgfsetlinewidth{0.000000pt}%
\definecolor{currentstroke}{rgb}{0.000000,0.000000,0.000000}%
\pgfsetstrokecolor{currentstroke}%
\pgfsetdash{}{0pt}%
\pgfpathmoveto{\pgfqpoint{3.677388in}{1.631177in}}%
\pgfpathlineto{\pgfqpoint{3.711284in}{1.534809in}}%
\pgfpathlineto{\pgfqpoint{3.665072in}{2.212212in}}%
\pgfpathlineto{\pgfqpoint{3.677388in}{1.631177in}}%
\pgfpathclose%
\pgfusepath{fill}%
\end{pgfscope}%
\begin{pgfscope}%
\pgfpathrectangle{\pgfqpoint{0.708921in}{0.208778in}}{\pgfqpoint{3.800000in}{3.800000in}}%
\pgfusepath{clip}%
\pgfsetbuttcap%
\pgfsetroundjoin%
\definecolor{currentfill}{rgb}{0.042579,0.163449,0.247234}%
\pgfsetfillcolor{currentfill}%
\pgfsetlinewidth{0.000000pt}%
\definecolor{currentstroke}{rgb}{0.000000,0.000000,0.000000}%
\pgfsetstrokecolor{currentstroke}%
\pgfsetdash{}{0pt}%
\pgfpathmoveto{\pgfqpoint{1.809458in}{1.448342in}}%
\pgfpathlineto{\pgfqpoint{2.056510in}{1.372555in}}%
\pgfpathlineto{\pgfqpoint{1.873361in}{1.548607in}}%
\pgfpathlineto{\pgfqpoint{1.809458in}{1.448342in}}%
\pgfpathclose%
\pgfusepath{fill}%
\end{pgfscope}%
\begin{pgfscope}%
\pgfpathrectangle{\pgfqpoint{0.708921in}{0.208778in}}{\pgfqpoint{3.800000in}{3.800000in}}%
\pgfusepath{clip}%
\pgfsetbuttcap%
\pgfsetroundjoin%
\definecolor{currentfill}{rgb}{0.042579,0.163449,0.247234}%
\pgfsetfillcolor{currentfill}%
\pgfsetlinewidth{0.000000pt}%
\definecolor{currentstroke}{rgb}{0.000000,0.000000,0.000000}%
\pgfsetstrokecolor{currentstroke}%
\pgfsetdash{}{0pt}%
\pgfpathmoveto{\pgfqpoint{3.447183in}{1.548607in}}%
\pgfpathlineto{\pgfqpoint{3.264034in}{1.372555in}}%
\pgfpathlineto{\pgfqpoint{3.511086in}{1.448342in}}%
\pgfpathlineto{\pgfqpoint{3.447183in}{1.548607in}}%
\pgfpathclose%
\pgfusepath{fill}%
\end{pgfscope}%
\begin{pgfscope}%
\pgfpathrectangle{\pgfqpoint{0.708921in}{0.208778in}}{\pgfqpoint{3.800000in}{3.800000in}}%
\pgfusepath{clip}%
\pgfsetbuttcap%
\pgfsetroundjoin%
\definecolor{currentfill}{rgb}{0.052493,0.201505,0.304798}%
\pgfsetfillcolor{currentfill}%
\pgfsetlinewidth{0.000000pt}%
\definecolor{currentstroke}{rgb}{0.000000,0.000000,0.000000}%
\pgfsetstrokecolor{currentstroke}%
\pgfsetdash{}{0pt}%
\pgfpathmoveto{\pgfqpoint{1.451224in}{2.252417in}}%
\pgfpathlineto{\pgfqpoint{1.465384in}{1.716873in}}%
\pgfpathlineto{\pgfqpoint{1.655472in}{2.212212in}}%
\pgfpathlineto{\pgfqpoint{1.451224in}{2.252417in}}%
\pgfpathclose%
\pgfusepath{fill}%
\end{pgfscope}%
\begin{pgfscope}%
\pgfpathrectangle{\pgfqpoint{0.708921in}{0.208778in}}{\pgfqpoint{3.800000in}{3.800000in}}%
\pgfusepath{clip}%
\pgfsetbuttcap%
\pgfsetroundjoin%
\definecolor{currentfill}{rgb}{0.052493,0.201505,0.304798}%
\pgfsetfillcolor{currentfill}%
\pgfsetlinewidth{0.000000pt}%
\definecolor{currentstroke}{rgb}{0.000000,0.000000,0.000000}%
\pgfsetstrokecolor{currentstroke}%
\pgfsetdash{}{0pt}%
\pgfpathmoveto{\pgfqpoint{3.665072in}{2.212212in}}%
\pgfpathlineto{\pgfqpoint{3.855160in}{1.716873in}}%
\pgfpathlineto{\pgfqpoint{3.869320in}{2.252417in}}%
\pgfpathlineto{\pgfqpoint{3.665072in}{2.212212in}}%
\pgfpathclose%
\pgfusepath{fill}%
\end{pgfscope}%
\begin{pgfscope}%
\pgfpathrectangle{\pgfqpoint{0.708921in}{0.208778in}}{\pgfqpoint{3.800000in}{3.800000in}}%
\pgfusepath{clip}%
\pgfsetbuttcap%
\pgfsetroundjoin%
\definecolor{currentfill}{rgb}{0.082280,0.315849,0.477755}%
\pgfsetfillcolor{currentfill}%
\pgfsetlinewidth{0.000000pt}%
\definecolor{currentstroke}{rgb}{0.000000,0.000000,0.000000}%
\pgfsetstrokecolor{currentstroke}%
\pgfsetdash{}{0pt}%
\pgfpathmoveto{\pgfqpoint{3.265560in}{3.134503in}}%
\pgfpathlineto{\pgfqpoint{2.781403in}{3.562584in}}%
\pgfpathlineto{\pgfqpoint{2.833633in}{2.956025in}}%
\pgfpathlineto{\pgfqpoint{3.265560in}{3.134503in}}%
\pgfpathclose%
\pgfusepath{fill}%
\end{pgfscope}%
\begin{pgfscope}%
\pgfpathrectangle{\pgfqpoint{0.708921in}{0.208778in}}{\pgfqpoint{3.800000in}{3.800000in}}%
\pgfusepath{clip}%
\pgfsetbuttcap%
\pgfsetroundjoin%
\definecolor{currentfill}{rgb}{0.082280,0.315849,0.477755}%
\pgfsetfillcolor{currentfill}%
\pgfsetlinewidth{0.000000pt}%
\definecolor{currentstroke}{rgb}{0.000000,0.000000,0.000000}%
\pgfsetstrokecolor{currentstroke}%
\pgfsetdash{}{0pt}%
\pgfpathmoveto{\pgfqpoint{2.486911in}{2.956025in}}%
\pgfpathlineto{\pgfqpoint{2.539141in}{3.562584in}}%
\pgfpathlineto{\pgfqpoint{2.054984in}{3.134503in}}%
\pgfpathlineto{\pgfqpoint{2.486911in}{2.956025in}}%
\pgfpathclose%
\pgfusepath{fill}%
\end{pgfscope}%
\begin{pgfscope}%
\pgfpathrectangle{\pgfqpoint{0.708921in}{0.208778in}}{\pgfqpoint{3.800000in}{3.800000in}}%
\pgfusepath{clip}%
\pgfsetbuttcap%
\pgfsetroundjoin%
\definecolor{currentfill}{rgb}{0.045702,0.175435,0.265364}%
\pgfsetfillcolor{currentfill}%
\pgfsetlinewidth{0.000000pt}%
\definecolor{currentstroke}{rgb}{0.000000,0.000000,0.000000}%
\pgfsetstrokecolor{currentstroke}%
\pgfsetdash{}{0pt}%
\pgfpathmoveto{\pgfqpoint{1.643156in}{1.631177in}}%
\pgfpathlineto{\pgfqpoint{1.809458in}{1.448342in}}%
\pgfpathlineto{\pgfqpoint{1.928629in}{2.174201in}}%
\pgfpathlineto{\pgfqpoint{1.643156in}{1.631177in}}%
\pgfpathclose%
\pgfusepath{fill}%
\end{pgfscope}%
\begin{pgfscope}%
\pgfpathrectangle{\pgfqpoint{0.708921in}{0.208778in}}{\pgfqpoint{3.800000in}{3.800000in}}%
\pgfusepath{clip}%
\pgfsetbuttcap%
\pgfsetroundjoin%
\definecolor{currentfill}{rgb}{0.045702,0.175435,0.265364}%
\pgfsetfillcolor{currentfill}%
\pgfsetlinewidth{0.000000pt}%
\definecolor{currentstroke}{rgb}{0.000000,0.000000,0.000000}%
\pgfsetstrokecolor{currentstroke}%
\pgfsetdash{}{0pt}%
\pgfpathmoveto{\pgfqpoint{3.391915in}{2.174201in}}%
\pgfpathlineto{\pgfqpoint{3.511086in}{1.448342in}}%
\pgfpathlineto{\pgfqpoint{3.677388in}{1.631177in}}%
\pgfpathlineto{\pgfqpoint{3.391915in}{2.174201in}}%
\pgfpathclose%
\pgfusepath{fill}%
\end{pgfscope}%
\begin{pgfscope}%
\pgfpathrectangle{\pgfqpoint{0.708921in}{0.208778in}}{\pgfqpoint{3.800000in}{3.800000in}}%
\pgfusepath{clip}%
\pgfsetbuttcap%
\pgfsetroundjoin%
\definecolor{currentfill}{rgb}{0.040669,0.156116,0.236142}%
\pgfsetfillcolor{currentfill}%
\pgfsetlinewidth{0.000000pt}%
\definecolor{currentstroke}{rgb}{0.000000,0.000000,0.000000}%
\pgfsetstrokecolor{currentstroke}%
\pgfsetdash{}{0pt}%
\pgfpathmoveto{\pgfqpoint{2.158398in}{1.480874in}}%
\pgfpathlineto{\pgfqpoint{2.056510in}{1.372555in}}%
\pgfpathlineto{\pgfqpoint{2.345611in}{1.319560in}}%
\pgfpathlineto{\pgfqpoint{2.158398in}{1.480874in}}%
\pgfpathclose%
\pgfusepath{fill}%
\end{pgfscope}%
\begin{pgfscope}%
\pgfpathrectangle{\pgfqpoint{0.708921in}{0.208778in}}{\pgfqpoint{3.800000in}{3.800000in}}%
\pgfusepath{clip}%
\pgfsetbuttcap%
\pgfsetroundjoin%
\definecolor{currentfill}{rgb}{0.040669,0.156116,0.236142}%
\pgfsetfillcolor{currentfill}%
\pgfsetlinewidth{0.000000pt}%
\definecolor{currentstroke}{rgb}{0.000000,0.000000,0.000000}%
\pgfsetstrokecolor{currentstroke}%
\pgfsetdash{}{0pt}%
\pgfpathmoveto{\pgfqpoint{2.974933in}{1.319560in}}%
\pgfpathlineto{\pgfqpoint{3.264034in}{1.372555in}}%
\pgfpathlineto{\pgfqpoint{3.162146in}{1.480874in}}%
\pgfpathlineto{\pgfqpoint{2.974933in}{1.319560in}}%
\pgfpathclose%
\pgfusepath{fill}%
\end{pgfscope}%
\begin{pgfscope}%
\pgfpathrectangle{\pgfqpoint{0.708921in}{0.208778in}}{\pgfqpoint{3.800000in}{3.800000in}}%
\pgfusepath{clip}%
\pgfsetbuttcap%
\pgfsetroundjoin%
\definecolor{currentfill}{rgb}{0.063981,0.245604,0.371502}%
\pgfsetfillcolor{currentfill}%
\pgfsetlinewidth{0.000000pt}%
\definecolor{currentstroke}{rgb}{0.000000,0.000000,0.000000}%
\pgfsetstrokecolor{currentstroke}%
\pgfsetdash{}{0pt}%
\pgfpathmoveto{\pgfqpoint{1.928629in}{2.174201in}}%
\pgfpathlineto{\pgfqpoint{1.809458in}{1.448342in}}%
\pgfpathlineto{\pgfqpoint{1.873361in}{1.548607in}}%
\pgfpathlineto{\pgfqpoint{1.928629in}{2.174201in}}%
\pgfpathclose%
\pgfusepath{fill}%
\end{pgfscope}%
\begin{pgfscope}%
\pgfpathrectangle{\pgfqpoint{0.708921in}{0.208778in}}{\pgfqpoint{3.800000in}{3.800000in}}%
\pgfusepath{clip}%
\pgfsetbuttcap%
\pgfsetroundjoin%
\definecolor{currentfill}{rgb}{0.063981,0.245604,0.371502}%
\pgfsetfillcolor{currentfill}%
\pgfsetlinewidth{0.000000pt}%
\definecolor{currentstroke}{rgb}{0.000000,0.000000,0.000000}%
\pgfsetstrokecolor{currentstroke}%
\pgfsetdash{}{0pt}%
\pgfpathmoveto{\pgfqpoint{3.447183in}{1.548607in}}%
\pgfpathlineto{\pgfqpoint{3.511086in}{1.448342in}}%
\pgfpathlineto{\pgfqpoint{3.391915in}{2.174201in}}%
\pgfpathlineto{\pgfqpoint{3.447183in}{1.548607in}}%
\pgfpathclose%
\pgfusepath{fill}%
\end{pgfscope}%
\begin{pgfscope}%
\pgfpathrectangle{\pgfqpoint{0.708921in}{0.208778in}}{\pgfqpoint{3.800000in}{3.800000in}}%
\pgfusepath{clip}%
\pgfsetbuttcap%
\pgfsetroundjoin%
\definecolor{currentfill}{rgb}{0.060942,0.233938,0.353856}%
\pgfsetfillcolor{currentfill}%
\pgfsetlinewidth{0.000000pt}%
\definecolor{currentstroke}{rgb}{0.000000,0.000000,0.000000}%
\pgfsetstrokecolor{currentstroke}%
\pgfsetdash{}{0pt}%
\pgfpathmoveto{\pgfqpoint{1.655472in}{2.212212in}}%
\pgfpathlineto{\pgfqpoint{1.557147in}{2.466299in}}%
\pgfpathlineto{\pgfqpoint{1.451224in}{2.252417in}}%
\pgfpathlineto{\pgfqpoint{1.655472in}{2.212212in}}%
\pgfpathclose%
\pgfusepath{fill}%
\end{pgfscope}%
\begin{pgfscope}%
\pgfpathrectangle{\pgfqpoint{0.708921in}{0.208778in}}{\pgfqpoint{3.800000in}{3.800000in}}%
\pgfusepath{clip}%
\pgfsetbuttcap%
\pgfsetroundjoin%
\definecolor{currentfill}{rgb}{0.060942,0.233938,0.353856}%
\pgfsetfillcolor{currentfill}%
\pgfsetlinewidth{0.000000pt}%
\definecolor{currentstroke}{rgb}{0.000000,0.000000,0.000000}%
\pgfsetstrokecolor{currentstroke}%
\pgfsetdash{}{0pt}%
\pgfpathmoveto{\pgfqpoint{3.869320in}{2.252417in}}%
\pgfpathlineto{\pgfqpoint{3.763397in}{2.466299in}}%
\pgfpathlineto{\pgfqpoint{3.665072in}{2.212212in}}%
\pgfpathlineto{\pgfqpoint{3.869320in}{2.252417in}}%
\pgfpathclose%
\pgfusepath{fill}%
\end{pgfscope}%
\begin{pgfscope}%
\pgfpathrectangle{\pgfqpoint{0.708921in}{0.208778in}}{\pgfqpoint{3.800000in}{3.800000in}}%
\pgfusepath{clip}%
\pgfsetbuttcap%
\pgfsetroundjoin%
\definecolor{currentfill}{rgb}{0.081954,0.314596,0.475860}%
\pgfsetfillcolor{currentfill}%
\pgfsetlinewidth{0.000000pt}%
\definecolor{currentstroke}{rgb}{0.000000,0.000000,0.000000}%
\pgfsetstrokecolor{currentstroke}%
\pgfsetdash{}{0pt}%
\pgfpathmoveto{\pgfqpoint{2.486911in}{2.956025in}}%
\pgfpathlineto{\pgfqpoint{2.833633in}{2.956025in}}%
\pgfpathlineto{\pgfqpoint{2.781403in}{3.562584in}}%
\pgfpathlineto{\pgfqpoint{2.486911in}{2.956025in}}%
\pgfpathclose%
\pgfusepath{fill}%
\end{pgfscope}%
\begin{pgfscope}%
\pgfpathrectangle{\pgfqpoint{0.708921in}{0.208778in}}{\pgfqpoint{3.800000in}{3.800000in}}%
\pgfusepath{clip}%
\pgfsetbuttcap%
\pgfsetroundjoin%
\definecolor{currentfill}{rgb}{0.039595,0.151995,0.229908}%
\pgfsetfillcolor{currentfill}%
\pgfsetlinewidth{0.000000pt}%
\definecolor{currentstroke}{rgb}{0.000000,0.000000,0.000000}%
\pgfsetstrokecolor{currentstroke}%
\pgfsetdash{}{0pt}%
\pgfpathmoveto{\pgfqpoint{2.974933in}{1.319560in}}%
\pgfpathlineto{\pgfqpoint{2.833257in}{1.441955in}}%
\pgfpathlineto{\pgfqpoint{2.660272in}{1.300375in}}%
\pgfpathlineto{\pgfqpoint{2.974933in}{1.319560in}}%
\pgfpathclose%
\pgfusepath{fill}%
\end{pgfscope}%
\begin{pgfscope}%
\pgfpathrectangle{\pgfqpoint{0.708921in}{0.208778in}}{\pgfqpoint{3.800000in}{3.800000in}}%
\pgfusepath{clip}%
\pgfsetbuttcap%
\pgfsetroundjoin%
\definecolor{currentfill}{rgb}{0.039595,0.151995,0.229908}%
\pgfsetfillcolor{currentfill}%
\pgfsetlinewidth{0.000000pt}%
\definecolor{currentstroke}{rgb}{0.000000,0.000000,0.000000}%
\pgfsetstrokecolor{currentstroke}%
\pgfsetdash{}{0pt}%
\pgfpathmoveto{\pgfqpoint{2.660272in}{1.300375in}}%
\pgfpathlineto{\pgfqpoint{2.487287in}{1.441955in}}%
\pgfpathlineto{\pgfqpoint{2.345611in}{1.319560in}}%
\pgfpathlineto{\pgfqpoint{2.660272in}{1.300375in}}%
\pgfpathclose%
\pgfusepath{fill}%
\end{pgfscope}%
\begin{pgfscope}%
\pgfpathrectangle{\pgfqpoint{0.708921in}{0.208778in}}{\pgfqpoint{3.800000in}{3.800000in}}%
\pgfusepath{clip}%
\pgfsetbuttcap%
\pgfsetroundjoin%
\definecolor{currentfill}{rgb}{0.075436,0.289576,0.438014}%
\pgfsetfillcolor{currentfill}%
\pgfsetlinewidth{0.000000pt}%
\definecolor{currentstroke}{rgb}{0.000000,0.000000,0.000000}%
\pgfsetstrokecolor{currentstroke}%
\pgfsetdash{}{0pt}%
\pgfpathmoveto{\pgfqpoint{2.157344in}{2.945086in}}%
\pgfpathlineto{\pgfqpoint{2.054984in}{3.134503in}}%
\pgfpathlineto{\pgfqpoint{1.871807in}{2.926052in}}%
\pgfpathlineto{\pgfqpoint{2.157344in}{2.945086in}}%
\pgfpathclose%
\pgfusepath{fill}%
\end{pgfscope}%
\begin{pgfscope}%
\pgfpathrectangle{\pgfqpoint{0.708921in}{0.208778in}}{\pgfqpoint{3.800000in}{3.800000in}}%
\pgfusepath{clip}%
\pgfsetbuttcap%
\pgfsetroundjoin%
\definecolor{currentfill}{rgb}{0.075436,0.289576,0.438014}%
\pgfsetfillcolor{currentfill}%
\pgfsetlinewidth{0.000000pt}%
\definecolor{currentstroke}{rgb}{0.000000,0.000000,0.000000}%
\pgfsetstrokecolor{currentstroke}%
\pgfsetdash{}{0pt}%
\pgfpathmoveto{\pgfqpoint{3.448737in}{2.926052in}}%
\pgfpathlineto{\pgfqpoint{3.265560in}{3.134503in}}%
\pgfpathlineto{\pgfqpoint{3.163200in}{2.945086in}}%
\pgfpathlineto{\pgfqpoint{3.448737in}{2.926052in}}%
\pgfpathclose%
\pgfusepath{fill}%
\end{pgfscope}%
\begin{pgfscope}%
\pgfpathrectangle{\pgfqpoint{0.708921in}{0.208778in}}{\pgfqpoint{3.800000in}{3.800000in}}%
\pgfusepath{clip}%
\pgfsetbuttcap%
\pgfsetroundjoin%
\definecolor{currentfill}{rgb}{0.062760,0.240916,0.364410}%
\pgfsetfillcolor{currentfill}%
\pgfsetlinewidth{0.000000pt}%
\definecolor{currentstroke}{rgb}{0.000000,0.000000,0.000000}%
\pgfsetstrokecolor{currentstroke}%
\pgfsetdash{}{0pt}%
\pgfpathmoveto{\pgfqpoint{1.655472in}{2.212212in}}%
\pgfpathlineto{\pgfqpoint{1.871807in}{2.926052in}}%
\pgfpathlineto{\pgfqpoint{1.557147in}{2.466299in}}%
\pgfpathlineto{\pgfqpoint{1.655472in}{2.212212in}}%
\pgfpathclose%
\pgfusepath{fill}%
\end{pgfscope}%
\begin{pgfscope}%
\pgfpathrectangle{\pgfqpoint{0.708921in}{0.208778in}}{\pgfqpoint{3.800000in}{3.800000in}}%
\pgfusepath{clip}%
\pgfsetbuttcap%
\pgfsetroundjoin%
\definecolor{currentfill}{rgb}{0.062760,0.240916,0.364410}%
\pgfsetfillcolor{currentfill}%
\pgfsetlinewidth{0.000000pt}%
\definecolor{currentstroke}{rgb}{0.000000,0.000000,0.000000}%
\pgfsetstrokecolor{currentstroke}%
\pgfsetdash{}{0pt}%
\pgfpathmoveto{\pgfqpoint{3.763397in}{2.466299in}}%
\pgfpathlineto{\pgfqpoint{3.448737in}{2.926052in}}%
\pgfpathlineto{\pgfqpoint{3.665072in}{2.212212in}}%
\pgfpathlineto{\pgfqpoint{3.763397in}{2.466299in}}%
\pgfpathclose%
\pgfusepath{fill}%
\end{pgfscope}%
\begin{pgfscope}%
\pgfpathrectangle{\pgfqpoint{0.708921in}{0.208778in}}{\pgfqpoint{3.800000in}{3.800000in}}%
\pgfusepath{clip}%
\pgfsetbuttcap%
\pgfsetroundjoin%
\definecolor{currentfill}{rgb}{0.043508,0.167016,0.252629}%
\pgfsetfillcolor{currentfill}%
\pgfsetlinewidth{0.000000pt}%
\definecolor{currentstroke}{rgb}{0.000000,0.000000,0.000000}%
\pgfsetstrokecolor{currentstroke}%
\pgfsetdash{}{0pt}%
\pgfpathmoveto{\pgfqpoint{1.873361in}{1.548607in}}%
\pgfpathlineto{\pgfqpoint{2.056510in}{1.372555in}}%
\pgfpathlineto{\pgfqpoint{2.100897in}{1.893599in}}%
\pgfpathlineto{\pgfqpoint{1.873361in}{1.548607in}}%
\pgfpathclose%
\pgfusepath{fill}%
\end{pgfscope}%
\begin{pgfscope}%
\pgfpathrectangle{\pgfqpoint{0.708921in}{0.208778in}}{\pgfqpoint{3.800000in}{3.800000in}}%
\pgfusepath{clip}%
\pgfsetbuttcap%
\pgfsetroundjoin%
\definecolor{currentfill}{rgb}{0.043508,0.167016,0.252629}%
\pgfsetfillcolor{currentfill}%
\pgfsetlinewidth{0.000000pt}%
\definecolor{currentstroke}{rgb}{0.000000,0.000000,0.000000}%
\pgfsetstrokecolor{currentstroke}%
\pgfsetdash{}{0pt}%
\pgfpathmoveto{\pgfqpoint{3.219647in}{1.893599in}}%
\pgfpathlineto{\pgfqpoint{3.264034in}{1.372555in}}%
\pgfpathlineto{\pgfqpoint{3.447183in}{1.548607in}}%
\pgfpathlineto{\pgfqpoint{3.219647in}{1.893599in}}%
\pgfpathclose%
\pgfusepath{fill}%
\end{pgfscope}%
\begin{pgfscope}%
\pgfpathrectangle{\pgfqpoint{0.708921in}{0.208778in}}{\pgfqpoint{3.800000in}{3.800000in}}%
\pgfusepath{clip}%
\pgfsetbuttcap%
\pgfsetroundjoin%
\definecolor{currentfill}{rgb}{0.050011,0.191979,0.290388}%
\pgfsetfillcolor{currentfill}%
\pgfsetlinewidth{0.000000pt}%
\definecolor{currentstroke}{rgb}{0.000000,0.000000,0.000000}%
\pgfsetstrokecolor{currentstroke}%
\pgfsetdash{}{0pt}%
\pgfpathmoveto{\pgfqpoint{1.655472in}{2.212212in}}%
\pgfpathlineto{\pgfqpoint{1.643156in}{1.631177in}}%
\pgfpathlineto{\pgfqpoint{1.928629in}{2.174201in}}%
\pgfpathlineto{\pgfqpoint{1.655472in}{2.212212in}}%
\pgfpathclose%
\pgfusepath{fill}%
\end{pgfscope}%
\begin{pgfscope}%
\pgfpathrectangle{\pgfqpoint{0.708921in}{0.208778in}}{\pgfqpoint{3.800000in}{3.800000in}}%
\pgfusepath{clip}%
\pgfsetbuttcap%
\pgfsetroundjoin%
\definecolor{currentfill}{rgb}{0.050011,0.191979,0.290388}%
\pgfsetfillcolor{currentfill}%
\pgfsetlinewidth{0.000000pt}%
\definecolor{currentstroke}{rgb}{0.000000,0.000000,0.000000}%
\pgfsetstrokecolor{currentstroke}%
\pgfsetdash{}{0pt}%
\pgfpathmoveto{\pgfqpoint{3.391915in}{2.174201in}}%
\pgfpathlineto{\pgfqpoint{3.677388in}{1.631177in}}%
\pgfpathlineto{\pgfqpoint{3.665072in}{2.212212in}}%
\pgfpathlineto{\pgfqpoint{3.391915in}{2.174201in}}%
\pgfpathclose%
\pgfusepath{fill}%
\end{pgfscope}%
\begin{pgfscope}%
\pgfpathrectangle{\pgfqpoint{0.708921in}{0.208778in}}{\pgfqpoint{3.800000in}{3.800000in}}%
\pgfusepath{clip}%
\pgfsetbuttcap%
\pgfsetroundjoin%
\definecolor{currentfill}{rgb}{0.049941,0.191710,0.289982}%
\pgfsetfillcolor{currentfill}%
\pgfsetlinewidth{0.000000pt}%
\definecolor{currentstroke}{rgb}{0.000000,0.000000,0.000000}%
\pgfsetstrokecolor{currentstroke}%
\pgfsetdash{}{0pt}%
\pgfpathmoveto{\pgfqpoint{2.100897in}{1.893599in}}%
\pgfpathlineto{\pgfqpoint{2.056510in}{1.372555in}}%
\pgfpathlineto{\pgfqpoint{2.158398in}{1.480874in}}%
\pgfpathlineto{\pgfqpoint{2.100897in}{1.893599in}}%
\pgfpathclose%
\pgfusepath{fill}%
\end{pgfscope}%
\begin{pgfscope}%
\pgfpathrectangle{\pgfqpoint{0.708921in}{0.208778in}}{\pgfqpoint{3.800000in}{3.800000in}}%
\pgfusepath{clip}%
\pgfsetbuttcap%
\pgfsetroundjoin%
\definecolor{currentfill}{rgb}{0.049941,0.191710,0.289982}%
\pgfsetfillcolor{currentfill}%
\pgfsetlinewidth{0.000000pt}%
\definecolor{currentstroke}{rgb}{0.000000,0.000000,0.000000}%
\pgfsetstrokecolor{currentstroke}%
\pgfsetdash{}{0pt}%
\pgfpathmoveto{\pgfqpoint{3.162146in}{1.480874in}}%
\pgfpathlineto{\pgfqpoint{3.264034in}{1.372555in}}%
\pgfpathlineto{\pgfqpoint{3.219647in}{1.893599in}}%
\pgfpathlineto{\pgfqpoint{3.162146in}{1.480874in}}%
\pgfpathclose%
\pgfusepath{fill}%
\end{pgfscope}%
\begin{pgfscope}%
\pgfpathrectangle{\pgfqpoint{0.708921in}{0.208778in}}{\pgfqpoint{3.800000in}{3.800000in}}%
\pgfusepath{clip}%
\pgfsetbuttcap%
\pgfsetroundjoin%
\definecolor{currentfill}{rgb}{0.078663,0.301965,0.456754}%
\pgfsetfillcolor{currentfill}%
\pgfsetlinewidth{0.000000pt}%
\definecolor{currentstroke}{rgb}{0.000000,0.000000,0.000000}%
\pgfsetstrokecolor{currentstroke}%
\pgfsetdash{}{0pt}%
\pgfpathmoveto{\pgfqpoint{3.163200in}{2.945086in}}%
\pgfpathlineto{\pgfqpoint{3.265560in}{3.134503in}}%
\pgfpathlineto{\pgfqpoint{2.833633in}{2.956025in}}%
\pgfpathlineto{\pgfqpoint{3.163200in}{2.945086in}}%
\pgfpathclose%
\pgfusepath{fill}%
\end{pgfscope}%
\begin{pgfscope}%
\pgfpathrectangle{\pgfqpoint{0.708921in}{0.208778in}}{\pgfqpoint{3.800000in}{3.800000in}}%
\pgfusepath{clip}%
\pgfsetbuttcap%
\pgfsetroundjoin%
\definecolor{currentfill}{rgb}{0.078663,0.301965,0.456754}%
\pgfsetfillcolor{currentfill}%
\pgfsetlinewidth{0.000000pt}%
\definecolor{currentstroke}{rgb}{0.000000,0.000000,0.000000}%
\pgfsetstrokecolor{currentstroke}%
\pgfsetdash{}{0pt}%
\pgfpathmoveto{\pgfqpoint{2.486911in}{2.956025in}}%
\pgfpathlineto{\pgfqpoint{2.054984in}{3.134503in}}%
\pgfpathlineto{\pgfqpoint{2.157344in}{2.945086in}}%
\pgfpathlineto{\pgfqpoint{2.486911in}{2.956025in}}%
\pgfpathclose%
\pgfusepath{fill}%
\end{pgfscope}%
\begin{pgfscope}%
\pgfpathrectangle{\pgfqpoint{0.708921in}{0.208778in}}{\pgfqpoint{3.800000in}{3.800000in}}%
\pgfusepath{clip}%
\pgfsetbuttcap%
\pgfsetroundjoin%
\definecolor{currentfill}{rgb}{0.064954,0.249341,0.377155}%
\pgfsetfillcolor{currentfill}%
\pgfsetlinewidth{0.000000pt}%
\definecolor{currentstroke}{rgb}{0.000000,0.000000,0.000000}%
\pgfsetstrokecolor{currentstroke}%
\pgfsetdash{}{0pt}%
\pgfpathmoveto{\pgfqpoint{1.928629in}{2.174201in}}%
\pgfpathlineto{\pgfqpoint{1.871807in}{2.926052in}}%
\pgfpathlineto{\pgfqpoint{1.655472in}{2.212212in}}%
\pgfpathlineto{\pgfqpoint{1.928629in}{2.174201in}}%
\pgfpathclose%
\pgfusepath{fill}%
\end{pgfscope}%
\begin{pgfscope}%
\pgfpathrectangle{\pgfqpoint{0.708921in}{0.208778in}}{\pgfqpoint{3.800000in}{3.800000in}}%
\pgfusepath{clip}%
\pgfsetbuttcap%
\pgfsetroundjoin%
\definecolor{currentfill}{rgb}{0.064954,0.249341,0.377155}%
\pgfsetfillcolor{currentfill}%
\pgfsetlinewidth{0.000000pt}%
\definecolor{currentstroke}{rgb}{0.000000,0.000000,0.000000}%
\pgfsetstrokecolor{currentstroke}%
\pgfsetdash{}{0pt}%
\pgfpathmoveto{\pgfqpoint{3.665072in}{2.212212in}}%
\pgfpathlineto{\pgfqpoint{3.448737in}{2.926052in}}%
\pgfpathlineto{\pgfqpoint{3.391915in}{2.174201in}}%
\pgfpathlineto{\pgfqpoint{3.665072in}{2.212212in}}%
\pgfpathclose%
\pgfusepath{fill}%
\end{pgfscope}%
\begin{pgfscope}%
\pgfpathrectangle{\pgfqpoint{0.708921in}{0.208778in}}{\pgfqpoint{3.800000in}{3.800000in}}%
\pgfusepath{clip}%
\pgfsetbuttcap%
\pgfsetroundjoin%
\definecolor{currentfill}{rgb}{0.042669,0.163794,0.247755}%
\pgfsetfillcolor{currentfill}%
\pgfsetlinewidth{0.000000pt}%
\definecolor{currentstroke}{rgb}{0.000000,0.000000,0.000000}%
\pgfsetstrokecolor{currentstroke}%
\pgfsetdash{}{0pt}%
\pgfpathmoveto{\pgfqpoint{3.162146in}{1.480874in}}%
\pgfpathlineto{\pgfqpoint{2.854641in}{1.862594in}}%
\pgfpathlineto{\pgfqpoint{2.974933in}{1.319560in}}%
\pgfpathlineto{\pgfqpoint{3.162146in}{1.480874in}}%
\pgfpathclose%
\pgfusepath{fill}%
\end{pgfscope}%
\begin{pgfscope}%
\pgfpathrectangle{\pgfqpoint{0.708921in}{0.208778in}}{\pgfqpoint{3.800000in}{3.800000in}}%
\pgfusepath{clip}%
\pgfsetbuttcap%
\pgfsetroundjoin%
\definecolor{currentfill}{rgb}{0.042669,0.163794,0.247755}%
\pgfsetfillcolor{currentfill}%
\pgfsetlinewidth{0.000000pt}%
\definecolor{currentstroke}{rgb}{0.000000,0.000000,0.000000}%
\pgfsetstrokecolor{currentstroke}%
\pgfsetdash{}{0pt}%
\pgfpathmoveto{\pgfqpoint{2.345611in}{1.319560in}}%
\pgfpathlineto{\pgfqpoint{2.465903in}{1.862594in}}%
\pgfpathlineto{\pgfqpoint{2.158398in}{1.480874in}}%
\pgfpathlineto{\pgfqpoint{2.345611in}{1.319560in}}%
\pgfpathclose%
\pgfusepath{fill}%
\end{pgfscope}%
\begin{pgfscope}%
\pgfpathrectangle{\pgfqpoint{0.708921in}{0.208778in}}{\pgfqpoint{3.800000in}{3.800000in}}%
\pgfusepath{clip}%
\pgfsetbuttcap%
\pgfsetroundjoin%
\definecolor{currentfill}{rgb}{0.068541,0.263111,0.397982}%
\pgfsetfillcolor{currentfill}%
\pgfsetlinewidth{0.000000pt}%
\definecolor{currentstroke}{rgb}{0.000000,0.000000,0.000000}%
\pgfsetstrokecolor{currentstroke}%
\pgfsetdash{}{0pt}%
\pgfpathmoveto{\pgfqpoint{2.157344in}{2.945086in}}%
\pgfpathlineto{\pgfqpoint{1.871807in}{2.926052in}}%
\pgfpathlineto{\pgfqpoint{2.294171in}{2.707660in}}%
\pgfpathlineto{\pgfqpoint{2.157344in}{2.945086in}}%
\pgfpathclose%
\pgfusepath{fill}%
\end{pgfscope}%
\begin{pgfscope}%
\pgfpathrectangle{\pgfqpoint{0.708921in}{0.208778in}}{\pgfqpoint{3.800000in}{3.800000in}}%
\pgfusepath{clip}%
\pgfsetbuttcap%
\pgfsetroundjoin%
\definecolor{currentfill}{rgb}{0.068541,0.263111,0.397982}%
\pgfsetfillcolor{currentfill}%
\pgfsetlinewidth{0.000000pt}%
\definecolor{currentstroke}{rgb}{0.000000,0.000000,0.000000}%
\pgfsetstrokecolor{currentstroke}%
\pgfsetdash{}{0pt}%
\pgfpathmoveto{\pgfqpoint{3.026373in}{2.707660in}}%
\pgfpathlineto{\pgfqpoint{3.448737in}{2.926052in}}%
\pgfpathlineto{\pgfqpoint{3.163200in}{2.945086in}}%
\pgfpathlineto{\pgfqpoint{3.026373in}{2.707660in}}%
\pgfpathclose%
\pgfusepath{fill}%
\end{pgfscope}%
\begin{pgfscope}%
\pgfpathrectangle{\pgfqpoint{0.708921in}{0.208778in}}{\pgfqpoint{3.800000in}{3.800000in}}%
\pgfusepath{clip}%
\pgfsetbuttcap%
\pgfsetroundjoin%
\definecolor{currentfill}{rgb}{0.047555,0.182548,0.276123}%
\pgfsetfillcolor{currentfill}%
\pgfsetlinewidth{0.000000pt}%
\definecolor{currentstroke}{rgb}{0.000000,0.000000,0.000000}%
\pgfsetstrokecolor{currentstroke}%
\pgfsetdash{}{0pt}%
\pgfpathmoveto{\pgfqpoint{2.345611in}{1.319560in}}%
\pgfpathlineto{\pgfqpoint{2.487287in}{1.441955in}}%
\pgfpathlineto{\pgfqpoint{2.465903in}{1.862594in}}%
\pgfpathlineto{\pgfqpoint{2.345611in}{1.319560in}}%
\pgfpathclose%
\pgfusepath{fill}%
\end{pgfscope}%
\begin{pgfscope}%
\pgfpathrectangle{\pgfqpoint{0.708921in}{0.208778in}}{\pgfqpoint{3.800000in}{3.800000in}}%
\pgfusepath{clip}%
\pgfsetbuttcap%
\pgfsetroundjoin%
\definecolor{currentfill}{rgb}{0.047555,0.182548,0.276123}%
\pgfsetfillcolor{currentfill}%
\pgfsetlinewidth{0.000000pt}%
\definecolor{currentstroke}{rgb}{0.000000,0.000000,0.000000}%
\pgfsetstrokecolor{currentstroke}%
\pgfsetdash{}{0pt}%
\pgfpathmoveto{\pgfqpoint{2.854641in}{1.862594in}}%
\pgfpathlineto{\pgfqpoint{2.833257in}{1.441955in}}%
\pgfpathlineto{\pgfqpoint{2.974933in}{1.319560in}}%
\pgfpathlineto{\pgfqpoint{2.854641in}{1.862594in}}%
\pgfpathclose%
\pgfusepath{fill}%
\end{pgfscope}%
\begin{pgfscope}%
\pgfpathrectangle{\pgfqpoint{0.708921in}{0.208778in}}{\pgfqpoint{3.800000in}{3.800000in}}%
\pgfusepath{clip}%
\pgfsetbuttcap%
\pgfsetroundjoin%
\definecolor{currentfill}{rgb}{0.046101,0.176968,0.267683}%
\pgfsetfillcolor{currentfill}%
\pgfsetlinewidth{0.000000pt}%
\definecolor{currentstroke}{rgb}{0.000000,0.000000,0.000000}%
\pgfsetstrokecolor{currentstroke}%
\pgfsetdash{}{0pt}%
\pgfpathmoveto{\pgfqpoint{2.660272in}{1.300375in}}%
\pgfpathlineto{\pgfqpoint{2.465903in}{1.862594in}}%
\pgfpathlineto{\pgfqpoint{2.487287in}{1.441955in}}%
\pgfpathlineto{\pgfqpoint{2.660272in}{1.300375in}}%
\pgfpathclose%
\pgfusepath{fill}%
\end{pgfscope}%
\begin{pgfscope}%
\pgfpathrectangle{\pgfqpoint{0.708921in}{0.208778in}}{\pgfqpoint{3.800000in}{3.800000in}}%
\pgfusepath{clip}%
\pgfsetbuttcap%
\pgfsetroundjoin%
\definecolor{currentfill}{rgb}{0.046101,0.176968,0.267683}%
\pgfsetfillcolor{currentfill}%
\pgfsetlinewidth{0.000000pt}%
\definecolor{currentstroke}{rgb}{0.000000,0.000000,0.000000}%
\pgfsetstrokecolor{currentstroke}%
\pgfsetdash{}{0pt}%
\pgfpathmoveto{\pgfqpoint{2.833257in}{1.441955in}}%
\pgfpathlineto{\pgfqpoint{2.854641in}{1.862594in}}%
\pgfpathlineto{\pgfqpoint{2.660272in}{1.300375in}}%
\pgfpathlineto{\pgfqpoint{2.833257in}{1.441955in}}%
\pgfpathclose%
\pgfusepath{fill}%
\end{pgfscope}%
\begin{pgfscope}%
\pgfpathrectangle{\pgfqpoint{0.708921in}{0.208778in}}{\pgfqpoint{3.800000in}{3.800000in}}%
\pgfusepath{clip}%
\pgfsetbuttcap%
\pgfsetroundjoin%
\definecolor{currentfill}{rgb}{0.051850,0.199036,0.301063}%
\pgfsetfillcolor{currentfill}%
\pgfsetlinewidth{0.000000pt}%
\definecolor{currentstroke}{rgb}{0.000000,0.000000,0.000000}%
\pgfsetstrokecolor{currentstroke}%
\pgfsetdash{}{0pt}%
\pgfpathmoveto{\pgfqpoint{1.873361in}{1.548607in}}%
\pgfpathlineto{\pgfqpoint{2.100897in}{1.893599in}}%
\pgfpathlineto{\pgfqpoint{1.928629in}{2.174201in}}%
\pgfpathlineto{\pgfqpoint{1.873361in}{1.548607in}}%
\pgfpathclose%
\pgfusepath{fill}%
\end{pgfscope}%
\begin{pgfscope}%
\pgfpathrectangle{\pgfqpoint{0.708921in}{0.208778in}}{\pgfqpoint{3.800000in}{3.800000in}}%
\pgfusepath{clip}%
\pgfsetbuttcap%
\pgfsetroundjoin%
\definecolor{currentfill}{rgb}{0.051850,0.199036,0.301063}%
\pgfsetfillcolor{currentfill}%
\pgfsetlinewidth{0.000000pt}%
\definecolor{currentstroke}{rgb}{0.000000,0.000000,0.000000}%
\pgfsetstrokecolor{currentstroke}%
\pgfsetdash{}{0pt}%
\pgfpathmoveto{\pgfqpoint{3.391915in}{2.174201in}}%
\pgfpathlineto{\pgfqpoint{3.219647in}{1.893599in}}%
\pgfpathlineto{\pgfqpoint{3.447183in}{1.548607in}}%
\pgfpathlineto{\pgfqpoint{3.391915in}{2.174201in}}%
\pgfpathclose%
\pgfusepath{fill}%
\end{pgfscope}%
\begin{pgfscope}%
\pgfpathrectangle{\pgfqpoint{0.708921in}{0.208778in}}{\pgfqpoint{3.800000in}{3.800000in}}%
\pgfusepath{clip}%
\pgfsetbuttcap%
\pgfsetroundjoin%
\definecolor{currentfill}{rgb}{0.064759,0.248590,0.376018}%
\pgfsetfillcolor{currentfill}%
\pgfsetlinewidth{0.000000pt}%
\definecolor{currentstroke}{rgb}{0.000000,0.000000,0.000000}%
\pgfsetstrokecolor{currentstroke}%
\pgfsetdash{}{0pt}%
\pgfpathmoveto{\pgfqpoint{1.928629in}{2.174201in}}%
\pgfpathlineto{\pgfqpoint{2.294171in}{2.707660in}}%
\pgfpathlineto{\pgfqpoint{1.871807in}{2.926052in}}%
\pgfpathlineto{\pgfqpoint{1.928629in}{2.174201in}}%
\pgfpathclose%
\pgfusepath{fill}%
\end{pgfscope}%
\begin{pgfscope}%
\pgfpathrectangle{\pgfqpoint{0.708921in}{0.208778in}}{\pgfqpoint{3.800000in}{3.800000in}}%
\pgfusepath{clip}%
\pgfsetbuttcap%
\pgfsetroundjoin%
\definecolor{currentfill}{rgb}{0.064759,0.248590,0.376018}%
\pgfsetfillcolor{currentfill}%
\pgfsetlinewidth{0.000000pt}%
\definecolor{currentstroke}{rgb}{0.000000,0.000000,0.000000}%
\pgfsetstrokecolor{currentstroke}%
\pgfsetdash{}{0pt}%
\pgfpathmoveto{\pgfqpoint{3.448737in}{2.926052in}}%
\pgfpathlineto{\pgfqpoint{3.026373in}{2.707660in}}%
\pgfpathlineto{\pgfqpoint{3.391915in}{2.174201in}}%
\pgfpathlineto{\pgfqpoint{3.448737in}{2.926052in}}%
\pgfpathclose%
\pgfusepath{fill}%
\end{pgfscope}%
\begin{pgfscope}%
\pgfpathrectangle{\pgfqpoint{0.708921in}{0.208778in}}{\pgfqpoint{3.800000in}{3.800000in}}%
\pgfusepath{clip}%
\pgfsetbuttcap%
\pgfsetroundjoin%
\definecolor{currentfill}{rgb}{0.071694,0.275212,0.416288}%
\pgfsetfillcolor{currentfill}%
\pgfsetlinewidth{0.000000pt}%
\definecolor{currentstroke}{rgb}{0.000000,0.000000,0.000000}%
\pgfsetstrokecolor{currentstroke}%
\pgfsetdash{}{0pt}%
\pgfpathmoveto{\pgfqpoint{2.294171in}{2.707660in}}%
\pgfpathlineto{\pgfqpoint{2.486911in}{2.956025in}}%
\pgfpathlineto{\pgfqpoint{2.157344in}{2.945086in}}%
\pgfpathlineto{\pgfqpoint{2.294171in}{2.707660in}}%
\pgfpathclose%
\pgfusepath{fill}%
\end{pgfscope}%
\begin{pgfscope}%
\pgfpathrectangle{\pgfqpoint{0.708921in}{0.208778in}}{\pgfqpoint{3.800000in}{3.800000in}}%
\pgfusepath{clip}%
\pgfsetbuttcap%
\pgfsetroundjoin%
\definecolor{currentfill}{rgb}{0.071694,0.275212,0.416288}%
\pgfsetfillcolor{currentfill}%
\pgfsetlinewidth{0.000000pt}%
\definecolor{currentstroke}{rgb}{0.000000,0.000000,0.000000}%
\pgfsetstrokecolor{currentstroke}%
\pgfsetdash{}{0pt}%
\pgfpathmoveto{\pgfqpoint{3.163200in}{2.945086in}}%
\pgfpathlineto{\pgfqpoint{2.833633in}{2.956025in}}%
\pgfpathlineto{\pgfqpoint{3.026373in}{2.707660in}}%
\pgfpathlineto{\pgfqpoint{3.163200in}{2.945086in}}%
\pgfpathclose%
\pgfusepath{fill}%
\end{pgfscope}%
\begin{pgfscope}%
\pgfpathrectangle{\pgfqpoint{0.708921in}{0.208778in}}{\pgfqpoint{3.800000in}{3.800000in}}%
\pgfusepath{clip}%
\pgfsetbuttcap%
\pgfsetroundjoin%
\definecolor{currentfill}{rgb}{0.071636,0.274990,0.415951}%
\pgfsetfillcolor{currentfill}%
\pgfsetlinewidth{0.000000pt}%
\definecolor{currentstroke}{rgb}{0.000000,0.000000,0.000000}%
\pgfsetstrokecolor{currentstroke}%
\pgfsetdash{}{0pt}%
\pgfpathmoveto{\pgfqpoint{2.660272in}{2.709310in}}%
\pgfpathlineto{\pgfqpoint{2.833633in}{2.956025in}}%
\pgfpathlineto{\pgfqpoint{2.486911in}{2.956025in}}%
\pgfpathlineto{\pgfqpoint{2.660272in}{2.709310in}}%
\pgfpathclose%
\pgfusepath{fill}%
\end{pgfscope}%
\begin{pgfscope}%
\pgfpathrectangle{\pgfqpoint{0.708921in}{0.208778in}}{\pgfqpoint{3.800000in}{3.800000in}}%
\pgfusepath{clip}%
\pgfsetbuttcap%
\pgfsetroundjoin%
\definecolor{currentfill}{rgb}{0.045820,0.175891,0.266053}%
\pgfsetfillcolor{currentfill}%
\pgfsetlinewidth{0.000000pt}%
\definecolor{currentstroke}{rgb}{0.000000,0.000000,0.000000}%
\pgfsetstrokecolor{currentstroke}%
\pgfsetdash{}{0pt}%
\pgfpathmoveto{\pgfqpoint{2.158398in}{1.480874in}}%
\pgfpathlineto{\pgfqpoint{2.271375in}{2.145805in}}%
\pgfpathlineto{\pgfqpoint{2.100897in}{1.893599in}}%
\pgfpathlineto{\pgfqpoint{2.158398in}{1.480874in}}%
\pgfpathclose%
\pgfusepath{fill}%
\end{pgfscope}%
\begin{pgfscope}%
\pgfpathrectangle{\pgfqpoint{0.708921in}{0.208778in}}{\pgfqpoint{3.800000in}{3.800000in}}%
\pgfusepath{clip}%
\pgfsetbuttcap%
\pgfsetroundjoin%
\definecolor{currentfill}{rgb}{0.045820,0.175891,0.266053}%
\pgfsetfillcolor{currentfill}%
\pgfsetlinewidth{0.000000pt}%
\definecolor{currentstroke}{rgb}{0.000000,0.000000,0.000000}%
\pgfsetstrokecolor{currentstroke}%
\pgfsetdash{}{0pt}%
\pgfpathmoveto{\pgfqpoint{3.219647in}{1.893599in}}%
\pgfpathlineto{\pgfqpoint{3.049169in}{2.145805in}}%
\pgfpathlineto{\pgfqpoint{3.162146in}{1.480874in}}%
\pgfpathlineto{\pgfqpoint{3.219647in}{1.893599in}}%
\pgfpathclose%
\pgfusepath{fill}%
\end{pgfscope}%
\begin{pgfscope}%
\pgfpathrectangle{\pgfqpoint{0.708921in}{0.208778in}}{\pgfqpoint{3.800000in}{3.800000in}}%
\pgfusepath{clip}%
\pgfsetbuttcap%
\pgfsetroundjoin%
\definecolor{currentfill}{rgb}{0.046814,0.179706,0.271825}%
\pgfsetfillcolor{currentfill}%
\pgfsetlinewidth{0.000000pt}%
\definecolor{currentstroke}{rgb}{0.000000,0.000000,0.000000}%
\pgfsetstrokecolor{currentstroke}%
\pgfsetdash{}{0pt}%
\pgfpathmoveto{\pgfqpoint{2.465903in}{1.862594in}}%
\pgfpathlineto{\pgfqpoint{2.660272in}{1.300375in}}%
\pgfpathlineto{\pgfqpoint{2.660272in}{2.135109in}}%
\pgfpathlineto{\pgfqpoint{2.465903in}{1.862594in}}%
\pgfpathclose%
\pgfusepath{fill}%
\end{pgfscope}%
\begin{pgfscope}%
\pgfpathrectangle{\pgfqpoint{0.708921in}{0.208778in}}{\pgfqpoint{3.800000in}{3.800000in}}%
\pgfusepath{clip}%
\pgfsetbuttcap%
\pgfsetroundjoin%
\definecolor{currentfill}{rgb}{0.046814,0.179706,0.271825}%
\pgfsetfillcolor{currentfill}%
\pgfsetlinewidth{0.000000pt}%
\definecolor{currentstroke}{rgb}{0.000000,0.000000,0.000000}%
\pgfsetstrokecolor{currentstroke}%
\pgfsetdash{}{0pt}%
\pgfpathmoveto{\pgfqpoint{2.660272in}{2.135109in}}%
\pgfpathlineto{\pgfqpoint{2.660272in}{1.300375in}}%
\pgfpathlineto{\pgfqpoint{2.854641in}{1.862594in}}%
\pgfpathlineto{\pgfqpoint{2.660272in}{2.135109in}}%
\pgfpathclose%
\pgfusepath{fill}%
\end{pgfscope}%
\begin{pgfscope}%
\pgfpathrectangle{\pgfqpoint{0.708921in}{0.208778in}}{\pgfqpoint{3.800000in}{3.800000in}}%
\pgfusepath{clip}%
\pgfsetbuttcap%
\pgfsetroundjoin%
\definecolor{currentfill}{rgb}{0.069261,0.265872,0.402159}%
\pgfsetfillcolor{currentfill}%
\pgfsetlinewidth{0.000000pt}%
\definecolor{currentstroke}{rgb}{0.000000,0.000000,0.000000}%
\pgfsetstrokecolor{currentstroke}%
\pgfsetdash{}{0pt}%
\pgfpathmoveto{\pgfqpoint{2.660272in}{2.709310in}}%
\pgfpathlineto{\pgfqpoint{2.486911in}{2.956025in}}%
\pgfpathlineto{\pgfqpoint{2.294171in}{2.707660in}}%
\pgfpathlineto{\pgfqpoint{2.660272in}{2.709310in}}%
\pgfpathclose%
\pgfusepath{fill}%
\end{pgfscope}%
\begin{pgfscope}%
\pgfpathrectangle{\pgfqpoint{0.708921in}{0.208778in}}{\pgfqpoint{3.800000in}{3.800000in}}%
\pgfusepath{clip}%
\pgfsetbuttcap%
\pgfsetroundjoin%
\definecolor{currentfill}{rgb}{0.069261,0.265872,0.402159}%
\pgfsetfillcolor{currentfill}%
\pgfsetlinewidth{0.000000pt}%
\definecolor{currentstroke}{rgb}{0.000000,0.000000,0.000000}%
\pgfsetstrokecolor{currentstroke}%
\pgfsetdash{}{0pt}%
\pgfpathmoveto{\pgfqpoint{3.026373in}{2.707660in}}%
\pgfpathlineto{\pgfqpoint{2.833633in}{2.956025in}}%
\pgfpathlineto{\pgfqpoint{2.660272in}{2.709310in}}%
\pgfpathlineto{\pgfqpoint{3.026373in}{2.707660in}}%
\pgfpathclose%
\pgfusepath{fill}%
\end{pgfscope}%
\begin{pgfscope}%
\pgfpathrectangle{\pgfqpoint{0.708921in}{0.208778in}}{\pgfqpoint{3.800000in}{3.800000in}}%
\pgfusepath{clip}%
\pgfsetbuttcap%
\pgfsetroundjoin%
\definecolor{currentfill}{rgb}{0.049465,0.189883,0.287218}%
\pgfsetfillcolor{currentfill}%
\pgfsetlinewidth{0.000000pt}%
\definecolor{currentstroke}{rgb}{0.000000,0.000000,0.000000}%
\pgfsetstrokecolor{currentstroke}%
\pgfsetdash{}{0pt}%
\pgfpathmoveto{\pgfqpoint{2.158398in}{1.480874in}}%
\pgfpathlineto{\pgfqpoint{2.465903in}{1.862594in}}%
\pgfpathlineto{\pgfqpoint{2.271375in}{2.145805in}}%
\pgfpathlineto{\pgfqpoint{2.158398in}{1.480874in}}%
\pgfpathclose%
\pgfusepath{fill}%
\end{pgfscope}%
\begin{pgfscope}%
\pgfpathrectangle{\pgfqpoint{0.708921in}{0.208778in}}{\pgfqpoint{3.800000in}{3.800000in}}%
\pgfusepath{clip}%
\pgfsetbuttcap%
\pgfsetroundjoin%
\definecolor{currentfill}{rgb}{0.049465,0.189883,0.287218}%
\pgfsetfillcolor{currentfill}%
\pgfsetlinewidth{0.000000pt}%
\definecolor{currentstroke}{rgb}{0.000000,0.000000,0.000000}%
\pgfsetstrokecolor{currentstroke}%
\pgfsetdash{}{0pt}%
\pgfpathmoveto{\pgfqpoint{3.049169in}{2.145805in}}%
\pgfpathlineto{\pgfqpoint{2.854641in}{1.862594in}}%
\pgfpathlineto{\pgfqpoint{3.162146in}{1.480874in}}%
\pgfpathlineto{\pgfqpoint{3.049169in}{2.145805in}}%
\pgfpathclose%
\pgfusepath{fill}%
\end{pgfscope}%
\begin{pgfscope}%
\pgfpathrectangle{\pgfqpoint{0.708921in}{0.208778in}}{\pgfqpoint{3.800000in}{3.800000in}}%
\pgfusepath{clip}%
\pgfsetbuttcap%
\pgfsetroundjoin%
\definecolor{currentfill}{rgb}{0.061576,0.236373,0.357539}%
\pgfsetfillcolor{currentfill}%
\pgfsetlinewidth{0.000000pt}%
\definecolor{currentstroke}{rgb}{0.000000,0.000000,0.000000}%
\pgfsetstrokecolor{currentstroke}%
\pgfsetdash{}{0pt}%
\pgfpathmoveto{\pgfqpoint{2.271375in}{2.145805in}}%
\pgfpathlineto{\pgfqpoint{2.294171in}{2.707660in}}%
\pgfpathlineto{\pgfqpoint{1.928629in}{2.174201in}}%
\pgfpathlineto{\pgfqpoint{2.271375in}{2.145805in}}%
\pgfpathclose%
\pgfusepath{fill}%
\end{pgfscope}%
\begin{pgfscope}%
\pgfpathrectangle{\pgfqpoint{0.708921in}{0.208778in}}{\pgfqpoint{3.800000in}{3.800000in}}%
\pgfusepath{clip}%
\pgfsetbuttcap%
\pgfsetroundjoin%
\definecolor{currentfill}{rgb}{0.061576,0.236373,0.357539}%
\pgfsetfillcolor{currentfill}%
\pgfsetlinewidth{0.000000pt}%
\definecolor{currentstroke}{rgb}{0.000000,0.000000,0.000000}%
\pgfsetstrokecolor{currentstroke}%
\pgfsetdash{}{0pt}%
\pgfpathmoveto{\pgfqpoint{3.391915in}{2.174201in}}%
\pgfpathlineto{\pgfqpoint{3.026373in}{2.707660in}}%
\pgfpathlineto{\pgfqpoint{3.049169in}{2.145805in}}%
\pgfpathlineto{\pgfqpoint{3.391915in}{2.174201in}}%
\pgfpathclose%
\pgfusepath{fill}%
\end{pgfscope}%
\begin{pgfscope}%
\pgfpathrectangle{\pgfqpoint{0.708921in}{0.208778in}}{\pgfqpoint{3.800000in}{3.800000in}}%
\pgfusepath{clip}%
\pgfsetbuttcap%
\pgfsetroundjoin%
\definecolor{currentfill}{rgb}{0.053541,0.205528,0.310883}%
\pgfsetfillcolor{currentfill}%
\pgfsetlinewidth{0.000000pt}%
\definecolor{currentstroke}{rgb}{0.000000,0.000000,0.000000}%
\pgfsetstrokecolor{currentstroke}%
\pgfsetdash{}{0pt}%
\pgfpathmoveto{\pgfqpoint{1.928629in}{2.174201in}}%
\pgfpathlineto{\pgfqpoint{2.100897in}{1.893599in}}%
\pgfpathlineto{\pgfqpoint{2.271375in}{2.145805in}}%
\pgfpathlineto{\pgfqpoint{1.928629in}{2.174201in}}%
\pgfpathclose%
\pgfusepath{fill}%
\end{pgfscope}%
\begin{pgfscope}%
\pgfpathrectangle{\pgfqpoint{0.708921in}{0.208778in}}{\pgfqpoint{3.800000in}{3.800000in}}%
\pgfusepath{clip}%
\pgfsetbuttcap%
\pgfsetroundjoin%
\definecolor{currentfill}{rgb}{0.053541,0.205528,0.310883}%
\pgfsetfillcolor{currentfill}%
\pgfsetlinewidth{0.000000pt}%
\definecolor{currentstroke}{rgb}{0.000000,0.000000,0.000000}%
\pgfsetstrokecolor{currentstroke}%
\pgfsetdash{}{0pt}%
\pgfpathmoveto{\pgfqpoint{3.049169in}{2.145805in}}%
\pgfpathlineto{\pgfqpoint{3.219647in}{1.893599in}}%
\pgfpathlineto{\pgfqpoint{3.391915in}{2.174201in}}%
\pgfpathlineto{\pgfqpoint{3.049169in}{2.145805in}}%
\pgfpathclose%
\pgfusepath{fill}%
\end{pgfscope}%
\begin{pgfscope}%
\pgfpathrectangle{\pgfqpoint{0.708921in}{0.208778in}}{\pgfqpoint{3.800000in}{3.800000in}}%
\pgfusepath{clip}%
\pgfsetbuttcap%
\pgfsetroundjoin%
\definecolor{currentfill}{rgb}{0.060634,0.232757,0.352069}%
\pgfsetfillcolor{currentfill}%
\pgfsetlinewidth{0.000000pt}%
\definecolor{currentstroke}{rgb}{0.000000,0.000000,0.000000}%
\pgfsetstrokecolor{currentstroke}%
\pgfsetdash{}{0pt}%
\pgfpathmoveto{\pgfqpoint{2.294171in}{2.707660in}}%
\pgfpathlineto{\pgfqpoint{2.271375in}{2.145805in}}%
\pgfpathlineto{\pgfqpoint{2.660272in}{2.709310in}}%
\pgfpathlineto{\pgfqpoint{2.294171in}{2.707660in}}%
\pgfpathclose%
\pgfusepath{fill}%
\end{pgfscope}%
\begin{pgfscope}%
\pgfpathrectangle{\pgfqpoint{0.708921in}{0.208778in}}{\pgfqpoint{3.800000in}{3.800000in}}%
\pgfusepath{clip}%
\pgfsetbuttcap%
\pgfsetroundjoin%
\definecolor{currentfill}{rgb}{0.060634,0.232757,0.352069}%
\pgfsetfillcolor{currentfill}%
\pgfsetlinewidth{0.000000pt}%
\definecolor{currentstroke}{rgb}{0.000000,0.000000,0.000000}%
\pgfsetstrokecolor{currentstroke}%
\pgfsetdash{}{0pt}%
\pgfpathmoveto{\pgfqpoint{2.660272in}{2.709310in}}%
\pgfpathlineto{\pgfqpoint{3.049169in}{2.145805in}}%
\pgfpathlineto{\pgfqpoint{3.026373in}{2.707660in}}%
\pgfpathlineto{\pgfqpoint{2.660272in}{2.709310in}}%
\pgfpathclose%
\pgfusepath{fill}%
\end{pgfscope}%
\begin{pgfscope}%
\pgfpathrectangle{\pgfqpoint{0.708921in}{0.208778in}}{\pgfqpoint{3.800000in}{3.800000in}}%
\pgfusepath{clip}%
\pgfsetbuttcap%
\pgfsetroundjoin%
\definecolor{currentfill}{rgb}{0.060773,0.233289,0.352874}%
\pgfsetfillcolor{currentfill}%
\pgfsetlinewidth{0.000000pt}%
\definecolor{currentstroke}{rgb}{0.000000,0.000000,0.000000}%
\pgfsetstrokecolor{currentstroke}%
\pgfsetdash{}{0pt}%
\pgfpathmoveto{\pgfqpoint{2.660272in}{2.135109in}}%
\pgfpathlineto{\pgfqpoint{2.660272in}{2.709310in}}%
\pgfpathlineto{\pgfqpoint{2.271375in}{2.145805in}}%
\pgfpathlineto{\pgfqpoint{2.660272in}{2.135109in}}%
\pgfpathclose%
\pgfusepath{fill}%
\end{pgfscope}%
\begin{pgfscope}%
\pgfpathrectangle{\pgfqpoint{0.708921in}{0.208778in}}{\pgfqpoint{3.800000in}{3.800000in}}%
\pgfusepath{clip}%
\pgfsetbuttcap%
\pgfsetroundjoin%
\definecolor{currentfill}{rgb}{0.060773,0.233289,0.352874}%
\pgfsetfillcolor{currentfill}%
\pgfsetlinewidth{0.000000pt}%
\definecolor{currentstroke}{rgb}{0.000000,0.000000,0.000000}%
\pgfsetstrokecolor{currentstroke}%
\pgfsetdash{}{0pt}%
\pgfpathmoveto{\pgfqpoint{3.049169in}{2.145805in}}%
\pgfpathlineto{\pgfqpoint{2.660272in}{2.709310in}}%
\pgfpathlineto{\pgfqpoint{2.660272in}{2.135109in}}%
\pgfpathlineto{\pgfqpoint{3.049169in}{2.145805in}}%
\pgfpathclose%
\pgfusepath{fill}%
\end{pgfscope}%
\begin{pgfscope}%
\pgfpathrectangle{\pgfqpoint{0.708921in}{0.208778in}}{\pgfqpoint{3.800000in}{3.800000in}}%
\pgfusepath{clip}%
\pgfsetbuttcap%
\pgfsetroundjoin%
\definecolor{currentfill}{rgb}{0.052607,0.201942,0.305459}%
\pgfsetfillcolor{currentfill}%
\pgfsetlinewidth{0.000000pt}%
\definecolor{currentstroke}{rgb}{0.000000,0.000000,0.000000}%
\pgfsetstrokecolor{currentstroke}%
\pgfsetdash{}{0pt}%
\pgfpathmoveto{\pgfqpoint{2.271375in}{2.145805in}}%
\pgfpathlineto{\pgfqpoint{2.465903in}{1.862594in}}%
\pgfpathlineto{\pgfqpoint{2.660272in}{2.135109in}}%
\pgfpathlineto{\pgfqpoint{2.271375in}{2.145805in}}%
\pgfpathclose%
\pgfusepath{fill}%
\end{pgfscope}%
\begin{pgfscope}%
\pgfpathrectangle{\pgfqpoint{0.708921in}{0.208778in}}{\pgfqpoint{3.800000in}{3.800000in}}%
\pgfusepath{clip}%
\pgfsetbuttcap%
\pgfsetroundjoin%
\definecolor{currentfill}{rgb}{0.052607,0.201942,0.305459}%
\pgfsetfillcolor{currentfill}%
\pgfsetlinewidth{0.000000pt}%
\definecolor{currentstroke}{rgb}{0.000000,0.000000,0.000000}%
\pgfsetstrokecolor{currentstroke}%
\pgfsetdash{}{0pt}%
\pgfpathmoveto{\pgfqpoint{2.660272in}{2.135109in}}%
\pgfpathlineto{\pgfqpoint{2.854641in}{1.862594in}}%
\pgfpathlineto{\pgfqpoint{3.049169in}{2.145805in}}%
\pgfpathlineto{\pgfqpoint{2.660272in}{2.135109in}}%
\pgfpathclose%
\pgfusepath{fill}%
\end{pgfscope}%
\begin{pgfscope}%
\pgfpathrectangle{\pgfqpoint{0.708921in}{0.208778in}}{\pgfqpoint{3.800000in}{3.800000in}}%
\pgfusepath{clip}%
\pgfsetbuttcap%
\pgfsetroundjoin%
\definecolor{currentfill}{rgb}{0.839216,0.152941,0.156863}%
\pgfsetfillcolor{currentfill}%
\pgfsetfillopacity{0.300000}%
\pgfsetlinewidth{1.003750pt}%
\definecolor{currentstroke}{rgb}{0.839216,0.152941,0.156863}%
\pgfsetstrokecolor{currentstroke}%
\pgfsetstrokeopacity{0.300000}%
\pgfsetdash{}{0pt}%
\pgfpathmoveto{\pgfqpoint{1.835792in}{1.331622in}}%
\pgfpathcurveto{\pgfqpoint{1.845879in}{1.331622in}}{\pgfqpoint{1.855555in}{1.335630in}}{\pgfqpoint{1.862688in}{1.342762in}}%
\pgfpathcurveto{\pgfqpoint{1.869820in}{1.349895in}}{\pgfqpoint{1.873828in}{1.359571in}}{\pgfqpoint{1.873828in}{1.369658in}}%
\pgfpathcurveto{\pgfqpoint{1.873828in}{1.379745in}}{\pgfqpoint{1.869820in}{1.389421in}}{\pgfqpoint{1.862688in}{1.396554in}}%
\pgfpathcurveto{\pgfqpoint{1.855555in}{1.403687in}}{\pgfqpoint{1.845879in}{1.407694in}}{\pgfqpoint{1.835792in}{1.407694in}}%
\pgfpathcurveto{\pgfqpoint{1.825705in}{1.407694in}}{\pgfqpoint{1.816029in}{1.403687in}}{\pgfqpoint{1.808896in}{1.396554in}}%
\pgfpathcurveto{\pgfqpoint{1.801763in}{1.389421in}}{\pgfqpoint{1.797756in}{1.379745in}}{\pgfqpoint{1.797756in}{1.369658in}}%
\pgfpathcurveto{\pgfqpoint{1.797756in}{1.359571in}}{\pgfqpoint{1.801763in}{1.349895in}}{\pgfqpoint{1.808896in}{1.342762in}}%
\pgfpathcurveto{\pgfqpoint{1.816029in}{1.335630in}}{\pgfqpoint{1.825705in}{1.331622in}}{\pgfqpoint{1.835792in}{1.331622in}}%
\pgfpathlineto{\pgfqpoint{1.835792in}{1.331622in}}%
\pgfpathclose%
\pgfusepath{stroke,fill}%
\end{pgfscope}%
\begin{pgfscope}%
\pgfpathrectangle{\pgfqpoint{0.708921in}{0.208778in}}{\pgfqpoint{3.800000in}{3.800000in}}%
\pgfusepath{clip}%
\pgfsetbuttcap%
\pgfsetroundjoin%
\definecolor{currentfill}{rgb}{0.839216,0.152941,0.156863}%
\pgfsetfillcolor{currentfill}%
\pgfsetfillopacity{0.383610}%
\pgfsetlinewidth{1.003750pt}%
\definecolor{currentstroke}{rgb}{0.839216,0.152941,0.156863}%
\pgfsetstrokecolor{currentstroke}%
\pgfsetstrokeopacity{0.383610}%
\pgfsetdash{}{0pt}%
\pgfpathmoveto{\pgfqpoint{1.808197in}{1.421460in}}%
\pgfpathcurveto{\pgfqpoint{1.818284in}{1.421460in}}{\pgfqpoint{1.827960in}{1.425468in}}{\pgfqpoint{1.835093in}{1.432601in}}%
\pgfpathcurveto{\pgfqpoint{1.842225in}{1.439734in}}{\pgfqpoint{1.846233in}{1.449409in}}{\pgfqpoint{1.846233in}{1.459497in}}%
\pgfpathcurveto{\pgfqpoint{1.846233in}{1.469584in}}{\pgfqpoint{1.842225in}{1.479259in}}{\pgfqpoint{1.835093in}{1.486392in}}%
\pgfpathcurveto{\pgfqpoint{1.827960in}{1.493525in}}{\pgfqpoint{1.818284in}{1.497533in}}{\pgfqpoint{1.808197in}{1.497533in}}%
\pgfpathcurveto{\pgfqpoint{1.798110in}{1.497533in}}{\pgfqpoint{1.788434in}{1.493525in}}{\pgfqpoint{1.781301in}{1.486392in}}%
\pgfpathcurveto{\pgfqpoint{1.774168in}{1.479259in}}{\pgfqpoint{1.770161in}{1.469584in}}{\pgfqpoint{1.770161in}{1.459497in}}%
\pgfpathcurveto{\pgfqpoint{1.770161in}{1.449409in}}{\pgfqpoint{1.774168in}{1.439734in}}{\pgfqpoint{1.781301in}{1.432601in}}%
\pgfpathcurveto{\pgfqpoint{1.788434in}{1.425468in}}{\pgfqpoint{1.798110in}{1.421460in}}{\pgfqpoint{1.808197in}{1.421460in}}%
\pgfpathlineto{\pgfqpoint{1.808197in}{1.421460in}}%
\pgfpathclose%
\pgfusepath{stroke,fill}%
\end{pgfscope}%
\begin{pgfscope}%
\pgfpathrectangle{\pgfqpoint{0.708921in}{0.208778in}}{\pgfqpoint{3.800000in}{3.800000in}}%
\pgfusepath{clip}%
\pgfsetbuttcap%
\pgfsetroundjoin%
\definecolor{currentfill}{rgb}{0.839216,0.152941,0.156863}%
\pgfsetfillcolor{currentfill}%
\pgfsetfillopacity{0.457533}%
\pgfsetlinewidth{1.003750pt}%
\definecolor{currentstroke}{rgb}{0.839216,0.152941,0.156863}%
\pgfsetstrokecolor{currentstroke}%
\pgfsetstrokeopacity{0.457533}%
\pgfsetdash{}{0pt}%
\pgfpathmoveto{\pgfqpoint{2.024941in}{1.356079in}}%
\pgfpathcurveto{\pgfqpoint{2.035028in}{1.356079in}}{\pgfqpoint{2.044704in}{1.360087in}}{\pgfqpoint{2.051837in}{1.367220in}}%
\pgfpathcurveto{\pgfqpoint{2.058970in}{1.374353in}}{\pgfqpoint{2.062977in}{1.384028in}}{\pgfqpoint{2.062977in}{1.394116in}}%
\pgfpathcurveto{\pgfqpoint{2.062977in}{1.404203in}}{\pgfqpoint{2.058970in}{1.413878in}}{\pgfqpoint{2.051837in}{1.421011in}}%
\pgfpathcurveto{\pgfqpoint{2.044704in}{1.428144in}}{\pgfqpoint{2.035028in}{1.432152in}}{\pgfqpoint{2.024941in}{1.432152in}}%
\pgfpathcurveto{\pgfqpoint{2.014854in}{1.432152in}}{\pgfqpoint{2.005178in}{1.428144in}}{\pgfqpoint{1.998045in}{1.421011in}}%
\pgfpathcurveto{\pgfqpoint{1.990912in}{1.413878in}}{\pgfqpoint{1.986905in}{1.404203in}}{\pgfqpoint{1.986905in}{1.394116in}}%
\pgfpathcurveto{\pgfqpoint{1.986905in}{1.384028in}}{\pgfqpoint{1.990912in}{1.374353in}}{\pgfqpoint{1.998045in}{1.367220in}}%
\pgfpathcurveto{\pgfqpoint{2.005178in}{1.360087in}}{\pgfqpoint{2.014854in}{1.356079in}}{\pgfqpoint{2.024941in}{1.356079in}}%
\pgfpathlineto{\pgfqpoint{2.024941in}{1.356079in}}%
\pgfpathclose%
\pgfusepath{stroke,fill}%
\end{pgfscope}%
\begin{pgfscope}%
\pgfpathrectangle{\pgfqpoint{0.708921in}{0.208778in}}{\pgfqpoint{3.800000in}{3.800000in}}%
\pgfusepath{clip}%
\pgfsetbuttcap%
\pgfsetroundjoin%
\definecolor{currentfill}{rgb}{0.839216,0.152941,0.156863}%
\pgfsetfillcolor{currentfill}%
\pgfsetfillopacity{0.492303}%
\pgfsetlinewidth{1.003750pt}%
\definecolor{currentstroke}{rgb}{0.839216,0.152941,0.156863}%
\pgfsetstrokecolor{currentstroke}%
\pgfsetstrokeopacity{0.492303}%
\pgfsetdash{}{0pt}%
\pgfpathmoveto{\pgfqpoint{1.575764in}{2.118941in}}%
\pgfpathcurveto{\pgfqpoint{1.585851in}{2.118941in}}{\pgfqpoint{1.595527in}{2.122949in}}{\pgfqpoint{1.602659in}{2.130082in}}%
\pgfpathcurveto{\pgfqpoint{1.609792in}{2.137215in}}{\pgfqpoint{1.613800in}{2.146890in}}{\pgfqpoint{1.613800in}{2.156978in}}%
\pgfpathcurveto{\pgfqpoint{1.613800in}{2.167065in}}{\pgfqpoint{1.609792in}{2.176740in}}{\pgfqpoint{1.602659in}{2.183873in}}%
\pgfpathcurveto{\pgfqpoint{1.595527in}{2.191006in}}{\pgfqpoint{1.585851in}{2.195014in}}{\pgfqpoint{1.575764in}{2.195014in}}%
\pgfpathcurveto{\pgfqpoint{1.565676in}{2.195014in}}{\pgfqpoint{1.556001in}{2.191006in}}{\pgfqpoint{1.548868in}{2.183873in}}%
\pgfpathcurveto{\pgfqpoint{1.541735in}{2.176740in}}{\pgfqpoint{1.537727in}{2.167065in}}{\pgfqpoint{1.537727in}{2.156978in}}%
\pgfpathcurveto{\pgfqpoint{1.537727in}{2.146890in}}{\pgfqpoint{1.541735in}{2.137215in}}{\pgfqpoint{1.548868in}{2.130082in}}%
\pgfpathcurveto{\pgfqpoint{1.556001in}{2.122949in}}{\pgfqpoint{1.565676in}{2.118941in}}{\pgfqpoint{1.575764in}{2.118941in}}%
\pgfpathlineto{\pgfqpoint{1.575764in}{2.118941in}}%
\pgfpathclose%
\pgfusepath{stroke,fill}%
\end{pgfscope}%
\begin{pgfscope}%
\pgfpathrectangle{\pgfqpoint{0.708921in}{0.208778in}}{\pgfqpoint{3.800000in}{3.800000in}}%
\pgfusepath{clip}%
\pgfsetbuttcap%
\pgfsetroundjoin%
\definecolor{currentfill}{rgb}{0.839216,0.152941,0.156863}%
\pgfsetfillcolor{currentfill}%
\pgfsetfillopacity{0.498590}%
\pgfsetlinewidth{1.003750pt}%
\definecolor{currentstroke}{rgb}{0.839216,0.152941,0.156863}%
\pgfsetstrokecolor{currentstroke}%
\pgfsetstrokeopacity{0.498590}%
\pgfsetdash{}{0pt}%
\pgfpathmoveto{\pgfqpoint{3.864236in}{2.271734in}}%
\pgfpathcurveto{\pgfqpoint{3.874324in}{2.271734in}}{\pgfqpoint{3.883999in}{2.275742in}}{\pgfqpoint{3.891132in}{2.282875in}}%
\pgfpathcurveto{\pgfqpoint{3.898265in}{2.290008in}}{\pgfqpoint{3.902273in}{2.299683in}}{\pgfqpoint{3.902273in}{2.309771in}}%
\pgfpathcurveto{\pgfqpoint{3.902273in}{2.319858in}}{\pgfqpoint{3.898265in}{2.329533in}}{\pgfqpoint{3.891132in}{2.336666in}}%
\pgfpathcurveto{\pgfqpoint{3.883999in}{2.343799in}}{\pgfqpoint{3.874324in}{2.347807in}}{\pgfqpoint{3.864236in}{2.347807in}}%
\pgfpathcurveto{\pgfqpoint{3.854149in}{2.347807in}}{\pgfqpoint{3.844473in}{2.343799in}}{\pgfqpoint{3.837341in}{2.336666in}}%
\pgfpathcurveto{\pgfqpoint{3.830208in}{2.329533in}}{\pgfqpoint{3.826200in}{2.319858in}}{\pgfqpoint{3.826200in}{2.309771in}}%
\pgfpathcurveto{\pgfqpoint{3.826200in}{2.299683in}}{\pgfqpoint{3.830208in}{2.290008in}}{\pgfqpoint{3.837341in}{2.282875in}}%
\pgfpathcurveto{\pgfqpoint{3.844473in}{2.275742in}}{\pgfqpoint{3.854149in}{2.271734in}}{\pgfqpoint{3.864236in}{2.271734in}}%
\pgfpathlineto{\pgfqpoint{3.864236in}{2.271734in}}%
\pgfpathclose%
\pgfusepath{stroke,fill}%
\end{pgfscope}%
\begin{pgfscope}%
\pgfpathrectangle{\pgfqpoint{0.708921in}{0.208778in}}{\pgfqpoint{3.800000in}{3.800000in}}%
\pgfusepath{clip}%
\pgfsetbuttcap%
\pgfsetroundjoin%
\definecolor{currentfill}{rgb}{0.839216,0.152941,0.156863}%
\pgfsetfillcolor{currentfill}%
\pgfsetfillopacity{0.612876}%
\pgfsetlinewidth{1.003750pt}%
\definecolor{currentstroke}{rgb}{0.839216,0.152941,0.156863}%
\pgfsetstrokecolor{currentstroke}%
\pgfsetstrokeopacity{0.612876}%
\pgfsetdash{}{0pt}%
\pgfpathmoveto{\pgfqpoint{2.041957in}{1.379714in}}%
\pgfpathcurveto{\pgfqpoint{2.052044in}{1.379714in}}{\pgfqpoint{2.061720in}{1.383721in}}{\pgfqpoint{2.068853in}{1.390854in}}%
\pgfpathcurveto{\pgfqpoint{2.075985in}{1.397987in}}{\pgfqpoint{2.079993in}{1.407663in}}{\pgfqpoint{2.079993in}{1.417750in}}%
\pgfpathcurveto{\pgfqpoint{2.079993in}{1.427837in}}{\pgfqpoint{2.075985in}{1.437513in}}{\pgfqpoint{2.068853in}{1.444646in}}%
\pgfpathcurveto{\pgfqpoint{2.061720in}{1.451778in}}{\pgfqpoint{2.052044in}{1.455786in}}{\pgfqpoint{2.041957in}{1.455786in}}%
\pgfpathcurveto{\pgfqpoint{2.031870in}{1.455786in}}{\pgfqpoint{2.022194in}{1.451778in}}{\pgfqpoint{2.015061in}{1.444646in}}%
\pgfpathcurveto{\pgfqpoint{2.007928in}{1.437513in}}{\pgfqpoint{2.003921in}{1.427837in}}{\pgfqpoint{2.003921in}{1.417750in}}%
\pgfpathcurveto{\pgfqpoint{2.003921in}{1.407663in}}{\pgfqpoint{2.007928in}{1.397987in}}{\pgfqpoint{2.015061in}{1.390854in}}%
\pgfpathcurveto{\pgfqpoint{2.022194in}{1.383721in}}{\pgfqpoint{2.031870in}{1.379714in}}{\pgfqpoint{2.041957in}{1.379714in}}%
\pgfpathlineto{\pgfqpoint{2.041957in}{1.379714in}}%
\pgfpathclose%
\pgfusepath{stroke,fill}%
\end{pgfscope}%
\begin{pgfscope}%
\pgfpathrectangle{\pgfqpoint{0.708921in}{0.208778in}}{\pgfqpoint{3.800000in}{3.800000in}}%
\pgfusepath{clip}%
\pgfsetbuttcap%
\pgfsetroundjoin%
\definecolor{currentfill}{rgb}{0.839216,0.152941,0.156863}%
\pgfsetfillcolor{currentfill}%
\pgfsetfillopacity{0.625674}%
\pgfsetlinewidth{1.003750pt}%
\definecolor{currentstroke}{rgb}{0.839216,0.152941,0.156863}%
\pgfsetstrokecolor{currentstroke}%
\pgfsetstrokeopacity{0.625674}%
\pgfsetdash{}{0pt}%
\pgfpathmoveto{\pgfqpoint{3.262967in}{3.025545in}}%
\pgfpathcurveto{\pgfqpoint{3.273054in}{3.025545in}}{\pgfqpoint{3.282730in}{3.029552in}}{\pgfqpoint{3.289862in}{3.036685in}}%
\pgfpathcurveto{\pgfqpoint{3.296995in}{3.043818in}}{\pgfqpoint{3.301003in}{3.053494in}}{\pgfqpoint{3.301003in}{3.063581in}}%
\pgfpathcurveto{\pgfqpoint{3.301003in}{3.073668in}}{\pgfqpoint{3.296995in}{3.083344in}}{\pgfqpoint{3.289862in}{3.090477in}}%
\pgfpathcurveto{\pgfqpoint{3.282730in}{3.097609in}}{\pgfqpoint{3.273054in}{3.101617in}}{\pgfqpoint{3.262967in}{3.101617in}}%
\pgfpathcurveto{\pgfqpoint{3.252879in}{3.101617in}}{\pgfqpoint{3.243204in}{3.097609in}}{\pgfqpoint{3.236071in}{3.090477in}}%
\pgfpathcurveto{\pgfqpoint{3.228938in}{3.083344in}}{\pgfqpoint{3.224930in}{3.073668in}}{\pgfqpoint{3.224930in}{3.063581in}}%
\pgfpathcurveto{\pgfqpoint{3.224930in}{3.053494in}}{\pgfqpoint{3.228938in}{3.043818in}}{\pgfqpoint{3.236071in}{3.036685in}}%
\pgfpathcurveto{\pgfqpoint{3.243204in}{3.029552in}}{\pgfqpoint{3.252879in}{3.025545in}}{\pgfqpoint{3.262967in}{3.025545in}}%
\pgfpathlineto{\pgfqpoint{3.262967in}{3.025545in}}%
\pgfpathclose%
\pgfusepath{stroke,fill}%
\end{pgfscope}%
\begin{pgfscope}%
\pgfpathrectangle{\pgfqpoint{0.708921in}{0.208778in}}{\pgfqpoint{3.800000in}{3.800000in}}%
\pgfusepath{clip}%
\pgfsetbuttcap%
\pgfsetroundjoin%
\definecolor{currentfill}{rgb}{0.839216,0.152941,0.156863}%
\pgfsetfillcolor{currentfill}%
\pgfsetfillopacity{0.631635}%
\pgfsetlinewidth{1.003750pt}%
\definecolor{currentstroke}{rgb}{0.839216,0.152941,0.156863}%
\pgfsetstrokecolor{currentstroke}%
\pgfsetstrokeopacity{0.631635}%
\pgfsetdash{}{0pt}%
\pgfpathmoveto{\pgfqpoint{3.250605in}{2.942909in}}%
\pgfpathcurveto{\pgfqpoint{3.260692in}{2.942909in}}{\pgfqpoint{3.270368in}{2.946917in}}{\pgfqpoint{3.277501in}{2.954050in}}%
\pgfpathcurveto{\pgfqpoint{3.284634in}{2.961183in}}{\pgfqpoint{3.288641in}{2.970858in}}{\pgfqpoint{3.288641in}{2.980946in}}%
\pgfpathcurveto{\pgfqpoint{3.288641in}{2.991033in}}{\pgfqpoint{3.284634in}{3.000709in}}{\pgfqpoint{3.277501in}{3.007841in}}%
\pgfpathcurveto{\pgfqpoint{3.270368in}{3.014974in}}{\pgfqpoint{3.260692in}{3.018982in}}{\pgfqpoint{3.250605in}{3.018982in}}%
\pgfpathcurveto{\pgfqpoint{3.240518in}{3.018982in}}{\pgfqpoint{3.230842in}{3.014974in}}{\pgfqpoint{3.223709in}{3.007841in}}%
\pgfpathcurveto{\pgfqpoint{3.216577in}{3.000709in}}{\pgfqpoint{3.212569in}{2.991033in}}{\pgfqpoint{3.212569in}{2.980946in}}%
\pgfpathcurveto{\pgfqpoint{3.212569in}{2.970858in}}{\pgfqpoint{3.216577in}{2.961183in}}{\pgfqpoint{3.223709in}{2.954050in}}%
\pgfpathcurveto{\pgfqpoint{3.230842in}{2.946917in}}{\pgfqpoint{3.240518in}{2.942909in}}{\pgfqpoint{3.250605in}{2.942909in}}%
\pgfpathlineto{\pgfqpoint{3.250605in}{2.942909in}}%
\pgfpathclose%
\pgfusepath{stroke,fill}%
\end{pgfscope}%
\begin{pgfscope}%
\pgfpathrectangle{\pgfqpoint{0.708921in}{0.208778in}}{\pgfqpoint{3.800000in}{3.800000in}}%
\pgfusepath{clip}%
\pgfsetbuttcap%
\pgfsetroundjoin%
\definecolor{currentfill}{rgb}{0.839216,0.152941,0.156863}%
\pgfsetfillcolor{currentfill}%
\pgfsetfillopacity{0.634032}%
\pgfsetlinewidth{1.003750pt}%
\definecolor{currentstroke}{rgb}{0.839216,0.152941,0.156863}%
\pgfsetstrokecolor{currentstroke}%
\pgfsetstrokeopacity{0.634032}%
\pgfsetdash{}{0pt}%
\pgfpathmoveto{\pgfqpoint{3.627084in}{2.149100in}}%
\pgfpathcurveto{\pgfqpoint{3.637171in}{2.149100in}}{\pgfqpoint{3.646847in}{2.153108in}}{\pgfqpoint{3.653980in}{2.160241in}}%
\pgfpathcurveto{\pgfqpoint{3.661113in}{2.167374in}}{\pgfqpoint{3.665120in}{2.177049in}}{\pgfqpoint{3.665120in}{2.187137in}}%
\pgfpathcurveto{\pgfqpoint{3.665120in}{2.197224in}}{\pgfqpoint{3.661113in}{2.206900in}}{\pgfqpoint{3.653980in}{2.214032in}}%
\pgfpathcurveto{\pgfqpoint{3.646847in}{2.221165in}}{\pgfqpoint{3.637171in}{2.225173in}}{\pgfqpoint{3.627084in}{2.225173in}}%
\pgfpathcurveto{\pgfqpoint{3.616997in}{2.225173in}}{\pgfqpoint{3.607321in}{2.221165in}}{\pgfqpoint{3.600188in}{2.214032in}}%
\pgfpathcurveto{\pgfqpoint{3.593056in}{2.206900in}}{\pgfqpoint{3.589048in}{2.197224in}}{\pgfqpoint{3.589048in}{2.187137in}}%
\pgfpathcurveto{\pgfqpoint{3.589048in}{2.177049in}}{\pgfqpoint{3.593056in}{2.167374in}}{\pgfqpoint{3.600188in}{2.160241in}}%
\pgfpathcurveto{\pgfqpoint{3.607321in}{2.153108in}}{\pgfqpoint{3.616997in}{2.149100in}}{\pgfqpoint{3.627084in}{2.149100in}}%
\pgfpathlineto{\pgfqpoint{3.627084in}{2.149100in}}%
\pgfpathclose%
\pgfusepath{stroke,fill}%
\end{pgfscope}%
\begin{pgfscope}%
\pgfpathrectangle{\pgfqpoint{0.708921in}{0.208778in}}{\pgfqpoint{3.800000in}{3.800000in}}%
\pgfusepath{clip}%
\pgfsetbuttcap%
\pgfsetroundjoin%
\definecolor{currentfill}{rgb}{0.839216,0.152941,0.156863}%
\pgfsetfillcolor{currentfill}%
\pgfsetfillopacity{0.652064}%
\pgfsetlinewidth{1.003750pt}%
\definecolor{currentstroke}{rgb}{0.839216,0.152941,0.156863}%
\pgfsetstrokecolor{currentstroke}%
\pgfsetstrokeopacity{0.652064}%
\pgfsetdash{}{0pt}%
\pgfpathmoveto{\pgfqpoint{2.395222in}{1.341889in}}%
\pgfpathcurveto{\pgfqpoint{2.405309in}{1.341889in}}{\pgfqpoint{2.414985in}{1.345897in}}{\pgfqpoint{2.422118in}{1.353030in}}%
\pgfpathcurveto{\pgfqpoint{2.429250in}{1.360163in}}{\pgfqpoint{2.433258in}{1.369838in}}{\pgfqpoint{2.433258in}{1.379926in}}%
\pgfpathcurveto{\pgfqpoint{2.433258in}{1.390013in}}{\pgfqpoint{2.429250in}{1.399688in}}{\pgfqpoint{2.422118in}{1.406821in}}%
\pgfpathcurveto{\pgfqpoint{2.414985in}{1.413954in}}{\pgfqpoint{2.405309in}{1.417962in}}{\pgfqpoint{2.395222in}{1.417962in}}%
\pgfpathcurveto{\pgfqpoint{2.385134in}{1.417962in}}{\pgfqpoint{2.375459in}{1.413954in}}{\pgfqpoint{2.368326in}{1.406821in}}%
\pgfpathcurveto{\pgfqpoint{2.361193in}{1.399688in}}{\pgfqpoint{2.357186in}{1.390013in}}{\pgfqpoint{2.357186in}{1.379926in}}%
\pgfpathcurveto{\pgfqpoint{2.357186in}{1.369838in}}{\pgfqpoint{2.361193in}{1.360163in}}{\pgfqpoint{2.368326in}{1.353030in}}%
\pgfpathcurveto{\pgfqpoint{2.375459in}{1.345897in}}{\pgfqpoint{2.385134in}{1.341889in}}{\pgfqpoint{2.395222in}{1.341889in}}%
\pgfpathlineto{\pgfqpoint{2.395222in}{1.341889in}}%
\pgfpathclose%
\pgfusepath{stroke,fill}%
\end{pgfscope}%
\begin{pgfscope}%
\pgfpathrectangle{\pgfqpoint{0.708921in}{0.208778in}}{\pgfqpoint{3.800000in}{3.800000in}}%
\pgfusepath{clip}%
\pgfsetbuttcap%
\pgfsetroundjoin%
\definecolor{currentfill}{rgb}{0.839216,0.152941,0.156863}%
\pgfsetfillcolor{currentfill}%
\pgfsetfillopacity{0.738747}%
\pgfsetlinewidth{1.003750pt}%
\definecolor{currentstroke}{rgb}{0.839216,0.152941,0.156863}%
\pgfsetstrokecolor{currentstroke}%
\pgfsetstrokeopacity{0.738747}%
\pgfsetdash{}{0pt}%
\pgfpathmoveto{\pgfqpoint{1.547941in}{2.113471in}}%
\pgfpathcurveto{\pgfqpoint{1.558029in}{2.113471in}}{\pgfqpoint{1.567704in}{2.117479in}}{\pgfqpoint{1.574837in}{2.124612in}}%
\pgfpathcurveto{\pgfqpoint{1.581970in}{2.131744in}}{\pgfqpoint{1.585978in}{2.141420in}}{\pgfqpoint{1.585978in}{2.151507in}}%
\pgfpathcurveto{\pgfqpoint{1.585978in}{2.161595in}}{\pgfqpoint{1.581970in}{2.171270in}}{\pgfqpoint{1.574837in}{2.178403in}}%
\pgfpathcurveto{\pgfqpoint{1.567704in}{2.185536in}}{\pgfqpoint{1.558029in}{2.189544in}}{\pgfqpoint{1.547941in}{2.189544in}}%
\pgfpathcurveto{\pgfqpoint{1.537854in}{2.189544in}}{\pgfqpoint{1.528178in}{2.185536in}}{\pgfqpoint{1.521046in}{2.178403in}}%
\pgfpathcurveto{\pgfqpoint{1.513913in}{2.171270in}}{\pgfqpoint{1.509905in}{2.161595in}}{\pgfqpoint{1.509905in}{2.151507in}}%
\pgfpathcurveto{\pgfqpoint{1.509905in}{2.141420in}}{\pgfqpoint{1.513913in}{2.131744in}}{\pgfqpoint{1.521046in}{2.124612in}}%
\pgfpathcurveto{\pgfqpoint{1.528178in}{2.117479in}}{\pgfqpoint{1.537854in}{2.113471in}}{\pgfqpoint{1.547941in}{2.113471in}}%
\pgfpathlineto{\pgfqpoint{1.547941in}{2.113471in}}%
\pgfpathclose%
\pgfusepath{stroke,fill}%
\end{pgfscope}%
\begin{pgfscope}%
\pgfpathrectangle{\pgfqpoint{0.708921in}{0.208778in}}{\pgfqpoint{3.800000in}{3.800000in}}%
\pgfusepath{clip}%
\pgfsetbuttcap%
\pgfsetroundjoin%
\definecolor{currentfill}{rgb}{0.839216,0.152941,0.156863}%
\pgfsetfillcolor{currentfill}%
\pgfsetfillopacity{0.791813}%
\pgfsetlinewidth{1.003750pt}%
\definecolor{currentstroke}{rgb}{0.839216,0.152941,0.156863}%
\pgfsetstrokecolor{currentstroke}%
\pgfsetstrokeopacity{0.791813}%
\pgfsetdash{}{0pt}%
\pgfpathmoveto{\pgfqpoint{2.233031in}{2.817079in}}%
\pgfpathcurveto{\pgfqpoint{2.243119in}{2.817079in}}{\pgfqpoint{2.252794in}{2.821087in}}{\pgfqpoint{2.259927in}{2.828220in}}%
\pgfpathcurveto{\pgfqpoint{2.267060in}{2.835353in}}{\pgfqpoint{2.271068in}{2.845028in}}{\pgfqpoint{2.271068in}{2.855116in}}%
\pgfpathcurveto{\pgfqpoint{2.271068in}{2.865203in}}{\pgfqpoint{2.267060in}{2.874879in}}{\pgfqpoint{2.259927in}{2.882011in}}%
\pgfpathcurveto{\pgfqpoint{2.252794in}{2.889144in}}{\pgfqpoint{2.243119in}{2.893152in}}{\pgfqpoint{2.233031in}{2.893152in}}%
\pgfpathcurveto{\pgfqpoint{2.222944in}{2.893152in}}{\pgfqpoint{2.213268in}{2.889144in}}{\pgfqpoint{2.206136in}{2.882011in}}%
\pgfpathcurveto{\pgfqpoint{2.199003in}{2.874879in}}{\pgfqpoint{2.194995in}{2.865203in}}{\pgfqpoint{2.194995in}{2.855116in}}%
\pgfpathcurveto{\pgfqpoint{2.194995in}{2.845028in}}{\pgfqpoint{2.199003in}{2.835353in}}{\pgfqpoint{2.206136in}{2.828220in}}%
\pgfpathcurveto{\pgfqpoint{2.213268in}{2.821087in}}{\pgfqpoint{2.222944in}{2.817079in}}{\pgfqpoint{2.233031in}{2.817079in}}%
\pgfpathlineto{\pgfqpoint{2.233031in}{2.817079in}}%
\pgfpathclose%
\pgfusepath{stroke,fill}%
\end{pgfscope}%
\begin{pgfscope}%
\pgfpathrectangle{\pgfqpoint{0.708921in}{0.208778in}}{\pgfqpoint{3.800000in}{3.800000in}}%
\pgfusepath{clip}%
\pgfsetbuttcap%
\pgfsetroundjoin%
\definecolor{currentfill}{rgb}{0.839216,0.152941,0.156863}%
\pgfsetfillcolor{currentfill}%
\pgfsetfillopacity{0.864233}%
\pgfsetlinewidth{1.003750pt}%
\definecolor{currentstroke}{rgb}{0.839216,0.152941,0.156863}%
\pgfsetstrokecolor{currentstroke}%
\pgfsetstrokeopacity{0.864233}%
\pgfsetdash{}{0pt}%
\pgfpathmoveto{\pgfqpoint{3.158724in}{2.695780in}}%
\pgfpathcurveto{\pgfqpoint{3.168811in}{2.695780in}}{\pgfqpoint{3.178487in}{2.699788in}}{\pgfqpoint{3.185619in}{2.706921in}}%
\pgfpathcurveto{\pgfqpoint{3.192752in}{2.714054in}}{\pgfqpoint{3.196760in}{2.723729in}}{\pgfqpoint{3.196760in}{2.733816in}}%
\pgfpathcurveto{\pgfqpoint{3.196760in}{2.743904in}}{\pgfqpoint{3.192752in}{2.753579in}}{\pgfqpoint{3.185619in}{2.760712in}}%
\pgfpathcurveto{\pgfqpoint{3.178487in}{2.767845in}}{\pgfqpoint{3.168811in}{2.771853in}}{\pgfqpoint{3.158724in}{2.771853in}}%
\pgfpathcurveto{\pgfqpoint{3.148636in}{2.771853in}}{\pgfqpoint{3.138961in}{2.767845in}}{\pgfqpoint{3.131828in}{2.760712in}}%
\pgfpathcurveto{\pgfqpoint{3.124695in}{2.753579in}}{\pgfqpoint{3.120687in}{2.743904in}}{\pgfqpoint{3.120687in}{2.733816in}}%
\pgfpathcurveto{\pgfqpoint{3.120687in}{2.723729in}}{\pgfqpoint{3.124695in}{2.714054in}}{\pgfqpoint{3.131828in}{2.706921in}}%
\pgfpathcurveto{\pgfqpoint{3.138961in}{2.699788in}}{\pgfqpoint{3.148636in}{2.695780in}}{\pgfqpoint{3.158724in}{2.695780in}}%
\pgfpathlineto{\pgfqpoint{3.158724in}{2.695780in}}%
\pgfpathclose%
\pgfusepath{stroke,fill}%
\end{pgfscope}%
\begin{pgfscope}%
\pgfpathrectangle{\pgfqpoint{0.708921in}{0.208778in}}{\pgfqpoint{3.800000in}{3.800000in}}%
\pgfusepath{clip}%
\pgfsetbuttcap%
\pgfsetroundjoin%
\definecolor{currentfill}{rgb}{0.839216,0.152941,0.156863}%
\pgfsetfillcolor{currentfill}%
\pgfsetfillopacity{0.929084}%
\pgfsetlinewidth{1.003750pt}%
\definecolor{currentstroke}{rgb}{0.839216,0.152941,0.156863}%
\pgfsetstrokecolor{currentstroke}%
\pgfsetstrokeopacity{0.929084}%
\pgfsetdash{}{0pt}%
\pgfpathmoveto{\pgfqpoint{3.241263in}{1.809510in}}%
\pgfpathcurveto{\pgfqpoint{3.251350in}{1.809510in}}{\pgfqpoint{3.261026in}{1.813517in}}{\pgfqpoint{3.268159in}{1.820650in}}%
\pgfpathcurveto{\pgfqpoint{3.275291in}{1.827783in}}{\pgfqpoint{3.279299in}{1.837459in}}{\pgfqpoint{3.279299in}{1.847546in}}%
\pgfpathcurveto{\pgfqpoint{3.279299in}{1.857633in}}{\pgfqpoint{3.275291in}{1.867309in}}{\pgfqpoint{3.268159in}{1.874442in}}%
\pgfpathcurveto{\pgfqpoint{3.261026in}{1.881574in}}{\pgfqpoint{3.251350in}{1.885582in}}{\pgfqpoint{3.241263in}{1.885582in}}%
\pgfpathcurveto{\pgfqpoint{3.231175in}{1.885582in}}{\pgfqpoint{3.221500in}{1.881574in}}{\pgfqpoint{3.214367in}{1.874442in}}%
\pgfpathcurveto{\pgfqpoint{3.207234in}{1.867309in}}{\pgfqpoint{3.203227in}{1.857633in}}{\pgfqpoint{3.203227in}{1.847546in}}%
\pgfpathcurveto{\pgfqpoint{3.203227in}{1.837459in}}{\pgfqpoint{3.207234in}{1.827783in}}{\pgfqpoint{3.214367in}{1.820650in}}%
\pgfpathcurveto{\pgfqpoint{3.221500in}{1.813517in}}{\pgfqpoint{3.231175in}{1.809510in}}{\pgfqpoint{3.241263in}{1.809510in}}%
\pgfpathlineto{\pgfqpoint{3.241263in}{1.809510in}}%
\pgfpathclose%
\pgfusepath{stroke,fill}%
\end{pgfscope}%
\begin{pgfscope}%
\pgfpathrectangle{\pgfqpoint{0.708921in}{0.208778in}}{\pgfqpoint{3.800000in}{3.800000in}}%
\pgfusepath{clip}%
\pgfsetbuttcap%
\pgfsetroundjoin%
\definecolor{currentfill}{rgb}{0.839216,0.152941,0.156863}%
\pgfsetfillcolor{currentfill}%
\pgfsetlinewidth{1.003750pt}%
\definecolor{currentstroke}{rgb}{0.839216,0.152941,0.156863}%
\pgfsetstrokecolor{currentstroke}%
\pgfsetdash{}{0pt}%
\pgfpathmoveto{\pgfqpoint{3.074505in}{2.112077in}}%
\pgfpathcurveto{\pgfqpoint{3.084592in}{2.112077in}}{\pgfqpoint{3.094268in}{2.116085in}}{\pgfqpoint{3.101401in}{2.123218in}}%
\pgfpathcurveto{\pgfqpoint{3.108533in}{2.130350in}}{\pgfqpoint{3.112541in}{2.140026in}}{\pgfqpoint{3.112541in}{2.150113in}}%
\pgfpathcurveto{\pgfqpoint{3.112541in}{2.160201in}}{\pgfqpoint{3.108533in}{2.169876in}}{\pgfqpoint{3.101401in}{2.177009in}}%
\pgfpathcurveto{\pgfqpoint{3.094268in}{2.184142in}}{\pgfqpoint{3.084592in}{2.188150in}}{\pgfqpoint{3.074505in}{2.188150in}}%
\pgfpathcurveto{\pgfqpoint{3.064417in}{2.188150in}}{\pgfqpoint{3.054742in}{2.184142in}}{\pgfqpoint{3.047609in}{2.177009in}}%
\pgfpathcurveto{\pgfqpoint{3.040476in}{2.169876in}}{\pgfqpoint{3.036469in}{2.160201in}}{\pgfqpoint{3.036469in}{2.150113in}}%
\pgfpathcurveto{\pgfqpoint{3.036469in}{2.140026in}}{\pgfqpoint{3.040476in}{2.130350in}}{\pgfqpoint{3.047609in}{2.123218in}}%
\pgfpathcurveto{\pgfqpoint{3.054742in}{2.116085in}}{\pgfqpoint{3.064417in}{2.112077in}}{\pgfqpoint{3.074505in}{2.112077in}}%
\pgfpathlineto{\pgfqpoint{3.074505in}{2.112077in}}%
\pgfpathclose%
\pgfusepath{stroke,fill}%
\end{pgfscope}%
\begin{pgfscope}%
\pgfpathrectangle{\pgfqpoint{0.708921in}{0.208778in}}{\pgfqpoint{3.800000in}{3.800000in}}%
\pgfusepath{clip}%
\pgfsetbuttcap%
\pgfsetroundjoin%
\definecolor{currentfill}{rgb}{0.071067,0.258424,0.071067}%
\pgfsetfillcolor{currentfill}%
\pgfsetlinewidth{0.000000pt}%
\definecolor{currentstroke}{rgb}{0.000000,0.000000,0.000000}%
\pgfsetstrokecolor{currentstroke}%
\pgfsetdash{}{0pt}%
\pgfpathmoveto{\pgfqpoint{1.324860in}{1.516856in}}%
\pgfpathlineto{\pgfqpoint{1.190909in}{1.711315in}}%
\pgfpathlineto{\pgfqpoint{1.191706in}{1.609418in}}%
\pgfpathlineto{\pgfqpoint{1.324860in}{1.516856in}}%
\pgfpathclose%
\pgfusepath{fill}%
\end{pgfscope}%
\begin{pgfscope}%
\pgfpathrectangle{\pgfqpoint{0.708921in}{0.208778in}}{\pgfqpoint{3.800000in}{3.800000in}}%
\pgfusepath{clip}%
\pgfsetbuttcap%
\pgfsetroundjoin%
\definecolor{currentfill}{rgb}{0.071067,0.258424,0.071067}%
\pgfsetfillcolor{currentfill}%
\pgfsetlinewidth{0.000000pt}%
\definecolor{currentstroke}{rgb}{0.000000,0.000000,0.000000}%
\pgfsetstrokecolor{currentstroke}%
\pgfsetdash{}{0pt}%
\pgfpathmoveto{\pgfqpoint{4.129635in}{1.711315in}}%
\pgfpathlineto{\pgfqpoint{3.995684in}{1.516856in}}%
\pgfpathlineto{\pgfqpoint{4.128838in}{1.609418in}}%
\pgfpathlineto{\pgfqpoint{4.129635in}{1.711315in}}%
\pgfpathclose%
\pgfusepath{fill}%
\end{pgfscope}%
\begin{pgfscope}%
\pgfpathrectangle{\pgfqpoint{0.708921in}{0.208778in}}{\pgfqpoint{3.800000in}{3.800000in}}%
\pgfusepath{clip}%
\pgfsetbuttcap%
\pgfsetroundjoin%
\definecolor{currentfill}{rgb}{0.128601,0.467641,0.128601}%
\pgfsetfillcolor{currentfill}%
\pgfsetlinewidth{0.000000pt}%
\definecolor{currentstroke}{rgb}{0.000000,0.000000,0.000000}%
\pgfsetstrokecolor{currentstroke}%
\pgfsetdash{}{0pt}%
\pgfpathmoveto{\pgfqpoint{2.526320in}{3.661884in}}%
\pgfpathlineto{\pgfqpoint{2.794224in}{3.661884in}}%
\pgfpathlineto{\pgfqpoint{2.660272in}{3.754163in}}%
\pgfpathlineto{\pgfqpoint{2.526320in}{3.661884in}}%
\pgfpathclose%
\pgfusepath{fill}%
\end{pgfscope}%
\begin{pgfscope}%
\pgfpathrectangle{\pgfqpoint{0.708921in}{0.208778in}}{\pgfqpoint{3.800000in}{3.800000in}}%
\pgfusepath{clip}%
\pgfsetbuttcap%
\pgfsetroundjoin%
\definecolor{currentfill}{rgb}{0.067488,0.245410,0.067488}%
\pgfsetfillcolor{currentfill}%
\pgfsetlinewidth{0.000000pt}%
\definecolor{currentstroke}{rgb}{0.000000,0.000000,0.000000}%
\pgfsetstrokecolor{currentstroke}%
\pgfsetdash{}{0pt}%
\pgfpathmoveto{\pgfqpoint{1.497851in}{1.419312in}}%
\pgfpathlineto{\pgfqpoint{1.338610in}{1.620589in}}%
\pgfpathlineto{\pgfqpoint{1.324860in}{1.516856in}}%
\pgfpathlineto{\pgfqpoint{1.497851in}{1.419312in}}%
\pgfpathclose%
\pgfusepath{fill}%
\end{pgfscope}%
\begin{pgfscope}%
\pgfpathrectangle{\pgfqpoint{0.708921in}{0.208778in}}{\pgfqpoint{3.800000in}{3.800000in}}%
\pgfusepath{clip}%
\pgfsetbuttcap%
\pgfsetroundjoin%
\definecolor{currentfill}{rgb}{0.067488,0.245410,0.067488}%
\pgfsetfillcolor{currentfill}%
\pgfsetlinewidth{0.000000pt}%
\definecolor{currentstroke}{rgb}{0.000000,0.000000,0.000000}%
\pgfsetstrokecolor{currentstroke}%
\pgfsetdash{}{0pt}%
\pgfpathmoveto{\pgfqpoint{3.995684in}{1.516856in}}%
\pgfpathlineto{\pgfqpoint{3.981934in}{1.620589in}}%
\pgfpathlineto{\pgfqpoint{3.822693in}{1.419312in}}%
\pgfpathlineto{\pgfqpoint{3.995684in}{1.516856in}}%
\pgfpathclose%
\pgfusepath{fill}%
\end{pgfscope}%
\begin{pgfscope}%
\pgfpathrectangle{\pgfqpoint{0.708921in}{0.208778in}}{\pgfqpoint{3.800000in}{3.800000in}}%
\pgfusepath{clip}%
\pgfsetbuttcap%
\pgfsetroundjoin%
\definecolor{currentfill}{rgb}{0.069492,0.252698,0.069492}%
\pgfsetfillcolor{currentfill}%
\pgfsetlinewidth{0.000000pt}%
\definecolor{currentstroke}{rgb}{0.000000,0.000000,0.000000}%
\pgfsetstrokecolor{currentstroke}%
\pgfsetdash{}{0pt}%
\pgfpathmoveto{\pgfqpoint{1.190909in}{1.711315in}}%
\pgfpathlineto{\pgfqpoint{1.324860in}{1.516856in}}%
\pgfpathlineto{\pgfqpoint{1.321794in}{2.212598in}}%
\pgfpathlineto{\pgfqpoint{1.190909in}{1.711315in}}%
\pgfpathclose%
\pgfusepath{fill}%
\end{pgfscope}%
\begin{pgfscope}%
\pgfpathrectangle{\pgfqpoint{0.708921in}{0.208778in}}{\pgfqpoint{3.800000in}{3.800000in}}%
\pgfusepath{clip}%
\pgfsetbuttcap%
\pgfsetroundjoin%
\definecolor{currentfill}{rgb}{0.069492,0.252698,0.069492}%
\pgfsetfillcolor{currentfill}%
\pgfsetlinewidth{0.000000pt}%
\definecolor{currentstroke}{rgb}{0.000000,0.000000,0.000000}%
\pgfsetstrokecolor{currentstroke}%
\pgfsetdash{}{0pt}%
\pgfpathmoveto{\pgfqpoint{4.129635in}{1.711315in}}%
\pgfpathlineto{\pgfqpoint{3.998750in}{2.212598in}}%
\pgfpathlineto{\pgfqpoint{3.995684in}{1.516856in}}%
\pgfpathlineto{\pgfqpoint{4.129635in}{1.711315in}}%
\pgfpathclose%
\pgfusepath{fill}%
\end{pgfscope}%
\begin{pgfscope}%
\pgfpathrectangle{\pgfqpoint{0.708921in}{0.208778in}}{\pgfqpoint{3.800000in}{3.800000in}}%
\pgfusepath{clip}%
\pgfsetbuttcap%
\pgfsetroundjoin%
\definecolor{currentfill}{rgb}{0.099716,0.362602,0.099716}%
\pgfsetfillcolor{currentfill}%
\pgfsetlinewidth{0.000000pt}%
\definecolor{currentstroke}{rgb}{0.000000,0.000000,0.000000}%
\pgfsetstrokecolor{currentstroke}%
\pgfsetdash{}{0pt}%
\pgfpathmoveto{\pgfqpoint{1.321794in}{2.212598in}}%
\pgfpathlineto{\pgfqpoint{1.324860in}{1.516856in}}%
\pgfpathlineto{\pgfqpoint{1.338610in}{1.620589in}}%
\pgfpathlineto{\pgfqpoint{1.321794in}{2.212598in}}%
\pgfpathclose%
\pgfusepath{fill}%
\end{pgfscope}%
\begin{pgfscope}%
\pgfpathrectangle{\pgfqpoint{0.708921in}{0.208778in}}{\pgfqpoint{3.800000in}{3.800000in}}%
\pgfusepath{clip}%
\pgfsetbuttcap%
\pgfsetroundjoin%
\definecolor{currentfill}{rgb}{0.099716,0.362602,0.099716}%
\pgfsetfillcolor{currentfill}%
\pgfsetlinewidth{0.000000pt}%
\definecolor{currentstroke}{rgb}{0.000000,0.000000,0.000000}%
\pgfsetstrokecolor{currentstroke}%
\pgfsetdash{}{0pt}%
\pgfpathmoveto{\pgfqpoint{3.981934in}{1.620589in}}%
\pgfpathlineto{\pgfqpoint{3.995684in}{1.516856in}}%
\pgfpathlineto{\pgfqpoint{3.998750in}{2.212598in}}%
\pgfpathlineto{\pgfqpoint{3.981934in}{1.620589in}}%
\pgfpathclose%
\pgfusepath{fill}%
\end{pgfscope}%
\begin{pgfscope}%
\pgfpathrectangle{\pgfqpoint{0.708921in}{0.208778in}}{\pgfqpoint{3.800000in}{3.800000in}}%
\pgfusepath{clip}%
\pgfsetbuttcap%
\pgfsetroundjoin%
\definecolor{currentfill}{rgb}{0.063840,0.232145,0.063840}%
\pgfsetfillcolor{currentfill}%
\pgfsetlinewidth{0.000000pt}%
\definecolor{currentstroke}{rgb}{0.000000,0.000000,0.000000}%
\pgfsetstrokecolor{currentstroke}%
\pgfsetdash{}{0pt}%
\pgfpathmoveto{\pgfqpoint{1.718862in}{1.323116in}}%
\pgfpathlineto{\pgfqpoint{1.534653in}{1.525226in}}%
\pgfpathlineto{\pgfqpoint{1.497851in}{1.419312in}}%
\pgfpathlineto{\pgfqpoint{1.718862in}{1.323116in}}%
\pgfpathclose%
\pgfusepath{fill}%
\end{pgfscope}%
\begin{pgfscope}%
\pgfpathrectangle{\pgfqpoint{0.708921in}{0.208778in}}{\pgfqpoint{3.800000in}{3.800000in}}%
\pgfusepath{clip}%
\pgfsetbuttcap%
\pgfsetroundjoin%
\definecolor{currentfill}{rgb}{0.063840,0.232145,0.063840}%
\pgfsetfillcolor{currentfill}%
\pgfsetlinewidth{0.000000pt}%
\definecolor{currentstroke}{rgb}{0.000000,0.000000,0.000000}%
\pgfsetstrokecolor{currentstroke}%
\pgfsetdash{}{0pt}%
\pgfpathmoveto{\pgfqpoint{3.822693in}{1.419312in}}%
\pgfpathlineto{\pgfqpoint{3.785891in}{1.525226in}}%
\pgfpathlineto{\pgfqpoint{3.601682in}{1.323116in}}%
\pgfpathlineto{\pgfqpoint{3.822693in}{1.419312in}}%
\pgfpathclose%
\pgfusepath{fill}%
\end{pgfscope}%
\begin{pgfscope}%
\pgfpathrectangle{\pgfqpoint{0.708921in}{0.208778in}}{\pgfqpoint{3.800000in}{3.800000in}}%
\pgfusepath{clip}%
\pgfsetbuttcap%
\pgfsetroundjoin%
\definecolor{currentfill}{rgb}{0.116321,0.422987,0.116321}%
\pgfsetfillcolor{currentfill}%
\pgfsetlinewidth{0.000000pt}%
\definecolor{currentstroke}{rgb}{0.000000,0.000000,0.000000}%
\pgfsetstrokecolor{currentstroke}%
\pgfsetdash{}{0pt}%
\pgfpathmoveto{\pgfqpoint{2.794224in}{3.661884in}}%
\pgfpathlineto{\pgfqpoint{2.526320in}{3.661884in}}%
\pgfpathlineto{\pgfqpoint{2.468151in}{2.991915in}}%
\pgfpathlineto{\pgfqpoint{2.794224in}{3.661884in}}%
\pgfpathclose%
\pgfusepath{fill}%
\end{pgfscope}%
\begin{pgfscope}%
\pgfpathrectangle{\pgfqpoint{0.708921in}{0.208778in}}{\pgfqpoint{3.800000in}{3.800000in}}%
\pgfusepath{clip}%
\pgfsetbuttcap%
\pgfsetroundjoin%
\definecolor{currentfill}{rgb}{0.067061,0.243857,0.067061}%
\pgfsetfillcolor{currentfill}%
\pgfsetlinewidth{0.000000pt}%
\definecolor{currentstroke}{rgb}{0.000000,0.000000,0.000000}%
\pgfsetstrokecolor{currentstroke}%
\pgfsetdash{}{0pt}%
\pgfpathmoveto{\pgfqpoint{1.338610in}{1.620589in}}%
\pgfpathlineto{\pgfqpoint{1.497851in}{1.419312in}}%
\pgfpathlineto{\pgfqpoint{1.547135in}{2.167772in}}%
\pgfpathlineto{\pgfqpoint{1.338610in}{1.620589in}}%
\pgfpathclose%
\pgfusepath{fill}%
\end{pgfscope}%
\begin{pgfscope}%
\pgfpathrectangle{\pgfqpoint{0.708921in}{0.208778in}}{\pgfqpoint{3.800000in}{3.800000in}}%
\pgfusepath{clip}%
\pgfsetbuttcap%
\pgfsetroundjoin%
\definecolor{currentfill}{rgb}{0.067061,0.243857,0.067061}%
\pgfsetfillcolor{currentfill}%
\pgfsetlinewidth{0.000000pt}%
\definecolor{currentstroke}{rgb}{0.000000,0.000000,0.000000}%
\pgfsetstrokecolor{currentstroke}%
\pgfsetdash{}{0pt}%
\pgfpathmoveto{\pgfqpoint{3.773409in}{2.167772in}}%
\pgfpathlineto{\pgfqpoint{3.822693in}{1.419312in}}%
\pgfpathlineto{\pgfqpoint{3.981934in}{1.620589in}}%
\pgfpathlineto{\pgfqpoint{3.773409in}{2.167772in}}%
\pgfpathclose%
\pgfusepath{fill}%
\end{pgfscope}%
\begin{pgfscope}%
\pgfpathrectangle{\pgfqpoint{0.708921in}{0.208778in}}{\pgfqpoint{3.800000in}{3.800000in}}%
\pgfusepath{clip}%
\pgfsetbuttcap%
\pgfsetroundjoin%
\definecolor{currentfill}{rgb}{0.095351,0.346729,0.095351}%
\pgfsetfillcolor{currentfill}%
\pgfsetlinewidth{0.000000pt}%
\definecolor{currentstroke}{rgb}{0.000000,0.000000,0.000000}%
\pgfsetstrokecolor{currentstroke}%
\pgfsetdash{}{0pt}%
\pgfpathmoveto{\pgfqpoint{1.547135in}{2.167772in}}%
\pgfpathlineto{\pgfqpoint{1.497851in}{1.419312in}}%
\pgfpathlineto{\pgfqpoint{1.534653in}{1.525226in}}%
\pgfpathlineto{\pgfqpoint{1.547135in}{2.167772in}}%
\pgfpathclose%
\pgfusepath{fill}%
\end{pgfscope}%
\begin{pgfscope}%
\pgfpathrectangle{\pgfqpoint{0.708921in}{0.208778in}}{\pgfqpoint{3.800000in}{3.800000in}}%
\pgfusepath{clip}%
\pgfsetbuttcap%
\pgfsetroundjoin%
\definecolor{currentfill}{rgb}{0.095351,0.346729,0.095351}%
\pgfsetfillcolor{currentfill}%
\pgfsetlinewidth{0.000000pt}%
\definecolor{currentstroke}{rgb}{0.000000,0.000000,0.000000}%
\pgfsetstrokecolor{currentstroke}%
\pgfsetdash{}{0pt}%
\pgfpathmoveto{\pgfqpoint{3.785891in}{1.525226in}}%
\pgfpathlineto{\pgfqpoint{3.822693in}{1.419312in}}%
\pgfpathlineto{\pgfqpoint{3.773409in}{2.167772in}}%
\pgfpathlineto{\pgfqpoint{3.785891in}{1.525226in}}%
\pgfpathclose%
\pgfusepath{fill}%
\end{pgfscope}%
\begin{pgfscope}%
\pgfpathrectangle{\pgfqpoint{0.708921in}{0.208778in}}{\pgfqpoint{3.800000in}{3.800000in}}%
\pgfusepath{clip}%
\pgfsetbuttcap%
\pgfsetroundjoin%
\definecolor{currentfill}{rgb}{0.060435,0.219763,0.060435}%
\pgfsetfillcolor{currentfill}%
\pgfsetlinewidth{0.000000pt}%
\definecolor{currentstroke}{rgb}{0.000000,0.000000,0.000000}%
\pgfsetstrokecolor{currentstroke}%
\pgfsetdash{}{0pt}%
\pgfpathmoveto{\pgfqpoint{1.718862in}{1.323116in}}%
\pgfpathlineto{\pgfqpoint{1.991967in}{1.238733in}}%
\pgfpathlineto{\pgfqpoint{1.788977in}{1.433246in}}%
\pgfpathlineto{\pgfqpoint{1.718862in}{1.323116in}}%
\pgfpathclose%
\pgfusepath{fill}%
\end{pgfscope}%
\begin{pgfscope}%
\pgfpathrectangle{\pgfqpoint{0.708921in}{0.208778in}}{\pgfqpoint{3.800000in}{3.800000in}}%
\pgfusepath{clip}%
\pgfsetbuttcap%
\pgfsetroundjoin%
\definecolor{currentfill}{rgb}{0.060435,0.219763,0.060435}%
\pgfsetfillcolor{currentfill}%
\pgfsetlinewidth{0.000000pt}%
\definecolor{currentstroke}{rgb}{0.000000,0.000000,0.000000}%
\pgfsetstrokecolor{currentstroke}%
\pgfsetdash{}{0pt}%
\pgfpathmoveto{\pgfqpoint{3.531567in}{1.433246in}}%
\pgfpathlineto{\pgfqpoint{3.328577in}{1.238733in}}%
\pgfpathlineto{\pgfqpoint{3.601682in}{1.323116in}}%
\pgfpathlineto{\pgfqpoint{3.531567in}{1.433246in}}%
\pgfpathclose%
\pgfusepath{fill}%
\end{pgfscope}%
\begin{pgfscope}%
\pgfpathrectangle{\pgfqpoint{0.708921in}{0.208778in}}{\pgfqpoint{3.800000in}{3.800000in}}%
\pgfusepath{clip}%
\pgfsetbuttcap%
\pgfsetroundjoin%
\definecolor{currentfill}{rgb}{0.074506,0.270932,0.074506}%
\pgfsetfillcolor{currentfill}%
\pgfsetlinewidth{0.000000pt}%
\definecolor{currentstroke}{rgb}{0.000000,0.000000,0.000000}%
\pgfsetstrokecolor{currentstroke}%
\pgfsetdash{}{0pt}%
\pgfpathmoveto{\pgfqpoint{1.321794in}{2.212598in}}%
\pgfpathlineto{\pgfqpoint{1.338610in}{1.620589in}}%
\pgfpathlineto{\pgfqpoint{1.547135in}{2.167772in}}%
\pgfpathlineto{\pgfqpoint{1.321794in}{2.212598in}}%
\pgfpathclose%
\pgfusepath{fill}%
\end{pgfscope}%
\begin{pgfscope}%
\pgfpathrectangle{\pgfqpoint{0.708921in}{0.208778in}}{\pgfqpoint{3.800000in}{3.800000in}}%
\pgfusepath{clip}%
\pgfsetbuttcap%
\pgfsetroundjoin%
\definecolor{currentfill}{rgb}{0.074506,0.270932,0.074506}%
\pgfsetfillcolor{currentfill}%
\pgfsetlinewidth{0.000000pt}%
\definecolor{currentstroke}{rgb}{0.000000,0.000000,0.000000}%
\pgfsetstrokecolor{currentstroke}%
\pgfsetdash{}{0pt}%
\pgfpathmoveto{\pgfqpoint{3.773409in}{2.167772in}}%
\pgfpathlineto{\pgfqpoint{3.981934in}{1.620589in}}%
\pgfpathlineto{\pgfqpoint{3.998750in}{2.212598in}}%
\pgfpathlineto{\pgfqpoint{3.773409in}{2.167772in}}%
\pgfpathclose%
\pgfusepath{fill}%
\end{pgfscope}%
\begin{pgfscope}%
\pgfpathrectangle{\pgfqpoint{0.708921in}{0.208778in}}{\pgfqpoint{3.800000in}{3.800000in}}%
\pgfusepath{clip}%
\pgfsetbuttcap%
\pgfsetroundjoin%
\definecolor{currentfill}{rgb}{0.116785,0.424671,0.116785}%
\pgfsetfillcolor{currentfill}%
\pgfsetlinewidth{0.000000pt}%
\definecolor{currentstroke}{rgb}{0.000000,0.000000,0.000000}%
\pgfsetstrokecolor{currentstroke}%
\pgfsetdash{}{0pt}%
\pgfpathmoveto{\pgfqpoint{3.330448in}{3.189189in}}%
\pgfpathlineto{\pgfqpoint{2.794224in}{3.661884in}}%
\pgfpathlineto{\pgfqpoint{2.852393in}{2.991915in}}%
\pgfpathlineto{\pgfqpoint{3.330448in}{3.189189in}}%
\pgfpathclose%
\pgfusepath{fill}%
\end{pgfscope}%
\begin{pgfscope}%
\pgfpathrectangle{\pgfqpoint{0.708921in}{0.208778in}}{\pgfqpoint{3.800000in}{3.800000in}}%
\pgfusepath{clip}%
\pgfsetbuttcap%
\pgfsetroundjoin%
\definecolor{currentfill}{rgb}{0.116785,0.424671,0.116785}%
\pgfsetfillcolor{currentfill}%
\pgfsetlinewidth{0.000000pt}%
\definecolor{currentstroke}{rgb}{0.000000,0.000000,0.000000}%
\pgfsetstrokecolor{currentstroke}%
\pgfsetdash{}{0pt}%
\pgfpathmoveto{\pgfqpoint{2.468151in}{2.991915in}}%
\pgfpathlineto{\pgfqpoint{2.526320in}{3.661884in}}%
\pgfpathlineto{\pgfqpoint{1.990096in}{3.189189in}}%
\pgfpathlineto{\pgfqpoint{2.468151in}{2.991915in}}%
\pgfpathclose%
\pgfusepath{fill}%
\end{pgfscope}%
\begin{pgfscope}%
\pgfpathrectangle{\pgfqpoint{0.708921in}{0.208778in}}{\pgfqpoint{3.800000in}{3.800000in}}%
\pgfusepath{clip}%
\pgfsetbuttcap%
\pgfsetroundjoin%
\definecolor{currentfill}{rgb}{0.064867,0.235879,0.064867}%
\pgfsetfillcolor{currentfill}%
\pgfsetlinewidth{0.000000pt}%
\definecolor{currentstroke}{rgb}{0.000000,0.000000,0.000000}%
\pgfsetstrokecolor{currentstroke}%
\pgfsetdash{}{0pt}%
\pgfpathmoveto{\pgfqpoint{1.534653in}{1.525226in}}%
\pgfpathlineto{\pgfqpoint{1.718862in}{1.323116in}}%
\pgfpathlineto{\pgfqpoint{1.849213in}{2.125336in}}%
\pgfpathlineto{\pgfqpoint{1.534653in}{1.525226in}}%
\pgfpathclose%
\pgfusepath{fill}%
\end{pgfscope}%
\begin{pgfscope}%
\pgfpathrectangle{\pgfqpoint{0.708921in}{0.208778in}}{\pgfqpoint{3.800000in}{3.800000in}}%
\pgfusepath{clip}%
\pgfsetbuttcap%
\pgfsetroundjoin%
\definecolor{currentfill}{rgb}{0.064867,0.235879,0.064867}%
\pgfsetfillcolor{currentfill}%
\pgfsetlinewidth{0.000000pt}%
\definecolor{currentstroke}{rgb}{0.000000,0.000000,0.000000}%
\pgfsetstrokecolor{currentstroke}%
\pgfsetdash{}{0pt}%
\pgfpathmoveto{\pgfqpoint{3.471331in}{2.125336in}}%
\pgfpathlineto{\pgfqpoint{3.601682in}{1.323116in}}%
\pgfpathlineto{\pgfqpoint{3.785891in}{1.525226in}}%
\pgfpathlineto{\pgfqpoint{3.471331in}{2.125336in}}%
\pgfpathclose%
\pgfusepath{fill}%
\end{pgfscope}%
\begin{pgfscope}%
\pgfpathrectangle{\pgfqpoint{0.708921in}{0.208778in}}{\pgfqpoint{3.800000in}{3.800000in}}%
\pgfusepath{clip}%
\pgfsetbuttcap%
\pgfsetroundjoin%
\definecolor{currentfill}{rgb}{0.057724,0.209904,0.057724}%
\pgfsetfillcolor{currentfill}%
\pgfsetlinewidth{0.000000pt}%
\definecolor{currentstroke}{rgb}{0.000000,0.000000,0.000000}%
\pgfsetstrokecolor{currentstroke}%
\pgfsetdash{}{0pt}%
\pgfpathmoveto{\pgfqpoint{2.104350in}{1.357725in}}%
\pgfpathlineto{\pgfqpoint{1.991967in}{1.238733in}}%
\pgfpathlineto{\pgfqpoint{2.311881in}{1.179689in}}%
\pgfpathlineto{\pgfqpoint{2.104350in}{1.357725in}}%
\pgfpathclose%
\pgfusepath{fill}%
\end{pgfscope}%
\begin{pgfscope}%
\pgfpathrectangle{\pgfqpoint{0.708921in}{0.208778in}}{\pgfqpoint{3.800000in}{3.800000in}}%
\pgfusepath{clip}%
\pgfsetbuttcap%
\pgfsetroundjoin%
\definecolor{currentfill}{rgb}{0.057724,0.209904,0.057724}%
\pgfsetfillcolor{currentfill}%
\pgfsetlinewidth{0.000000pt}%
\definecolor{currentstroke}{rgb}{0.000000,0.000000,0.000000}%
\pgfsetstrokecolor{currentstroke}%
\pgfsetdash{}{0pt}%
\pgfpathmoveto{\pgfqpoint{3.008663in}{1.179689in}}%
\pgfpathlineto{\pgfqpoint{3.328577in}{1.238733in}}%
\pgfpathlineto{\pgfqpoint{3.216194in}{1.357725in}}%
\pgfpathlineto{\pgfqpoint{3.008663in}{1.179689in}}%
\pgfpathclose%
\pgfusepath{fill}%
\end{pgfscope}%
\begin{pgfscope}%
\pgfpathrectangle{\pgfqpoint{0.708921in}{0.208778in}}{\pgfqpoint{3.800000in}{3.800000in}}%
\pgfusepath{clip}%
\pgfsetbuttcap%
\pgfsetroundjoin%
\definecolor{currentfill}{rgb}{0.086498,0.314539,0.086498}%
\pgfsetfillcolor{currentfill}%
\pgfsetlinewidth{0.000000pt}%
\definecolor{currentstroke}{rgb}{0.000000,0.000000,0.000000}%
\pgfsetstrokecolor{currentstroke}%
\pgfsetdash{}{0pt}%
\pgfpathmoveto{\pgfqpoint{3.998750in}{2.212598in}}%
\pgfpathlineto{\pgfqpoint{3.881863in}{2.449323in}}%
\pgfpathlineto{\pgfqpoint{3.773409in}{2.167772in}}%
\pgfpathlineto{\pgfqpoint{3.998750in}{2.212598in}}%
\pgfpathclose%
\pgfusepath{fill}%
\end{pgfscope}%
\begin{pgfscope}%
\pgfpathrectangle{\pgfqpoint{0.708921in}{0.208778in}}{\pgfqpoint{3.800000in}{3.800000in}}%
\pgfusepath{clip}%
\pgfsetbuttcap%
\pgfsetroundjoin%
\definecolor{currentfill}{rgb}{0.086498,0.314539,0.086498}%
\pgfsetfillcolor{currentfill}%
\pgfsetlinewidth{0.000000pt}%
\definecolor{currentstroke}{rgb}{0.000000,0.000000,0.000000}%
\pgfsetstrokecolor{currentstroke}%
\pgfsetdash{}{0pt}%
\pgfpathmoveto{\pgfqpoint{1.547135in}{2.167772in}}%
\pgfpathlineto{\pgfqpoint{1.438681in}{2.449323in}}%
\pgfpathlineto{\pgfqpoint{1.321794in}{2.212598in}}%
\pgfpathlineto{\pgfqpoint{1.547135in}{2.167772in}}%
\pgfpathclose%
\pgfusepath{fill}%
\end{pgfscope}%
\begin{pgfscope}%
\pgfpathrectangle{\pgfqpoint{0.708921in}{0.208778in}}{\pgfqpoint{3.800000in}{3.800000in}}%
\pgfusepath{clip}%
\pgfsetbuttcap%
\pgfsetroundjoin%
\definecolor{currentfill}{rgb}{0.090812,0.330224,0.090812}%
\pgfsetfillcolor{currentfill}%
\pgfsetlinewidth{0.000000pt}%
\definecolor{currentstroke}{rgb}{0.000000,0.000000,0.000000}%
\pgfsetstrokecolor{currentstroke}%
\pgfsetdash{}{0pt}%
\pgfpathmoveto{\pgfqpoint{1.849213in}{2.125336in}}%
\pgfpathlineto{\pgfqpoint{1.718862in}{1.323116in}}%
\pgfpathlineto{\pgfqpoint{1.788977in}{1.433246in}}%
\pgfpathlineto{\pgfqpoint{1.849213in}{2.125336in}}%
\pgfpathclose%
\pgfusepath{fill}%
\end{pgfscope}%
\begin{pgfscope}%
\pgfpathrectangle{\pgfqpoint{0.708921in}{0.208778in}}{\pgfqpoint{3.800000in}{3.800000in}}%
\pgfusepath{clip}%
\pgfsetbuttcap%
\pgfsetroundjoin%
\definecolor{currentfill}{rgb}{0.090812,0.330224,0.090812}%
\pgfsetfillcolor{currentfill}%
\pgfsetlinewidth{0.000000pt}%
\definecolor{currentstroke}{rgb}{0.000000,0.000000,0.000000}%
\pgfsetstrokecolor{currentstroke}%
\pgfsetdash{}{0pt}%
\pgfpathmoveto{\pgfqpoint{3.531567in}{1.433246in}}%
\pgfpathlineto{\pgfqpoint{3.601682in}{1.323116in}}%
\pgfpathlineto{\pgfqpoint{3.471331in}{2.125336in}}%
\pgfpathlineto{\pgfqpoint{3.531567in}{1.433246in}}%
\pgfpathclose%
\pgfusepath{fill}%
\end{pgfscope}%
\begin{pgfscope}%
\pgfpathrectangle{\pgfqpoint{0.708921in}{0.208778in}}{\pgfqpoint{3.800000in}{3.800000in}}%
\pgfusepath{clip}%
\pgfsetbuttcap%
\pgfsetroundjoin%
\definecolor{currentfill}{rgb}{0.116321,0.422987,0.116321}%
\pgfsetfillcolor{currentfill}%
\pgfsetlinewidth{0.000000pt}%
\definecolor{currentstroke}{rgb}{0.000000,0.000000,0.000000}%
\pgfsetstrokecolor{currentstroke}%
\pgfsetdash{}{0pt}%
\pgfpathmoveto{\pgfqpoint{2.468151in}{2.991915in}}%
\pgfpathlineto{\pgfqpoint{2.852393in}{2.991915in}}%
\pgfpathlineto{\pgfqpoint{2.794224in}{3.661884in}}%
\pgfpathlineto{\pgfqpoint{2.468151in}{2.991915in}}%
\pgfpathclose%
\pgfusepath{fill}%
\end{pgfscope}%
\begin{pgfscope}%
\pgfpathrectangle{\pgfqpoint{0.708921in}{0.208778in}}{\pgfqpoint{3.800000in}{3.800000in}}%
\pgfusepath{clip}%
\pgfsetbuttcap%
\pgfsetroundjoin%
\definecolor{currentfill}{rgb}{0.056200,0.204363,0.056200}%
\pgfsetfillcolor{currentfill}%
\pgfsetlinewidth{0.000000pt}%
\definecolor{currentstroke}{rgb}{0.000000,0.000000,0.000000}%
\pgfsetstrokecolor{currentstroke}%
\pgfsetdash{}{0pt}%
\pgfpathmoveto{\pgfqpoint{2.660272in}{1.158306in}}%
\pgfpathlineto{\pgfqpoint{2.468612in}{1.314303in}}%
\pgfpathlineto{\pgfqpoint{2.311881in}{1.179689in}}%
\pgfpathlineto{\pgfqpoint{2.660272in}{1.158306in}}%
\pgfpathclose%
\pgfusepath{fill}%
\end{pgfscope}%
\begin{pgfscope}%
\pgfpathrectangle{\pgfqpoint{0.708921in}{0.208778in}}{\pgfqpoint{3.800000in}{3.800000in}}%
\pgfusepath{clip}%
\pgfsetbuttcap%
\pgfsetroundjoin%
\definecolor{currentfill}{rgb}{0.056200,0.204363,0.056200}%
\pgfsetfillcolor{currentfill}%
\pgfsetlinewidth{0.000000pt}%
\definecolor{currentstroke}{rgb}{0.000000,0.000000,0.000000}%
\pgfsetstrokecolor{currentstroke}%
\pgfsetdash{}{0pt}%
\pgfpathmoveto{\pgfqpoint{3.008663in}{1.179689in}}%
\pgfpathlineto{\pgfqpoint{2.851932in}{1.314303in}}%
\pgfpathlineto{\pgfqpoint{2.660272in}{1.158306in}}%
\pgfpathlineto{\pgfqpoint{3.008663in}{1.179689in}}%
\pgfpathclose%
\pgfusepath{fill}%
\end{pgfscope}%
\begin{pgfscope}%
\pgfpathrectangle{\pgfqpoint{0.708921in}{0.208778in}}{\pgfqpoint{3.800000in}{3.800000in}}%
\pgfusepath{clip}%
\pgfsetbuttcap%
\pgfsetroundjoin%
\definecolor{currentfill}{rgb}{0.107070,0.389346,0.107070}%
\pgfsetfillcolor{currentfill}%
\pgfsetlinewidth{0.000000pt}%
\definecolor{currentstroke}{rgb}{0.000000,0.000000,0.000000}%
\pgfsetstrokecolor{currentstroke}%
\pgfsetdash{}{0pt}%
\pgfpathmoveto{\pgfqpoint{3.533473in}{2.958473in}}%
\pgfpathlineto{\pgfqpoint{3.330448in}{3.189189in}}%
\pgfpathlineto{\pgfqpoint{3.217488in}{2.979704in}}%
\pgfpathlineto{\pgfqpoint{3.533473in}{2.958473in}}%
\pgfpathclose%
\pgfusepath{fill}%
\end{pgfscope}%
\begin{pgfscope}%
\pgfpathrectangle{\pgfqpoint{0.708921in}{0.208778in}}{\pgfqpoint{3.800000in}{3.800000in}}%
\pgfusepath{clip}%
\pgfsetbuttcap%
\pgfsetroundjoin%
\definecolor{currentfill}{rgb}{0.107070,0.389346,0.107070}%
\pgfsetfillcolor{currentfill}%
\pgfsetlinewidth{0.000000pt}%
\definecolor{currentstroke}{rgb}{0.000000,0.000000,0.000000}%
\pgfsetstrokecolor{currentstroke}%
\pgfsetdash{}{0pt}%
\pgfpathmoveto{\pgfqpoint{2.103056in}{2.979704in}}%
\pgfpathlineto{\pgfqpoint{1.990096in}{3.189189in}}%
\pgfpathlineto{\pgfqpoint{1.787071in}{2.958473in}}%
\pgfpathlineto{\pgfqpoint{2.103056in}{2.979704in}}%
\pgfpathclose%
\pgfusepath{fill}%
\end{pgfscope}%
\begin{pgfscope}%
\pgfpathrectangle{\pgfqpoint{0.708921in}{0.208778in}}{\pgfqpoint{3.800000in}{3.800000in}}%
\pgfusepath{clip}%
\pgfsetbuttcap%
\pgfsetroundjoin%
\definecolor{currentfill}{rgb}{0.089078,0.323920,0.089078}%
\pgfsetfillcolor{currentfill}%
\pgfsetlinewidth{0.000000pt}%
\definecolor{currentstroke}{rgb}{0.000000,0.000000,0.000000}%
\pgfsetstrokecolor{currentstroke}%
\pgfsetdash{}{0pt}%
\pgfpathmoveto{\pgfqpoint{1.547135in}{2.167772in}}%
\pgfpathlineto{\pgfqpoint{1.787071in}{2.958473in}}%
\pgfpathlineto{\pgfqpoint{1.438681in}{2.449323in}}%
\pgfpathlineto{\pgfqpoint{1.547135in}{2.167772in}}%
\pgfpathclose%
\pgfusepath{fill}%
\end{pgfscope}%
\begin{pgfscope}%
\pgfpathrectangle{\pgfqpoint{0.708921in}{0.208778in}}{\pgfqpoint{3.800000in}{3.800000in}}%
\pgfusepath{clip}%
\pgfsetbuttcap%
\pgfsetroundjoin%
\definecolor{currentfill}{rgb}{0.089078,0.323920,0.089078}%
\pgfsetfillcolor{currentfill}%
\pgfsetlinewidth{0.000000pt}%
\definecolor{currentstroke}{rgb}{0.000000,0.000000,0.000000}%
\pgfsetstrokecolor{currentstroke}%
\pgfsetdash{}{0pt}%
\pgfpathmoveto{\pgfqpoint{3.881863in}{2.449323in}}%
\pgfpathlineto{\pgfqpoint{3.533473in}{2.958473in}}%
\pgfpathlineto{\pgfqpoint{3.773409in}{2.167772in}}%
\pgfpathlineto{\pgfqpoint{3.881863in}{2.449323in}}%
\pgfpathclose%
\pgfusepath{fill}%
\end{pgfscope}%
\begin{pgfscope}%
\pgfpathrectangle{\pgfqpoint{0.708921in}{0.208778in}}{\pgfqpoint{3.800000in}{3.800000in}}%
\pgfusepath{clip}%
\pgfsetbuttcap%
\pgfsetroundjoin%
\definecolor{currentfill}{rgb}{0.061754,0.224559,0.061754}%
\pgfsetfillcolor{currentfill}%
\pgfsetlinewidth{0.000000pt}%
\definecolor{currentstroke}{rgb}{0.000000,0.000000,0.000000}%
\pgfsetstrokecolor{currentstroke}%
\pgfsetdash{}{0pt}%
\pgfpathmoveto{\pgfqpoint{1.788977in}{1.433246in}}%
\pgfpathlineto{\pgfqpoint{1.991967in}{1.238733in}}%
\pgfpathlineto{\pgfqpoint{2.040110in}{1.814186in}}%
\pgfpathlineto{\pgfqpoint{1.788977in}{1.433246in}}%
\pgfpathclose%
\pgfusepath{fill}%
\end{pgfscope}%
\begin{pgfscope}%
\pgfpathrectangle{\pgfqpoint{0.708921in}{0.208778in}}{\pgfqpoint{3.800000in}{3.800000in}}%
\pgfusepath{clip}%
\pgfsetbuttcap%
\pgfsetroundjoin%
\definecolor{currentfill}{rgb}{0.061754,0.224559,0.061754}%
\pgfsetfillcolor{currentfill}%
\pgfsetlinewidth{0.000000pt}%
\definecolor{currentstroke}{rgb}{0.000000,0.000000,0.000000}%
\pgfsetstrokecolor{currentstroke}%
\pgfsetdash{}{0pt}%
\pgfpathmoveto{\pgfqpoint{3.280434in}{1.814186in}}%
\pgfpathlineto{\pgfqpoint{3.328577in}{1.238733in}}%
\pgfpathlineto{\pgfqpoint{3.531567in}{1.433246in}}%
\pgfpathlineto{\pgfqpoint{3.280434in}{1.814186in}}%
\pgfpathclose%
\pgfusepath{fill}%
\end{pgfscope}%
\begin{pgfscope}%
\pgfpathrectangle{\pgfqpoint{0.708921in}{0.208778in}}{\pgfqpoint{3.800000in}{3.800000in}}%
\pgfusepath{clip}%
\pgfsetbuttcap%
\pgfsetroundjoin%
\definecolor{currentfill}{rgb}{0.070984,0.258123,0.070984}%
\pgfsetfillcolor{currentfill}%
\pgfsetlinewidth{0.000000pt}%
\definecolor{currentstroke}{rgb}{0.000000,0.000000,0.000000}%
\pgfsetstrokecolor{currentstroke}%
\pgfsetdash{}{0pt}%
\pgfpathmoveto{\pgfqpoint{1.547135in}{2.167772in}}%
\pgfpathlineto{\pgfqpoint{1.534653in}{1.525226in}}%
\pgfpathlineto{\pgfqpoint{1.849213in}{2.125336in}}%
\pgfpathlineto{\pgfqpoint{1.547135in}{2.167772in}}%
\pgfpathclose%
\pgfusepath{fill}%
\end{pgfscope}%
\begin{pgfscope}%
\pgfpathrectangle{\pgfqpoint{0.708921in}{0.208778in}}{\pgfqpoint{3.800000in}{3.800000in}}%
\pgfusepath{clip}%
\pgfsetbuttcap%
\pgfsetroundjoin%
\definecolor{currentfill}{rgb}{0.070984,0.258123,0.070984}%
\pgfsetfillcolor{currentfill}%
\pgfsetlinewidth{0.000000pt}%
\definecolor{currentstroke}{rgb}{0.000000,0.000000,0.000000}%
\pgfsetstrokecolor{currentstroke}%
\pgfsetdash{}{0pt}%
\pgfpathmoveto{\pgfqpoint{3.471331in}{2.125336in}}%
\pgfpathlineto{\pgfqpoint{3.785891in}{1.525226in}}%
\pgfpathlineto{\pgfqpoint{3.773409in}{2.167772in}}%
\pgfpathlineto{\pgfqpoint{3.471331in}{2.125336in}}%
\pgfpathclose%
\pgfusepath{fill}%
\end{pgfscope}%
\begin{pgfscope}%
\pgfpathrectangle{\pgfqpoint{0.708921in}{0.208778in}}{\pgfqpoint{3.800000in}{3.800000in}}%
\pgfusepath{clip}%
\pgfsetbuttcap%
\pgfsetroundjoin%
\definecolor{currentfill}{rgb}{0.070885,0.257762,0.070885}%
\pgfsetfillcolor{currentfill}%
\pgfsetlinewidth{0.000000pt}%
\definecolor{currentstroke}{rgb}{0.000000,0.000000,0.000000}%
\pgfsetstrokecolor{currentstroke}%
\pgfsetdash{}{0pt}%
\pgfpathmoveto{\pgfqpoint{2.040110in}{1.814186in}}%
\pgfpathlineto{\pgfqpoint{1.991967in}{1.238733in}}%
\pgfpathlineto{\pgfqpoint{2.104350in}{1.357725in}}%
\pgfpathlineto{\pgfqpoint{2.040110in}{1.814186in}}%
\pgfpathclose%
\pgfusepath{fill}%
\end{pgfscope}%
\begin{pgfscope}%
\pgfpathrectangle{\pgfqpoint{0.708921in}{0.208778in}}{\pgfqpoint{3.800000in}{3.800000in}}%
\pgfusepath{clip}%
\pgfsetbuttcap%
\pgfsetroundjoin%
\definecolor{currentfill}{rgb}{0.070885,0.257762,0.070885}%
\pgfsetfillcolor{currentfill}%
\pgfsetlinewidth{0.000000pt}%
\definecolor{currentstroke}{rgb}{0.000000,0.000000,0.000000}%
\pgfsetstrokecolor{currentstroke}%
\pgfsetdash{}{0pt}%
\pgfpathmoveto{\pgfqpoint{3.216194in}{1.357725in}}%
\pgfpathlineto{\pgfqpoint{3.328577in}{1.238733in}}%
\pgfpathlineto{\pgfqpoint{3.280434in}{1.814186in}}%
\pgfpathlineto{\pgfqpoint{3.216194in}{1.357725in}}%
\pgfpathclose%
\pgfusepath{fill}%
\end{pgfscope}%
\begin{pgfscope}%
\pgfpathrectangle{\pgfqpoint{0.708921in}{0.208778in}}{\pgfqpoint{3.800000in}{3.800000in}}%
\pgfusepath{clip}%
\pgfsetbuttcap%
\pgfsetroundjoin%
\definecolor{currentfill}{rgb}{0.111651,0.406004,0.111651}%
\pgfsetfillcolor{currentfill}%
\pgfsetlinewidth{0.000000pt}%
\definecolor{currentstroke}{rgb}{0.000000,0.000000,0.000000}%
\pgfsetstrokecolor{currentstroke}%
\pgfsetdash{}{0pt}%
\pgfpathmoveto{\pgfqpoint{3.217488in}{2.979704in}}%
\pgfpathlineto{\pgfqpoint{3.330448in}{3.189189in}}%
\pgfpathlineto{\pgfqpoint{2.852393in}{2.991915in}}%
\pgfpathlineto{\pgfqpoint{3.217488in}{2.979704in}}%
\pgfpathclose%
\pgfusepath{fill}%
\end{pgfscope}%
\begin{pgfscope}%
\pgfpathrectangle{\pgfqpoint{0.708921in}{0.208778in}}{\pgfqpoint{3.800000in}{3.800000in}}%
\pgfusepath{clip}%
\pgfsetbuttcap%
\pgfsetroundjoin%
\definecolor{currentfill}{rgb}{0.111651,0.406004,0.111651}%
\pgfsetfillcolor{currentfill}%
\pgfsetlinewidth{0.000000pt}%
\definecolor{currentstroke}{rgb}{0.000000,0.000000,0.000000}%
\pgfsetstrokecolor{currentstroke}%
\pgfsetdash{}{0pt}%
\pgfpathmoveto{\pgfqpoint{2.468151in}{2.991915in}}%
\pgfpathlineto{\pgfqpoint{1.990096in}{3.189189in}}%
\pgfpathlineto{\pgfqpoint{2.103056in}{2.979704in}}%
\pgfpathlineto{\pgfqpoint{2.468151in}{2.991915in}}%
\pgfpathclose%
\pgfusepath{fill}%
\end{pgfscope}%
\begin{pgfscope}%
\pgfpathrectangle{\pgfqpoint{0.708921in}{0.208778in}}{\pgfqpoint{3.800000in}{3.800000in}}%
\pgfusepath{clip}%
\pgfsetbuttcap%
\pgfsetroundjoin%
\definecolor{currentfill}{rgb}{0.092193,0.335248,0.092193}%
\pgfsetfillcolor{currentfill}%
\pgfsetlinewidth{0.000000pt}%
\definecolor{currentstroke}{rgb}{0.000000,0.000000,0.000000}%
\pgfsetstrokecolor{currentstroke}%
\pgfsetdash{}{0pt}%
\pgfpathmoveto{\pgfqpoint{1.849213in}{2.125336in}}%
\pgfpathlineto{\pgfqpoint{1.787071in}{2.958473in}}%
\pgfpathlineto{\pgfqpoint{1.547135in}{2.167772in}}%
\pgfpathlineto{\pgfqpoint{1.849213in}{2.125336in}}%
\pgfpathclose%
\pgfusepath{fill}%
\end{pgfscope}%
\begin{pgfscope}%
\pgfpathrectangle{\pgfqpoint{0.708921in}{0.208778in}}{\pgfqpoint{3.800000in}{3.800000in}}%
\pgfusepath{clip}%
\pgfsetbuttcap%
\pgfsetroundjoin%
\definecolor{currentfill}{rgb}{0.092193,0.335248,0.092193}%
\pgfsetfillcolor{currentfill}%
\pgfsetlinewidth{0.000000pt}%
\definecolor{currentstroke}{rgb}{0.000000,0.000000,0.000000}%
\pgfsetstrokecolor{currentstroke}%
\pgfsetdash{}{0pt}%
\pgfpathmoveto{\pgfqpoint{3.773409in}{2.167772in}}%
\pgfpathlineto{\pgfqpoint{3.533473in}{2.958473in}}%
\pgfpathlineto{\pgfqpoint{3.471331in}{2.125336in}}%
\pgfpathlineto{\pgfqpoint{3.773409in}{2.167772in}}%
\pgfpathclose%
\pgfusepath{fill}%
\end{pgfscope}%
\begin{pgfscope}%
\pgfpathrectangle{\pgfqpoint{0.708921in}{0.208778in}}{\pgfqpoint{3.800000in}{3.800000in}}%
\pgfusepath{clip}%
\pgfsetbuttcap%
\pgfsetroundjoin%
\definecolor{currentfill}{rgb}{0.060562,0.220227,0.060562}%
\pgfsetfillcolor{currentfill}%
\pgfsetlinewidth{0.000000pt}%
\definecolor{currentstroke}{rgb}{0.000000,0.000000,0.000000}%
\pgfsetstrokecolor{currentstroke}%
\pgfsetdash{}{0pt}%
\pgfpathmoveto{\pgfqpoint{2.311881in}{1.179689in}}%
\pgfpathlineto{\pgfqpoint{2.444709in}{1.779529in}}%
\pgfpathlineto{\pgfqpoint{2.104350in}{1.357725in}}%
\pgfpathlineto{\pgfqpoint{2.311881in}{1.179689in}}%
\pgfpathclose%
\pgfusepath{fill}%
\end{pgfscope}%
\begin{pgfscope}%
\pgfpathrectangle{\pgfqpoint{0.708921in}{0.208778in}}{\pgfqpoint{3.800000in}{3.800000in}}%
\pgfusepath{clip}%
\pgfsetbuttcap%
\pgfsetroundjoin%
\definecolor{currentfill}{rgb}{0.060562,0.220227,0.060562}%
\pgfsetfillcolor{currentfill}%
\pgfsetlinewidth{0.000000pt}%
\definecolor{currentstroke}{rgb}{0.000000,0.000000,0.000000}%
\pgfsetstrokecolor{currentstroke}%
\pgfsetdash{}{0pt}%
\pgfpathmoveto{\pgfqpoint{3.216194in}{1.357725in}}%
\pgfpathlineto{\pgfqpoint{2.875835in}{1.779529in}}%
\pgfpathlineto{\pgfqpoint{3.008663in}{1.179689in}}%
\pgfpathlineto{\pgfqpoint{3.216194in}{1.357725in}}%
\pgfpathclose%
\pgfusepath{fill}%
\end{pgfscope}%
\begin{pgfscope}%
\pgfpathrectangle{\pgfqpoint{0.708921in}{0.208778in}}{\pgfqpoint{3.800000in}{3.800000in}}%
\pgfusepath{clip}%
\pgfsetbuttcap%
\pgfsetroundjoin%
\definecolor{currentfill}{rgb}{0.097285,0.353762,0.097285}%
\pgfsetfillcolor{currentfill}%
\pgfsetlinewidth{0.000000pt}%
\definecolor{currentstroke}{rgb}{0.000000,0.000000,0.000000}%
\pgfsetstrokecolor{currentstroke}%
\pgfsetdash{}{0pt}%
\pgfpathmoveto{\pgfqpoint{2.103056in}{2.979704in}}%
\pgfpathlineto{\pgfqpoint{1.787071in}{2.958473in}}%
\pgfpathlineto{\pgfqpoint{2.254399in}{2.716708in}}%
\pgfpathlineto{\pgfqpoint{2.103056in}{2.979704in}}%
\pgfpathclose%
\pgfusepath{fill}%
\end{pgfscope}%
\begin{pgfscope}%
\pgfpathrectangle{\pgfqpoint{0.708921in}{0.208778in}}{\pgfqpoint{3.800000in}{3.800000in}}%
\pgfusepath{clip}%
\pgfsetbuttcap%
\pgfsetroundjoin%
\definecolor{currentfill}{rgb}{0.097285,0.353762,0.097285}%
\pgfsetfillcolor{currentfill}%
\pgfsetlinewidth{0.000000pt}%
\definecolor{currentstroke}{rgb}{0.000000,0.000000,0.000000}%
\pgfsetstrokecolor{currentstroke}%
\pgfsetdash{}{0pt}%
\pgfpathmoveto{\pgfqpoint{3.066145in}{2.716708in}}%
\pgfpathlineto{\pgfqpoint{3.533473in}{2.958473in}}%
\pgfpathlineto{\pgfqpoint{3.217488in}{2.979704in}}%
\pgfpathlineto{\pgfqpoint{3.066145in}{2.716708in}}%
\pgfpathclose%
\pgfusepath{fill}%
\end{pgfscope}%
\begin{pgfscope}%
\pgfpathrectangle{\pgfqpoint{0.708921in}{0.208778in}}{\pgfqpoint{3.800000in}{3.800000in}}%
\pgfusepath{clip}%
\pgfsetbuttcap%
\pgfsetroundjoin%
\definecolor{currentfill}{rgb}{0.067497,0.245443,0.067497}%
\pgfsetfillcolor{currentfill}%
\pgfsetlinewidth{0.000000pt}%
\definecolor{currentstroke}{rgb}{0.000000,0.000000,0.000000}%
\pgfsetstrokecolor{currentstroke}%
\pgfsetdash{}{0pt}%
\pgfpathmoveto{\pgfqpoint{2.875835in}{1.779529in}}%
\pgfpathlineto{\pgfqpoint{2.851932in}{1.314303in}}%
\pgfpathlineto{\pgfqpoint{3.008663in}{1.179689in}}%
\pgfpathlineto{\pgfqpoint{2.875835in}{1.779529in}}%
\pgfpathclose%
\pgfusepath{fill}%
\end{pgfscope}%
\begin{pgfscope}%
\pgfpathrectangle{\pgfqpoint{0.708921in}{0.208778in}}{\pgfqpoint{3.800000in}{3.800000in}}%
\pgfusepath{clip}%
\pgfsetbuttcap%
\pgfsetroundjoin%
\definecolor{currentfill}{rgb}{0.067497,0.245443,0.067497}%
\pgfsetfillcolor{currentfill}%
\pgfsetlinewidth{0.000000pt}%
\definecolor{currentstroke}{rgb}{0.000000,0.000000,0.000000}%
\pgfsetstrokecolor{currentstroke}%
\pgfsetdash{}{0pt}%
\pgfpathmoveto{\pgfqpoint{2.311881in}{1.179689in}}%
\pgfpathlineto{\pgfqpoint{2.468612in}{1.314303in}}%
\pgfpathlineto{\pgfqpoint{2.444709in}{1.779529in}}%
\pgfpathlineto{\pgfqpoint{2.311881in}{1.179689in}}%
\pgfpathclose%
\pgfusepath{fill}%
\end{pgfscope}%
\begin{pgfscope}%
\pgfpathrectangle{\pgfqpoint{0.708921in}{0.208778in}}{\pgfqpoint{3.800000in}{3.800000in}}%
\pgfusepath{clip}%
\pgfsetbuttcap%
\pgfsetroundjoin%
\definecolor{currentfill}{rgb}{0.065434,0.237940,0.065434}%
\pgfsetfillcolor{currentfill}%
\pgfsetlinewidth{0.000000pt}%
\definecolor{currentstroke}{rgb}{0.000000,0.000000,0.000000}%
\pgfsetstrokecolor{currentstroke}%
\pgfsetdash{}{0pt}%
\pgfpathmoveto{\pgfqpoint{2.851932in}{1.314303in}}%
\pgfpathlineto{\pgfqpoint{2.875835in}{1.779529in}}%
\pgfpathlineto{\pgfqpoint{2.660272in}{1.158306in}}%
\pgfpathlineto{\pgfqpoint{2.851932in}{1.314303in}}%
\pgfpathclose%
\pgfusepath{fill}%
\end{pgfscope}%
\begin{pgfscope}%
\pgfpathrectangle{\pgfqpoint{0.708921in}{0.208778in}}{\pgfqpoint{3.800000in}{3.800000in}}%
\pgfusepath{clip}%
\pgfsetbuttcap%
\pgfsetroundjoin%
\definecolor{currentfill}{rgb}{0.065434,0.237940,0.065434}%
\pgfsetfillcolor{currentfill}%
\pgfsetlinewidth{0.000000pt}%
\definecolor{currentstroke}{rgb}{0.000000,0.000000,0.000000}%
\pgfsetstrokecolor{currentstroke}%
\pgfsetdash{}{0pt}%
\pgfpathmoveto{\pgfqpoint{2.660272in}{1.158306in}}%
\pgfpathlineto{\pgfqpoint{2.444709in}{1.779529in}}%
\pgfpathlineto{\pgfqpoint{2.468612in}{1.314303in}}%
\pgfpathlineto{\pgfqpoint{2.660272in}{1.158306in}}%
\pgfpathclose%
\pgfusepath{fill}%
\end{pgfscope}%
\begin{pgfscope}%
\pgfpathrectangle{\pgfqpoint{0.708921in}{0.208778in}}{\pgfqpoint{3.800000in}{3.800000in}}%
\pgfusepath{clip}%
\pgfsetbuttcap%
\pgfsetroundjoin%
\definecolor{currentfill}{rgb}{0.073593,0.267612,0.073593}%
\pgfsetfillcolor{currentfill}%
\pgfsetlinewidth{0.000000pt}%
\definecolor{currentstroke}{rgb}{0.000000,0.000000,0.000000}%
\pgfsetstrokecolor{currentstroke}%
\pgfsetdash{}{0pt}%
\pgfpathmoveto{\pgfqpoint{1.788977in}{1.433246in}}%
\pgfpathlineto{\pgfqpoint{2.040110in}{1.814186in}}%
\pgfpathlineto{\pgfqpoint{1.849213in}{2.125336in}}%
\pgfpathlineto{\pgfqpoint{1.788977in}{1.433246in}}%
\pgfpathclose%
\pgfusepath{fill}%
\end{pgfscope}%
\begin{pgfscope}%
\pgfpathrectangle{\pgfqpoint{0.708921in}{0.208778in}}{\pgfqpoint{3.800000in}{3.800000in}}%
\pgfusepath{clip}%
\pgfsetbuttcap%
\pgfsetroundjoin%
\definecolor{currentfill}{rgb}{0.073593,0.267612,0.073593}%
\pgfsetfillcolor{currentfill}%
\pgfsetlinewidth{0.000000pt}%
\definecolor{currentstroke}{rgb}{0.000000,0.000000,0.000000}%
\pgfsetstrokecolor{currentstroke}%
\pgfsetdash{}{0pt}%
\pgfpathmoveto{\pgfqpoint{3.471331in}{2.125336in}}%
\pgfpathlineto{\pgfqpoint{3.280434in}{1.814186in}}%
\pgfpathlineto{\pgfqpoint{3.531567in}{1.433246in}}%
\pgfpathlineto{\pgfqpoint{3.471331in}{2.125336in}}%
\pgfpathclose%
\pgfusepath{fill}%
\end{pgfscope}%
\begin{pgfscope}%
\pgfpathrectangle{\pgfqpoint{0.708921in}{0.208778in}}{\pgfqpoint{3.800000in}{3.800000in}}%
\pgfusepath{clip}%
\pgfsetbuttcap%
\pgfsetroundjoin%
\definecolor{currentfill}{rgb}{0.091915,0.334238,0.091915}%
\pgfsetfillcolor{currentfill}%
\pgfsetlinewidth{0.000000pt}%
\definecolor{currentstroke}{rgb}{0.000000,0.000000,0.000000}%
\pgfsetstrokecolor{currentstroke}%
\pgfsetdash{}{0pt}%
\pgfpathmoveto{\pgfqpoint{1.849213in}{2.125336in}}%
\pgfpathlineto{\pgfqpoint{2.254399in}{2.716708in}}%
\pgfpathlineto{\pgfqpoint{1.787071in}{2.958473in}}%
\pgfpathlineto{\pgfqpoint{1.849213in}{2.125336in}}%
\pgfpathclose%
\pgfusepath{fill}%
\end{pgfscope}%
\begin{pgfscope}%
\pgfpathrectangle{\pgfqpoint{0.708921in}{0.208778in}}{\pgfqpoint{3.800000in}{3.800000in}}%
\pgfusepath{clip}%
\pgfsetbuttcap%
\pgfsetroundjoin%
\definecolor{currentfill}{rgb}{0.091915,0.334238,0.091915}%
\pgfsetfillcolor{currentfill}%
\pgfsetlinewidth{0.000000pt}%
\definecolor{currentstroke}{rgb}{0.000000,0.000000,0.000000}%
\pgfsetstrokecolor{currentstroke}%
\pgfsetdash{}{0pt}%
\pgfpathmoveto{\pgfqpoint{3.533473in}{2.958473in}}%
\pgfpathlineto{\pgfqpoint{3.066145in}{2.716708in}}%
\pgfpathlineto{\pgfqpoint{3.471331in}{2.125336in}}%
\pgfpathlineto{\pgfqpoint{3.533473in}{2.958473in}}%
\pgfpathclose%
\pgfusepath{fill}%
\end{pgfscope}%
\begin{pgfscope}%
\pgfpathrectangle{\pgfqpoint{0.708921in}{0.208778in}}{\pgfqpoint{3.800000in}{3.800000in}}%
\pgfusepath{clip}%
\pgfsetbuttcap%
\pgfsetroundjoin%
\definecolor{currentfill}{rgb}{0.101759,0.370033,0.101759}%
\pgfsetfillcolor{currentfill}%
\pgfsetlinewidth{0.000000pt}%
\definecolor{currentstroke}{rgb}{0.000000,0.000000,0.000000}%
\pgfsetstrokecolor{currentstroke}%
\pgfsetdash{}{0pt}%
\pgfpathmoveto{\pgfqpoint{2.254399in}{2.716708in}}%
\pgfpathlineto{\pgfqpoint{2.468151in}{2.991915in}}%
\pgfpathlineto{\pgfqpoint{2.103056in}{2.979704in}}%
\pgfpathlineto{\pgfqpoint{2.254399in}{2.716708in}}%
\pgfpathclose%
\pgfusepath{fill}%
\end{pgfscope}%
\begin{pgfscope}%
\pgfpathrectangle{\pgfqpoint{0.708921in}{0.208778in}}{\pgfqpoint{3.800000in}{3.800000in}}%
\pgfusepath{clip}%
\pgfsetbuttcap%
\pgfsetroundjoin%
\definecolor{currentfill}{rgb}{0.101759,0.370033,0.101759}%
\pgfsetfillcolor{currentfill}%
\pgfsetlinewidth{0.000000pt}%
\definecolor{currentstroke}{rgb}{0.000000,0.000000,0.000000}%
\pgfsetstrokecolor{currentstroke}%
\pgfsetdash{}{0pt}%
\pgfpathmoveto{\pgfqpoint{3.217488in}{2.979704in}}%
\pgfpathlineto{\pgfqpoint{2.852393in}{2.991915in}}%
\pgfpathlineto{\pgfqpoint{3.066145in}{2.716708in}}%
\pgfpathlineto{\pgfqpoint{3.217488in}{2.979704in}}%
\pgfpathclose%
\pgfusepath{fill}%
\end{pgfscope}%
\begin{pgfscope}%
\pgfpathrectangle{\pgfqpoint{0.708921in}{0.208778in}}{\pgfqpoint{3.800000in}{3.800000in}}%
\pgfusepath{clip}%
\pgfsetbuttcap%
\pgfsetroundjoin%
\definecolor{currentfill}{rgb}{0.101677,0.369734,0.101677}%
\pgfsetfillcolor{currentfill}%
\pgfsetlinewidth{0.000000pt}%
\definecolor{currentstroke}{rgb}{0.000000,0.000000,0.000000}%
\pgfsetstrokecolor{currentstroke}%
\pgfsetdash{}{0pt}%
\pgfpathmoveto{\pgfqpoint{2.660272in}{2.718551in}}%
\pgfpathlineto{\pgfqpoint{2.852393in}{2.991915in}}%
\pgfpathlineto{\pgfqpoint{2.468151in}{2.991915in}}%
\pgfpathlineto{\pgfqpoint{2.660272in}{2.718551in}}%
\pgfpathclose%
\pgfusepath{fill}%
\end{pgfscope}%
\begin{pgfscope}%
\pgfpathrectangle{\pgfqpoint{0.708921in}{0.208778in}}{\pgfqpoint{3.800000in}{3.800000in}}%
\pgfusepath{clip}%
\pgfsetbuttcap%
\pgfsetroundjoin%
\definecolor{currentfill}{rgb}{0.065035,0.236492,0.065035}%
\pgfsetfillcolor{currentfill}%
\pgfsetlinewidth{0.000000pt}%
\definecolor{currentstroke}{rgb}{0.000000,0.000000,0.000000}%
\pgfsetstrokecolor{currentstroke}%
\pgfsetdash{}{0pt}%
\pgfpathmoveto{\pgfqpoint{2.104350in}{1.357725in}}%
\pgfpathlineto{\pgfqpoint{2.228951in}{2.093598in}}%
\pgfpathlineto{\pgfqpoint{2.040110in}{1.814186in}}%
\pgfpathlineto{\pgfqpoint{2.104350in}{1.357725in}}%
\pgfpathclose%
\pgfusepath{fill}%
\end{pgfscope}%
\begin{pgfscope}%
\pgfpathrectangle{\pgfqpoint{0.708921in}{0.208778in}}{\pgfqpoint{3.800000in}{3.800000in}}%
\pgfusepath{clip}%
\pgfsetbuttcap%
\pgfsetroundjoin%
\definecolor{currentfill}{rgb}{0.065035,0.236492,0.065035}%
\pgfsetfillcolor{currentfill}%
\pgfsetlinewidth{0.000000pt}%
\definecolor{currentstroke}{rgb}{0.000000,0.000000,0.000000}%
\pgfsetstrokecolor{currentstroke}%
\pgfsetdash{}{0pt}%
\pgfpathmoveto{\pgfqpoint{3.280434in}{1.814186in}}%
\pgfpathlineto{\pgfqpoint{3.091593in}{2.093598in}}%
\pgfpathlineto{\pgfqpoint{3.216194in}{1.357725in}}%
\pgfpathlineto{\pgfqpoint{3.280434in}{1.814186in}}%
\pgfpathclose%
\pgfusepath{fill}%
\end{pgfscope}%
\begin{pgfscope}%
\pgfpathrectangle{\pgfqpoint{0.708921in}{0.208778in}}{\pgfqpoint{3.800000in}{3.800000in}}%
\pgfusepath{clip}%
\pgfsetbuttcap%
\pgfsetroundjoin%
\definecolor{currentfill}{rgb}{0.066446,0.241622,0.066446}%
\pgfsetfillcolor{currentfill}%
\pgfsetlinewidth{0.000000pt}%
\definecolor{currentstroke}{rgb}{0.000000,0.000000,0.000000}%
\pgfsetstrokecolor{currentstroke}%
\pgfsetdash{}{0pt}%
\pgfpathmoveto{\pgfqpoint{2.444709in}{1.779529in}}%
\pgfpathlineto{\pgfqpoint{2.660272in}{1.158306in}}%
\pgfpathlineto{\pgfqpoint{2.660272in}{2.081635in}}%
\pgfpathlineto{\pgfqpoint{2.444709in}{1.779529in}}%
\pgfpathclose%
\pgfusepath{fill}%
\end{pgfscope}%
\begin{pgfscope}%
\pgfpathrectangle{\pgfqpoint{0.708921in}{0.208778in}}{\pgfqpoint{3.800000in}{3.800000in}}%
\pgfusepath{clip}%
\pgfsetbuttcap%
\pgfsetroundjoin%
\definecolor{currentfill}{rgb}{0.066446,0.241622,0.066446}%
\pgfsetfillcolor{currentfill}%
\pgfsetlinewidth{0.000000pt}%
\definecolor{currentstroke}{rgb}{0.000000,0.000000,0.000000}%
\pgfsetstrokecolor{currentstroke}%
\pgfsetdash{}{0pt}%
\pgfpathmoveto{\pgfqpoint{2.660272in}{2.081635in}}%
\pgfpathlineto{\pgfqpoint{2.660272in}{1.158306in}}%
\pgfpathlineto{\pgfqpoint{2.875835in}{1.779529in}}%
\pgfpathlineto{\pgfqpoint{2.660272in}{2.081635in}}%
\pgfpathclose%
\pgfusepath{fill}%
\end{pgfscope}%
\begin{pgfscope}%
\pgfpathrectangle{\pgfqpoint{0.708921in}{0.208778in}}{\pgfqpoint{3.800000in}{3.800000in}}%
\pgfusepath{clip}%
\pgfsetbuttcap%
\pgfsetroundjoin%
\definecolor{currentfill}{rgb}{0.098306,0.357475,0.098306}%
\pgfsetfillcolor{currentfill}%
\pgfsetlinewidth{0.000000pt}%
\definecolor{currentstroke}{rgb}{0.000000,0.000000,0.000000}%
\pgfsetstrokecolor{currentstroke}%
\pgfsetdash{}{0pt}%
\pgfpathmoveto{\pgfqpoint{2.660272in}{2.718551in}}%
\pgfpathlineto{\pgfqpoint{2.468151in}{2.991915in}}%
\pgfpathlineto{\pgfqpoint{2.254399in}{2.716708in}}%
\pgfpathlineto{\pgfqpoint{2.660272in}{2.718551in}}%
\pgfpathclose%
\pgfusepath{fill}%
\end{pgfscope}%
\begin{pgfscope}%
\pgfpathrectangle{\pgfqpoint{0.708921in}{0.208778in}}{\pgfqpoint{3.800000in}{3.800000in}}%
\pgfusepath{clip}%
\pgfsetbuttcap%
\pgfsetroundjoin%
\definecolor{currentfill}{rgb}{0.098306,0.357475,0.098306}%
\pgfsetfillcolor{currentfill}%
\pgfsetlinewidth{0.000000pt}%
\definecolor{currentstroke}{rgb}{0.000000,0.000000,0.000000}%
\pgfsetstrokecolor{currentstroke}%
\pgfsetdash{}{0pt}%
\pgfpathmoveto{\pgfqpoint{3.066145in}{2.716708in}}%
\pgfpathlineto{\pgfqpoint{2.852393in}{2.991915in}}%
\pgfpathlineto{\pgfqpoint{2.660272in}{2.718551in}}%
\pgfpathlineto{\pgfqpoint{3.066145in}{2.716708in}}%
\pgfpathclose%
\pgfusepath{fill}%
\end{pgfscope}%
\begin{pgfscope}%
\pgfpathrectangle{\pgfqpoint{0.708921in}{0.208778in}}{\pgfqpoint{3.800000in}{3.800000in}}%
\pgfusepath{clip}%
\pgfsetbuttcap%
\pgfsetroundjoin%
\definecolor{currentfill}{rgb}{0.070209,0.255305,0.070209}%
\pgfsetfillcolor{currentfill}%
\pgfsetlinewidth{0.000000pt}%
\definecolor{currentstroke}{rgb}{0.000000,0.000000,0.000000}%
\pgfsetstrokecolor{currentstroke}%
\pgfsetdash{}{0pt}%
\pgfpathmoveto{\pgfqpoint{2.104350in}{1.357725in}}%
\pgfpathlineto{\pgfqpoint{2.444709in}{1.779529in}}%
\pgfpathlineto{\pgfqpoint{2.228951in}{2.093598in}}%
\pgfpathlineto{\pgfqpoint{2.104350in}{1.357725in}}%
\pgfpathclose%
\pgfusepath{fill}%
\end{pgfscope}%
\begin{pgfscope}%
\pgfpathrectangle{\pgfqpoint{0.708921in}{0.208778in}}{\pgfqpoint{3.800000in}{3.800000in}}%
\pgfusepath{clip}%
\pgfsetbuttcap%
\pgfsetroundjoin%
\definecolor{currentfill}{rgb}{0.070209,0.255305,0.070209}%
\pgfsetfillcolor{currentfill}%
\pgfsetlinewidth{0.000000pt}%
\definecolor{currentstroke}{rgb}{0.000000,0.000000,0.000000}%
\pgfsetstrokecolor{currentstroke}%
\pgfsetdash{}{0pt}%
\pgfpathmoveto{\pgfqpoint{3.091593in}{2.093598in}}%
\pgfpathlineto{\pgfqpoint{2.875835in}{1.779529in}}%
\pgfpathlineto{\pgfqpoint{3.216194in}{1.357725in}}%
\pgfpathlineto{\pgfqpoint{3.091593in}{2.093598in}}%
\pgfpathclose%
\pgfusepath{fill}%
\end{pgfscope}%
\begin{pgfscope}%
\pgfpathrectangle{\pgfqpoint{0.708921in}{0.208778in}}{\pgfqpoint{3.800000in}{3.800000in}}%
\pgfusepath{clip}%
\pgfsetbuttcap%
\pgfsetroundjoin%
\definecolor{currentfill}{rgb}{0.087398,0.317812,0.087398}%
\pgfsetfillcolor{currentfill}%
\pgfsetlinewidth{0.000000pt}%
\definecolor{currentstroke}{rgb}{0.000000,0.000000,0.000000}%
\pgfsetstrokecolor{currentstroke}%
\pgfsetdash{}{0pt}%
\pgfpathmoveto{\pgfqpoint{2.228951in}{2.093598in}}%
\pgfpathlineto{\pgfqpoint{2.254399in}{2.716708in}}%
\pgfpathlineto{\pgfqpoint{1.849213in}{2.125336in}}%
\pgfpathlineto{\pgfqpoint{2.228951in}{2.093598in}}%
\pgfpathclose%
\pgfusepath{fill}%
\end{pgfscope}%
\begin{pgfscope}%
\pgfpathrectangle{\pgfqpoint{0.708921in}{0.208778in}}{\pgfqpoint{3.800000in}{3.800000in}}%
\pgfusepath{clip}%
\pgfsetbuttcap%
\pgfsetroundjoin%
\definecolor{currentfill}{rgb}{0.087398,0.317812,0.087398}%
\pgfsetfillcolor{currentfill}%
\pgfsetlinewidth{0.000000pt}%
\definecolor{currentstroke}{rgb}{0.000000,0.000000,0.000000}%
\pgfsetstrokecolor{currentstroke}%
\pgfsetdash{}{0pt}%
\pgfpathmoveto{\pgfqpoint{3.471331in}{2.125336in}}%
\pgfpathlineto{\pgfqpoint{3.066145in}{2.716708in}}%
\pgfpathlineto{\pgfqpoint{3.091593in}{2.093598in}}%
\pgfpathlineto{\pgfqpoint{3.471331in}{2.125336in}}%
\pgfpathclose%
\pgfusepath{fill}%
\end{pgfscope}%
\begin{pgfscope}%
\pgfpathrectangle{\pgfqpoint{0.708921in}{0.208778in}}{\pgfqpoint{3.800000in}{3.800000in}}%
\pgfusepath{clip}%
\pgfsetbuttcap%
\pgfsetroundjoin%
\definecolor{currentfill}{rgb}{0.075994,0.276341,0.075994}%
\pgfsetfillcolor{currentfill}%
\pgfsetlinewidth{0.000000pt}%
\definecolor{currentstroke}{rgb}{0.000000,0.000000,0.000000}%
\pgfsetstrokecolor{currentstroke}%
\pgfsetdash{}{0pt}%
\pgfpathmoveto{\pgfqpoint{1.849213in}{2.125336in}}%
\pgfpathlineto{\pgfqpoint{2.040110in}{1.814186in}}%
\pgfpathlineto{\pgfqpoint{2.228951in}{2.093598in}}%
\pgfpathlineto{\pgfqpoint{1.849213in}{2.125336in}}%
\pgfpathclose%
\pgfusepath{fill}%
\end{pgfscope}%
\begin{pgfscope}%
\pgfpathrectangle{\pgfqpoint{0.708921in}{0.208778in}}{\pgfqpoint{3.800000in}{3.800000in}}%
\pgfusepath{clip}%
\pgfsetbuttcap%
\pgfsetroundjoin%
\definecolor{currentfill}{rgb}{0.075994,0.276341,0.075994}%
\pgfsetfillcolor{currentfill}%
\pgfsetlinewidth{0.000000pt}%
\definecolor{currentstroke}{rgb}{0.000000,0.000000,0.000000}%
\pgfsetstrokecolor{currentstroke}%
\pgfsetdash{}{0pt}%
\pgfpathmoveto{\pgfqpoint{3.091593in}{2.093598in}}%
\pgfpathlineto{\pgfqpoint{3.280434in}{1.814186in}}%
\pgfpathlineto{\pgfqpoint{3.471331in}{2.125336in}}%
\pgfpathlineto{\pgfqpoint{3.091593in}{2.093598in}}%
\pgfpathclose%
\pgfusepath{fill}%
\end{pgfscope}%
\begin{pgfscope}%
\pgfpathrectangle{\pgfqpoint{0.708921in}{0.208778in}}{\pgfqpoint{3.800000in}{3.800000in}}%
\pgfusepath{clip}%
\pgfsetbuttcap%
\pgfsetroundjoin%
\definecolor{currentfill}{rgb}{0.086061,0.312950,0.086061}%
\pgfsetfillcolor{currentfill}%
\pgfsetlinewidth{0.000000pt}%
\definecolor{currentstroke}{rgb}{0.000000,0.000000,0.000000}%
\pgfsetstrokecolor{currentstroke}%
\pgfsetdash{}{0pt}%
\pgfpathmoveto{\pgfqpoint{2.254399in}{2.716708in}}%
\pgfpathlineto{\pgfqpoint{2.228951in}{2.093598in}}%
\pgfpathlineto{\pgfqpoint{2.660272in}{2.718551in}}%
\pgfpathlineto{\pgfqpoint{2.254399in}{2.716708in}}%
\pgfpathclose%
\pgfusepath{fill}%
\end{pgfscope}%
\begin{pgfscope}%
\pgfpathrectangle{\pgfqpoint{0.708921in}{0.208778in}}{\pgfqpoint{3.800000in}{3.800000in}}%
\pgfusepath{clip}%
\pgfsetbuttcap%
\pgfsetroundjoin%
\definecolor{currentfill}{rgb}{0.086061,0.312950,0.086061}%
\pgfsetfillcolor{currentfill}%
\pgfsetlinewidth{0.000000pt}%
\definecolor{currentstroke}{rgb}{0.000000,0.000000,0.000000}%
\pgfsetstrokecolor{currentstroke}%
\pgfsetdash{}{0pt}%
\pgfpathmoveto{\pgfqpoint{2.660272in}{2.718551in}}%
\pgfpathlineto{\pgfqpoint{3.091593in}{2.093598in}}%
\pgfpathlineto{\pgfqpoint{3.066145in}{2.716708in}}%
\pgfpathlineto{\pgfqpoint{2.660272in}{2.718551in}}%
\pgfpathclose%
\pgfusepath{fill}%
\end{pgfscope}%
\begin{pgfscope}%
\pgfpathrectangle{\pgfqpoint{0.708921in}{0.208778in}}{\pgfqpoint{3.800000in}{3.800000in}}%
\pgfusepath{clip}%
\pgfsetbuttcap%
\pgfsetroundjoin%
\definecolor{currentfill}{rgb}{0.086258,0.313666,0.086258}%
\pgfsetfillcolor{currentfill}%
\pgfsetlinewidth{0.000000pt}%
\definecolor{currentstroke}{rgb}{0.000000,0.000000,0.000000}%
\pgfsetstrokecolor{currentstroke}%
\pgfsetdash{}{0pt}%
\pgfpathmoveto{\pgfqpoint{2.660272in}{2.081635in}}%
\pgfpathlineto{\pgfqpoint{2.660272in}{2.718551in}}%
\pgfpathlineto{\pgfqpoint{2.228951in}{2.093598in}}%
\pgfpathlineto{\pgfqpoint{2.660272in}{2.081635in}}%
\pgfpathclose%
\pgfusepath{fill}%
\end{pgfscope}%
\begin{pgfscope}%
\pgfpathrectangle{\pgfqpoint{0.708921in}{0.208778in}}{\pgfqpoint{3.800000in}{3.800000in}}%
\pgfusepath{clip}%
\pgfsetbuttcap%
\pgfsetroundjoin%
\definecolor{currentfill}{rgb}{0.086258,0.313666,0.086258}%
\pgfsetfillcolor{currentfill}%
\pgfsetlinewidth{0.000000pt}%
\definecolor{currentstroke}{rgb}{0.000000,0.000000,0.000000}%
\pgfsetstrokecolor{currentstroke}%
\pgfsetdash{}{0pt}%
\pgfpathmoveto{\pgfqpoint{3.091593in}{2.093598in}}%
\pgfpathlineto{\pgfqpoint{2.660272in}{2.718551in}}%
\pgfpathlineto{\pgfqpoint{2.660272in}{2.081635in}}%
\pgfpathlineto{\pgfqpoint{3.091593in}{2.093598in}}%
\pgfpathclose%
\pgfusepath{fill}%
\end{pgfscope}%
\begin{pgfscope}%
\pgfpathrectangle{\pgfqpoint{0.708921in}{0.208778in}}{\pgfqpoint{3.800000in}{3.800000in}}%
\pgfusepath{clip}%
\pgfsetbuttcap%
\pgfsetroundjoin%
\definecolor{currentfill}{rgb}{0.074668,0.271519,0.074668}%
\pgfsetfillcolor{currentfill}%
\pgfsetlinewidth{0.000000pt}%
\definecolor{currentstroke}{rgb}{0.000000,0.000000,0.000000}%
\pgfsetstrokecolor{currentstroke}%
\pgfsetdash{}{0pt}%
\pgfpathmoveto{\pgfqpoint{2.660272in}{2.081635in}}%
\pgfpathlineto{\pgfqpoint{2.875835in}{1.779529in}}%
\pgfpathlineto{\pgfqpoint{3.091593in}{2.093598in}}%
\pgfpathlineto{\pgfqpoint{2.660272in}{2.081635in}}%
\pgfpathclose%
\pgfusepath{fill}%
\end{pgfscope}%
\begin{pgfscope}%
\pgfpathrectangle{\pgfqpoint{0.708921in}{0.208778in}}{\pgfqpoint{3.800000in}{3.800000in}}%
\pgfusepath{clip}%
\pgfsetbuttcap%
\pgfsetroundjoin%
\definecolor{currentfill}{rgb}{0.074668,0.271519,0.074668}%
\pgfsetfillcolor{currentfill}%
\pgfsetlinewidth{0.000000pt}%
\definecolor{currentstroke}{rgb}{0.000000,0.000000,0.000000}%
\pgfsetstrokecolor{currentstroke}%
\pgfsetdash{}{0pt}%
\pgfpathmoveto{\pgfqpoint{2.228951in}{2.093598in}}%
\pgfpathlineto{\pgfqpoint{2.444709in}{1.779529in}}%
\pgfpathlineto{\pgfqpoint{2.660272in}{2.081635in}}%
\pgfpathlineto{\pgfqpoint{2.228951in}{2.093598in}}%
\pgfpathclose%
\pgfusepath{fill}%
\end{pgfscope}%
\end{pgfpicture}%
\makeatother%
\endgroup%
}
  \caption{From left to right: An opaque Pareto front; a translucent Pareto front showing the interior points above a sub-optimal front; and the sub-optimal front hiding the interior points from a different angle.}
  \label{fig:3d-mga}
\end{figure}