\section{\acs{temoa} and \acs{pygen}}
\label{section:temoa}

This thesis uses the tools \ac{temoa} and \ac{pygen} to establish benchmark
results for a typical \ac{esom}. \ac{temoa} is an open-source \ac{esom}
developed at North Carolina State University that uses \ac{milp} to develop
capacity-expansion scenarios \cite{decarolis_temoa_2010}. The key benefits of
\ac{temoa} are its open-source code, open data, and built-in uncertainty
analysis capabilities. These features address the need for greater transparency
in \ac{esom} modeling and robust assessment of future uncertainties
\cite{hunter_modeling_2013, fattahi_systemic_2020}. 
\ac{pygen} is another
open-source code, developed by this thesis' author, that wraps around \ac{temoa} 
facilitating rapid development of \ac{temoa} models and
enabling sensitivity analyses using a templated approach
\cite{dotson_influence_2022, dotson_python_2021}. These features of \ac{pygen}
reduce the learning curve and the cost of producing unique models in \ac{temoa}
\cite{dotson_influence_2022}.

A single \ac{temoa} run minimizes total system cost \cite{decarolis_temoa_2010},

\begin{align}
  C_{total} &= C_{loans} + C_{fixed} + C_{variable}
  \intertext{where}
  C_{loans} &= \text{the sum of all investment loan costs},\nonumber\\
  C_{fixed} &= \text{the sum of all fixed operating costs},\nonumber\\
  C_{variable} &= \text{the sum of all variable operating costs}.\nonumber
\end{align}

Each of these terms is amortized over the model time horizon. The decision
variables include the generation from each technology at time, $t$, and the
capacity of each technology in year, $y$. The dispatch model deviates slightly
from the model described in Section \ref{section:dispatch} by making the initial
storage level for energy storage technologies a decision variable, whereas the
dispatch model used in this thesis does not optimize initial storage and assumes
energy storage starts at zero. The detailed formulation of \ac{temoa}'s
constraints and equations are available online \cite{decarolis_temoa_2010}.
% (\textcolor{red}{maybe in an appendix?}). 

\subsection{\acl{mga}}
\label{section:mga}
\ac{temoa}'s built-in method for uncertainty analysis is the \ac{hsj}
formulation of \ac{mga}. This algorithm is designed to handle
\textit{structural} uncertainty, which presumes to account for unmodeled
objectives. The steps for \ac{hsj} are \cite{decarolis_using_2011,
dotson_influence_2022}:
\begin{enumerate}
  \item obtain an optimal solution by any method,
  \item add a user-specified amount of slack to the objective function value
  from the first step,
  \item use the adjusted objective function value as an upper bound constraint,
  \item generate a new objective function that minimizes the sum of all decision
  variables,
  \item iterate the procedure,
  \item stop the \ac{mga} when no significant changes are observed.
\end{enumerate}
The mathematical formulation of this algorithm is to
\begin{align}
  \intertext{minimize:}
  p &= \sum_{k\in K} x_k,
  \intertext{subject to:}
  f_j\left(\vec{x}\right) &\leq T_j \quad\forall \quad j,\\
  \vec{x}&\in X,
  \intertext{where}
  p &= \text{the new objective function}\nonumber,\\
  x_k &= \text{the $k^{th}$ decision variable with a nonzero value in previous solutions}\nonumber,\\
  f_j\left(\vec{x}\right) &= \text{the $j^{th}$ original objective function},\nonumber\\
  T_j &= \text{the slack-adjusted target value},\nonumber\\
  X &= \text{the set of all feasible solutions}.\nonumber
\end{align}
This procedure results in a small set of maximally different solutions for
modelers to interpret. In this way, \ac{mga} efficiently proposes alternatives
that may capture unmodeled objectives, such as political expediency or social
acceptance. However, this method depends on a single objective function which
does not guarantee that these alternative solutions will be optimal or 
near-optimal for any other measurable objective.


I created \ac{pygen} as an initial exploration on repeatable analysis and
extending the functionality of an existing \ac{esom}. While successful in
that regard, \ac{pygen} could not overcome \ac{temoa}'s inherent limitations
on optimizing multiple objectives and the inability to modify its objective 
function. Addressing these limits led to my developing \ac{osier}.

