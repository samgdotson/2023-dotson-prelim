\section{\acs{temoa} and \acs{pygen}}
\label{section:temoa}

This thesis uses the tools \ac{temoa} and \ac{pygen} 
to establish benchmark results for a typical \acp{esom}.
\ac{temoa} is an open-source \ac{esom} developed at North
Carolina State University that uses \ac{milp} to develop
capacity-expansion scenarios \cite{decarolis_temoa_2010}. The 
key benefits of \ac{temoa} are its open-source code, open data, 
and built-in uncertainty analysis capabilities. These features 
address the need for greater transparency in \ac{esom} modeling 
and robust assessment of future uncertainties \cite{hunter_modeling_2013, fattahi_systemic_2020}.

A single \ac{temoa} run minimizes total system cost \cite{decarolis_temoa_2010},

\begin{align}
  C_{total} &= C_{loans} + C_{fixed} + C_{variable}
  \intertext{where}
  C_{loans} &= \text{the sum of all investment loan costs},\nonumber\\
  C_{fixed} &= \text{the sum of all fixed operating costs},\nonumber\\
  C_{variable} &= \text{the sum of all variable operating costs}.\nonumber
\end{align}

Each of these terms is amortized over the model time horizon. The decision 
variables include the generation from each technology at time, $t$, along with 
the capacity of each technology in
year, $y$. The dispatch model deviates slightly from the model described in 
Section \ref{section:dispatch} by making the initial storage level for energy
storage technologies a decision variable, whereas the dispatch model used in this 
thesis does not optimize initial storage and assumes energy storage starts at zero. 
The detailed formulation of \ac{temoa}'s constraints and equations are available 
online \cite{decarolis_temoa_2010} (\textcolor{red}{maybe in an appendix?}). 
\ac{temoa}'s built-in method for uncertainty analysis is the \ac{hsj} formulation of
\ac{mga}. This algorithm is designed to handle \textit{structural} uncertainty which 
assumes unmodeled objectives.