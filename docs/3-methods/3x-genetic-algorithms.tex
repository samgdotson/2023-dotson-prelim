\section{Genetic Algorithms}
\label{section:genetic-algorithms}

Rather than rely on \ac{lp} to model future capacity requirements, in this
thesis, \acp{ga} assume the role of investment optimizer. \acp{ga} share a
fundamental algorithmic structure, which is \cite{blank_pymoo_2020}
\begin{enumerate}
    \item \textbf{Initialize} a starting population of $N_p$ individuals, where
    each individual has a set of ``genes'' that are randomly chosen from the
    bounds of the decision variables.
    \item Each individual in the population is \textbf{evaluated} for
    ``fitness.'' 
    \item The \textbf{fittest}, $N_f$ individuals ``survive'' and persist in the
    next generation.
    \item A ``selection'' operator \textbf{chooses} among the surviving
    individuals to mate.
    \item The parents are \textbf{combined} using a ``crossover'' operator,
    thereby filling the remaining $N_p - N_f$ individuals for the next
    generation.
    \item The offspring are finally \textbf{mutated} with some probability,
    $\mu$, to improve genetic diversity.
\end{enumerate}
\noindent
Figure \ref{fig:genetic-alg} illustrates the flow of these steps applied to an
energy systems model.

\begin{figure}[ht]
        \centering
        \begin{tikzpicture}[node distance=1.7cm]
                \tikzstyle{every node}=[font=\small] \node (1) [lbblock]
                {\textbf{Create initial population\\ of capacity sets}}; \node
                (2) [lbblock, below of=1] {\textbf{Evaluate dispatch model and
                calculate objectives}}; \node (3) [lbblock, below of=2]
                {\textbf{Survival}}; \node (4) [lbblock, below of=3]
                {\textbf{Selection}}; \node (5) [lbblock, below of=4]
                {\textbf{Crossover}}; \node (6) [lbblock, below of=5]
                {\textbf{Mutation}}; \node (7) [lbblock, below of=6] {\textbf{Is
                the termination \\ criteria satisfied?}}; \node (8) [loblock,
                below of=7] {\textbf{Done}}; \draw [arrow] (1) -- (2); \draw
                [arrow] (2) -- (3); \draw [arrow] (3) -- (4); \draw [arrow] (4)
                -- (5); \draw [arrow] (5) -- (6); \draw [arrow] (6) -- (7);
                \draw [arrow] (7) -- (8); \draw [arrow] (7) -- node[anchor=east]
                {yes} (8); \draw [arrow] (7) -- ([shift={(0.5cm,0cm)}]7.east)--
                node[anchor=west] {no} ([shift={(0.5cm,0cm)}]2.east)--(2);
        \end{tikzpicture}
        \caption{The basic flow of the \ac{ga} used in this thesis.}
        \label{fig:genetic-alg}
\end{figure}

\subsection{Specific \Aclp{ga}} The variety of \acp{ga} comes from different
types of operators being applied to the selection, crossover, and mutation
steps. Section \ref{section:moo-in-energy} showed that \ac{nsga2} is a popular
genetic algorithm choice. However, this algorithm performs poorly with greater
than three objectives \cite{deb_fast_2002, seada_unified_2016}. In this thesis,
I use a more modern algorithm, \ac{unsga3}. \ac{unsga3} builds on its
predecessors \ac{nsga2} and \ac{nsga3} by unifying efficient solutions of mono-,
multi-, and many-objective problems in a single algorithm.


\ac{nsga2} improves on the basic \ac{ga} by introducing a more sophisticated
mating and selection algorithms. Instead of random selection, the individuals
are sorted by rank (i.e. fitness) and crowding distance in binary tournament
mating selection. The crowding distance is simply the Manhattan distance between
individuals. A greater crowding distance is desirable to preserve diversity and
since the extreme points are maximally diverse they should always persist and
are therefore assigned a crowding distance of infinity \cite{deb_fast_2002}.

The successor to \ac{nsga2}, \ac{nsga3}, enhances the many-objective
capabilities of the former by introducing reference directions. Reference
directions are used for initialization and the survival steps. In addition to
fitness, individuals are chosen based on their proximity to a reference line,
thus ensuring population diversity which greatly important for many-objective
problems. Since diversity is handled by reference directions, individuals are
selected randomly for mating. References directions are rays passing through
uniformly spaced points on the unit simplex \cite{seada_unified_2016,
blank_generating_2021}. In this thesis, I use the Riesz s- Energy method
described by Blank et al. to calculate these points for a problem with an
arbitrary number of objectives \cite{blank_generating_2021}. Figure
\ref{fig:ref-dirs} illustrates a set of initialized reference directions.

\begin{figure}[h]
  \centering
  \resizebox{0.6\columnwidth}{!}{%% Creator: Matplotlib, PGF backend
%%
%% To include the figure in your LaTeX document, write
%%   \input{<filename>.pgf}
%%
%% Make sure the required packages are loaded in your preamble
%%   \usepackage{pgf}
%%
%% Also ensure that all the required font packages are loaded; for instance,
%% the lmodern package is sometimes necessary when using math font.
%%   \usepackage{lmodern}
%%
%% Figures using additional raster images can only be included by \input if
%% they are in the same directory as the main LaTeX file. For loading figures
%% from other directories you can use the `import` package
%%   \usepackage{import}
%%
%% and then include the figures with
%%   \import{<path to file>}{<filename>.pgf}
%%
%% Matplotlib used the following preamble
%%   \usepackage{fontspec}
%%   \setmainfont{DejaVuSerif.ttf}[Path=\detokenize{C:/Users/samgd/anaconda3/Lib/site-packages/matplotlib/mpl-data/fonts/ttf/}]
%%   \setsansfont{DejaVuSans.ttf}[Path=\detokenize{C:/Users/samgd/anaconda3/Lib/site-packages/matplotlib/mpl-data/fonts/ttf/}]
%%   \setmonofont{DejaVuSansMono.ttf}[Path=\detokenize{C:/Users/samgd/anaconda3/Lib/site-packages/matplotlib/mpl-data/fonts/ttf/}]
%%
\begingroup%
\makeatletter%
\begin{pgfpicture}%
\pgfpathrectangle{\pgfpointorigin}{\pgfqpoint{6.594429in}{6.401491in}}%
\pgfusepath{use as bounding box, clip}%
\begin{pgfscope}%
\pgfsetbuttcap%
\pgfsetmiterjoin%
\definecolor{currentfill}{rgb}{1.000000,1.000000,1.000000}%
\pgfsetfillcolor{currentfill}%
\pgfsetlinewidth{0.000000pt}%
\definecolor{currentstroke}{rgb}{0.000000,0.000000,0.000000}%
\pgfsetstrokecolor{currentstroke}%
\pgfsetdash{}{0pt}%
\pgfpathmoveto{\pgfqpoint{0.000000in}{0.000000in}}%
\pgfpathlineto{\pgfqpoint{6.594429in}{0.000000in}}%
\pgfpathlineto{\pgfqpoint{6.594429in}{6.401491in}}%
\pgfpathlineto{\pgfqpoint{0.000000in}{6.401491in}}%
\pgfpathlineto{\pgfqpoint{0.000000in}{0.000000in}}%
\pgfpathclose%
\pgfusepath{fill}%
\end{pgfscope}%
\begin{pgfscope}%
\pgfsetbuttcap%
\pgfsetmiterjoin%
\definecolor{currentfill}{rgb}{1.000000,1.000000,1.000000}%
\pgfsetfillcolor{currentfill}%
\pgfsetlinewidth{0.000000pt}%
\definecolor{currentstroke}{rgb}{0.000000,0.000000,0.000000}%
\pgfsetstrokecolor{currentstroke}%
\pgfsetstrokeopacity{0.000000}%
\pgfsetdash{}{0pt}%
\pgfpathmoveto{\pgfqpoint{0.454429in}{0.261491in}}%
\pgfpathlineto{\pgfqpoint{6.494429in}{0.261491in}}%
\pgfpathlineto{\pgfqpoint{6.494429in}{6.301491in}}%
\pgfpathlineto{\pgfqpoint{0.454429in}{6.301491in}}%
\pgfpathlineto{\pgfqpoint{0.454429in}{0.261491in}}%
\pgfpathclose%
\pgfusepath{fill}%
\end{pgfscope}%
\begin{pgfscope}%
\pgfsetbuttcap%
\pgfsetmiterjoin%
\definecolor{currentfill}{rgb}{0.950000,0.950000,0.950000}%
\pgfsetfillcolor{currentfill}%
\pgfsetfillopacity{0.500000}%
\pgfsetlinewidth{1.003750pt}%
\definecolor{currentstroke}{rgb}{0.950000,0.950000,0.950000}%
\pgfsetstrokecolor{currentstroke}%
\pgfsetstrokeopacity{0.500000}%
\pgfsetdash{}{0pt}%
\pgfpathmoveto{\pgfqpoint{3.556051in}{4.199655in}}%
\pgfpathlineto{\pgfqpoint{6.220291in}{2.364023in}}%
\pgfpathlineto{\pgfqpoint{6.393980in}{4.427336in}}%
\pgfpathlineto{\pgfqpoint{3.556051in}{6.256232in}}%
\pgfusepath{stroke,fill}%
\end{pgfscope}%
\begin{pgfscope}%
\pgfsetbuttcap%
\pgfsetmiterjoin%
\definecolor{currentfill}{rgb}{0.900000,0.900000,0.900000}%
\pgfsetfillcolor{currentfill}%
\pgfsetfillopacity{0.500000}%
\pgfsetlinewidth{1.003750pt}%
\definecolor{currentstroke}{rgb}{0.900000,0.900000,0.900000}%
\pgfsetstrokecolor{currentstroke}%
\pgfsetstrokeopacity{0.500000}%
\pgfsetdash{}{0pt}%
\pgfpathmoveto{\pgfqpoint{3.556051in}{4.199655in}}%
\pgfpathlineto{\pgfqpoint{0.891811in}{2.364023in}}%
\pgfpathlineto{\pgfqpoint{0.718122in}{4.427336in}}%
\pgfpathlineto{\pgfqpoint{3.556051in}{6.256232in}}%
\pgfusepath{stroke,fill}%
\end{pgfscope}%
\begin{pgfscope}%
\pgfsetbuttcap%
\pgfsetmiterjoin%
\definecolor{currentfill}{rgb}{0.925000,0.925000,0.925000}%
\pgfsetfillcolor{currentfill}%
\pgfsetfillopacity{0.500000}%
\pgfsetlinewidth{1.003750pt}%
\definecolor{currentstroke}{rgb}{0.925000,0.925000,0.925000}%
\pgfsetstrokecolor{currentstroke}%
\pgfsetstrokeopacity{0.500000}%
\pgfsetdash{}{0pt}%
\pgfpathmoveto{\pgfqpoint{3.556051in}{4.199655in}}%
\pgfpathlineto{\pgfqpoint{0.891811in}{2.364023in}}%
\pgfpathlineto{\pgfqpoint{3.556051in}{0.303578in}}%
\pgfpathlineto{\pgfqpoint{6.220291in}{2.364023in}}%
\pgfusepath{stroke,fill}%
\end{pgfscope}%
\begin{pgfscope}%
\pgfsetrectcap%
\pgfsetroundjoin%
\pgfsetlinewidth{0.803000pt}%
\definecolor{currentstroke}{rgb}{0.000000,0.000000,0.000000}%
\pgfsetstrokecolor{currentstroke}%
\pgfsetdash{}{0pt}%
\pgfpathmoveto{\pgfqpoint{6.220291in}{2.364023in}}%
\pgfpathlineto{\pgfqpoint{3.556051in}{0.303578in}}%
\pgfusepath{stroke}%
\end{pgfscope}%
\begin{pgfscope}%
\definecolor{textcolor}{rgb}{0.000000,0.000000,0.000000}%
\pgfsetstrokecolor{textcolor}%
\pgfsetfillcolor{textcolor}%
\pgftext[x=5.194308in, y=0.828835in, left, base,rotate=37.717305]{\color{textcolor}\rmfamily\fontsize{14.000000}{16.800000}\selectfont \(\displaystyle f_1\)}%
\end{pgfscope}%
\begin{pgfscope}%
\pgfsetbuttcap%
\pgfsetroundjoin%
\pgfsetlinewidth{0.803000pt}%
\definecolor{currentstroke}{rgb}{0.690196,0.690196,0.690196}%
\pgfsetstrokecolor{currentstroke}%
\pgfsetdash{}{0pt}%
\pgfpathmoveto{\pgfqpoint{6.059943in}{2.240014in}}%
\pgfpathlineto{\pgfqpoint{3.395199in}{4.088829in}}%
\pgfpathlineto{\pgfqpoint{3.385284in}{6.146181in}}%
\pgfusepath{stroke}%
\end{pgfscope}%
\begin{pgfscope}%
\pgfsetbuttcap%
\pgfsetroundjoin%
\pgfsetlinewidth{0.803000pt}%
\definecolor{currentstroke}{rgb}{0.690196,0.690196,0.690196}%
\pgfsetstrokecolor{currentstroke}%
\pgfsetdash{}{0pt}%
\pgfpathmoveto{\pgfqpoint{5.614001in}{1.895136in}}%
\pgfpathlineto{\pgfqpoint{2.948198in}{3.780850in}}%
\pgfpathlineto{\pgfqpoint{2.910343in}{5.840107in}}%
\pgfusepath{stroke}%
\end{pgfscope}%
\begin{pgfscope}%
\pgfsetbuttcap%
\pgfsetroundjoin%
\pgfsetlinewidth{0.803000pt}%
\definecolor{currentstroke}{rgb}{0.690196,0.690196,0.690196}%
\pgfsetstrokecolor{currentstroke}%
\pgfsetdash{}{0pt}%
\pgfpathmoveto{\pgfqpoint{5.158858in}{1.543142in}}%
\pgfpathlineto{\pgfqpoint{2.492487in}{3.466871in}}%
\pgfpathlineto{\pgfqpoint{2.425567in}{5.527694in}}%
\pgfusepath{stroke}%
\end{pgfscope}%
\begin{pgfscope}%
\pgfsetbuttcap%
\pgfsetroundjoin%
\pgfsetlinewidth{0.803000pt}%
\definecolor{currentstroke}{rgb}{0.690196,0.690196,0.690196}%
\pgfsetstrokecolor{currentstroke}%
\pgfsetdash{}{0pt}%
\pgfpathmoveto{\pgfqpoint{4.694225in}{1.183808in}}%
\pgfpathlineto{\pgfqpoint{2.027811in}{3.146714in}}%
\pgfpathlineto{\pgfqpoint{1.930646in}{5.208744in}}%
\pgfusepath{stroke}%
\end{pgfscope}%
\begin{pgfscope}%
\pgfsetbuttcap%
\pgfsetroundjoin%
\pgfsetlinewidth{0.803000pt}%
\definecolor{currentstroke}{rgb}{0.690196,0.690196,0.690196}%
\pgfsetstrokecolor{currentstroke}%
\pgfsetdash{}{0pt}%
\pgfpathmoveto{\pgfqpoint{4.219802in}{0.816904in}}%
\pgfpathlineto{\pgfqpoint{1.553901in}{2.820196in}}%
\pgfpathlineto{\pgfqpoint{1.425259in}{4.883049in}}%
\pgfusepath{stroke}%
\end{pgfscope}%
\begin{pgfscope}%
\pgfsetbuttcap%
\pgfsetroundjoin%
\pgfsetlinewidth{0.803000pt}%
\definecolor{currentstroke}{rgb}{0.690196,0.690196,0.690196}%
\pgfsetstrokecolor{currentstroke}%
\pgfsetdash{}{0pt}%
\pgfpathmoveto{\pgfqpoint{3.735277in}{0.442186in}}%
\pgfpathlineto{\pgfqpoint{1.070480in}{2.487124in}}%
\pgfpathlineto{\pgfqpoint{0.909071in}{4.550392in}}%
\pgfusepath{stroke}%
\end{pgfscope}%
\begin{pgfscope}%
\pgfsetrectcap%
\pgfsetroundjoin%
\pgfsetlinewidth{0.803000pt}%
\definecolor{currentstroke}{rgb}{0.000000,0.000000,0.000000}%
\pgfsetstrokecolor{currentstroke}%
\pgfsetdash{}{0pt}%
\pgfpathmoveto{\pgfqpoint{6.037401in}{2.255654in}}%
\pgfpathlineto{\pgfqpoint{6.105090in}{2.208691in}}%
\pgfusepath{stroke}%
\end{pgfscope}%
\begin{pgfscope}%
\definecolor{textcolor}{rgb}{0.000000,0.000000,0.000000}%
\pgfsetstrokecolor{textcolor}%
\pgfsetfillcolor{textcolor}%
\pgftext[x=6.195241in,y=2.042420in,,top]{\color{textcolor}\rmfamily\fontsize{10.000000}{12.000000}\selectfont 0.0}%
\end{pgfscope}%
\begin{pgfscope}%
\pgfsetrectcap%
\pgfsetroundjoin%
\pgfsetlinewidth{0.803000pt}%
\definecolor{currentstroke}{rgb}{0.000000,0.000000,0.000000}%
\pgfsetstrokecolor{currentstroke}%
\pgfsetdash{}{0pt}%
\pgfpathmoveto{\pgfqpoint{5.591438in}{1.911097in}}%
\pgfpathlineto{\pgfqpoint{5.659192in}{1.863169in}}%
\pgfusepath{stroke}%
\end{pgfscope}%
\begin{pgfscope}%
\definecolor{textcolor}{rgb}{0.000000,0.000000,0.000000}%
\pgfsetstrokecolor{textcolor}%
\pgfsetfillcolor{textcolor}%
\pgftext[x=5.750761in,y=1.695586in,,top]{\color{textcolor}\rmfamily\fontsize{10.000000}{12.000000}\selectfont 0.2}%
\end{pgfscope}%
\begin{pgfscope}%
\pgfsetrectcap%
\pgfsetroundjoin%
\pgfsetlinewidth{0.803000pt}%
\definecolor{currentstroke}{rgb}{0.000000,0.000000,0.000000}%
\pgfsetstrokecolor{currentstroke}%
\pgfsetdash{}{0pt}%
\pgfpathmoveto{\pgfqpoint{5.136277in}{1.559433in}}%
\pgfpathlineto{\pgfqpoint{5.204084in}{1.510512in}}%
\pgfusepath{stroke}%
\end{pgfscope}%
\begin{pgfscope}%
\definecolor{textcolor}{rgb}{0.000000,0.000000,0.000000}%
\pgfsetstrokecolor{textcolor}%
\pgfsetfillcolor{textcolor}%
\pgftext[x=5.297112in,y=1.341597in,,top]{\color{textcolor}\rmfamily\fontsize{10.000000}{12.000000}\selectfont 0.4}%
\end{pgfscope}%
\begin{pgfscope}%
\pgfsetrectcap%
\pgfsetroundjoin%
\pgfsetlinewidth{0.803000pt}%
\definecolor{currentstroke}{rgb}{0.000000,0.000000,0.000000}%
\pgfsetstrokecolor{currentstroke}%
\pgfsetdash{}{0pt}%
\pgfpathmoveto{\pgfqpoint{4.671630in}{1.200441in}}%
\pgfpathlineto{\pgfqpoint{4.739479in}{1.150494in}}%
\pgfusepath{stroke}%
\end{pgfscope}%
\begin{pgfscope}%
\definecolor{textcolor}{rgb}{0.000000,0.000000,0.000000}%
\pgfsetstrokecolor{textcolor}%
\pgfsetfillcolor{textcolor}%
\pgftext[x=4.834005in,y=0.980229in,,top]{\color{textcolor}\rmfamily\fontsize{10.000000}{12.000000}\selectfont 0.6}%
\end{pgfscope}%
\begin{pgfscope}%
\pgfsetrectcap%
\pgfsetroundjoin%
\pgfsetlinewidth{0.803000pt}%
\definecolor{currentstroke}{rgb}{0.000000,0.000000,0.000000}%
\pgfsetstrokecolor{currentstroke}%
\pgfsetdash{}{0pt}%
\pgfpathmoveto{\pgfqpoint{4.197199in}{0.833889in}}%
\pgfpathlineto{\pgfqpoint{4.265075in}{0.782884in}}%
\pgfusepath{stroke}%
\end{pgfscope}%
\begin{pgfscope}%
\definecolor{textcolor}{rgb}{0.000000,0.000000,0.000000}%
\pgfsetstrokecolor{textcolor}%
\pgfsetfillcolor{textcolor}%
\pgftext[x=4.361143in,y=0.611249in,,top]{\color{textcolor}\rmfamily\fontsize{10.000000}{12.000000}\selectfont 0.8}%
\end{pgfscope}%
\begin{pgfscope}%
\pgfsetrectcap%
\pgfsetroundjoin%
\pgfsetlinewidth{0.803000pt}%
\definecolor{currentstroke}{rgb}{0.000000,0.000000,0.000000}%
\pgfsetstrokecolor{currentstroke}%
\pgfsetdash{}{0pt}%
\pgfpathmoveto{\pgfqpoint{3.712669in}{0.459535in}}%
\pgfpathlineto{\pgfqpoint{3.780559in}{0.407438in}}%
\pgfusepath{stroke}%
\end{pgfscope}%
\begin{pgfscope}%
\definecolor{textcolor}{rgb}{0.000000,0.000000,0.000000}%
\pgfsetstrokecolor{textcolor}%
\pgfsetfillcolor{textcolor}%
\pgftext[x=3.878213in,y=0.234413in,,top]{\color{textcolor}\rmfamily\fontsize{10.000000}{12.000000}\selectfont 1.0}%
\end{pgfscope}%
\begin{pgfscope}%
\pgfsetrectcap%
\pgfsetroundjoin%
\pgfsetlinewidth{0.803000pt}%
\definecolor{currentstroke}{rgb}{0.000000,0.000000,0.000000}%
\pgfsetstrokecolor{currentstroke}%
\pgfsetdash{}{0pt}%
\pgfpathmoveto{\pgfqpoint{0.891811in}{2.364023in}}%
\pgfpathlineto{\pgfqpoint{3.556051in}{0.303578in}}%
\pgfusepath{stroke}%
\end{pgfscope}%
\begin{pgfscope}%
\definecolor{textcolor}{rgb}{0.000000,0.000000,0.000000}%
\pgfsetstrokecolor{textcolor}%
\pgfsetfillcolor{textcolor}%
\pgftext[x=1.781011in, y=0.934619in, left, base,rotate=322.282695]{\color{textcolor}\rmfamily\fontsize{14.000000}{16.800000}\selectfont \(\displaystyle f_2\)}%
\end{pgfscope}%
\begin{pgfscope}%
\pgfsetbuttcap%
\pgfsetroundjoin%
\pgfsetlinewidth{0.803000pt}%
\definecolor{currentstroke}{rgb}{0.690196,0.690196,0.690196}%
\pgfsetstrokecolor{currentstroke}%
\pgfsetdash{}{0pt}%
\pgfpathmoveto{\pgfqpoint{3.726818in}{6.146181in}}%
\pgfpathlineto{\pgfqpoint{3.716903in}{4.088829in}}%
\pgfpathlineto{\pgfqpoint{1.052159in}{2.240014in}}%
\pgfusepath{stroke}%
\end{pgfscope}%
\begin{pgfscope}%
\pgfsetbuttcap%
\pgfsetroundjoin%
\pgfsetlinewidth{0.803000pt}%
\definecolor{currentstroke}{rgb}{0.690196,0.690196,0.690196}%
\pgfsetstrokecolor{currentstroke}%
\pgfsetdash{}{0pt}%
\pgfpathmoveto{\pgfqpoint{4.201759in}{5.840107in}}%
\pgfpathlineto{\pgfqpoint{4.163904in}{3.780850in}}%
\pgfpathlineto{\pgfqpoint{1.498101in}{1.895136in}}%
\pgfusepath{stroke}%
\end{pgfscope}%
\begin{pgfscope}%
\pgfsetbuttcap%
\pgfsetroundjoin%
\pgfsetlinewidth{0.803000pt}%
\definecolor{currentstroke}{rgb}{0.690196,0.690196,0.690196}%
\pgfsetstrokecolor{currentstroke}%
\pgfsetdash{}{0pt}%
\pgfpathmoveto{\pgfqpoint{4.686535in}{5.527694in}}%
\pgfpathlineto{\pgfqpoint{4.619615in}{3.466871in}}%
\pgfpathlineto{\pgfqpoint{1.953244in}{1.543142in}}%
\pgfusepath{stroke}%
\end{pgfscope}%
\begin{pgfscope}%
\pgfsetbuttcap%
\pgfsetroundjoin%
\pgfsetlinewidth{0.803000pt}%
\definecolor{currentstroke}{rgb}{0.690196,0.690196,0.690196}%
\pgfsetstrokecolor{currentstroke}%
\pgfsetdash{}{0pt}%
\pgfpathmoveto{\pgfqpoint{5.181456in}{5.208744in}}%
\pgfpathlineto{\pgfqpoint{5.084291in}{3.146714in}}%
\pgfpathlineto{\pgfqpoint{2.417877in}{1.183808in}}%
\pgfusepath{stroke}%
\end{pgfscope}%
\begin{pgfscope}%
\pgfsetbuttcap%
\pgfsetroundjoin%
\pgfsetlinewidth{0.803000pt}%
\definecolor{currentstroke}{rgb}{0.690196,0.690196,0.690196}%
\pgfsetstrokecolor{currentstroke}%
\pgfsetdash{}{0pt}%
\pgfpathmoveto{\pgfqpoint{5.686843in}{4.883049in}}%
\pgfpathlineto{\pgfqpoint{5.558201in}{2.820196in}}%
\pgfpathlineto{\pgfqpoint{2.892300in}{0.816904in}}%
\pgfusepath{stroke}%
\end{pgfscope}%
\begin{pgfscope}%
\pgfsetbuttcap%
\pgfsetroundjoin%
\pgfsetlinewidth{0.803000pt}%
\definecolor{currentstroke}{rgb}{0.690196,0.690196,0.690196}%
\pgfsetstrokecolor{currentstroke}%
\pgfsetdash{}{0pt}%
\pgfpathmoveto{\pgfqpoint{6.203031in}{4.550392in}}%
\pgfpathlineto{\pgfqpoint{6.041622in}{2.487124in}}%
\pgfpathlineto{\pgfqpoint{3.376825in}{0.442186in}}%
\pgfusepath{stroke}%
\end{pgfscope}%
\begin{pgfscope}%
\pgfsetrectcap%
\pgfsetroundjoin%
\pgfsetlinewidth{0.803000pt}%
\definecolor{currentstroke}{rgb}{0.000000,0.000000,0.000000}%
\pgfsetstrokecolor{currentstroke}%
\pgfsetdash{}{0pt}%
\pgfpathmoveto{\pgfqpoint{1.074701in}{2.255654in}}%
\pgfpathlineto{\pgfqpoint{1.007012in}{2.208691in}}%
\pgfusepath{stroke}%
\end{pgfscope}%
\begin{pgfscope}%
\definecolor{textcolor}{rgb}{0.000000,0.000000,0.000000}%
\pgfsetstrokecolor{textcolor}%
\pgfsetfillcolor{textcolor}%
\pgftext[x=0.916861in,y=2.042420in,,top]{\color{textcolor}\rmfamily\fontsize{10.000000}{12.000000}\selectfont 0.0}%
\end{pgfscope}%
\begin{pgfscope}%
\pgfsetrectcap%
\pgfsetroundjoin%
\pgfsetlinewidth{0.803000pt}%
\definecolor{currentstroke}{rgb}{0.000000,0.000000,0.000000}%
\pgfsetstrokecolor{currentstroke}%
\pgfsetdash{}{0pt}%
\pgfpathmoveto{\pgfqpoint{1.520664in}{1.911097in}}%
\pgfpathlineto{\pgfqpoint{1.452910in}{1.863169in}}%
\pgfusepath{stroke}%
\end{pgfscope}%
\begin{pgfscope}%
\definecolor{textcolor}{rgb}{0.000000,0.000000,0.000000}%
\pgfsetstrokecolor{textcolor}%
\pgfsetfillcolor{textcolor}%
\pgftext[x=1.361341in,y=1.695586in,,top]{\color{textcolor}\rmfamily\fontsize{10.000000}{12.000000}\selectfont 0.2}%
\end{pgfscope}%
\begin{pgfscope}%
\pgfsetrectcap%
\pgfsetroundjoin%
\pgfsetlinewidth{0.803000pt}%
\definecolor{currentstroke}{rgb}{0.000000,0.000000,0.000000}%
\pgfsetstrokecolor{currentstroke}%
\pgfsetdash{}{0pt}%
\pgfpathmoveto{\pgfqpoint{1.975825in}{1.559433in}}%
\pgfpathlineto{\pgfqpoint{1.908018in}{1.510512in}}%
\pgfusepath{stroke}%
\end{pgfscope}%
\begin{pgfscope}%
\definecolor{textcolor}{rgb}{0.000000,0.000000,0.000000}%
\pgfsetstrokecolor{textcolor}%
\pgfsetfillcolor{textcolor}%
\pgftext[x=1.814990in,y=1.341597in,,top]{\color{textcolor}\rmfamily\fontsize{10.000000}{12.000000}\selectfont 0.4}%
\end{pgfscope}%
\begin{pgfscope}%
\pgfsetrectcap%
\pgfsetroundjoin%
\pgfsetlinewidth{0.803000pt}%
\definecolor{currentstroke}{rgb}{0.000000,0.000000,0.000000}%
\pgfsetstrokecolor{currentstroke}%
\pgfsetdash{}{0pt}%
\pgfpathmoveto{\pgfqpoint{2.440472in}{1.200441in}}%
\pgfpathlineto{\pgfqpoint{2.372623in}{1.150494in}}%
\pgfusepath{stroke}%
\end{pgfscope}%
\begin{pgfscope}%
\definecolor{textcolor}{rgb}{0.000000,0.000000,0.000000}%
\pgfsetstrokecolor{textcolor}%
\pgfsetfillcolor{textcolor}%
\pgftext[x=2.278097in,y=0.980229in,,top]{\color{textcolor}\rmfamily\fontsize{10.000000}{12.000000}\selectfont 0.6}%
\end{pgfscope}%
\begin{pgfscope}%
\pgfsetrectcap%
\pgfsetroundjoin%
\pgfsetlinewidth{0.803000pt}%
\definecolor{currentstroke}{rgb}{0.000000,0.000000,0.000000}%
\pgfsetstrokecolor{currentstroke}%
\pgfsetdash{}{0pt}%
\pgfpathmoveto{\pgfqpoint{2.914903in}{0.833889in}}%
\pgfpathlineto{\pgfqpoint{2.847027in}{0.782884in}}%
\pgfusepath{stroke}%
\end{pgfscope}%
\begin{pgfscope}%
\definecolor{textcolor}{rgb}{0.000000,0.000000,0.000000}%
\pgfsetstrokecolor{textcolor}%
\pgfsetfillcolor{textcolor}%
\pgftext[x=2.750959in,y=0.611249in,,top]{\color{textcolor}\rmfamily\fontsize{10.000000}{12.000000}\selectfont 0.8}%
\end{pgfscope}%
\begin{pgfscope}%
\pgfsetrectcap%
\pgfsetroundjoin%
\pgfsetlinewidth{0.803000pt}%
\definecolor{currentstroke}{rgb}{0.000000,0.000000,0.000000}%
\pgfsetstrokecolor{currentstroke}%
\pgfsetdash{}{0pt}%
\pgfpathmoveto{\pgfqpoint{3.399432in}{0.459535in}}%
\pgfpathlineto{\pgfqpoint{3.331543in}{0.407438in}}%
\pgfusepath{stroke}%
\end{pgfscope}%
\begin{pgfscope}%
\definecolor{textcolor}{rgb}{0.000000,0.000000,0.000000}%
\pgfsetstrokecolor{textcolor}%
\pgfsetfillcolor{textcolor}%
\pgftext[x=3.233889in,y=0.234413in,,top]{\color{textcolor}\rmfamily\fontsize{10.000000}{12.000000}\selectfont 1.0}%
\end{pgfscope}%
\begin{pgfscope}%
\pgfsetrectcap%
\pgfsetroundjoin%
\pgfsetlinewidth{0.803000pt}%
\definecolor{currentstroke}{rgb}{0.000000,0.000000,0.000000}%
\pgfsetstrokecolor{currentstroke}%
\pgfsetdash{}{0pt}%
\pgfpathmoveto{\pgfqpoint{0.891811in}{2.364023in}}%
\pgfpathlineto{\pgfqpoint{0.718122in}{4.427336in}}%
\pgfusepath{stroke}%
\end{pgfscope}%
\begin{pgfscope}%
\definecolor{textcolor}{rgb}{0.000000,0.000000,0.000000}%
\pgfsetstrokecolor{textcolor}%
\pgfsetfillcolor{textcolor}%
\pgftext[x=0.140303in, y=3.444766in, left, base,rotate=274.811779]{\color{textcolor}\rmfamily\fontsize{14.000000}{16.800000}\selectfont \(\displaystyle f_3\)}%
\end{pgfscope}%
\begin{pgfscope}%
\pgfsetbuttcap%
\pgfsetroundjoin%
\pgfsetlinewidth{0.803000pt}%
\definecolor{currentstroke}{rgb}{0.690196,0.690196,0.690196}%
\pgfsetstrokecolor{currentstroke}%
\pgfsetdash{}{0pt}%
\pgfpathmoveto{\pgfqpoint{0.881394in}{2.487770in}}%
\pgfpathlineto{\pgfqpoint{3.556051in}{4.323411in}}%
\pgfpathlineto{\pgfqpoint{6.230708in}{2.487770in}}%
\pgfusepath{stroke}%
\end{pgfscope}%
\begin{pgfscope}%
\pgfsetbuttcap%
\pgfsetroundjoin%
\pgfsetlinewidth{0.803000pt}%
\definecolor{currentstroke}{rgb}{0.690196,0.690196,0.690196}%
\pgfsetstrokecolor{currentstroke}%
\pgfsetdash{}{0pt}%
\pgfpathmoveto{\pgfqpoint{0.852399in}{2.832214in}}%
\pgfpathlineto{\pgfqpoint{3.556051in}{4.667599in}}%
\pgfpathlineto{\pgfqpoint{6.259703in}{2.832214in}}%
\pgfusepath{stroke}%
\end{pgfscope}%
\begin{pgfscope}%
\pgfsetbuttcap%
\pgfsetroundjoin%
\pgfsetlinewidth{0.803000pt}%
\definecolor{currentstroke}{rgb}{0.690196,0.690196,0.690196}%
\pgfsetstrokecolor{currentstroke}%
\pgfsetdash{}{0pt}%
\pgfpathmoveto{\pgfqpoint{0.822768in}{3.184207in}}%
\pgfpathlineto{\pgfqpoint{3.556051in}{5.018910in}}%
\pgfpathlineto{\pgfqpoint{6.289334in}{3.184207in}}%
\pgfusepath{stroke}%
\end{pgfscope}%
\begin{pgfscope}%
\pgfsetbuttcap%
\pgfsetroundjoin%
\pgfsetlinewidth{0.803000pt}%
\definecolor{currentstroke}{rgb}{0.690196,0.690196,0.690196}%
\pgfsetstrokecolor{currentstroke}%
\pgfsetdash{}{0pt}%
\pgfpathmoveto{\pgfqpoint{0.792481in}{3.544001in}}%
\pgfpathlineto{\pgfqpoint{3.556051in}{5.377568in}}%
\pgfpathlineto{\pgfqpoint{6.319621in}{3.544001in}}%
\pgfusepath{stroke}%
\end{pgfscope}%
\begin{pgfscope}%
\pgfsetbuttcap%
\pgfsetroundjoin%
\pgfsetlinewidth{0.803000pt}%
\definecolor{currentstroke}{rgb}{0.690196,0.690196,0.690196}%
\pgfsetstrokecolor{currentstroke}%
\pgfsetdash{}{0pt}%
\pgfpathmoveto{\pgfqpoint{0.761515in}{3.911858in}}%
\pgfpathlineto{\pgfqpoint{3.556051in}{5.743804in}}%
\pgfpathlineto{\pgfqpoint{6.350587in}{3.911858in}}%
\pgfusepath{stroke}%
\end{pgfscope}%
\begin{pgfscope}%
\pgfsetbuttcap%
\pgfsetroundjoin%
\pgfsetlinewidth{0.803000pt}%
\definecolor{currentstroke}{rgb}{0.690196,0.690196,0.690196}%
\pgfsetstrokecolor{currentstroke}%
\pgfsetdash{}{0pt}%
\pgfpathmoveto{\pgfqpoint{0.729847in}{4.288052in}}%
\pgfpathlineto{\pgfqpoint{3.556051in}{6.117861in}}%
\pgfpathlineto{\pgfqpoint{6.382255in}{4.288052in}}%
\pgfusepath{stroke}%
\end{pgfscope}%
\begin{pgfscope}%
\pgfsetrectcap%
\pgfsetroundjoin%
\pgfsetlinewidth{0.803000pt}%
\definecolor{currentstroke}{rgb}{0.000000,0.000000,0.000000}%
\pgfsetstrokecolor{currentstroke}%
\pgfsetdash{}{0pt}%
\pgfpathmoveto{\pgfqpoint{0.904020in}{2.503299in}}%
\pgfpathlineto{\pgfqpoint{0.836078in}{2.456670in}}%
\pgfusepath{stroke}%
\end{pgfscope}%
\begin{pgfscope}%
\definecolor{textcolor}{rgb}{0.000000,0.000000,0.000000}%
\pgfsetstrokecolor{textcolor}%
\pgfsetfillcolor{textcolor}%
\pgftext[x=0.609795in,y=2.487770in,,top]{\color{textcolor}\rmfamily\fontsize{10.000000}{12.000000}\selectfont 0.0}%
\end{pgfscope}%
\begin{pgfscope}%
\pgfsetrectcap%
\pgfsetroundjoin%
\pgfsetlinewidth{0.803000pt}%
\definecolor{currentstroke}{rgb}{0.000000,0.000000,0.000000}%
\pgfsetstrokecolor{currentstroke}%
\pgfsetdash{}{0pt}%
\pgfpathmoveto{\pgfqpoint{0.875284in}{2.847749in}}%
\pgfpathlineto{\pgfqpoint{0.806564in}{2.801099in}}%
\pgfusepath{stroke}%
\end{pgfscope}%
\begin{pgfscope}%
\definecolor{textcolor}{rgb}{0.000000,0.000000,0.000000}%
\pgfsetstrokecolor{textcolor}%
\pgfsetfillcolor{textcolor}%
\pgftext[x=0.577856in,y=2.832214in,,top]{\color{textcolor}\rmfamily\fontsize{10.000000}{12.000000}\selectfont 0.2}%
\end{pgfscope}%
\begin{pgfscope}%
\pgfsetrectcap%
\pgfsetroundjoin%
\pgfsetlinewidth{0.803000pt}%
\definecolor{currentstroke}{rgb}{0.000000,0.000000,0.000000}%
\pgfsetstrokecolor{currentstroke}%
\pgfsetdash{}{0pt}%
\pgfpathmoveto{\pgfqpoint{0.845918in}{3.199746in}}%
\pgfpathlineto{\pgfqpoint{0.776403in}{3.153084in}}%
\pgfusepath{stroke}%
\end{pgfscope}%
\begin{pgfscope}%
\definecolor{textcolor}{rgb}{0.000000,0.000000,0.000000}%
\pgfsetstrokecolor{textcolor}%
\pgfsetfillcolor{textcolor}%
\pgftext[x=0.545216in,y=3.184207in,,top]{\color{textcolor}\rmfamily\fontsize{10.000000}{12.000000}\selectfont 0.4}%
\end{pgfscope}%
\begin{pgfscope}%
\pgfsetrectcap%
\pgfsetroundjoin%
\pgfsetlinewidth{0.803000pt}%
\definecolor{currentstroke}{rgb}{0.000000,0.000000,0.000000}%
\pgfsetstrokecolor{currentstroke}%
\pgfsetdash{}{0pt}%
\pgfpathmoveto{\pgfqpoint{0.815902in}{3.559540in}}%
\pgfpathlineto{\pgfqpoint{0.745572in}{3.512878in}}%
\pgfusepath{stroke}%
\end{pgfscope}%
\begin{pgfscope}%
\definecolor{textcolor}{rgb}{0.000000,0.000000,0.000000}%
\pgfsetstrokecolor{textcolor}%
\pgfsetfillcolor{textcolor}%
\pgftext[x=0.511854in,y=3.544001in,,top]{\color{textcolor}\rmfamily\fontsize{10.000000}{12.000000}\selectfont 0.6}%
\end{pgfscope}%
\begin{pgfscope}%
\pgfsetrectcap%
\pgfsetroundjoin%
\pgfsetlinewidth{0.803000pt}%
\definecolor{currentstroke}{rgb}{0.000000,0.000000,0.000000}%
\pgfsetstrokecolor{currentstroke}%
\pgfsetdash{}{0pt}%
\pgfpathmoveto{\pgfqpoint{0.785213in}{3.927393in}}%
\pgfpathlineto{\pgfqpoint{0.714050in}{3.880743in}}%
\pgfusepath{stroke}%
\end{pgfscope}%
\begin{pgfscope}%
\definecolor{textcolor}{rgb}{0.000000,0.000000,0.000000}%
\pgfsetstrokecolor{textcolor}%
\pgfsetfillcolor{textcolor}%
\pgftext[x=0.477743in,y=3.911858in,,top]{\color{textcolor}\rmfamily\fontsize{10.000000}{12.000000}\selectfont 0.8}%
\end{pgfscope}%
\begin{pgfscope}%
\pgfsetrectcap%
\pgfsetroundjoin%
\pgfsetlinewidth{0.803000pt}%
\definecolor{currentstroke}{rgb}{0.000000,0.000000,0.000000}%
\pgfsetstrokecolor{currentstroke}%
\pgfsetdash{}{0pt}%
\pgfpathmoveto{\pgfqpoint{0.753829in}{4.303579in}}%
\pgfpathlineto{\pgfqpoint{0.681813in}{4.256953in}}%
\pgfusepath{stroke}%
\end{pgfscope}%
\begin{pgfscope}%
\definecolor{textcolor}{rgb}{0.000000,0.000000,0.000000}%
\pgfsetstrokecolor{textcolor}%
\pgfsetfillcolor{textcolor}%
\pgftext[x=0.442860in,y=4.288052in,,top]{\color{textcolor}\rmfamily\fontsize{10.000000}{12.000000}\selectfont 1.0}%
\end{pgfscope}%
\begin{pgfscope}%
\pgfpathrectangle{\pgfqpoint{0.454429in}{0.261491in}}{\pgfqpoint{6.040000in}{6.040000in}}%
\pgfusepath{clip}%
\pgfsetbuttcap%
\pgfsetroundjoin%
\definecolor{currentfill}{rgb}{0.121569,0.466667,0.705882}%
\pgfsetfillcolor{currentfill}%
\pgfsetlinewidth{1.003750pt}%
\definecolor{currentstroke}{rgb}{0.121569,0.466667,0.705882}%
\pgfsetstrokecolor{currentstroke}%
\pgfsetdash{}{0pt}%
\pgfpathmoveto{\pgfqpoint{1.221805in}{2.438666in}}%
\pgfpathcurveto{\pgfqpoint{1.234827in}{2.438666in}}{\pgfqpoint{1.247318in}{2.443840in}}{\pgfqpoint{1.256527in}{2.453048in}}%
\pgfpathcurveto{\pgfqpoint{1.265735in}{2.462257in}}{\pgfqpoint{1.270909in}{2.474748in}}{\pgfqpoint{1.270909in}{2.487770in}}%
\pgfpathcurveto{\pgfqpoint{1.270909in}{2.500793in}}{\pgfqpoint{1.265735in}{2.513284in}}{\pgfqpoint{1.256527in}{2.522493in}}%
\pgfpathcurveto{\pgfqpoint{1.247318in}{2.531701in}}{\pgfqpoint{1.234827in}{2.536875in}}{\pgfqpoint{1.221805in}{2.536875in}}%
\pgfpathcurveto{\pgfqpoint{1.208782in}{2.536875in}}{\pgfqpoint{1.196291in}{2.531701in}}{\pgfqpoint{1.187082in}{2.522493in}}%
\pgfpathcurveto{\pgfqpoint{1.177874in}{2.513284in}}{\pgfqpoint{1.172700in}{2.500793in}}{\pgfqpoint{1.172700in}{2.487770in}}%
\pgfpathcurveto{\pgfqpoint{1.172700in}{2.474748in}}{\pgfqpoint{1.177874in}{2.462257in}}{\pgfqpoint{1.187082in}{2.453048in}}%
\pgfpathcurveto{\pgfqpoint{1.196291in}{2.443840in}}{\pgfqpoint{1.208782in}{2.438666in}}{\pgfqpoint{1.221805in}{2.438666in}}%
\pgfpathlineto{\pgfqpoint{1.221805in}{2.438666in}}%
\pgfpathclose%
\pgfusepath{stroke,fill}%
\end{pgfscope}%
\begin{pgfscope}%
\pgfpathrectangle{\pgfqpoint{0.454429in}{0.261491in}}{\pgfqpoint{6.040000in}{6.040000in}}%
\pgfusepath{clip}%
\pgfsetbuttcap%
\pgfsetroundjoin%
\definecolor{currentfill}{rgb}{0.121569,0.466667,0.705882}%
\pgfsetfillcolor{currentfill}%
\pgfsetlinewidth{1.003750pt}%
\definecolor{currentstroke}{rgb}{0.121569,0.466667,0.705882}%
\pgfsetstrokecolor{currentstroke}%
\pgfsetdash{}{0pt}%
\pgfpathmoveto{\pgfqpoint{2.417765in}{2.438666in}}%
\pgfpathcurveto{\pgfqpoint{2.430788in}{2.438666in}}{\pgfqpoint{2.443279in}{2.443840in}}{\pgfqpoint{2.452487in}{2.453048in}}%
\pgfpathcurveto{\pgfqpoint{2.461696in}{2.462257in}}{\pgfqpoint{2.466870in}{2.474748in}}{\pgfqpoint{2.466870in}{2.487770in}}%
\pgfpathcurveto{\pgfqpoint{2.466870in}{2.500793in}}{\pgfqpoint{2.461696in}{2.513284in}}{\pgfqpoint{2.452487in}{2.522493in}}%
\pgfpathcurveto{\pgfqpoint{2.443279in}{2.531701in}}{\pgfqpoint{2.430788in}{2.536875in}}{\pgfqpoint{2.417765in}{2.536875in}}%
\pgfpathcurveto{\pgfqpoint{2.404742in}{2.536875in}}{\pgfqpoint{2.392251in}{2.531701in}}{\pgfqpoint{2.383043in}{2.522493in}}%
\pgfpathcurveto{\pgfqpoint{2.373835in}{2.513284in}}{\pgfqpoint{2.368661in}{2.500793in}}{\pgfqpoint{2.368661in}{2.487770in}}%
\pgfpathcurveto{\pgfqpoint{2.368661in}{2.474748in}}{\pgfqpoint{2.373835in}{2.462257in}}{\pgfqpoint{2.383043in}{2.453048in}}%
\pgfpathcurveto{\pgfqpoint{2.392251in}{2.443840in}}{\pgfqpoint{2.404742in}{2.438666in}}{\pgfqpoint{2.417765in}{2.438666in}}%
\pgfpathlineto{\pgfqpoint{2.417765in}{2.438666in}}%
\pgfpathclose%
\pgfusepath{stroke,fill}%
\end{pgfscope}%
\begin{pgfscope}%
\pgfpathrectangle{\pgfqpoint{0.454429in}{0.261491in}}{\pgfqpoint{6.040000in}{6.040000in}}%
\pgfusepath{clip}%
\pgfsetbuttcap%
\pgfsetroundjoin%
\definecolor{currentfill}{rgb}{0.121569,0.466667,0.705882}%
\pgfsetfillcolor{currentfill}%
\pgfsetlinewidth{1.003750pt}%
\definecolor{currentstroke}{rgb}{0.121569,0.466667,0.705882}%
\pgfsetstrokecolor{currentstroke}%
\pgfsetdash{}{0pt}%
\pgfpathmoveto{\pgfqpoint{5.379798in}{2.438666in}}%
\pgfpathcurveto{\pgfqpoint{5.392821in}{2.438666in}}{\pgfqpoint{5.405312in}{2.443840in}}{\pgfqpoint{5.414520in}{2.453048in}}%
\pgfpathcurveto{\pgfqpoint{5.423729in}{2.462257in}}{\pgfqpoint{5.428903in}{2.474748in}}{\pgfqpoint{5.428903in}{2.487770in}}%
\pgfpathcurveto{\pgfqpoint{5.428903in}{2.500793in}}{\pgfqpoint{5.423729in}{2.513284in}}{\pgfqpoint{5.414520in}{2.522493in}}%
\pgfpathcurveto{\pgfqpoint{5.405312in}{2.531701in}}{\pgfqpoint{5.392821in}{2.536875in}}{\pgfqpoint{5.379798in}{2.536875in}}%
\pgfpathcurveto{\pgfqpoint{5.366776in}{2.536875in}}{\pgfqpoint{5.354284in}{2.531701in}}{\pgfqpoint{5.345076in}{2.522493in}}%
\pgfpathcurveto{\pgfqpoint{5.335868in}{2.513284in}}{\pgfqpoint{5.330694in}{2.500793in}}{\pgfqpoint{5.330694in}{2.487770in}}%
\pgfpathcurveto{\pgfqpoint{5.330694in}{2.474748in}}{\pgfqpoint{5.335868in}{2.462257in}}{\pgfqpoint{5.345076in}{2.453048in}}%
\pgfpathcurveto{\pgfqpoint{5.354284in}{2.443840in}}{\pgfqpoint{5.366776in}{2.438666in}}{\pgfqpoint{5.379798in}{2.438666in}}%
\pgfpathlineto{\pgfqpoint{5.379798in}{2.438666in}}%
\pgfpathclose%
\pgfusepath{stroke,fill}%
\end{pgfscope}%
\begin{pgfscope}%
\pgfpathrectangle{\pgfqpoint{0.454429in}{0.261491in}}{\pgfqpoint{6.040000in}{6.040000in}}%
\pgfusepath{clip}%
\pgfsetbuttcap%
\pgfsetroundjoin%
\definecolor{currentfill}{rgb}{0.121569,0.466667,0.705882}%
\pgfsetfillcolor{currentfill}%
\pgfsetlinewidth{1.003750pt}%
\definecolor{currentstroke}{rgb}{0.121569,0.466667,0.705882}%
\pgfsetstrokecolor{currentstroke}%
\pgfsetdash{}{0pt}%
\pgfpathmoveto{\pgfqpoint{2.815143in}{2.438666in}}%
\pgfpathcurveto{\pgfqpoint{2.828166in}{2.438666in}}{\pgfqpoint{2.840657in}{2.443840in}}{\pgfqpoint{2.849866in}{2.453048in}}%
\pgfpathcurveto{\pgfqpoint{2.859074in}{2.462257in}}{\pgfqpoint{2.864248in}{2.474748in}}{\pgfqpoint{2.864248in}{2.487770in}}%
\pgfpathcurveto{\pgfqpoint{2.864248in}{2.500793in}}{\pgfqpoint{2.859074in}{2.513284in}}{\pgfqpoint{2.849866in}{2.522493in}}%
\pgfpathcurveto{\pgfqpoint{2.840657in}{2.531701in}}{\pgfqpoint{2.828166in}{2.536875in}}{\pgfqpoint{2.815143in}{2.536875in}}%
\pgfpathcurveto{\pgfqpoint{2.802121in}{2.536875in}}{\pgfqpoint{2.789630in}{2.531701in}}{\pgfqpoint{2.780421in}{2.522493in}}%
\pgfpathcurveto{\pgfqpoint{2.771213in}{2.513284in}}{\pgfqpoint{2.766039in}{2.500793in}}{\pgfqpoint{2.766039in}{2.487770in}}%
\pgfpathcurveto{\pgfqpoint{2.766039in}{2.474748in}}{\pgfqpoint{2.771213in}{2.462257in}}{\pgfqpoint{2.780421in}{2.453048in}}%
\pgfpathcurveto{\pgfqpoint{2.789630in}{2.443840in}}{\pgfqpoint{2.802121in}{2.438666in}}{\pgfqpoint{2.815143in}{2.438666in}}%
\pgfpathlineto{\pgfqpoint{2.815143in}{2.438666in}}%
\pgfpathclose%
\pgfusepath{stroke,fill}%
\end{pgfscope}%
\begin{pgfscope}%
\pgfpathrectangle{\pgfqpoint{0.454429in}{0.261491in}}{\pgfqpoint{6.040000in}{6.040000in}}%
\pgfusepath{clip}%
\pgfsetbuttcap%
\pgfsetroundjoin%
\definecolor{currentfill}{rgb}{0.121569,0.466667,0.705882}%
\pgfsetfillcolor{currentfill}%
\pgfsetlinewidth{1.003750pt}%
\definecolor{currentstroke}{rgb}{0.121569,0.466667,0.705882}%
\pgfsetstrokecolor{currentstroke}%
\pgfsetdash{}{0pt}%
\pgfpathmoveto{\pgfqpoint{3.600801in}{2.438666in}}%
\pgfpathcurveto{\pgfqpoint{3.613824in}{2.438666in}}{\pgfqpoint{3.626315in}{2.443840in}}{\pgfqpoint{3.635523in}{2.453048in}}%
\pgfpathcurveto{\pgfqpoint{3.644732in}{2.462257in}}{\pgfqpoint{3.649906in}{2.474748in}}{\pgfqpoint{3.649906in}{2.487770in}}%
\pgfpathcurveto{\pgfqpoint{3.649906in}{2.500793in}}{\pgfqpoint{3.644732in}{2.513284in}}{\pgfqpoint{3.635523in}{2.522493in}}%
\pgfpathcurveto{\pgfqpoint{3.626315in}{2.531701in}}{\pgfqpoint{3.613824in}{2.536875in}}{\pgfqpoint{3.600801in}{2.536875in}}%
\pgfpathcurveto{\pgfqpoint{3.587778in}{2.536875in}}{\pgfqpoint{3.575287in}{2.531701in}}{\pgfqpoint{3.566079in}{2.522493in}}%
\pgfpathcurveto{\pgfqpoint{3.556870in}{2.513284in}}{\pgfqpoint{3.551696in}{2.500793in}}{\pgfqpoint{3.551696in}{2.487770in}}%
\pgfpathcurveto{\pgfqpoint{3.551696in}{2.474748in}}{\pgfqpoint{3.556870in}{2.462257in}}{\pgfqpoint{3.566079in}{2.453048in}}%
\pgfpathcurveto{\pgfqpoint{3.575287in}{2.443840in}}{\pgfqpoint{3.587778in}{2.438666in}}{\pgfqpoint{3.600801in}{2.438666in}}%
\pgfpathlineto{\pgfqpoint{3.600801in}{2.438666in}}%
\pgfpathclose%
\pgfusepath{stroke,fill}%
\end{pgfscope}%
\begin{pgfscope}%
\pgfpathrectangle{\pgfqpoint{0.454429in}{0.261491in}}{\pgfqpoint{6.040000in}{6.040000in}}%
\pgfusepath{clip}%
\pgfsetbuttcap%
\pgfsetroundjoin%
\definecolor{currentfill}{rgb}{0.121569,0.466667,0.705882}%
\pgfsetfillcolor{currentfill}%
\pgfsetlinewidth{1.003750pt}%
\definecolor{currentstroke}{rgb}{0.121569,0.466667,0.705882}%
\pgfsetstrokecolor{currentstroke}%
\pgfsetdash{}{0pt}%
\pgfpathmoveto{\pgfqpoint{1.629195in}{2.438666in}}%
\pgfpathcurveto{\pgfqpoint{1.642218in}{2.438666in}}{\pgfqpoint{1.654709in}{2.443840in}}{\pgfqpoint{1.663917in}{2.453048in}}%
\pgfpathcurveto{\pgfqpoint{1.673126in}{2.462257in}}{\pgfqpoint{1.678300in}{2.474748in}}{\pgfqpoint{1.678300in}{2.487770in}}%
\pgfpathcurveto{\pgfqpoint{1.678300in}{2.500793in}}{\pgfqpoint{1.673126in}{2.513284in}}{\pgfqpoint{1.663917in}{2.522493in}}%
\pgfpathcurveto{\pgfqpoint{1.654709in}{2.531701in}}{\pgfqpoint{1.642218in}{2.536875in}}{\pgfqpoint{1.629195in}{2.536875in}}%
\pgfpathcurveto{\pgfqpoint{1.616172in}{2.536875in}}{\pgfqpoint{1.603681in}{2.531701in}}{\pgfqpoint{1.594473in}{2.522493in}}%
\pgfpathcurveto{\pgfqpoint{1.585265in}{2.513284in}}{\pgfqpoint{1.580091in}{2.500793in}}{\pgfqpoint{1.580091in}{2.487770in}}%
\pgfpathcurveto{\pgfqpoint{1.580091in}{2.474748in}}{\pgfqpoint{1.585265in}{2.462257in}}{\pgfqpoint{1.594473in}{2.453048in}}%
\pgfpathcurveto{\pgfqpoint{1.603681in}{2.443840in}}{\pgfqpoint{1.616172in}{2.438666in}}{\pgfqpoint{1.629195in}{2.438666in}}%
\pgfpathlineto{\pgfqpoint{1.629195in}{2.438666in}}%
\pgfpathclose%
\pgfusepath{stroke,fill}%
\end{pgfscope}%
\begin{pgfscope}%
\pgfpathrectangle{\pgfqpoint{0.454429in}{0.261491in}}{\pgfqpoint{6.040000in}{6.040000in}}%
\pgfusepath{clip}%
\pgfsetbuttcap%
\pgfsetroundjoin%
\definecolor{currentfill}{rgb}{0.121569,0.466667,0.705882}%
\pgfsetfillcolor{currentfill}%
\pgfsetlinewidth{1.003750pt}%
\definecolor{currentstroke}{rgb}{0.121569,0.466667,0.705882}%
\pgfsetstrokecolor{currentstroke}%
\pgfsetdash{}{0pt}%
\pgfpathmoveto{\pgfqpoint{4.869614in}{2.438666in}}%
\pgfpathcurveto{\pgfqpoint{4.882637in}{2.438666in}}{\pgfqpoint{4.895128in}{2.443840in}}{\pgfqpoint{4.904336in}{2.453048in}}%
\pgfpathcurveto{\pgfqpoint{4.913544in}{2.462257in}}{\pgfqpoint{4.918718in}{2.474748in}}{\pgfqpoint{4.918718in}{2.487770in}}%
\pgfpathcurveto{\pgfqpoint{4.918718in}{2.500793in}}{\pgfqpoint{4.913544in}{2.513284in}}{\pgfqpoint{4.904336in}{2.522493in}}%
\pgfpathcurveto{\pgfqpoint{4.895128in}{2.531701in}}{\pgfqpoint{4.882637in}{2.536875in}}{\pgfqpoint{4.869614in}{2.536875in}}%
\pgfpathcurveto{\pgfqpoint{4.856591in}{2.536875in}}{\pgfqpoint{4.844100in}{2.531701in}}{\pgfqpoint{4.834892in}{2.522493in}}%
\pgfpathcurveto{\pgfqpoint{4.825683in}{2.513284in}}{\pgfqpoint{4.820509in}{2.500793in}}{\pgfqpoint{4.820509in}{2.487770in}}%
\pgfpathcurveto{\pgfqpoint{4.820509in}{2.474748in}}{\pgfqpoint{4.825683in}{2.462257in}}{\pgfqpoint{4.834892in}{2.453048in}}%
\pgfpathcurveto{\pgfqpoint{4.844100in}{2.443840in}}{\pgfqpoint{4.856591in}{2.438666in}}{\pgfqpoint{4.869614in}{2.438666in}}%
\pgfpathlineto{\pgfqpoint{4.869614in}{2.438666in}}%
\pgfpathclose%
\pgfusepath{stroke,fill}%
\end{pgfscope}%
\begin{pgfscope}%
\pgfpathrectangle{\pgfqpoint{0.454429in}{0.261491in}}{\pgfqpoint{6.040000in}{6.040000in}}%
\pgfusepath{clip}%
\pgfsetbuttcap%
\pgfsetroundjoin%
\definecolor{currentfill}{rgb}{0.121569,0.466667,0.705882}%
\pgfsetfillcolor{currentfill}%
\pgfsetlinewidth{1.003750pt}%
\definecolor{currentstroke}{rgb}{0.121569,0.466667,0.705882}%
\pgfsetstrokecolor{currentstroke}%
\pgfsetdash{}{0pt}%
\pgfpathmoveto{\pgfqpoint{5.890297in}{2.438666in}}%
\pgfpathcurveto{\pgfqpoint{5.903320in}{2.438666in}}{\pgfqpoint{5.915811in}{2.443840in}}{\pgfqpoint{5.925019in}{2.453048in}}%
\pgfpathcurveto{\pgfqpoint{5.934228in}{2.462257in}}{\pgfqpoint{5.939402in}{2.474748in}}{\pgfqpoint{5.939402in}{2.487770in}}%
\pgfpathcurveto{\pgfqpoint{5.939402in}{2.500793in}}{\pgfqpoint{5.934228in}{2.513284in}}{\pgfqpoint{5.925019in}{2.522493in}}%
\pgfpathcurveto{\pgfqpoint{5.915811in}{2.531701in}}{\pgfqpoint{5.903320in}{2.536875in}}{\pgfqpoint{5.890297in}{2.536875in}}%
\pgfpathcurveto{\pgfqpoint{5.877275in}{2.536875in}}{\pgfqpoint{5.864783in}{2.531701in}}{\pgfqpoint{5.855575in}{2.522493in}}%
\pgfpathcurveto{\pgfqpoint{5.846367in}{2.513284in}}{\pgfqpoint{5.841193in}{2.500793in}}{\pgfqpoint{5.841193in}{2.487770in}}%
\pgfpathcurveto{\pgfqpoint{5.841193in}{2.474748in}}{\pgfqpoint{5.846367in}{2.462257in}}{\pgfqpoint{5.855575in}{2.453048in}}%
\pgfpathcurveto{\pgfqpoint{5.864783in}{2.443840in}}{\pgfqpoint{5.877275in}{2.438666in}}{\pgfqpoint{5.890297in}{2.438666in}}%
\pgfpathlineto{\pgfqpoint{5.890297in}{2.438666in}}%
\pgfpathclose%
\pgfusepath{stroke,fill}%
\end{pgfscope}%
\begin{pgfscope}%
\pgfpathrectangle{\pgfqpoint{0.454429in}{0.261491in}}{\pgfqpoint{6.040000in}{6.040000in}}%
\pgfusepath{clip}%
\pgfsetbuttcap%
\pgfsetroundjoin%
\definecolor{currentfill}{rgb}{0.121569,0.466667,0.705882}%
\pgfsetfillcolor{currentfill}%
\pgfsetlinewidth{1.003750pt}%
\definecolor{currentstroke}{rgb}{0.121569,0.466667,0.705882}%
\pgfsetstrokecolor{currentstroke}%
\pgfsetdash{}{0pt}%
\pgfpathmoveto{\pgfqpoint{4.407715in}{2.438666in}}%
\pgfpathcurveto{\pgfqpoint{4.420737in}{2.438666in}}{\pgfqpoint{4.433229in}{2.443840in}}{\pgfqpoint{4.442437in}{2.453048in}}%
\pgfpathcurveto{\pgfqpoint{4.451645in}{2.462257in}}{\pgfqpoint{4.456819in}{2.474748in}}{\pgfqpoint{4.456819in}{2.487770in}}%
\pgfpathcurveto{\pgfqpoint{4.456819in}{2.500793in}}{\pgfqpoint{4.451645in}{2.513284in}}{\pgfqpoint{4.442437in}{2.522493in}}%
\pgfpathcurveto{\pgfqpoint{4.433229in}{2.531701in}}{\pgfqpoint{4.420737in}{2.536875in}}{\pgfqpoint{4.407715in}{2.536875in}}%
\pgfpathcurveto{\pgfqpoint{4.394692in}{2.536875in}}{\pgfqpoint{4.382201in}{2.531701in}}{\pgfqpoint{4.372993in}{2.522493in}}%
\pgfpathcurveto{\pgfqpoint{4.363784in}{2.513284in}}{\pgfqpoint{4.358610in}{2.500793in}}{\pgfqpoint{4.358610in}{2.487770in}}%
\pgfpathcurveto{\pgfqpoint{4.358610in}{2.474748in}}{\pgfqpoint{4.363784in}{2.462257in}}{\pgfqpoint{4.372993in}{2.453048in}}%
\pgfpathcurveto{\pgfqpoint{4.382201in}{2.443840in}}{\pgfqpoint{4.394692in}{2.438666in}}{\pgfqpoint{4.407715in}{2.438666in}}%
\pgfpathlineto{\pgfqpoint{4.407715in}{2.438666in}}%
\pgfpathclose%
\pgfusepath{stroke,fill}%
\end{pgfscope}%
\begin{pgfscope}%
\pgfpathrectangle{\pgfqpoint{0.454429in}{0.261491in}}{\pgfqpoint{6.040000in}{6.040000in}}%
\pgfusepath{clip}%
\pgfsetbuttcap%
\pgfsetroundjoin%
\definecolor{currentfill}{rgb}{0.121569,0.466667,0.705882}%
\pgfsetfillcolor{currentfill}%
\pgfsetlinewidth{1.003750pt}%
\definecolor{currentstroke}{rgb}{0.121569,0.466667,0.705882}%
\pgfsetstrokecolor{currentstroke}%
\pgfsetdash{}{0pt}%
\pgfpathmoveto{\pgfqpoint{2.036687in}{2.438666in}}%
\pgfpathcurveto{\pgfqpoint{2.049710in}{2.438666in}}{\pgfqpoint{2.062201in}{2.443840in}}{\pgfqpoint{2.071410in}{2.453048in}}%
\pgfpathcurveto{\pgfqpoint{2.080618in}{2.462257in}}{\pgfqpoint{2.085792in}{2.474748in}}{\pgfqpoint{2.085792in}{2.487770in}}%
\pgfpathcurveto{\pgfqpoint{2.085792in}{2.500793in}}{\pgfqpoint{2.080618in}{2.513284in}}{\pgfqpoint{2.071410in}{2.522493in}}%
\pgfpathcurveto{\pgfqpoint{2.062201in}{2.531701in}}{\pgfqpoint{2.049710in}{2.536875in}}{\pgfqpoint{2.036687in}{2.536875in}}%
\pgfpathcurveto{\pgfqpoint{2.023665in}{2.536875in}}{\pgfqpoint{2.011174in}{2.531701in}}{\pgfqpoint{2.001965in}{2.522493in}}%
\pgfpathcurveto{\pgfqpoint{1.992757in}{2.513284in}}{\pgfqpoint{1.987583in}{2.500793in}}{\pgfqpoint{1.987583in}{2.487770in}}%
\pgfpathcurveto{\pgfqpoint{1.987583in}{2.474748in}}{\pgfqpoint{1.992757in}{2.462257in}}{\pgfqpoint{2.001965in}{2.453048in}}%
\pgfpathcurveto{\pgfqpoint{2.011174in}{2.443840in}}{\pgfqpoint{2.023665in}{2.438666in}}{\pgfqpoint{2.036687in}{2.438666in}}%
\pgfpathlineto{\pgfqpoint{2.036687in}{2.438666in}}%
\pgfpathclose%
\pgfusepath{stroke,fill}%
\end{pgfscope}%
\begin{pgfscope}%
\pgfpathrectangle{\pgfqpoint{0.454429in}{0.261491in}}{\pgfqpoint{6.040000in}{6.040000in}}%
\pgfusepath{clip}%
\pgfsetbuttcap%
\pgfsetroundjoin%
\definecolor{currentfill}{rgb}{0.121569,0.466667,0.705882}%
\pgfsetfillcolor{currentfill}%
\pgfsetlinewidth{1.003750pt}%
\definecolor{currentstroke}{rgb}{0.121569,0.466667,0.705882}%
\pgfsetstrokecolor{currentstroke}%
\pgfsetdash{}{0pt}%
\pgfpathmoveto{\pgfqpoint{3.987193in}{2.438666in}}%
\pgfpathcurveto{\pgfqpoint{4.000215in}{2.438666in}}{\pgfqpoint{4.012706in}{2.443840in}}{\pgfqpoint{4.021915in}{2.453048in}}%
\pgfpathcurveto{\pgfqpoint{4.031123in}{2.462257in}}{\pgfqpoint{4.036297in}{2.474748in}}{\pgfqpoint{4.036297in}{2.487770in}}%
\pgfpathcurveto{\pgfqpoint{4.036297in}{2.500793in}}{\pgfqpoint{4.031123in}{2.513284in}}{\pgfqpoint{4.021915in}{2.522493in}}%
\pgfpathcurveto{\pgfqpoint{4.012706in}{2.531701in}}{\pgfqpoint{4.000215in}{2.536875in}}{\pgfqpoint{3.987193in}{2.536875in}}%
\pgfpathcurveto{\pgfqpoint{3.974170in}{2.536875in}}{\pgfqpoint{3.961679in}{2.531701in}}{\pgfqpoint{3.952470in}{2.522493in}}%
\pgfpathcurveto{\pgfqpoint{3.943262in}{2.513284in}}{\pgfqpoint{3.938088in}{2.500793in}}{\pgfqpoint{3.938088in}{2.487770in}}%
\pgfpathcurveto{\pgfqpoint{3.938088in}{2.474748in}}{\pgfqpoint{3.943262in}{2.462257in}}{\pgfqpoint{3.952470in}{2.453048in}}%
\pgfpathcurveto{\pgfqpoint{3.961679in}{2.443840in}}{\pgfqpoint{3.974170in}{2.438666in}}{\pgfqpoint{3.987193in}{2.438666in}}%
\pgfpathlineto{\pgfqpoint{3.987193in}{2.438666in}}%
\pgfpathclose%
\pgfusepath{stroke,fill}%
\end{pgfscope}%
\begin{pgfscope}%
\pgfpathrectangle{\pgfqpoint{0.454429in}{0.261491in}}{\pgfqpoint{6.040000in}{6.040000in}}%
\pgfusepath{clip}%
\pgfsetbuttcap%
\pgfsetroundjoin%
\definecolor{currentfill}{rgb}{0.121569,0.466667,0.705882}%
\pgfsetfillcolor{currentfill}%
\pgfsetlinewidth{1.003750pt}%
\definecolor{currentstroke}{rgb}{0.121569,0.466667,0.705882}%
\pgfsetstrokecolor{currentstroke}%
\pgfsetdash{}{0pt}%
\pgfpathmoveto{\pgfqpoint{3.222157in}{2.438666in}}%
\pgfpathcurveto{\pgfqpoint{3.235180in}{2.438666in}}{\pgfqpoint{3.247671in}{2.443840in}}{\pgfqpoint{3.256879in}{2.453048in}}%
\pgfpathcurveto{\pgfqpoint{3.266087in}{2.462257in}}{\pgfqpoint{3.271261in}{2.474748in}}{\pgfqpoint{3.271261in}{2.487770in}}%
\pgfpathcurveto{\pgfqpoint{3.271261in}{2.500793in}}{\pgfqpoint{3.266087in}{2.513284in}}{\pgfqpoint{3.256879in}{2.522493in}}%
\pgfpathcurveto{\pgfqpoint{3.247671in}{2.531701in}}{\pgfqpoint{3.235180in}{2.536875in}}{\pgfqpoint{3.222157in}{2.536875in}}%
\pgfpathcurveto{\pgfqpoint{3.209134in}{2.536875in}}{\pgfqpoint{3.196643in}{2.531701in}}{\pgfqpoint{3.187435in}{2.522493in}}%
\pgfpathcurveto{\pgfqpoint{3.178226in}{2.513284in}}{\pgfqpoint{3.173052in}{2.500793in}}{\pgfqpoint{3.173052in}{2.487770in}}%
\pgfpathcurveto{\pgfqpoint{3.173052in}{2.474748in}}{\pgfqpoint{3.178226in}{2.462257in}}{\pgfqpoint{3.187435in}{2.453048in}}%
\pgfpathcurveto{\pgfqpoint{3.196643in}{2.443840in}}{\pgfqpoint{3.209134in}{2.438666in}}{\pgfqpoint{3.222157in}{2.438666in}}%
\pgfpathlineto{\pgfqpoint{3.222157in}{2.438666in}}%
\pgfpathclose%
\pgfusepath{stroke,fill}%
\end{pgfscope}%
\begin{pgfscope}%
\pgfpathrectangle{\pgfqpoint{0.454429in}{0.261491in}}{\pgfqpoint{6.040000in}{6.040000in}}%
\pgfusepath{clip}%
\pgfsetbuttcap%
\pgfsetroundjoin%
\definecolor{currentfill}{rgb}{0.121569,0.466667,0.705882}%
\pgfsetfillcolor{currentfill}%
\pgfsetlinewidth{1.003750pt}%
\definecolor{currentstroke}{rgb}{0.121569,0.466667,0.705882}%
\pgfsetstrokecolor{currentstroke}%
\pgfsetdash{}{0pt}%
\pgfpathmoveto{\pgfqpoint{4.187779in}{2.713449in}}%
\pgfpathcurveto{\pgfqpoint{4.200802in}{2.713449in}}{\pgfqpoint{4.213293in}{2.718623in}}{\pgfqpoint{4.222502in}{2.727831in}}%
\pgfpathcurveto{\pgfqpoint{4.231710in}{2.737039in}}{\pgfqpoint{4.236884in}{2.749531in}}{\pgfqpoint{4.236884in}{2.762553in}}%
\pgfpathcurveto{\pgfqpoint{4.236884in}{2.775576in}}{\pgfqpoint{4.231710in}{2.788067in}}{\pgfqpoint{4.222502in}{2.797275in}}%
\pgfpathcurveto{\pgfqpoint{4.213293in}{2.806484in}}{\pgfqpoint{4.200802in}{2.811658in}}{\pgfqpoint{4.187779in}{2.811658in}}%
\pgfpathcurveto{\pgfqpoint{4.174757in}{2.811658in}}{\pgfqpoint{4.162266in}{2.806484in}}{\pgfqpoint{4.153057in}{2.797275in}}%
\pgfpathcurveto{\pgfqpoint{4.143849in}{2.788067in}}{\pgfqpoint{4.138675in}{2.775576in}}{\pgfqpoint{4.138675in}{2.762553in}}%
\pgfpathcurveto{\pgfqpoint{4.138675in}{2.749531in}}{\pgfqpoint{4.143849in}{2.737039in}}{\pgfqpoint{4.153057in}{2.727831in}}%
\pgfpathcurveto{\pgfqpoint{4.162266in}{2.718623in}}{\pgfqpoint{4.174757in}{2.713449in}}{\pgfqpoint{4.187779in}{2.713449in}}%
\pgfpathlineto{\pgfqpoint{4.187779in}{2.713449in}}%
\pgfpathclose%
\pgfusepath{stroke,fill}%
\end{pgfscope}%
\begin{pgfscope}%
\pgfpathrectangle{\pgfqpoint{0.454429in}{0.261491in}}{\pgfqpoint{6.040000in}{6.040000in}}%
\pgfusepath{clip}%
\pgfsetbuttcap%
\pgfsetroundjoin%
\definecolor{currentfill}{rgb}{0.121569,0.466667,0.705882}%
\pgfsetfillcolor{currentfill}%
\pgfsetlinewidth{1.003750pt}%
\definecolor{currentstroke}{rgb}{0.121569,0.466667,0.705882}%
\pgfsetstrokecolor{currentstroke}%
\pgfsetdash{}{0pt}%
\pgfpathmoveto{\pgfqpoint{4.584629in}{2.714360in}}%
\pgfpathcurveto{\pgfqpoint{4.597652in}{2.714360in}}{\pgfqpoint{4.610143in}{2.719534in}}{\pgfqpoint{4.619351in}{2.728743in}}%
\pgfpathcurveto{\pgfqpoint{4.628560in}{2.737951in}}{\pgfqpoint{4.633734in}{2.750442in}}{\pgfqpoint{4.633734in}{2.763465in}}%
\pgfpathcurveto{\pgfqpoint{4.633734in}{2.776488in}}{\pgfqpoint{4.628560in}{2.788979in}}{\pgfqpoint{4.619351in}{2.798187in}}%
\pgfpathcurveto{\pgfqpoint{4.610143in}{2.807396in}}{\pgfqpoint{4.597652in}{2.812570in}}{\pgfqpoint{4.584629in}{2.812570in}}%
\pgfpathcurveto{\pgfqpoint{4.571606in}{2.812570in}}{\pgfqpoint{4.559115in}{2.807396in}}{\pgfqpoint{4.549907in}{2.798187in}}%
\pgfpathcurveto{\pgfqpoint{4.540698in}{2.788979in}}{\pgfqpoint{4.535524in}{2.776488in}}{\pgfqpoint{4.535524in}{2.763465in}}%
\pgfpathcurveto{\pgfqpoint{4.535524in}{2.750442in}}{\pgfqpoint{4.540698in}{2.737951in}}{\pgfqpoint{4.549907in}{2.728743in}}%
\pgfpathcurveto{\pgfqpoint{4.559115in}{2.719534in}}{\pgfqpoint{4.571606in}{2.714360in}}{\pgfqpoint{4.584629in}{2.714360in}}%
\pgfpathlineto{\pgfqpoint{4.584629in}{2.714360in}}%
\pgfpathclose%
\pgfusepath{stroke,fill}%
\end{pgfscope}%
\begin{pgfscope}%
\pgfpathrectangle{\pgfqpoint{0.454429in}{0.261491in}}{\pgfqpoint{6.040000in}{6.040000in}}%
\pgfusepath{clip}%
\pgfsetbuttcap%
\pgfsetroundjoin%
\definecolor{currentfill}{rgb}{0.121569,0.466667,0.705882}%
\pgfsetfillcolor{currentfill}%
\pgfsetlinewidth{1.003750pt}%
\definecolor{currentstroke}{rgb}{0.121569,0.466667,0.705882}%
\pgfsetstrokecolor{currentstroke}%
\pgfsetdash{}{0pt}%
\pgfpathmoveto{\pgfqpoint{3.003501in}{2.714746in}}%
\pgfpathcurveto{\pgfqpoint{3.016523in}{2.714746in}}{\pgfqpoint{3.029014in}{2.719920in}}{\pgfqpoint{3.038223in}{2.729129in}}%
\pgfpathcurveto{\pgfqpoint{3.047431in}{2.738337in}}{\pgfqpoint{3.052605in}{2.750828in}}{\pgfqpoint{3.052605in}{2.763851in}}%
\pgfpathcurveto{\pgfqpoint{3.052605in}{2.776874in}}{\pgfqpoint{3.047431in}{2.789365in}}{\pgfqpoint{3.038223in}{2.798573in}}%
\pgfpathcurveto{\pgfqpoint{3.029014in}{2.807782in}}{\pgfqpoint{3.016523in}{2.812956in}}{\pgfqpoint{3.003501in}{2.812956in}}%
\pgfpathcurveto{\pgfqpoint{2.990478in}{2.812956in}}{\pgfqpoint{2.977987in}{2.807782in}}{\pgfqpoint{2.968778in}{2.798573in}}%
\pgfpathcurveto{\pgfqpoint{2.959570in}{2.789365in}}{\pgfqpoint{2.954396in}{2.776874in}}{\pgfqpoint{2.954396in}{2.763851in}}%
\pgfpathcurveto{\pgfqpoint{2.954396in}{2.750828in}}{\pgfqpoint{2.959570in}{2.738337in}}{\pgfqpoint{2.968778in}{2.729129in}}%
\pgfpathcurveto{\pgfqpoint{2.977987in}{2.719920in}}{\pgfqpoint{2.990478in}{2.714746in}}{\pgfqpoint{3.003501in}{2.714746in}}%
\pgfpathlineto{\pgfqpoint{3.003501in}{2.714746in}}%
\pgfpathclose%
\pgfusepath{stroke,fill}%
\end{pgfscope}%
\begin{pgfscope}%
\pgfpathrectangle{\pgfqpoint{0.454429in}{0.261491in}}{\pgfqpoint{6.040000in}{6.040000in}}%
\pgfusepath{clip}%
\pgfsetbuttcap%
\pgfsetroundjoin%
\definecolor{currentfill}{rgb}{0.121569,0.466667,0.705882}%
\pgfsetfillcolor{currentfill}%
\pgfsetlinewidth{1.003750pt}%
\definecolor{currentstroke}{rgb}{0.121569,0.466667,0.705882}%
\pgfsetstrokecolor{currentstroke}%
\pgfsetdash{}{0pt}%
\pgfpathmoveto{\pgfqpoint{2.606030in}{2.717087in}}%
\pgfpathcurveto{\pgfqpoint{2.619053in}{2.717087in}}{\pgfqpoint{2.631544in}{2.722261in}}{\pgfqpoint{2.640753in}{2.731469in}}%
\pgfpathcurveto{\pgfqpoint{2.649961in}{2.740678in}}{\pgfqpoint{2.655135in}{2.753169in}}{\pgfqpoint{2.655135in}{2.766192in}}%
\pgfpathcurveto{\pgfqpoint{2.655135in}{2.779214in}}{\pgfqpoint{2.649961in}{2.791705in}}{\pgfqpoint{2.640753in}{2.800914in}}%
\pgfpathcurveto{\pgfqpoint{2.631544in}{2.810122in}}{\pgfqpoint{2.619053in}{2.815296in}}{\pgfqpoint{2.606030in}{2.815296in}}%
\pgfpathcurveto{\pgfqpoint{2.593008in}{2.815296in}}{\pgfqpoint{2.580517in}{2.810122in}}{\pgfqpoint{2.571308in}{2.800914in}}%
\pgfpathcurveto{\pgfqpoint{2.562100in}{2.791705in}}{\pgfqpoint{2.556926in}{2.779214in}}{\pgfqpoint{2.556926in}{2.766192in}}%
\pgfpathcurveto{\pgfqpoint{2.556926in}{2.753169in}}{\pgfqpoint{2.562100in}{2.740678in}}{\pgfqpoint{2.571308in}{2.731469in}}%
\pgfpathcurveto{\pgfqpoint{2.580517in}{2.722261in}}{\pgfqpoint{2.593008in}{2.717087in}}{\pgfqpoint{2.606030in}{2.717087in}}%
\pgfpathlineto{\pgfqpoint{2.606030in}{2.717087in}}%
\pgfpathclose%
\pgfusepath{stroke,fill}%
\end{pgfscope}%
\begin{pgfscope}%
\pgfpathrectangle{\pgfqpoint{0.454429in}{0.261491in}}{\pgfqpoint{6.040000in}{6.040000in}}%
\pgfusepath{clip}%
\pgfsetbuttcap%
\pgfsetroundjoin%
\definecolor{currentfill}{rgb}{0.121569,0.466667,0.705882}%
\pgfsetfillcolor{currentfill}%
\pgfsetlinewidth{1.003750pt}%
\definecolor{currentstroke}{rgb}{0.121569,0.466667,0.705882}%
\pgfsetstrokecolor{currentstroke}%
\pgfsetdash{}{0pt}%
\pgfpathmoveto{\pgfqpoint{3.791102in}{2.719834in}}%
\pgfpathcurveto{\pgfqpoint{3.804125in}{2.719834in}}{\pgfqpoint{3.816616in}{2.725008in}}{\pgfqpoint{3.825824in}{2.734217in}}%
\pgfpathcurveto{\pgfqpoint{3.835033in}{2.743425in}}{\pgfqpoint{3.840207in}{2.755916in}}{\pgfqpoint{3.840207in}{2.768939in}}%
\pgfpathcurveto{\pgfqpoint{3.840207in}{2.781962in}}{\pgfqpoint{3.835033in}{2.794453in}}{\pgfqpoint{3.825824in}{2.803661in}}%
\pgfpathcurveto{\pgfqpoint{3.816616in}{2.812870in}}{\pgfqpoint{3.804125in}{2.818044in}}{\pgfqpoint{3.791102in}{2.818044in}}%
\pgfpathcurveto{\pgfqpoint{3.778079in}{2.818044in}}{\pgfqpoint{3.765588in}{2.812870in}}{\pgfqpoint{3.756380in}{2.803661in}}%
\pgfpathcurveto{\pgfqpoint{3.747171in}{2.794453in}}{\pgfqpoint{3.741997in}{2.781962in}}{\pgfqpoint{3.741997in}{2.768939in}}%
\pgfpathcurveto{\pgfqpoint{3.741997in}{2.755916in}}{\pgfqpoint{3.747171in}{2.743425in}}{\pgfqpoint{3.756380in}{2.734217in}}%
\pgfpathcurveto{\pgfqpoint{3.765588in}{2.725008in}}{\pgfqpoint{3.778079in}{2.719834in}}{\pgfqpoint{3.791102in}{2.719834in}}%
\pgfpathlineto{\pgfqpoint{3.791102in}{2.719834in}}%
\pgfpathclose%
\pgfusepath{stroke,fill}%
\end{pgfscope}%
\begin{pgfscope}%
\pgfpathrectangle{\pgfqpoint{0.454429in}{0.261491in}}{\pgfqpoint{6.040000in}{6.040000in}}%
\pgfusepath{clip}%
\pgfsetbuttcap%
\pgfsetroundjoin%
\definecolor{currentfill}{rgb}{0.121569,0.466667,0.705882}%
\pgfsetfillcolor{currentfill}%
\pgfsetlinewidth{1.003750pt}%
\definecolor{currentstroke}{rgb}{0.121569,0.466667,0.705882}%
\pgfsetstrokecolor{currentstroke}%
\pgfsetdash{}{0pt}%
\pgfpathmoveto{\pgfqpoint{3.398118in}{2.721078in}}%
\pgfpathcurveto{\pgfqpoint{3.411141in}{2.721078in}}{\pgfqpoint{3.423632in}{2.726252in}}{\pgfqpoint{3.432841in}{2.735461in}}%
\pgfpathcurveto{\pgfqpoint{3.442049in}{2.744669in}}{\pgfqpoint{3.447223in}{2.757160in}}{\pgfqpoint{3.447223in}{2.770183in}}%
\pgfpathcurveto{\pgfqpoint{3.447223in}{2.783206in}}{\pgfqpoint{3.442049in}{2.795697in}}{\pgfqpoint{3.432841in}{2.804905in}}%
\pgfpathcurveto{\pgfqpoint{3.423632in}{2.814114in}}{\pgfqpoint{3.411141in}{2.819288in}}{\pgfqpoint{3.398118in}{2.819288in}}%
\pgfpathcurveto{\pgfqpoint{3.385096in}{2.819288in}}{\pgfqpoint{3.372605in}{2.814114in}}{\pgfqpoint{3.363396in}{2.804905in}}%
\pgfpathcurveto{\pgfqpoint{3.354188in}{2.795697in}}{\pgfqpoint{3.349014in}{2.783206in}}{\pgfqpoint{3.349014in}{2.770183in}}%
\pgfpathcurveto{\pgfqpoint{3.349014in}{2.757160in}}{\pgfqpoint{3.354188in}{2.744669in}}{\pgfqpoint{3.363396in}{2.735461in}}%
\pgfpathcurveto{\pgfqpoint{3.372605in}{2.726252in}}{\pgfqpoint{3.385096in}{2.721078in}}{\pgfqpoint{3.398118in}{2.721078in}}%
\pgfpathlineto{\pgfqpoint{3.398118in}{2.721078in}}%
\pgfpathclose%
\pgfusepath{stroke,fill}%
\end{pgfscope}%
\begin{pgfscope}%
\pgfpathrectangle{\pgfqpoint{0.454429in}{0.261491in}}{\pgfqpoint{6.040000in}{6.040000in}}%
\pgfusepath{clip}%
\pgfsetbuttcap%
\pgfsetroundjoin%
\definecolor{currentfill}{rgb}{0.121569,0.466667,0.705882}%
\pgfsetfillcolor{currentfill}%
\pgfsetlinewidth{1.003750pt}%
\definecolor{currentstroke}{rgb}{0.121569,0.466667,0.705882}%
\pgfsetstrokecolor{currentstroke}%
\pgfsetdash{}{0pt}%
\pgfpathmoveto{\pgfqpoint{1.815791in}{2.721471in}}%
\pgfpathcurveto{\pgfqpoint{1.828814in}{2.721471in}}{\pgfqpoint{1.841305in}{2.726645in}}{\pgfqpoint{1.850513in}{2.735853in}}%
\pgfpathcurveto{\pgfqpoint{1.859722in}{2.745062in}}{\pgfqpoint{1.864896in}{2.757553in}}{\pgfqpoint{1.864896in}{2.770576in}}%
\pgfpathcurveto{\pgfqpoint{1.864896in}{2.783598in}}{\pgfqpoint{1.859722in}{2.796089in}}{\pgfqpoint{1.850513in}{2.805298in}}%
\pgfpathcurveto{\pgfqpoint{1.841305in}{2.814506in}}{\pgfqpoint{1.828814in}{2.819680in}}{\pgfqpoint{1.815791in}{2.819680in}}%
\pgfpathcurveto{\pgfqpoint{1.802768in}{2.819680in}}{\pgfqpoint{1.790277in}{2.814506in}}{\pgfqpoint{1.781069in}{2.805298in}}%
\pgfpathcurveto{\pgfqpoint{1.771861in}{2.796089in}}{\pgfqpoint{1.766687in}{2.783598in}}{\pgfqpoint{1.766687in}{2.770576in}}%
\pgfpathcurveto{\pgfqpoint{1.766687in}{2.757553in}}{\pgfqpoint{1.771861in}{2.745062in}}{\pgfqpoint{1.781069in}{2.735853in}}%
\pgfpathcurveto{\pgfqpoint{1.790277in}{2.726645in}}{\pgfqpoint{1.802768in}{2.721471in}}{\pgfqpoint{1.815791in}{2.721471in}}%
\pgfpathlineto{\pgfqpoint{1.815791in}{2.721471in}}%
\pgfpathclose%
\pgfusepath{stroke,fill}%
\end{pgfscope}%
\begin{pgfscope}%
\pgfpathrectangle{\pgfqpoint{0.454429in}{0.261491in}}{\pgfqpoint{6.040000in}{6.040000in}}%
\pgfusepath{clip}%
\pgfsetbuttcap%
\pgfsetroundjoin%
\definecolor{currentfill}{rgb}{0.121569,0.466667,0.705882}%
\pgfsetfillcolor{currentfill}%
\pgfsetlinewidth{1.003750pt}%
\definecolor{currentstroke}{rgb}{0.121569,0.466667,0.705882}%
\pgfsetstrokecolor{currentstroke}%
\pgfsetdash{}{0pt}%
\pgfpathmoveto{\pgfqpoint{2.210774in}{2.721879in}}%
\pgfpathcurveto{\pgfqpoint{2.223797in}{2.721879in}}{\pgfqpoint{2.236288in}{2.727053in}}{\pgfqpoint{2.245496in}{2.736262in}}%
\pgfpathcurveto{\pgfqpoint{2.254705in}{2.745470in}}{\pgfqpoint{2.259879in}{2.757961in}}{\pgfqpoint{2.259879in}{2.770984in}}%
\pgfpathcurveto{\pgfqpoint{2.259879in}{2.784006in}}{\pgfqpoint{2.254705in}{2.796498in}}{\pgfqpoint{2.245496in}{2.805706in}}%
\pgfpathcurveto{\pgfqpoint{2.236288in}{2.814914in}}{\pgfqpoint{2.223797in}{2.820088in}}{\pgfqpoint{2.210774in}{2.820088in}}%
\pgfpathcurveto{\pgfqpoint{2.197751in}{2.820088in}}{\pgfqpoint{2.185260in}{2.814914in}}{\pgfqpoint{2.176052in}{2.805706in}}%
\pgfpathcurveto{\pgfqpoint{2.166844in}{2.796498in}}{\pgfqpoint{2.161670in}{2.784006in}}{\pgfqpoint{2.161670in}{2.770984in}}%
\pgfpathcurveto{\pgfqpoint{2.161670in}{2.757961in}}{\pgfqpoint{2.166844in}{2.745470in}}{\pgfqpoint{2.176052in}{2.736262in}}%
\pgfpathcurveto{\pgfqpoint{2.185260in}{2.727053in}}{\pgfqpoint{2.197751in}{2.721879in}}{\pgfqpoint{2.210774in}{2.721879in}}%
\pgfpathlineto{\pgfqpoint{2.210774in}{2.721879in}}%
\pgfpathclose%
\pgfusepath{stroke,fill}%
\end{pgfscope}%
\begin{pgfscope}%
\pgfpathrectangle{\pgfqpoint{0.454429in}{0.261491in}}{\pgfqpoint{6.040000in}{6.040000in}}%
\pgfusepath{clip}%
\pgfsetbuttcap%
\pgfsetroundjoin%
\definecolor{currentfill}{rgb}{0.121569,0.466667,0.705882}%
\pgfsetfillcolor{currentfill}%
\pgfsetlinewidth{1.003750pt}%
\definecolor{currentstroke}{rgb}{0.121569,0.466667,0.705882}%
\pgfsetstrokecolor{currentstroke}%
\pgfsetdash{}{0pt}%
\pgfpathmoveto{\pgfqpoint{5.691078in}{2.729612in}}%
\pgfpathcurveto{\pgfqpoint{5.704101in}{2.729612in}}{\pgfqpoint{5.716592in}{2.734786in}}{\pgfqpoint{5.725800in}{2.743995in}}%
\pgfpathcurveto{\pgfqpoint{5.735009in}{2.753203in}}{\pgfqpoint{5.740183in}{2.765694in}}{\pgfqpoint{5.740183in}{2.778717in}}%
\pgfpathcurveto{\pgfqpoint{5.740183in}{2.791740in}}{\pgfqpoint{5.735009in}{2.804231in}}{\pgfqpoint{5.725800in}{2.813439in}}%
\pgfpathcurveto{\pgfqpoint{5.716592in}{2.822648in}}{\pgfqpoint{5.704101in}{2.827822in}}{\pgfqpoint{5.691078in}{2.827822in}}%
\pgfpathcurveto{\pgfqpoint{5.678055in}{2.827822in}}{\pgfqpoint{5.665564in}{2.822648in}}{\pgfqpoint{5.656356in}{2.813439in}}%
\pgfpathcurveto{\pgfqpoint{5.647147in}{2.804231in}}{\pgfqpoint{5.641973in}{2.791740in}}{\pgfqpoint{5.641973in}{2.778717in}}%
\pgfpathcurveto{\pgfqpoint{5.641973in}{2.765694in}}{\pgfqpoint{5.647147in}{2.753203in}}{\pgfqpoint{5.656356in}{2.743995in}}%
\pgfpathcurveto{\pgfqpoint{5.665564in}{2.734786in}}{\pgfqpoint{5.678055in}{2.729612in}}{\pgfqpoint{5.691078in}{2.729612in}}%
\pgfpathlineto{\pgfqpoint{5.691078in}{2.729612in}}%
\pgfpathclose%
\pgfusepath{stroke,fill}%
\end{pgfscope}%
\begin{pgfscope}%
\pgfpathrectangle{\pgfqpoint{0.454429in}{0.261491in}}{\pgfqpoint{6.040000in}{6.040000in}}%
\pgfusepath{clip}%
\pgfsetbuttcap%
\pgfsetroundjoin%
\definecolor{currentfill}{rgb}{0.121569,0.466667,0.705882}%
\pgfsetfillcolor{currentfill}%
\pgfsetlinewidth{1.003750pt}%
\definecolor{currentstroke}{rgb}{0.121569,0.466667,0.705882}%
\pgfsetstrokecolor{currentstroke}%
\pgfsetdash{}{0pt}%
\pgfpathmoveto{\pgfqpoint{4.961182in}{2.739370in}}%
\pgfpathcurveto{\pgfqpoint{4.974205in}{2.739370in}}{\pgfqpoint{4.986696in}{2.744544in}}{\pgfqpoint{4.995904in}{2.753753in}}%
\pgfpathcurveto{\pgfqpoint{5.005113in}{2.762961in}}{\pgfqpoint{5.010287in}{2.775452in}}{\pgfqpoint{5.010287in}{2.788475in}}%
\pgfpathcurveto{\pgfqpoint{5.010287in}{2.801498in}}{\pgfqpoint{5.005113in}{2.813989in}}{\pgfqpoint{4.995904in}{2.823197in}}%
\pgfpathcurveto{\pgfqpoint{4.986696in}{2.832406in}}{\pgfqpoint{4.974205in}{2.837580in}}{\pgfqpoint{4.961182in}{2.837580in}}%
\pgfpathcurveto{\pgfqpoint{4.948159in}{2.837580in}}{\pgfqpoint{4.935668in}{2.832406in}}{\pgfqpoint{4.926460in}{2.823197in}}%
\pgfpathcurveto{\pgfqpoint{4.917251in}{2.813989in}}{\pgfqpoint{4.912078in}{2.801498in}}{\pgfqpoint{4.912078in}{2.788475in}}%
\pgfpathcurveto{\pgfqpoint{4.912078in}{2.775452in}}{\pgfqpoint{4.917251in}{2.762961in}}{\pgfqpoint{4.926460in}{2.753753in}}%
\pgfpathcurveto{\pgfqpoint{4.935668in}{2.744544in}}{\pgfqpoint{4.948159in}{2.739370in}}{\pgfqpoint{4.961182in}{2.739370in}}%
\pgfpathlineto{\pgfqpoint{4.961182in}{2.739370in}}%
\pgfpathclose%
\pgfusepath{stroke,fill}%
\end{pgfscope}%
\begin{pgfscope}%
\pgfpathrectangle{\pgfqpoint{0.454429in}{0.261491in}}{\pgfqpoint{6.040000in}{6.040000in}}%
\pgfusepath{clip}%
\pgfsetbuttcap%
\pgfsetroundjoin%
\definecolor{currentfill}{rgb}{0.121569,0.466667,0.705882}%
\pgfsetfillcolor{currentfill}%
\pgfsetlinewidth{1.003750pt}%
\definecolor{currentstroke}{rgb}{0.121569,0.466667,0.705882}%
\pgfsetstrokecolor{currentstroke}%
\pgfsetdash{}{0pt}%
\pgfpathmoveto{\pgfqpoint{1.428012in}{2.739818in}}%
\pgfpathcurveto{\pgfqpoint{1.441034in}{2.739818in}}{\pgfqpoint{1.453526in}{2.744992in}}{\pgfqpoint{1.462734in}{2.754200in}}%
\pgfpathcurveto{\pgfqpoint{1.471942in}{2.763409in}}{\pgfqpoint{1.477116in}{2.775900in}}{\pgfqpoint{1.477116in}{2.788922in}}%
\pgfpathcurveto{\pgfqpoint{1.477116in}{2.801945in}}{\pgfqpoint{1.471942in}{2.814436in}}{\pgfqpoint{1.462734in}{2.823645in}}%
\pgfpathcurveto{\pgfqpoint{1.453526in}{2.832853in}}{\pgfqpoint{1.441034in}{2.838027in}}{\pgfqpoint{1.428012in}{2.838027in}}%
\pgfpathcurveto{\pgfqpoint{1.414989in}{2.838027in}}{\pgfqpoint{1.402498in}{2.832853in}}{\pgfqpoint{1.393290in}{2.823645in}}%
\pgfpathcurveto{\pgfqpoint{1.384081in}{2.814436in}}{\pgfqpoint{1.378907in}{2.801945in}}{\pgfqpoint{1.378907in}{2.788922in}}%
\pgfpathcurveto{\pgfqpoint{1.378907in}{2.775900in}}{\pgfqpoint{1.384081in}{2.763409in}}{\pgfqpoint{1.393290in}{2.754200in}}%
\pgfpathcurveto{\pgfqpoint{1.402498in}{2.744992in}}{\pgfqpoint{1.414989in}{2.739818in}}{\pgfqpoint{1.428012in}{2.739818in}}%
\pgfpathlineto{\pgfqpoint{1.428012in}{2.739818in}}%
\pgfpathclose%
\pgfusepath{stroke,fill}%
\end{pgfscope}%
\begin{pgfscope}%
\pgfpathrectangle{\pgfqpoint{0.454429in}{0.261491in}}{\pgfqpoint{6.040000in}{6.040000in}}%
\pgfusepath{clip}%
\pgfsetbuttcap%
\pgfsetroundjoin%
\definecolor{currentfill}{rgb}{0.121569,0.466667,0.705882}%
\pgfsetfillcolor{currentfill}%
\pgfsetlinewidth{1.003750pt}%
\definecolor{currentstroke}{rgb}{0.121569,0.466667,0.705882}%
\pgfsetstrokecolor{currentstroke}%
\pgfsetdash{}{0pt}%
\pgfpathmoveto{\pgfqpoint{5.323851in}{2.751864in}}%
\pgfpathcurveto{\pgfqpoint{5.336874in}{2.751864in}}{\pgfqpoint{5.349365in}{2.757038in}}{\pgfqpoint{5.358573in}{2.766247in}}%
\pgfpathcurveto{\pgfqpoint{5.367782in}{2.775455in}}{\pgfqpoint{5.372956in}{2.787946in}}{\pgfqpoint{5.372956in}{2.800969in}}%
\pgfpathcurveto{\pgfqpoint{5.372956in}{2.813992in}}{\pgfqpoint{5.367782in}{2.826483in}}{\pgfqpoint{5.358573in}{2.835691in}}%
\pgfpathcurveto{\pgfqpoint{5.349365in}{2.844900in}}{\pgfqpoint{5.336874in}{2.850074in}}{\pgfqpoint{5.323851in}{2.850074in}}%
\pgfpathcurveto{\pgfqpoint{5.310828in}{2.850074in}}{\pgfqpoint{5.298337in}{2.844900in}}{\pgfqpoint{5.289129in}{2.835691in}}%
\pgfpathcurveto{\pgfqpoint{5.279920in}{2.826483in}}{\pgfqpoint{5.274746in}{2.813992in}}{\pgfqpoint{5.274746in}{2.800969in}}%
\pgfpathcurveto{\pgfqpoint{5.274746in}{2.787946in}}{\pgfqpoint{5.279920in}{2.775455in}}{\pgfqpoint{5.289129in}{2.766247in}}%
\pgfpathcurveto{\pgfqpoint{5.298337in}{2.757038in}}{\pgfqpoint{5.310828in}{2.751864in}}{\pgfqpoint{5.323851in}{2.751864in}}%
\pgfpathlineto{\pgfqpoint{5.323851in}{2.751864in}}%
\pgfpathclose%
\pgfusepath{stroke,fill}%
\end{pgfscope}%
\begin{pgfscope}%
\pgfpathrectangle{\pgfqpoint{0.454429in}{0.261491in}}{\pgfqpoint{6.040000in}{6.040000in}}%
\pgfusepath{clip}%
\pgfsetbuttcap%
\pgfsetroundjoin%
\definecolor{currentfill}{rgb}{0.121569,0.466667,0.705882}%
\pgfsetfillcolor{currentfill}%
\pgfsetlinewidth{1.003750pt}%
\definecolor{currentstroke}{rgb}{0.121569,0.466667,0.705882}%
\pgfsetstrokecolor{currentstroke}%
\pgfsetdash{}{0pt}%
\pgfpathmoveto{\pgfqpoint{2.794486in}{2.996250in}}%
\pgfpathcurveto{\pgfqpoint{2.807508in}{2.996250in}}{\pgfqpoint{2.819999in}{3.001424in}}{\pgfqpoint{2.829208in}{3.010633in}}%
\pgfpathcurveto{\pgfqpoint{2.838416in}{3.019841in}}{\pgfqpoint{2.843590in}{3.032332in}}{\pgfqpoint{2.843590in}{3.045355in}}%
\pgfpathcurveto{\pgfqpoint{2.843590in}{3.058378in}}{\pgfqpoint{2.838416in}{3.070869in}}{\pgfqpoint{2.829208in}{3.080077in}}%
\pgfpathcurveto{\pgfqpoint{2.819999in}{3.089286in}}{\pgfqpoint{2.807508in}{3.094460in}}{\pgfqpoint{2.794486in}{3.094460in}}%
\pgfpathcurveto{\pgfqpoint{2.781463in}{3.094460in}}{\pgfqpoint{2.768972in}{3.089286in}}{\pgfqpoint{2.759763in}{3.080077in}}%
\pgfpathcurveto{\pgfqpoint{2.750555in}{3.070869in}}{\pgfqpoint{2.745381in}{3.058378in}}{\pgfqpoint{2.745381in}{3.045355in}}%
\pgfpathcurveto{\pgfqpoint{2.745381in}{3.032332in}}{\pgfqpoint{2.750555in}{3.019841in}}{\pgfqpoint{2.759763in}{3.010633in}}%
\pgfpathcurveto{\pgfqpoint{2.768972in}{3.001424in}}{\pgfqpoint{2.781463in}{2.996250in}}{\pgfqpoint{2.794486in}{2.996250in}}%
\pgfpathlineto{\pgfqpoint{2.794486in}{2.996250in}}%
\pgfpathclose%
\pgfusepath{stroke,fill}%
\end{pgfscope}%
\begin{pgfscope}%
\pgfpathrectangle{\pgfqpoint{0.454429in}{0.261491in}}{\pgfqpoint{6.040000in}{6.040000in}}%
\pgfusepath{clip}%
\pgfsetbuttcap%
\pgfsetroundjoin%
\definecolor{currentfill}{rgb}{0.121569,0.466667,0.705882}%
\pgfsetfillcolor{currentfill}%
\pgfsetlinewidth{1.003750pt}%
\definecolor{currentstroke}{rgb}{0.121569,0.466667,0.705882}%
\pgfsetstrokecolor{currentstroke}%
\pgfsetdash{}{0pt}%
\pgfpathmoveto{\pgfqpoint{3.187133in}{2.998088in}}%
\pgfpathcurveto{\pgfqpoint{3.200155in}{2.998088in}}{\pgfqpoint{3.212646in}{3.003262in}}{\pgfqpoint{3.221855in}{3.012471in}}%
\pgfpathcurveto{\pgfqpoint{3.231063in}{3.021679in}}{\pgfqpoint{3.236237in}{3.034170in}}{\pgfqpoint{3.236237in}{3.047193in}}%
\pgfpathcurveto{\pgfqpoint{3.236237in}{3.060215in}}{\pgfqpoint{3.231063in}{3.072707in}}{\pgfqpoint{3.221855in}{3.081915in}}%
\pgfpathcurveto{\pgfqpoint{3.212646in}{3.091123in}}{\pgfqpoint{3.200155in}{3.096297in}}{\pgfqpoint{3.187133in}{3.096297in}}%
\pgfpathcurveto{\pgfqpoint{3.174110in}{3.096297in}}{\pgfqpoint{3.161619in}{3.091123in}}{\pgfqpoint{3.152410in}{3.081915in}}%
\pgfpathcurveto{\pgfqpoint{3.143202in}{3.072707in}}{\pgfqpoint{3.138028in}{3.060215in}}{\pgfqpoint{3.138028in}{3.047193in}}%
\pgfpathcurveto{\pgfqpoint{3.138028in}{3.034170in}}{\pgfqpoint{3.143202in}{3.021679in}}{\pgfqpoint{3.152410in}{3.012471in}}%
\pgfpathcurveto{\pgfqpoint{3.161619in}{3.003262in}}{\pgfqpoint{3.174110in}{2.998088in}}{\pgfqpoint{3.187133in}{2.998088in}}%
\pgfpathlineto{\pgfqpoint{3.187133in}{2.998088in}}%
\pgfpathclose%
\pgfusepath{stroke,fill}%
\end{pgfscope}%
\begin{pgfscope}%
\pgfpathrectangle{\pgfqpoint{0.454429in}{0.261491in}}{\pgfqpoint{6.040000in}{6.040000in}}%
\pgfusepath{clip}%
\pgfsetbuttcap%
\pgfsetroundjoin%
\definecolor{currentfill}{rgb}{0.121569,0.466667,0.705882}%
\pgfsetfillcolor{currentfill}%
\pgfsetlinewidth{1.003750pt}%
\definecolor{currentstroke}{rgb}{0.121569,0.466667,0.705882}%
\pgfsetstrokecolor{currentstroke}%
\pgfsetdash{}{0pt}%
\pgfpathmoveto{\pgfqpoint{2.400556in}{3.000676in}}%
\pgfpathcurveto{\pgfqpoint{2.413579in}{3.000676in}}{\pgfqpoint{2.426070in}{3.005850in}}{\pgfqpoint{2.435279in}{3.015058in}}%
\pgfpathcurveto{\pgfqpoint{2.444487in}{3.024267in}}{\pgfqpoint{2.449661in}{3.036758in}}{\pgfqpoint{2.449661in}{3.049781in}}%
\pgfpathcurveto{\pgfqpoint{2.449661in}{3.062803in}}{\pgfqpoint{2.444487in}{3.075294in}}{\pgfqpoint{2.435279in}{3.084503in}}%
\pgfpathcurveto{\pgfqpoint{2.426070in}{3.093711in}}{\pgfqpoint{2.413579in}{3.098885in}}{\pgfqpoint{2.400556in}{3.098885in}}%
\pgfpathcurveto{\pgfqpoint{2.387534in}{3.098885in}}{\pgfqpoint{2.375043in}{3.093711in}}{\pgfqpoint{2.365834in}{3.084503in}}%
\pgfpathcurveto{\pgfqpoint{2.356626in}{3.075294in}}{\pgfqpoint{2.351452in}{3.062803in}}{\pgfqpoint{2.351452in}{3.049781in}}%
\pgfpathcurveto{\pgfqpoint{2.351452in}{3.036758in}}{\pgfqpoint{2.356626in}{3.024267in}}{\pgfqpoint{2.365834in}{3.015058in}}%
\pgfpathcurveto{\pgfqpoint{2.375043in}{3.005850in}}{\pgfqpoint{2.387534in}{3.000676in}}{\pgfqpoint{2.400556in}{3.000676in}}%
\pgfpathlineto{\pgfqpoint{2.400556in}{3.000676in}}%
\pgfpathclose%
\pgfusepath{stroke,fill}%
\end{pgfscope}%
\begin{pgfscope}%
\pgfpathrectangle{\pgfqpoint{0.454429in}{0.261491in}}{\pgfqpoint{6.040000in}{6.040000in}}%
\pgfusepath{clip}%
\pgfsetbuttcap%
\pgfsetroundjoin%
\definecolor{currentfill}{rgb}{0.121569,0.466667,0.705882}%
\pgfsetfillcolor{currentfill}%
\pgfsetlinewidth{1.003750pt}%
\definecolor{currentstroke}{rgb}{0.121569,0.466667,0.705882}%
\pgfsetstrokecolor{currentstroke}%
\pgfsetdash{}{0pt}%
\pgfpathmoveto{\pgfqpoint{3.971805in}{3.001238in}}%
\pgfpathcurveto{\pgfqpoint{3.984828in}{3.001238in}}{\pgfqpoint{3.997319in}{3.006412in}}{\pgfqpoint{4.006527in}{3.015620in}}%
\pgfpathcurveto{\pgfqpoint{4.015736in}{3.024829in}}{\pgfqpoint{4.020910in}{3.037320in}}{\pgfqpoint{4.020910in}{3.050342in}}%
\pgfpathcurveto{\pgfqpoint{4.020910in}{3.063365in}}{\pgfqpoint{4.015736in}{3.075856in}}{\pgfqpoint{4.006527in}{3.085065in}}%
\pgfpathcurveto{\pgfqpoint{3.997319in}{3.094273in}}{\pgfqpoint{3.984828in}{3.099447in}}{\pgfqpoint{3.971805in}{3.099447in}}%
\pgfpathcurveto{\pgfqpoint{3.958782in}{3.099447in}}{\pgfqpoint{3.946291in}{3.094273in}}{\pgfqpoint{3.937083in}{3.085065in}}%
\pgfpathcurveto{\pgfqpoint{3.927874in}{3.075856in}}{\pgfqpoint{3.922700in}{3.063365in}}{\pgfqpoint{3.922700in}{3.050342in}}%
\pgfpathcurveto{\pgfqpoint{3.922700in}{3.037320in}}{\pgfqpoint{3.927874in}{3.024829in}}{\pgfqpoint{3.937083in}{3.015620in}}%
\pgfpathcurveto{\pgfqpoint{3.946291in}{3.006412in}}{\pgfqpoint{3.958782in}{3.001238in}}{\pgfqpoint{3.971805in}{3.001238in}}%
\pgfpathlineto{\pgfqpoint{3.971805in}{3.001238in}}%
\pgfpathclose%
\pgfusepath{stroke,fill}%
\end{pgfscope}%
\begin{pgfscope}%
\pgfpathrectangle{\pgfqpoint{0.454429in}{0.261491in}}{\pgfqpoint{6.040000in}{6.040000in}}%
\pgfusepath{clip}%
\pgfsetbuttcap%
\pgfsetroundjoin%
\definecolor{currentfill}{rgb}{0.121569,0.466667,0.705882}%
\pgfsetfillcolor{currentfill}%
\pgfsetlinewidth{1.003750pt}%
\definecolor{currentstroke}{rgb}{0.121569,0.466667,0.705882}%
\pgfsetstrokecolor{currentstroke}%
\pgfsetdash{}{0pt}%
\pgfpathmoveto{\pgfqpoint{3.581246in}{3.001820in}}%
\pgfpathcurveto{\pgfqpoint{3.594269in}{3.001820in}}{\pgfqpoint{3.606760in}{3.006994in}}{\pgfqpoint{3.615968in}{3.016202in}}%
\pgfpathcurveto{\pgfqpoint{3.625177in}{3.025411in}}{\pgfqpoint{3.630351in}{3.037902in}}{\pgfqpoint{3.630351in}{3.050924in}}%
\pgfpathcurveto{\pgfqpoint{3.630351in}{3.063947in}}{\pgfqpoint{3.625177in}{3.076438in}}{\pgfqpoint{3.615968in}{3.085647in}}%
\pgfpathcurveto{\pgfqpoint{3.606760in}{3.094855in}}{\pgfqpoint{3.594269in}{3.100029in}}{\pgfqpoint{3.581246in}{3.100029in}}%
\pgfpathcurveto{\pgfqpoint{3.568223in}{3.100029in}}{\pgfqpoint{3.555732in}{3.094855in}}{\pgfqpoint{3.546524in}{3.085647in}}%
\pgfpathcurveto{\pgfqpoint{3.537315in}{3.076438in}}{\pgfqpoint{3.532141in}{3.063947in}}{\pgfqpoint{3.532141in}{3.050924in}}%
\pgfpathcurveto{\pgfqpoint{3.532141in}{3.037902in}}{\pgfqpoint{3.537315in}{3.025411in}}{\pgfqpoint{3.546524in}{3.016202in}}%
\pgfpathcurveto{\pgfqpoint{3.555732in}{3.006994in}}{\pgfqpoint{3.568223in}{3.001820in}}{\pgfqpoint{3.581246in}{3.001820in}}%
\pgfpathlineto{\pgfqpoint{3.581246in}{3.001820in}}%
\pgfpathclose%
\pgfusepath{stroke,fill}%
\end{pgfscope}%
\begin{pgfscope}%
\pgfpathrectangle{\pgfqpoint{0.454429in}{0.261491in}}{\pgfqpoint{6.040000in}{6.040000in}}%
\pgfusepath{clip}%
\pgfsetbuttcap%
\pgfsetroundjoin%
\definecolor{currentfill}{rgb}{0.121569,0.466667,0.705882}%
\pgfsetfillcolor{currentfill}%
\pgfsetlinewidth{1.003750pt}%
\definecolor{currentstroke}{rgb}{0.121569,0.466667,0.705882}%
\pgfsetstrokecolor{currentstroke}%
\pgfsetdash{}{0pt}%
\pgfpathmoveto{\pgfqpoint{4.355064in}{3.003907in}}%
\pgfpathcurveto{\pgfqpoint{4.368087in}{3.003907in}}{\pgfqpoint{4.380578in}{3.009081in}}{\pgfqpoint{4.389786in}{3.018289in}}%
\pgfpathcurveto{\pgfqpoint{4.398995in}{3.027497in}}{\pgfqpoint{4.404169in}{3.039989in}}{\pgfqpoint{4.404169in}{3.053011in}}%
\pgfpathcurveto{\pgfqpoint{4.404169in}{3.066034in}}{\pgfqpoint{4.398995in}{3.078525in}}{\pgfqpoint{4.389786in}{3.087733in}}%
\pgfpathcurveto{\pgfqpoint{4.380578in}{3.096942in}}{\pgfqpoint{4.368087in}{3.102116in}}{\pgfqpoint{4.355064in}{3.102116in}}%
\pgfpathcurveto{\pgfqpoint{4.342041in}{3.102116in}}{\pgfqpoint{4.329550in}{3.096942in}}{\pgfqpoint{4.320342in}{3.087733in}}%
\pgfpathcurveto{\pgfqpoint{4.311133in}{3.078525in}}{\pgfqpoint{4.305959in}{3.066034in}}{\pgfqpoint{4.305959in}{3.053011in}}%
\pgfpathcurveto{\pgfqpoint{4.305959in}{3.039989in}}{\pgfqpoint{4.311133in}{3.027497in}}{\pgfqpoint{4.320342in}{3.018289in}}%
\pgfpathcurveto{\pgfqpoint{4.329550in}{3.009081in}}{\pgfqpoint{4.342041in}{3.003907in}}{\pgfqpoint{4.355064in}{3.003907in}}%
\pgfpathlineto{\pgfqpoint{4.355064in}{3.003907in}}%
\pgfpathclose%
\pgfusepath{stroke,fill}%
\end{pgfscope}%
\begin{pgfscope}%
\pgfpathrectangle{\pgfqpoint{0.454429in}{0.261491in}}{\pgfqpoint{6.040000in}{6.040000in}}%
\pgfusepath{clip}%
\pgfsetbuttcap%
\pgfsetroundjoin%
\definecolor{currentfill}{rgb}{0.121569,0.466667,0.705882}%
\pgfsetfillcolor{currentfill}%
\pgfsetlinewidth{1.003750pt}%
\definecolor{currentstroke}{rgb}{0.121569,0.466667,0.705882}%
\pgfsetstrokecolor{currentstroke}%
\pgfsetdash{}{0pt}%
\pgfpathmoveto{\pgfqpoint{2.007501in}{3.004360in}}%
\pgfpathcurveto{\pgfqpoint{2.020524in}{3.004360in}}{\pgfqpoint{2.033015in}{3.009534in}}{\pgfqpoint{2.042223in}{3.018742in}}%
\pgfpathcurveto{\pgfqpoint{2.051432in}{3.027951in}}{\pgfqpoint{2.056606in}{3.040442in}}{\pgfqpoint{2.056606in}{3.053464in}}%
\pgfpathcurveto{\pgfqpoint{2.056606in}{3.066487in}}{\pgfqpoint{2.051432in}{3.078978in}}{\pgfqpoint{2.042223in}{3.088187in}}%
\pgfpathcurveto{\pgfqpoint{2.033015in}{3.097395in}}{\pgfqpoint{2.020524in}{3.102569in}}{\pgfqpoint{2.007501in}{3.102569in}}%
\pgfpathcurveto{\pgfqpoint{1.994478in}{3.102569in}}{\pgfqpoint{1.981987in}{3.097395in}}{\pgfqpoint{1.972779in}{3.088187in}}%
\pgfpathcurveto{\pgfqpoint{1.963571in}{3.078978in}}{\pgfqpoint{1.958397in}{3.066487in}}{\pgfqpoint{1.958397in}{3.053464in}}%
\pgfpathcurveto{\pgfqpoint{1.958397in}{3.040442in}}{\pgfqpoint{1.963571in}{3.027951in}}{\pgfqpoint{1.972779in}{3.018742in}}%
\pgfpathcurveto{\pgfqpoint{1.981987in}{3.009534in}}{\pgfqpoint{1.994478in}{3.004360in}}{\pgfqpoint{2.007501in}{3.004360in}}%
\pgfpathlineto{\pgfqpoint{2.007501in}{3.004360in}}%
\pgfpathclose%
\pgfusepath{stroke,fill}%
\end{pgfscope}%
\begin{pgfscope}%
\pgfpathrectangle{\pgfqpoint{0.454429in}{0.261491in}}{\pgfqpoint{6.040000in}{6.040000in}}%
\pgfusepath{clip}%
\pgfsetbuttcap%
\pgfsetroundjoin%
\definecolor{currentfill}{rgb}{0.121569,0.466667,0.705882}%
\pgfsetfillcolor{currentfill}%
\pgfsetlinewidth{1.003750pt}%
\definecolor{currentstroke}{rgb}{0.121569,0.466667,0.705882}%
\pgfsetstrokecolor{currentstroke}%
\pgfsetdash{}{0pt}%
\pgfpathmoveto{\pgfqpoint{4.730464in}{3.021171in}}%
\pgfpathcurveto{\pgfqpoint{4.743487in}{3.021171in}}{\pgfqpoint{4.755978in}{3.026345in}}{\pgfqpoint{4.765187in}{3.035553in}}%
\pgfpathcurveto{\pgfqpoint{4.774395in}{3.044761in}}{\pgfqpoint{4.779569in}{3.057253in}}{\pgfqpoint{4.779569in}{3.070275in}}%
\pgfpathcurveto{\pgfqpoint{4.779569in}{3.083298in}}{\pgfqpoint{4.774395in}{3.095789in}}{\pgfqpoint{4.765187in}{3.104997in}}%
\pgfpathcurveto{\pgfqpoint{4.755978in}{3.114206in}}{\pgfqpoint{4.743487in}{3.119380in}}{\pgfqpoint{4.730464in}{3.119380in}}%
\pgfpathcurveto{\pgfqpoint{4.717442in}{3.119380in}}{\pgfqpoint{4.704951in}{3.114206in}}{\pgfqpoint{4.695742in}{3.104997in}}%
\pgfpathcurveto{\pgfqpoint{4.686534in}{3.095789in}}{\pgfqpoint{4.681360in}{3.083298in}}{\pgfqpoint{4.681360in}{3.070275in}}%
\pgfpathcurveto{\pgfqpoint{4.681360in}{3.057253in}}{\pgfqpoint{4.686534in}{3.044761in}}{\pgfqpoint{4.695742in}{3.035553in}}%
\pgfpathcurveto{\pgfqpoint{4.704951in}{3.026345in}}{\pgfqpoint{4.717442in}{3.021171in}}{\pgfqpoint{4.730464in}{3.021171in}}%
\pgfpathlineto{\pgfqpoint{4.730464in}{3.021171in}}%
\pgfpathclose%
\pgfusepath{stroke,fill}%
\end{pgfscope}%
\begin{pgfscope}%
\pgfpathrectangle{\pgfqpoint{0.454429in}{0.261491in}}{\pgfqpoint{6.040000in}{6.040000in}}%
\pgfusepath{clip}%
\pgfsetbuttcap%
\pgfsetroundjoin%
\definecolor{currentfill}{rgb}{0.121569,0.466667,0.705882}%
\pgfsetfillcolor{currentfill}%
\pgfsetlinewidth{1.003750pt}%
\definecolor{currentstroke}{rgb}{0.121569,0.466667,0.705882}%
\pgfsetstrokecolor{currentstroke}%
\pgfsetdash{}{0pt}%
\pgfpathmoveto{\pgfqpoint{1.623002in}{3.024589in}}%
\pgfpathcurveto{\pgfqpoint{1.636025in}{3.024589in}}{\pgfqpoint{1.648516in}{3.029763in}}{\pgfqpoint{1.657725in}{3.038971in}}%
\pgfpathcurveto{\pgfqpoint{1.666933in}{3.048180in}}{\pgfqpoint{1.672107in}{3.060671in}}{\pgfqpoint{1.672107in}{3.073694in}}%
\pgfpathcurveto{\pgfqpoint{1.672107in}{3.086716in}}{\pgfqpoint{1.666933in}{3.099207in}}{\pgfqpoint{1.657725in}{3.108416in}}%
\pgfpathcurveto{\pgfqpoint{1.648516in}{3.117624in}}{\pgfqpoint{1.636025in}{3.122798in}}{\pgfqpoint{1.623002in}{3.122798in}}%
\pgfpathcurveto{\pgfqpoint{1.609980in}{3.122798in}}{\pgfqpoint{1.597489in}{3.117624in}}{\pgfqpoint{1.588280in}{3.108416in}}%
\pgfpathcurveto{\pgfqpoint{1.579072in}{3.099207in}}{\pgfqpoint{1.573898in}{3.086716in}}{\pgfqpoint{1.573898in}{3.073694in}}%
\pgfpathcurveto{\pgfqpoint{1.573898in}{3.060671in}}{\pgfqpoint{1.579072in}{3.048180in}}{\pgfqpoint{1.588280in}{3.038971in}}%
\pgfpathcurveto{\pgfqpoint{1.597489in}{3.029763in}}{\pgfqpoint{1.609980in}{3.024589in}}{\pgfqpoint{1.623002in}{3.024589in}}%
\pgfpathlineto{\pgfqpoint{1.623002in}{3.024589in}}%
\pgfpathclose%
\pgfusepath{stroke,fill}%
\end{pgfscope}%
\begin{pgfscope}%
\pgfpathrectangle{\pgfqpoint{0.454429in}{0.261491in}}{\pgfqpoint{6.040000in}{6.040000in}}%
\pgfusepath{clip}%
\pgfsetbuttcap%
\pgfsetroundjoin%
\definecolor{currentfill}{rgb}{0.121569,0.466667,0.705882}%
\pgfsetfillcolor{currentfill}%
\pgfsetlinewidth{1.003750pt}%
\definecolor{currentstroke}{rgb}{0.121569,0.466667,0.705882}%
\pgfsetstrokecolor{currentstroke}%
\pgfsetdash{}{0pt}%
\pgfpathmoveto{\pgfqpoint{5.101697in}{3.049958in}}%
\pgfpathcurveto{\pgfqpoint{5.114720in}{3.049958in}}{\pgfqpoint{5.127211in}{3.055132in}}{\pgfqpoint{5.136419in}{3.064340in}}%
\pgfpathcurveto{\pgfqpoint{5.145628in}{3.073549in}}{\pgfqpoint{5.150802in}{3.086040in}}{\pgfqpoint{5.150802in}{3.099062in}}%
\pgfpathcurveto{\pgfqpoint{5.150802in}{3.112085in}}{\pgfqpoint{5.145628in}{3.124576in}}{\pgfqpoint{5.136419in}{3.133785in}}%
\pgfpathcurveto{\pgfqpoint{5.127211in}{3.142993in}}{\pgfqpoint{5.114720in}{3.148167in}}{\pgfqpoint{5.101697in}{3.148167in}}%
\pgfpathcurveto{\pgfqpoint{5.088674in}{3.148167in}}{\pgfqpoint{5.076183in}{3.142993in}}{\pgfqpoint{5.066975in}{3.133785in}}%
\pgfpathcurveto{\pgfqpoint{5.057766in}{3.124576in}}{\pgfqpoint{5.052592in}{3.112085in}}{\pgfqpoint{5.052592in}{3.099062in}}%
\pgfpathcurveto{\pgfqpoint{5.052592in}{3.086040in}}{\pgfqpoint{5.057766in}{3.073549in}}{\pgfqpoint{5.066975in}{3.064340in}}%
\pgfpathcurveto{\pgfqpoint{5.076183in}{3.055132in}}{\pgfqpoint{5.088674in}{3.049958in}}{\pgfqpoint{5.101697in}{3.049958in}}%
\pgfpathlineto{\pgfqpoint{5.101697in}{3.049958in}}%
\pgfpathclose%
\pgfusepath{stroke,fill}%
\end{pgfscope}%
\begin{pgfscope}%
\pgfpathrectangle{\pgfqpoint{0.454429in}{0.261491in}}{\pgfqpoint{6.040000in}{6.040000in}}%
\pgfusepath{clip}%
\pgfsetbuttcap%
\pgfsetroundjoin%
\definecolor{currentfill}{rgb}{0.121569,0.466667,0.705882}%
\pgfsetfillcolor{currentfill}%
\pgfsetlinewidth{1.003750pt}%
\definecolor{currentstroke}{rgb}{0.121569,0.466667,0.705882}%
\pgfsetstrokecolor{currentstroke}%
\pgfsetdash{}{0pt}%
\pgfpathmoveto{\pgfqpoint{5.466723in}{3.057269in}}%
\pgfpathcurveto{\pgfqpoint{5.479746in}{3.057269in}}{\pgfqpoint{5.492237in}{3.062443in}}{\pgfqpoint{5.501445in}{3.071651in}}%
\pgfpathcurveto{\pgfqpoint{5.510653in}{3.080860in}}{\pgfqpoint{5.515827in}{3.093351in}}{\pgfqpoint{5.515827in}{3.106373in}}%
\pgfpathcurveto{\pgfqpoint{5.515827in}{3.119396in}}{\pgfqpoint{5.510653in}{3.131887in}}{\pgfqpoint{5.501445in}{3.141096in}}%
\pgfpathcurveto{\pgfqpoint{5.492237in}{3.150304in}}{\pgfqpoint{5.479746in}{3.155478in}}{\pgfqpoint{5.466723in}{3.155478in}}%
\pgfpathcurveto{\pgfqpoint{5.453700in}{3.155478in}}{\pgfqpoint{5.441209in}{3.150304in}}{\pgfqpoint{5.432001in}{3.141096in}}%
\pgfpathcurveto{\pgfqpoint{5.422792in}{3.131887in}}{\pgfqpoint{5.417618in}{3.119396in}}{\pgfqpoint{5.417618in}{3.106373in}}%
\pgfpathcurveto{\pgfqpoint{5.417618in}{3.093351in}}{\pgfqpoint{5.422792in}{3.080860in}}{\pgfqpoint{5.432001in}{3.071651in}}%
\pgfpathcurveto{\pgfqpoint{5.441209in}{3.062443in}}{\pgfqpoint{5.453700in}{3.057269in}}{\pgfqpoint{5.466723in}{3.057269in}}%
\pgfpathlineto{\pgfqpoint{5.466723in}{3.057269in}}%
\pgfpathclose%
\pgfusepath{stroke,fill}%
\end{pgfscope}%
\begin{pgfscope}%
\pgfpathrectangle{\pgfqpoint{0.454429in}{0.261491in}}{\pgfqpoint{6.040000in}{6.040000in}}%
\pgfusepath{clip}%
\pgfsetbuttcap%
\pgfsetroundjoin%
\definecolor{currentfill}{rgb}{0.121569,0.466667,0.705882}%
\pgfsetfillcolor{currentfill}%
\pgfsetlinewidth{1.003750pt}%
\definecolor{currentstroke}{rgb}{0.121569,0.466667,0.705882}%
\pgfsetstrokecolor{currentstroke}%
\pgfsetdash{}{0pt}%
\pgfpathmoveto{\pgfqpoint{2.980741in}{3.278238in}}%
\pgfpathcurveto{\pgfqpoint{2.993763in}{3.278238in}}{\pgfqpoint{3.006254in}{3.283412in}}{\pgfqpoint{3.015463in}{3.292620in}}%
\pgfpathcurveto{\pgfqpoint{3.024671in}{3.301829in}}{\pgfqpoint{3.029845in}{3.314320in}}{\pgfqpoint{3.029845in}{3.327343in}}%
\pgfpathcurveto{\pgfqpoint{3.029845in}{3.340365in}}{\pgfqpoint{3.024671in}{3.352856in}}{\pgfqpoint{3.015463in}{3.362065in}}%
\pgfpathcurveto{\pgfqpoint{3.006254in}{3.371273in}}{\pgfqpoint{2.993763in}{3.376447in}}{\pgfqpoint{2.980741in}{3.376447in}}%
\pgfpathcurveto{\pgfqpoint{2.967718in}{3.376447in}}{\pgfqpoint{2.955227in}{3.371273in}}{\pgfqpoint{2.946018in}{3.362065in}}%
\pgfpathcurveto{\pgfqpoint{2.936810in}{3.352856in}}{\pgfqpoint{2.931636in}{3.340365in}}{\pgfqpoint{2.931636in}{3.327343in}}%
\pgfpathcurveto{\pgfqpoint{2.931636in}{3.314320in}}{\pgfqpoint{2.936810in}{3.301829in}}{\pgfqpoint{2.946018in}{3.292620in}}%
\pgfpathcurveto{\pgfqpoint{2.955227in}{3.283412in}}{\pgfqpoint{2.967718in}{3.278238in}}{\pgfqpoint{2.980741in}{3.278238in}}%
\pgfpathlineto{\pgfqpoint{2.980741in}{3.278238in}}%
\pgfpathclose%
\pgfusepath{stroke,fill}%
\end{pgfscope}%
\begin{pgfscope}%
\pgfpathrectangle{\pgfqpoint{0.454429in}{0.261491in}}{\pgfqpoint{6.040000in}{6.040000in}}%
\pgfusepath{clip}%
\pgfsetbuttcap%
\pgfsetroundjoin%
\definecolor{currentfill}{rgb}{0.121569,0.466667,0.705882}%
\pgfsetfillcolor{currentfill}%
\pgfsetlinewidth{1.003750pt}%
\definecolor{currentstroke}{rgb}{0.121569,0.466667,0.705882}%
\pgfsetstrokecolor{currentstroke}%
\pgfsetdash{}{0pt}%
\pgfpathmoveto{\pgfqpoint{2.588700in}{3.279932in}}%
\pgfpathcurveto{\pgfqpoint{2.601723in}{3.279932in}}{\pgfqpoint{2.614214in}{3.285106in}}{\pgfqpoint{2.623423in}{3.294315in}}%
\pgfpathcurveto{\pgfqpoint{2.632631in}{3.303523in}}{\pgfqpoint{2.637805in}{3.316014in}}{\pgfqpoint{2.637805in}{3.329037in}}%
\pgfpathcurveto{\pgfqpoint{2.637805in}{3.342060in}}{\pgfqpoint{2.632631in}{3.354551in}}{\pgfqpoint{2.623423in}{3.363759in}}%
\pgfpathcurveto{\pgfqpoint{2.614214in}{3.372967in}}{\pgfqpoint{2.601723in}{3.378141in}}{\pgfqpoint{2.588700in}{3.378141in}}%
\pgfpathcurveto{\pgfqpoint{2.575678in}{3.378141in}}{\pgfqpoint{2.563187in}{3.372967in}}{\pgfqpoint{2.553978in}{3.363759in}}%
\pgfpathcurveto{\pgfqpoint{2.544770in}{3.354551in}}{\pgfqpoint{2.539596in}{3.342060in}}{\pgfqpoint{2.539596in}{3.329037in}}%
\pgfpathcurveto{\pgfqpoint{2.539596in}{3.316014in}}{\pgfqpoint{2.544770in}{3.303523in}}{\pgfqpoint{2.553978in}{3.294315in}}%
\pgfpathcurveto{\pgfqpoint{2.563187in}{3.285106in}}{\pgfqpoint{2.575678in}{3.279932in}}{\pgfqpoint{2.588700in}{3.279932in}}%
\pgfpathlineto{\pgfqpoint{2.588700in}{3.279932in}}%
\pgfpathclose%
\pgfusepath{stroke,fill}%
\end{pgfscope}%
\begin{pgfscope}%
\pgfpathrectangle{\pgfqpoint{0.454429in}{0.261491in}}{\pgfqpoint{6.040000in}{6.040000in}}%
\pgfusepath{clip}%
\pgfsetbuttcap%
\pgfsetroundjoin%
\definecolor{currentfill}{rgb}{0.121569,0.466667,0.705882}%
\pgfsetfillcolor{currentfill}%
\pgfsetlinewidth{1.003750pt}%
\definecolor{currentstroke}{rgb}{0.121569,0.466667,0.705882}%
\pgfsetstrokecolor{currentstroke}%
\pgfsetdash{}{0pt}%
\pgfpathmoveto{\pgfqpoint{3.372322in}{3.281233in}}%
\pgfpathcurveto{\pgfqpoint{3.385344in}{3.281233in}}{\pgfqpoint{3.397836in}{3.286407in}}{\pgfqpoint{3.407044in}{3.295616in}}%
\pgfpathcurveto{\pgfqpoint{3.416252in}{3.304824in}}{\pgfqpoint{3.421426in}{3.317315in}}{\pgfqpoint{3.421426in}{3.330338in}}%
\pgfpathcurveto{\pgfqpoint{3.421426in}{3.343361in}}{\pgfqpoint{3.416252in}{3.355852in}}{\pgfqpoint{3.407044in}{3.365060in}}%
\pgfpathcurveto{\pgfqpoint{3.397836in}{3.374269in}}{\pgfqpoint{3.385344in}{3.379442in}}{\pgfqpoint{3.372322in}{3.379442in}}%
\pgfpathcurveto{\pgfqpoint{3.359299in}{3.379442in}}{\pgfqpoint{3.346808in}{3.374269in}}{\pgfqpoint{3.337600in}{3.365060in}}%
\pgfpathcurveto{\pgfqpoint{3.328391in}{3.355852in}}{\pgfqpoint{3.323217in}{3.343361in}}{\pgfqpoint{3.323217in}{3.330338in}}%
\pgfpathcurveto{\pgfqpoint{3.323217in}{3.317315in}}{\pgfqpoint{3.328391in}{3.304824in}}{\pgfqpoint{3.337600in}{3.295616in}}%
\pgfpathcurveto{\pgfqpoint{3.346808in}{3.286407in}}{\pgfqpoint{3.359299in}{3.281233in}}{\pgfqpoint{3.372322in}{3.281233in}}%
\pgfpathlineto{\pgfqpoint{3.372322in}{3.281233in}}%
\pgfpathclose%
\pgfusepath{stroke,fill}%
\end{pgfscope}%
\begin{pgfscope}%
\pgfpathrectangle{\pgfqpoint{0.454429in}{0.261491in}}{\pgfqpoint{6.040000in}{6.040000in}}%
\pgfusepath{clip}%
\pgfsetbuttcap%
\pgfsetroundjoin%
\definecolor{currentfill}{rgb}{0.121569,0.466667,0.705882}%
\pgfsetfillcolor{currentfill}%
\pgfsetlinewidth{1.003750pt}%
\definecolor{currentstroke}{rgb}{0.121569,0.466667,0.705882}%
\pgfsetstrokecolor{currentstroke}%
\pgfsetdash{}{0pt}%
\pgfpathmoveto{\pgfqpoint{3.761146in}{3.286088in}}%
\pgfpathcurveto{\pgfqpoint{3.774169in}{3.286088in}}{\pgfqpoint{3.786660in}{3.291261in}}{\pgfqpoint{3.795868in}{3.300470in}}%
\pgfpathcurveto{\pgfqpoint{3.805077in}{3.309678in}}{\pgfqpoint{3.810251in}{3.322169in}}{\pgfqpoint{3.810251in}{3.335192in}}%
\pgfpathcurveto{\pgfqpoint{3.810251in}{3.348215in}}{\pgfqpoint{3.805077in}{3.360706in}}{\pgfqpoint{3.795868in}{3.369914in}}%
\pgfpathcurveto{\pgfqpoint{3.786660in}{3.379123in}}{\pgfqpoint{3.774169in}{3.384297in}}{\pgfqpoint{3.761146in}{3.384297in}}%
\pgfpathcurveto{\pgfqpoint{3.748124in}{3.384297in}}{\pgfqpoint{3.735632in}{3.379123in}}{\pgfqpoint{3.726424in}{3.369914in}}%
\pgfpathcurveto{\pgfqpoint{3.717216in}{3.360706in}}{\pgfqpoint{3.712042in}{3.348215in}}{\pgfqpoint{3.712042in}{3.335192in}}%
\pgfpathcurveto{\pgfqpoint{3.712042in}{3.322169in}}{\pgfqpoint{3.717216in}{3.309678in}}{\pgfqpoint{3.726424in}{3.300470in}}%
\pgfpathcurveto{\pgfqpoint{3.735632in}{3.291261in}}{\pgfqpoint{3.748124in}{3.286088in}}{\pgfqpoint{3.761146in}{3.286088in}}%
\pgfpathlineto{\pgfqpoint{3.761146in}{3.286088in}}%
\pgfpathclose%
\pgfusepath{stroke,fill}%
\end{pgfscope}%
\begin{pgfscope}%
\pgfpathrectangle{\pgfqpoint{0.454429in}{0.261491in}}{\pgfqpoint{6.040000in}{6.040000in}}%
\pgfusepath{clip}%
\pgfsetbuttcap%
\pgfsetroundjoin%
\definecolor{currentfill}{rgb}{0.121569,0.466667,0.705882}%
\pgfsetfillcolor{currentfill}%
\pgfsetlinewidth{1.003750pt}%
\definecolor{currentstroke}{rgb}{0.121569,0.466667,0.705882}%
\pgfsetstrokecolor{currentstroke}%
\pgfsetdash{}{0pt}%
\pgfpathmoveto{\pgfqpoint{2.196243in}{3.286137in}}%
\pgfpathcurveto{\pgfqpoint{2.209266in}{3.286137in}}{\pgfqpoint{2.221757in}{3.291311in}}{\pgfqpoint{2.230965in}{3.300519in}}%
\pgfpathcurveto{\pgfqpoint{2.240174in}{3.309728in}}{\pgfqpoint{2.245348in}{3.322219in}}{\pgfqpoint{2.245348in}{3.335242in}}%
\pgfpathcurveto{\pgfqpoint{2.245348in}{3.348264in}}{\pgfqpoint{2.240174in}{3.360755in}}{\pgfqpoint{2.230965in}{3.369964in}}%
\pgfpathcurveto{\pgfqpoint{2.221757in}{3.379172in}}{\pgfqpoint{2.209266in}{3.384346in}}{\pgfqpoint{2.196243in}{3.384346in}}%
\pgfpathcurveto{\pgfqpoint{2.183220in}{3.384346in}}{\pgfqpoint{2.170729in}{3.379172in}}{\pgfqpoint{2.161521in}{3.369964in}}%
\pgfpathcurveto{\pgfqpoint{2.152312in}{3.360755in}}{\pgfqpoint{2.147138in}{3.348264in}}{\pgfqpoint{2.147138in}{3.335242in}}%
\pgfpathcurveto{\pgfqpoint{2.147138in}{3.322219in}}{\pgfqpoint{2.152312in}{3.309728in}}{\pgfqpoint{2.161521in}{3.300519in}}%
\pgfpathcurveto{\pgfqpoint{2.170729in}{3.291311in}}{\pgfqpoint{2.183220in}{3.286137in}}{\pgfqpoint{2.196243in}{3.286137in}}%
\pgfpathlineto{\pgfqpoint{2.196243in}{3.286137in}}%
\pgfpathclose%
\pgfusepath{stroke,fill}%
\end{pgfscope}%
\begin{pgfscope}%
\pgfpathrectangle{\pgfqpoint{0.454429in}{0.261491in}}{\pgfqpoint{6.040000in}{6.040000in}}%
\pgfusepath{clip}%
\pgfsetbuttcap%
\pgfsetroundjoin%
\definecolor{currentfill}{rgb}{0.121569,0.466667,0.705882}%
\pgfsetfillcolor{currentfill}%
\pgfsetlinewidth{1.003750pt}%
\definecolor{currentstroke}{rgb}{0.121569,0.466667,0.705882}%
\pgfsetstrokecolor{currentstroke}%
\pgfsetdash{}{0pt}%
\pgfpathmoveto{\pgfqpoint{4.143655in}{3.292664in}}%
\pgfpathcurveto{\pgfqpoint{4.156678in}{3.292664in}}{\pgfqpoint{4.169169in}{3.297838in}}{\pgfqpoint{4.178377in}{3.307047in}}%
\pgfpathcurveto{\pgfqpoint{4.187586in}{3.316255in}}{\pgfqpoint{4.192760in}{3.328746in}}{\pgfqpoint{4.192760in}{3.341769in}}%
\pgfpathcurveto{\pgfqpoint{4.192760in}{3.354792in}}{\pgfqpoint{4.187586in}{3.367283in}}{\pgfqpoint{4.178377in}{3.376491in}}%
\pgfpathcurveto{\pgfqpoint{4.169169in}{3.385700in}}{\pgfqpoint{4.156678in}{3.390874in}}{\pgfqpoint{4.143655in}{3.390874in}}%
\pgfpathcurveto{\pgfqpoint{4.130632in}{3.390874in}}{\pgfqpoint{4.118141in}{3.385700in}}{\pgfqpoint{4.108933in}{3.376491in}}%
\pgfpathcurveto{\pgfqpoint{4.099724in}{3.367283in}}{\pgfqpoint{4.094550in}{3.354792in}}{\pgfqpoint{4.094550in}{3.341769in}}%
\pgfpathcurveto{\pgfqpoint{4.094550in}{3.328746in}}{\pgfqpoint{4.099724in}{3.316255in}}{\pgfqpoint{4.108933in}{3.307047in}}%
\pgfpathcurveto{\pgfqpoint{4.118141in}{3.297838in}}{\pgfqpoint{4.130632in}{3.292664in}}{\pgfqpoint{4.143655in}{3.292664in}}%
\pgfpathlineto{\pgfqpoint{4.143655in}{3.292664in}}%
\pgfpathclose%
\pgfusepath{stroke,fill}%
\end{pgfscope}%
\begin{pgfscope}%
\pgfpathrectangle{\pgfqpoint{0.454429in}{0.261491in}}{\pgfqpoint{6.040000in}{6.040000in}}%
\pgfusepath{clip}%
\pgfsetbuttcap%
\pgfsetroundjoin%
\definecolor{currentfill}{rgb}{0.121569,0.466667,0.705882}%
\pgfsetfillcolor{currentfill}%
\pgfsetlinewidth{1.003750pt}%
\definecolor{currentstroke}{rgb}{0.121569,0.466667,0.705882}%
\pgfsetstrokecolor{currentstroke}%
\pgfsetdash{}{0pt}%
\pgfpathmoveto{\pgfqpoint{1.809790in}{3.297381in}}%
\pgfpathcurveto{\pgfqpoint{1.822813in}{3.297381in}}{\pgfqpoint{1.835304in}{3.302555in}}{\pgfqpoint{1.844513in}{3.311763in}}%
\pgfpathcurveto{\pgfqpoint{1.853721in}{3.320972in}}{\pgfqpoint{1.858895in}{3.333463in}}{\pgfqpoint{1.858895in}{3.346486in}}%
\pgfpathcurveto{\pgfqpoint{1.858895in}{3.359508in}}{\pgfqpoint{1.853721in}{3.371999in}}{\pgfqpoint{1.844513in}{3.381208in}}%
\pgfpathcurveto{\pgfqpoint{1.835304in}{3.390416in}}{\pgfqpoint{1.822813in}{3.395590in}}{\pgfqpoint{1.809790in}{3.395590in}}%
\pgfpathcurveto{\pgfqpoint{1.796768in}{3.395590in}}{\pgfqpoint{1.784277in}{3.390416in}}{\pgfqpoint{1.775068in}{3.381208in}}%
\pgfpathcurveto{\pgfqpoint{1.765860in}{3.371999in}}{\pgfqpoint{1.760686in}{3.359508in}}{\pgfqpoint{1.760686in}{3.346486in}}%
\pgfpathcurveto{\pgfqpoint{1.760686in}{3.333463in}}{\pgfqpoint{1.765860in}{3.320972in}}{\pgfqpoint{1.775068in}{3.311763in}}%
\pgfpathcurveto{\pgfqpoint{1.784277in}{3.302555in}}{\pgfqpoint{1.796768in}{3.297381in}}{\pgfqpoint{1.809790in}{3.297381in}}%
\pgfpathlineto{\pgfqpoint{1.809790in}{3.297381in}}%
\pgfpathclose%
\pgfusepath{stroke,fill}%
\end{pgfscope}%
\begin{pgfscope}%
\pgfpathrectangle{\pgfqpoint{0.454429in}{0.261491in}}{\pgfqpoint{6.040000in}{6.040000in}}%
\pgfusepath{clip}%
\pgfsetbuttcap%
\pgfsetroundjoin%
\definecolor{currentfill}{rgb}{0.121569,0.466667,0.705882}%
\pgfsetfillcolor{currentfill}%
\pgfsetlinewidth{1.003750pt}%
\definecolor{currentstroke}{rgb}{0.121569,0.466667,0.705882}%
\pgfsetstrokecolor{currentstroke}%
\pgfsetdash{}{0pt}%
\pgfpathmoveto{\pgfqpoint{4.519812in}{3.306737in}}%
\pgfpathcurveto{\pgfqpoint{4.532835in}{3.306737in}}{\pgfqpoint{4.545326in}{3.311911in}}{\pgfqpoint{4.554534in}{3.321119in}}%
\pgfpathcurveto{\pgfqpoint{4.563743in}{3.330328in}}{\pgfqpoint{4.568916in}{3.342819in}}{\pgfqpoint{4.568916in}{3.355842in}}%
\pgfpathcurveto{\pgfqpoint{4.568916in}{3.368864in}}{\pgfqpoint{4.563743in}{3.381355in}}{\pgfqpoint{4.554534in}{3.390564in}}%
\pgfpathcurveto{\pgfqpoint{4.545326in}{3.399772in}}{\pgfqpoint{4.532835in}{3.404946in}}{\pgfqpoint{4.519812in}{3.404946in}}%
\pgfpathcurveto{\pgfqpoint{4.506789in}{3.404946in}}{\pgfqpoint{4.494298in}{3.399772in}}{\pgfqpoint{4.485090in}{3.390564in}}%
\pgfpathcurveto{\pgfqpoint{4.475881in}{3.381355in}}{\pgfqpoint{4.470707in}{3.368864in}}{\pgfqpoint{4.470707in}{3.355842in}}%
\pgfpathcurveto{\pgfqpoint{4.470707in}{3.342819in}}{\pgfqpoint{4.475881in}{3.330328in}}{\pgfqpoint{4.485090in}{3.321119in}}%
\pgfpathcurveto{\pgfqpoint{4.494298in}{3.311911in}}{\pgfqpoint{4.506789in}{3.306737in}}{\pgfqpoint{4.519812in}{3.306737in}}%
\pgfpathlineto{\pgfqpoint{4.519812in}{3.306737in}}%
\pgfpathclose%
\pgfusepath{stroke,fill}%
\end{pgfscope}%
\begin{pgfscope}%
\pgfpathrectangle{\pgfqpoint{0.454429in}{0.261491in}}{\pgfqpoint{6.040000in}{6.040000in}}%
\pgfusepath{clip}%
\pgfsetbuttcap%
\pgfsetroundjoin%
\definecolor{currentfill}{rgb}{0.121569,0.466667,0.705882}%
\pgfsetfillcolor{currentfill}%
\pgfsetlinewidth{1.003750pt}%
\definecolor{currentstroke}{rgb}{0.121569,0.466667,0.705882}%
\pgfsetstrokecolor{currentstroke}%
\pgfsetdash{}{0pt}%
\pgfpathmoveto{\pgfqpoint{4.890853in}{3.333506in}}%
\pgfpathcurveto{\pgfqpoint{4.903876in}{3.333506in}}{\pgfqpoint{4.916367in}{3.338680in}}{\pgfqpoint{4.925576in}{3.347888in}}%
\pgfpathcurveto{\pgfqpoint{4.934784in}{3.357096in}}{\pgfqpoint{4.939958in}{3.369587in}}{\pgfqpoint{4.939958in}{3.382610in}}%
\pgfpathcurveto{\pgfqpoint{4.939958in}{3.395633in}}{\pgfqpoint{4.934784in}{3.408124in}}{\pgfqpoint{4.925576in}{3.417332in}}%
\pgfpathcurveto{\pgfqpoint{4.916367in}{3.426541in}}{\pgfqpoint{4.903876in}{3.431715in}}{\pgfqpoint{4.890853in}{3.431715in}}%
\pgfpathcurveto{\pgfqpoint{4.877831in}{3.431715in}}{\pgfqpoint{4.865340in}{3.426541in}}{\pgfqpoint{4.856131in}{3.417332in}}%
\pgfpathcurveto{\pgfqpoint{4.846923in}{3.408124in}}{\pgfqpoint{4.841749in}{3.395633in}}{\pgfqpoint{4.841749in}{3.382610in}}%
\pgfpathcurveto{\pgfqpoint{4.841749in}{3.369587in}}{\pgfqpoint{4.846923in}{3.357096in}}{\pgfqpoint{4.856131in}{3.347888in}}%
\pgfpathcurveto{\pgfqpoint{4.865340in}{3.338680in}}{\pgfqpoint{4.877831in}{3.333506in}}{\pgfqpoint{4.890853in}{3.333506in}}%
\pgfpathlineto{\pgfqpoint{4.890853in}{3.333506in}}%
\pgfpathclose%
\pgfusepath{stroke,fill}%
\end{pgfscope}%
\begin{pgfscope}%
\pgfpathrectangle{\pgfqpoint{0.454429in}{0.261491in}}{\pgfqpoint{6.040000in}{6.040000in}}%
\pgfusepath{clip}%
\pgfsetbuttcap%
\pgfsetroundjoin%
\definecolor{currentfill}{rgb}{0.121569,0.466667,0.705882}%
\pgfsetfillcolor{currentfill}%
\pgfsetlinewidth{1.003750pt}%
\definecolor{currentstroke}{rgb}{0.121569,0.466667,0.705882}%
\pgfsetstrokecolor{currentstroke}%
\pgfsetdash{}{0pt}%
\pgfpathmoveto{\pgfqpoint{5.257462in}{3.362881in}}%
\pgfpathcurveto{\pgfqpoint{5.270485in}{3.362881in}}{\pgfqpoint{5.282976in}{3.368055in}}{\pgfqpoint{5.292184in}{3.377263in}}%
\pgfpathcurveto{\pgfqpoint{5.301393in}{3.386471in}}{\pgfqpoint{5.306567in}{3.398962in}}{\pgfqpoint{5.306567in}{3.411985in}}%
\pgfpathcurveto{\pgfqpoint{5.306567in}{3.425008in}}{\pgfqpoint{5.301393in}{3.437499in}}{\pgfqpoint{5.292184in}{3.446707in}}%
\pgfpathcurveto{\pgfqpoint{5.282976in}{3.455916in}}{\pgfqpoint{5.270485in}{3.461090in}}{\pgfqpoint{5.257462in}{3.461090in}}%
\pgfpathcurveto{\pgfqpoint{5.244439in}{3.461090in}}{\pgfqpoint{5.231948in}{3.455916in}}{\pgfqpoint{5.222740in}{3.446707in}}%
\pgfpathcurveto{\pgfqpoint{5.213531in}{3.437499in}}{\pgfqpoint{5.208357in}{3.425008in}}{\pgfqpoint{5.208357in}{3.411985in}}%
\pgfpathcurveto{\pgfqpoint{5.208357in}{3.398962in}}{\pgfqpoint{5.213531in}{3.386471in}}{\pgfqpoint{5.222740in}{3.377263in}}%
\pgfpathcurveto{\pgfqpoint{5.231948in}{3.368055in}}{\pgfqpoint{5.244439in}{3.362881in}}{\pgfqpoint{5.257462in}{3.362881in}}%
\pgfpathlineto{\pgfqpoint{5.257462in}{3.362881in}}%
\pgfpathclose%
\pgfusepath{stroke,fill}%
\end{pgfscope}%
\begin{pgfscope}%
\pgfpathrectangle{\pgfqpoint{0.454429in}{0.261491in}}{\pgfqpoint{6.040000in}{6.040000in}}%
\pgfusepath{clip}%
\pgfsetbuttcap%
\pgfsetroundjoin%
\definecolor{currentfill}{rgb}{0.121569,0.466667,0.705882}%
\pgfsetfillcolor{currentfill}%
\pgfsetlinewidth{1.003750pt}%
\definecolor{currentstroke}{rgb}{0.121569,0.466667,0.705882}%
\pgfsetstrokecolor{currentstroke}%
\pgfsetdash{}{0pt}%
\pgfpathmoveto{\pgfqpoint{2.778527in}{3.560087in}}%
\pgfpathcurveto{\pgfqpoint{2.791550in}{3.560087in}}{\pgfqpoint{2.804041in}{3.565261in}}{\pgfqpoint{2.813250in}{3.574469in}}%
\pgfpathcurveto{\pgfqpoint{2.822458in}{3.583678in}}{\pgfqpoint{2.827632in}{3.596169in}}{\pgfqpoint{2.827632in}{3.609192in}}%
\pgfpathcurveto{\pgfqpoint{2.827632in}{3.622214in}}{\pgfqpoint{2.822458in}{3.634705in}}{\pgfqpoint{2.813250in}{3.643914in}}%
\pgfpathcurveto{\pgfqpoint{2.804041in}{3.653122in}}{\pgfqpoint{2.791550in}{3.658296in}}{\pgfqpoint{2.778527in}{3.658296in}}%
\pgfpathcurveto{\pgfqpoint{2.765505in}{3.658296in}}{\pgfqpoint{2.753013in}{3.653122in}}{\pgfqpoint{2.743805in}{3.643914in}}%
\pgfpathcurveto{\pgfqpoint{2.734597in}{3.634705in}}{\pgfqpoint{2.729423in}{3.622214in}}{\pgfqpoint{2.729423in}{3.609192in}}%
\pgfpathcurveto{\pgfqpoint{2.729423in}{3.596169in}}{\pgfqpoint{2.734597in}{3.583678in}}{\pgfqpoint{2.743805in}{3.574469in}}%
\pgfpathcurveto{\pgfqpoint{2.753013in}{3.565261in}}{\pgfqpoint{2.765505in}{3.560087in}}{\pgfqpoint{2.778527in}{3.560087in}}%
\pgfpathlineto{\pgfqpoint{2.778527in}{3.560087in}}%
\pgfpathclose%
\pgfusepath{stroke,fill}%
\end{pgfscope}%
\begin{pgfscope}%
\pgfpathrectangle{\pgfqpoint{0.454429in}{0.261491in}}{\pgfqpoint{6.040000in}{6.040000in}}%
\pgfusepath{clip}%
\pgfsetbuttcap%
\pgfsetroundjoin%
\definecolor{currentfill}{rgb}{0.121569,0.466667,0.705882}%
\pgfsetfillcolor{currentfill}%
\pgfsetlinewidth{1.003750pt}%
\definecolor{currentstroke}{rgb}{0.121569,0.466667,0.705882}%
\pgfsetstrokecolor{currentstroke}%
\pgfsetdash{}{0pt}%
\pgfpathmoveto{\pgfqpoint{3.168219in}{3.561486in}}%
\pgfpathcurveto{\pgfqpoint{3.181242in}{3.561486in}}{\pgfqpoint{3.193733in}{3.566660in}}{\pgfqpoint{3.202942in}{3.575869in}}%
\pgfpathcurveto{\pgfqpoint{3.212150in}{3.585077in}}{\pgfqpoint{3.217324in}{3.597568in}}{\pgfqpoint{3.217324in}{3.610591in}}%
\pgfpathcurveto{\pgfqpoint{3.217324in}{3.623614in}}{\pgfqpoint{3.212150in}{3.636105in}}{\pgfqpoint{3.202942in}{3.645313in}}%
\pgfpathcurveto{\pgfqpoint{3.193733in}{3.654522in}}{\pgfqpoint{3.181242in}{3.659696in}}{\pgfqpoint{3.168219in}{3.659696in}}%
\pgfpathcurveto{\pgfqpoint{3.155197in}{3.659696in}}{\pgfqpoint{3.142706in}{3.654522in}}{\pgfqpoint{3.133497in}{3.645313in}}%
\pgfpathcurveto{\pgfqpoint{3.124289in}{3.636105in}}{\pgfqpoint{3.119115in}{3.623614in}}{\pgfqpoint{3.119115in}{3.610591in}}%
\pgfpathcurveto{\pgfqpoint{3.119115in}{3.597568in}}{\pgfqpoint{3.124289in}{3.585077in}}{\pgfqpoint{3.133497in}{3.575869in}}%
\pgfpathcurveto{\pgfqpoint{3.142706in}{3.566660in}}{\pgfqpoint{3.155197in}{3.561486in}}{\pgfqpoint{3.168219in}{3.561486in}}%
\pgfpathlineto{\pgfqpoint{3.168219in}{3.561486in}}%
\pgfpathclose%
\pgfusepath{stroke,fill}%
\end{pgfscope}%
\begin{pgfscope}%
\pgfpathrectangle{\pgfqpoint{0.454429in}{0.261491in}}{\pgfqpoint{6.040000in}{6.040000in}}%
\pgfusepath{clip}%
\pgfsetbuttcap%
\pgfsetroundjoin%
\definecolor{currentfill}{rgb}{0.121569,0.466667,0.705882}%
\pgfsetfillcolor{currentfill}%
\pgfsetlinewidth{1.003750pt}%
\definecolor{currentstroke}{rgb}{0.121569,0.466667,0.705882}%
\pgfsetstrokecolor{currentstroke}%
\pgfsetdash{}{0pt}%
\pgfpathmoveto{\pgfqpoint{2.387795in}{3.565206in}}%
\pgfpathcurveto{\pgfqpoint{2.400817in}{3.565206in}}{\pgfqpoint{2.413308in}{3.570380in}}{\pgfqpoint{2.422517in}{3.579589in}}%
\pgfpathcurveto{\pgfqpoint{2.431725in}{3.588797in}}{\pgfqpoint{2.436899in}{3.601288in}}{\pgfqpoint{2.436899in}{3.614311in}}%
\pgfpathcurveto{\pgfqpoint{2.436899in}{3.627334in}}{\pgfqpoint{2.431725in}{3.639825in}}{\pgfqpoint{2.422517in}{3.649033in}}%
\pgfpathcurveto{\pgfqpoint{2.413308in}{3.658242in}}{\pgfqpoint{2.400817in}{3.663416in}}{\pgfqpoint{2.387795in}{3.663416in}}%
\pgfpathcurveto{\pgfqpoint{2.374772in}{3.663416in}}{\pgfqpoint{2.362281in}{3.658242in}}{\pgfqpoint{2.353072in}{3.649033in}}%
\pgfpathcurveto{\pgfqpoint{2.343864in}{3.639825in}}{\pgfqpoint{2.338690in}{3.627334in}}{\pgfqpoint{2.338690in}{3.614311in}}%
\pgfpathcurveto{\pgfqpoint{2.338690in}{3.601288in}}{\pgfqpoint{2.343864in}{3.588797in}}{\pgfqpoint{2.353072in}{3.579589in}}%
\pgfpathcurveto{\pgfqpoint{2.362281in}{3.570380in}}{\pgfqpoint{2.374772in}{3.565206in}}{\pgfqpoint{2.387795in}{3.565206in}}%
\pgfpathlineto{\pgfqpoint{2.387795in}{3.565206in}}%
\pgfpathclose%
\pgfusepath{stroke,fill}%
\end{pgfscope}%
\begin{pgfscope}%
\pgfpathrectangle{\pgfqpoint{0.454429in}{0.261491in}}{\pgfqpoint{6.040000in}{6.040000in}}%
\pgfusepath{clip}%
\pgfsetbuttcap%
\pgfsetroundjoin%
\definecolor{currentfill}{rgb}{0.121569,0.466667,0.705882}%
\pgfsetfillcolor{currentfill}%
\pgfsetlinewidth{1.003750pt}%
\definecolor{currentstroke}{rgb}{0.121569,0.466667,0.705882}%
\pgfsetstrokecolor{currentstroke}%
\pgfsetdash{}{0pt}%
\pgfpathmoveto{\pgfqpoint{3.555728in}{3.567409in}}%
\pgfpathcurveto{\pgfqpoint{3.568751in}{3.567409in}}{\pgfqpoint{3.581242in}{3.572583in}}{\pgfqpoint{3.590451in}{3.581791in}}%
\pgfpathcurveto{\pgfqpoint{3.599659in}{3.591000in}}{\pgfqpoint{3.604833in}{3.603491in}}{\pgfqpoint{3.604833in}{3.616514in}}%
\pgfpathcurveto{\pgfqpoint{3.604833in}{3.629536in}}{\pgfqpoint{3.599659in}{3.642027in}}{\pgfqpoint{3.590451in}{3.651236in}}%
\pgfpathcurveto{\pgfqpoint{3.581242in}{3.660444in}}{\pgfqpoint{3.568751in}{3.665618in}}{\pgfqpoint{3.555728in}{3.665618in}}%
\pgfpathcurveto{\pgfqpoint{3.542706in}{3.665618in}}{\pgfqpoint{3.530215in}{3.660444in}}{\pgfqpoint{3.521006in}{3.651236in}}%
\pgfpathcurveto{\pgfqpoint{3.511798in}{3.642027in}}{\pgfqpoint{3.506624in}{3.629536in}}{\pgfqpoint{3.506624in}{3.616514in}}%
\pgfpathcurveto{\pgfqpoint{3.506624in}{3.603491in}}{\pgfqpoint{3.511798in}{3.591000in}}{\pgfqpoint{3.521006in}{3.581791in}}%
\pgfpathcurveto{\pgfqpoint{3.530215in}{3.572583in}}{\pgfqpoint{3.542706in}{3.567409in}}{\pgfqpoint{3.555728in}{3.567409in}}%
\pgfpathlineto{\pgfqpoint{3.555728in}{3.567409in}}%
\pgfpathclose%
\pgfusepath{stroke,fill}%
\end{pgfscope}%
\begin{pgfscope}%
\pgfpathrectangle{\pgfqpoint{0.454429in}{0.261491in}}{\pgfqpoint{6.040000in}{6.040000in}}%
\pgfusepath{clip}%
\pgfsetbuttcap%
\pgfsetroundjoin%
\definecolor{currentfill}{rgb}{0.121569,0.466667,0.705882}%
\pgfsetfillcolor{currentfill}%
\pgfsetlinewidth{1.003750pt}%
\definecolor{currentstroke}{rgb}{0.121569,0.466667,0.705882}%
\pgfsetstrokecolor{currentstroke}%
\pgfsetdash{}{0pt}%
\pgfpathmoveto{\pgfqpoint{3.939718in}{3.577078in}}%
\pgfpathcurveto{\pgfqpoint{3.952741in}{3.577078in}}{\pgfqpoint{3.965232in}{3.582252in}}{\pgfqpoint{3.974441in}{3.591460in}}%
\pgfpathcurveto{\pgfqpoint{3.983649in}{3.600669in}}{\pgfqpoint{3.988823in}{3.613160in}}{\pgfqpoint{3.988823in}{3.626182in}}%
\pgfpathcurveto{\pgfqpoint{3.988823in}{3.639205in}}{\pgfqpoint{3.983649in}{3.651696in}}{\pgfqpoint{3.974441in}{3.660905in}}%
\pgfpathcurveto{\pgfqpoint{3.965232in}{3.670113in}}{\pgfqpoint{3.952741in}{3.675287in}}{\pgfqpoint{3.939718in}{3.675287in}}%
\pgfpathcurveto{\pgfqpoint{3.926696in}{3.675287in}}{\pgfqpoint{3.914205in}{3.670113in}}{\pgfqpoint{3.904996in}{3.660905in}}%
\pgfpathcurveto{\pgfqpoint{3.895788in}{3.651696in}}{\pgfqpoint{3.890614in}{3.639205in}}{\pgfqpoint{3.890614in}{3.626182in}}%
\pgfpathcurveto{\pgfqpoint{3.890614in}{3.613160in}}{\pgfqpoint{3.895788in}{3.600669in}}{\pgfqpoint{3.904996in}{3.591460in}}%
\pgfpathcurveto{\pgfqpoint{3.914205in}{3.582252in}}{\pgfqpoint{3.926696in}{3.577078in}}{\pgfqpoint{3.939718in}{3.577078in}}%
\pgfpathlineto{\pgfqpoint{3.939718in}{3.577078in}}%
\pgfpathclose%
\pgfusepath{stroke,fill}%
\end{pgfscope}%
\begin{pgfscope}%
\pgfpathrectangle{\pgfqpoint{0.454429in}{0.261491in}}{\pgfqpoint{6.040000in}{6.040000in}}%
\pgfusepath{clip}%
\pgfsetbuttcap%
\pgfsetroundjoin%
\definecolor{currentfill}{rgb}{0.121569,0.466667,0.705882}%
\pgfsetfillcolor{currentfill}%
\pgfsetlinewidth{1.003750pt}%
\definecolor{currentstroke}{rgb}{0.121569,0.466667,0.705882}%
\pgfsetstrokecolor{currentstroke}%
\pgfsetdash{}{0pt}%
\pgfpathmoveto{\pgfqpoint{2.003048in}{3.579622in}}%
\pgfpathcurveto{\pgfqpoint{2.016071in}{3.579622in}}{\pgfqpoint{2.028562in}{3.584796in}}{\pgfqpoint{2.037771in}{3.594004in}}%
\pgfpathcurveto{\pgfqpoint{2.046979in}{3.603213in}}{\pgfqpoint{2.052153in}{3.615704in}}{\pgfqpoint{2.052153in}{3.628726in}}%
\pgfpathcurveto{\pgfqpoint{2.052153in}{3.641749in}}{\pgfqpoint{2.046979in}{3.654240in}}{\pgfqpoint{2.037771in}{3.663449in}}%
\pgfpathcurveto{\pgfqpoint{2.028562in}{3.672657in}}{\pgfqpoint{2.016071in}{3.677831in}}{\pgfqpoint{2.003048in}{3.677831in}}%
\pgfpathcurveto{\pgfqpoint{1.990026in}{3.677831in}}{\pgfqpoint{1.977535in}{3.672657in}}{\pgfqpoint{1.968326in}{3.663449in}}%
\pgfpathcurveto{\pgfqpoint{1.959118in}{3.654240in}}{\pgfqpoint{1.953944in}{3.641749in}}{\pgfqpoint{1.953944in}{3.628726in}}%
\pgfpathcurveto{\pgfqpoint{1.953944in}{3.615704in}}{\pgfqpoint{1.959118in}{3.603213in}}{\pgfqpoint{1.968326in}{3.594004in}}%
\pgfpathcurveto{\pgfqpoint{1.977535in}{3.584796in}}{\pgfqpoint{1.990026in}{3.579622in}}{\pgfqpoint{2.003048in}{3.579622in}}%
\pgfpathlineto{\pgfqpoint{2.003048in}{3.579622in}}%
\pgfpathclose%
\pgfusepath{stroke,fill}%
\end{pgfscope}%
\begin{pgfscope}%
\pgfpathrectangle{\pgfqpoint{0.454429in}{0.261491in}}{\pgfqpoint{6.040000in}{6.040000in}}%
\pgfusepath{clip}%
\pgfsetbuttcap%
\pgfsetroundjoin%
\definecolor{currentfill}{rgb}{0.121569,0.466667,0.705882}%
\pgfsetfillcolor{currentfill}%
\pgfsetlinewidth{1.003750pt}%
\definecolor{currentstroke}{rgb}{0.121569,0.466667,0.705882}%
\pgfsetstrokecolor{currentstroke}%
\pgfsetdash{}{0pt}%
\pgfpathmoveto{\pgfqpoint{4.318779in}{3.591626in}}%
\pgfpathcurveto{\pgfqpoint{4.331802in}{3.591626in}}{\pgfqpoint{4.344293in}{3.596800in}}{\pgfqpoint{4.353501in}{3.606008in}}%
\pgfpathcurveto{\pgfqpoint{4.362709in}{3.615217in}}{\pgfqpoint{4.367883in}{3.627708in}}{\pgfqpoint{4.367883in}{3.640731in}}%
\pgfpathcurveto{\pgfqpoint{4.367883in}{3.653753in}}{\pgfqpoint{4.362709in}{3.666244in}}{\pgfqpoint{4.353501in}{3.675453in}}%
\pgfpathcurveto{\pgfqpoint{4.344293in}{3.684661in}}{\pgfqpoint{4.331802in}{3.689835in}}{\pgfqpoint{4.318779in}{3.689835in}}%
\pgfpathcurveto{\pgfqpoint{4.305756in}{3.689835in}}{\pgfqpoint{4.293265in}{3.684661in}}{\pgfqpoint{4.284057in}{3.675453in}}%
\pgfpathcurveto{\pgfqpoint{4.274848in}{3.666244in}}{\pgfqpoint{4.269674in}{3.653753in}}{\pgfqpoint{4.269674in}{3.640731in}}%
\pgfpathcurveto{\pgfqpoint{4.269674in}{3.627708in}}{\pgfqpoint{4.274848in}{3.615217in}}{\pgfqpoint{4.284057in}{3.606008in}}%
\pgfpathcurveto{\pgfqpoint{4.293265in}{3.596800in}}{\pgfqpoint{4.305756in}{3.591626in}}{\pgfqpoint{4.318779in}{3.591626in}}%
\pgfpathlineto{\pgfqpoint{4.318779in}{3.591626in}}%
\pgfpathclose%
\pgfusepath{stroke,fill}%
\end{pgfscope}%
\begin{pgfscope}%
\pgfpathrectangle{\pgfqpoint{0.454429in}{0.261491in}}{\pgfqpoint{6.040000in}{6.040000in}}%
\pgfusepath{clip}%
\pgfsetbuttcap%
\pgfsetroundjoin%
\definecolor{currentfill}{rgb}{0.121569,0.466667,0.705882}%
\pgfsetfillcolor{currentfill}%
\pgfsetlinewidth{1.003750pt}%
\definecolor{currentstroke}{rgb}{0.121569,0.466667,0.705882}%
\pgfsetstrokecolor{currentstroke}%
\pgfsetdash{}{0pt}%
\pgfpathmoveto{\pgfqpoint{4.692412in}{3.617461in}}%
\pgfpathcurveto{\pgfqpoint{4.705435in}{3.617461in}}{\pgfqpoint{4.717926in}{3.622635in}}{\pgfqpoint{4.727134in}{3.631843in}}%
\pgfpathcurveto{\pgfqpoint{4.736342in}{3.641052in}}{\pgfqpoint{4.741516in}{3.653543in}}{\pgfqpoint{4.741516in}{3.666565in}}%
\pgfpathcurveto{\pgfqpoint{4.741516in}{3.679588in}}{\pgfqpoint{4.736342in}{3.692079in}}{\pgfqpoint{4.727134in}{3.701288in}}%
\pgfpathcurveto{\pgfqpoint{4.717926in}{3.710496in}}{\pgfqpoint{4.705435in}{3.715670in}}{\pgfqpoint{4.692412in}{3.715670in}}%
\pgfpathcurveto{\pgfqpoint{4.679389in}{3.715670in}}{\pgfqpoint{4.666898in}{3.710496in}}{\pgfqpoint{4.657690in}{3.701288in}}%
\pgfpathcurveto{\pgfqpoint{4.648481in}{3.692079in}}{\pgfqpoint{4.643307in}{3.679588in}}{\pgfqpoint{4.643307in}{3.666565in}}%
\pgfpathcurveto{\pgfqpoint{4.643307in}{3.653543in}}{\pgfqpoint{4.648481in}{3.641052in}}{\pgfqpoint{4.657690in}{3.631843in}}%
\pgfpathcurveto{\pgfqpoint{4.666898in}{3.622635in}}{\pgfqpoint{4.679389in}{3.617461in}}{\pgfqpoint{4.692412in}{3.617461in}}%
\pgfpathlineto{\pgfqpoint{4.692412in}{3.617461in}}%
\pgfpathclose%
\pgfusepath{stroke,fill}%
\end{pgfscope}%
\begin{pgfscope}%
\pgfpathrectangle{\pgfqpoint{0.454429in}{0.261491in}}{\pgfqpoint{6.040000in}{6.040000in}}%
\pgfusepath{clip}%
\pgfsetbuttcap%
\pgfsetroundjoin%
\definecolor{currentfill}{rgb}{0.121569,0.466667,0.705882}%
\pgfsetfillcolor{currentfill}%
\pgfsetlinewidth{1.003750pt}%
\definecolor{currentstroke}{rgb}{0.121569,0.466667,0.705882}%
\pgfsetstrokecolor{currentstroke}%
\pgfsetdash{}{0pt}%
\pgfpathmoveto{\pgfqpoint{5.060091in}{3.651128in}}%
\pgfpathcurveto{\pgfqpoint{5.073114in}{3.651128in}}{\pgfqpoint{5.085605in}{3.656302in}}{\pgfqpoint{5.094813in}{3.665511in}}%
\pgfpathcurveto{\pgfqpoint{5.104022in}{3.674719in}}{\pgfqpoint{5.109196in}{3.687210in}}{\pgfqpoint{5.109196in}{3.700233in}}%
\pgfpathcurveto{\pgfqpoint{5.109196in}{3.713256in}}{\pgfqpoint{5.104022in}{3.725747in}}{\pgfqpoint{5.094813in}{3.734955in}}%
\pgfpathcurveto{\pgfqpoint{5.085605in}{3.744163in}}{\pgfqpoint{5.073114in}{3.749337in}}{\pgfqpoint{5.060091in}{3.749337in}}%
\pgfpathcurveto{\pgfqpoint{5.047068in}{3.749337in}}{\pgfqpoint{5.034577in}{3.744163in}}{\pgfqpoint{5.025369in}{3.734955in}}%
\pgfpathcurveto{\pgfqpoint{5.016160in}{3.725747in}}{\pgfqpoint{5.010986in}{3.713256in}}{\pgfqpoint{5.010986in}{3.700233in}}%
\pgfpathcurveto{\pgfqpoint{5.010986in}{3.687210in}}{\pgfqpoint{5.016160in}{3.674719in}}{\pgfqpoint{5.025369in}{3.665511in}}%
\pgfpathcurveto{\pgfqpoint{5.034577in}{3.656302in}}{\pgfqpoint{5.047068in}{3.651128in}}{\pgfqpoint{5.060091in}{3.651128in}}%
\pgfpathlineto{\pgfqpoint{5.060091in}{3.651128in}}%
\pgfpathclose%
\pgfusepath{stroke,fill}%
\end{pgfscope}%
\begin{pgfscope}%
\pgfpathrectangle{\pgfqpoint{0.454429in}{0.261491in}}{\pgfqpoint{6.040000in}{6.040000in}}%
\pgfusepath{clip}%
\pgfsetbuttcap%
\pgfsetroundjoin%
\definecolor{currentfill}{rgb}{0.121569,0.466667,0.705882}%
\pgfsetfillcolor{currentfill}%
\pgfsetlinewidth{1.003750pt}%
\definecolor{currentstroke}{rgb}{0.121569,0.466667,0.705882}%
\pgfsetstrokecolor{currentstroke}%
\pgfsetdash{}{0pt}%
\pgfpathmoveto{\pgfqpoint{2.968728in}{3.842485in}}%
\pgfpathcurveto{\pgfqpoint{2.981751in}{3.842485in}}{\pgfqpoint{2.994242in}{3.847659in}}{\pgfqpoint{3.003451in}{3.856867in}}%
\pgfpathcurveto{\pgfqpoint{3.012659in}{3.866076in}}{\pgfqpoint{3.017833in}{3.878567in}}{\pgfqpoint{3.017833in}{3.891590in}}%
\pgfpathcurveto{\pgfqpoint{3.017833in}{3.904612in}}{\pgfqpoint{3.012659in}{3.917103in}}{\pgfqpoint{3.003451in}{3.926312in}}%
\pgfpathcurveto{\pgfqpoint{2.994242in}{3.935520in}}{\pgfqpoint{2.981751in}{3.940694in}}{\pgfqpoint{2.968728in}{3.940694in}}%
\pgfpathcurveto{\pgfqpoint{2.955706in}{3.940694in}}{\pgfqpoint{2.943215in}{3.935520in}}{\pgfqpoint{2.934006in}{3.926312in}}%
\pgfpathcurveto{\pgfqpoint{2.924798in}{3.917103in}}{\pgfqpoint{2.919624in}{3.904612in}}{\pgfqpoint{2.919624in}{3.891590in}}%
\pgfpathcurveto{\pgfqpoint{2.919624in}{3.878567in}}{\pgfqpoint{2.924798in}{3.866076in}}{\pgfqpoint{2.934006in}{3.856867in}}%
\pgfpathcurveto{\pgfqpoint{2.943215in}{3.847659in}}{\pgfqpoint{2.955706in}{3.842485in}}{\pgfqpoint{2.968728in}{3.842485in}}%
\pgfpathlineto{\pgfqpoint{2.968728in}{3.842485in}}%
\pgfpathclose%
\pgfusepath{stroke,fill}%
\end{pgfscope}%
\begin{pgfscope}%
\pgfpathrectangle{\pgfqpoint{0.454429in}{0.261491in}}{\pgfqpoint{6.040000in}{6.040000in}}%
\pgfusepath{clip}%
\pgfsetbuttcap%
\pgfsetroundjoin%
\definecolor{currentfill}{rgb}{0.121569,0.466667,0.705882}%
\pgfsetfillcolor{currentfill}%
\pgfsetlinewidth{1.003750pt}%
\definecolor{currentstroke}{rgb}{0.121569,0.466667,0.705882}%
\pgfsetstrokecolor{currentstroke}%
\pgfsetdash{}{0pt}%
\pgfpathmoveto{\pgfqpoint{2.579774in}{3.842867in}}%
\pgfpathcurveto{\pgfqpoint{2.592796in}{3.842867in}}{\pgfqpoint{2.605288in}{3.848041in}}{\pgfqpoint{2.614496in}{3.857249in}}%
\pgfpathcurveto{\pgfqpoint{2.623704in}{3.866458in}}{\pgfqpoint{2.628878in}{3.878949in}}{\pgfqpoint{2.628878in}{3.891971in}}%
\pgfpathcurveto{\pgfqpoint{2.628878in}{3.904994in}}{\pgfqpoint{2.623704in}{3.917485in}}{\pgfqpoint{2.614496in}{3.926694in}}%
\pgfpathcurveto{\pgfqpoint{2.605288in}{3.935902in}}{\pgfqpoint{2.592796in}{3.941076in}}{\pgfqpoint{2.579774in}{3.941076in}}%
\pgfpathcurveto{\pgfqpoint{2.566751in}{3.941076in}}{\pgfqpoint{2.554260in}{3.935902in}}{\pgfqpoint{2.545052in}{3.926694in}}%
\pgfpathcurveto{\pgfqpoint{2.535843in}{3.917485in}}{\pgfqpoint{2.530669in}{3.904994in}}{\pgfqpoint{2.530669in}{3.891971in}}%
\pgfpathcurveto{\pgfqpoint{2.530669in}{3.878949in}}{\pgfqpoint{2.535843in}{3.866458in}}{\pgfqpoint{2.545052in}{3.857249in}}%
\pgfpathcurveto{\pgfqpoint{2.554260in}{3.848041in}}{\pgfqpoint{2.566751in}{3.842867in}}{\pgfqpoint{2.579774in}{3.842867in}}%
\pgfpathlineto{\pgfqpoint{2.579774in}{3.842867in}}%
\pgfpathclose%
\pgfusepath{stroke,fill}%
\end{pgfscope}%
\begin{pgfscope}%
\pgfpathrectangle{\pgfqpoint{0.454429in}{0.261491in}}{\pgfqpoint{6.040000in}{6.040000in}}%
\pgfusepath{clip}%
\pgfsetbuttcap%
\pgfsetroundjoin%
\definecolor{currentfill}{rgb}{0.121569,0.466667,0.705882}%
\pgfsetfillcolor{currentfill}%
\pgfsetlinewidth{1.003750pt}%
\definecolor{currentstroke}{rgb}{0.121569,0.466667,0.705882}%
\pgfsetstrokecolor{currentstroke}%
\pgfsetdash{}{0pt}%
\pgfpathmoveto{\pgfqpoint{3.355599in}{3.848075in}}%
\pgfpathcurveto{\pgfqpoint{3.368622in}{3.848075in}}{\pgfqpoint{3.381113in}{3.853249in}}{\pgfqpoint{3.390321in}{3.862457in}}%
\pgfpathcurveto{\pgfqpoint{3.399530in}{3.871666in}}{\pgfqpoint{3.404704in}{3.884157in}}{\pgfqpoint{3.404704in}{3.897180in}}%
\pgfpathcurveto{\pgfqpoint{3.404704in}{3.910202in}}{\pgfqpoint{3.399530in}{3.922693in}}{\pgfqpoint{3.390321in}{3.931902in}}%
\pgfpathcurveto{\pgfqpoint{3.381113in}{3.941110in}}{\pgfqpoint{3.368622in}{3.946284in}}{\pgfqpoint{3.355599in}{3.946284in}}%
\pgfpathcurveto{\pgfqpoint{3.342576in}{3.946284in}}{\pgfqpoint{3.330085in}{3.941110in}}{\pgfqpoint{3.320877in}{3.931902in}}%
\pgfpathcurveto{\pgfqpoint{3.311668in}{3.922693in}}{\pgfqpoint{3.306494in}{3.910202in}}{\pgfqpoint{3.306494in}{3.897180in}}%
\pgfpathcurveto{\pgfqpoint{3.306494in}{3.884157in}}{\pgfqpoint{3.311668in}{3.871666in}}{\pgfqpoint{3.320877in}{3.862457in}}%
\pgfpathcurveto{\pgfqpoint{3.330085in}{3.853249in}}{\pgfqpoint{3.342576in}{3.848075in}}{\pgfqpoint{3.355599in}{3.848075in}}%
\pgfpathlineto{\pgfqpoint{3.355599in}{3.848075in}}%
\pgfpathclose%
\pgfusepath{stroke,fill}%
\end{pgfscope}%
\begin{pgfscope}%
\pgfpathrectangle{\pgfqpoint{0.454429in}{0.261491in}}{\pgfqpoint{6.040000in}{6.040000in}}%
\pgfusepath{clip}%
\pgfsetbuttcap%
\pgfsetroundjoin%
\definecolor{currentfill}{rgb}{0.121569,0.466667,0.705882}%
\pgfsetfillcolor{currentfill}%
\pgfsetlinewidth{1.003750pt}%
\definecolor{currentstroke}{rgb}{0.121569,0.466667,0.705882}%
\pgfsetstrokecolor{currentstroke}%
\pgfsetdash{}{0pt}%
\pgfpathmoveto{\pgfqpoint{2.193058in}{3.857119in}}%
\pgfpathcurveto{\pgfqpoint{2.206081in}{3.857119in}}{\pgfqpoint{2.218572in}{3.862293in}}{\pgfqpoint{2.227781in}{3.871501in}}%
\pgfpathcurveto{\pgfqpoint{2.236989in}{3.880710in}}{\pgfqpoint{2.242163in}{3.893201in}}{\pgfqpoint{2.242163in}{3.906224in}}%
\pgfpathcurveto{\pgfqpoint{2.242163in}{3.919246in}}{\pgfqpoint{2.236989in}{3.931737in}}{\pgfqpoint{2.227781in}{3.940946in}}%
\pgfpathcurveto{\pgfqpoint{2.218572in}{3.950154in}}{\pgfqpoint{2.206081in}{3.955328in}}{\pgfqpoint{2.193058in}{3.955328in}}%
\pgfpathcurveto{\pgfqpoint{2.180036in}{3.955328in}}{\pgfqpoint{2.167545in}{3.950154in}}{\pgfqpoint{2.158336in}{3.940946in}}%
\pgfpathcurveto{\pgfqpoint{2.149128in}{3.931737in}}{\pgfqpoint{2.143954in}{3.919246in}}{\pgfqpoint{2.143954in}{3.906224in}}%
\pgfpathcurveto{\pgfqpoint{2.143954in}{3.893201in}}{\pgfqpoint{2.149128in}{3.880710in}}{\pgfqpoint{2.158336in}{3.871501in}}%
\pgfpathcurveto{\pgfqpoint{2.167545in}{3.862293in}}{\pgfqpoint{2.180036in}{3.857119in}}{\pgfqpoint{2.193058in}{3.857119in}}%
\pgfpathlineto{\pgfqpoint{2.193058in}{3.857119in}}%
\pgfpathclose%
\pgfusepath{stroke,fill}%
\end{pgfscope}%
\begin{pgfscope}%
\pgfpathrectangle{\pgfqpoint{0.454429in}{0.261491in}}{\pgfqpoint{6.040000in}{6.040000in}}%
\pgfusepath{clip}%
\pgfsetbuttcap%
\pgfsetroundjoin%
\definecolor{currentfill}{rgb}{0.121569,0.466667,0.705882}%
\pgfsetfillcolor{currentfill}%
\pgfsetlinewidth{1.003750pt}%
\definecolor{currentstroke}{rgb}{0.121569,0.466667,0.705882}%
\pgfsetstrokecolor{currentstroke}%
\pgfsetdash{}{0pt}%
\pgfpathmoveto{\pgfqpoint{3.739727in}{3.858481in}}%
\pgfpathcurveto{\pgfqpoint{3.752749in}{3.858481in}}{\pgfqpoint{3.765241in}{3.863655in}}{\pgfqpoint{3.774449in}{3.872863in}}%
\pgfpathcurveto{\pgfqpoint{3.783657in}{3.882072in}}{\pgfqpoint{3.788831in}{3.894563in}}{\pgfqpoint{3.788831in}{3.907586in}}%
\pgfpathcurveto{\pgfqpoint{3.788831in}{3.920608in}}{\pgfqpoint{3.783657in}{3.933099in}}{\pgfqpoint{3.774449in}{3.942308in}}%
\pgfpathcurveto{\pgfqpoint{3.765241in}{3.951516in}}{\pgfqpoint{3.752749in}{3.956690in}}{\pgfqpoint{3.739727in}{3.956690in}}%
\pgfpathcurveto{\pgfqpoint{3.726704in}{3.956690in}}{\pgfqpoint{3.714213in}{3.951516in}}{\pgfqpoint{3.705005in}{3.942308in}}%
\pgfpathcurveto{\pgfqpoint{3.695796in}{3.933099in}}{\pgfqpoint{3.690622in}{3.920608in}}{\pgfqpoint{3.690622in}{3.907586in}}%
\pgfpathcurveto{\pgfqpoint{3.690622in}{3.894563in}}{\pgfqpoint{3.695796in}{3.882072in}}{\pgfqpoint{3.705005in}{3.872863in}}%
\pgfpathcurveto{\pgfqpoint{3.714213in}{3.863655in}}{\pgfqpoint{3.726704in}{3.858481in}}{\pgfqpoint{3.739727in}{3.858481in}}%
\pgfpathlineto{\pgfqpoint{3.739727in}{3.858481in}}%
\pgfpathclose%
\pgfusepath{stroke,fill}%
\end{pgfscope}%
\begin{pgfscope}%
\pgfpathrectangle{\pgfqpoint{0.454429in}{0.261491in}}{\pgfqpoint{6.040000in}{6.040000in}}%
\pgfusepath{clip}%
\pgfsetbuttcap%
\pgfsetroundjoin%
\definecolor{currentfill}{rgb}{0.121569,0.466667,0.705882}%
\pgfsetfillcolor{currentfill}%
\pgfsetlinewidth{1.003750pt}%
\definecolor{currentstroke}{rgb}{0.121569,0.466667,0.705882}%
\pgfsetstrokecolor{currentstroke}%
\pgfsetdash{}{0pt}%
\pgfpathmoveto{\pgfqpoint{4.120336in}{3.873810in}}%
\pgfpathcurveto{\pgfqpoint{4.133359in}{3.873810in}}{\pgfqpoint{4.145850in}{3.878984in}}{\pgfqpoint{4.155058in}{3.888192in}}%
\pgfpathcurveto{\pgfqpoint{4.164267in}{3.897401in}}{\pgfqpoint{4.169441in}{3.909892in}}{\pgfqpoint{4.169441in}{3.922915in}}%
\pgfpathcurveto{\pgfqpoint{4.169441in}{3.935937in}}{\pgfqpoint{4.164267in}{3.948428in}}{\pgfqpoint{4.155058in}{3.957637in}}%
\pgfpathcurveto{\pgfqpoint{4.145850in}{3.966845in}}{\pgfqpoint{4.133359in}{3.972019in}}{\pgfqpoint{4.120336in}{3.972019in}}%
\pgfpathcurveto{\pgfqpoint{4.107313in}{3.972019in}}{\pgfqpoint{4.094822in}{3.966845in}}{\pgfqpoint{4.085614in}{3.957637in}}%
\pgfpathcurveto{\pgfqpoint{4.076405in}{3.948428in}}{\pgfqpoint{4.071231in}{3.935937in}}{\pgfqpoint{4.071231in}{3.922915in}}%
\pgfpathcurveto{\pgfqpoint{4.071231in}{3.909892in}}{\pgfqpoint{4.076405in}{3.897401in}}{\pgfqpoint{4.085614in}{3.888192in}}%
\pgfpathcurveto{\pgfqpoint{4.094822in}{3.878984in}}{\pgfqpoint{4.107313in}{3.873810in}}{\pgfqpoint{4.120336in}{3.873810in}}%
\pgfpathlineto{\pgfqpoint{4.120336in}{3.873810in}}%
\pgfpathclose%
\pgfusepath{stroke,fill}%
\end{pgfscope}%
\begin{pgfscope}%
\pgfpathrectangle{\pgfqpoint{0.454429in}{0.261491in}}{\pgfqpoint{6.040000in}{6.040000in}}%
\pgfusepath{clip}%
\pgfsetbuttcap%
\pgfsetroundjoin%
\definecolor{currentfill}{rgb}{0.121569,0.466667,0.705882}%
\pgfsetfillcolor{currentfill}%
\pgfsetlinewidth{1.003750pt}%
\definecolor{currentstroke}{rgb}{0.121569,0.466667,0.705882}%
\pgfsetstrokecolor{currentstroke}%
\pgfsetdash{}{0pt}%
\pgfpathmoveto{\pgfqpoint{4.495982in}{3.898411in}}%
\pgfpathcurveto{\pgfqpoint{4.509005in}{3.898411in}}{\pgfqpoint{4.521496in}{3.903585in}}{\pgfqpoint{4.530704in}{3.912793in}}%
\pgfpathcurveto{\pgfqpoint{4.539913in}{3.922002in}}{\pgfqpoint{4.545087in}{3.934493in}}{\pgfqpoint{4.545087in}{3.947516in}}%
\pgfpathcurveto{\pgfqpoint{4.545087in}{3.960538in}}{\pgfqpoint{4.539913in}{3.973029in}}{\pgfqpoint{4.530704in}{3.982238in}}%
\pgfpathcurveto{\pgfqpoint{4.521496in}{3.991446in}}{\pgfqpoint{4.509005in}{3.996620in}}{\pgfqpoint{4.495982in}{3.996620in}}%
\pgfpathcurveto{\pgfqpoint{4.482959in}{3.996620in}}{\pgfqpoint{4.470468in}{3.991446in}}{\pgfqpoint{4.461260in}{3.982238in}}%
\pgfpathcurveto{\pgfqpoint{4.452051in}{3.973029in}}{\pgfqpoint{4.446877in}{3.960538in}}{\pgfqpoint{4.446877in}{3.947516in}}%
\pgfpathcurveto{\pgfqpoint{4.446877in}{3.934493in}}{\pgfqpoint{4.452051in}{3.922002in}}{\pgfqpoint{4.461260in}{3.912793in}}%
\pgfpathcurveto{\pgfqpoint{4.470468in}{3.903585in}}{\pgfqpoint{4.482959in}{3.898411in}}{\pgfqpoint{4.495982in}{3.898411in}}%
\pgfpathlineto{\pgfqpoint{4.495982in}{3.898411in}}%
\pgfpathclose%
\pgfusepath{stroke,fill}%
\end{pgfscope}%
\begin{pgfscope}%
\pgfpathrectangle{\pgfqpoint{0.454429in}{0.261491in}}{\pgfqpoint{6.040000in}{6.040000in}}%
\pgfusepath{clip}%
\pgfsetbuttcap%
\pgfsetroundjoin%
\definecolor{currentfill}{rgb}{0.121569,0.466667,0.705882}%
\pgfsetfillcolor{currentfill}%
\pgfsetlinewidth{1.003750pt}%
\definecolor{currentstroke}{rgb}{0.121569,0.466667,0.705882}%
\pgfsetstrokecolor{currentstroke}%
\pgfsetdash{}{0pt}%
\pgfpathmoveto{\pgfqpoint{4.864700in}{3.936484in}}%
\pgfpathcurveto{\pgfqpoint{4.877723in}{3.936484in}}{\pgfqpoint{4.890214in}{3.941658in}}{\pgfqpoint{4.899423in}{3.950866in}}%
\pgfpathcurveto{\pgfqpoint{4.908631in}{3.960075in}}{\pgfqpoint{4.913805in}{3.972566in}}{\pgfqpoint{4.913805in}{3.985588in}}%
\pgfpathcurveto{\pgfqpoint{4.913805in}{3.998611in}}{\pgfqpoint{4.908631in}{4.011102in}}{\pgfqpoint{4.899423in}{4.020311in}}%
\pgfpathcurveto{\pgfqpoint{4.890214in}{4.029519in}}{\pgfqpoint{4.877723in}{4.034693in}}{\pgfqpoint{4.864700in}{4.034693in}}%
\pgfpathcurveto{\pgfqpoint{4.851678in}{4.034693in}}{\pgfqpoint{4.839187in}{4.029519in}}{\pgfqpoint{4.829978in}{4.020311in}}%
\pgfpathcurveto{\pgfqpoint{4.820770in}{4.011102in}}{\pgfqpoint{4.815596in}{3.998611in}}{\pgfqpoint{4.815596in}{3.985588in}}%
\pgfpathcurveto{\pgfqpoint{4.815596in}{3.972566in}}{\pgfqpoint{4.820770in}{3.960075in}}{\pgfqpoint{4.829978in}{3.950866in}}%
\pgfpathcurveto{\pgfqpoint{4.839187in}{3.941658in}}{\pgfqpoint{4.851678in}{3.936484in}}{\pgfqpoint{4.864700in}{3.936484in}}%
\pgfpathlineto{\pgfqpoint{4.864700in}{3.936484in}}%
\pgfpathclose%
\pgfusepath{stroke,fill}%
\end{pgfscope}%
\begin{pgfscope}%
\pgfpathrectangle{\pgfqpoint{0.454429in}{0.261491in}}{\pgfqpoint{6.040000in}{6.040000in}}%
\pgfusepath{clip}%
\pgfsetbuttcap%
\pgfsetroundjoin%
\definecolor{currentfill}{rgb}{0.121569,0.466667,0.705882}%
\pgfsetfillcolor{currentfill}%
\pgfsetlinewidth{1.003750pt}%
\definecolor{currentstroke}{rgb}{0.121569,0.466667,0.705882}%
\pgfsetstrokecolor{currentstroke}%
\pgfsetdash{}{0pt}%
\pgfpathmoveto{\pgfqpoint{2.768067in}{4.122698in}}%
\pgfpathcurveto{\pgfqpoint{2.781090in}{4.122698in}}{\pgfqpoint{2.793581in}{4.127872in}}{\pgfqpoint{2.802789in}{4.137080in}}%
\pgfpathcurveto{\pgfqpoint{2.811997in}{4.146289in}}{\pgfqpoint{2.817171in}{4.158780in}}{\pgfqpoint{2.817171in}{4.171802in}}%
\pgfpathcurveto{\pgfqpoint{2.817171in}{4.184825in}}{\pgfqpoint{2.811997in}{4.197316in}}{\pgfqpoint{2.802789in}{4.206525in}}%
\pgfpathcurveto{\pgfqpoint{2.793581in}{4.215733in}}{\pgfqpoint{2.781090in}{4.220907in}}{\pgfqpoint{2.768067in}{4.220907in}}%
\pgfpathcurveto{\pgfqpoint{2.755044in}{4.220907in}}{\pgfqpoint{2.742553in}{4.215733in}}{\pgfqpoint{2.733345in}{4.206525in}}%
\pgfpathcurveto{\pgfqpoint{2.724136in}{4.197316in}}{\pgfqpoint{2.718962in}{4.184825in}}{\pgfqpoint{2.718962in}{4.171802in}}%
\pgfpathcurveto{\pgfqpoint{2.718962in}{4.158780in}}{\pgfqpoint{2.724136in}{4.146289in}}{\pgfqpoint{2.733345in}{4.137080in}}%
\pgfpathcurveto{\pgfqpoint{2.742553in}{4.127872in}}{\pgfqpoint{2.755044in}{4.122698in}}{\pgfqpoint{2.768067in}{4.122698in}}%
\pgfpathlineto{\pgfqpoint{2.768067in}{4.122698in}}%
\pgfpathclose%
\pgfusepath{stroke,fill}%
\end{pgfscope}%
\begin{pgfscope}%
\pgfpathrectangle{\pgfqpoint{0.454429in}{0.261491in}}{\pgfqpoint{6.040000in}{6.040000in}}%
\pgfusepath{clip}%
\pgfsetbuttcap%
\pgfsetroundjoin%
\definecolor{currentfill}{rgb}{0.121569,0.466667,0.705882}%
\pgfsetfillcolor{currentfill}%
\pgfsetlinewidth{1.003750pt}%
\definecolor{currentstroke}{rgb}{0.121569,0.466667,0.705882}%
\pgfsetstrokecolor{currentstroke}%
\pgfsetdash{}{0pt}%
\pgfpathmoveto{\pgfqpoint{2.377142in}{4.125961in}}%
\pgfpathcurveto{\pgfqpoint{2.390165in}{4.125961in}}{\pgfqpoint{2.402656in}{4.131135in}}{\pgfqpoint{2.411864in}{4.140344in}}%
\pgfpathcurveto{\pgfqpoint{2.421073in}{4.149552in}}{\pgfqpoint{2.426247in}{4.162043in}}{\pgfqpoint{2.426247in}{4.175066in}}%
\pgfpathcurveto{\pgfqpoint{2.426247in}{4.188089in}}{\pgfqpoint{2.421073in}{4.200580in}}{\pgfqpoint{2.411864in}{4.209788in}}%
\pgfpathcurveto{\pgfqpoint{2.402656in}{4.218997in}}{\pgfqpoint{2.390165in}{4.224171in}}{\pgfqpoint{2.377142in}{4.224171in}}%
\pgfpathcurveto{\pgfqpoint{2.364119in}{4.224171in}}{\pgfqpoint{2.351628in}{4.218997in}}{\pgfqpoint{2.342420in}{4.209788in}}%
\pgfpathcurveto{\pgfqpoint{2.333211in}{4.200580in}}{\pgfqpoint{2.328037in}{4.188089in}}{\pgfqpoint{2.328037in}{4.175066in}}%
\pgfpathcurveto{\pgfqpoint{2.328037in}{4.162043in}}{\pgfqpoint{2.333211in}{4.149552in}}{\pgfqpoint{2.342420in}{4.140344in}}%
\pgfpathcurveto{\pgfqpoint{2.351628in}{4.131135in}}{\pgfqpoint{2.364119in}{4.125961in}}{\pgfqpoint{2.377142in}{4.125961in}}%
\pgfpathlineto{\pgfqpoint{2.377142in}{4.125961in}}%
\pgfpathclose%
\pgfusepath{stroke,fill}%
\end{pgfscope}%
\begin{pgfscope}%
\pgfpathrectangle{\pgfqpoint{0.454429in}{0.261491in}}{\pgfqpoint{6.040000in}{6.040000in}}%
\pgfusepath{clip}%
\pgfsetbuttcap%
\pgfsetroundjoin%
\definecolor{currentfill}{rgb}{0.121569,0.466667,0.705882}%
\pgfsetfillcolor{currentfill}%
\pgfsetlinewidth{1.003750pt}%
\definecolor{currentstroke}{rgb}{0.121569,0.466667,0.705882}%
\pgfsetstrokecolor{currentstroke}%
\pgfsetdash{}{0pt}%
\pgfpathmoveto{\pgfqpoint{3.156680in}{4.127922in}}%
\pgfpathcurveto{\pgfqpoint{3.169703in}{4.127922in}}{\pgfqpoint{3.182194in}{4.133096in}}{\pgfqpoint{3.191403in}{4.142305in}}%
\pgfpathcurveto{\pgfqpoint{3.200611in}{4.151513in}}{\pgfqpoint{3.205785in}{4.164004in}}{\pgfqpoint{3.205785in}{4.177027in}}%
\pgfpathcurveto{\pgfqpoint{3.205785in}{4.190050in}}{\pgfqpoint{3.200611in}{4.202541in}}{\pgfqpoint{3.191403in}{4.211749in}}%
\pgfpathcurveto{\pgfqpoint{3.182194in}{4.220958in}}{\pgfqpoint{3.169703in}{4.226132in}}{\pgfqpoint{3.156680in}{4.226132in}}%
\pgfpathcurveto{\pgfqpoint{3.143658in}{4.226132in}}{\pgfqpoint{3.131167in}{4.220958in}}{\pgfqpoint{3.121958in}{4.211749in}}%
\pgfpathcurveto{\pgfqpoint{3.112750in}{4.202541in}}{\pgfqpoint{3.107576in}{4.190050in}}{\pgfqpoint{3.107576in}{4.177027in}}%
\pgfpathcurveto{\pgfqpoint{3.107576in}{4.164004in}}{\pgfqpoint{3.112750in}{4.151513in}}{\pgfqpoint{3.121958in}{4.142305in}}%
\pgfpathcurveto{\pgfqpoint{3.131167in}{4.133096in}}{\pgfqpoint{3.143658in}{4.127922in}}{\pgfqpoint{3.156680in}{4.127922in}}%
\pgfpathlineto{\pgfqpoint{3.156680in}{4.127922in}}%
\pgfpathclose%
\pgfusepath{stroke,fill}%
\end{pgfscope}%
\begin{pgfscope}%
\pgfpathrectangle{\pgfqpoint{0.454429in}{0.261491in}}{\pgfqpoint{6.040000in}{6.040000in}}%
\pgfusepath{clip}%
\pgfsetbuttcap%
\pgfsetroundjoin%
\definecolor{currentfill}{rgb}{0.121569,0.466667,0.705882}%
\pgfsetfillcolor{currentfill}%
\pgfsetlinewidth{1.003750pt}%
\definecolor{currentstroke}{rgb}{0.121569,0.466667,0.705882}%
\pgfsetstrokecolor{currentstroke}%
\pgfsetdash{}{0pt}%
\pgfpathmoveto{\pgfqpoint{3.541612in}{4.138380in}}%
\pgfpathcurveto{\pgfqpoint{3.554635in}{4.138380in}}{\pgfqpoint{3.567126in}{4.143554in}}{\pgfqpoint{3.576334in}{4.152763in}}%
\pgfpathcurveto{\pgfqpoint{3.585543in}{4.161971in}}{\pgfqpoint{3.590717in}{4.174462in}}{\pgfqpoint{3.590717in}{4.187485in}}%
\pgfpathcurveto{\pgfqpoint{3.590717in}{4.200508in}}{\pgfqpoint{3.585543in}{4.212999in}}{\pgfqpoint{3.576334in}{4.222207in}}%
\pgfpathcurveto{\pgfqpoint{3.567126in}{4.231416in}}{\pgfqpoint{3.554635in}{4.236590in}}{\pgfqpoint{3.541612in}{4.236590in}}%
\pgfpathcurveto{\pgfqpoint{3.528590in}{4.236590in}}{\pgfqpoint{3.516098in}{4.231416in}}{\pgfqpoint{3.506890in}{4.222207in}}%
\pgfpathcurveto{\pgfqpoint{3.497682in}{4.212999in}}{\pgfqpoint{3.492508in}{4.200508in}}{\pgfqpoint{3.492508in}{4.187485in}}%
\pgfpathcurveto{\pgfqpoint{3.492508in}{4.174462in}}{\pgfqpoint{3.497682in}{4.161971in}}{\pgfqpoint{3.506890in}{4.152763in}}%
\pgfpathcurveto{\pgfqpoint{3.516098in}{4.143554in}}{\pgfqpoint{3.528590in}{4.138380in}}{\pgfqpoint{3.541612in}{4.138380in}}%
\pgfpathlineto{\pgfqpoint{3.541612in}{4.138380in}}%
\pgfpathclose%
\pgfusepath{stroke,fill}%
\end{pgfscope}%
\begin{pgfscope}%
\pgfpathrectangle{\pgfqpoint{0.454429in}{0.261491in}}{\pgfqpoint{6.040000in}{6.040000in}}%
\pgfusepath{clip}%
\pgfsetbuttcap%
\pgfsetroundjoin%
\definecolor{currentfill}{rgb}{0.121569,0.466667,0.705882}%
\pgfsetfillcolor{currentfill}%
\pgfsetlinewidth{1.003750pt}%
\definecolor{currentstroke}{rgb}{0.121569,0.466667,0.705882}%
\pgfsetstrokecolor{currentstroke}%
\pgfsetdash{}{0pt}%
\pgfpathmoveto{\pgfqpoint{3.924415in}{4.153756in}}%
\pgfpathcurveto{\pgfqpoint{3.937437in}{4.153756in}}{\pgfqpoint{3.949928in}{4.158930in}}{\pgfqpoint{3.959137in}{4.168139in}}%
\pgfpathcurveto{\pgfqpoint{3.968345in}{4.177347in}}{\pgfqpoint{3.973519in}{4.189838in}}{\pgfqpoint{3.973519in}{4.202861in}}%
\pgfpathcurveto{\pgfqpoint{3.973519in}{4.215884in}}{\pgfqpoint{3.968345in}{4.228375in}}{\pgfqpoint{3.959137in}{4.237583in}}%
\pgfpathcurveto{\pgfqpoint{3.949928in}{4.246792in}}{\pgfqpoint{3.937437in}{4.251966in}}{\pgfqpoint{3.924415in}{4.251966in}}%
\pgfpathcurveto{\pgfqpoint{3.911392in}{4.251966in}}{\pgfqpoint{3.898901in}{4.246792in}}{\pgfqpoint{3.889692in}{4.237583in}}%
\pgfpathcurveto{\pgfqpoint{3.880484in}{4.228375in}}{\pgfqpoint{3.875310in}{4.215884in}}{\pgfqpoint{3.875310in}{4.202861in}}%
\pgfpathcurveto{\pgfqpoint{3.875310in}{4.189838in}}{\pgfqpoint{3.880484in}{4.177347in}}{\pgfqpoint{3.889692in}{4.168139in}}%
\pgfpathcurveto{\pgfqpoint{3.898901in}{4.158930in}}{\pgfqpoint{3.911392in}{4.153756in}}{\pgfqpoint{3.924415in}{4.153756in}}%
\pgfpathlineto{\pgfqpoint{3.924415in}{4.153756in}}%
\pgfpathclose%
\pgfusepath{stroke,fill}%
\end{pgfscope}%
\begin{pgfscope}%
\pgfpathrectangle{\pgfqpoint{0.454429in}{0.261491in}}{\pgfqpoint{6.040000in}{6.040000in}}%
\pgfusepath{clip}%
\pgfsetbuttcap%
\pgfsetroundjoin%
\definecolor{currentfill}{rgb}{0.121569,0.466667,0.705882}%
\pgfsetfillcolor{currentfill}%
\pgfsetlinewidth{1.003750pt}%
\definecolor{currentstroke}{rgb}{0.121569,0.466667,0.705882}%
\pgfsetstrokecolor{currentstroke}%
\pgfsetdash{}{0pt}%
\pgfpathmoveto{\pgfqpoint{4.302652in}{4.177310in}}%
\pgfpathcurveto{\pgfqpoint{4.315675in}{4.177310in}}{\pgfqpoint{4.328166in}{4.182484in}}{\pgfqpoint{4.337375in}{4.191693in}}%
\pgfpathcurveto{\pgfqpoint{4.346583in}{4.200901in}}{\pgfqpoint{4.351757in}{4.213392in}}{\pgfqpoint{4.351757in}{4.226415in}}%
\pgfpathcurveto{\pgfqpoint{4.351757in}{4.239438in}}{\pgfqpoint{4.346583in}{4.251929in}}{\pgfqpoint{4.337375in}{4.261137in}}%
\pgfpathcurveto{\pgfqpoint{4.328166in}{4.270346in}}{\pgfqpoint{4.315675in}{4.275520in}}{\pgfqpoint{4.302652in}{4.275520in}}%
\pgfpathcurveto{\pgfqpoint{4.289630in}{4.275520in}}{\pgfqpoint{4.277139in}{4.270346in}}{\pgfqpoint{4.267930in}{4.261137in}}%
\pgfpathcurveto{\pgfqpoint{4.258722in}{4.251929in}}{\pgfqpoint{4.253548in}{4.239438in}}{\pgfqpoint{4.253548in}{4.226415in}}%
\pgfpathcurveto{\pgfqpoint{4.253548in}{4.213392in}}{\pgfqpoint{4.258722in}{4.200901in}}{\pgfqpoint{4.267930in}{4.191693in}}%
\pgfpathcurveto{\pgfqpoint{4.277139in}{4.182484in}}{\pgfqpoint{4.289630in}{4.177310in}}{\pgfqpoint{4.302652in}{4.177310in}}%
\pgfpathlineto{\pgfqpoint{4.302652in}{4.177310in}}%
\pgfpathclose%
\pgfusepath{stroke,fill}%
\end{pgfscope}%
\begin{pgfscope}%
\pgfpathrectangle{\pgfqpoint{0.454429in}{0.261491in}}{\pgfqpoint{6.040000in}{6.040000in}}%
\pgfusepath{clip}%
\pgfsetbuttcap%
\pgfsetroundjoin%
\definecolor{currentfill}{rgb}{0.121569,0.466667,0.705882}%
\pgfsetfillcolor{currentfill}%
\pgfsetlinewidth{1.003750pt}%
\definecolor{currentstroke}{rgb}{0.121569,0.466667,0.705882}%
\pgfsetstrokecolor{currentstroke}%
\pgfsetdash{}{0pt}%
\pgfpathmoveto{\pgfqpoint{4.673342in}{4.215950in}}%
\pgfpathcurveto{\pgfqpoint{4.686365in}{4.215950in}}{\pgfqpoint{4.698856in}{4.221124in}}{\pgfqpoint{4.708064in}{4.230333in}}%
\pgfpathcurveto{\pgfqpoint{4.717273in}{4.239541in}}{\pgfqpoint{4.722447in}{4.252032in}}{\pgfqpoint{4.722447in}{4.265055in}}%
\pgfpathcurveto{\pgfqpoint{4.722447in}{4.278078in}}{\pgfqpoint{4.717273in}{4.290569in}}{\pgfqpoint{4.708064in}{4.299777in}}%
\pgfpathcurveto{\pgfqpoint{4.698856in}{4.308985in}}{\pgfqpoint{4.686365in}{4.314159in}}{\pgfqpoint{4.673342in}{4.314159in}}%
\pgfpathcurveto{\pgfqpoint{4.660319in}{4.314159in}}{\pgfqpoint{4.647828in}{4.308985in}}{\pgfqpoint{4.638620in}{4.299777in}}%
\pgfpathcurveto{\pgfqpoint{4.629411in}{4.290569in}}{\pgfqpoint{4.624237in}{4.278078in}}{\pgfqpoint{4.624237in}{4.265055in}}%
\pgfpathcurveto{\pgfqpoint{4.624237in}{4.252032in}}{\pgfqpoint{4.629411in}{4.239541in}}{\pgfqpoint{4.638620in}{4.230333in}}%
\pgfpathcurveto{\pgfqpoint{4.647828in}{4.221124in}}{\pgfqpoint{4.660319in}{4.215950in}}{\pgfqpoint{4.673342in}{4.215950in}}%
\pgfpathlineto{\pgfqpoint{4.673342in}{4.215950in}}%
\pgfpathclose%
\pgfusepath{stroke,fill}%
\end{pgfscope}%
\begin{pgfscope}%
\pgfpathrectangle{\pgfqpoint{0.454429in}{0.261491in}}{\pgfqpoint{6.040000in}{6.040000in}}%
\pgfusepath{clip}%
\pgfsetbuttcap%
\pgfsetroundjoin%
\definecolor{currentfill}{rgb}{0.121569,0.466667,0.705882}%
\pgfsetfillcolor{currentfill}%
\pgfsetlinewidth{1.003750pt}%
\definecolor{currentstroke}{rgb}{0.121569,0.466667,0.705882}%
\pgfsetstrokecolor{currentstroke}%
\pgfsetdash{}{0pt}%
\pgfpathmoveto{\pgfqpoint{2.563435in}{4.398031in}}%
\pgfpathcurveto{\pgfqpoint{2.576458in}{4.398031in}}{\pgfqpoint{2.588949in}{4.403205in}}{\pgfqpoint{2.598157in}{4.412413in}}%
\pgfpathcurveto{\pgfqpoint{2.607366in}{4.421621in}}{\pgfqpoint{2.612540in}{4.434112in}}{\pgfqpoint{2.612540in}{4.447135in}}%
\pgfpathcurveto{\pgfqpoint{2.612540in}{4.460158in}}{\pgfqpoint{2.607366in}{4.472649in}}{\pgfqpoint{2.598157in}{4.481857in}}%
\pgfpathcurveto{\pgfqpoint{2.588949in}{4.491066in}}{\pgfqpoint{2.576458in}{4.496240in}}{\pgfqpoint{2.563435in}{4.496240in}}%
\pgfpathcurveto{\pgfqpoint{2.550413in}{4.496240in}}{\pgfqpoint{2.537921in}{4.491066in}}{\pgfqpoint{2.528713in}{4.481857in}}%
\pgfpathcurveto{\pgfqpoint{2.519505in}{4.472649in}}{\pgfqpoint{2.514331in}{4.460158in}}{\pgfqpoint{2.514331in}{4.447135in}}%
\pgfpathcurveto{\pgfqpoint{2.514331in}{4.434112in}}{\pgfqpoint{2.519505in}{4.421621in}}{\pgfqpoint{2.528713in}{4.412413in}}%
\pgfpathcurveto{\pgfqpoint{2.537921in}{4.403205in}}{\pgfqpoint{2.550413in}{4.398031in}}{\pgfqpoint{2.563435in}{4.398031in}}%
\pgfpathlineto{\pgfqpoint{2.563435in}{4.398031in}}%
\pgfpathclose%
\pgfusepath{stroke,fill}%
\end{pgfscope}%
\begin{pgfscope}%
\pgfpathrectangle{\pgfqpoint{0.454429in}{0.261491in}}{\pgfqpoint{6.040000in}{6.040000in}}%
\pgfusepath{clip}%
\pgfsetbuttcap%
\pgfsetroundjoin%
\definecolor{currentfill}{rgb}{0.121569,0.466667,0.705882}%
\pgfsetfillcolor{currentfill}%
\pgfsetlinewidth{1.003750pt}%
\definecolor{currentstroke}{rgb}{0.121569,0.466667,0.705882}%
\pgfsetstrokecolor{currentstroke}%
\pgfsetdash{}{0pt}%
\pgfpathmoveto{\pgfqpoint{2.957401in}{4.404989in}}%
\pgfpathcurveto{\pgfqpoint{2.970424in}{4.404989in}}{\pgfqpoint{2.982915in}{4.410163in}}{\pgfqpoint{2.992124in}{4.419372in}}%
\pgfpathcurveto{\pgfqpoint{3.001332in}{4.428580in}}{\pgfqpoint{3.006506in}{4.441071in}}{\pgfqpoint{3.006506in}{4.454094in}}%
\pgfpathcurveto{\pgfqpoint{3.006506in}{4.467117in}}{\pgfqpoint{3.001332in}{4.479608in}}{\pgfqpoint{2.992124in}{4.488816in}}%
\pgfpathcurveto{\pgfqpoint{2.982915in}{4.498025in}}{\pgfqpoint{2.970424in}{4.503199in}}{\pgfqpoint{2.957401in}{4.503199in}}%
\pgfpathcurveto{\pgfqpoint{2.944379in}{4.503199in}}{\pgfqpoint{2.931888in}{4.498025in}}{\pgfqpoint{2.922679in}{4.488816in}}%
\pgfpathcurveto{\pgfqpoint{2.913471in}{4.479608in}}{\pgfqpoint{2.908297in}{4.467117in}}{\pgfqpoint{2.908297in}{4.454094in}}%
\pgfpathcurveto{\pgfqpoint{2.908297in}{4.441071in}}{\pgfqpoint{2.913471in}{4.428580in}}{\pgfqpoint{2.922679in}{4.419372in}}%
\pgfpathcurveto{\pgfqpoint{2.931888in}{4.410163in}}{\pgfqpoint{2.944379in}{4.404989in}}{\pgfqpoint{2.957401in}{4.404989in}}%
\pgfpathlineto{\pgfqpoint{2.957401in}{4.404989in}}%
\pgfpathclose%
\pgfusepath{stroke,fill}%
\end{pgfscope}%
\begin{pgfscope}%
\pgfpathrectangle{\pgfqpoint{0.454429in}{0.261491in}}{\pgfqpoint{6.040000in}{6.040000in}}%
\pgfusepath{clip}%
\pgfsetbuttcap%
\pgfsetroundjoin%
\definecolor{currentfill}{rgb}{0.121569,0.466667,0.705882}%
\pgfsetfillcolor{currentfill}%
\pgfsetlinewidth{1.003750pt}%
\definecolor{currentstroke}{rgb}{0.121569,0.466667,0.705882}%
\pgfsetstrokecolor{currentstroke}%
\pgfsetdash{}{0pt}%
\pgfpathmoveto{\pgfqpoint{3.344216in}{4.417272in}}%
\pgfpathcurveto{\pgfqpoint{3.357238in}{4.417272in}}{\pgfqpoint{3.369729in}{4.422446in}}{\pgfqpoint{3.378938in}{4.431655in}}%
\pgfpathcurveto{\pgfqpoint{3.388146in}{4.440863in}}{\pgfqpoint{3.393320in}{4.453354in}}{\pgfqpoint{3.393320in}{4.466377in}}%
\pgfpathcurveto{\pgfqpoint{3.393320in}{4.479400in}}{\pgfqpoint{3.388146in}{4.491891in}}{\pgfqpoint{3.378938in}{4.501099in}}%
\pgfpathcurveto{\pgfqpoint{3.369729in}{4.510308in}}{\pgfqpoint{3.357238in}{4.515482in}}{\pgfqpoint{3.344216in}{4.515482in}}%
\pgfpathcurveto{\pgfqpoint{3.331193in}{4.515482in}}{\pgfqpoint{3.318702in}{4.510308in}}{\pgfqpoint{3.309493in}{4.501099in}}%
\pgfpathcurveto{\pgfqpoint{3.300285in}{4.491891in}}{\pgfqpoint{3.295111in}{4.479400in}}{\pgfqpoint{3.295111in}{4.466377in}}%
\pgfpathcurveto{\pgfqpoint{3.295111in}{4.453354in}}{\pgfqpoint{3.300285in}{4.440863in}}{\pgfqpoint{3.309493in}{4.431655in}}%
\pgfpathcurveto{\pgfqpoint{3.318702in}{4.422446in}}{\pgfqpoint{3.331193in}{4.417272in}}{\pgfqpoint{3.344216in}{4.417272in}}%
\pgfpathlineto{\pgfqpoint{3.344216in}{4.417272in}}%
\pgfpathclose%
\pgfusepath{stroke,fill}%
\end{pgfscope}%
\begin{pgfscope}%
\pgfpathrectangle{\pgfqpoint{0.454429in}{0.261491in}}{\pgfqpoint{6.040000in}{6.040000in}}%
\pgfusepath{clip}%
\pgfsetbuttcap%
\pgfsetroundjoin%
\definecolor{currentfill}{rgb}{0.121569,0.466667,0.705882}%
\pgfsetfillcolor{currentfill}%
\pgfsetlinewidth{1.003750pt}%
\definecolor{currentstroke}{rgb}{0.121569,0.466667,0.705882}%
\pgfsetstrokecolor{currentstroke}%
\pgfsetdash{}{0pt}%
\pgfpathmoveto{\pgfqpoint{3.729204in}{4.431142in}}%
\pgfpathcurveto{\pgfqpoint{3.742227in}{4.431142in}}{\pgfqpoint{3.754718in}{4.436316in}}{\pgfqpoint{3.763927in}{4.445524in}}%
\pgfpathcurveto{\pgfqpoint{3.773135in}{4.454733in}}{\pgfqpoint{3.778309in}{4.467224in}}{\pgfqpoint{3.778309in}{4.480246in}}%
\pgfpathcurveto{\pgfqpoint{3.778309in}{4.493269in}}{\pgfqpoint{3.773135in}{4.505760in}}{\pgfqpoint{3.763927in}{4.514969in}}%
\pgfpathcurveto{\pgfqpoint{3.754718in}{4.524177in}}{\pgfqpoint{3.742227in}{4.529351in}}{\pgfqpoint{3.729204in}{4.529351in}}%
\pgfpathcurveto{\pgfqpoint{3.716182in}{4.529351in}}{\pgfqpoint{3.703691in}{4.524177in}}{\pgfqpoint{3.694482in}{4.514969in}}%
\pgfpathcurveto{\pgfqpoint{3.685274in}{4.505760in}}{\pgfqpoint{3.680100in}{4.493269in}}{\pgfqpoint{3.680100in}{4.480246in}}%
\pgfpathcurveto{\pgfqpoint{3.680100in}{4.467224in}}{\pgfqpoint{3.685274in}{4.454733in}}{\pgfqpoint{3.694482in}{4.445524in}}%
\pgfpathcurveto{\pgfqpoint{3.703691in}{4.436316in}}{\pgfqpoint{3.716182in}{4.431142in}}{\pgfqpoint{3.729204in}{4.431142in}}%
\pgfpathlineto{\pgfqpoint{3.729204in}{4.431142in}}%
\pgfpathclose%
\pgfusepath{stroke,fill}%
\end{pgfscope}%
\begin{pgfscope}%
\pgfpathrectangle{\pgfqpoint{0.454429in}{0.261491in}}{\pgfqpoint{6.040000in}{6.040000in}}%
\pgfusepath{clip}%
\pgfsetbuttcap%
\pgfsetroundjoin%
\definecolor{currentfill}{rgb}{0.121569,0.466667,0.705882}%
\pgfsetfillcolor{currentfill}%
\pgfsetlinewidth{1.003750pt}%
\definecolor{currentstroke}{rgb}{0.121569,0.466667,0.705882}%
\pgfsetstrokecolor{currentstroke}%
\pgfsetdash{}{0pt}%
\pgfpathmoveto{\pgfqpoint{4.111123in}{4.454319in}}%
\pgfpathcurveto{\pgfqpoint{4.124145in}{4.454319in}}{\pgfqpoint{4.136637in}{4.459493in}}{\pgfqpoint{4.145845in}{4.468701in}}%
\pgfpathcurveto{\pgfqpoint{4.155053in}{4.477910in}}{\pgfqpoint{4.160227in}{4.490401in}}{\pgfqpoint{4.160227in}{4.503424in}}%
\pgfpathcurveto{\pgfqpoint{4.160227in}{4.516446in}}{\pgfqpoint{4.155053in}{4.528937in}}{\pgfqpoint{4.145845in}{4.538146in}}%
\pgfpathcurveto{\pgfqpoint{4.136637in}{4.547354in}}{\pgfqpoint{4.124145in}{4.552528in}}{\pgfqpoint{4.111123in}{4.552528in}}%
\pgfpathcurveto{\pgfqpoint{4.098100in}{4.552528in}}{\pgfqpoint{4.085609in}{4.547354in}}{\pgfqpoint{4.076401in}{4.538146in}}%
\pgfpathcurveto{\pgfqpoint{4.067192in}{4.528937in}}{\pgfqpoint{4.062018in}{4.516446in}}{\pgfqpoint{4.062018in}{4.503424in}}%
\pgfpathcurveto{\pgfqpoint{4.062018in}{4.490401in}}{\pgfqpoint{4.067192in}{4.477910in}}{\pgfqpoint{4.076401in}{4.468701in}}%
\pgfpathcurveto{\pgfqpoint{4.085609in}{4.459493in}}{\pgfqpoint{4.098100in}{4.454319in}}{\pgfqpoint{4.111123in}{4.454319in}}%
\pgfpathlineto{\pgfqpoint{4.111123in}{4.454319in}}%
\pgfpathclose%
\pgfusepath{stroke,fill}%
\end{pgfscope}%
\begin{pgfscope}%
\pgfpathrectangle{\pgfqpoint{0.454429in}{0.261491in}}{\pgfqpoint{6.040000in}{6.040000in}}%
\pgfusepath{clip}%
\pgfsetbuttcap%
\pgfsetroundjoin%
\definecolor{currentfill}{rgb}{0.121569,0.466667,0.705882}%
\pgfsetfillcolor{currentfill}%
\pgfsetlinewidth{1.003750pt}%
\definecolor{currentstroke}{rgb}{0.121569,0.466667,0.705882}%
\pgfsetstrokecolor{currentstroke}%
\pgfsetdash{}{0pt}%
\pgfpathmoveto{\pgfqpoint{4.484696in}{4.491455in}}%
\pgfpathcurveto{\pgfqpoint{4.497719in}{4.491455in}}{\pgfqpoint{4.510210in}{4.496629in}}{\pgfqpoint{4.519418in}{4.505838in}}%
\pgfpathcurveto{\pgfqpoint{4.528627in}{4.515046in}}{\pgfqpoint{4.533801in}{4.527537in}}{\pgfqpoint{4.533801in}{4.540560in}}%
\pgfpathcurveto{\pgfqpoint{4.533801in}{4.553583in}}{\pgfqpoint{4.528627in}{4.566074in}}{\pgfqpoint{4.519418in}{4.575282in}}%
\pgfpathcurveto{\pgfqpoint{4.510210in}{4.584490in}}{\pgfqpoint{4.497719in}{4.589664in}}{\pgfqpoint{4.484696in}{4.589664in}}%
\pgfpathcurveto{\pgfqpoint{4.471674in}{4.589664in}}{\pgfqpoint{4.459182in}{4.584490in}}{\pgfqpoint{4.449974in}{4.575282in}}%
\pgfpathcurveto{\pgfqpoint{4.440766in}{4.566074in}}{\pgfqpoint{4.435592in}{4.553583in}}{\pgfqpoint{4.435592in}{4.540560in}}%
\pgfpathcurveto{\pgfqpoint{4.435592in}{4.527537in}}{\pgfqpoint{4.440766in}{4.515046in}}{\pgfqpoint{4.449974in}{4.505838in}}%
\pgfpathcurveto{\pgfqpoint{4.459182in}{4.496629in}}{\pgfqpoint{4.471674in}{4.491455in}}{\pgfqpoint{4.484696in}{4.491455in}}%
\pgfpathlineto{\pgfqpoint{4.484696in}{4.491455in}}%
\pgfpathclose%
\pgfusepath{stroke,fill}%
\end{pgfscope}%
\begin{pgfscope}%
\pgfpathrectangle{\pgfqpoint{0.454429in}{0.261491in}}{\pgfqpoint{6.040000in}{6.040000in}}%
\pgfusepath{clip}%
\pgfsetbuttcap%
\pgfsetroundjoin%
\definecolor{currentfill}{rgb}{0.121569,0.466667,0.705882}%
\pgfsetfillcolor{currentfill}%
\pgfsetlinewidth{1.003750pt}%
\definecolor{currentstroke}{rgb}{0.121569,0.466667,0.705882}%
\pgfsetstrokecolor{currentstroke}%
\pgfsetdash{}{0pt}%
\pgfpathmoveto{\pgfqpoint{2.751825in}{4.673162in}}%
\pgfpathcurveto{\pgfqpoint{2.764848in}{4.673162in}}{\pgfqpoint{2.777339in}{4.678336in}}{\pgfqpoint{2.786547in}{4.687544in}}%
\pgfpathcurveto{\pgfqpoint{2.795756in}{4.696753in}}{\pgfqpoint{2.800930in}{4.709244in}}{\pgfqpoint{2.800930in}{4.722266in}}%
\pgfpathcurveto{\pgfqpoint{2.800930in}{4.735289in}}{\pgfqpoint{2.795756in}{4.747780in}}{\pgfqpoint{2.786547in}{4.756989in}}%
\pgfpathcurveto{\pgfqpoint{2.777339in}{4.766197in}}{\pgfqpoint{2.764848in}{4.771371in}}{\pgfqpoint{2.751825in}{4.771371in}}%
\pgfpathcurveto{\pgfqpoint{2.738802in}{4.771371in}}{\pgfqpoint{2.726311in}{4.766197in}}{\pgfqpoint{2.717103in}{4.756989in}}%
\pgfpathcurveto{\pgfqpoint{2.707894in}{4.747780in}}{\pgfqpoint{2.702720in}{4.735289in}}{\pgfqpoint{2.702720in}{4.722266in}}%
\pgfpathcurveto{\pgfqpoint{2.702720in}{4.709244in}}{\pgfqpoint{2.707894in}{4.696753in}}{\pgfqpoint{2.717103in}{4.687544in}}%
\pgfpathcurveto{\pgfqpoint{2.726311in}{4.678336in}}{\pgfqpoint{2.738802in}{4.673162in}}{\pgfqpoint{2.751825in}{4.673162in}}%
\pgfpathlineto{\pgfqpoint{2.751825in}{4.673162in}}%
\pgfpathclose%
\pgfusepath{stroke,fill}%
\end{pgfscope}%
\begin{pgfscope}%
\pgfpathrectangle{\pgfqpoint{0.454429in}{0.261491in}}{\pgfqpoint{6.040000in}{6.040000in}}%
\pgfusepath{clip}%
\pgfsetbuttcap%
\pgfsetroundjoin%
\definecolor{currentfill}{rgb}{0.121569,0.466667,0.705882}%
\pgfsetfillcolor{currentfill}%
\pgfsetlinewidth{1.003750pt}%
\definecolor{currentstroke}{rgb}{0.121569,0.466667,0.705882}%
\pgfsetstrokecolor{currentstroke}%
\pgfsetdash{}{0pt}%
\pgfpathmoveto{\pgfqpoint{3.144478in}{4.693420in}}%
\pgfpathcurveto{\pgfqpoint{3.157500in}{4.693420in}}{\pgfqpoint{3.169991in}{4.698594in}}{\pgfqpoint{3.179200in}{4.707803in}}%
\pgfpathcurveto{\pgfqpoint{3.188408in}{4.717011in}}{\pgfqpoint{3.193582in}{4.729502in}}{\pgfqpoint{3.193582in}{4.742525in}}%
\pgfpathcurveto{\pgfqpoint{3.193582in}{4.755548in}}{\pgfqpoint{3.188408in}{4.768039in}}{\pgfqpoint{3.179200in}{4.777247in}}%
\pgfpathcurveto{\pgfqpoint{3.169991in}{4.786456in}}{\pgfqpoint{3.157500in}{4.791630in}}{\pgfqpoint{3.144478in}{4.791630in}}%
\pgfpathcurveto{\pgfqpoint{3.131455in}{4.791630in}}{\pgfqpoint{3.118964in}{4.786456in}}{\pgfqpoint{3.109755in}{4.777247in}}%
\pgfpathcurveto{\pgfqpoint{3.100547in}{4.768039in}}{\pgfqpoint{3.095373in}{4.755548in}}{\pgfqpoint{3.095373in}{4.742525in}}%
\pgfpathcurveto{\pgfqpoint{3.095373in}{4.729502in}}{\pgfqpoint{3.100547in}{4.717011in}}{\pgfqpoint{3.109755in}{4.707803in}}%
\pgfpathcurveto{\pgfqpoint{3.118964in}{4.698594in}}{\pgfqpoint{3.131455in}{4.693420in}}{\pgfqpoint{3.144478in}{4.693420in}}%
\pgfpathlineto{\pgfqpoint{3.144478in}{4.693420in}}%
\pgfpathclose%
\pgfusepath{stroke,fill}%
\end{pgfscope}%
\begin{pgfscope}%
\pgfpathrectangle{\pgfqpoint{0.454429in}{0.261491in}}{\pgfqpoint{6.040000in}{6.040000in}}%
\pgfusepath{clip}%
\pgfsetbuttcap%
\pgfsetroundjoin%
\definecolor{currentfill}{rgb}{0.121569,0.466667,0.705882}%
\pgfsetfillcolor{currentfill}%
\pgfsetlinewidth{1.003750pt}%
\definecolor{currentstroke}{rgb}{0.121569,0.466667,0.705882}%
\pgfsetstrokecolor{currentstroke}%
\pgfsetdash{}{0pt}%
\pgfpathmoveto{\pgfqpoint{3.534142in}{4.707301in}}%
\pgfpathcurveto{\pgfqpoint{3.547164in}{4.707301in}}{\pgfqpoint{3.559655in}{4.712475in}}{\pgfqpoint{3.568864in}{4.721683in}}%
\pgfpathcurveto{\pgfqpoint{3.578072in}{4.730892in}}{\pgfqpoint{3.583246in}{4.743383in}}{\pgfqpoint{3.583246in}{4.756405in}}%
\pgfpathcurveto{\pgfqpoint{3.583246in}{4.769428in}}{\pgfqpoint{3.578072in}{4.781919in}}{\pgfqpoint{3.568864in}{4.791128in}}%
\pgfpathcurveto{\pgfqpoint{3.559655in}{4.800336in}}{\pgfqpoint{3.547164in}{4.805510in}}{\pgfqpoint{3.534142in}{4.805510in}}%
\pgfpathcurveto{\pgfqpoint{3.521119in}{4.805510in}}{\pgfqpoint{3.508628in}{4.800336in}}{\pgfqpoint{3.499419in}{4.791128in}}%
\pgfpathcurveto{\pgfqpoint{3.490211in}{4.781919in}}{\pgfqpoint{3.485037in}{4.769428in}}{\pgfqpoint{3.485037in}{4.756405in}}%
\pgfpathcurveto{\pgfqpoint{3.485037in}{4.743383in}}{\pgfqpoint{3.490211in}{4.730892in}}{\pgfqpoint{3.499419in}{4.721683in}}%
\pgfpathcurveto{\pgfqpoint{3.508628in}{4.712475in}}{\pgfqpoint{3.521119in}{4.707301in}}{\pgfqpoint{3.534142in}{4.707301in}}%
\pgfpathlineto{\pgfqpoint{3.534142in}{4.707301in}}%
\pgfpathclose%
\pgfusepath{stroke,fill}%
\end{pgfscope}%
\begin{pgfscope}%
\pgfpathrectangle{\pgfqpoint{0.454429in}{0.261491in}}{\pgfqpoint{6.040000in}{6.040000in}}%
\pgfusepath{clip}%
\pgfsetbuttcap%
\pgfsetroundjoin%
\definecolor{currentfill}{rgb}{0.121569,0.466667,0.705882}%
\pgfsetfillcolor{currentfill}%
\pgfsetlinewidth{1.003750pt}%
\definecolor{currentstroke}{rgb}{0.121569,0.466667,0.705882}%
\pgfsetstrokecolor{currentstroke}%
\pgfsetdash{}{0pt}%
\pgfpathmoveto{\pgfqpoint{3.918050in}{4.727538in}}%
\pgfpathcurveto{\pgfqpoint{3.931073in}{4.727538in}}{\pgfqpoint{3.943564in}{4.732712in}}{\pgfqpoint{3.952772in}{4.741921in}}%
\pgfpathcurveto{\pgfqpoint{3.961981in}{4.751129in}}{\pgfqpoint{3.967155in}{4.763620in}}{\pgfqpoint{3.967155in}{4.776643in}}%
\pgfpathcurveto{\pgfqpoint{3.967155in}{4.789666in}}{\pgfqpoint{3.961981in}{4.802157in}}{\pgfqpoint{3.952772in}{4.811365in}}%
\pgfpathcurveto{\pgfqpoint{3.943564in}{4.820574in}}{\pgfqpoint{3.931073in}{4.825748in}}{\pgfqpoint{3.918050in}{4.825748in}}%
\pgfpathcurveto{\pgfqpoint{3.905028in}{4.825748in}}{\pgfqpoint{3.892536in}{4.820574in}}{\pgfqpoint{3.883328in}{4.811365in}}%
\pgfpathcurveto{\pgfqpoint{3.874120in}{4.802157in}}{\pgfqpoint{3.868946in}{4.789666in}}{\pgfqpoint{3.868946in}{4.776643in}}%
\pgfpathcurveto{\pgfqpoint{3.868946in}{4.763620in}}{\pgfqpoint{3.874120in}{4.751129in}}{\pgfqpoint{3.883328in}{4.741921in}}%
\pgfpathcurveto{\pgfqpoint{3.892536in}{4.732712in}}{\pgfqpoint{3.905028in}{4.727538in}}{\pgfqpoint{3.918050in}{4.727538in}}%
\pgfpathlineto{\pgfqpoint{3.918050in}{4.727538in}}%
\pgfpathclose%
\pgfusepath{stroke,fill}%
\end{pgfscope}%
\begin{pgfscope}%
\pgfpathrectangle{\pgfqpoint{0.454429in}{0.261491in}}{\pgfqpoint{6.040000in}{6.040000in}}%
\pgfusepath{clip}%
\pgfsetbuttcap%
\pgfsetroundjoin%
\definecolor{currentfill}{rgb}{0.121569,0.466667,0.705882}%
\pgfsetfillcolor{currentfill}%
\pgfsetlinewidth{1.003750pt}%
\definecolor{currentstroke}{rgb}{0.121569,0.466667,0.705882}%
\pgfsetstrokecolor{currentstroke}%
\pgfsetdash{}{0pt}%
\pgfpathmoveto{\pgfqpoint{4.297592in}{4.764709in}}%
\pgfpathcurveto{\pgfqpoint{4.310614in}{4.764709in}}{\pgfqpoint{4.323105in}{4.769883in}}{\pgfqpoint{4.332314in}{4.779092in}}%
\pgfpathcurveto{\pgfqpoint{4.341522in}{4.788300in}}{\pgfqpoint{4.346696in}{4.800791in}}{\pgfqpoint{4.346696in}{4.813814in}}%
\pgfpathcurveto{\pgfqpoint{4.346696in}{4.826837in}}{\pgfqpoint{4.341522in}{4.839328in}}{\pgfqpoint{4.332314in}{4.848536in}}%
\pgfpathcurveto{\pgfqpoint{4.323105in}{4.857745in}}{\pgfqpoint{4.310614in}{4.862919in}}{\pgfqpoint{4.297592in}{4.862919in}}%
\pgfpathcurveto{\pgfqpoint{4.284569in}{4.862919in}}{\pgfqpoint{4.272078in}{4.857745in}}{\pgfqpoint{4.262869in}{4.848536in}}%
\pgfpathcurveto{\pgfqpoint{4.253661in}{4.839328in}}{\pgfqpoint{4.248487in}{4.826837in}}{\pgfqpoint{4.248487in}{4.813814in}}%
\pgfpathcurveto{\pgfqpoint{4.248487in}{4.800791in}}{\pgfqpoint{4.253661in}{4.788300in}}{\pgfqpoint{4.262869in}{4.779092in}}%
\pgfpathcurveto{\pgfqpoint{4.272078in}{4.769883in}}{\pgfqpoint{4.284569in}{4.764709in}}{\pgfqpoint{4.297592in}{4.764709in}}%
\pgfpathlineto{\pgfqpoint{4.297592in}{4.764709in}}%
\pgfpathclose%
\pgfusepath{stroke,fill}%
\end{pgfscope}%
\begin{pgfscope}%
\pgfpathrectangle{\pgfqpoint{0.454429in}{0.261491in}}{\pgfqpoint{6.040000in}{6.040000in}}%
\pgfusepath{clip}%
\pgfsetbuttcap%
\pgfsetroundjoin%
\definecolor{currentfill}{rgb}{0.121569,0.466667,0.705882}%
\pgfsetfillcolor{currentfill}%
\pgfsetlinewidth{1.003750pt}%
\definecolor{currentstroke}{rgb}{0.121569,0.466667,0.705882}%
\pgfsetstrokecolor{currentstroke}%
\pgfsetdash{}{0pt}%
\pgfpathmoveto{\pgfqpoint{2.953127in}{4.967151in}}%
\pgfpathcurveto{\pgfqpoint{2.966150in}{4.967151in}}{\pgfqpoint{2.978641in}{4.972325in}}{\pgfqpoint{2.987850in}{4.981533in}}%
\pgfpathcurveto{\pgfqpoint{2.997058in}{4.990742in}}{\pgfqpoint{3.002232in}{5.003233in}}{\pgfqpoint{3.002232in}{5.016256in}}%
\pgfpathcurveto{\pgfqpoint{3.002232in}{5.029278in}}{\pgfqpoint{2.997058in}{5.041769in}}{\pgfqpoint{2.987850in}{5.050978in}}%
\pgfpathcurveto{\pgfqpoint{2.978641in}{5.060186in}}{\pgfqpoint{2.966150in}{5.065360in}}{\pgfqpoint{2.953127in}{5.065360in}}%
\pgfpathcurveto{\pgfqpoint{2.940105in}{5.065360in}}{\pgfqpoint{2.927614in}{5.060186in}}{\pgfqpoint{2.918405in}{5.050978in}}%
\pgfpathcurveto{\pgfqpoint{2.909197in}{5.041769in}}{\pgfqpoint{2.904023in}{5.029278in}}{\pgfqpoint{2.904023in}{5.016256in}}%
\pgfpathcurveto{\pgfqpoint{2.904023in}{5.003233in}}{\pgfqpoint{2.909197in}{4.990742in}}{\pgfqpoint{2.918405in}{4.981533in}}%
\pgfpathcurveto{\pgfqpoint{2.927614in}{4.972325in}}{\pgfqpoint{2.940105in}{4.967151in}}{\pgfqpoint{2.953127in}{4.967151in}}%
\pgfpathlineto{\pgfqpoint{2.953127in}{4.967151in}}%
\pgfpathclose%
\pgfusepath{stroke,fill}%
\end{pgfscope}%
\begin{pgfscope}%
\pgfpathrectangle{\pgfqpoint{0.454429in}{0.261491in}}{\pgfqpoint{6.040000in}{6.040000in}}%
\pgfusepath{clip}%
\pgfsetbuttcap%
\pgfsetroundjoin%
\definecolor{currentfill}{rgb}{0.121569,0.466667,0.705882}%
\pgfsetfillcolor{currentfill}%
\pgfsetlinewidth{1.003750pt}%
\definecolor{currentstroke}{rgb}{0.121569,0.466667,0.705882}%
\pgfsetstrokecolor{currentstroke}%
\pgfsetdash{}{0pt}%
\pgfpathmoveto{\pgfqpoint{3.344864in}{4.981730in}}%
\pgfpathcurveto{\pgfqpoint{3.357887in}{4.981730in}}{\pgfqpoint{3.370378in}{4.986904in}}{\pgfqpoint{3.379587in}{4.996112in}}%
\pgfpathcurveto{\pgfqpoint{3.388795in}{5.005321in}}{\pgfqpoint{3.393969in}{5.017812in}}{\pgfqpoint{3.393969in}{5.030834in}}%
\pgfpathcurveto{\pgfqpoint{3.393969in}{5.043857in}}{\pgfqpoint{3.388795in}{5.056348in}}{\pgfqpoint{3.379587in}{5.065557in}}%
\pgfpathcurveto{\pgfqpoint{3.370378in}{5.074765in}}{\pgfqpoint{3.357887in}{5.079939in}}{\pgfqpoint{3.344864in}{5.079939in}}%
\pgfpathcurveto{\pgfqpoint{3.331842in}{5.079939in}}{\pgfqpoint{3.319351in}{5.074765in}}{\pgfqpoint{3.310142in}{5.065557in}}%
\pgfpathcurveto{\pgfqpoint{3.300934in}{5.056348in}}{\pgfqpoint{3.295760in}{5.043857in}}{\pgfqpoint{3.295760in}{5.030834in}}%
\pgfpathcurveto{\pgfqpoint{3.295760in}{5.017812in}}{\pgfqpoint{3.300934in}{5.005321in}}{\pgfqpoint{3.310142in}{4.996112in}}%
\pgfpathcurveto{\pgfqpoint{3.319351in}{4.986904in}}{\pgfqpoint{3.331842in}{4.981730in}}{\pgfqpoint{3.344864in}{4.981730in}}%
\pgfpathlineto{\pgfqpoint{3.344864in}{4.981730in}}%
\pgfpathclose%
\pgfusepath{stroke,fill}%
\end{pgfscope}%
\begin{pgfscope}%
\pgfpathrectangle{\pgfqpoint{0.454429in}{0.261491in}}{\pgfqpoint{6.040000in}{6.040000in}}%
\pgfusepath{clip}%
\pgfsetbuttcap%
\pgfsetroundjoin%
\definecolor{currentfill}{rgb}{0.121569,0.466667,0.705882}%
\pgfsetfillcolor{currentfill}%
\pgfsetlinewidth{1.003750pt}%
\definecolor{currentstroke}{rgb}{0.121569,0.466667,0.705882}%
\pgfsetstrokecolor{currentstroke}%
\pgfsetdash{}{0pt}%
\pgfpathmoveto{\pgfqpoint{3.731397in}{5.002130in}}%
\pgfpathcurveto{\pgfqpoint{3.744420in}{5.002130in}}{\pgfqpoint{3.756911in}{5.007304in}}{\pgfqpoint{3.766119in}{5.016513in}}%
\pgfpathcurveto{\pgfqpoint{3.775328in}{5.025721in}}{\pgfqpoint{3.780502in}{5.038212in}}{\pgfqpoint{3.780502in}{5.051235in}}%
\pgfpathcurveto{\pgfqpoint{3.780502in}{5.064258in}}{\pgfqpoint{3.775328in}{5.076749in}}{\pgfqpoint{3.766119in}{5.085957in}}%
\pgfpathcurveto{\pgfqpoint{3.756911in}{5.095166in}}{\pgfqpoint{3.744420in}{5.100339in}}{\pgfqpoint{3.731397in}{5.100339in}}%
\pgfpathcurveto{\pgfqpoint{3.718374in}{5.100339in}}{\pgfqpoint{3.705883in}{5.095166in}}{\pgfqpoint{3.696675in}{5.085957in}}%
\pgfpathcurveto{\pgfqpoint{3.687466in}{5.076749in}}{\pgfqpoint{3.682292in}{5.064258in}}{\pgfqpoint{3.682292in}{5.051235in}}%
\pgfpathcurveto{\pgfqpoint{3.682292in}{5.038212in}}{\pgfqpoint{3.687466in}{5.025721in}}{\pgfqpoint{3.696675in}{5.016513in}}%
\pgfpathcurveto{\pgfqpoint{3.705883in}{5.007304in}}{\pgfqpoint{3.718374in}{5.002130in}}{\pgfqpoint{3.731397in}{5.002130in}}%
\pgfpathlineto{\pgfqpoint{3.731397in}{5.002130in}}%
\pgfpathclose%
\pgfusepath{stroke,fill}%
\end{pgfscope}%
\begin{pgfscope}%
\pgfpathrectangle{\pgfqpoint{0.454429in}{0.261491in}}{\pgfqpoint{6.040000in}{6.040000in}}%
\pgfusepath{clip}%
\pgfsetbuttcap%
\pgfsetroundjoin%
\definecolor{currentfill}{rgb}{0.121569,0.466667,0.705882}%
\pgfsetfillcolor{currentfill}%
\pgfsetlinewidth{1.003750pt}%
\definecolor{currentstroke}{rgb}{0.121569,0.466667,0.705882}%
\pgfsetstrokecolor{currentstroke}%
\pgfsetdash{}{0pt}%
\pgfpathmoveto{\pgfqpoint{4.119239in}{5.025182in}}%
\pgfpathcurveto{\pgfqpoint{4.132262in}{5.025182in}}{\pgfqpoint{4.144753in}{5.030356in}}{\pgfqpoint{4.153961in}{5.039564in}}%
\pgfpathcurveto{\pgfqpoint{4.163170in}{5.048772in}}{\pgfqpoint{4.168344in}{5.061263in}}{\pgfqpoint{4.168344in}{5.074286in}}%
\pgfpathcurveto{\pgfqpoint{4.168344in}{5.087309in}}{\pgfqpoint{4.163170in}{5.099800in}}{\pgfqpoint{4.153961in}{5.109008in}}%
\pgfpathcurveto{\pgfqpoint{4.144753in}{5.118217in}}{\pgfqpoint{4.132262in}{5.123391in}}{\pgfqpoint{4.119239in}{5.123391in}}%
\pgfpathcurveto{\pgfqpoint{4.106217in}{5.123391in}}{\pgfqpoint{4.093725in}{5.118217in}}{\pgfqpoint{4.084517in}{5.109008in}}%
\pgfpathcurveto{\pgfqpoint{4.075309in}{5.099800in}}{\pgfqpoint{4.070135in}{5.087309in}}{\pgfqpoint{4.070135in}{5.074286in}}%
\pgfpathcurveto{\pgfqpoint{4.070135in}{5.061263in}}{\pgfqpoint{4.075309in}{5.048772in}}{\pgfqpoint{4.084517in}{5.039564in}}%
\pgfpathcurveto{\pgfqpoint{4.093725in}{5.030356in}}{\pgfqpoint{4.106217in}{5.025182in}}{\pgfqpoint{4.119239in}{5.025182in}}%
\pgfpathlineto{\pgfqpoint{4.119239in}{5.025182in}}%
\pgfpathclose%
\pgfusepath{stroke,fill}%
\end{pgfscope}%
\begin{pgfscope}%
\pgfpathrectangle{\pgfqpoint{0.454429in}{0.261491in}}{\pgfqpoint{6.040000in}{6.040000in}}%
\pgfusepath{clip}%
\pgfsetbuttcap%
\pgfsetroundjoin%
\definecolor{currentfill}{rgb}{0.121569,0.466667,0.705882}%
\pgfsetfillcolor{currentfill}%
\pgfsetlinewidth{1.003750pt}%
\definecolor{currentstroke}{rgb}{0.121569,0.466667,0.705882}%
\pgfsetstrokecolor{currentstroke}%
\pgfsetdash{}{0pt}%
\pgfpathmoveto{\pgfqpoint{3.144993in}{5.247359in}}%
\pgfpathcurveto{\pgfqpoint{3.158016in}{5.247359in}}{\pgfqpoint{3.170507in}{5.252533in}}{\pgfqpoint{3.179716in}{5.261741in}}%
\pgfpathcurveto{\pgfqpoint{3.188924in}{5.270950in}}{\pgfqpoint{3.194098in}{5.283441in}}{\pgfqpoint{3.194098in}{5.296463in}}%
\pgfpathcurveto{\pgfqpoint{3.194098in}{5.309486in}}{\pgfqpoint{3.188924in}{5.321977in}}{\pgfqpoint{3.179716in}{5.331186in}}%
\pgfpathcurveto{\pgfqpoint{3.170507in}{5.340394in}}{\pgfqpoint{3.158016in}{5.345568in}}{\pgfqpoint{3.144993in}{5.345568in}}%
\pgfpathcurveto{\pgfqpoint{3.131971in}{5.345568in}}{\pgfqpoint{3.119480in}{5.340394in}}{\pgfqpoint{3.110271in}{5.331186in}}%
\pgfpathcurveto{\pgfqpoint{3.101063in}{5.321977in}}{\pgfqpoint{3.095889in}{5.309486in}}{\pgfqpoint{3.095889in}{5.296463in}}%
\pgfpathcurveto{\pgfqpoint{3.095889in}{5.283441in}}{\pgfqpoint{3.101063in}{5.270950in}}{\pgfqpoint{3.110271in}{5.261741in}}%
\pgfpathcurveto{\pgfqpoint{3.119480in}{5.252533in}}{\pgfqpoint{3.131971in}{5.247359in}}{\pgfqpoint{3.144993in}{5.247359in}}%
\pgfpathlineto{\pgfqpoint{3.144993in}{5.247359in}}%
\pgfpathclose%
\pgfusepath{stroke,fill}%
\end{pgfscope}%
\begin{pgfscope}%
\pgfpathrectangle{\pgfqpoint{0.454429in}{0.261491in}}{\pgfqpoint{6.040000in}{6.040000in}}%
\pgfusepath{clip}%
\pgfsetbuttcap%
\pgfsetroundjoin%
\definecolor{currentfill}{rgb}{0.121569,0.466667,0.705882}%
\pgfsetfillcolor{currentfill}%
\pgfsetlinewidth{1.003750pt}%
\definecolor{currentstroke}{rgb}{0.121569,0.466667,0.705882}%
\pgfsetstrokecolor{currentstroke}%
\pgfsetdash{}{0pt}%
\pgfpathmoveto{\pgfqpoint{3.541812in}{5.276227in}}%
\pgfpathcurveto{\pgfqpoint{3.554834in}{5.276227in}}{\pgfqpoint{3.567326in}{5.281401in}}{\pgfqpoint{3.576534in}{5.290609in}}%
\pgfpathcurveto{\pgfqpoint{3.585742in}{5.299817in}}{\pgfqpoint{3.590916in}{5.312308in}}{\pgfqpoint{3.590916in}{5.325331in}}%
\pgfpathcurveto{\pgfqpoint{3.590916in}{5.338354in}}{\pgfqpoint{3.585742in}{5.350845in}}{\pgfqpoint{3.576534in}{5.360053in}}%
\pgfpathcurveto{\pgfqpoint{3.567326in}{5.369262in}}{\pgfqpoint{3.554834in}{5.374436in}}{\pgfqpoint{3.541812in}{5.374436in}}%
\pgfpathcurveto{\pgfqpoint{3.528789in}{5.374436in}}{\pgfqpoint{3.516298in}{5.369262in}}{\pgfqpoint{3.507090in}{5.360053in}}%
\pgfpathcurveto{\pgfqpoint{3.497881in}{5.350845in}}{\pgfqpoint{3.492707in}{5.338354in}}{\pgfqpoint{3.492707in}{5.325331in}}%
\pgfpathcurveto{\pgfqpoint{3.492707in}{5.312308in}}{\pgfqpoint{3.497881in}{5.299817in}}{\pgfqpoint{3.507090in}{5.290609in}}%
\pgfpathcurveto{\pgfqpoint{3.516298in}{5.281401in}}{\pgfqpoint{3.528789in}{5.276227in}}{\pgfqpoint{3.541812in}{5.276227in}}%
\pgfpathlineto{\pgfqpoint{3.541812in}{5.276227in}}%
\pgfpathclose%
\pgfusepath{stroke,fill}%
\end{pgfscope}%
\begin{pgfscope}%
\pgfpathrectangle{\pgfqpoint{0.454429in}{0.261491in}}{\pgfqpoint{6.040000in}{6.040000in}}%
\pgfusepath{clip}%
\pgfsetbuttcap%
\pgfsetroundjoin%
\definecolor{currentfill}{rgb}{0.121569,0.466667,0.705882}%
\pgfsetfillcolor{currentfill}%
\pgfsetlinewidth{1.003750pt}%
\definecolor{currentstroke}{rgb}{0.121569,0.466667,0.705882}%
\pgfsetstrokecolor{currentstroke}%
\pgfsetdash{}{0pt}%
\pgfpathmoveto{\pgfqpoint{3.938851in}{5.288627in}}%
\pgfpathcurveto{\pgfqpoint{3.951874in}{5.288627in}}{\pgfqpoint{3.964365in}{5.293801in}}{\pgfqpoint{3.973573in}{5.303009in}}%
\pgfpathcurveto{\pgfqpoint{3.982782in}{5.312218in}}{\pgfqpoint{3.987956in}{5.324709in}}{\pgfqpoint{3.987956in}{5.337732in}}%
\pgfpathcurveto{\pgfqpoint{3.987956in}{5.350754in}}{\pgfqpoint{3.982782in}{5.363245in}}{\pgfqpoint{3.973573in}{5.372454in}}%
\pgfpathcurveto{\pgfqpoint{3.964365in}{5.381662in}}{\pgfqpoint{3.951874in}{5.386836in}}{\pgfqpoint{3.938851in}{5.386836in}}%
\pgfpathcurveto{\pgfqpoint{3.925828in}{5.386836in}}{\pgfqpoint{3.913337in}{5.381662in}}{\pgfqpoint{3.904129in}{5.372454in}}%
\pgfpathcurveto{\pgfqpoint{3.894920in}{5.363245in}}{\pgfqpoint{3.889746in}{5.350754in}}{\pgfqpoint{3.889746in}{5.337732in}}%
\pgfpathcurveto{\pgfqpoint{3.889746in}{5.324709in}}{\pgfqpoint{3.894920in}{5.312218in}}{\pgfqpoint{3.904129in}{5.303009in}}%
\pgfpathcurveto{\pgfqpoint{3.913337in}{5.293801in}}{\pgfqpoint{3.925828in}{5.288627in}}{\pgfqpoint{3.938851in}{5.288627in}}%
\pgfpathlineto{\pgfqpoint{3.938851in}{5.288627in}}%
\pgfpathclose%
\pgfusepath{stroke,fill}%
\end{pgfscope}%
\begin{pgfscope}%
\pgfpathrectangle{\pgfqpoint{0.454429in}{0.261491in}}{\pgfqpoint{6.040000in}{6.040000in}}%
\pgfusepath{clip}%
\pgfsetbuttcap%
\pgfsetroundjoin%
\definecolor{currentfill}{rgb}{0.121569,0.466667,0.705882}%
\pgfsetfillcolor{currentfill}%
\pgfsetlinewidth{1.003750pt}%
\definecolor{currentstroke}{rgb}{0.121569,0.466667,0.705882}%
\pgfsetstrokecolor{currentstroke}%
\pgfsetdash{}{0pt}%
\pgfpathmoveto{\pgfqpoint{3.349130in}{5.545487in}}%
\pgfpathcurveto{\pgfqpoint{3.362153in}{5.545487in}}{\pgfqpoint{3.374644in}{5.550661in}}{\pgfqpoint{3.383852in}{5.559869in}}%
\pgfpathcurveto{\pgfqpoint{3.393061in}{5.569078in}}{\pgfqpoint{3.398235in}{5.581569in}}{\pgfqpoint{3.398235in}{5.594592in}}%
\pgfpathcurveto{\pgfqpoint{3.398235in}{5.607614in}}{\pgfqpoint{3.393061in}{5.620105in}}{\pgfqpoint{3.383852in}{5.629314in}}%
\pgfpathcurveto{\pgfqpoint{3.374644in}{5.638522in}}{\pgfqpoint{3.362153in}{5.643696in}}{\pgfqpoint{3.349130in}{5.643696in}}%
\pgfpathcurveto{\pgfqpoint{3.336107in}{5.643696in}}{\pgfqpoint{3.323616in}{5.638522in}}{\pgfqpoint{3.314408in}{5.629314in}}%
\pgfpathcurveto{\pgfqpoint{3.305199in}{5.620105in}}{\pgfqpoint{3.300025in}{5.607614in}}{\pgfqpoint{3.300025in}{5.594592in}}%
\pgfpathcurveto{\pgfqpoint{3.300025in}{5.581569in}}{\pgfqpoint{3.305199in}{5.569078in}}{\pgfqpoint{3.314408in}{5.559869in}}%
\pgfpathcurveto{\pgfqpoint{3.323616in}{5.550661in}}{\pgfqpoint{3.336107in}{5.545487in}}{\pgfqpoint{3.349130in}{5.545487in}}%
\pgfpathlineto{\pgfqpoint{3.349130in}{5.545487in}}%
\pgfpathclose%
\pgfusepath{stroke,fill}%
\end{pgfscope}%
\begin{pgfscope}%
\pgfpathrectangle{\pgfqpoint{0.454429in}{0.261491in}}{\pgfqpoint{6.040000in}{6.040000in}}%
\pgfusepath{clip}%
\pgfsetbuttcap%
\pgfsetroundjoin%
\definecolor{currentfill}{rgb}{0.121569,0.466667,0.705882}%
\pgfsetfillcolor{currentfill}%
\pgfsetlinewidth{1.003750pt}%
\definecolor{currentstroke}{rgb}{0.121569,0.466667,0.705882}%
\pgfsetstrokecolor{currentstroke}%
\pgfsetdash{}{0pt}%
\pgfpathmoveto{\pgfqpoint{3.750730in}{5.563366in}}%
\pgfpathcurveto{\pgfqpoint{3.763752in}{5.563366in}}{\pgfqpoint{3.776244in}{5.568540in}}{\pgfqpoint{3.785452in}{5.577748in}}%
\pgfpathcurveto{\pgfqpoint{3.794660in}{5.586957in}}{\pgfqpoint{3.799834in}{5.599448in}}{\pgfqpoint{3.799834in}{5.612471in}}%
\pgfpathcurveto{\pgfqpoint{3.799834in}{5.625493in}}{\pgfqpoint{3.794660in}{5.637984in}}{\pgfqpoint{3.785452in}{5.647193in}}%
\pgfpathcurveto{\pgfqpoint{3.776244in}{5.656401in}}{\pgfqpoint{3.763752in}{5.661575in}}{\pgfqpoint{3.750730in}{5.661575in}}%
\pgfpathcurveto{\pgfqpoint{3.737707in}{5.661575in}}{\pgfqpoint{3.725216in}{5.656401in}}{\pgfqpoint{3.716008in}{5.647193in}}%
\pgfpathcurveto{\pgfqpoint{3.706799in}{5.637984in}}{\pgfqpoint{3.701625in}{5.625493in}}{\pgfqpoint{3.701625in}{5.612471in}}%
\pgfpathcurveto{\pgfqpoint{3.701625in}{5.599448in}}{\pgfqpoint{3.706799in}{5.586957in}}{\pgfqpoint{3.716008in}{5.577748in}}%
\pgfpathcurveto{\pgfqpoint{3.725216in}{5.568540in}}{\pgfqpoint{3.737707in}{5.563366in}}{\pgfqpoint{3.750730in}{5.563366in}}%
\pgfpathlineto{\pgfqpoint{3.750730in}{5.563366in}}%
\pgfpathclose%
\pgfusepath{stroke,fill}%
\end{pgfscope}%
\begin{pgfscope}%
\pgfpathrectangle{\pgfqpoint{0.454429in}{0.261491in}}{\pgfqpoint{6.040000in}{6.040000in}}%
\pgfusepath{clip}%
\pgfsetbuttcap%
\pgfsetroundjoin%
\definecolor{currentfill}{rgb}{0.121569,0.466667,0.705882}%
\pgfsetfillcolor{currentfill}%
\pgfsetlinewidth{1.003750pt}%
\definecolor{currentstroke}{rgb}{0.121569,0.466667,0.705882}%
\pgfsetstrokecolor{currentstroke}%
\pgfsetdash{}{0pt}%
\pgfpathmoveto{\pgfqpoint{3.556051in}{5.847682in}}%
\pgfpathcurveto{\pgfqpoint{3.569074in}{5.847682in}}{\pgfqpoint{3.581565in}{5.852856in}}{\pgfqpoint{3.590773in}{5.862064in}}%
\pgfpathcurveto{\pgfqpoint{3.599982in}{5.871273in}}{\pgfqpoint{3.605156in}{5.883764in}}{\pgfqpoint{3.605156in}{5.896786in}}%
\pgfpathcurveto{\pgfqpoint{3.605156in}{5.909809in}}{\pgfqpoint{3.599982in}{5.922300in}}{\pgfqpoint{3.590773in}{5.931509in}}%
\pgfpathcurveto{\pgfqpoint{3.581565in}{5.940717in}}{\pgfqpoint{3.569074in}{5.945891in}}{\pgfqpoint{3.556051in}{5.945891in}}%
\pgfpathcurveto{\pgfqpoint{3.543028in}{5.945891in}}{\pgfqpoint{3.530537in}{5.940717in}}{\pgfqpoint{3.521329in}{5.931509in}}%
\pgfpathcurveto{\pgfqpoint{3.512120in}{5.922300in}}{\pgfqpoint{3.506946in}{5.909809in}}{\pgfqpoint{3.506946in}{5.896786in}}%
\pgfpathcurveto{\pgfqpoint{3.506946in}{5.883764in}}{\pgfqpoint{3.512120in}{5.871273in}}{\pgfqpoint{3.521329in}{5.862064in}}%
\pgfpathcurveto{\pgfqpoint{3.530537in}{5.852856in}}{\pgfqpoint{3.543028in}{5.847682in}}{\pgfqpoint{3.556051in}{5.847682in}}%
\pgfpathlineto{\pgfqpoint{3.556051in}{5.847682in}}%
\pgfpathclose%
\pgfusepath{stroke,fill}%
\end{pgfscope}%
\end{pgfpicture}%
\makeatother%
\endgroup%
}
  \caption{A set of reference directions for a three-objective problem.}
  \label{fig:ref-dirs}
\end{figure}

\ac{nsga2} is useful for mono- and multi-objective functions while \ac{nsga3} is
better for many-objective problems. \ac{unsga3} can handle any number of
objectives by introducing the binary tournament from \ac{nsga2} and reducing to
the most efficient algorithm for the problem at hand \cite{seada_unified_2016}.
Chapter \ref{chapter:benchmark-results} demonstrates these three algorithms in 

\subsection{Hyperparameter Tuning}
Similar to other machine learning models, \acp{ga} have several hyperparameters
that must be tuned for optimal behavior. These hyperparameters include
probabilities for mutation, crossover, and selection, as well as the number of
parents, number of offspring, and population size. Determining ideal
hyperparameters is often performed using either a grid search or random sampling
\cite{bergstra_random_2012}. This thesis adopts the approach from Blank and Deb
\cite{blank_pymoo_2020} using a genetic algorithm to identify the ideal
hyperparameters. A problem is converted into a single objective problem using
the desired algorithm, then a second genetic algorithm drives the problem where
the decision variables are hyperparameters of the desired algorithm.

\subsection{Convergence}
There are several ways to stop a simulation in \ac{pymoo}. A simulation may end
after reaching
\begin{enumerate}
    \item a specified end time (e.g., 100 minutes),
    \label{it:convergence1}
    \item a specified number of evaluations or iterations (e.g., 500 individual
    evaluations or 20 generations),
    \label{it:convergence2}
    \item a tolerance value in the design space,
    \label{it:convergence3}    
    \item a tolerance value in the objective space.
    \label{it:convergence4}
\end{enumerate}

It is possible that criteria \ref{it:convergence3} and \ref{it:convergence4}
will never be met; therefore, they are often combined with either of the first
two criteria. The fourth convergence criterion is the most interesting due to
the challenge of calculating an appropriate metric. This thesis uses the weakly
Pareto-compliant algorithm \ac{igdp} over the more common hypervolume
calculation due to its reduced computational requirements
\cite{ishibuchi_modified_2015}.

\subsection{\acs{pymoo} and \acs{deap}}

The \ac{esom} framework developed in this thesis is built on top of \ac{pymoo}
and \ac{deap}. \ac{pymoo} is an open-source library for \acp{ga} developed by
the creators of \ac{nsga2} and \ac{unsga3} \cite{blank_pymoo_2020}. This package
implements several \acp{ga} out-of-the-box and offers a set of visualization
tools and hyperparameter tuning. \ac{deap} is another open-source library
offering a toolkit for constructing \acp{ga} and therefore has fewer prepackaged
algorithms than \ac{pymoo}. There are robust reasons to use both libraries, so
\ac{osier} facilitates both.




