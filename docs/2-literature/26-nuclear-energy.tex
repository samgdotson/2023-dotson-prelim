\subsection{\textcolor{red}{If nuclear energy can solve climate change, where are all the reactors?}}

\begin{enumerate}
\item What do proponents of nuclear energy say about nuclear power?
\begin{itemize}
    \item Nuclear engineers generally understand that discomfort and fear 
    around nuclear power come from fears about nuclear weapons and fears 
    about radiation. As such, they view these fears as irrational and placatable
    by "educating" and ignorant public.
    \item Some general benefits of nuclear energy (high energy density, low
    material requirements, low carbon footprint, "safe," low land requirements,
    reliable, "resilient").
\end{itemize}
\item What are the objections to nuclear?
\begin{itemize}
    \item Nuclear accidents
    \item Waste (high level waste is a tremendous issue.)
    \item Ethical issues around mining (what historical harms have been done
    to mining communities? What about sourcing uranium from places like 
    Kazakhstan (allied to Russia) and directly funding the invasion of Ukraine?
    What reparations have been made to those harmed? How will future harms be prevented?)
    \item Nuclear weapons proliferation (discuss the relationship between
    nuclear energy and nuclear power. Additionally, although nuclear power 
    doesn't necessarily lead to weapons programs, the need to keep careful
    track of nuclear materials and prevent its release into the biosphere, 
    malicious or otherwise, presents a profound responsibility with some
    intergenerational inequities).
\end{itemize}
\item Why are engineering solutions insufficient?
\begin{itemize}
    \item Case study on yucca mountain + sweden + finland
    \item Are advanced reactors actually being designed in a way that incorporates people's preferences.
    Perhaps preferences toward nuclear energy are not so dependent on the probability of an accident, but
    on the trust between reactor owners and host communities. Non-experts don't have the expertise to assess
    the importance of neutron spectra, fuel form, or other technical design parameters are important for safety.
    Asserting a low accident probability does not inspire trust. If the reactor is so safe, why not put your money
    where your mouth is (so-to-speak) and engage in profit sharing with the community? 
    ``Communities are not the final arbiters of safety, determining safety is the purview of the Nuclear Regulatory Commission.''


    \textcolor{red}{Should I include a question about the benefit of profit-sharing in human-subjects interviews? 
    What are their concerns with energy? Are they focused only on the production of energy or also the lifecycle 
    (extraction + disposal) as well?}
\end{itemize}
\end{enumerate}