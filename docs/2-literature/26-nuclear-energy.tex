\subsection{\textcolor{red}{If nuclear energy can solve climate change, where are all the reactors?}}

\begin{enumerate}
\item What do proponents of nuclear energy say about nuclear power?
\begin{itemize}
    \item Nuclear engineers generally understand that discomfort and fear 
    around nuclear power come from fears about nuclear weapons and fears 
    about radiation. As such, they view these fears as irrational and placatable
    by "educating" and ignorant public.
    \item Some general benefits of nuclear energy (high energy density, low
    material requirements, low carbon footprint, "safe," low land requirements,
    reliable, "resilient").
    \item Additionally, advocates for nuclear energy point to the sustainability 
    of nuclear energy due to its high energy density which in turn reduces the 
    amount of harmful externalities associated with its fuel cycle, relative to
    other technologies.
    \item Finally, the nuclear industry has a clear understanding of its fuel cycle, the ways
    nuclear materials may be reused and/or disposed of, and the measures needed to ensure its
    safety.
\end{itemize}
\item What are the technical objections to nuclear?
\begin{itemize}
    \item Nuclear accidents
    \item Waste (high level waste is a tremendous issue.)
    \item Ethical issues around mining (what historical harms have been done
    to mining communities? What about sourcing uranium from places like 
    Kazakhstan (allied to Russia) and directly funding the invasion of Ukraine?
    What reparations have been made to those harmed? How will future harms be prevented?)
    \item Nuclear weapons proliferation (discuss the relationship between
    nuclear energy and nuclear power. Additionally, although nuclear power 
    doesn't necessarily lead to weapons programs, the need to keep careful
    track of nuclear materials and prevent its release into the biosphere, 
    malicious or otherwise, presents a profound responsibility with some
    intergenerational inequities).
    \item Nuclear energy is expensive.
\end{itemize}
\item What are some non-technical critiques of nuclear power?
\item Why are engineering solutions insufficient?
\begin{itemize}
    \item Case study on yucca mountain + sweden + finland
    \item Are advanced reactors actually being designed in a way that incorporates people's preferences.
    Perhaps preferences toward nuclear energy are not so dependent on the probability of an accident, but
    on the trust between reactor owners and host communities. Non-experts don't have the expertise to assess
    the importance of neutron spectra, fuel form, or other technical design parameters that are important for safety.
    Asserting a low accident probability does not inspire trust. If the reactor is so safe, why not put your money
    where your mouth is (so-to-speak) and engage in profit sharing with the community? 
    ``Communities are not the final arbiters of safety, determining safety is the purview of the Nuclear Regulatory Commission.''


    \textcolor{red}{Should I include a question about the benefit of profit-sharing in human-subjects interviews? 
    What are their concerns with energy? Are they focused only on the production of energy or also the lifecycle 
    (extraction + disposal) as well?}
\end{itemize}
\end{enumerate}



%What do proponents of nuclear energy say about nuclear power?

In spite of its complicated history, nuclear energy has a variety of unique benefits that researchers and
advocates cite to support its continued and expanded use. First, uranium has an enormous energy density. This
fact has a number of important consequences that favor the use of nuclear energy, such as low land use 
\cite{lovering_land-use_2022,van_zalk_spatial_2018}, high \ac{eroi} \cite{weisbach_energy_2013,murphy_energy_2022},
and a low mass and volume of waste byproducts relative to other sources 
\cite{liu_wind_2017,chowdhury_overview_2020,holdsworth_spent_2023,taebi_recycle_2008}. Second, due to the nature of
nuclear fission, nuclear power plants emit zero carbon emissions, making them among the ``cleanest'' sources of 
energy along with solar panels and wind turbines \cite{nicholson_life_2021,intergovernmental_panel_on_climate_change_climate_2021}.