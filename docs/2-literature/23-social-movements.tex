% \section{Social Movements and the Modeling Gap}





% \textcolor{blue}{This tool is needed because conventional energy planning and
% modeling have not yielded a single best course of action to address climate
% change. Indeed, there is no best answer, yet decision-makers employing
% traditional \ac{esom} techniques would lead you to believe so. There are many
% papers identifying nuclear energy as necessary to resolve climate change, and
% others arguing for 100\% renewable energy. \Acp{esom} demonstrate the
% feasibility of using both. How are decision-makers and their constituents
% supposed to choose among apparently equally feasible alternatives? Thus, a
% novel framework for energy modeling is necessary. One that clearly identifies
% trade-offs among conflicting objectives.}



% \begin{enumerate} \item In the absence of corruption (defined as making
%     decisions for personal gain at the expense of fulfilling the duties of a
%     job), policy-makers largely base their decisions on economic merits. \item
%     Why is there a gap between the demonstrated need for a transition away
%     from fossil fuels and the adoption of clean energy? \end{enumerate}



% 8. What do surveys suggest is the reason for opposition to energy projects?
%    \textcolor{red}{\acp{esom} typically model energy transitions at a national
%    or sub-national level, yet siting, permitting, and budgetary decisions are
%    typically made at the local level (subject to regulatory review).}


% \textcolor{orange}{It seems like \ac{osier} could turn into a \ac{pve} tool.
% Previously, I had envisioned users creating their own objectives, or a
% community engagement might start by identifying important objectives to that
% community. That might still be an aspect of the process, and an important one
% for assessing community preferences for different technologies, however later
% iterations of the process, participants may just be given a set of policies to
% rank or choose from. One reason that traditional \acp{esom} do not work with
% the \ac{pve} paradigm, is that total cost cannot be set as a constraint -- it
% is virtually always the thing to be minimized.}

% \noindent\hrulefill

% Addressing ``structural uncertainty'' with \ac{mga} in \acp{esom} is ignorant
% of the significant body of social acceptance research. Social acceptance is an
% essential determinant of the success or failure of an energy project (cite).

% \noindent\hrulefill
\section{Characterizing the Problem of Climate Change}
\label{section:climate-change-risk}
% Questions

% \begin{enumerate}

% \item What are the risks of climate change and failure to respond?

Risk is generally understood as the ``potential for adverse consequences''
\cite{reisinger_concept_2020}. However, due to the complexity of climate change,
the \ac{ipcc} developed a three-tenet framework to discuss risk
\cite{reisinger_concept_2020}: hazard, exposure, and vulnerability.
\textit{Hazards} are mediated by physical features, such as climate and
topography \cite{dorkenoo_critical_2022, simpson_framework_2021}.  Climate
change is already producing more significant hazards, like forest fires,
hurricanes, storms, floods, droughts, and heat waves
\cite{reidmiller_fourth_2018,
intergovernmental_panel_on_climate_change_climate_2021, dahl_killer_2019}.
\textit{Exposure} refers to the scale and duration of the subjection of people,
infrastructure, and social wealth to a particular hazard
\cite{simpson_framework_2021,reisinger_concept_2020,li_understanding_2021}.
\textit{Vulnerability} is the ability of a system to cope, recover, and adapt
after exposure to a hazard. Although climate change is a worldwide phenomenon,
vulnerabilities to its hazards are not uniformly distributed. On the contrary,
the people and communities most likely to be harmed by climate change are
already harmed by social inequities \cite{islam_climate_2017}. For example, 
low-income communities have fewer resources to respond to natural hazards,
such as hurricanes, floods, or fires, and therefore take longer to recover, compared
to a communities with relatively greater wealth. Recent work from
Simpson et al. \cite{simpson_framework_2021} expanded on this definition of risk
by including \textit{responses} to risk as itself a driver of risk. This
framework is illustrated in Figure \ref{fig:risk-framework} using infrastructure
risk as an instructive example. Considering the actions taken (or not) in
response to climate change is vital for a holistic understanding of risk because
it encompasses benefits and mitigating outcomes, not just negative, inflammatory
ones. Additionally, heterogeneous stakeholders perceive the costs and benefits
of (in)action differently. Therefore, including response as a driver of risk is
essential for making choices more transparent and actionable within
decision-making structures \cite{simpson_framework_2021}. Responses to climate
change risk come in myriad forms,  and at multiple scales, from individual
choices (e.g., demand response) \cite{seck_embedding_2020,rinaldi_what_2022,
dehghanpour_agent-based_2018} to community responses
\cite{paterson_community-based_2019, elmallah_frontlining_2022}, national
level policies \cite{roelfsema_taking_2020, fawzy_strategies_2020}, and levels in between. 
Paterson and
Charles \cite{paterson_community-based_2019} developed a descriptive typology
for community-based hazard responses that also applies to national and global
scales. The five response categories making up this typology are:
\cite{paterson_community-based_2019}
\begin{enumerate}
    \item individual and material well-being, which seek to meet individuals'
    basic needs such as food, water, and shelter, as well as livelihood and
    health.
    \item relational well-being emphasizes community and support networks and
    could include evacuation or relocation.
    \item awareness involves monitoring and stock-taking of potential hazards.
    \item governance relates to decision-making structures around human-hazard
    interactions.
    \item infrastructure refers to the physical defense against hazards using
    engineered tools or ecological characteristics.
\end{enumerate} 
Figure \ref{fig:risk-response} shows the breakdown of the categories. Although
this framework could help assess policies to mitigate climate change, these
response categories are related to specific climatic hazards rather than climate
change mitigation.


\begin{figure}
    \centering
    \includegraphics{figures/simpson-risk-framework.jpg}
    \caption{A framework for decomposing risk into its parts: hazard, exposure,
    vulnerability, and response, using risk to infrastructure as an illustrative
    example. Reproduced from Simpson et al. (2021)
    \cite{simpson_framework_2021}.}
    \label{fig:risk-framework}
\end{figure}

\begin{figure}
    \centering
    \includegraphics[width=\columnwidth]{figures/risk-response.png}
    \caption{A categorization schema for various responses to climate risks.
    Reproduced from Paterson et al. (2019)
    \cite{paterson_community-based_2019}.}
    \label{fig:risk-response}
\end{figure}

% \noindent\hrulefill \item What are the responses to climate change? How
% successful have climate policies been at achieving climate goals (and
% ultimately achieving net zero carbon emissions)?

Based on the net-zero carbon emissions target set by the 2016 Paris Agreement,
myriad countries, states, and companies have set climate policies covering
two-thirds of the global economy \cite{hale_assessing_2022}. Reducing CO$_2$ (or
CO$_{2eq}$ in some cases) emissions is the primary focus for most of these
policies \cite{fawzy_strategies_2020, roelfsema_taking_2020,
hale_assessing_2022}, which includes the following broad strategies
\cite{fawzy_strategies_2020}:
\begin{enumerate}
    \item Reducing \ac{ghg} emissions by transitioning from fossil-fueled to
    clean energy.
    \item Removing CO$_2$ from the atmosphere using \ac{ccs} and other
    sequestration techniques.
    \item Altering the Earth's energy balance by increasing its albedo and other
    geoengineering concepts.
\end{enumerate}
Despite this, only around five percent of these policies are consistent with
the \ac{un} ``Race to Zero'' campaign
\cite{hale_assessing_2022}. Further, even the full implementation of national
climate policies leaves approximately  a 28 GtCO$_{2eq}$ gap in \ac{ghg}
emissions \cite{roelfsema_taking_2020} (with the implicit goal of zero emissions). 
This gap and the fundamental assumptions
about carbon sequestration from the 2016 Paris Agreement suggest that the world
is on track to overshoot these emissions targets
\cite{roelfsema_taking_2020,taylor_managing_2021}. Carley et al. (2018)
developed a quantitative framework for assessing the vulnerabilities associated
with energy policies or responses \cite{carley_framework_2018}.

% \noindent\hrulefill \item What are the impacts of climate change?

Risk analysis is the first step to a more encompassing understanding of the climate
crisis. The literature on disproportionality further distinguishes
\textit{risks} and \textit{impacts} \cite{dorkenoo_critical_2022}. Consistent
with previous work, a risk is the aggregate of hazards, exposures,
vulnerabilities, and responses. Impacts, then, are the realizations of risk in
terms of loss and damages. This distinction is essential. Responses to
\textit{impacts} are always made \textit{ex post facto}. Differences in
vulnerability to a hazard, often arbitrated by socio-economic status, manifest
as differential impacts. Access to resources conditions an individual's or
community's ability to respond to the impacts of a hazard. Since losses from
impacts disproportionately affect those with the fewest resources, their
vulnerability to future hazards increases in a ``vicious cycle''
\cite{islam_climate_2017, dorkenoo_critical_2022}. In purely economic terms,
studies estimate the loss of ecosystem services from land use change associated
with climate change and other human activities at \$4 - \$20 trillion per year
(in 2011 \$US) globally, \cite{costanza_changes_2014} and the poorest third of
U.S. counties will experience financial damages between 2 and 20 percent of
their annual income \cite{hsiang_estimating_2017}. However, impacts also have
cultural and psychological dimensions \cite{dorkenoo_critical_2022} that cannot
be captured by accounting for ``externalities.''

% \noindent\hrulefill

% \item How are the damages of climate change distributed, and \textit{why} are
% they distributed this way?

Dorkenoo et al. \cite{dorkenoo_critical_2022} establish \textit{burdens},
injustices arising from social, political, or economic power imbalances, as a
third theme paramount for a holistic understanding of disproportionality.
Burdens influence all aspects of risk and affect access to resources which
condition impacts. Dorkenoo et al. wrote, ``[p]rocesses of marginalization and
exclusion influenced by power struggles [...] influence the distribution of
burdens and consequently responsibilities, in addition to the different
dimensions of climate risk (hazard, exposure, vulnerability [, response])''
\cite{dorkenoo_critical_2022}. Figure \ref{fig:risk-impact-burden} demonstrates
the mutually reinforcing relationships among risks, impacts, and burdens. A
particularly relevant example of burden is the persistence of energy burden,
where low-income households pay the highest percentage of their income on energy
bills relative to other income groups \cite{brown_high_2020,
cong_unveiling_2022}. Energy burden interferes with electricity access, thereby
increasing vulnerability to extreme heat events \cite{cong_unveiling_2022,
klinenberg_heat_2003}. The risk assessment literature and the energy system
modeling literature typically adopt an apolitical framing of vulnerabilities.
That is to say, these literature to analyze their respective systems independent
of any sociopolitical context.
However, inequities do not arise in a vacuum but through processes of
marginalization and exclusion \cite{thomas_explaining_2019}. Often the
distribution of burdens falls along class, race, and gendered lines
\cite{thomas_explaining_2019,mohai_which_2015}. Research on siting patterns of
polluting facilities indicates these projects frequently developed in areas with
people of color and low-income populations \cite{mohai_which_2015}. Pollution
from these facilities creates additional burdens for nearby communities. The
energy justice and environmental justice literature offer insights to contrast
this neutral framing and facilitate normative questions about alternative
distributions \cite{dorkenoo_critical_2022, thomas_explaining_2019}.

\begin{figure}
    \centering
    \includegraphics[width=\columnwidth]{figures/dorkenoo-disproportionality.jpg}
    \caption{The relationships among risks, impacts, and burdens. Reproduced
    from Dorkenoo et al. (2022) \cite{dorkenoo_critical_2022}.}
    \label{fig:risk-impact-burden}
\end{figure}


% \noindent\hrulefill

% \textcolor{red}{Subsection: Climate and Energy Systems}
\section{Climate and Energy Systems}

Climate change is driven by the buildup of additional \acp{ghg} in our atmosphere
from human activities.
\acp{ghg} are a significant byproduct of the
infrastructure that creates, delivers, and consumes energy (i.e., energy
infrastructure). The total decarbonization of the global economy
will lead to greater electricity demand, even when accounting for efficiency
improvements
\cite{national_academies_of_sciences_engineering_and_medicine_accelerating_2021,
mai_electrification_2018}, decarbonizing our electricity production is one of
the most critical issues to resolving climate change. Therefore, producing
electricity with zero \acp{ghg} will initiate a cascade of deeper
decarbonization throughout the economy. However, this will require expanded electrical
infrastructure to accomodate new energy technologies. Since, energy production contributes
significantly to climate change, and new energy infrastructure is required to reduce carbon
emissions in other sectors (e.g. heat and transportation), accelerating the adoption of clean energy technology
(i.e. technologies that do not release \acp{ghg} to the atmosphere) is
essential for achieving a stable climate \cite{roelfsema_taking_2020,
taylor_managing_2021}.
The next section discusses the range of technical solutions for accomplishing the
goals established above.

% \textcolor{red}{Subsubsection: Technical solutions to energy decarbonization}
\subsection{Technical Solutions to Energy Decarbonization}
% \item What are the technical solutions for decarbonizing the electricity
% sector?

Many studies show that global and local economies can be supported by 100\%
\ac{vre}, such as wind, hydro, solar power, and storage \cite{jacobson_100_2015,
bussar_optimal_2014,brown_response_2018,dorotic_integration_2019,wallsgrove_emerging_2021,
cochran_la100_2021,cosic_100_2012,traber_economically_2021,bogdanov_full_2021,
bogdanov_north-east_2016,
esteban_100_2018,yue_least_2020,neumann_near-optimal_2021}. Yet some countries
that transition to majority \ac{vre} observe higher carbon emissions or a
slower-than-expected reduction due to greater dependence on natural gas brought on
by the relative unpredictability of natural energy sources
\cite{wagner_co2_2021}. Other studies demonstrate that firm baseload power, such
as nuclear power, is necessary for the deep decarbonization of our energy
systems
\cite{wagner_co2_2021,shaner_geophysical_2018,dotson_influence_2022,greene_enhancing_2019,kim_carbon_2021,
lehtveer_how_2015,vaillancourt_role_2008,
de_sisternes_value_2016,alzbutas_uncertainty_2012,brook_why_2014,
epiney_economic_2020,petti_future_2018, patrizio_socially_2020}. While some
countries are building new nuclear reactors, and the \ac{nrc} just licensed the design
of the first small modular reactor design from NuScale
\cite{office_of_nuclear_energy_science_and_technology_nrc_2023}, other places
are shutting down their operating nuclear plants \cite{johnson_new_2021}. 
% In the latter cases, places that shut down nuclear plants always saw a
% subsequent rise in carbon emissions due to greater dependence on natural gas.
% \textcolor{red}{Show plot of carbon emissions vs time for New York?}. 
Further, the only examples of highly decarbonized electrical grids are places
with a high penetration of hydro or nuclear power and the former is widely
considered exhausted. There is nearly universal agreement that decarbonizing
electricity requires phasing out fossil-fueled power plants and a significant
expansion of clean electricity generators. Although many studies show the
\textit{feasibility} of a variety of energy mixes, the following is strongly
debated in the literature.
\begin{enumerate}
    \item Whether energy systems should be 100\% renewable or if nuclear power
    and \ac{ccs} should be included \cite{heard_burden_2017,
    brown_response_2018,elmallah_frontlining_2022, brook_why_2014}.
    \item What the role of distributed and decentralized energy sources in
    expanding our energy infrastructure should be
    \cite{pitt_assessing_2015,rinaldi_what_2022,parag_electricity_2016,wang_modeling_2020,
    morvaj_decarbonizing_2017,gilbert_can_2020,li_economic_2016,falke_multi-objective_2016}.
\end{enumerate}
The strength of the technical arguments on both sides of these discussions
combined with the distinct lack of sufficient policy agendas pursuing any of
them \cite{roelfsema_taking_2020,hale_assessing_2022}, suggests the existence of
poorly articulated trade-offs and that technical solutions cannot be assessed
from an engineering perspective, alone. Some researchers and policymakers
disagree on technical grounds, while others disagree on the basis of
institutional or systemic injustices. There are also differences in values.
Indeed, the cultural theory of risk argues that our social constructions, rather
than risks themselves, dictate what threats are recognized and their
corresponding liabilities and benefits \cite{mcneeley_cultural_2014,
van_de_graaff_understanding_2016}. Clean technologies like nuclear power and
renewables, such as solar or wind power, are not only different in how they
produce electricity but also in the values and paradigms they represent.
Sometimes, communication fails because the question being discussed is not
agreed upon either. Often, feasibility studies address the positivist question,
``what is the least-cost pathway to the energy transition,'' while others
consider more normative questions, such as ``how should we proceed equitably?''
Normative questions are qualitative and, therefore, inherently challenging to
answer and require the application of ethics. Indeed there are many more
normative questions than positive ones. \textcolor{black}{Is perfect the enemy
of good? How do we balance stakeholder preferences, upstream and downstream
effects, and the necessity to respond quickly to climate change? Will this mix
of influences lead to paralysis or inaction?} 
Engineers typically do not possess the training nor the expertise to answer these questions
thoroughly.
\textcolor{black}{Therefore, given climate
change's complex, interacting, and disproportionate nature, engineering alone is
ill-equipped to resolve the problem. Ideas from the environmental and energy
justice literature offer a social perspective for addressing the risks and
impacts of climate change hazards.}
The next section introduces the concept of energy justice and how this area of
scholarship understands challenges related to climate change and energy systems.

