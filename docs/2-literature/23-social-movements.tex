% \section{Social Movements and the Modeling Gap}





% \textcolor{blue}{This tool is needed because conventional energy planning and
% modeling have not yielded a single best course of action to address climate
% change. Indeed, there is no best answer, yet decision-makers employing
% traditional \ac{esom} techniques would lead you to believe so. There are many
% papers identifying nuclear energy as necessary to resolve climate change, and
% others arguing for 100\% renewable energy. \Acp{esom} demonstrate the
% feasibility of using both. How are decision-makers and their constituents
% supposed to choose among apparently equally feasible alternatives? Thus, a
% novel framework for energy modeling is necessary. One that clearly identifies
% trade-offs among conflicting objectives.}



% \begin{enumerate} \item In the absence of corruption (defined as making
%     decisions for personal gain at the expense of fulfilling the duties of a
%     job), policy-makers largely base their decisions on economic merits. \item
%     Why is there a gap between the demonstrated need for a transition away
%     from fossil fuels and the adoption of clean energy? \end{enumerate}



% 8. What do surveys suggest is the reason for opposition to energy projects?
%    \textcolor{red}{\acp{esom} typically model energy transitions at a national
%    or sub-national level, yet siting, permitting, and budgetary decisions are
%    typically made at the local level (subject to regulatory review).}


% \textcolor{orange}{It seems like \ac{osier} could turn into a \ac{pve} tool.
% Previously, I had envisioned users creating their own objectives, or a
% community engagement might start by identifying important objectives to that
% community. That might still be an aspect of the process, and an important one
% for assessing community preferences for different technologies, however later
% iterations of the process, participants may just be given a set of policies to
% rank or choose from. One reason that traditional \acp{esom} do not work with
% the \ac{pve} paradigm, is that total cost cannot be set as a constraint -- it
% is virtually always the thing to be minimized.}

% \noindent\hrulefill

% Addressing ``structural uncertainty'' with \ac{mga} in \acp{esom} is ignorant
% of the significant body of social acceptance research. Social acceptance is an
% essential determinant of the success or failure of an energy project (cite).

% \noindent\hrulefill
\section{Characterizing the Problem of Climate Change}

% Questions

% \begin{enumerate}

% \item What are the risks of climate change and failure to respond?

Risk is generally understood as the ``potential for adverse consequences''
\cite{reisinger_concept_2020}. However, due to the complexity of climate change,
the \ac{ipcc} developed a three-tenet framework to discuss risk
\cite{reisinger_concept_2020}: hazard, exposure, and vulnerability.
\textit{Hazards} are mediated by physical features, such as climate and
topography \cite{dorkenoo_critical_2022, simpson_framework_2021}.  Climate
change is already producing stronger hazards, like forest fires, hurricanes,
storms, floods, droughts, and heat waves \cite{reidmiller_fourth_2018,
intergovernmental_panel_on_climate_change_climate_2021, dahl_killer_2019}.
\textit{Exposure} refers to the scale and duration of the subjection of people,
infrastructure, and social wealth to a particular hazard
\cite{simpson_framework_2021,reisinger_concept_2020,li_understanding_2021}.
\textit{Vulnerability} is the ability of a system to cope, recover, and adapt
after exposure to a hazard. Although climate change is a worldwide phenomenon,
vulnerabilities to its hazards are not uniformly distributed. On the contrary,
the people and communities most likely to be harmed by climate change are those
already being harmed by social inequities \cite{islam_climate_2017}. Recent work
from Simpson et al. \cite{simpson_framework_2021} expanded on this definition of
risk by including \textit{responses} to risk as itself a driver of risk.
Considering the actions taken (or not) in response to climate change is
important for a holistic understanding of risk because it encompasses benefits
and mitigating outcomes, not just negative, inflammatory ones. Additionally,
various stakeholders perceive the costs and benefits of (in)action differently,
and therefore, including response as a driver of risk is essential for making
choices more transparent and actionable within decision-making structures
\cite{simpson_framework_2021}. Responses to climate change risk come in myriad
forms,  and at multiple scales, from individual choices (e.g. demand response)
\cite{seck_embedding_2020,rinaldi_what_2022, dehghanpour_agent-based_2018}, to
communities \cite{paterson_community-based_2019, elmallah_frontlining_2022}, and
at the national level \cite{roelfsema_taking_2020, fawzy_strategies_2020}.
Paterson and Charles \cite{paterson_community-based_2019} developed a
descriptive typology for community-based hazard responses that also applies to
national and global scales. The five response categories making up this typology
are individual material well-being, relational well-being, awareness,
governance, and infrastructure.

\noindent\hrulefill
% \item What are the responses to climate change? How successful have climate
% policies been at achieving climate goals (and ultimately achieving net zero
% carbon emissions)?

Based on the net-zero carbon emissions target set by the 2015 Paris Agreement,
thousands of countries, states, and companies have set climate policies covering
two-thirds of the global economy \cite{hale_assessing_2022}. Reducing CO$_2$ (or
CO$_{2eq}$ in some cases) emissions is the primary focus for most of these
policies \cite{fawzy_strategies_2020, roelfsema_taking_2020,
hale_assessing_2022}, which includes the following broad strategies
\cite{fawzy_strategies_2020}:
\begin{enumerate}
    \item Reducing \ac{ghg} emissions by transitioning from fossil-fueled to
    clean energy.
    \item Removing CO$_2$ from the atmosphere using \ac{ccs} and other
    sequestration techniques.
    \item Altering the Earth's energy balance by increasing its albedo and other
    geoengineering concepts.
\end{enumerate}

In spite of this, only around five percent of these policies are considered
robust according to their consistency with the \ac{un} ``Race to Zero'' campaign
\cite{hale_assessing_2022}. Further, even the full implementation of national
climate policies leaves a 28 GtCO$_{2eq}$ gap in \ac{ghg} emissions
\cite{roelfsema_taking_2020}. This gap, as well as the fundamental assumptions
about carbon sequestration from the 2015 Paris Agreement, suggests that the
world is on track to overshoot these emissions targets
\cite{roelfsema_taking_2020,taylor_managing_2021}. Carley et al. (2018)
developed a quantitative framework for assessing the vulnerabilities associated
with energy policies, or responses \cite{carley_framework_2018}.

\noindent\hrulefill
% \item What are the impacts of climate change?

The literature on disproportionality further distinguishes \textit{risks} and
\textit{impacts} \cite{dorkenoo_critical_2022}. Consistent with previous work, a
risk is the aggregate of hazards, exposures, vulnerabilities, and responses.
Impacts, then, are the realizations of risk in terms of loss and damages. This
distinction is essential. Responses to \textit{impacts} are always done
\textit{ex post facto}. Differences in vulnerability to a hazard, often
arbitrated by socio-economic status, manifest as differential impacts. Access to
resources conditions an individual's or community's ability to respond to the
impacts of a hazard. Since losses from impacts disproportionately affect those
with the fewest resources, their vulnerability to future hazards increases in a
``vicious cycle'' \cite{islam_climate_2017, dorkenoo_critical_2022}. In purely
economic terms, the loss of ecosystem services from land use change associated
with climate change and other human activities is estimated at \$4 - \$20
trillion per year (in 2011 \$US), globally, \cite{costanza_changes_2014} and the
poorest third of U.S. counties will experience financial damages between 2 and
20 percent of their annual income \cite{hsiang_estimating_2017}. However,
impacts also have cultural and psychological dimensions
\cite{dorkenoo_critical_2022} that cannot be captured by accounting for
``externalities.''

\noindent\hrulefill

% \item How are the damages of climate change distributed, and \textit{why} are
% they distributed this way?

Dorkenoo et al. \cite{dorkenoo_critical_2022} establish \textit{burdens},
injustices arising from imbalances in social, political, or economic power, as a
third theme that is paramount for a holistic understanding of
disproportionality. Burdens influence all aspects of risk and affect access to
resources which condition impacts. Dorkenoo et al. wrote, ``[p]rocesses of
marginalization and exclusion influenced by power struggles [...] influence the
distribution of burdens and consequently responsibilities, in addition to the
different dimensions of climate risk (hazard, exposure, vulnerability [,
response])'' \cite{dorkenoo_critical_2022}. A particularly relevant example is
the persistence of energy burden, where low-income households pay the highest
percentage of their income on energy bills, relative to other income groups
\cite{brown_high_2020, cong_unveiling_2022}. Energy burden interferes with
electricity access, thereby increasing vulnerability to extreme heat events
\cite{cong_unveiling_2022, klinenberg_heat_2003}. The risk assessment
literature, as well as the energy system modeling literature, typically adopt an
apolitical framing of vulnerabilities. However, inequities do not arise in a
vacuum but through processes of marginalization and exclusion
\cite{thomas_explaining_2019}. Often the distribution of burdens falls along
class, race, and gendered lines \cite{thomas_explaining_2019,mohai_which_2015}.
Research on siting patterns of polluting facilities indicates these projects
frequently developed in areas with people of color and low-income populations
\cite{mohai_which_2015}. Pollution from these facilities creates additional
burdens for nearby communities. The energy and environmental justice literature
offer insights to contrast this neutral framing and facilitate normative
questions about alternative distributions \cite{dorkenoo_critical_2022,
thomas_explaining_2019}.

\noindent\hrulefill

% \textcolor{red}{Subsection: Climate and Energy Systems}
\section{Climate and Energy Systems}

Climate change is driven by \acp{ghg}, a significant byproduct of the
infrastructure that creates, delivers, and consumes energy (i.e. energy
infrastructure). Specifically, due to the fraction of carbon emissions already
coming from electricity generation (25 percent in the United States
\cite{us_epa_sources_2020}) and the full decarbonization of the global economy
requiring electrification of other sectors such as transportation and heat,
thereby increasing the demand for electricity, even when accounting for
efficiency improvements
\cite{national_academies_of_sciences_engineering_and_medicine_accelerating_2021,
mai_electrification_2018}, decarbonizing our electricity production is one of
the most important issues to resolve climate change. Therefore, producing
electricity with zero \acp{ghg} will initiate a cascade of deeper
decarbonization throughout the economy but will require expanded electrical
infrastructure. Accelerating the adoption of clean energy technology is
essential for achieving a stable climate \cite{roelfsema_taking_2020,
taylor_managing_2021}.

% \textcolor{red}{Subsubsection: Technical solutions to energy decarbonization}
\subsection{Technical Solutions to Energy Decarbonization}
% \item What are the technical solutions for decarbonizing the electricity
% sector?

Many studies show that global and local economies can be supported by 100\%
\ac{vre}, such as wind, hydro, and solar power \cite{jacobson_100_2015,
bussar_optimal_2014,brown_response_2018,dorotic_integration_2019,wallsgrove_emerging_2021,
cochran_la100_2021,cosic_100_2012,traber_economically_2021,bogdanov_full_2021,
bogdanov_north-east_2016,
esteban_100_2018,yue_least_2020,neumann_near-optimal_2021}. Yet some countries
that transition to majority \ac{vre} observe greater carbon emissions or a
slower-than-expected reduction due to greater dependence on natural gas
\cite{wagner_co2_2021}. Other studies demonstrate that nuclear power is
necessary for the deep decarbonization of our energy systems
\cite{wagner_co2_2021,dotson_influence_2022,greene_enhancing_2019,kim_carbon_2021,
lehtveer_how_2015,
de_sisternes_value_2016,alzbutas_uncertainty_2012,brook_why_2014,
epiney_economic_2020,petti_future_2018, patrizio_socially_2020}. While some
countries are building new nuclear reactors and the \ac{nrc} just approved the
first small modular reactor design from NuScale
\cite{office_of_nuclear_energy_science_and_technology_nrc_2023}, other places
are shutting down their operating nuclear plants \cite{johnson_new_2021}. In the
latter cases, places that shut down nuclear plants always saw a subsequent rise
in carbon emissions, due to greater dependence on natural gas.
\textcolor{red}{Show plot of carbon emissions vs time for New York?}. Further,
the only examples of highly decarbonized electrical grids are places with a high
penetration of hydro or nuclear power, and the former is widely considered
exhausted. There is nearly universal agreement that decarbonizing electricity
requires phasing out fossil-fueled power plants and a significant expansion of
clean electricity generators. Although many studies show the
\textit{feasibility} for a variety of energy mixes, the following is strongly
debated in the literature.
\begin{enumerate}
    \item Whether energy systems should be 100\% renewable
    \cite{heard_burden_2017, brown_response_2018}.
    \item Whether nuclear power and \ac{ccs} are ``false solutions''
    \cite{elmallah_frontlining_2022, brook_why_2014} .
    \item The role of distributed and decentralized energy sources in expanding
    our energy infrastructure
    \cite{pitt_assessing_2015,rinaldi_what_2022,parag_electricity_2016,wang_modeling_2020,
    morvaj_decarbonizing_2017,gilbert_can_2020,li_economic_2016,falke_multi-objective_2016}.
\end{enumerate}

Some researchers and policymakers disagree on technical grounds, while others
disagree on the basis of institutional or systemic injustices. There are also
differences in values. Clean technologies like nuclear power and renewables,
such as solar or wind power, are not only different in how they produce
electricity but also in the values and paradigms they represent. Sometimes,
communication fails because the question being discussed is not agreed upon
either. Often, feasibility studies address the positivist question, ``what is
the least-cost pathway to the energy transition,'' while others consider more
normative questions, such as ``how should we proceed equitably?'' Normative
questions are qualitative and, therefore, inherently challenging to answer and
require the application of ethics. Indeed there are many more normative
questions than positive ones. \textcolor{black}{Is perfect the enemy of good?
How do we balance stakeholder preferences, upstream and downstream effects, and
the necessity to respond quickly to climate change? Will this mix of influences
lead to paralysis or inaction?} \textcolor{black}{Given climate change's
complex, interacting, and disproportionate nature, engineering alone is
ill-equipped to resolve the problem. We need ideas from the environmental and
energy justice literature to help address the risks and impacts of climate
change hazards.}
% \noindent\hrulefill \textcolor{red}{Subsubsection: Energy justice and the
% Boundaries of Energy Systems.}
\subsection{Energy Justice and the Boundaries of Energy Systems}

% \item What is energy justice?


Energy justice is a conceptual and analytical tool regarding the ethical or
normative dimensions of energy systems and addresses the systemic causes of
burdens, and inequities \cite{sovacool_energy_2015}. 

% \begin{enumerate} \item What is justice?
    
    There are many conceptions of justice; however, the most popular framework
    for discussing justice is a three-faceted approach originating from David
    Schlosberg: distributional, recognitional, and procedural justice
    \cite{schlosberg_2_2007}.

    % \item What is distributional justice?
    
\noindent\hrulefill

    This form of justice relates to the fair distribution of resources, burdens,
    and responsibilities. Studies on distributional justice seek to address the
    normative question: how should a just society distribute the benefits it
    produces and \textit{the burdens required to maintain it}
    \cite{brighouse_justice_2004}. Additionally, distributional justice
    considers \textit{how} poor distributions are created
    \cite{schlosberg_2_2007}.
    
\noindent\hrulefill
    % \item What is procedural justice?

    Procedural (in)justice is defined as the presence of (un)fair and
    (in)equitable institutional processes of the state \cite{schlosberg_2_2007}.
    In other words, how decisions of societal import are made and who is
    involved in those decisions. Sovacool and Dworkin (2015) outline four
    elements of procedural justice: transparency, meaningful participation,
    impartiality, and avenues for redress \cite{sovacool_energy_2015}.
    
\noindent\hrulefill
    % \item What is recognition justice?

    Justice of recognition is the vaguest of the three tenets of justice and is
    frequently reduced to a component of either distribution or procedural
    justice \cite{schlosberg_2_2007, van_uffelen_revisiting_2022}. A common
    argument for this consolidation is that recognition is a precondition for
    achieving distributional justice or that achieving procedural justice
    necessarily includes recognition \cite{schlosberg_2_2007}. However,
    recognition is unique from distributive and procedural justices because it
    is concerned with a different family of injustice, namely,
    \textit{misrecognition} \cite{van_uffelen_revisiting_2022}. van Uffelen
    (2022) suggests a nuanced definition of recognition as ``the adequate
    recognition of all actors through love, law, and the status order''
    \cite{van_uffelen_revisiting_2022}.
% \end{enumerate} \noindent\hrulefill
\textcolor{red}{Next, I examine the specific ways the social science literature understands how energy systems and their infrastructure (artifacts) contribute to the distribution of burdens.}

\noindent\hrulefill
% \item What is an energy system?

Previous work defined energy systems in purely technical terms as spatially,
temporally, and topologically complex machines that coordinate the supply and
demand of energy, especially electricity \cite{dotson_influence_2022}. However,
this definition neglects the ways energy systems may be used to construct and
maintain power relations that contribute to inequitable distributions of
burdens. Energy access is necessary to support complex modern economies and
therefore possesses political power \cite{jones_building_2013,
bridge_energy_2018}. The literature on the political economy of energy
infrastructure locates this political influence in five distinct ways
\cite{bridge_energy_2018}. First, energy infrastructure affects competition and
collaboration among nation-states in the geo-political sphere. The current
situation in Ukraine makes this especially salient
\cite{figueiredo_impacts_2022}. 

The second subset of the literature focuses on the process of energy
infrastructure development and how these processes create social inequities. For
example, energy policies that subsidize residential solar panels have not led to
more equitable adoption of solar energy, with greater adoption in areas with
higher income, among other social indicators \cite{reames_distributional_2020}.
Other popular arguments in favor of renewable energy assert that these energy
sources are necessarily more egalitarian because the Sun and the wind cannot be
(or have not yet been) privatized. Another is the urgency of climate change.
While true, ignores or minimizes the potential environmental and social
consequences of energy planning that does not consider energy justice
\cite{jones_building_2013}. Large-scale energy projects in the Global South have
already led to the dispossession of nearby indigenous communities and other key
actors \cite{yenneti_spatial_2016, barragan-contreras_procedural_2022}.

Third, the development of energy infrastructure is not simply conducted via
policy measures, but also in the manner governments activate the public
imagination in favor of these policies
\cite{bridge_energy_2018,jasanoff_containing_2009}. Jasanoff and Kim (2009)
articulate this concept as `socio-technical imaginaries,' which are
simultaneously descriptive and prescriptive of possible energy futures
established by governments in the national zeitgeist
\cite{jasanoff_containing_2009}. This concept is demonstrated by the discourse
surrounding nuclear energy in the United States and South Korea
\cite{jasanoff_containing_2009} as well as in Japan
\cite{valentine_energy_2019}. Governments can employ `grand narratives' related
to national security, climate change, or modernization to enhance public support
while minimizing genuine participation \cite{bridge_energy_2018}.

Fourth, the political power of energy infrastructure can be traced further to
the cultural values and policy choices embedded in the design and operation of
seemingly technical systems \cite{bridge_energy_2018}. In other words, the
design and implementation of energy infrastructure may be used as a vehicle for
apparently unrelated agendas, a form of ``policy-making by other means''
\cite{bridge_energy_2018, clausewitz_chapter_1918}. Edwards and Hecht (2010)
refer to the co-constitution of technological and political order as
`\textit{technopolitics},' demonstrating the tangible material and political
outcomes of technological systems \cite{edwards_history_2010}.

Finally, energy systems and their infrastructure possess a unifying quality
through which new political identities may evolve \cite{bridge_energy_2018}.

From these various perspectives, we can observe that confining an energy system
to its technical characteristics is woefully incomplete. I propose that an
energy system is a spatially, temporally, and topologically complex machine that
coordinates the supply and demand of energy and acts as an important mediator of
burdens that influence climate change risk. This thesis takes the important step
of analyzing energy system planning and policy with this expanded definition.


\textcolor{red}{Climate change \textit{is} a complex issue with multiple
interacting and interwoven layers. However, it can and must be understood
holistically instead of stripping it of its complexity and exclusively
relegating solutions to the realms of economics and engineering. This
over-simplification is done out of a misplaced sense of pragmatism and either an
inability or unwillingness to completely apprehend the problem.}

\noindent\hrulefill
% \item What are the obstacles to building more clean energy infrastructure?
% What is preventing the transition to a ``qualitatively new type of
% [environmental] society'' \cite{bluhdorn_legitimation_2020}?

% \textcolor{red}{This is where you can introduce different schools of thought
% about transitions.}

% \textcolor{blue}{Interesting that you chose to frame public acceptance as an
% ``obstacle'' rather than a source of greater accountability and
% participation.} \begin{enumerate} \item Legitimation crisis of democracy
% \cite{bluhdorn_legitimation_2020}. \item Social acceptance literature
% \end{enumerate}

% \end{enumerate}

\section{Calls for a Just-Transition}

Att