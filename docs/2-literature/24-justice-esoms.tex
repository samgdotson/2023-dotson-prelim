\section{Modeling and Quantifying Energy Justice}

The dearth of studies that incorporate energy justice into \acp{esom} highlights
the challenge of combining these techniques. The literature on energy justice and
socio-technical transitions tend to derogate modeling efforts as cold and
calculating \cite{sovacool_energy_2015,sovacool_energy_2016}, and most models do
not account for energy justice in either equations or analysis. However, there
have been some notable attempts to bridge this gap. The following studies by Patrizio 
et al. (2020) \cite{patrizio_socially_2020} and Neumann \& Brown (2021) \cite{neumann_near-optimal_2021}
explicitly use \acp{esom} in their analyses. Although the works by Chapman et al. (2018)
\cite{chapman_prioritizing_2018} and Mayfield et al. (2019) \cite{mayfield_quantifying_2019}
do not use \acp{esom} as described in Section \ref{section:esoms}, these contributions
quantify some features of their respective energy systems and how they relate to
notions of energy justice and equity.

Patrizio et al. (2020) conducted a technology-agnostic `social equity' scenario
that maximized the \ac{gva} of several countries' energy systems rather than
minimizing the total cost \cite{patrizio_socially_2020}. \Ac{gva} is also
distinct from social welfare because it measures contributions to \ac{gdp} from
individual producers rather than maximizing surplus. This metric enables
sector-specific analysis of the impacts of energy infrastructure on employment
and sales. Equity, in this context, is identical to socioeconomic development as
measured by \ac{gdp}. The researchers looked at a socio-technical transition for
three countries: Spain, the United Kingdom, and Poland. They found that a 100\%
renewable energy system would reduce labor compensation by 50-60\% in the UK and
Poland but could increase benefits in Spain. They argue this is due to the
outsourcing of manufacturing and mining jobs in the former cases, while Spain
has enough domestic resources to accommodate the transition. The researchers did
not analyze possible shifts in power dynamics related to the energy systems, but
they did identify that there is no one-size-fits-all solution to achieving
net-zero carbon emissions.

Neumann \& Brown (2021) performed a detailed analysis of the European energy
system considering the expansion of transmission networks and energy producers
for a 100\% renewable energy system under cost minimization
\cite{neumann_near-optimal_2021}. They also used a novel formulation of \ac{mga}
to identify the boundaries of the feasible space for each technology within
different levels of tolerance. This study uses Lorenz curves and Gini
coefficients to measure the uniformity of the distribution of energy production
and consumption. In other words, the most equitable distribution of energy
resources would accord with energy consumption \cite{neumann_near-optimal_2021}.
The researchers conclude that wind power and greater transmission capacity are
associated with less regional equity, while solar power and storage technologies
lead to a more even distribution of the power supply. This is useful for
measuring the distribution of energy benefits from the energy system but does
not consider the distribution of costs nor consider regional preferences. 

Chapman et al. (2018) looked at the energy justice implications of transitioning
coal plants to renewable energy projects for the nearby communities
\cite{chapman_prioritizing_2018}. They measure distributional justice with
``relative equity'' and ``policy burden.'' Relative equity accounts for factors
such as \ac{ghg} reduction, employment, electricity cost, and health impacts.
Policy burden is a weighted value according to the income level of each
community. These two quantities were plotted together to identify a retirement
schedule that maximizes equity outcomes and ensures that burdens are borne by
the ablest communities \cite{chapman_prioritizing_2018}. Additionally, the
researchers argue that by using equity measures to inform policy choices, those
policy decisions are more procedurally just. However, this neglects meaningful
participation and may or may not address decision-making transparency
\cite{sovacool_energy_2015}. Further, this study does not consider how replacing
dispatchable suppliers with \ac{vre} will affect the availability and
affordability of electricity \cite{sovacool_energy_2015}.

Mayfield et al. (2019) quantified the social equity implications for the
expansion of natural gas infrastructure in Appalachia using spatial and temporal
metrics such as job-years generated by greater gas development, premature deaths
caused by air pollution, changes in poverty and income, and the distribution of
these various benefits along regional, racial, and economic lines. Additionally,
they identified some of the intergenerational equity impacts of climate change
and expanded gas infrastructure.

\subsection{Enabling Procedural Justice Through Energy Models}

Traditionally, \acp{esom} are used to inform policy-makers \cite{li_open_2020}
in order to infuse policy choices with an appearance of objectivity. Indeed,
some of the studies reviewed in the previous section argue that this infusion
will lead to greater procedural and recognitional justice outcomes as long as
the policies maximize some measure of energy justice
\cite{chapman_prioritizing_2018, heffron_resolving_2015}. However, these types
of detailed analyses may also be used to dismiss concerns or opposition from the
public due to insufficient `technical expertise' \cite{johnson_dakota_2021}.
Further, without meaningful participation from the affected public, this
approach is further entrenches procedural injustices. To credit the energy
modeling community, there is significant awareness of the importance of
transparency and repeatability in the space \cite{decarolis_case_2012,
pfenninger_energy_2014, pfenninger_openmod_nodate, forster_open_2022,
hilpert_open_2018}. Yet these two goals are challenged by the computational
resources required to run the more complex and detailed models, as well as the
learning curve necessary to understand and modify the model inputs themselves.
There has been some effort to reduce this learning curve and make modeling
itself more accessible. Frameworks such as METIS, EnergyRT, and \ac{pygen} all
emphasize reproducibility, user-friendliness, and a shallower learning curve
\cite{sakellaris_metis_2018, lugovoy_energyrt_2022, dotson_python_2021}. The
creators of METIS state their goal is to ``close the gap between modelers and
policy-makers, enabling policy-makers to become modelers''
\cite{sakellaris_metis_2018}. However, these frameworks do not offer
computational resources to run their models. The \ac{temoa} project offers
limited cloud computing capabilities, free of charge
\cite{temoa_project_temoa_2023}. However, the responsibility for creating an
input file still falls to the user, which can be overwhelming even for
experienced modelers. Finally, it's not clear that perfectly accessible and
transparent modeling tools will translate to more procedurally just
policy-making. The next section outlines one method used to address this
challenge.

\subsection{\Acl{pve}}

Even if the public could use modeling tools, their testimony may still be
dismissed due to a `lack of expertise.' However, the public has preferences that
should be incorporated into decision-making. Additionally, community members are
frequently able to assess trade-offs when presented with them. \Acf{pve} is one
method for translating community preferences into just policy outcomes.
Researchers in the Netherlands developed this method to enhance democratic
participation and infuse policies with genuine feedback from constituents
\cite{mouter_introduction_2019}. They observed that a common method of assessing
social impacts is \ac{wtp}, which is the maximum price an individual is willing
to pay for a good or service, yet individual purchasing habits do not
necessarily reflect their views on public policy due to the relative salience of
moral considerations \cite{mouter_introduction_2019}. With \ac{pve},
participants can allocate a specific amount of the public budget for certain
policies, including levying or reducing taxes for greater or lesser government
spending \cite{mouter_introduction_2019}. Researchers applied \ac{pve} in three
different settings, mobility and transportation \cite{mouter_contrasting_2021},
flood risk projects (i.e., a climate hazard \textit{infrastructure} response)
\cite{dekker_economics_2019}, and with a phaseout of natural gas
\cite{mouter_including_2021}. Importantly, the studies also measured the impact
of these interventions and found that \ac{pve} enables participation from people
that do not typically participate (recognition), the results were useful for
decision-making and participation was meaningful for the majority of subjects
\cite{mouter_including_2021}. Although previous applications of \ac{pve} focused
on economic policy levers, this approach offers a promising pathway toward
identifying equitable and just energy mixes for the future.

In summary, climate change is a multi-dimensional existential threat to society.
Transitioning to a zero-carbon economy by decarbonizing our energy systems may
prevent the worst outcomes of climate change. However, energy systems do not
only transport electrons and gas but also mediate socio-political power.
Therefore this transition must be done equitably in order to avoid entrenching
further injustices. The existing energy system modeling tools and literature
routinely ignore the social dimensions of these systems and forego true
trade-off analysis. Additionally, it's unclear whether improving these modeling
practices will correspond to just energy policy outcomes. This thesis attempts
to bridge the gap between energy system modeling and energy justice by
developing a novel framework that allows multiple, and perhaps non-economic,
objectives and is designed for transparency and usability by non-modelers to
inform energy policy decisions. A framework such as the one developed in this
thesis may be used in conjunction with a policy process like \ac{pve} to fully
enclose the triumvirate of energy justice tenets: distribution, procedure, and
recognition.
