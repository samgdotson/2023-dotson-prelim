\section{\Acl{moo}}
\label{section:moo-in-energy}
A multi-objective problem may be formulated as
\begin{align}
\label{eqn:generic-moop}
&\text{min}\{F_1(x), F_2(x), ... , F_i(x)\},
\intertext{subject to:}
&g(x, p) \leq 0.\nonumber\\
&x \in \vec{X}\nonumber
\intertext{where}
&F_i \text{ is an arbitrary objective function,}\nonumber\\
&g \text{ is a constraint,}\nonumber\\
&p \text{ is an arbitrary parameter of $g$,}\nonumber\\
&\vec{X} \text{ is the set of decision variables.}
\end{align}
\noindent
Where Equation \ref{eqn:generic_objective} had a single objective $F(x)$ to
minimize, Equation \ref{eqn:generic-moop} has a \textit{set} of objectives,
$\{F_i(x)\}$. Rather than identifying a global minimum point, the solution to
Equation \ref{eqn:generic-moop} is a \textit{set} of non-dominated points called a Pareto-front. Each point on this frontier cannot
improve one objective without making another objective worse, hence
``non-dominated.'' Generally, for competing objectives, there will be an
infeasible space that is not attainable by the given combination of objectives.
For a minimization problem, the space above the Pareto-front is the sub-optimal
feasible space. This is the space that \ac{mga} promises to search for a
corresponding single-objective problem. Figure \ref{fig:truss-pareto}
illustrates a set of solutions along a Pareto-front for an example problem from
\ac{pymoo} \cite{blank_pymoo_2020,deb_omni-optimizer_2008}.

\begin{figure}[H]
  \centering
  \resizebox{0.6\columnwidth}{!}{%% Creator: Matplotlib, PGF backend
%%
%% To include the figure in your LaTeX document, write
%%   \input{<filename>.pgf}
%%
%% Make sure the required packages are loaded in your preamble
%%   \usepackage{pgf}
%%
%% Also ensure that all the required font packages are loaded; for instance,
%% the lmodern package is sometimes necessary when using math font.
%%   \usepackage{lmodern}
%%
%% Figures using additional raster images can only be included by \input if
%% they are in the same directory as the main LaTeX file. For loading figures
%% from other directories you can use the `import` package
%%   \usepackage{import}
%%
%% and then include the figures with
%%   \import{<path to file>}{<filename>.pgf}
%%
%% Matplotlib used the following preamble
%%
\begingroup%
\makeatletter%
\begin{pgfpicture}%
\pgfpathrectangle{\pgfpointorigin}{\pgfqpoint{7.147223in}{5.232237in}}%
\pgfusepath{use as bounding box, clip}%
\begin{pgfscope}%
\pgfsetbuttcap%
\pgfsetmiterjoin%
\definecolor{currentfill}{rgb}{1.000000,1.000000,1.000000}%
\pgfsetfillcolor{currentfill}%
\pgfsetlinewidth{0.000000pt}%
\definecolor{currentstroke}{rgb}{0.000000,0.000000,0.000000}%
\pgfsetstrokecolor{currentstroke}%
\pgfsetdash{}{0pt}%
\pgfpathmoveto{\pgfqpoint{0.000000in}{0.000000in}}%
\pgfpathlineto{\pgfqpoint{7.147223in}{0.000000in}}%
\pgfpathlineto{\pgfqpoint{7.147223in}{5.232237in}}%
\pgfpathlineto{\pgfqpoint{0.000000in}{5.232237in}}%
\pgfpathlineto{\pgfqpoint{0.000000in}{0.000000in}}%
\pgfpathclose%
\pgfusepath{fill}%
\end{pgfscope}%
\begin{pgfscope}%
\pgfsetbuttcap%
\pgfsetmiterjoin%
\definecolor{currentfill}{rgb}{1.000000,1.000000,1.000000}%
\pgfsetfillcolor{currentfill}%
\pgfsetlinewidth{0.000000pt}%
\definecolor{currentstroke}{rgb}{0.000000,0.000000,0.000000}%
\pgfsetstrokecolor{currentstroke}%
\pgfsetstrokeopacity{0.000000}%
\pgfsetdash{}{0pt}%
\pgfpathmoveto{\pgfqpoint{0.847223in}{0.554012in}}%
\pgfpathlineto{\pgfqpoint{7.047223in}{0.554012in}}%
\pgfpathlineto{\pgfqpoint{7.047223in}{5.084012in}}%
\pgfpathlineto{\pgfqpoint{0.847223in}{5.084012in}}%
\pgfpathlineto{\pgfqpoint{0.847223in}{0.554012in}}%
\pgfpathclose%
\pgfusepath{fill}%
\end{pgfscope}%
\begin{pgfscope}%
\pgfpathrectangle{\pgfqpoint{0.847223in}{0.554012in}}{\pgfqpoint{6.200000in}{4.530000in}}%
\pgfusepath{clip}%
\pgfsetbuttcap%
\pgfsetroundjoin%
\pgfsetlinewidth{1.003750pt}%
\definecolor{currentstroke}{rgb}{1.000000,0.000000,0.000000}%
\pgfsetstrokecolor{currentstroke}%
\pgfsetdash{}{0pt}%
\pgfpathmoveto{\pgfqpoint{0.847223in}{5.042345in}}%
\pgfpathcurveto{\pgfqpoint{0.858273in}{5.042345in}}{\pgfqpoint{0.868872in}{5.046735in}}{\pgfqpoint{0.876686in}{5.054549in}}%
\pgfpathcurveto{\pgfqpoint{0.884499in}{5.062363in}}{\pgfqpoint{0.888890in}{5.072962in}}{\pgfqpoint{0.888890in}{5.084012in}}%
\pgfpathcurveto{\pgfqpoint{0.888890in}{5.095062in}}{\pgfqpoint{0.884499in}{5.105661in}}{\pgfqpoint{0.876686in}{5.113475in}}%
\pgfpathcurveto{\pgfqpoint{0.868872in}{5.121288in}}{\pgfqpoint{0.858273in}{5.125678in}}{\pgfqpoint{0.847223in}{5.125678in}}%
\pgfpathcurveto{\pgfqpoint{0.836173in}{5.125678in}}{\pgfqpoint{0.825574in}{5.121288in}}{\pgfqpoint{0.817760in}{5.113475in}}%
\pgfpathcurveto{\pgfqpoint{0.809947in}{5.105661in}}{\pgfqpoint{0.805556in}{5.095062in}}{\pgfqpoint{0.805556in}{5.084012in}}%
\pgfpathcurveto{\pgfqpoint{0.805556in}{5.072962in}}{\pgfqpoint{0.809947in}{5.062363in}}{\pgfqpoint{0.817760in}{5.054549in}}%
\pgfpathcurveto{\pgfqpoint{0.825574in}{5.046735in}}{\pgfqpoint{0.836173in}{5.042345in}}{\pgfqpoint{0.847223in}{5.042345in}}%
\pgfpathlineto{\pgfqpoint{0.847223in}{5.042345in}}%
\pgfpathclose%
\pgfusepath{stroke}%
\end{pgfscope}%
\begin{pgfscope}%
\pgfpathrectangle{\pgfqpoint{0.847223in}{0.554012in}}{\pgfqpoint{6.200000in}{4.530000in}}%
\pgfusepath{clip}%
\pgfsetbuttcap%
\pgfsetroundjoin%
\pgfsetlinewidth{1.003750pt}%
\definecolor{currentstroke}{rgb}{1.000000,0.000000,0.000000}%
\pgfsetstrokecolor{currentstroke}%
\pgfsetdash{}{0pt}%
\pgfpathmoveto{\pgfqpoint{0.852556in}{4.992439in}}%
\pgfpathcurveto{\pgfqpoint{0.863606in}{4.992439in}}{\pgfqpoint{0.874205in}{4.996830in}}{\pgfqpoint{0.882019in}{5.004643in}}%
\pgfpathcurveto{\pgfqpoint{0.889833in}{5.012457in}}{\pgfqpoint{0.894223in}{5.023056in}}{\pgfqpoint{0.894223in}{5.034106in}}%
\pgfpathcurveto{\pgfqpoint{0.894223in}{5.045156in}}{\pgfqpoint{0.889833in}{5.055755in}}{\pgfqpoint{0.882019in}{5.063569in}}%
\pgfpathcurveto{\pgfqpoint{0.874205in}{5.071382in}}{\pgfqpoint{0.863606in}{5.075773in}}{\pgfqpoint{0.852556in}{5.075773in}}%
\pgfpathcurveto{\pgfqpoint{0.841506in}{5.075773in}}{\pgfqpoint{0.830907in}{5.071382in}}{\pgfqpoint{0.823093in}{5.063569in}}%
\pgfpathcurveto{\pgfqpoint{0.815280in}{5.055755in}}{\pgfqpoint{0.810890in}{5.045156in}}{\pgfqpoint{0.810890in}{5.034106in}}%
\pgfpathcurveto{\pgfqpoint{0.810890in}{5.023056in}}{\pgfqpoint{0.815280in}{5.012457in}}{\pgfqpoint{0.823093in}{5.004643in}}%
\pgfpathcurveto{\pgfqpoint{0.830907in}{4.996830in}}{\pgfqpoint{0.841506in}{4.992439in}}{\pgfqpoint{0.852556in}{4.992439in}}%
\pgfpathlineto{\pgfqpoint{0.852556in}{4.992439in}}%
\pgfpathclose%
\pgfusepath{stroke}%
\end{pgfscope}%
\begin{pgfscope}%
\pgfpathrectangle{\pgfqpoint{0.847223in}{0.554012in}}{\pgfqpoint{6.200000in}{4.530000in}}%
\pgfusepath{clip}%
\pgfsetbuttcap%
\pgfsetroundjoin%
\pgfsetlinewidth{1.003750pt}%
\definecolor{currentstroke}{rgb}{1.000000,0.000000,0.000000}%
\pgfsetstrokecolor{currentstroke}%
\pgfsetdash{}{0pt}%
\pgfpathmoveto{\pgfqpoint{0.857889in}{4.943530in}}%
\pgfpathcurveto{\pgfqpoint{0.868940in}{4.943530in}}{\pgfqpoint{0.879539in}{4.947921in}}{\pgfqpoint{0.887352in}{4.955734in}}%
\pgfpathcurveto{\pgfqpoint{0.895166in}{4.963548in}}{\pgfqpoint{0.899556in}{4.974147in}}{\pgfqpoint{0.899556in}{4.985197in}}%
\pgfpathcurveto{\pgfqpoint{0.899556in}{4.996247in}}{\pgfqpoint{0.895166in}{5.006846in}}{\pgfqpoint{0.887352in}{5.014660in}}%
\pgfpathcurveto{\pgfqpoint{0.879539in}{5.022473in}}{\pgfqpoint{0.868940in}{5.026864in}}{\pgfqpoint{0.857889in}{5.026864in}}%
\pgfpathcurveto{\pgfqpoint{0.846839in}{5.026864in}}{\pgfqpoint{0.836240in}{5.022473in}}{\pgfqpoint{0.828427in}{5.014660in}}%
\pgfpathcurveto{\pgfqpoint{0.820613in}{5.006846in}}{\pgfqpoint{0.816223in}{4.996247in}}{\pgfqpoint{0.816223in}{4.985197in}}%
\pgfpathcurveto{\pgfqpoint{0.816223in}{4.974147in}}{\pgfqpoint{0.820613in}{4.963548in}}{\pgfqpoint{0.828427in}{4.955734in}}%
\pgfpathcurveto{\pgfqpoint{0.836240in}{4.947921in}}{\pgfqpoint{0.846839in}{4.943530in}}{\pgfqpoint{0.857889in}{4.943530in}}%
\pgfpathlineto{\pgfqpoint{0.857889in}{4.943530in}}%
\pgfpathclose%
\pgfusepath{stroke}%
\end{pgfscope}%
\begin{pgfscope}%
\pgfpathrectangle{\pgfqpoint{0.847223in}{0.554012in}}{\pgfqpoint{6.200000in}{4.530000in}}%
\pgfusepath{clip}%
\pgfsetbuttcap%
\pgfsetroundjoin%
\pgfsetlinewidth{1.003750pt}%
\definecolor{currentstroke}{rgb}{1.000000,0.000000,0.000000}%
\pgfsetstrokecolor{currentstroke}%
\pgfsetdash{}{0pt}%
\pgfpathmoveto{\pgfqpoint{0.863223in}{4.895588in}}%
\pgfpathcurveto{\pgfqpoint{0.874273in}{4.895588in}}{\pgfqpoint{0.884872in}{4.899979in}}{\pgfqpoint{0.892685in}{4.907792in}}%
\pgfpathcurveto{\pgfqpoint{0.900499in}{4.915606in}}{\pgfqpoint{0.904889in}{4.926205in}}{\pgfqpoint{0.904889in}{4.937255in}}%
\pgfpathcurveto{\pgfqpoint{0.904889in}{4.948305in}}{\pgfqpoint{0.900499in}{4.958904in}}{\pgfqpoint{0.892685in}{4.966718in}}%
\pgfpathcurveto{\pgfqpoint{0.884872in}{4.974532in}}{\pgfqpoint{0.874273in}{4.978922in}}{\pgfqpoint{0.863223in}{4.978922in}}%
\pgfpathcurveto{\pgfqpoint{0.852173in}{4.978922in}}{\pgfqpoint{0.841574in}{4.974532in}}{\pgfqpoint{0.833760in}{4.966718in}}%
\pgfpathcurveto{\pgfqpoint{0.825946in}{4.958904in}}{\pgfqpoint{0.821556in}{4.948305in}}{\pgfqpoint{0.821556in}{4.937255in}}%
\pgfpathcurveto{\pgfqpoint{0.821556in}{4.926205in}}{\pgfqpoint{0.825946in}{4.915606in}}{\pgfqpoint{0.833760in}{4.907792in}}%
\pgfpathcurveto{\pgfqpoint{0.841574in}{4.899979in}}{\pgfqpoint{0.852173in}{4.895588in}}{\pgfqpoint{0.863223in}{4.895588in}}%
\pgfpathlineto{\pgfqpoint{0.863223in}{4.895588in}}%
\pgfpathclose%
\pgfusepath{stroke}%
\end{pgfscope}%
\begin{pgfscope}%
\pgfpathrectangle{\pgfqpoint{0.847223in}{0.554012in}}{\pgfqpoint{6.200000in}{4.530000in}}%
\pgfusepath{clip}%
\pgfsetbuttcap%
\pgfsetroundjoin%
\pgfsetlinewidth{1.003750pt}%
\definecolor{currentstroke}{rgb}{1.000000,0.000000,0.000000}%
\pgfsetstrokecolor{currentstroke}%
\pgfsetdash{}{0pt}%
\pgfpathmoveto{\pgfqpoint{0.868556in}{4.848586in}}%
\pgfpathcurveto{\pgfqpoint{0.879606in}{4.848586in}}{\pgfqpoint{0.890205in}{4.852976in}}{\pgfqpoint{0.898019in}{4.860789in}}%
\pgfpathcurveto{\pgfqpoint{0.905832in}{4.868603in}}{\pgfqpoint{0.910223in}{4.879202in}}{\pgfqpoint{0.910223in}{4.890252in}}%
\pgfpathcurveto{\pgfqpoint{0.910223in}{4.901302in}}{\pgfqpoint{0.905832in}{4.911901in}}{\pgfqpoint{0.898019in}{4.919715in}}%
\pgfpathcurveto{\pgfqpoint{0.890205in}{4.927529in}}{\pgfqpoint{0.879606in}{4.931919in}}{\pgfqpoint{0.868556in}{4.931919in}}%
\pgfpathcurveto{\pgfqpoint{0.857506in}{4.931919in}}{\pgfqpoint{0.846907in}{4.927529in}}{\pgfqpoint{0.839093in}{4.919715in}}%
\pgfpathcurveto{\pgfqpoint{0.831280in}{4.911901in}}{\pgfqpoint{0.826889in}{4.901302in}}{\pgfqpoint{0.826889in}{4.890252in}}%
\pgfpathcurveto{\pgfqpoint{0.826889in}{4.879202in}}{\pgfqpoint{0.831280in}{4.868603in}}{\pgfqpoint{0.839093in}{4.860789in}}%
\pgfpathcurveto{\pgfqpoint{0.846907in}{4.852976in}}{\pgfqpoint{0.857506in}{4.848586in}}{\pgfqpoint{0.868556in}{4.848586in}}%
\pgfpathlineto{\pgfqpoint{0.868556in}{4.848586in}}%
\pgfpathclose%
\pgfusepath{stroke}%
\end{pgfscope}%
\begin{pgfscope}%
\pgfpathrectangle{\pgfqpoint{0.847223in}{0.554012in}}{\pgfqpoint{6.200000in}{4.530000in}}%
\pgfusepath{clip}%
\pgfsetbuttcap%
\pgfsetroundjoin%
\pgfsetlinewidth{1.003750pt}%
\definecolor{currentstroke}{rgb}{1.000000,0.000000,0.000000}%
\pgfsetstrokecolor{currentstroke}%
\pgfsetdash{}{0pt}%
\pgfpathmoveto{\pgfqpoint{0.873889in}{4.802494in}}%
\pgfpathcurveto{\pgfqpoint{0.884939in}{4.802494in}}{\pgfqpoint{0.895538in}{4.806884in}}{\pgfqpoint{0.903352in}{4.814698in}}%
\pgfpathcurveto{\pgfqpoint{0.911166in}{4.822512in}}{\pgfqpoint{0.915556in}{4.833111in}}{\pgfqpoint{0.915556in}{4.844161in}}%
\pgfpathcurveto{\pgfqpoint{0.915556in}{4.855211in}}{\pgfqpoint{0.911166in}{4.865810in}}{\pgfqpoint{0.903352in}{4.873624in}}%
\pgfpathcurveto{\pgfqpoint{0.895538in}{4.881437in}}{\pgfqpoint{0.884939in}{4.885827in}}{\pgfqpoint{0.873889in}{4.885827in}}%
\pgfpathcurveto{\pgfqpoint{0.862839in}{4.885827in}}{\pgfqpoint{0.852240in}{4.881437in}}{\pgfqpoint{0.844426in}{4.873624in}}%
\pgfpathcurveto{\pgfqpoint{0.836613in}{4.865810in}}{\pgfqpoint{0.832222in}{4.855211in}}{\pgfqpoint{0.832222in}{4.844161in}}%
\pgfpathcurveto{\pgfqpoint{0.832222in}{4.833111in}}{\pgfqpoint{0.836613in}{4.822512in}}{\pgfqpoint{0.844426in}{4.814698in}}%
\pgfpathcurveto{\pgfqpoint{0.852240in}{4.806884in}}{\pgfqpoint{0.862839in}{4.802494in}}{\pgfqpoint{0.873889in}{4.802494in}}%
\pgfpathlineto{\pgfqpoint{0.873889in}{4.802494in}}%
\pgfpathclose%
\pgfusepath{stroke}%
\end{pgfscope}%
\begin{pgfscope}%
\pgfpathrectangle{\pgfqpoint{0.847223in}{0.554012in}}{\pgfqpoint{6.200000in}{4.530000in}}%
\pgfusepath{clip}%
\pgfsetbuttcap%
\pgfsetroundjoin%
\pgfsetlinewidth{1.003750pt}%
\definecolor{currentstroke}{rgb}{1.000000,0.000000,0.000000}%
\pgfsetstrokecolor{currentstroke}%
\pgfsetdash{}{0pt}%
\pgfpathmoveto{\pgfqpoint{0.879222in}{4.757288in}}%
\pgfpathcurveto{\pgfqpoint{0.890272in}{4.757288in}}{\pgfqpoint{0.900872in}{4.761678in}}{\pgfqpoint{0.908685in}{4.769492in}}%
\pgfpathcurveto{\pgfqpoint{0.916499in}{4.777305in}}{\pgfqpoint{0.920889in}{4.787904in}}{\pgfqpoint{0.920889in}{4.798955in}}%
\pgfpathcurveto{\pgfqpoint{0.920889in}{4.810005in}}{\pgfqpoint{0.916499in}{4.820604in}}{\pgfqpoint{0.908685in}{4.828417in}}%
\pgfpathcurveto{\pgfqpoint{0.900872in}{4.836231in}}{\pgfqpoint{0.890272in}{4.840621in}}{\pgfqpoint{0.879222in}{4.840621in}}%
\pgfpathcurveto{\pgfqpoint{0.868172in}{4.840621in}}{\pgfqpoint{0.857573in}{4.836231in}}{\pgfqpoint{0.849760in}{4.828417in}}%
\pgfpathcurveto{\pgfqpoint{0.841946in}{4.820604in}}{\pgfqpoint{0.837556in}{4.810005in}}{\pgfqpoint{0.837556in}{4.798955in}}%
\pgfpathcurveto{\pgfqpoint{0.837556in}{4.787904in}}{\pgfqpoint{0.841946in}{4.777305in}}{\pgfqpoint{0.849760in}{4.769492in}}%
\pgfpathcurveto{\pgfqpoint{0.857573in}{4.761678in}}{\pgfqpoint{0.868172in}{4.757288in}}{\pgfqpoint{0.879222in}{4.757288in}}%
\pgfpathlineto{\pgfqpoint{0.879222in}{4.757288in}}%
\pgfpathclose%
\pgfusepath{stroke}%
\end{pgfscope}%
\begin{pgfscope}%
\pgfpathrectangle{\pgfqpoint{0.847223in}{0.554012in}}{\pgfqpoint{6.200000in}{4.530000in}}%
\pgfusepath{clip}%
\pgfsetbuttcap%
\pgfsetroundjoin%
\pgfsetlinewidth{1.003750pt}%
\definecolor{currentstroke}{rgb}{1.000000,0.000000,0.000000}%
\pgfsetstrokecolor{currentstroke}%
\pgfsetdash{}{0pt}%
\pgfpathmoveto{\pgfqpoint{0.884556in}{4.712942in}}%
\pgfpathcurveto{\pgfqpoint{0.895606in}{4.712942in}}{\pgfqpoint{0.906205in}{4.717332in}}{\pgfqpoint{0.914018in}{4.725146in}}%
\pgfpathcurveto{\pgfqpoint{0.921832in}{4.732959in}}{\pgfqpoint{0.926222in}{4.743558in}}{\pgfqpoint{0.926222in}{4.754608in}}%
\pgfpathcurveto{\pgfqpoint{0.926222in}{4.765659in}}{\pgfqpoint{0.921832in}{4.776258in}}{\pgfqpoint{0.914018in}{4.784071in}}%
\pgfpathcurveto{\pgfqpoint{0.906205in}{4.791885in}}{\pgfqpoint{0.895606in}{4.796275in}}{\pgfqpoint{0.884556in}{4.796275in}}%
\pgfpathcurveto{\pgfqpoint{0.873505in}{4.796275in}}{\pgfqpoint{0.862906in}{4.791885in}}{\pgfqpoint{0.855093in}{4.784071in}}%
\pgfpathcurveto{\pgfqpoint{0.847279in}{4.776258in}}{\pgfqpoint{0.842889in}{4.765659in}}{\pgfqpoint{0.842889in}{4.754608in}}%
\pgfpathcurveto{\pgfqpoint{0.842889in}{4.743558in}}{\pgfqpoint{0.847279in}{4.732959in}}{\pgfqpoint{0.855093in}{4.725146in}}%
\pgfpathcurveto{\pgfqpoint{0.862906in}{4.717332in}}{\pgfqpoint{0.873505in}{4.712942in}}{\pgfqpoint{0.884556in}{4.712942in}}%
\pgfpathlineto{\pgfqpoint{0.884556in}{4.712942in}}%
\pgfpathclose%
\pgfusepath{stroke}%
\end{pgfscope}%
\begin{pgfscope}%
\pgfpathrectangle{\pgfqpoint{0.847223in}{0.554012in}}{\pgfqpoint{6.200000in}{4.530000in}}%
\pgfusepath{clip}%
\pgfsetbuttcap%
\pgfsetroundjoin%
\pgfsetlinewidth{1.003750pt}%
\definecolor{currentstroke}{rgb}{1.000000,0.000000,0.000000}%
\pgfsetstrokecolor{currentstroke}%
\pgfsetdash{}{0pt}%
\pgfpathmoveto{\pgfqpoint{0.889889in}{4.669431in}}%
\pgfpathcurveto{\pgfqpoint{0.900939in}{4.669431in}}{\pgfqpoint{0.911538in}{4.673822in}}{\pgfqpoint{0.919352in}{4.681635in}}%
\pgfpathcurveto{\pgfqpoint{0.927165in}{4.689449in}}{\pgfqpoint{0.931555in}{4.700048in}}{\pgfqpoint{0.931555in}{4.711098in}}%
\pgfpathcurveto{\pgfqpoint{0.931555in}{4.722148in}}{\pgfqpoint{0.927165in}{4.732747in}}{\pgfqpoint{0.919352in}{4.740561in}}%
\pgfpathcurveto{\pgfqpoint{0.911538in}{4.748374in}}{\pgfqpoint{0.900939in}{4.752765in}}{\pgfqpoint{0.889889in}{4.752765in}}%
\pgfpathcurveto{\pgfqpoint{0.878839in}{4.752765in}}{\pgfqpoint{0.868240in}{4.748374in}}{\pgfqpoint{0.860426in}{4.740561in}}%
\pgfpathcurveto{\pgfqpoint{0.852612in}{4.732747in}}{\pgfqpoint{0.848222in}{4.722148in}}{\pgfqpoint{0.848222in}{4.711098in}}%
\pgfpathcurveto{\pgfqpoint{0.848222in}{4.700048in}}{\pgfqpoint{0.852612in}{4.689449in}}{\pgfqpoint{0.860426in}{4.681635in}}%
\pgfpathcurveto{\pgfqpoint{0.868240in}{4.673822in}}{\pgfqpoint{0.878839in}{4.669431in}}{\pgfqpoint{0.889889in}{4.669431in}}%
\pgfpathlineto{\pgfqpoint{0.889889in}{4.669431in}}%
\pgfpathclose%
\pgfusepath{stroke}%
\end{pgfscope}%
\begin{pgfscope}%
\pgfpathrectangle{\pgfqpoint{0.847223in}{0.554012in}}{\pgfqpoint{6.200000in}{4.530000in}}%
\pgfusepath{clip}%
\pgfsetbuttcap%
\pgfsetroundjoin%
\pgfsetlinewidth{1.003750pt}%
\definecolor{currentstroke}{rgb}{1.000000,0.000000,0.000000}%
\pgfsetstrokecolor{currentstroke}%
\pgfsetdash{}{0pt}%
\pgfpathmoveto{\pgfqpoint{0.895222in}{4.626733in}}%
\pgfpathcurveto{\pgfqpoint{0.906272in}{4.626733in}}{\pgfqpoint{0.916871in}{4.631123in}}{\pgfqpoint{0.924685in}{4.638937in}}%
\pgfpathcurveto{\pgfqpoint{0.932498in}{4.646751in}}{\pgfqpoint{0.936889in}{4.657350in}}{\pgfqpoint{0.936889in}{4.668400in}}%
\pgfpathcurveto{\pgfqpoint{0.936889in}{4.679450in}}{\pgfqpoint{0.932498in}{4.690049in}}{\pgfqpoint{0.924685in}{4.697863in}}%
\pgfpathcurveto{\pgfqpoint{0.916871in}{4.705676in}}{\pgfqpoint{0.906272in}{4.710066in}}{\pgfqpoint{0.895222in}{4.710066in}}%
\pgfpathcurveto{\pgfqpoint{0.884172in}{4.710066in}}{\pgfqpoint{0.873573in}{4.705676in}}{\pgfqpoint{0.865759in}{4.697863in}}%
\pgfpathcurveto{\pgfqpoint{0.857946in}{4.690049in}}{\pgfqpoint{0.853555in}{4.679450in}}{\pgfqpoint{0.853555in}{4.668400in}}%
\pgfpathcurveto{\pgfqpoint{0.853555in}{4.657350in}}{\pgfqpoint{0.857946in}{4.646751in}}{\pgfqpoint{0.865759in}{4.638937in}}%
\pgfpathcurveto{\pgfqpoint{0.873573in}{4.631123in}}{\pgfqpoint{0.884172in}{4.626733in}}{\pgfqpoint{0.895222in}{4.626733in}}%
\pgfpathlineto{\pgfqpoint{0.895222in}{4.626733in}}%
\pgfpathclose%
\pgfusepath{stroke}%
\end{pgfscope}%
\begin{pgfscope}%
\pgfpathrectangle{\pgfqpoint{0.847223in}{0.554012in}}{\pgfqpoint{6.200000in}{4.530000in}}%
\pgfusepath{clip}%
\pgfsetbuttcap%
\pgfsetroundjoin%
\pgfsetlinewidth{1.003750pt}%
\definecolor{currentstroke}{rgb}{1.000000,0.000000,0.000000}%
\pgfsetstrokecolor{currentstroke}%
\pgfsetdash{}{0pt}%
\pgfpathmoveto{\pgfqpoint{0.900555in}{4.584825in}}%
\pgfpathcurveto{\pgfqpoint{0.911605in}{4.584825in}}{\pgfqpoint{0.922204in}{4.589215in}}{\pgfqpoint{0.930018in}{4.597029in}}%
\pgfpathcurveto{\pgfqpoint{0.937832in}{4.604842in}}{\pgfqpoint{0.942222in}{4.615441in}}{\pgfqpoint{0.942222in}{4.626491in}}%
\pgfpathcurveto{\pgfqpoint{0.942222in}{4.637541in}}{\pgfqpoint{0.937832in}{4.648141in}}{\pgfqpoint{0.930018in}{4.655954in}}%
\pgfpathcurveto{\pgfqpoint{0.922204in}{4.663768in}}{\pgfqpoint{0.911605in}{4.668158in}}{\pgfqpoint{0.900555in}{4.668158in}}%
\pgfpathcurveto{\pgfqpoint{0.889505in}{4.668158in}}{\pgfqpoint{0.878906in}{4.663768in}}{\pgfqpoint{0.871092in}{4.655954in}}%
\pgfpathcurveto{\pgfqpoint{0.863279in}{4.648141in}}{\pgfqpoint{0.858889in}{4.637541in}}{\pgfqpoint{0.858889in}{4.626491in}}%
\pgfpathcurveto{\pgfqpoint{0.858889in}{4.615441in}}{\pgfqpoint{0.863279in}{4.604842in}}{\pgfqpoint{0.871092in}{4.597029in}}%
\pgfpathcurveto{\pgfqpoint{0.878906in}{4.589215in}}{\pgfqpoint{0.889505in}{4.584825in}}{\pgfqpoint{0.900555in}{4.584825in}}%
\pgfpathlineto{\pgfqpoint{0.900555in}{4.584825in}}%
\pgfpathclose%
\pgfusepath{stroke}%
\end{pgfscope}%
\begin{pgfscope}%
\pgfpathrectangle{\pgfqpoint{0.847223in}{0.554012in}}{\pgfqpoint{6.200000in}{4.530000in}}%
\pgfusepath{clip}%
\pgfsetbuttcap%
\pgfsetroundjoin%
\pgfsetlinewidth{1.003750pt}%
\definecolor{currentstroke}{rgb}{1.000000,0.000000,0.000000}%
\pgfsetstrokecolor{currentstroke}%
\pgfsetdash{}{0pt}%
\pgfpathmoveto{\pgfqpoint{0.905888in}{4.543684in}}%
\pgfpathcurveto{\pgfqpoint{0.916939in}{4.543684in}}{\pgfqpoint{0.927538in}{4.548075in}}{\pgfqpoint{0.935351in}{4.555888in}}%
\pgfpathcurveto{\pgfqpoint{0.943165in}{4.563702in}}{\pgfqpoint{0.947555in}{4.574301in}}{\pgfqpoint{0.947555in}{4.585351in}}%
\pgfpathcurveto{\pgfqpoint{0.947555in}{4.596401in}}{\pgfqpoint{0.943165in}{4.607000in}}{\pgfqpoint{0.935351in}{4.614814in}}%
\pgfpathcurveto{\pgfqpoint{0.927538in}{4.622627in}}{\pgfqpoint{0.916939in}{4.627018in}}{\pgfqpoint{0.905888in}{4.627018in}}%
\pgfpathcurveto{\pgfqpoint{0.894838in}{4.627018in}}{\pgfqpoint{0.884239in}{4.622627in}}{\pgfqpoint{0.876426in}{4.614814in}}%
\pgfpathcurveto{\pgfqpoint{0.868612in}{4.607000in}}{\pgfqpoint{0.864222in}{4.596401in}}{\pgfqpoint{0.864222in}{4.585351in}}%
\pgfpathcurveto{\pgfqpoint{0.864222in}{4.574301in}}{\pgfqpoint{0.868612in}{4.563702in}}{\pgfqpoint{0.876426in}{4.555888in}}%
\pgfpathcurveto{\pgfqpoint{0.884239in}{4.548075in}}{\pgfqpoint{0.894838in}{4.543684in}}{\pgfqpoint{0.905888in}{4.543684in}}%
\pgfpathlineto{\pgfqpoint{0.905888in}{4.543684in}}%
\pgfpathclose%
\pgfusepath{stroke}%
\end{pgfscope}%
\begin{pgfscope}%
\pgfpathrectangle{\pgfqpoint{0.847223in}{0.554012in}}{\pgfqpoint{6.200000in}{4.530000in}}%
\pgfusepath{clip}%
\pgfsetbuttcap%
\pgfsetroundjoin%
\pgfsetlinewidth{1.003750pt}%
\definecolor{currentstroke}{rgb}{1.000000,0.000000,0.000000}%
\pgfsetstrokecolor{currentstroke}%
\pgfsetdash{}{0pt}%
\pgfpathmoveto{\pgfqpoint{0.911222in}{4.503291in}}%
\pgfpathcurveto{\pgfqpoint{0.922272in}{4.503291in}}{\pgfqpoint{0.932871in}{4.507681in}}{\pgfqpoint{0.940684in}{4.515495in}}%
\pgfpathcurveto{\pgfqpoint{0.948498in}{4.523309in}}{\pgfqpoint{0.952888in}{4.533908in}}{\pgfqpoint{0.952888in}{4.544958in}}%
\pgfpathcurveto{\pgfqpoint{0.952888in}{4.556008in}}{\pgfqpoint{0.948498in}{4.566607in}}{\pgfqpoint{0.940684in}{4.574421in}}%
\pgfpathcurveto{\pgfqpoint{0.932871in}{4.582234in}}{\pgfqpoint{0.922272in}{4.586624in}}{\pgfqpoint{0.911222in}{4.586624in}}%
\pgfpathcurveto{\pgfqpoint{0.900172in}{4.586624in}}{\pgfqpoint{0.889572in}{4.582234in}}{\pgfqpoint{0.881759in}{4.574421in}}%
\pgfpathcurveto{\pgfqpoint{0.873945in}{4.566607in}}{\pgfqpoint{0.869555in}{4.556008in}}{\pgfqpoint{0.869555in}{4.544958in}}%
\pgfpathcurveto{\pgfqpoint{0.869555in}{4.533908in}}{\pgfqpoint{0.873945in}{4.523309in}}{\pgfqpoint{0.881759in}{4.515495in}}%
\pgfpathcurveto{\pgfqpoint{0.889572in}{4.507681in}}{\pgfqpoint{0.900172in}{4.503291in}}{\pgfqpoint{0.911222in}{4.503291in}}%
\pgfpathlineto{\pgfqpoint{0.911222in}{4.503291in}}%
\pgfpathclose%
\pgfusepath{stroke}%
\end{pgfscope}%
\begin{pgfscope}%
\pgfpathrectangle{\pgfqpoint{0.847223in}{0.554012in}}{\pgfqpoint{6.200000in}{4.530000in}}%
\pgfusepath{clip}%
\pgfsetbuttcap%
\pgfsetroundjoin%
\pgfsetlinewidth{1.003750pt}%
\definecolor{currentstroke}{rgb}{1.000000,0.000000,0.000000}%
\pgfsetstrokecolor{currentstroke}%
\pgfsetdash{}{0pt}%
\pgfpathmoveto{\pgfqpoint{0.916555in}{4.463625in}}%
\pgfpathcurveto{\pgfqpoint{0.927605in}{4.463625in}}{\pgfqpoint{0.938204in}{4.468015in}}{\pgfqpoint{0.946018in}{4.475829in}}%
\pgfpathcurveto{\pgfqpoint{0.953831in}{4.483642in}}{\pgfqpoint{0.958222in}{4.494241in}}{\pgfqpoint{0.958222in}{4.505291in}}%
\pgfpathcurveto{\pgfqpoint{0.958222in}{4.516342in}}{\pgfqpoint{0.953831in}{4.526941in}}{\pgfqpoint{0.946018in}{4.534754in}}%
\pgfpathcurveto{\pgfqpoint{0.938204in}{4.542568in}}{\pgfqpoint{0.927605in}{4.546958in}}{\pgfqpoint{0.916555in}{4.546958in}}%
\pgfpathcurveto{\pgfqpoint{0.905505in}{4.546958in}}{\pgfqpoint{0.894906in}{4.542568in}}{\pgfqpoint{0.887092in}{4.534754in}}%
\pgfpathcurveto{\pgfqpoint{0.879278in}{4.526941in}}{\pgfqpoint{0.874888in}{4.516342in}}{\pgfqpoint{0.874888in}{4.505291in}}%
\pgfpathcurveto{\pgfqpoint{0.874888in}{4.494241in}}{\pgfqpoint{0.879278in}{4.483642in}}{\pgfqpoint{0.887092in}{4.475829in}}%
\pgfpathcurveto{\pgfqpoint{0.894906in}{4.468015in}}{\pgfqpoint{0.905505in}{4.463625in}}{\pgfqpoint{0.916555in}{4.463625in}}%
\pgfpathlineto{\pgfqpoint{0.916555in}{4.463625in}}%
\pgfpathclose%
\pgfusepath{stroke}%
\end{pgfscope}%
\begin{pgfscope}%
\pgfpathrectangle{\pgfqpoint{0.847223in}{0.554012in}}{\pgfqpoint{6.200000in}{4.530000in}}%
\pgfusepath{clip}%
\pgfsetbuttcap%
\pgfsetroundjoin%
\pgfsetlinewidth{1.003750pt}%
\definecolor{currentstroke}{rgb}{1.000000,0.000000,0.000000}%
\pgfsetstrokecolor{currentstroke}%
\pgfsetdash{}{0pt}%
\pgfpathmoveto{\pgfqpoint{0.921888in}{4.424666in}}%
\pgfpathcurveto{\pgfqpoint{0.932938in}{4.424666in}}{\pgfqpoint{0.943537in}{4.429056in}}{\pgfqpoint{0.951351in}{4.436870in}}%
\pgfpathcurveto{\pgfqpoint{0.959164in}{4.444683in}}{\pgfqpoint{0.963555in}{4.455282in}}{\pgfqpoint{0.963555in}{4.466333in}}%
\pgfpathcurveto{\pgfqpoint{0.963555in}{4.477383in}}{\pgfqpoint{0.959164in}{4.487982in}}{\pgfqpoint{0.951351in}{4.495795in}}%
\pgfpathcurveto{\pgfqpoint{0.943537in}{4.503609in}}{\pgfqpoint{0.932938in}{4.507999in}}{\pgfqpoint{0.921888in}{4.507999in}}%
\pgfpathcurveto{\pgfqpoint{0.910838in}{4.507999in}}{\pgfqpoint{0.900239in}{4.503609in}}{\pgfqpoint{0.892425in}{4.495795in}}%
\pgfpathcurveto{\pgfqpoint{0.884612in}{4.487982in}}{\pgfqpoint{0.880221in}{4.477383in}}{\pgfqpoint{0.880221in}{4.466333in}}%
\pgfpathcurveto{\pgfqpoint{0.880221in}{4.455282in}}{\pgfqpoint{0.884612in}{4.444683in}}{\pgfqpoint{0.892425in}{4.436870in}}%
\pgfpathcurveto{\pgfqpoint{0.900239in}{4.429056in}}{\pgfqpoint{0.910838in}{4.424666in}}{\pgfqpoint{0.921888in}{4.424666in}}%
\pgfpathlineto{\pgfqpoint{0.921888in}{4.424666in}}%
\pgfpathclose%
\pgfusepath{stroke}%
\end{pgfscope}%
\begin{pgfscope}%
\pgfpathrectangle{\pgfqpoint{0.847223in}{0.554012in}}{\pgfqpoint{6.200000in}{4.530000in}}%
\pgfusepath{clip}%
\pgfsetbuttcap%
\pgfsetroundjoin%
\pgfsetlinewidth{1.003750pt}%
\definecolor{currentstroke}{rgb}{1.000000,0.000000,0.000000}%
\pgfsetstrokecolor{currentstroke}%
\pgfsetdash{}{0pt}%
\pgfpathmoveto{\pgfqpoint{0.927221in}{4.386396in}}%
\pgfpathcurveto{\pgfqpoint{0.938271in}{4.386396in}}{\pgfqpoint{0.948870in}{4.390786in}}{\pgfqpoint{0.956684in}{4.398600in}}%
\pgfpathcurveto{\pgfqpoint{0.964498in}{4.406413in}}{\pgfqpoint{0.968888in}{4.417012in}}{\pgfqpoint{0.968888in}{4.428063in}}%
\pgfpathcurveto{\pgfqpoint{0.968888in}{4.439113in}}{\pgfqpoint{0.964498in}{4.449712in}}{\pgfqpoint{0.956684in}{4.457525in}}%
\pgfpathcurveto{\pgfqpoint{0.948870in}{4.465339in}}{\pgfqpoint{0.938271in}{4.469729in}}{\pgfqpoint{0.927221in}{4.469729in}}%
\pgfpathcurveto{\pgfqpoint{0.916171in}{4.469729in}}{\pgfqpoint{0.905572in}{4.465339in}}{\pgfqpoint{0.897759in}{4.457525in}}%
\pgfpathcurveto{\pgfqpoint{0.889945in}{4.449712in}}{\pgfqpoint{0.885555in}{4.439113in}}{\pgfqpoint{0.885555in}{4.428063in}}%
\pgfpathcurveto{\pgfqpoint{0.885555in}{4.417012in}}{\pgfqpoint{0.889945in}{4.406413in}}{\pgfqpoint{0.897759in}{4.398600in}}%
\pgfpathcurveto{\pgfqpoint{0.905572in}{4.390786in}}{\pgfqpoint{0.916171in}{4.386396in}}{\pgfqpoint{0.927221in}{4.386396in}}%
\pgfpathlineto{\pgfqpoint{0.927221in}{4.386396in}}%
\pgfpathclose%
\pgfusepath{stroke}%
\end{pgfscope}%
\begin{pgfscope}%
\pgfpathrectangle{\pgfqpoint{0.847223in}{0.554012in}}{\pgfqpoint{6.200000in}{4.530000in}}%
\pgfusepath{clip}%
\pgfsetbuttcap%
\pgfsetroundjoin%
\pgfsetlinewidth{1.003750pt}%
\definecolor{currentstroke}{rgb}{1.000000,0.000000,0.000000}%
\pgfsetstrokecolor{currentstroke}%
\pgfsetdash{}{0pt}%
\pgfpathmoveto{\pgfqpoint{0.932555in}{4.348796in}}%
\pgfpathcurveto{\pgfqpoint{0.943605in}{4.348796in}}{\pgfqpoint{0.954204in}{4.353187in}}{\pgfqpoint{0.962017in}{4.361000in}}%
\pgfpathcurveto{\pgfqpoint{0.969831in}{4.368814in}}{\pgfqpoint{0.974221in}{4.379413in}}{\pgfqpoint{0.974221in}{4.390463in}}%
\pgfpathcurveto{\pgfqpoint{0.974221in}{4.401513in}}{\pgfqpoint{0.969831in}{4.412112in}}{\pgfqpoint{0.962017in}{4.419926in}}%
\pgfpathcurveto{\pgfqpoint{0.954204in}{4.427740in}}{\pgfqpoint{0.943605in}{4.432130in}}{\pgfqpoint{0.932555in}{4.432130in}}%
\pgfpathcurveto{\pgfqpoint{0.921504in}{4.432130in}}{\pgfqpoint{0.910905in}{4.427740in}}{\pgfqpoint{0.903092in}{4.419926in}}%
\pgfpathcurveto{\pgfqpoint{0.895278in}{4.412112in}}{\pgfqpoint{0.890888in}{4.401513in}}{\pgfqpoint{0.890888in}{4.390463in}}%
\pgfpathcurveto{\pgfqpoint{0.890888in}{4.379413in}}{\pgfqpoint{0.895278in}{4.368814in}}{\pgfqpoint{0.903092in}{4.361000in}}%
\pgfpathcurveto{\pgfqpoint{0.910905in}{4.353187in}}{\pgfqpoint{0.921504in}{4.348796in}}{\pgfqpoint{0.932555in}{4.348796in}}%
\pgfpathlineto{\pgfqpoint{0.932555in}{4.348796in}}%
\pgfpathclose%
\pgfusepath{stroke}%
\end{pgfscope}%
\begin{pgfscope}%
\pgfpathrectangle{\pgfqpoint{0.847223in}{0.554012in}}{\pgfqpoint{6.200000in}{4.530000in}}%
\pgfusepath{clip}%
\pgfsetbuttcap%
\pgfsetroundjoin%
\pgfsetlinewidth{1.003750pt}%
\definecolor{currentstroke}{rgb}{1.000000,0.000000,0.000000}%
\pgfsetstrokecolor{currentstroke}%
\pgfsetdash{}{0pt}%
\pgfpathmoveto{\pgfqpoint{0.937888in}{4.311850in}}%
\pgfpathcurveto{\pgfqpoint{0.948938in}{4.311850in}}{\pgfqpoint{0.959537in}{4.316240in}}{\pgfqpoint{0.967351in}{4.324054in}}%
\pgfpathcurveto{\pgfqpoint{0.975164in}{4.331868in}}{\pgfqpoint{0.979554in}{4.342467in}}{\pgfqpoint{0.979554in}{4.353517in}}%
\pgfpathcurveto{\pgfqpoint{0.979554in}{4.364567in}}{\pgfqpoint{0.975164in}{4.375166in}}{\pgfqpoint{0.967351in}{4.382980in}}%
\pgfpathcurveto{\pgfqpoint{0.959537in}{4.390793in}}{\pgfqpoint{0.948938in}{4.395184in}}{\pgfqpoint{0.937888in}{4.395184in}}%
\pgfpathcurveto{\pgfqpoint{0.926838in}{4.395184in}}{\pgfqpoint{0.916239in}{4.390793in}}{\pgfqpoint{0.908425in}{4.382980in}}%
\pgfpathcurveto{\pgfqpoint{0.900611in}{4.375166in}}{\pgfqpoint{0.896221in}{4.364567in}}{\pgfqpoint{0.896221in}{4.353517in}}%
\pgfpathcurveto{\pgfqpoint{0.896221in}{4.342467in}}{\pgfqpoint{0.900611in}{4.331868in}}{\pgfqpoint{0.908425in}{4.324054in}}%
\pgfpathcurveto{\pgfqpoint{0.916239in}{4.316240in}}{\pgfqpoint{0.926838in}{4.311850in}}{\pgfqpoint{0.937888in}{4.311850in}}%
\pgfpathlineto{\pgfqpoint{0.937888in}{4.311850in}}%
\pgfpathclose%
\pgfusepath{stroke}%
\end{pgfscope}%
\begin{pgfscope}%
\pgfpathrectangle{\pgfqpoint{0.847223in}{0.554012in}}{\pgfqpoint{6.200000in}{4.530000in}}%
\pgfusepath{clip}%
\pgfsetbuttcap%
\pgfsetroundjoin%
\pgfsetlinewidth{1.003750pt}%
\definecolor{currentstroke}{rgb}{1.000000,0.000000,0.000000}%
\pgfsetstrokecolor{currentstroke}%
\pgfsetdash{}{0pt}%
\pgfpathmoveto{\pgfqpoint{0.943221in}{4.275540in}}%
\pgfpathcurveto{\pgfqpoint{0.954271in}{4.275540in}}{\pgfqpoint{0.964870in}{4.279931in}}{\pgfqpoint{0.972684in}{4.287744in}}%
\pgfpathcurveto{\pgfqpoint{0.980497in}{4.295558in}}{\pgfqpoint{0.984888in}{4.306157in}}{\pgfqpoint{0.984888in}{4.317207in}}%
\pgfpathcurveto{\pgfqpoint{0.984888in}{4.328257in}}{\pgfqpoint{0.980497in}{4.338856in}}{\pgfqpoint{0.972684in}{4.346670in}}%
\pgfpathcurveto{\pgfqpoint{0.964870in}{4.354483in}}{\pgfqpoint{0.954271in}{4.358874in}}{\pgfqpoint{0.943221in}{4.358874in}}%
\pgfpathcurveto{\pgfqpoint{0.932171in}{4.358874in}}{\pgfqpoint{0.921572in}{4.354483in}}{\pgfqpoint{0.913758in}{4.346670in}}%
\pgfpathcurveto{\pgfqpoint{0.905945in}{4.338856in}}{\pgfqpoint{0.901554in}{4.328257in}}{\pgfqpoint{0.901554in}{4.317207in}}%
\pgfpathcurveto{\pgfqpoint{0.901554in}{4.306157in}}{\pgfqpoint{0.905945in}{4.295558in}}{\pgfqpoint{0.913758in}{4.287744in}}%
\pgfpathcurveto{\pgfqpoint{0.921572in}{4.279931in}}{\pgfqpoint{0.932171in}{4.275540in}}{\pgfqpoint{0.943221in}{4.275540in}}%
\pgfpathlineto{\pgfqpoint{0.943221in}{4.275540in}}%
\pgfpathclose%
\pgfusepath{stroke}%
\end{pgfscope}%
\begin{pgfscope}%
\pgfpathrectangle{\pgfqpoint{0.847223in}{0.554012in}}{\pgfqpoint{6.200000in}{4.530000in}}%
\pgfusepath{clip}%
\pgfsetbuttcap%
\pgfsetroundjoin%
\pgfsetlinewidth{1.003750pt}%
\definecolor{currentstroke}{rgb}{1.000000,0.000000,0.000000}%
\pgfsetstrokecolor{currentstroke}%
\pgfsetdash{}{0pt}%
\pgfpathmoveto{\pgfqpoint{0.948554in}{4.239850in}}%
\pgfpathcurveto{\pgfqpoint{0.959604in}{4.239850in}}{\pgfqpoint{0.970203in}{4.244241in}}{\pgfqpoint{0.978017in}{4.252054in}}%
\pgfpathcurveto{\pgfqpoint{0.985831in}{4.259868in}}{\pgfqpoint{0.990221in}{4.270467in}}{\pgfqpoint{0.990221in}{4.281517in}}%
\pgfpathcurveto{\pgfqpoint{0.990221in}{4.292567in}}{\pgfqpoint{0.985831in}{4.303166in}}{\pgfqpoint{0.978017in}{4.310980in}}%
\pgfpathcurveto{\pgfqpoint{0.970203in}{4.318793in}}{\pgfqpoint{0.959604in}{4.323184in}}{\pgfqpoint{0.948554in}{4.323184in}}%
\pgfpathcurveto{\pgfqpoint{0.937504in}{4.323184in}}{\pgfqpoint{0.926905in}{4.318793in}}{\pgfqpoint{0.919091in}{4.310980in}}%
\pgfpathcurveto{\pgfqpoint{0.911278in}{4.303166in}}{\pgfqpoint{0.906887in}{4.292567in}}{\pgfqpoint{0.906887in}{4.281517in}}%
\pgfpathcurveto{\pgfqpoint{0.906887in}{4.270467in}}{\pgfqpoint{0.911278in}{4.259868in}}{\pgfqpoint{0.919091in}{4.252054in}}%
\pgfpathcurveto{\pgfqpoint{0.926905in}{4.244241in}}{\pgfqpoint{0.937504in}{4.239850in}}{\pgfqpoint{0.948554in}{4.239850in}}%
\pgfpathlineto{\pgfqpoint{0.948554in}{4.239850in}}%
\pgfpathclose%
\pgfusepath{stroke}%
\end{pgfscope}%
\begin{pgfscope}%
\pgfpathrectangle{\pgfqpoint{0.847223in}{0.554012in}}{\pgfqpoint{6.200000in}{4.530000in}}%
\pgfusepath{clip}%
\pgfsetbuttcap%
\pgfsetroundjoin%
\pgfsetlinewidth{1.003750pt}%
\definecolor{currentstroke}{rgb}{1.000000,0.000000,0.000000}%
\pgfsetstrokecolor{currentstroke}%
\pgfsetdash{}{0pt}%
\pgfpathmoveto{\pgfqpoint{0.953887in}{4.204765in}}%
\pgfpathcurveto{\pgfqpoint{0.964937in}{4.204765in}}{\pgfqpoint{0.975537in}{4.209155in}}{\pgfqpoint{0.983350in}{4.216968in}}%
\pgfpathcurveto{\pgfqpoint{0.991164in}{4.224782in}}{\pgfqpoint{0.995554in}{4.235381in}}{\pgfqpoint{0.995554in}{4.246431in}}%
\pgfpathcurveto{\pgfqpoint{0.995554in}{4.257481in}}{\pgfqpoint{0.991164in}{4.268080in}}{\pgfqpoint{0.983350in}{4.275894in}}%
\pgfpathcurveto{\pgfqpoint{0.975537in}{4.283708in}}{\pgfqpoint{0.964937in}{4.288098in}}{\pgfqpoint{0.953887in}{4.288098in}}%
\pgfpathcurveto{\pgfqpoint{0.942837in}{4.288098in}}{\pgfqpoint{0.932238in}{4.283708in}}{\pgfqpoint{0.924425in}{4.275894in}}%
\pgfpathcurveto{\pgfqpoint{0.916611in}{4.268080in}}{\pgfqpoint{0.912221in}{4.257481in}}{\pgfqpoint{0.912221in}{4.246431in}}%
\pgfpathcurveto{\pgfqpoint{0.912221in}{4.235381in}}{\pgfqpoint{0.916611in}{4.224782in}}{\pgfqpoint{0.924425in}{4.216968in}}%
\pgfpathcurveto{\pgfqpoint{0.932238in}{4.209155in}}{\pgfqpoint{0.942837in}{4.204765in}}{\pgfqpoint{0.953887in}{4.204765in}}%
\pgfpathlineto{\pgfqpoint{0.953887in}{4.204765in}}%
\pgfpathclose%
\pgfusepath{stroke}%
\end{pgfscope}%
\begin{pgfscope}%
\pgfpathrectangle{\pgfqpoint{0.847223in}{0.554012in}}{\pgfqpoint{6.200000in}{4.530000in}}%
\pgfusepath{clip}%
\pgfsetbuttcap%
\pgfsetroundjoin%
\pgfsetlinewidth{1.003750pt}%
\definecolor{currentstroke}{rgb}{1.000000,0.000000,0.000000}%
\pgfsetstrokecolor{currentstroke}%
\pgfsetdash{}{0pt}%
\pgfpathmoveto{\pgfqpoint{0.959221in}{4.170268in}}%
\pgfpathcurveto{\pgfqpoint{0.970271in}{4.170268in}}{\pgfqpoint{0.980870in}{4.174658in}}{\pgfqpoint{0.988683in}{4.182472in}}%
\pgfpathcurveto{\pgfqpoint{0.996497in}{4.190285in}}{\pgfqpoint{1.000887in}{4.200884in}}{\pgfqpoint{1.000887in}{4.211935in}}%
\pgfpathcurveto{\pgfqpoint{1.000887in}{4.222985in}}{\pgfqpoint{0.996497in}{4.233584in}}{\pgfqpoint{0.988683in}{4.241397in}}%
\pgfpathcurveto{\pgfqpoint{0.980870in}{4.249211in}}{\pgfqpoint{0.970271in}{4.253601in}}{\pgfqpoint{0.959221in}{4.253601in}}%
\pgfpathcurveto{\pgfqpoint{0.948170in}{4.253601in}}{\pgfqpoint{0.937571in}{4.249211in}}{\pgfqpoint{0.929758in}{4.241397in}}%
\pgfpathcurveto{\pgfqpoint{0.921944in}{4.233584in}}{\pgfqpoint{0.917554in}{4.222985in}}{\pgfqpoint{0.917554in}{4.211935in}}%
\pgfpathcurveto{\pgfqpoint{0.917554in}{4.200884in}}{\pgfqpoint{0.921944in}{4.190285in}}{\pgfqpoint{0.929758in}{4.182472in}}%
\pgfpathcurveto{\pgfqpoint{0.937571in}{4.174658in}}{\pgfqpoint{0.948170in}{4.170268in}}{\pgfqpoint{0.959221in}{4.170268in}}%
\pgfpathlineto{\pgfqpoint{0.959221in}{4.170268in}}%
\pgfpathclose%
\pgfusepath{stroke}%
\end{pgfscope}%
\begin{pgfscope}%
\pgfpathrectangle{\pgfqpoint{0.847223in}{0.554012in}}{\pgfqpoint{6.200000in}{4.530000in}}%
\pgfusepath{clip}%
\pgfsetbuttcap%
\pgfsetroundjoin%
\pgfsetlinewidth{1.003750pt}%
\definecolor{currentstroke}{rgb}{1.000000,0.000000,0.000000}%
\pgfsetstrokecolor{currentstroke}%
\pgfsetdash{}{0pt}%
\pgfpathmoveto{\pgfqpoint{0.964554in}{4.136345in}}%
\pgfpathcurveto{\pgfqpoint{0.975604in}{4.136345in}}{\pgfqpoint{0.986203in}{4.140736in}}{\pgfqpoint{0.994017in}{4.148549in}}%
\pgfpathcurveto{\pgfqpoint{1.001830in}{4.156363in}}{\pgfqpoint{1.006220in}{4.166962in}}{\pgfqpoint{1.006220in}{4.178012in}}%
\pgfpathcurveto{\pgfqpoint{1.006220in}{4.189062in}}{\pgfqpoint{1.001830in}{4.199661in}}{\pgfqpoint{0.994017in}{4.207475in}}%
\pgfpathcurveto{\pgfqpoint{0.986203in}{4.215289in}}{\pgfqpoint{0.975604in}{4.219679in}}{\pgfqpoint{0.964554in}{4.219679in}}%
\pgfpathcurveto{\pgfqpoint{0.953504in}{4.219679in}}{\pgfqpoint{0.942905in}{4.215289in}}{\pgfqpoint{0.935091in}{4.207475in}}%
\pgfpathcurveto{\pgfqpoint{0.927277in}{4.199661in}}{\pgfqpoint{0.922887in}{4.189062in}}{\pgfqpoint{0.922887in}{4.178012in}}%
\pgfpathcurveto{\pgfqpoint{0.922887in}{4.166962in}}{\pgfqpoint{0.927277in}{4.156363in}}{\pgfqpoint{0.935091in}{4.148549in}}%
\pgfpathcurveto{\pgfqpoint{0.942905in}{4.140736in}}{\pgfqpoint{0.953504in}{4.136345in}}{\pgfqpoint{0.964554in}{4.136345in}}%
\pgfpathlineto{\pgfqpoint{0.964554in}{4.136345in}}%
\pgfpathclose%
\pgfusepath{stroke}%
\end{pgfscope}%
\begin{pgfscope}%
\pgfpathrectangle{\pgfqpoint{0.847223in}{0.554012in}}{\pgfqpoint{6.200000in}{4.530000in}}%
\pgfusepath{clip}%
\pgfsetbuttcap%
\pgfsetroundjoin%
\pgfsetlinewidth{1.003750pt}%
\definecolor{currentstroke}{rgb}{1.000000,0.000000,0.000000}%
\pgfsetstrokecolor{currentstroke}%
\pgfsetdash{}{0pt}%
\pgfpathmoveto{\pgfqpoint{0.969887in}{4.102983in}}%
\pgfpathcurveto{\pgfqpoint{0.980937in}{4.102983in}}{\pgfqpoint{0.991536in}{4.107373in}}{\pgfqpoint{0.999350in}{4.115187in}}%
\pgfpathcurveto{\pgfqpoint{1.007163in}{4.123001in}}{\pgfqpoint{1.011554in}{4.133600in}}{\pgfqpoint{1.011554in}{4.144650in}}%
\pgfpathcurveto{\pgfqpoint{1.011554in}{4.155700in}}{\pgfqpoint{1.007163in}{4.166299in}}{\pgfqpoint{0.999350in}{4.174113in}}%
\pgfpathcurveto{\pgfqpoint{0.991536in}{4.181926in}}{\pgfqpoint{0.980937in}{4.186317in}}{\pgfqpoint{0.969887in}{4.186317in}}%
\pgfpathcurveto{\pgfqpoint{0.958837in}{4.186317in}}{\pgfqpoint{0.948238in}{4.181926in}}{\pgfqpoint{0.940424in}{4.174113in}}%
\pgfpathcurveto{\pgfqpoint{0.932611in}{4.166299in}}{\pgfqpoint{0.928220in}{4.155700in}}{\pgfqpoint{0.928220in}{4.144650in}}%
\pgfpathcurveto{\pgfqpoint{0.928220in}{4.133600in}}{\pgfqpoint{0.932611in}{4.123001in}}{\pgfqpoint{0.940424in}{4.115187in}}%
\pgfpathcurveto{\pgfqpoint{0.948238in}{4.107373in}}{\pgfqpoint{0.958837in}{4.102983in}}{\pgfqpoint{0.969887in}{4.102983in}}%
\pgfpathlineto{\pgfqpoint{0.969887in}{4.102983in}}%
\pgfpathclose%
\pgfusepath{stroke}%
\end{pgfscope}%
\begin{pgfscope}%
\pgfpathrectangle{\pgfqpoint{0.847223in}{0.554012in}}{\pgfqpoint{6.200000in}{4.530000in}}%
\pgfusepath{clip}%
\pgfsetbuttcap%
\pgfsetroundjoin%
\pgfsetlinewidth{1.003750pt}%
\definecolor{currentstroke}{rgb}{1.000000,0.000000,0.000000}%
\pgfsetstrokecolor{currentstroke}%
\pgfsetdash{}{0pt}%
\pgfpathmoveto{\pgfqpoint{0.975220in}{4.070167in}}%
\pgfpathcurveto{\pgfqpoint{0.986270in}{4.070167in}}{\pgfqpoint{0.996869in}{4.074558in}}{\pgfqpoint{1.004683in}{4.082371in}}%
\pgfpathcurveto{\pgfqpoint{1.012497in}{4.090185in}}{\pgfqpoint{1.016887in}{4.100784in}}{\pgfqpoint{1.016887in}{4.111834in}}%
\pgfpathcurveto{\pgfqpoint{1.016887in}{4.122884in}}{\pgfqpoint{1.012497in}{4.133483in}}{\pgfqpoint{1.004683in}{4.141297in}}%
\pgfpathcurveto{\pgfqpoint{0.996869in}{4.149110in}}{\pgfqpoint{0.986270in}{4.153501in}}{\pgfqpoint{0.975220in}{4.153501in}}%
\pgfpathcurveto{\pgfqpoint{0.964170in}{4.153501in}}{\pgfqpoint{0.953571in}{4.149110in}}{\pgfqpoint{0.945757in}{4.141297in}}%
\pgfpathcurveto{\pgfqpoint{0.937944in}{4.133483in}}{\pgfqpoint{0.933554in}{4.122884in}}{\pgfqpoint{0.933554in}{4.111834in}}%
\pgfpathcurveto{\pgfqpoint{0.933554in}{4.100784in}}{\pgfqpoint{0.937944in}{4.090185in}}{\pgfqpoint{0.945757in}{4.082371in}}%
\pgfpathcurveto{\pgfqpoint{0.953571in}{4.074558in}}{\pgfqpoint{0.964170in}{4.070167in}}{\pgfqpoint{0.975220in}{4.070167in}}%
\pgfpathlineto{\pgfqpoint{0.975220in}{4.070167in}}%
\pgfpathclose%
\pgfusepath{stroke}%
\end{pgfscope}%
\begin{pgfscope}%
\pgfpathrectangle{\pgfqpoint{0.847223in}{0.554012in}}{\pgfqpoint{6.200000in}{4.530000in}}%
\pgfusepath{clip}%
\pgfsetbuttcap%
\pgfsetroundjoin%
\pgfsetlinewidth{1.003750pt}%
\definecolor{currentstroke}{rgb}{1.000000,0.000000,0.000000}%
\pgfsetstrokecolor{currentstroke}%
\pgfsetdash{}{0pt}%
\pgfpathmoveto{\pgfqpoint{0.980553in}{4.037884in}}%
\pgfpathcurveto{\pgfqpoint{0.991604in}{4.037884in}}{\pgfqpoint{1.002203in}{4.042275in}}{\pgfqpoint{1.010016in}{4.050088in}}%
\pgfpathcurveto{\pgfqpoint{1.017830in}{4.057902in}}{\pgfqpoint{1.022220in}{4.068501in}}{\pgfqpoint{1.022220in}{4.079551in}}%
\pgfpathcurveto{\pgfqpoint{1.022220in}{4.090601in}}{\pgfqpoint{1.017830in}{4.101200in}}{\pgfqpoint{1.010016in}{4.109014in}}%
\pgfpathcurveto{\pgfqpoint{1.002203in}{4.116827in}}{\pgfqpoint{0.991604in}{4.121218in}}{\pgfqpoint{0.980553in}{4.121218in}}%
\pgfpathcurveto{\pgfqpoint{0.969503in}{4.121218in}}{\pgfqpoint{0.958904in}{4.116827in}}{\pgfqpoint{0.951091in}{4.109014in}}%
\pgfpathcurveto{\pgfqpoint{0.943277in}{4.101200in}}{\pgfqpoint{0.938887in}{4.090601in}}{\pgfqpoint{0.938887in}{4.079551in}}%
\pgfpathcurveto{\pgfqpoint{0.938887in}{4.068501in}}{\pgfqpoint{0.943277in}{4.057902in}}{\pgfqpoint{0.951091in}{4.050088in}}%
\pgfpathcurveto{\pgfqpoint{0.958904in}{4.042275in}}{\pgfqpoint{0.969503in}{4.037884in}}{\pgfqpoint{0.980553in}{4.037884in}}%
\pgfpathlineto{\pgfqpoint{0.980553in}{4.037884in}}%
\pgfpathclose%
\pgfusepath{stroke}%
\end{pgfscope}%
\begin{pgfscope}%
\pgfpathrectangle{\pgfqpoint{0.847223in}{0.554012in}}{\pgfqpoint{6.200000in}{4.530000in}}%
\pgfusepath{clip}%
\pgfsetbuttcap%
\pgfsetroundjoin%
\pgfsetlinewidth{1.003750pt}%
\definecolor{currentstroke}{rgb}{1.000000,0.000000,0.000000}%
\pgfsetstrokecolor{currentstroke}%
\pgfsetdash{}{0pt}%
\pgfpathmoveto{\pgfqpoint{0.985887in}{4.006122in}}%
\pgfpathcurveto{\pgfqpoint{0.996937in}{4.006122in}}{\pgfqpoint{1.007536in}{4.010512in}}{\pgfqpoint{1.015349in}{4.018325in}}%
\pgfpathcurveto{\pgfqpoint{1.023163in}{4.026139in}}{\pgfqpoint{1.027553in}{4.036738in}}{\pgfqpoint{1.027553in}{4.047788in}}%
\pgfpathcurveto{\pgfqpoint{1.027553in}{4.058838in}}{\pgfqpoint{1.023163in}{4.069437in}}{\pgfqpoint{1.015349in}{4.077251in}}%
\pgfpathcurveto{\pgfqpoint{1.007536in}{4.085065in}}{\pgfqpoint{0.996937in}{4.089455in}}{\pgfqpoint{0.985887in}{4.089455in}}%
\pgfpathcurveto{\pgfqpoint{0.974837in}{4.089455in}}{\pgfqpoint{0.964238in}{4.085065in}}{\pgfqpoint{0.956424in}{4.077251in}}%
\pgfpathcurveto{\pgfqpoint{0.948610in}{4.069437in}}{\pgfqpoint{0.944220in}{4.058838in}}{\pgfqpoint{0.944220in}{4.047788in}}%
\pgfpathcurveto{\pgfqpoint{0.944220in}{4.036738in}}{\pgfqpoint{0.948610in}{4.026139in}}{\pgfqpoint{0.956424in}{4.018325in}}%
\pgfpathcurveto{\pgfqpoint{0.964238in}{4.010512in}}{\pgfqpoint{0.974837in}{4.006122in}}{\pgfqpoint{0.985887in}{4.006122in}}%
\pgfpathlineto{\pgfqpoint{0.985887in}{4.006122in}}%
\pgfpathclose%
\pgfusepath{stroke}%
\end{pgfscope}%
\begin{pgfscope}%
\pgfpathrectangle{\pgfqpoint{0.847223in}{0.554012in}}{\pgfqpoint{6.200000in}{4.530000in}}%
\pgfusepath{clip}%
\pgfsetbuttcap%
\pgfsetroundjoin%
\pgfsetlinewidth{1.003750pt}%
\definecolor{currentstroke}{rgb}{1.000000,0.000000,0.000000}%
\pgfsetstrokecolor{currentstroke}%
\pgfsetdash{}{0pt}%
\pgfpathmoveto{\pgfqpoint{0.991220in}{3.974866in}}%
\pgfpathcurveto{\pgfqpoint{1.002270in}{3.974866in}}{\pgfqpoint{1.012869in}{3.979257in}}{\pgfqpoint{1.020683in}{3.987070in}}%
\pgfpathcurveto{\pgfqpoint{1.028496in}{3.994884in}}{\pgfqpoint{1.032887in}{4.005483in}}{\pgfqpoint{1.032887in}{4.016533in}}%
\pgfpathcurveto{\pgfqpoint{1.032887in}{4.027583in}}{\pgfqpoint{1.028496in}{4.038182in}}{\pgfqpoint{1.020683in}{4.045996in}}%
\pgfpathcurveto{\pgfqpoint{1.012869in}{4.053810in}}{\pgfqpoint{1.002270in}{4.058200in}}{\pgfqpoint{0.991220in}{4.058200in}}%
\pgfpathcurveto{\pgfqpoint{0.980170in}{4.058200in}}{\pgfqpoint{0.969571in}{4.053810in}}{\pgfqpoint{0.961757in}{4.045996in}}%
\pgfpathcurveto{\pgfqpoint{0.953943in}{4.038182in}}{\pgfqpoint{0.949553in}{4.027583in}}{\pgfqpoint{0.949553in}{4.016533in}}%
\pgfpathcurveto{\pgfqpoint{0.949553in}{4.005483in}}{\pgfqpoint{0.953943in}{3.994884in}}{\pgfqpoint{0.961757in}{3.987070in}}%
\pgfpathcurveto{\pgfqpoint{0.969571in}{3.979257in}}{\pgfqpoint{0.980170in}{3.974866in}}{\pgfqpoint{0.991220in}{3.974866in}}%
\pgfpathlineto{\pgfqpoint{0.991220in}{3.974866in}}%
\pgfpathclose%
\pgfusepath{stroke}%
\end{pgfscope}%
\begin{pgfscope}%
\pgfpathrectangle{\pgfqpoint{0.847223in}{0.554012in}}{\pgfqpoint{6.200000in}{4.530000in}}%
\pgfusepath{clip}%
\pgfsetbuttcap%
\pgfsetroundjoin%
\pgfsetlinewidth{1.003750pt}%
\definecolor{currentstroke}{rgb}{1.000000,0.000000,0.000000}%
\pgfsetstrokecolor{currentstroke}%
\pgfsetdash{}{0pt}%
\pgfpathmoveto{\pgfqpoint{0.996553in}{3.944107in}}%
\pgfpathcurveto{\pgfqpoint{1.007603in}{3.944107in}}{\pgfqpoint{1.018202in}{3.948497in}}{\pgfqpoint{1.026016in}{3.956311in}}%
\pgfpathcurveto{\pgfqpoint{1.033829in}{3.964125in}}{\pgfqpoint{1.038220in}{3.974724in}}{\pgfqpoint{1.038220in}{3.985774in}}%
\pgfpathcurveto{\pgfqpoint{1.038220in}{3.996824in}}{\pgfqpoint{1.033829in}{4.007423in}}{\pgfqpoint{1.026016in}{4.015236in}}%
\pgfpathcurveto{\pgfqpoint{1.018202in}{4.023050in}}{\pgfqpoint{1.007603in}{4.027440in}}{\pgfqpoint{0.996553in}{4.027440in}}%
\pgfpathcurveto{\pgfqpoint{0.985503in}{4.027440in}}{\pgfqpoint{0.974904in}{4.023050in}}{\pgfqpoint{0.967090in}{4.015236in}}%
\pgfpathcurveto{\pgfqpoint{0.959277in}{4.007423in}}{\pgfqpoint{0.954886in}{3.996824in}}{\pgfqpoint{0.954886in}{3.985774in}}%
\pgfpathcurveto{\pgfqpoint{0.954886in}{3.974724in}}{\pgfqpoint{0.959277in}{3.964125in}}{\pgfqpoint{0.967090in}{3.956311in}}%
\pgfpathcurveto{\pgfqpoint{0.974904in}{3.948497in}}{\pgfqpoint{0.985503in}{3.944107in}}{\pgfqpoint{0.996553in}{3.944107in}}%
\pgfpathlineto{\pgfqpoint{0.996553in}{3.944107in}}%
\pgfpathclose%
\pgfusepath{stroke}%
\end{pgfscope}%
\begin{pgfscope}%
\pgfpathrectangle{\pgfqpoint{0.847223in}{0.554012in}}{\pgfqpoint{6.200000in}{4.530000in}}%
\pgfusepath{clip}%
\pgfsetbuttcap%
\pgfsetroundjoin%
\pgfsetlinewidth{1.003750pt}%
\definecolor{currentstroke}{rgb}{1.000000,0.000000,0.000000}%
\pgfsetstrokecolor{currentstroke}%
\pgfsetdash{}{0pt}%
\pgfpathmoveto{\pgfqpoint{1.001886in}{3.913831in}}%
\pgfpathcurveto{\pgfqpoint{1.012936in}{3.913831in}}{\pgfqpoint{1.023535in}{3.918222in}}{\pgfqpoint{1.031349in}{3.926035in}}%
\pgfpathcurveto{\pgfqpoint{1.039163in}{3.933849in}}{\pgfqpoint{1.043553in}{3.944448in}}{\pgfqpoint{1.043553in}{3.955498in}}%
\pgfpathcurveto{\pgfqpoint{1.043553in}{3.966548in}}{\pgfqpoint{1.039163in}{3.977147in}}{\pgfqpoint{1.031349in}{3.984961in}}%
\pgfpathcurveto{\pgfqpoint{1.023535in}{3.992774in}}{\pgfqpoint{1.012936in}{3.997165in}}{\pgfqpoint{1.001886in}{3.997165in}}%
\pgfpathcurveto{\pgfqpoint{0.990836in}{3.997165in}}{\pgfqpoint{0.980237in}{3.992774in}}{\pgfqpoint{0.972424in}{3.984961in}}%
\pgfpathcurveto{\pgfqpoint{0.964610in}{3.977147in}}{\pgfqpoint{0.960220in}{3.966548in}}{\pgfqpoint{0.960220in}{3.955498in}}%
\pgfpathcurveto{\pgfqpoint{0.960220in}{3.944448in}}{\pgfqpoint{0.964610in}{3.933849in}}{\pgfqpoint{0.972424in}{3.926035in}}%
\pgfpathcurveto{\pgfqpoint{0.980237in}{3.918222in}}{\pgfqpoint{0.990836in}{3.913831in}}{\pgfqpoint{1.001886in}{3.913831in}}%
\pgfpathlineto{\pgfqpoint{1.001886in}{3.913831in}}%
\pgfpathclose%
\pgfusepath{stroke}%
\end{pgfscope}%
\begin{pgfscope}%
\pgfpathrectangle{\pgfqpoint{0.847223in}{0.554012in}}{\pgfqpoint{6.200000in}{4.530000in}}%
\pgfusepath{clip}%
\pgfsetbuttcap%
\pgfsetroundjoin%
\pgfsetlinewidth{1.003750pt}%
\definecolor{currentstroke}{rgb}{1.000000,0.000000,0.000000}%
\pgfsetstrokecolor{currentstroke}%
\pgfsetdash{}{0pt}%
\pgfpathmoveto{\pgfqpoint{1.007220in}{3.884028in}}%
\pgfpathcurveto{\pgfqpoint{1.018270in}{3.884028in}}{\pgfqpoint{1.028869in}{3.888419in}}{\pgfqpoint{1.036682in}{3.896232in}}%
\pgfpathcurveto{\pgfqpoint{1.044496in}{3.904046in}}{\pgfqpoint{1.048886in}{3.914645in}}{\pgfqpoint{1.048886in}{3.925695in}}%
\pgfpathcurveto{\pgfqpoint{1.048886in}{3.936745in}}{\pgfqpoint{1.044496in}{3.947344in}}{\pgfqpoint{1.036682in}{3.955158in}}%
\pgfpathcurveto{\pgfqpoint{1.028869in}{3.962971in}}{\pgfqpoint{1.018270in}{3.967362in}}{\pgfqpoint{1.007220in}{3.967362in}}%
\pgfpathcurveto{\pgfqpoint{0.996169in}{3.967362in}}{\pgfqpoint{0.985570in}{3.962971in}}{\pgfqpoint{0.977757in}{3.955158in}}%
\pgfpathcurveto{\pgfqpoint{0.969943in}{3.947344in}}{\pgfqpoint{0.965553in}{3.936745in}}{\pgfqpoint{0.965553in}{3.925695in}}%
\pgfpathcurveto{\pgfqpoint{0.965553in}{3.914645in}}{\pgfqpoint{0.969943in}{3.904046in}}{\pgfqpoint{0.977757in}{3.896232in}}%
\pgfpathcurveto{\pgfqpoint{0.985570in}{3.888419in}}{\pgfqpoint{0.996169in}{3.884028in}}{\pgfqpoint{1.007220in}{3.884028in}}%
\pgfpathlineto{\pgfqpoint{1.007220in}{3.884028in}}%
\pgfpathclose%
\pgfusepath{stroke}%
\end{pgfscope}%
\begin{pgfscope}%
\pgfpathrectangle{\pgfqpoint{0.847223in}{0.554012in}}{\pgfqpoint{6.200000in}{4.530000in}}%
\pgfusepath{clip}%
\pgfsetbuttcap%
\pgfsetroundjoin%
\pgfsetlinewidth{1.003750pt}%
\definecolor{currentstroke}{rgb}{1.000000,0.000000,0.000000}%
\pgfsetstrokecolor{currentstroke}%
\pgfsetdash{}{0pt}%
\pgfpathmoveto{\pgfqpoint{1.012553in}{3.854687in}}%
\pgfpathcurveto{\pgfqpoint{1.023603in}{3.854687in}}{\pgfqpoint{1.034202in}{3.859077in}}{\pgfqpoint{1.042016in}{3.866891in}}%
\pgfpathcurveto{\pgfqpoint{1.049829in}{3.874704in}}{\pgfqpoint{1.054219in}{3.885304in}}{\pgfqpoint{1.054219in}{3.896354in}}%
\pgfpathcurveto{\pgfqpoint{1.054219in}{3.907404in}}{\pgfqpoint{1.049829in}{3.918003in}}{\pgfqpoint{1.042016in}{3.925816in}}%
\pgfpathcurveto{\pgfqpoint{1.034202in}{3.933630in}}{\pgfqpoint{1.023603in}{3.938020in}}{\pgfqpoint{1.012553in}{3.938020in}}%
\pgfpathcurveto{\pgfqpoint{1.001503in}{3.938020in}}{\pgfqpoint{0.990904in}{3.933630in}}{\pgfqpoint{0.983090in}{3.925816in}}%
\pgfpathcurveto{\pgfqpoint{0.975276in}{3.918003in}}{\pgfqpoint{0.970886in}{3.907404in}}{\pgfqpoint{0.970886in}{3.896354in}}%
\pgfpathcurveto{\pgfqpoint{0.970886in}{3.885304in}}{\pgfqpoint{0.975276in}{3.874704in}}{\pgfqpoint{0.983090in}{3.866891in}}%
\pgfpathcurveto{\pgfqpoint{0.990904in}{3.859077in}}{\pgfqpoint{1.001503in}{3.854687in}}{\pgfqpoint{1.012553in}{3.854687in}}%
\pgfpathlineto{\pgfqpoint{1.012553in}{3.854687in}}%
\pgfpathclose%
\pgfusepath{stroke}%
\end{pgfscope}%
\begin{pgfscope}%
\pgfpathrectangle{\pgfqpoint{0.847223in}{0.554012in}}{\pgfqpoint{6.200000in}{4.530000in}}%
\pgfusepath{clip}%
\pgfsetbuttcap%
\pgfsetroundjoin%
\pgfsetlinewidth{1.003750pt}%
\definecolor{currentstroke}{rgb}{1.000000,0.000000,0.000000}%
\pgfsetstrokecolor{currentstroke}%
\pgfsetdash{}{0pt}%
\pgfpathmoveto{\pgfqpoint{1.017886in}{3.825797in}}%
\pgfpathcurveto{\pgfqpoint{1.028936in}{3.825797in}}{\pgfqpoint{1.039535in}{3.830187in}}{\pgfqpoint{1.047349in}{3.838000in}}%
\pgfpathcurveto{\pgfqpoint{1.055162in}{3.845814in}}{\pgfqpoint{1.059553in}{3.856413in}}{\pgfqpoint{1.059553in}{3.867463in}}%
\pgfpathcurveto{\pgfqpoint{1.059553in}{3.878513in}}{\pgfqpoint{1.055162in}{3.889112in}}{\pgfqpoint{1.047349in}{3.896926in}}%
\pgfpathcurveto{\pgfqpoint{1.039535in}{3.904740in}}{\pgfqpoint{1.028936in}{3.909130in}}{\pgfqpoint{1.017886in}{3.909130in}}%
\pgfpathcurveto{\pgfqpoint{1.006836in}{3.909130in}}{\pgfqpoint{0.996237in}{3.904740in}}{\pgfqpoint{0.988423in}{3.896926in}}%
\pgfpathcurveto{\pgfqpoint{0.980610in}{3.889112in}}{\pgfqpoint{0.976219in}{3.878513in}}{\pgfqpoint{0.976219in}{3.867463in}}%
\pgfpathcurveto{\pgfqpoint{0.976219in}{3.856413in}}{\pgfqpoint{0.980610in}{3.845814in}}{\pgfqpoint{0.988423in}{3.838000in}}%
\pgfpathcurveto{\pgfqpoint{0.996237in}{3.830187in}}{\pgfqpoint{1.006836in}{3.825797in}}{\pgfqpoint{1.017886in}{3.825797in}}%
\pgfpathlineto{\pgfqpoint{1.017886in}{3.825797in}}%
\pgfpathclose%
\pgfusepath{stroke}%
\end{pgfscope}%
\begin{pgfscope}%
\pgfpathrectangle{\pgfqpoint{0.847223in}{0.554012in}}{\pgfqpoint{6.200000in}{4.530000in}}%
\pgfusepath{clip}%
\pgfsetbuttcap%
\pgfsetroundjoin%
\pgfsetlinewidth{1.003750pt}%
\definecolor{currentstroke}{rgb}{1.000000,0.000000,0.000000}%
\pgfsetstrokecolor{currentstroke}%
\pgfsetdash{}{0pt}%
\pgfpathmoveto{\pgfqpoint{1.023219in}{3.797347in}}%
\pgfpathcurveto{\pgfqpoint{1.034269in}{3.797347in}}{\pgfqpoint{1.044868in}{3.801737in}}{\pgfqpoint{1.052682in}{3.809551in}}%
\pgfpathcurveto{\pgfqpoint{1.060496in}{3.817364in}}{\pgfqpoint{1.064886in}{3.827963in}}{\pgfqpoint{1.064886in}{3.839013in}}%
\pgfpathcurveto{\pgfqpoint{1.064886in}{3.850063in}}{\pgfqpoint{1.060496in}{3.860663in}}{\pgfqpoint{1.052682in}{3.868476in}}%
\pgfpathcurveto{\pgfqpoint{1.044868in}{3.876290in}}{\pgfqpoint{1.034269in}{3.880680in}}{\pgfqpoint{1.023219in}{3.880680in}}%
\pgfpathcurveto{\pgfqpoint{1.012169in}{3.880680in}}{\pgfqpoint{1.001570in}{3.876290in}}{\pgfqpoint{0.993756in}{3.868476in}}%
\pgfpathcurveto{\pgfqpoint{0.985943in}{3.860663in}}{\pgfqpoint{0.981553in}{3.850063in}}{\pgfqpoint{0.981553in}{3.839013in}}%
\pgfpathcurveto{\pgfqpoint{0.981553in}{3.827963in}}{\pgfqpoint{0.985943in}{3.817364in}}{\pgfqpoint{0.993756in}{3.809551in}}%
\pgfpathcurveto{\pgfqpoint{1.001570in}{3.801737in}}{\pgfqpoint{1.012169in}{3.797347in}}{\pgfqpoint{1.023219in}{3.797347in}}%
\pgfpathlineto{\pgfqpoint{1.023219in}{3.797347in}}%
\pgfpathclose%
\pgfusepath{stroke}%
\end{pgfscope}%
\begin{pgfscope}%
\pgfpathrectangle{\pgfqpoint{0.847223in}{0.554012in}}{\pgfqpoint{6.200000in}{4.530000in}}%
\pgfusepath{clip}%
\pgfsetbuttcap%
\pgfsetroundjoin%
\pgfsetlinewidth{1.003750pt}%
\definecolor{currentstroke}{rgb}{1.000000,0.000000,0.000000}%
\pgfsetstrokecolor{currentstroke}%
\pgfsetdash{}{0pt}%
\pgfpathmoveto{\pgfqpoint{1.028552in}{3.769328in}}%
\pgfpathcurveto{\pgfqpoint{1.039603in}{3.769328in}}{\pgfqpoint{1.050202in}{3.773718in}}{\pgfqpoint{1.058015in}{3.781531in}}%
\pgfpathcurveto{\pgfqpoint{1.065829in}{3.789345in}}{\pgfqpoint{1.070219in}{3.799944in}}{\pgfqpoint{1.070219in}{3.810994in}}%
\pgfpathcurveto{\pgfqpoint{1.070219in}{3.822044in}}{\pgfqpoint{1.065829in}{3.832643in}}{\pgfqpoint{1.058015in}{3.840457in}}%
\pgfpathcurveto{\pgfqpoint{1.050202in}{3.848271in}}{\pgfqpoint{1.039603in}{3.852661in}}{\pgfqpoint{1.028552in}{3.852661in}}%
\pgfpathcurveto{\pgfqpoint{1.017502in}{3.852661in}}{\pgfqpoint{1.006903in}{3.848271in}}{\pgfqpoint{0.999090in}{3.840457in}}%
\pgfpathcurveto{\pgfqpoint{0.991276in}{3.832643in}}{\pgfqpoint{0.986886in}{3.822044in}}{\pgfqpoint{0.986886in}{3.810994in}}%
\pgfpathcurveto{\pgfqpoint{0.986886in}{3.799944in}}{\pgfqpoint{0.991276in}{3.789345in}}{\pgfqpoint{0.999090in}{3.781531in}}%
\pgfpathcurveto{\pgfqpoint{1.006903in}{3.773718in}}{\pgfqpoint{1.017502in}{3.769328in}}{\pgfqpoint{1.028552in}{3.769328in}}%
\pgfpathlineto{\pgfqpoint{1.028552in}{3.769328in}}%
\pgfpathclose%
\pgfusepath{stroke}%
\end{pgfscope}%
\begin{pgfscope}%
\pgfpathrectangle{\pgfqpoint{0.847223in}{0.554012in}}{\pgfqpoint{6.200000in}{4.530000in}}%
\pgfusepath{clip}%
\pgfsetbuttcap%
\pgfsetroundjoin%
\pgfsetlinewidth{1.003750pt}%
\definecolor{currentstroke}{rgb}{1.000000,0.000000,0.000000}%
\pgfsetstrokecolor{currentstroke}%
\pgfsetdash{}{0pt}%
\pgfpathmoveto{\pgfqpoint{1.033886in}{3.741729in}}%
\pgfpathcurveto{\pgfqpoint{1.044936in}{3.741729in}}{\pgfqpoint{1.055535in}{3.746120in}}{\pgfqpoint{1.063348in}{3.753933in}}%
\pgfpathcurveto{\pgfqpoint{1.071162in}{3.761747in}}{\pgfqpoint{1.075552in}{3.772346in}}{\pgfqpoint{1.075552in}{3.783396in}}%
\pgfpathcurveto{\pgfqpoint{1.075552in}{3.794446in}}{\pgfqpoint{1.071162in}{3.805045in}}{\pgfqpoint{1.063348in}{3.812859in}}%
\pgfpathcurveto{\pgfqpoint{1.055535in}{3.820672in}}{\pgfqpoint{1.044936in}{3.825063in}}{\pgfqpoint{1.033886in}{3.825063in}}%
\pgfpathcurveto{\pgfqpoint{1.022835in}{3.825063in}}{\pgfqpoint{1.012236in}{3.820672in}}{\pgfqpoint{1.004423in}{3.812859in}}%
\pgfpathcurveto{\pgfqpoint{0.996609in}{3.805045in}}{\pgfqpoint{0.992219in}{3.794446in}}{\pgfqpoint{0.992219in}{3.783396in}}%
\pgfpathcurveto{\pgfqpoint{0.992219in}{3.772346in}}{\pgfqpoint{0.996609in}{3.761747in}}{\pgfqpoint{1.004423in}{3.753933in}}%
\pgfpathcurveto{\pgfqpoint{1.012236in}{3.746120in}}{\pgfqpoint{1.022835in}{3.741729in}}{\pgfqpoint{1.033886in}{3.741729in}}%
\pgfpathlineto{\pgfqpoint{1.033886in}{3.741729in}}%
\pgfpathclose%
\pgfusepath{stroke}%
\end{pgfscope}%
\begin{pgfscope}%
\pgfpathrectangle{\pgfqpoint{0.847223in}{0.554012in}}{\pgfqpoint{6.200000in}{4.530000in}}%
\pgfusepath{clip}%
\pgfsetbuttcap%
\pgfsetroundjoin%
\pgfsetlinewidth{1.003750pt}%
\definecolor{currentstroke}{rgb}{1.000000,0.000000,0.000000}%
\pgfsetstrokecolor{currentstroke}%
\pgfsetdash{}{0pt}%
\pgfpathmoveto{\pgfqpoint{1.039219in}{3.714542in}}%
\pgfpathcurveto{\pgfqpoint{1.050269in}{3.714542in}}{\pgfqpoint{1.060868in}{3.718933in}}{\pgfqpoint{1.068682in}{3.726746in}}%
\pgfpathcurveto{\pgfqpoint{1.076495in}{3.734560in}}{\pgfqpoint{1.080885in}{3.745159in}}{\pgfqpoint{1.080885in}{3.756209in}}%
\pgfpathcurveto{\pgfqpoint{1.080885in}{3.767259in}}{\pgfqpoint{1.076495in}{3.777858in}}{\pgfqpoint{1.068682in}{3.785672in}}%
\pgfpathcurveto{\pgfqpoint{1.060868in}{3.793486in}}{\pgfqpoint{1.050269in}{3.797876in}}{\pgfqpoint{1.039219in}{3.797876in}}%
\pgfpathcurveto{\pgfqpoint{1.028169in}{3.797876in}}{\pgfqpoint{1.017570in}{3.793486in}}{\pgfqpoint{1.009756in}{3.785672in}}%
\pgfpathcurveto{\pgfqpoint{1.001942in}{3.777858in}}{\pgfqpoint{0.997552in}{3.767259in}}{\pgfqpoint{0.997552in}{3.756209in}}%
\pgfpathcurveto{\pgfqpoint{0.997552in}{3.745159in}}{\pgfqpoint{1.001942in}{3.734560in}}{\pgfqpoint{1.009756in}{3.726746in}}%
\pgfpathcurveto{\pgfqpoint{1.017570in}{3.718933in}}{\pgfqpoint{1.028169in}{3.714542in}}{\pgfqpoint{1.039219in}{3.714542in}}%
\pgfpathlineto{\pgfqpoint{1.039219in}{3.714542in}}%
\pgfpathclose%
\pgfusepath{stroke}%
\end{pgfscope}%
\begin{pgfscope}%
\pgfpathrectangle{\pgfqpoint{0.847223in}{0.554012in}}{\pgfqpoint{6.200000in}{4.530000in}}%
\pgfusepath{clip}%
\pgfsetbuttcap%
\pgfsetroundjoin%
\pgfsetlinewidth{1.003750pt}%
\definecolor{currentstroke}{rgb}{1.000000,0.000000,0.000000}%
\pgfsetstrokecolor{currentstroke}%
\pgfsetdash{}{0pt}%
\pgfpathmoveto{\pgfqpoint{1.044552in}{3.687758in}}%
\pgfpathcurveto{\pgfqpoint{1.055602in}{3.687758in}}{\pgfqpoint{1.066201in}{3.692148in}}{\pgfqpoint{1.074015in}{3.699962in}}%
\pgfpathcurveto{\pgfqpoint{1.081828in}{3.707776in}}{\pgfqpoint{1.086219in}{3.718375in}}{\pgfqpoint{1.086219in}{3.729425in}}%
\pgfpathcurveto{\pgfqpoint{1.086219in}{3.740475in}}{\pgfqpoint{1.081828in}{3.751074in}}{\pgfqpoint{1.074015in}{3.758888in}}%
\pgfpathcurveto{\pgfqpoint{1.066201in}{3.766701in}}{\pgfqpoint{1.055602in}{3.771091in}}{\pgfqpoint{1.044552in}{3.771091in}}%
\pgfpathcurveto{\pgfqpoint{1.033502in}{3.771091in}}{\pgfqpoint{1.022903in}{3.766701in}}{\pgfqpoint{1.015089in}{3.758888in}}%
\pgfpathcurveto{\pgfqpoint{1.007276in}{3.751074in}}{\pgfqpoint{1.002885in}{3.740475in}}{\pgfqpoint{1.002885in}{3.729425in}}%
\pgfpathcurveto{\pgfqpoint{1.002885in}{3.718375in}}{\pgfqpoint{1.007276in}{3.707776in}}{\pgfqpoint{1.015089in}{3.699962in}}%
\pgfpathcurveto{\pgfqpoint{1.022903in}{3.692148in}}{\pgfqpoint{1.033502in}{3.687758in}}{\pgfqpoint{1.044552in}{3.687758in}}%
\pgfpathlineto{\pgfqpoint{1.044552in}{3.687758in}}%
\pgfpathclose%
\pgfusepath{stroke}%
\end{pgfscope}%
\begin{pgfscope}%
\pgfpathrectangle{\pgfqpoint{0.847223in}{0.554012in}}{\pgfqpoint{6.200000in}{4.530000in}}%
\pgfusepath{clip}%
\pgfsetbuttcap%
\pgfsetroundjoin%
\pgfsetlinewidth{1.003750pt}%
\definecolor{currentstroke}{rgb}{1.000000,0.000000,0.000000}%
\pgfsetstrokecolor{currentstroke}%
\pgfsetdash{}{0pt}%
\pgfpathmoveto{\pgfqpoint{1.049885in}{3.661367in}}%
\pgfpathcurveto{\pgfqpoint{1.060935in}{3.661367in}}{\pgfqpoint{1.071534in}{3.665758in}}{\pgfqpoint{1.079348in}{3.673571in}}%
\pgfpathcurveto{\pgfqpoint{1.087162in}{3.681385in}}{\pgfqpoint{1.091552in}{3.691984in}}{\pgfqpoint{1.091552in}{3.703034in}}%
\pgfpathcurveto{\pgfqpoint{1.091552in}{3.714084in}}{\pgfqpoint{1.087162in}{3.724683in}}{\pgfqpoint{1.079348in}{3.732497in}}%
\pgfpathcurveto{\pgfqpoint{1.071534in}{3.740310in}}{\pgfqpoint{1.060935in}{3.744701in}}{\pgfqpoint{1.049885in}{3.744701in}}%
\pgfpathcurveto{\pgfqpoint{1.038835in}{3.744701in}}{\pgfqpoint{1.028236in}{3.740310in}}{\pgfqpoint{1.020422in}{3.732497in}}%
\pgfpathcurveto{\pgfqpoint{1.012609in}{3.724683in}}{\pgfqpoint{1.008219in}{3.714084in}}{\pgfqpoint{1.008219in}{3.703034in}}%
\pgfpathcurveto{\pgfqpoint{1.008219in}{3.691984in}}{\pgfqpoint{1.012609in}{3.681385in}}{\pgfqpoint{1.020422in}{3.673571in}}%
\pgfpathcurveto{\pgfqpoint{1.028236in}{3.665758in}}{\pgfqpoint{1.038835in}{3.661367in}}{\pgfqpoint{1.049885in}{3.661367in}}%
\pgfpathlineto{\pgfqpoint{1.049885in}{3.661367in}}%
\pgfpathclose%
\pgfusepath{stroke}%
\end{pgfscope}%
\begin{pgfscope}%
\pgfpathrectangle{\pgfqpoint{0.847223in}{0.554012in}}{\pgfqpoint{6.200000in}{4.530000in}}%
\pgfusepath{clip}%
\pgfsetbuttcap%
\pgfsetroundjoin%
\pgfsetlinewidth{1.003750pt}%
\definecolor{currentstroke}{rgb}{1.000000,0.000000,0.000000}%
\pgfsetstrokecolor{currentstroke}%
\pgfsetdash{}{0pt}%
\pgfpathmoveto{\pgfqpoint{1.055218in}{3.635361in}}%
\pgfpathcurveto{\pgfqpoint{1.066269in}{3.635361in}}{\pgfqpoint{1.076868in}{3.639752in}}{\pgfqpoint{1.084681in}{3.647565in}}%
\pgfpathcurveto{\pgfqpoint{1.092495in}{3.655379in}}{\pgfqpoint{1.096885in}{3.665978in}}{\pgfqpoint{1.096885in}{3.677028in}}%
\pgfpathcurveto{\pgfqpoint{1.096885in}{3.688078in}}{\pgfqpoint{1.092495in}{3.698677in}}{\pgfqpoint{1.084681in}{3.706491in}}%
\pgfpathcurveto{\pgfqpoint{1.076868in}{3.714304in}}{\pgfqpoint{1.066269in}{3.718695in}}{\pgfqpoint{1.055218in}{3.718695in}}%
\pgfpathcurveto{\pgfqpoint{1.044168in}{3.718695in}}{\pgfqpoint{1.033569in}{3.714304in}}{\pgfqpoint{1.025756in}{3.706491in}}%
\pgfpathcurveto{\pgfqpoint{1.017942in}{3.698677in}}{\pgfqpoint{1.013552in}{3.688078in}}{\pgfqpoint{1.013552in}{3.677028in}}%
\pgfpathcurveto{\pgfqpoint{1.013552in}{3.665978in}}{\pgfqpoint{1.017942in}{3.655379in}}{\pgfqpoint{1.025756in}{3.647565in}}%
\pgfpathcurveto{\pgfqpoint{1.033569in}{3.639752in}}{\pgfqpoint{1.044168in}{3.635361in}}{\pgfqpoint{1.055218in}{3.635361in}}%
\pgfpathlineto{\pgfqpoint{1.055218in}{3.635361in}}%
\pgfpathclose%
\pgfusepath{stroke}%
\end{pgfscope}%
\begin{pgfscope}%
\pgfpathrectangle{\pgfqpoint{0.847223in}{0.554012in}}{\pgfqpoint{6.200000in}{4.530000in}}%
\pgfusepath{clip}%
\pgfsetbuttcap%
\pgfsetroundjoin%
\pgfsetlinewidth{1.003750pt}%
\definecolor{currentstroke}{rgb}{1.000000,0.000000,0.000000}%
\pgfsetstrokecolor{currentstroke}%
\pgfsetdash{}{0pt}%
\pgfpathmoveto{\pgfqpoint{1.060552in}{3.609732in}}%
\pgfpathcurveto{\pgfqpoint{1.071602in}{3.609732in}}{\pgfqpoint{1.082201in}{3.614122in}}{\pgfqpoint{1.090014in}{3.621936in}}%
\pgfpathcurveto{\pgfqpoint{1.097828in}{3.629749in}}{\pgfqpoint{1.102218in}{3.640348in}}{\pgfqpoint{1.102218in}{3.651399in}}%
\pgfpathcurveto{\pgfqpoint{1.102218in}{3.662449in}}{\pgfqpoint{1.097828in}{3.673048in}}{\pgfqpoint{1.090014in}{3.680861in}}%
\pgfpathcurveto{\pgfqpoint{1.082201in}{3.688675in}}{\pgfqpoint{1.071602in}{3.693065in}}{\pgfqpoint{1.060552in}{3.693065in}}%
\pgfpathcurveto{\pgfqpoint{1.049502in}{3.693065in}}{\pgfqpoint{1.038903in}{3.688675in}}{\pgfqpoint{1.031089in}{3.680861in}}%
\pgfpathcurveto{\pgfqpoint{1.023275in}{3.673048in}}{\pgfqpoint{1.018885in}{3.662449in}}{\pgfqpoint{1.018885in}{3.651399in}}%
\pgfpathcurveto{\pgfqpoint{1.018885in}{3.640348in}}{\pgfqpoint{1.023275in}{3.629749in}}{\pgfqpoint{1.031089in}{3.621936in}}%
\pgfpathcurveto{\pgfqpoint{1.038903in}{3.614122in}}{\pgfqpoint{1.049502in}{3.609732in}}{\pgfqpoint{1.060552in}{3.609732in}}%
\pgfpathlineto{\pgfqpoint{1.060552in}{3.609732in}}%
\pgfpathclose%
\pgfusepath{stroke}%
\end{pgfscope}%
\begin{pgfscope}%
\pgfpathrectangle{\pgfqpoint{0.847223in}{0.554012in}}{\pgfqpoint{6.200000in}{4.530000in}}%
\pgfusepath{clip}%
\pgfsetbuttcap%
\pgfsetroundjoin%
\pgfsetlinewidth{1.003750pt}%
\definecolor{currentstroke}{rgb}{1.000000,0.000000,0.000000}%
\pgfsetstrokecolor{currentstroke}%
\pgfsetdash{}{0pt}%
\pgfpathmoveto{\pgfqpoint{1.065885in}{3.584471in}}%
\pgfpathcurveto{\pgfqpoint{1.076935in}{3.584471in}}{\pgfqpoint{1.087534in}{3.588861in}}{\pgfqpoint{1.095348in}{3.596675in}}%
\pgfpathcurveto{\pgfqpoint{1.103161in}{3.604488in}}{\pgfqpoint{1.107552in}{3.615087in}}{\pgfqpoint{1.107552in}{3.626138in}}%
\pgfpathcurveto{\pgfqpoint{1.107552in}{3.637188in}}{\pgfqpoint{1.103161in}{3.647787in}}{\pgfqpoint{1.095348in}{3.655600in}}%
\pgfpathcurveto{\pgfqpoint{1.087534in}{3.663414in}}{\pgfqpoint{1.076935in}{3.667804in}}{\pgfqpoint{1.065885in}{3.667804in}}%
\pgfpathcurveto{\pgfqpoint{1.054835in}{3.667804in}}{\pgfqpoint{1.044236in}{3.663414in}}{\pgfqpoint{1.036422in}{3.655600in}}%
\pgfpathcurveto{\pgfqpoint{1.028608in}{3.647787in}}{\pgfqpoint{1.024218in}{3.637188in}}{\pgfqpoint{1.024218in}{3.626138in}}%
\pgfpathcurveto{\pgfqpoint{1.024218in}{3.615087in}}{\pgfqpoint{1.028608in}{3.604488in}}{\pgfqpoint{1.036422in}{3.596675in}}%
\pgfpathcurveto{\pgfqpoint{1.044236in}{3.588861in}}{\pgfqpoint{1.054835in}{3.584471in}}{\pgfqpoint{1.065885in}{3.584471in}}%
\pgfpathlineto{\pgfqpoint{1.065885in}{3.584471in}}%
\pgfpathclose%
\pgfusepath{stroke}%
\end{pgfscope}%
\begin{pgfscope}%
\pgfpathrectangle{\pgfqpoint{0.847223in}{0.554012in}}{\pgfqpoint{6.200000in}{4.530000in}}%
\pgfusepath{clip}%
\pgfsetbuttcap%
\pgfsetroundjoin%
\pgfsetlinewidth{1.003750pt}%
\definecolor{currentstroke}{rgb}{1.000000,0.000000,0.000000}%
\pgfsetstrokecolor{currentstroke}%
\pgfsetdash{}{0pt}%
\pgfpathmoveto{\pgfqpoint{1.071218in}{3.559570in}}%
\pgfpathcurveto{\pgfqpoint{1.082268in}{3.559570in}}{\pgfqpoint{1.092867in}{3.563961in}}{\pgfqpoint{1.100681in}{3.571774in}}%
\pgfpathcurveto{\pgfqpoint{1.108495in}{3.579588in}}{\pgfqpoint{1.112885in}{3.590187in}}{\pgfqpoint{1.112885in}{3.601237in}}%
\pgfpathcurveto{\pgfqpoint{1.112885in}{3.612287in}}{\pgfqpoint{1.108495in}{3.622886in}}{\pgfqpoint{1.100681in}{3.630700in}}%
\pgfpathcurveto{\pgfqpoint{1.092867in}{3.638514in}}{\pgfqpoint{1.082268in}{3.642904in}}{\pgfqpoint{1.071218in}{3.642904in}}%
\pgfpathcurveto{\pgfqpoint{1.060168in}{3.642904in}}{\pgfqpoint{1.049569in}{3.638514in}}{\pgfqpoint{1.041755in}{3.630700in}}%
\pgfpathcurveto{\pgfqpoint{1.033942in}{3.622886in}}{\pgfqpoint{1.029551in}{3.612287in}}{\pgfqpoint{1.029551in}{3.601237in}}%
\pgfpathcurveto{\pgfqpoint{1.029551in}{3.590187in}}{\pgfqpoint{1.033942in}{3.579588in}}{\pgfqpoint{1.041755in}{3.571774in}}%
\pgfpathcurveto{\pgfqpoint{1.049569in}{3.563961in}}{\pgfqpoint{1.060168in}{3.559570in}}{\pgfqpoint{1.071218in}{3.559570in}}%
\pgfpathlineto{\pgfqpoint{1.071218in}{3.559570in}}%
\pgfpathclose%
\pgfusepath{stroke}%
\end{pgfscope}%
\begin{pgfscope}%
\pgfpathrectangle{\pgfqpoint{0.847223in}{0.554012in}}{\pgfqpoint{6.200000in}{4.530000in}}%
\pgfusepath{clip}%
\pgfsetbuttcap%
\pgfsetroundjoin%
\pgfsetlinewidth{1.003750pt}%
\definecolor{currentstroke}{rgb}{1.000000,0.000000,0.000000}%
\pgfsetstrokecolor{currentstroke}%
\pgfsetdash{}{0pt}%
\pgfpathmoveto{\pgfqpoint{1.076551in}{3.535023in}}%
\pgfpathcurveto{\pgfqpoint{1.087601in}{3.535023in}}{\pgfqpoint{1.098200in}{3.539413in}}{\pgfqpoint{1.106014in}{3.547227in}}%
\pgfpathcurveto{\pgfqpoint{1.113828in}{3.555040in}}{\pgfqpoint{1.118218in}{3.565639in}}{\pgfqpoint{1.118218in}{3.576690in}}%
\pgfpathcurveto{\pgfqpoint{1.118218in}{3.587740in}}{\pgfqpoint{1.113828in}{3.598339in}}{\pgfqpoint{1.106014in}{3.606152in}}%
\pgfpathcurveto{\pgfqpoint{1.098200in}{3.613966in}}{\pgfqpoint{1.087601in}{3.618356in}}{\pgfqpoint{1.076551in}{3.618356in}}%
\pgfpathcurveto{\pgfqpoint{1.065501in}{3.618356in}}{\pgfqpoint{1.054902in}{3.613966in}}{\pgfqpoint{1.047089in}{3.606152in}}%
\pgfpathcurveto{\pgfqpoint{1.039275in}{3.598339in}}{\pgfqpoint{1.034885in}{3.587740in}}{\pgfqpoint{1.034885in}{3.576690in}}%
\pgfpathcurveto{\pgfqpoint{1.034885in}{3.565639in}}{\pgfqpoint{1.039275in}{3.555040in}}{\pgfqpoint{1.047089in}{3.547227in}}%
\pgfpathcurveto{\pgfqpoint{1.054902in}{3.539413in}}{\pgfqpoint{1.065501in}{3.535023in}}{\pgfqpoint{1.076551in}{3.535023in}}%
\pgfpathlineto{\pgfqpoint{1.076551in}{3.535023in}}%
\pgfpathclose%
\pgfusepath{stroke}%
\end{pgfscope}%
\begin{pgfscope}%
\pgfpathrectangle{\pgfqpoint{0.847223in}{0.554012in}}{\pgfqpoint{6.200000in}{4.530000in}}%
\pgfusepath{clip}%
\pgfsetbuttcap%
\pgfsetroundjoin%
\pgfsetlinewidth{1.003750pt}%
\definecolor{currentstroke}{rgb}{1.000000,0.000000,0.000000}%
\pgfsetstrokecolor{currentstroke}%
\pgfsetdash{}{0pt}%
\pgfpathmoveto{\pgfqpoint{1.081885in}{3.510821in}}%
\pgfpathcurveto{\pgfqpoint{1.092935in}{3.510821in}}{\pgfqpoint{1.103534in}{3.515211in}}{\pgfqpoint{1.111347in}{3.523025in}}%
\pgfpathcurveto{\pgfqpoint{1.119161in}{3.530838in}}{\pgfqpoint{1.123551in}{3.541437in}}{\pgfqpoint{1.123551in}{3.552487in}}%
\pgfpathcurveto{\pgfqpoint{1.123551in}{3.563537in}}{\pgfqpoint{1.119161in}{3.574137in}}{\pgfqpoint{1.111347in}{3.581950in}}%
\pgfpathcurveto{\pgfqpoint{1.103534in}{3.589764in}}{\pgfqpoint{1.092935in}{3.594154in}}{\pgfqpoint{1.081885in}{3.594154in}}%
\pgfpathcurveto{\pgfqpoint{1.070834in}{3.594154in}}{\pgfqpoint{1.060235in}{3.589764in}}{\pgfqpoint{1.052422in}{3.581950in}}%
\pgfpathcurveto{\pgfqpoint{1.044608in}{3.574137in}}{\pgfqpoint{1.040218in}{3.563537in}}{\pgfqpoint{1.040218in}{3.552487in}}%
\pgfpathcurveto{\pgfqpoint{1.040218in}{3.541437in}}{\pgfqpoint{1.044608in}{3.530838in}}{\pgfqpoint{1.052422in}{3.523025in}}%
\pgfpathcurveto{\pgfqpoint{1.060235in}{3.515211in}}{\pgfqpoint{1.070834in}{3.510821in}}{\pgfqpoint{1.081885in}{3.510821in}}%
\pgfpathlineto{\pgfqpoint{1.081885in}{3.510821in}}%
\pgfpathclose%
\pgfusepath{stroke}%
\end{pgfscope}%
\begin{pgfscope}%
\pgfpathrectangle{\pgfqpoint{0.847223in}{0.554012in}}{\pgfqpoint{6.200000in}{4.530000in}}%
\pgfusepath{clip}%
\pgfsetbuttcap%
\pgfsetroundjoin%
\pgfsetlinewidth{1.003750pt}%
\definecolor{currentstroke}{rgb}{1.000000,0.000000,0.000000}%
\pgfsetstrokecolor{currentstroke}%
\pgfsetdash{}{0pt}%
\pgfpathmoveto{\pgfqpoint{1.087218in}{3.486957in}}%
\pgfpathcurveto{\pgfqpoint{1.098268in}{3.486957in}}{\pgfqpoint{1.108867in}{3.491347in}}{\pgfqpoint{1.116681in}{3.499161in}}%
\pgfpathcurveto{\pgfqpoint{1.124494in}{3.506974in}}{\pgfqpoint{1.128884in}{3.517573in}}{\pgfqpoint{1.128884in}{3.528623in}}%
\pgfpathcurveto{\pgfqpoint{1.128884in}{3.539674in}}{\pgfqpoint{1.124494in}{3.550273in}}{\pgfqpoint{1.116681in}{3.558086in}}%
\pgfpathcurveto{\pgfqpoint{1.108867in}{3.565900in}}{\pgfqpoint{1.098268in}{3.570290in}}{\pgfqpoint{1.087218in}{3.570290in}}%
\pgfpathcurveto{\pgfqpoint{1.076168in}{3.570290in}}{\pgfqpoint{1.065569in}{3.565900in}}{\pgfqpoint{1.057755in}{3.558086in}}%
\pgfpathcurveto{\pgfqpoint{1.049941in}{3.550273in}}{\pgfqpoint{1.045551in}{3.539674in}}{\pgfqpoint{1.045551in}{3.528623in}}%
\pgfpathcurveto{\pgfqpoint{1.045551in}{3.517573in}}{\pgfqpoint{1.049941in}{3.506974in}}{\pgfqpoint{1.057755in}{3.499161in}}%
\pgfpathcurveto{\pgfqpoint{1.065569in}{3.491347in}}{\pgfqpoint{1.076168in}{3.486957in}}{\pgfqpoint{1.087218in}{3.486957in}}%
\pgfpathlineto{\pgfqpoint{1.087218in}{3.486957in}}%
\pgfpathclose%
\pgfusepath{stroke}%
\end{pgfscope}%
\begin{pgfscope}%
\pgfpathrectangle{\pgfqpoint{0.847223in}{0.554012in}}{\pgfqpoint{6.200000in}{4.530000in}}%
\pgfusepath{clip}%
\pgfsetbuttcap%
\pgfsetroundjoin%
\pgfsetlinewidth{1.003750pt}%
\definecolor{currentstroke}{rgb}{1.000000,0.000000,0.000000}%
\pgfsetstrokecolor{currentstroke}%
\pgfsetdash{}{0pt}%
\pgfpathmoveto{\pgfqpoint{1.092551in}{3.463424in}}%
\pgfpathcurveto{\pgfqpoint{1.103601in}{3.463424in}}{\pgfqpoint{1.114200in}{3.467814in}}{\pgfqpoint{1.122014in}{3.475628in}}%
\pgfpathcurveto{\pgfqpoint{1.129827in}{3.483441in}}{\pgfqpoint{1.134218in}{3.494040in}}{\pgfqpoint{1.134218in}{3.505091in}}%
\pgfpathcurveto{\pgfqpoint{1.134218in}{3.516141in}}{\pgfqpoint{1.129827in}{3.526740in}}{\pgfqpoint{1.122014in}{3.534553in}}%
\pgfpathcurveto{\pgfqpoint{1.114200in}{3.542367in}}{\pgfqpoint{1.103601in}{3.546757in}}{\pgfqpoint{1.092551in}{3.546757in}}%
\pgfpathcurveto{\pgfqpoint{1.081501in}{3.546757in}}{\pgfqpoint{1.070902in}{3.542367in}}{\pgfqpoint{1.063088in}{3.534553in}}%
\pgfpathcurveto{\pgfqpoint{1.055275in}{3.526740in}}{\pgfqpoint{1.050884in}{3.516141in}}{\pgfqpoint{1.050884in}{3.505091in}}%
\pgfpathcurveto{\pgfqpoint{1.050884in}{3.494040in}}{\pgfqpoint{1.055275in}{3.483441in}}{\pgfqpoint{1.063088in}{3.475628in}}%
\pgfpathcurveto{\pgfqpoint{1.070902in}{3.467814in}}{\pgfqpoint{1.081501in}{3.463424in}}{\pgfqpoint{1.092551in}{3.463424in}}%
\pgfpathlineto{\pgfqpoint{1.092551in}{3.463424in}}%
\pgfpathclose%
\pgfusepath{stroke}%
\end{pgfscope}%
\begin{pgfscope}%
\pgfpathrectangle{\pgfqpoint{0.847223in}{0.554012in}}{\pgfqpoint{6.200000in}{4.530000in}}%
\pgfusepath{clip}%
\pgfsetbuttcap%
\pgfsetroundjoin%
\pgfsetlinewidth{1.003750pt}%
\definecolor{currentstroke}{rgb}{1.000000,0.000000,0.000000}%
\pgfsetstrokecolor{currentstroke}%
\pgfsetdash{}{0pt}%
\pgfpathmoveto{\pgfqpoint{1.097884in}{3.440215in}}%
\pgfpathcurveto{\pgfqpoint{1.108934in}{3.440215in}}{\pgfqpoint{1.119533in}{3.444606in}}{\pgfqpoint{1.127347in}{3.452419in}}%
\pgfpathcurveto{\pgfqpoint{1.135161in}{3.460233in}}{\pgfqpoint{1.139551in}{3.470832in}}{\pgfqpoint{1.139551in}{3.481882in}}%
\pgfpathcurveto{\pgfqpoint{1.139551in}{3.492932in}}{\pgfqpoint{1.135161in}{3.503531in}}{\pgfqpoint{1.127347in}{3.511345in}}%
\pgfpathcurveto{\pgfqpoint{1.119533in}{3.519158in}}{\pgfqpoint{1.108934in}{3.523549in}}{\pgfqpoint{1.097884in}{3.523549in}}%
\pgfpathcurveto{\pgfqpoint{1.086834in}{3.523549in}}{\pgfqpoint{1.076235in}{3.519158in}}{\pgfqpoint{1.068421in}{3.511345in}}%
\pgfpathcurveto{\pgfqpoint{1.060608in}{3.503531in}}{\pgfqpoint{1.056218in}{3.492932in}}{\pgfqpoint{1.056218in}{3.481882in}}%
\pgfpathcurveto{\pgfqpoint{1.056218in}{3.470832in}}{\pgfqpoint{1.060608in}{3.460233in}}{\pgfqpoint{1.068421in}{3.452419in}}%
\pgfpathcurveto{\pgfqpoint{1.076235in}{3.444606in}}{\pgfqpoint{1.086834in}{3.440215in}}{\pgfqpoint{1.097884in}{3.440215in}}%
\pgfpathlineto{\pgfqpoint{1.097884in}{3.440215in}}%
\pgfpathclose%
\pgfusepath{stroke}%
\end{pgfscope}%
\begin{pgfscope}%
\pgfpathrectangle{\pgfqpoint{0.847223in}{0.554012in}}{\pgfqpoint{6.200000in}{4.530000in}}%
\pgfusepath{clip}%
\pgfsetbuttcap%
\pgfsetroundjoin%
\pgfsetlinewidth{1.003750pt}%
\definecolor{currentstroke}{rgb}{1.000000,0.000000,0.000000}%
\pgfsetstrokecolor{currentstroke}%
\pgfsetdash{}{0pt}%
\pgfpathmoveto{\pgfqpoint{1.103217in}{3.417324in}}%
\pgfpathcurveto{\pgfqpoint{1.114268in}{3.417324in}}{\pgfqpoint{1.124867in}{3.421715in}}{\pgfqpoint{1.132680in}{3.429528in}}%
\pgfpathcurveto{\pgfqpoint{1.140494in}{3.437342in}}{\pgfqpoint{1.144884in}{3.447941in}}{\pgfqpoint{1.144884in}{3.458991in}}%
\pgfpathcurveto{\pgfqpoint{1.144884in}{3.470041in}}{\pgfqpoint{1.140494in}{3.480640in}}{\pgfqpoint{1.132680in}{3.488454in}}%
\pgfpathcurveto{\pgfqpoint{1.124867in}{3.496268in}}{\pgfqpoint{1.114268in}{3.500658in}}{\pgfqpoint{1.103217in}{3.500658in}}%
\pgfpathcurveto{\pgfqpoint{1.092167in}{3.500658in}}{\pgfqpoint{1.081568in}{3.496268in}}{\pgfqpoint{1.073755in}{3.488454in}}%
\pgfpathcurveto{\pgfqpoint{1.065941in}{3.480640in}}{\pgfqpoint{1.061551in}{3.470041in}}{\pgfqpoint{1.061551in}{3.458991in}}%
\pgfpathcurveto{\pgfqpoint{1.061551in}{3.447941in}}{\pgfqpoint{1.065941in}{3.437342in}}{\pgfqpoint{1.073755in}{3.429528in}}%
\pgfpathcurveto{\pgfqpoint{1.081568in}{3.421715in}}{\pgfqpoint{1.092167in}{3.417324in}}{\pgfqpoint{1.103217in}{3.417324in}}%
\pgfpathlineto{\pgfqpoint{1.103217in}{3.417324in}}%
\pgfpathclose%
\pgfusepath{stroke}%
\end{pgfscope}%
\begin{pgfscope}%
\pgfpathrectangle{\pgfqpoint{0.847223in}{0.554012in}}{\pgfqpoint{6.200000in}{4.530000in}}%
\pgfusepath{clip}%
\pgfsetbuttcap%
\pgfsetroundjoin%
\pgfsetlinewidth{1.003750pt}%
\definecolor{currentstroke}{rgb}{1.000000,0.000000,0.000000}%
\pgfsetstrokecolor{currentstroke}%
\pgfsetdash{}{0pt}%
\pgfpathmoveto{\pgfqpoint{1.108551in}{3.394745in}}%
\pgfpathcurveto{\pgfqpoint{1.119601in}{3.394745in}}{\pgfqpoint{1.130200in}{3.399135in}}{\pgfqpoint{1.138013in}{3.406949in}}%
\pgfpathcurveto{\pgfqpoint{1.145827in}{3.414762in}}{\pgfqpoint{1.150217in}{3.425361in}}{\pgfqpoint{1.150217in}{3.436411in}}%
\pgfpathcurveto{\pgfqpoint{1.150217in}{3.447462in}}{\pgfqpoint{1.145827in}{3.458061in}}{\pgfqpoint{1.138013in}{3.465874in}}%
\pgfpathcurveto{\pgfqpoint{1.130200in}{3.473688in}}{\pgfqpoint{1.119601in}{3.478078in}}{\pgfqpoint{1.108551in}{3.478078in}}%
\pgfpathcurveto{\pgfqpoint{1.097500in}{3.478078in}}{\pgfqpoint{1.086901in}{3.473688in}}{\pgfqpoint{1.079088in}{3.465874in}}%
\pgfpathcurveto{\pgfqpoint{1.071274in}{3.458061in}}{\pgfqpoint{1.066884in}{3.447462in}}{\pgfqpoint{1.066884in}{3.436411in}}%
\pgfpathcurveto{\pgfqpoint{1.066884in}{3.425361in}}{\pgfqpoint{1.071274in}{3.414762in}}{\pgfqpoint{1.079088in}{3.406949in}}%
\pgfpathcurveto{\pgfqpoint{1.086901in}{3.399135in}}{\pgfqpoint{1.097500in}{3.394745in}}{\pgfqpoint{1.108551in}{3.394745in}}%
\pgfpathlineto{\pgfqpoint{1.108551in}{3.394745in}}%
\pgfpathclose%
\pgfusepath{stroke}%
\end{pgfscope}%
\begin{pgfscope}%
\pgfpathrectangle{\pgfqpoint{0.847223in}{0.554012in}}{\pgfqpoint{6.200000in}{4.530000in}}%
\pgfusepath{clip}%
\pgfsetbuttcap%
\pgfsetroundjoin%
\pgfsetlinewidth{1.003750pt}%
\definecolor{currentstroke}{rgb}{1.000000,0.000000,0.000000}%
\pgfsetstrokecolor{currentstroke}%
\pgfsetdash{}{0pt}%
\pgfpathmoveto{\pgfqpoint{1.113884in}{3.372470in}}%
\pgfpathcurveto{\pgfqpoint{1.124934in}{3.372470in}}{\pgfqpoint{1.135533in}{3.376860in}}{\pgfqpoint{1.143347in}{3.384674in}}%
\pgfpathcurveto{\pgfqpoint{1.151160in}{3.392487in}}{\pgfqpoint{1.155551in}{3.403086in}}{\pgfqpoint{1.155551in}{3.414137in}}%
\pgfpathcurveto{\pgfqpoint{1.155551in}{3.425187in}}{\pgfqpoint{1.151160in}{3.435786in}}{\pgfqpoint{1.143347in}{3.443599in}}%
\pgfpathcurveto{\pgfqpoint{1.135533in}{3.451413in}}{\pgfqpoint{1.124934in}{3.455803in}}{\pgfqpoint{1.113884in}{3.455803in}}%
\pgfpathcurveto{\pgfqpoint{1.102834in}{3.455803in}}{\pgfqpoint{1.092235in}{3.451413in}}{\pgfqpoint{1.084421in}{3.443599in}}%
\pgfpathcurveto{\pgfqpoint{1.076607in}{3.435786in}}{\pgfqpoint{1.072217in}{3.425187in}}{\pgfqpoint{1.072217in}{3.414137in}}%
\pgfpathcurveto{\pgfqpoint{1.072217in}{3.403086in}}{\pgfqpoint{1.076607in}{3.392487in}}{\pgfqpoint{1.084421in}{3.384674in}}%
\pgfpathcurveto{\pgfqpoint{1.092235in}{3.376860in}}{\pgfqpoint{1.102834in}{3.372470in}}{\pgfqpoint{1.113884in}{3.372470in}}%
\pgfpathlineto{\pgfqpoint{1.113884in}{3.372470in}}%
\pgfpathclose%
\pgfusepath{stroke}%
\end{pgfscope}%
\begin{pgfscope}%
\pgfpathrectangle{\pgfqpoint{0.847223in}{0.554012in}}{\pgfqpoint{6.200000in}{4.530000in}}%
\pgfusepath{clip}%
\pgfsetbuttcap%
\pgfsetroundjoin%
\pgfsetlinewidth{1.003750pt}%
\definecolor{currentstroke}{rgb}{1.000000,0.000000,0.000000}%
\pgfsetstrokecolor{currentstroke}%
\pgfsetdash{}{0pt}%
\pgfpathmoveto{\pgfqpoint{1.119217in}{3.350494in}}%
\pgfpathcurveto{\pgfqpoint{1.130267in}{3.350494in}}{\pgfqpoint{1.140866in}{3.354884in}}{\pgfqpoint{1.148680in}{3.362698in}}%
\pgfpathcurveto{\pgfqpoint{1.156493in}{3.370511in}}{\pgfqpoint{1.160884in}{3.381110in}}{\pgfqpoint{1.160884in}{3.392160in}}%
\pgfpathcurveto{\pgfqpoint{1.160884in}{3.403211in}}{\pgfqpoint{1.156493in}{3.413810in}}{\pgfqpoint{1.148680in}{3.421623in}}%
\pgfpathcurveto{\pgfqpoint{1.140866in}{3.429437in}}{\pgfqpoint{1.130267in}{3.433827in}}{\pgfqpoint{1.119217in}{3.433827in}}%
\pgfpathcurveto{\pgfqpoint{1.108167in}{3.433827in}}{\pgfqpoint{1.097568in}{3.429437in}}{\pgfqpoint{1.089754in}{3.421623in}}%
\pgfpathcurveto{\pgfqpoint{1.081941in}{3.413810in}}{\pgfqpoint{1.077550in}{3.403211in}}{\pgfqpoint{1.077550in}{3.392160in}}%
\pgfpathcurveto{\pgfqpoint{1.077550in}{3.381110in}}{\pgfqpoint{1.081941in}{3.370511in}}{\pgfqpoint{1.089754in}{3.362698in}}%
\pgfpathcurveto{\pgfqpoint{1.097568in}{3.354884in}}{\pgfqpoint{1.108167in}{3.350494in}}{\pgfqpoint{1.119217in}{3.350494in}}%
\pgfpathlineto{\pgfqpoint{1.119217in}{3.350494in}}%
\pgfpathclose%
\pgfusepath{stroke}%
\end{pgfscope}%
\begin{pgfscope}%
\pgfpathrectangle{\pgfqpoint{0.847223in}{0.554012in}}{\pgfqpoint{6.200000in}{4.530000in}}%
\pgfusepath{clip}%
\pgfsetbuttcap%
\pgfsetroundjoin%
\pgfsetlinewidth{1.003750pt}%
\definecolor{currentstroke}{rgb}{1.000000,0.000000,0.000000}%
\pgfsetstrokecolor{currentstroke}%
\pgfsetdash{}{0pt}%
\pgfpathmoveto{\pgfqpoint{1.124550in}{3.328810in}}%
\pgfpathcurveto{\pgfqpoint{1.135600in}{3.328810in}}{\pgfqpoint{1.146199in}{3.333201in}}{\pgfqpoint{1.154013in}{3.341014in}}%
\pgfpathcurveto{\pgfqpoint{1.161827in}{3.348828in}}{\pgfqpoint{1.166217in}{3.359427in}}{\pgfqpoint{1.166217in}{3.370477in}}%
\pgfpathcurveto{\pgfqpoint{1.166217in}{3.381527in}}{\pgfqpoint{1.161827in}{3.392126in}}{\pgfqpoint{1.154013in}{3.399940in}}%
\pgfpathcurveto{\pgfqpoint{1.146199in}{3.407753in}}{\pgfqpoint{1.135600in}{3.412144in}}{\pgfqpoint{1.124550in}{3.412144in}}%
\pgfpathcurveto{\pgfqpoint{1.113500in}{3.412144in}}{\pgfqpoint{1.102901in}{3.407753in}}{\pgfqpoint{1.095087in}{3.399940in}}%
\pgfpathcurveto{\pgfqpoint{1.087274in}{3.392126in}}{\pgfqpoint{1.082884in}{3.381527in}}{\pgfqpoint{1.082884in}{3.370477in}}%
\pgfpathcurveto{\pgfqpoint{1.082884in}{3.359427in}}{\pgfqpoint{1.087274in}{3.348828in}}{\pgfqpoint{1.095087in}{3.341014in}}%
\pgfpathcurveto{\pgfqpoint{1.102901in}{3.333201in}}{\pgfqpoint{1.113500in}{3.328810in}}{\pgfqpoint{1.124550in}{3.328810in}}%
\pgfpathlineto{\pgfqpoint{1.124550in}{3.328810in}}%
\pgfpathclose%
\pgfusepath{stroke}%
\end{pgfscope}%
\begin{pgfscope}%
\pgfpathrectangle{\pgfqpoint{0.847223in}{0.554012in}}{\pgfqpoint{6.200000in}{4.530000in}}%
\pgfusepath{clip}%
\pgfsetbuttcap%
\pgfsetroundjoin%
\pgfsetlinewidth{1.003750pt}%
\definecolor{currentstroke}{rgb}{1.000000,0.000000,0.000000}%
\pgfsetstrokecolor{currentstroke}%
\pgfsetdash{}{0pt}%
\pgfpathmoveto{\pgfqpoint{1.129883in}{3.307414in}}%
\pgfpathcurveto{\pgfqpoint{1.140934in}{3.307414in}}{\pgfqpoint{1.151533in}{3.311804in}}{\pgfqpoint{1.159346in}{3.319618in}}%
\pgfpathcurveto{\pgfqpoint{1.167160in}{3.327431in}}{\pgfqpoint{1.171550in}{3.338030in}}{\pgfqpoint{1.171550in}{3.349081in}}%
\pgfpathcurveto{\pgfqpoint{1.171550in}{3.360131in}}{\pgfqpoint{1.167160in}{3.370730in}}{\pgfqpoint{1.159346in}{3.378543in}}%
\pgfpathcurveto{\pgfqpoint{1.151533in}{3.386357in}}{\pgfqpoint{1.140934in}{3.390747in}}{\pgfqpoint{1.129883in}{3.390747in}}%
\pgfpathcurveto{\pgfqpoint{1.118833in}{3.390747in}}{\pgfqpoint{1.108234in}{3.386357in}}{\pgfqpoint{1.100421in}{3.378543in}}%
\pgfpathcurveto{\pgfqpoint{1.092607in}{3.370730in}}{\pgfqpoint{1.088217in}{3.360131in}}{\pgfqpoint{1.088217in}{3.349081in}}%
\pgfpathcurveto{\pgfqpoint{1.088217in}{3.338030in}}{\pgfqpoint{1.092607in}{3.327431in}}{\pgfqpoint{1.100421in}{3.319618in}}%
\pgfpathcurveto{\pgfqpoint{1.108234in}{3.311804in}}{\pgfqpoint{1.118833in}{3.307414in}}{\pgfqpoint{1.129883in}{3.307414in}}%
\pgfpathlineto{\pgfqpoint{1.129883in}{3.307414in}}%
\pgfpathclose%
\pgfusepath{stroke}%
\end{pgfscope}%
\begin{pgfscope}%
\pgfpathrectangle{\pgfqpoint{0.847223in}{0.554012in}}{\pgfqpoint{6.200000in}{4.530000in}}%
\pgfusepath{clip}%
\pgfsetbuttcap%
\pgfsetroundjoin%
\pgfsetlinewidth{1.003750pt}%
\definecolor{currentstroke}{rgb}{1.000000,0.000000,0.000000}%
\pgfsetstrokecolor{currentstroke}%
\pgfsetdash{}{0pt}%
\pgfpathmoveto{\pgfqpoint{1.135217in}{3.286299in}}%
\pgfpathcurveto{\pgfqpoint{1.146267in}{3.286299in}}{\pgfqpoint{1.156866in}{3.290689in}}{\pgfqpoint{1.164679in}{3.298503in}}%
\pgfpathcurveto{\pgfqpoint{1.172493in}{3.306316in}}{\pgfqpoint{1.176883in}{3.316915in}}{\pgfqpoint{1.176883in}{3.327965in}}%
\pgfpathcurveto{\pgfqpoint{1.176883in}{3.339016in}}{\pgfqpoint{1.172493in}{3.349615in}}{\pgfqpoint{1.164679in}{3.357428in}}%
\pgfpathcurveto{\pgfqpoint{1.156866in}{3.365242in}}{\pgfqpoint{1.146267in}{3.369632in}}{\pgfqpoint{1.135217in}{3.369632in}}%
\pgfpathcurveto{\pgfqpoint{1.124167in}{3.369632in}}{\pgfqpoint{1.113568in}{3.365242in}}{\pgfqpoint{1.105754in}{3.357428in}}%
\pgfpathcurveto{\pgfqpoint{1.097940in}{3.349615in}}{\pgfqpoint{1.093550in}{3.339016in}}{\pgfqpoint{1.093550in}{3.327965in}}%
\pgfpathcurveto{\pgfqpoint{1.093550in}{3.316915in}}{\pgfqpoint{1.097940in}{3.306316in}}{\pgfqpoint{1.105754in}{3.298503in}}%
\pgfpathcurveto{\pgfqpoint{1.113568in}{3.290689in}}{\pgfqpoint{1.124167in}{3.286299in}}{\pgfqpoint{1.135217in}{3.286299in}}%
\pgfpathlineto{\pgfqpoint{1.135217in}{3.286299in}}%
\pgfpathclose%
\pgfusepath{stroke}%
\end{pgfscope}%
\begin{pgfscope}%
\pgfpathrectangle{\pgfqpoint{0.847223in}{0.554012in}}{\pgfqpoint{6.200000in}{4.530000in}}%
\pgfusepath{clip}%
\pgfsetbuttcap%
\pgfsetroundjoin%
\pgfsetlinewidth{1.003750pt}%
\definecolor{currentstroke}{rgb}{1.000000,0.000000,0.000000}%
\pgfsetstrokecolor{currentstroke}%
\pgfsetdash{}{0pt}%
\pgfpathmoveto{\pgfqpoint{1.140550in}{3.265459in}}%
\pgfpathcurveto{\pgfqpoint{1.151600in}{3.265459in}}{\pgfqpoint{1.162199in}{3.269850in}}{\pgfqpoint{1.170013in}{3.277663in}}%
\pgfpathcurveto{\pgfqpoint{1.177826in}{3.285477in}}{\pgfqpoint{1.182217in}{3.296076in}}{\pgfqpoint{1.182217in}{3.307126in}}%
\pgfpathcurveto{\pgfqpoint{1.182217in}{3.318176in}}{\pgfqpoint{1.177826in}{3.328775in}}{\pgfqpoint{1.170013in}{3.336589in}}%
\pgfpathcurveto{\pgfqpoint{1.162199in}{3.344403in}}{\pgfqpoint{1.151600in}{3.348793in}}{\pgfqpoint{1.140550in}{3.348793in}}%
\pgfpathcurveto{\pgfqpoint{1.129500in}{3.348793in}}{\pgfqpoint{1.118901in}{3.344403in}}{\pgfqpoint{1.111087in}{3.336589in}}%
\pgfpathcurveto{\pgfqpoint{1.103274in}{3.328775in}}{\pgfqpoint{1.098883in}{3.318176in}}{\pgfqpoint{1.098883in}{3.307126in}}%
\pgfpathcurveto{\pgfqpoint{1.098883in}{3.296076in}}{\pgfqpoint{1.103274in}{3.285477in}}{\pgfqpoint{1.111087in}{3.277663in}}%
\pgfpathcurveto{\pgfqpoint{1.118901in}{3.269850in}}{\pgfqpoint{1.129500in}{3.265459in}}{\pgfqpoint{1.140550in}{3.265459in}}%
\pgfpathlineto{\pgfqpoint{1.140550in}{3.265459in}}%
\pgfpathclose%
\pgfusepath{stroke}%
\end{pgfscope}%
\begin{pgfscope}%
\pgfpathrectangle{\pgfqpoint{0.847223in}{0.554012in}}{\pgfqpoint{6.200000in}{4.530000in}}%
\pgfusepath{clip}%
\pgfsetbuttcap%
\pgfsetroundjoin%
\pgfsetlinewidth{1.003750pt}%
\definecolor{currentstroke}{rgb}{1.000000,0.000000,0.000000}%
\pgfsetstrokecolor{currentstroke}%
\pgfsetdash{}{0pt}%
\pgfpathmoveto{\pgfqpoint{1.145883in}{3.244891in}}%
\pgfpathcurveto{\pgfqpoint{1.156933in}{3.244891in}}{\pgfqpoint{1.167532in}{3.249281in}}{\pgfqpoint{1.175346in}{3.257094in}}%
\pgfpathcurveto{\pgfqpoint{1.183160in}{3.264908in}}{\pgfqpoint{1.187550in}{3.275507in}}{\pgfqpoint{1.187550in}{3.286557in}}%
\pgfpathcurveto{\pgfqpoint{1.187550in}{3.297607in}}{\pgfqpoint{1.183160in}{3.308206in}}{\pgfqpoint{1.175346in}{3.316020in}}%
\pgfpathcurveto{\pgfqpoint{1.167532in}{3.323834in}}{\pgfqpoint{1.156933in}{3.328224in}}{\pgfqpoint{1.145883in}{3.328224in}}%
\pgfpathcurveto{\pgfqpoint{1.134833in}{3.328224in}}{\pgfqpoint{1.124234in}{3.323834in}}{\pgfqpoint{1.116420in}{3.316020in}}%
\pgfpathcurveto{\pgfqpoint{1.108607in}{3.308206in}}{\pgfqpoint{1.104216in}{3.297607in}}{\pgfqpoint{1.104216in}{3.286557in}}%
\pgfpathcurveto{\pgfqpoint{1.104216in}{3.275507in}}{\pgfqpoint{1.108607in}{3.264908in}}{\pgfqpoint{1.116420in}{3.257094in}}%
\pgfpathcurveto{\pgfqpoint{1.124234in}{3.249281in}}{\pgfqpoint{1.134833in}{3.244891in}}{\pgfqpoint{1.145883in}{3.244891in}}%
\pgfpathlineto{\pgfqpoint{1.145883in}{3.244891in}}%
\pgfpathclose%
\pgfusepath{stroke}%
\end{pgfscope}%
\begin{pgfscope}%
\pgfpathrectangle{\pgfqpoint{0.847223in}{0.554012in}}{\pgfqpoint{6.200000in}{4.530000in}}%
\pgfusepath{clip}%
\pgfsetbuttcap%
\pgfsetroundjoin%
\pgfsetlinewidth{1.003750pt}%
\definecolor{currentstroke}{rgb}{1.000000,0.000000,0.000000}%
\pgfsetstrokecolor{currentstroke}%
\pgfsetdash{}{0pt}%
\pgfpathmoveto{\pgfqpoint{1.151216in}{3.224587in}}%
\pgfpathcurveto{\pgfqpoint{1.162266in}{3.224587in}}{\pgfqpoint{1.172866in}{3.228977in}}{\pgfqpoint{1.180679in}{3.236791in}}%
\pgfpathcurveto{\pgfqpoint{1.188493in}{3.244604in}}{\pgfqpoint{1.192883in}{3.255203in}}{\pgfqpoint{1.192883in}{3.266253in}}%
\pgfpathcurveto{\pgfqpoint{1.192883in}{3.277304in}}{\pgfqpoint{1.188493in}{3.287903in}}{\pgfqpoint{1.180679in}{3.295716in}}%
\pgfpathcurveto{\pgfqpoint{1.172866in}{3.303530in}}{\pgfqpoint{1.162266in}{3.307920in}}{\pgfqpoint{1.151216in}{3.307920in}}%
\pgfpathcurveto{\pgfqpoint{1.140166in}{3.307920in}}{\pgfqpoint{1.129567in}{3.303530in}}{\pgfqpoint{1.121754in}{3.295716in}}%
\pgfpathcurveto{\pgfqpoint{1.113940in}{3.287903in}}{\pgfqpoint{1.109550in}{3.277304in}}{\pgfqpoint{1.109550in}{3.266253in}}%
\pgfpathcurveto{\pgfqpoint{1.109550in}{3.255203in}}{\pgfqpoint{1.113940in}{3.244604in}}{\pgfqpoint{1.121754in}{3.236791in}}%
\pgfpathcurveto{\pgfqpoint{1.129567in}{3.228977in}}{\pgfqpoint{1.140166in}{3.224587in}}{\pgfqpoint{1.151216in}{3.224587in}}%
\pgfpathlineto{\pgfqpoint{1.151216in}{3.224587in}}%
\pgfpathclose%
\pgfusepath{stroke}%
\end{pgfscope}%
\begin{pgfscope}%
\pgfpathrectangle{\pgfqpoint{0.847223in}{0.554012in}}{\pgfqpoint{6.200000in}{4.530000in}}%
\pgfusepath{clip}%
\pgfsetbuttcap%
\pgfsetroundjoin%
\pgfsetlinewidth{1.003750pt}%
\definecolor{currentstroke}{rgb}{1.000000,0.000000,0.000000}%
\pgfsetstrokecolor{currentstroke}%
\pgfsetdash{}{0pt}%
\pgfpathmoveto{\pgfqpoint{1.156550in}{3.204543in}}%
\pgfpathcurveto{\pgfqpoint{1.167600in}{3.204543in}}{\pgfqpoint{1.178199in}{3.208933in}}{\pgfqpoint{1.186012in}{3.216747in}}%
\pgfpathcurveto{\pgfqpoint{1.193826in}{3.224561in}}{\pgfqpoint{1.198216in}{3.235160in}}{\pgfqpoint{1.198216in}{3.246210in}}%
\pgfpathcurveto{\pgfqpoint{1.198216in}{3.257260in}}{\pgfqpoint{1.193826in}{3.267859in}}{\pgfqpoint{1.186012in}{3.275673in}}%
\pgfpathcurveto{\pgfqpoint{1.178199in}{3.283486in}}{\pgfqpoint{1.167600in}{3.287876in}}{\pgfqpoint{1.156550in}{3.287876in}}%
\pgfpathcurveto{\pgfqpoint{1.145499in}{3.287876in}}{\pgfqpoint{1.134900in}{3.283486in}}{\pgfqpoint{1.127087in}{3.275673in}}%
\pgfpathcurveto{\pgfqpoint{1.119273in}{3.267859in}}{\pgfqpoint{1.114883in}{3.257260in}}{\pgfqpoint{1.114883in}{3.246210in}}%
\pgfpathcurveto{\pgfqpoint{1.114883in}{3.235160in}}{\pgfqpoint{1.119273in}{3.224561in}}{\pgfqpoint{1.127087in}{3.216747in}}%
\pgfpathcurveto{\pgfqpoint{1.134900in}{3.208933in}}{\pgfqpoint{1.145499in}{3.204543in}}{\pgfqpoint{1.156550in}{3.204543in}}%
\pgfpathlineto{\pgfqpoint{1.156550in}{3.204543in}}%
\pgfpathclose%
\pgfusepath{stroke}%
\end{pgfscope}%
\begin{pgfscope}%
\pgfpathrectangle{\pgfqpoint{0.847223in}{0.554012in}}{\pgfqpoint{6.200000in}{4.530000in}}%
\pgfusepath{clip}%
\pgfsetbuttcap%
\pgfsetroundjoin%
\pgfsetlinewidth{1.003750pt}%
\definecolor{currentstroke}{rgb}{1.000000,0.000000,0.000000}%
\pgfsetstrokecolor{currentstroke}%
\pgfsetdash{}{0pt}%
\pgfpathmoveto{\pgfqpoint{1.161883in}{3.184755in}}%
\pgfpathcurveto{\pgfqpoint{1.172933in}{3.184755in}}{\pgfqpoint{1.183532in}{3.189145in}}{\pgfqpoint{1.191346in}{3.196958in}}%
\pgfpathcurveto{\pgfqpoint{1.199159in}{3.204772in}}{\pgfqpoint{1.203549in}{3.215371in}}{\pgfqpoint{1.203549in}{3.226421in}}%
\pgfpathcurveto{\pgfqpoint{1.203549in}{3.237471in}}{\pgfqpoint{1.199159in}{3.248070in}}{\pgfqpoint{1.191346in}{3.255884in}}%
\pgfpathcurveto{\pgfqpoint{1.183532in}{3.263698in}}{\pgfqpoint{1.172933in}{3.268088in}}{\pgfqpoint{1.161883in}{3.268088in}}%
\pgfpathcurveto{\pgfqpoint{1.150833in}{3.268088in}}{\pgfqpoint{1.140234in}{3.263698in}}{\pgfqpoint{1.132420in}{3.255884in}}%
\pgfpathcurveto{\pgfqpoint{1.124606in}{3.248070in}}{\pgfqpoint{1.120216in}{3.237471in}}{\pgfqpoint{1.120216in}{3.226421in}}%
\pgfpathcurveto{\pgfqpoint{1.120216in}{3.215371in}}{\pgfqpoint{1.124606in}{3.204772in}}{\pgfqpoint{1.132420in}{3.196958in}}%
\pgfpathcurveto{\pgfqpoint{1.140234in}{3.189145in}}{\pgfqpoint{1.150833in}{3.184755in}}{\pgfqpoint{1.161883in}{3.184755in}}%
\pgfpathlineto{\pgfqpoint{1.161883in}{3.184755in}}%
\pgfpathclose%
\pgfusepath{stroke}%
\end{pgfscope}%
\begin{pgfscope}%
\pgfpathrectangle{\pgfqpoint{0.847223in}{0.554012in}}{\pgfqpoint{6.200000in}{4.530000in}}%
\pgfusepath{clip}%
\pgfsetbuttcap%
\pgfsetroundjoin%
\pgfsetlinewidth{1.003750pt}%
\definecolor{currentstroke}{rgb}{1.000000,0.000000,0.000000}%
\pgfsetstrokecolor{currentstroke}%
\pgfsetdash{}{0pt}%
\pgfpathmoveto{\pgfqpoint{1.167216in}{3.165216in}}%
\pgfpathcurveto{\pgfqpoint{1.178266in}{3.165216in}}{\pgfqpoint{1.188865in}{3.169607in}}{\pgfqpoint{1.196679in}{3.177420in}}%
\pgfpathcurveto{\pgfqpoint{1.204492in}{3.185234in}}{\pgfqpoint{1.208883in}{3.195833in}}{\pgfqpoint{1.208883in}{3.206883in}}%
\pgfpathcurveto{\pgfqpoint{1.208883in}{3.217933in}}{\pgfqpoint{1.204492in}{3.228532in}}{\pgfqpoint{1.196679in}{3.236346in}}%
\pgfpathcurveto{\pgfqpoint{1.188865in}{3.244159in}}{\pgfqpoint{1.178266in}{3.248550in}}{\pgfqpoint{1.167216in}{3.248550in}}%
\pgfpathcurveto{\pgfqpoint{1.156166in}{3.248550in}}{\pgfqpoint{1.145567in}{3.244159in}}{\pgfqpoint{1.137753in}{3.236346in}}%
\pgfpathcurveto{\pgfqpoint{1.129940in}{3.228532in}}{\pgfqpoint{1.125549in}{3.217933in}}{\pgfqpoint{1.125549in}{3.206883in}}%
\pgfpathcurveto{\pgfqpoint{1.125549in}{3.195833in}}{\pgfqpoint{1.129940in}{3.185234in}}{\pgfqpoint{1.137753in}{3.177420in}}%
\pgfpathcurveto{\pgfqpoint{1.145567in}{3.169607in}}{\pgfqpoint{1.156166in}{3.165216in}}{\pgfqpoint{1.167216in}{3.165216in}}%
\pgfpathlineto{\pgfqpoint{1.167216in}{3.165216in}}%
\pgfpathclose%
\pgfusepath{stroke}%
\end{pgfscope}%
\begin{pgfscope}%
\pgfpathrectangle{\pgfqpoint{0.847223in}{0.554012in}}{\pgfqpoint{6.200000in}{4.530000in}}%
\pgfusepath{clip}%
\pgfsetbuttcap%
\pgfsetroundjoin%
\pgfsetlinewidth{1.003750pt}%
\definecolor{currentstroke}{rgb}{1.000000,0.000000,0.000000}%
\pgfsetstrokecolor{currentstroke}%
\pgfsetdash{}{0pt}%
\pgfpathmoveto{\pgfqpoint{1.172549in}{3.145924in}}%
\pgfpathcurveto{\pgfqpoint{1.183599in}{3.145924in}}{\pgfqpoint{1.194198in}{3.150314in}}{\pgfqpoint{1.202012in}{3.158127in}}%
\pgfpathcurveto{\pgfqpoint{1.209826in}{3.165941in}}{\pgfqpoint{1.214216in}{3.176540in}}{\pgfqpoint{1.214216in}{3.187590in}}%
\pgfpathcurveto{\pgfqpoint{1.214216in}{3.198640in}}{\pgfqpoint{1.209826in}{3.209239in}}{\pgfqpoint{1.202012in}{3.217053in}}%
\pgfpathcurveto{\pgfqpoint{1.194198in}{3.224867in}}{\pgfqpoint{1.183599in}{3.229257in}}{\pgfqpoint{1.172549in}{3.229257in}}%
\pgfpathcurveto{\pgfqpoint{1.161499in}{3.229257in}}{\pgfqpoint{1.150900in}{3.224867in}}{\pgfqpoint{1.143086in}{3.217053in}}%
\pgfpathcurveto{\pgfqpoint{1.135273in}{3.209239in}}{\pgfqpoint{1.130883in}{3.198640in}}{\pgfqpoint{1.130883in}{3.187590in}}%
\pgfpathcurveto{\pgfqpoint{1.130883in}{3.176540in}}{\pgfqpoint{1.135273in}{3.165941in}}{\pgfqpoint{1.143086in}{3.158127in}}%
\pgfpathcurveto{\pgfqpoint{1.150900in}{3.150314in}}{\pgfqpoint{1.161499in}{3.145924in}}{\pgfqpoint{1.172549in}{3.145924in}}%
\pgfpathlineto{\pgfqpoint{1.172549in}{3.145924in}}%
\pgfpathclose%
\pgfusepath{stroke}%
\end{pgfscope}%
\begin{pgfscope}%
\pgfpathrectangle{\pgfqpoint{0.847223in}{0.554012in}}{\pgfqpoint{6.200000in}{4.530000in}}%
\pgfusepath{clip}%
\pgfsetbuttcap%
\pgfsetroundjoin%
\pgfsetlinewidth{1.003750pt}%
\definecolor{currentstroke}{rgb}{1.000000,0.000000,0.000000}%
\pgfsetstrokecolor{currentstroke}%
\pgfsetdash{}{0pt}%
\pgfpathmoveto{\pgfqpoint{1.177882in}{3.126872in}}%
\pgfpathcurveto{\pgfqpoint{1.188933in}{3.126872in}}{\pgfqpoint{1.199532in}{3.131262in}}{\pgfqpoint{1.207345in}{3.139076in}}%
\pgfpathcurveto{\pgfqpoint{1.215159in}{3.146889in}}{\pgfqpoint{1.219549in}{3.157488in}}{\pgfqpoint{1.219549in}{3.168538in}}%
\pgfpathcurveto{\pgfqpoint{1.219549in}{3.179589in}}{\pgfqpoint{1.215159in}{3.190188in}}{\pgfqpoint{1.207345in}{3.198001in}}%
\pgfpathcurveto{\pgfqpoint{1.199532in}{3.205815in}}{\pgfqpoint{1.188933in}{3.210205in}}{\pgfqpoint{1.177882in}{3.210205in}}%
\pgfpathcurveto{\pgfqpoint{1.166832in}{3.210205in}}{\pgfqpoint{1.156233in}{3.205815in}}{\pgfqpoint{1.148420in}{3.198001in}}%
\pgfpathcurveto{\pgfqpoint{1.140606in}{3.190188in}}{\pgfqpoint{1.136216in}{3.179589in}}{\pgfqpoint{1.136216in}{3.168538in}}%
\pgfpathcurveto{\pgfqpoint{1.136216in}{3.157488in}}{\pgfqpoint{1.140606in}{3.146889in}}{\pgfqpoint{1.148420in}{3.139076in}}%
\pgfpathcurveto{\pgfqpoint{1.156233in}{3.131262in}}{\pgfqpoint{1.166832in}{3.126872in}}{\pgfqpoint{1.177882in}{3.126872in}}%
\pgfpathlineto{\pgfqpoint{1.177882in}{3.126872in}}%
\pgfpathclose%
\pgfusepath{stroke}%
\end{pgfscope}%
\begin{pgfscope}%
\pgfpathrectangle{\pgfqpoint{0.847223in}{0.554012in}}{\pgfqpoint{6.200000in}{4.530000in}}%
\pgfusepath{clip}%
\pgfsetbuttcap%
\pgfsetroundjoin%
\pgfsetlinewidth{1.003750pt}%
\definecolor{currentstroke}{rgb}{1.000000,0.000000,0.000000}%
\pgfsetstrokecolor{currentstroke}%
\pgfsetdash{}{0pt}%
\pgfpathmoveto{\pgfqpoint{1.183216in}{3.108057in}}%
\pgfpathcurveto{\pgfqpoint{1.194266in}{3.108057in}}{\pgfqpoint{1.204865in}{3.112447in}}{\pgfqpoint{1.212678in}{3.120260in}}%
\pgfpathcurveto{\pgfqpoint{1.220492in}{3.128074in}}{\pgfqpoint{1.224882in}{3.138673in}}{\pgfqpoint{1.224882in}{3.149723in}}%
\pgfpathcurveto{\pgfqpoint{1.224882in}{3.160773in}}{\pgfqpoint{1.220492in}{3.171372in}}{\pgfqpoint{1.212678in}{3.179186in}}%
\pgfpathcurveto{\pgfqpoint{1.204865in}{3.187000in}}{\pgfqpoint{1.194266in}{3.191390in}}{\pgfqpoint{1.183216in}{3.191390in}}%
\pgfpathcurveto{\pgfqpoint{1.172166in}{3.191390in}}{\pgfqpoint{1.161566in}{3.187000in}}{\pgfqpoint{1.153753in}{3.179186in}}%
\pgfpathcurveto{\pgfqpoint{1.145939in}{3.171372in}}{\pgfqpoint{1.141549in}{3.160773in}}{\pgfqpoint{1.141549in}{3.149723in}}%
\pgfpathcurveto{\pgfqpoint{1.141549in}{3.138673in}}{\pgfqpoint{1.145939in}{3.128074in}}{\pgfqpoint{1.153753in}{3.120260in}}%
\pgfpathcurveto{\pgfqpoint{1.161566in}{3.112447in}}{\pgfqpoint{1.172166in}{3.108057in}}{\pgfqpoint{1.183216in}{3.108057in}}%
\pgfpathlineto{\pgfqpoint{1.183216in}{3.108057in}}%
\pgfpathclose%
\pgfusepath{stroke}%
\end{pgfscope}%
\begin{pgfscope}%
\pgfpathrectangle{\pgfqpoint{0.847223in}{0.554012in}}{\pgfqpoint{6.200000in}{4.530000in}}%
\pgfusepath{clip}%
\pgfsetbuttcap%
\pgfsetroundjoin%
\pgfsetlinewidth{1.003750pt}%
\definecolor{currentstroke}{rgb}{1.000000,0.000000,0.000000}%
\pgfsetstrokecolor{currentstroke}%
\pgfsetdash{}{0pt}%
\pgfpathmoveto{\pgfqpoint{1.188549in}{3.089473in}}%
\pgfpathcurveto{\pgfqpoint{1.199599in}{3.089473in}}{\pgfqpoint{1.210198in}{3.093864in}}{\pgfqpoint{1.218012in}{3.101677in}}%
\pgfpathcurveto{\pgfqpoint{1.225825in}{3.109491in}}{\pgfqpoint{1.230216in}{3.120090in}}{\pgfqpoint{1.230216in}{3.131140in}}%
\pgfpathcurveto{\pgfqpoint{1.230216in}{3.142190in}}{\pgfqpoint{1.225825in}{3.152789in}}{\pgfqpoint{1.218012in}{3.160603in}}%
\pgfpathcurveto{\pgfqpoint{1.210198in}{3.168416in}}{\pgfqpoint{1.199599in}{3.172807in}}{\pgfqpoint{1.188549in}{3.172807in}}%
\pgfpathcurveto{\pgfqpoint{1.177499in}{3.172807in}}{\pgfqpoint{1.166900in}{3.168416in}}{\pgfqpoint{1.159086in}{3.160603in}}%
\pgfpathcurveto{\pgfqpoint{1.151272in}{3.152789in}}{\pgfqpoint{1.146882in}{3.142190in}}{\pgfqpoint{1.146882in}{3.131140in}}%
\pgfpathcurveto{\pgfqpoint{1.146882in}{3.120090in}}{\pgfqpoint{1.151272in}{3.109491in}}{\pgfqpoint{1.159086in}{3.101677in}}%
\pgfpathcurveto{\pgfqpoint{1.166900in}{3.093864in}}{\pgfqpoint{1.177499in}{3.089473in}}{\pgfqpoint{1.188549in}{3.089473in}}%
\pgfpathlineto{\pgfqpoint{1.188549in}{3.089473in}}%
\pgfpathclose%
\pgfusepath{stroke}%
\end{pgfscope}%
\begin{pgfscope}%
\pgfpathrectangle{\pgfqpoint{0.847223in}{0.554012in}}{\pgfqpoint{6.200000in}{4.530000in}}%
\pgfusepath{clip}%
\pgfsetbuttcap%
\pgfsetroundjoin%
\pgfsetlinewidth{1.003750pt}%
\definecolor{currentstroke}{rgb}{1.000000,0.000000,0.000000}%
\pgfsetstrokecolor{currentstroke}%
\pgfsetdash{}{0pt}%
\pgfpathmoveto{\pgfqpoint{1.193882in}{3.071118in}}%
\pgfpathcurveto{\pgfqpoint{1.204932in}{3.071118in}}{\pgfqpoint{1.215531in}{3.075508in}}{\pgfqpoint{1.223345in}{3.083322in}}%
\pgfpathcurveto{\pgfqpoint{1.231158in}{3.091136in}}{\pgfqpoint{1.235549in}{3.101735in}}{\pgfqpoint{1.235549in}{3.112785in}}%
\pgfpathcurveto{\pgfqpoint{1.235549in}{3.123835in}}{\pgfqpoint{1.231158in}{3.134434in}}{\pgfqpoint{1.223345in}{3.142248in}}%
\pgfpathcurveto{\pgfqpoint{1.215531in}{3.150061in}}{\pgfqpoint{1.204932in}{3.154451in}}{\pgfqpoint{1.193882in}{3.154451in}}%
\pgfpathcurveto{\pgfqpoint{1.182832in}{3.154451in}}{\pgfqpoint{1.172233in}{3.150061in}}{\pgfqpoint{1.164419in}{3.142248in}}%
\pgfpathcurveto{\pgfqpoint{1.156606in}{3.134434in}}{\pgfqpoint{1.152215in}{3.123835in}}{\pgfqpoint{1.152215in}{3.112785in}}%
\pgfpathcurveto{\pgfqpoint{1.152215in}{3.101735in}}{\pgfqpoint{1.156606in}{3.091136in}}{\pgfqpoint{1.164419in}{3.083322in}}%
\pgfpathcurveto{\pgfqpoint{1.172233in}{3.075508in}}{\pgfqpoint{1.182832in}{3.071118in}}{\pgfqpoint{1.193882in}{3.071118in}}%
\pgfpathlineto{\pgfqpoint{1.193882in}{3.071118in}}%
\pgfpathclose%
\pgfusepath{stroke}%
\end{pgfscope}%
\begin{pgfscope}%
\pgfpathrectangle{\pgfqpoint{0.847223in}{0.554012in}}{\pgfqpoint{6.200000in}{4.530000in}}%
\pgfusepath{clip}%
\pgfsetbuttcap%
\pgfsetroundjoin%
\pgfsetlinewidth{1.003750pt}%
\definecolor{currentstroke}{rgb}{1.000000,0.000000,0.000000}%
\pgfsetstrokecolor{currentstroke}%
\pgfsetdash{}{0pt}%
\pgfpathmoveto{\pgfqpoint{1.199215in}{3.052986in}}%
\pgfpathcurveto{\pgfqpoint{1.210265in}{3.052986in}}{\pgfqpoint{1.220864in}{3.057377in}}{\pgfqpoint{1.228678in}{3.065190in}}%
\pgfpathcurveto{\pgfqpoint{1.236492in}{3.073004in}}{\pgfqpoint{1.240882in}{3.083603in}}{\pgfqpoint{1.240882in}{3.094653in}}%
\pgfpathcurveto{\pgfqpoint{1.240882in}{3.105703in}}{\pgfqpoint{1.236492in}{3.116302in}}{\pgfqpoint{1.228678in}{3.124116in}}%
\pgfpathcurveto{\pgfqpoint{1.220864in}{3.131929in}}{\pgfqpoint{1.210265in}{3.136320in}}{\pgfqpoint{1.199215in}{3.136320in}}%
\pgfpathcurveto{\pgfqpoint{1.188165in}{3.136320in}}{\pgfqpoint{1.177566in}{3.131929in}}{\pgfqpoint{1.169753in}{3.124116in}}%
\pgfpathcurveto{\pgfqpoint{1.161939in}{3.116302in}}{\pgfqpoint{1.157549in}{3.105703in}}{\pgfqpoint{1.157549in}{3.094653in}}%
\pgfpathcurveto{\pgfqpoint{1.157549in}{3.083603in}}{\pgfqpoint{1.161939in}{3.073004in}}{\pgfqpoint{1.169753in}{3.065190in}}%
\pgfpathcurveto{\pgfqpoint{1.177566in}{3.057377in}}{\pgfqpoint{1.188165in}{3.052986in}}{\pgfqpoint{1.199215in}{3.052986in}}%
\pgfpathlineto{\pgfqpoint{1.199215in}{3.052986in}}%
\pgfpathclose%
\pgfusepath{stroke}%
\end{pgfscope}%
\begin{pgfscope}%
\pgfpathrectangle{\pgfqpoint{0.847223in}{0.554012in}}{\pgfqpoint{6.200000in}{4.530000in}}%
\pgfusepath{clip}%
\pgfsetbuttcap%
\pgfsetroundjoin%
\pgfsetlinewidth{1.003750pt}%
\definecolor{currentstroke}{rgb}{1.000000,0.000000,0.000000}%
\pgfsetstrokecolor{currentstroke}%
\pgfsetdash{}{0pt}%
\pgfpathmoveto{\pgfqpoint{1.204549in}{3.035074in}}%
\pgfpathcurveto{\pgfqpoint{1.215599in}{3.035074in}}{\pgfqpoint{1.226198in}{3.039465in}}{\pgfqpoint{1.234011in}{3.047278in}}%
\pgfpathcurveto{\pgfqpoint{1.241825in}{3.055092in}}{\pgfqpoint{1.246215in}{3.065691in}}{\pgfqpoint{1.246215in}{3.076741in}}%
\pgfpathcurveto{\pgfqpoint{1.246215in}{3.087791in}}{\pgfqpoint{1.241825in}{3.098390in}}{\pgfqpoint{1.234011in}{3.106204in}}%
\pgfpathcurveto{\pgfqpoint{1.226198in}{3.114017in}}{\pgfqpoint{1.215599in}{3.118408in}}{\pgfqpoint{1.204549in}{3.118408in}}%
\pgfpathcurveto{\pgfqpoint{1.193498in}{3.118408in}}{\pgfqpoint{1.182899in}{3.114017in}}{\pgfqpoint{1.175086in}{3.106204in}}%
\pgfpathcurveto{\pgfqpoint{1.167272in}{3.098390in}}{\pgfqpoint{1.162882in}{3.087791in}}{\pgfqpoint{1.162882in}{3.076741in}}%
\pgfpathcurveto{\pgfqpoint{1.162882in}{3.065691in}}{\pgfqpoint{1.167272in}{3.055092in}}{\pgfqpoint{1.175086in}{3.047278in}}%
\pgfpathcurveto{\pgfqpoint{1.182899in}{3.039465in}}{\pgfqpoint{1.193498in}{3.035074in}}{\pgfqpoint{1.204549in}{3.035074in}}%
\pgfpathlineto{\pgfqpoint{1.204549in}{3.035074in}}%
\pgfpathclose%
\pgfusepath{stroke}%
\end{pgfscope}%
\begin{pgfscope}%
\pgfpathrectangle{\pgfqpoint{0.847223in}{0.554012in}}{\pgfqpoint{6.200000in}{4.530000in}}%
\pgfusepath{clip}%
\pgfsetbuttcap%
\pgfsetroundjoin%
\pgfsetlinewidth{1.003750pt}%
\definecolor{currentstroke}{rgb}{1.000000,0.000000,0.000000}%
\pgfsetstrokecolor{currentstroke}%
\pgfsetdash{}{0pt}%
\pgfpathmoveto{\pgfqpoint{1.209882in}{3.017378in}}%
\pgfpathcurveto{\pgfqpoint{1.220932in}{3.017378in}}{\pgfqpoint{1.231531in}{3.021768in}}{\pgfqpoint{1.239345in}{3.029582in}}%
\pgfpathcurveto{\pgfqpoint{1.247158in}{3.037395in}}{\pgfqpoint{1.251548in}{3.047994in}}{\pgfqpoint{1.251548in}{3.059045in}}%
\pgfpathcurveto{\pgfqpoint{1.251548in}{3.070095in}}{\pgfqpoint{1.247158in}{3.080694in}}{\pgfqpoint{1.239345in}{3.088507in}}%
\pgfpathcurveto{\pgfqpoint{1.231531in}{3.096321in}}{\pgfqpoint{1.220932in}{3.100711in}}{\pgfqpoint{1.209882in}{3.100711in}}%
\pgfpathcurveto{\pgfqpoint{1.198832in}{3.100711in}}{\pgfqpoint{1.188233in}{3.096321in}}{\pgfqpoint{1.180419in}{3.088507in}}%
\pgfpathcurveto{\pgfqpoint{1.172605in}{3.080694in}}{\pgfqpoint{1.168215in}{3.070095in}}{\pgfqpoint{1.168215in}{3.059045in}}%
\pgfpathcurveto{\pgfqpoint{1.168215in}{3.047994in}}{\pgfqpoint{1.172605in}{3.037395in}}{\pgfqpoint{1.180419in}{3.029582in}}%
\pgfpathcurveto{\pgfqpoint{1.188233in}{3.021768in}}{\pgfqpoint{1.198832in}{3.017378in}}{\pgfqpoint{1.209882in}{3.017378in}}%
\pgfpathlineto{\pgfqpoint{1.209882in}{3.017378in}}%
\pgfpathclose%
\pgfusepath{stroke}%
\end{pgfscope}%
\begin{pgfscope}%
\pgfpathrectangle{\pgfqpoint{0.847223in}{0.554012in}}{\pgfqpoint{6.200000in}{4.530000in}}%
\pgfusepath{clip}%
\pgfsetbuttcap%
\pgfsetroundjoin%
\pgfsetlinewidth{1.003750pt}%
\definecolor{currentstroke}{rgb}{1.000000,0.000000,0.000000}%
\pgfsetstrokecolor{currentstroke}%
\pgfsetdash{}{0pt}%
\pgfpathmoveto{\pgfqpoint{1.215215in}{2.999893in}}%
\pgfpathcurveto{\pgfqpoint{1.226265in}{2.999893in}}{\pgfqpoint{1.236864in}{3.004284in}}{\pgfqpoint{1.244678in}{3.012097in}}%
\pgfpathcurveto{\pgfqpoint{1.252491in}{3.019911in}}{\pgfqpoint{1.256882in}{3.030510in}}{\pgfqpoint{1.256882in}{3.041560in}}%
\pgfpathcurveto{\pgfqpoint{1.256882in}{3.052610in}}{\pgfqpoint{1.252491in}{3.063209in}}{\pgfqpoint{1.244678in}{3.071023in}}%
\pgfpathcurveto{\pgfqpoint{1.236864in}{3.078836in}}{\pgfqpoint{1.226265in}{3.083227in}}{\pgfqpoint{1.215215in}{3.083227in}}%
\pgfpathcurveto{\pgfqpoint{1.204165in}{3.083227in}}{\pgfqpoint{1.193566in}{3.078836in}}{\pgfqpoint{1.185752in}{3.071023in}}%
\pgfpathcurveto{\pgfqpoint{1.177939in}{3.063209in}}{\pgfqpoint{1.173548in}{3.052610in}}{\pgfqpoint{1.173548in}{3.041560in}}%
\pgfpathcurveto{\pgfqpoint{1.173548in}{3.030510in}}{\pgfqpoint{1.177939in}{3.019911in}}{\pgfqpoint{1.185752in}{3.012097in}}%
\pgfpathcurveto{\pgfqpoint{1.193566in}{3.004284in}}{\pgfqpoint{1.204165in}{2.999893in}}{\pgfqpoint{1.215215in}{2.999893in}}%
\pgfpathlineto{\pgfqpoint{1.215215in}{2.999893in}}%
\pgfpathclose%
\pgfusepath{stroke}%
\end{pgfscope}%
\begin{pgfscope}%
\pgfpathrectangle{\pgfqpoint{0.847223in}{0.554012in}}{\pgfqpoint{6.200000in}{4.530000in}}%
\pgfusepath{clip}%
\pgfsetbuttcap%
\pgfsetroundjoin%
\pgfsetlinewidth{1.003750pt}%
\definecolor{currentstroke}{rgb}{1.000000,0.000000,0.000000}%
\pgfsetstrokecolor{currentstroke}%
\pgfsetdash{}{0pt}%
\pgfpathmoveto{\pgfqpoint{1.220548in}{2.982617in}}%
\pgfpathcurveto{\pgfqpoint{1.231598in}{2.982617in}}{\pgfqpoint{1.242197in}{2.987007in}}{\pgfqpoint{1.250011in}{2.994820in}}%
\pgfpathcurveto{\pgfqpoint{1.257825in}{3.002634in}}{\pgfqpoint{1.262215in}{3.013233in}}{\pgfqpoint{1.262215in}{3.024283in}}%
\pgfpathcurveto{\pgfqpoint{1.262215in}{3.035333in}}{\pgfqpoint{1.257825in}{3.045932in}}{\pgfqpoint{1.250011in}{3.053746in}}%
\pgfpathcurveto{\pgfqpoint{1.242197in}{3.061560in}}{\pgfqpoint{1.231598in}{3.065950in}}{\pgfqpoint{1.220548in}{3.065950in}}%
\pgfpathcurveto{\pgfqpoint{1.209498in}{3.065950in}}{\pgfqpoint{1.198899in}{3.061560in}}{\pgfqpoint{1.191085in}{3.053746in}}%
\pgfpathcurveto{\pgfqpoint{1.183272in}{3.045932in}}{\pgfqpoint{1.178881in}{3.035333in}}{\pgfqpoint{1.178881in}{3.024283in}}%
\pgfpathcurveto{\pgfqpoint{1.178881in}{3.013233in}}{\pgfqpoint{1.183272in}{3.002634in}}{\pgfqpoint{1.191085in}{2.994820in}}%
\pgfpathcurveto{\pgfqpoint{1.198899in}{2.987007in}}{\pgfqpoint{1.209498in}{2.982617in}}{\pgfqpoint{1.220548in}{2.982617in}}%
\pgfpathlineto{\pgfqpoint{1.220548in}{2.982617in}}%
\pgfpathclose%
\pgfusepath{stroke}%
\end{pgfscope}%
\begin{pgfscope}%
\pgfpathrectangle{\pgfqpoint{0.847223in}{0.554012in}}{\pgfqpoint{6.200000in}{4.530000in}}%
\pgfusepath{clip}%
\pgfsetbuttcap%
\pgfsetroundjoin%
\pgfsetlinewidth{1.003750pt}%
\definecolor{currentstroke}{rgb}{1.000000,0.000000,0.000000}%
\pgfsetstrokecolor{currentstroke}%
\pgfsetdash{}{0pt}%
\pgfpathmoveto{\pgfqpoint{1.225881in}{2.965544in}}%
\pgfpathcurveto{\pgfqpoint{1.236931in}{2.965544in}}{\pgfqpoint{1.247531in}{2.969934in}}{\pgfqpoint{1.255344in}{2.977748in}}%
\pgfpathcurveto{\pgfqpoint{1.263158in}{2.985562in}}{\pgfqpoint{1.267548in}{2.996161in}}{\pgfqpoint{1.267548in}{3.007211in}}%
\pgfpathcurveto{\pgfqpoint{1.267548in}{3.018261in}}{\pgfqpoint{1.263158in}{3.028860in}}{\pgfqpoint{1.255344in}{3.036674in}}%
\pgfpathcurveto{\pgfqpoint{1.247531in}{3.044487in}}{\pgfqpoint{1.236931in}{3.048878in}}{\pgfqpoint{1.225881in}{3.048878in}}%
\pgfpathcurveto{\pgfqpoint{1.214831in}{3.048878in}}{\pgfqpoint{1.204232in}{3.044487in}}{\pgfqpoint{1.196419in}{3.036674in}}%
\pgfpathcurveto{\pgfqpoint{1.188605in}{3.028860in}}{\pgfqpoint{1.184215in}{3.018261in}}{\pgfqpoint{1.184215in}{3.007211in}}%
\pgfpathcurveto{\pgfqpoint{1.184215in}{2.996161in}}{\pgfqpoint{1.188605in}{2.985562in}}{\pgfqpoint{1.196419in}{2.977748in}}%
\pgfpathcurveto{\pgfqpoint{1.204232in}{2.969934in}}{\pgfqpoint{1.214831in}{2.965544in}}{\pgfqpoint{1.225881in}{2.965544in}}%
\pgfpathlineto{\pgfqpoint{1.225881in}{2.965544in}}%
\pgfpathclose%
\pgfusepath{stroke}%
\end{pgfscope}%
\begin{pgfscope}%
\pgfpathrectangle{\pgfqpoint{0.847223in}{0.554012in}}{\pgfqpoint{6.200000in}{4.530000in}}%
\pgfusepath{clip}%
\pgfsetbuttcap%
\pgfsetroundjoin%
\pgfsetlinewidth{1.003750pt}%
\definecolor{currentstroke}{rgb}{1.000000,0.000000,0.000000}%
\pgfsetstrokecolor{currentstroke}%
\pgfsetdash{}{0pt}%
\pgfpathmoveto{\pgfqpoint{1.231215in}{2.948673in}}%
\pgfpathcurveto{\pgfqpoint{1.242265in}{2.948673in}}{\pgfqpoint{1.252864in}{2.953063in}}{\pgfqpoint{1.260677in}{2.960876in}}%
\pgfpathcurveto{\pgfqpoint{1.268491in}{2.968690in}}{\pgfqpoint{1.272881in}{2.979289in}}{\pgfqpoint{1.272881in}{2.990339in}}%
\pgfpathcurveto{\pgfqpoint{1.272881in}{3.001389in}}{\pgfqpoint{1.268491in}{3.011988in}}{\pgfqpoint{1.260677in}{3.019802in}}%
\pgfpathcurveto{\pgfqpoint{1.252864in}{3.027616in}}{\pgfqpoint{1.242265in}{3.032006in}}{\pgfqpoint{1.231215in}{3.032006in}}%
\pgfpathcurveto{\pgfqpoint{1.220164in}{3.032006in}}{\pgfqpoint{1.209565in}{3.027616in}}{\pgfqpoint{1.201752in}{3.019802in}}%
\pgfpathcurveto{\pgfqpoint{1.193938in}{3.011988in}}{\pgfqpoint{1.189548in}{3.001389in}}{\pgfqpoint{1.189548in}{2.990339in}}%
\pgfpathcurveto{\pgfqpoint{1.189548in}{2.979289in}}{\pgfqpoint{1.193938in}{2.968690in}}{\pgfqpoint{1.201752in}{2.960876in}}%
\pgfpathcurveto{\pgfqpoint{1.209565in}{2.953063in}}{\pgfqpoint{1.220164in}{2.948673in}}{\pgfqpoint{1.231215in}{2.948673in}}%
\pgfpathlineto{\pgfqpoint{1.231215in}{2.948673in}}%
\pgfpathclose%
\pgfusepath{stroke}%
\end{pgfscope}%
\begin{pgfscope}%
\pgfpathrectangle{\pgfqpoint{0.847223in}{0.554012in}}{\pgfqpoint{6.200000in}{4.530000in}}%
\pgfusepath{clip}%
\pgfsetbuttcap%
\pgfsetroundjoin%
\pgfsetlinewidth{1.003750pt}%
\definecolor{currentstroke}{rgb}{1.000000,0.000000,0.000000}%
\pgfsetstrokecolor{currentstroke}%
\pgfsetdash{}{0pt}%
\pgfpathmoveto{\pgfqpoint{1.236548in}{2.931998in}}%
\pgfpathcurveto{\pgfqpoint{1.247598in}{2.931998in}}{\pgfqpoint{1.258197in}{2.936388in}}{\pgfqpoint{1.266011in}{2.944202in}}%
\pgfpathcurveto{\pgfqpoint{1.273824in}{2.952016in}}{\pgfqpoint{1.278214in}{2.962615in}}{\pgfqpoint{1.278214in}{2.973665in}}%
\pgfpathcurveto{\pgfqpoint{1.278214in}{2.984715in}}{\pgfqpoint{1.273824in}{2.995314in}}{\pgfqpoint{1.266011in}{3.003128in}}%
\pgfpathcurveto{\pgfqpoint{1.258197in}{3.010941in}}{\pgfqpoint{1.247598in}{3.015331in}}{\pgfqpoint{1.236548in}{3.015331in}}%
\pgfpathcurveto{\pgfqpoint{1.225498in}{3.015331in}}{\pgfqpoint{1.214899in}{3.010941in}}{\pgfqpoint{1.207085in}{3.003128in}}%
\pgfpathcurveto{\pgfqpoint{1.199271in}{2.995314in}}{\pgfqpoint{1.194881in}{2.984715in}}{\pgfqpoint{1.194881in}{2.973665in}}%
\pgfpathcurveto{\pgfqpoint{1.194881in}{2.962615in}}{\pgfqpoint{1.199271in}{2.952016in}}{\pgfqpoint{1.207085in}{2.944202in}}%
\pgfpathcurveto{\pgfqpoint{1.214899in}{2.936388in}}{\pgfqpoint{1.225498in}{2.931998in}}{\pgfqpoint{1.236548in}{2.931998in}}%
\pgfpathlineto{\pgfqpoint{1.236548in}{2.931998in}}%
\pgfpathclose%
\pgfusepath{stroke}%
\end{pgfscope}%
\begin{pgfscope}%
\pgfpathrectangle{\pgfqpoint{0.847223in}{0.554012in}}{\pgfqpoint{6.200000in}{4.530000in}}%
\pgfusepath{clip}%
\pgfsetbuttcap%
\pgfsetroundjoin%
\pgfsetlinewidth{1.003750pt}%
\definecolor{currentstroke}{rgb}{1.000000,0.000000,0.000000}%
\pgfsetstrokecolor{currentstroke}%
\pgfsetdash{}{0pt}%
\pgfpathmoveto{\pgfqpoint{1.241881in}{2.915517in}}%
\pgfpathcurveto{\pgfqpoint{1.252931in}{2.915517in}}{\pgfqpoint{1.263530in}{2.919908in}}{\pgfqpoint{1.271344in}{2.927721in}}%
\pgfpathcurveto{\pgfqpoint{1.279157in}{2.935535in}}{\pgfqpoint{1.283548in}{2.946134in}}{\pgfqpoint{1.283548in}{2.957184in}}%
\pgfpathcurveto{\pgfqpoint{1.283548in}{2.968234in}}{\pgfqpoint{1.279157in}{2.978833in}}{\pgfqpoint{1.271344in}{2.986647in}}%
\pgfpathcurveto{\pgfqpoint{1.263530in}{2.994460in}}{\pgfqpoint{1.252931in}{2.998851in}}{\pgfqpoint{1.241881in}{2.998851in}}%
\pgfpathcurveto{\pgfqpoint{1.230831in}{2.998851in}}{\pgfqpoint{1.220232in}{2.994460in}}{\pgfqpoint{1.212418in}{2.986647in}}%
\pgfpathcurveto{\pgfqpoint{1.204605in}{2.978833in}}{\pgfqpoint{1.200214in}{2.968234in}}{\pgfqpoint{1.200214in}{2.957184in}}%
\pgfpathcurveto{\pgfqpoint{1.200214in}{2.946134in}}{\pgfqpoint{1.204605in}{2.935535in}}{\pgfqpoint{1.212418in}{2.927721in}}%
\pgfpathcurveto{\pgfqpoint{1.220232in}{2.919908in}}{\pgfqpoint{1.230831in}{2.915517in}}{\pgfqpoint{1.241881in}{2.915517in}}%
\pgfpathlineto{\pgfqpoint{1.241881in}{2.915517in}}%
\pgfpathclose%
\pgfusepath{stroke}%
\end{pgfscope}%
\begin{pgfscope}%
\pgfpathrectangle{\pgfqpoint{0.847223in}{0.554012in}}{\pgfqpoint{6.200000in}{4.530000in}}%
\pgfusepath{clip}%
\pgfsetbuttcap%
\pgfsetroundjoin%
\pgfsetlinewidth{1.003750pt}%
\definecolor{currentstroke}{rgb}{1.000000,0.000000,0.000000}%
\pgfsetstrokecolor{currentstroke}%
\pgfsetdash{}{0pt}%
\pgfpathmoveto{\pgfqpoint{1.247214in}{2.899227in}}%
\pgfpathcurveto{\pgfqpoint{1.258264in}{2.899227in}}{\pgfqpoint{1.268863in}{2.903617in}}{\pgfqpoint{1.276677in}{2.911431in}}%
\pgfpathcurveto{\pgfqpoint{1.284491in}{2.919244in}}{\pgfqpoint{1.288881in}{2.929844in}}{\pgfqpoint{1.288881in}{2.940894in}}%
\pgfpathcurveto{\pgfqpoint{1.288881in}{2.951944in}}{\pgfqpoint{1.284491in}{2.962543in}}{\pgfqpoint{1.276677in}{2.970356in}}%
\pgfpathcurveto{\pgfqpoint{1.268863in}{2.978170in}}{\pgfqpoint{1.258264in}{2.982560in}}{\pgfqpoint{1.247214in}{2.982560in}}%
\pgfpathcurveto{\pgfqpoint{1.236164in}{2.982560in}}{\pgfqpoint{1.225565in}{2.978170in}}{\pgfqpoint{1.217751in}{2.970356in}}%
\pgfpathcurveto{\pgfqpoint{1.209938in}{2.962543in}}{\pgfqpoint{1.205548in}{2.951944in}}{\pgfqpoint{1.205548in}{2.940894in}}%
\pgfpathcurveto{\pgfqpoint{1.205548in}{2.929844in}}{\pgfqpoint{1.209938in}{2.919244in}}{\pgfqpoint{1.217751in}{2.911431in}}%
\pgfpathcurveto{\pgfqpoint{1.225565in}{2.903617in}}{\pgfqpoint{1.236164in}{2.899227in}}{\pgfqpoint{1.247214in}{2.899227in}}%
\pgfpathlineto{\pgfqpoint{1.247214in}{2.899227in}}%
\pgfpathclose%
\pgfusepath{stroke}%
\end{pgfscope}%
\begin{pgfscope}%
\pgfpathrectangle{\pgfqpoint{0.847223in}{0.554012in}}{\pgfqpoint{6.200000in}{4.530000in}}%
\pgfusepath{clip}%
\pgfsetbuttcap%
\pgfsetroundjoin%
\pgfsetlinewidth{1.003750pt}%
\definecolor{currentstroke}{rgb}{1.000000,0.000000,0.000000}%
\pgfsetstrokecolor{currentstroke}%
\pgfsetdash{}{0pt}%
\pgfpathmoveto{\pgfqpoint{1.252547in}{2.883124in}}%
\pgfpathcurveto{\pgfqpoint{1.263598in}{2.883124in}}{\pgfqpoint{1.274197in}{2.887514in}}{\pgfqpoint{1.282010in}{2.895328in}}%
\pgfpathcurveto{\pgfqpoint{1.289824in}{2.903141in}}{\pgfqpoint{1.294214in}{2.913740in}}{\pgfqpoint{1.294214in}{2.924790in}}%
\pgfpathcurveto{\pgfqpoint{1.294214in}{2.935841in}}{\pgfqpoint{1.289824in}{2.946440in}}{\pgfqpoint{1.282010in}{2.954253in}}%
\pgfpathcurveto{\pgfqpoint{1.274197in}{2.962067in}}{\pgfqpoint{1.263598in}{2.966457in}}{\pgfqpoint{1.252547in}{2.966457in}}%
\pgfpathcurveto{\pgfqpoint{1.241497in}{2.966457in}}{\pgfqpoint{1.230898in}{2.962067in}}{\pgfqpoint{1.223085in}{2.954253in}}%
\pgfpathcurveto{\pgfqpoint{1.215271in}{2.946440in}}{\pgfqpoint{1.210881in}{2.935841in}}{\pgfqpoint{1.210881in}{2.924790in}}%
\pgfpathcurveto{\pgfqpoint{1.210881in}{2.913740in}}{\pgfqpoint{1.215271in}{2.903141in}}{\pgfqpoint{1.223085in}{2.895328in}}%
\pgfpathcurveto{\pgfqpoint{1.230898in}{2.887514in}}{\pgfqpoint{1.241497in}{2.883124in}}{\pgfqpoint{1.252547in}{2.883124in}}%
\pgfpathlineto{\pgfqpoint{1.252547in}{2.883124in}}%
\pgfpathclose%
\pgfusepath{stroke}%
\end{pgfscope}%
\begin{pgfscope}%
\pgfpathrectangle{\pgfqpoint{0.847223in}{0.554012in}}{\pgfqpoint{6.200000in}{4.530000in}}%
\pgfusepath{clip}%
\pgfsetbuttcap%
\pgfsetroundjoin%
\pgfsetlinewidth{1.003750pt}%
\definecolor{currentstroke}{rgb}{1.000000,0.000000,0.000000}%
\pgfsetstrokecolor{currentstroke}%
\pgfsetdash{}{0pt}%
\pgfpathmoveto{\pgfqpoint{1.257881in}{2.867204in}}%
\pgfpathcurveto{\pgfqpoint{1.268931in}{2.867204in}}{\pgfqpoint{1.279530in}{2.871595in}}{\pgfqpoint{1.287343in}{2.879408in}}%
\pgfpathcurveto{\pgfqpoint{1.295157in}{2.887222in}}{\pgfqpoint{1.299547in}{2.897821in}}{\pgfqpoint{1.299547in}{2.908871in}}%
\pgfpathcurveto{\pgfqpoint{1.299547in}{2.919921in}}{\pgfqpoint{1.295157in}{2.930520in}}{\pgfqpoint{1.287343in}{2.938334in}}%
\pgfpathcurveto{\pgfqpoint{1.279530in}{2.946147in}}{\pgfqpoint{1.268931in}{2.950538in}}{\pgfqpoint{1.257881in}{2.950538in}}%
\pgfpathcurveto{\pgfqpoint{1.246831in}{2.950538in}}{\pgfqpoint{1.236232in}{2.946147in}}{\pgfqpoint{1.228418in}{2.938334in}}%
\pgfpathcurveto{\pgfqpoint{1.220604in}{2.930520in}}{\pgfqpoint{1.216214in}{2.919921in}}{\pgfqpoint{1.216214in}{2.908871in}}%
\pgfpathcurveto{\pgfqpoint{1.216214in}{2.897821in}}{\pgfqpoint{1.220604in}{2.887222in}}{\pgfqpoint{1.228418in}{2.879408in}}%
\pgfpathcurveto{\pgfqpoint{1.236232in}{2.871595in}}{\pgfqpoint{1.246831in}{2.867204in}}{\pgfqpoint{1.257881in}{2.867204in}}%
\pgfpathlineto{\pgfqpoint{1.257881in}{2.867204in}}%
\pgfpathclose%
\pgfusepath{stroke}%
\end{pgfscope}%
\begin{pgfscope}%
\pgfpathrectangle{\pgfqpoint{0.847223in}{0.554012in}}{\pgfqpoint{6.200000in}{4.530000in}}%
\pgfusepath{clip}%
\pgfsetbuttcap%
\pgfsetroundjoin%
\pgfsetlinewidth{1.003750pt}%
\definecolor{currentstroke}{rgb}{1.000000,0.000000,0.000000}%
\pgfsetstrokecolor{currentstroke}%
\pgfsetdash{}{0pt}%
\pgfpathmoveto{\pgfqpoint{1.263214in}{2.851466in}}%
\pgfpathcurveto{\pgfqpoint{1.274264in}{2.851466in}}{\pgfqpoint{1.284863in}{2.855856in}}{\pgfqpoint{1.292677in}{2.863670in}}%
\pgfpathcurveto{\pgfqpoint{1.300490in}{2.871483in}}{\pgfqpoint{1.304881in}{2.882082in}}{\pgfqpoint{1.304881in}{2.893132in}}%
\pgfpathcurveto{\pgfqpoint{1.304881in}{2.904183in}}{\pgfqpoint{1.300490in}{2.914782in}}{\pgfqpoint{1.292677in}{2.922595in}}%
\pgfpathcurveto{\pgfqpoint{1.284863in}{2.930409in}}{\pgfqpoint{1.274264in}{2.934799in}}{\pgfqpoint{1.263214in}{2.934799in}}%
\pgfpathcurveto{\pgfqpoint{1.252164in}{2.934799in}}{\pgfqpoint{1.241565in}{2.930409in}}{\pgfqpoint{1.233751in}{2.922595in}}%
\pgfpathcurveto{\pgfqpoint{1.225937in}{2.914782in}}{\pgfqpoint{1.221547in}{2.904183in}}{\pgfqpoint{1.221547in}{2.893132in}}%
\pgfpathcurveto{\pgfqpoint{1.221547in}{2.882082in}}{\pgfqpoint{1.225937in}{2.871483in}}{\pgfqpoint{1.233751in}{2.863670in}}%
\pgfpathcurveto{\pgfqpoint{1.241565in}{2.855856in}}{\pgfqpoint{1.252164in}{2.851466in}}{\pgfqpoint{1.263214in}{2.851466in}}%
\pgfpathlineto{\pgfqpoint{1.263214in}{2.851466in}}%
\pgfpathclose%
\pgfusepath{stroke}%
\end{pgfscope}%
\begin{pgfscope}%
\pgfpathrectangle{\pgfqpoint{0.847223in}{0.554012in}}{\pgfqpoint{6.200000in}{4.530000in}}%
\pgfusepath{clip}%
\pgfsetbuttcap%
\pgfsetroundjoin%
\pgfsetlinewidth{1.003750pt}%
\definecolor{currentstroke}{rgb}{1.000000,0.000000,0.000000}%
\pgfsetstrokecolor{currentstroke}%
\pgfsetdash{}{0pt}%
\pgfpathmoveto{\pgfqpoint{1.268547in}{2.835905in}}%
\pgfpathcurveto{\pgfqpoint{1.279597in}{2.835905in}}{\pgfqpoint{1.290196in}{2.840295in}}{\pgfqpoint{1.298010in}{2.848109in}}%
\pgfpathcurveto{\pgfqpoint{1.305823in}{2.855922in}}{\pgfqpoint{1.310214in}{2.866521in}}{\pgfqpoint{1.310214in}{2.877572in}}%
\pgfpathcurveto{\pgfqpoint{1.310214in}{2.888622in}}{\pgfqpoint{1.305823in}{2.899221in}}{\pgfqpoint{1.298010in}{2.907034in}}%
\pgfpathcurveto{\pgfqpoint{1.290196in}{2.914848in}}{\pgfqpoint{1.279597in}{2.919238in}}{\pgfqpoint{1.268547in}{2.919238in}}%
\pgfpathcurveto{\pgfqpoint{1.257497in}{2.919238in}}{\pgfqpoint{1.246898in}{2.914848in}}{\pgfqpoint{1.239084in}{2.907034in}}%
\pgfpathcurveto{\pgfqpoint{1.231271in}{2.899221in}}{\pgfqpoint{1.226880in}{2.888622in}}{\pgfqpoint{1.226880in}{2.877572in}}%
\pgfpathcurveto{\pgfqpoint{1.226880in}{2.866521in}}{\pgfqpoint{1.231271in}{2.855922in}}{\pgfqpoint{1.239084in}{2.848109in}}%
\pgfpathcurveto{\pgfqpoint{1.246898in}{2.840295in}}{\pgfqpoint{1.257497in}{2.835905in}}{\pgfqpoint{1.268547in}{2.835905in}}%
\pgfpathlineto{\pgfqpoint{1.268547in}{2.835905in}}%
\pgfpathclose%
\pgfusepath{stroke}%
\end{pgfscope}%
\begin{pgfscope}%
\pgfpathrectangle{\pgfqpoint{0.847223in}{0.554012in}}{\pgfqpoint{6.200000in}{4.530000in}}%
\pgfusepath{clip}%
\pgfsetbuttcap%
\pgfsetroundjoin%
\pgfsetlinewidth{1.003750pt}%
\definecolor{currentstroke}{rgb}{1.000000,0.000000,0.000000}%
\pgfsetstrokecolor{currentstroke}%
\pgfsetdash{}{0pt}%
\pgfpathmoveto{\pgfqpoint{1.273880in}{2.820519in}}%
\pgfpathcurveto{\pgfqpoint{1.284930in}{2.820519in}}{\pgfqpoint{1.295529in}{2.824909in}}{\pgfqpoint{1.303343in}{2.832723in}}%
\pgfpathcurveto{\pgfqpoint{1.311157in}{2.840536in}}{\pgfqpoint{1.315547in}{2.851135in}}{\pgfqpoint{1.315547in}{2.862185in}}%
\pgfpathcurveto{\pgfqpoint{1.315547in}{2.873236in}}{\pgfqpoint{1.311157in}{2.883835in}}{\pgfqpoint{1.303343in}{2.891648in}}%
\pgfpathcurveto{\pgfqpoint{1.295529in}{2.899462in}}{\pgfqpoint{1.284930in}{2.903852in}}{\pgfqpoint{1.273880in}{2.903852in}}%
\pgfpathcurveto{\pgfqpoint{1.262830in}{2.903852in}}{\pgfqpoint{1.252231in}{2.899462in}}{\pgfqpoint{1.244418in}{2.891648in}}%
\pgfpathcurveto{\pgfqpoint{1.236604in}{2.883835in}}{\pgfqpoint{1.232214in}{2.873236in}}{\pgfqpoint{1.232214in}{2.862185in}}%
\pgfpathcurveto{\pgfqpoint{1.232214in}{2.851135in}}{\pgfqpoint{1.236604in}{2.840536in}}{\pgfqpoint{1.244418in}{2.832723in}}%
\pgfpathcurveto{\pgfqpoint{1.252231in}{2.824909in}}{\pgfqpoint{1.262830in}{2.820519in}}{\pgfqpoint{1.273880in}{2.820519in}}%
\pgfpathlineto{\pgfqpoint{1.273880in}{2.820519in}}%
\pgfpathclose%
\pgfusepath{stroke}%
\end{pgfscope}%
\begin{pgfscope}%
\pgfpathrectangle{\pgfqpoint{0.847223in}{0.554012in}}{\pgfqpoint{6.200000in}{4.530000in}}%
\pgfusepath{clip}%
\pgfsetbuttcap%
\pgfsetroundjoin%
\pgfsetlinewidth{1.003750pt}%
\definecolor{currentstroke}{rgb}{1.000000,0.000000,0.000000}%
\pgfsetstrokecolor{currentstroke}%
\pgfsetdash{}{0pt}%
\pgfpathmoveto{\pgfqpoint{1.279214in}{2.805304in}}%
\pgfpathcurveto{\pgfqpoint{1.290264in}{2.805304in}}{\pgfqpoint{1.300863in}{2.809695in}}{\pgfqpoint{1.308676in}{2.817508in}}%
\pgfpathcurveto{\pgfqpoint{1.316490in}{2.825322in}}{\pgfqpoint{1.320880in}{2.835921in}}{\pgfqpoint{1.320880in}{2.846971in}}%
\pgfpathcurveto{\pgfqpoint{1.320880in}{2.858021in}}{\pgfqpoint{1.316490in}{2.868620in}}{\pgfqpoint{1.308676in}{2.876434in}}%
\pgfpathcurveto{\pgfqpoint{1.300863in}{2.884247in}}{\pgfqpoint{1.290264in}{2.888638in}}{\pgfqpoint{1.279214in}{2.888638in}}%
\pgfpathcurveto{\pgfqpoint{1.268163in}{2.888638in}}{\pgfqpoint{1.257564in}{2.884247in}}{\pgfqpoint{1.249751in}{2.876434in}}%
\pgfpathcurveto{\pgfqpoint{1.241937in}{2.868620in}}{\pgfqpoint{1.237547in}{2.858021in}}{\pgfqpoint{1.237547in}{2.846971in}}%
\pgfpathcurveto{\pgfqpoint{1.237547in}{2.835921in}}{\pgfqpoint{1.241937in}{2.825322in}}{\pgfqpoint{1.249751in}{2.817508in}}%
\pgfpathcurveto{\pgfqpoint{1.257564in}{2.809695in}}{\pgfqpoint{1.268163in}{2.805304in}}{\pgfqpoint{1.279214in}{2.805304in}}%
\pgfpathlineto{\pgfqpoint{1.279214in}{2.805304in}}%
\pgfpathclose%
\pgfusepath{stroke}%
\end{pgfscope}%
\begin{pgfscope}%
\pgfpathrectangle{\pgfqpoint{0.847223in}{0.554012in}}{\pgfqpoint{6.200000in}{4.530000in}}%
\pgfusepath{clip}%
\pgfsetbuttcap%
\pgfsetroundjoin%
\pgfsetlinewidth{1.003750pt}%
\definecolor{currentstroke}{rgb}{1.000000,0.000000,0.000000}%
\pgfsetstrokecolor{currentstroke}%
\pgfsetdash{}{0pt}%
\pgfpathmoveto{\pgfqpoint{1.284547in}{2.790259in}}%
\pgfpathcurveto{\pgfqpoint{1.295597in}{2.790259in}}{\pgfqpoint{1.306196in}{2.794649in}}{\pgfqpoint{1.314010in}{2.802463in}}%
\pgfpathcurveto{\pgfqpoint{1.321823in}{2.810276in}}{\pgfqpoint{1.326213in}{2.820875in}}{\pgfqpoint{1.326213in}{2.831926in}}%
\pgfpathcurveto{\pgfqpoint{1.326213in}{2.842976in}}{\pgfqpoint{1.321823in}{2.853575in}}{\pgfqpoint{1.314010in}{2.861388in}}%
\pgfpathcurveto{\pgfqpoint{1.306196in}{2.869202in}}{\pgfqpoint{1.295597in}{2.873592in}}{\pgfqpoint{1.284547in}{2.873592in}}%
\pgfpathcurveto{\pgfqpoint{1.273497in}{2.873592in}}{\pgfqpoint{1.262898in}{2.869202in}}{\pgfqpoint{1.255084in}{2.861388in}}%
\pgfpathcurveto{\pgfqpoint{1.247270in}{2.853575in}}{\pgfqpoint{1.242880in}{2.842976in}}{\pgfqpoint{1.242880in}{2.831926in}}%
\pgfpathcurveto{\pgfqpoint{1.242880in}{2.820875in}}{\pgfqpoint{1.247270in}{2.810276in}}{\pgfqpoint{1.255084in}{2.802463in}}%
\pgfpathcurveto{\pgfqpoint{1.262898in}{2.794649in}}{\pgfqpoint{1.273497in}{2.790259in}}{\pgfqpoint{1.284547in}{2.790259in}}%
\pgfpathlineto{\pgfqpoint{1.284547in}{2.790259in}}%
\pgfpathclose%
\pgfusepath{stroke}%
\end{pgfscope}%
\begin{pgfscope}%
\pgfpathrectangle{\pgfqpoint{0.847223in}{0.554012in}}{\pgfqpoint{6.200000in}{4.530000in}}%
\pgfusepath{clip}%
\pgfsetbuttcap%
\pgfsetroundjoin%
\pgfsetlinewidth{1.003750pt}%
\definecolor{currentstroke}{rgb}{1.000000,0.000000,0.000000}%
\pgfsetstrokecolor{currentstroke}%
\pgfsetdash{}{0pt}%
\pgfpathmoveto{\pgfqpoint{1.289880in}{2.775380in}}%
\pgfpathcurveto{\pgfqpoint{1.300930in}{2.775380in}}{\pgfqpoint{1.311529in}{2.779770in}}{\pgfqpoint{1.319343in}{2.787583in}}%
\pgfpathcurveto{\pgfqpoint{1.327156in}{2.795397in}}{\pgfqpoint{1.331547in}{2.805996in}}{\pgfqpoint{1.331547in}{2.817046in}}%
\pgfpathcurveto{\pgfqpoint{1.331547in}{2.828096in}}{\pgfqpoint{1.327156in}{2.838695in}}{\pgfqpoint{1.319343in}{2.846509in}}%
\pgfpathcurveto{\pgfqpoint{1.311529in}{2.854323in}}{\pgfqpoint{1.300930in}{2.858713in}}{\pgfqpoint{1.289880in}{2.858713in}}%
\pgfpathcurveto{\pgfqpoint{1.278830in}{2.858713in}}{\pgfqpoint{1.268231in}{2.854323in}}{\pgfqpoint{1.260417in}{2.846509in}}%
\pgfpathcurveto{\pgfqpoint{1.252604in}{2.838695in}}{\pgfqpoint{1.248213in}{2.828096in}}{\pgfqpoint{1.248213in}{2.817046in}}%
\pgfpathcurveto{\pgfqpoint{1.248213in}{2.805996in}}{\pgfqpoint{1.252604in}{2.795397in}}{\pgfqpoint{1.260417in}{2.787583in}}%
\pgfpathcurveto{\pgfqpoint{1.268231in}{2.779770in}}{\pgfqpoint{1.278830in}{2.775380in}}{\pgfqpoint{1.289880in}{2.775380in}}%
\pgfpathlineto{\pgfqpoint{1.289880in}{2.775380in}}%
\pgfpathclose%
\pgfusepath{stroke}%
\end{pgfscope}%
\begin{pgfscope}%
\pgfpathrectangle{\pgfqpoint{0.847223in}{0.554012in}}{\pgfqpoint{6.200000in}{4.530000in}}%
\pgfusepath{clip}%
\pgfsetbuttcap%
\pgfsetroundjoin%
\pgfsetlinewidth{1.003750pt}%
\definecolor{currentstroke}{rgb}{1.000000,0.000000,0.000000}%
\pgfsetstrokecolor{currentstroke}%
\pgfsetdash{}{0pt}%
\pgfpathmoveto{\pgfqpoint{1.295213in}{2.760664in}}%
\pgfpathcurveto{\pgfqpoint{1.306263in}{2.760664in}}{\pgfqpoint{1.316862in}{2.765054in}}{\pgfqpoint{1.324676in}{2.772868in}}%
\pgfpathcurveto{\pgfqpoint{1.332490in}{2.780681in}}{\pgfqpoint{1.336880in}{2.791280in}}{\pgfqpoint{1.336880in}{2.802330in}}%
\pgfpathcurveto{\pgfqpoint{1.336880in}{2.813380in}}{\pgfqpoint{1.332490in}{2.823979in}}{\pgfqpoint{1.324676in}{2.831793in}}%
\pgfpathcurveto{\pgfqpoint{1.316862in}{2.839607in}}{\pgfqpoint{1.306263in}{2.843997in}}{\pgfqpoint{1.295213in}{2.843997in}}%
\pgfpathcurveto{\pgfqpoint{1.284163in}{2.843997in}}{\pgfqpoint{1.273564in}{2.839607in}}{\pgfqpoint{1.265750in}{2.831793in}}%
\pgfpathcurveto{\pgfqpoint{1.257937in}{2.823979in}}{\pgfqpoint{1.253547in}{2.813380in}}{\pgfqpoint{1.253547in}{2.802330in}}%
\pgfpathcurveto{\pgfqpoint{1.253547in}{2.791280in}}{\pgfqpoint{1.257937in}{2.780681in}}{\pgfqpoint{1.265750in}{2.772868in}}%
\pgfpathcurveto{\pgfqpoint{1.273564in}{2.765054in}}{\pgfqpoint{1.284163in}{2.760664in}}{\pgfqpoint{1.295213in}{2.760664in}}%
\pgfpathlineto{\pgfqpoint{1.295213in}{2.760664in}}%
\pgfpathclose%
\pgfusepath{stroke}%
\end{pgfscope}%
\begin{pgfscope}%
\pgfpathrectangle{\pgfqpoint{0.847223in}{0.554012in}}{\pgfqpoint{6.200000in}{4.530000in}}%
\pgfusepath{clip}%
\pgfsetbuttcap%
\pgfsetroundjoin%
\pgfsetlinewidth{1.003750pt}%
\definecolor{currentstroke}{rgb}{1.000000,0.000000,0.000000}%
\pgfsetstrokecolor{currentstroke}%
\pgfsetdash{}{0pt}%
\pgfpathmoveto{\pgfqpoint{1.300546in}{2.746108in}}%
\pgfpathcurveto{\pgfqpoint{1.311597in}{2.746108in}}{\pgfqpoint{1.322196in}{2.750499in}}{\pgfqpoint{1.330009in}{2.758312in}}%
\pgfpathcurveto{\pgfqpoint{1.337823in}{2.766126in}}{\pgfqpoint{1.342213in}{2.776725in}}{\pgfqpoint{1.342213in}{2.787775in}}%
\pgfpathcurveto{\pgfqpoint{1.342213in}{2.798825in}}{\pgfqpoint{1.337823in}{2.809424in}}{\pgfqpoint{1.330009in}{2.817238in}}%
\pgfpathcurveto{\pgfqpoint{1.322196in}{2.825051in}}{\pgfqpoint{1.311597in}{2.829442in}}{\pgfqpoint{1.300546in}{2.829442in}}%
\pgfpathcurveto{\pgfqpoint{1.289496in}{2.829442in}}{\pgfqpoint{1.278897in}{2.825051in}}{\pgfqpoint{1.271084in}{2.817238in}}%
\pgfpathcurveto{\pgfqpoint{1.263270in}{2.809424in}}{\pgfqpoint{1.258880in}{2.798825in}}{\pgfqpoint{1.258880in}{2.787775in}}%
\pgfpathcurveto{\pgfqpoint{1.258880in}{2.776725in}}{\pgfqpoint{1.263270in}{2.766126in}}{\pgfqpoint{1.271084in}{2.758312in}}%
\pgfpathcurveto{\pgfqpoint{1.278897in}{2.750499in}}{\pgfqpoint{1.289496in}{2.746108in}}{\pgfqpoint{1.300546in}{2.746108in}}%
\pgfpathlineto{\pgfqpoint{1.300546in}{2.746108in}}%
\pgfpathclose%
\pgfusepath{stroke}%
\end{pgfscope}%
\begin{pgfscope}%
\pgfpathrectangle{\pgfqpoint{0.847223in}{0.554012in}}{\pgfqpoint{6.200000in}{4.530000in}}%
\pgfusepath{clip}%
\pgfsetbuttcap%
\pgfsetroundjoin%
\pgfsetlinewidth{1.003750pt}%
\definecolor{currentstroke}{rgb}{1.000000,0.000000,0.000000}%
\pgfsetstrokecolor{currentstroke}%
\pgfsetdash{}{0pt}%
\pgfpathmoveto{\pgfqpoint{1.305880in}{2.731711in}}%
\pgfpathcurveto{\pgfqpoint{1.316930in}{2.731711in}}{\pgfqpoint{1.327529in}{2.736102in}}{\pgfqpoint{1.335342in}{2.743915in}}%
\pgfpathcurveto{\pgfqpoint{1.343156in}{2.751729in}}{\pgfqpoint{1.347546in}{2.762328in}}{\pgfqpoint{1.347546in}{2.773378in}}%
\pgfpathcurveto{\pgfqpoint{1.347546in}{2.784428in}}{\pgfqpoint{1.343156in}{2.795027in}}{\pgfqpoint{1.335342in}{2.802841in}}%
\pgfpathcurveto{\pgfqpoint{1.327529in}{2.810654in}}{\pgfqpoint{1.316930in}{2.815045in}}{\pgfqpoint{1.305880in}{2.815045in}}%
\pgfpathcurveto{\pgfqpoint{1.294829in}{2.815045in}}{\pgfqpoint{1.284230in}{2.810654in}}{\pgfqpoint{1.276417in}{2.802841in}}%
\pgfpathcurveto{\pgfqpoint{1.268603in}{2.795027in}}{\pgfqpoint{1.264213in}{2.784428in}}{\pgfqpoint{1.264213in}{2.773378in}}%
\pgfpathcurveto{\pgfqpoint{1.264213in}{2.762328in}}{\pgfqpoint{1.268603in}{2.751729in}}{\pgfqpoint{1.276417in}{2.743915in}}%
\pgfpathcurveto{\pgfqpoint{1.284230in}{2.736102in}}{\pgfqpoint{1.294829in}{2.731711in}}{\pgfqpoint{1.305880in}{2.731711in}}%
\pgfpathlineto{\pgfqpoint{1.305880in}{2.731711in}}%
\pgfpathclose%
\pgfusepath{stroke}%
\end{pgfscope}%
\begin{pgfscope}%
\pgfpathrectangle{\pgfqpoint{0.847223in}{0.554012in}}{\pgfqpoint{6.200000in}{4.530000in}}%
\pgfusepath{clip}%
\pgfsetbuttcap%
\pgfsetroundjoin%
\pgfsetlinewidth{1.003750pt}%
\definecolor{currentstroke}{rgb}{1.000000,0.000000,0.000000}%
\pgfsetstrokecolor{currentstroke}%
\pgfsetdash{}{0pt}%
\pgfpathmoveto{\pgfqpoint{1.311213in}{2.717470in}}%
\pgfpathcurveto{\pgfqpoint{1.322263in}{2.717470in}}{\pgfqpoint{1.332862in}{2.721860in}}{\pgfqpoint{1.340676in}{2.729674in}}%
\pgfpathcurveto{\pgfqpoint{1.348489in}{2.737487in}}{\pgfqpoint{1.352879in}{2.748086in}}{\pgfqpoint{1.352879in}{2.759136in}}%
\pgfpathcurveto{\pgfqpoint{1.352879in}{2.770187in}}{\pgfqpoint{1.348489in}{2.780786in}}{\pgfqpoint{1.340676in}{2.788599in}}%
\pgfpathcurveto{\pgfqpoint{1.332862in}{2.796413in}}{\pgfqpoint{1.322263in}{2.800803in}}{\pgfqpoint{1.311213in}{2.800803in}}%
\pgfpathcurveto{\pgfqpoint{1.300163in}{2.800803in}}{\pgfqpoint{1.289564in}{2.796413in}}{\pgfqpoint{1.281750in}{2.788599in}}%
\pgfpathcurveto{\pgfqpoint{1.273936in}{2.780786in}}{\pgfqpoint{1.269546in}{2.770187in}}{\pgfqpoint{1.269546in}{2.759136in}}%
\pgfpathcurveto{\pgfqpoint{1.269546in}{2.748086in}}{\pgfqpoint{1.273936in}{2.737487in}}{\pgfqpoint{1.281750in}{2.729674in}}%
\pgfpathcurveto{\pgfqpoint{1.289564in}{2.721860in}}{\pgfqpoint{1.300163in}{2.717470in}}{\pgfqpoint{1.311213in}{2.717470in}}%
\pgfpathlineto{\pgfqpoint{1.311213in}{2.717470in}}%
\pgfpathclose%
\pgfusepath{stroke}%
\end{pgfscope}%
\begin{pgfscope}%
\pgfpathrectangle{\pgfqpoint{0.847223in}{0.554012in}}{\pgfqpoint{6.200000in}{4.530000in}}%
\pgfusepath{clip}%
\pgfsetbuttcap%
\pgfsetroundjoin%
\pgfsetlinewidth{1.003750pt}%
\definecolor{currentstroke}{rgb}{1.000000,0.000000,0.000000}%
\pgfsetstrokecolor{currentstroke}%
\pgfsetdash{}{0pt}%
\pgfpathmoveto{\pgfqpoint{1.316546in}{2.703381in}}%
\pgfpathcurveto{\pgfqpoint{1.327596in}{2.703381in}}{\pgfqpoint{1.338195in}{2.707771in}}{\pgfqpoint{1.346009in}{2.715585in}}%
\pgfpathcurveto{\pgfqpoint{1.353822in}{2.723399in}}{\pgfqpoint{1.358213in}{2.733998in}}{\pgfqpoint{1.358213in}{2.745048in}}%
\pgfpathcurveto{\pgfqpoint{1.358213in}{2.756098in}}{\pgfqpoint{1.353822in}{2.766697in}}{\pgfqpoint{1.346009in}{2.774511in}}%
\pgfpathcurveto{\pgfqpoint{1.338195in}{2.782324in}}{\pgfqpoint{1.327596in}{2.786714in}}{\pgfqpoint{1.316546in}{2.786714in}}%
\pgfpathcurveto{\pgfqpoint{1.305496in}{2.786714in}}{\pgfqpoint{1.294897in}{2.782324in}}{\pgfqpoint{1.287083in}{2.774511in}}%
\pgfpathcurveto{\pgfqpoint{1.279270in}{2.766697in}}{\pgfqpoint{1.274879in}{2.756098in}}{\pgfqpoint{1.274879in}{2.745048in}}%
\pgfpathcurveto{\pgfqpoint{1.274879in}{2.733998in}}{\pgfqpoint{1.279270in}{2.723399in}}{\pgfqpoint{1.287083in}{2.715585in}}%
\pgfpathcurveto{\pgfqpoint{1.294897in}{2.707771in}}{\pgfqpoint{1.305496in}{2.703381in}}{\pgfqpoint{1.316546in}{2.703381in}}%
\pgfpathlineto{\pgfqpoint{1.316546in}{2.703381in}}%
\pgfpathclose%
\pgfusepath{stroke}%
\end{pgfscope}%
\begin{pgfscope}%
\pgfpathrectangle{\pgfqpoint{0.847223in}{0.554012in}}{\pgfqpoint{6.200000in}{4.530000in}}%
\pgfusepath{clip}%
\pgfsetbuttcap%
\pgfsetroundjoin%
\pgfsetlinewidth{1.003750pt}%
\definecolor{currentstroke}{rgb}{1.000000,0.000000,0.000000}%
\pgfsetstrokecolor{currentstroke}%
\pgfsetdash{}{0pt}%
\pgfpathmoveto{\pgfqpoint{1.321879in}{2.689443in}}%
\pgfpathcurveto{\pgfqpoint{1.332929in}{2.689443in}}{\pgfqpoint{1.343528in}{2.693833in}}{\pgfqpoint{1.351342in}{2.701647in}}%
\pgfpathcurveto{\pgfqpoint{1.359156in}{2.709461in}}{\pgfqpoint{1.363546in}{2.720060in}}{\pgfqpoint{1.363546in}{2.731110in}}%
\pgfpathcurveto{\pgfqpoint{1.363546in}{2.742160in}}{\pgfqpoint{1.359156in}{2.752759in}}{\pgfqpoint{1.351342in}{2.760573in}}%
\pgfpathcurveto{\pgfqpoint{1.343528in}{2.768386in}}{\pgfqpoint{1.332929in}{2.772776in}}{\pgfqpoint{1.321879in}{2.772776in}}%
\pgfpathcurveto{\pgfqpoint{1.310829in}{2.772776in}}{\pgfqpoint{1.300230in}{2.768386in}}{\pgfqpoint{1.292416in}{2.760573in}}%
\pgfpathcurveto{\pgfqpoint{1.284603in}{2.752759in}}{\pgfqpoint{1.280213in}{2.742160in}}{\pgfqpoint{1.280213in}{2.731110in}}%
\pgfpathcurveto{\pgfqpoint{1.280213in}{2.720060in}}{\pgfqpoint{1.284603in}{2.709461in}}{\pgfqpoint{1.292416in}{2.701647in}}%
\pgfpathcurveto{\pgfqpoint{1.300230in}{2.693833in}}{\pgfqpoint{1.310829in}{2.689443in}}{\pgfqpoint{1.321879in}{2.689443in}}%
\pgfpathlineto{\pgfqpoint{1.321879in}{2.689443in}}%
\pgfpathclose%
\pgfusepath{stroke}%
\end{pgfscope}%
\begin{pgfscope}%
\pgfpathrectangle{\pgfqpoint{0.847223in}{0.554012in}}{\pgfqpoint{6.200000in}{4.530000in}}%
\pgfusepath{clip}%
\pgfsetbuttcap%
\pgfsetroundjoin%
\pgfsetlinewidth{1.003750pt}%
\definecolor{currentstroke}{rgb}{1.000000,0.000000,0.000000}%
\pgfsetstrokecolor{currentstroke}%
\pgfsetdash{}{0pt}%
\pgfpathmoveto{\pgfqpoint{1.327212in}{2.675653in}}%
\pgfpathcurveto{\pgfqpoint{1.338263in}{2.675653in}}{\pgfqpoint{1.348862in}{2.680044in}}{\pgfqpoint{1.356675in}{2.687857in}}%
\pgfpathcurveto{\pgfqpoint{1.364489in}{2.695671in}}{\pgfqpoint{1.368879in}{2.706270in}}{\pgfqpoint{1.368879in}{2.717320in}}%
\pgfpathcurveto{\pgfqpoint{1.368879in}{2.728370in}}{\pgfqpoint{1.364489in}{2.738969in}}{\pgfqpoint{1.356675in}{2.746783in}}%
\pgfpathcurveto{\pgfqpoint{1.348862in}{2.754596in}}{\pgfqpoint{1.338263in}{2.758987in}}{\pgfqpoint{1.327212in}{2.758987in}}%
\pgfpathcurveto{\pgfqpoint{1.316162in}{2.758987in}}{\pgfqpoint{1.305563in}{2.754596in}}{\pgfqpoint{1.297750in}{2.746783in}}%
\pgfpathcurveto{\pgfqpoint{1.289936in}{2.738969in}}{\pgfqpoint{1.285546in}{2.728370in}}{\pgfqpoint{1.285546in}{2.717320in}}%
\pgfpathcurveto{\pgfqpoint{1.285546in}{2.706270in}}{\pgfqpoint{1.289936in}{2.695671in}}{\pgfqpoint{1.297750in}{2.687857in}}%
\pgfpathcurveto{\pgfqpoint{1.305563in}{2.680044in}}{\pgfqpoint{1.316162in}{2.675653in}}{\pgfqpoint{1.327212in}{2.675653in}}%
\pgfpathlineto{\pgfqpoint{1.327212in}{2.675653in}}%
\pgfpathclose%
\pgfusepath{stroke}%
\end{pgfscope}%
\begin{pgfscope}%
\pgfpathrectangle{\pgfqpoint{0.847223in}{0.554012in}}{\pgfqpoint{6.200000in}{4.530000in}}%
\pgfusepath{clip}%
\pgfsetbuttcap%
\pgfsetroundjoin%
\pgfsetlinewidth{1.003750pt}%
\definecolor{currentstroke}{rgb}{1.000000,0.000000,0.000000}%
\pgfsetstrokecolor{currentstroke}%
\pgfsetdash{}{0pt}%
\pgfpathmoveto{\pgfqpoint{1.332546in}{2.662009in}}%
\pgfpathcurveto{\pgfqpoint{1.343596in}{2.662009in}}{\pgfqpoint{1.354195in}{2.666400in}}{\pgfqpoint{1.362008in}{2.674213in}}%
\pgfpathcurveto{\pgfqpoint{1.369822in}{2.682027in}}{\pgfqpoint{1.374212in}{2.692626in}}{\pgfqpoint{1.374212in}{2.703676in}}%
\pgfpathcurveto{\pgfqpoint{1.374212in}{2.714726in}}{\pgfqpoint{1.369822in}{2.725325in}}{\pgfqpoint{1.362008in}{2.733139in}}%
\pgfpathcurveto{\pgfqpoint{1.354195in}{2.740952in}}{\pgfqpoint{1.343596in}{2.745343in}}{\pgfqpoint{1.332546in}{2.745343in}}%
\pgfpathcurveto{\pgfqpoint{1.321496in}{2.745343in}}{\pgfqpoint{1.310897in}{2.740952in}}{\pgfqpoint{1.303083in}{2.733139in}}%
\pgfpathcurveto{\pgfqpoint{1.295269in}{2.725325in}}{\pgfqpoint{1.290879in}{2.714726in}}{\pgfqpoint{1.290879in}{2.703676in}}%
\pgfpathcurveto{\pgfqpoint{1.290879in}{2.692626in}}{\pgfqpoint{1.295269in}{2.682027in}}{\pgfqpoint{1.303083in}{2.674213in}}%
\pgfpathcurveto{\pgfqpoint{1.310897in}{2.666400in}}{\pgfqpoint{1.321496in}{2.662009in}}{\pgfqpoint{1.332546in}{2.662009in}}%
\pgfpathlineto{\pgfqpoint{1.332546in}{2.662009in}}%
\pgfpathclose%
\pgfusepath{stroke}%
\end{pgfscope}%
\begin{pgfscope}%
\pgfpathrectangle{\pgfqpoint{0.847223in}{0.554012in}}{\pgfqpoint{6.200000in}{4.530000in}}%
\pgfusepath{clip}%
\pgfsetbuttcap%
\pgfsetroundjoin%
\pgfsetlinewidth{1.003750pt}%
\definecolor{currentstroke}{rgb}{1.000000,0.000000,0.000000}%
\pgfsetstrokecolor{currentstroke}%
\pgfsetdash{}{0pt}%
\pgfpathmoveto{\pgfqpoint{1.337879in}{2.648509in}}%
\pgfpathcurveto{\pgfqpoint{1.348929in}{2.648509in}}{\pgfqpoint{1.359528in}{2.652899in}}{\pgfqpoint{1.367342in}{2.660713in}}%
\pgfpathcurveto{\pgfqpoint{1.375155in}{2.668526in}}{\pgfqpoint{1.379546in}{2.679125in}}{\pgfqpoint{1.379546in}{2.690176in}}%
\pgfpathcurveto{\pgfqpoint{1.379546in}{2.701226in}}{\pgfqpoint{1.375155in}{2.711825in}}{\pgfqpoint{1.367342in}{2.719638in}}%
\pgfpathcurveto{\pgfqpoint{1.359528in}{2.727452in}}{\pgfqpoint{1.348929in}{2.731842in}}{\pgfqpoint{1.337879in}{2.731842in}}%
\pgfpathcurveto{\pgfqpoint{1.326829in}{2.731842in}}{\pgfqpoint{1.316230in}{2.727452in}}{\pgfqpoint{1.308416in}{2.719638in}}%
\pgfpathcurveto{\pgfqpoint{1.300602in}{2.711825in}}{\pgfqpoint{1.296212in}{2.701226in}}{\pgfqpoint{1.296212in}{2.690176in}}%
\pgfpathcurveto{\pgfqpoint{1.296212in}{2.679125in}}{\pgfqpoint{1.300602in}{2.668526in}}{\pgfqpoint{1.308416in}{2.660713in}}%
\pgfpathcurveto{\pgfqpoint{1.316230in}{2.652899in}}{\pgfqpoint{1.326829in}{2.648509in}}{\pgfqpoint{1.337879in}{2.648509in}}%
\pgfpathlineto{\pgfqpoint{1.337879in}{2.648509in}}%
\pgfpathclose%
\pgfusepath{stroke}%
\end{pgfscope}%
\begin{pgfscope}%
\pgfpathrectangle{\pgfqpoint{0.847223in}{0.554012in}}{\pgfqpoint{6.200000in}{4.530000in}}%
\pgfusepath{clip}%
\pgfsetbuttcap%
\pgfsetroundjoin%
\pgfsetlinewidth{1.003750pt}%
\definecolor{currentstroke}{rgb}{1.000000,0.000000,0.000000}%
\pgfsetstrokecolor{currentstroke}%
\pgfsetdash{}{0pt}%
\pgfpathmoveto{\pgfqpoint{1.343212in}{2.635150in}}%
\pgfpathcurveto{\pgfqpoint{1.354262in}{2.635150in}}{\pgfqpoint{1.364861in}{2.639540in}}{\pgfqpoint{1.372675in}{2.647354in}}%
\pgfpathcurveto{\pgfqpoint{1.380489in}{2.655167in}}{\pgfqpoint{1.384879in}{2.665766in}}{\pgfqpoint{1.384879in}{2.676816in}}%
\pgfpathcurveto{\pgfqpoint{1.384879in}{2.687867in}}{\pgfqpoint{1.380489in}{2.698466in}}{\pgfqpoint{1.372675in}{2.706279in}}%
\pgfpathcurveto{\pgfqpoint{1.364861in}{2.714093in}}{\pgfqpoint{1.354262in}{2.718483in}}{\pgfqpoint{1.343212in}{2.718483in}}%
\pgfpathcurveto{\pgfqpoint{1.332162in}{2.718483in}}{\pgfqpoint{1.321563in}{2.714093in}}{\pgfqpoint{1.313749in}{2.706279in}}%
\pgfpathcurveto{\pgfqpoint{1.305936in}{2.698466in}}{\pgfqpoint{1.301545in}{2.687867in}}{\pgfqpoint{1.301545in}{2.676816in}}%
\pgfpathcurveto{\pgfqpoint{1.301545in}{2.665766in}}{\pgfqpoint{1.305936in}{2.655167in}}{\pgfqpoint{1.313749in}{2.647354in}}%
\pgfpathcurveto{\pgfqpoint{1.321563in}{2.639540in}}{\pgfqpoint{1.332162in}{2.635150in}}{\pgfqpoint{1.343212in}{2.635150in}}%
\pgfpathlineto{\pgfqpoint{1.343212in}{2.635150in}}%
\pgfpathclose%
\pgfusepath{stroke}%
\end{pgfscope}%
\begin{pgfscope}%
\pgfpathrectangle{\pgfqpoint{0.847223in}{0.554012in}}{\pgfqpoint{6.200000in}{4.530000in}}%
\pgfusepath{clip}%
\pgfsetbuttcap%
\pgfsetroundjoin%
\pgfsetlinewidth{1.003750pt}%
\definecolor{currentstroke}{rgb}{1.000000,0.000000,0.000000}%
\pgfsetstrokecolor{currentstroke}%
\pgfsetdash{}{0pt}%
\pgfpathmoveto{\pgfqpoint{1.348545in}{2.621930in}}%
\pgfpathcurveto{\pgfqpoint{1.359595in}{2.621930in}}{\pgfqpoint{1.370194in}{2.626320in}}{\pgfqpoint{1.378008in}{2.634133in}}%
\pgfpathcurveto{\pgfqpoint{1.385822in}{2.641947in}}{\pgfqpoint{1.390212in}{2.652546in}}{\pgfqpoint{1.390212in}{2.663596in}}%
\pgfpathcurveto{\pgfqpoint{1.390212in}{2.674646in}}{\pgfqpoint{1.385822in}{2.685245in}}{\pgfqpoint{1.378008in}{2.693059in}}%
\pgfpathcurveto{\pgfqpoint{1.370194in}{2.700873in}}{\pgfqpoint{1.359595in}{2.705263in}}{\pgfqpoint{1.348545in}{2.705263in}}%
\pgfpathcurveto{\pgfqpoint{1.337495in}{2.705263in}}{\pgfqpoint{1.326896in}{2.700873in}}{\pgfqpoint{1.319083in}{2.693059in}}%
\pgfpathcurveto{\pgfqpoint{1.311269in}{2.685245in}}{\pgfqpoint{1.306879in}{2.674646in}}{\pgfqpoint{1.306879in}{2.663596in}}%
\pgfpathcurveto{\pgfqpoint{1.306879in}{2.652546in}}{\pgfqpoint{1.311269in}{2.641947in}}{\pgfqpoint{1.319083in}{2.634133in}}%
\pgfpathcurveto{\pgfqpoint{1.326896in}{2.626320in}}{\pgfqpoint{1.337495in}{2.621930in}}{\pgfqpoint{1.348545in}{2.621930in}}%
\pgfpathlineto{\pgfqpoint{1.348545in}{2.621930in}}%
\pgfpathclose%
\pgfusepath{stroke}%
\end{pgfscope}%
\begin{pgfscope}%
\pgfpathrectangle{\pgfqpoint{0.847223in}{0.554012in}}{\pgfqpoint{6.200000in}{4.530000in}}%
\pgfusepath{clip}%
\pgfsetbuttcap%
\pgfsetroundjoin%
\pgfsetlinewidth{1.003750pt}%
\definecolor{currentstroke}{rgb}{1.000000,0.000000,0.000000}%
\pgfsetstrokecolor{currentstroke}%
\pgfsetdash{}{0pt}%
\pgfpathmoveto{\pgfqpoint{1.353879in}{2.608846in}}%
\pgfpathcurveto{\pgfqpoint{1.364929in}{2.608846in}}{\pgfqpoint{1.375528in}{2.613237in}}{\pgfqpoint{1.383341in}{2.621050in}}%
\pgfpathcurveto{\pgfqpoint{1.391155in}{2.628864in}}{\pgfqpoint{1.395545in}{2.639463in}}{\pgfqpoint{1.395545in}{2.650513in}}%
\pgfpathcurveto{\pgfqpoint{1.395545in}{2.661563in}}{\pgfqpoint{1.391155in}{2.672162in}}{\pgfqpoint{1.383341in}{2.679976in}}%
\pgfpathcurveto{\pgfqpoint{1.375528in}{2.687789in}}{\pgfqpoint{1.364929in}{2.692180in}}{\pgfqpoint{1.353879in}{2.692180in}}%
\pgfpathcurveto{\pgfqpoint{1.342828in}{2.692180in}}{\pgfqpoint{1.332229in}{2.687789in}}{\pgfqpoint{1.324416in}{2.679976in}}%
\pgfpathcurveto{\pgfqpoint{1.316602in}{2.672162in}}{\pgfqpoint{1.312212in}{2.661563in}}{\pgfqpoint{1.312212in}{2.650513in}}%
\pgfpathcurveto{\pgfqpoint{1.312212in}{2.639463in}}{\pgfqpoint{1.316602in}{2.628864in}}{\pgfqpoint{1.324416in}{2.621050in}}%
\pgfpathcurveto{\pgfqpoint{1.332229in}{2.613237in}}{\pgfqpoint{1.342828in}{2.608846in}}{\pgfqpoint{1.353879in}{2.608846in}}%
\pgfpathlineto{\pgfqpoint{1.353879in}{2.608846in}}%
\pgfpathclose%
\pgfusepath{stroke}%
\end{pgfscope}%
\begin{pgfscope}%
\pgfpathrectangle{\pgfqpoint{0.847223in}{0.554012in}}{\pgfqpoint{6.200000in}{4.530000in}}%
\pgfusepath{clip}%
\pgfsetbuttcap%
\pgfsetroundjoin%
\pgfsetlinewidth{1.003750pt}%
\definecolor{currentstroke}{rgb}{1.000000,0.000000,0.000000}%
\pgfsetstrokecolor{currentstroke}%
\pgfsetdash{}{0pt}%
\pgfpathmoveto{\pgfqpoint{1.359212in}{2.595898in}}%
\pgfpathcurveto{\pgfqpoint{1.370262in}{2.595898in}}{\pgfqpoint{1.380861in}{2.600288in}}{\pgfqpoint{1.388675in}{2.608102in}}%
\pgfpathcurveto{\pgfqpoint{1.396488in}{2.615915in}}{\pgfqpoint{1.400878in}{2.626514in}}{\pgfqpoint{1.400878in}{2.637565in}}%
\pgfpathcurveto{\pgfqpoint{1.400878in}{2.648615in}}{\pgfqpoint{1.396488in}{2.659214in}}{\pgfqpoint{1.388675in}{2.667027in}}%
\pgfpathcurveto{\pgfqpoint{1.380861in}{2.674841in}}{\pgfqpoint{1.370262in}{2.679231in}}{\pgfqpoint{1.359212in}{2.679231in}}%
\pgfpathcurveto{\pgfqpoint{1.348162in}{2.679231in}}{\pgfqpoint{1.337563in}{2.674841in}}{\pgfqpoint{1.329749in}{2.667027in}}%
\pgfpathcurveto{\pgfqpoint{1.321935in}{2.659214in}}{\pgfqpoint{1.317545in}{2.648615in}}{\pgfqpoint{1.317545in}{2.637565in}}%
\pgfpathcurveto{\pgfqpoint{1.317545in}{2.626514in}}{\pgfqpoint{1.321935in}{2.615915in}}{\pgfqpoint{1.329749in}{2.608102in}}%
\pgfpathcurveto{\pgfqpoint{1.337563in}{2.600288in}}{\pgfqpoint{1.348162in}{2.595898in}}{\pgfqpoint{1.359212in}{2.595898in}}%
\pgfpathlineto{\pgfqpoint{1.359212in}{2.595898in}}%
\pgfpathclose%
\pgfusepath{stroke}%
\end{pgfscope}%
\begin{pgfscope}%
\pgfpathrectangle{\pgfqpoint{0.847223in}{0.554012in}}{\pgfqpoint{6.200000in}{4.530000in}}%
\pgfusepath{clip}%
\pgfsetbuttcap%
\pgfsetroundjoin%
\pgfsetlinewidth{1.003750pt}%
\definecolor{currentstroke}{rgb}{1.000000,0.000000,0.000000}%
\pgfsetstrokecolor{currentstroke}%
\pgfsetdash{}{0pt}%
\pgfpathmoveto{\pgfqpoint{1.364545in}{2.583082in}}%
\pgfpathcurveto{\pgfqpoint{1.375595in}{2.583082in}}{\pgfqpoint{1.386194in}{2.587472in}}{\pgfqpoint{1.394008in}{2.595286in}}%
\pgfpathcurveto{\pgfqpoint{1.401821in}{2.603100in}}{\pgfqpoint{1.406212in}{2.613699in}}{\pgfqpoint{1.406212in}{2.624749in}}%
\pgfpathcurveto{\pgfqpoint{1.406212in}{2.635799in}}{\pgfqpoint{1.401821in}{2.646398in}}{\pgfqpoint{1.394008in}{2.654212in}}%
\pgfpathcurveto{\pgfqpoint{1.386194in}{2.662025in}}{\pgfqpoint{1.375595in}{2.666416in}}{\pgfqpoint{1.364545in}{2.666416in}}%
\pgfpathcurveto{\pgfqpoint{1.353495in}{2.666416in}}{\pgfqpoint{1.342896in}{2.662025in}}{\pgfqpoint{1.335082in}{2.654212in}}%
\pgfpathcurveto{\pgfqpoint{1.327269in}{2.646398in}}{\pgfqpoint{1.322878in}{2.635799in}}{\pgfqpoint{1.322878in}{2.624749in}}%
\pgfpathcurveto{\pgfqpoint{1.322878in}{2.613699in}}{\pgfqpoint{1.327269in}{2.603100in}}{\pgfqpoint{1.335082in}{2.595286in}}%
\pgfpathcurveto{\pgfqpoint{1.342896in}{2.587472in}}{\pgfqpoint{1.353495in}{2.583082in}}{\pgfqpoint{1.364545in}{2.583082in}}%
\pgfpathlineto{\pgfqpoint{1.364545in}{2.583082in}}%
\pgfpathclose%
\pgfusepath{stroke}%
\end{pgfscope}%
\begin{pgfscope}%
\pgfpathrectangle{\pgfqpoint{0.847223in}{0.554012in}}{\pgfqpoint{6.200000in}{4.530000in}}%
\pgfusepath{clip}%
\pgfsetbuttcap%
\pgfsetroundjoin%
\pgfsetlinewidth{1.003750pt}%
\definecolor{currentstroke}{rgb}{1.000000,0.000000,0.000000}%
\pgfsetstrokecolor{currentstroke}%
\pgfsetdash{}{0pt}%
\pgfpathmoveto{\pgfqpoint{1.369878in}{2.570397in}}%
\pgfpathcurveto{\pgfqpoint{1.380928in}{2.570397in}}{\pgfqpoint{1.391527in}{2.574787in}}{\pgfqpoint{1.399341in}{2.582601in}}%
\pgfpathcurveto{\pgfqpoint{1.407155in}{2.590415in}}{\pgfqpoint{1.411545in}{2.601014in}}{\pgfqpoint{1.411545in}{2.612064in}}%
\pgfpathcurveto{\pgfqpoint{1.411545in}{2.623114in}}{\pgfqpoint{1.407155in}{2.633713in}}{\pgfqpoint{1.399341in}{2.641527in}}%
\pgfpathcurveto{\pgfqpoint{1.391527in}{2.649340in}}{\pgfqpoint{1.380928in}{2.653730in}}{\pgfqpoint{1.369878in}{2.653730in}}%
\pgfpathcurveto{\pgfqpoint{1.358828in}{2.653730in}}{\pgfqpoint{1.348229in}{2.649340in}}{\pgfqpoint{1.340415in}{2.641527in}}%
\pgfpathcurveto{\pgfqpoint{1.332602in}{2.633713in}}{\pgfqpoint{1.328212in}{2.623114in}}{\pgfqpoint{1.328212in}{2.612064in}}%
\pgfpathcurveto{\pgfqpoint{1.328212in}{2.601014in}}{\pgfqpoint{1.332602in}{2.590415in}}{\pgfqpoint{1.340415in}{2.582601in}}%
\pgfpathcurveto{\pgfqpoint{1.348229in}{2.574787in}}{\pgfqpoint{1.358828in}{2.570397in}}{\pgfqpoint{1.369878in}{2.570397in}}%
\pgfpathlineto{\pgfqpoint{1.369878in}{2.570397in}}%
\pgfpathclose%
\pgfusepath{stroke}%
\end{pgfscope}%
\begin{pgfscope}%
\pgfpathrectangle{\pgfqpoint{0.847223in}{0.554012in}}{\pgfqpoint{6.200000in}{4.530000in}}%
\pgfusepath{clip}%
\pgfsetbuttcap%
\pgfsetroundjoin%
\pgfsetlinewidth{1.003750pt}%
\definecolor{currentstroke}{rgb}{1.000000,0.000000,0.000000}%
\pgfsetstrokecolor{currentstroke}%
\pgfsetdash{}{0pt}%
\pgfpathmoveto{\pgfqpoint{1.375211in}{2.557841in}}%
\pgfpathcurveto{\pgfqpoint{1.386262in}{2.557841in}}{\pgfqpoint{1.396861in}{2.562231in}}{\pgfqpoint{1.404674in}{2.570045in}}%
\pgfpathcurveto{\pgfqpoint{1.412488in}{2.577858in}}{\pgfqpoint{1.416878in}{2.588457in}}{\pgfqpoint{1.416878in}{2.599507in}}%
\pgfpathcurveto{\pgfqpoint{1.416878in}{2.610558in}}{\pgfqpoint{1.412488in}{2.621157in}}{\pgfqpoint{1.404674in}{2.628970in}}%
\pgfpathcurveto{\pgfqpoint{1.396861in}{2.636784in}}{\pgfqpoint{1.386262in}{2.641174in}}{\pgfqpoint{1.375211in}{2.641174in}}%
\pgfpathcurveto{\pgfqpoint{1.364161in}{2.641174in}}{\pgfqpoint{1.353562in}{2.636784in}}{\pgfqpoint{1.345749in}{2.628970in}}%
\pgfpathcurveto{\pgfqpoint{1.337935in}{2.621157in}}{\pgfqpoint{1.333545in}{2.610558in}}{\pgfqpoint{1.333545in}{2.599507in}}%
\pgfpathcurveto{\pgfqpoint{1.333545in}{2.588457in}}{\pgfqpoint{1.337935in}{2.577858in}}{\pgfqpoint{1.345749in}{2.570045in}}%
\pgfpathcurveto{\pgfqpoint{1.353562in}{2.562231in}}{\pgfqpoint{1.364161in}{2.557841in}}{\pgfqpoint{1.375211in}{2.557841in}}%
\pgfpathlineto{\pgfqpoint{1.375211in}{2.557841in}}%
\pgfpathclose%
\pgfusepath{stroke}%
\end{pgfscope}%
\begin{pgfscope}%
\pgfpathrectangle{\pgfqpoint{0.847223in}{0.554012in}}{\pgfqpoint{6.200000in}{4.530000in}}%
\pgfusepath{clip}%
\pgfsetbuttcap%
\pgfsetroundjoin%
\pgfsetlinewidth{1.003750pt}%
\definecolor{currentstroke}{rgb}{1.000000,0.000000,0.000000}%
\pgfsetstrokecolor{currentstroke}%
\pgfsetdash{}{0pt}%
\pgfpathmoveto{\pgfqpoint{1.380545in}{2.545411in}}%
\pgfpathcurveto{\pgfqpoint{1.391595in}{2.545411in}}{\pgfqpoint{1.402194in}{2.549801in}}{\pgfqpoint{1.410007in}{2.557615in}}%
\pgfpathcurveto{\pgfqpoint{1.417821in}{2.565429in}}{\pgfqpoint{1.422211in}{2.576028in}}{\pgfqpoint{1.422211in}{2.587078in}}%
\pgfpathcurveto{\pgfqpoint{1.422211in}{2.598128in}}{\pgfqpoint{1.417821in}{2.608727in}}{\pgfqpoint{1.410007in}{2.616541in}}%
\pgfpathcurveto{\pgfqpoint{1.402194in}{2.624354in}}{\pgfqpoint{1.391595in}{2.628745in}}{\pgfqpoint{1.380545in}{2.628745in}}%
\pgfpathcurveto{\pgfqpoint{1.369494in}{2.628745in}}{\pgfqpoint{1.358895in}{2.624354in}}{\pgfqpoint{1.351082in}{2.616541in}}%
\pgfpathcurveto{\pgfqpoint{1.343268in}{2.608727in}}{\pgfqpoint{1.338878in}{2.598128in}}{\pgfqpoint{1.338878in}{2.587078in}}%
\pgfpathcurveto{\pgfqpoint{1.338878in}{2.576028in}}{\pgfqpoint{1.343268in}{2.565429in}}{\pgfqpoint{1.351082in}{2.557615in}}%
\pgfpathcurveto{\pgfqpoint{1.358895in}{2.549801in}}{\pgfqpoint{1.369494in}{2.545411in}}{\pgfqpoint{1.380545in}{2.545411in}}%
\pgfpathlineto{\pgfqpoint{1.380545in}{2.545411in}}%
\pgfpathclose%
\pgfusepath{stroke}%
\end{pgfscope}%
\begin{pgfscope}%
\pgfpathrectangle{\pgfqpoint{0.847223in}{0.554012in}}{\pgfqpoint{6.200000in}{4.530000in}}%
\pgfusepath{clip}%
\pgfsetbuttcap%
\pgfsetroundjoin%
\pgfsetlinewidth{1.003750pt}%
\definecolor{currentstroke}{rgb}{1.000000,0.000000,0.000000}%
\pgfsetstrokecolor{currentstroke}%
\pgfsetdash{}{0pt}%
\pgfpathmoveto{\pgfqpoint{1.385878in}{2.533106in}}%
\pgfpathcurveto{\pgfqpoint{1.396928in}{2.533106in}}{\pgfqpoint{1.407527in}{2.537497in}}{\pgfqpoint{1.415341in}{2.545310in}}%
\pgfpathcurveto{\pgfqpoint{1.423154in}{2.553124in}}{\pgfqpoint{1.427545in}{2.563723in}}{\pgfqpoint{1.427545in}{2.574773in}}%
\pgfpathcurveto{\pgfqpoint{1.427545in}{2.585823in}}{\pgfqpoint{1.423154in}{2.596422in}}{\pgfqpoint{1.415341in}{2.604236in}}%
\pgfpathcurveto{\pgfqpoint{1.407527in}{2.612050in}}{\pgfqpoint{1.396928in}{2.616440in}}{\pgfqpoint{1.385878in}{2.616440in}}%
\pgfpathcurveto{\pgfqpoint{1.374828in}{2.616440in}}{\pgfqpoint{1.364229in}{2.612050in}}{\pgfqpoint{1.356415in}{2.604236in}}%
\pgfpathcurveto{\pgfqpoint{1.348601in}{2.596422in}}{\pgfqpoint{1.344211in}{2.585823in}}{\pgfqpoint{1.344211in}{2.574773in}}%
\pgfpathcurveto{\pgfqpoint{1.344211in}{2.563723in}}{\pgfqpoint{1.348601in}{2.553124in}}{\pgfqpoint{1.356415in}{2.545310in}}%
\pgfpathcurveto{\pgfqpoint{1.364229in}{2.537497in}}{\pgfqpoint{1.374828in}{2.533106in}}{\pgfqpoint{1.385878in}{2.533106in}}%
\pgfpathlineto{\pgfqpoint{1.385878in}{2.533106in}}%
\pgfpathclose%
\pgfusepath{stroke}%
\end{pgfscope}%
\begin{pgfscope}%
\pgfpathrectangle{\pgfqpoint{0.847223in}{0.554012in}}{\pgfqpoint{6.200000in}{4.530000in}}%
\pgfusepath{clip}%
\pgfsetbuttcap%
\pgfsetroundjoin%
\pgfsetlinewidth{1.003750pt}%
\definecolor{currentstroke}{rgb}{1.000000,0.000000,0.000000}%
\pgfsetstrokecolor{currentstroke}%
\pgfsetdash{}{0pt}%
\pgfpathmoveto{\pgfqpoint{1.391211in}{2.520925in}}%
\pgfpathcurveto{\pgfqpoint{1.402261in}{2.520925in}}{\pgfqpoint{1.412860in}{2.525315in}}{\pgfqpoint{1.420674in}{2.533129in}}%
\pgfpathcurveto{\pgfqpoint{1.428487in}{2.540942in}}{\pgfqpoint{1.432878in}{2.551541in}}{\pgfqpoint{1.432878in}{2.562591in}}%
\pgfpathcurveto{\pgfqpoint{1.432878in}{2.573641in}}{\pgfqpoint{1.428487in}{2.584241in}}{\pgfqpoint{1.420674in}{2.592054in}}%
\pgfpathcurveto{\pgfqpoint{1.412860in}{2.599868in}}{\pgfqpoint{1.402261in}{2.604258in}}{\pgfqpoint{1.391211in}{2.604258in}}%
\pgfpathcurveto{\pgfqpoint{1.380161in}{2.604258in}}{\pgfqpoint{1.369562in}{2.599868in}}{\pgfqpoint{1.361748in}{2.592054in}}%
\pgfpathcurveto{\pgfqpoint{1.353935in}{2.584241in}}{\pgfqpoint{1.349544in}{2.573641in}}{\pgfqpoint{1.349544in}{2.562591in}}%
\pgfpathcurveto{\pgfqpoint{1.349544in}{2.551541in}}{\pgfqpoint{1.353935in}{2.540942in}}{\pgfqpoint{1.361748in}{2.533129in}}%
\pgfpathcurveto{\pgfqpoint{1.369562in}{2.525315in}}{\pgfqpoint{1.380161in}{2.520925in}}{\pgfqpoint{1.391211in}{2.520925in}}%
\pgfpathlineto{\pgfqpoint{1.391211in}{2.520925in}}%
\pgfpathclose%
\pgfusepath{stroke}%
\end{pgfscope}%
\begin{pgfscope}%
\pgfpathrectangle{\pgfqpoint{0.847223in}{0.554012in}}{\pgfqpoint{6.200000in}{4.530000in}}%
\pgfusepath{clip}%
\pgfsetbuttcap%
\pgfsetroundjoin%
\pgfsetlinewidth{1.003750pt}%
\definecolor{currentstroke}{rgb}{1.000000,0.000000,0.000000}%
\pgfsetstrokecolor{currentstroke}%
\pgfsetdash{}{0pt}%
\pgfpathmoveto{\pgfqpoint{1.396544in}{2.508864in}}%
\pgfpathcurveto{\pgfqpoint{1.407594in}{2.508864in}}{\pgfqpoint{1.418193in}{2.513254in}}{\pgfqpoint{1.426007in}{2.521068in}}%
\pgfpathcurveto{\pgfqpoint{1.433821in}{2.528882in}}{\pgfqpoint{1.438211in}{2.539481in}}{\pgfqpoint{1.438211in}{2.550531in}}%
\pgfpathcurveto{\pgfqpoint{1.438211in}{2.561581in}}{\pgfqpoint{1.433821in}{2.572180in}}{\pgfqpoint{1.426007in}{2.579993in}}%
\pgfpathcurveto{\pgfqpoint{1.418193in}{2.587807in}}{\pgfqpoint{1.407594in}{2.592197in}}{\pgfqpoint{1.396544in}{2.592197in}}%
\pgfpathcurveto{\pgfqpoint{1.385494in}{2.592197in}}{\pgfqpoint{1.374895in}{2.587807in}}{\pgfqpoint{1.367081in}{2.579993in}}%
\pgfpathcurveto{\pgfqpoint{1.359268in}{2.572180in}}{\pgfqpoint{1.354878in}{2.561581in}}{\pgfqpoint{1.354878in}{2.550531in}}%
\pgfpathcurveto{\pgfqpoint{1.354878in}{2.539481in}}{\pgfqpoint{1.359268in}{2.528882in}}{\pgfqpoint{1.367081in}{2.521068in}}%
\pgfpathcurveto{\pgfqpoint{1.374895in}{2.513254in}}{\pgfqpoint{1.385494in}{2.508864in}}{\pgfqpoint{1.396544in}{2.508864in}}%
\pgfpathlineto{\pgfqpoint{1.396544in}{2.508864in}}%
\pgfpathclose%
\pgfusepath{stroke}%
\end{pgfscope}%
\begin{pgfscope}%
\pgfpathrectangle{\pgfqpoint{0.847223in}{0.554012in}}{\pgfqpoint{6.200000in}{4.530000in}}%
\pgfusepath{clip}%
\pgfsetbuttcap%
\pgfsetroundjoin%
\pgfsetlinewidth{1.003750pt}%
\definecolor{currentstroke}{rgb}{1.000000,0.000000,0.000000}%
\pgfsetstrokecolor{currentstroke}%
\pgfsetdash{}{0pt}%
\pgfpathmoveto{\pgfqpoint{1.401877in}{2.496923in}}%
\pgfpathcurveto{\pgfqpoint{1.412928in}{2.496923in}}{\pgfqpoint{1.423527in}{2.501313in}}{\pgfqpoint{1.431340in}{2.509127in}}%
\pgfpathcurveto{\pgfqpoint{1.439154in}{2.516940in}}{\pgfqpoint{1.443544in}{2.527539in}}{\pgfqpoint{1.443544in}{2.538589in}}%
\pgfpathcurveto{\pgfqpoint{1.443544in}{2.549640in}}{\pgfqpoint{1.439154in}{2.560239in}}{\pgfqpoint{1.431340in}{2.568052in}}%
\pgfpathcurveto{\pgfqpoint{1.423527in}{2.575866in}}{\pgfqpoint{1.412928in}{2.580256in}}{\pgfqpoint{1.401877in}{2.580256in}}%
\pgfpathcurveto{\pgfqpoint{1.390827in}{2.580256in}}{\pgfqpoint{1.380228in}{2.575866in}}{\pgfqpoint{1.372415in}{2.568052in}}%
\pgfpathcurveto{\pgfqpoint{1.364601in}{2.560239in}}{\pgfqpoint{1.360211in}{2.549640in}}{\pgfqpoint{1.360211in}{2.538589in}}%
\pgfpathcurveto{\pgfqpoint{1.360211in}{2.527539in}}{\pgfqpoint{1.364601in}{2.516940in}}{\pgfqpoint{1.372415in}{2.509127in}}%
\pgfpathcurveto{\pgfqpoint{1.380228in}{2.501313in}}{\pgfqpoint{1.390827in}{2.496923in}}{\pgfqpoint{1.401877in}{2.496923in}}%
\pgfpathlineto{\pgfqpoint{1.401877in}{2.496923in}}%
\pgfpathclose%
\pgfusepath{stroke}%
\end{pgfscope}%
\begin{pgfscope}%
\pgfpathrectangle{\pgfqpoint{0.847223in}{0.554012in}}{\pgfqpoint{6.200000in}{4.530000in}}%
\pgfusepath{clip}%
\pgfsetbuttcap%
\pgfsetroundjoin%
\pgfsetlinewidth{1.003750pt}%
\definecolor{currentstroke}{rgb}{1.000000,0.000000,0.000000}%
\pgfsetstrokecolor{currentstroke}%
\pgfsetdash{}{0pt}%
\pgfpathmoveto{\pgfqpoint{1.407211in}{2.485099in}}%
\pgfpathcurveto{\pgfqpoint{1.418261in}{2.485099in}}{\pgfqpoint{1.428860in}{2.489489in}}{\pgfqpoint{1.436673in}{2.497303in}}%
\pgfpathcurveto{\pgfqpoint{1.444487in}{2.505116in}}{\pgfqpoint{1.448877in}{2.515716in}}{\pgfqpoint{1.448877in}{2.526766in}}%
\pgfpathcurveto{\pgfqpoint{1.448877in}{2.537816in}}{\pgfqpoint{1.444487in}{2.548415in}}{\pgfqpoint{1.436673in}{2.556228in}}%
\pgfpathcurveto{\pgfqpoint{1.428860in}{2.564042in}}{\pgfqpoint{1.418261in}{2.568432in}}{\pgfqpoint{1.407211in}{2.568432in}}%
\pgfpathcurveto{\pgfqpoint{1.396161in}{2.568432in}}{\pgfqpoint{1.385562in}{2.564042in}}{\pgfqpoint{1.377748in}{2.556228in}}%
\pgfpathcurveto{\pgfqpoint{1.369934in}{2.548415in}}{\pgfqpoint{1.365544in}{2.537816in}}{\pgfqpoint{1.365544in}{2.526766in}}%
\pgfpathcurveto{\pgfqpoint{1.365544in}{2.515716in}}{\pgfqpoint{1.369934in}{2.505116in}}{\pgfqpoint{1.377748in}{2.497303in}}%
\pgfpathcurveto{\pgfqpoint{1.385562in}{2.489489in}}{\pgfqpoint{1.396161in}{2.485099in}}{\pgfqpoint{1.407211in}{2.485099in}}%
\pgfpathlineto{\pgfqpoint{1.407211in}{2.485099in}}%
\pgfpathclose%
\pgfusepath{stroke}%
\end{pgfscope}%
\begin{pgfscope}%
\pgfpathrectangle{\pgfqpoint{0.847223in}{0.554012in}}{\pgfqpoint{6.200000in}{4.530000in}}%
\pgfusepath{clip}%
\pgfsetbuttcap%
\pgfsetroundjoin%
\pgfsetlinewidth{1.003750pt}%
\definecolor{currentstroke}{rgb}{1.000000,0.000000,0.000000}%
\pgfsetstrokecolor{currentstroke}%
\pgfsetdash{}{0pt}%
\pgfpathmoveto{\pgfqpoint{1.412544in}{2.473391in}}%
\pgfpathcurveto{\pgfqpoint{1.423594in}{2.473391in}}{\pgfqpoint{1.434193in}{2.477781in}}{\pgfqpoint{1.442007in}{2.485595in}}%
\pgfpathcurveto{\pgfqpoint{1.449820in}{2.493409in}}{\pgfqpoint{1.454211in}{2.504008in}}{\pgfqpoint{1.454211in}{2.515058in}}%
\pgfpathcurveto{\pgfqpoint{1.454211in}{2.526108in}}{\pgfqpoint{1.449820in}{2.536707in}}{\pgfqpoint{1.442007in}{2.544521in}}%
\pgfpathcurveto{\pgfqpoint{1.434193in}{2.552334in}}{\pgfqpoint{1.423594in}{2.556724in}}{\pgfqpoint{1.412544in}{2.556724in}}%
\pgfpathcurveto{\pgfqpoint{1.401494in}{2.556724in}}{\pgfqpoint{1.390895in}{2.552334in}}{\pgfqpoint{1.383081in}{2.544521in}}%
\pgfpathcurveto{\pgfqpoint{1.375268in}{2.536707in}}{\pgfqpoint{1.370877in}{2.526108in}}{\pgfqpoint{1.370877in}{2.515058in}}%
\pgfpathcurveto{\pgfqpoint{1.370877in}{2.504008in}}{\pgfqpoint{1.375268in}{2.493409in}}{\pgfqpoint{1.383081in}{2.485595in}}%
\pgfpathcurveto{\pgfqpoint{1.390895in}{2.477781in}}{\pgfqpoint{1.401494in}{2.473391in}}{\pgfqpoint{1.412544in}{2.473391in}}%
\pgfpathlineto{\pgfqpoint{1.412544in}{2.473391in}}%
\pgfpathclose%
\pgfusepath{stroke}%
\end{pgfscope}%
\begin{pgfscope}%
\pgfpathrectangle{\pgfqpoint{0.847223in}{0.554012in}}{\pgfqpoint{6.200000in}{4.530000in}}%
\pgfusepath{clip}%
\pgfsetbuttcap%
\pgfsetroundjoin%
\pgfsetlinewidth{1.003750pt}%
\definecolor{currentstroke}{rgb}{1.000000,0.000000,0.000000}%
\pgfsetstrokecolor{currentstroke}%
\pgfsetdash{}{0pt}%
\pgfpathmoveto{\pgfqpoint{1.417877in}{2.461797in}}%
\pgfpathcurveto{\pgfqpoint{1.428927in}{2.461797in}}{\pgfqpoint{1.439526in}{2.466188in}}{\pgfqpoint{1.447340in}{2.474001in}}%
\pgfpathcurveto{\pgfqpoint{1.455154in}{2.481815in}}{\pgfqpoint{1.459544in}{2.492414in}}{\pgfqpoint{1.459544in}{2.503464in}}%
\pgfpathcurveto{\pgfqpoint{1.459544in}{2.514514in}}{\pgfqpoint{1.455154in}{2.525113in}}{\pgfqpoint{1.447340in}{2.532927in}}%
\pgfpathcurveto{\pgfqpoint{1.439526in}{2.540740in}}{\pgfqpoint{1.428927in}{2.545131in}}{\pgfqpoint{1.417877in}{2.545131in}}%
\pgfpathcurveto{\pgfqpoint{1.406827in}{2.545131in}}{\pgfqpoint{1.396228in}{2.540740in}}{\pgfqpoint{1.388414in}{2.532927in}}%
\pgfpathcurveto{\pgfqpoint{1.380601in}{2.525113in}}{\pgfqpoint{1.376210in}{2.514514in}}{\pgfqpoint{1.376210in}{2.503464in}}%
\pgfpathcurveto{\pgfqpoint{1.376210in}{2.492414in}}{\pgfqpoint{1.380601in}{2.481815in}}{\pgfqpoint{1.388414in}{2.474001in}}%
\pgfpathcurveto{\pgfqpoint{1.396228in}{2.466188in}}{\pgfqpoint{1.406827in}{2.461797in}}{\pgfqpoint{1.417877in}{2.461797in}}%
\pgfpathlineto{\pgfqpoint{1.417877in}{2.461797in}}%
\pgfpathclose%
\pgfusepath{stroke}%
\end{pgfscope}%
\begin{pgfscope}%
\pgfpathrectangle{\pgfqpoint{0.847223in}{0.554012in}}{\pgfqpoint{6.200000in}{4.530000in}}%
\pgfusepath{clip}%
\pgfsetbuttcap%
\pgfsetroundjoin%
\pgfsetlinewidth{1.003750pt}%
\definecolor{currentstroke}{rgb}{1.000000,0.000000,0.000000}%
\pgfsetstrokecolor{currentstroke}%
\pgfsetdash{}{0pt}%
\pgfpathmoveto{\pgfqpoint{1.423210in}{2.450316in}}%
\pgfpathcurveto{\pgfqpoint{1.434260in}{2.450316in}}{\pgfqpoint{1.444860in}{2.454706in}}{\pgfqpoint{1.452673in}{2.462520in}}%
\pgfpathcurveto{\pgfqpoint{1.460487in}{2.470334in}}{\pgfqpoint{1.464877in}{2.480933in}}{\pgfqpoint{1.464877in}{2.491983in}}%
\pgfpathcurveto{\pgfqpoint{1.464877in}{2.503033in}}{\pgfqpoint{1.460487in}{2.513632in}}{\pgfqpoint{1.452673in}{2.521446in}}%
\pgfpathcurveto{\pgfqpoint{1.444860in}{2.529259in}}{\pgfqpoint{1.434260in}{2.533649in}}{\pgfqpoint{1.423210in}{2.533649in}}%
\pgfpathcurveto{\pgfqpoint{1.412160in}{2.533649in}}{\pgfqpoint{1.401561in}{2.529259in}}{\pgfqpoint{1.393748in}{2.521446in}}%
\pgfpathcurveto{\pgfqpoint{1.385934in}{2.513632in}}{\pgfqpoint{1.381544in}{2.503033in}}{\pgfqpoint{1.381544in}{2.491983in}}%
\pgfpathcurveto{\pgfqpoint{1.381544in}{2.480933in}}{\pgfqpoint{1.385934in}{2.470334in}}{\pgfqpoint{1.393748in}{2.462520in}}%
\pgfpathcurveto{\pgfqpoint{1.401561in}{2.454706in}}{\pgfqpoint{1.412160in}{2.450316in}}{\pgfqpoint{1.423210in}{2.450316in}}%
\pgfpathlineto{\pgfqpoint{1.423210in}{2.450316in}}%
\pgfpathclose%
\pgfusepath{stroke}%
\end{pgfscope}%
\begin{pgfscope}%
\pgfpathrectangle{\pgfqpoint{0.847223in}{0.554012in}}{\pgfqpoint{6.200000in}{4.530000in}}%
\pgfusepath{clip}%
\pgfsetbuttcap%
\pgfsetroundjoin%
\pgfsetlinewidth{1.003750pt}%
\definecolor{currentstroke}{rgb}{1.000000,0.000000,0.000000}%
\pgfsetstrokecolor{currentstroke}%
\pgfsetdash{}{0pt}%
\pgfpathmoveto{\pgfqpoint{1.428544in}{2.438946in}}%
\pgfpathcurveto{\pgfqpoint{1.439594in}{2.438946in}}{\pgfqpoint{1.450193in}{2.443336in}}{\pgfqpoint{1.458006in}{2.451150in}}%
\pgfpathcurveto{\pgfqpoint{1.465820in}{2.458963in}}{\pgfqpoint{1.470210in}{2.469562in}}{\pgfqpoint{1.470210in}{2.480612in}}%
\pgfpathcurveto{\pgfqpoint{1.470210in}{2.491662in}}{\pgfqpoint{1.465820in}{2.502262in}}{\pgfqpoint{1.458006in}{2.510075in}}%
\pgfpathcurveto{\pgfqpoint{1.450193in}{2.517889in}}{\pgfqpoint{1.439594in}{2.522279in}}{\pgfqpoint{1.428544in}{2.522279in}}%
\pgfpathcurveto{\pgfqpoint{1.417493in}{2.522279in}}{\pgfqpoint{1.406894in}{2.517889in}}{\pgfqpoint{1.399081in}{2.510075in}}%
\pgfpathcurveto{\pgfqpoint{1.391267in}{2.502262in}}{\pgfqpoint{1.386877in}{2.491662in}}{\pgfqpoint{1.386877in}{2.480612in}}%
\pgfpathcurveto{\pgfqpoint{1.386877in}{2.469562in}}{\pgfqpoint{1.391267in}{2.458963in}}{\pgfqpoint{1.399081in}{2.451150in}}%
\pgfpathcurveto{\pgfqpoint{1.406894in}{2.443336in}}{\pgfqpoint{1.417493in}{2.438946in}}{\pgfqpoint{1.428544in}{2.438946in}}%
\pgfpathlineto{\pgfqpoint{1.428544in}{2.438946in}}%
\pgfpathclose%
\pgfusepath{stroke}%
\end{pgfscope}%
\begin{pgfscope}%
\pgfpathrectangle{\pgfqpoint{0.847223in}{0.554012in}}{\pgfqpoint{6.200000in}{4.530000in}}%
\pgfusepath{clip}%
\pgfsetbuttcap%
\pgfsetroundjoin%
\pgfsetlinewidth{1.003750pt}%
\definecolor{currentstroke}{rgb}{1.000000,0.000000,0.000000}%
\pgfsetstrokecolor{currentstroke}%
\pgfsetdash{}{0pt}%
\pgfpathmoveto{\pgfqpoint{1.433877in}{2.427685in}}%
\pgfpathcurveto{\pgfqpoint{1.444927in}{2.427685in}}{\pgfqpoint{1.455526in}{2.432075in}}{\pgfqpoint{1.463340in}{2.439888in}}%
\pgfpathcurveto{\pgfqpoint{1.471153in}{2.447702in}}{\pgfqpoint{1.475543in}{2.458301in}}{\pgfqpoint{1.475543in}{2.469351in}}%
\pgfpathcurveto{\pgfqpoint{1.475543in}{2.480401in}}{\pgfqpoint{1.471153in}{2.491000in}}{\pgfqpoint{1.463340in}{2.498814in}}%
\pgfpathcurveto{\pgfqpoint{1.455526in}{2.506628in}}{\pgfqpoint{1.444927in}{2.511018in}}{\pgfqpoint{1.433877in}{2.511018in}}%
\pgfpathcurveto{\pgfqpoint{1.422827in}{2.511018in}}{\pgfqpoint{1.412228in}{2.506628in}}{\pgfqpoint{1.404414in}{2.498814in}}%
\pgfpathcurveto{\pgfqpoint{1.396600in}{2.491000in}}{\pgfqpoint{1.392210in}{2.480401in}}{\pgfqpoint{1.392210in}{2.469351in}}%
\pgfpathcurveto{\pgfqpoint{1.392210in}{2.458301in}}{\pgfqpoint{1.396600in}{2.447702in}}{\pgfqpoint{1.404414in}{2.439888in}}%
\pgfpathcurveto{\pgfqpoint{1.412228in}{2.432075in}}{\pgfqpoint{1.422827in}{2.427685in}}{\pgfqpoint{1.433877in}{2.427685in}}%
\pgfpathlineto{\pgfqpoint{1.433877in}{2.427685in}}%
\pgfpathclose%
\pgfusepath{stroke}%
\end{pgfscope}%
\begin{pgfscope}%
\pgfpathrectangle{\pgfqpoint{0.847223in}{0.554012in}}{\pgfqpoint{6.200000in}{4.530000in}}%
\pgfusepath{clip}%
\pgfsetbuttcap%
\pgfsetroundjoin%
\pgfsetlinewidth{1.003750pt}%
\definecolor{currentstroke}{rgb}{1.000000,0.000000,0.000000}%
\pgfsetstrokecolor{currentstroke}%
\pgfsetdash{}{0pt}%
\pgfpathmoveto{\pgfqpoint{1.439210in}{2.416531in}}%
\pgfpathcurveto{\pgfqpoint{1.450260in}{2.416531in}}{\pgfqpoint{1.460859in}{2.420921in}}{\pgfqpoint{1.468673in}{2.428735in}}%
\pgfpathcurveto{\pgfqpoint{1.476486in}{2.436549in}}{\pgfqpoint{1.480877in}{2.447148in}}{\pgfqpoint{1.480877in}{2.458198in}}%
\pgfpathcurveto{\pgfqpoint{1.480877in}{2.469248in}}{\pgfqpoint{1.476486in}{2.479847in}}{\pgfqpoint{1.468673in}{2.487661in}}%
\pgfpathcurveto{\pgfqpoint{1.460859in}{2.495474in}}{\pgfqpoint{1.450260in}{2.499864in}}{\pgfqpoint{1.439210in}{2.499864in}}%
\pgfpathcurveto{\pgfqpoint{1.428160in}{2.499864in}}{\pgfqpoint{1.417561in}{2.495474in}}{\pgfqpoint{1.409747in}{2.487661in}}%
\pgfpathcurveto{\pgfqpoint{1.401934in}{2.479847in}}{\pgfqpoint{1.397543in}{2.469248in}}{\pgfqpoint{1.397543in}{2.458198in}}%
\pgfpathcurveto{\pgfqpoint{1.397543in}{2.447148in}}{\pgfqpoint{1.401934in}{2.436549in}}{\pgfqpoint{1.409747in}{2.428735in}}%
\pgfpathcurveto{\pgfqpoint{1.417561in}{2.420921in}}{\pgfqpoint{1.428160in}{2.416531in}}{\pgfqpoint{1.439210in}{2.416531in}}%
\pgfpathlineto{\pgfqpoint{1.439210in}{2.416531in}}%
\pgfpathclose%
\pgfusepath{stroke}%
\end{pgfscope}%
\begin{pgfscope}%
\pgfpathrectangle{\pgfqpoint{0.847223in}{0.554012in}}{\pgfqpoint{6.200000in}{4.530000in}}%
\pgfusepath{clip}%
\pgfsetbuttcap%
\pgfsetroundjoin%
\pgfsetlinewidth{1.003750pt}%
\definecolor{currentstroke}{rgb}{1.000000,0.000000,0.000000}%
\pgfsetstrokecolor{currentstroke}%
\pgfsetdash{}{0pt}%
\pgfpathmoveto{\pgfqpoint{1.444543in}{2.405484in}}%
\pgfpathcurveto{\pgfqpoint{1.455593in}{2.405484in}}{\pgfqpoint{1.466192in}{2.409874in}}{\pgfqpoint{1.474006in}{2.417688in}}%
\pgfpathcurveto{\pgfqpoint{1.481820in}{2.425501in}}{\pgfqpoint{1.486210in}{2.436100in}}{\pgfqpoint{1.486210in}{2.447150in}}%
\pgfpathcurveto{\pgfqpoint{1.486210in}{2.458201in}}{\pgfqpoint{1.481820in}{2.468800in}}{\pgfqpoint{1.474006in}{2.476613in}}%
\pgfpathcurveto{\pgfqpoint{1.466192in}{2.484427in}}{\pgfqpoint{1.455593in}{2.488817in}}{\pgfqpoint{1.444543in}{2.488817in}}%
\pgfpathcurveto{\pgfqpoint{1.433493in}{2.488817in}}{\pgfqpoint{1.422894in}{2.484427in}}{\pgfqpoint{1.415080in}{2.476613in}}%
\pgfpathcurveto{\pgfqpoint{1.407267in}{2.468800in}}{\pgfqpoint{1.402877in}{2.458201in}}{\pgfqpoint{1.402877in}{2.447150in}}%
\pgfpathcurveto{\pgfqpoint{1.402877in}{2.436100in}}{\pgfqpoint{1.407267in}{2.425501in}}{\pgfqpoint{1.415080in}{2.417688in}}%
\pgfpathcurveto{\pgfqpoint{1.422894in}{2.409874in}}{\pgfqpoint{1.433493in}{2.405484in}}{\pgfqpoint{1.444543in}{2.405484in}}%
\pgfpathlineto{\pgfqpoint{1.444543in}{2.405484in}}%
\pgfpathclose%
\pgfusepath{stroke}%
\end{pgfscope}%
\begin{pgfscope}%
\pgfpathrectangle{\pgfqpoint{0.847223in}{0.554012in}}{\pgfqpoint{6.200000in}{4.530000in}}%
\pgfusepath{clip}%
\pgfsetbuttcap%
\pgfsetroundjoin%
\pgfsetlinewidth{1.003750pt}%
\definecolor{currentstroke}{rgb}{1.000000,0.000000,0.000000}%
\pgfsetstrokecolor{currentstroke}%
\pgfsetdash{}{0pt}%
\pgfpathmoveto{\pgfqpoint{1.449876in}{2.394541in}}%
\pgfpathcurveto{\pgfqpoint{1.460927in}{2.394541in}}{\pgfqpoint{1.471526in}{2.398931in}}{\pgfqpoint{1.479339in}{2.406745in}}%
\pgfpathcurveto{\pgfqpoint{1.487153in}{2.414559in}}{\pgfqpoint{1.491543in}{2.425158in}}{\pgfqpoint{1.491543in}{2.436208in}}%
\pgfpathcurveto{\pgfqpoint{1.491543in}{2.447258in}}{\pgfqpoint{1.487153in}{2.457857in}}{\pgfqpoint{1.479339in}{2.465671in}}%
\pgfpathcurveto{\pgfqpoint{1.471526in}{2.473484in}}{\pgfqpoint{1.460927in}{2.477875in}}{\pgfqpoint{1.449876in}{2.477875in}}%
\pgfpathcurveto{\pgfqpoint{1.438826in}{2.477875in}}{\pgfqpoint{1.428227in}{2.473484in}}{\pgfqpoint{1.420414in}{2.465671in}}%
\pgfpathcurveto{\pgfqpoint{1.412600in}{2.457857in}}{\pgfqpoint{1.408210in}{2.447258in}}{\pgfqpoint{1.408210in}{2.436208in}}%
\pgfpathcurveto{\pgfqpoint{1.408210in}{2.425158in}}{\pgfqpoint{1.412600in}{2.414559in}}{\pgfqpoint{1.420414in}{2.406745in}}%
\pgfpathcurveto{\pgfqpoint{1.428227in}{2.398931in}}{\pgfqpoint{1.438826in}{2.394541in}}{\pgfqpoint{1.449876in}{2.394541in}}%
\pgfpathlineto{\pgfqpoint{1.449876in}{2.394541in}}%
\pgfpathclose%
\pgfusepath{stroke}%
\end{pgfscope}%
\begin{pgfscope}%
\pgfpathrectangle{\pgfqpoint{0.847223in}{0.554012in}}{\pgfqpoint{6.200000in}{4.530000in}}%
\pgfusepath{clip}%
\pgfsetbuttcap%
\pgfsetroundjoin%
\pgfsetlinewidth{1.003750pt}%
\definecolor{currentstroke}{rgb}{1.000000,0.000000,0.000000}%
\pgfsetstrokecolor{currentstroke}%
\pgfsetdash{}{0pt}%
\pgfpathmoveto{\pgfqpoint{1.455210in}{2.383702in}}%
\pgfpathcurveto{\pgfqpoint{1.466260in}{2.383702in}}{\pgfqpoint{1.476859in}{2.388092in}}{\pgfqpoint{1.484672in}{2.395906in}}%
\pgfpathcurveto{\pgfqpoint{1.492486in}{2.403719in}}{\pgfqpoint{1.496876in}{2.414318in}}{\pgfqpoint{1.496876in}{2.425368in}}%
\pgfpathcurveto{\pgfqpoint{1.496876in}{2.436419in}}{\pgfqpoint{1.492486in}{2.447018in}}{\pgfqpoint{1.484672in}{2.454831in}}%
\pgfpathcurveto{\pgfqpoint{1.476859in}{2.462645in}}{\pgfqpoint{1.466260in}{2.467035in}}{\pgfqpoint{1.455210in}{2.467035in}}%
\pgfpathcurveto{\pgfqpoint{1.444160in}{2.467035in}}{\pgfqpoint{1.433560in}{2.462645in}}{\pgfqpoint{1.425747in}{2.454831in}}%
\pgfpathcurveto{\pgfqpoint{1.417933in}{2.447018in}}{\pgfqpoint{1.413543in}{2.436419in}}{\pgfqpoint{1.413543in}{2.425368in}}%
\pgfpathcurveto{\pgfqpoint{1.413543in}{2.414318in}}{\pgfqpoint{1.417933in}{2.403719in}}{\pgfqpoint{1.425747in}{2.395906in}}%
\pgfpathcurveto{\pgfqpoint{1.433560in}{2.388092in}}{\pgfqpoint{1.444160in}{2.383702in}}{\pgfqpoint{1.455210in}{2.383702in}}%
\pgfpathlineto{\pgfqpoint{1.455210in}{2.383702in}}%
\pgfpathclose%
\pgfusepath{stroke}%
\end{pgfscope}%
\begin{pgfscope}%
\pgfpathrectangle{\pgfqpoint{0.847223in}{0.554012in}}{\pgfqpoint{6.200000in}{4.530000in}}%
\pgfusepath{clip}%
\pgfsetbuttcap%
\pgfsetroundjoin%
\pgfsetlinewidth{1.003750pt}%
\definecolor{currentstroke}{rgb}{1.000000,0.000000,0.000000}%
\pgfsetstrokecolor{currentstroke}%
\pgfsetdash{}{0pt}%
\pgfpathmoveto{\pgfqpoint{1.460543in}{2.372964in}}%
\pgfpathcurveto{\pgfqpoint{1.471593in}{2.372964in}}{\pgfqpoint{1.482192in}{2.377354in}}{\pgfqpoint{1.490006in}{2.385168in}}%
\pgfpathcurveto{\pgfqpoint{1.497819in}{2.392981in}}{\pgfqpoint{1.502210in}{2.403581in}}{\pgfqpoint{1.502210in}{2.414631in}}%
\pgfpathcurveto{\pgfqpoint{1.502210in}{2.425681in}}{\pgfqpoint{1.497819in}{2.436280in}}{\pgfqpoint{1.490006in}{2.444093in}}%
\pgfpathcurveto{\pgfqpoint{1.482192in}{2.451907in}}{\pgfqpoint{1.471593in}{2.456297in}}{\pgfqpoint{1.460543in}{2.456297in}}%
\pgfpathcurveto{\pgfqpoint{1.449493in}{2.456297in}}{\pgfqpoint{1.438894in}{2.451907in}}{\pgfqpoint{1.431080in}{2.444093in}}%
\pgfpathcurveto{\pgfqpoint{1.423266in}{2.436280in}}{\pgfqpoint{1.418876in}{2.425681in}}{\pgfqpoint{1.418876in}{2.414631in}}%
\pgfpathcurveto{\pgfqpoint{1.418876in}{2.403581in}}{\pgfqpoint{1.423266in}{2.392981in}}{\pgfqpoint{1.431080in}{2.385168in}}%
\pgfpathcurveto{\pgfqpoint{1.438894in}{2.377354in}}{\pgfqpoint{1.449493in}{2.372964in}}{\pgfqpoint{1.460543in}{2.372964in}}%
\pgfpathlineto{\pgfqpoint{1.460543in}{2.372964in}}%
\pgfpathclose%
\pgfusepath{stroke}%
\end{pgfscope}%
\begin{pgfscope}%
\pgfpathrectangle{\pgfqpoint{0.847223in}{0.554012in}}{\pgfqpoint{6.200000in}{4.530000in}}%
\pgfusepath{clip}%
\pgfsetbuttcap%
\pgfsetroundjoin%
\pgfsetlinewidth{1.003750pt}%
\definecolor{currentstroke}{rgb}{1.000000,0.000000,0.000000}%
\pgfsetstrokecolor{currentstroke}%
\pgfsetdash{}{0pt}%
\pgfpathmoveto{\pgfqpoint{1.465876in}{2.362327in}}%
\pgfpathcurveto{\pgfqpoint{1.476926in}{2.362327in}}{\pgfqpoint{1.487525in}{2.366717in}}{\pgfqpoint{1.495339in}{2.374530in}}%
\pgfpathcurveto{\pgfqpoint{1.503152in}{2.382344in}}{\pgfqpoint{1.507543in}{2.392943in}}{\pgfqpoint{1.507543in}{2.403993in}}%
\pgfpathcurveto{\pgfqpoint{1.507543in}{2.415043in}}{\pgfqpoint{1.503152in}{2.425642in}}{\pgfqpoint{1.495339in}{2.433456in}}%
\pgfpathcurveto{\pgfqpoint{1.487525in}{2.441270in}}{\pgfqpoint{1.476926in}{2.445660in}}{\pgfqpoint{1.465876in}{2.445660in}}%
\pgfpathcurveto{\pgfqpoint{1.454826in}{2.445660in}}{\pgfqpoint{1.444227in}{2.441270in}}{\pgfqpoint{1.436413in}{2.433456in}}%
\pgfpathcurveto{\pgfqpoint{1.428600in}{2.425642in}}{\pgfqpoint{1.424209in}{2.415043in}}{\pgfqpoint{1.424209in}{2.403993in}}%
\pgfpathcurveto{\pgfqpoint{1.424209in}{2.392943in}}{\pgfqpoint{1.428600in}{2.382344in}}{\pgfqpoint{1.436413in}{2.374530in}}%
\pgfpathcurveto{\pgfqpoint{1.444227in}{2.366717in}}{\pgfqpoint{1.454826in}{2.362327in}}{\pgfqpoint{1.465876in}{2.362327in}}%
\pgfpathlineto{\pgfqpoint{1.465876in}{2.362327in}}%
\pgfpathclose%
\pgfusepath{stroke}%
\end{pgfscope}%
\begin{pgfscope}%
\pgfpathrectangle{\pgfqpoint{0.847223in}{0.554012in}}{\pgfqpoint{6.200000in}{4.530000in}}%
\pgfusepath{clip}%
\pgfsetbuttcap%
\pgfsetroundjoin%
\pgfsetlinewidth{1.003750pt}%
\definecolor{currentstroke}{rgb}{1.000000,0.000000,0.000000}%
\pgfsetstrokecolor{currentstroke}%
\pgfsetdash{}{0pt}%
\pgfpathmoveto{\pgfqpoint{1.471209in}{2.351788in}}%
\pgfpathcurveto{\pgfqpoint{1.482259in}{2.351788in}}{\pgfqpoint{1.492858in}{2.356178in}}{\pgfqpoint{1.500672in}{2.363992in}}%
\pgfpathcurveto{\pgfqpoint{1.508486in}{2.371805in}}{\pgfqpoint{1.512876in}{2.382404in}}{\pgfqpoint{1.512876in}{2.393455in}}%
\pgfpathcurveto{\pgfqpoint{1.512876in}{2.404505in}}{\pgfqpoint{1.508486in}{2.415104in}}{\pgfqpoint{1.500672in}{2.422917in}}%
\pgfpathcurveto{\pgfqpoint{1.492858in}{2.430731in}}{\pgfqpoint{1.482259in}{2.435121in}}{\pgfqpoint{1.471209in}{2.435121in}}%
\pgfpathcurveto{\pgfqpoint{1.460159in}{2.435121in}}{\pgfqpoint{1.449560in}{2.430731in}}{\pgfqpoint{1.441747in}{2.422917in}}%
\pgfpathcurveto{\pgfqpoint{1.433933in}{2.415104in}}{\pgfqpoint{1.429543in}{2.404505in}}{\pgfqpoint{1.429543in}{2.393455in}}%
\pgfpathcurveto{\pgfqpoint{1.429543in}{2.382404in}}{\pgfqpoint{1.433933in}{2.371805in}}{\pgfqpoint{1.441747in}{2.363992in}}%
\pgfpathcurveto{\pgfqpoint{1.449560in}{2.356178in}}{\pgfqpoint{1.460159in}{2.351788in}}{\pgfqpoint{1.471209in}{2.351788in}}%
\pgfpathlineto{\pgfqpoint{1.471209in}{2.351788in}}%
\pgfpathclose%
\pgfusepath{stroke}%
\end{pgfscope}%
\begin{pgfscope}%
\pgfpathrectangle{\pgfqpoint{0.847223in}{0.554012in}}{\pgfqpoint{6.200000in}{4.530000in}}%
\pgfusepath{clip}%
\pgfsetbuttcap%
\pgfsetroundjoin%
\pgfsetlinewidth{1.003750pt}%
\definecolor{currentstroke}{rgb}{1.000000,0.000000,0.000000}%
\pgfsetstrokecolor{currentstroke}%
\pgfsetdash{}{0pt}%
\pgfpathmoveto{\pgfqpoint{1.476543in}{2.341347in}}%
\pgfpathcurveto{\pgfqpoint{1.487593in}{2.341347in}}{\pgfqpoint{1.498192in}{2.345737in}}{\pgfqpoint{1.506005in}{2.353551in}}%
\pgfpathcurveto{\pgfqpoint{1.513819in}{2.361364in}}{\pgfqpoint{1.518209in}{2.371963in}}{\pgfqpoint{1.518209in}{2.383014in}}%
\pgfpathcurveto{\pgfqpoint{1.518209in}{2.394064in}}{\pgfqpoint{1.513819in}{2.404663in}}{\pgfqpoint{1.506005in}{2.412476in}}%
\pgfpathcurveto{\pgfqpoint{1.498192in}{2.420290in}}{\pgfqpoint{1.487593in}{2.424680in}}{\pgfqpoint{1.476543in}{2.424680in}}%
\pgfpathcurveto{\pgfqpoint{1.465492in}{2.424680in}}{\pgfqpoint{1.454893in}{2.420290in}}{\pgfqpoint{1.447080in}{2.412476in}}%
\pgfpathcurveto{\pgfqpoint{1.439266in}{2.404663in}}{\pgfqpoint{1.434876in}{2.394064in}}{\pgfqpoint{1.434876in}{2.383014in}}%
\pgfpathcurveto{\pgfqpoint{1.434876in}{2.371963in}}{\pgfqpoint{1.439266in}{2.361364in}}{\pgfqpoint{1.447080in}{2.353551in}}%
\pgfpathcurveto{\pgfqpoint{1.454893in}{2.345737in}}{\pgfqpoint{1.465492in}{2.341347in}}{\pgfqpoint{1.476543in}{2.341347in}}%
\pgfpathlineto{\pgfqpoint{1.476543in}{2.341347in}}%
\pgfpathclose%
\pgfusepath{stroke}%
\end{pgfscope}%
\begin{pgfscope}%
\pgfpathrectangle{\pgfqpoint{0.847223in}{0.554012in}}{\pgfqpoint{6.200000in}{4.530000in}}%
\pgfusepath{clip}%
\pgfsetbuttcap%
\pgfsetroundjoin%
\pgfsetlinewidth{1.003750pt}%
\definecolor{currentstroke}{rgb}{1.000000,0.000000,0.000000}%
\pgfsetstrokecolor{currentstroke}%
\pgfsetdash{}{0pt}%
\pgfpathmoveto{\pgfqpoint{1.481876in}{2.331002in}}%
\pgfpathcurveto{\pgfqpoint{1.492926in}{2.331002in}}{\pgfqpoint{1.503525in}{2.335392in}}{\pgfqpoint{1.511339in}{2.343206in}}%
\pgfpathcurveto{\pgfqpoint{1.519152in}{2.351020in}}{\pgfqpoint{1.523542in}{2.361619in}}{\pgfqpoint{1.523542in}{2.372669in}}%
\pgfpathcurveto{\pgfqpoint{1.523542in}{2.383719in}}{\pgfqpoint{1.519152in}{2.394318in}}{\pgfqpoint{1.511339in}{2.402131in}}%
\pgfpathcurveto{\pgfqpoint{1.503525in}{2.409945in}}{\pgfqpoint{1.492926in}{2.414335in}}{\pgfqpoint{1.481876in}{2.414335in}}%
\pgfpathcurveto{\pgfqpoint{1.470826in}{2.414335in}}{\pgfqpoint{1.460227in}{2.409945in}}{\pgfqpoint{1.452413in}{2.402131in}}%
\pgfpathcurveto{\pgfqpoint{1.444599in}{2.394318in}}{\pgfqpoint{1.440209in}{2.383719in}}{\pgfqpoint{1.440209in}{2.372669in}}%
\pgfpathcurveto{\pgfqpoint{1.440209in}{2.361619in}}{\pgfqpoint{1.444599in}{2.351020in}}{\pgfqpoint{1.452413in}{2.343206in}}%
\pgfpathcurveto{\pgfqpoint{1.460227in}{2.335392in}}{\pgfqpoint{1.470826in}{2.331002in}}{\pgfqpoint{1.481876in}{2.331002in}}%
\pgfpathlineto{\pgfqpoint{1.481876in}{2.331002in}}%
\pgfpathclose%
\pgfusepath{stroke}%
\end{pgfscope}%
\begin{pgfscope}%
\pgfpathrectangle{\pgfqpoint{0.847223in}{0.554012in}}{\pgfqpoint{6.200000in}{4.530000in}}%
\pgfusepath{clip}%
\pgfsetbuttcap%
\pgfsetroundjoin%
\pgfsetlinewidth{1.003750pt}%
\definecolor{currentstroke}{rgb}{1.000000,0.000000,0.000000}%
\pgfsetstrokecolor{currentstroke}%
\pgfsetdash{}{0pt}%
\pgfpathmoveto{\pgfqpoint{1.487209in}{2.320752in}}%
\pgfpathcurveto{\pgfqpoint{1.498259in}{2.320752in}}{\pgfqpoint{1.508858in}{2.325142in}}{\pgfqpoint{1.516672in}{2.332956in}}%
\pgfpathcurveto{\pgfqpoint{1.524485in}{2.340770in}}{\pgfqpoint{1.528876in}{2.351369in}}{\pgfqpoint{1.528876in}{2.362419in}}%
\pgfpathcurveto{\pgfqpoint{1.528876in}{2.373469in}}{\pgfqpoint{1.524485in}{2.384068in}}{\pgfqpoint{1.516672in}{2.391881in}}%
\pgfpathcurveto{\pgfqpoint{1.508858in}{2.399695in}}{\pgfqpoint{1.498259in}{2.404085in}}{\pgfqpoint{1.487209in}{2.404085in}}%
\pgfpathcurveto{\pgfqpoint{1.476159in}{2.404085in}}{\pgfqpoint{1.465560in}{2.399695in}}{\pgfqpoint{1.457746in}{2.391881in}}%
\pgfpathcurveto{\pgfqpoint{1.449933in}{2.384068in}}{\pgfqpoint{1.445542in}{2.373469in}}{\pgfqpoint{1.445542in}{2.362419in}}%
\pgfpathcurveto{\pgfqpoint{1.445542in}{2.351369in}}{\pgfqpoint{1.449933in}{2.340770in}}{\pgfqpoint{1.457746in}{2.332956in}}%
\pgfpathcurveto{\pgfqpoint{1.465560in}{2.325142in}}{\pgfqpoint{1.476159in}{2.320752in}}{\pgfqpoint{1.487209in}{2.320752in}}%
\pgfpathlineto{\pgfqpoint{1.487209in}{2.320752in}}%
\pgfpathclose%
\pgfusepath{stroke}%
\end{pgfscope}%
\begin{pgfscope}%
\pgfpathrectangle{\pgfqpoint{0.847223in}{0.554012in}}{\pgfqpoint{6.200000in}{4.530000in}}%
\pgfusepath{clip}%
\pgfsetbuttcap%
\pgfsetroundjoin%
\pgfsetlinewidth{1.003750pt}%
\definecolor{currentstroke}{rgb}{1.000000,0.000000,0.000000}%
\pgfsetstrokecolor{currentstroke}%
\pgfsetdash{}{0pt}%
\pgfpathmoveto{\pgfqpoint{1.492542in}{2.310596in}}%
\pgfpathcurveto{\pgfqpoint{1.503592in}{2.310596in}}{\pgfqpoint{1.514191in}{2.314986in}}{\pgfqpoint{1.522005in}{2.322799in}}%
\pgfpathcurveto{\pgfqpoint{1.529819in}{2.330613in}}{\pgfqpoint{1.534209in}{2.341212in}}{\pgfqpoint{1.534209in}{2.352262in}}%
\pgfpathcurveto{\pgfqpoint{1.534209in}{2.363312in}}{\pgfqpoint{1.529819in}{2.373911in}}{\pgfqpoint{1.522005in}{2.381725in}}%
\pgfpathcurveto{\pgfqpoint{1.514191in}{2.389539in}}{\pgfqpoint{1.503592in}{2.393929in}}{\pgfqpoint{1.492542in}{2.393929in}}%
\pgfpathcurveto{\pgfqpoint{1.481492in}{2.393929in}}{\pgfqpoint{1.470893in}{2.389539in}}{\pgfqpoint{1.463079in}{2.381725in}}%
\pgfpathcurveto{\pgfqpoint{1.455266in}{2.373911in}}{\pgfqpoint{1.450875in}{2.363312in}}{\pgfqpoint{1.450875in}{2.352262in}}%
\pgfpathcurveto{\pgfqpoint{1.450875in}{2.341212in}}{\pgfqpoint{1.455266in}{2.330613in}}{\pgfqpoint{1.463079in}{2.322799in}}%
\pgfpathcurveto{\pgfqpoint{1.470893in}{2.314986in}}{\pgfqpoint{1.481492in}{2.310596in}}{\pgfqpoint{1.492542in}{2.310596in}}%
\pgfpathlineto{\pgfqpoint{1.492542in}{2.310596in}}%
\pgfpathclose%
\pgfusepath{stroke}%
\end{pgfscope}%
\begin{pgfscope}%
\pgfpathrectangle{\pgfqpoint{0.847223in}{0.554012in}}{\pgfqpoint{6.200000in}{4.530000in}}%
\pgfusepath{clip}%
\pgfsetbuttcap%
\pgfsetroundjoin%
\pgfsetlinewidth{1.003750pt}%
\definecolor{currentstroke}{rgb}{1.000000,0.000000,0.000000}%
\pgfsetstrokecolor{currentstroke}%
\pgfsetdash{}{0pt}%
\pgfpathmoveto{\pgfqpoint{1.497875in}{2.300531in}}%
\pgfpathcurveto{\pgfqpoint{1.508925in}{2.300531in}}{\pgfqpoint{1.519525in}{2.304922in}}{\pgfqpoint{1.527338in}{2.312735in}}%
\pgfpathcurveto{\pgfqpoint{1.535152in}{2.320549in}}{\pgfqpoint{1.539542in}{2.331148in}}{\pgfqpoint{1.539542in}{2.342198in}}%
\pgfpathcurveto{\pgfqpoint{1.539542in}{2.353248in}}{\pgfqpoint{1.535152in}{2.363847in}}{\pgfqpoint{1.527338in}{2.371661in}}%
\pgfpathcurveto{\pgfqpoint{1.519525in}{2.379474in}}{\pgfqpoint{1.508925in}{2.383865in}}{\pgfqpoint{1.497875in}{2.383865in}}%
\pgfpathcurveto{\pgfqpoint{1.486825in}{2.383865in}}{\pgfqpoint{1.476226in}{2.379474in}}{\pgfqpoint{1.468413in}{2.371661in}}%
\pgfpathcurveto{\pgfqpoint{1.460599in}{2.363847in}}{\pgfqpoint{1.456209in}{2.353248in}}{\pgfqpoint{1.456209in}{2.342198in}}%
\pgfpathcurveto{\pgfqpoint{1.456209in}{2.331148in}}{\pgfqpoint{1.460599in}{2.320549in}}{\pgfqpoint{1.468413in}{2.312735in}}%
\pgfpathcurveto{\pgfqpoint{1.476226in}{2.304922in}}{\pgfqpoint{1.486825in}{2.300531in}}{\pgfqpoint{1.497875in}{2.300531in}}%
\pgfpathlineto{\pgfqpoint{1.497875in}{2.300531in}}%
\pgfpathclose%
\pgfusepath{stroke}%
\end{pgfscope}%
\begin{pgfscope}%
\pgfpathrectangle{\pgfqpoint{0.847223in}{0.554012in}}{\pgfqpoint{6.200000in}{4.530000in}}%
\pgfusepath{clip}%
\pgfsetbuttcap%
\pgfsetroundjoin%
\pgfsetlinewidth{1.003750pt}%
\definecolor{currentstroke}{rgb}{1.000000,0.000000,0.000000}%
\pgfsetstrokecolor{currentstroke}%
\pgfsetdash{}{0pt}%
\pgfpathmoveto{\pgfqpoint{1.503209in}{2.290558in}}%
\pgfpathcurveto{\pgfqpoint{1.514259in}{2.290558in}}{\pgfqpoint{1.524858in}{2.294948in}}{\pgfqpoint{1.532671in}{2.302762in}}%
\pgfpathcurveto{\pgfqpoint{1.540485in}{2.310576in}}{\pgfqpoint{1.544875in}{2.321175in}}{\pgfqpoint{1.544875in}{2.332225in}}%
\pgfpathcurveto{\pgfqpoint{1.544875in}{2.343275in}}{\pgfqpoint{1.540485in}{2.353874in}}{\pgfqpoint{1.532671in}{2.361688in}}%
\pgfpathcurveto{\pgfqpoint{1.524858in}{2.369501in}}{\pgfqpoint{1.514259in}{2.373892in}}{\pgfqpoint{1.503209in}{2.373892in}}%
\pgfpathcurveto{\pgfqpoint{1.492158in}{2.373892in}}{\pgfqpoint{1.481559in}{2.369501in}}{\pgfqpoint{1.473746in}{2.361688in}}%
\pgfpathcurveto{\pgfqpoint{1.465932in}{2.353874in}}{\pgfqpoint{1.461542in}{2.343275in}}{\pgfqpoint{1.461542in}{2.332225in}}%
\pgfpathcurveto{\pgfqpoint{1.461542in}{2.321175in}}{\pgfqpoint{1.465932in}{2.310576in}}{\pgfqpoint{1.473746in}{2.302762in}}%
\pgfpathcurveto{\pgfqpoint{1.481559in}{2.294948in}}{\pgfqpoint{1.492158in}{2.290558in}}{\pgfqpoint{1.503209in}{2.290558in}}%
\pgfpathlineto{\pgfqpoint{1.503209in}{2.290558in}}%
\pgfpathclose%
\pgfusepath{stroke}%
\end{pgfscope}%
\begin{pgfscope}%
\pgfpathrectangle{\pgfqpoint{0.847223in}{0.554012in}}{\pgfqpoint{6.200000in}{4.530000in}}%
\pgfusepath{clip}%
\pgfsetbuttcap%
\pgfsetroundjoin%
\pgfsetlinewidth{1.003750pt}%
\definecolor{currentstroke}{rgb}{1.000000,0.000000,0.000000}%
\pgfsetstrokecolor{currentstroke}%
\pgfsetdash{}{0pt}%
\pgfpathmoveto{\pgfqpoint{1.508542in}{2.280675in}}%
\pgfpathcurveto{\pgfqpoint{1.519592in}{2.280675in}}{\pgfqpoint{1.530191in}{2.285065in}}{\pgfqpoint{1.538005in}{2.292879in}}%
\pgfpathcurveto{\pgfqpoint{1.545818in}{2.300692in}}{\pgfqpoint{1.550208in}{2.311291in}}{\pgfqpoint{1.550208in}{2.322342in}}%
\pgfpathcurveto{\pgfqpoint{1.550208in}{2.333392in}}{\pgfqpoint{1.545818in}{2.343991in}}{\pgfqpoint{1.538005in}{2.351804in}}%
\pgfpathcurveto{\pgfqpoint{1.530191in}{2.359618in}}{\pgfqpoint{1.519592in}{2.364008in}}{\pgfqpoint{1.508542in}{2.364008in}}%
\pgfpathcurveto{\pgfqpoint{1.497492in}{2.364008in}}{\pgfqpoint{1.486893in}{2.359618in}}{\pgfqpoint{1.479079in}{2.351804in}}%
\pgfpathcurveto{\pgfqpoint{1.471265in}{2.343991in}}{\pgfqpoint{1.466875in}{2.333392in}}{\pgfqpoint{1.466875in}{2.322342in}}%
\pgfpathcurveto{\pgfqpoint{1.466875in}{2.311291in}}{\pgfqpoint{1.471265in}{2.300692in}}{\pgfqpoint{1.479079in}{2.292879in}}%
\pgfpathcurveto{\pgfqpoint{1.486893in}{2.285065in}}{\pgfqpoint{1.497492in}{2.280675in}}{\pgfqpoint{1.508542in}{2.280675in}}%
\pgfpathlineto{\pgfqpoint{1.508542in}{2.280675in}}%
\pgfpathclose%
\pgfusepath{stroke}%
\end{pgfscope}%
\begin{pgfscope}%
\pgfpathrectangle{\pgfqpoint{0.847223in}{0.554012in}}{\pgfqpoint{6.200000in}{4.530000in}}%
\pgfusepath{clip}%
\pgfsetbuttcap%
\pgfsetroundjoin%
\pgfsetlinewidth{1.003750pt}%
\definecolor{currentstroke}{rgb}{1.000000,0.000000,0.000000}%
\pgfsetstrokecolor{currentstroke}%
\pgfsetdash{}{0pt}%
\pgfpathmoveto{\pgfqpoint{1.513875in}{2.270880in}}%
\pgfpathcurveto{\pgfqpoint{1.524925in}{2.270880in}}{\pgfqpoint{1.535524in}{2.275270in}}{\pgfqpoint{1.543338in}{2.283084in}}%
\pgfpathcurveto{\pgfqpoint{1.551151in}{2.290898in}}{\pgfqpoint{1.555542in}{2.301497in}}{\pgfqpoint{1.555542in}{2.312547in}}%
\pgfpathcurveto{\pgfqpoint{1.555542in}{2.323597in}}{\pgfqpoint{1.551151in}{2.334196in}}{\pgfqpoint{1.543338in}{2.342010in}}%
\pgfpathcurveto{\pgfqpoint{1.535524in}{2.349823in}}{\pgfqpoint{1.524925in}{2.354213in}}{\pgfqpoint{1.513875in}{2.354213in}}%
\pgfpathcurveto{\pgfqpoint{1.502825in}{2.354213in}}{\pgfqpoint{1.492226in}{2.349823in}}{\pgfqpoint{1.484412in}{2.342010in}}%
\pgfpathcurveto{\pgfqpoint{1.476599in}{2.334196in}}{\pgfqpoint{1.472208in}{2.323597in}}{\pgfqpoint{1.472208in}{2.312547in}}%
\pgfpathcurveto{\pgfqpoint{1.472208in}{2.301497in}}{\pgfqpoint{1.476599in}{2.290898in}}{\pgfqpoint{1.484412in}{2.283084in}}%
\pgfpathcurveto{\pgfqpoint{1.492226in}{2.275270in}}{\pgfqpoint{1.502825in}{2.270880in}}{\pgfqpoint{1.513875in}{2.270880in}}%
\pgfpathlineto{\pgfqpoint{1.513875in}{2.270880in}}%
\pgfpathclose%
\pgfusepath{stroke}%
\end{pgfscope}%
\begin{pgfscope}%
\pgfpathrectangle{\pgfqpoint{0.847223in}{0.554012in}}{\pgfqpoint{6.200000in}{4.530000in}}%
\pgfusepath{clip}%
\pgfsetbuttcap%
\pgfsetroundjoin%
\pgfsetlinewidth{1.003750pt}%
\definecolor{currentstroke}{rgb}{1.000000,0.000000,0.000000}%
\pgfsetstrokecolor{currentstroke}%
\pgfsetdash{}{0pt}%
\pgfpathmoveto{\pgfqpoint{1.519208in}{2.261173in}}%
\pgfpathcurveto{\pgfqpoint{1.530258in}{2.261173in}}{\pgfqpoint{1.540857in}{2.265563in}}{\pgfqpoint{1.548671in}{2.273377in}}%
\pgfpathcurveto{\pgfqpoint{1.556485in}{2.281190in}}{\pgfqpoint{1.560875in}{2.291789in}}{\pgfqpoint{1.560875in}{2.302839in}}%
\pgfpathcurveto{\pgfqpoint{1.560875in}{2.313889in}}{\pgfqpoint{1.556485in}{2.324489in}}{\pgfqpoint{1.548671in}{2.332302in}}%
\pgfpathcurveto{\pgfqpoint{1.540857in}{2.340116in}}{\pgfqpoint{1.530258in}{2.344506in}}{\pgfqpoint{1.519208in}{2.344506in}}%
\pgfpathcurveto{\pgfqpoint{1.508158in}{2.344506in}}{\pgfqpoint{1.497559in}{2.340116in}}{\pgfqpoint{1.489745in}{2.332302in}}%
\pgfpathcurveto{\pgfqpoint{1.481932in}{2.324489in}}{\pgfqpoint{1.477542in}{2.313889in}}{\pgfqpoint{1.477542in}{2.302839in}}%
\pgfpathcurveto{\pgfqpoint{1.477542in}{2.291789in}}{\pgfqpoint{1.481932in}{2.281190in}}{\pgfqpoint{1.489745in}{2.273377in}}%
\pgfpathcurveto{\pgfqpoint{1.497559in}{2.265563in}}{\pgfqpoint{1.508158in}{2.261173in}}{\pgfqpoint{1.519208in}{2.261173in}}%
\pgfpathlineto{\pgfqpoint{1.519208in}{2.261173in}}%
\pgfpathclose%
\pgfusepath{stroke}%
\end{pgfscope}%
\begin{pgfscope}%
\pgfpathrectangle{\pgfqpoint{0.847223in}{0.554012in}}{\pgfqpoint{6.200000in}{4.530000in}}%
\pgfusepath{clip}%
\pgfsetbuttcap%
\pgfsetroundjoin%
\pgfsetlinewidth{1.003750pt}%
\definecolor{currentstroke}{rgb}{1.000000,0.000000,0.000000}%
\pgfsetstrokecolor{currentstroke}%
\pgfsetdash{}{0pt}%
\pgfpathmoveto{\pgfqpoint{1.524541in}{2.251552in}}%
\pgfpathcurveto{\pgfqpoint{1.535592in}{2.251552in}}{\pgfqpoint{1.546191in}{2.255942in}}{\pgfqpoint{1.554004in}{2.263755in}}%
\pgfpathcurveto{\pgfqpoint{1.561818in}{2.271569in}}{\pgfqpoint{1.566208in}{2.282168in}}{\pgfqpoint{1.566208in}{2.293218in}}%
\pgfpathcurveto{\pgfqpoint{1.566208in}{2.304268in}}{\pgfqpoint{1.561818in}{2.314867in}}{\pgfqpoint{1.554004in}{2.322681in}}%
\pgfpathcurveto{\pgfqpoint{1.546191in}{2.330495in}}{\pgfqpoint{1.535592in}{2.334885in}}{\pgfqpoint{1.524541in}{2.334885in}}%
\pgfpathcurveto{\pgfqpoint{1.513491in}{2.334885in}}{\pgfqpoint{1.502892in}{2.330495in}}{\pgfqpoint{1.495079in}{2.322681in}}%
\pgfpathcurveto{\pgfqpoint{1.487265in}{2.314867in}}{\pgfqpoint{1.482875in}{2.304268in}}{\pgfqpoint{1.482875in}{2.293218in}}%
\pgfpathcurveto{\pgfqpoint{1.482875in}{2.282168in}}{\pgfqpoint{1.487265in}{2.271569in}}{\pgfqpoint{1.495079in}{2.263755in}}%
\pgfpathcurveto{\pgfqpoint{1.502892in}{2.255942in}}{\pgfqpoint{1.513491in}{2.251552in}}{\pgfqpoint{1.524541in}{2.251552in}}%
\pgfpathlineto{\pgfqpoint{1.524541in}{2.251552in}}%
\pgfpathclose%
\pgfusepath{stroke}%
\end{pgfscope}%
\begin{pgfscope}%
\pgfpathrectangle{\pgfqpoint{0.847223in}{0.554012in}}{\pgfqpoint{6.200000in}{4.530000in}}%
\pgfusepath{clip}%
\pgfsetbuttcap%
\pgfsetroundjoin%
\pgfsetlinewidth{1.003750pt}%
\definecolor{currentstroke}{rgb}{1.000000,0.000000,0.000000}%
\pgfsetstrokecolor{currentstroke}%
\pgfsetdash{}{0pt}%
\pgfpathmoveto{\pgfqpoint{1.529875in}{2.242015in}}%
\pgfpathcurveto{\pgfqpoint{1.540925in}{2.242015in}}{\pgfqpoint{1.551524in}{2.246406in}}{\pgfqpoint{1.559337in}{2.254219in}}%
\pgfpathcurveto{\pgfqpoint{1.567151in}{2.262033in}}{\pgfqpoint{1.571541in}{2.272632in}}{\pgfqpoint{1.571541in}{2.283682in}}%
\pgfpathcurveto{\pgfqpoint{1.571541in}{2.294732in}}{\pgfqpoint{1.567151in}{2.305331in}}{\pgfqpoint{1.559337in}{2.313145in}}%
\pgfpathcurveto{\pgfqpoint{1.551524in}{2.320959in}}{\pgfqpoint{1.540925in}{2.325349in}}{\pgfqpoint{1.529875in}{2.325349in}}%
\pgfpathcurveto{\pgfqpoint{1.518825in}{2.325349in}}{\pgfqpoint{1.508225in}{2.320959in}}{\pgfqpoint{1.500412in}{2.313145in}}%
\pgfpathcurveto{\pgfqpoint{1.492598in}{2.305331in}}{\pgfqpoint{1.488208in}{2.294732in}}{\pgfqpoint{1.488208in}{2.283682in}}%
\pgfpathcurveto{\pgfqpoint{1.488208in}{2.272632in}}{\pgfqpoint{1.492598in}{2.262033in}}{\pgfqpoint{1.500412in}{2.254219in}}%
\pgfpathcurveto{\pgfqpoint{1.508225in}{2.246406in}}{\pgfqpoint{1.518825in}{2.242015in}}{\pgfqpoint{1.529875in}{2.242015in}}%
\pgfpathlineto{\pgfqpoint{1.529875in}{2.242015in}}%
\pgfpathclose%
\pgfusepath{stroke}%
\end{pgfscope}%
\begin{pgfscope}%
\pgfpathrectangle{\pgfqpoint{0.847223in}{0.554012in}}{\pgfqpoint{6.200000in}{4.530000in}}%
\pgfusepath{clip}%
\pgfsetbuttcap%
\pgfsetroundjoin%
\pgfsetlinewidth{1.003750pt}%
\definecolor{currentstroke}{rgb}{1.000000,0.000000,0.000000}%
\pgfsetstrokecolor{currentstroke}%
\pgfsetdash{}{0pt}%
\pgfpathmoveto{\pgfqpoint{1.535208in}{2.232563in}}%
\pgfpathcurveto{\pgfqpoint{1.546258in}{2.232563in}}{\pgfqpoint{1.556857in}{2.236954in}}{\pgfqpoint{1.564671in}{2.244767in}}%
\pgfpathcurveto{\pgfqpoint{1.572484in}{2.252581in}}{\pgfqpoint{1.576875in}{2.263180in}}{\pgfqpoint{1.576875in}{2.274230in}}%
\pgfpathcurveto{\pgfqpoint{1.576875in}{2.285280in}}{\pgfqpoint{1.572484in}{2.295879in}}{\pgfqpoint{1.564671in}{2.303693in}}%
\pgfpathcurveto{\pgfqpoint{1.556857in}{2.311506in}}{\pgfqpoint{1.546258in}{2.315897in}}{\pgfqpoint{1.535208in}{2.315897in}}%
\pgfpathcurveto{\pgfqpoint{1.524158in}{2.315897in}}{\pgfqpoint{1.513559in}{2.311506in}}{\pgfqpoint{1.505745in}{2.303693in}}%
\pgfpathcurveto{\pgfqpoint{1.497931in}{2.295879in}}{\pgfqpoint{1.493541in}{2.285280in}}{\pgfqpoint{1.493541in}{2.274230in}}%
\pgfpathcurveto{\pgfqpoint{1.493541in}{2.263180in}}{\pgfqpoint{1.497931in}{2.252581in}}{\pgfqpoint{1.505745in}{2.244767in}}%
\pgfpathcurveto{\pgfqpoint{1.513559in}{2.236954in}}{\pgfqpoint{1.524158in}{2.232563in}}{\pgfqpoint{1.535208in}{2.232563in}}%
\pgfpathlineto{\pgfqpoint{1.535208in}{2.232563in}}%
\pgfpathclose%
\pgfusepath{stroke}%
\end{pgfscope}%
\begin{pgfscope}%
\pgfpathrectangle{\pgfqpoint{0.847223in}{0.554012in}}{\pgfqpoint{6.200000in}{4.530000in}}%
\pgfusepath{clip}%
\pgfsetbuttcap%
\pgfsetroundjoin%
\pgfsetlinewidth{1.003750pt}%
\definecolor{currentstroke}{rgb}{1.000000,0.000000,0.000000}%
\pgfsetstrokecolor{currentstroke}%
\pgfsetdash{}{0pt}%
\pgfpathmoveto{\pgfqpoint{1.540541in}{2.223194in}}%
\pgfpathcurveto{\pgfqpoint{1.551591in}{2.223194in}}{\pgfqpoint{1.562190in}{2.227584in}}{\pgfqpoint{1.570004in}{2.235398in}}%
\pgfpathcurveto{\pgfqpoint{1.577817in}{2.243212in}}{\pgfqpoint{1.582208in}{2.253811in}}{\pgfqpoint{1.582208in}{2.264861in}}%
\pgfpathcurveto{\pgfqpoint{1.582208in}{2.275911in}}{\pgfqpoint{1.577817in}{2.286510in}}{\pgfqpoint{1.570004in}{2.294324in}}%
\pgfpathcurveto{\pgfqpoint{1.562190in}{2.302137in}}{\pgfqpoint{1.551591in}{2.306528in}}{\pgfqpoint{1.540541in}{2.306528in}}%
\pgfpathcurveto{\pgfqpoint{1.529491in}{2.306528in}}{\pgfqpoint{1.518892in}{2.302137in}}{\pgfqpoint{1.511078in}{2.294324in}}%
\pgfpathcurveto{\pgfqpoint{1.503265in}{2.286510in}}{\pgfqpoint{1.498874in}{2.275911in}}{\pgfqpoint{1.498874in}{2.264861in}}%
\pgfpathcurveto{\pgfqpoint{1.498874in}{2.253811in}}{\pgfqpoint{1.503265in}{2.243212in}}{\pgfqpoint{1.511078in}{2.235398in}}%
\pgfpathcurveto{\pgfqpoint{1.518892in}{2.227584in}}{\pgfqpoint{1.529491in}{2.223194in}}{\pgfqpoint{1.540541in}{2.223194in}}%
\pgfpathlineto{\pgfqpoint{1.540541in}{2.223194in}}%
\pgfpathclose%
\pgfusepath{stroke}%
\end{pgfscope}%
\begin{pgfscope}%
\pgfpathrectangle{\pgfqpoint{0.847223in}{0.554012in}}{\pgfqpoint{6.200000in}{4.530000in}}%
\pgfusepath{clip}%
\pgfsetbuttcap%
\pgfsetroundjoin%
\pgfsetlinewidth{1.003750pt}%
\definecolor{currentstroke}{rgb}{1.000000,0.000000,0.000000}%
\pgfsetstrokecolor{currentstroke}%
\pgfsetdash{}{0pt}%
\pgfpathmoveto{\pgfqpoint{1.545874in}{2.213907in}}%
\pgfpathcurveto{\pgfqpoint{1.556924in}{2.213907in}}{\pgfqpoint{1.567523in}{2.218297in}}{\pgfqpoint{1.575337in}{2.226111in}}%
\pgfpathcurveto{\pgfqpoint{1.583151in}{2.233924in}}{\pgfqpoint{1.587541in}{2.244523in}}{\pgfqpoint{1.587541in}{2.255573in}}%
\pgfpathcurveto{\pgfqpoint{1.587541in}{2.266624in}}{\pgfqpoint{1.583151in}{2.277223in}}{\pgfqpoint{1.575337in}{2.285036in}}%
\pgfpathcurveto{\pgfqpoint{1.567523in}{2.292850in}}{\pgfqpoint{1.556924in}{2.297240in}}{\pgfqpoint{1.545874in}{2.297240in}}%
\pgfpathcurveto{\pgfqpoint{1.534824in}{2.297240in}}{\pgfqpoint{1.524225in}{2.292850in}}{\pgfqpoint{1.516412in}{2.285036in}}%
\pgfpathcurveto{\pgfqpoint{1.508598in}{2.277223in}}{\pgfqpoint{1.504208in}{2.266624in}}{\pgfqpoint{1.504208in}{2.255573in}}%
\pgfpathcurveto{\pgfqpoint{1.504208in}{2.244523in}}{\pgfqpoint{1.508598in}{2.233924in}}{\pgfqpoint{1.516412in}{2.226111in}}%
\pgfpathcurveto{\pgfqpoint{1.524225in}{2.218297in}}{\pgfqpoint{1.534824in}{2.213907in}}{\pgfqpoint{1.545874in}{2.213907in}}%
\pgfpathlineto{\pgfqpoint{1.545874in}{2.213907in}}%
\pgfpathclose%
\pgfusepath{stroke}%
\end{pgfscope}%
\begin{pgfscope}%
\pgfpathrectangle{\pgfqpoint{0.847223in}{0.554012in}}{\pgfqpoint{6.200000in}{4.530000in}}%
\pgfusepath{clip}%
\pgfsetbuttcap%
\pgfsetroundjoin%
\pgfsetlinewidth{1.003750pt}%
\definecolor{currentstroke}{rgb}{1.000000,0.000000,0.000000}%
\pgfsetstrokecolor{currentstroke}%
\pgfsetdash{}{0pt}%
\pgfpathmoveto{\pgfqpoint{1.551208in}{2.204700in}}%
\pgfpathcurveto{\pgfqpoint{1.562258in}{2.204700in}}{\pgfqpoint{1.572857in}{2.209090in}}{\pgfqpoint{1.580670in}{2.216904in}}%
\pgfpathcurveto{\pgfqpoint{1.588484in}{2.224718in}}{\pgfqpoint{1.592874in}{2.235317in}}{\pgfqpoint{1.592874in}{2.246367in}}%
\pgfpathcurveto{\pgfqpoint{1.592874in}{2.257417in}}{\pgfqpoint{1.588484in}{2.268016in}}{\pgfqpoint{1.580670in}{2.275829in}}%
\pgfpathcurveto{\pgfqpoint{1.572857in}{2.283643in}}{\pgfqpoint{1.562258in}{2.288033in}}{\pgfqpoint{1.551208in}{2.288033in}}%
\pgfpathcurveto{\pgfqpoint{1.540157in}{2.288033in}}{\pgfqpoint{1.529558in}{2.283643in}}{\pgfqpoint{1.521745in}{2.275829in}}%
\pgfpathcurveto{\pgfqpoint{1.513931in}{2.268016in}}{\pgfqpoint{1.509541in}{2.257417in}}{\pgfqpoint{1.509541in}{2.246367in}}%
\pgfpathcurveto{\pgfqpoint{1.509541in}{2.235317in}}{\pgfqpoint{1.513931in}{2.224718in}}{\pgfqpoint{1.521745in}{2.216904in}}%
\pgfpathcurveto{\pgfqpoint{1.529558in}{2.209090in}}{\pgfqpoint{1.540157in}{2.204700in}}{\pgfqpoint{1.551208in}{2.204700in}}%
\pgfpathlineto{\pgfqpoint{1.551208in}{2.204700in}}%
\pgfpathclose%
\pgfusepath{stroke}%
\end{pgfscope}%
\begin{pgfscope}%
\pgfpathrectangle{\pgfqpoint{0.847223in}{0.554012in}}{\pgfqpoint{6.200000in}{4.530000in}}%
\pgfusepath{clip}%
\pgfsetbuttcap%
\pgfsetroundjoin%
\pgfsetlinewidth{1.003750pt}%
\definecolor{currentstroke}{rgb}{1.000000,0.000000,0.000000}%
\pgfsetstrokecolor{currentstroke}%
\pgfsetdash{}{0pt}%
\pgfpathmoveto{\pgfqpoint{1.556541in}{2.195573in}}%
\pgfpathcurveto{\pgfqpoint{1.567591in}{2.195573in}}{\pgfqpoint{1.578190in}{2.199963in}}{\pgfqpoint{1.586004in}{2.207777in}}%
\pgfpathcurveto{\pgfqpoint{1.593817in}{2.215590in}}{\pgfqpoint{1.598207in}{2.226190in}}{\pgfqpoint{1.598207in}{2.237240in}}%
\pgfpathcurveto{\pgfqpoint{1.598207in}{2.248290in}}{\pgfqpoint{1.593817in}{2.258889in}}{\pgfqpoint{1.586004in}{2.266702in}}%
\pgfpathcurveto{\pgfqpoint{1.578190in}{2.274516in}}{\pgfqpoint{1.567591in}{2.278906in}}{\pgfqpoint{1.556541in}{2.278906in}}%
\pgfpathcurveto{\pgfqpoint{1.545491in}{2.278906in}}{\pgfqpoint{1.534892in}{2.274516in}}{\pgfqpoint{1.527078in}{2.266702in}}%
\pgfpathcurveto{\pgfqpoint{1.519264in}{2.258889in}}{\pgfqpoint{1.514874in}{2.248290in}}{\pgfqpoint{1.514874in}{2.237240in}}%
\pgfpathcurveto{\pgfqpoint{1.514874in}{2.226190in}}{\pgfqpoint{1.519264in}{2.215590in}}{\pgfqpoint{1.527078in}{2.207777in}}%
\pgfpathcurveto{\pgfqpoint{1.534892in}{2.199963in}}{\pgfqpoint{1.545491in}{2.195573in}}{\pgfqpoint{1.556541in}{2.195573in}}%
\pgfpathlineto{\pgfqpoint{1.556541in}{2.195573in}}%
\pgfpathclose%
\pgfusepath{stroke}%
\end{pgfscope}%
\begin{pgfscope}%
\pgfpathrectangle{\pgfqpoint{0.847223in}{0.554012in}}{\pgfqpoint{6.200000in}{4.530000in}}%
\pgfusepath{clip}%
\pgfsetbuttcap%
\pgfsetroundjoin%
\pgfsetlinewidth{1.003750pt}%
\definecolor{currentstroke}{rgb}{1.000000,0.000000,0.000000}%
\pgfsetstrokecolor{currentstroke}%
\pgfsetdash{}{0pt}%
\pgfpathmoveto{\pgfqpoint{1.561874in}{2.186525in}}%
\pgfpathcurveto{\pgfqpoint{1.572924in}{2.186525in}}{\pgfqpoint{1.583523in}{2.190915in}}{\pgfqpoint{1.591337in}{2.198728in}}%
\pgfpathcurveto{\pgfqpoint{1.599150in}{2.206542in}}{\pgfqpoint{1.603541in}{2.217141in}}{\pgfqpoint{1.603541in}{2.228191in}}%
\pgfpathcurveto{\pgfqpoint{1.603541in}{2.239241in}}{\pgfqpoint{1.599150in}{2.249840in}}{\pgfqpoint{1.591337in}{2.257654in}}%
\pgfpathcurveto{\pgfqpoint{1.583523in}{2.265468in}}{\pgfqpoint{1.572924in}{2.269858in}}{\pgfqpoint{1.561874in}{2.269858in}}%
\pgfpathcurveto{\pgfqpoint{1.550824in}{2.269858in}}{\pgfqpoint{1.540225in}{2.265468in}}{\pgfqpoint{1.532411in}{2.257654in}}%
\pgfpathcurveto{\pgfqpoint{1.524598in}{2.249840in}}{\pgfqpoint{1.520207in}{2.239241in}}{\pgfqpoint{1.520207in}{2.228191in}}%
\pgfpathcurveto{\pgfqpoint{1.520207in}{2.217141in}}{\pgfqpoint{1.524598in}{2.206542in}}{\pgfqpoint{1.532411in}{2.198728in}}%
\pgfpathcurveto{\pgfqpoint{1.540225in}{2.190915in}}{\pgfqpoint{1.550824in}{2.186525in}}{\pgfqpoint{1.561874in}{2.186525in}}%
\pgfpathlineto{\pgfqpoint{1.561874in}{2.186525in}}%
\pgfpathclose%
\pgfusepath{stroke}%
\end{pgfscope}%
\begin{pgfscope}%
\pgfpathrectangle{\pgfqpoint{0.847223in}{0.554012in}}{\pgfqpoint{6.200000in}{4.530000in}}%
\pgfusepath{clip}%
\pgfsetbuttcap%
\pgfsetroundjoin%
\pgfsetlinewidth{1.003750pt}%
\definecolor{currentstroke}{rgb}{1.000000,0.000000,0.000000}%
\pgfsetstrokecolor{currentstroke}%
\pgfsetdash{}{0pt}%
\pgfpathmoveto{\pgfqpoint{1.567207in}{2.177554in}}%
\pgfpathcurveto{\pgfqpoint{1.578257in}{2.177554in}}{\pgfqpoint{1.588856in}{2.181944in}}{\pgfqpoint{1.596670in}{2.189758in}}%
\pgfpathcurveto{\pgfqpoint{1.604484in}{2.197571in}}{\pgfqpoint{1.608874in}{2.208170in}}{\pgfqpoint{1.608874in}{2.219220in}}%
\pgfpathcurveto{\pgfqpoint{1.608874in}{2.230271in}}{\pgfqpoint{1.604484in}{2.240870in}}{\pgfqpoint{1.596670in}{2.248683in}}%
\pgfpathcurveto{\pgfqpoint{1.588856in}{2.256497in}}{\pgfqpoint{1.578257in}{2.260887in}}{\pgfqpoint{1.567207in}{2.260887in}}%
\pgfpathcurveto{\pgfqpoint{1.556157in}{2.260887in}}{\pgfqpoint{1.545558in}{2.256497in}}{\pgfqpoint{1.537744in}{2.248683in}}%
\pgfpathcurveto{\pgfqpoint{1.529931in}{2.240870in}}{\pgfqpoint{1.525541in}{2.230271in}}{\pgfqpoint{1.525541in}{2.219220in}}%
\pgfpathcurveto{\pgfqpoint{1.525541in}{2.208170in}}{\pgfqpoint{1.529931in}{2.197571in}}{\pgfqpoint{1.537744in}{2.189758in}}%
\pgfpathcurveto{\pgfqpoint{1.545558in}{2.181944in}}{\pgfqpoint{1.556157in}{2.177554in}}{\pgfqpoint{1.567207in}{2.177554in}}%
\pgfpathlineto{\pgfqpoint{1.567207in}{2.177554in}}%
\pgfpathclose%
\pgfusepath{stroke}%
\end{pgfscope}%
\begin{pgfscope}%
\pgfpathrectangle{\pgfqpoint{0.847223in}{0.554012in}}{\pgfqpoint{6.200000in}{4.530000in}}%
\pgfusepath{clip}%
\pgfsetbuttcap%
\pgfsetroundjoin%
\pgfsetlinewidth{1.003750pt}%
\definecolor{currentstroke}{rgb}{1.000000,0.000000,0.000000}%
\pgfsetstrokecolor{currentstroke}%
\pgfsetdash{}{0pt}%
\pgfpathmoveto{\pgfqpoint{1.572540in}{2.168660in}}%
\pgfpathcurveto{\pgfqpoint{1.583591in}{2.168660in}}{\pgfqpoint{1.594190in}{2.173050in}}{\pgfqpoint{1.602003in}{2.180864in}}%
\pgfpathcurveto{\pgfqpoint{1.609817in}{2.188677in}}{\pgfqpoint{1.614207in}{2.199276in}}{\pgfqpoint{1.614207in}{2.210326in}}%
\pgfpathcurveto{\pgfqpoint{1.614207in}{2.221376in}}{\pgfqpoint{1.609817in}{2.231975in}}{\pgfqpoint{1.602003in}{2.239789in}}%
\pgfpathcurveto{\pgfqpoint{1.594190in}{2.247603in}}{\pgfqpoint{1.583591in}{2.251993in}}{\pgfqpoint{1.572540in}{2.251993in}}%
\pgfpathcurveto{\pgfqpoint{1.561490in}{2.251993in}}{\pgfqpoint{1.550891in}{2.247603in}}{\pgfqpoint{1.543078in}{2.239789in}}%
\pgfpathcurveto{\pgfqpoint{1.535264in}{2.231975in}}{\pgfqpoint{1.530874in}{2.221376in}}{\pgfqpoint{1.530874in}{2.210326in}}%
\pgfpathcurveto{\pgfqpoint{1.530874in}{2.199276in}}{\pgfqpoint{1.535264in}{2.188677in}}{\pgfqpoint{1.543078in}{2.180864in}}%
\pgfpathcurveto{\pgfqpoint{1.550891in}{2.173050in}}{\pgfqpoint{1.561490in}{2.168660in}}{\pgfqpoint{1.572540in}{2.168660in}}%
\pgfpathlineto{\pgfqpoint{1.572540in}{2.168660in}}%
\pgfpathclose%
\pgfusepath{stroke}%
\end{pgfscope}%
\begin{pgfscope}%
\pgfpathrectangle{\pgfqpoint{0.847223in}{0.554012in}}{\pgfqpoint{6.200000in}{4.530000in}}%
\pgfusepath{clip}%
\pgfsetbuttcap%
\pgfsetroundjoin%
\pgfsetlinewidth{1.003750pt}%
\definecolor{currentstroke}{rgb}{1.000000,0.000000,0.000000}%
\pgfsetstrokecolor{currentstroke}%
\pgfsetdash{}{0pt}%
\pgfpathmoveto{\pgfqpoint{1.577874in}{2.159841in}}%
\pgfpathcurveto{\pgfqpoint{1.588924in}{2.159841in}}{\pgfqpoint{1.599523in}{2.164231in}}{\pgfqpoint{1.607336in}{2.172045in}}%
\pgfpathcurveto{\pgfqpoint{1.615150in}{2.179859in}}{\pgfqpoint{1.619540in}{2.190458in}}{\pgfqpoint{1.619540in}{2.201508in}}%
\pgfpathcurveto{\pgfqpoint{1.619540in}{2.212558in}}{\pgfqpoint{1.615150in}{2.223157in}}{\pgfqpoint{1.607336in}{2.230971in}}%
\pgfpathcurveto{\pgfqpoint{1.599523in}{2.238784in}}{\pgfqpoint{1.588924in}{2.243174in}}{\pgfqpoint{1.577874in}{2.243174in}}%
\pgfpathcurveto{\pgfqpoint{1.566823in}{2.243174in}}{\pgfqpoint{1.556224in}{2.238784in}}{\pgfqpoint{1.548411in}{2.230971in}}%
\pgfpathcurveto{\pgfqpoint{1.540597in}{2.223157in}}{\pgfqpoint{1.536207in}{2.212558in}}{\pgfqpoint{1.536207in}{2.201508in}}%
\pgfpathcurveto{\pgfqpoint{1.536207in}{2.190458in}}{\pgfqpoint{1.540597in}{2.179859in}}{\pgfqpoint{1.548411in}{2.172045in}}%
\pgfpathcurveto{\pgfqpoint{1.556224in}{2.164231in}}{\pgfqpoint{1.566823in}{2.159841in}}{\pgfqpoint{1.577874in}{2.159841in}}%
\pgfpathlineto{\pgfqpoint{1.577874in}{2.159841in}}%
\pgfpathclose%
\pgfusepath{stroke}%
\end{pgfscope}%
\begin{pgfscope}%
\pgfpathrectangle{\pgfqpoint{0.847223in}{0.554012in}}{\pgfqpoint{6.200000in}{4.530000in}}%
\pgfusepath{clip}%
\pgfsetbuttcap%
\pgfsetroundjoin%
\pgfsetlinewidth{1.003750pt}%
\definecolor{currentstroke}{rgb}{1.000000,0.000000,0.000000}%
\pgfsetstrokecolor{currentstroke}%
\pgfsetdash{}{0pt}%
\pgfpathmoveto{\pgfqpoint{1.583207in}{2.151097in}}%
\pgfpathcurveto{\pgfqpoint{1.594257in}{2.151097in}}{\pgfqpoint{1.604856in}{2.155488in}}{\pgfqpoint{1.612670in}{2.163301in}}%
\pgfpathcurveto{\pgfqpoint{1.620483in}{2.171115in}}{\pgfqpoint{1.624873in}{2.181714in}}{\pgfqpoint{1.624873in}{2.192764in}}%
\pgfpathcurveto{\pgfqpoint{1.624873in}{2.203814in}}{\pgfqpoint{1.620483in}{2.214413in}}{\pgfqpoint{1.612670in}{2.222227in}}%
\pgfpathcurveto{\pgfqpoint{1.604856in}{2.230040in}}{\pgfqpoint{1.594257in}{2.234431in}}{\pgfqpoint{1.583207in}{2.234431in}}%
\pgfpathcurveto{\pgfqpoint{1.572157in}{2.234431in}}{\pgfqpoint{1.561558in}{2.230040in}}{\pgfqpoint{1.553744in}{2.222227in}}%
\pgfpathcurveto{\pgfqpoint{1.545930in}{2.214413in}}{\pgfqpoint{1.541540in}{2.203814in}}{\pgfqpoint{1.541540in}{2.192764in}}%
\pgfpathcurveto{\pgfqpoint{1.541540in}{2.181714in}}{\pgfqpoint{1.545930in}{2.171115in}}{\pgfqpoint{1.553744in}{2.163301in}}%
\pgfpathcurveto{\pgfqpoint{1.561558in}{2.155488in}}{\pgfqpoint{1.572157in}{2.151097in}}{\pgfqpoint{1.583207in}{2.151097in}}%
\pgfpathlineto{\pgfqpoint{1.583207in}{2.151097in}}%
\pgfpathclose%
\pgfusepath{stroke}%
\end{pgfscope}%
\begin{pgfscope}%
\pgfpathrectangle{\pgfqpoint{0.847223in}{0.554012in}}{\pgfqpoint{6.200000in}{4.530000in}}%
\pgfusepath{clip}%
\pgfsetbuttcap%
\pgfsetroundjoin%
\pgfsetlinewidth{1.003750pt}%
\definecolor{currentstroke}{rgb}{1.000000,0.000000,0.000000}%
\pgfsetstrokecolor{currentstroke}%
\pgfsetdash{}{0pt}%
\pgfpathmoveto{\pgfqpoint{1.588540in}{2.142427in}}%
\pgfpathcurveto{\pgfqpoint{1.599590in}{2.142427in}}{\pgfqpoint{1.610189in}{2.146818in}}{\pgfqpoint{1.618003in}{2.154631in}}%
\pgfpathcurveto{\pgfqpoint{1.625816in}{2.162445in}}{\pgfqpoint{1.630207in}{2.173044in}}{\pgfqpoint{1.630207in}{2.184094in}}%
\pgfpathcurveto{\pgfqpoint{1.630207in}{2.195144in}}{\pgfqpoint{1.625816in}{2.205743in}}{\pgfqpoint{1.618003in}{2.213557in}}%
\pgfpathcurveto{\pgfqpoint{1.610189in}{2.221370in}}{\pgfqpoint{1.599590in}{2.225761in}}{\pgfqpoint{1.588540in}{2.225761in}}%
\pgfpathcurveto{\pgfqpoint{1.577490in}{2.225761in}}{\pgfqpoint{1.566891in}{2.221370in}}{\pgfqpoint{1.559077in}{2.213557in}}%
\pgfpathcurveto{\pgfqpoint{1.551264in}{2.205743in}}{\pgfqpoint{1.546873in}{2.195144in}}{\pgfqpoint{1.546873in}{2.184094in}}%
\pgfpathcurveto{\pgfqpoint{1.546873in}{2.173044in}}{\pgfqpoint{1.551264in}{2.162445in}}{\pgfqpoint{1.559077in}{2.154631in}}%
\pgfpathcurveto{\pgfqpoint{1.566891in}{2.146818in}}{\pgfqpoint{1.577490in}{2.142427in}}{\pgfqpoint{1.588540in}{2.142427in}}%
\pgfpathlineto{\pgfqpoint{1.588540in}{2.142427in}}%
\pgfpathclose%
\pgfusepath{stroke}%
\end{pgfscope}%
\begin{pgfscope}%
\pgfpathrectangle{\pgfqpoint{0.847223in}{0.554012in}}{\pgfqpoint{6.200000in}{4.530000in}}%
\pgfusepath{clip}%
\pgfsetbuttcap%
\pgfsetroundjoin%
\pgfsetlinewidth{1.003750pt}%
\definecolor{currentstroke}{rgb}{1.000000,0.000000,0.000000}%
\pgfsetstrokecolor{currentstroke}%
\pgfsetdash{}{0pt}%
\pgfpathmoveto{\pgfqpoint{1.593873in}{2.133830in}}%
\pgfpathcurveto{\pgfqpoint{1.604923in}{2.133830in}}{\pgfqpoint{1.615522in}{2.138220in}}{\pgfqpoint{1.623336in}{2.146034in}}%
\pgfpathcurveto{\pgfqpoint{1.631150in}{2.153848in}}{\pgfqpoint{1.635540in}{2.164447in}}{\pgfqpoint{1.635540in}{2.175497in}}%
\pgfpathcurveto{\pgfqpoint{1.635540in}{2.186547in}}{\pgfqpoint{1.631150in}{2.197146in}}{\pgfqpoint{1.623336in}{2.204959in}}%
\pgfpathcurveto{\pgfqpoint{1.615522in}{2.212773in}}{\pgfqpoint{1.604923in}{2.217163in}}{\pgfqpoint{1.593873in}{2.217163in}}%
\pgfpathcurveto{\pgfqpoint{1.582823in}{2.217163in}}{\pgfqpoint{1.572224in}{2.212773in}}{\pgfqpoint{1.564410in}{2.204959in}}%
\pgfpathcurveto{\pgfqpoint{1.556597in}{2.197146in}}{\pgfqpoint{1.552207in}{2.186547in}}{\pgfqpoint{1.552207in}{2.175497in}}%
\pgfpathcurveto{\pgfqpoint{1.552207in}{2.164447in}}{\pgfqpoint{1.556597in}{2.153848in}}{\pgfqpoint{1.564410in}{2.146034in}}%
\pgfpathcurveto{\pgfqpoint{1.572224in}{2.138220in}}{\pgfqpoint{1.582823in}{2.133830in}}{\pgfqpoint{1.593873in}{2.133830in}}%
\pgfpathlineto{\pgfqpoint{1.593873in}{2.133830in}}%
\pgfpathclose%
\pgfusepath{stroke}%
\end{pgfscope}%
\begin{pgfscope}%
\pgfpathrectangle{\pgfqpoint{0.847223in}{0.554012in}}{\pgfqpoint{6.200000in}{4.530000in}}%
\pgfusepath{clip}%
\pgfsetbuttcap%
\pgfsetroundjoin%
\pgfsetlinewidth{1.003750pt}%
\definecolor{currentstroke}{rgb}{1.000000,0.000000,0.000000}%
\pgfsetstrokecolor{currentstroke}%
\pgfsetdash{}{0pt}%
\pgfpathmoveto{\pgfqpoint{1.599206in}{2.125305in}}%
\pgfpathcurveto{\pgfqpoint{1.610257in}{2.125305in}}{\pgfqpoint{1.620856in}{2.129695in}}{\pgfqpoint{1.628669in}{2.137509in}}%
\pgfpathcurveto{\pgfqpoint{1.636483in}{2.145322in}}{\pgfqpoint{1.640873in}{2.155921in}}{\pgfqpoint{1.640873in}{2.166971in}}%
\pgfpathcurveto{\pgfqpoint{1.640873in}{2.178021in}}{\pgfqpoint{1.636483in}{2.188620in}}{\pgfqpoint{1.628669in}{2.196434in}}%
\pgfpathcurveto{\pgfqpoint{1.620856in}{2.204248in}}{\pgfqpoint{1.610257in}{2.208638in}}{\pgfqpoint{1.599206in}{2.208638in}}%
\pgfpathcurveto{\pgfqpoint{1.588156in}{2.208638in}}{\pgfqpoint{1.577557in}{2.204248in}}{\pgfqpoint{1.569744in}{2.196434in}}%
\pgfpathcurveto{\pgfqpoint{1.561930in}{2.188620in}}{\pgfqpoint{1.557540in}{2.178021in}}{\pgfqpoint{1.557540in}{2.166971in}}%
\pgfpathcurveto{\pgfqpoint{1.557540in}{2.155921in}}{\pgfqpoint{1.561930in}{2.145322in}}{\pgfqpoint{1.569744in}{2.137509in}}%
\pgfpathcurveto{\pgfqpoint{1.577557in}{2.129695in}}{\pgfqpoint{1.588156in}{2.125305in}}{\pgfqpoint{1.599206in}{2.125305in}}%
\pgfpathlineto{\pgfqpoint{1.599206in}{2.125305in}}%
\pgfpathclose%
\pgfusepath{stroke}%
\end{pgfscope}%
\begin{pgfscope}%
\pgfpathrectangle{\pgfqpoint{0.847223in}{0.554012in}}{\pgfqpoint{6.200000in}{4.530000in}}%
\pgfusepath{clip}%
\pgfsetbuttcap%
\pgfsetroundjoin%
\pgfsetlinewidth{1.003750pt}%
\definecolor{currentstroke}{rgb}{1.000000,0.000000,0.000000}%
\pgfsetstrokecolor{currentstroke}%
\pgfsetdash{}{0pt}%
\pgfpathmoveto{\pgfqpoint{1.604540in}{2.116850in}}%
\pgfpathcurveto{\pgfqpoint{1.615590in}{2.116850in}}{\pgfqpoint{1.626189in}{2.121241in}}{\pgfqpoint{1.634002in}{2.129054in}}%
\pgfpathcurveto{\pgfqpoint{1.641816in}{2.136868in}}{\pgfqpoint{1.646206in}{2.147467in}}{\pgfqpoint{1.646206in}{2.158517in}}%
\pgfpathcurveto{\pgfqpoint{1.646206in}{2.169567in}}{\pgfqpoint{1.641816in}{2.180166in}}{\pgfqpoint{1.634002in}{2.187980in}}%
\pgfpathcurveto{\pgfqpoint{1.626189in}{2.195793in}}{\pgfqpoint{1.615590in}{2.200184in}}{\pgfqpoint{1.604540in}{2.200184in}}%
\pgfpathcurveto{\pgfqpoint{1.593490in}{2.200184in}}{\pgfqpoint{1.582891in}{2.195793in}}{\pgfqpoint{1.575077in}{2.187980in}}%
\pgfpathcurveto{\pgfqpoint{1.567263in}{2.180166in}}{\pgfqpoint{1.562873in}{2.169567in}}{\pgfqpoint{1.562873in}{2.158517in}}%
\pgfpathcurveto{\pgfqpoint{1.562873in}{2.147467in}}{\pgfqpoint{1.567263in}{2.136868in}}{\pgfqpoint{1.575077in}{2.129054in}}%
\pgfpathcurveto{\pgfqpoint{1.582891in}{2.121241in}}{\pgfqpoint{1.593490in}{2.116850in}}{\pgfqpoint{1.604540in}{2.116850in}}%
\pgfpathlineto{\pgfqpoint{1.604540in}{2.116850in}}%
\pgfpathclose%
\pgfusepath{stroke}%
\end{pgfscope}%
\begin{pgfscope}%
\pgfpathrectangle{\pgfqpoint{0.847223in}{0.554012in}}{\pgfqpoint{6.200000in}{4.530000in}}%
\pgfusepath{clip}%
\pgfsetbuttcap%
\pgfsetroundjoin%
\pgfsetlinewidth{1.003750pt}%
\definecolor{currentstroke}{rgb}{1.000000,0.000000,0.000000}%
\pgfsetstrokecolor{currentstroke}%
\pgfsetdash{}{0pt}%
\pgfpathmoveto{\pgfqpoint{1.609873in}{2.108466in}}%
\pgfpathcurveto{\pgfqpoint{1.620923in}{2.108466in}}{\pgfqpoint{1.631522in}{2.112856in}}{\pgfqpoint{1.639336in}{2.120670in}}%
\pgfpathcurveto{\pgfqpoint{1.647149in}{2.128484in}}{\pgfqpoint{1.651540in}{2.139083in}}{\pgfqpoint{1.651540in}{2.150133in}}%
\pgfpathcurveto{\pgfqpoint{1.651540in}{2.161183in}}{\pgfqpoint{1.647149in}{2.171782in}}{\pgfqpoint{1.639336in}{2.179596in}}%
\pgfpathcurveto{\pgfqpoint{1.631522in}{2.187409in}}{\pgfqpoint{1.620923in}{2.191799in}}{\pgfqpoint{1.609873in}{2.191799in}}%
\pgfpathcurveto{\pgfqpoint{1.598823in}{2.191799in}}{\pgfqpoint{1.588224in}{2.187409in}}{\pgfqpoint{1.580410in}{2.179596in}}%
\pgfpathcurveto{\pgfqpoint{1.572596in}{2.171782in}}{\pgfqpoint{1.568206in}{2.161183in}}{\pgfqpoint{1.568206in}{2.150133in}}%
\pgfpathcurveto{\pgfqpoint{1.568206in}{2.139083in}}{\pgfqpoint{1.572596in}{2.128484in}}{\pgfqpoint{1.580410in}{2.120670in}}%
\pgfpathcurveto{\pgfqpoint{1.588224in}{2.112856in}}{\pgfqpoint{1.598823in}{2.108466in}}{\pgfqpoint{1.609873in}{2.108466in}}%
\pgfpathlineto{\pgfqpoint{1.609873in}{2.108466in}}%
\pgfpathclose%
\pgfusepath{stroke}%
\end{pgfscope}%
\begin{pgfscope}%
\pgfpathrectangle{\pgfqpoint{0.847223in}{0.554012in}}{\pgfqpoint{6.200000in}{4.530000in}}%
\pgfusepath{clip}%
\pgfsetbuttcap%
\pgfsetroundjoin%
\pgfsetlinewidth{1.003750pt}%
\definecolor{currentstroke}{rgb}{1.000000,0.000000,0.000000}%
\pgfsetstrokecolor{currentstroke}%
\pgfsetdash{}{0pt}%
\pgfpathmoveto{\pgfqpoint{1.615206in}{2.100151in}}%
\pgfpathcurveto{\pgfqpoint{1.626256in}{2.100151in}}{\pgfqpoint{1.636855in}{2.104541in}}{\pgfqpoint{1.644669in}{2.112355in}}%
\pgfpathcurveto{\pgfqpoint{1.652483in}{2.120169in}}{\pgfqpoint{1.656873in}{2.130768in}}{\pgfqpoint{1.656873in}{2.141818in}}%
\pgfpathcurveto{\pgfqpoint{1.656873in}{2.152868in}}{\pgfqpoint{1.652483in}{2.163467in}}{\pgfqpoint{1.644669in}{2.171281in}}%
\pgfpathcurveto{\pgfqpoint{1.636855in}{2.179094in}}{\pgfqpoint{1.626256in}{2.183484in}}{\pgfqpoint{1.615206in}{2.183484in}}%
\pgfpathcurveto{\pgfqpoint{1.604156in}{2.183484in}}{\pgfqpoint{1.593557in}{2.179094in}}{\pgfqpoint{1.585743in}{2.171281in}}%
\pgfpathcurveto{\pgfqpoint{1.577930in}{2.163467in}}{\pgfqpoint{1.573539in}{2.152868in}}{\pgfqpoint{1.573539in}{2.141818in}}%
\pgfpathcurveto{\pgfqpoint{1.573539in}{2.130768in}}{\pgfqpoint{1.577930in}{2.120169in}}{\pgfqpoint{1.585743in}{2.112355in}}%
\pgfpathcurveto{\pgfqpoint{1.593557in}{2.104541in}}{\pgfqpoint{1.604156in}{2.100151in}}{\pgfqpoint{1.615206in}{2.100151in}}%
\pgfpathlineto{\pgfqpoint{1.615206in}{2.100151in}}%
\pgfpathclose%
\pgfusepath{stroke}%
\end{pgfscope}%
\begin{pgfscope}%
\pgfpathrectangle{\pgfqpoint{0.847223in}{0.554012in}}{\pgfqpoint{6.200000in}{4.530000in}}%
\pgfusepath{clip}%
\pgfsetbuttcap%
\pgfsetroundjoin%
\pgfsetlinewidth{1.003750pt}%
\definecolor{currentstroke}{rgb}{1.000000,0.000000,0.000000}%
\pgfsetstrokecolor{currentstroke}%
\pgfsetdash{}{0pt}%
\pgfpathmoveto{\pgfqpoint{1.620539in}{2.091905in}}%
\pgfpathcurveto{\pgfqpoint{1.631589in}{2.091905in}}{\pgfqpoint{1.642188in}{2.096295in}}{\pgfqpoint{1.650002in}{2.104108in}}%
\pgfpathcurveto{\pgfqpoint{1.657816in}{2.111922in}}{\pgfqpoint{1.662206in}{2.122521in}}{\pgfqpoint{1.662206in}{2.133571in}}%
\pgfpathcurveto{\pgfqpoint{1.662206in}{2.144621in}}{\pgfqpoint{1.657816in}{2.155220in}}{\pgfqpoint{1.650002in}{2.163034in}}%
\pgfpathcurveto{\pgfqpoint{1.642188in}{2.170848in}}{\pgfqpoint{1.631589in}{2.175238in}}{\pgfqpoint{1.620539in}{2.175238in}}%
\pgfpathcurveto{\pgfqpoint{1.609489in}{2.175238in}}{\pgfqpoint{1.598890in}{2.170848in}}{\pgfqpoint{1.591077in}{2.163034in}}%
\pgfpathcurveto{\pgfqpoint{1.583263in}{2.155220in}}{\pgfqpoint{1.578873in}{2.144621in}}{\pgfqpoint{1.578873in}{2.133571in}}%
\pgfpathcurveto{\pgfqpoint{1.578873in}{2.122521in}}{\pgfqpoint{1.583263in}{2.111922in}}{\pgfqpoint{1.591077in}{2.104108in}}%
\pgfpathcurveto{\pgfqpoint{1.598890in}{2.096295in}}{\pgfqpoint{1.609489in}{2.091905in}}{\pgfqpoint{1.620539in}{2.091905in}}%
\pgfpathlineto{\pgfqpoint{1.620539in}{2.091905in}}%
\pgfpathclose%
\pgfusepath{stroke}%
\end{pgfscope}%
\begin{pgfscope}%
\pgfpathrectangle{\pgfqpoint{0.847223in}{0.554012in}}{\pgfqpoint{6.200000in}{4.530000in}}%
\pgfusepath{clip}%
\pgfsetbuttcap%
\pgfsetroundjoin%
\pgfsetlinewidth{1.003750pt}%
\definecolor{currentstroke}{rgb}{1.000000,0.000000,0.000000}%
\pgfsetstrokecolor{currentstroke}%
\pgfsetdash{}{0pt}%
\pgfpathmoveto{\pgfqpoint{1.625873in}{2.083726in}}%
\pgfpathcurveto{\pgfqpoint{1.636923in}{2.083726in}}{\pgfqpoint{1.647522in}{2.088116in}}{\pgfqpoint{1.655335in}{2.095929in}}%
\pgfpathcurveto{\pgfqpoint{1.663149in}{2.103743in}}{\pgfqpoint{1.667539in}{2.114342in}}{\pgfqpoint{1.667539in}{2.125392in}}%
\pgfpathcurveto{\pgfqpoint{1.667539in}{2.136442in}}{\pgfqpoint{1.663149in}{2.147041in}}{\pgfqpoint{1.655335in}{2.154855in}}%
\pgfpathcurveto{\pgfqpoint{1.647522in}{2.162669in}}{\pgfqpoint{1.636923in}{2.167059in}}{\pgfqpoint{1.625873in}{2.167059in}}%
\pgfpathcurveto{\pgfqpoint{1.614822in}{2.167059in}}{\pgfqpoint{1.604223in}{2.162669in}}{\pgfqpoint{1.596410in}{2.154855in}}%
\pgfpathcurveto{\pgfqpoint{1.588596in}{2.147041in}}{\pgfqpoint{1.584206in}{2.136442in}}{\pgfqpoint{1.584206in}{2.125392in}}%
\pgfpathcurveto{\pgfqpoint{1.584206in}{2.114342in}}{\pgfqpoint{1.588596in}{2.103743in}}{\pgfqpoint{1.596410in}{2.095929in}}%
\pgfpathcurveto{\pgfqpoint{1.604223in}{2.088116in}}{\pgfqpoint{1.614822in}{2.083726in}}{\pgfqpoint{1.625873in}{2.083726in}}%
\pgfpathlineto{\pgfqpoint{1.625873in}{2.083726in}}%
\pgfpathclose%
\pgfusepath{stroke}%
\end{pgfscope}%
\begin{pgfscope}%
\pgfpathrectangle{\pgfqpoint{0.847223in}{0.554012in}}{\pgfqpoint{6.200000in}{4.530000in}}%
\pgfusepath{clip}%
\pgfsetbuttcap%
\pgfsetroundjoin%
\pgfsetlinewidth{1.003750pt}%
\definecolor{currentstroke}{rgb}{1.000000,0.000000,0.000000}%
\pgfsetstrokecolor{currentstroke}%
\pgfsetdash{}{0pt}%
\pgfpathmoveto{\pgfqpoint{1.631206in}{2.075613in}}%
\pgfpathcurveto{\pgfqpoint{1.642256in}{2.075613in}}{\pgfqpoint{1.652855in}{2.080004in}}{\pgfqpoint{1.660669in}{2.087817in}}%
\pgfpathcurveto{\pgfqpoint{1.668482in}{2.095631in}}{\pgfqpoint{1.672872in}{2.106230in}}{\pgfqpoint{1.672872in}{2.117280in}}%
\pgfpathcurveto{\pgfqpoint{1.672872in}{2.128330in}}{\pgfqpoint{1.668482in}{2.138929in}}{\pgfqpoint{1.660669in}{2.146743in}}%
\pgfpathcurveto{\pgfqpoint{1.652855in}{2.154556in}}{\pgfqpoint{1.642256in}{2.158947in}}{\pgfqpoint{1.631206in}{2.158947in}}%
\pgfpathcurveto{\pgfqpoint{1.620156in}{2.158947in}}{\pgfqpoint{1.609557in}{2.154556in}}{\pgfqpoint{1.601743in}{2.146743in}}%
\pgfpathcurveto{\pgfqpoint{1.593929in}{2.138929in}}{\pgfqpoint{1.589539in}{2.128330in}}{\pgfqpoint{1.589539in}{2.117280in}}%
\pgfpathcurveto{\pgfqpoint{1.589539in}{2.106230in}}{\pgfqpoint{1.593929in}{2.095631in}}{\pgfqpoint{1.601743in}{2.087817in}}%
\pgfpathcurveto{\pgfqpoint{1.609557in}{2.080004in}}{\pgfqpoint{1.620156in}{2.075613in}}{\pgfqpoint{1.631206in}{2.075613in}}%
\pgfpathlineto{\pgfqpoint{1.631206in}{2.075613in}}%
\pgfpathclose%
\pgfusepath{stroke}%
\end{pgfscope}%
\begin{pgfscope}%
\pgfpathrectangle{\pgfqpoint{0.847223in}{0.554012in}}{\pgfqpoint{6.200000in}{4.530000in}}%
\pgfusepath{clip}%
\pgfsetbuttcap%
\pgfsetroundjoin%
\pgfsetlinewidth{1.003750pt}%
\definecolor{currentstroke}{rgb}{1.000000,0.000000,0.000000}%
\pgfsetstrokecolor{currentstroke}%
\pgfsetdash{}{0pt}%
\pgfpathmoveto{\pgfqpoint{1.636539in}{2.067567in}}%
\pgfpathcurveto{\pgfqpoint{1.647589in}{2.067567in}}{\pgfqpoint{1.658188in}{2.071957in}}{\pgfqpoint{1.666002in}{2.079771in}}%
\pgfpathcurveto{\pgfqpoint{1.673815in}{2.087584in}}{\pgfqpoint{1.678206in}{2.098184in}}{\pgfqpoint{1.678206in}{2.109234in}}%
\pgfpathcurveto{\pgfqpoint{1.678206in}{2.120284in}}{\pgfqpoint{1.673815in}{2.130883in}}{\pgfqpoint{1.666002in}{2.138696in}}%
\pgfpathcurveto{\pgfqpoint{1.658188in}{2.146510in}}{\pgfqpoint{1.647589in}{2.150900in}}{\pgfqpoint{1.636539in}{2.150900in}}%
\pgfpathcurveto{\pgfqpoint{1.625489in}{2.150900in}}{\pgfqpoint{1.614890in}{2.146510in}}{\pgfqpoint{1.607076in}{2.138696in}}%
\pgfpathcurveto{\pgfqpoint{1.599263in}{2.130883in}}{\pgfqpoint{1.594872in}{2.120284in}}{\pgfqpoint{1.594872in}{2.109234in}}%
\pgfpathcurveto{\pgfqpoint{1.594872in}{2.098184in}}{\pgfqpoint{1.599263in}{2.087584in}}{\pgfqpoint{1.607076in}{2.079771in}}%
\pgfpathcurveto{\pgfqpoint{1.614890in}{2.071957in}}{\pgfqpoint{1.625489in}{2.067567in}}{\pgfqpoint{1.636539in}{2.067567in}}%
\pgfpathlineto{\pgfqpoint{1.636539in}{2.067567in}}%
\pgfpathclose%
\pgfusepath{stroke}%
\end{pgfscope}%
\begin{pgfscope}%
\pgfpathrectangle{\pgfqpoint{0.847223in}{0.554012in}}{\pgfqpoint{6.200000in}{4.530000in}}%
\pgfusepath{clip}%
\pgfsetbuttcap%
\pgfsetroundjoin%
\pgfsetlinewidth{1.003750pt}%
\definecolor{currentstroke}{rgb}{1.000000,0.000000,0.000000}%
\pgfsetstrokecolor{currentstroke}%
\pgfsetdash{}{0pt}%
\pgfpathmoveto{\pgfqpoint{1.641872in}{2.059586in}}%
\pgfpathcurveto{\pgfqpoint{1.652922in}{2.059586in}}{\pgfqpoint{1.663521in}{2.063976in}}{\pgfqpoint{1.671335in}{2.071790in}}%
\pgfpathcurveto{\pgfqpoint{1.679149in}{2.079603in}}{\pgfqpoint{1.683539in}{2.090202in}}{\pgfqpoint{1.683539in}{2.101252in}}%
\pgfpathcurveto{\pgfqpoint{1.683539in}{2.112303in}}{\pgfqpoint{1.679149in}{2.122902in}}{\pgfqpoint{1.671335in}{2.130715in}}%
\pgfpathcurveto{\pgfqpoint{1.663521in}{2.138529in}}{\pgfqpoint{1.652922in}{2.142919in}}{\pgfqpoint{1.641872in}{2.142919in}}%
\pgfpathcurveto{\pgfqpoint{1.630822in}{2.142919in}}{\pgfqpoint{1.620223in}{2.138529in}}{\pgfqpoint{1.612409in}{2.130715in}}%
\pgfpathcurveto{\pgfqpoint{1.604596in}{2.122902in}}{\pgfqpoint{1.600206in}{2.112303in}}{\pgfqpoint{1.600206in}{2.101252in}}%
\pgfpathcurveto{\pgfqpoint{1.600206in}{2.090202in}}{\pgfqpoint{1.604596in}{2.079603in}}{\pgfqpoint{1.612409in}{2.071790in}}%
\pgfpathcurveto{\pgfqpoint{1.620223in}{2.063976in}}{\pgfqpoint{1.630822in}{2.059586in}}{\pgfqpoint{1.641872in}{2.059586in}}%
\pgfpathlineto{\pgfqpoint{1.641872in}{2.059586in}}%
\pgfpathclose%
\pgfusepath{stroke}%
\end{pgfscope}%
\begin{pgfscope}%
\pgfpathrectangle{\pgfqpoint{0.847223in}{0.554012in}}{\pgfqpoint{6.200000in}{4.530000in}}%
\pgfusepath{clip}%
\pgfsetbuttcap%
\pgfsetroundjoin%
\pgfsetlinewidth{1.003750pt}%
\definecolor{currentstroke}{rgb}{1.000000,0.000000,0.000000}%
\pgfsetstrokecolor{currentstroke}%
\pgfsetdash{}{0pt}%
\pgfpathmoveto{\pgfqpoint{1.647205in}{2.051669in}}%
\pgfpathcurveto{\pgfqpoint{1.658256in}{2.051669in}}{\pgfqpoint{1.668855in}{2.056059in}}{\pgfqpoint{1.676668in}{2.063873in}}%
\pgfpathcurveto{\pgfqpoint{1.684482in}{2.071686in}}{\pgfqpoint{1.688872in}{2.082285in}}{\pgfqpoint{1.688872in}{2.093335in}}%
\pgfpathcurveto{\pgfqpoint{1.688872in}{2.104386in}}{\pgfqpoint{1.684482in}{2.114985in}}{\pgfqpoint{1.676668in}{2.122798in}}%
\pgfpathcurveto{\pgfqpoint{1.668855in}{2.130612in}}{\pgfqpoint{1.658256in}{2.135002in}}{\pgfqpoint{1.647205in}{2.135002in}}%
\pgfpathcurveto{\pgfqpoint{1.636155in}{2.135002in}}{\pgfqpoint{1.625556in}{2.130612in}}{\pgfqpoint{1.617743in}{2.122798in}}%
\pgfpathcurveto{\pgfqpoint{1.609929in}{2.114985in}}{\pgfqpoint{1.605539in}{2.104386in}}{\pgfqpoint{1.605539in}{2.093335in}}%
\pgfpathcurveto{\pgfqpoint{1.605539in}{2.082285in}}{\pgfqpoint{1.609929in}{2.071686in}}{\pgfqpoint{1.617743in}{2.063873in}}%
\pgfpathcurveto{\pgfqpoint{1.625556in}{2.056059in}}{\pgfqpoint{1.636155in}{2.051669in}}{\pgfqpoint{1.647205in}{2.051669in}}%
\pgfpathlineto{\pgfqpoint{1.647205in}{2.051669in}}%
\pgfpathclose%
\pgfusepath{stroke}%
\end{pgfscope}%
\begin{pgfscope}%
\pgfpathrectangle{\pgfqpoint{0.847223in}{0.554012in}}{\pgfqpoint{6.200000in}{4.530000in}}%
\pgfusepath{clip}%
\pgfsetbuttcap%
\pgfsetroundjoin%
\pgfsetlinewidth{1.003750pt}%
\definecolor{currentstroke}{rgb}{1.000000,0.000000,0.000000}%
\pgfsetstrokecolor{currentstroke}%
\pgfsetdash{}{0pt}%
\pgfpathmoveto{\pgfqpoint{1.652539in}{2.043815in}}%
\pgfpathcurveto{\pgfqpoint{1.663589in}{2.043815in}}{\pgfqpoint{1.674188in}{2.048206in}}{\pgfqpoint{1.682001in}{2.056019in}}%
\pgfpathcurveto{\pgfqpoint{1.689815in}{2.063833in}}{\pgfqpoint{1.694205in}{2.074432in}}{\pgfqpoint{1.694205in}{2.085482in}}%
\pgfpathcurveto{\pgfqpoint{1.694205in}{2.096532in}}{\pgfqpoint{1.689815in}{2.107131in}}{\pgfqpoint{1.682001in}{2.114945in}}%
\pgfpathcurveto{\pgfqpoint{1.674188in}{2.122759in}}{\pgfqpoint{1.663589in}{2.127149in}}{\pgfqpoint{1.652539in}{2.127149in}}%
\pgfpathcurveto{\pgfqpoint{1.641488in}{2.127149in}}{\pgfqpoint{1.630889in}{2.122759in}}{\pgfqpoint{1.623076in}{2.114945in}}%
\pgfpathcurveto{\pgfqpoint{1.615262in}{2.107131in}}{\pgfqpoint{1.610872in}{2.096532in}}{\pgfqpoint{1.610872in}{2.085482in}}%
\pgfpathcurveto{\pgfqpoint{1.610872in}{2.074432in}}{\pgfqpoint{1.615262in}{2.063833in}}{\pgfqpoint{1.623076in}{2.056019in}}%
\pgfpathcurveto{\pgfqpoint{1.630889in}{2.048206in}}{\pgfqpoint{1.641488in}{2.043815in}}{\pgfqpoint{1.652539in}{2.043815in}}%
\pgfpathlineto{\pgfqpoint{1.652539in}{2.043815in}}%
\pgfpathclose%
\pgfusepath{stroke}%
\end{pgfscope}%
\begin{pgfscope}%
\pgfpathrectangle{\pgfqpoint{0.847223in}{0.554012in}}{\pgfqpoint{6.200000in}{4.530000in}}%
\pgfusepath{clip}%
\pgfsetbuttcap%
\pgfsetroundjoin%
\pgfsetlinewidth{1.003750pt}%
\definecolor{currentstroke}{rgb}{1.000000,0.000000,0.000000}%
\pgfsetstrokecolor{currentstroke}%
\pgfsetdash{}{0pt}%
\pgfpathmoveto{\pgfqpoint{1.657872in}{2.036025in}}%
\pgfpathcurveto{\pgfqpoint{1.668922in}{2.036025in}}{\pgfqpoint{1.679521in}{2.040415in}}{\pgfqpoint{1.687335in}{2.048229in}}%
\pgfpathcurveto{\pgfqpoint{1.695148in}{2.056042in}}{\pgfqpoint{1.699539in}{2.066641in}}{\pgfqpoint{1.699539in}{2.077692in}}%
\pgfpathcurveto{\pgfqpoint{1.699539in}{2.088742in}}{\pgfqpoint{1.695148in}{2.099341in}}{\pgfqpoint{1.687335in}{2.107154in}}%
\pgfpathcurveto{\pgfqpoint{1.679521in}{2.114968in}}{\pgfqpoint{1.668922in}{2.119358in}}{\pgfqpoint{1.657872in}{2.119358in}}%
\pgfpathcurveto{\pgfqpoint{1.646822in}{2.119358in}}{\pgfqpoint{1.636223in}{2.114968in}}{\pgfqpoint{1.628409in}{2.107154in}}%
\pgfpathcurveto{\pgfqpoint{1.620595in}{2.099341in}}{\pgfqpoint{1.616205in}{2.088742in}}{\pgfqpoint{1.616205in}{2.077692in}}%
\pgfpathcurveto{\pgfqpoint{1.616205in}{2.066641in}}{\pgfqpoint{1.620595in}{2.056042in}}{\pgfqpoint{1.628409in}{2.048229in}}%
\pgfpathcurveto{\pgfqpoint{1.636223in}{2.040415in}}{\pgfqpoint{1.646822in}{2.036025in}}{\pgfqpoint{1.657872in}{2.036025in}}%
\pgfpathlineto{\pgfqpoint{1.657872in}{2.036025in}}%
\pgfpathclose%
\pgfusepath{stroke}%
\end{pgfscope}%
\begin{pgfscope}%
\pgfpathrectangle{\pgfqpoint{0.847223in}{0.554012in}}{\pgfqpoint{6.200000in}{4.530000in}}%
\pgfusepath{clip}%
\pgfsetbuttcap%
\pgfsetroundjoin%
\pgfsetlinewidth{1.003750pt}%
\definecolor{currentstroke}{rgb}{1.000000,0.000000,0.000000}%
\pgfsetstrokecolor{currentstroke}%
\pgfsetdash{}{0pt}%
\pgfpathmoveto{\pgfqpoint{1.663205in}{2.028296in}}%
\pgfpathcurveto{\pgfqpoint{1.674255in}{2.028296in}}{\pgfqpoint{1.684854in}{2.032687in}}{\pgfqpoint{1.692668in}{2.040500in}}%
\pgfpathcurveto{\pgfqpoint{1.700481in}{2.048314in}}{\pgfqpoint{1.704872in}{2.058913in}}{\pgfqpoint{1.704872in}{2.069963in}}%
\pgfpathcurveto{\pgfqpoint{1.704872in}{2.081013in}}{\pgfqpoint{1.700481in}{2.091612in}}{\pgfqpoint{1.692668in}{2.099426in}}%
\pgfpathcurveto{\pgfqpoint{1.684854in}{2.107239in}}{\pgfqpoint{1.674255in}{2.111630in}}{\pgfqpoint{1.663205in}{2.111630in}}%
\pgfpathcurveto{\pgfqpoint{1.652155in}{2.111630in}}{\pgfqpoint{1.641556in}{2.107239in}}{\pgfqpoint{1.633742in}{2.099426in}}%
\pgfpathcurveto{\pgfqpoint{1.625929in}{2.091612in}}{\pgfqpoint{1.621538in}{2.081013in}}{\pgfqpoint{1.621538in}{2.069963in}}%
\pgfpathcurveto{\pgfqpoint{1.621538in}{2.058913in}}{\pgfqpoint{1.625929in}{2.048314in}}{\pgfqpoint{1.633742in}{2.040500in}}%
\pgfpathcurveto{\pgfqpoint{1.641556in}{2.032687in}}{\pgfqpoint{1.652155in}{2.028296in}}{\pgfqpoint{1.663205in}{2.028296in}}%
\pgfpathlineto{\pgfqpoint{1.663205in}{2.028296in}}%
\pgfpathclose%
\pgfusepath{stroke}%
\end{pgfscope}%
\begin{pgfscope}%
\pgfpathrectangle{\pgfqpoint{0.847223in}{0.554012in}}{\pgfqpoint{6.200000in}{4.530000in}}%
\pgfusepath{clip}%
\pgfsetbuttcap%
\pgfsetroundjoin%
\pgfsetlinewidth{1.003750pt}%
\definecolor{currentstroke}{rgb}{1.000000,0.000000,0.000000}%
\pgfsetstrokecolor{currentstroke}%
\pgfsetdash{}{0pt}%
\pgfpathmoveto{\pgfqpoint{1.668538in}{2.020629in}}%
\pgfpathcurveto{\pgfqpoint{1.679588in}{2.020629in}}{\pgfqpoint{1.690187in}{2.025019in}}{\pgfqpoint{1.698001in}{2.032833in}}%
\pgfpathcurveto{\pgfqpoint{1.705815in}{2.040647in}}{\pgfqpoint{1.710205in}{2.051246in}}{\pgfqpoint{1.710205in}{2.062296in}}%
\pgfpathcurveto{\pgfqpoint{1.710205in}{2.073346in}}{\pgfqpoint{1.705815in}{2.083945in}}{\pgfqpoint{1.698001in}{2.091759in}}%
\pgfpathcurveto{\pgfqpoint{1.690187in}{2.099572in}}{\pgfqpoint{1.679588in}{2.103963in}}{\pgfqpoint{1.668538in}{2.103963in}}%
\pgfpathcurveto{\pgfqpoint{1.657488in}{2.103963in}}{\pgfqpoint{1.646889in}{2.099572in}}{\pgfqpoint{1.639075in}{2.091759in}}%
\pgfpathcurveto{\pgfqpoint{1.631262in}{2.083945in}}{\pgfqpoint{1.626872in}{2.073346in}}{\pgfqpoint{1.626872in}{2.062296in}}%
\pgfpathcurveto{\pgfqpoint{1.626872in}{2.051246in}}{\pgfqpoint{1.631262in}{2.040647in}}{\pgfqpoint{1.639075in}{2.032833in}}%
\pgfpathcurveto{\pgfqpoint{1.646889in}{2.025019in}}{\pgfqpoint{1.657488in}{2.020629in}}{\pgfqpoint{1.668538in}{2.020629in}}%
\pgfpathlineto{\pgfqpoint{1.668538in}{2.020629in}}%
\pgfpathclose%
\pgfusepath{stroke}%
\end{pgfscope}%
\begin{pgfscope}%
\pgfpathrectangle{\pgfqpoint{0.847223in}{0.554012in}}{\pgfqpoint{6.200000in}{4.530000in}}%
\pgfusepath{clip}%
\pgfsetbuttcap%
\pgfsetroundjoin%
\pgfsetlinewidth{1.003750pt}%
\definecolor{currentstroke}{rgb}{1.000000,0.000000,0.000000}%
\pgfsetstrokecolor{currentstroke}%
\pgfsetdash{}{0pt}%
\pgfpathmoveto{\pgfqpoint{1.673871in}{2.013023in}}%
\pgfpathcurveto{\pgfqpoint{1.684922in}{2.013023in}}{\pgfqpoint{1.695521in}{2.017413in}}{\pgfqpoint{1.703334in}{2.025226in}}%
\pgfpathcurveto{\pgfqpoint{1.711148in}{2.033040in}}{\pgfqpoint{1.715538in}{2.043639in}}{\pgfqpoint{1.715538in}{2.054689in}}%
\pgfpathcurveto{\pgfqpoint{1.715538in}{2.065739in}}{\pgfqpoint{1.711148in}{2.076338in}}{\pgfqpoint{1.703334in}{2.084152in}}%
\pgfpathcurveto{\pgfqpoint{1.695521in}{2.091966in}}{\pgfqpoint{1.684922in}{2.096356in}}{\pgfqpoint{1.673871in}{2.096356in}}%
\pgfpathcurveto{\pgfqpoint{1.662821in}{2.096356in}}{\pgfqpoint{1.652222in}{2.091966in}}{\pgfqpoint{1.644409in}{2.084152in}}%
\pgfpathcurveto{\pgfqpoint{1.636595in}{2.076338in}}{\pgfqpoint{1.632205in}{2.065739in}}{\pgfqpoint{1.632205in}{2.054689in}}%
\pgfpathcurveto{\pgfqpoint{1.632205in}{2.043639in}}{\pgfqpoint{1.636595in}{2.033040in}}{\pgfqpoint{1.644409in}{2.025226in}}%
\pgfpathcurveto{\pgfqpoint{1.652222in}{2.017413in}}{\pgfqpoint{1.662821in}{2.013023in}}{\pgfqpoint{1.673871in}{2.013023in}}%
\pgfpathlineto{\pgfqpoint{1.673871in}{2.013023in}}%
\pgfpathclose%
\pgfusepath{stroke}%
\end{pgfscope}%
\begin{pgfscope}%
\pgfpathrectangle{\pgfqpoint{0.847223in}{0.554012in}}{\pgfqpoint{6.200000in}{4.530000in}}%
\pgfusepath{clip}%
\pgfsetbuttcap%
\pgfsetroundjoin%
\pgfsetlinewidth{1.003750pt}%
\definecolor{currentstroke}{rgb}{1.000000,0.000000,0.000000}%
\pgfsetstrokecolor{currentstroke}%
\pgfsetdash{}{0pt}%
\pgfpathmoveto{\pgfqpoint{1.679205in}{2.005476in}}%
\pgfpathcurveto{\pgfqpoint{1.690255in}{2.005476in}}{\pgfqpoint{1.700854in}{2.009866in}}{\pgfqpoint{1.708667in}{2.017680in}}%
\pgfpathcurveto{\pgfqpoint{1.716481in}{2.025493in}}{\pgfqpoint{1.720871in}{2.036092in}}{\pgfqpoint{1.720871in}{2.047142in}}%
\pgfpathcurveto{\pgfqpoint{1.720871in}{2.058193in}}{\pgfqpoint{1.716481in}{2.068792in}}{\pgfqpoint{1.708667in}{2.076605in}}%
\pgfpathcurveto{\pgfqpoint{1.700854in}{2.084419in}}{\pgfqpoint{1.690255in}{2.088809in}}{\pgfqpoint{1.679205in}{2.088809in}}%
\pgfpathcurveto{\pgfqpoint{1.668155in}{2.088809in}}{\pgfqpoint{1.657556in}{2.084419in}}{\pgfqpoint{1.649742in}{2.076605in}}%
\pgfpathcurveto{\pgfqpoint{1.641928in}{2.068792in}}{\pgfqpoint{1.637538in}{2.058193in}}{\pgfqpoint{1.637538in}{2.047142in}}%
\pgfpathcurveto{\pgfqpoint{1.637538in}{2.036092in}}{\pgfqpoint{1.641928in}{2.025493in}}{\pgfqpoint{1.649742in}{2.017680in}}%
\pgfpathcurveto{\pgfqpoint{1.657556in}{2.009866in}}{\pgfqpoint{1.668155in}{2.005476in}}{\pgfqpoint{1.679205in}{2.005476in}}%
\pgfpathlineto{\pgfqpoint{1.679205in}{2.005476in}}%
\pgfpathclose%
\pgfusepath{stroke}%
\end{pgfscope}%
\begin{pgfscope}%
\pgfpathrectangle{\pgfqpoint{0.847223in}{0.554012in}}{\pgfqpoint{6.200000in}{4.530000in}}%
\pgfusepath{clip}%
\pgfsetbuttcap%
\pgfsetroundjoin%
\pgfsetlinewidth{1.003750pt}%
\definecolor{currentstroke}{rgb}{1.000000,0.000000,0.000000}%
\pgfsetstrokecolor{currentstroke}%
\pgfsetdash{}{0pt}%
\pgfpathmoveto{\pgfqpoint{1.684538in}{1.997988in}}%
\pgfpathcurveto{\pgfqpoint{1.695588in}{1.997988in}}{\pgfqpoint{1.706187in}{2.002378in}}{\pgfqpoint{1.714001in}{2.010192in}}%
\pgfpathcurveto{\pgfqpoint{1.721814in}{2.018006in}}{\pgfqpoint{1.726205in}{2.028605in}}{\pgfqpoint{1.726205in}{2.039655in}}%
\pgfpathcurveto{\pgfqpoint{1.726205in}{2.050705in}}{\pgfqpoint{1.721814in}{2.061304in}}{\pgfqpoint{1.714001in}{2.069118in}}%
\pgfpathcurveto{\pgfqpoint{1.706187in}{2.076931in}}{\pgfqpoint{1.695588in}{2.081322in}}{\pgfqpoint{1.684538in}{2.081322in}}%
\pgfpathcurveto{\pgfqpoint{1.673488in}{2.081322in}}{\pgfqpoint{1.662889in}{2.076931in}}{\pgfqpoint{1.655075in}{2.069118in}}%
\pgfpathcurveto{\pgfqpoint{1.647262in}{2.061304in}}{\pgfqpoint{1.642871in}{2.050705in}}{\pgfqpoint{1.642871in}{2.039655in}}%
\pgfpathcurveto{\pgfqpoint{1.642871in}{2.028605in}}{\pgfqpoint{1.647262in}{2.018006in}}{\pgfqpoint{1.655075in}{2.010192in}}%
\pgfpathcurveto{\pgfqpoint{1.662889in}{2.002378in}}{\pgfqpoint{1.673488in}{1.997988in}}{\pgfqpoint{1.684538in}{1.997988in}}%
\pgfpathlineto{\pgfqpoint{1.684538in}{1.997988in}}%
\pgfpathclose%
\pgfusepath{stroke}%
\end{pgfscope}%
\begin{pgfscope}%
\pgfpathrectangle{\pgfqpoint{0.847223in}{0.554012in}}{\pgfqpoint{6.200000in}{4.530000in}}%
\pgfusepath{clip}%
\pgfsetbuttcap%
\pgfsetroundjoin%
\pgfsetlinewidth{1.003750pt}%
\definecolor{currentstroke}{rgb}{1.000000,0.000000,0.000000}%
\pgfsetstrokecolor{currentstroke}%
\pgfsetdash{}{0pt}%
\pgfpathmoveto{\pgfqpoint{1.689871in}{1.990559in}}%
\pgfpathcurveto{\pgfqpoint{1.700921in}{1.990559in}}{\pgfqpoint{1.711520in}{1.994949in}}{\pgfqpoint{1.719334in}{2.002763in}}%
\pgfpathcurveto{\pgfqpoint{1.727148in}{2.010577in}}{\pgfqpoint{1.731538in}{2.021176in}}{\pgfqpoint{1.731538in}{2.032226in}}%
\pgfpathcurveto{\pgfqpoint{1.731538in}{2.043276in}}{\pgfqpoint{1.727148in}{2.053875in}}{\pgfqpoint{1.719334in}{2.061689in}}%
\pgfpathcurveto{\pgfqpoint{1.711520in}{2.069502in}}{\pgfqpoint{1.700921in}{2.073892in}}{\pgfqpoint{1.689871in}{2.073892in}}%
\pgfpathcurveto{\pgfqpoint{1.678821in}{2.073892in}}{\pgfqpoint{1.668222in}{2.069502in}}{\pgfqpoint{1.660408in}{2.061689in}}%
\pgfpathcurveto{\pgfqpoint{1.652595in}{2.053875in}}{\pgfqpoint{1.648204in}{2.043276in}}{\pgfqpoint{1.648204in}{2.032226in}}%
\pgfpathcurveto{\pgfqpoint{1.648204in}{2.021176in}}{\pgfqpoint{1.652595in}{2.010577in}}{\pgfqpoint{1.660408in}{2.002763in}}%
\pgfpathcurveto{\pgfqpoint{1.668222in}{1.994949in}}{\pgfqpoint{1.678821in}{1.990559in}}{\pgfqpoint{1.689871in}{1.990559in}}%
\pgfpathlineto{\pgfqpoint{1.689871in}{1.990559in}}%
\pgfpathclose%
\pgfusepath{stroke}%
\end{pgfscope}%
\begin{pgfscope}%
\pgfpathrectangle{\pgfqpoint{0.847223in}{0.554012in}}{\pgfqpoint{6.200000in}{4.530000in}}%
\pgfusepath{clip}%
\pgfsetbuttcap%
\pgfsetroundjoin%
\pgfsetlinewidth{1.003750pt}%
\definecolor{currentstroke}{rgb}{1.000000,0.000000,0.000000}%
\pgfsetstrokecolor{currentstroke}%
\pgfsetdash{}{0pt}%
\pgfpathmoveto{\pgfqpoint{1.695204in}{1.983188in}}%
\pgfpathcurveto{\pgfqpoint{1.706254in}{1.983188in}}{\pgfqpoint{1.716854in}{1.987578in}}{\pgfqpoint{1.724667in}{1.995392in}}%
\pgfpathcurveto{\pgfqpoint{1.732481in}{2.003205in}}{\pgfqpoint{1.736871in}{2.013804in}}{\pgfqpoint{1.736871in}{2.024854in}}%
\pgfpathcurveto{\pgfqpoint{1.736871in}{2.035905in}}{\pgfqpoint{1.732481in}{2.046504in}}{\pgfqpoint{1.724667in}{2.054317in}}%
\pgfpathcurveto{\pgfqpoint{1.716854in}{2.062131in}}{\pgfqpoint{1.706254in}{2.066521in}}{\pgfqpoint{1.695204in}{2.066521in}}%
\pgfpathcurveto{\pgfqpoint{1.684154in}{2.066521in}}{\pgfqpoint{1.673555in}{2.062131in}}{\pgfqpoint{1.665742in}{2.054317in}}%
\pgfpathcurveto{\pgfqpoint{1.657928in}{2.046504in}}{\pgfqpoint{1.653538in}{2.035905in}}{\pgfqpoint{1.653538in}{2.024854in}}%
\pgfpathcurveto{\pgfqpoint{1.653538in}{2.013804in}}{\pgfqpoint{1.657928in}{2.003205in}}{\pgfqpoint{1.665742in}{1.995392in}}%
\pgfpathcurveto{\pgfqpoint{1.673555in}{1.987578in}}{\pgfqpoint{1.684154in}{1.983188in}}{\pgfqpoint{1.695204in}{1.983188in}}%
\pgfpathlineto{\pgfqpoint{1.695204in}{1.983188in}}%
\pgfpathclose%
\pgfusepath{stroke}%
\end{pgfscope}%
\begin{pgfscope}%
\pgfpathrectangle{\pgfqpoint{0.847223in}{0.554012in}}{\pgfqpoint{6.200000in}{4.530000in}}%
\pgfusepath{clip}%
\pgfsetbuttcap%
\pgfsetroundjoin%
\pgfsetlinewidth{1.003750pt}%
\definecolor{currentstroke}{rgb}{1.000000,0.000000,0.000000}%
\pgfsetstrokecolor{currentstroke}%
\pgfsetdash{}{0pt}%
\pgfpathmoveto{\pgfqpoint{1.700538in}{1.975874in}}%
\pgfpathcurveto{\pgfqpoint{1.711588in}{1.975874in}}{\pgfqpoint{1.722187in}{1.980264in}}{\pgfqpoint{1.730000in}{1.988077in}}%
\pgfpathcurveto{\pgfqpoint{1.737814in}{1.995891in}}{\pgfqpoint{1.742204in}{2.006490in}}{\pgfqpoint{1.742204in}{2.017540in}}%
\pgfpathcurveto{\pgfqpoint{1.742204in}{2.028590in}}{\pgfqpoint{1.737814in}{2.039189in}}{\pgfqpoint{1.730000in}{2.047003in}}%
\pgfpathcurveto{\pgfqpoint{1.722187in}{2.054817in}}{\pgfqpoint{1.711588in}{2.059207in}}{\pgfqpoint{1.700538in}{2.059207in}}%
\pgfpathcurveto{\pgfqpoint{1.689487in}{2.059207in}}{\pgfqpoint{1.678888in}{2.054817in}}{\pgfqpoint{1.671075in}{2.047003in}}%
\pgfpathcurveto{\pgfqpoint{1.663261in}{2.039189in}}{\pgfqpoint{1.658871in}{2.028590in}}{\pgfqpoint{1.658871in}{2.017540in}}%
\pgfpathcurveto{\pgfqpoint{1.658871in}{2.006490in}}{\pgfqpoint{1.663261in}{1.995891in}}{\pgfqpoint{1.671075in}{1.988077in}}%
\pgfpathcurveto{\pgfqpoint{1.678888in}{1.980264in}}{\pgfqpoint{1.689487in}{1.975874in}}{\pgfqpoint{1.700538in}{1.975874in}}%
\pgfpathlineto{\pgfqpoint{1.700538in}{1.975874in}}%
\pgfpathclose%
\pgfusepath{stroke}%
\end{pgfscope}%
\begin{pgfscope}%
\pgfpathrectangle{\pgfqpoint{0.847223in}{0.554012in}}{\pgfqpoint{6.200000in}{4.530000in}}%
\pgfusepath{clip}%
\pgfsetbuttcap%
\pgfsetroundjoin%
\pgfsetlinewidth{1.003750pt}%
\definecolor{currentstroke}{rgb}{1.000000,0.000000,0.000000}%
\pgfsetstrokecolor{currentstroke}%
\pgfsetdash{}{0pt}%
\pgfpathmoveto{\pgfqpoint{1.705871in}{1.968616in}}%
\pgfpathcurveto{\pgfqpoint{1.716921in}{1.968616in}}{\pgfqpoint{1.727520in}{1.973006in}}{\pgfqpoint{1.735334in}{1.980820in}}%
\pgfpathcurveto{\pgfqpoint{1.743147in}{1.988633in}}{\pgfqpoint{1.747537in}{1.999232in}}{\pgfqpoint{1.747537in}{2.010282in}}%
\pgfpathcurveto{\pgfqpoint{1.747537in}{2.021333in}}{\pgfqpoint{1.743147in}{2.031932in}}{\pgfqpoint{1.735334in}{2.039745in}}%
\pgfpathcurveto{\pgfqpoint{1.727520in}{2.047559in}}{\pgfqpoint{1.716921in}{2.051949in}}{\pgfqpoint{1.705871in}{2.051949in}}%
\pgfpathcurveto{\pgfqpoint{1.694821in}{2.051949in}}{\pgfqpoint{1.684222in}{2.047559in}}{\pgfqpoint{1.676408in}{2.039745in}}%
\pgfpathcurveto{\pgfqpoint{1.668594in}{2.031932in}}{\pgfqpoint{1.664204in}{2.021333in}}{\pgfqpoint{1.664204in}{2.010282in}}%
\pgfpathcurveto{\pgfqpoint{1.664204in}{1.999232in}}{\pgfqpoint{1.668594in}{1.988633in}}{\pgfqpoint{1.676408in}{1.980820in}}%
\pgfpathcurveto{\pgfqpoint{1.684222in}{1.973006in}}{\pgfqpoint{1.694821in}{1.968616in}}{\pgfqpoint{1.705871in}{1.968616in}}%
\pgfpathlineto{\pgfqpoint{1.705871in}{1.968616in}}%
\pgfpathclose%
\pgfusepath{stroke}%
\end{pgfscope}%
\begin{pgfscope}%
\pgfpathrectangle{\pgfqpoint{0.847223in}{0.554012in}}{\pgfqpoint{6.200000in}{4.530000in}}%
\pgfusepath{clip}%
\pgfsetbuttcap%
\pgfsetroundjoin%
\pgfsetlinewidth{1.003750pt}%
\definecolor{currentstroke}{rgb}{1.000000,0.000000,0.000000}%
\pgfsetstrokecolor{currentstroke}%
\pgfsetdash{}{0pt}%
\pgfpathmoveto{\pgfqpoint{1.711204in}{1.961414in}}%
\pgfpathcurveto{\pgfqpoint{1.722254in}{1.961414in}}{\pgfqpoint{1.732853in}{1.965804in}}{\pgfqpoint{1.740667in}{1.973618in}}%
\pgfpathcurveto{\pgfqpoint{1.748480in}{1.981431in}}{\pgfqpoint{1.752871in}{1.992030in}}{\pgfqpoint{1.752871in}{2.003080in}}%
\pgfpathcurveto{\pgfqpoint{1.752871in}{2.014131in}}{\pgfqpoint{1.748480in}{2.024730in}}{\pgfqpoint{1.740667in}{2.032543in}}%
\pgfpathcurveto{\pgfqpoint{1.732853in}{2.040357in}}{\pgfqpoint{1.722254in}{2.044747in}}{\pgfqpoint{1.711204in}{2.044747in}}%
\pgfpathcurveto{\pgfqpoint{1.700154in}{2.044747in}}{\pgfqpoint{1.689555in}{2.040357in}}{\pgfqpoint{1.681741in}{2.032543in}}%
\pgfpathcurveto{\pgfqpoint{1.673928in}{2.024730in}}{\pgfqpoint{1.669537in}{2.014131in}}{\pgfqpoint{1.669537in}{2.003080in}}%
\pgfpathcurveto{\pgfqpoint{1.669537in}{1.992030in}}{\pgfqpoint{1.673928in}{1.981431in}}{\pgfqpoint{1.681741in}{1.973618in}}%
\pgfpathcurveto{\pgfqpoint{1.689555in}{1.965804in}}{\pgfqpoint{1.700154in}{1.961414in}}{\pgfqpoint{1.711204in}{1.961414in}}%
\pgfpathlineto{\pgfqpoint{1.711204in}{1.961414in}}%
\pgfpathclose%
\pgfusepath{stroke}%
\end{pgfscope}%
\begin{pgfscope}%
\pgfpathrectangle{\pgfqpoint{0.847223in}{0.554012in}}{\pgfqpoint{6.200000in}{4.530000in}}%
\pgfusepath{clip}%
\pgfsetbuttcap%
\pgfsetroundjoin%
\pgfsetlinewidth{1.003750pt}%
\definecolor{currentstroke}{rgb}{1.000000,0.000000,0.000000}%
\pgfsetstrokecolor{currentstroke}%
\pgfsetdash{}{0pt}%
\pgfpathmoveto{\pgfqpoint{1.716537in}{1.954267in}}%
\pgfpathcurveto{\pgfqpoint{1.727587in}{1.954267in}}{\pgfqpoint{1.738186in}{1.958657in}}{\pgfqpoint{1.746000in}{1.966471in}}%
\pgfpathcurveto{\pgfqpoint{1.753814in}{1.974284in}}{\pgfqpoint{1.758204in}{1.984883in}}{\pgfqpoint{1.758204in}{1.995934in}}%
\pgfpathcurveto{\pgfqpoint{1.758204in}{2.006984in}}{\pgfqpoint{1.753814in}{2.017583in}}{\pgfqpoint{1.746000in}{2.025396in}}%
\pgfpathcurveto{\pgfqpoint{1.738186in}{2.033210in}}{\pgfqpoint{1.727587in}{2.037600in}}{\pgfqpoint{1.716537in}{2.037600in}}%
\pgfpathcurveto{\pgfqpoint{1.705487in}{2.037600in}}{\pgfqpoint{1.694888in}{2.033210in}}{\pgfqpoint{1.687074in}{2.025396in}}%
\pgfpathcurveto{\pgfqpoint{1.679261in}{2.017583in}}{\pgfqpoint{1.674871in}{2.006984in}}{\pgfqpoint{1.674871in}{1.995934in}}%
\pgfpathcurveto{\pgfqpoint{1.674871in}{1.984883in}}{\pgfqpoint{1.679261in}{1.974284in}}{\pgfqpoint{1.687074in}{1.966471in}}%
\pgfpathcurveto{\pgfqpoint{1.694888in}{1.958657in}}{\pgfqpoint{1.705487in}{1.954267in}}{\pgfqpoint{1.716537in}{1.954267in}}%
\pgfpathlineto{\pgfqpoint{1.716537in}{1.954267in}}%
\pgfpathclose%
\pgfusepath{stroke}%
\end{pgfscope}%
\begin{pgfscope}%
\pgfpathrectangle{\pgfqpoint{0.847223in}{0.554012in}}{\pgfqpoint{6.200000in}{4.530000in}}%
\pgfusepath{clip}%
\pgfsetbuttcap%
\pgfsetroundjoin%
\pgfsetlinewidth{1.003750pt}%
\definecolor{currentstroke}{rgb}{1.000000,0.000000,0.000000}%
\pgfsetstrokecolor{currentstroke}%
\pgfsetdash{}{0pt}%
\pgfpathmoveto{\pgfqpoint{1.721870in}{1.947175in}}%
\pgfpathcurveto{\pgfqpoint{1.732921in}{1.947175in}}{\pgfqpoint{1.743520in}{1.951565in}}{\pgfqpoint{1.751333in}{1.959379in}}%
\pgfpathcurveto{\pgfqpoint{1.759147in}{1.967192in}}{\pgfqpoint{1.763537in}{1.977791in}}{\pgfqpoint{1.763537in}{1.988841in}}%
\pgfpathcurveto{\pgfqpoint{1.763537in}{1.999891in}}{\pgfqpoint{1.759147in}{2.010490in}}{\pgfqpoint{1.751333in}{2.018304in}}%
\pgfpathcurveto{\pgfqpoint{1.743520in}{2.026118in}}{\pgfqpoint{1.732921in}{2.030508in}}{\pgfqpoint{1.721870in}{2.030508in}}%
\pgfpathcurveto{\pgfqpoint{1.710820in}{2.030508in}}{\pgfqpoint{1.700221in}{2.026118in}}{\pgfqpoint{1.692408in}{2.018304in}}%
\pgfpathcurveto{\pgfqpoint{1.684594in}{2.010490in}}{\pgfqpoint{1.680204in}{1.999891in}}{\pgfqpoint{1.680204in}{1.988841in}}%
\pgfpathcurveto{\pgfqpoint{1.680204in}{1.977791in}}{\pgfqpoint{1.684594in}{1.967192in}}{\pgfqpoint{1.692408in}{1.959379in}}%
\pgfpathcurveto{\pgfqpoint{1.700221in}{1.951565in}}{\pgfqpoint{1.710820in}{1.947175in}}{\pgfqpoint{1.721870in}{1.947175in}}%
\pgfpathlineto{\pgfqpoint{1.721870in}{1.947175in}}%
\pgfpathclose%
\pgfusepath{stroke}%
\end{pgfscope}%
\begin{pgfscope}%
\pgfpathrectangle{\pgfqpoint{0.847223in}{0.554012in}}{\pgfqpoint{6.200000in}{4.530000in}}%
\pgfusepath{clip}%
\pgfsetbuttcap%
\pgfsetroundjoin%
\pgfsetlinewidth{1.003750pt}%
\definecolor{currentstroke}{rgb}{1.000000,0.000000,0.000000}%
\pgfsetstrokecolor{currentstroke}%
\pgfsetdash{}{0pt}%
\pgfpathmoveto{\pgfqpoint{1.727204in}{1.940136in}}%
\pgfpathcurveto{\pgfqpoint{1.738254in}{1.940136in}}{\pgfqpoint{1.748853in}{1.944527in}}{\pgfqpoint{1.756666in}{1.952340in}}%
\pgfpathcurveto{\pgfqpoint{1.764480in}{1.960154in}}{\pgfqpoint{1.768870in}{1.970753in}}{\pgfqpoint{1.768870in}{1.981803in}}%
\pgfpathcurveto{\pgfqpoint{1.768870in}{1.992853in}}{\pgfqpoint{1.764480in}{2.003452in}}{\pgfqpoint{1.756666in}{2.011266in}}%
\pgfpathcurveto{\pgfqpoint{1.748853in}{2.019079in}}{\pgfqpoint{1.738254in}{2.023470in}}{\pgfqpoint{1.727204in}{2.023470in}}%
\pgfpathcurveto{\pgfqpoint{1.716154in}{2.023470in}}{\pgfqpoint{1.705554in}{2.019079in}}{\pgfqpoint{1.697741in}{2.011266in}}%
\pgfpathcurveto{\pgfqpoint{1.689927in}{2.003452in}}{\pgfqpoint{1.685537in}{1.992853in}}{\pgfqpoint{1.685537in}{1.981803in}}%
\pgfpathcurveto{\pgfqpoint{1.685537in}{1.970753in}}{\pgfqpoint{1.689927in}{1.960154in}}{\pgfqpoint{1.697741in}{1.952340in}}%
\pgfpathcurveto{\pgfqpoint{1.705554in}{1.944527in}}{\pgfqpoint{1.716154in}{1.940136in}}{\pgfqpoint{1.727204in}{1.940136in}}%
\pgfpathlineto{\pgfqpoint{1.727204in}{1.940136in}}%
\pgfpathclose%
\pgfusepath{stroke}%
\end{pgfscope}%
\begin{pgfscope}%
\pgfpathrectangle{\pgfqpoint{0.847223in}{0.554012in}}{\pgfqpoint{6.200000in}{4.530000in}}%
\pgfusepath{clip}%
\pgfsetbuttcap%
\pgfsetroundjoin%
\pgfsetlinewidth{1.003750pt}%
\definecolor{currentstroke}{rgb}{1.000000,0.000000,0.000000}%
\pgfsetstrokecolor{currentstroke}%
\pgfsetdash{}{0pt}%
\pgfpathmoveto{\pgfqpoint{1.732537in}{1.933151in}}%
\pgfpathcurveto{\pgfqpoint{1.743587in}{1.933151in}}{\pgfqpoint{1.754186in}{1.937541in}}{\pgfqpoint{1.762000in}{1.945355in}}%
\pgfpathcurveto{\pgfqpoint{1.769813in}{1.953169in}}{\pgfqpoint{1.774204in}{1.963768in}}{\pgfqpoint{1.774204in}{1.974818in}}%
\pgfpathcurveto{\pgfqpoint{1.774204in}{1.985868in}}{\pgfqpoint{1.769813in}{1.996467in}}{\pgfqpoint{1.762000in}{2.004281in}}%
\pgfpathcurveto{\pgfqpoint{1.754186in}{2.012094in}}{\pgfqpoint{1.743587in}{2.016484in}}{\pgfqpoint{1.732537in}{2.016484in}}%
\pgfpathcurveto{\pgfqpoint{1.721487in}{2.016484in}}{\pgfqpoint{1.710888in}{2.012094in}}{\pgfqpoint{1.703074in}{2.004281in}}%
\pgfpathcurveto{\pgfqpoint{1.695260in}{1.996467in}}{\pgfqpoint{1.690870in}{1.985868in}}{\pgfqpoint{1.690870in}{1.974818in}}%
\pgfpathcurveto{\pgfqpoint{1.690870in}{1.963768in}}{\pgfqpoint{1.695260in}{1.953169in}}{\pgfqpoint{1.703074in}{1.945355in}}%
\pgfpathcurveto{\pgfqpoint{1.710888in}{1.937541in}}{\pgfqpoint{1.721487in}{1.933151in}}{\pgfqpoint{1.732537in}{1.933151in}}%
\pgfpathlineto{\pgfqpoint{1.732537in}{1.933151in}}%
\pgfpathclose%
\pgfusepath{stroke}%
\end{pgfscope}%
\begin{pgfscope}%
\pgfpathrectangle{\pgfqpoint{0.847223in}{0.554012in}}{\pgfqpoint{6.200000in}{4.530000in}}%
\pgfusepath{clip}%
\pgfsetbuttcap%
\pgfsetroundjoin%
\pgfsetlinewidth{1.003750pt}%
\definecolor{currentstroke}{rgb}{1.000000,0.000000,0.000000}%
\pgfsetstrokecolor{currentstroke}%
\pgfsetdash{}{0pt}%
\pgfpathmoveto{\pgfqpoint{1.737870in}{1.926219in}}%
\pgfpathcurveto{\pgfqpoint{1.748920in}{1.926219in}}{\pgfqpoint{1.759519in}{1.930609in}}{\pgfqpoint{1.767333in}{1.938423in}}%
\pgfpathcurveto{\pgfqpoint{1.775146in}{1.946236in}}{\pgfqpoint{1.779537in}{1.956835in}}{\pgfqpoint{1.779537in}{1.967885in}}%
\pgfpathcurveto{\pgfqpoint{1.779537in}{1.978936in}}{\pgfqpoint{1.775146in}{1.989535in}}{\pgfqpoint{1.767333in}{1.997348in}}%
\pgfpathcurveto{\pgfqpoint{1.759519in}{2.005162in}}{\pgfqpoint{1.748920in}{2.009552in}}{\pgfqpoint{1.737870in}{2.009552in}}%
\pgfpathcurveto{\pgfqpoint{1.726820in}{2.009552in}}{\pgfqpoint{1.716221in}{2.005162in}}{\pgfqpoint{1.708407in}{1.997348in}}%
\pgfpathcurveto{\pgfqpoint{1.700594in}{1.989535in}}{\pgfqpoint{1.696203in}{1.978936in}}{\pgfqpoint{1.696203in}{1.967885in}}%
\pgfpathcurveto{\pgfqpoint{1.696203in}{1.956835in}}{\pgfqpoint{1.700594in}{1.946236in}}{\pgfqpoint{1.708407in}{1.938423in}}%
\pgfpathcurveto{\pgfqpoint{1.716221in}{1.930609in}}{\pgfqpoint{1.726820in}{1.926219in}}{\pgfqpoint{1.737870in}{1.926219in}}%
\pgfpathlineto{\pgfqpoint{1.737870in}{1.926219in}}%
\pgfpathclose%
\pgfusepath{stroke}%
\end{pgfscope}%
\begin{pgfscope}%
\pgfpathrectangle{\pgfqpoint{0.847223in}{0.554012in}}{\pgfqpoint{6.200000in}{4.530000in}}%
\pgfusepath{clip}%
\pgfsetbuttcap%
\pgfsetroundjoin%
\pgfsetlinewidth{1.003750pt}%
\definecolor{currentstroke}{rgb}{1.000000,0.000000,0.000000}%
\pgfsetstrokecolor{currentstroke}%
\pgfsetdash{}{0pt}%
\pgfpathmoveto{\pgfqpoint{1.743203in}{1.919338in}}%
\pgfpathcurveto{\pgfqpoint{1.754253in}{1.919338in}}{\pgfqpoint{1.764852in}{1.923729in}}{\pgfqpoint{1.772666in}{1.931542in}}%
\pgfpathcurveto{\pgfqpoint{1.780480in}{1.939356in}}{\pgfqpoint{1.784870in}{1.949955in}}{\pgfqpoint{1.784870in}{1.961005in}}%
\pgfpathcurveto{\pgfqpoint{1.784870in}{1.972055in}}{\pgfqpoint{1.780480in}{1.982654in}}{\pgfqpoint{1.772666in}{1.990468in}}%
\pgfpathcurveto{\pgfqpoint{1.764852in}{1.998282in}}{\pgfqpoint{1.754253in}{2.002672in}}{\pgfqpoint{1.743203in}{2.002672in}}%
\pgfpathcurveto{\pgfqpoint{1.732153in}{2.002672in}}{\pgfqpoint{1.721554in}{1.998282in}}{\pgfqpoint{1.713741in}{1.990468in}}%
\pgfpathcurveto{\pgfqpoint{1.705927in}{1.982654in}}{\pgfqpoint{1.701537in}{1.972055in}}{\pgfqpoint{1.701537in}{1.961005in}}%
\pgfpathcurveto{\pgfqpoint{1.701537in}{1.949955in}}{\pgfqpoint{1.705927in}{1.939356in}}{\pgfqpoint{1.713741in}{1.931542in}}%
\pgfpathcurveto{\pgfqpoint{1.721554in}{1.923729in}}{\pgfqpoint{1.732153in}{1.919338in}}{\pgfqpoint{1.743203in}{1.919338in}}%
\pgfpathlineto{\pgfqpoint{1.743203in}{1.919338in}}%
\pgfpathclose%
\pgfusepath{stroke}%
\end{pgfscope}%
\begin{pgfscope}%
\pgfpathrectangle{\pgfqpoint{0.847223in}{0.554012in}}{\pgfqpoint{6.200000in}{4.530000in}}%
\pgfusepath{clip}%
\pgfsetbuttcap%
\pgfsetroundjoin%
\pgfsetlinewidth{1.003750pt}%
\definecolor{currentstroke}{rgb}{1.000000,0.000000,0.000000}%
\pgfsetstrokecolor{currentstroke}%
\pgfsetdash{}{0pt}%
\pgfpathmoveto{\pgfqpoint{1.748537in}{1.912510in}}%
\pgfpathcurveto{\pgfqpoint{1.759587in}{1.912510in}}{\pgfqpoint{1.770186in}{1.916900in}}{\pgfqpoint{1.777999in}{1.924714in}}%
\pgfpathcurveto{\pgfqpoint{1.785813in}{1.932527in}}{\pgfqpoint{1.790203in}{1.943126in}}{\pgfqpoint{1.790203in}{1.954176in}}%
\pgfpathcurveto{\pgfqpoint{1.790203in}{1.965226in}}{\pgfqpoint{1.785813in}{1.975826in}}{\pgfqpoint{1.777999in}{1.983639in}}%
\pgfpathcurveto{\pgfqpoint{1.770186in}{1.991453in}}{\pgfqpoint{1.759587in}{1.995843in}}{\pgfqpoint{1.748537in}{1.995843in}}%
\pgfpathcurveto{\pgfqpoint{1.737486in}{1.995843in}}{\pgfqpoint{1.726887in}{1.991453in}}{\pgfqpoint{1.719074in}{1.983639in}}%
\pgfpathcurveto{\pgfqpoint{1.711260in}{1.975826in}}{\pgfqpoint{1.706870in}{1.965226in}}{\pgfqpoint{1.706870in}{1.954176in}}%
\pgfpathcurveto{\pgfqpoint{1.706870in}{1.943126in}}{\pgfqpoint{1.711260in}{1.932527in}}{\pgfqpoint{1.719074in}{1.924714in}}%
\pgfpathcurveto{\pgfqpoint{1.726887in}{1.916900in}}{\pgfqpoint{1.737486in}{1.912510in}}{\pgfqpoint{1.748537in}{1.912510in}}%
\pgfpathlineto{\pgfqpoint{1.748537in}{1.912510in}}%
\pgfpathclose%
\pgfusepath{stroke}%
\end{pgfscope}%
\begin{pgfscope}%
\pgfpathrectangle{\pgfqpoint{0.847223in}{0.554012in}}{\pgfqpoint{6.200000in}{4.530000in}}%
\pgfusepath{clip}%
\pgfsetbuttcap%
\pgfsetroundjoin%
\pgfsetlinewidth{1.003750pt}%
\definecolor{currentstroke}{rgb}{1.000000,0.000000,0.000000}%
\pgfsetstrokecolor{currentstroke}%
\pgfsetdash{}{0pt}%
\pgfpathmoveto{\pgfqpoint{1.753870in}{1.905732in}}%
\pgfpathcurveto{\pgfqpoint{1.764920in}{1.905732in}}{\pgfqpoint{1.775519in}{1.910122in}}{\pgfqpoint{1.783333in}{1.917936in}}%
\pgfpathcurveto{\pgfqpoint{1.791146in}{1.925749in}}{\pgfqpoint{1.795536in}{1.936348in}}{\pgfqpoint{1.795536in}{1.947398in}}%
\pgfpathcurveto{\pgfqpoint{1.795536in}{1.958449in}}{\pgfqpoint{1.791146in}{1.969048in}}{\pgfqpoint{1.783333in}{1.976861in}}%
\pgfpathcurveto{\pgfqpoint{1.775519in}{1.984675in}}{\pgfqpoint{1.764920in}{1.989065in}}{\pgfqpoint{1.753870in}{1.989065in}}%
\pgfpathcurveto{\pgfqpoint{1.742820in}{1.989065in}}{\pgfqpoint{1.732221in}{1.984675in}}{\pgfqpoint{1.724407in}{1.976861in}}%
\pgfpathcurveto{\pgfqpoint{1.716593in}{1.969048in}}{\pgfqpoint{1.712203in}{1.958449in}}{\pgfqpoint{1.712203in}{1.947398in}}%
\pgfpathcurveto{\pgfqpoint{1.712203in}{1.936348in}}{\pgfqpoint{1.716593in}{1.925749in}}{\pgfqpoint{1.724407in}{1.917936in}}%
\pgfpathcurveto{\pgfqpoint{1.732221in}{1.910122in}}{\pgfqpoint{1.742820in}{1.905732in}}{\pgfqpoint{1.753870in}{1.905732in}}%
\pgfpathlineto{\pgfqpoint{1.753870in}{1.905732in}}%
\pgfpathclose%
\pgfusepath{stroke}%
\end{pgfscope}%
\begin{pgfscope}%
\pgfpathrectangle{\pgfqpoint{0.847223in}{0.554012in}}{\pgfqpoint{6.200000in}{4.530000in}}%
\pgfusepath{clip}%
\pgfsetbuttcap%
\pgfsetroundjoin%
\pgfsetlinewidth{1.003750pt}%
\definecolor{currentstroke}{rgb}{1.000000,0.000000,0.000000}%
\pgfsetstrokecolor{currentstroke}%
\pgfsetdash{}{0pt}%
\pgfpathmoveto{\pgfqpoint{1.759203in}{1.899004in}}%
\pgfpathcurveto{\pgfqpoint{1.770253in}{1.899004in}}{\pgfqpoint{1.780852in}{1.903395in}}{\pgfqpoint{1.788666in}{1.911208in}}%
\pgfpathcurveto{\pgfqpoint{1.796479in}{1.919022in}}{\pgfqpoint{1.800870in}{1.929621in}}{\pgfqpoint{1.800870in}{1.940671in}}%
\pgfpathcurveto{\pgfqpoint{1.800870in}{1.951721in}}{\pgfqpoint{1.796479in}{1.962320in}}{\pgfqpoint{1.788666in}{1.970134in}}%
\pgfpathcurveto{\pgfqpoint{1.780852in}{1.977947in}}{\pgfqpoint{1.770253in}{1.982338in}}{\pgfqpoint{1.759203in}{1.982338in}}%
\pgfpathcurveto{\pgfqpoint{1.748153in}{1.982338in}}{\pgfqpoint{1.737554in}{1.977947in}}{\pgfqpoint{1.729740in}{1.970134in}}%
\pgfpathcurveto{\pgfqpoint{1.721927in}{1.962320in}}{\pgfqpoint{1.717536in}{1.951721in}}{\pgfqpoint{1.717536in}{1.940671in}}%
\pgfpathcurveto{\pgfqpoint{1.717536in}{1.929621in}}{\pgfqpoint{1.721927in}{1.919022in}}{\pgfqpoint{1.729740in}{1.911208in}}%
\pgfpathcurveto{\pgfqpoint{1.737554in}{1.903395in}}{\pgfqpoint{1.748153in}{1.899004in}}{\pgfqpoint{1.759203in}{1.899004in}}%
\pgfpathlineto{\pgfqpoint{1.759203in}{1.899004in}}%
\pgfpathclose%
\pgfusepath{stroke}%
\end{pgfscope}%
\begin{pgfscope}%
\pgfpathrectangle{\pgfqpoint{0.847223in}{0.554012in}}{\pgfqpoint{6.200000in}{4.530000in}}%
\pgfusepath{clip}%
\pgfsetbuttcap%
\pgfsetroundjoin%
\pgfsetlinewidth{1.003750pt}%
\definecolor{currentstroke}{rgb}{1.000000,0.000000,0.000000}%
\pgfsetstrokecolor{currentstroke}%
\pgfsetdash{}{0pt}%
\pgfpathmoveto{\pgfqpoint{1.764536in}{1.892327in}}%
\pgfpathcurveto{\pgfqpoint{1.775586in}{1.892327in}}{\pgfqpoint{1.786185in}{1.896717in}}{\pgfqpoint{1.793999in}{1.904531in}}%
\pgfpathcurveto{\pgfqpoint{1.801813in}{1.912344in}}{\pgfqpoint{1.806203in}{1.922943in}}{\pgfqpoint{1.806203in}{1.933993in}}%
\pgfpathcurveto{\pgfqpoint{1.806203in}{1.945043in}}{\pgfqpoint{1.801813in}{1.955642in}}{\pgfqpoint{1.793999in}{1.963456in}}%
\pgfpathcurveto{\pgfqpoint{1.786185in}{1.971270in}}{\pgfqpoint{1.775586in}{1.975660in}}{\pgfqpoint{1.764536in}{1.975660in}}%
\pgfpathcurveto{\pgfqpoint{1.753486in}{1.975660in}}{\pgfqpoint{1.742887in}{1.971270in}}{\pgfqpoint{1.735073in}{1.963456in}}%
\pgfpathcurveto{\pgfqpoint{1.727260in}{1.955642in}}{\pgfqpoint{1.722869in}{1.945043in}}{\pgfqpoint{1.722869in}{1.933993in}}%
\pgfpathcurveto{\pgfqpoint{1.722869in}{1.922943in}}{\pgfqpoint{1.727260in}{1.912344in}}{\pgfqpoint{1.735073in}{1.904531in}}%
\pgfpathcurveto{\pgfqpoint{1.742887in}{1.896717in}}{\pgfqpoint{1.753486in}{1.892327in}}{\pgfqpoint{1.764536in}{1.892327in}}%
\pgfpathlineto{\pgfqpoint{1.764536in}{1.892327in}}%
\pgfpathclose%
\pgfusepath{stroke}%
\end{pgfscope}%
\begin{pgfscope}%
\pgfpathrectangle{\pgfqpoint{0.847223in}{0.554012in}}{\pgfqpoint{6.200000in}{4.530000in}}%
\pgfusepath{clip}%
\pgfsetbuttcap%
\pgfsetroundjoin%
\pgfsetlinewidth{1.003750pt}%
\definecolor{currentstroke}{rgb}{1.000000,0.000000,0.000000}%
\pgfsetstrokecolor{currentstroke}%
\pgfsetdash{}{0pt}%
\pgfpathmoveto{\pgfqpoint{1.769869in}{1.885698in}}%
\pgfpathcurveto{\pgfqpoint{1.780919in}{1.885698in}}{\pgfqpoint{1.791519in}{1.890089in}}{\pgfqpoint{1.799332in}{1.897902in}}%
\pgfpathcurveto{\pgfqpoint{1.807146in}{1.905716in}}{\pgfqpoint{1.811536in}{1.916315in}}{\pgfqpoint{1.811536in}{1.927365in}}%
\pgfpathcurveto{\pgfqpoint{1.811536in}{1.938415in}}{\pgfqpoint{1.807146in}{1.949014in}}{\pgfqpoint{1.799332in}{1.956828in}}%
\pgfpathcurveto{\pgfqpoint{1.791519in}{1.964641in}}{\pgfqpoint{1.780919in}{1.969032in}}{\pgfqpoint{1.769869in}{1.969032in}}%
\pgfpathcurveto{\pgfqpoint{1.758819in}{1.969032in}}{\pgfqpoint{1.748220in}{1.964641in}}{\pgfqpoint{1.740407in}{1.956828in}}%
\pgfpathcurveto{\pgfqpoint{1.732593in}{1.949014in}}{\pgfqpoint{1.728203in}{1.938415in}}{\pgfqpoint{1.728203in}{1.927365in}}%
\pgfpathcurveto{\pgfqpoint{1.728203in}{1.916315in}}{\pgfqpoint{1.732593in}{1.905716in}}{\pgfqpoint{1.740407in}{1.897902in}}%
\pgfpathcurveto{\pgfqpoint{1.748220in}{1.890089in}}{\pgfqpoint{1.758819in}{1.885698in}}{\pgfqpoint{1.769869in}{1.885698in}}%
\pgfpathlineto{\pgfqpoint{1.769869in}{1.885698in}}%
\pgfpathclose%
\pgfusepath{stroke}%
\end{pgfscope}%
\begin{pgfscope}%
\pgfpathrectangle{\pgfqpoint{0.847223in}{0.554012in}}{\pgfqpoint{6.200000in}{4.530000in}}%
\pgfusepath{clip}%
\pgfsetbuttcap%
\pgfsetroundjoin%
\pgfsetlinewidth{1.003750pt}%
\definecolor{currentstroke}{rgb}{1.000000,0.000000,0.000000}%
\pgfsetstrokecolor{currentstroke}%
\pgfsetdash{}{0pt}%
\pgfpathmoveto{\pgfqpoint{1.775203in}{1.879119in}}%
\pgfpathcurveto{\pgfqpoint{1.786253in}{1.879119in}}{\pgfqpoint{1.796852in}{1.883509in}}{\pgfqpoint{1.804665in}{1.891322in}}%
\pgfpathcurveto{\pgfqpoint{1.812479in}{1.899136in}}{\pgfqpoint{1.816869in}{1.909735in}}{\pgfqpoint{1.816869in}{1.920785in}}%
\pgfpathcurveto{\pgfqpoint{1.816869in}{1.931835in}}{\pgfqpoint{1.812479in}{1.942434in}}{\pgfqpoint{1.804665in}{1.950248in}}%
\pgfpathcurveto{\pgfqpoint{1.796852in}{1.958062in}}{\pgfqpoint{1.786253in}{1.962452in}}{\pgfqpoint{1.775203in}{1.962452in}}%
\pgfpathcurveto{\pgfqpoint{1.764152in}{1.962452in}}{\pgfqpoint{1.753553in}{1.958062in}}{\pgfqpoint{1.745740in}{1.950248in}}%
\pgfpathcurveto{\pgfqpoint{1.737926in}{1.942434in}}{\pgfqpoint{1.733536in}{1.931835in}}{\pgfqpoint{1.733536in}{1.920785in}}%
\pgfpathcurveto{\pgfqpoint{1.733536in}{1.909735in}}{\pgfqpoint{1.737926in}{1.899136in}}{\pgfqpoint{1.745740in}{1.891322in}}%
\pgfpathcurveto{\pgfqpoint{1.753553in}{1.883509in}}{\pgfqpoint{1.764152in}{1.879119in}}{\pgfqpoint{1.775203in}{1.879119in}}%
\pgfpathlineto{\pgfqpoint{1.775203in}{1.879119in}}%
\pgfpathclose%
\pgfusepath{stroke}%
\end{pgfscope}%
\begin{pgfscope}%
\pgfpathrectangle{\pgfqpoint{0.847223in}{0.554012in}}{\pgfqpoint{6.200000in}{4.530000in}}%
\pgfusepath{clip}%
\pgfsetbuttcap%
\pgfsetroundjoin%
\pgfsetlinewidth{1.003750pt}%
\definecolor{currentstroke}{rgb}{1.000000,0.000000,0.000000}%
\pgfsetstrokecolor{currentstroke}%
\pgfsetdash{}{0pt}%
\pgfpathmoveto{\pgfqpoint{1.780536in}{1.872587in}}%
\pgfpathcurveto{\pgfqpoint{1.791586in}{1.872587in}}{\pgfqpoint{1.802185in}{1.876977in}}{\pgfqpoint{1.809999in}{1.884791in}}%
\pgfpathcurveto{\pgfqpoint{1.817812in}{1.892605in}}{\pgfqpoint{1.822202in}{1.903204in}}{\pgfqpoint{1.822202in}{1.914254in}}%
\pgfpathcurveto{\pgfqpoint{1.822202in}{1.925304in}}{\pgfqpoint{1.817812in}{1.935903in}}{\pgfqpoint{1.809999in}{1.943716in}}%
\pgfpathcurveto{\pgfqpoint{1.802185in}{1.951530in}}{\pgfqpoint{1.791586in}{1.955920in}}{\pgfqpoint{1.780536in}{1.955920in}}%
\pgfpathcurveto{\pgfqpoint{1.769486in}{1.955920in}}{\pgfqpoint{1.758887in}{1.951530in}}{\pgfqpoint{1.751073in}{1.943716in}}%
\pgfpathcurveto{\pgfqpoint{1.743259in}{1.935903in}}{\pgfqpoint{1.738869in}{1.925304in}}{\pgfqpoint{1.738869in}{1.914254in}}%
\pgfpathcurveto{\pgfqpoint{1.738869in}{1.903204in}}{\pgfqpoint{1.743259in}{1.892605in}}{\pgfqpoint{1.751073in}{1.884791in}}%
\pgfpathcurveto{\pgfqpoint{1.758887in}{1.876977in}}{\pgfqpoint{1.769486in}{1.872587in}}{\pgfqpoint{1.780536in}{1.872587in}}%
\pgfpathlineto{\pgfqpoint{1.780536in}{1.872587in}}%
\pgfpathclose%
\pgfusepath{stroke}%
\end{pgfscope}%
\begin{pgfscope}%
\pgfpathrectangle{\pgfqpoint{0.847223in}{0.554012in}}{\pgfqpoint{6.200000in}{4.530000in}}%
\pgfusepath{clip}%
\pgfsetbuttcap%
\pgfsetroundjoin%
\pgfsetlinewidth{1.003750pt}%
\definecolor{currentstroke}{rgb}{1.000000,0.000000,0.000000}%
\pgfsetstrokecolor{currentstroke}%
\pgfsetdash{}{0pt}%
\pgfpathmoveto{\pgfqpoint{1.785869in}{1.866103in}}%
\pgfpathcurveto{\pgfqpoint{1.796919in}{1.866103in}}{\pgfqpoint{1.807518in}{1.870493in}}{\pgfqpoint{1.815332in}{1.878307in}}%
\pgfpathcurveto{\pgfqpoint{1.823145in}{1.886121in}}{\pgfqpoint{1.827536in}{1.896720in}}{\pgfqpoint{1.827536in}{1.907770in}}%
\pgfpathcurveto{\pgfqpoint{1.827536in}{1.918820in}}{\pgfqpoint{1.823145in}{1.929419in}}{\pgfqpoint{1.815332in}{1.937233in}}%
\pgfpathcurveto{\pgfqpoint{1.807518in}{1.945046in}}{\pgfqpoint{1.796919in}{1.949436in}}{\pgfqpoint{1.785869in}{1.949436in}}%
\pgfpathcurveto{\pgfqpoint{1.774819in}{1.949436in}}{\pgfqpoint{1.764220in}{1.945046in}}{\pgfqpoint{1.756406in}{1.937233in}}%
\pgfpathcurveto{\pgfqpoint{1.748593in}{1.929419in}}{\pgfqpoint{1.744202in}{1.918820in}}{\pgfqpoint{1.744202in}{1.907770in}}%
\pgfpathcurveto{\pgfqpoint{1.744202in}{1.896720in}}{\pgfqpoint{1.748593in}{1.886121in}}{\pgfqpoint{1.756406in}{1.878307in}}%
\pgfpathcurveto{\pgfqpoint{1.764220in}{1.870493in}}{\pgfqpoint{1.774819in}{1.866103in}}{\pgfqpoint{1.785869in}{1.866103in}}%
\pgfpathlineto{\pgfqpoint{1.785869in}{1.866103in}}%
\pgfpathclose%
\pgfusepath{stroke}%
\end{pgfscope}%
\begin{pgfscope}%
\pgfpathrectangle{\pgfqpoint{0.847223in}{0.554012in}}{\pgfqpoint{6.200000in}{4.530000in}}%
\pgfusepath{clip}%
\pgfsetbuttcap%
\pgfsetroundjoin%
\pgfsetlinewidth{1.003750pt}%
\definecolor{currentstroke}{rgb}{1.000000,0.000000,0.000000}%
\pgfsetstrokecolor{currentstroke}%
\pgfsetdash{}{0pt}%
\pgfpathmoveto{\pgfqpoint{1.791202in}{1.859666in}}%
\pgfpathcurveto{\pgfqpoint{1.802252in}{1.859666in}}{\pgfqpoint{1.812851in}{1.864057in}}{\pgfqpoint{1.820665in}{1.871870in}}%
\pgfpathcurveto{\pgfqpoint{1.828479in}{1.879684in}}{\pgfqpoint{1.832869in}{1.890283in}}{\pgfqpoint{1.832869in}{1.901333in}}%
\pgfpathcurveto{\pgfqpoint{1.832869in}{1.912383in}}{\pgfqpoint{1.828479in}{1.922982in}}{\pgfqpoint{1.820665in}{1.930796in}}%
\pgfpathcurveto{\pgfqpoint{1.812851in}{1.938609in}}{\pgfqpoint{1.802252in}{1.943000in}}{\pgfqpoint{1.791202in}{1.943000in}}%
\pgfpathcurveto{\pgfqpoint{1.780152in}{1.943000in}}{\pgfqpoint{1.769553in}{1.938609in}}{\pgfqpoint{1.761739in}{1.930796in}}%
\pgfpathcurveto{\pgfqpoint{1.753926in}{1.922982in}}{\pgfqpoint{1.749536in}{1.912383in}}{\pgfqpoint{1.749536in}{1.901333in}}%
\pgfpathcurveto{\pgfqpoint{1.749536in}{1.890283in}}{\pgfqpoint{1.753926in}{1.879684in}}{\pgfqpoint{1.761739in}{1.871870in}}%
\pgfpathcurveto{\pgfqpoint{1.769553in}{1.864057in}}{\pgfqpoint{1.780152in}{1.859666in}}{\pgfqpoint{1.791202in}{1.859666in}}%
\pgfpathlineto{\pgfqpoint{1.791202in}{1.859666in}}%
\pgfpathclose%
\pgfusepath{stroke}%
\end{pgfscope}%
\begin{pgfscope}%
\pgfpathrectangle{\pgfqpoint{0.847223in}{0.554012in}}{\pgfqpoint{6.200000in}{4.530000in}}%
\pgfusepath{clip}%
\pgfsetbuttcap%
\pgfsetroundjoin%
\pgfsetlinewidth{1.003750pt}%
\definecolor{currentstroke}{rgb}{1.000000,0.000000,0.000000}%
\pgfsetstrokecolor{currentstroke}%
\pgfsetdash{}{0pt}%
\pgfpathmoveto{\pgfqpoint{1.796535in}{1.853276in}}%
\pgfpathcurveto{\pgfqpoint{1.807586in}{1.853276in}}{\pgfqpoint{1.818185in}{1.857667in}}{\pgfqpoint{1.825998in}{1.865480in}}%
\pgfpathcurveto{\pgfqpoint{1.833812in}{1.873294in}}{\pgfqpoint{1.838202in}{1.883893in}}{\pgfqpoint{1.838202in}{1.894943in}}%
\pgfpathcurveto{\pgfqpoint{1.838202in}{1.905993in}}{\pgfqpoint{1.833812in}{1.916592in}}{\pgfqpoint{1.825998in}{1.924406in}}%
\pgfpathcurveto{\pgfqpoint{1.818185in}{1.932219in}}{\pgfqpoint{1.807586in}{1.936610in}}{\pgfqpoint{1.796535in}{1.936610in}}%
\pgfpathcurveto{\pgfqpoint{1.785485in}{1.936610in}}{\pgfqpoint{1.774886in}{1.932219in}}{\pgfqpoint{1.767073in}{1.924406in}}%
\pgfpathcurveto{\pgfqpoint{1.759259in}{1.916592in}}{\pgfqpoint{1.754869in}{1.905993in}}{\pgfqpoint{1.754869in}{1.894943in}}%
\pgfpathcurveto{\pgfqpoint{1.754869in}{1.883893in}}{\pgfqpoint{1.759259in}{1.873294in}}{\pgfqpoint{1.767073in}{1.865480in}}%
\pgfpathcurveto{\pgfqpoint{1.774886in}{1.857667in}}{\pgfqpoint{1.785485in}{1.853276in}}{\pgfqpoint{1.796535in}{1.853276in}}%
\pgfpathlineto{\pgfqpoint{1.796535in}{1.853276in}}%
\pgfpathclose%
\pgfusepath{stroke}%
\end{pgfscope}%
\begin{pgfscope}%
\pgfpathrectangle{\pgfqpoint{0.847223in}{0.554012in}}{\pgfqpoint{6.200000in}{4.530000in}}%
\pgfusepath{clip}%
\pgfsetbuttcap%
\pgfsetroundjoin%
\pgfsetlinewidth{1.003750pt}%
\definecolor{currentstroke}{rgb}{1.000000,0.000000,0.000000}%
\pgfsetstrokecolor{currentstroke}%
\pgfsetdash{}{0pt}%
\pgfpathmoveto{\pgfqpoint{1.801869in}{1.846932in}}%
\pgfpathcurveto{\pgfqpoint{1.812919in}{1.846932in}}{\pgfqpoint{1.823518in}{1.851323in}}{\pgfqpoint{1.831331in}{1.859136in}}%
\pgfpathcurveto{\pgfqpoint{1.839145in}{1.866950in}}{\pgfqpoint{1.843535in}{1.877549in}}{\pgfqpoint{1.843535in}{1.888599in}}%
\pgfpathcurveto{\pgfqpoint{1.843535in}{1.899649in}}{\pgfqpoint{1.839145in}{1.910248in}}{\pgfqpoint{1.831331in}{1.918062in}}%
\pgfpathcurveto{\pgfqpoint{1.823518in}{1.925875in}}{\pgfqpoint{1.812919in}{1.930266in}}{\pgfqpoint{1.801869in}{1.930266in}}%
\pgfpathcurveto{\pgfqpoint{1.790819in}{1.930266in}}{\pgfqpoint{1.780219in}{1.925875in}}{\pgfqpoint{1.772406in}{1.918062in}}%
\pgfpathcurveto{\pgfqpoint{1.764592in}{1.910248in}}{\pgfqpoint{1.760202in}{1.899649in}}{\pgfqpoint{1.760202in}{1.888599in}}%
\pgfpathcurveto{\pgfqpoint{1.760202in}{1.877549in}}{\pgfqpoint{1.764592in}{1.866950in}}{\pgfqpoint{1.772406in}{1.859136in}}%
\pgfpathcurveto{\pgfqpoint{1.780219in}{1.851323in}}{\pgfqpoint{1.790819in}{1.846932in}}{\pgfqpoint{1.801869in}{1.846932in}}%
\pgfpathlineto{\pgfqpoint{1.801869in}{1.846932in}}%
\pgfpathclose%
\pgfusepath{stroke}%
\end{pgfscope}%
\begin{pgfscope}%
\pgfpathrectangle{\pgfqpoint{0.847223in}{0.554012in}}{\pgfqpoint{6.200000in}{4.530000in}}%
\pgfusepath{clip}%
\pgfsetbuttcap%
\pgfsetroundjoin%
\pgfsetlinewidth{1.003750pt}%
\definecolor{currentstroke}{rgb}{1.000000,0.000000,0.000000}%
\pgfsetstrokecolor{currentstroke}%
\pgfsetdash{}{0pt}%
\pgfpathmoveto{\pgfqpoint{1.807202in}{1.840634in}}%
\pgfpathcurveto{\pgfqpoint{1.818252in}{1.840634in}}{\pgfqpoint{1.828851in}{1.845024in}}{\pgfqpoint{1.836665in}{1.852838in}}%
\pgfpathcurveto{\pgfqpoint{1.844478in}{1.860651in}}{\pgfqpoint{1.848869in}{1.871250in}}{\pgfqpoint{1.848869in}{1.882301in}}%
\pgfpathcurveto{\pgfqpoint{1.848869in}{1.893351in}}{\pgfqpoint{1.844478in}{1.903950in}}{\pgfqpoint{1.836665in}{1.911763in}}%
\pgfpathcurveto{\pgfqpoint{1.828851in}{1.919577in}}{\pgfqpoint{1.818252in}{1.923967in}}{\pgfqpoint{1.807202in}{1.923967in}}%
\pgfpathcurveto{\pgfqpoint{1.796152in}{1.923967in}}{\pgfqpoint{1.785553in}{1.919577in}}{\pgfqpoint{1.777739in}{1.911763in}}%
\pgfpathcurveto{\pgfqpoint{1.769925in}{1.903950in}}{\pgfqpoint{1.765535in}{1.893351in}}{\pgfqpoint{1.765535in}{1.882301in}}%
\pgfpathcurveto{\pgfqpoint{1.765535in}{1.871250in}}{\pgfqpoint{1.769925in}{1.860651in}}{\pgfqpoint{1.777739in}{1.852838in}}%
\pgfpathcurveto{\pgfqpoint{1.785553in}{1.845024in}}{\pgfqpoint{1.796152in}{1.840634in}}{\pgfqpoint{1.807202in}{1.840634in}}%
\pgfpathlineto{\pgfqpoint{1.807202in}{1.840634in}}%
\pgfpathclose%
\pgfusepath{stroke}%
\end{pgfscope}%
\begin{pgfscope}%
\pgfpathrectangle{\pgfqpoint{0.847223in}{0.554012in}}{\pgfqpoint{6.200000in}{4.530000in}}%
\pgfusepath{clip}%
\pgfsetbuttcap%
\pgfsetroundjoin%
\pgfsetlinewidth{1.003750pt}%
\definecolor{currentstroke}{rgb}{1.000000,0.000000,0.000000}%
\pgfsetstrokecolor{currentstroke}%
\pgfsetdash{}{0pt}%
\pgfpathmoveto{\pgfqpoint{1.812535in}{1.834381in}}%
\pgfpathcurveto{\pgfqpoint{1.823585in}{1.834381in}}{\pgfqpoint{1.834184in}{1.838771in}}{\pgfqpoint{1.841998in}{1.846585in}}%
\pgfpathcurveto{\pgfqpoint{1.849811in}{1.854398in}}{\pgfqpoint{1.854202in}{1.864997in}}{\pgfqpoint{1.854202in}{1.876047in}}%
\pgfpathcurveto{\pgfqpoint{1.854202in}{1.887097in}}{\pgfqpoint{1.849811in}{1.897697in}}{\pgfqpoint{1.841998in}{1.905510in}}%
\pgfpathcurveto{\pgfqpoint{1.834184in}{1.913324in}}{\pgfqpoint{1.823585in}{1.917714in}}{\pgfqpoint{1.812535in}{1.917714in}}%
\pgfpathcurveto{\pgfqpoint{1.801485in}{1.917714in}}{\pgfqpoint{1.790886in}{1.913324in}}{\pgfqpoint{1.783072in}{1.905510in}}%
\pgfpathcurveto{\pgfqpoint{1.775259in}{1.897697in}}{\pgfqpoint{1.770868in}{1.887097in}}{\pgfqpoint{1.770868in}{1.876047in}}%
\pgfpathcurveto{\pgfqpoint{1.770868in}{1.864997in}}{\pgfqpoint{1.775259in}{1.854398in}}{\pgfqpoint{1.783072in}{1.846585in}}%
\pgfpathcurveto{\pgfqpoint{1.790886in}{1.838771in}}{\pgfqpoint{1.801485in}{1.834381in}}{\pgfqpoint{1.812535in}{1.834381in}}%
\pgfpathlineto{\pgfqpoint{1.812535in}{1.834381in}}%
\pgfpathclose%
\pgfusepath{stroke}%
\end{pgfscope}%
\begin{pgfscope}%
\pgfpathrectangle{\pgfqpoint{0.847223in}{0.554012in}}{\pgfqpoint{6.200000in}{4.530000in}}%
\pgfusepath{clip}%
\pgfsetbuttcap%
\pgfsetroundjoin%
\pgfsetlinewidth{1.003750pt}%
\definecolor{currentstroke}{rgb}{1.000000,0.000000,0.000000}%
\pgfsetstrokecolor{currentstroke}%
\pgfsetdash{}{0pt}%
\pgfpathmoveto{\pgfqpoint{1.817868in}{1.828172in}}%
\pgfpathcurveto{\pgfqpoint{1.828918in}{1.828172in}}{\pgfqpoint{1.839517in}{1.832562in}}{\pgfqpoint{1.847331in}{1.840376in}}%
\pgfpathcurveto{\pgfqpoint{1.855145in}{1.848190in}}{\pgfqpoint{1.859535in}{1.858789in}}{\pgfqpoint{1.859535in}{1.869839in}}%
\pgfpathcurveto{\pgfqpoint{1.859535in}{1.880889in}}{\pgfqpoint{1.855145in}{1.891488in}}{\pgfqpoint{1.847331in}{1.899302in}}%
\pgfpathcurveto{\pgfqpoint{1.839517in}{1.907115in}}{\pgfqpoint{1.828918in}{1.911505in}}{\pgfqpoint{1.817868in}{1.911505in}}%
\pgfpathcurveto{\pgfqpoint{1.806818in}{1.911505in}}{\pgfqpoint{1.796219in}{1.907115in}}{\pgfqpoint{1.788406in}{1.899302in}}%
\pgfpathcurveto{\pgfqpoint{1.780592in}{1.891488in}}{\pgfqpoint{1.776202in}{1.880889in}}{\pgfqpoint{1.776202in}{1.869839in}}%
\pgfpathcurveto{\pgfqpoint{1.776202in}{1.858789in}}{\pgfqpoint{1.780592in}{1.848190in}}{\pgfqpoint{1.788406in}{1.840376in}}%
\pgfpathcurveto{\pgfqpoint{1.796219in}{1.832562in}}{\pgfqpoint{1.806818in}{1.828172in}}{\pgfqpoint{1.817868in}{1.828172in}}%
\pgfpathlineto{\pgfqpoint{1.817868in}{1.828172in}}%
\pgfpathclose%
\pgfusepath{stroke}%
\end{pgfscope}%
\begin{pgfscope}%
\pgfpathrectangle{\pgfqpoint{0.847223in}{0.554012in}}{\pgfqpoint{6.200000in}{4.530000in}}%
\pgfusepath{clip}%
\pgfsetbuttcap%
\pgfsetroundjoin%
\pgfsetlinewidth{1.003750pt}%
\definecolor{currentstroke}{rgb}{1.000000,0.000000,0.000000}%
\pgfsetstrokecolor{currentstroke}%
\pgfsetdash{}{0pt}%
\pgfpathmoveto{\pgfqpoint{1.823202in}{1.822008in}}%
\pgfpathcurveto{\pgfqpoint{1.834252in}{1.822008in}}{\pgfqpoint{1.844851in}{1.826398in}}{\pgfqpoint{1.852664in}{1.834212in}}%
\pgfpathcurveto{\pgfqpoint{1.860478in}{1.842025in}}{\pgfqpoint{1.864868in}{1.852624in}}{\pgfqpoint{1.864868in}{1.863674in}}%
\pgfpathcurveto{\pgfqpoint{1.864868in}{1.874724in}}{\pgfqpoint{1.860478in}{1.885323in}}{\pgfqpoint{1.852664in}{1.893137in}}%
\pgfpathcurveto{\pgfqpoint{1.844851in}{1.900951in}}{\pgfqpoint{1.834252in}{1.905341in}}{\pgfqpoint{1.823202in}{1.905341in}}%
\pgfpathcurveto{\pgfqpoint{1.812151in}{1.905341in}}{\pgfqpoint{1.801552in}{1.900951in}}{\pgfqpoint{1.793739in}{1.893137in}}%
\pgfpathcurveto{\pgfqpoint{1.785925in}{1.885323in}}{\pgfqpoint{1.781535in}{1.874724in}}{\pgfqpoint{1.781535in}{1.863674in}}%
\pgfpathcurveto{\pgfqpoint{1.781535in}{1.852624in}}{\pgfqpoint{1.785925in}{1.842025in}}{\pgfqpoint{1.793739in}{1.834212in}}%
\pgfpathcurveto{\pgfqpoint{1.801552in}{1.826398in}}{\pgfqpoint{1.812151in}{1.822008in}}{\pgfqpoint{1.823202in}{1.822008in}}%
\pgfpathlineto{\pgfqpoint{1.823202in}{1.822008in}}%
\pgfpathclose%
\pgfusepath{stroke}%
\end{pgfscope}%
\begin{pgfscope}%
\pgfpathrectangle{\pgfqpoint{0.847223in}{0.554012in}}{\pgfqpoint{6.200000in}{4.530000in}}%
\pgfusepath{clip}%
\pgfsetbuttcap%
\pgfsetroundjoin%
\pgfsetlinewidth{1.003750pt}%
\definecolor{currentstroke}{rgb}{1.000000,0.000000,0.000000}%
\pgfsetstrokecolor{currentstroke}%
\pgfsetdash{}{0pt}%
\pgfpathmoveto{\pgfqpoint{1.828535in}{1.815887in}}%
\pgfpathcurveto{\pgfqpoint{1.839585in}{1.815887in}}{\pgfqpoint{1.850184in}{1.820277in}}{\pgfqpoint{1.857998in}{1.828091in}}%
\pgfpathcurveto{\pgfqpoint{1.865811in}{1.835904in}}{\pgfqpoint{1.870201in}{1.846503in}}{\pgfqpoint{1.870201in}{1.857554in}}%
\pgfpathcurveto{\pgfqpoint{1.870201in}{1.868604in}}{\pgfqpoint{1.865811in}{1.879203in}}{\pgfqpoint{1.857998in}{1.887016in}}%
\pgfpathcurveto{\pgfqpoint{1.850184in}{1.894830in}}{\pgfqpoint{1.839585in}{1.899220in}}{\pgfqpoint{1.828535in}{1.899220in}}%
\pgfpathcurveto{\pgfqpoint{1.817485in}{1.899220in}}{\pgfqpoint{1.806886in}{1.894830in}}{\pgfqpoint{1.799072in}{1.887016in}}%
\pgfpathcurveto{\pgfqpoint{1.791258in}{1.879203in}}{\pgfqpoint{1.786868in}{1.868604in}}{\pgfqpoint{1.786868in}{1.857554in}}%
\pgfpathcurveto{\pgfqpoint{1.786868in}{1.846503in}}{\pgfqpoint{1.791258in}{1.835904in}}{\pgfqpoint{1.799072in}{1.828091in}}%
\pgfpathcurveto{\pgfqpoint{1.806886in}{1.820277in}}{\pgfqpoint{1.817485in}{1.815887in}}{\pgfqpoint{1.828535in}{1.815887in}}%
\pgfpathlineto{\pgfqpoint{1.828535in}{1.815887in}}%
\pgfpathclose%
\pgfusepath{stroke}%
\end{pgfscope}%
\begin{pgfscope}%
\pgfpathrectangle{\pgfqpoint{0.847223in}{0.554012in}}{\pgfqpoint{6.200000in}{4.530000in}}%
\pgfusepath{clip}%
\pgfsetbuttcap%
\pgfsetroundjoin%
\pgfsetlinewidth{1.003750pt}%
\definecolor{currentstroke}{rgb}{1.000000,0.000000,0.000000}%
\pgfsetstrokecolor{currentstroke}%
\pgfsetdash{}{0pt}%
\pgfpathmoveto{\pgfqpoint{1.833868in}{1.809809in}}%
\pgfpathcurveto{\pgfqpoint{1.844918in}{1.809809in}}{\pgfqpoint{1.855517in}{1.814200in}}{\pgfqpoint{1.863331in}{1.822013in}}%
\pgfpathcurveto{\pgfqpoint{1.871144in}{1.829827in}}{\pgfqpoint{1.875535in}{1.840426in}}{\pgfqpoint{1.875535in}{1.851476in}}%
\pgfpathcurveto{\pgfqpoint{1.875535in}{1.862526in}}{\pgfqpoint{1.871144in}{1.873125in}}{\pgfqpoint{1.863331in}{1.880939in}}%
\pgfpathcurveto{\pgfqpoint{1.855517in}{1.888752in}}{\pgfqpoint{1.844918in}{1.893143in}}{\pgfqpoint{1.833868in}{1.893143in}}%
\pgfpathcurveto{\pgfqpoint{1.822818in}{1.893143in}}{\pgfqpoint{1.812219in}{1.888752in}}{\pgfqpoint{1.804405in}{1.880939in}}%
\pgfpathcurveto{\pgfqpoint{1.796592in}{1.873125in}}{\pgfqpoint{1.792201in}{1.862526in}}{\pgfqpoint{1.792201in}{1.851476in}}%
\pgfpathcurveto{\pgfqpoint{1.792201in}{1.840426in}}{\pgfqpoint{1.796592in}{1.829827in}}{\pgfqpoint{1.804405in}{1.822013in}}%
\pgfpathcurveto{\pgfqpoint{1.812219in}{1.814200in}}{\pgfqpoint{1.822818in}{1.809809in}}{\pgfqpoint{1.833868in}{1.809809in}}%
\pgfpathlineto{\pgfqpoint{1.833868in}{1.809809in}}%
\pgfpathclose%
\pgfusepath{stroke}%
\end{pgfscope}%
\begin{pgfscope}%
\pgfpathrectangle{\pgfqpoint{0.847223in}{0.554012in}}{\pgfqpoint{6.200000in}{4.530000in}}%
\pgfusepath{clip}%
\pgfsetbuttcap%
\pgfsetroundjoin%
\pgfsetlinewidth{1.003750pt}%
\definecolor{currentstroke}{rgb}{1.000000,0.000000,0.000000}%
\pgfsetstrokecolor{currentstroke}%
\pgfsetdash{}{0pt}%
\pgfpathmoveto{\pgfqpoint{1.839201in}{1.803775in}}%
\pgfpathcurveto{\pgfqpoint{1.850251in}{1.803775in}}{\pgfqpoint{1.860850in}{1.808165in}}{\pgfqpoint{1.868664in}{1.815979in}}%
\pgfpathcurveto{\pgfqpoint{1.876478in}{1.823792in}}{\pgfqpoint{1.880868in}{1.834391in}}{\pgfqpoint{1.880868in}{1.845441in}}%
\pgfpathcurveto{\pgfqpoint{1.880868in}{1.856491in}}{\pgfqpoint{1.876478in}{1.867091in}}{\pgfqpoint{1.868664in}{1.874904in}}%
\pgfpathcurveto{\pgfqpoint{1.860850in}{1.882718in}}{\pgfqpoint{1.850251in}{1.887108in}}{\pgfqpoint{1.839201in}{1.887108in}}%
\pgfpathcurveto{\pgfqpoint{1.828151in}{1.887108in}}{\pgfqpoint{1.817552in}{1.882718in}}{\pgfqpoint{1.809738in}{1.874904in}}%
\pgfpathcurveto{\pgfqpoint{1.801925in}{1.867091in}}{\pgfqpoint{1.797535in}{1.856491in}}{\pgfqpoint{1.797535in}{1.845441in}}%
\pgfpathcurveto{\pgfqpoint{1.797535in}{1.834391in}}{\pgfqpoint{1.801925in}{1.823792in}}{\pgfqpoint{1.809738in}{1.815979in}}%
\pgfpathcurveto{\pgfqpoint{1.817552in}{1.808165in}}{\pgfqpoint{1.828151in}{1.803775in}}{\pgfqpoint{1.839201in}{1.803775in}}%
\pgfpathlineto{\pgfqpoint{1.839201in}{1.803775in}}%
\pgfpathclose%
\pgfusepath{stroke}%
\end{pgfscope}%
\begin{pgfscope}%
\pgfpathrectangle{\pgfqpoint{0.847223in}{0.554012in}}{\pgfqpoint{6.200000in}{4.530000in}}%
\pgfusepath{clip}%
\pgfsetbuttcap%
\pgfsetroundjoin%
\pgfsetlinewidth{1.003750pt}%
\definecolor{currentstroke}{rgb}{1.000000,0.000000,0.000000}%
\pgfsetstrokecolor{currentstroke}%
\pgfsetdash{}{0pt}%
\pgfpathmoveto{\pgfqpoint{1.844534in}{1.797782in}}%
\pgfpathcurveto{\pgfqpoint{1.855585in}{1.797782in}}{\pgfqpoint{1.866184in}{1.802173in}}{\pgfqpoint{1.873997in}{1.809986in}}%
\pgfpathcurveto{\pgfqpoint{1.881811in}{1.817800in}}{\pgfqpoint{1.886201in}{1.828399in}}{\pgfqpoint{1.886201in}{1.839449in}}%
\pgfpathcurveto{\pgfqpoint{1.886201in}{1.850499in}}{\pgfqpoint{1.881811in}{1.861098in}}{\pgfqpoint{1.873997in}{1.868912in}}%
\pgfpathcurveto{\pgfqpoint{1.866184in}{1.876725in}}{\pgfqpoint{1.855585in}{1.881116in}}{\pgfqpoint{1.844534in}{1.881116in}}%
\pgfpathcurveto{\pgfqpoint{1.833484in}{1.881116in}}{\pgfqpoint{1.822885in}{1.876725in}}{\pgfqpoint{1.815072in}{1.868912in}}%
\pgfpathcurveto{\pgfqpoint{1.807258in}{1.861098in}}{\pgfqpoint{1.802868in}{1.850499in}}{\pgfqpoint{1.802868in}{1.839449in}}%
\pgfpathcurveto{\pgfqpoint{1.802868in}{1.828399in}}{\pgfqpoint{1.807258in}{1.817800in}}{\pgfqpoint{1.815072in}{1.809986in}}%
\pgfpathcurveto{\pgfqpoint{1.822885in}{1.802173in}}{\pgfqpoint{1.833484in}{1.797782in}}{\pgfqpoint{1.844534in}{1.797782in}}%
\pgfpathlineto{\pgfqpoint{1.844534in}{1.797782in}}%
\pgfpathclose%
\pgfusepath{stroke}%
\end{pgfscope}%
\begin{pgfscope}%
\pgfpathrectangle{\pgfqpoint{0.847223in}{0.554012in}}{\pgfqpoint{6.200000in}{4.530000in}}%
\pgfusepath{clip}%
\pgfsetbuttcap%
\pgfsetroundjoin%
\pgfsetlinewidth{1.003750pt}%
\definecolor{currentstroke}{rgb}{1.000000,0.000000,0.000000}%
\pgfsetstrokecolor{currentstroke}%
\pgfsetdash{}{0pt}%
\pgfpathmoveto{\pgfqpoint{1.849868in}{1.791832in}}%
\pgfpathcurveto{\pgfqpoint{1.860918in}{1.791832in}}{\pgfqpoint{1.871517in}{1.796222in}}{\pgfqpoint{1.879330in}{1.804036in}}%
\pgfpathcurveto{\pgfqpoint{1.887144in}{1.811849in}}{\pgfqpoint{1.891534in}{1.822448in}}{\pgfqpoint{1.891534in}{1.833498in}}%
\pgfpathcurveto{\pgfqpoint{1.891534in}{1.844549in}}{\pgfqpoint{1.887144in}{1.855148in}}{\pgfqpoint{1.879330in}{1.862961in}}%
\pgfpathcurveto{\pgfqpoint{1.871517in}{1.870775in}}{\pgfqpoint{1.860918in}{1.875165in}}{\pgfqpoint{1.849868in}{1.875165in}}%
\pgfpathcurveto{\pgfqpoint{1.838817in}{1.875165in}}{\pgfqpoint{1.828218in}{1.870775in}}{\pgfqpoint{1.820405in}{1.862961in}}%
\pgfpathcurveto{\pgfqpoint{1.812591in}{1.855148in}}{\pgfqpoint{1.808201in}{1.844549in}}{\pgfqpoint{1.808201in}{1.833498in}}%
\pgfpathcurveto{\pgfqpoint{1.808201in}{1.822448in}}{\pgfqpoint{1.812591in}{1.811849in}}{\pgfqpoint{1.820405in}{1.804036in}}%
\pgfpathcurveto{\pgfqpoint{1.828218in}{1.796222in}}{\pgfqpoint{1.838817in}{1.791832in}}{\pgfqpoint{1.849868in}{1.791832in}}%
\pgfpathlineto{\pgfqpoint{1.849868in}{1.791832in}}%
\pgfpathclose%
\pgfusepath{stroke}%
\end{pgfscope}%
\begin{pgfscope}%
\pgfpathrectangle{\pgfqpoint{0.847223in}{0.554012in}}{\pgfqpoint{6.200000in}{4.530000in}}%
\pgfusepath{clip}%
\pgfsetbuttcap%
\pgfsetroundjoin%
\pgfsetlinewidth{1.003750pt}%
\definecolor{currentstroke}{rgb}{1.000000,0.000000,0.000000}%
\pgfsetstrokecolor{currentstroke}%
\pgfsetdash{}{0pt}%
\pgfpathmoveto{\pgfqpoint{1.855201in}{1.785923in}}%
\pgfpathcurveto{\pgfqpoint{1.866251in}{1.785923in}}{\pgfqpoint{1.876850in}{1.790313in}}{\pgfqpoint{1.884664in}{1.798127in}}%
\pgfpathcurveto{\pgfqpoint{1.892477in}{1.805940in}}{\pgfqpoint{1.896867in}{1.816539in}}{\pgfqpoint{1.896867in}{1.827589in}}%
\pgfpathcurveto{\pgfqpoint{1.896867in}{1.838640in}}{\pgfqpoint{1.892477in}{1.849239in}}{\pgfqpoint{1.884664in}{1.857052in}}%
\pgfpathcurveto{\pgfqpoint{1.876850in}{1.864866in}}{\pgfqpoint{1.866251in}{1.869256in}}{\pgfqpoint{1.855201in}{1.869256in}}%
\pgfpathcurveto{\pgfqpoint{1.844151in}{1.869256in}}{\pgfqpoint{1.833552in}{1.864866in}}{\pgfqpoint{1.825738in}{1.857052in}}%
\pgfpathcurveto{\pgfqpoint{1.817924in}{1.849239in}}{\pgfqpoint{1.813534in}{1.838640in}}{\pgfqpoint{1.813534in}{1.827589in}}%
\pgfpathcurveto{\pgfqpoint{1.813534in}{1.816539in}}{\pgfqpoint{1.817924in}{1.805940in}}{\pgfqpoint{1.825738in}{1.798127in}}%
\pgfpathcurveto{\pgfqpoint{1.833552in}{1.790313in}}{\pgfqpoint{1.844151in}{1.785923in}}{\pgfqpoint{1.855201in}{1.785923in}}%
\pgfpathlineto{\pgfqpoint{1.855201in}{1.785923in}}%
\pgfpathclose%
\pgfusepath{stroke}%
\end{pgfscope}%
\begin{pgfscope}%
\pgfpathrectangle{\pgfqpoint{0.847223in}{0.554012in}}{\pgfqpoint{6.200000in}{4.530000in}}%
\pgfusepath{clip}%
\pgfsetbuttcap%
\pgfsetroundjoin%
\pgfsetlinewidth{1.003750pt}%
\definecolor{currentstroke}{rgb}{1.000000,0.000000,0.000000}%
\pgfsetstrokecolor{currentstroke}%
\pgfsetdash{}{0pt}%
\pgfpathmoveto{\pgfqpoint{1.860534in}{1.780055in}}%
\pgfpathcurveto{\pgfqpoint{1.871584in}{1.780055in}}{\pgfqpoint{1.882183in}{1.784445in}}{\pgfqpoint{1.889997in}{1.792259in}}%
\pgfpathcurveto{\pgfqpoint{1.897810in}{1.800072in}}{\pgfqpoint{1.902201in}{1.810671in}}{\pgfqpoint{1.902201in}{1.821721in}}%
\pgfpathcurveto{\pgfqpoint{1.902201in}{1.832771in}}{\pgfqpoint{1.897810in}{1.843371in}}{\pgfqpoint{1.889997in}{1.851184in}}%
\pgfpathcurveto{\pgfqpoint{1.882183in}{1.858998in}}{\pgfqpoint{1.871584in}{1.863388in}}{\pgfqpoint{1.860534in}{1.863388in}}%
\pgfpathcurveto{\pgfqpoint{1.849484in}{1.863388in}}{\pgfqpoint{1.838885in}{1.858998in}}{\pgfqpoint{1.831071in}{1.851184in}}%
\pgfpathcurveto{\pgfqpoint{1.823258in}{1.843371in}}{\pgfqpoint{1.818867in}{1.832771in}}{\pgfqpoint{1.818867in}{1.821721in}}%
\pgfpathcurveto{\pgfqpoint{1.818867in}{1.810671in}}{\pgfqpoint{1.823258in}{1.800072in}}{\pgfqpoint{1.831071in}{1.792259in}}%
\pgfpathcurveto{\pgfqpoint{1.838885in}{1.784445in}}{\pgfqpoint{1.849484in}{1.780055in}}{\pgfqpoint{1.860534in}{1.780055in}}%
\pgfpathlineto{\pgfqpoint{1.860534in}{1.780055in}}%
\pgfpathclose%
\pgfusepath{stroke}%
\end{pgfscope}%
\begin{pgfscope}%
\pgfpathrectangle{\pgfqpoint{0.847223in}{0.554012in}}{\pgfqpoint{6.200000in}{4.530000in}}%
\pgfusepath{clip}%
\pgfsetbuttcap%
\pgfsetroundjoin%
\pgfsetlinewidth{1.003750pt}%
\definecolor{currentstroke}{rgb}{1.000000,0.000000,0.000000}%
\pgfsetstrokecolor{currentstroke}%
\pgfsetdash{}{0pt}%
\pgfpathmoveto{\pgfqpoint{1.865867in}{1.774227in}}%
\pgfpathcurveto{\pgfqpoint{1.876917in}{1.774227in}}{\pgfqpoint{1.887516in}{1.778617in}}{\pgfqpoint{1.895330in}{1.786431in}}%
\pgfpathcurveto{\pgfqpoint{1.903144in}{1.794245in}}{\pgfqpoint{1.907534in}{1.804844in}}{\pgfqpoint{1.907534in}{1.815894in}}%
\pgfpathcurveto{\pgfqpoint{1.907534in}{1.826944in}}{\pgfqpoint{1.903144in}{1.837543in}}{\pgfqpoint{1.895330in}{1.845357in}}%
\pgfpathcurveto{\pgfqpoint{1.887516in}{1.853170in}}{\pgfqpoint{1.876917in}{1.857561in}}{\pgfqpoint{1.865867in}{1.857561in}}%
\pgfpathcurveto{\pgfqpoint{1.854817in}{1.857561in}}{\pgfqpoint{1.844218in}{1.853170in}}{\pgfqpoint{1.836404in}{1.845357in}}%
\pgfpathcurveto{\pgfqpoint{1.828591in}{1.837543in}}{\pgfqpoint{1.824201in}{1.826944in}}{\pgfqpoint{1.824201in}{1.815894in}}%
\pgfpathcurveto{\pgfqpoint{1.824201in}{1.804844in}}{\pgfqpoint{1.828591in}{1.794245in}}{\pgfqpoint{1.836404in}{1.786431in}}%
\pgfpathcurveto{\pgfqpoint{1.844218in}{1.778617in}}{\pgfqpoint{1.854817in}{1.774227in}}{\pgfqpoint{1.865867in}{1.774227in}}%
\pgfpathlineto{\pgfqpoint{1.865867in}{1.774227in}}%
\pgfpathclose%
\pgfusepath{stroke}%
\end{pgfscope}%
\begin{pgfscope}%
\pgfpathrectangle{\pgfqpoint{0.847223in}{0.554012in}}{\pgfqpoint{6.200000in}{4.530000in}}%
\pgfusepath{clip}%
\pgfsetbuttcap%
\pgfsetroundjoin%
\pgfsetlinewidth{1.003750pt}%
\definecolor{currentstroke}{rgb}{1.000000,0.000000,0.000000}%
\pgfsetstrokecolor{currentstroke}%
\pgfsetdash{}{0pt}%
\pgfpathmoveto{\pgfqpoint{1.871200in}{1.768440in}}%
\pgfpathcurveto{\pgfqpoint{1.882251in}{1.768440in}}{\pgfqpoint{1.892850in}{1.772830in}}{\pgfqpoint{1.900663in}{1.780644in}}%
\pgfpathcurveto{\pgfqpoint{1.908477in}{1.788457in}}{\pgfqpoint{1.912867in}{1.799056in}}{\pgfqpoint{1.912867in}{1.810107in}}%
\pgfpathcurveto{\pgfqpoint{1.912867in}{1.821157in}}{\pgfqpoint{1.908477in}{1.831756in}}{\pgfqpoint{1.900663in}{1.839569in}}%
\pgfpathcurveto{\pgfqpoint{1.892850in}{1.847383in}}{\pgfqpoint{1.882251in}{1.851773in}}{\pgfqpoint{1.871200in}{1.851773in}}%
\pgfpathcurveto{\pgfqpoint{1.860150in}{1.851773in}}{\pgfqpoint{1.849551in}{1.847383in}}{\pgfqpoint{1.841738in}{1.839569in}}%
\pgfpathcurveto{\pgfqpoint{1.833924in}{1.831756in}}{\pgfqpoint{1.829534in}{1.821157in}}{\pgfqpoint{1.829534in}{1.810107in}}%
\pgfpathcurveto{\pgfqpoint{1.829534in}{1.799056in}}{\pgfqpoint{1.833924in}{1.788457in}}{\pgfqpoint{1.841738in}{1.780644in}}%
\pgfpathcurveto{\pgfqpoint{1.849551in}{1.772830in}}{\pgfqpoint{1.860150in}{1.768440in}}{\pgfqpoint{1.871200in}{1.768440in}}%
\pgfpathlineto{\pgfqpoint{1.871200in}{1.768440in}}%
\pgfpathclose%
\pgfusepath{stroke}%
\end{pgfscope}%
\begin{pgfscope}%
\pgfpathrectangle{\pgfqpoint{0.847223in}{0.554012in}}{\pgfqpoint{6.200000in}{4.530000in}}%
\pgfusepath{clip}%
\pgfsetbuttcap%
\pgfsetroundjoin%
\pgfsetlinewidth{1.003750pt}%
\definecolor{currentstroke}{rgb}{1.000000,0.000000,0.000000}%
\pgfsetstrokecolor{currentstroke}%
\pgfsetdash{}{0pt}%
\pgfpathmoveto{\pgfqpoint{1.876534in}{1.762692in}}%
\pgfpathcurveto{\pgfqpoint{1.887584in}{1.762692in}}{\pgfqpoint{1.898183in}{1.767083in}}{\pgfqpoint{1.905996in}{1.774896in}}%
\pgfpathcurveto{\pgfqpoint{1.913810in}{1.782710in}}{\pgfqpoint{1.918200in}{1.793309in}}{\pgfqpoint{1.918200in}{1.804359in}}%
\pgfpathcurveto{\pgfqpoint{1.918200in}{1.815409in}}{\pgfqpoint{1.913810in}{1.826008in}}{\pgfqpoint{1.905996in}{1.833822in}}%
\pgfpathcurveto{\pgfqpoint{1.898183in}{1.841635in}}{\pgfqpoint{1.887584in}{1.846026in}}{\pgfqpoint{1.876534in}{1.846026in}}%
\pgfpathcurveto{\pgfqpoint{1.865484in}{1.846026in}}{\pgfqpoint{1.854885in}{1.841635in}}{\pgfqpoint{1.847071in}{1.833822in}}%
\pgfpathcurveto{\pgfqpoint{1.839257in}{1.826008in}}{\pgfqpoint{1.834867in}{1.815409in}}{\pgfqpoint{1.834867in}{1.804359in}}%
\pgfpathcurveto{\pgfqpoint{1.834867in}{1.793309in}}{\pgfqpoint{1.839257in}{1.782710in}}{\pgfqpoint{1.847071in}{1.774896in}}%
\pgfpathcurveto{\pgfqpoint{1.854885in}{1.767083in}}{\pgfqpoint{1.865484in}{1.762692in}}{\pgfqpoint{1.876534in}{1.762692in}}%
\pgfpathlineto{\pgfqpoint{1.876534in}{1.762692in}}%
\pgfpathclose%
\pgfusepath{stroke}%
\end{pgfscope}%
\begin{pgfscope}%
\pgfpathrectangle{\pgfqpoint{0.847223in}{0.554012in}}{\pgfqpoint{6.200000in}{4.530000in}}%
\pgfusepath{clip}%
\pgfsetbuttcap%
\pgfsetroundjoin%
\pgfsetlinewidth{1.003750pt}%
\definecolor{currentstroke}{rgb}{1.000000,0.000000,0.000000}%
\pgfsetstrokecolor{currentstroke}%
\pgfsetdash{}{0pt}%
\pgfpathmoveto{\pgfqpoint{1.881867in}{1.756984in}}%
\pgfpathcurveto{\pgfqpoint{1.892917in}{1.756984in}}{\pgfqpoint{1.903516in}{1.761375in}}{\pgfqpoint{1.911330in}{1.769188in}}%
\pgfpathcurveto{\pgfqpoint{1.919143in}{1.777002in}}{\pgfqpoint{1.923534in}{1.787601in}}{\pgfqpoint{1.923534in}{1.798651in}}%
\pgfpathcurveto{\pgfqpoint{1.923534in}{1.809701in}}{\pgfqpoint{1.919143in}{1.820300in}}{\pgfqpoint{1.911330in}{1.828114in}}%
\pgfpathcurveto{\pgfqpoint{1.903516in}{1.835927in}}{\pgfqpoint{1.892917in}{1.840318in}}{\pgfqpoint{1.881867in}{1.840318in}}%
\pgfpathcurveto{\pgfqpoint{1.870817in}{1.840318in}}{\pgfqpoint{1.860218in}{1.835927in}}{\pgfqpoint{1.852404in}{1.828114in}}%
\pgfpathcurveto{\pgfqpoint{1.844590in}{1.820300in}}{\pgfqpoint{1.840200in}{1.809701in}}{\pgfqpoint{1.840200in}{1.798651in}}%
\pgfpathcurveto{\pgfqpoint{1.840200in}{1.787601in}}{\pgfqpoint{1.844590in}{1.777002in}}{\pgfqpoint{1.852404in}{1.769188in}}%
\pgfpathcurveto{\pgfqpoint{1.860218in}{1.761375in}}{\pgfqpoint{1.870817in}{1.756984in}}{\pgfqpoint{1.881867in}{1.756984in}}%
\pgfpathlineto{\pgfqpoint{1.881867in}{1.756984in}}%
\pgfpathclose%
\pgfusepath{stroke}%
\end{pgfscope}%
\begin{pgfscope}%
\pgfpathrectangle{\pgfqpoint{0.847223in}{0.554012in}}{\pgfqpoint{6.200000in}{4.530000in}}%
\pgfusepath{clip}%
\pgfsetbuttcap%
\pgfsetroundjoin%
\pgfsetlinewidth{1.003750pt}%
\definecolor{currentstroke}{rgb}{1.000000,0.000000,0.000000}%
\pgfsetstrokecolor{currentstroke}%
\pgfsetdash{}{0pt}%
\pgfpathmoveto{\pgfqpoint{1.887200in}{1.751315in}}%
\pgfpathcurveto{\pgfqpoint{1.898250in}{1.751315in}}{\pgfqpoint{1.908849in}{1.755705in}}{\pgfqpoint{1.916663in}{1.763519in}}%
\pgfpathcurveto{\pgfqpoint{1.924477in}{1.771333in}}{\pgfqpoint{1.928867in}{1.781932in}}{\pgfqpoint{1.928867in}{1.792982in}}%
\pgfpathcurveto{\pgfqpoint{1.928867in}{1.804032in}}{\pgfqpoint{1.924477in}{1.814631in}}{\pgfqpoint{1.916663in}{1.822444in}}%
\pgfpathcurveto{\pgfqpoint{1.908849in}{1.830258in}}{\pgfqpoint{1.898250in}{1.834648in}}{\pgfqpoint{1.887200in}{1.834648in}}%
\pgfpathcurveto{\pgfqpoint{1.876150in}{1.834648in}}{\pgfqpoint{1.865551in}{1.830258in}}{\pgfqpoint{1.857737in}{1.822444in}}%
\pgfpathcurveto{\pgfqpoint{1.849924in}{1.814631in}}{\pgfqpoint{1.845533in}{1.804032in}}{\pgfqpoint{1.845533in}{1.792982in}}%
\pgfpathcurveto{\pgfqpoint{1.845533in}{1.781932in}}{\pgfqpoint{1.849924in}{1.771333in}}{\pgfqpoint{1.857737in}{1.763519in}}%
\pgfpathcurveto{\pgfqpoint{1.865551in}{1.755705in}}{\pgfqpoint{1.876150in}{1.751315in}}{\pgfqpoint{1.887200in}{1.751315in}}%
\pgfpathlineto{\pgfqpoint{1.887200in}{1.751315in}}%
\pgfpathclose%
\pgfusepath{stroke}%
\end{pgfscope}%
\begin{pgfscope}%
\pgfpathrectangle{\pgfqpoint{0.847223in}{0.554012in}}{\pgfqpoint{6.200000in}{4.530000in}}%
\pgfusepath{clip}%
\pgfsetbuttcap%
\pgfsetroundjoin%
\pgfsetlinewidth{1.003750pt}%
\definecolor{currentstroke}{rgb}{1.000000,0.000000,0.000000}%
\pgfsetstrokecolor{currentstroke}%
\pgfsetdash{}{0pt}%
\pgfpathmoveto{\pgfqpoint{1.892533in}{1.745684in}}%
\pgfpathcurveto{\pgfqpoint{1.903583in}{1.745684in}}{\pgfqpoint{1.914182in}{1.750075in}}{\pgfqpoint{1.921996in}{1.757888in}}%
\pgfpathcurveto{\pgfqpoint{1.929810in}{1.765702in}}{\pgfqpoint{1.934200in}{1.776301in}}{\pgfqpoint{1.934200in}{1.787351in}}%
\pgfpathcurveto{\pgfqpoint{1.934200in}{1.798401in}}{\pgfqpoint{1.929810in}{1.809000in}}{\pgfqpoint{1.921996in}{1.816814in}}%
\pgfpathcurveto{\pgfqpoint{1.914182in}{1.824627in}}{\pgfqpoint{1.903583in}{1.829018in}}{\pgfqpoint{1.892533in}{1.829018in}}%
\pgfpathcurveto{\pgfqpoint{1.881483in}{1.829018in}}{\pgfqpoint{1.870884in}{1.824627in}}{\pgfqpoint{1.863071in}{1.816814in}}%
\pgfpathcurveto{\pgfqpoint{1.855257in}{1.809000in}}{\pgfqpoint{1.850867in}{1.798401in}}{\pgfqpoint{1.850867in}{1.787351in}}%
\pgfpathcurveto{\pgfqpoint{1.850867in}{1.776301in}}{\pgfqpoint{1.855257in}{1.765702in}}{\pgfqpoint{1.863071in}{1.757888in}}%
\pgfpathcurveto{\pgfqpoint{1.870884in}{1.750075in}}{\pgfqpoint{1.881483in}{1.745684in}}{\pgfqpoint{1.892533in}{1.745684in}}%
\pgfpathlineto{\pgfqpoint{1.892533in}{1.745684in}}%
\pgfpathclose%
\pgfusepath{stroke}%
\end{pgfscope}%
\begin{pgfscope}%
\pgfpathrectangle{\pgfqpoint{0.847223in}{0.554012in}}{\pgfqpoint{6.200000in}{4.530000in}}%
\pgfusepath{clip}%
\pgfsetbuttcap%
\pgfsetroundjoin%
\pgfsetlinewidth{1.003750pt}%
\definecolor{currentstroke}{rgb}{1.000000,0.000000,0.000000}%
\pgfsetstrokecolor{currentstroke}%
\pgfsetdash{}{0pt}%
\pgfpathmoveto{\pgfqpoint{1.897867in}{1.740092in}}%
\pgfpathcurveto{\pgfqpoint{1.908917in}{1.740092in}}{\pgfqpoint{1.919516in}{1.744482in}}{\pgfqpoint{1.927329in}{1.752296in}}%
\pgfpathcurveto{\pgfqpoint{1.935143in}{1.760109in}}{\pgfqpoint{1.939533in}{1.770708in}}{\pgfqpoint{1.939533in}{1.781758in}}%
\pgfpathcurveto{\pgfqpoint{1.939533in}{1.792809in}}{\pgfqpoint{1.935143in}{1.803408in}}{\pgfqpoint{1.927329in}{1.811221in}}%
\pgfpathcurveto{\pgfqpoint{1.919516in}{1.819035in}}{\pgfqpoint{1.908917in}{1.823425in}}{\pgfqpoint{1.897867in}{1.823425in}}%
\pgfpathcurveto{\pgfqpoint{1.886816in}{1.823425in}}{\pgfqpoint{1.876217in}{1.819035in}}{\pgfqpoint{1.868404in}{1.811221in}}%
\pgfpathcurveto{\pgfqpoint{1.860590in}{1.803408in}}{\pgfqpoint{1.856200in}{1.792809in}}{\pgfqpoint{1.856200in}{1.781758in}}%
\pgfpathcurveto{\pgfqpoint{1.856200in}{1.770708in}}{\pgfqpoint{1.860590in}{1.760109in}}{\pgfqpoint{1.868404in}{1.752296in}}%
\pgfpathcurveto{\pgfqpoint{1.876217in}{1.744482in}}{\pgfqpoint{1.886816in}{1.740092in}}{\pgfqpoint{1.897867in}{1.740092in}}%
\pgfpathlineto{\pgfqpoint{1.897867in}{1.740092in}}%
\pgfpathclose%
\pgfusepath{stroke}%
\end{pgfscope}%
\begin{pgfscope}%
\pgfpathrectangle{\pgfqpoint{0.847223in}{0.554012in}}{\pgfqpoint{6.200000in}{4.530000in}}%
\pgfusepath{clip}%
\pgfsetbuttcap%
\pgfsetroundjoin%
\pgfsetlinewidth{1.003750pt}%
\definecolor{currentstroke}{rgb}{1.000000,0.000000,0.000000}%
\pgfsetstrokecolor{currentstroke}%
\pgfsetdash{}{0pt}%
\pgfpathmoveto{\pgfqpoint{1.903200in}{1.734537in}}%
\pgfpathcurveto{\pgfqpoint{1.914250in}{1.734537in}}{\pgfqpoint{1.924849in}{1.738927in}}{\pgfqpoint{1.932663in}{1.746741in}}%
\pgfpathcurveto{\pgfqpoint{1.940476in}{1.754555in}}{\pgfqpoint{1.944866in}{1.765154in}}{\pgfqpoint{1.944866in}{1.776204in}}%
\pgfpathcurveto{\pgfqpoint{1.944866in}{1.787254in}}{\pgfqpoint{1.940476in}{1.797853in}}{\pgfqpoint{1.932663in}{1.805667in}}%
\pgfpathcurveto{\pgfqpoint{1.924849in}{1.813480in}}{\pgfqpoint{1.914250in}{1.817870in}}{\pgfqpoint{1.903200in}{1.817870in}}%
\pgfpathcurveto{\pgfqpoint{1.892150in}{1.817870in}}{\pgfqpoint{1.881551in}{1.813480in}}{\pgfqpoint{1.873737in}{1.805667in}}%
\pgfpathcurveto{\pgfqpoint{1.865923in}{1.797853in}}{\pgfqpoint{1.861533in}{1.787254in}}{\pgfqpoint{1.861533in}{1.776204in}}%
\pgfpathcurveto{\pgfqpoint{1.861533in}{1.765154in}}{\pgfqpoint{1.865923in}{1.754555in}}{\pgfqpoint{1.873737in}{1.746741in}}%
\pgfpathcurveto{\pgfqpoint{1.881551in}{1.738927in}}{\pgfqpoint{1.892150in}{1.734537in}}{\pgfqpoint{1.903200in}{1.734537in}}%
\pgfpathlineto{\pgfqpoint{1.903200in}{1.734537in}}%
\pgfpathclose%
\pgfusepath{stroke}%
\end{pgfscope}%
\begin{pgfscope}%
\pgfpathrectangle{\pgfqpoint{0.847223in}{0.554012in}}{\pgfqpoint{6.200000in}{4.530000in}}%
\pgfusepath{clip}%
\pgfsetbuttcap%
\pgfsetroundjoin%
\pgfsetlinewidth{1.003750pt}%
\definecolor{currentstroke}{rgb}{1.000000,0.000000,0.000000}%
\pgfsetstrokecolor{currentstroke}%
\pgfsetdash{}{0pt}%
\pgfpathmoveto{\pgfqpoint{1.908533in}{1.729020in}}%
\pgfpathcurveto{\pgfqpoint{1.919583in}{1.729020in}}{\pgfqpoint{1.930182in}{1.733410in}}{\pgfqpoint{1.937996in}{1.741224in}}%
\pgfpathcurveto{\pgfqpoint{1.945809in}{1.749037in}}{\pgfqpoint{1.950200in}{1.759636in}}{\pgfqpoint{1.950200in}{1.770686in}}%
\pgfpathcurveto{\pgfqpoint{1.950200in}{1.781737in}}{\pgfqpoint{1.945809in}{1.792336in}}{\pgfqpoint{1.937996in}{1.800149in}}%
\pgfpathcurveto{\pgfqpoint{1.930182in}{1.807963in}}{\pgfqpoint{1.919583in}{1.812353in}}{\pgfqpoint{1.908533in}{1.812353in}}%
\pgfpathcurveto{\pgfqpoint{1.897483in}{1.812353in}}{\pgfqpoint{1.886884in}{1.807963in}}{\pgfqpoint{1.879070in}{1.800149in}}%
\pgfpathcurveto{\pgfqpoint{1.871257in}{1.792336in}}{\pgfqpoint{1.866866in}{1.781737in}}{\pgfqpoint{1.866866in}{1.770686in}}%
\pgfpathcurveto{\pgfqpoint{1.866866in}{1.759636in}}{\pgfqpoint{1.871257in}{1.749037in}}{\pgfqpoint{1.879070in}{1.741224in}}%
\pgfpathcurveto{\pgfqpoint{1.886884in}{1.733410in}}{\pgfqpoint{1.897483in}{1.729020in}}{\pgfqpoint{1.908533in}{1.729020in}}%
\pgfpathlineto{\pgfqpoint{1.908533in}{1.729020in}}%
\pgfpathclose%
\pgfusepath{stroke}%
\end{pgfscope}%
\begin{pgfscope}%
\pgfpathrectangle{\pgfqpoint{0.847223in}{0.554012in}}{\pgfqpoint{6.200000in}{4.530000in}}%
\pgfusepath{clip}%
\pgfsetbuttcap%
\pgfsetroundjoin%
\pgfsetlinewidth{1.003750pt}%
\definecolor{currentstroke}{rgb}{1.000000,0.000000,0.000000}%
\pgfsetstrokecolor{currentstroke}%
\pgfsetdash{}{0pt}%
\pgfpathmoveto{\pgfqpoint{1.913866in}{1.723539in}}%
\pgfpathcurveto{\pgfqpoint{1.924916in}{1.723539in}}{\pgfqpoint{1.935515in}{1.727930in}}{\pgfqpoint{1.943329in}{1.735743in}}%
\pgfpathcurveto{\pgfqpoint{1.951143in}{1.743557in}}{\pgfqpoint{1.955533in}{1.754156in}}{\pgfqpoint{1.955533in}{1.765206in}}%
\pgfpathcurveto{\pgfqpoint{1.955533in}{1.776256in}}{\pgfqpoint{1.951143in}{1.786855in}}{\pgfqpoint{1.943329in}{1.794669in}}%
\pgfpathcurveto{\pgfqpoint{1.935515in}{1.802482in}}{\pgfqpoint{1.924916in}{1.806873in}}{\pgfqpoint{1.913866in}{1.806873in}}%
\pgfpathcurveto{\pgfqpoint{1.902816in}{1.806873in}}{\pgfqpoint{1.892217in}{1.802482in}}{\pgfqpoint{1.884403in}{1.794669in}}%
\pgfpathcurveto{\pgfqpoint{1.876590in}{1.786855in}}{\pgfqpoint{1.872200in}{1.776256in}}{\pgfqpoint{1.872200in}{1.765206in}}%
\pgfpathcurveto{\pgfqpoint{1.872200in}{1.754156in}}{\pgfqpoint{1.876590in}{1.743557in}}{\pgfqpoint{1.884403in}{1.735743in}}%
\pgfpathcurveto{\pgfqpoint{1.892217in}{1.727930in}}{\pgfqpoint{1.902816in}{1.723539in}}{\pgfqpoint{1.913866in}{1.723539in}}%
\pgfpathlineto{\pgfqpoint{1.913866in}{1.723539in}}%
\pgfpathclose%
\pgfusepath{stroke}%
\end{pgfscope}%
\begin{pgfscope}%
\pgfpathrectangle{\pgfqpoint{0.847223in}{0.554012in}}{\pgfqpoint{6.200000in}{4.530000in}}%
\pgfusepath{clip}%
\pgfsetbuttcap%
\pgfsetroundjoin%
\pgfsetlinewidth{1.003750pt}%
\definecolor{currentstroke}{rgb}{1.000000,0.000000,0.000000}%
\pgfsetstrokecolor{currentstroke}%
\pgfsetdash{}{0pt}%
\pgfpathmoveto{\pgfqpoint{1.919199in}{1.718096in}}%
\pgfpathcurveto{\pgfqpoint{1.930250in}{1.718096in}}{\pgfqpoint{1.940849in}{1.722486in}}{\pgfqpoint{1.948662in}{1.730300in}}%
\pgfpathcurveto{\pgfqpoint{1.956476in}{1.738113in}}{\pgfqpoint{1.960866in}{1.748712in}}{\pgfqpoint{1.960866in}{1.759762in}}%
\pgfpathcurveto{\pgfqpoint{1.960866in}{1.770812in}}{\pgfqpoint{1.956476in}{1.781411in}}{\pgfqpoint{1.948662in}{1.789225in}}%
\pgfpathcurveto{\pgfqpoint{1.940849in}{1.797039in}}{\pgfqpoint{1.930250in}{1.801429in}}{\pgfqpoint{1.919199in}{1.801429in}}%
\pgfpathcurveto{\pgfqpoint{1.908149in}{1.801429in}}{\pgfqpoint{1.897550in}{1.797039in}}{\pgfqpoint{1.889737in}{1.789225in}}%
\pgfpathcurveto{\pgfqpoint{1.881923in}{1.781411in}}{\pgfqpoint{1.877533in}{1.770812in}}{\pgfqpoint{1.877533in}{1.759762in}}%
\pgfpathcurveto{\pgfqpoint{1.877533in}{1.748712in}}{\pgfqpoint{1.881923in}{1.738113in}}{\pgfqpoint{1.889737in}{1.730300in}}%
\pgfpathcurveto{\pgfqpoint{1.897550in}{1.722486in}}{\pgfqpoint{1.908149in}{1.718096in}}{\pgfqpoint{1.919199in}{1.718096in}}%
\pgfpathlineto{\pgfqpoint{1.919199in}{1.718096in}}%
\pgfpathclose%
\pgfusepath{stroke}%
\end{pgfscope}%
\begin{pgfscope}%
\pgfpathrectangle{\pgfqpoint{0.847223in}{0.554012in}}{\pgfqpoint{6.200000in}{4.530000in}}%
\pgfusepath{clip}%
\pgfsetbuttcap%
\pgfsetroundjoin%
\pgfsetlinewidth{1.003750pt}%
\definecolor{currentstroke}{rgb}{1.000000,0.000000,0.000000}%
\pgfsetstrokecolor{currentstroke}%
\pgfsetdash{}{0pt}%
\pgfpathmoveto{\pgfqpoint{1.924533in}{1.712688in}}%
\pgfpathcurveto{\pgfqpoint{1.935583in}{1.712688in}}{\pgfqpoint{1.946182in}{1.717078in}}{\pgfqpoint{1.953995in}{1.724892in}}%
\pgfpathcurveto{\pgfqpoint{1.961809in}{1.732706in}}{\pgfqpoint{1.966199in}{1.743305in}}{\pgfqpoint{1.966199in}{1.754355in}}%
\pgfpathcurveto{\pgfqpoint{1.966199in}{1.765405in}}{\pgfqpoint{1.961809in}{1.776004in}}{\pgfqpoint{1.953995in}{1.783818in}}%
\pgfpathcurveto{\pgfqpoint{1.946182in}{1.791631in}}{\pgfqpoint{1.935583in}{1.796022in}}{\pgfqpoint{1.924533in}{1.796022in}}%
\pgfpathcurveto{\pgfqpoint{1.913482in}{1.796022in}}{\pgfqpoint{1.902883in}{1.791631in}}{\pgfqpoint{1.895070in}{1.783818in}}%
\pgfpathcurveto{\pgfqpoint{1.887256in}{1.776004in}}{\pgfqpoint{1.882866in}{1.765405in}}{\pgfqpoint{1.882866in}{1.754355in}}%
\pgfpathcurveto{\pgfqpoint{1.882866in}{1.743305in}}{\pgfqpoint{1.887256in}{1.732706in}}{\pgfqpoint{1.895070in}{1.724892in}}%
\pgfpathcurveto{\pgfqpoint{1.902883in}{1.717078in}}{\pgfqpoint{1.913482in}{1.712688in}}{\pgfqpoint{1.924533in}{1.712688in}}%
\pgfpathlineto{\pgfqpoint{1.924533in}{1.712688in}}%
\pgfpathclose%
\pgfusepath{stroke}%
\end{pgfscope}%
\begin{pgfscope}%
\pgfpathrectangle{\pgfqpoint{0.847223in}{0.554012in}}{\pgfqpoint{6.200000in}{4.530000in}}%
\pgfusepath{clip}%
\pgfsetbuttcap%
\pgfsetroundjoin%
\pgfsetlinewidth{1.003750pt}%
\definecolor{currentstroke}{rgb}{1.000000,0.000000,0.000000}%
\pgfsetstrokecolor{currentstroke}%
\pgfsetdash{}{0pt}%
\pgfpathmoveto{\pgfqpoint{1.929866in}{1.707317in}}%
\pgfpathcurveto{\pgfqpoint{1.940916in}{1.707317in}}{\pgfqpoint{1.951515in}{1.711707in}}{\pgfqpoint{1.959329in}{1.719521in}}%
\pgfpathcurveto{\pgfqpoint{1.967142in}{1.727334in}}{\pgfqpoint{1.971533in}{1.737933in}}{\pgfqpoint{1.971533in}{1.748983in}}%
\pgfpathcurveto{\pgfqpoint{1.971533in}{1.760034in}}{\pgfqpoint{1.967142in}{1.770633in}}{\pgfqpoint{1.959329in}{1.778446in}}%
\pgfpathcurveto{\pgfqpoint{1.951515in}{1.786260in}}{\pgfqpoint{1.940916in}{1.790650in}}{\pgfqpoint{1.929866in}{1.790650in}}%
\pgfpathcurveto{\pgfqpoint{1.918816in}{1.790650in}}{\pgfqpoint{1.908217in}{1.786260in}}{\pgfqpoint{1.900403in}{1.778446in}}%
\pgfpathcurveto{\pgfqpoint{1.892589in}{1.770633in}}{\pgfqpoint{1.888199in}{1.760034in}}{\pgfqpoint{1.888199in}{1.748983in}}%
\pgfpathcurveto{\pgfqpoint{1.888199in}{1.737933in}}{\pgfqpoint{1.892589in}{1.727334in}}{\pgfqpoint{1.900403in}{1.719521in}}%
\pgfpathcurveto{\pgfqpoint{1.908217in}{1.711707in}}{\pgfqpoint{1.918816in}{1.707317in}}{\pgfqpoint{1.929866in}{1.707317in}}%
\pgfpathlineto{\pgfqpoint{1.929866in}{1.707317in}}%
\pgfpathclose%
\pgfusepath{stroke}%
\end{pgfscope}%
\begin{pgfscope}%
\pgfpathrectangle{\pgfqpoint{0.847223in}{0.554012in}}{\pgfqpoint{6.200000in}{4.530000in}}%
\pgfusepath{clip}%
\pgfsetbuttcap%
\pgfsetroundjoin%
\pgfsetlinewidth{1.003750pt}%
\definecolor{currentstroke}{rgb}{1.000000,0.000000,0.000000}%
\pgfsetstrokecolor{currentstroke}%
\pgfsetdash{}{0pt}%
\pgfpathmoveto{\pgfqpoint{1.935199in}{1.701981in}}%
\pgfpathcurveto{\pgfqpoint{1.946249in}{1.701981in}}{\pgfqpoint{1.956848in}{1.706371in}}{\pgfqpoint{1.964662in}{1.714185in}}%
\pgfpathcurveto{\pgfqpoint{1.972475in}{1.721998in}}{\pgfqpoint{1.976866in}{1.732597in}}{\pgfqpoint{1.976866in}{1.743647in}}%
\pgfpathcurveto{\pgfqpoint{1.976866in}{1.754698in}}{\pgfqpoint{1.972475in}{1.765297in}}{\pgfqpoint{1.964662in}{1.773110in}}%
\pgfpathcurveto{\pgfqpoint{1.956848in}{1.780924in}}{\pgfqpoint{1.946249in}{1.785314in}}{\pgfqpoint{1.935199in}{1.785314in}}%
\pgfpathcurveto{\pgfqpoint{1.924149in}{1.785314in}}{\pgfqpoint{1.913550in}{1.780924in}}{\pgfqpoint{1.905736in}{1.773110in}}%
\pgfpathcurveto{\pgfqpoint{1.897923in}{1.765297in}}{\pgfqpoint{1.893532in}{1.754698in}}{\pgfqpoint{1.893532in}{1.743647in}}%
\pgfpathcurveto{\pgfqpoint{1.893532in}{1.732597in}}{\pgfqpoint{1.897923in}{1.721998in}}{\pgfqpoint{1.905736in}{1.714185in}}%
\pgfpathcurveto{\pgfqpoint{1.913550in}{1.706371in}}{\pgfqpoint{1.924149in}{1.701981in}}{\pgfqpoint{1.935199in}{1.701981in}}%
\pgfpathlineto{\pgfqpoint{1.935199in}{1.701981in}}%
\pgfpathclose%
\pgfusepath{stroke}%
\end{pgfscope}%
\begin{pgfscope}%
\pgfpathrectangle{\pgfqpoint{0.847223in}{0.554012in}}{\pgfqpoint{6.200000in}{4.530000in}}%
\pgfusepath{clip}%
\pgfsetbuttcap%
\pgfsetroundjoin%
\pgfsetlinewidth{1.003750pt}%
\definecolor{currentstroke}{rgb}{1.000000,0.000000,0.000000}%
\pgfsetstrokecolor{currentstroke}%
\pgfsetdash{}{0pt}%
\pgfpathmoveto{\pgfqpoint{1.940532in}{1.696680in}}%
\pgfpathcurveto{\pgfqpoint{1.951582in}{1.696680in}}{\pgfqpoint{1.962181in}{1.701070in}}{\pgfqpoint{1.969995in}{1.708884in}}%
\pgfpathcurveto{\pgfqpoint{1.977809in}{1.716698in}}{\pgfqpoint{1.982199in}{1.727297in}}{\pgfqpoint{1.982199in}{1.738347in}}%
\pgfpathcurveto{\pgfqpoint{1.982199in}{1.749397in}}{\pgfqpoint{1.977809in}{1.759996in}}{\pgfqpoint{1.969995in}{1.767809in}}%
\pgfpathcurveto{\pgfqpoint{1.962181in}{1.775623in}}{\pgfqpoint{1.951582in}{1.780013in}}{\pgfqpoint{1.940532in}{1.780013in}}%
\pgfpathcurveto{\pgfqpoint{1.929482in}{1.780013in}}{\pgfqpoint{1.918883in}{1.775623in}}{\pgfqpoint{1.911069in}{1.767809in}}%
\pgfpathcurveto{\pgfqpoint{1.903256in}{1.759996in}}{\pgfqpoint{1.898866in}{1.749397in}}{\pgfqpoint{1.898866in}{1.738347in}}%
\pgfpathcurveto{\pgfqpoint{1.898866in}{1.727297in}}{\pgfqpoint{1.903256in}{1.716698in}}{\pgfqpoint{1.911069in}{1.708884in}}%
\pgfpathcurveto{\pgfqpoint{1.918883in}{1.701070in}}{\pgfqpoint{1.929482in}{1.696680in}}{\pgfqpoint{1.940532in}{1.696680in}}%
\pgfpathlineto{\pgfqpoint{1.940532in}{1.696680in}}%
\pgfpathclose%
\pgfusepath{stroke}%
\end{pgfscope}%
\begin{pgfscope}%
\pgfpathrectangle{\pgfqpoint{0.847223in}{0.554012in}}{\pgfqpoint{6.200000in}{4.530000in}}%
\pgfusepath{clip}%
\pgfsetbuttcap%
\pgfsetroundjoin%
\pgfsetlinewidth{1.003750pt}%
\definecolor{currentstroke}{rgb}{1.000000,0.000000,0.000000}%
\pgfsetstrokecolor{currentstroke}%
\pgfsetdash{}{0pt}%
\pgfpathmoveto{\pgfqpoint{1.945865in}{1.691414in}}%
\pgfpathcurveto{\pgfqpoint{1.956916in}{1.691414in}}{\pgfqpoint{1.967515in}{1.695804in}}{\pgfqpoint{1.975328in}{1.703618in}}%
\pgfpathcurveto{\pgfqpoint{1.983142in}{1.711432in}}{\pgfqpoint{1.987532in}{1.722031in}}{\pgfqpoint{1.987532in}{1.733081in}}%
\pgfpathcurveto{\pgfqpoint{1.987532in}{1.744131in}}{\pgfqpoint{1.983142in}{1.754730in}}{\pgfqpoint{1.975328in}{1.762544in}}%
\pgfpathcurveto{\pgfqpoint{1.967515in}{1.770357in}}{\pgfqpoint{1.956916in}{1.774747in}}{\pgfqpoint{1.945865in}{1.774747in}}%
\pgfpathcurveto{\pgfqpoint{1.934815in}{1.774747in}}{\pgfqpoint{1.924216in}{1.770357in}}{\pgfqpoint{1.916403in}{1.762544in}}%
\pgfpathcurveto{\pgfqpoint{1.908589in}{1.754730in}}{\pgfqpoint{1.904199in}{1.744131in}}{\pgfqpoint{1.904199in}{1.733081in}}%
\pgfpathcurveto{\pgfqpoint{1.904199in}{1.722031in}}{\pgfqpoint{1.908589in}{1.711432in}}{\pgfqpoint{1.916403in}{1.703618in}}%
\pgfpathcurveto{\pgfqpoint{1.924216in}{1.695804in}}{\pgfqpoint{1.934815in}{1.691414in}}{\pgfqpoint{1.945865in}{1.691414in}}%
\pgfpathlineto{\pgfqpoint{1.945865in}{1.691414in}}%
\pgfpathclose%
\pgfusepath{stroke}%
\end{pgfscope}%
\begin{pgfscope}%
\pgfpathrectangle{\pgfqpoint{0.847223in}{0.554012in}}{\pgfqpoint{6.200000in}{4.530000in}}%
\pgfusepath{clip}%
\pgfsetbuttcap%
\pgfsetroundjoin%
\pgfsetlinewidth{1.003750pt}%
\definecolor{currentstroke}{rgb}{1.000000,0.000000,0.000000}%
\pgfsetstrokecolor{currentstroke}%
\pgfsetdash{}{0pt}%
\pgfpathmoveto{\pgfqpoint{1.951199in}{1.686183in}}%
\pgfpathcurveto{\pgfqpoint{1.962249in}{1.686183in}}{\pgfqpoint{1.972848in}{1.690573in}}{\pgfqpoint{1.980661in}{1.698387in}}%
\pgfpathcurveto{\pgfqpoint{1.988475in}{1.706200in}}{\pgfqpoint{1.992865in}{1.716799in}}{\pgfqpoint{1.992865in}{1.727849in}}%
\pgfpathcurveto{\pgfqpoint{1.992865in}{1.738900in}}{\pgfqpoint{1.988475in}{1.749499in}}{\pgfqpoint{1.980661in}{1.757312in}}%
\pgfpathcurveto{\pgfqpoint{1.972848in}{1.765126in}}{\pgfqpoint{1.962249in}{1.769516in}}{\pgfqpoint{1.951199in}{1.769516in}}%
\pgfpathcurveto{\pgfqpoint{1.940149in}{1.769516in}}{\pgfqpoint{1.929550in}{1.765126in}}{\pgfqpoint{1.921736in}{1.757312in}}%
\pgfpathcurveto{\pgfqpoint{1.913922in}{1.749499in}}{\pgfqpoint{1.909532in}{1.738900in}}{\pgfqpoint{1.909532in}{1.727849in}}%
\pgfpathcurveto{\pgfqpoint{1.909532in}{1.716799in}}{\pgfqpoint{1.913922in}{1.706200in}}{\pgfqpoint{1.921736in}{1.698387in}}%
\pgfpathcurveto{\pgfqpoint{1.929550in}{1.690573in}}{\pgfqpoint{1.940149in}{1.686183in}}{\pgfqpoint{1.951199in}{1.686183in}}%
\pgfpathlineto{\pgfqpoint{1.951199in}{1.686183in}}%
\pgfpathclose%
\pgfusepath{stroke}%
\end{pgfscope}%
\begin{pgfscope}%
\pgfpathrectangle{\pgfqpoint{0.847223in}{0.554012in}}{\pgfqpoint{6.200000in}{4.530000in}}%
\pgfusepath{clip}%
\pgfsetbuttcap%
\pgfsetroundjoin%
\pgfsetlinewidth{1.003750pt}%
\definecolor{currentstroke}{rgb}{1.000000,0.000000,0.000000}%
\pgfsetstrokecolor{currentstroke}%
\pgfsetdash{}{0pt}%
\pgfpathmoveto{\pgfqpoint{1.956532in}{1.680986in}}%
\pgfpathcurveto{\pgfqpoint{1.967582in}{1.680986in}}{\pgfqpoint{1.978181in}{1.685376in}}{\pgfqpoint{1.985995in}{1.693189in}}%
\pgfpathcurveto{\pgfqpoint{1.993808in}{1.701003in}}{\pgfqpoint{1.998199in}{1.711602in}}{\pgfqpoint{1.998199in}{1.722652in}}%
\pgfpathcurveto{\pgfqpoint{1.998199in}{1.733702in}}{\pgfqpoint{1.993808in}{1.744301in}}{\pgfqpoint{1.985995in}{1.752115in}}%
\pgfpathcurveto{\pgfqpoint{1.978181in}{1.759929in}}{\pgfqpoint{1.967582in}{1.764319in}}{\pgfqpoint{1.956532in}{1.764319in}}%
\pgfpathcurveto{\pgfqpoint{1.945482in}{1.764319in}}{\pgfqpoint{1.934883in}{1.759929in}}{\pgfqpoint{1.927069in}{1.752115in}}%
\pgfpathcurveto{\pgfqpoint{1.919256in}{1.744301in}}{\pgfqpoint{1.914865in}{1.733702in}}{\pgfqpoint{1.914865in}{1.722652in}}%
\pgfpathcurveto{\pgfqpoint{1.914865in}{1.711602in}}{\pgfqpoint{1.919256in}{1.701003in}}{\pgfqpoint{1.927069in}{1.693189in}}%
\pgfpathcurveto{\pgfqpoint{1.934883in}{1.685376in}}{\pgfqpoint{1.945482in}{1.680986in}}{\pgfqpoint{1.956532in}{1.680986in}}%
\pgfpathlineto{\pgfqpoint{1.956532in}{1.680986in}}%
\pgfpathclose%
\pgfusepath{stroke}%
\end{pgfscope}%
\begin{pgfscope}%
\pgfpathrectangle{\pgfqpoint{0.847223in}{0.554012in}}{\pgfqpoint{6.200000in}{4.530000in}}%
\pgfusepath{clip}%
\pgfsetbuttcap%
\pgfsetroundjoin%
\pgfsetlinewidth{1.003750pt}%
\definecolor{currentstroke}{rgb}{1.000000,0.000000,0.000000}%
\pgfsetstrokecolor{currentstroke}%
\pgfsetdash{}{0pt}%
\pgfpathmoveto{\pgfqpoint{1.961865in}{1.675822in}}%
\pgfpathcurveto{\pgfqpoint{1.972915in}{1.675822in}}{\pgfqpoint{1.983514in}{1.680212in}}{\pgfqpoint{1.991328in}{1.688026in}}%
\pgfpathcurveto{\pgfqpoint{1.999142in}{1.695840in}}{\pgfqpoint{2.003532in}{1.706439in}}{\pgfqpoint{2.003532in}{1.717489in}}%
\pgfpathcurveto{\pgfqpoint{2.003532in}{1.728539in}}{\pgfqpoint{1.999142in}{1.739138in}}{\pgfqpoint{1.991328in}{1.746952in}}%
\pgfpathcurveto{\pgfqpoint{1.983514in}{1.754765in}}{\pgfqpoint{1.972915in}{1.759156in}}{\pgfqpoint{1.961865in}{1.759156in}}%
\pgfpathcurveto{\pgfqpoint{1.950815in}{1.759156in}}{\pgfqpoint{1.940216in}{1.754765in}}{\pgfqpoint{1.932402in}{1.746952in}}%
\pgfpathcurveto{\pgfqpoint{1.924589in}{1.739138in}}{\pgfqpoint{1.920198in}{1.728539in}}{\pgfqpoint{1.920198in}{1.717489in}}%
\pgfpathcurveto{\pgfqpoint{1.920198in}{1.706439in}}{\pgfqpoint{1.924589in}{1.695840in}}{\pgfqpoint{1.932402in}{1.688026in}}%
\pgfpathcurveto{\pgfqpoint{1.940216in}{1.680212in}}{\pgfqpoint{1.950815in}{1.675822in}}{\pgfqpoint{1.961865in}{1.675822in}}%
\pgfpathlineto{\pgfqpoint{1.961865in}{1.675822in}}%
\pgfpathclose%
\pgfusepath{stroke}%
\end{pgfscope}%
\begin{pgfscope}%
\pgfpathrectangle{\pgfqpoint{0.847223in}{0.554012in}}{\pgfqpoint{6.200000in}{4.530000in}}%
\pgfusepath{clip}%
\pgfsetbuttcap%
\pgfsetroundjoin%
\pgfsetlinewidth{1.003750pt}%
\definecolor{currentstroke}{rgb}{1.000000,0.000000,0.000000}%
\pgfsetstrokecolor{currentstroke}%
\pgfsetdash{}{0pt}%
\pgfpathmoveto{\pgfqpoint{1.967198in}{1.670692in}}%
\pgfpathcurveto{\pgfqpoint{1.978248in}{1.670692in}}{\pgfqpoint{1.988848in}{1.675083in}}{\pgfqpoint{1.996661in}{1.682896in}}%
\pgfpathcurveto{\pgfqpoint{2.004475in}{1.690710in}}{\pgfqpoint{2.008865in}{1.701309in}}{\pgfqpoint{2.008865in}{1.712359in}}%
\pgfpathcurveto{\pgfqpoint{2.008865in}{1.723409in}}{\pgfqpoint{2.004475in}{1.734008in}}{\pgfqpoint{1.996661in}{1.741822in}}%
\pgfpathcurveto{\pgfqpoint{1.988848in}{1.749635in}}{\pgfqpoint{1.978248in}{1.754026in}}{\pgfqpoint{1.967198in}{1.754026in}}%
\pgfpathcurveto{\pgfqpoint{1.956148in}{1.754026in}}{\pgfqpoint{1.945549in}{1.749635in}}{\pgfqpoint{1.937736in}{1.741822in}}%
\pgfpathcurveto{\pgfqpoint{1.929922in}{1.734008in}}{\pgfqpoint{1.925532in}{1.723409in}}{\pgfqpoint{1.925532in}{1.712359in}}%
\pgfpathcurveto{\pgfqpoint{1.925532in}{1.701309in}}{\pgfqpoint{1.929922in}{1.690710in}}{\pgfqpoint{1.937736in}{1.682896in}}%
\pgfpathcurveto{\pgfqpoint{1.945549in}{1.675083in}}{\pgfqpoint{1.956148in}{1.670692in}}{\pgfqpoint{1.967198in}{1.670692in}}%
\pgfpathlineto{\pgfqpoint{1.967198in}{1.670692in}}%
\pgfpathclose%
\pgfusepath{stroke}%
\end{pgfscope}%
\begin{pgfscope}%
\pgfpathrectangle{\pgfqpoint{0.847223in}{0.554012in}}{\pgfqpoint{6.200000in}{4.530000in}}%
\pgfusepath{clip}%
\pgfsetbuttcap%
\pgfsetroundjoin%
\pgfsetlinewidth{1.003750pt}%
\definecolor{currentstroke}{rgb}{1.000000,0.000000,0.000000}%
\pgfsetstrokecolor{currentstroke}%
\pgfsetdash{}{0pt}%
\pgfpathmoveto{\pgfqpoint{1.972532in}{1.665596in}}%
\pgfpathcurveto{\pgfqpoint{1.983582in}{1.665596in}}{\pgfqpoint{1.994181in}{1.669986in}}{\pgfqpoint{2.001994in}{1.677800in}}%
\pgfpathcurveto{\pgfqpoint{2.009808in}{1.685613in}}{\pgfqpoint{2.014198in}{1.696212in}}{\pgfqpoint{2.014198in}{1.707262in}}%
\pgfpathcurveto{\pgfqpoint{2.014198in}{1.718312in}}{\pgfqpoint{2.009808in}{1.728912in}}{\pgfqpoint{2.001994in}{1.736725in}}%
\pgfpathcurveto{\pgfqpoint{1.994181in}{1.744539in}}{\pgfqpoint{1.983582in}{1.748929in}}{\pgfqpoint{1.972532in}{1.748929in}}%
\pgfpathcurveto{\pgfqpoint{1.961481in}{1.748929in}}{\pgfqpoint{1.950882in}{1.744539in}}{\pgfqpoint{1.943069in}{1.736725in}}%
\pgfpathcurveto{\pgfqpoint{1.935255in}{1.728912in}}{\pgfqpoint{1.930865in}{1.718312in}}{\pgfqpoint{1.930865in}{1.707262in}}%
\pgfpathcurveto{\pgfqpoint{1.930865in}{1.696212in}}{\pgfqpoint{1.935255in}{1.685613in}}{\pgfqpoint{1.943069in}{1.677800in}}%
\pgfpathcurveto{\pgfqpoint{1.950882in}{1.669986in}}{\pgfqpoint{1.961481in}{1.665596in}}{\pgfqpoint{1.972532in}{1.665596in}}%
\pgfpathlineto{\pgfqpoint{1.972532in}{1.665596in}}%
\pgfpathclose%
\pgfusepath{stroke}%
\end{pgfscope}%
\begin{pgfscope}%
\pgfpathrectangle{\pgfqpoint{0.847223in}{0.554012in}}{\pgfqpoint{6.200000in}{4.530000in}}%
\pgfusepath{clip}%
\pgfsetbuttcap%
\pgfsetroundjoin%
\pgfsetlinewidth{1.003750pt}%
\definecolor{currentstroke}{rgb}{1.000000,0.000000,0.000000}%
\pgfsetstrokecolor{currentstroke}%
\pgfsetdash{}{0pt}%
\pgfpathmoveto{\pgfqpoint{1.977865in}{1.660532in}}%
\pgfpathcurveto{\pgfqpoint{1.988915in}{1.660532in}}{\pgfqpoint{1.999514in}{1.664922in}}{\pgfqpoint{2.007328in}{1.672736in}}%
\pgfpathcurveto{\pgfqpoint{2.015141in}{1.680549in}}{\pgfqpoint{2.019531in}{1.691148in}}{\pgfqpoint{2.019531in}{1.702199in}}%
\pgfpathcurveto{\pgfqpoint{2.019531in}{1.713249in}}{\pgfqpoint{2.015141in}{1.723848in}}{\pgfqpoint{2.007328in}{1.731661in}}%
\pgfpathcurveto{\pgfqpoint{1.999514in}{1.739475in}}{\pgfqpoint{1.988915in}{1.743865in}}{\pgfqpoint{1.977865in}{1.743865in}}%
\pgfpathcurveto{\pgfqpoint{1.966815in}{1.743865in}}{\pgfqpoint{1.956216in}{1.739475in}}{\pgfqpoint{1.948402in}{1.731661in}}%
\pgfpathcurveto{\pgfqpoint{1.940588in}{1.723848in}}{\pgfqpoint{1.936198in}{1.713249in}}{\pgfqpoint{1.936198in}{1.702199in}}%
\pgfpathcurveto{\pgfqpoint{1.936198in}{1.691148in}}{\pgfqpoint{1.940588in}{1.680549in}}{\pgfqpoint{1.948402in}{1.672736in}}%
\pgfpathcurveto{\pgfqpoint{1.956216in}{1.664922in}}{\pgfqpoint{1.966815in}{1.660532in}}{\pgfqpoint{1.977865in}{1.660532in}}%
\pgfpathlineto{\pgfqpoint{1.977865in}{1.660532in}}%
\pgfpathclose%
\pgfusepath{stroke}%
\end{pgfscope}%
\begin{pgfscope}%
\pgfpathrectangle{\pgfqpoint{0.847223in}{0.554012in}}{\pgfqpoint{6.200000in}{4.530000in}}%
\pgfusepath{clip}%
\pgfsetbuttcap%
\pgfsetroundjoin%
\pgfsetlinewidth{1.003750pt}%
\definecolor{currentstroke}{rgb}{1.000000,0.000000,0.000000}%
\pgfsetstrokecolor{currentstroke}%
\pgfsetdash{}{0pt}%
\pgfpathmoveto{\pgfqpoint{1.983198in}{1.655501in}}%
\pgfpathcurveto{\pgfqpoint{1.994248in}{1.655501in}}{\pgfqpoint{2.004847in}{1.659891in}}{\pgfqpoint{2.012661in}{1.667705in}}%
\pgfpathcurveto{\pgfqpoint{2.020474in}{1.675518in}}{\pgfqpoint{2.024865in}{1.686117in}}{\pgfqpoint{2.024865in}{1.697167in}}%
\pgfpathcurveto{\pgfqpoint{2.024865in}{1.708217in}}{\pgfqpoint{2.020474in}{1.718816in}}{\pgfqpoint{2.012661in}{1.726630in}}%
\pgfpathcurveto{\pgfqpoint{2.004847in}{1.734444in}}{\pgfqpoint{1.994248in}{1.738834in}}{\pgfqpoint{1.983198in}{1.738834in}}%
\pgfpathcurveto{\pgfqpoint{1.972148in}{1.738834in}}{\pgfqpoint{1.961549in}{1.734444in}}{\pgfqpoint{1.953735in}{1.726630in}}%
\pgfpathcurveto{\pgfqpoint{1.945922in}{1.718816in}}{\pgfqpoint{1.941531in}{1.708217in}}{\pgfqpoint{1.941531in}{1.697167in}}%
\pgfpathcurveto{\pgfqpoint{1.941531in}{1.686117in}}{\pgfqpoint{1.945922in}{1.675518in}}{\pgfqpoint{1.953735in}{1.667705in}}%
\pgfpathcurveto{\pgfqpoint{1.961549in}{1.659891in}}{\pgfqpoint{1.972148in}{1.655501in}}{\pgfqpoint{1.983198in}{1.655501in}}%
\pgfpathlineto{\pgfqpoint{1.983198in}{1.655501in}}%
\pgfpathclose%
\pgfusepath{stroke}%
\end{pgfscope}%
\begin{pgfscope}%
\pgfpathrectangle{\pgfqpoint{0.847223in}{0.554012in}}{\pgfqpoint{6.200000in}{4.530000in}}%
\pgfusepath{clip}%
\pgfsetbuttcap%
\pgfsetroundjoin%
\pgfsetlinewidth{1.003750pt}%
\definecolor{currentstroke}{rgb}{1.000000,0.000000,0.000000}%
\pgfsetstrokecolor{currentstroke}%
\pgfsetdash{}{0pt}%
\pgfpathmoveto{\pgfqpoint{1.988531in}{1.650502in}}%
\pgfpathcurveto{\pgfqpoint{1.999581in}{1.650502in}}{\pgfqpoint{2.010180in}{1.654892in}}{\pgfqpoint{2.017994in}{1.662706in}}%
\pgfpathcurveto{\pgfqpoint{2.025808in}{1.670519in}}{\pgfqpoint{2.030198in}{1.681118in}}{\pgfqpoint{2.030198in}{1.692168in}}%
\pgfpathcurveto{\pgfqpoint{2.030198in}{1.703218in}}{\pgfqpoint{2.025808in}{1.713817in}}{\pgfqpoint{2.017994in}{1.721631in}}%
\pgfpathcurveto{\pgfqpoint{2.010180in}{1.729445in}}{\pgfqpoint{1.999581in}{1.733835in}}{\pgfqpoint{1.988531in}{1.733835in}}%
\pgfpathcurveto{\pgfqpoint{1.977481in}{1.733835in}}{\pgfqpoint{1.966882in}{1.729445in}}{\pgfqpoint{1.959068in}{1.721631in}}%
\pgfpathcurveto{\pgfqpoint{1.951255in}{1.713817in}}{\pgfqpoint{1.946865in}{1.703218in}}{\pgfqpoint{1.946865in}{1.692168in}}%
\pgfpathcurveto{\pgfqpoint{1.946865in}{1.681118in}}{\pgfqpoint{1.951255in}{1.670519in}}{\pgfqpoint{1.959068in}{1.662706in}}%
\pgfpathcurveto{\pgfqpoint{1.966882in}{1.654892in}}{\pgfqpoint{1.977481in}{1.650502in}}{\pgfqpoint{1.988531in}{1.650502in}}%
\pgfpathlineto{\pgfqpoint{1.988531in}{1.650502in}}%
\pgfpathclose%
\pgfusepath{stroke}%
\end{pgfscope}%
\begin{pgfscope}%
\pgfpathrectangle{\pgfqpoint{0.847223in}{0.554012in}}{\pgfqpoint{6.200000in}{4.530000in}}%
\pgfusepath{clip}%
\pgfsetbuttcap%
\pgfsetroundjoin%
\pgfsetlinewidth{1.003750pt}%
\definecolor{currentstroke}{rgb}{1.000000,0.000000,0.000000}%
\pgfsetstrokecolor{currentstroke}%
\pgfsetdash{}{0pt}%
\pgfpathmoveto{\pgfqpoint{1.993864in}{1.645535in}}%
\pgfpathcurveto{\pgfqpoint{2.004915in}{1.645535in}}{\pgfqpoint{2.015514in}{1.649925in}}{\pgfqpoint{2.023327in}{1.657738in}}%
\pgfpathcurveto{\pgfqpoint{2.031141in}{1.665552in}}{\pgfqpoint{2.035531in}{1.676151in}}{\pgfqpoint{2.035531in}{1.687201in}}%
\pgfpathcurveto{\pgfqpoint{2.035531in}{1.698251in}}{\pgfqpoint{2.031141in}{1.708850in}}{\pgfqpoint{2.023327in}{1.716664in}}%
\pgfpathcurveto{\pgfqpoint{2.015514in}{1.724478in}}{\pgfqpoint{2.004915in}{1.728868in}}{\pgfqpoint{1.993864in}{1.728868in}}%
\pgfpathcurveto{\pgfqpoint{1.982814in}{1.728868in}}{\pgfqpoint{1.972215in}{1.724478in}}{\pgfqpoint{1.964402in}{1.716664in}}%
\pgfpathcurveto{\pgfqpoint{1.956588in}{1.708850in}}{\pgfqpoint{1.952198in}{1.698251in}}{\pgfqpoint{1.952198in}{1.687201in}}%
\pgfpathcurveto{\pgfqpoint{1.952198in}{1.676151in}}{\pgfqpoint{1.956588in}{1.665552in}}{\pgfqpoint{1.964402in}{1.657738in}}%
\pgfpathcurveto{\pgfqpoint{1.972215in}{1.649925in}}{\pgfqpoint{1.982814in}{1.645535in}}{\pgfqpoint{1.993864in}{1.645535in}}%
\pgfpathlineto{\pgfqpoint{1.993864in}{1.645535in}}%
\pgfpathclose%
\pgfusepath{stroke}%
\end{pgfscope}%
\begin{pgfscope}%
\pgfpathrectangle{\pgfqpoint{0.847223in}{0.554012in}}{\pgfqpoint{6.200000in}{4.530000in}}%
\pgfusepath{clip}%
\pgfsetbuttcap%
\pgfsetroundjoin%
\pgfsetlinewidth{1.003750pt}%
\definecolor{currentstroke}{rgb}{1.000000,0.000000,0.000000}%
\pgfsetstrokecolor{currentstroke}%
\pgfsetdash{}{0pt}%
\pgfpathmoveto{\pgfqpoint{1.999198in}{1.640599in}}%
\pgfpathcurveto{\pgfqpoint{2.010248in}{1.640599in}}{\pgfqpoint{2.020847in}{1.644989in}}{\pgfqpoint{2.028660in}{1.652803in}}%
\pgfpathcurveto{\pgfqpoint{2.036474in}{1.660617in}}{\pgfqpoint{2.040864in}{1.671216in}}{\pgfqpoint{2.040864in}{1.682266in}}%
\pgfpathcurveto{\pgfqpoint{2.040864in}{1.693316in}}{\pgfqpoint{2.036474in}{1.703915in}}{\pgfqpoint{2.028660in}{1.711729in}}%
\pgfpathcurveto{\pgfqpoint{2.020847in}{1.719542in}}{\pgfqpoint{2.010248in}{1.723932in}}{\pgfqpoint{1.999198in}{1.723932in}}%
\pgfpathcurveto{\pgfqpoint{1.988148in}{1.723932in}}{\pgfqpoint{1.977548in}{1.719542in}}{\pgfqpoint{1.969735in}{1.711729in}}%
\pgfpathcurveto{\pgfqpoint{1.961921in}{1.703915in}}{\pgfqpoint{1.957531in}{1.693316in}}{\pgfqpoint{1.957531in}{1.682266in}}%
\pgfpathcurveto{\pgfqpoint{1.957531in}{1.671216in}}{\pgfqpoint{1.961921in}{1.660617in}}{\pgfqpoint{1.969735in}{1.652803in}}%
\pgfpathcurveto{\pgfqpoint{1.977548in}{1.644989in}}{\pgfqpoint{1.988148in}{1.640599in}}{\pgfqpoint{1.999198in}{1.640599in}}%
\pgfpathlineto{\pgfqpoint{1.999198in}{1.640599in}}%
\pgfpathclose%
\pgfusepath{stroke}%
\end{pgfscope}%
\begin{pgfscope}%
\pgfpathrectangle{\pgfqpoint{0.847223in}{0.554012in}}{\pgfqpoint{6.200000in}{4.530000in}}%
\pgfusepath{clip}%
\pgfsetbuttcap%
\pgfsetroundjoin%
\pgfsetlinewidth{1.003750pt}%
\definecolor{currentstroke}{rgb}{1.000000,0.000000,0.000000}%
\pgfsetstrokecolor{currentstroke}%
\pgfsetdash{}{0pt}%
\pgfpathmoveto{\pgfqpoint{2.004531in}{1.635695in}}%
\pgfpathcurveto{\pgfqpoint{2.015581in}{1.635695in}}{\pgfqpoint{2.026180in}{1.640085in}}{\pgfqpoint{2.033994in}{1.647899in}}%
\pgfpathcurveto{\pgfqpoint{2.041807in}{1.655713in}}{\pgfqpoint{2.046198in}{1.666312in}}{\pgfqpoint{2.046198in}{1.677362in}}%
\pgfpathcurveto{\pgfqpoint{2.046198in}{1.688412in}}{\pgfqpoint{2.041807in}{1.699011in}}{\pgfqpoint{2.033994in}{1.706824in}}%
\pgfpathcurveto{\pgfqpoint{2.026180in}{1.714638in}}{\pgfqpoint{2.015581in}{1.719028in}}{\pgfqpoint{2.004531in}{1.719028in}}%
\pgfpathcurveto{\pgfqpoint{1.993481in}{1.719028in}}{\pgfqpoint{1.982882in}{1.714638in}}{\pgfqpoint{1.975068in}{1.706824in}}%
\pgfpathcurveto{\pgfqpoint{1.967254in}{1.699011in}}{\pgfqpoint{1.962864in}{1.688412in}}{\pgfqpoint{1.962864in}{1.677362in}}%
\pgfpathcurveto{\pgfqpoint{1.962864in}{1.666312in}}{\pgfqpoint{1.967254in}{1.655713in}}{\pgfqpoint{1.975068in}{1.647899in}}%
\pgfpathcurveto{\pgfqpoint{1.982882in}{1.640085in}}{\pgfqpoint{1.993481in}{1.635695in}}{\pgfqpoint{2.004531in}{1.635695in}}%
\pgfpathlineto{\pgfqpoint{2.004531in}{1.635695in}}%
\pgfpathclose%
\pgfusepath{stroke}%
\end{pgfscope}%
\begin{pgfscope}%
\pgfpathrectangle{\pgfqpoint{0.847223in}{0.554012in}}{\pgfqpoint{6.200000in}{4.530000in}}%
\pgfusepath{clip}%
\pgfsetbuttcap%
\pgfsetroundjoin%
\pgfsetlinewidth{1.003750pt}%
\definecolor{currentstroke}{rgb}{1.000000,0.000000,0.000000}%
\pgfsetstrokecolor{currentstroke}%
\pgfsetdash{}{0pt}%
\pgfpathmoveto{\pgfqpoint{2.009864in}{1.630822in}}%
\pgfpathcurveto{\pgfqpoint{2.020914in}{1.630822in}}{\pgfqpoint{2.031513in}{1.635212in}}{\pgfqpoint{2.039327in}{1.643026in}}%
\pgfpathcurveto{\pgfqpoint{2.047140in}{1.650839in}}{\pgfqpoint{2.051531in}{1.661438in}}{\pgfqpoint{2.051531in}{1.672489in}}%
\pgfpathcurveto{\pgfqpoint{2.051531in}{1.683539in}}{\pgfqpoint{2.047140in}{1.694138in}}{\pgfqpoint{2.039327in}{1.701951in}}%
\pgfpathcurveto{\pgfqpoint{2.031513in}{1.709765in}}{\pgfqpoint{2.020914in}{1.714155in}}{\pgfqpoint{2.009864in}{1.714155in}}%
\pgfpathcurveto{\pgfqpoint{1.998814in}{1.714155in}}{\pgfqpoint{1.988215in}{1.709765in}}{\pgfqpoint{1.980401in}{1.701951in}}%
\pgfpathcurveto{\pgfqpoint{1.972588in}{1.694138in}}{\pgfqpoint{1.968197in}{1.683539in}}{\pgfqpoint{1.968197in}{1.672489in}}%
\pgfpathcurveto{\pgfqpoint{1.968197in}{1.661438in}}{\pgfqpoint{1.972588in}{1.650839in}}{\pgfqpoint{1.980401in}{1.643026in}}%
\pgfpathcurveto{\pgfqpoint{1.988215in}{1.635212in}}{\pgfqpoint{1.998814in}{1.630822in}}{\pgfqpoint{2.009864in}{1.630822in}}%
\pgfpathlineto{\pgfqpoint{2.009864in}{1.630822in}}%
\pgfpathclose%
\pgfusepath{stroke}%
\end{pgfscope}%
\begin{pgfscope}%
\pgfpathrectangle{\pgfqpoint{0.847223in}{0.554012in}}{\pgfqpoint{6.200000in}{4.530000in}}%
\pgfusepath{clip}%
\pgfsetbuttcap%
\pgfsetroundjoin%
\pgfsetlinewidth{1.003750pt}%
\definecolor{currentstroke}{rgb}{1.000000,0.000000,0.000000}%
\pgfsetstrokecolor{currentstroke}%
\pgfsetdash{}{0pt}%
\pgfpathmoveto{\pgfqpoint{2.015197in}{1.625980in}}%
\pgfpathcurveto{\pgfqpoint{2.026247in}{1.625980in}}{\pgfqpoint{2.036846in}{1.630370in}}{\pgfqpoint{2.044660in}{1.638183in}}%
\pgfpathcurveto{\pgfqpoint{2.052474in}{1.645997in}}{\pgfqpoint{2.056864in}{1.656596in}}{\pgfqpoint{2.056864in}{1.667646in}}%
\pgfpathcurveto{\pgfqpoint{2.056864in}{1.678696in}}{\pgfqpoint{2.052474in}{1.689295in}}{\pgfqpoint{2.044660in}{1.697109in}}%
\pgfpathcurveto{\pgfqpoint{2.036846in}{1.704923in}}{\pgfqpoint{2.026247in}{1.709313in}}{\pgfqpoint{2.015197in}{1.709313in}}%
\pgfpathcurveto{\pgfqpoint{2.004147in}{1.709313in}}{\pgfqpoint{1.993548in}{1.704923in}}{\pgfqpoint{1.985735in}{1.697109in}}%
\pgfpathcurveto{\pgfqpoint{1.977921in}{1.689295in}}{\pgfqpoint{1.973531in}{1.678696in}}{\pgfqpoint{1.973531in}{1.667646in}}%
\pgfpathcurveto{\pgfqpoint{1.973531in}{1.656596in}}{\pgfqpoint{1.977921in}{1.645997in}}{\pgfqpoint{1.985735in}{1.638183in}}%
\pgfpathcurveto{\pgfqpoint{1.993548in}{1.630370in}}{\pgfqpoint{2.004147in}{1.625980in}}{\pgfqpoint{2.015197in}{1.625980in}}%
\pgfpathlineto{\pgfqpoint{2.015197in}{1.625980in}}%
\pgfpathclose%
\pgfusepath{stroke}%
\end{pgfscope}%
\begin{pgfscope}%
\pgfpathrectangle{\pgfqpoint{0.847223in}{0.554012in}}{\pgfqpoint{6.200000in}{4.530000in}}%
\pgfusepath{clip}%
\pgfsetbuttcap%
\pgfsetroundjoin%
\pgfsetlinewidth{1.003750pt}%
\definecolor{currentstroke}{rgb}{1.000000,0.000000,0.000000}%
\pgfsetstrokecolor{currentstroke}%
\pgfsetdash{}{0pt}%
\pgfpathmoveto{\pgfqpoint{2.020531in}{1.621168in}}%
\pgfpathcurveto{\pgfqpoint{2.031581in}{1.621168in}}{\pgfqpoint{2.042180in}{1.625558in}}{\pgfqpoint{2.049993in}{1.633372in}}%
\pgfpathcurveto{\pgfqpoint{2.057807in}{1.641185in}}{\pgfqpoint{2.062197in}{1.651784in}}{\pgfqpoint{2.062197in}{1.662834in}}%
\pgfpathcurveto{\pgfqpoint{2.062197in}{1.673884in}}{\pgfqpoint{2.057807in}{1.684483in}}{\pgfqpoint{2.049993in}{1.692297in}}%
\pgfpathcurveto{\pgfqpoint{2.042180in}{1.700111in}}{\pgfqpoint{2.031581in}{1.704501in}}{\pgfqpoint{2.020531in}{1.704501in}}%
\pgfpathcurveto{\pgfqpoint{2.009480in}{1.704501in}}{\pgfqpoint{1.998881in}{1.700111in}}{\pgfqpoint{1.991068in}{1.692297in}}%
\pgfpathcurveto{\pgfqpoint{1.983254in}{1.684483in}}{\pgfqpoint{1.978864in}{1.673884in}}{\pgfqpoint{1.978864in}{1.662834in}}%
\pgfpathcurveto{\pgfqpoint{1.978864in}{1.651784in}}{\pgfqpoint{1.983254in}{1.641185in}}{\pgfqpoint{1.991068in}{1.633372in}}%
\pgfpathcurveto{\pgfqpoint{1.998881in}{1.625558in}}{\pgfqpoint{2.009480in}{1.621168in}}{\pgfqpoint{2.020531in}{1.621168in}}%
\pgfpathlineto{\pgfqpoint{2.020531in}{1.621168in}}%
\pgfpathclose%
\pgfusepath{stroke}%
\end{pgfscope}%
\begin{pgfscope}%
\pgfpathrectangle{\pgfqpoint{0.847223in}{0.554012in}}{\pgfqpoint{6.200000in}{4.530000in}}%
\pgfusepath{clip}%
\pgfsetbuttcap%
\pgfsetroundjoin%
\pgfsetlinewidth{1.003750pt}%
\definecolor{currentstroke}{rgb}{1.000000,0.000000,0.000000}%
\pgfsetstrokecolor{currentstroke}%
\pgfsetdash{}{0pt}%
\pgfpathmoveto{\pgfqpoint{2.025864in}{1.616386in}}%
\pgfpathcurveto{\pgfqpoint{2.036914in}{1.616386in}}{\pgfqpoint{2.047513in}{1.620776in}}{\pgfqpoint{2.055326in}{1.628590in}}%
\pgfpathcurveto{\pgfqpoint{2.063140in}{1.636403in}}{\pgfqpoint{2.067530in}{1.647002in}}{\pgfqpoint{2.067530in}{1.658053in}}%
\pgfpathcurveto{\pgfqpoint{2.067530in}{1.669103in}}{\pgfqpoint{2.063140in}{1.679702in}}{\pgfqpoint{2.055326in}{1.687515in}}%
\pgfpathcurveto{\pgfqpoint{2.047513in}{1.695329in}}{\pgfqpoint{2.036914in}{1.699719in}}{\pgfqpoint{2.025864in}{1.699719in}}%
\pgfpathcurveto{\pgfqpoint{2.014814in}{1.699719in}}{\pgfqpoint{2.004215in}{1.695329in}}{\pgfqpoint{1.996401in}{1.687515in}}%
\pgfpathcurveto{\pgfqpoint{1.988587in}{1.679702in}}{\pgfqpoint{1.984197in}{1.669103in}}{\pgfqpoint{1.984197in}{1.658053in}}%
\pgfpathcurveto{\pgfqpoint{1.984197in}{1.647002in}}{\pgfqpoint{1.988587in}{1.636403in}}{\pgfqpoint{1.996401in}{1.628590in}}%
\pgfpathcurveto{\pgfqpoint{2.004215in}{1.620776in}}{\pgfqpoint{2.014814in}{1.616386in}}{\pgfqpoint{2.025864in}{1.616386in}}%
\pgfpathlineto{\pgfqpoint{2.025864in}{1.616386in}}%
\pgfpathclose%
\pgfusepath{stroke}%
\end{pgfscope}%
\begin{pgfscope}%
\pgfpathrectangle{\pgfqpoint{0.847223in}{0.554012in}}{\pgfqpoint{6.200000in}{4.530000in}}%
\pgfusepath{clip}%
\pgfsetbuttcap%
\pgfsetroundjoin%
\pgfsetlinewidth{1.003750pt}%
\definecolor{currentstroke}{rgb}{1.000000,0.000000,0.000000}%
\pgfsetstrokecolor{currentstroke}%
\pgfsetdash{}{0pt}%
\pgfpathmoveto{\pgfqpoint{2.031197in}{1.611634in}}%
\pgfpathcurveto{\pgfqpoint{2.042247in}{1.611634in}}{\pgfqpoint{2.052846in}{1.616024in}}{\pgfqpoint{2.060660in}{1.623838in}}%
\pgfpathcurveto{\pgfqpoint{2.068473in}{1.631652in}}{\pgfqpoint{2.072864in}{1.642251in}}{\pgfqpoint{2.072864in}{1.653301in}}%
\pgfpathcurveto{\pgfqpoint{2.072864in}{1.664351in}}{\pgfqpoint{2.068473in}{1.674950in}}{\pgfqpoint{2.060660in}{1.682763in}}%
\pgfpathcurveto{\pgfqpoint{2.052846in}{1.690577in}}{\pgfqpoint{2.042247in}{1.694967in}}{\pgfqpoint{2.031197in}{1.694967in}}%
\pgfpathcurveto{\pgfqpoint{2.020147in}{1.694967in}}{\pgfqpoint{2.009548in}{1.690577in}}{\pgfqpoint{2.001734in}{1.682763in}}%
\pgfpathcurveto{\pgfqpoint{1.993921in}{1.674950in}}{\pgfqpoint{1.989530in}{1.664351in}}{\pgfqpoint{1.989530in}{1.653301in}}%
\pgfpathcurveto{\pgfqpoint{1.989530in}{1.642251in}}{\pgfqpoint{1.993921in}{1.631652in}}{\pgfqpoint{2.001734in}{1.623838in}}%
\pgfpathcurveto{\pgfqpoint{2.009548in}{1.616024in}}{\pgfqpoint{2.020147in}{1.611634in}}{\pgfqpoint{2.031197in}{1.611634in}}%
\pgfpathlineto{\pgfqpoint{2.031197in}{1.611634in}}%
\pgfpathclose%
\pgfusepath{stroke}%
\end{pgfscope}%
\begin{pgfscope}%
\pgfpathrectangle{\pgfqpoint{0.847223in}{0.554012in}}{\pgfqpoint{6.200000in}{4.530000in}}%
\pgfusepath{clip}%
\pgfsetbuttcap%
\pgfsetroundjoin%
\pgfsetlinewidth{1.003750pt}%
\definecolor{currentstroke}{rgb}{1.000000,0.000000,0.000000}%
\pgfsetstrokecolor{currentstroke}%
\pgfsetdash{}{0pt}%
\pgfpathmoveto{\pgfqpoint{2.036530in}{1.606912in}}%
\pgfpathcurveto{\pgfqpoint{2.047580in}{1.606912in}}{\pgfqpoint{2.058179in}{1.611302in}}{\pgfqpoint{2.065993in}{1.619116in}}%
\pgfpathcurveto{\pgfqpoint{2.073807in}{1.626929in}}{\pgfqpoint{2.078197in}{1.637528in}}{\pgfqpoint{2.078197in}{1.648578in}}%
\pgfpathcurveto{\pgfqpoint{2.078197in}{1.659629in}}{\pgfqpoint{2.073807in}{1.670228in}}{\pgfqpoint{2.065993in}{1.678041in}}%
\pgfpathcurveto{\pgfqpoint{2.058179in}{1.685855in}}{\pgfqpoint{2.047580in}{1.690245in}}{\pgfqpoint{2.036530in}{1.690245in}}%
\pgfpathcurveto{\pgfqpoint{2.025480in}{1.690245in}}{\pgfqpoint{2.014881in}{1.685855in}}{\pgfqpoint{2.007067in}{1.678041in}}%
\pgfpathcurveto{\pgfqpoint{1.999254in}{1.670228in}}{\pgfqpoint{1.994863in}{1.659629in}}{\pgfqpoint{1.994863in}{1.648578in}}%
\pgfpathcurveto{\pgfqpoint{1.994863in}{1.637528in}}{\pgfqpoint{1.999254in}{1.626929in}}{\pgfqpoint{2.007067in}{1.619116in}}%
\pgfpathcurveto{\pgfqpoint{2.014881in}{1.611302in}}{\pgfqpoint{2.025480in}{1.606912in}}{\pgfqpoint{2.036530in}{1.606912in}}%
\pgfpathlineto{\pgfqpoint{2.036530in}{1.606912in}}%
\pgfpathclose%
\pgfusepath{stroke}%
\end{pgfscope}%
\begin{pgfscope}%
\pgfpathrectangle{\pgfqpoint{0.847223in}{0.554012in}}{\pgfqpoint{6.200000in}{4.530000in}}%
\pgfusepath{clip}%
\pgfsetbuttcap%
\pgfsetroundjoin%
\pgfsetlinewidth{1.003750pt}%
\definecolor{currentstroke}{rgb}{1.000000,0.000000,0.000000}%
\pgfsetstrokecolor{currentstroke}%
\pgfsetdash{}{0pt}%
\pgfpathmoveto{\pgfqpoint{2.041863in}{1.602219in}}%
\pgfpathcurveto{\pgfqpoint{2.052913in}{1.602219in}}{\pgfqpoint{2.063513in}{1.606609in}}{\pgfqpoint{2.071326in}{1.614423in}}%
\pgfpathcurveto{\pgfqpoint{2.079140in}{1.622236in}}{\pgfqpoint{2.083530in}{1.632835in}}{\pgfqpoint{2.083530in}{1.643885in}}%
\pgfpathcurveto{\pgfqpoint{2.083530in}{1.654936in}}{\pgfqpoint{2.079140in}{1.665535in}}{\pgfqpoint{2.071326in}{1.673348in}}%
\pgfpathcurveto{\pgfqpoint{2.063513in}{1.681162in}}{\pgfqpoint{2.052913in}{1.685552in}}{\pgfqpoint{2.041863in}{1.685552in}}%
\pgfpathcurveto{\pgfqpoint{2.030813in}{1.685552in}}{\pgfqpoint{2.020214in}{1.681162in}}{\pgfqpoint{2.012401in}{1.673348in}}%
\pgfpathcurveto{\pgfqpoint{2.004587in}{1.665535in}}{\pgfqpoint{2.000197in}{1.654936in}}{\pgfqpoint{2.000197in}{1.643885in}}%
\pgfpathcurveto{\pgfqpoint{2.000197in}{1.632835in}}{\pgfqpoint{2.004587in}{1.622236in}}{\pgfqpoint{2.012401in}{1.614423in}}%
\pgfpathcurveto{\pgfqpoint{2.020214in}{1.606609in}}{\pgfqpoint{2.030813in}{1.602219in}}{\pgfqpoint{2.041863in}{1.602219in}}%
\pgfpathlineto{\pgfqpoint{2.041863in}{1.602219in}}%
\pgfpathclose%
\pgfusepath{stroke}%
\end{pgfscope}%
\begin{pgfscope}%
\pgfpathrectangle{\pgfqpoint{0.847223in}{0.554012in}}{\pgfqpoint{6.200000in}{4.530000in}}%
\pgfusepath{clip}%
\pgfsetbuttcap%
\pgfsetroundjoin%
\pgfsetlinewidth{1.003750pt}%
\definecolor{currentstroke}{rgb}{1.000000,0.000000,0.000000}%
\pgfsetstrokecolor{currentstroke}%
\pgfsetdash{}{0pt}%
\pgfpathmoveto{\pgfqpoint{2.047197in}{1.597555in}}%
\pgfpathcurveto{\pgfqpoint{2.058247in}{1.597555in}}{\pgfqpoint{2.068846in}{1.601945in}}{\pgfqpoint{2.076659in}{1.609759in}}%
\pgfpathcurveto{\pgfqpoint{2.084473in}{1.617572in}}{\pgfqpoint{2.088863in}{1.628171in}}{\pgfqpoint{2.088863in}{1.639221in}}%
\pgfpathcurveto{\pgfqpoint{2.088863in}{1.650272in}}{\pgfqpoint{2.084473in}{1.660871in}}{\pgfqpoint{2.076659in}{1.668684in}}%
\pgfpathcurveto{\pgfqpoint{2.068846in}{1.676498in}}{\pgfqpoint{2.058247in}{1.680888in}}{\pgfqpoint{2.047197in}{1.680888in}}%
\pgfpathcurveto{\pgfqpoint{2.036146in}{1.680888in}}{\pgfqpoint{2.025547in}{1.676498in}}{\pgfqpoint{2.017734in}{1.668684in}}%
\pgfpathcurveto{\pgfqpoint{2.009920in}{1.660871in}}{\pgfqpoint{2.005530in}{1.650272in}}{\pgfqpoint{2.005530in}{1.639221in}}%
\pgfpathcurveto{\pgfqpoint{2.005530in}{1.628171in}}{\pgfqpoint{2.009920in}{1.617572in}}{\pgfqpoint{2.017734in}{1.609759in}}%
\pgfpathcurveto{\pgfqpoint{2.025547in}{1.601945in}}{\pgfqpoint{2.036146in}{1.597555in}}{\pgfqpoint{2.047197in}{1.597555in}}%
\pgfpathlineto{\pgfqpoint{2.047197in}{1.597555in}}%
\pgfpathclose%
\pgfusepath{stroke}%
\end{pgfscope}%
\begin{pgfscope}%
\pgfpathrectangle{\pgfqpoint{0.847223in}{0.554012in}}{\pgfqpoint{6.200000in}{4.530000in}}%
\pgfusepath{clip}%
\pgfsetbuttcap%
\pgfsetroundjoin%
\pgfsetlinewidth{1.003750pt}%
\definecolor{currentstroke}{rgb}{1.000000,0.000000,0.000000}%
\pgfsetstrokecolor{currentstroke}%
\pgfsetdash{}{0pt}%
\pgfpathmoveto{\pgfqpoint{2.052530in}{1.592920in}}%
\pgfpathcurveto{\pgfqpoint{2.063580in}{1.592920in}}{\pgfqpoint{2.074179in}{1.597310in}}{\pgfqpoint{2.081993in}{1.605124in}}%
\pgfpathcurveto{\pgfqpoint{2.089806in}{1.612937in}}{\pgfqpoint{2.094196in}{1.623536in}}{\pgfqpoint{2.094196in}{1.634586in}}%
\pgfpathcurveto{\pgfqpoint{2.094196in}{1.645636in}}{\pgfqpoint{2.089806in}{1.656235in}}{\pgfqpoint{2.081993in}{1.664049in}}%
\pgfpathcurveto{\pgfqpoint{2.074179in}{1.671863in}}{\pgfqpoint{2.063580in}{1.676253in}}{\pgfqpoint{2.052530in}{1.676253in}}%
\pgfpathcurveto{\pgfqpoint{2.041480in}{1.676253in}}{\pgfqpoint{2.030881in}{1.671863in}}{\pgfqpoint{2.023067in}{1.664049in}}%
\pgfpathcurveto{\pgfqpoint{2.015253in}{1.656235in}}{\pgfqpoint{2.010863in}{1.645636in}}{\pgfqpoint{2.010863in}{1.634586in}}%
\pgfpathcurveto{\pgfqpoint{2.010863in}{1.623536in}}{\pgfqpoint{2.015253in}{1.612937in}}{\pgfqpoint{2.023067in}{1.605124in}}%
\pgfpathcurveto{\pgfqpoint{2.030881in}{1.597310in}}{\pgfqpoint{2.041480in}{1.592920in}}{\pgfqpoint{2.052530in}{1.592920in}}%
\pgfpathlineto{\pgfqpoint{2.052530in}{1.592920in}}%
\pgfpathclose%
\pgfusepath{stroke}%
\end{pgfscope}%
\begin{pgfscope}%
\pgfpathrectangle{\pgfqpoint{0.847223in}{0.554012in}}{\pgfqpoint{6.200000in}{4.530000in}}%
\pgfusepath{clip}%
\pgfsetbuttcap%
\pgfsetroundjoin%
\pgfsetlinewidth{1.003750pt}%
\definecolor{currentstroke}{rgb}{1.000000,0.000000,0.000000}%
\pgfsetstrokecolor{currentstroke}%
\pgfsetdash{}{0pt}%
\pgfpathmoveto{\pgfqpoint{2.057863in}{1.588313in}}%
\pgfpathcurveto{\pgfqpoint{2.068913in}{1.588313in}}{\pgfqpoint{2.079512in}{1.592703in}}{\pgfqpoint{2.087326in}{1.600517in}}%
\pgfpathcurveto{\pgfqpoint{2.095139in}{1.608331in}}{\pgfqpoint{2.099530in}{1.618930in}}{\pgfqpoint{2.099530in}{1.629980in}}%
\pgfpathcurveto{\pgfqpoint{2.099530in}{1.641030in}}{\pgfqpoint{2.095139in}{1.651629in}}{\pgfqpoint{2.087326in}{1.659442in}}%
\pgfpathcurveto{\pgfqpoint{2.079512in}{1.667256in}}{\pgfqpoint{2.068913in}{1.671646in}}{\pgfqpoint{2.057863in}{1.671646in}}%
\pgfpathcurveto{\pgfqpoint{2.046813in}{1.671646in}}{\pgfqpoint{2.036214in}{1.667256in}}{\pgfqpoint{2.028400in}{1.659442in}}%
\pgfpathcurveto{\pgfqpoint{2.020587in}{1.651629in}}{\pgfqpoint{2.016196in}{1.641030in}}{\pgfqpoint{2.016196in}{1.629980in}}%
\pgfpathcurveto{\pgfqpoint{2.016196in}{1.618930in}}{\pgfqpoint{2.020587in}{1.608331in}}{\pgfqpoint{2.028400in}{1.600517in}}%
\pgfpathcurveto{\pgfqpoint{2.036214in}{1.592703in}}{\pgfqpoint{2.046813in}{1.588313in}}{\pgfqpoint{2.057863in}{1.588313in}}%
\pgfpathlineto{\pgfqpoint{2.057863in}{1.588313in}}%
\pgfpathclose%
\pgfusepath{stroke}%
\end{pgfscope}%
\begin{pgfscope}%
\pgfpathrectangle{\pgfqpoint{0.847223in}{0.554012in}}{\pgfqpoint{6.200000in}{4.530000in}}%
\pgfusepath{clip}%
\pgfsetbuttcap%
\pgfsetroundjoin%
\pgfsetlinewidth{1.003750pt}%
\definecolor{currentstroke}{rgb}{1.000000,0.000000,0.000000}%
\pgfsetstrokecolor{currentstroke}%
\pgfsetdash{}{0pt}%
\pgfpathmoveto{\pgfqpoint{2.063196in}{1.583735in}}%
\pgfpathcurveto{\pgfqpoint{2.074246in}{1.583735in}}{\pgfqpoint{2.084845in}{1.588125in}}{\pgfqpoint{2.092659in}{1.595939in}}%
\pgfpathcurveto{\pgfqpoint{2.100473in}{1.603752in}}{\pgfqpoint{2.104863in}{1.614351in}}{\pgfqpoint{2.104863in}{1.625401in}}%
\pgfpathcurveto{\pgfqpoint{2.104863in}{1.636451in}}{\pgfqpoint{2.100473in}{1.647050in}}{\pgfqpoint{2.092659in}{1.654864in}}%
\pgfpathcurveto{\pgfqpoint{2.084845in}{1.662678in}}{\pgfqpoint{2.074246in}{1.667068in}}{\pgfqpoint{2.063196in}{1.667068in}}%
\pgfpathcurveto{\pgfqpoint{2.052146in}{1.667068in}}{\pgfqpoint{2.041547in}{1.662678in}}{\pgfqpoint{2.033733in}{1.654864in}}%
\pgfpathcurveto{\pgfqpoint{2.025920in}{1.647050in}}{\pgfqpoint{2.021530in}{1.636451in}}{\pgfqpoint{2.021530in}{1.625401in}}%
\pgfpathcurveto{\pgfqpoint{2.021530in}{1.614351in}}{\pgfqpoint{2.025920in}{1.603752in}}{\pgfqpoint{2.033733in}{1.595939in}}%
\pgfpathcurveto{\pgfqpoint{2.041547in}{1.588125in}}{\pgfqpoint{2.052146in}{1.583735in}}{\pgfqpoint{2.063196in}{1.583735in}}%
\pgfpathlineto{\pgfqpoint{2.063196in}{1.583735in}}%
\pgfpathclose%
\pgfusepath{stroke}%
\end{pgfscope}%
\begin{pgfscope}%
\pgfpathrectangle{\pgfqpoint{0.847223in}{0.554012in}}{\pgfqpoint{6.200000in}{4.530000in}}%
\pgfusepath{clip}%
\pgfsetbuttcap%
\pgfsetroundjoin%
\pgfsetlinewidth{1.003750pt}%
\definecolor{currentstroke}{rgb}{1.000000,0.000000,0.000000}%
\pgfsetstrokecolor{currentstroke}%
\pgfsetdash{}{0pt}%
\pgfpathmoveto{\pgfqpoint{2.068529in}{1.579184in}}%
\pgfpathcurveto{\pgfqpoint{2.079580in}{1.579184in}}{\pgfqpoint{2.090179in}{1.583574in}}{\pgfqpoint{2.097992in}{1.591388in}}%
\pgfpathcurveto{\pgfqpoint{2.105806in}{1.599202in}}{\pgfqpoint{2.110196in}{1.609801in}}{\pgfqpoint{2.110196in}{1.620851in}}%
\pgfpathcurveto{\pgfqpoint{2.110196in}{1.631901in}}{\pgfqpoint{2.105806in}{1.642500in}}{\pgfqpoint{2.097992in}{1.650314in}}%
\pgfpathcurveto{\pgfqpoint{2.090179in}{1.658127in}}{\pgfqpoint{2.079580in}{1.662518in}}{\pgfqpoint{2.068529in}{1.662518in}}%
\pgfpathcurveto{\pgfqpoint{2.057479in}{1.662518in}}{\pgfqpoint{2.046880in}{1.658127in}}{\pgfqpoint{2.039067in}{1.650314in}}%
\pgfpathcurveto{\pgfqpoint{2.031253in}{1.642500in}}{\pgfqpoint{2.026863in}{1.631901in}}{\pgfqpoint{2.026863in}{1.620851in}}%
\pgfpathcurveto{\pgfqpoint{2.026863in}{1.609801in}}{\pgfqpoint{2.031253in}{1.599202in}}{\pgfqpoint{2.039067in}{1.591388in}}%
\pgfpathcurveto{\pgfqpoint{2.046880in}{1.583574in}}{\pgfqpoint{2.057479in}{1.579184in}}{\pgfqpoint{2.068529in}{1.579184in}}%
\pgfpathlineto{\pgfqpoint{2.068529in}{1.579184in}}%
\pgfpathclose%
\pgfusepath{stroke}%
\end{pgfscope}%
\begin{pgfscope}%
\pgfpathrectangle{\pgfqpoint{0.847223in}{0.554012in}}{\pgfqpoint{6.200000in}{4.530000in}}%
\pgfusepath{clip}%
\pgfsetbuttcap%
\pgfsetroundjoin%
\pgfsetlinewidth{1.003750pt}%
\definecolor{currentstroke}{rgb}{1.000000,0.000000,0.000000}%
\pgfsetstrokecolor{currentstroke}%
\pgfsetdash{}{0pt}%
\pgfpathmoveto{\pgfqpoint{2.073863in}{1.574662in}}%
\pgfpathcurveto{\pgfqpoint{2.084913in}{1.574662in}}{\pgfqpoint{2.095512in}{1.579052in}}{\pgfqpoint{2.103325in}{1.586865in}}%
\pgfpathcurveto{\pgfqpoint{2.111139in}{1.594679in}}{\pgfqpoint{2.115529in}{1.605278in}}{\pgfqpoint{2.115529in}{1.616328in}}%
\pgfpathcurveto{\pgfqpoint{2.115529in}{1.627378in}}{\pgfqpoint{2.111139in}{1.637977in}}{\pgfqpoint{2.103325in}{1.645791in}}%
\pgfpathcurveto{\pgfqpoint{2.095512in}{1.653605in}}{\pgfqpoint{2.084913in}{1.657995in}}{\pgfqpoint{2.073863in}{1.657995in}}%
\pgfpathcurveto{\pgfqpoint{2.062813in}{1.657995in}}{\pgfqpoint{2.052213in}{1.653605in}}{\pgfqpoint{2.044400in}{1.645791in}}%
\pgfpathcurveto{\pgfqpoint{2.036586in}{1.637977in}}{\pgfqpoint{2.032196in}{1.627378in}}{\pgfqpoint{2.032196in}{1.616328in}}%
\pgfpathcurveto{\pgfqpoint{2.032196in}{1.605278in}}{\pgfqpoint{2.036586in}{1.594679in}}{\pgfqpoint{2.044400in}{1.586865in}}%
\pgfpathcurveto{\pgfqpoint{2.052213in}{1.579052in}}{\pgfqpoint{2.062813in}{1.574662in}}{\pgfqpoint{2.073863in}{1.574662in}}%
\pgfpathlineto{\pgfqpoint{2.073863in}{1.574662in}}%
\pgfpathclose%
\pgfusepath{stroke}%
\end{pgfscope}%
\begin{pgfscope}%
\pgfpathrectangle{\pgfqpoint{0.847223in}{0.554012in}}{\pgfqpoint{6.200000in}{4.530000in}}%
\pgfusepath{clip}%
\pgfsetbuttcap%
\pgfsetroundjoin%
\pgfsetlinewidth{1.003750pt}%
\definecolor{currentstroke}{rgb}{1.000000,0.000000,0.000000}%
\pgfsetstrokecolor{currentstroke}%
\pgfsetdash{}{0pt}%
\pgfpathmoveto{\pgfqpoint{2.079196in}{1.570166in}}%
\pgfpathcurveto{\pgfqpoint{2.090246in}{1.570166in}}{\pgfqpoint{2.100845in}{1.574557in}}{\pgfqpoint{2.108659in}{1.582370in}}%
\pgfpathcurveto{\pgfqpoint{2.116472in}{1.590184in}}{\pgfqpoint{2.120863in}{1.600783in}}{\pgfqpoint{2.120863in}{1.611833in}}%
\pgfpathcurveto{\pgfqpoint{2.120863in}{1.622883in}}{\pgfqpoint{2.116472in}{1.633482in}}{\pgfqpoint{2.108659in}{1.641296in}}%
\pgfpathcurveto{\pgfqpoint{2.100845in}{1.649109in}}{\pgfqpoint{2.090246in}{1.653500in}}{\pgfqpoint{2.079196in}{1.653500in}}%
\pgfpathcurveto{\pgfqpoint{2.068146in}{1.653500in}}{\pgfqpoint{2.057547in}{1.649109in}}{\pgfqpoint{2.049733in}{1.641296in}}%
\pgfpathcurveto{\pgfqpoint{2.041919in}{1.633482in}}{\pgfqpoint{2.037529in}{1.622883in}}{\pgfqpoint{2.037529in}{1.611833in}}%
\pgfpathcurveto{\pgfqpoint{2.037529in}{1.600783in}}{\pgfqpoint{2.041919in}{1.590184in}}{\pgfqpoint{2.049733in}{1.582370in}}%
\pgfpathcurveto{\pgfqpoint{2.057547in}{1.574557in}}{\pgfqpoint{2.068146in}{1.570166in}}{\pgfqpoint{2.079196in}{1.570166in}}%
\pgfpathlineto{\pgfqpoint{2.079196in}{1.570166in}}%
\pgfpathclose%
\pgfusepath{stroke}%
\end{pgfscope}%
\begin{pgfscope}%
\pgfpathrectangle{\pgfqpoint{0.847223in}{0.554012in}}{\pgfqpoint{6.200000in}{4.530000in}}%
\pgfusepath{clip}%
\pgfsetbuttcap%
\pgfsetroundjoin%
\pgfsetlinewidth{1.003750pt}%
\definecolor{currentstroke}{rgb}{1.000000,0.000000,0.000000}%
\pgfsetstrokecolor{currentstroke}%
\pgfsetdash{}{0pt}%
\pgfpathmoveto{\pgfqpoint{2.084529in}{1.565698in}}%
\pgfpathcurveto{\pgfqpoint{2.095579in}{1.565698in}}{\pgfqpoint{2.106178in}{1.570089in}}{\pgfqpoint{2.113992in}{1.577902in}}%
\pgfpathcurveto{\pgfqpoint{2.121805in}{1.585716in}}{\pgfqpoint{2.126196in}{1.596315in}}{\pgfqpoint{2.126196in}{1.607365in}}%
\pgfpathcurveto{\pgfqpoint{2.126196in}{1.618415in}}{\pgfqpoint{2.121805in}{1.629014in}}{\pgfqpoint{2.113992in}{1.636828in}}%
\pgfpathcurveto{\pgfqpoint{2.106178in}{1.644642in}}{\pgfqpoint{2.095579in}{1.649032in}}{\pgfqpoint{2.084529in}{1.649032in}}%
\pgfpathcurveto{\pgfqpoint{2.073479in}{1.649032in}}{\pgfqpoint{2.062880in}{1.644642in}}{\pgfqpoint{2.055066in}{1.636828in}}%
\pgfpathcurveto{\pgfqpoint{2.047253in}{1.629014in}}{\pgfqpoint{2.042862in}{1.618415in}}{\pgfqpoint{2.042862in}{1.607365in}}%
\pgfpathcurveto{\pgfqpoint{2.042862in}{1.596315in}}{\pgfqpoint{2.047253in}{1.585716in}}{\pgfqpoint{2.055066in}{1.577902in}}%
\pgfpathcurveto{\pgfqpoint{2.062880in}{1.570089in}}{\pgfqpoint{2.073479in}{1.565698in}}{\pgfqpoint{2.084529in}{1.565698in}}%
\pgfpathlineto{\pgfqpoint{2.084529in}{1.565698in}}%
\pgfpathclose%
\pgfusepath{stroke}%
\end{pgfscope}%
\begin{pgfscope}%
\pgfpathrectangle{\pgfqpoint{0.847223in}{0.554012in}}{\pgfqpoint{6.200000in}{4.530000in}}%
\pgfusepath{clip}%
\pgfsetbuttcap%
\pgfsetroundjoin%
\pgfsetlinewidth{1.003750pt}%
\definecolor{currentstroke}{rgb}{1.000000,0.000000,0.000000}%
\pgfsetstrokecolor{currentstroke}%
\pgfsetdash{}{0pt}%
\pgfpathmoveto{\pgfqpoint{2.089862in}{1.561257in}}%
\pgfpathcurveto{\pgfqpoint{2.100912in}{1.561257in}}{\pgfqpoint{2.111511in}{1.565648in}}{\pgfqpoint{2.119325in}{1.573461in}}%
\pgfpathcurveto{\pgfqpoint{2.127139in}{1.581275in}}{\pgfqpoint{2.131529in}{1.591874in}}{\pgfqpoint{2.131529in}{1.602924in}}%
\pgfpathcurveto{\pgfqpoint{2.131529in}{1.613974in}}{\pgfqpoint{2.127139in}{1.624573in}}{\pgfqpoint{2.119325in}{1.632387in}}%
\pgfpathcurveto{\pgfqpoint{2.111511in}{1.640201in}}{\pgfqpoint{2.100912in}{1.644591in}}{\pgfqpoint{2.089862in}{1.644591in}}%
\pgfpathcurveto{\pgfqpoint{2.078812in}{1.644591in}}{\pgfqpoint{2.068213in}{1.640201in}}{\pgfqpoint{2.060400in}{1.632387in}}%
\pgfpathcurveto{\pgfqpoint{2.052586in}{1.624573in}}{\pgfqpoint{2.048196in}{1.613974in}}{\pgfqpoint{2.048196in}{1.602924in}}%
\pgfpathcurveto{\pgfqpoint{2.048196in}{1.591874in}}{\pgfqpoint{2.052586in}{1.581275in}}{\pgfqpoint{2.060400in}{1.573461in}}%
\pgfpathcurveto{\pgfqpoint{2.068213in}{1.565648in}}{\pgfqpoint{2.078812in}{1.561257in}}{\pgfqpoint{2.089862in}{1.561257in}}%
\pgfpathlineto{\pgfqpoint{2.089862in}{1.561257in}}%
\pgfpathclose%
\pgfusepath{stroke}%
\end{pgfscope}%
\begin{pgfscope}%
\pgfpathrectangle{\pgfqpoint{0.847223in}{0.554012in}}{\pgfqpoint{6.200000in}{4.530000in}}%
\pgfusepath{clip}%
\pgfsetbuttcap%
\pgfsetroundjoin%
\pgfsetlinewidth{1.003750pt}%
\definecolor{currentstroke}{rgb}{1.000000,0.000000,0.000000}%
\pgfsetstrokecolor{currentstroke}%
\pgfsetdash{}{0pt}%
\pgfpathmoveto{\pgfqpoint{2.095196in}{1.556843in}}%
\pgfpathcurveto{\pgfqpoint{2.106246in}{1.556843in}}{\pgfqpoint{2.116845in}{1.561234in}}{\pgfqpoint{2.124658in}{1.569047in}}%
\pgfpathcurveto{\pgfqpoint{2.132472in}{1.576861in}}{\pgfqpoint{2.136862in}{1.587460in}}{\pgfqpoint{2.136862in}{1.598510in}}%
\pgfpathcurveto{\pgfqpoint{2.136862in}{1.609560in}}{\pgfqpoint{2.132472in}{1.620159in}}{\pgfqpoint{2.124658in}{1.627973in}}%
\pgfpathcurveto{\pgfqpoint{2.116845in}{1.635786in}}{\pgfqpoint{2.106246in}{1.640177in}}{\pgfqpoint{2.095196in}{1.640177in}}%
\pgfpathcurveto{\pgfqpoint{2.084145in}{1.640177in}}{\pgfqpoint{2.073546in}{1.635786in}}{\pgfqpoint{2.065733in}{1.627973in}}%
\pgfpathcurveto{\pgfqpoint{2.057919in}{1.620159in}}{\pgfqpoint{2.053529in}{1.609560in}}{\pgfqpoint{2.053529in}{1.598510in}}%
\pgfpathcurveto{\pgfqpoint{2.053529in}{1.587460in}}{\pgfqpoint{2.057919in}{1.576861in}}{\pgfqpoint{2.065733in}{1.569047in}}%
\pgfpathcurveto{\pgfqpoint{2.073546in}{1.561234in}}{\pgfqpoint{2.084145in}{1.556843in}}{\pgfqpoint{2.095196in}{1.556843in}}%
\pgfpathlineto{\pgfqpoint{2.095196in}{1.556843in}}%
\pgfpathclose%
\pgfusepath{stroke}%
\end{pgfscope}%
\begin{pgfscope}%
\pgfpathrectangle{\pgfqpoint{0.847223in}{0.554012in}}{\pgfqpoint{6.200000in}{4.530000in}}%
\pgfusepath{clip}%
\pgfsetbuttcap%
\pgfsetroundjoin%
\pgfsetlinewidth{1.003750pt}%
\definecolor{currentstroke}{rgb}{1.000000,0.000000,0.000000}%
\pgfsetstrokecolor{currentstroke}%
\pgfsetdash{}{0pt}%
\pgfpathmoveto{\pgfqpoint{2.100529in}{1.552456in}}%
\pgfpathcurveto{\pgfqpoint{2.111579in}{1.552456in}}{\pgfqpoint{2.122178in}{1.556846in}}{\pgfqpoint{2.129992in}{1.564659in}}%
\pgfpathcurveto{\pgfqpoint{2.137805in}{1.572473in}}{\pgfqpoint{2.142195in}{1.583072in}}{\pgfqpoint{2.142195in}{1.594122in}}%
\pgfpathcurveto{\pgfqpoint{2.142195in}{1.605172in}}{\pgfqpoint{2.137805in}{1.615771in}}{\pgfqpoint{2.129992in}{1.623585in}}%
\pgfpathcurveto{\pgfqpoint{2.122178in}{1.631399in}}{\pgfqpoint{2.111579in}{1.635789in}}{\pgfqpoint{2.100529in}{1.635789in}}%
\pgfpathcurveto{\pgfqpoint{2.089479in}{1.635789in}}{\pgfqpoint{2.078880in}{1.631399in}}{\pgfqpoint{2.071066in}{1.623585in}}%
\pgfpathcurveto{\pgfqpoint{2.063252in}{1.615771in}}{\pgfqpoint{2.058862in}{1.605172in}}{\pgfqpoint{2.058862in}{1.594122in}}%
\pgfpathcurveto{\pgfqpoint{2.058862in}{1.583072in}}{\pgfqpoint{2.063252in}{1.572473in}}{\pgfqpoint{2.071066in}{1.564659in}}%
\pgfpathcurveto{\pgfqpoint{2.078880in}{1.556846in}}{\pgfqpoint{2.089479in}{1.552456in}}{\pgfqpoint{2.100529in}{1.552456in}}%
\pgfpathlineto{\pgfqpoint{2.100529in}{1.552456in}}%
\pgfpathclose%
\pgfusepath{stroke}%
\end{pgfscope}%
\begin{pgfscope}%
\pgfpathrectangle{\pgfqpoint{0.847223in}{0.554012in}}{\pgfqpoint{6.200000in}{4.530000in}}%
\pgfusepath{clip}%
\pgfsetbuttcap%
\pgfsetroundjoin%
\pgfsetlinewidth{1.003750pt}%
\definecolor{currentstroke}{rgb}{1.000000,0.000000,0.000000}%
\pgfsetstrokecolor{currentstroke}%
\pgfsetdash{}{0pt}%
\pgfpathmoveto{\pgfqpoint{2.105862in}{1.548094in}}%
\pgfpathcurveto{\pgfqpoint{2.116912in}{1.548094in}}{\pgfqpoint{2.127511in}{1.552484in}}{\pgfqpoint{2.135325in}{1.560298in}}%
\pgfpathcurveto{\pgfqpoint{2.143138in}{1.568112in}}{\pgfqpoint{2.147529in}{1.578711in}}{\pgfqpoint{2.147529in}{1.589761in}}%
\pgfpathcurveto{\pgfqpoint{2.147529in}{1.600811in}}{\pgfqpoint{2.143138in}{1.611410in}}{\pgfqpoint{2.135325in}{1.619224in}}%
\pgfpathcurveto{\pgfqpoint{2.127511in}{1.627037in}}{\pgfqpoint{2.116912in}{1.631427in}}{\pgfqpoint{2.105862in}{1.631427in}}%
\pgfpathcurveto{\pgfqpoint{2.094812in}{1.631427in}}{\pgfqpoint{2.084213in}{1.627037in}}{\pgfqpoint{2.076399in}{1.619224in}}%
\pgfpathcurveto{\pgfqpoint{2.068586in}{1.611410in}}{\pgfqpoint{2.064195in}{1.600811in}}{\pgfqpoint{2.064195in}{1.589761in}}%
\pgfpathcurveto{\pgfqpoint{2.064195in}{1.578711in}}{\pgfqpoint{2.068586in}{1.568112in}}{\pgfqpoint{2.076399in}{1.560298in}}%
\pgfpathcurveto{\pgfqpoint{2.084213in}{1.552484in}}{\pgfqpoint{2.094812in}{1.548094in}}{\pgfqpoint{2.105862in}{1.548094in}}%
\pgfpathlineto{\pgfqpoint{2.105862in}{1.548094in}}%
\pgfpathclose%
\pgfusepath{stroke}%
\end{pgfscope}%
\begin{pgfscope}%
\pgfpathrectangle{\pgfqpoint{0.847223in}{0.554012in}}{\pgfqpoint{6.200000in}{4.530000in}}%
\pgfusepath{clip}%
\pgfsetbuttcap%
\pgfsetroundjoin%
\pgfsetlinewidth{1.003750pt}%
\definecolor{currentstroke}{rgb}{1.000000,0.000000,0.000000}%
\pgfsetstrokecolor{currentstroke}%
\pgfsetdash{}{0pt}%
\pgfpathmoveto{\pgfqpoint{2.111195in}{1.543759in}}%
\pgfpathcurveto{\pgfqpoint{2.122245in}{1.543759in}}{\pgfqpoint{2.132844in}{1.548149in}}{\pgfqpoint{2.140658in}{1.555963in}}%
\pgfpathcurveto{\pgfqpoint{2.148472in}{1.563776in}}{\pgfqpoint{2.152862in}{1.574375in}}{\pgfqpoint{2.152862in}{1.585425in}}%
\pgfpathcurveto{\pgfqpoint{2.152862in}{1.596475in}}{\pgfqpoint{2.148472in}{1.607075in}}{\pgfqpoint{2.140658in}{1.614888in}}%
\pgfpathcurveto{\pgfqpoint{2.132844in}{1.622702in}}{\pgfqpoint{2.122245in}{1.627092in}}{\pgfqpoint{2.111195in}{1.627092in}}%
\pgfpathcurveto{\pgfqpoint{2.100145in}{1.627092in}}{\pgfqpoint{2.089546in}{1.622702in}}{\pgfqpoint{2.081732in}{1.614888in}}%
\pgfpathcurveto{\pgfqpoint{2.073919in}{1.607075in}}{\pgfqpoint{2.069529in}{1.596475in}}{\pgfqpoint{2.069529in}{1.585425in}}%
\pgfpathcurveto{\pgfqpoint{2.069529in}{1.574375in}}{\pgfqpoint{2.073919in}{1.563776in}}{\pgfqpoint{2.081732in}{1.555963in}}%
\pgfpathcurveto{\pgfqpoint{2.089546in}{1.548149in}}{\pgfqpoint{2.100145in}{1.543759in}}{\pgfqpoint{2.111195in}{1.543759in}}%
\pgfpathlineto{\pgfqpoint{2.111195in}{1.543759in}}%
\pgfpathclose%
\pgfusepath{stroke}%
\end{pgfscope}%
\begin{pgfscope}%
\pgfpathrectangle{\pgfqpoint{0.847223in}{0.554012in}}{\pgfqpoint{6.200000in}{4.530000in}}%
\pgfusepath{clip}%
\pgfsetbuttcap%
\pgfsetroundjoin%
\pgfsetlinewidth{1.003750pt}%
\definecolor{currentstroke}{rgb}{1.000000,0.000000,0.000000}%
\pgfsetstrokecolor{currentstroke}%
\pgfsetdash{}{0pt}%
\pgfpathmoveto{\pgfqpoint{2.116528in}{1.539449in}}%
\pgfpathcurveto{\pgfqpoint{2.127579in}{1.539449in}}{\pgfqpoint{2.138178in}{1.543839in}}{\pgfqpoint{2.145991in}{1.551653in}}%
\pgfpathcurveto{\pgfqpoint{2.153805in}{1.559467in}}{\pgfqpoint{2.158195in}{1.570066in}}{\pgfqpoint{2.158195in}{1.581116in}}%
\pgfpathcurveto{\pgfqpoint{2.158195in}{1.592166in}}{\pgfqpoint{2.153805in}{1.602765in}}{\pgfqpoint{2.145991in}{1.610578in}}%
\pgfpathcurveto{\pgfqpoint{2.138178in}{1.618392in}}{\pgfqpoint{2.127579in}{1.622782in}}{\pgfqpoint{2.116528in}{1.622782in}}%
\pgfpathcurveto{\pgfqpoint{2.105478in}{1.622782in}}{\pgfqpoint{2.094879in}{1.618392in}}{\pgfqpoint{2.087066in}{1.610578in}}%
\pgfpathcurveto{\pgfqpoint{2.079252in}{1.602765in}}{\pgfqpoint{2.074862in}{1.592166in}}{\pgfqpoint{2.074862in}{1.581116in}}%
\pgfpathcurveto{\pgfqpoint{2.074862in}{1.570066in}}{\pgfqpoint{2.079252in}{1.559467in}}{\pgfqpoint{2.087066in}{1.551653in}}%
\pgfpathcurveto{\pgfqpoint{2.094879in}{1.543839in}}{\pgfqpoint{2.105478in}{1.539449in}}{\pgfqpoint{2.116528in}{1.539449in}}%
\pgfpathlineto{\pgfqpoint{2.116528in}{1.539449in}}%
\pgfpathclose%
\pgfusepath{stroke}%
\end{pgfscope}%
\begin{pgfscope}%
\pgfpathrectangle{\pgfqpoint{0.847223in}{0.554012in}}{\pgfqpoint{6.200000in}{4.530000in}}%
\pgfusepath{clip}%
\pgfsetbuttcap%
\pgfsetroundjoin%
\pgfsetlinewidth{1.003750pt}%
\definecolor{currentstroke}{rgb}{1.000000,0.000000,0.000000}%
\pgfsetstrokecolor{currentstroke}%
\pgfsetdash{}{0pt}%
\pgfpathmoveto{\pgfqpoint{2.121862in}{1.535165in}}%
\pgfpathcurveto{\pgfqpoint{2.132912in}{1.535165in}}{\pgfqpoint{2.143511in}{1.539555in}}{\pgfqpoint{2.151324in}{1.547369in}}%
\pgfpathcurveto{\pgfqpoint{2.159138in}{1.555182in}}{\pgfqpoint{2.163528in}{1.565782in}}{\pgfqpoint{2.163528in}{1.576832in}}%
\pgfpathcurveto{\pgfqpoint{2.163528in}{1.587882in}}{\pgfqpoint{2.159138in}{1.598481in}}{\pgfqpoint{2.151324in}{1.606294in}}%
\pgfpathcurveto{\pgfqpoint{2.143511in}{1.614108in}}{\pgfqpoint{2.132912in}{1.618498in}}{\pgfqpoint{2.121862in}{1.618498in}}%
\pgfpathcurveto{\pgfqpoint{2.110811in}{1.618498in}}{\pgfqpoint{2.100212in}{1.614108in}}{\pgfqpoint{2.092399in}{1.606294in}}%
\pgfpathcurveto{\pgfqpoint{2.084585in}{1.598481in}}{\pgfqpoint{2.080195in}{1.587882in}}{\pgfqpoint{2.080195in}{1.576832in}}%
\pgfpathcurveto{\pgfqpoint{2.080195in}{1.565782in}}{\pgfqpoint{2.084585in}{1.555182in}}{\pgfqpoint{2.092399in}{1.547369in}}%
\pgfpathcurveto{\pgfqpoint{2.100212in}{1.539555in}}{\pgfqpoint{2.110811in}{1.535165in}}{\pgfqpoint{2.121862in}{1.535165in}}%
\pgfpathlineto{\pgfqpoint{2.121862in}{1.535165in}}%
\pgfpathclose%
\pgfusepath{stroke}%
\end{pgfscope}%
\begin{pgfscope}%
\pgfpathrectangle{\pgfqpoint{0.847223in}{0.554012in}}{\pgfqpoint{6.200000in}{4.530000in}}%
\pgfusepath{clip}%
\pgfsetbuttcap%
\pgfsetroundjoin%
\pgfsetlinewidth{1.003750pt}%
\definecolor{currentstroke}{rgb}{1.000000,0.000000,0.000000}%
\pgfsetstrokecolor{currentstroke}%
\pgfsetdash{}{0pt}%
\pgfpathmoveto{\pgfqpoint{2.127195in}{1.530906in}}%
\pgfpathcurveto{\pgfqpoint{2.138245in}{1.530906in}}{\pgfqpoint{2.148844in}{1.535297in}}{\pgfqpoint{2.156658in}{1.543110in}}%
\pgfpathcurveto{\pgfqpoint{2.164471in}{1.550924in}}{\pgfqpoint{2.168861in}{1.561523in}}{\pgfqpoint{2.168861in}{1.572573in}}%
\pgfpathcurveto{\pgfqpoint{2.168861in}{1.583623in}}{\pgfqpoint{2.164471in}{1.594222in}}{\pgfqpoint{2.156658in}{1.602036in}}%
\pgfpathcurveto{\pgfqpoint{2.148844in}{1.609849in}}{\pgfqpoint{2.138245in}{1.614240in}}{\pgfqpoint{2.127195in}{1.614240in}}%
\pgfpathcurveto{\pgfqpoint{2.116145in}{1.614240in}}{\pgfqpoint{2.105546in}{1.609849in}}{\pgfqpoint{2.097732in}{1.602036in}}%
\pgfpathcurveto{\pgfqpoint{2.089918in}{1.594222in}}{\pgfqpoint{2.085528in}{1.583623in}}{\pgfqpoint{2.085528in}{1.572573in}}%
\pgfpathcurveto{\pgfqpoint{2.085528in}{1.561523in}}{\pgfqpoint{2.089918in}{1.550924in}}{\pgfqpoint{2.097732in}{1.543110in}}%
\pgfpathcurveto{\pgfqpoint{2.105546in}{1.535297in}}{\pgfqpoint{2.116145in}{1.530906in}}{\pgfqpoint{2.127195in}{1.530906in}}%
\pgfpathlineto{\pgfqpoint{2.127195in}{1.530906in}}%
\pgfpathclose%
\pgfusepath{stroke}%
\end{pgfscope}%
\begin{pgfscope}%
\pgfpathrectangle{\pgfqpoint{0.847223in}{0.554012in}}{\pgfqpoint{6.200000in}{4.530000in}}%
\pgfusepath{clip}%
\pgfsetbuttcap%
\pgfsetroundjoin%
\pgfsetlinewidth{1.003750pt}%
\definecolor{currentstroke}{rgb}{1.000000,0.000000,0.000000}%
\pgfsetstrokecolor{currentstroke}%
\pgfsetdash{}{0pt}%
\pgfpathmoveto{\pgfqpoint{2.132528in}{1.526673in}}%
\pgfpathcurveto{\pgfqpoint{2.143578in}{1.526673in}}{\pgfqpoint{2.154177in}{1.531063in}}{\pgfqpoint{2.161991in}{1.538877in}}%
\pgfpathcurveto{\pgfqpoint{2.169804in}{1.546690in}}{\pgfqpoint{2.174195in}{1.557289in}}{\pgfqpoint{2.174195in}{1.568339in}}%
\pgfpathcurveto{\pgfqpoint{2.174195in}{1.579389in}}{\pgfqpoint{2.169804in}{1.589988in}}{\pgfqpoint{2.161991in}{1.597802in}}%
\pgfpathcurveto{\pgfqpoint{2.154177in}{1.605616in}}{\pgfqpoint{2.143578in}{1.610006in}}{\pgfqpoint{2.132528in}{1.610006in}}%
\pgfpathcurveto{\pgfqpoint{2.121478in}{1.610006in}}{\pgfqpoint{2.110879in}{1.605616in}}{\pgfqpoint{2.103065in}{1.597802in}}%
\pgfpathcurveto{\pgfqpoint{2.095252in}{1.589988in}}{\pgfqpoint{2.090861in}{1.579389in}}{\pgfqpoint{2.090861in}{1.568339in}}%
\pgfpathcurveto{\pgfqpoint{2.090861in}{1.557289in}}{\pgfqpoint{2.095252in}{1.546690in}}{\pgfqpoint{2.103065in}{1.538877in}}%
\pgfpathcurveto{\pgfqpoint{2.110879in}{1.531063in}}{\pgfqpoint{2.121478in}{1.526673in}}{\pgfqpoint{2.132528in}{1.526673in}}%
\pgfpathlineto{\pgfqpoint{2.132528in}{1.526673in}}%
\pgfpathclose%
\pgfusepath{stroke}%
\end{pgfscope}%
\begin{pgfscope}%
\pgfpathrectangle{\pgfqpoint{0.847223in}{0.554012in}}{\pgfqpoint{6.200000in}{4.530000in}}%
\pgfusepath{clip}%
\pgfsetbuttcap%
\pgfsetroundjoin%
\pgfsetlinewidth{1.003750pt}%
\definecolor{currentstroke}{rgb}{1.000000,0.000000,0.000000}%
\pgfsetstrokecolor{currentstroke}%
\pgfsetdash{}{0pt}%
\pgfpathmoveto{\pgfqpoint{2.137861in}{1.522464in}}%
\pgfpathcurveto{\pgfqpoint{2.148911in}{1.522464in}}{\pgfqpoint{2.159510in}{1.526854in}}{\pgfqpoint{2.167324in}{1.534668in}}%
\pgfpathcurveto{\pgfqpoint{2.175138in}{1.542481in}}{\pgfqpoint{2.179528in}{1.553080in}}{\pgfqpoint{2.179528in}{1.564131in}}%
\pgfpathcurveto{\pgfqpoint{2.179528in}{1.575181in}}{\pgfqpoint{2.175138in}{1.585780in}}{\pgfqpoint{2.167324in}{1.593593in}}%
\pgfpathcurveto{\pgfqpoint{2.159510in}{1.601407in}}{\pgfqpoint{2.148911in}{1.605797in}}{\pgfqpoint{2.137861in}{1.605797in}}%
\pgfpathcurveto{\pgfqpoint{2.126811in}{1.605797in}}{\pgfqpoint{2.116212in}{1.601407in}}{\pgfqpoint{2.108398in}{1.593593in}}%
\pgfpathcurveto{\pgfqpoint{2.100585in}{1.585780in}}{\pgfqpoint{2.096195in}{1.575181in}}{\pgfqpoint{2.096195in}{1.564131in}}%
\pgfpathcurveto{\pgfqpoint{2.096195in}{1.553080in}}{\pgfqpoint{2.100585in}{1.542481in}}{\pgfqpoint{2.108398in}{1.534668in}}%
\pgfpathcurveto{\pgfqpoint{2.116212in}{1.526854in}}{\pgfqpoint{2.126811in}{1.522464in}}{\pgfqpoint{2.137861in}{1.522464in}}%
\pgfpathlineto{\pgfqpoint{2.137861in}{1.522464in}}%
\pgfpathclose%
\pgfusepath{stroke}%
\end{pgfscope}%
\begin{pgfscope}%
\pgfpathrectangle{\pgfqpoint{0.847223in}{0.554012in}}{\pgfqpoint{6.200000in}{4.530000in}}%
\pgfusepath{clip}%
\pgfsetbuttcap%
\pgfsetroundjoin%
\pgfsetlinewidth{1.003750pt}%
\definecolor{currentstroke}{rgb}{1.000000,0.000000,0.000000}%
\pgfsetstrokecolor{currentstroke}%
\pgfsetdash{}{0pt}%
\pgfpathmoveto{\pgfqpoint{2.143194in}{1.518280in}}%
\pgfpathcurveto{\pgfqpoint{2.154245in}{1.518280in}}{\pgfqpoint{2.164844in}{1.522670in}}{\pgfqpoint{2.172657in}{1.530484in}}%
\pgfpathcurveto{\pgfqpoint{2.180471in}{1.538297in}}{\pgfqpoint{2.184861in}{1.548896in}}{\pgfqpoint{2.184861in}{1.559947in}}%
\pgfpathcurveto{\pgfqpoint{2.184861in}{1.570997in}}{\pgfqpoint{2.180471in}{1.581596in}}{\pgfqpoint{2.172657in}{1.589409in}}%
\pgfpathcurveto{\pgfqpoint{2.164844in}{1.597223in}}{\pgfqpoint{2.154245in}{1.601613in}}{\pgfqpoint{2.143194in}{1.601613in}}%
\pgfpathcurveto{\pgfqpoint{2.132144in}{1.601613in}}{\pgfqpoint{2.121545in}{1.597223in}}{\pgfqpoint{2.113732in}{1.589409in}}%
\pgfpathcurveto{\pgfqpoint{2.105918in}{1.581596in}}{\pgfqpoint{2.101528in}{1.570997in}}{\pgfqpoint{2.101528in}{1.559947in}}%
\pgfpathcurveto{\pgfqpoint{2.101528in}{1.548896in}}{\pgfqpoint{2.105918in}{1.538297in}}{\pgfqpoint{2.113732in}{1.530484in}}%
\pgfpathcurveto{\pgfqpoint{2.121545in}{1.522670in}}{\pgfqpoint{2.132144in}{1.518280in}}{\pgfqpoint{2.143194in}{1.518280in}}%
\pgfpathlineto{\pgfqpoint{2.143194in}{1.518280in}}%
\pgfpathclose%
\pgfusepath{stroke}%
\end{pgfscope}%
\begin{pgfscope}%
\pgfpathrectangle{\pgfqpoint{0.847223in}{0.554012in}}{\pgfqpoint{6.200000in}{4.530000in}}%
\pgfusepath{clip}%
\pgfsetbuttcap%
\pgfsetroundjoin%
\pgfsetlinewidth{1.003750pt}%
\definecolor{currentstroke}{rgb}{1.000000,0.000000,0.000000}%
\pgfsetstrokecolor{currentstroke}%
\pgfsetdash{}{0pt}%
\pgfpathmoveto{\pgfqpoint{2.148528in}{1.514120in}}%
\pgfpathcurveto{\pgfqpoint{2.159578in}{1.514120in}}{\pgfqpoint{2.170177in}{1.518511in}}{\pgfqpoint{2.177990in}{1.526324in}}%
\pgfpathcurveto{\pgfqpoint{2.185804in}{1.534138in}}{\pgfqpoint{2.190194in}{1.544737in}}{\pgfqpoint{2.190194in}{1.555787in}}%
\pgfpathcurveto{\pgfqpoint{2.190194in}{1.566837in}}{\pgfqpoint{2.185804in}{1.577436in}}{\pgfqpoint{2.177990in}{1.585250in}}%
\pgfpathcurveto{\pgfqpoint{2.170177in}{1.593063in}}{\pgfqpoint{2.159578in}{1.597454in}}{\pgfqpoint{2.148528in}{1.597454in}}%
\pgfpathcurveto{\pgfqpoint{2.137478in}{1.597454in}}{\pgfqpoint{2.126879in}{1.593063in}}{\pgfqpoint{2.119065in}{1.585250in}}%
\pgfpathcurveto{\pgfqpoint{2.111251in}{1.577436in}}{\pgfqpoint{2.106861in}{1.566837in}}{\pgfqpoint{2.106861in}{1.555787in}}%
\pgfpathcurveto{\pgfqpoint{2.106861in}{1.544737in}}{\pgfqpoint{2.111251in}{1.534138in}}{\pgfqpoint{2.119065in}{1.526324in}}%
\pgfpathcurveto{\pgfqpoint{2.126879in}{1.518511in}}{\pgfqpoint{2.137478in}{1.514120in}}{\pgfqpoint{2.148528in}{1.514120in}}%
\pgfpathlineto{\pgfqpoint{2.148528in}{1.514120in}}%
\pgfpathclose%
\pgfusepath{stroke}%
\end{pgfscope}%
\begin{pgfscope}%
\pgfpathrectangle{\pgfqpoint{0.847223in}{0.554012in}}{\pgfqpoint{6.200000in}{4.530000in}}%
\pgfusepath{clip}%
\pgfsetbuttcap%
\pgfsetroundjoin%
\pgfsetlinewidth{1.003750pt}%
\definecolor{currentstroke}{rgb}{1.000000,0.000000,0.000000}%
\pgfsetstrokecolor{currentstroke}%
\pgfsetdash{}{0pt}%
\pgfpathmoveto{\pgfqpoint{2.153861in}{1.509985in}}%
\pgfpathcurveto{\pgfqpoint{2.164911in}{1.509985in}}{\pgfqpoint{2.175510in}{1.514375in}}{\pgfqpoint{2.183324in}{1.522189in}}%
\pgfpathcurveto{\pgfqpoint{2.191137in}{1.530003in}}{\pgfqpoint{2.195528in}{1.540602in}}{\pgfqpoint{2.195528in}{1.551652in}}%
\pgfpathcurveto{\pgfqpoint{2.195528in}{1.562702in}}{\pgfqpoint{2.191137in}{1.573301in}}{\pgfqpoint{2.183324in}{1.581114in}}%
\pgfpathcurveto{\pgfqpoint{2.175510in}{1.588928in}}{\pgfqpoint{2.164911in}{1.593318in}}{\pgfqpoint{2.153861in}{1.593318in}}%
\pgfpathcurveto{\pgfqpoint{2.142811in}{1.593318in}}{\pgfqpoint{2.132212in}{1.588928in}}{\pgfqpoint{2.124398in}{1.581114in}}%
\pgfpathcurveto{\pgfqpoint{2.116584in}{1.573301in}}{\pgfqpoint{2.112194in}{1.562702in}}{\pgfqpoint{2.112194in}{1.551652in}}%
\pgfpathcurveto{\pgfqpoint{2.112194in}{1.540602in}}{\pgfqpoint{2.116584in}{1.530003in}}{\pgfqpoint{2.124398in}{1.522189in}}%
\pgfpathcurveto{\pgfqpoint{2.132212in}{1.514375in}}{\pgfqpoint{2.142811in}{1.509985in}}{\pgfqpoint{2.153861in}{1.509985in}}%
\pgfpathlineto{\pgfqpoint{2.153861in}{1.509985in}}%
\pgfpathclose%
\pgfusepath{stroke}%
\end{pgfscope}%
\begin{pgfscope}%
\pgfpathrectangle{\pgfqpoint{0.847223in}{0.554012in}}{\pgfqpoint{6.200000in}{4.530000in}}%
\pgfusepath{clip}%
\pgfsetbuttcap%
\pgfsetroundjoin%
\pgfsetlinewidth{1.003750pt}%
\definecolor{currentstroke}{rgb}{1.000000,0.000000,0.000000}%
\pgfsetstrokecolor{currentstroke}%
\pgfsetdash{}{0pt}%
\pgfpathmoveto{\pgfqpoint{2.159194in}{1.505874in}}%
\pgfpathcurveto{\pgfqpoint{2.170244in}{1.505874in}}{\pgfqpoint{2.180843in}{1.510264in}}{\pgfqpoint{2.188657in}{1.518078in}}%
\pgfpathcurveto{\pgfqpoint{2.196471in}{1.525891in}}{\pgfqpoint{2.200861in}{1.536490in}}{\pgfqpoint{2.200861in}{1.547540in}}%
\pgfpathcurveto{\pgfqpoint{2.200861in}{1.558590in}}{\pgfqpoint{2.196471in}{1.569190in}}{\pgfqpoint{2.188657in}{1.577003in}}%
\pgfpathcurveto{\pgfqpoint{2.180843in}{1.584817in}}{\pgfqpoint{2.170244in}{1.589207in}}{\pgfqpoint{2.159194in}{1.589207in}}%
\pgfpathcurveto{\pgfqpoint{2.148144in}{1.589207in}}{\pgfqpoint{2.137545in}{1.584817in}}{\pgfqpoint{2.129731in}{1.577003in}}%
\pgfpathcurveto{\pgfqpoint{2.121918in}{1.569190in}}{\pgfqpoint{2.117527in}{1.558590in}}{\pgfqpoint{2.117527in}{1.547540in}}%
\pgfpathcurveto{\pgfqpoint{2.117527in}{1.536490in}}{\pgfqpoint{2.121918in}{1.525891in}}{\pgfqpoint{2.129731in}{1.518078in}}%
\pgfpathcurveto{\pgfqpoint{2.137545in}{1.510264in}}{\pgfqpoint{2.148144in}{1.505874in}}{\pgfqpoint{2.159194in}{1.505874in}}%
\pgfpathlineto{\pgfqpoint{2.159194in}{1.505874in}}%
\pgfpathclose%
\pgfusepath{stroke}%
\end{pgfscope}%
\begin{pgfscope}%
\pgfpathrectangle{\pgfqpoint{0.847223in}{0.554012in}}{\pgfqpoint{6.200000in}{4.530000in}}%
\pgfusepath{clip}%
\pgfsetbuttcap%
\pgfsetroundjoin%
\pgfsetlinewidth{1.003750pt}%
\definecolor{currentstroke}{rgb}{1.000000,0.000000,0.000000}%
\pgfsetstrokecolor{currentstroke}%
\pgfsetdash{}{0pt}%
\pgfpathmoveto{\pgfqpoint{2.164527in}{1.501786in}}%
\pgfpathcurveto{\pgfqpoint{2.175577in}{1.501786in}}{\pgfqpoint{2.186176in}{1.506176in}}{\pgfqpoint{2.193990in}{1.513990in}}%
\pgfpathcurveto{\pgfqpoint{2.201804in}{1.521804in}}{\pgfqpoint{2.206194in}{1.532403in}}{\pgfqpoint{2.206194in}{1.543453in}}%
\pgfpathcurveto{\pgfqpoint{2.206194in}{1.554503in}}{\pgfqpoint{2.201804in}{1.565102in}}{\pgfqpoint{2.193990in}{1.572916in}}%
\pgfpathcurveto{\pgfqpoint{2.186176in}{1.580729in}}{\pgfqpoint{2.175577in}{1.585120in}}{\pgfqpoint{2.164527in}{1.585120in}}%
\pgfpathcurveto{\pgfqpoint{2.153477in}{1.585120in}}{\pgfqpoint{2.142878in}{1.580729in}}{\pgfqpoint{2.135065in}{1.572916in}}%
\pgfpathcurveto{\pgfqpoint{2.127251in}{1.565102in}}{\pgfqpoint{2.122861in}{1.554503in}}{\pgfqpoint{2.122861in}{1.543453in}}%
\pgfpathcurveto{\pgfqpoint{2.122861in}{1.532403in}}{\pgfqpoint{2.127251in}{1.521804in}}{\pgfqpoint{2.135065in}{1.513990in}}%
\pgfpathcurveto{\pgfqpoint{2.142878in}{1.506176in}}{\pgfqpoint{2.153477in}{1.501786in}}{\pgfqpoint{2.164527in}{1.501786in}}%
\pgfpathlineto{\pgfqpoint{2.164527in}{1.501786in}}%
\pgfpathclose%
\pgfusepath{stroke}%
\end{pgfscope}%
\begin{pgfscope}%
\pgfpathrectangle{\pgfqpoint{0.847223in}{0.554012in}}{\pgfqpoint{6.200000in}{4.530000in}}%
\pgfusepath{clip}%
\pgfsetbuttcap%
\pgfsetroundjoin%
\pgfsetlinewidth{1.003750pt}%
\definecolor{currentstroke}{rgb}{1.000000,0.000000,0.000000}%
\pgfsetstrokecolor{currentstroke}%
\pgfsetdash{}{0pt}%
\pgfpathmoveto{\pgfqpoint{2.169861in}{1.497722in}}%
\pgfpathcurveto{\pgfqpoint{2.180911in}{1.497722in}}{\pgfqpoint{2.191510in}{1.502113in}}{\pgfqpoint{2.199323in}{1.509926in}}%
\pgfpathcurveto{\pgfqpoint{2.207137in}{1.517740in}}{\pgfqpoint{2.211527in}{1.528339in}}{\pgfqpoint{2.211527in}{1.539389in}}%
\pgfpathcurveto{\pgfqpoint{2.211527in}{1.550439in}}{\pgfqpoint{2.207137in}{1.561038in}}{\pgfqpoint{2.199323in}{1.568852in}}%
\pgfpathcurveto{\pgfqpoint{2.191510in}{1.576665in}}{\pgfqpoint{2.180911in}{1.581056in}}{\pgfqpoint{2.169861in}{1.581056in}}%
\pgfpathcurveto{\pgfqpoint{2.158810in}{1.581056in}}{\pgfqpoint{2.148211in}{1.576665in}}{\pgfqpoint{2.140398in}{1.568852in}}%
\pgfpathcurveto{\pgfqpoint{2.132584in}{1.561038in}}{\pgfqpoint{2.128194in}{1.550439in}}{\pgfqpoint{2.128194in}{1.539389in}}%
\pgfpathcurveto{\pgfqpoint{2.128194in}{1.528339in}}{\pgfqpoint{2.132584in}{1.517740in}}{\pgfqpoint{2.140398in}{1.509926in}}%
\pgfpathcurveto{\pgfqpoint{2.148211in}{1.502113in}}{\pgfqpoint{2.158810in}{1.497722in}}{\pgfqpoint{2.169861in}{1.497722in}}%
\pgfpathlineto{\pgfqpoint{2.169861in}{1.497722in}}%
\pgfpathclose%
\pgfusepath{stroke}%
\end{pgfscope}%
\begin{pgfscope}%
\pgfpathrectangle{\pgfqpoint{0.847223in}{0.554012in}}{\pgfqpoint{6.200000in}{4.530000in}}%
\pgfusepath{clip}%
\pgfsetbuttcap%
\pgfsetroundjoin%
\pgfsetlinewidth{1.003750pt}%
\definecolor{currentstroke}{rgb}{1.000000,0.000000,0.000000}%
\pgfsetstrokecolor{currentstroke}%
\pgfsetdash{}{0pt}%
\pgfpathmoveto{\pgfqpoint{2.175194in}{1.493682in}}%
\pgfpathcurveto{\pgfqpoint{2.186244in}{1.493682in}}{\pgfqpoint{2.196843in}{1.498072in}}{\pgfqpoint{2.204657in}{1.505886in}}%
\pgfpathcurveto{\pgfqpoint{2.212470in}{1.513699in}}{\pgfqpoint{2.216860in}{1.524298in}}{\pgfqpoint{2.216860in}{1.535349in}}%
\pgfpathcurveto{\pgfqpoint{2.216860in}{1.546399in}}{\pgfqpoint{2.212470in}{1.556998in}}{\pgfqpoint{2.204657in}{1.564811in}}%
\pgfpathcurveto{\pgfqpoint{2.196843in}{1.572625in}}{\pgfqpoint{2.186244in}{1.577015in}}{\pgfqpoint{2.175194in}{1.577015in}}%
\pgfpathcurveto{\pgfqpoint{2.164144in}{1.577015in}}{\pgfqpoint{2.153545in}{1.572625in}}{\pgfqpoint{2.145731in}{1.564811in}}%
\pgfpathcurveto{\pgfqpoint{2.137917in}{1.556998in}}{\pgfqpoint{2.133527in}{1.546399in}}{\pgfqpoint{2.133527in}{1.535349in}}%
\pgfpathcurveto{\pgfqpoint{2.133527in}{1.524298in}}{\pgfqpoint{2.137917in}{1.513699in}}{\pgfqpoint{2.145731in}{1.505886in}}%
\pgfpathcurveto{\pgfqpoint{2.153545in}{1.498072in}}{\pgfqpoint{2.164144in}{1.493682in}}{\pgfqpoint{2.175194in}{1.493682in}}%
\pgfpathlineto{\pgfqpoint{2.175194in}{1.493682in}}%
\pgfpathclose%
\pgfusepath{stroke}%
\end{pgfscope}%
\begin{pgfscope}%
\pgfpathrectangle{\pgfqpoint{0.847223in}{0.554012in}}{\pgfqpoint{6.200000in}{4.530000in}}%
\pgfusepath{clip}%
\pgfsetbuttcap%
\pgfsetroundjoin%
\pgfsetlinewidth{1.003750pt}%
\definecolor{currentstroke}{rgb}{1.000000,0.000000,0.000000}%
\pgfsetstrokecolor{currentstroke}%
\pgfsetdash{}{0pt}%
\pgfpathmoveto{\pgfqpoint{2.180527in}{1.489665in}}%
\pgfpathcurveto{\pgfqpoint{2.191577in}{1.489665in}}{\pgfqpoint{2.202176in}{1.494055in}}{\pgfqpoint{2.209990in}{1.501869in}}%
\pgfpathcurveto{\pgfqpoint{2.217803in}{1.509682in}}{\pgfqpoint{2.222194in}{1.520281in}}{\pgfqpoint{2.222194in}{1.531331in}}%
\pgfpathcurveto{\pgfqpoint{2.222194in}{1.542381in}}{\pgfqpoint{2.217803in}{1.552981in}}{\pgfqpoint{2.209990in}{1.560794in}}%
\pgfpathcurveto{\pgfqpoint{2.202176in}{1.568608in}}{\pgfqpoint{2.191577in}{1.572998in}}{\pgfqpoint{2.180527in}{1.572998in}}%
\pgfpathcurveto{\pgfqpoint{2.169477in}{1.572998in}}{\pgfqpoint{2.158878in}{1.568608in}}{\pgfqpoint{2.151064in}{1.560794in}}%
\pgfpathcurveto{\pgfqpoint{2.143251in}{1.552981in}}{\pgfqpoint{2.138860in}{1.542381in}}{\pgfqpoint{2.138860in}{1.531331in}}%
\pgfpathcurveto{\pgfqpoint{2.138860in}{1.520281in}}{\pgfqpoint{2.143251in}{1.509682in}}{\pgfqpoint{2.151064in}{1.501869in}}%
\pgfpathcurveto{\pgfqpoint{2.158878in}{1.494055in}}{\pgfqpoint{2.169477in}{1.489665in}}{\pgfqpoint{2.180527in}{1.489665in}}%
\pgfpathlineto{\pgfqpoint{2.180527in}{1.489665in}}%
\pgfpathclose%
\pgfusepath{stroke}%
\end{pgfscope}%
\begin{pgfscope}%
\pgfpathrectangle{\pgfqpoint{0.847223in}{0.554012in}}{\pgfqpoint{6.200000in}{4.530000in}}%
\pgfusepath{clip}%
\pgfsetbuttcap%
\pgfsetroundjoin%
\pgfsetlinewidth{1.003750pt}%
\definecolor{currentstroke}{rgb}{1.000000,0.000000,0.000000}%
\pgfsetstrokecolor{currentstroke}%
\pgfsetdash{}{0pt}%
\pgfpathmoveto{\pgfqpoint{2.185860in}{1.485670in}}%
\pgfpathcurveto{\pgfqpoint{2.196910in}{1.485670in}}{\pgfqpoint{2.207509in}{1.490061in}}{\pgfqpoint{2.215323in}{1.497874in}}%
\pgfpathcurveto{\pgfqpoint{2.223137in}{1.505688in}}{\pgfqpoint{2.227527in}{1.516287in}}{\pgfqpoint{2.227527in}{1.527337in}}%
\pgfpathcurveto{\pgfqpoint{2.227527in}{1.538387in}}{\pgfqpoint{2.223137in}{1.548986in}}{\pgfqpoint{2.215323in}{1.556800in}}%
\pgfpathcurveto{\pgfqpoint{2.207509in}{1.564614in}}{\pgfqpoint{2.196910in}{1.569004in}}{\pgfqpoint{2.185860in}{1.569004in}}%
\pgfpathcurveto{\pgfqpoint{2.174810in}{1.569004in}}{\pgfqpoint{2.164211in}{1.564614in}}{\pgfqpoint{2.156397in}{1.556800in}}%
\pgfpathcurveto{\pgfqpoint{2.148584in}{1.548986in}}{\pgfqpoint{2.144194in}{1.538387in}}{\pgfqpoint{2.144194in}{1.527337in}}%
\pgfpathcurveto{\pgfqpoint{2.144194in}{1.516287in}}{\pgfqpoint{2.148584in}{1.505688in}}{\pgfqpoint{2.156397in}{1.497874in}}%
\pgfpathcurveto{\pgfqpoint{2.164211in}{1.490061in}}{\pgfqpoint{2.174810in}{1.485670in}}{\pgfqpoint{2.185860in}{1.485670in}}%
\pgfpathlineto{\pgfqpoint{2.185860in}{1.485670in}}%
\pgfpathclose%
\pgfusepath{stroke}%
\end{pgfscope}%
\begin{pgfscope}%
\pgfpathrectangle{\pgfqpoint{0.847223in}{0.554012in}}{\pgfqpoint{6.200000in}{4.530000in}}%
\pgfusepath{clip}%
\pgfsetbuttcap%
\pgfsetroundjoin%
\pgfsetlinewidth{1.003750pt}%
\definecolor{currentstroke}{rgb}{1.000000,0.000000,0.000000}%
\pgfsetstrokecolor{currentstroke}%
\pgfsetdash{}{0pt}%
\pgfpathmoveto{\pgfqpoint{2.191193in}{1.481699in}}%
\pgfpathcurveto{\pgfqpoint{2.202244in}{1.481699in}}{\pgfqpoint{2.212843in}{1.486089in}}{\pgfqpoint{2.220656in}{1.493903in}}%
\pgfpathcurveto{\pgfqpoint{2.228470in}{1.501717in}}{\pgfqpoint{2.232860in}{1.512316in}}{\pgfqpoint{2.232860in}{1.523366in}}%
\pgfpathcurveto{\pgfqpoint{2.232860in}{1.534416in}}{\pgfqpoint{2.228470in}{1.545015in}}{\pgfqpoint{2.220656in}{1.552829in}}%
\pgfpathcurveto{\pgfqpoint{2.212843in}{1.560642in}}{\pgfqpoint{2.202244in}{1.565032in}}{\pgfqpoint{2.191193in}{1.565032in}}%
\pgfpathcurveto{\pgfqpoint{2.180143in}{1.565032in}}{\pgfqpoint{2.169544in}{1.560642in}}{\pgfqpoint{2.161731in}{1.552829in}}%
\pgfpathcurveto{\pgfqpoint{2.153917in}{1.545015in}}{\pgfqpoint{2.149527in}{1.534416in}}{\pgfqpoint{2.149527in}{1.523366in}}%
\pgfpathcurveto{\pgfqpoint{2.149527in}{1.512316in}}{\pgfqpoint{2.153917in}{1.501717in}}{\pgfqpoint{2.161731in}{1.493903in}}%
\pgfpathcurveto{\pgfqpoint{2.169544in}{1.486089in}}{\pgfqpoint{2.180143in}{1.481699in}}{\pgfqpoint{2.191193in}{1.481699in}}%
\pgfpathlineto{\pgfqpoint{2.191193in}{1.481699in}}%
\pgfpathclose%
\pgfusepath{stroke}%
\end{pgfscope}%
\begin{pgfscope}%
\pgfpathrectangle{\pgfqpoint{0.847223in}{0.554012in}}{\pgfqpoint{6.200000in}{4.530000in}}%
\pgfusepath{clip}%
\pgfsetbuttcap%
\pgfsetroundjoin%
\pgfsetlinewidth{1.003750pt}%
\definecolor{currentstroke}{rgb}{1.000000,0.000000,0.000000}%
\pgfsetstrokecolor{currentstroke}%
\pgfsetdash{}{0pt}%
\pgfpathmoveto{\pgfqpoint{2.196527in}{1.477750in}}%
\pgfpathcurveto{\pgfqpoint{2.207577in}{1.477750in}}{\pgfqpoint{2.218176in}{1.482141in}}{\pgfqpoint{2.225989in}{1.489954in}}%
\pgfpathcurveto{\pgfqpoint{2.233803in}{1.497768in}}{\pgfqpoint{2.238193in}{1.508367in}}{\pgfqpoint{2.238193in}{1.519417in}}%
\pgfpathcurveto{\pgfqpoint{2.238193in}{1.530467in}}{\pgfqpoint{2.233803in}{1.541066in}}{\pgfqpoint{2.225989in}{1.548880in}}%
\pgfpathcurveto{\pgfqpoint{2.218176in}{1.556693in}}{\pgfqpoint{2.207577in}{1.561084in}}{\pgfqpoint{2.196527in}{1.561084in}}%
\pgfpathcurveto{\pgfqpoint{2.185476in}{1.561084in}}{\pgfqpoint{2.174877in}{1.556693in}}{\pgfqpoint{2.167064in}{1.548880in}}%
\pgfpathcurveto{\pgfqpoint{2.159250in}{1.541066in}}{\pgfqpoint{2.154860in}{1.530467in}}{\pgfqpoint{2.154860in}{1.519417in}}%
\pgfpathcurveto{\pgfqpoint{2.154860in}{1.508367in}}{\pgfqpoint{2.159250in}{1.497768in}}{\pgfqpoint{2.167064in}{1.489954in}}%
\pgfpathcurveto{\pgfqpoint{2.174877in}{1.482141in}}{\pgfqpoint{2.185476in}{1.477750in}}{\pgfqpoint{2.196527in}{1.477750in}}%
\pgfpathlineto{\pgfqpoint{2.196527in}{1.477750in}}%
\pgfpathclose%
\pgfusepath{stroke}%
\end{pgfscope}%
\begin{pgfscope}%
\pgfpathrectangle{\pgfqpoint{0.847223in}{0.554012in}}{\pgfqpoint{6.200000in}{4.530000in}}%
\pgfusepath{clip}%
\pgfsetbuttcap%
\pgfsetroundjoin%
\pgfsetlinewidth{1.003750pt}%
\definecolor{currentstroke}{rgb}{1.000000,0.000000,0.000000}%
\pgfsetstrokecolor{currentstroke}%
\pgfsetdash{}{0pt}%
\pgfpathmoveto{\pgfqpoint{2.201860in}{1.473824in}}%
\pgfpathcurveto{\pgfqpoint{2.212910in}{1.473824in}}{\pgfqpoint{2.223509in}{1.478214in}}{\pgfqpoint{2.231323in}{1.486028in}}%
\pgfpathcurveto{\pgfqpoint{2.239136in}{1.493841in}}{\pgfqpoint{2.243526in}{1.504440in}}{\pgfqpoint{2.243526in}{1.515491in}}%
\pgfpathcurveto{\pgfqpoint{2.243526in}{1.526541in}}{\pgfqpoint{2.239136in}{1.537140in}}{\pgfqpoint{2.231323in}{1.544953in}}%
\pgfpathcurveto{\pgfqpoint{2.223509in}{1.552767in}}{\pgfqpoint{2.212910in}{1.557157in}}{\pgfqpoint{2.201860in}{1.557157in}}%
\pgfpathcurveto{\pgfqpoint{2.190810in}{1.557157in}}{\pgfqpoint{2.180211in}{1.552767in}}{\pgfqpoint{2.172397in}{1.544953in}}%
\pgfpathcurveto{\pgfqpoint{2.164583in}{1.537140in}}{\pgfqpoint{2.160193in}{1.526541in}}{\pgfqpoint{2.160193in}{1.515491in}}%
\pgfpathcurveto{\pgfqpoint{2.160193in}{1.504440in}}{\pgfqpoint{2.164583in}{1.493841in}}{\pgfqpoint{2.172397in}{1.486028in}}%
\pgfpathcurveto{\pgfqpoint{2.180211in}{1.478214in}}{\pgfqpoint{2.190810in}{1.473824in}}{\pgfqpoint{2.201860in}{1.473824in}}%
\pgfpathlineto{\pgfqpoint{2.201860in}{1.473824in}}%
\pgfpathclose%
\pgfusepath{stroke}%
\end{pgfscope}%
\begin{pgfscope}%
\pgfpathrectangle{\pgfqpoint{0.847223in}{0.554012in}}{\pgfqpoint{6.200000in}{4.530000in}}%
\pgfusepath{clip}%
\pgfsetbuttcap%
\pgfsetroundjoin%
\pgfsetlinewidth{1.003750pt}%
\definecolor{currentstroke}{rgb}{1.000000,0.000000,0.000000}%
\pgfsetstrokecolor{currentstroke}%
\pgfsetdash{}{0pt}%
\pgfpathmoveto{\pgfqpoint{2.207193in}{1.469920in}}%
\pgfpathcurveto{\pgfqpoint{2.218243in}{1.469920in}}{\pgfqpoint{2.228842in}{1.474310in}}{\pgfqpoint{2.236656in}{1.482124in}}%
\pgfpathcurveto{\pgfqpoint{2.244469in}{1.489937in}}{\pgfqpoint{2.248860in}{1.500536in}}{\pgfqpoint{2.248860in}{1.511586in}}%
\pgfpathcurveto{\pgfqpoint{2.248860in}{1.522637in}}{\pgfqpoint{2.244469in}{1.533236in}}{\pgfqpoint{2.236656in}{1.541049in}}%
\pgfpathcurveto{\pgfqpoint{2.228842in}{1.548863in}}{\pgfqpoint{2.218243in}{1.553253in}}{\pgfqpoint{2.207193in}{1.553253in}}%
\pgfpathcurveto{\pgfqpoint{2.196143in}{1.553253in}}{\pgfqpoint{2.185544in}{1.548863in}}{\pgfqpoint{2.177730in}{1.541049in}}%
\pgfpathcurveto{\pgfqpoint{2.169917in}{1.533236in}}{\pgfqpoint{2.165526in}{1.522637in}}{\pgfqpoint{2.165526in}{1.511586in}}%
\pgfpathcurveto{\pgfqpoint{2.165526in}{1.500536in}}{\pgfqpoint{2.169917in}{1.489937in}}{\pgfqpoint{2.177730in}{1.482124in}}%
\pgfpathcurveto{\pgfqpoint{2.185544in}{1.474310in}}{\pgfqpoint{2.196143in}{1.469920in}}{\pgfqpoint{2.207193in}{1.469920in}}%
\pgfpathlineto{\pgfqpoint{2.207193in}{1.469920in}}%
\pgfpathclose%
\pgfusepath{stroke}%
\end{pgfscope}%
\begin{pgfscope}%
\pgfpathrectangle{\pgfqpoint{0.847223in}{0.554012in}}{\pgfqpoint{6.200000in}{4.530000in}}%
\pgfusepath{clip}%
\pgfsetbuttcap%
\pgfsetroundjoin%
\pgfsetlinewidth{1.003750pt}%
\definecolor{currentstroke}{rgb}{1.000000,0.000000,0.000000}%
\pgfsetstrokecolor{currentstroke}%
\pgfsetdash{}{0pt}%
\pgfpathmoveto{\pgfqpoint{2.212526in}{1.466038in}}%
\pgfpathcurveto{\pgfqpoint{2.223576in}{1.466038in}}{\pgfqpoint{2.234175in}{1.470428in}}{\pgfqpoint{2.241989in}{1.478242in}}%
\pgfpathcurveto{\pgfqpoint{2.249803in}{1.486055in}}{\pgfqpoint{2.254193in}{1.496654in}}{\pgfqpoint{2.254193in}{1.507704in}}%
\pgfpathcurveto{\pgfqpoint{2.254193in}{1.518755in}}{\pgfqpoint{2.249803in}{1.529354in}}{\pgfqpoint{2.241989in}{1.537167in}}%
\pgfpathcurveto{\pgfqpoint{2.234175in}{1.544981in}}{\pgfqpoint{2.223576in}{1.549371in}}{\pgfqpoint{2.212526in}{1.549371in}}%
\pgfpathcurveto{\pgfqpoint{2.201476in}{1.549371in}}{\pgfqpoint{2.190877in}{1.544981in}}{\pgfqpoint{2.183063in}{1.537167in}}%
\pgfpathcurveto{\pgfqpoint{2.175250in}{1.529354in}}{\pgfqpoint{2.170860in}{1.518755in}}{\pgfqpoint{2.170860in}{1.507704in}}%
\pgfpathcurveto{\pgfqpoint{2.170860in}{1.496654in}}{\pgfqpoint{2.175250in}{1.486055in}}{\pgfqpoint{2.183063in}{1.478242in}}%
\pgfpathcurveto{\pgfqpoint{2.190877in}{1.470428in}}{\pgfqpoint{2.201476in}{1.466038in}}{\pgfqpoint{2.212526in}{1.466038in}}%
\pgfpathlineto{\pgfqpoint{2.212526in}{1.466038in}}%
\pgfpathclose%
\pgfusepath{stroke}%
\end{pgfscope}%
\begin{pgfscope}%
\pgfpathrectangle{\pgfqpoint{0.847223in}{0.554012in}}{\pgfqpoint{6.200000in}{4.530000in}}%
\pgfusepath{clip}%
\pgfsetbuttcap%
\pgfsetroundjoin%
\pgfsetlinewidth{1.003750pt}%
\definecolor{currentstroke}{rgb}{1.000000,0.000000,0.000000}%
\pgfsetstrokecolor{currentstroke}%
\pgfsetdash{}{0pt}%
\pgfpathmoveto{\pgfqpoint{2.217859in}{1.462178in}}%
\pgfpathcurveto{\pgfqpoint{2.228910in}{1.462178in}}{\pgfqpoint{2.239509in}{1.466568in}}{\pgfqpoint{2.247322in}{1.474381in}}%
\pgfpathcurveto{\pgfqpoint{2.255136in}{1.482195in}}{\pgfqpoint{2.259526in}{1.492794in}}{\pgfqpoint{2.259526in}{1.503844in}}%
\pgfpathcurveto{\pgfqpoint{2.259526in}{1.514894in}}{\pgfqpoint{2.255136in}{1.525493in}}{\pgfqpoint{2.247322in}{1.533307in}}%
\pgfpathcurveto{\pgfqpoint{2.239509in}{1.541121in}}{\pgfqpoint{2.228910in}{1.545511in}}{\pgfqpoint{2.217859in}{1.545511in}}%
\pgfpathcurveto{\pgfqpoint{2.206809in}{1.545511in}}{\pgfqpoint{2.196210in}{1.541121in}}{\pgfqpoint{2.188397in}{1.533307in}}%
\pgfpathcurveto{\pgfqpoint{2.180583in}{1.525493in}}{\pgfqpoint{2.176193in}{1.514894in}}{\pgfqpoint{2.176193in}{1.503844in}}%
\pgfpathcurveto{\pgfqpoint{2.176193in}{1.492794in}}{\pgfqpoint{2.180583in}{1.482195in}}{\pgfqpoint{2.188397in}{1.474381in}}%
\pgfpathcurveto{\pgfqpoint{2.196210in}{1.466568in}}{\pgfqpoint{2.206809in}{1.462178in}}{\pgfqpoint{2.217859in}{1.462178in}}%
\pgfpathlineto{\pgfqpoint{2.217859in}{1.462178in}}%
\pgfpathclose%
\pgfusepath{stroke}%
\end{pgfscope}%
\begin{pgfscope}%
\pgfpathrectangle{\pgfqpoint{0.847223in}{0.554012in}}{\pgfqpoint{6.200000in}{4.530000in}}%
\pgfusepath{clip}%
\pgfsetbuttcap%
\pgfsetroundjoin%
\pgfsetlinewidth{1.003750pt}%
\definecolor{currentstroke}{rgb}{1.000000,0.000000,0.000000}%
\pgfsetstrokecolor{currentstroke}%
\pgfsetdash{}{0pt}%
\pgfpathmoveto{\pgfqpoint{2.223193in}{1.458339in}}%
\pgfpathcurveto{\pgfqpoint{2.234243in}{1.458339in}}{\pgfqpoint{2.244842in}{1.462729in}}{\pgfqpoint{2.252655in}{1.470543in}}%
\pgfpathcurveto{\pgfqpoint{2.260469in}{1.478357in}}{\pgfqpoint{2.264859in}{1.488956in}}{\pgfqpoint{2.264859in}{1.500006in}}%
\pgfpathcurveto{\pgfqpoint{2.264859in}{1.511056in}}{\pgfqpoint{2.260469in}{1.521655in}}{\pgfqpoint{2.252655in}{1.529468in}}%
\pgfpathcurveto{\pgfqpoint{2.244842in}{1.537282in}}{\pgfqpoint{2.234243in}{1.541672in}}{\pgfqpoint{2.223193in}{1.541672in}}%
\pgfpathcurveto{\pgfqpoint{2.212143in}{1.541672in}}{\pgfqpoint{2.201544in}{1.537282in}}{\pgfqpoint{2.193730in}{1.529468in}}%
\pgfpathcurveto{\pgfqpoint{2.185916in}{1.521655in}}{\pgfqpoint{2.181526in}{1.511056in}}{\pgfqpoint{2.181526in}{1.500006in}}%
\pgfpathcurveto{\pgfqpoint{2.181526in}{1.488956in}}{\pgfqpoint{2.185916in}{1.478357in}}{\pgfqpoint{2.193730in}{1.470543in}}%
\pgfpathcurveto{\pgfqpoint{2.201544in}{1.462729in}}{\pgfqpoint{2.212143in}{1.458339in}}{\pgfqpoint{2.223193in}{1.458339in}}%
\pgfpathlineto{\pgfqpoint{2.223193in}{1.458339in}}%
\pgfpathclose%
\pgfusepath{stroke}%
\end{pgfscope}%
\begin{pgfscope}%
\pgfpathrectangle{\pgfqpoint{0.847223in}{0.554012in}}{\pgfqpoint{6.200000in}{4.530000in}}%
\pgfusepath{clip}%
\pgfsetbuttcap%
\pgfsetroundjoin%
\pgfsetlinewidth{1.003750pt}%
\definecolor{currentstroke}{rgb}{1.000000,0.000000,0.000000}%
\pgfsetstrokecolor{currentstroke}%
\pgfsetdash{}{0pt}%
\pgfpathmoveto{\pgfqpoint{2.228526in}{1.454522in}}%
\pgfpathcurveto{\pgfqpoint{2.239576in}{1.454522in}}{\pgfqpoint{2.250175in}{1.458912in}}{\pgfqpoint{2.257989in}{1.466726in}}%
\pgfpathcurveto{\pgfqpoint{2.265802in}{1.474539in}}{\pgfqpoint{2.270193in}{1.485139in}}{\pgfqpoint{2.270193in}{1.496189in}}%
\pgfpathcurveto{\pgfqpoint{2.270193in}{1.507239in}}{\pgfqpoint{2.265802in}{1.517838in}}{\pgfqpoint{2.257989in}{1.525651in}}%
\pgfpathcurveto{\pgfqpoint{2.250175in}{1.533465in}}{\pgfqpoint{2.239576in}{1.537855in}}{\pgfqpoint{2.228526in}{1.537855in}}%
\pgfpathcurveto{\pgfqpoint{2.217476in}{1.537855in}}{\pgfqpoint{2.206877in}{1.533465in}}{\pgfqpoint{2.199063in}{1.525651in}}%
\pgfpathcurveto{\pgfqpoint{2.191250in}{1.517838in}}{\pgfqpoint{2.186859in}{1.507239in}}{\pgfqpoint{2.186859in}{1.496189in}}%
\pgfpathcurveto{\pgfqpoint{2.186859in}{1.485139in}}{\pgfqpoint{2.191250in}{1.474539in}}{\pgfqpoint{2.199063in}{1.466726in}}%
\pgfpathcurveto{\pgfqpoint{2.206877in}{1.458912in}}{\pgfqpoint{2.217476in}{1.454522in}}{\pgfqpoint{2.228526in}{1.454522in}}%
\pgfpathlineto{\pgfqpoint{2.228526in}{1.454522in}}%
\pgfpathclose%
\pgfusepath{stroke}%
\end{pgfscope}%
\begin{pgfscope}%
\pgfpathrectangle{\pgfqpoint{0.847223in}{0.554012in}}{\pgfqpoint{6.200000in}{4.530000in}}%
\pgfusepath{clip}%
\pgfsetbuttcap%
\pgfsetroundjoin%
\pgfsetlinewidth{1.003750pt}%
\definecolor{currentstroke}{rgb}{1.000000,0.000000,0.000000}%
\pgfsetstrokecolor{currentstroke}%
\pgfsetdash{}{0pt}%
\pgfpathmoveto{\pgfqpoint{2.233859in}{1.450726in}}%
\pgfpathcurveto{\pgfqpoint{2.244909in}{1.450726in}}{\pgfqpoint{2.255508in}{1.455117in}}{\pgfqpoint{2.263322in}{1.462930in}}%
\pgfpathcurveto{\pgfqpoint{2.271136in}{1.470744in}}{\pgfqpoint{2.275526in}{1.481343in}}{\pgfqpoint{2.275526in}{1.492393in}}%
\pgfpathcurveto{\pgfqpoint{2.275526in}{1.503443in}}{\pgfqpoint{2.271136in}{1.514042in}}{\pgfqpoint{2.263322in}{1.521856in}}%
\pgfpathcurveto{\pgfqpoint{2.255508in}{1.529669in}}{\pgfqpoint{2.244909in}{1.534060in}}{\pgfqpoint{2.233859in}{1.534060in}}%
\pgfpathcurveto{\pgfqpoint{2.222809in}{1.534060in}}{\pgfqpoint{2.212210in}{1.529669in}}{\pgfqpoint{2.204396in}{1.521856in}}%
\pgfpathcurveto{\pgfqpoint{2.196583in}{1.514042in}}{\pgfqpoint{2.192192in}{1.503443in}}{\pgfqpoint{2.192192in}{1.492393in}}%
\pgfpathcurveto{\pgfqpoint{2.192192in}{1.481343in}}{\pgfqpoint{2.196583in}{1.470744in}}{\pgfqpoint{2.204396in}{1.462930in}}%
\pgfpathcurveto{\pgfqpoint{2.212210in}{1.455117in}}{\pgfqpoint{2.222809in}{1.450726in}}{\pgfqpoint{2.233859in}{1.450726in}}%
\pgfpathlineto{\pgfqpoint{2.233859in}{1.450726in}}%
\pgfpathclose%
\pgfusepath{stroke}%
\end{pgfscope}%
\begin{pgfscope}%
\pgfpathrectangle{\pgfqpoint{0.847223in}{0.554012in}}{\pgfqpoint{6.200000in}{4.530000in}}%
\pgfusepath{clip}%
\pgfsetbuttcap%
\pgfsetroundjoin%
\pgfsetlinewidth{1.003750pt}%
\definecolor{currentstroke}{rgb}{1.000000,0.000000,0.000000}%
\pgfsetstrokecolor{currentstroke}%
\pgfsetdash{}{0pt}%
\pgfpathmoveto{\pgfqpoint{2.239192in}{1.446952in}}%
\pgfpathcurveto{\pgfqpoint{2.250242in}{1.446952in}}{\pgfqpoint{2.260842in}{1.451342in}}{\pgfqpoint{2.268655in}{1.459156in}}%
\pgfpathcurveto{\pgfqpoint{2.276469in}{1.466969in}}{\pgfqpoint{2.280859in}{1.477568in}}{\pgfqpoint{2.280859in}{1.488618in}}%
\pgfpathcurveto{\pgfqpoint{2.280859in}{1.499668in}}{\pgfqpoint{2.276469in}{1.510268in}}{\pgfqpoint{2.268655in}{1.518081in}}%
\pgfpathcurveto{\pgfqpoint{2.260842in}{1.525895in}}{\pgfqpoint{2.250242in}{1.530285in}}{\pgfqpoint{2.239192in}{1.530285in}}%
\pgfpathcurveto{\pgfqpoint{2.228142in}{1.530285in}}{\pgfqpoint{2.217543in}{1.525895in}}{\pgfqpoint{2.209730in}{1.518081in}}%
\pgfpathcurveto{\pgfqpoint{2.201916in}{1.510268in}}{\pgfqpoint{2.197526in}{1.499668in}}{\pgfqpoint{2.197526in}{1.488618in}}%
\pgfpathcurveto{\pgfqpoint{2.197526in}{1.477568in}}{\pgfqpoint{2.201916in}{1.466969in}}{\pgfqpoint{2.209730in}{1.459156in}}%
\pgfpathcurveto{\pgfqpoint{2.217543in}{1.451342in}}{\pgfqpoint{2.228142in}{1.446952in}}{\pgfqpoint{2.239192in}{1.446952in}}%
\pgfpathlineto{\pgfqpoint{2.239192in}{1.446952in}}%
\pgfpathclose%
\pgfusepath{stroke}%
\end{pgfscope}%
\begin{pgfscope}%
\pgfpathrectangle{\pgfqpoint{0.847223in}{0.554012in}}{\pgfqpoint{6.200000in}{4.530000in}}%
\pgfusepath{clip}%
\pgfsetbuttcap%
\pgfsetroundjoin%
\pgfsetlinewidth{1.003750pt}%
\definecolor{currentstroke}{rgb}{1.000000,0.000000,0.000000}%
\pgfsetstrokecolor{currentstroke}%
\pgfsetdash{}{0pt}%
\pgfpathmoveto{\pgfqpoint{2.244526in}{1.443198in}}%
\pgfpathcurveto{\pgfqpoint{2.255576in}{1.443198in}}{\pgfqpoint{2.266175in}{1.447588in}}{\pgfqpoint{2.273988in}{1.455402in}}%
\pgfpathcurveto{\pgfqpoint{2.281802in}{1.463216in}}{\pgfqpoint{2.286192in}{1.473815in}}{\pgfqpoint{2.286192in}{1.484865in}}%
\pgfpathcurveto{\pgfqpoint{2.286192in}{1.495915in}}{\pgfqpoint{2.281802in}{1.506514in}}{\pgfqpoint{2.273988in}{1.514328in}}%
\pgfpathcurveto{\pgfqpoint{2.266175in}{1.522141in}}{\pgfqpoint{2.255576in}{1.526531in}}{\pgfqpoint{2.244526in}{1.526531in}}%
\pgfpathcurveto{\pgfqpoint{2.233475in}{1.526531in}}{\pgfqpoint{2.222876in}{1.522141in}}{\pgfqpoint{2.215063in}{1.514328in}}%
\pgfpathcurveto{\pgfqpoint{2.207249in}{1.506514in}}{\pgfqpoint{2.202859in}{1.495915in}}{\pgfqpoint{2.202859in}{1.484865in}}%
\pgfpathcurveto{\pgfqpoint{2.202859in}{1.473815in}}{\pgfqpoint{2.207249in}{1.463216in}}{\pgfqpoint{2.215063in}{1.455402in}}%
\pgfpathcurveto{\pgfqpoint{2.222876in}{1.447588in}}{\pgfqpoint{2.233475in}{1.443198in}}{\pgfqpoint{2.244526in}{1.443198in}}%
\pgfpathlineto{\pgfqpoint{2.244526in}{1.443198in}}%
\pgfpathclose%
\pgfusepath{stroke}%
\end{pgfscope}%
\begin{pgfscope}%
\pgfpathrectangle{\pgfqpoint{0.847223in}{0.554012in}}{\pgfqpoint{6.200000in}{4.530000in}}%
\pgfusepath{clip}%
\pgfsetbuttcap%
\pgfsetroundjoin%
\pgfsetlinewidth{1.003750pt}%
\definecolor{currentstroke}{rgb}{1.000000,0.000000,0.000000}%
\pgfsetstrokecolor{currentstroke}%
\pgfsetdash{}{0pt}%
\pgfpathmoveto{\pgfqpoint{2.249859in}{1.439465in}}%
\pgfpathcurveto{\pgfqpoint{2.260909in}{1.439465in}}{\pgfqpoint{2.271508in}{1.443856in}}{\pgfqpoint{2.279322in}{1.451669in}}%
\pgfpathcurveto{\pgfqpoint{2.287135in}{1.459483in}}{\pgfqpoint{2.291525in}{1.470082in}}{\pgfqpoint{2.291525in}{1.481132in}}%
\pgfpathcurveto{\pgfqpoint{2.291525in}{1.492182in}}{\pgfqpoint{2.287135in}{1.502781in}}{\pgfqpoint{2.279322in}{1.510595in}}%
\pgfpathcurveto{\pgfqpoint{2.271508in}{1.518408in}}{\pgfqpoint{2.260909in}{1.522799in}}{\pgfqpoint{2.249859in}{1.522799in}}%
\pgfpathcurveto{\pgfqpoint{2.238809in}{1.522799in}}{\pgfqpoint{2.228210in}{1.518408in}}{\pgfqpoint{2.220396in}{1.510595in}}%
\pgfpathcurveto{\pgfqpoint{2.212582in}{1.502781in}}{\pgfqpoint{2.208192in}{1.492182in}}{\pgfqpoint{2.208192in}{1.481132in}}%
\pgfpathcurveto{\pgfqpoint{2.208192in}{1.470082in}}{\pgfqpoint{2.212582in}{1.459483in}}{\pgfqpoint{2.220396in}{1.451669in}}%
\pgfpathcurveto{\pgfqpoint{2.228210in}{1.443856in}}{\pgfqpoint{2.238809in}{1.439465in}}{\pgfqpoint{2.249859in}{1.439465in}}%
\pgfpathlineto{\pgfqpoint{2.249859in}{1.439465in}}%
\pgfpathclose%
\pgfusepath{stroke}%
\end{pgfscope}%
\begin{pgfscope}%
\pgfpathrectangle{\pgfqpoint{0.847223in}{0.554012in}}{\pgfqpoint{6.200000in}{4.530000in}}%
\pgfusepath{clip}%
\pgfsetbuttcap%
\pgfsetroundjoin%
\pgfsetlinewidth{1.003750pt}%
\definecolor{currentstroke}{rgb}{1.000000,0.000000,0.000000}%
\pgfsetstrokecolor{currentstroke}%
\pgfsetdash{}{0pt}%
\pgfpathmoveto{\pgfqpoint{2.255192in}{1.435753in}}%
\pgfpathcurveto{\pgfqpoint{2.266242in}{1.435753in}}{\pgfqpoint{2.276841in}{1.440143in}}{\pgfqpoint{2.284655in}{1.447957in}}%
\pgfpathcurveto{\pgfqpoint{2.292468in}{1.455771in}}{\pgfqpoint{2.296859in}{1.466370in}}{\pgfqpoint{2.296859in}{1.477420in}}%
\pgfpathcurveto{\pgfqpoint{2.296859in}{1.488470in}}{\pgfqpoint{2.292468in}{1.499069in}}{\pgfqpoint{2.284655in}{1.506883in}}%
\pgfpathcurveto{\pgfqpoint{2.276841in}{1.514696in}}{\pgfqpoint{2.266242in}{1.519086in}}{\pgfqpoint{2.255192in}{1.519086in}}%
\pgfpathcurveto{\pgfqpoint{2.244142in}{1.519086in}}{\pgfqpoint{2.233543in}{1.514696in}}{\pgfqpoint{2.225729in}{1.506883in}}%
\pgfpathcurveto{\pgfqpoint{2.217916in}{1.499069in}}{\pgfqpoint{2.213525in}{1.488470in}}{\pgfqpoint{2.213525in}{1.477420in}}%
\pgfpathcurveto{\pgfqpoint{2.213525in}{1.466370in}}{\pgfqpoint{2.217916in}{1.455771in}}{\pgfqpoint{2.225729in}{1.447957in}}%
\pgfpathcurveto{\pgfqpoint{2.233543in}{1.440143in}}{\pgfqpoint{2.244142in}{1.435753in}}{\pgfqpoint{2.255192in}{1.435753in}}%
\pgfpathlineto{\pgfqpoint{2.255192in}{1.435753in}}%
\pgfpathclose%
\pgfusepath{stroke}%
\end{pgfscope}%
\begin{pgfscope}%
\pgfpathrectangle{\pgfqpoint{0.847223in}{0.554012in}}{\pgfqpoint{6.200000in}{4.530000in}}%
\pgfusepath{clip}%
\pgfsetbuttcap%
\pgfsetroundjoin%
\pgfsetlinewidth{1.003750pt}%
\definecolor{currentstroke}{rgb}{1.000000,0.000000,0.000000}%
\pgfsetstrokecolor{currentstroke}%
\pgfsetdash{}{0pt}%
\pgfpathmoveto{\pgfqpoint{2.260525in}{1.432061in}}%
\pgfpathcurveto{\pgfqpoint{2.271575in}{1.432061in}}{\pgfqpoint{2.282174in}{1.436452in}}{\pgfqpoint{2.289988in}{1.444265in}}%
\pgfpathcurveto{\pgfqpoint{2.297802in}{1.452079in}}{\pgfqpoint{2.302192in}{1.462678in}}{\pgfqpoint{2.302192in}{1.473728in}}%
\pgfpathcurveto{\pgfqpoint{2.302192in}{1.484778in}}{\pgfqpoint{2.297802in}{1.495377in}}{\pgfqpoint{2.289988in}{1.503191in}}%
\pgfpathcurveto{\pgfqpoint{2.282174in}{1.511004in}}{\pgfqpoint{2.271575in}{1.515395in}}{\pgfqpoint{2.260525in}{1.515395in}}%
\pgfpathcurveto{\pgfqpoint{2.249475in}{1.515395in}}{\pgfqpoint{2.238876in}{1.511004in}}{\pgfqpoint{2.231062in}{1.503191in}}%
\pgfpathcurveto{\pgfqpoint{2.223249in}{1.495377in}}{\pgfqpoint{2.218859in}{1.484778in}}{\pgfqpoint{2.218859in}{1.473728in}}%
\pgfpathcurveto{\pgfqpoint{2.218859in}{1.462678in}}{\pgfqpoint{2.223249in}{1.452079in}}{\pgfqpoint{2.231062in}{1.444265in}}%
\pgfpathcurveto{\pgfqpoint{2.238876in}{1.436452in}}{\pgfqpoint{2.249475in}{1.432061in}}{\pgfqpoint{2.260525in}{1.432061in}}%
\pgfpathlineto{\pgfqpoint{2.260525in}{1.432061in}}%
\pgfpathclose%
\pgfusepath{stroke}%
\end{pgfscope}%
\begin{pgfscope}%
\pgfpathrectangle{\pgfqpoint{0.847223in}{0.554012in}}{\pgfqpoint{6.200000in}{4.530000in}}%
\pgfusepath{clip}%
\pgfsetbuttcap%
\pgfsetroundjoin%
\pgfsetlinewidth{1.003750pt}%
\definecolor{currentstroke}{rgb}{1.000000,0.000000,0.000000}%
\pgfsetstrokecolor{currentstroke}%
\pgfsetdash{}{0pt}%
\pgfpathmoveto{\pgfqpoint{2.265858in}{1.428390in}}%
\pgfpathcurveto{\pgfqpoint{2.276909in}{1.428390in}}{\pgfqpoint{2.287508in}{1.432780in}}{\pgfqpoint{2.295321in}{1.440594in}}%
\pgfpathcurveto{\pgfqpoint{2.303135in}{1.448407in}}{\pgfqpoint{2.307525in}{1.459006in}}{\pgfqpoint{2.307525in}{1.470056in}}%
\pgfpathcurveto{\pgfqpoint{2.307525in}{1.481107in}}{\pgfqpoint{2.303135in}{1.491706in}}{\pgfqpoint{2.295321in}{1.499519in}}%
\pgfpathcurveto{\pgfqpoint{2.287508in}{1.507333in}}{\pgfqpoint{2.276909in}{1.511723in}}{\pgfqpoint{2.265858in}{1.511723in}}%
\pgfpathcurveto{\pgfqpoint{2.254808in}{1.511723in}}{\pgfqpoint{2.244209in}{1.507333in}}{\pgfqpoint{2.236396in}{1.499519in}}%
\pgfpathcurveto{\pgfqpoint{2.228582in}{1.491706in}}{\pgfqpoint{2.224192in}{1.481107in}}{\pgfqpoint{2.224192in}{1.470056in}}%
\pgfpathcurveto{\pgfqpoint{2.224192in}{1.459006in}}{\pgfqpoint{2.228582in}{1.448407in}}{\pgfqpoint{2.236396in}{1.440594in}}%
\pgfpathcurveto{\pgfqpoint{2.244209in}{1.432780in}}{\pgfqpoint{2.254808in}{1.428390in}}{\pgfqpoint{2.265858in}{1.428390in}}%
\pgfpathlineto{\pgfqpoint{2.265858in}{1.428390in}}%
\pgfpathclose%
\pgfusepath{stroke}%
\end{pgfscope}%
\begin{pgfscope}%
\pgfpathrectangle{\pgfqpoint{0.847223in}{0.554012in}}{\pgfqpoint{6.200000in}{4.530000in}}%
\pgfusepath{clip}%
\pgfsetbuttcap%
\pgfsetroundjoin%
\pgfsetlinewidth{1.003750pt}%
\definecolor{currentstroke}{rgb}{1.000000,0.000000,0.000000}%
\pgfsetstrokecolor{currentstroke}%
\pgfsetdash{}{0pt}%
\pgfpathmoveto{\pgfqpoint{2.271192in}{1.424738in}}%
\pgfpathcurveto{\pgfqpoint{2.282242in}{1.424738in}}{\pgfqpoint{2.292841in}{1.429129in}}{\pgfqpoint{2.300654in}{1.436942in}}%
\pgfpathcurveto{\pgfqpoint{2.308468in}{1.444756in}}{\pgfqpoint{2.312858in}{1.455355in}}{\pgfqpoint{2.312858in}{1.466405in}}%
\pgfpathcurveto{\pgfqpoint{2.312858in}{1.477455in}}{\pgfqpoint{2.308468in}{1.488054in}}{\pgfqpoint{2.300654in}{1.495868in}}%
\pgfpathcurveto{\pgfqpoint{2.292841in}{1.503682in}}{\pgfqpoint{2.282242in}{1.508072in}}{\pgfqpoint{2.271192in}{1.508072in}}%
\pgfpathcurveto{\pgfqpoint{2.260142in}{1.508072in}}{\pgfqpoint{2.249542in}{1.503682in}}{\pgfqpoint{2.241729in}{1.495868in}}%
\pgfpathcurveto{\pgfqpoint{2.233915in}{1.488054in}}{\pgfqpoint{2.229525in}{1.477455in}}{\pgfqpoint{2.229525in}{1.466405in}}%
\pgfpathcurveto{\pgfqpoint{2.229525in}{1.455355in}}{\pgfqpoint{2.233915in}{1.444756in}}{\pgfqpoint{2.241729in}{1.436942in}}%
\pgfpathcurveto{\pgfqpoint{2.249542in}{1.429129in}}{\pgfqpoint{2.260142in}{1.424738in}}{\pgfqpoint{2.271192in}{1.424738in}}%
\pgfpathlineto{\pgfqpoint{2.271192in}{1.424738in}}%
\pgfpathclose%
\pgfusepath{stroke}%
\end{pgfscope}%
\begin{pgfscope}%
\pgfpathrectangle{\pgfqpoint{0.847223in}{0.554012in}}{\pgfqpoint{6.200000in}{4.530000in}}%
\pgfusepath{clip}%
\pgfsetbuttcap%
\pgfsetroundjoin%
\pgfsetlinewidth{1.003750pt}%
\definecolor{currentstroke}{rgb}{1.000000,0.000000,0.000000}%
\pgfsetstrokecolor{currentstroke}%
\pgfsetdash{}{0pt}%
\pgfpathmoveto{\pgfqpoint{2.276525in}{1.421107in}}%
\pgfpathcurveto{\pgfqpoint{2.287575in}{1.421107in}}{\pgfqpoint{2.298174in}{1.425497in}}{\pgfqpoint{2.305988in}{1.433311in}}%
\pgfpathcurveto{\pgfqpoint{2.313801in}{1.441125in}}{\pgfqpoint{2.318192in}{1.451724in}}{\pgfqpoint{2.318192in}{1.462774in}}%
\pgfpathcurveto{\pgfqpoint{2.318192in}{1.473824in}}{\pgfqpoint{2.313801in}{1.484423in}}{\pgfqpoint{2.305988in}{1.492236in}}%
\pgfpathcurveto{\pgfqpoint{2.298174in}{1.500050in}}{\pgfqpoint{2.287575in}{1.504440in}}{\pgfqpoint{2.276525in}{1.504440in}}%
\pgfpathcurveto{\pgfqpoint{2.265475in}{1.504440in}}{\pgfqpoint{2.254876in}{1.500050in}}{\pgfqpoint{2.247062in}{1.492236in}}%
\pgfpathcurveto{\pgfqpoint{2.239248in}{1.484423in}}{\pgfqpoint{2.234858in}{1.473824in}}{\pgfqpoint{2.234858in}{1.462774in}}%
\pgfpathcurveto{\pgfqpoint{2.234858in}{1.451724in}}{\pgfqpoint{2.239248in}{1.441125in}}{\pgfqpoint{2.247062in}{1.433311in}}%
\pgfpathcurveto{\pgfqpoint{2.254876in}{1.425497in}}{\pgfqpoint{2.265475in}{1.421107in}}{\pgfqpoint{2.276525in}{1.421107in}}%
\pgfpathlineto{\pgfqpoint{2.276525in}{1.421107in}}%
\pgfpathclose%
\pgfusepath{stroke}%
\end{pgfscope}%
\begin{pgfscope}%
\pgfpathrectangle{\pgfqpoint{0.847223in}{0.554012in}}{\pgfqpoint{6.200000in}{4.530000in}}%
\pgfusepath{clip}%
\pgfsetbuttcap%
\pgfsetroundjoin%
\pgfsetlinewidth{1.003750pt}%
\definecolor{currentstroke}{rgb}{1.000000,0.000000,0.000000}%
\pgfsetstrokecolor{currentstroke}%
\pgfsetdash{}{0pt}%
\pgfpathmoveto{\pgfqpoint{2.281858in}{1.417495in}}%
\pgfpathcurveto{\pgfqpoint{2.292908in}{1.417495in}}{\pgfqpoint{2.303507in}{1.421886in}}{\pgfqpoint{2.311321in}{1.429699in}}%
\pgfpathcurveto{\pgfqpoint{2.319134in}{1.437513in}}{\pgfqpoint{2.323525in}{1.448112in}}{\pgfqpoint{2.323525in}{1.459162in}}%
\pgfpathcurveto{\pgfqpoint{2.323525in}{1.470212in}}{\pgfqpoint{2.319134in}{1.480811in}}{\pgfqpoint{2.311321in}{1.488625in}}%
\pgfpathcurveto{\pgfqpoint{2.303507in}{1.496438in}}{\pgfqpoint{2.292908in}{1.500829in}}{\pgfqpoint{2.281858in}{1.500829in}}%
\pgfpathcurveto{\pgfqpoint{2.270808in}{1.500829in}}{\pgfqpoint{2.260209in}{1.496438in}}{\pgfqpoint{2.252395in}{1.488625in}}%
\pgfpathcurveto{\pgfqpoint{2.244582in}{1.480811in}}{\pgfqpoint{2.240191in}{1.470212in}}{\pgfqpoint{2.240191in}{1.459162in}}%
\pgfpathcurveto{\pgfqpoint{2.240191in}{1.448112in}}{\pgfqpoint{2.244582in}{1.437513in}}{\pgfqpoint{2.252395in}{1.429699in}}%
\pgfpathcurveto{\pgfqpoint{2.260209in}{1.421886in}}{\pgfqpoint{2.270808in}{1.417495in}}{\pgfqpoint{2.281858in}{1.417495in}}%
\pgfpathlineto{\pgfqpoint{2.281858in}{1.417495in}}%
\pgfpathclose%
\pgfusepath{stroke}%
\end{pgfscope}%
\begin{pgfscope}%
\pgfpathrectangle{\pgfqpoint{0.847223in}{0.554012in}}{\pgfqpoint{6.200000in}{4.530000in}}%
\pgfusepath{clip}%
\pgfsetbuttcap%
\pgfsetroundjoin%
\pgfsetlinewidth{1.003750pt}%
\definecolor{currentstroke}{rgb}{1.000000,0.000000,0.000000}%
\pgfsetstrokecolor{currentstroke}%
\pgfsetdash{}{0pt}%
\pgfpathmoveto{\pgfqpoint{2.287191in}{1.413903in}}%
\pgfpathcurveto{\pgfqpoint{2.298241in}{1.413903in}}{\pgfqpoint{2.308840in}{1.418294in}}{\pgfqpoint{2.316654in}{1.426107in}}%
\pgfpathcurveto{\pgfqpoint{2.324468in}{1.433921in}}{\pgfqpoint{2.328858in}{1.444520in}}{\pgfqpoint{2.328858in}{1.455570in}}%
\pgfpathcurveto{\pgfqpoint{2.328858in}{1.466620in}}{\pgfqpoint{2.324468in}{1.477219in}}{\pgfqpoint{2.316654in}{1.485033in}}%
\pgfpathcurveto{\pgfqpoint{2.308840in}{1.492846in}}{\pgfqpoint{2.298241in}{1.497237in}}{\pgfqpoint{2.287191in}{1.497237in}}%
\pgfpathcurveto{\pgfqpoint{2.276141in}{1.497237in}}{\pgfqpoint{2.265542in}{1.492846in}}{\pgfqpoint{2.257729in}{1.485033in}}%
\pgfpathcurveto{\pgfqpoint{2.249915in}{1.477219in}}{\pgfqpoint{2.245525in}{1.466620in}}{\pgfqpoint{2.245525in}{1.455570in}}%
\pgfpathcurveto{\pgfqpoint{2.245525in}{1.444520in}}{\pgfqpoint{2.249915in}{1.433921in}}{\pgfqpoint{2.257729in}{1.426107in}}%
\pgfpathcurveto{\pgfqpoint{2.265542in}{1.418294in}}{\pgfqpoint{2.276141in}{1.413903in}}{\pgfqpoint{2.287191in}{1.413903in}}%
\pgfpathlineto{\pgfqpoint{2.287191in}{1.413903in}}%
\pgfpathclose%
\pgfusepath{stroke}%
\end{pgfscope}%
\begin{pgfscope}%
\pgfpathrectangle{\pgfqpoint{0.847223in}{0.554012in}}{\pgfqpoint{6.200000in}{4.530000in}}%
\pgfusepath{clip}%
\pgfsetbuttcap%
\pgfsetroundjoin%
\pgfsetlinewidth{1.003750pt}%
\definecolor{currentstroke}{rgb}{1.000000,0.000000,0.000000}%
\pgfsetstrokecolor{currentstroke}%
\pgfsetdash{}{0pt}%
\pgfpathmoveto{\pgfqpoint{2.292524in}{1.410331in}}%
\pgfpathcurveto{\pgfqpoint{2.303575in}{1.410331in}}{\pgfqpoint{2.314174in}{1.414721in}}{\pgfqpoint{2.321987in}{1.422535in}}%
\pgfpathcurveto{\pgfqpoint{2.329801in}{1.430348in}}{\pgfqpoint{2.334191in}{1.440947in}}{\pgfqpoint{2.334191in}{1.451997in}}%
\pgfpathcurveto{\pgfqpoint{2.334191in}{1.463048in}}{\pgfqpoint{2.329801in}{1.473647in}}{\pgfqpoint{2.321987in}{1.481460in}}%
\pgfpathcurveto{\pgfqpoint{2.314174in}{1.489274in}}{\pgfqpoint{2.303575in}{1.493664in}}{\pgfqpoint{2.292524in}{1.493664in}}%
\pgfpathcurveto{\pgfqpoint{2.281474in}{1.493664in}}{\pgfqpoint{2.270875in}{1.489274in}}{\pgfqpoint{2.263062in}{1.481460in}}%
\pgfpathcurveto{\pgfqpoint{2.255248in}{1.473647in}}{\pgfqpoint{2.250858in}{1.463048in}}{\pgfqpoint{2.250858in}{1.451997in}}%
\pgfpathcurveto{\pgfqpoint{2.250858in}{1.440947in}}{\pgfqpoint{2.255248in}{1.430348in}}{\pgfqpoint{2.263062in}{1.422535in}}%
\pgfpathcurveto{\pgfqpoint{2.270875in}{1.414721in}}{\pgfqpoint{2.281474in}{1.410331in}}{\pgfqpoint{2.292524in}{1.410331in}}%
\pgfpathlineto{\pgfqpoint{2.292524in}{1.410331in}}%
\pgfpathclose%
\pgfusepath{stroke}%
\end{pgfscope}%
\begin{pgfscope}%
\pgfpathrectangle{\pgfqpoint{0.847223in}{0.554012in}}{\pgfqpoint{6.200000in}{4.530000in}}%
\pgfusepath{clip}%
\pgfsetbuttcap%
\pgfsetroundjoin%
\pgfsetlinewidth{1.003750pt}%
\definecolor{currentstroke}{rgb}{1.000000,0.000000,0.000000}%
\pgfsetstrokecolor{currentstroke}%
\pgfsetdash{}{0pt}%
\pgfpathmoveto{\pgfqpoint{2.297858in}{1.406778in}}%
\pgfpathcurveto{\pgfqpoint{2.308908in}{1.406778in}}{\pgfqpoint{2.319507in}{1.411168in}}{\pgfqpoint{2.327320in}{1.418981in}}%
\pgfpathcurveto{\pgfqpoint{2.335134in}{1.426795in}}{\pgfqpoint{2.339524in}{1.437394in}}{\pgfqpoint{2.339524in}{1.448444in}}%
\pgfpathcurveto{\pgfqpoint{2.339524in}{1.459494in}}{\pgfqpoint{2.335134in}{1.470093in}}{\pgfqpoint{2.327320in}{1.477907in}}%
\pgfpathcurveto{\pgfqpoint{2.319507in}{1.485721in}}{\pgfqpoint{2.308908in}{1.490111in}}{\pgfqpoint{2.297858in}{1.490111in}}%
\pgfpathcurveto{\pgfqpoint{2.286808in}{1.490111in}}{\pgfqpoint{2.276209in}{1.485721in}}{\pgfqpoint{2.268395in}{1.477907in}}%
\pgfpathcurveto{\pgfqpoint{2.260581in}{1.470093in}}{\pgfqpoint{2.256191in}{1.459494in}}{\pgfqpoint{2.256191in}{1.448444in}}%
\pgfpathcurveto{\pgfqpoint{2.256191in}{1.437394in}}{\pgfqpoint{2.260581in}{1.426795in}}{\pgfqpoint{2.268395in}{1.418981in}}%
\pgfpathcurveto{\pgfqpoint{2.276209in}{1.411168in}}{\pgfqpoint{2.286808in}{1.406778in}}{\pgfqpoint{2.297858in}{1.406778in}}%
\pgfpathlineto{\pgfqpoint{2.297858in}{1.406778in}}%
\pgfpathclose%
\pgfusepath{stroke}%
\end{pgfscope}%
\begin{pgfscope}%
\pgfpathrectangle{\pgfqpoint{0.847223in}{0.554012in}}{\pgfqpoint{6.200000in}{4.530000in}}%
\pgfusepath{clip}%
\pgfsetbuttcap%
\pgfsetroundjoin%
\pgfsetlinewidth{1.003750pt}%
\definecolor{currentstroke}{rgb}{1.000000,0.000000,0.000000}%
\pgfsetstrokecolor{currentstroke}%
\pgfsetdash{}{0pt}%
\pgfpathmoveto{\pgfqpoint{2.303191in}{1.403243in}}%
\pgfpathcurveto{\pgfqpoint{2.314241in}{1.403243in}}{\pgfqpoint{2.324840in}{1.407634in}}{\pgfqpoint{2.332654in}{1.415447in}}%
\pgfpathcurveto{\pgfqpoint{2.340467in}{1.423261in}}{\pgfqpoint{2.344858in}{1.433860in}}{\pgfqpoint{2.344858in}{1.444910in}}%
\pgfpathcurveto{\pgfqpoint{2.344858in}{1.455960in}}{\pgfqpoint{2.340467in}{1.466559in}}{\pgfqpoint{2.332654in}{1.474373in}}%
\pgfpathcurveto{\pgfqpoint{2.324840in}{1.482186in}}{\pgfqpoint{2.314241in}{1.486577in}}{\pgfqpoint{2.303191in}{1.486577in}}%
\pgfpathcurveto{\pgfqpoint{2.292141in}{1.486577in}}{\pgfqpoint{2.281542in}{1.482186in}}{\pgfqpoint{2.273728in}{1.474373in}}%
\pgfpathcurveto{\pgfqpoint{2.265915in}{1.466559in}}{\pgfqpoint{2.261524in}{1.455960in}}{\pgfqpoint{2.261524in}{1.444910in}}%
\pgfpathcurveto{\pgfqpoint{2.261524in}{1.433860in}}{\pgfqpoint{2.265915in}{1.423261in}}{\pgfqpoint{2.273728in}{1.415447in}}%
\pgfpathcurveto{\pgfqpoint{2.281542in}{1.407634in}}{\pgfqpoint{2.292141in}{1.403243in}}{\pgfqpoint{2.303191in}{1.403243in}}%
\pgfpathlineto{\pgfqpoint{2.303191in}{1.403243in}}%
\pgfpathclose%
\pgfusepath{stroke}%
\end{pgfscope}%
\begin{pgfscope}%
\pgfpathrectangle{\pgfqpoint{0.847223in}{0.554012in}}{\pgfqpoint{6.200000in}{4.530000in}}%
\pgfusepath{clip}%
\pgfsetbuttcap%
\pgfsetroundjoin%
\pgfsetlinewidth{1.003750pt}%
\definecolor{currentstroke}{rgb}{1.000000,0.000000,0.000000}%
\pgfsetstrokecolor{currentstroke}%
\pgfsetdash{}{0pt}%
\pgfpathmoveto{\pgfqpoint{2.308524in}{1.399728in}}%
\pgfpathcurveto{\pgfqpoint{2.319574in}{1.399728in}}{\pgfqpoint{2.330173in}{1.404119in}}{\pgfqpoint{2.337987in}{1.411932in}}%
\pgfpathcurveto{\pgfqpoint{2.345801in}{1.419746in}}{\pgfqpoint{2.350191in}{1.430345in}}{\pgfqpoint{2.350191in}{1.441395in}}%
\pgfpathcurveto{\pgfqpoint{2.350191in}{1.452445in}}{\pgfqpoint{2.345801in}{1.463044in}}{\pgfqpoint{2.337987in}{1.470858in}}%
\pgfpathcurveto{\pgfqpoint{2.330173in}{1.478671in}}{\pgfqpoint{2.319574in}{1.483062in}}{\pgfqpoint{2.308524in}{1.483062in}}%
\pgfpathcurveto{\pgfqpoint{2.297474in}{1.483062in}}{\pgfqpoint{2.286875in}{1.478671in}}{\pgfqpoint{2.279061in}{1.470858in}}%
\pgfpathcurveto{\pgfqpoint{2.271248in}{1.463044in}}{\pgfqpoint{2.266857in}{1.452445in}}{\pgfqpoint{2.266857in}{1.441395in}}%
\pgfpathcurveto{\pgfqpoint{2.266857in}{1.430345in}}{\pgfqpoint{2.271248in}{1.419746in}}{\pgfqpoint{2.279061in}{1.411932in}}%
\pgfpathcurveto{\pgfqpoint{2.286875in}{1.404119in}}{\pgfqpoint{2.297474in}{1.399728in}}{\pgfqpoint{2.308524in}{1.399728in}}%
\pgfpathlineto{\pgfqpoint{2.308524in}{1.399728in}}%
\pgfpathclose%
\pgfusepath{stroke}%
\end{pgfscope}%
\begin{pgfscope}%
\pgfpathrectangle{\pgfqpoint{0.847223in}{0.554012in}}{\pgfqpoint{6.200000in}{4.530000in}}%
\pgfusepath{clip}%
\pgfsetbuttcap%
\pgfsetroundjoin%
\pgfsetlinewidth{1.003750pt}%
\definecolor{currentstroke}{rgb}{1.000000,0.000000,0.000000}%
\pgfsetstrokecolor{currentstroke}%
\pgfsetdash{}{0pt}%
\pgfpathmoveto{\pgfqpoint{2.313857in}{1.396232in}}%
\pgfpathcurveto{\pgfqpoint{2.324907in}{1.396232in}}{\pgfqpoint{2.335507in}{1.400622in}}{\pgfqpoint{2.343320in}{1.408436in}}%
\pgfpathcurveto{\pgfqpoint{2.351134in}{1.416250in}}{\pgfqpoint{2.355524in}{1.426849in}}{\pgfqpoint{2.355524in}{1.437899in}}%
\pgfpathcurveto{\pgfqpoint{2.355524in}{1.448949in}}{\pgfqpoint{2.351134in}{1.459548in}}{\pgfqpoint{2.343320in}{1.467361in}}%
\pgfpathcurveto{\pgfqpoint{2.335507in}{1.475175in}}{\pgfqpoint{2.324907in}{1.479565in}}{\pgfqpoint{2.313857in}{1.479565in}}%
\pgfpathcurveto{\pgfqpoint{2.302807in}{1.479565in}}{\pgfqpoint{2.292208in}{1.475175in}}{\pgfqpoint{2.284395in}{1.467361in}}%
\pgfpathcurveto{\pgfqpoint{2.276581in}{1.459548in}}{\pgfqpoint{2.272191in}{1.448949in}}{\pgfqpoint{2.272191in}{1.437899in}}%
\pgfpathcurveto{\pgfqpoint{2.272191in}{1.426849in}}{\pgfqpoint{2.276581in}{1.416250in}}{\pgfqpoint{2.284395in}{1.408436in}}%
\pgfpathcurveto{\pgfqpoint{2.292208in}{1.400622in}}{\pgfqpoint{2.302807in}{1.396232in}}{\pgfqpoint{2.313857in}{1.396232in}}%
\pgfpathlineto{\pgfqpoint{2.313857in}{1.396232in}}%
\pgfpathclose%
\pgfusepath{stroke}%
\end{pgfscope}%
\begin{pgfscope}%
\pgfpathrectangle{\pgfqpoint{0.847223in}{0.554012in}}{\pgfqpoint{6.200000in}{4.530000in}}%
\pgfusepath{clip}%
\pgfsetbuttcap%
\pgfsetroundjoin%
\pgfsetlinewidth{1.003750pt}%
\definecolor{currentstroke}{rgb}{1.000000,0.000000,0.000000}%
\pgfsetstrokecolor{currentstroke}%
\pgfsetdash{}{0pt}%
\pgfpathmoveto{\pgfqpoint{2.319191in}{1.392754in}}%
\pgfpathcurveto{\pgfqpoint{2.330241in}{1.392754in}}{\pgfqpoint{2.340840in}{1.397145in}}{\pgfqpoint{2.348653in}{1.404958in}}%
\pgfpathcurveto{\pgfqpoint{2.356467in}{1.412772in}}{\pgfqpoint{2.360857in}{1.423371in}}{\pgfqpoint{2.360857in}{1.434421in}}%
\pgfpathcurveto{\pgfqpoint{2.360857in}{1.445471in}}{\pgfqpoint{2.356467in}{1.456070in}}{\pgfqpoint{2.348653in}{1.463884in}}%
\pgfpathcurveto{\pgfqpoint{2.340840in}{1.471697in}}{\pgfqpoint{2.330241in}{1.476088in}}{\pgfqpoint{2.319191in}{1.476088in}}%
\pgfpathcurveto{\pgfqpoint{2.308140in}{1.476088in}}{\pgfqpoint{2.297541in}{1.471697in}}{\pgfqpoint{2.289728in}{1.463884in}}%
\pgfpathcurveto{\pgfqpoint{2.281914in}{1.456070in}}{\pgfqpoint{2.277524in}{1.445471in}}{\pgfqpoint{2.277524in}{1.434421in}}%
\pgfpathcurveto{\pgfqpoint{2.277524in}{1.423371in}}{\pgfqpoint{2.281914in}{1.412772in}}{\pgfqpoint{2.289728in}{1.404958in}}%
\pgfpathcurveto{\pgfqpoint{2.297541in}{1.397145in}}{\pgfqpoint{2.308140in}{1.392754in}}{\pgfqpoint{2.319191in}{1.392754in}}%
\pgfpathlineto{\pgfqpoint{2.319191in}{1.392754in}}%
\pgfpathclose%
\pgfusepath{stroke}%
\end{pgfscope}%
\begin{pgfscope}%
\pgfpathrectangle{\pgfqpoint{0.847223in}{0.554012in}}{\pgfqpoint{6.200000in}{4.530000in}}%
\pgfusepath{clip}%
\pgfsetbuttcap%
\pgfsetroundjoin%
\pgfsetlinewidth{1.003750pt}%
\definecolor{currentstroke}{rgb}{1.000000,0.000000,0.000000}%
\pgfsetstrokecolor{currentstroke}%
\pgfsetdash{}{0pt}%
\pgfpathmoveto{\pgfqpoint{2.324524in}{1.389295in}}%
\pgfpathcurveto{\pgfqpoint{2.335574in}{1.389295in}}{\pgfqpoint{2.346173in}{1.393686in}}{\pgfqpoint{2.353987in}{1.401499in}}%
\pgfpathcurveto{\pgfqpoint{2.361800in}{1.409313in}}{\pgfqpoint{2.366190in}{1.419912in}}{\pgfqpoint{2.366190in}{1.430962in}}%
\pgfpathcurveto{\pgfqpoint{2.366190in}{1.442012in}}{\pgfqpoint{2.361800in}{1.452611in}}{\pgfqpoint{2.353987in}{1.460425in}}%
\pgfpathcurveto{\pgfqpoint{2.346173in}{1.468238in}}{\pgfqpoint{2.335574in}{1.472629in}}{\pgfqpoint{2.324524in}{1.472629in}}%
\pgfpathcurveto{\pgfqpoint{2.313474in}{1.472629in}}{\pgfqpoint{2.302875in}{1.468238in}}{\pgfqpoint{2.295061in}{1.460425in}}%
\pgfpathcurveto{\pgfqpoint{2.287247in}{1.452611in}}{\pgfqpoint{2.282857in}{1.442012in}}{\pgfqpoint{2.282857in}{1.430962in}}%
\pgfpathcurveto{\pgfqpoint{2.282857in}{1.419912in}}{\pgfqpoint{2.287247in}{1.409313in}}{\pgfqpoint{2.295061in}{1.401499in}}%
\pgfpathcurveto{\pgfqpoint{2.302875in}{1.393686in}}{\pgfqpoint{2.313474in}{1.389295in}}{\pgfqpoint{2.324524in}{1.389295in}}%
\pgfpathlineto{\pgfqpoint{2.324524in}{1.389295in}}%
\pgfpathclose%
\pgfusepath{stroke}%
\end{pgfscope}%
\begin{pgfscope}%
\pgfpathrectangle{\pgfqpoint{0.847223in}{0.554012in}}{\pgfqpoint{6.200000in}{4.530000in}}%
\pgfusepath{clip}%
\pgfsetbuttcap%
\pgfsetroundjoin%
\pgfsetlinewidth{1.003750pt}%
\definecolor{currentstroke}{rgb}{1.000000,0.000000,0.000000}%
\pgfsetstrokecolor{currentstroke}%
\pgfsetdash{}{0pt}%
\pgfpathmoveto{\pgfqpoint{2.329857in}{1.385855in}}%
\pgfpathcurveto{\pgfqpoint{2.340907in}{1.385855in}}{\pgfqpoint{2.351506in}{1.390245in}}{\pgfqpoint{2.359320in}{1.398059in}}%
\pgfpathcurveto{\pgfqpoint{2.367133in}{1.405872in}}{\pgfqpoint{2.371524in}{1.416471in}}{\pgfqpoint{2.371524in}{1.427521in}}%
\pgfpathcurveto{\pgfqpoint{2.371524in}{1.438572in}}{\pgfqpoint{2.367133in}{1.449171in}}{\pgfqpoint{2.359320in}{1.456984in}}%
\pgfpathcurveto{\pgfqpoint{2.351506in}{1.464798in}}{\pgfqpoint{2.340907in}{1.469188in}}{\pgfqpoint{2.329857in}{1.469188in}}%
\pgfpathcurveto{\pgfqpoint{2.318807in}{1.469188in}}{\pgfqpoint{2.308208in}{1.464798in}}{\pgfqpoint{2.300394in}{1.456984in}}%
\pgfpathcurveto{\pgfqpoint{2.292581in}{1.449171in}}{\pgfqpoint{2.288190in}{1.438572in}}{\pgfqpoint{2.288190in}{1.427521in}}%
\pgfpathcurveto{\pgfqpoint{2.288190in}{1.416471in}}{\pgfqpoint{2.292581in}{1.405872in}}{\pgfqpoint{2.300394in}{1.398059in}}%
\pgfpathcurveto{\pgfqpoint{2.308208in}{1.390245in}}{\pgfqpoint{2.318807in}{1.385855in}}{\pgfqpoint{2.329857in}{1.385855in}}%
\pgfpathlineto{\pgfqpoint{2.329857in}{1.385855in}}%
\pgfpathclose%
\pgfusepath{stroke}%
\end{pgfscope}%
\begin{pgfscope}%
\pgfpathrectangle{\pgfqpoint{0.847223in}{0.554012in}}{\pgfqpoint{6.200000in}{4.530000in}}%
\pgfusepath{clip}%
\pgfsetbuttcap%
\pgfsetroundjoin%
\pgfsetlinewidth{1.003750pt}%
\definecolor{currentstroke}{rgb}{1.000000,0.000000,0.000000}%
\pgfsetstrokecolor{currentstroke}%
\pgfsetdash{}{0pt}%
\pgfpathmoveto{\pgfqpoint{2.335190in}{1.382432in}}%
\pgfpathcurveto{\pgfqpoint{2.346240in}{1.382432in}}{\pgfqpoint{2.356839in}{1.386823in}}{\pgfqpoint{2.364653in}{1.394636in}}%
\pgfpathcurveto{\pgfqpoint{2.372467in}{1.402450in}}{\pgfqpoint{2.376857in}{1.413049in}}{\pgfqpoint{2.376857in}{1.424099in}}%
\pgfpathcurveto{\pgfqpoint{2.376857in}{1.435149in}}{\pgfqpoint{2.372467in}{1.445748in}}{\pgfqpoint{2.364653in}{1.453562in}}%
\pgfpathcurveto{\pgfqpoint{2.356839in}{1.461375in}}{\pgfqpoint{2.346240in}{1.465766in}}{\pgfqpoint{2.335190in}{1.465766in}}%
\pgfpathcurveto{\pgfqpoint{2.324140in}{1.465766in}}{\pgfqpoint{2.313541in}{1.461375in}}{\pgfqpoint{2.305727in}{1.453562in}}%
\pgfpathcurveto{\pgfqpoint{2.297914in}{1.445748in}}{\pgfqpoint{2.293524in}{1.435149in}}{\pgfqpoint{2.293524in}{1.424099in}}%
\pgfpathcurveto{\pgfqpoint{2.293524in}{1.413049in}}{\pgfqpoint{2.297914in}{1.402450in}}{\pgfqpoint{2.305727in}{1.394636in}}%
\pgfpathcurveto{\pgfqpoint{2.313541in}{1.386823in}}{\pgfqpoint{2.324140in}{1.382432in}}{\pgfqpoint{2.335190in}{1.382432in}}%
\pgfpathlineto{\pgfqpoint{2.335190in}{1.382432in}}%
\pgfpathclose%
\pgfusepath{stroke}%
\end{pgfscope}%
\begin{pgfscope}%
\pgfpathrectangle{\pgfqpoint{0.847223in}{0.554012in}}{\pgfqpoint{6.200000in}{4.530000in}}%
\pgfusepath{clip}%
\pgfsetbuttcap%
\pgfsetroundjoin%
\pgfsetlinewidth{1.003750pt}%
\definecolor{currentstroke}{rgb}{1.000000,0.000000,0.000000}%
\pgfsetstrokecolor{currentstroke}%
\pgfsetdash{}{0pt}%
\pgfpathmoveto{\pgfqpoint{2.340523in}{1.379028in}}%
\pgfpathcurveto{\pgfqpoint{2.351574in}{1.379028in}}{\pgfqpoint{2.362173in}{1.383418in}}{\pgfqpoint{2.369986in}{1.391232in}}%
\pgfpathcurveto{\pgfqpoint{2.377800in}{1.399046in}}{\pgfqpoint{2.382190in}{1.409645in}}{\pgfqpoint{2.382190in}{1.420695in}}%
\pgfpathcurveto{\pgfqpoint{2.382190in}{1.431745in}}{\pgfqpoint{2.377800in}{1.442344in}}{\pgfqpoint{2.369986in}{1.450157in}}%
\pgfpathcurveto{\pgfqpoint{2.362173in}{1.457971in}}{\pgfqpoint{2.351574in}{1.462361in}}{\pgfqpoint{2.340523in}{1.462361in}}%
\pgfpathcurveto{\pgfqpoint{2.329473in}{1.462361in}}{\pgfqpoint{2.318874in}{1.457971in}}{\pgfqpoint{2.311061in}{1.450157in}}%
\pgfpathcurveto{\pgfqpoint{2.303247in}{1.442344in}}{\pgfqpoint{2.298857in}{1.431745in}}{\pgfqpoint{2.298857in}{1.420695in}}%
\pgfpathcurveto{\pgfqpoint{2.298857in}{1.409645in}}{\pgfqpoint{2.303247in}{1.399046in}}{\pgfqpoint{2.311061in}{1.391232in}}%
\pgfpathcurveto{\pgfqpoint{2.318874in}{1.383418in}}{\pgfqpoint{2.329473in}{1.379028in}}{\pgfqpoint{2.340523in}{1.379028in}}%
\pgfpathlineto{\pgfqpoint{2.340523in}{1.379028in}}%
\pgfpathclose%
\pgfusepath{stroke}%
\end{pgfscope}%
\begin{pgfscope}%
\pgfpathrectangle{\pgfqpoint{0.847223in}{0.554012in}}{\pgfqpoint{6.200000in}{4.530000in}}%
\pgfusepath{clip}%
\pgfsetbuttcap%
\pgfsetroundjoin%
\pgfsetlinewidth{1.003750pt}%
\definecolor{currentstroke}{rgb}{1.000000,0.000000,0.000000}%
\pgfsetstrokecolor{currentstroke}%
\pgfsetdash{}{0pt}%
\pgfpathmoveto{\pgfqpoint{2.345857in}{1.375642in}}%
\pgfpathcurveto{\pgfqpoint{2.356907in}{1.375642in}}{\pgfqpoint{2.367506in}{1.380032in}}{\pgfqpoint{2.375319in}{1.387846in}}%
\pgfpathcurveto{\pgfqpoint{2.383133in}{1.395659in}}{\pgfqpoint{2.387523in}{1.406258in}}{\pgfqpoint{2.387523in}{1.417308in}}%
\pgfpathcurveto{\pgfqpoint{2.387523in}{1.428358in}}{\pgfqpoint{2.383133in}{1.438958in}}{\pgfqpoint{2.375319in}{1.446771in}}%
\pgfpathcurveto{\pgfqpoint{2.367506in}{1.454585in}}{\pgfqpoint{2.356907in}{1.458975in}}{\pgfqpoint{2.345857in}{1.458975in}}%
\pgfpathcurveto{\pgfqpoint{2.334807in}{1.458975in}}{\pgfqpoint{2.324207in}{1.454585in}}{\pgfqpoint{2.316394in}{1.446771in}}%
\pgfpathcurveto{\pgfqpoint{2.308580in}{1.438958in}}{\pgfqpoint{2.304190in}{1.428358in}}{\pgfqpoint{2.304190in}{1.417308in}}%
\pgfpathcurveto{\pgfqpoint{2.304190in}{1.406258in}}{\pgfqpoint{2.308580in}{1.395659in}}{\pgfqpoint{2.316394in}{1.387846in}}%
\pgfpathcurveto{\pgfqpoint{2.324207in}{1.380032in}}{\pgfqpoint{2.334807in}{1.375642in}}{\pgfqpoint{2.345857in}{1.375642in}}%
\pgfpathlineto{\pgfqpoint{2.345857in}{1.375642in}}%
\pgfpathclose%
\pgfusepath{stroke}%
\end{pgfscope}%
\begin{pgfscope}%
\pgfpathrectangle{\pgfqpoint{0.847223in}{0.554012in}}{\pgfqpoint{6.200000in}{4.530000in}}%
\pgfusepath{clip}%
\pgfsetbuttcap%
\pgfsetroundjoin%
\pgfsetlinewidth{1.003750pt}%
\definecolor{currentstroke}{rgb}{1.000000,0.000000,0.000000}%
\pgfsetstrokecolor{currentstroke}%
\pgfsetdash{}{0pt}%
\pgfpathmoveto{\pgfqpoint{2.351190in}{1.372273in}}%
\pgfpathcurveto{\pgfqpoint{2.362240in}{1.372273in}}{\pgfqpoint{2.372839in}{1.376663in}}{\pgfqpoint{2.380653in}{1.384477in}}%
\pgfpathcurveto{\pgfqpoint{2.388466in}{1.392291in}}{\pgfqpoint{2.392857in}{1.402890in}}{\pgfqpoint{2.392857in}{1.413940in}}%
\pgfpathcurveto{\pgfqpoint{2.392857in}{1.424990in}}{\pgfqpoint{2.388466in}{1.435589in}}{\pgfqpoint{2.380653in}{1.443403in}}%
\pgfpathcurveto{\pgfqpoint{2.372839in}{1.451216in}}{\pgfqpoint{2.362240in}{1.455606in}}{\pgfqpoint{2.351190in}{1.455606in}}%
\pgfpathcurveto{\pgfqpoint{2.340140in}{1.455606in}}{\pgfqpoint{2.329541in}{1.451216in}}{\pgfqpoint{2.321727in}{1.443403in}}%
\pgfpathcurveto{\pgfqpoint{2.313913in}{1.435589in}}{\pgfqpoint{2.309523in}{1.424990in}}{\pgfqpoint{2.309523in}{1.413940in}}%
\pgfpathcurveto{\pgfqpoint{2.309523in}{1.402890in}}{\pgfqpoint{2.313913in}{1.392291in}}{\pgfqpoint{2.321727in}{1.384477in}}%
\pgfpathcurveto{\pgfqpoint{2.329541in}{1.376663in}}{\pgfqpoint{2.340140in}{1.372273in}}{\pgfqpoint{2.351190in}{1.372273in}}%
\pgfpathlineto{\pgfqpoint{2.351190in}{1.372273in}}%
\pgfpathclose%
\pgfusepath{stroke}%
\end{pgfscope}%
\begin{pgfscope}%
\pgfpathrectangle{\pgfqpoint{0.847223in}{0.554012in}}{\pgfqpoint{6.200000in}{4.530000in}}%
\pgfusepath{clip}%
\pgfsetbuttcap%
\pgfsetroundjoin%
\pgfsetlinewidth{1.003750pt}%
\definecolor{currentstroke}{rgb}{1.000000,0.000000,0.000000}%
\pgfsetstrokecolor{currentstroke}%
\pgfsetdash{}{0pt}%
\pgfpathmoveto{\pgfqpoint{2.356523in}{1.368922in}}%
\pgfpathcurveto{\pgfqpoint{2.367573in}{1.368922in}}{\pgfqpoint{2.378172in}{1.373313in}}{\pgfqpoint{2.385986in}{1.381126in}}%
\pgfpathcurveto{\pgfqpoint{2.393799in}{1.388940in}}{\pgfqpoint{2.398190in}{1.399539in}}{\pgfqpoint{2.398190in}{1.410589in}}%
\pgfpathcurveto{\pgfqpoint{2.398190in}{1.421639in}}{\pgfqpoint{2.393799in}{1.432238in}}{\pgfqpoint{2.385986in}{1.440052in}}%
\pgfpathcurveto{\pgfqpoint{2.378172in}{1.447865in}}{\pgfqpoint{2.367573in}{1.452256in}}{\pgfqpoint{2.356523in}{1.452256in}}%
\pgfpathcurveto{\pgfqpoint{2.345473in}{1.452256in}}{\pgfqpoint{2.334874in}{1.447865in}}{\pgfqpoint{2.327060in}{1.440052in}}%
\pgfpathcurveto{\pgfqpoint{2.319247in}{1.432238in}}{\pgfqpoint{2.314856in}{1.421639in}}{\pgfqpoint{2.314856in}{1.410589in}}%
\pgfpathcurveto{\pgfqpoint{2.314856in}{1.399539in}}{\pgfqpoint{2.319247in}{1.388940in}}{\pgfqpoint{2.327060in}{1.381126in}}%
\pgfpathcurveto{\pgfqpoint{2.334874in}{1.373313in}}{\pgfqpoint{2.345473in}{1.368922in}}{\pgfqpoint{2.356523in}{1.368922in}}%
\pgfpathlineto{\pgfqpoint{2.356523in}{1.368922in}}%
\pgfpathclose%
\pgfusepath{stroke}%
\end{pgfscope}%
\begin{pgfscope}%
\pgfpathrectangle{\pgfqpoint{0.847223in}{0.554012in}}{\pgfqpoint{6.200000in}{4.530000in}}%
\pgfusepath{clip}%
\pgfsetbuttcap%
\pgfsetroundjoin%
\pgfsetlinewidth{1.003750pt}%
\definecolor{currentstroke}{rgb}{1.000000,0.000000,0.000000}%
\pgfsetstrokecolor{currentstroke}%
\pgfsetdash{}{0pt}%
\pgfpathmoveto{\pgfqpoint{2.361856in}{1.365589in}}%
\pgfpathcurveto{\pgfqpoint{2.372906in}{1.365589in}}{\pgfqpoint{2.383505in}{1.369979in}}{\pgfqpoint{2.391319in}{1.377793in}}%
\pgfpathcurveto{\pgfqpoint{2.399133in}{1.385606in}}{\pgfqpoint{2.403523in}{1.396206in}}{\pgfqpoint{2.403523in}{1.407256in}}%
\pgfpathcurveto{\pgfqpoint{2.403523in}{1.418306in}}{\pgfqpoint{2.399133in}{1.428905in}}{\pgfqpoint{2.391319in}{1.436718in}}%
\pgfpathcurveto{\pgfqpoint{2.383505in}{1.444532in}}{\pgfqpoint{2.372906in}{1.448922in}}{\pgfqpoint{2.361856in}{1.448922in}}%
\pgfpathcurveto{\pgfqpoint{2.350806in}{1.448922in}}{\pgfqpoint{2.340207in}{1.444532in}}{\pgfqpoint{2.332394in}{1.436718in}}%
\pgfpathcurveto{\pgfqpoint{2.324580in}{1.428905in}}{\pgfqpoint{2.320190in}{1.418306in}}{\pgfqpoint{2.320190in}{1.407256in}}%
\pgfpathcurveto{\pgfqpoint{2.320190in}{1.396206in}}{\pgfqpoint{2.324580in}{1.385606in}}{\pgfqpoint{2.332394in}{1.377793in}}%
\pgfpathcurveto{\pgfqpoint{2.340207in}{1.369979in}}{\pgfqpoint{2.350806in}{1.365589in}}{\pgfqpoint{2.361856in}{1.365589in}}%
\pgfpathlineto{\pgfqpoint{2.361856in}{1.365589in}}%
\pgfpathclose%
\pgfusepath{stroke}%
\end{pgfscope}%
\begin{pgfscope}%
\pgfpathrectangle{\pgfqpoint{0.847223in}{0.554012in}}{\pgfqpoint{6.200000in}{4.530000in}}%
\pgfusepath{clip}%
\pgfsetbuttcap%
\pgfsetroundjoin%
\pgfsetlinewidth{1.003750pt}%
\definecolor{currentstroke}{rgb}{1.000000,0.000000,0.000000}%
\pgfsetstrokecolor{currentstroke}%
\pgfsetdash{}{0pt}%
\pgfpathmoveto{\pgfqpoint{2.367190in}{1.362273in}}%
\pgfpathcurveto{\pgfqpoint{2.378240in}{1.362273in}}{\pgfqpoint{2.388839in}{1.366663in}}{\pgfqpoint{2.396652in}{1.374477in}}%
\pgfpathcurveto{\pgfqpoint{2.404466in}{1.382291in}}{\pgfqpoint{2.408856in}{1.392890in}}{\pgfqpoint{2.408856in}{1.403940in}}%
\pgfpathcurveto{\pgfqpoint{2.408856in}{1.414990in}}{\pgfqpoint{2.404466in}{1.425589in}}{\pgfqpoint{2.396652in}{1.433403in}}%
\pgfpathcurveto{\pgfqpoint{2.388839in}{1.441216in}}{\pgfqpoint{2.378240in}{1.445606in}}{\pgfqpoint{2.367190in}{1.445606in}}%
\pgfpathcurveto{\pgfqpoint{2.356139in}{1.445606in}}{\pgfqpoint{2.345540in}{1.441216in}}{\pgfqpoint{2.337727in}{1.433403in}}%
\pgfpathcurveto{\pgfqpoint{2.329913in}{1.425589in}}{\pgfqpoint{2.325523in}{1.414990in}}{\pgfqpoint{2.325523in}{1.403940in}}%
\pgfpathcurveto{\pgfqpoint{2.325523in}{1.392890in}}{\pgfqpoint{2.329913in}{1.382291in}}{\pgfqpoint{2.337727in}{1.374477in}}%
\pgfpathcurveto{\pgfqpoint{2.345540in}{1.366663in}}{\pgfqpoint{2.356139in}{1.362273in}}{\pgfqpoint{2.367190in}{1.362273in}}%
\pgfpathlineto{\pgfqpoint{2.367190in}{1.362273in}}%
\pgfpathclose%
\pgfusepath{stroke}%
\end{pgfscope}%
\begin{pgfscope}%
\pgfpathrectangle{\pgfqpoint{0.847223in}{0.554012in}}{\pgfqpoint{6.200000in}{4.530000in}}%
\pgfusepath{clip}%
\pgfsetbuttcap%
\pgfsetroundjoin%
\pgfsetlinewidth{1.003750pt}%
\definecolor{currentstroke}{rgb}{1.000000,0.000000,0.000000}%
\pgfsetstrokecolor{currentstroke}%
\pgfsetdash{}{0pt}%
\pgfpathmoveto{\pgfqpoint{2.372523in}{1.358974in}}%
\pgfpathcurveto{\pgfqpoint{2.383573in}{1.358974in}}{\pgfqpoint{2.394172in}{1.363365in}}{\pgfqpoint{2.401986in}{1.371178in}}%
\pgfpathcurveto{\pgfqpoint{2.409799in}{1.378992in}}{\pgfqpoint{2.414189in}{1.389591in}}{\pgfqpoint{2.414189in}{1.400641in}}%
\pgfpathcurveto{\pgfqpoint{2.414189in}{1.411691in}}{\pgfqpoint{2.409799in}{1.422290in}}{\pgfqpoint{2.401986in}{1.430104in}}%
\pgfpathcurveto{\pgfqpoint{2.394172in}{1.437917in}}{\pgfqpoint{2.383573in}{1.442308in}}{\pgfqpoint{2.372523in}{1.442308in}}%
\pgfpathcurveto{\pgfqpoint{2.361473in}{1.442308in}}{\pgfqpoint{2.350874in}{1.437917in}}{\pgfqpoint{2.343060in}{1.430104in}}%
\pgfpathcurveto{\pgfqpoint{2.335246in}{1.422290in}}{\pgfqpoint{2.330856in}{1.411691in}}{\pgfqpoint{2.330856in}{1.400641in}}%
\pgfpathcurveto{\pgfqpoint{2.330856in}{1.389591in}}{\pgfqpoint{2.335246in}{1.378992in}}{\pgfqpoint{2.343060in}{1.371178in}}%
\pgfpathcurveto{\pgfqpoint{2.350874in}{1.363365in}}{\pgfqpoint{2.361473in}{1.358974in}}{\pgfqpoint{2.372523in}{1.358974in}}%
\pgfpathlineto{\pgfqpoint{2.372523in}{1.358974in}}%
\pgfpathclose%
\pgfusepath{stroke}%
\end{pgfscope}%
\begin{pgfscope}%
\pgfpathrectangle{\pgfqpoint{0.847223in}{0.554012in}}{\pgfqpoint{6.200000in}{4.530000in}}%
\pgfusepath{clip}%
\pgfsetbuttcap%
\pgfsetroundjoin%
\pgfsetlinewidth{1.003750pt}%
\definecolor{currentstroke}{rgb}{1.000000,0.000000,0.000000}%
\pgfsetstrokecolor{currentstroke}%
\pgfsetdash{}{0pt}%
\pgfpathmoveto{\pgfqpoint{2.377856in}{1.355693in}}%
\pgfpathcurveto{\pgfqpoint{2.388906in}{1.355693in}}{\pgfqpoint{2.399505in}{1.360083in}}{\pgfqpoint{2.407319in}{1.367897in}}%
\pgfpathcurveto{\pgfqpoint{2.415132in}{1.375710in}}{\pgfqpoint{2.419523in}{1.386309in}}{\pgfqpoint{2.419523in}{1.397360in}}%
\pgfpathcurveto{\pgfqpoint{2.419523in}{1.408410in}}{\pgfqpoint{2.415132in}{1.419009in}}{\pgfqpoint{2.407319in}{1.426822in}}%
\pgfpathcurveto{\pgfqpoint{2.399505in}{1.434636in}}{\pgfqpoint{2.388906in}{1.439026in}}{\pgfqpoint{2.377856in}{1.439026in}}%
\pgfpathcurveto{\pgfqpoint{2.366806in}{1.439026in}}{\pgfqpoint{2.356207in}{1.434636in}}{\pgfqpoint{2.348393in}{1.426822in}}%
\pgfpathcurveto{\pgfqpoint{2.340580in}{1.419009in}}{\pgfqpoint{2.336189in}{1.408410in}}{\pgfqpoint{2.336189in}{1.397360in}}%
\pgfpathcurveto{\pgfqpoint{2.336189in}{1.386309in}}{\pgfqpoint{2.340580in}{1.375710in}}{\pgfqpoint{2.348393in}{1.367897in}}%
\pgfpathcurveto{\pgfqpoint{2.356207in}{1.360083in}}{\pgfqpoint{2.366806in}{1.355693in}}{\pgfqpoint{2.377856in}{1.355693in}}%
\pgfpathlineto{\pgfqpoint{2.377856in}{1.355693in}}%
\pgfpathclose%
\pgfusepath{stroke}%
\end{pgfscope}%
\begin{pgfscope}%
\pgfpathrectangle{\pgfqpoint{0.847223in}{0.554012in}}{\pgfqpoint{6.200000in}{4.530000in}}%
\pgfusepath{clip}%
\pgfsetbuttcap%
\pgfsetroundjoin%
\pgfsetlinewidth{1.003750pt}%
\definecolor{currentstroke}{rgb}{1.000000,0.000000,0.000000}%
\pgfsetstrokecolor{currentstroke}%
\pgfsetdash{}{0pt}%
\pgfpathmoveto{\pgfqpoint{2.383189in}{1.352428in}}%
\pgfpathcurveto{\pgfqpoint{2.394239in}{1.352428in}}{\pgfqpoint{2.404838in}{1.356819in}}{\pgfqpoint{2.412652in}{1.364632in}}%
\pgfpathcurveto{\pgfqpoint{2.420466in}{1.372446in}}{\pgfqpoint{2.424856in}{1.383045in}}{\pgfqpoint{2.424856in}{1.394095in}}%
\pgfpathcurveto{\pgfqpoint{2.424856in}{1.405145in}}{\pgfqpoint{2.420466in}{1.415744in}}{\pgfqpoint{2.412652in}{1.423558in}}%
\pgfpathcurveto{\pgfqpoint{2.404838in}{1.431371in}}{\pgfqpoint{2.394239in}{1.435762in}}{\pgfqpoint{2.383189in}{1.435762in}}%
\pgfpathcurveto{\pgfqpoint{2.372139in}{1.435762in}}{\pgfqpoint{2.361540in}{1.431371in}}{\pgfqpoint{2.353726in}{1.423558in}}%
\pgfpathcurveto{\pgfqpoint{2.345913in}{1.415744in}}{\pgfqpoint{2.341522in}{1.405145in}}{\pgfqpoint{2.341522in}{1.394095in}}%
\pgfpathcurveto{\pgfqpoint{2.341522in}{1.383045in}}{\pgfqpoint{2.345913in}{1.372446in}}{\pgfqpoint{2.353726in}{1.364632in}}%
\pgfpathcurveto{\pgfqpoint{2.361540in}{1.356819in}}{\pgfqpoint{2.372139in}{1.352428in}}{\pgfqpoint{2.383189in}{1.352428in}}%
\pgfpathlineto{\pgfqpoint{2.383189in}{1.352428in}}%
\pgfpathclose%
\pgfusepath{stroke}%
\end{pgfscope}%
\begin{pgfscope}%
\pgfpathrectangle{\pgfqpoint{0.847223in}{0.554012in}}{\pgfqpoint{6.200000in}{4.530000in}}%
\pgfusepath{clip}%
\pgfsetbuttcap%
\pgfsetroundjoin%
\pgfsetlinewidth{1.003750pt}%
\definecolor{currentstroke}{rgb}{1.000000,0.000000,0.000000}%
\pgfsetstrokecolor{currentstroke}%
\pgfsetdash{}{0pt}%
\pgfpathmoveto{\pgfqpoint{2.388522in}{1.349181in}}%
\pgfpathcurveto{\pgfqpoint{2.399573in}{1.349181in}}{\pgfqpoint{2.410172in}{1.353571in}}{\pgfqpoint{2.417985in}{1.361385in}}%
\pgfpathcurveto{\pgfqpoint{2.425799in}{1.369198in}}{\pgfqpoint{2.430189in}{1.379797in}}{\pgfqpoint{2.430189in}{1.390847in}}%
\pgfpathcurveto{\pgfqpoint{2.430189in}{1.401898in}}{\pgfqpoint{2.425799in}{1.412497in}}{\pgfqpoint{2.417985in}{1.420310in}}%
\pgfpathcurveto{\pgfqpoint{2.410172in}{1.428124in}}{\pgfqpoint{2.399573in}{1.432514in}}{\pgfqpoint{2.388522in}{1.432514in}}%
\pgfpathcurveto{\pgfqpoint{2.377472in}{1.432514in}}{\pgfqpoint{2.366873in}{1.428124in}}{\pgfqpoint{2.359060in}{1.420310in}}%
\pgfpathcurveto{\pgfqpoint{2.351246in}{1.412497in}}{\pgfqpoint{2.346856in}{1.401898in}}{\pgfqpoint{2.346856in}{1.390847in}}%
\pgfpathcurveto{\pgfqpoint{2.346856in}{1.379797in}}{\pgfqpoint{2.351246in}{1.369198in}}{\pgfqpoint{2.359060in}{1.361385in}}%
\pgfpathcurveto{\pgfqpoint{2.366873in}{1.353571in}}{\pgfqpoint{2.377472in}{1.349181in}}{\pgfqpoint{2.388522in}{1.349181in}}%
\pgfpathlineto{\pgfqpoint{2.388522in}{1.349181in}}%
\pgfpathclose%
\pgfusepath{stroke}%
\end{pgfscope}%
\begin{pgfscope}%
\pgfpathrectangle{\pgfqpoint{0.847223in}{0.554012in}}{\pgfqpoint{6.200000in}{4.530000in}}%
\pgfusepath{clip}%
\pgfsetbuttcap%
\pgfsetroundjoin%
\pgfsetlinewidth{1.003750pt}%
\definecolor{currentstroke}{rgb}{1.000000,0.000000,0.000000}%
\pgfsetstrokecolor{currentstroke}%
\pgfsetdash{}{0pt}%
\pgfpathmoveto{\pgfqpoint{2.393856in}{1.345950in}}%
\pgfpathcurveto{\pgfqpoint{2.404906in}{1.345950in}}{\pgfqpoint{2.415505in}{1.350340in}}{\pgfqpoint{2.423318in}{1.358154in}}%
\pgfpathcurveto{\pgfqpoint{2.431132in}{1.365967in}}{\pgfqpoint{2.435522in}{1.376566in}}{\pgfqpoint{2.435522in}{1.387616in}}%
\pgfpathcurveto{\pgfqpoint{2.435522in}{1.398667in}}{\pgfqpoint{2.431132in}{1.409266in}}{\pgfqpoint{2.423318in}{1.417079in}}%
\pgfpathcurveto{\pgfqpoint{2.415505in}{1.424893in}}{\pgfqpoint{2.404906in}{1.429283in}}{\pgfqpoint{2.393856in}{1.429283in}}%
\pgfpathcurveto{\pgfqpoint{2.382805in}{1.429283in}}{\pgfqpoint{2.372206in}{1.424893in}}{\pgfqpoint{2.364393in}{1.417079in}}%
\pgfpathcurveto{\pgfqpoint{2.356579in}{1.409266in}}{\pgfqpoint{2.352189in}{1.398667in}}{\pgfqpoint{2.352189in}{1.387616in}}%
\pgfpathcurveto{\pgfqpoint{2.352189in}{1.376566in}}{\pgfqpoint{2.356579in}{1.365967in}}{\pgfqpoint{2.364393in}{1.358154in}}%
\pgfpathcurveto{\pgfqpoint{2.372206in}{1.350340in}}{\pgfqpoint{2.382805in}{1.345950in}}{\pgfqpoint{2.393856in}{1.345950in}}%
\pgfpathlineto{\pgfqpoint{2.393856in}{1.345950in}}%
\pgfpathclose%
\pgfusepath{stroke}%
\end{pgfscope}%
\begin{pgfscope}%
\pgfpathrectangle{\pgfqpoint{0.847223in}{0.554012in}}{\pgfqpoint{6.200000in}{4.530000in}}%
\pgfusepath{clip}%
\pgfsetbuttcap%
\pgfsetroundjoin%
\pgfsetlinewidth{1.003750pt}%
\definecolor{currentstroke}{rgb}{1.000000,0.000000,0.000000}%
\pgfsetstrokecolor{currentstroke}%
\pgfsetdash{}{0pt}%
\pgfpathmoveto{\pgfqpoint{2.399189in}{1.342735in}}%
\pgfpathcurveto{\pgfqpoint{2.410239in}{1.342735in}}{\pgfqpoint{2.420838in}{1.347126in}}{\pgfqpoint{2.428652in}{1.354939in}}%
\pgfpathcurveto{\pgfqpoint{2.436465in}{1.362753in}}{\pgfqpoint{2.440855in}{1.373352in}}{\pgfqpoint{2.440855in}{1.384402in}}%
\pgfpathcurveto{\pgfqpoint{2.440855in}{1.395452in}}{\pgfqpoint{2.436465in}{1.406051in}}{\pgfqpoint{2.428652in}{1.413865in}}%
\pgfpathcurveto{\pgfqpoint{2.420838in}{1.421679in}}{\pgfqpoint{2.410239in}{1.426069in}}{\pgfqpoint{2.399189in}{1.426069in}}%
\pgfpathcurveto{\pgfqpoint{2.388139in}{1.426069in}}{\pgfqpoint{2.377540in}{1.421679in}}{\pgfqpoint{2.369726in}{1.413865in}}%
\pgfpathcurveto{\pgfqpoint{2.361912in}{1.406051in}}{\pgfqpoint{2.357522in}{1.395452in}}{\pgfqpoint{2.357522in}{1.384402in}}%
\pgfpathcurveto{\pgfqpoint{2.357522in}{1.373352in}}{\pgfqpoint{2.361912in}{1.362753in}}{\pgfqpoint{2.369726in}{1.354939in}}%
\pgfpathcurveto{\pgfqpoint{2.377540in}{1.347126in}}{\pgfqpoint{2.388139in}{1.342735in}}{\pgfqpoint{2.399189in}{1.342735in}}%
\pgfpathlineto{\pgfqpoint{2.399189in}{1.342735in}}%
\pgfpathclose%
\pgfusepath{stroke}%
\end{pgfscope}%
\begin{pgfscope}%
\pgfpathrectangle{\pgfqpoint{0.847223in}{0.554012in}}{\pgfqpoint{6.200000in}{4.530000in}}%
\pgfusepath{clip}%
\pgfsetbuttcap%
\pgfsetroundjoin%
\pgfsetlinewidth{1.003750pt}%
\definecolor{currentstroke}{rgb}{1.000000,0.000000,0.000000}%
\pgfsetstrokecolor{currentstroke}%
\pgfsetdash{}{0pt}%
\pgfpathmoveto{\pgfqpoint{2.404522in}{1.339538in}}%
\pgfpathcurveto{\pgfqpoint{2.415572in}{1.339538in}}{\pgfqpoint{2.426171in}{1.343928in}}{\pgfqpoint{2.433985in}{1.351742in}}%
\pgfpathcurveto{\pgfqpoint{2.441798in}{1.359555in}}{\pgfqpoint{2.446189in}{1.370154in}}{\pgfqpoint{2.446189in}{1.381204in}}%
\pgfpathcurveto{\pgfqpoint{2.446189in}{1.392254in}}{\pgfqpoint{2.441798in}{1.402853in}}{\pgfqpoint{2.433985in}{1.410667in}}%
\pgfpathcurveto{\pgfqpoint{2.426171in}{1.418481in}}{\pgfqpoint{2.415572in}{1.422871in}}{\pgfqpoint{2.404522in}{1.422871in}}%
\pgfpathcurveto{\pgfqpoint{2.393472in}{1.422871in}}{\pgfqpoint{2.382873in}{1.418481in}}{\pgfqpoint{2.375059in}{1.410667in}}%
\pgfpathcurveto{\pgfqpoint{2.367246in}{1.402853in}}{\pgfqpoint{2.362855in}{1.392254in}}{\pgfqpoint{2.362855in}{1.381204in}}%
\pgfpathcurveto{\pgfqpoint{2.362855in}{1.370154in}}{\pgfqpoint{2.367246in}{1.359555in}}{\pgfqpoint{2.375059in}{1.351742in}}%
\pgfpathcurveto{\pgfqpoint{2.382873in}{1.343928in}}{\pgfqpoint{2.393472in}{1.339538in}}{\pgfqpoint{2.404522in}{1.339538in}}%
\pgfpathlineto{\pgfqpoint{2.404522in}{1.339538in}}%
\pgfpathclose%
\pgfusepath{stroke}%
\end{pgfscope}%
\begin{pgfscope}%
\pgfpathrectangle{\pgfqpoint{0.847223in}{0.554012in}}{\pgfqpoint{6.200000in}{4.530000in}}%
\pgfusepath{clip}%
\pgfsetbuttcap%
\pgfsetroundjoin%
\pgfsetlinewidth{1.003750pt}%
\definecolor{currentstroke}{rgb}{1.000000,0.000000,0.000000}%
\pgfsetstrokecolor{currentstroke}%
\pgfsetdash{}{0pt}%
\pgfpathmoveto{\pgfqpoint{2.409855in}{1.336356in}}%
\pgfpathcurveto{\pgfqpoint{2.420905in}{1.336356in}}{\pgfqpoint{2.431504in}{1.340746in}}{\pgfqpoint{2.439318in}{1.348560in}}%
\pgfpathcurveto{\pgfqpoint{2.447132in}{1.356374in}}{\pgfqpoint{2.451522in}{1.366973in}}{\pgfqpoint{2.451522in}{1.378023in}}%
\pgfpathcurveto{\pgfqpoint{2.451522in}{1.389073in}}{\pgfqpoint{2.447132in}{1.399672in}}{\pgfqpoint{2.439318in}{1.407486in}}%
\pgfpathcurveto{\pgfqpoint{2.431504in}{1.415299in}}{\pgfqpoint{2.420905in}{1.419689in}}{\pgfqpoint{2.409855in}{1.419689in}}%
\pgfpathcurveto{\pgfqpoint{2.398805in}{1.419689in}}{\pgfqpoint{2.388206in}{1.415299in}}{\pgfqpoint{2.380392in}{1.407486in}}%
\pgfpathcurveto{\pgfqpoint{2.372579in}{1.399672in}}{\pgfqpoint{2.368189in}{1.389073in}}{\pgfqpoint{2.368189in}{1.378023in}}%
\pgfpathcurveto{\pgfqpoint{2.368189in}{1.366973in}}{\pgfqpoint{2.372579in}{1.356374in}}{\pgfqpoint{2.380392in}{1.348560in}}%
\pgfpathcurveto{\pgfqpoint{2.388206in}{1.340746in}}{\pgfqpoint{2.398805in}{1.336356in}}{\pgfqpoint{2.409855in}{1.336356in}}%
\pgfpathlineto{\pgfqpoint{2.409855in}{1.336356in}}%
\pgfpathclose%
\pgfusepath{stroke}%
\end{pgfscope}%
\begin{pgfscope}%
\pgfpathrectangle{\pgfqpoint{0.847223in}{0.554012in}}{\pgfqpoint{6.200000in}{4.530000in}}%
\pgfusepath{clip}%
\pgfsetbuttcap%
\pgfsetroundjoin%
\pgfsetlinewidth{1.003750pt}%
\definecolor{currentstroke}{rgb}{1.000000,0.000000,0.000000}%
\pgfsetstrokecolor{currentstroke}%
\pgfsetdash{}{0pt}%
\pgfpathmoveto{\pgfqpoint{2.415188in}{1.333191in}}%
\pgfpathcurveto{\pgfqpoint{2.426239in}{1.333191in}}{\pgfqpoint{2.436838in}{1.337581in}}{\pgfqpoint{2.444651in}{1.345395in}}%
\pgfpathcurveto{\pgfqpoint{2.452465in}{1.353208in}}{\pgfqpoint{2.456855in}{1.363807in}}{\pgfqpoint{2.456855in}{1.374858in}}%
\pgfpathcurveto{\pgfqpoint{2.456855in}{1.385908in}}{\pgfqpoint{2.452465in}{1.396507in}}{\pgfqpoint{2.444651in}{1.404320in}}%
\pgfpathcurveto{\pgfqpoint{2.436838in}{1.412134in}}{\pgfqpoint{2.426239in}{1.416524in}}{\pgfqpoint{2.415188in}{1.416524in}}%
\pgfpathcurveto{\pgfqpoint{2.404138in}{1.416524in}}{\pgfqpoint{2.393539in}{1.412134in}}{\pgfqpoint{2.385726in}{1.404320in}}%
\pgfpathcurveto{\pgfqpoint{2.377912in}{1.396507in}}{\pgfqpoint{2.373522in}{1.385908in}}{\pgfqpoint{2.373522in}{1.374858in}}%
\pgfpathcurveto{\pgfqpoint{2.373522in}{1.363807in}}{\pgfqpoint{2.377912in}{1.353208in}}{\pgfqpoint{2.385726in}{1.345395in}}%
\pgfpathcurveto{\pgfqpoint{2.393539in}{1.337581in}}{\pgfqpoint{2.404138in}{1.333191in}}{\pgfqpoint{2.415188in}{1.333191in}}%
\pgfpathlineto{\pgfqpoint{2.415188in}{1.333191in}}%
\pgfpathclose%
\pgfusepath{stroke}%
\end{pgfscope}%
\begin{pgfscope}%
\pgfpathrectangle{\pgfqpoint{0.847223in}{0.554012in}}{\pgfqpoint{6.200000in}{4.530000in}}%
\pgfusepath{clip}%
\pgfsetbuttcap%
\pgfsetroundjoin%
\pgfsetlinewidth{1.003750pt}%
\definecolor{currentstroke}{rgb}{1.000000,0.000000,0.000000}%
\pgfsetstrokecolor{currentstroke}%
\pgfsetdash{}{0pt}%
\pgfpathmoveto{\pgfqpoint{2.420522in}{1.330042in}}%
\pgfpathcurveto{\pgfqpoint{2.431572in}{1.330042in}}{\pgfqpoint{2.442171in}{1.334432in}}{\pgfqpoint{2.449984in}{1.342246in}}%
\pgfpathcurveto{\pgfqpoint{2.457798in}{1.350059in}}{\pgfqpoint{2.462188in}{1.360658in}}{\pgfqpoint{2.462188in}{1.371708in}}%
\pgfpathcurveto{\pgfqpoint{2.462188in}{1.382759in}}{\pgfqpoint{2.457798in}{1.393358in}}{\pgfqpoint{2.449984in}{1.401171in}}%
\pgfpathcurveto{\pgfqpoint{2.442171in}{1.408985in}}{\pgfqpoint{2.431572in}{1.413375in}}{\pgfqpoint{2.420522in}{1.413375in}}%
\pgfpathcurveto{\pgfqpoint{2.409472in}{1.413375in}}{\pgfqpoint{2.398873in}{1.408985in}}{\pgfqpoint{2.391059in}{1.401171in}}%
\pgfpathcurveto{\pgfqpoint{2.383245in}{1.393358in}}{\pgfqpoint{2.378855in}{1.382759in}}{\pgfqpoint{2.378855in}{1.371708in}}%
\pgfpathcurveto{\pgfqpoint{2.378855in}{1.360658in}}{\pgfqpoint{2.383245in}{1.350059in}}{\pgfqpoint{2.391059in}{1.342246in}}%
\pgfpathcurveto{\pgfqpoint{2.398873in}{1.334432in}}{\pgfqpoint{2.409472in}{1.330042in}}{\pgfqpoint{2.420522in}{1.330042in}}%
\pgfpathlineto{\pgfqpoint{2.420522in}{1.330042in}}%
\pgfpathclose%
\pgfusepath{stroke}%
\end{pgfscope}%
\begin{pgfscope}%
\pgfpathrectangle{\pgfqpoint{0.847223in}{0.554012in}}{\pgfqpoint{6.200000in}{4.530000in}}%
\pgfusepath{clip}%
\pgfsetbuttcap%
\pgfsetroundjoin%
\pgfsetlinewidth{1.003750pt}%
\definecolor{currentstroke}{rgb}{1.000000,0.000000,0.000000}%
\pgfsetstrokecolor{currentstroke}%
\pgfsetdash{}{0pt}%
\pgfpathmoveto{\pgfqpoint{2.425855in}{1.326909in}}%
\pgfpathcurveto{\pgfqpoint{2.436905in}{1.326909in}}{\pgfqpoint{2.447504in}{1.331299in}}{\pgfqpoint{2.455318in}{1.339112in}}%
\pgfpathcurveto{\pgfqpoint{2.463131in}{1.346926in}}{\pgfqpoint{2.467522in}{1.357525in}}{\pgfqpoint{2.467522in}{1.368575in}}%
\pgfpathcurveto{\pgfqpoint{2.467522in}{1.379625in}}{\pgfqpoint{2.463131in}{1.390224in}}{\pgfqpoint{2.455318in}{1.398038in}}%
\pgfpathcurveto{\pgfqpoint{2.447504in}{1.405852in}}{\pgfqpoint{2.436905in}{1.410242in}}{\pgfqpoint{2.425855in}{1.410242in}}%
\pgfpathcurveto{\pgfqpoint{2.414805in}{1.410242in}}{\pgfqpoint{2.404206in}{1.405852in}}{\pgfqpoint{2.396392in}{1.398038in}}%
\pgfpathcurveto{\pgfqpoint{2.388578in}{1.390224in}}{\pgfqpoint{2.384188in}{1.379625in}}{\pgfqpoint{2.384188in}{1.368575in}}%
\pgfpathcurveto{\pgfqpoint{2.384188in}{1.357525in}}{\pgfqpoint{2.388578in}{1.346926in}}{\pgfqpoint{2.396392in}{1.339112in}}%
\pgfpathcurveto{\pgfqpoint{2.404206in}{1.331299in}}{\pgfqpoint{2.414805in}{1.326909in}}{\pgfqpoint{2.425855in}{1.326909in}}%
\pgfpathlineto{\pgfqpoint{2.425855in}{1.326909in}}%
\pgfpathclose%
\pgfusepath{stroke}%
\end{pgfscope}%
\begin{pgfscope}%
\pgfpathrectangle{\pgfqpoint{0.847223in}{0.554012in}}{\pgfqpoint{6.200000in}{4.530000in}}%
\pgfusepath{clip}%
\pgfsetbuttcap%
\pgfsetroundjoin%
\pgfsetlinewidth{1.003750pt}%
\definecolor{currentstroke}{rgb}{1.000000,0.000000,0.000000}%
\pgfsetstrokecolor{currentstroke}%
\pgfsetdash{}{0pt}%
\pgfpathmoveto{\pgfqpoint{2.431188in}{1.323791in}}%
\pgfpathcurveto{\pgfqpoint{2.442238in}{1.323791in}}{\pgfqpoint{2.452837in}{1.328182in}}{\pgfqpoint{2.460651in}{1.335995in}}%
\pgfpathcurveto{\pgfqpoint{2.468465in}{1.343809in}}{\pgfqpoint{2.472855in}{1.354408in}}{\pgfqpoint{2.472855in}{1.365458in}}%
\pgfpathcurveto{\pgfqpoint{2.472855in}{1.376508in}}{\pgfqpoint{2.468465in}{1.387107in}}{\pgfqpoint{2.460651in}{1.394921in}}%
\pgfpathcurveto{\pgfqpoint{2.452837in}{1.402734in}}{\pgfqpoint{2.442238in}{1.407125in}}{\pgfqpoint{2.431188in}{1.407125in}}%
\pgfpathcurveto{\pgfqpoint{2.420138in}{1.407125in}}{\pgfqpoint{2.409539in}{1.402734in}}{\pgfqpoint{2.401725in}{1.394921in}}%
\pgfpathcurveto{\pgfqpoint{2.393912in}{1.387107in}}{\pgfqpoint{2.389521in}{1.376508in}}{\pgfqpoint{2.389521in}{1.365458in}}%
\pgfpathcurveto{\pgfqpoint{2.389521in}{1.354408in}}{\pgfqpoint{2.393912in}{1.343809in}}{\pgfqpoint{2.401725in}{1.335995in}}%
\pgfpathcurveto{\pgfqpoint{2.409539in}{1.328182in}}{\pgfqpoint{2.420138in}{1.323791in}}{\pgfqpoint{2.431188in}{1.323791in}}%
\pgfpathlineto{\pgfqpoint{2.431188in}{1.323791in}}%
\pgfpathclose%
\pgfusepath{stroke}%
\end{pgfscope}%
\begin{pgfscope}%
\pgfpathrectangle{\pgfqpoint{0.847223in}{0.554012in}}{\pgfqpoint{6.200000in}{4.530000in}}%
\pgfusepath{clip}%
\pgfsetbuttcap%
\pgfsetroundjoin%
\pgfsetlinewidth{1.003750pt}%
\definecolor{currentstroke}{rgb}{1.000000,0.000000,0.000000}%
\pgfsetstrokecolor{currentstroke}%
\pgfsetdash{}{0pt}%
\pgfpathmoveto{\pgfqpoint{2.436521in}{1.320690in}}%
\pgfpathcurveto{\pgfqpoint{2.447571in}{1.320690in}}{\pgfqpoint{2.458170in}{1.325080in}}{\pgfqpoint{2.465984in}{1.332894in}}%
\pgfpathcurveto{\pgfqpoint{2.473798in}{1.340707in}}{\pgfqpoint{2.478188in}{1.351306in}}{\pgfqpoint{2.478188in}{1.362356in}}%
\pgfpathcurveto{\pgfqpoint{2.478188in}{1.373406in}}{\pgfqpoint{2.473798in}{1.384006in}}{\pgfqpoint{2.465984in}{1.391819in}}%
\pgfpathcurveto{\pgfqpoint{2.458170in}{1.399633in}}{\pgfqpoint{2.447571in}{1.404023in}}{\pgfqpoint{2.436521in}{1.404023in}}%
\pgfpathcurveto{\pgfqpoint{2.425471in}{1.404023in}}{\pgfqpoint{2.414872in}{1.399633in}}{\pgfqpoint{2.407059in}{1.391819in}}%
\pgfpathcurveto{\pgfqpoint{2.399245in}{1.384006in}}{\pgfqpoint{2.394855in}{1.373406in}}{\pgfqpoint{2.394855in}{1.362356in}}%
\pgfpathcurveto{\pgfqpoint{2.394855in}{1.351306in}}{\pgfqpoint{2.399245in}{1.340707in}}{\pgfqpoint{2.407059in}{1.332894in}}%
\pgfpathcurveto{\pgfqpoint{2.414872in}{1.325080in}}{\pgfqpoint{2.425471in}{1.320690in}}{\pgfqpoint{2.436521in}{1.320690in}}%
\pgfpathlineto{\pgfqpoint{2.436521in}{1.320690in}}%
\pgfpathclose%
\pgfusepath{stroke}%
\end{pgfscope}%
\begin{pgfscope}%
\pgfpathrectangle{\pgfqpoint{0.847223in}{0.554012in}}{\pgfqpoint{6.200000in}{4.530000in}}%
\pgfusepath{clip}%
\pgfsetbuttcap%
\pgfsetroundjoin%
\pgfsetlinewidth{1.003750pt}%
\definecolor{currentstroke}{rgb}{1.000000,0.000000,0.000000}%
\pgfsetstrokecolor{currentstroke}%
\pgfsetdash{}{0pt}%
\pgfpathmoveto{\pgfqpoint{2.441855in}{1.317604in}}%
\pgfpathcurveto{\pgfqpoint{2.452905in}{1.317604in}}{\pgfqpoint{2.463504in}{1.321994in}}{\pgfqpoint{2.471317in}{1.329808in}}%
\pgfpathcurveto{\pgfqpoint{2.479131in}{1.337621in}}{\pgfqpoint{2.483521in}{1.348220in}}{\pgfqpoint{2.483521in}{1.359270in}}%
\pgfpathcurveto{\pgfqpoint{2.483521in}{1.370321in}}{\pgfqpoint{2.479131in}{1.380920in}}{\pgfqpoint{2.471317in}{1.388733in}}%
\pgfpathcurveto{\pgfqpoint{2.463504in}{1.396547in}}{\pgfqpoint{2.452905in}{1.400937in}}{\pgfqpoint{2.441855in}{1.400937in}}%
\pgfpathcurveto{\pgfqpoint{2.430804in}{1.400937in}}{\pgfqpoint{2.420205in}{1.396547in}}{\pgfqpoint{2.412392in}{1.388733in}}%
\pgfpathcurveto{\pgfqpoint{2.404578in}{1.380920in}}{\pgfqpoint{2.400188in}{1.370321in}}{\pgfqpoint{2.400188in}{1.359270in}}%
\pgfpathcurveto{\pgfqpoint{2.400188in}{1.348220in}}{\pgfqpoint{2.404578in}{1.337621in}}{\pgfqpoint{2.412392in}{1.329808in}}%
\pgfpathcurveto{\pgfqpoint{2.420205in}{1.321994in}}{\pgfqpoint{2.430804in}{1.317604in}}{\pgfqpoint{2.441855in}{1.317604in}}%
\pgfpathlineto{\pgfqpoint{2.441855in}{1.317604in}}%
\pgfpathclose%
\pgfusepath{stroke}%
\end{pgfscope}%
\begin{pgfscope}%
\pgfpathrectangle{\pgfqpoint{0.847223in}{0.554012in}}{\pgfqpoint{6.200000in}{4.530000in}}%
\pgfusepath{clip}%
\pgfsetbuttcap%
\pgfsetroundjoin%
\pgfsetlinewidth{1.003750pt}%
\definecolor{currentstroke}{rgb}{1.000000,0.000000,0.000000}%
\pgfsetstrokecolor{currentstroke}%
\pgfsetdash{}{0pt}%
\pgfpathmoveto{\pgfqpoint{2.447188in}{1.314533in}}%
\pgfpathcurveto{\pgfqpoint{2.458238in}{1.314533in}}{\pgfqpoint{2.468837in}{1.318924in}}{\pgfqpoint{2.476651in}{1.326737in}}%
\pgfpathcurveto{\pgfqpoint{2.484464in}{1.334551in}}{\pgfqpoint{2.488854in}{1.345150in}}{\pgfqpoint{2.488854in}{1.356200in}}%
\pgfpathcurveto{\pgfqpoint{2.488854in}{1.367250in}}{\pgfqpoint{2.484464in}{1.377849in}}{\pgfqpoint{2.476651in}{1.385663in}}%
\pgfpathcurveto{\pgfqpoint{2.468837in}{1.393476in}}{\pgfqpoint{2.458238in}{1.397867in}}{\pgfqpoint{2.447188in}{1.397867in}}%
\pgfpathcurveto{\pgfqpoint{2.436138in}{1.397867in}}{\pgfqpoint{2.425539in}{1.393476in}}{\pgfqpoint{2.417725in}{1.385663in}}%
\pgfpathcurveto{\pgfqpoint{2.409911in}{1.377849in}}{\pgfqpoint{2.405521in}{1.367250in}}{\pgfqpoint{2.405521in}{1.356200in}}%
\pgfpathcurveto{\pgfqpoint{2.405521in}{1.345150in}}{\pgfqpoint{2.409911in}{1.334551in}}{\pgfqpoint{2.417725in}{1.326737in}}%
\pgfpathcurveto{\pgfqpoint{2.425539in}{1.318924in}}{\pgfqpoint{2.436138in}{1.314533in}}{\pgfqpoint{2.447188in}{1.314533in}}%
\pgfpathlineto{\pgfqpoint{2.447188in}{1.314533in}}%
\pgfpathclose%
\pgfusepath{stroke}%
\end{pgfscope}%
\begin{pgfscope}%
\pgfpathrectangle{\pgfqpoint{0.847223in}{0.554012in}}{\pgfqpoint{6.200000in}{4.530000in}}%
\pgfusepath{clip}%
\pgfsetbuttcap%
\pgfsetroundjoin%
\pgfsetlinewidth{1.003750pt}%
\definecolor{currentstroke}{rgb}{1.000000,0.000000,0.000000}%
\pgfsetstrokecolor{currentstroke}%
\pgfsetdash{}{0pt}%
\pgfpathmoveto{\pgfqpoint{2.452521in}{1.311478in}}%
\pgfpathcurveto{\pgfqpoint{2.463571in}{1.311478in}}{\pgfqpoint{2.474170in}{1.315868in}}{\pgfqpoint{2.481984in}{1.323682in}}%
\pgfpathcurveto{\pgfqpoint{2.489797in}{1.331496in}}{\pgfqpoint{2.494188in}{1.342095in}}{\pgfqpoint{2.494188in}{1.353145in}}%
\pgfpathcurveto{\pgfqpoint{2.494188in}{1.364195in}}{\pgfqpoint{2.489797in}{1.374794in}}{\pgfqpoint{2.481984in}{1.382608in}}%
\pgfpathcurveto{\pgfqpoint{2.474170in}{1.390421in}}{\pgfqpoint{2.463571in}{1.394812in}}{\pgfqpoint{2.452521in}{1.394812in}}%
\pgfpathcurveto{\pgfqpoint{2.441471in}{1.394812in}}{\pgfqpoint{2.430872in}{1.390421in}}{\pgfqpoint{2.423058in}{1.382608in}}%
\pgfpathcurveto{\pgfqpoint{2.415245in}{1.374794in}}{\pgfqpoint{2.410854in}{1.364195in}}{\pgfqpoint{2.410854in}{1.353145in}}%
\pgfpathcurveto{\pgfqpoint{2.410854in}{1.342095in}}{\pgfqpoint{2.415245in}{1.331496in}}{\pgfqpoint{2.423058in}{1.323682in}}%
\pgfpathcurveto{\pgfqpoint{2.430872in}{1.315868in}}{\pgfqpoint{2.441471in}{1.311478in}}{\pgfqpoint{2.452521in}{1.311478in}}%
\pgfpathlineto{\pgfqpoint{2.452521in}{1.311478in}}%
\pgfpathclose%
\pgfusepath{stroke}%
\end{pgfscope}%
\begin{pgfscope}%
\pgfpathrectangle{\pgfqpoint{0.847223in}{0.554012in}}{\pgfqpoint{6.200000in}{4.530000in}}%
\pgfusepath{clip}%
\pgfsetbuttcap%
\pgfsetroundjoin%
\pgfsetlinewidth{1.003750pt}%
\definecolor{currentstroke}{rgb}{1.000000,0.000000,0.000000}%
\pgfsetstrokecolor{currentstroke}%
\pgfsetdash{}{0pt}%
\pgfpathmoveto{\pgfqpoint{2.457854in}{1.308438in}}%
\pgfpathcurveto{\pgfqpoint{2.468904in}{1.308438in}}{\pgfqpoint{2.479503in}{1.312829in}}{\pgfqpoint{2.487317in}{1.320642in}}%
\pgfpathcurveto{\pgfqpoint{2.495131in}{1.328456in}}{\pgfqpoint{2.499521in}{1.339055in}}{\pgfqpoint{2.499521in}{1.350105in}}%
\pgfpathcurveto{\pgfqpoint{2.499521in}{1.361155in}}{\pgfqpoint{2.495131in}{1.371754in}}{\pgfqpoint{2.487317in}{1.379568in}}%
\pgfpathcurveto{\pgfqpoint{2.479503in}{1.387381in}}{\pgfqpoint{2.468904in}{1.391772in}}{\pgfqpoint{2.457854in}{1.391772in}}%
\pgfpathcurveto{\pgfqpoint{2.446804in}{1.391772in}}{\pgfqpoint{2.436205in}{1.387381in}}{\pgfqpoint{2.428391in}{1.379568in}}%
\pgfpathcurveto{\pgfqpoint{2.420578in}{1.371754in}}{\pgfqpoint{2.416188in}{1.361155in}}{\pgfqpoint{2.416188in}{1.350105in}}%
\pgfpathcurveto{\pgfqpoint{2.416188in}{1.339055in}}{\pgfqpoint{2.420578in}{1.328456in}}{\pgfqpoint{2.428391in}{1.320642in}}%
\pgfpathcurveto{\pgfqpoint{2.436205in}{1.312829in}}{\pgfqpoint{2.446804in}{1.308438in}}{\pgfqpoint{2.457854in}{1.308438in}}%
\pgfpathlineto{\pgfqpoint{2.457854in}{1.308438in}}%
\pgfpathclose%
\pgfusepath{stroke}%
\end{pgfscope}%
\begin{pgfscope}%
\pgfpathrectangle{\pgfqpoint{0.847223in}{0.554012in}}{\pgfqpoint{6.200000in}{4.530000in}}%
\pgfusepath{clip}%
\pgfsetbuttcap%
\pgfsetroundjoin%
\pgfsetlinewidth{1.003750pt}%
\definecolor{currentstroke}{rgb}{1.000000,0.000000,0.000000}%
\pgfsetstrokecolor{currentstroke}%
\pgfsetdash{}{0pt}%
\pgfpathmoveto{\pgfqpoint{2.463187in}{1.305414in}}%
\pgfpathcurveto{\pgfqpoint{2.474238in}{1.305414in}}{\pgfqpoint{2.484837in}{1.309804in}}{\pgfqpoint{2.492650in}{1.317618in}}%
\pgfpathcurveto{\pgfqpoint{2.500464in}{1.325431in}}{\pgfqpoint{2.504854in}{1.336030in}}{\pgfqpoint{2.504854in}{1.347080in}}%
\pgfpathcurveto{\pgfqpoint{2.504854in}{1.358131in}}{\pgfqpoint{2.500464in}{1.368730in}}{\pgfqpoint{2.492650in}{1.376543in}}%
\pgfpathcurveto{\pgfqpoint{2.484837in}{1.384357in}}{\pgfqpoint{2.474238in}{1.388747in}}{\pgfqpoint{2.463187in}{1.388747in}}%
\pgfpathcurveto{\pgfqpoint{2.452137in}{1.388747in}}{\pgfqpoint{2.441538in}{1.384357in}}{\pgfqpoint{2.433725in}{1.376543in}}%
\pgfpathcurveto{\pgfqpoint{2.425911in}{1.368730in}}{\pgfqpoint{2.421521in}{1.358131in}}{\pgfqpoint{2.421521in}{1.347080in}}%
\pgfpathcurveto{\pgfqpoint{2.421521in}{1.336030in}}{\pgfqpoint{2.425911in}{1.325431in}}{\pgfqpoint{2.433725in}{1.317618in}}%
\pgfpathcurveto{\pgfqpoint{2.441538in}{1.309804in}}{\pgfqpoint{2.452137in}{1.305414in}}{\pgfqpoint{2.463187in}{1.305414in}}%
\pgfpathlineto{\pgfqpoint{2.463187in}{1.305414in}}%
\pgfpathclose%
\pgfusepath{stroke}%
\end{pgfscope}%
\begin{pgfscope}%
\pgfpathrectangle{\pgfqpoint{0.847223in}{0.554012in}}{\pgfqpoint{6.200000in}{4.530000in}}%
\pgfusepath{clip}%
\pgfsetbuttcap%
\pgfsetroundjoin%
\pgfsetlinewidth{1.003750pt}%
\definecolor{currentstroke}{rgb}{1.000000,0.000000,0.000000}%
\pgfsetstrokecolor{currentstroke}%
\pgfsetdash{}{0pt}%
\pgfpathmoveto{\pgfqpoint{2.468521in}{1.302404in}}%
\pgfpathcurveto{\pgfqpoint{2.479571in}{1.302404in}}{\pgfqpoint{2.490170in}{1.306794in}}{\pgfqpoint{2.497983in}{1.314608in}}%
\pgfpathcurveto{\pgfqpoint{2.505797in}{1.322422in}}{\pgfqpoint{2.510187in}{1.333021in}}{\pgfqpoint{2.510187in}{1.344071in}}%
\pgfpathcurveto{\pgfqpoint{2.510187in}{1.355121in}}{\pgfqpoint{2.505797in}{1.365720in}}{\pgfqpoint{2.497983in}{1.373534in}}%
\pgfpathcurveto{\pgfqpoint{2.490170in}{1.381347in}}{\pgfqpoint{2.479571in}{1.385737in}}{\pgfqpoint{2.468521in}{1.385737in}}%
\pgfpathcurveto{\pgfqpoint{2.457470in}{1.385737in}}{\pgfqpoint{2.446871in}{1.381347in}}{\pgfqpoint{2.439058in}{1.373534in}}%
\pgfpathcurveto{\pgfqpoint{2.431244in}{1.365720in}}{\pgfqpoint{2.426854in}{1.355121in}}{\pgfqpoint{2.426854in}{1.344071in}}%
\pgfpathcurveto{\pgfqpoint{2.426854in}{1.333021in}}{\pgfqpoint{2.431244in}{1.322422in}}{\pgfqpoint{2.439058in}{1.314608in}}%
\pgfpathcurveto{\pgfqpoint{2.446871in}{1.306794in}}{\pgfqpoint{2.457470in}{1.302404in}}{\pgfqpoint{2.468521in}{1.302404in}}%
\pgfpathlineto{\pgfqpoint{2.468521in}{1.302404in}}%
\pgfpathclose%
\pgfusepath{stroke}%
\end{pgfscope}%
\begin{pgfscope}%
\pgfpathrectangle{\pgfqpoint{0.847223in}{0.554012in}}{\pgfqpoint{6.200000in}{4.530000in}}%
\pgfusepath{clip}%
\pgfsetbuttcap%
\pgfsetroundjoin%
\pgfsetlinewidth{1.003750pt}%
\definecolor{currentstroke}{rgb}{1.000000,0.000000,0.000000}%
\pgfsetstrokecolor{currentstroke}%
\pgfsetdash{}{0pt}%
\pgfpathmoveto{\pgfqpoint{2.473854in}{1.299409in}}%
\pgfpathcurveto{\pgfqpoint{2.484904in}{1.299409in}}{\pgfqpoint{2.495503in}{1.303800in}}{\pgfqpoint{2.503317in}{1.311613in}}%
\pgfpathcurveto{\pgfqpoint{2.511130in}{1.319427in}}{\pgfqpoint{2.515520in}{1.330026in}}{\pgfqpoint{2.515520in}{1.341076in}}%
\pgfpathcurveto{\pgfqpoint{2.515520in}{1.352126in}}{\pgfqpoint{2.511130in}{1.362725in}}{\pgfqpoint{2.503317in}{1.370539in}}%
\pgfpathcurveto{\pgfqpoint{2.495503in}{1.378353in}}{\pgfqpoint{2.484904in}{1.382743in}}{\pgfqpoint{2.473854in}{1.382743in}}%
\pgfpathcurveto{\pgfqpoint{2.462804in}{1.382743in}}{\pgfqpoint{2.452205in}{1.378353in}}{\pgfqpoint{2.444391in}{1.370539in}}%
\pgfpathcurveto{\pgfqpoint{2.436577in}{1.362725in}}{\pgfqpoint{2.432187in}{1.352126in}}{\pgfqpoint{2.432187in}{1.341076in}}%
\pgfpathcurveto{\pgfqpoint{2.432187in}{1.330026in}}{\pgfqpoint{2.436577in}{1.319427in}}{\pgfqpoint{2.444391in}{1.311613in}}%
\pgfpathcurveto{\pgfqpoint{2.452205in}{1.303800in}}{\pgfqpoint{2.462804in}{1.299409in}}{\pgfqpoint{2.473854in}{1.299409in}}%
\pgfpathlineto{\pgfqpoint{2.473854in}{1.299409in}}%
\pgfpathclose%
\pgfusepath{stroke}%
\end{pgfscope}%
\begin{pgfscope}%
\pgfpathrectangle{\pgfqpoint{0.847223in}{0.554012in}}{\pgfqpoint{6.200000in}{4.530000in}}%
\pgfusepath{clip}%
\pgfsetbuttcap%
\pgfsetroundjoin%
\pgfsetlinewidth{1.003750pt}%
\definecolor{currentstroke}{rgb}{1.000000,0.000000,0.000000}%
\pgfsetstrokecolor{currentstroke}%
\pgfsetdash{}{0pt}%
\pgfpathmoveto{\pgfqpoint{2.479187in}{1.296430in}}%
\pgfpathcurveto{\pgfqpoint{2.490237in}{1.296430in}}{\pgfqpoint{2.500836in}{1.300820in}}{\pgfqpoint{2.508650in}{1.308634in}}%
\pgfpathcurveto{\pgfqpoint{2.516463in}{1.316447in}}{\pgfqpoint{2.520854in}{1.327046in}}{\pgfqpoint{2.520854in}{1.338096in}}%
\pgfpathcurveto{\pgfqpoint{2.520854in}{1.349146in}}{\pgfqpoint{2.516463in}{1.359745in}}{\pgfqpoint{2.508650in}{1.367559in}}%
\pgfpathcurveto{\pgfqpoint{2.500836in}{1.375373in}}{\pgfqpoint{2.490237in}{1.379763in}}{\pgfqpoint{2.479187in}{1.379763in}}%
\pgfpathcurveto{\pgfqpoint{2.468137in}{1.379763in}}{\pgfqpoint{2.457538in}{1.375373in}}{\pgfqpoint{2.449724in}{1.367559in}}%
\pgfpathcurveto{\pgfqpoint{2.441911in}{1.359745in}}{\pgfqpoint{2.437520in}{1.349146in}}{\pgfqpoint{2.437520in}{1.338096in}}%
\pgfpathcurveto{\pgfqpoint{2.437520in}{1.327046in}}{\pgfqpoint{2.441911in}{1.316447in}}{\pgfqpoint{2.449724in}{1.308634in}}%
\pgfpathcurveto{\pgfqpoint{2.457538in}{1.300820in}}{\pgfqpoint{2.468137in}{1.296430in}}{\pgfqpoint{2.479187in}{1.296430in}}%
\pgfpathlineto{\pgfqpoint{2.479187in}{1.296430in}}%
\pgfpathclose%
\pgfusepath{stroke}%
\end{pgfscope}%
\begin{pgfscope}%
\pgfpathrectangle{\pgfqpoint{0.847223in}{0.554012in}}{\pgfqpoint{6.200000in}{4.530000in}}%
\pgfusepath{clip}%
\pgfsetbuttcap%
\pgfsetroundjoin%
\pgfsetlinewidth{1.003750pt}%
\definecolor{currentstroke}{rgb}{1.000000,0.000000,0.000000}%
\pgfsetstrokecolor{currentstroke}%
\pgfsetdash{}{0pt}%
\pgfpathmoveto{\pgfqpoint{2.484520in}{1.293464in}}%
\pgfpathcurveto{\pgfqpoint{2.495570in}{1.293464in}}{\pgfqpoint{2.506169in}{1.297855in}}{\pgfqpoint{2.513983in}{1.305668in}}%
\pgfpathcurveto{\pgfqpoint{2.521797in}{1.313482in}}{\pgfqpoint{2.526187in}{1.324081in}}{\pgfqpoint{2.526187in}{1.335131in}}%
\pgfpathcurveto{\pgfqpoint{2.526187in}{1.346181in}}{\pgfqpoint{2.521797in}{1.356780in}}{\pgfqpoint{2.513983in}{1.364594in}}%
\pgfpathcurveto{\pgfqpoint{2.506169in}{1.372408in}}{\pgfqpoint{2.495570in}{1.376798in}}{\pgfqpoint{2.484520in}{1.376798in}}%
\pgfpathcurveto{\pgfqpoint{2.473470in}{1.376798in}}{\pgfqpoint{2.462871in}{1.372408in}}{\pgfqpoint{2.455057in}{1.364594in}}%
\pgfpathcurveto{\pgfqpoint{2.447244in}{1.356780in}}{\pgfqpoint{2.442854in}{1.346181in}}{\pgfqpoint{2.442854in}{1.335131in}}%
\pgfpathcurveto{\pgfqpoint{2.442854in}{1.324081in}}{\pgfqpoint{2.447244in}{1.313482in}}{\pgfqpoint{2.455057in}{1.305668in}}%
\pgfpathcurveto{\pgfqpoint{2.462871in}{1.297855in}}{\pgfqpoint{2.473470in}{1.293464in}}{\pgfqpoint{2.484520in}{1.293464in}}%
\pgfpathlineto{\pgfqpoint{2.484520in}{1.293464in}}%
\pgfpathclose%
\pgfusepath{stroke}%
\end{pgfscope}%
\begin{pgfscope}%
\pgfpathrectangle{\pgfqpoint{0.847223in}{0.554012in}}{\pgfqpoint{6.200000in}{4.530000in}}%
\pgfusepath{clip}%
\pgfsetbuttcap%
\pgfsetroundjoin%
\pgfsetlinewidth{1.003750pt}%
\definecolor{currentstroke}{rgb}{1.000000,0.000000,0.000000}%
\pgfsetstrokecolor{currentstroke}%
\pgfsetdash{}{0pt}%
\pgfpathmoveto{\pgfqpoint{2.489853in}{1.290514in}}%
\pgfpathcurveto{\pgfqpoint{2.500904in}{1.290514in}}{\pgfqpoint{2.511503in}{1.294904in}}{\pgfqpoint{2.519316in}{1.302718in}}%
\pgfpathcurveto{\pgfqpoint{2.527130in}{1.310531in}}{\pgfqpoint{2.531520in}{1.321130in}}{\pgfqpoint{2.531520in}{1.332181in}}%
\pgfpathcurveto{\pgfqpoint{2.531520in}{1.343231in}}{\pgfqpoint{2.527130in}{1.353830in}}{\pgfqpoint{2.519316in}{1.361643in}}%
\pgfpathcurveto{\pgfqpoint{2.511503in}{1.369457in}}{\pgfqpoint{2.500904in}{1.373847in}}{\pgfqpoint{2.489853in}{1.373847in}}%
\pgfpathcurveto{\pgfqpoint{2.478803in}{1.373847in}}{\pgfqpoint{2.468204in}{1.369457in}}{\pgfqpoint{2.460391in}{1.361643in}}%
\pgfpathcurveto{\pgfqpoint{2.452577in}{1.353830in}}{\pgfqpoint{2.448187in}{1.343231in}}{\pgfqpoint{2.448187in}{1.332181in}}%
\pgfpathcurveto{\pgfqpoint{2.448187in}{1.321130in}}{\pgfqpoint{2.452577in}{1.310531in}}{\pgfqpoint{2.460391in}{1.302718in}}%
\pgfpathcurveto{\pgfqpoint{2.468204in}{1.294904in}}{\pgfqpoint{2.478803in}{1.290514in}}{\pgfqpoint{2.489853in}{1.290514in}}%
\pgfpathlineto{\pgfqpoint{2.489853in}{1.290514in}}%
\pgfpathclose%
\pgfusepath{stroke}%
\end{pgfscope}%
\begin{pgfscope}%
\pgfpathrectangle{\pgfqpoint{0.847223in}{0.554012in}}{\pgfqpoint{6.200000in}{4.530000in}}%
\pgfusepath{clip}%
\pgfsetbuttcap%
\pgfsetroundjoin%
\pgfsetlinewidth{1.003750pt}%
\definecolor{currentstroke}{rgb}{1.000000,0.000000,0.000000}%
\pgfsetstrokecolor{currentstroke}%
\pgfsetdash{}{0pt}%
\pgfpathmoveto{\pgfqpoint{2.495187in}{1.287578in}}%
\pgfpathcurveto{\pgfqpoint{2.506237in}{1.287578in}}{\pgfqpoint{2.516836in}{1.291968in}}{\pgfqpoint{2.524649in}{1.299782in}}%
\pgfpathcurveto{\pgfqpoint{2.532463in}{1.307595in}}{\pgfqpoint{2.536853in}{1.318194in}}{\pgfqpoint{2.536853in}{1.329245in}}%
\pgfpathcurveto{\pgfqpoint{2.536853in}{1.340295in}}{\pgfqpoint{2.532463in}{1.350894in}}{\pgfqpoint{2.524649in}{1.358707in}}%
\pgfpathcurveto{\pgfqpoint{2.516836in}{1.366521in}}{\pgfqpoint{2.506237in}{1.370911in}}{\pgfqpoint{2.495187in}{1.370911in}}%
\pgfpathcurveto{\pgfqpoint{2.484137in}{1.370911in}}{\pgfqpoint{2.473538in}{1.366521in}}{\pgfqpoint{2.465724in}{1.358707in}}%
\pgfpathcurveto{\pgfqpoint{2.457910in}{1.350894in}}{\pgfqpoint{2.453520in}{1.340295in}}{\pgfqpoint{2.453520in}{1.329245in}}%
\pgfpathcurveto{\pgfqpoint{2.453520in}{1.318194in}}{\pgfqpoint{2.457910in}{1.307595in}}{\pgfqpoint{2.465724in}{1.299782in}}%
\pgfpathcurveto{\pgfqpoint{2.473538in}{1.291968in}}{\pgfqpoint{2.484137in}{1.287578in}}{\pgfqpoint{2.495187in}{1.287578in}}%
\pgfpathlineto{\pgfqpoint{2.495187in}{1.287578in}}%
\pgfpathclose%
\pgfusepath{stroke}%
\end{pgfscope}%
\begin{pgfscope}%
\pgfpathrectangle{\pgfqpoint{0.847223in}{0.554012in}}{\pgfqpoint{6.200000in}{4.530000in}}%
\pgfusepath{clip}%
\pgfsetbuttcap%
\pgfsetroundjoin%
\pgfsetlinewidth{1.003750pt}%
\definecolor{currentstroke}{rgb}{1.000000,0.000000,0.000000}%
\pgfsetstrokecolor{currentstroke}%
\pgfsetdash{}{0pt}%
\pgfpathmoveto{\pgfqpoint{2.500520in}{1.284656in}}%
\pgfpathcurveto{\pgfqpoint{2.511570in}{1.284656in}}{\pgfqpoint{2.522169in}{1.289047in}}{\pgfqpoint{2.529983in}{1.296860in}}%
\pgfpathcurveto{\pgfqpoint{2.537796in}{1.304674in}}{\pgfqpoint{2.542187in}{1.315273in}}{\pgfqpoint{2.542187in}{1.326323in}}%
\pgfpathcurveto{\pgfqpoint{2.542187in}{1.337373in}}{\pgfqpoint{2.537796in}{1.347972in}}{\pgfqpoint{2.529983in}{1.355786in}}%
\pgfpathcurveto{\pgfqpoint{2.522169in}{1.363599in}}{\pgfqpoint{2.511570in}{1.367990in}}{\pgfqpoint{2.500520in}{1.367990in}}%
\pgfpathcurveto{\pgfqpoint{2.489470in}{1.367990in}}{\pgfqpoint{2.478871in}{1.363599in}}{\pgfqpoint{2.471057in}{1.355786in}}%
\pgfpathcurveto{\pgfqpoint{2.463244in}{1.347972in}}{\pgfqpoint{2.458853in}{1.337373in}}{\pgfqpoint{2.458853in}{1.326323in}}%
\pgfpathcurveto{\pgfqpoint{2.458853in}{1.315273in}}{\pgfqpoint{2.463244in}{1.304674in}}{\pgfqpoint{2.471057in}{1.296860in}}%
\pgfpathcurveto{\pgfqpoint{2.478871in}{1.289047in}}{\pgfqpoint{2.489470in}{1.284656in}}{\pgfqpoint{2.500520in}{1.284656in}}%
\pgfpathlineto{\pgfqpoint{2.500520in}{1.284656in}}%
\pgfpathclose%
\pgfusepath{stroke}%
\end{pgfscope}%
\begin{pgfscope}%
\pgfpathrectangle{\pgfqpoint{0.847223in}{0.554012in}}{\pgfqpoint{6.200000in}{4.530000in}}%
\pgfusepath{clip}%
\pgfsetbuttcap%
\pgfsetroundjoin%
\pgfsetlinewidth{1.003750pt}%
\definecolor{currentstroke}{rgb}{1.000000,0.000000,0.000000}%
\pgfsetstrokecolor{currentstroke}%
\pgfsetdash{}{0pt}%
\pgfpathmoveto{\pgfqpoint{2.505853in}{1.281749in}}%
\pgfpathcurveto{\pgfqpoint{2.516903in}{1.281749in}}{\pgfqpoint{2.527502in}{1.286139in}}{\pgfqpoint{2.535316in}{1.293953in}}%
\pgfpathcurveto{\pgfqpoint{2.543130in}{1.301766in}}{\pgfqpoint{2.547520in}{1.312365in}}{\pgfqpoint{2.547520in}{1.323416in}}%
\pgfpathcurveto{\pgfqpoint{2.547520in}{1.334466in}}{\pgfqpoint{2.543130in}{1.345065in}}{\pgfqpoint{2.535316in}{1.352878in}}%
\pgfpathcurveto{\pgfqpoint{2.527502in}{1.360692in}}{\pgfqpoint{2.516903in}{1.365082in}}{\pgfqpoint{2.505853in}{1.365082in}}%
\pgfpathcurveto{\pgfqpoint{2.494803in}{1.365082in}}{\pgfqpoint{2.484204in}{1.360692in}}{\pgfqpoint{2.476390in}{1.352878in}}%
\pgfpathcurveto{\pgfqpoint{2.468577in}{1.345065in}}{\pgfqpoint{2.464186in}{1.334466in}}{\pgfqpoint{2.464186in}{1.323416in}}%
\pgfpathcurveto{\pgfqpoint{2.464186in}{1.312365in}}{\pgfqpoint{2.468577in}{1.301766in}}{\pgfqpoint{2.476390in}{1.293953in}}%
\pgfpathcurveto{\pgfqpoint{2.484204in}{1.286139in}}{\pgfqpoint{2.494803in}{1.281749in}}{\pgfqpoint{2.505853in}{1.281749in}}%
\pgfpathlineto{\pgfqpoint{2.505853in}{1.281749in}}%
\pgfpathclose%
\pgfusepath{stroke}%
\end{pgfscope}%
\begin{pgfscope}%
\pgfpathrectangle{\pgfqpoint{0.847223in}{0.554012in}}{\pgfqpoint{6.200000in}{4.530000in}}%
\pgfusepath{clip}%
\pgfsetbuttcap%
\pgfsetroundjoin%
\pgfsetlinewidth{1.003750pt}%
\definecolor{currentstroke}{rgb}{1.000000,0.000000,0.000000}%
\pgfsetstrokecolor{currentstroke}%
\pgfsetdash{}{0pt}%
\pgfpathmoveto{\pgfqpoint{2.511186in}{1.278856in}}%
\pgfpathcurveto{\pgfqpoint{2.522236in}{1.278856in}}{\pgfqpoint{2.532836in}{1.283246in}}{\pgfqpoint{2.540649in}{1.291060in}}%
\pgfpathcurveto{\pgfqpoint{2.548463in}{1.298873in}}{\pgfqpoint{2.552853in}{1.309472in}}{\pgfqpoint{2.552853in}{1.320522in}}%
\pgfpathcurveto{\pgfqpoint{2.552853in}{1.331572in}}{\pgfqpoint{2.548463in}{1.342172in}}{\pgfqpoint{2.540649in}{1.349985in}}%
\pgfpathcurveto{\pgfqpoint{2.532836in}{1.357799in}}{\pgfqpoint{2.522236in}{1.362189in}}{\pgfqpoint{2.511186in}{1.362189in}}%
\pgfpathcurveto{\pgfqpoint{2.500136in}{1.362189in}}{\pgfqpoint{2.489537in}{1.357799in}}{\pgfqpoint{2.481724in}{1.349985in}}%
\pgfpathcurveto{\pgfqpoint{2.473910in}{1.342172in}}{\pgfqpoint{2.469520in}{1.331572in}}{\pgfqpoint{2.469520in}{1.320522in}}%
\pgfpathcurveto{\pgfqpoint{2.469520in}{1.309472in}}{\pgfqpoint{2.473910in}{1.298873in}}{\pgfqpoint{2.481724in}{1.291060in}}%
\pgfpathcurveto{\pgfqpoint{2.489537in}{1.283246in}}{\pgfqpoint{2.500136in}{1.278856in}}{\pgfqpoint{2.511186in}{1.278856in}}%
\pgfpathlineto{\pgfqpoint{2.511186in}{1.278856in}}%
\pgfpathclose%
\pgfusepath{stroke}%
\end{pgfscope}%
\begin{pgfscope}%
\pgfpathrectangle{\pgfqpoint{0.847223in}{0.554012in}}{\pgfqpoint{6.200000in}{4.530000in}}%
\pgfusepath{clip}%
\pgfsetbuttcap%
\pgfsetroundjoin%
\pgfsetlinewidth{1.003750pt}%
\definecolor{currentstroke}{rgb}{1.000000,0.000000,0.000000}%
\pgfsetstrokecolor{currentstroke}%
\pgfsetdash{}{0pt}%
\pgfpathmoveto{\pgfqpoint{2.516520in}{1.275977in}}%
\pgfpathcurveto{\pgfqpoint{2.527570in}{1.275977in}}{\pgfqpoint{2.538169in}{1.280367in}}{\pgfqpoint{2.545982in}{1.288180in}}%
\pgfpathcurveto{\pgfqpoint{2.553796in}{1.295994in}}{\pgfqpoint{2.558186in}{1.306593in}}{\pgfqpoint{2.558186in}{1.317643in}}%
\pgfpathcurveto{\pgfqpoint{2.558186in}{1.328693in}}{\pgfqpoint{2.553796in}{1.339292in}}{\pgfqpoint{2.545982in}{1.347106in}}%
\pgfpathcurveto{\pgfqpoint{2.538169in}{1.354920in}}{\pgfqpoint{2.527570in}{1.359310in}}{\pgfqpoint{2.516520in}{1.359310in}}%
\pgfpathcurveto{\pgfqpoint{2.505469in}{1.359310in}}{\pgfqpoint{2.494870in}{1.354920in}}{\pgfqpoint{2.487057in}{1.347106in}}%
\pgfpathcurveto{\pgfqpoint{2.479243in}{1.339292in}}{\pgfqpoint{2.474853in}{1.328693in}}{\pgfqpoint{2.474853in}{1.317643in}}%
\pgfpathcurveto{\pgfqpoint{2.474853in}{1.306593in}}{\pgfqpoint{2.479243in}{1.295994in}}{\pgfqpoint{2.487057in}{1.288180in}}%
\pgfpathcurveto{\pgfqpoint{2.494870in}{1.280367in}}{\pgfqpoint{2.505469in}{1.275977in}}{\pgfqpoint{2.516520in}{1.275977in}}%
\pgfpathlineto{\pgfqpoint{2.516520in}{1.275977in}}%
\pgfpathclose%
\pgfusepath{stroke}%
\end{pgfscope}%
\begin{pgfscope}%
\pgfpathrectangle{\pgfqpoint{0.847223in}{0.554012in}}{\pgfqpoint{6.200000in}{4.530000in}}%
\pgfusepath{clip}%
\pgfsetbuttcap%
\pgfsetroundjoin%
\pgfsetlinewidth{1.003750pt}%
\definecolor{currentstroke}{rgb}{1.000000,0.000000,0.000000}%
\pgfsetstrokecolor{currentstroke}%
\pgfsetdash{}{0pt}%
\pgfpathmoveto{\pgfqpoint{2.521853in}{1.273111in}}%
\pgfpathcurveto{\pgfqpoint{2.532903in}{1.273111in}}{\pgfqpoint{2.543502in}{1.277502in}}{\pgfqpoint{2.551316in}{1.285315in}}%
\pgfpathcurveto{\pgfqpoint{2.559129in}{1.293129in}}{\pgfqpoint{2.563519in}{1.303728in}}{\pgfqpoint{2.563519in}{1.314778in}}%
\pgfpathcurveto{\pgfqpoint{2.563519in}{1.325828in}}{\pgfqpoint{2.559129in}{1.336427in}}{\pgfqpoint{2.551316in}{1.344241in}}%
\pgfpathcurveto{\pgfqpoint{2.543502in}{1.352054in}}{\pgfqpoint{2.532903in}{1.356445in}}{\pgfqpoint{2.521853in}{1.356445in}}%
\pgfpathcurveto{\pgfqpoint{2.510803in}{1.356445in}}{\pgfqpoint{2.500204in}{1.352054in}}{\pgfqpoint{2.492390in}{1.344241in}}%
\pgfpathcurveto{\pgfqpoint{2.484576in}{1.336427in}}{\pgfqpoint{2.480186in}{1.325828in}}{\pgfqpoint{2.480186in}{1.314778in}}%
\pgfpathcurveto{\pgfqpoint{2.480186in}{1.303728in}}{\pgfqpoint{2.484576in}{1.293129in}}{\pgfqpoint{2.492390in}{1.285315in}}%
\pgfpathcurveto{\pgfqpoint{2.500204in}{1.277502in}}{\pgfqpoint{2.510803in}{1.273111in}}{\pgfqpoint{2.521853in}{1.273111in}}%
\pgfpathlineto{\pgfqpoint{2.521853in}{1.273111in}}%
\pgfpathclose%
\pgfusepath{stroke}%
\end{pgfscope}%
\begin{pgfscope}%
\pgfpathrectangle{\pgfqpoint{0.847223in}{0.554012in}}{\pgfqpoint{6.200000in}{4.530000in}}%
\pgfusepath{clip}%
\pgfsetbuttcap%
\pgfsetroundjoin%
\pgfsetlinewidth{1.003750pt}%
\definecolor{currentstroke}{rgb}{1.000000,0.000000,0.000000}%
\pgfsetstrokecolor{currentstroke}%
\pgfsetdash{}{0pt}%
\pgfpathmoveto{\pgfqpoint{2.527186in}{1.270260in}}%
\pgfpathcurveto{\pgfqpoint{2.538236in}{1.270260in}}{\pgfqpoint{2.548835in}{1.274650in}}{\pgfqpoint{2.556649in}{1.282464in}}%
\pgfpathcurveto{\pgfqpoint{2.564462in}{1.290278in}}{\pgfqpoint{2.568853in}{1.300877in}}{\pgfqpoint{2.568853in}{1.311927in}}%
\pgfpathcurveto{\pgfqpoint{2.568853in}{1.322977in}}{\pgfqpoint{2.564462in}{1.333576in}}{\pgfqpoint{2.556649in}{1.341390in}}%
\pgfpathcurveto{\pgfqpoint{2.548835in}{1.349203in}}{\pgfqpoint{2.538236in}{1.353593in}}{\pgfqpoint{2.527186in}{1.353593in}}%
\pgfpathcurveto{\pgfqpoint{2.516136in}{1.353593in}}{\pgfqpoint{2.505537in}{1.349203in}}{\pgfqpoint{2.497723in}{1.341390in}}%
\pgfpathcurveto{\pgfqpoint{2.489910in}{1.333576in}}{\pgfqpoint{2.485519in}{1.322977in}}{\pgfqpoint{2.485519in}{1.311927in}}%
\pgfpathcurveto{\pgfqpoint{2.485519in}{1.300877in}}{\pgfqpoint{2.489910in}{1.290278in}}{\pgfqpoint{2.497723in}{1.282464in}}%
\pgfpathcurveto{\pgfqpoint{2.505537in}{1.274650in}}{\pgfqpoint{2.516136in}{1.270260in}}{\pgfqpoint{2.527186in}{1.270260in}}%
\pgfpathlineto{\pgfqpoint{2.527186in}{1.270260in}}%
\pgfpathclose%
\pgfusepath{stroke}%
\end{pgfscope}%
\begin{pgfscope}%
\pgfpathrectangle{\pgfqpoint{0.847223in}{0.554012in}}{\pgfqpoint{6.200000in}{4.530000in}}%
\pgfusepath{clip}%
\pgfsetbuttcap%
\pgfsetroundjoin%
\pgfsetlinewidth{1.003750pt}%
\definecolor{currentstroke}{rgb}{1.000000,0.000000,0.000000}%
\pgfsetstrokecolor{currentstroke}%
\pgfsetdash{}{0pt}%
\pgfpathmoveto{\pgfqpoint{2.532519in}{1.267423in}}%
\pgfpathcurveto{\pgfqpoint{2.543569in}{1.267423in}}{\pgfqpoint{2.554168in}{1.271813in}}{\pgfqpoint{2.561982in}{1.279627in}}%
\pgfpathcurveto{\pgfqpoint{2.569796in}{1.287440in}}{\pgfqpoint{2.574186in}{1.298039in}}{\pgfqpoint{2.574186in}{1.309089in}}%
\pgfpathcurveto{\pgfqpoint{2.574186in}{1.320139in}}{\pgfqpoint{2.569796in}{1.330738in}}{\pgfqpoint{2.561982in}{1.338552in}}%
\pgfpathcurveto{\pgfqpoint{2.554168in}{1.346366in}}{\pgfqpoint{2.543569in}{1.350756in}}{\pgfqpoint{2.532519in}{1.350756in}}%
\pgfpathcurveto{\pgfqpoint{2.521469in}{1.350756in}}{\pgfqpoint{2.510870in}{1.346366in}}{\pgfqpoint{2.503056in}{1.338552in}}%
\pgfpathcurveto{\pgfqpoint{2.495243in}{1.330738in}}{\pgfqpoint{2.490853in}{1.320139in}}{\pgfqpoint{2.490853in}{1.309089in}}%
\pgfpathcurveto{\pgfqpoint{2.490853in}{1.298039in}}{\pgfqpoint{2.495243in}{1.287440in}}{\pgfqpoint{2.503056in}{1.279627in}}%
\pgfpathcurveto{\pgfqpoint{2.510870in}{1.271813in}}{\pgfqpoint{2.521469in}{1.267423in}}{\pgfqpoint{2.532519in}{1.267423in}}%
\pgfpathlineto{\pgfqpoint{2.532519in}{1.267423in}}%
\pgfpathclose%
\pgfusepath{stroke}%
\end{pgfscope}%
\begin{pgfscope}%
\pgfpathrectangle{\pgfqpoint{0.847223in}{0.554012in}}{\pgfqpoint{6.200000in}{4.530000in}}%
\pgfusepath{clip}%
\pgfsetbuttcap%
\pgfsetroundjoin%
\pgfsetlinewidth{1.003750pt}%
\definecolor{currentstroke}{rgb}{1.000000,0.000000,0.000000}%
\pgfsetstrokecolor{currentstroke}%
\pgfsetdash{}{0pt}%
\pgfpathmoveto{\pgfqpoint{2.537852in}{1.264599in}}%
\pgfpathcurveto{\pgfqpoint{2.548903in}{1.264599in}}{\pgfqpoint{2.559502in}{1.268989in}}{\pgfqpoint{2.567315in}{1.276803in}}%
\pgfpathcurveto{\pgfqpoint{2.575129in}{1.284616in}}{\pgfqpoint{2.579519in}{1.295215in}}{\pgfqpoint{2.579519in}{1.306265in}}%
\pgfpathcurveto{\pgfqpoint{2.579519in}{1.317316in}}{\pgfqpoint{2.575129in}{1.327915in}}{\pgfqpoint{2.567315in}{1.335728in}}%
\pgfpathcurveto{\pgfqpoint{2.559502in}{1.343542in}}{\pgfqpoint{2.548903in}{1.347932in}}{\pgfqpoint{2.537852in}{1.347932in}}%
\pgfpathcurveto{\pgfqpoint{2.526802in}{1.347932in}}{\pgfqpoint{2.516203in}{1.343542in}}{\pgfqpoint{2.508390in}{1.335728in}}%
\pgfpathcurveto{\pgfqpoint{2.500576in}{1.327915in}}{\pgfqpoint{2.496186in}{1.317316in}}{\pgfqpoint{2.496186in}{1.306265in}}%
\pgfpathcurveto{\pgfqpoint{2.496186in}{1.295215in}}{\pgfqpoint{2.500576in}{1.284616in}}{\pgfqpoint{2.508390in}{1.276803in}}%
\pgfpathcurveto{\pgfqpoint{2.516203in}{1.268989in}}{\pgfqpoint{2.526802in}{1.264599in}}{\pgfqpoint{2.537852in}{1.264599in}}%
\pgfpathlineto{\pgfqpoint{2.537852in}{1.264599in}}%
\pgfpathclose%
\pgfusepath{stroke}%
\end{pgfscope}%
\begin{pgfscope}%
\pgfpathrectangle{\pgfqpoint{0.847223in}{0.554012in}}{\pgfqpoint{6.200000in}{4.530000in}}%
\pgfusepath{clip}%
\pgfsetbuttcap%
\pgfsetroundjoin%
\pgfsetlinewidth{1.003750pt}%
\definecolor{currentstroke}{rgb}{1.000000,0.000000,0.000000}%
\pgfsetstrokecolor{currentstroke}%
\pgfsetdash{}{0pt}%
\pgfpathmoveto{\pgfqpoint{2.543186in}{1.261789in}}%
\pgfpathcurveto{\pgfqpoint{2.554236in}{1.261789in}}{\pgfqpoint{2.564835in}{1.266179in}}{\pgfqpoint{2.572648in}{1.273992in}}%
\pgfpathcurveto{\pgfqpoint{2.580462in}{1.281806in}}{\pgfqpoint{2.584852in}{1.292405in}}{\pgfqpoint{2.584852in}{1.303455in}}%
\pgfpathcurveto{\pgfqpoint{2.584852in}{1.314505in}}{\pgfqpoint{2.580462in}{1.325104in}}{\pgfqpoint{2.572648in}{1.332918in}}%
\pgfpathcurveto{\pgfqpoint{2.564835in}{1.340732in}}{\pgfqpoint{2.554236in}{1.345122in}}{\pgfqpoint{2.543186in}{1.345122in}}%
\pgfpathcurveto{\pgfqpoint{2.532136in}{1.345122in}}{\pgfqpoint{2.521536in}{1.340732in}}{\pgfqpoint{2.513723in}{1.332918in}}%
\pgfpathcurveto{\pgfqpoint{2.505909in}{1.325104in}}{\pgfqpoint{2.501519in}{1.314505in}}{\pgfqpoint{2.501519in}{1.303455in}}%
\pgfpathcurveto{\pgfqpoint{2.501519in}{1.292405in}}{\pgfqpoint{2.505909in}{1.281806in}}{\pgfqpoint{2.513723in}{1.273992in}}%
\pgfpathcurveto{\pgfqpoint{2.521536in}{1.266179in}}{\pgfqpoint{2.532136in}{1.261789in}}{\pgfqpoint{2.543186in}{1.261789in}}%
\pgfpathlineto{\pgfqpoint{2.543186in}{1.261789in}}%
\pgfpathclose%
\pgfusepath{stroke}%
\end{pgfscope}%
\begin{pgfscope}%
\pgfpathrectangle{\pgfqpoint{0.847223in}{0.554012in}}{\pgfqpoint{6.200000in}{4.530000in}}%
\pgfusepath{clip}%
\pgfsetbuttcap%
\pgfsetroundjoin%
\pgfsetlinewidth{1.003750pt}%
\definecolor{currentstroke}{rgb}{1.000000,0.000000,0.000000}%
\pgfsetstrokecolor{currentstroke}%
\pgfsetdash{}{0pt}%
\pgfpathmoveto{\pgfqpoint{2.548519in}{1.258992in}}%
\pgfpathcurveto{\pgfqpoint{2.559569in}{1.258992in}}{\pgfqpoint{2.570168in}{1.263382in}}{\pgfqpoint{2.577982in}{1.271196in}}%
\pgfpathcurveto{\pgfqpoint{2.585795in}{1.279009in}}{\pgfqpoint{2.590186in}{1.289608in}}{\pgfqpoint{2.590186in}{1.300658in}}%
\pgfpathcurveto{\pgfqpoint{2.590186in}{1.311709in}}{\pgfqpoint{2.585795in}{1.322308in}}{\pgfqpoint{2.577982in}{1.330121in}}%
\pgfpathcurveto{\pgfqpoint{2.570168in}{1.337935in}}{\pgfqpoint{2.559569in}{1.342325in}}{\pgfqpoint{2.548519in}{1.342325in}}%
\pgfpathcurveto{\pgfqpoint{2.537469in}{1.342325in}}{\pgfqpoint{2.526870in}{1.337935in}}{\pgfqpoint{2.519056in}{1.330121in}}%
\pgfpathcurveto{\pgfqpoint{2.511242in}{1.322308in}}{\pgfqpoint{2.506852in}{1.311709in}}{\pgfqpoint{2.506852in}{1.300658in}}%
\pgfpathcurveto{\pgfqpoint{2.506852in}{1.289608in}}{\pgfqpoint{2.511242in}{1.279009in}}{\pgfqpoint{2.519056in}{1.271196in}}%
\pgfpathcurveto{\pgfqpoint{2.526870in}{1.263382in}}{\pgfqpoint{2.537469in}{1.258992in}}{\pgfqpoint{2.548519in}{1.258992in}}%
\pgfpathlineto{\pgfqpoint{2.548519in}{1.258992in}}%
\pgfpathclose%
\pgfusepath{stroke}%
\end{pgfscope}%
\begin{pgfscope}%
\pgfpathrectangle{\pgfqpoint{0.847223in}{0.554012in}}{\pgfqpoint{6.200000in}{4.530000in}}%
\pgfusepath{clip}%
\pgfsetbuttcap%
\pgfsetroundjoin%
\pgfsetlinewidth{1.003750pt}%
\definecolor{currentstroke}{rgb}{1.000000,0.000000,0.000000}%
\pgfsetstrokecolor{currentstroke}%
\pgfsetdash{}{0pt}%
\pgfpathmoveto{\pgfqpoint{2.553852in}{1.256208in}}%
\pgfpathcurveto{\pgfqpoint{2.564902in}{1.256208in}}{\pgfqpoint{2.575501in}{1.260599in}}{\pgfqpoint{2.583315in}{1.268412in}}%
\pgfpathcurveto{\pgfqpoint{2.591128in}{1.276226in}}{\pgfqpoint{2.595519in}{1.286825in}}{\pgfqpoint{2.595519in}{1.297875in}}%
\pgfpathcurveto{\pgfqpoint{2.595519in}{1.308925in}}{\pgfqpoint{2.591128in}{1.319524in}}{\pgfqpoint{2.583315in}{1.327338in}}%
\pgfpathcurveto{\pgfqpoint{2.575501in}{1.335151in}}{\pgfqpoint{2.564902in}{1.339542in}}{\pgfqpoint{2.553852in}{1.339542in}}%
\pgfpathcurveto{\pgfqpoint{2.542802in}{1.339542in}}{\pgfqpoint{2.532203in}{1.335151in}}{\pgfqpoint{2.524389in}{1.327338in}}%
\pgfpathcurveto{\pgfqpoint{2.516576in}{1.319524in}}{\pgfqpoint{2.512185in}{1.308925in}}{\pgfqpoint{2.512185in}{1.297875in}}%
\pgfpathcurveto{\pgfqpoint{2.512185in}{1.286825in}}{\pgfqpoint{2.516576in}{1.276226in}}{\pgfqpoint{2.524389in}{1.268412in}}%
\pgfpathcurveto{\pgfqpoint{2.532203in}{1.260599in}}{\pgfqpoint{2.542802in}{1.256208in}}{\pgfqpoint{2.553852in}{1.256208in}}%
\pgfpathlineto{\pgfqpoint{2.553852in}{1.256208in}}%
\pgfpathclose%
\pgfusepath{stroke}%
\end{pgfscope}%
\begin{pgfscope}%
\pgfpathrectangle{\pgfqpoint{0.847223in}{0.554012in}}{\pgfqpoint{6.200000in}{4.530000in}}%
\pgfusepath{clip}%
\pgfsetbuttcap%
\pgfsetroundjoin%
\pgfsetlinewidth{1.003750pt}%
\definecolor{currentstroke}{rgb}{1.000000,0.000000,0.000000}%
\pgfsetstrokecolor{currentstroke}%
\pgfsetdash{}{0pt}%
\pgfpathmoveto{\pgfqpoint{2.559185in}{1.253438in}}%
\pgfpathcurveto{\pgfqpoint{2.570235in}{1.253438in}}{\pgfqpoint{2.580834in}{1.257828in}}{\pgfqpoint{2.588648in}{1.265642in}}%
\pgfpathcurveto{\pgfqpoint{2.596462in}{1.273456in}}{\pgfqpoint{2.600852in}{1.284055in}}{\pgfqpoint{2.600852in}{1.295105in}}%
\pgfpathcurveto{\pgfqpoint{2.600852in}{1.306155in}}{\pgfqpoint{2.596462in}{1.316754in}}{\pgfqpoint{2.588648in}{1.324568in}}%
\pgfpathcurveto{\pgfqpoint{2.580834in}{1.332381in}}{\pgfqpoint{2.570235in}{1.336772in}}{\pgfqpoint{2.559185in}{1.336772in}}%
\pgfpathcurveto{\pgfqpoint{2.548135in}{1.336772in}}{\pgfqpoint{2.537536in}{1.332381in}}{\pgfqpoint{2.529722in}{1.324568in}}%
\pgfpathcurveto{\pgfqpoint{2.521909in}{1.316754in}}{\pgfqpoint{2.517519in}{1.306155in}}{\pgfqpoint{2.517519in}{1.295105in}}%
\pgfpathcurveto{\pgfqpoint{2.517519in}{1.284055in}}{\pgfqpoint{2.521909in}{1.273456in}}{\pgfqpoint{2.529722in}{1.265642in}}%
\pgfpathcurveto{\pgfqpoint{2.537536in}{1.257828in}}{\pgfqpoint{2.548135in}{1.253438in}}{\pgfqpoint{2.559185in}{1.253438in}}%
\pgfpathlineto{\pgfqpoint{2.559185in}{1.253438in}}%
\pgfpathclose%
\pgfusepath{stroke}%
\end{pgfscope}%
\begin{pgfscope}%
\pgfpathrectangle{\pgfqpoint{0.847223in}{0.554012in}}{\pgfqpoint{6.200000in}{4.530000in}}%
\pgfusepath{clip}%
\pgfsetbuttcap%
\pgfsetroundjoin%
\pgfsetlinewidth{1.003750pt}%
\definecolor{currentstroke}{rgb}{1.000000,0.000000,0.000000}%
\pgfsetstrokecolor{currentstroke}%
\pgfsetdash{}{0pt}%
\pgfpathmoveto{\pgfqpoint{2.564518in}{1.250681in}}%
\pgfpathcurveto{\pgfqpoint{2.575569in}{1.250681in}}{\pgfqpoint{2.586168in}{1.255072in}}{\pgfqpoint{2.593981in}{1.262885in}}%
\pgfpathcurveto{\pgfqpoint{2.601795in}{1.270699in}}{\pgfqpoint{2.606185in}{1.281298in}}{\pgfqpoint{2.606185in}{1.292348in}}%
\pgfpathcurveto{\pgfqpoint{2.606185in}{1.303398in}}{\pgfqpoint{2.601795in}{1.313997in}}{\pgfqpoint{2.593981in}{1.321811in}}%
\pgfpathcurveto{\pgfqpoint{2.586168in}{1.329624in}}{\pgfqpoint{2.575569in}{1.334015in}}{\pgfqpoint{2.564518in}{1.334015in}}%
\pgfpathcurveto{\pgfqpoint{2.553468in}{1.334015in}}{\pgfqpoint{2.542869in}{1.329624in}}{\pgfqpoint{2.535056in}{1.321811in}}%
\pgfpathcurveto{\pgfqpoint{2.527242in}{1.313997in}}{\pgfqpoint{2.522852in}{1.303398in}}{\pgfqpoint{2.522852in}{1.292348in}}%
\pgfpathcurveto{\pgfqpoint{2.522852in}{1.281298in}}{\pgfqpoint{2.527242in}{1.270699in}}{\pgfqpoint{2.535056in}{1.262885in}}%
\pgfpathcurveto{\pgfqpoint{2.542869in}{1.255072in}}{\pgfqpoint{2.553468in}{1.250681in}}{\pgfqpoint{2.564518in}{1.250681in}}%
\pgfpathlineto{\pgfqpoint{2.564518in}{1.250681in}}%
\pgfpathclose%
\pgfusepath{stroke}%
\end{pgfscope}%
\begin{pgfscope}%
\pgfpathrectangle{\pgfqpoint{0.847223in}{0.554012in}}{\pgfqpoint{6.200000in}{4.530000in}}%
\pgfusepath{clip}%
\pgfsetbuttcap%
\pgfsetroundjoin%
\pgfsetlinewidth{1.003750pt}%
\definecolor{currentstroke}{rgb}{1.000000,0.000000,0.000000}%
\pgfsetstrokecolor{currentstroke}%
\pgfsetdash{}{0pt}%
\pgfpathmoveto{\pgfqpoint{2.569852in}{1.247937in}}%
\pgfpathcurveto{\pgfqpoint{2.580902in}{1.247937in}}{\pgfqpoint{2.591501in}{1.252328in}}{\pgfqpoint{2.599314in}{1.260141in}}%
\pgfpathcurveto{\pgfqpoint{2.607128in}{1.267955in}}{\pgfqpoint{2.611518in}{1.278554in}}{\pgfqpoint{2.611518in}{1.289604in}}%
\pgfpathcurveto{\pgfqpoint{2.611518in}{1.300654in}}{\pgfqpoint{2.607128in}{1.311253in}}{\pgfqpoint{2.599314in}{1.319067in}}%
\pgfpathcurveto{\pgfqpoint{2.591501in}{1.326880in}}{\pgfqpoint{2.580902in}{1.331271in}}{\pgfqpoint{2.569852in}{1.331271in}}%
\pgfpathcurveto{\pgfqpoint{2.558802in}{1.331271in}}{\pgfqpoint{2.548203in}{1.326880in}}{\pgfqpoint{2.540389in}{1.319067in}}%
\pgfpathcurveto{\pgfqpoint{2.532575in}{1.311253in}}{\pgfqpoint{2.528185in}{1.300654in}}{\pgfqpoint{2.528185in}{1.289604in}}%
\pgfpathcurveto{\pgfqpoint{2.528185in}{1.278554in}}{\pgfqpoint{2.532575in}{1.267955in}}{\pgfqpoint{2.540389in}{1.260141in}}%
\pgfpathcurveto{\pgfqpoint{2.548203in}{1.252328in}}{\pgfqpoint{2.558802in}{1.247937in}}{\pgfqpoint{2.569852in}{1.247937in}}%
\pgfpathlineto{\pgfqpoint{2.569852in}{1.247937in}}%
\pgfpathclose%
\pgfusepath{stroke}%
\end{pgfscope}%
\begin{pgfscope}%
\pgfpathrectangle{\pgfqpoint{0.847223in}{0.554012in}}{\pgfqpoint{6.200000in}{4.530000in}}%
\pgfusepath{clip}%
\pgfsetbuttcap%
\pgfsetroundjoin%
\pgfsetlinewidth{1.003750pt}%
\definecolor{currentstroke}{rgb}{1.000000,0.000000,0.000000}%
\pgfsetstrokecolor{currentstroke}%
\pgfsetdash{}{0pt}%
\pgfpathmoveto{\pgfqpoint{2.575185in}{1.245207in}}%
\pgfpathcurveto{\pgfqpoint{2.586235in}{1.245207in}}{\pgfqpoint{2.596834in}{1.249597in}}{\pgfqpoint{2.604648in}{1.257410in}}%
\pgfpathcurveto{\pgfqpoint{2.612461in}{1.265224in}}{\pgfqpoint{2.616852in}{1.275823in}}{\pgfqpoint{2.616852in}{1.286873in}}%
\pgfpathcurveto{\pgfqpoint{2.616852in}{1.297923in}}{\pgfqpoint{2.612461in}{1.308522in}}{\pgfqpoint{2.604648in}{1.316336in}}%
\pgfpathcurveto{\pgfqpoint{2.596834in}{1.324150in}}{\pgfqpoint{2.586235in}{1.328540in}}{\pgfqpoint{2.575185in}{1.328540in}}%
\pgfpathcurveto{\pgfqpoint{2.564135in}{1.328540in}}{\pgfqpoint{2.553536in}{1.324150in}}{\pgfqpoint{2.545722in}{1.316336in}}%
\pgfpathcurveto{\pgfqpoint{2.537909in}{1.308522in}}{\pgfqpoint{2.533518in}{1.297923in}}{\pgfqpoint{2.533518in}{1.286873in}}%
\pgfpathcurveto{\pgfqpoint{2.533518in}{1.275823in}}{\pgfqpoint{2.537909in}{1.265224in}}{\pgfqpoint{2.545722in}{1.257410in}}%
\pgfpathcurveto{\pgfqpoint{2.553536in}{1.249597in}}{\pgfqpoint{2.564135in}{1.245207in}}{\pgfqpoint{2.575185in}{1.245207in}}%
\pgfpathlineto{\pgfqpoint{2.575185in}{1.245207in}}%
\pgfpathclose%
\pgfusepath{stroke}%
\end{pgfscope}%
\begin{pgfscope}%
\pgfpathrectangle{\pgfqpoint{0.847223in}{0.554012in}}{\pgfqpoint{6.200000in}{4.530000in}}%
\pgfusepath{clip}%
\pgfsetbuttcap%
\pgfsetroundjoin%
\pgfsetlinewidth{1.003750pt}%
\definecolor{currentstroke}{rgb}{1.000000,0.000000,0.000000}%
\pgfsetstrokecolor{currentstroke}%
\pgfsetdash{}{0pt}%
\pgfpathmoveto{\pgfqpoint{2.580518in}{1.242489in}}%
\pgfpathcurveto{\pgfqpoint{2.591568in}{1.242489in}}{\pgfqpoint{2.602167in}{1.246879in}}{\pgfqpoint{2.609981in}{1.254692in}}%
\pgfpathcurveto{\pgfqpoint{2.617795in}{1.262506in}}{\pgfqpoint{2.622185in}{1.273105in}}{\pgfqpoint{2.622185in}{1.284155in}}%
\pgfpathcurveto{\pgfqpoint{2.622185in}{1.295205in}}{\pgfqpoint{2.617795in}{1.305804in}}{\pgfqpoint{2.609981in}{1.313618in}}%
\pgfpathcurveto{\pgfqpoint{2.602167in}{1.321432in}}{\pgfqpoint{2.591568in}{1.325822in}}{\pgfqpoint{2.580518in}{1.325822in}}%
\pgfpathcurveto{\pgfqpoint{2.569468in}{1.325822in}}{\pgfqpoint{2.558869in}{1.321432in}}{\pgfqpoint{2.551055in}{1.313618in}}%
\pgfpathcurveto{\pgfqpoint{2.543242in}{1.305804in}}{\pgfqpoint{2.538851in}{1.295205in}}{\pgfqpoint{2.538851in}{1.284155in}}%
\pgfpathcurveto{\pgfqpoint{2.538851in}{1.273105in}}{\pgfqpoint{2.543242in}{1.262506in}}{\pgfqpoint{2.551055in}{1.254692in}}%
\pgfpathcurveto{\pgfqpoint{2.558869in}{1.246879in}}{\pgfqpoint{2.569468in}{1.242489in}}{\pgfqpoint{2.580518in}{1.242489in}}%
\pgfpathlineto{\pgfqpoint{2.580518in}{1.242489in}}%
\pgfpathclose%
\pgfusepath{stroke}%
\end{pgfscope}%
\begin{pgfscope}%
\pgfpathrectangle{\pgfqpoint{0.847223in}{0.554012in}}{\pgfqpoint{6.200000in}{4.530000in}}%
\pgfusepath{clip}%
\pgfsetbuttcap%
\pgfsetroundjoin%
\pgfsetlinewidth{1.003750pt}%
\definecolor{currentstroke}{rgb}{1.000000,0.000000,0.000000}%
\pgfsetstrokecolor{currentstroke}%
\pgfsetdash{}{0pt}%
\pgfpathmoveto{\pgfqpoint{2.585851in}{1.239783in}}%
\pgfpathcurveto{\pgfqpoint{2.596901in}{1.239783in}}{\pgfqpoint{2.607501in}{1.244174in}}{\pgfqpoint{2.615314in}{1.251987in}}%
\pgfpathcurveto{\pgfqpoint{2.623128in}{1.259801in}}{\pgfqpoint{2.627518in}{1.270400in}}{\pgfqpoint{2.627518in}{1.281450in}}%
\pgfpathcurveto{\pgfqpoint{2.627518in}{1.292500in}}{\pgfqpoint{2.623128in}{1.303099in}}{\pgfqpoint{2.615314in}{1.310913in}}%
\pgfpathcurveto{\pgfqpoint{2.607501in}{1.318727in}}{\pgfqpoint{2.596901in}{1.323117in}}{\pgfqpoint{2.585851in}{1.323117in}}%
\pgfpathcurveto{\pgfqpoint{2.574801in}{1.323117in}}{\pgfqpoint{2.564202in}{1.318727in}}{\pgfqpoint{2.556389in}{1.310913in}}%
\pgfpathcurveto{\pgfqpoint{2.548575in}{1.303099in}}{\pgfqpoint{2.544185in}{1.292500in}}{\pgfqpoint{2.544185in}{1.281450in}}%
\pgfpathcurveto{\pgfqpoint{2.544185in}{1.270400in}}{\pgfqpoint{2.548575in}{1.259801in}}{\pgfqpoint{2.556389in}{1.251987in}}%
\pgfpathcurveto{\pgfqpoint{2.564202in}{1.244174in}}{\pgfqpoint{2.574801in}{1.239783in}}{\pgfqpoint{2.585851in}{1.239783in}}%
\pgfpathlineto{\pgfqpoint{2.585851in}{1.239783in}}%
\pgfpathclose%
\pgfusepath{stroke}%
\end{pgfscope}%
\begin{pgfscope}%
\pgfpathrectangle{\pgfqpoint{0.847223in}{0.554012in}}{\pgfqpoint{6.200000in}{4.530000in}}%
\pgfusepath{clip}%
\pgfsetbuttcap%
\pgfsetroundjoin%
\pgfsetlinewidth{1.003750pt}%
\definecolor{currentstroke}{rgb}{1.000000,0.000000,0.000000}%
\pgfsetstrokecolor{currentstroke}%
\pgfsetdash{}{0pt}%
\pgfpathmoveto{\pgfqpoint{2.591185in}{1.237091in}}%
\pgfpathcurveto{\pgfqpoint{2.602235in}{1.237091in}}{\pgfqpoint{2.612834in}{1.241481in}}{\pgfqpoint{2.620647in}{1.249295in}}%
\pgfpathcurveto{\pgfqpoint{2.628461in}{1.257109in}}{\pgfqpoint{2.632851in}{1.267708in}}{\pgfqpoint{2.632851in}{1.278758in}}%
\pgfpathcurveto{\pgfqpoint{2.632851in}{1.289808in}}{\pgfqpoint{2.628461in}{1.300407in}}{\pgfqpoint{2.620647in}{1.308221in}}%
\pgfpathcurveto{\pgfqpoint{2.612834in}{1.316034in}}{\pgfqpoint{2.602235in}{1.320424in}}{\pgfqpoint{2.591185in}{1.320424in}}%
\pgfpathcurveto{\pgfqpoint{2.580134in}{1.320424in}}{\pgfqpoint{2.569535in}{1.316034in}}{\pgfqpoint{2.561722in}{1.308221in}}%
\pgfpathcurveto{\pgfqpoint{2.553908in}{1.300407in}}{\pgfqpoint{2.549518in}{1.289808in}}{\pgfqpoint{2.549518in}{1.278758in}}%
\pgfpathcurveto{\pgfqpoint{2.549518in}{1.267708in}}{\pgfqpoint{2.553908in}{1.257109in}}{\pgfqpoint{2.561722in}{1.249295in}}%
\pgfpathcurveto{\pgfqpoint{2.569535in}{1.241481in}}{\pgfqpoint{2.580134in}{1.237091in}}{\pgfqpoint{2.591185in}{1.237091in}}%
\pgfpathlineto{\pgfqpoint{2.591185in}{1.237091in}}%
\pgfpathclose%
\pgfusepath{stroke}%
\end{pgfscope}%
\begin{pgfscope}%
\pgfpathrectangle{\pgfqpoint{0.847223in}{0.554012in}}{\pgfqpoint{6.200000in}{4.530000in}}%
\pgfusepath{clip}%
\pgfsetbuttcap%
\pgfsetroundjoin%
\pgfsetlinewidth{1.003750pt}%
\definecolor{currentstroke}{rgb}{1.000000,0.000000,0.000000}%
\pgfsetstrokecolor{currentstroke}%
\pgfsetdash{}{0pt}%
\pgfpathmoveto{\pgfqpoint{2.596518in}{1.234411in}}%
\pgfpathcurveto{\pgfqpoint{2.607568in}{1.234411in}}{\pgfqpoint{2.618167in}{1.238802in}}{\pgfqpoint{2.625981in}{1.246615in}}%
\pgfpathcurveto{\pgfqpoint{2.633794in}{1.254429in}}{\pgfqpoint{2.638184in}{1.265028in}}{\pgfqpoint{2.638184in}{1.276078in}}%
\pgfpathcurveto{\pgfqpoint{2.638184in}{1.287128in}}{\pgfqpoint{2.633794in}{1.297727in}}{\pgfqpoint{2.625981in}{1.305541in}}%
\pgfpathcurveto{\pgfqpoint{2.618167in}{1.313354in}}{\pgfqpoint{2.607568in}{1.317745in}}{\pgfqpoint{2.596518in}{1.317745in}}%
\pgfpathcurveto{\pgfqpoint{2.585468in}{1.317745in}}{\pgfqpoint{2.574869in}{1.313354in}}{\pgfqpoint{2.567055in}{1.305541in}}%
\pgfpathcurveto{\pgfqpoint{2.559241in}{1.297727in}}{\pgfqpoint{2.554851in}{1.287128in}}{\pgfqpoint{2.554851in}{1.276078in}}%
\pgfpathcurveto{\pgfqpoint{2.554851in}{1.265028in}}{\pgfqpoint{2.559241in}{1.254429in}}{\pgfqpoint{2.567055in}{1.246615in}}%
\pgfpathcurveto{\pgfqpoint{2.574869in}{1.238802in}}{\pgfqpoint{2.585468in}{1.234411in}}{\pgfqpoint{2.596518in}{1.234411in}}%
\pgfpathlineto{\pgfqpoint{2.596518in}{1.234411in}}%
\pgfpathclose%
\pgfusepath{stroke}%
\end{pgfscope}%
\begin{pgfscope}%
\pgfpathrectangle{\pgfqpoint{0.847223in}{0.554012in}}{\pgfqpoint{6.200000in}{4.530000in}}%
\pgfusepath{clip}%
\pgfsetbuttcap%
\pgfsetroundjoin%
\pgfsetlinewidth{1.003750pt}%
\definecolor{currentstroke}{rgb}{1.000000,0.000000,0.000000}%
\pgfsetstrokecolor{currentstroke}%
\pgfsetdash{}{0pt}%
\pgfpathmoveto{\pgfqpoint{2.601851in}{1.231744in}}%
\pgfpathcurveto{\pgfqpoint{2.612901in}{1.231744in}}{\pgfqpoint{2.623500in}{1.236134in}}{\pgfqpoint{2.631314in}{1.243948in}}%
\pgfpathcurveto{\pgfqpoint{2.639127in}{1.251762in}}{\pgfqpoint{2.643518in}{1.262361in}}{\pgfqpoint{2.643518in}{1.273411in}}%
\pgfpathcurveto{\pgfqpoint{2.643518in}{1.284461in}}{\pgfqpoint{2.639127in}{1.295060in}}{\pgfqpoint{2.631314in}{1.302874in}}%
\pgfpathcurveto{\pgfqpoint{2.623500in}{1.310687in}}{\pgfqpoint{2.612901in}{1.315077in}}{\pgfqpoint{2.601851in}{1.315077in}}%
\pgfpathcurveto{\pgfqpoint{2.590801in}{1.315077in}}{\pgfqpoint{2.580202in}{1.310687in}}{\pgfqpoint{2.572388in}{1.302874in}}%
\pgfpathcurveto{\pgfqpoint{2.564575in}{1.295060in}}{\pgfqpoint{2.560184in}{1.284461in}}{\pgfqpoint{2.560184in}{1.273411in}}%
\pgfpathcurveto{\pgfqpoint{2.560184in}{1.262361in}}{\pgfqpoint{2.564575in}{1.251762in}}{\pgfqpoint{2.572388in}{1.243948in}}%
\pgfpathcurveto{\pgfqpoint{2.580202in}{1.236134in}}{\pgfqpoint{2.590801in}{1.231744in}}{\pgfqpoint{2.601851in}{1.231744in}}%
\pgfpathlineto{\pgfqpoint{2.601851in}{1.231744in}}%
\pgfpathclose%
\pgfusepath{stroke}%
\end{pgfscope}%
\begin{pgfscope}%
\pgfpathrectangle{\pgfqpoint{0.847223in}{0.554012in}}{\pgfqpoint{6.200000in}{4.530000in}}%
\pgfusepath{clip}%
\pgfsetbuttcap%
\pgfsetroundjoin%
\pgfsetlinewidth{1.003750pt}%
\definecolor{currentstroke}{rgb}{1.000000,0.000000,0.000000}%
\pgfsetstrokecolor{currentstroke}%
\pgfsetdash{}{0pt}%
\pgfpathmoveto{\pgfqpoint{2.607184in}{1.229089in}}%
\pgfpathcurveto{\pgfqpoint{2.618234in}{1.229089in}}{\pgfqpoint{2.628833in}{1.233480in}}{\pgfqpoint{2.636647in}{1.241293in}}%
\pgfpathcurveto{\pgfqpoint{2.644461in}{1.249107in}}{\pgfqpoint{2.648851in}{1.259706in}}{\pgfqpoint{2.648851in}{1.270756in}}%
\pgfpathcurveto{\pgfqpoint{2.648851in}{1.281806in}}{\pgfqpoint{2.644461in}{1.292405in}}{\pgfqpoint{2.636647in}{1.300219in}}%
\pgfpathcurveto{\pgfqpoint{2.628833in}{1.308032in}}{\pgfqpoint{2.618234in}{1.312423in}}{\pgfqpoint{2.607184in}{1.312423in}}%
\pgfpathcurveto{\pgfqpoint{2.596134in}{1.312423in}}{\pgfqpoint{2.585535in}{1.308032in}}{\pgfqpoint{2.577721in}{1.300219in}}%
\pgfpathcurveto{\pgfqpoint{2.569908in}{1.292405in}}{\pgfqpoint{2.565518in}{1.281806in}}{\pgfqpoint{2.565518in}{1.270756in}}%
\pgfpathcurveto{\pgfqpoint{2.565518in}{1.259706in}}{\pgfqpoint{2.569908in}{1.249107in}}{\pgfqpoint{2.577721in}{1.241293in}}%
\pgfpathcurveto{\pgfqpoint{2.585535in}{1.233480in}}{\pgfqpoint{2.596134in}{1.229089in}}{\pgfqpoint{2.607184in}{1.229089in}}%
\pgfpathlineto{\pgfqpoint{2.607184in}{1.229089in}}%
\pgfpathclose%
\pgfusepath{stroke}%
\end{pgfscope}%
\begin{pgfscope}%
\pgfpathrectangle{\pgfqpoint{0.847223in}{0.554012in}}{\pgfqpoint{6.200000in}{4.530000in}}%
\pgfusepath{clip}%
\pgfsetbuttcap%
\pgfsetroundjoin%
\pgfsetlinewidth{1.003750pt}%
\definecolor{currentstroke}{rgb}{1.000000,0.000000,0.000000}%
\pgfsetstrokecolor{currentstroke}%
\pgfsetdash{}{0pt}%
\pgfpathmoveto{\pgfqpoint{2.612517in}{1.226447in}}%
\pgfpathcurveto{\pgfqpoint{2.623568in}{1.226447in}}{\pgfqpoint{2.634167in}{1.230837in}}{\pgfqpoint{2.641980in}{1.238651in}}%
\pgfpathcurveto{\pgfqpoint{2.649794in}{1.246464in}}{\pgfqpoint{2.654184in}{1.257063in}}{\pgfqpoint{2.654184in}{1.268114in}}%
\pgfpathcurveto{\pgfqpoint{2.654184in}{1.279164in}}{\pgfqpoint{2.649794in}{1.289763in}}{\pgfqpoint{2.641980in}{1.297576in}}%
\pgfpathcurveto{\pgfqpoint{2.634167in}{1.305390in}}{\pgfqpoint{2.623568in}{1.309780in}}{\pgfqpoint{2.612517in}{1.309780in}}%
\pgfpathcurveto{\pgfqpoint{2.601467in}{1.309780in}}{\pgfqpoint{2.590868in}{1.305390in}}{\pgfqpoint{2.583055in}{1.297576in}}%
\pgfpathcurveto{\pgfqpoint{2.575241in}{1.289763in}}{\pgfqpoint{2.570851in}{1.279164in}}{\pgfqpoint{2.570851in}{1.268114in}}%
\pgfpathcurveto{\pgfqpoint{2.570851in}{1.257063in}}{\pgfqpoint{2.575241in}{1.246464in}}{\pgfqpoint{2.583055in}{1.238651in}}%
\pgfpathcurveto{\pgfqpoint{2.590868in}{1.230837in}}{\pgfqpoint{2.601467in}{1.226447in}}{\pgfqpoint{2.612517in}{1.226447in}}%
\pgfpathlineto{\pgfqpoint{2.612517in}{1.226447in}}%
\pgfpathclose%
\pgfusepath{stroke}%
\end{pgfscope}%
\begin{pgfscope}%
\pgfpathrectangle{\pgfqpoint{0.847223in}{0.554012in}}{\pgfqpoint{6.200000in}{4.530000in}}%
\pgfusepath{clip}%
\pgfsetbuttcap%
\pgfsetroundjoin%
\pgfsetlinewidth{1.003750pt}%
\definecolor{currentstroke}{rgb}{1.000000,0.000000,0.000000}%
\pgfsetstrokecolor{currentstroke}%
\pgfsetdash{}{0pt}%
\pgfpathmoveto{\pgfqpoint{2.617851in}{1.223817in}}%
\pgfpathcurveto{\pgfqpoint{2.628901in}{1.223817in}}{\pgfqpoint{2.639500in}{1.228207in}}{\pgfqpoint{2.647313in}{1.236021in}}%
\pgfpathcurveto{\pgfqpoint{2.655127in}{1.243834in}}{\pgfqpoint{2.659517in}{1.254433in}}{\pgfqpoint{2.659517in}{1.265484in}}%
\pgfpathcurveto{\pgfqpoint{2.659517in}{1.276534in}}{\pgfqpoint{2.655127in}{1.287133in}}{\pgfqpoint{2.647313in}{1.294946in}}%
\pgfpathcurveto{\pgfqpoint{2.639500in}{1.302760in}}{\pgfqpoint{2.628901in}{1.307150in}}{\pgfqpoint{2.617851in}{1.307150in}}%
\pgfpathcurveto{\pgfqpoint{2.606801in}{1.307150in}}{\pgfqpoint{2.596201in}{1.302760in}}{\pgfqpoint{2.588388in}{1.294946in}}%
\pgfpathcurveto{\pgfqpoint{2.580574in}{1.287133in}}{\pgfqpoint{2.576184in}{1.276534in}}{\pgfqpoint{2.576184in}{1.265484in}}%
\pgfpathcurveto{\pgfqpoint{2.576184in}{1.254433in}}{\pgfqpoint{2.580574in}{1.243834in}}{\pgfqpoint{2.588388in}{1.236021in}}%
\pgfpathcurveto{\pgfqpoint{2.596201in}{1.228207in}}{\pgfqpoint{2.606801in}{1.223817in}}{\pgfqpoint{2.617851in}{1.223817in}}%
\pgfpathlineto{\pgfqpoint{2.617851in}{1.223817in}}%
\pgfpathclose%
\pgfusepath{stroke}%
\end{pgfscope}%
\begin{pgfscope}%
\pgfpathrectangle{\pgfqpoint{0.847223in}{0.554012in}}{\pgfqpoint{6.200000in}{4.530000in}}%
\pgfusepath{clip}%
\pgfsetbuttcap%
\pgfsetroundjoin%
\pgfsetlinewidth{1.003750pt}%
\definecolor{currentstroke}{rgb}{1.000000,0.000000,0.000000}%
\pgfsetstrokecolor{currentstroke}%
\pgfsetdash{}{0pt}%
\pgfpathmoveto{\pgfqpoint{2.623184in}{1.221199in}}%
\pgfpathcurveto{\pgfqpoint{2.634234in}{1.221199in}}{\pgfqpoint{2.644833in}{1.225589in}}{\pgfqpoint{2.652647in}{1.233403in}}%
\pgfpathcurveto{\pgfqpoint{2.660460in}{1.241216in}}{\pgfqpoint{2.664851in}{1.251815in}}{\pgfqpoint{2.664851in}{1.262866in}}%
\pgfpathcurveto{\pgfqpoint{2.664851in}{1.273916in}}{\pgfqpoint{2.660460in}{1.284515in}}{\pgfqpoint{2.652647in}{1.292328in}}%
\pgfpathcurveto{\pgfqpoint{2.644833in}{1.300142in}}{\pgfqpoint{2.634234in}{1.304532in}}{\pgfqpoint{2.623184in}{1.304532in}}%
\pgfpathcurveto{\pgfqpoint{2.612134in}{1.304532in}}{\pgfqpoint{2.601535in}{1.300142in}}{\pgfqpoint{2.593721in}{1.292328in}}%
\pgfpathcurveto{\pgfqpoint{2.585907in}{1.284515in}}{\pgfqpoint{2.581517in}{1.273916in}}{\pgfqpoint{2.581517in}{1.262866in}}%
\pgfpathcurveto{\pgfqpoint{2.581517in}{1.251815in}}{\pgfqpoint{2.585907in}{1.241216in}}{\pgfqpoint{2.593721in}{1.233403in}}%
\pgfpathcurveto{\pgfqpoint{2.601535in}{1.225589in}}{\pgfqpoint{2.612134in}{1.221199in}}{\pgfqpoint{2.623184in}{1.221199in}}%
\pgfpathlineto{\pgfqpoint{2.623184in}{1.221199in}}%
\pgfpathclose%
\pgfusepath{stroke}%
\end{pgfscope}%
\begin{pgfscope}%
\pgfpathrectangle{\pgfqpoint{0.847223in}{0.554012in}}{\pgfqpoint{6.200000in}{4.530000in}}%
\pgfusepath{clip}%
\pgfsetbuttcap%
\pgfsetroundjoin%
\pgfsetlinewidth{1.003750pt}%
\definecolor{currentstroke}{rgb}{1.000000,0.000000,0.000000}%
\pgfsetstrokecolor{currentstroke}%
\pgfsetdash{}{0pt}%
\pgfpathmoveto{\pgfqpoint{2.628517in}{1.218593in}}%
\pgfpathcurveto{\pgfqpoint{2.639567in}{1.218593in}}{\pgfqpoint{2.650166in}{1.222983in}}{\pgfqpoint{2.657980in}{1.230797in}}%
\pgfpathcurveto{\pgfqpoint{2.665793in}{1.238611in}}{\pgfqpoint{2.670184in}{1.249210in}}{\pgfqpoint{2.670184in}{1.260260in}}%
\pgfpathcurveto{\pgfqpoint{2.670184in}{1.271310in}}{\pgfqpoint{2.665793in}{1.281909in}}{\pgfqpoint{2.657980in}{1.289723in}}%
\pgfpathcurveto{\pgfqpoint{2.650166in}{1.297536in}}{\pgfqpoint{2.639567in}{1.301927in}}{\pgfqpoint{2.628517in}{1.301927in}}%
\pgfpathcurveto{\pgfqpoint{2.617467in}{1.301927in}}{\pgfqpoint{2.606868in}{1.297536in}}{\pgfqpoint{2.599054in}{1.289723in}}%
\pgfpathcurveto{\pgfqpoint{2.591241in}{1.281909in}}{\pgfqpoint{2.586850in}{1.271310in}}{\pgfqpoint{2.586850in}{1.260260in}}%
\pgfpathcurveto{\pgfqpoint{2.586850in}{1.249210in}}{\pgfqpoint{2.591241in}{1.238611in}}{\pgfqpoint{2.599054in}{1.230797in}}%
\pgfpathcurveto{\pgfqpoint{2.606868in}{1.222983in}}{\pgfqpoint{2.617467in}{1.218593in}}{\pgfqpoint{2.628517in}{1.218593in}}%
\pgfpathlineto{\pgfqpoint{2.628517in}{1.218593in}}%
\pgfpathclose%
\pgfusepath{stroke}%
\end{pgfscope}%
\begin{pgfscope}%
\pgfpathrectangle{\pgfqpoint{0.847223in}{0.554012in}}{\pgfqpoint{6.200000in}{4.530000in}}%
\pgfusepath{clip}%
\pgfsetbuttcap%
\pgfsetroundjoin%
\pgfsetlinewidth{1.003750pt}%
\definecolor{currentstroke}{rgb}{1.000000,0.000000,0.000000}%
\pgfsetstrokecolor{currentstroke}%
\pgfsetdash{}{0pt}%
\pgfpathmoveto{\pgfqpoint{2.633850in}{1.215999in}}%
\pgfpathcurveto{\pgfqpoint{2.644900in}{1.215999in}}{\pgfqpoint{2.655499in}{1.220390in}}{\pgfqpoint{2.663313in}{1.228203in}}%
\pgfpathcurveto{\pgfqpoint{2.671127in}{1.236017in}}{\pgfqpoint{2.675517in}{1.246616in}}{\pgfqpoint{2.675517in}{1.257666in}}%
\pgfpathcurveto{\pgfqpoint{2.675517in}{1.268716in}}{\pgfqpoint{2.671127in}{1.279315in}}{\pgfqpoint{2.663313in}{1.287129in}}%
\pgfpathcurveto{\pgfqpoint{2.655499in}{1.294942in}}{\pgfqpoint{2.644900in}{1.299333in}}{\pgfqpoint{2.633850in}{1.299333in}}%
\pgfpathcurveto{\pgfqpoint{2.622800in}{1.299333in}}{\pgfqpoint{2.612201in}{1.294942in}}{\pgfqpoint{2.604388in}{1.287129in}}%
\pgfpathcurveto{\pgfqpoint{2.596574in}{1.279315in}}{\pgfqpoint{2.592184in}{1.268716in}}{\pgfqpoint{2.592184in}{1.257666in}}%
\pgfpathcurveto{\pgfqpoint{2.592184in}{1.246616in}}{\pgfqpoint{2.596574in}{1.236017in}}{\pgfqpoint{2.604388in}{1.228203in}}%
\pgfpathcurveto{\pgfqpoint{2.612201in}{1.220390in}}{\pgfqpoint{2.622800in}{1.215999in}}{\pgfqpoint{2.633850in}{1.215999in}}%
\pgfpathlineto{\pgfqpoint{2.633850in}{1.215999in}}%
\pgfpathclose%
\pgfusepath{stroke}%
\end{pgfscope}%
\begin{pgfscope}%
\pgfpathrectangle{\pgfqpoint{0.847223in}{0.554012in}}{\pgfqpoint{6.200000in}{4.530000in}}%
\pgfusepath{clip}%
\pgfsetbuttcap%
\pgfsetroundjoin%
\pgfsetlinewidth{1.003750pt}%
\definecolor{currentstroke}{rgb}{1.000000,0.000000,0.000000}%
\pgfsetstrokecolor{currentstroke}%
\pgfsetdash{}{0pt}%
\pgfpathmoveto{\pgfqpoint{2.639184in}{1.213418in}}%
\pgfpathcurveto{\pgfqpoint{2.650234in}{1.213418in}}{\pgfqpoint{2.660833in}{1.217808in}}{\pgfqpoint{2.668646in}{1.225622in}}%
\pgfpathcurveto{\pgfqpoint{2.676460in}{1.233435in}}{\pgfqpoint{2.680850in}{1.244034in}}{\pgfqpoint{2.680850in}{1.255084in}}%
\pgfpathcurveto{\pgfqpoint{2.680850in}{1.266134in}}{\pgfqpoint{2.676460in}{1.276733in}}{\pgfqpoint{2.668646in}{1.284547in}}%
\pgfpathcurveto{\pgfqpoint{2.660833in}{1.292361in}}{\pgfqpoint{2.650234in}{1.296751in}}{\pgfqpoint{2.639184in}{1.296751in}}%
\pgfpathcurveto{\pgfqpoint{2.628133in}{1.296751in}}{\pgfqpoint{2.617534in}{1.292361in}}{\pgfqpoint{2.609721in}{1.284547in}}%
\pgfpathcurveto{\pgfqpoint{2.601907in}{1.276733in}}{\pgfqpoint{2.597517in}{1.266134in}}{\pgfqpoint{2.597517in}{1.255084in}}%
\pgfpathcurveto{\pgfqpoint{2.597517in}{1.244034in}}{\pgfqpoint{2.601907in}{1.233435in}}{\pgfqpoint{2.609721in}{1.225622in}}%
\pgfpathcurveto{\pgfqpoint{2.617534in}{1.217808in}}{\pgfqpoint{2.628133in}{1.213418in}}{\pgfqpoint{2.639184in}{1.213418in}}%
\pgfpathlineto{\pgfqpoint{2.639184in}{1.213418in}}%
\pgfpathclose%
\pgfusepath{stroke}%
\end{pgfscope}%
\begin{pgfscope}%
\pgfpathrectangle{\pgfqpoint{0.847223in}{0.554012in}}{\pgfqpoint{6.200000in}{4.530000in}}%
\pgfusepath{clip}%
\pgfsetbuttcap%
\pgfsetroundjoin%
\pgfsetlinewidth{1.003750pt}%
\definecolor{currentstroke}{rgb}{1.000000,0.000000,0.000000}%
\pgfsetstrokecolor{currentstroke}%
\pgfsetdash{}{0pt}%
\pgfpathmoveto{\pgfqpoint{2.644517in}{1.210848in}}%
\pgfpathcurveto{\pgfqpoint{2.655567in}{1.210848in}}{\pgfqpoint{2.666166in}{1.215238in}}{\pgfqpoint{2.673980in}{1.223052in}}%
\pgfpathcurveto{\pgfqpoint{2.681793in}{1.230865in}}{\pgfqpoint{2.686183in}{1.241464in}}{\pgfqpoint{2.686183in}{1.252514in}}%
\pgfpathcurveto{\pgfqpoint{2.686183in}{1.263564in}}{\pgfqpoint{2.681793in}{1.274164in}}{\pgfqpoint{2.673980in}{1.281977in}}%
\pgfpathcurveto{\pgfqpoint{2.666166in}{1.289791in}}{\pgfqpoint{2.655567in}{1.294181in}}{\pgfqpoint{2.644517in}{1.294181in}}%
\pgfpathcurveto{\pgfqpoint{2.633467in}{1.294181in}}{\pgfqpoint{2.622868in}{1.289791in}}{\pgfqpoint{2.615054in}{1.281977in}}%
\pgfpathcurveto{\pgfqpoint{2.607240in}{1.274164in}}{\pgfqpoint{2.602850in}{1.263564in}}{\pgfqpoint{2.602850in}{1.252514in}}%
\pgfpathcurveto{\pgfqpoint{2.602850in}{1.241464in}}{\pgfqpoint{2.607240in}{1.230865in}}{\pgfqpoint{2.615054in}{1.223052in}}%
\pgfpathcurveto{\pgfqpoint{2.622868in}{1.215238in}}{\pgfqpoint{2.633467in}{1.210848in}}{\pgfqpoint{2.644517in}{1.210848in}}%
\pgfpathlineto{\pgfqpoint{2.644517in}{1.210848in}}%
\pgfpathclose%
\pgfusepath{stroke}%
\end{pgfscope}%
\begin{pgfscope}%
\pgfpathrectangle{\pgfqpoint{0.847223in}{0.554012in}}{\pgfqpoint{6.200000in}{4.530000in}}%
\pgfusepath{clip}%
\pgfsetbuttcap%
\pgfsetroundjoin%
\pgfsetlinewidth{1.003750pt}%
\definecolor{currentstroke}{rgb}{1.000000,0.000000,0.000000}%
\pgfsetstrokecolor{currentstroke}%
\pgfsetdash{}{0pt}%
\pgfpathmoveto{\pgfqpoint{2.649850in}{1.208290in}}%
\pgfpathcurveto{\pgfqpoint{2.660900in}{1.208290in}}{\pgfqpoint{2.671499in}{1.212680in}}{\pgfqpoint{2.679313in}{1.220493in}}%
\pgfpathcurveto{\pgfqpoint{2.687126in}{1.228307in}}{\pgfqpoint{2.691517in}{1.238906in}}{\pgfqpoint{2.691517in}{1.249956in}}%
\pgfpathcurveto{\pgfqpoint{2.691517in}{1.261006in}}{\pgfqpoint{2.687126in}{1.271605in}}{\pgfqpoint{2.679313in}{1.279419in}}%
\pgfpathcurveto{\pgfqpoint{2.671499in}{1.287233in}}{\pgfqpoint{2.660900in}{1.291623in}}{\pgfqpoint{2.649850in}{1.291623in}}%
\pgfpathcurveto{\pgfqpoint{2.638800in}{1.291623in}}{\pgfqpoint{2.628201in}{1.287233in}}{\pgfqpoint{2.620387in}{1.279419in}}%
\pgfpathcurveto{\pgfqpoint{2.612574in}{1.271605in}}{\pgfqpoint{2.608183in}{1.261006in}}{\pgfqpoint{2.608183in}{1.249956in}}%
\pgfpathcurveto{\pgfqpoint{2.608183in}{1.238906in}}{\pgfqpoint{2.612574in}{1.228307in}}{\pgfqpoint{2.620387in}{1.220493in}}%
\pgfpathcurveto{\pgfqpoint{2.628201in}{1.212680in}}{\pgfqpoint{2.638800in}{1.208290in}}{\pgfqpoint{2.649850in}{1.208290in}}%
\pgfpathlineto{\pgfqpoint{2.649850in}{1.208290in}}%
\pgfpathclose%
\pgfusepath{stroke}%
\end{pgfscope}%
\begin{pgfscope}%
\pgfpathrectangle{\pgfqpoint{0.847223in}{0.554012in}}{\pgfqpoint{6.200000in}{4.530000in}}%
\pgfusepath{clip}%
\pgfsetbuttcap%
\pgfsetroundjoin%
\pgfsetlinewidth{1.003750pt}%
\definecolor{currentstroke}{rgb}{1.000000,0.000000,0.000000}%
\pgfsetstrokecolor{currentstroke}%
\pgfsetdash{}{0pt}%
\pgfpathmoveto{\pgfqpoint{2.655183in}{1.205743in}}%
\pgfpathcurveto{\pgfqpoint{2.666233in}{1.205743in}}{\pgfqpoint{2.676832in}{1.210133in}}{\pgfqpoint{2.684646in}{1.217947in}}%
\pgfpathcurveto{\pgfqpoint{2.692460in}{1.225761in}}{\pgfqpoint{2.696850in}{1.236360in}}{\pgfqpoint{2.696850in}{1.247410in}}%
\pgfpathcurveto{\pgfqpoint{2.696850in}{1.258460in}}{\pgfqpoint{2.692460in}{1.269059in}}{\pgfqpoint{2.684646in}{1.276873in}}%
\pgfpathcurveto{\pgfqpoint{2.676832in}{1.284686in}}{\pgfqpoint{2.666233in}{1.289076in}}{\pgfqpoint{2.655183in}{1.289076in}}%
\pgfpathcurveto{\pgfqpoint{2.644133in}{1.289076in}}{\pgfqpoint{2.633534in}{1.284686in}}{\pgfqpoint{2.625720in}{1.276873in}}%
\pgfpathcurveto{\pgfqpoint{2.617907in}{1.269059in}}{\pgfqpoint{2.613516in}{1.258460in}}{\pgfqpoint{2.613516in}{1.247410in}}%
\pgfpathcurveto{\pgfqpoint{2.613516in}{1.236360in}}{\pgfqpoint{2.617907in}{1.225761in}}{\pgfqpoint{2.625720in}{1.217947in}}%
\pgfpathcurveto{\pgfqpoint{2.633534in}{1.210133in}}{\pgfqpoint{2.644133in}{1.205743in}}{\pgfqpoint{2.655183in}{1.205743in}}%
\pgfpathlineto{\pgfqpoint{2.655183in}{1.205743in}}%
\pgfpathclose%
\pgfusepath{stroke}%
\end{pgfscope}%
\begin{pgfscope}%
\pgfpathrectangle{\pgfqpoint{0.847223in}{0.554012in}}{\pgfqpoint{6.200000in}{4.530000in}}%
\pgfusepath{clip}%
\pgfsetbuttcap%
\pgfsetroundjoin%
\pgfsetlinewidth{1.003750pt}%
\definecolor{currentstroke}{rgb}{1.000000,0.000000,0.000000}%
\pgfsetstrokecolor{currentstroke}%
\pgfsetdash{}{0pt}%
\pgfpathmoveto{\pgfqpoint{2.660516in}{1.203208in}}%
\pgfpathcurveto{\pgfqpoint{2.671567in}{1.203208in}}{\pgfqpoint{2.682166in}{1.207599in}}{\pgfqpoint{2.689979in}{1.215412in}}%
\pgfpathcurveto{\pgfqpoint{2.697793in}{1.223226in}}{\pgfqpoint{2.702183in}{1.233825in}}{\pgfqpoint{2.702183in}{1.244875in}}%
\pgfpathcurveto{\pgfqpoint{2.702183in}{1.255925in}}{\pgfqpoint{2.697793in}{1.266524in}}{\pgfqpoint{2.689979in}{1.274338in}}%
\pgfpathcurveto{\pgfqpoint{2.682166in}{1.282151in}}{\pgfqpoint{2.671567in}{1.286542in}}{\pgfqpoint{2.660516in}{1.286542in}}%
\pgfpathcurveto{\pgfqpoint{2.649466in}{1.286542in}}{\pgfqpoint{2.638867in}{1.282151in}}{\pgfqpoint{2.631054in}{1.274338in}}%
\pgfpathcurveto{\pgfqpoint{2.623240in}{1.266524in}}{\pgfqpoint{2.618850in}{1.255925in}}{\pgfqpoint{2.618850in}{1.244875in}}%
\pgfpathcurveto{\pgfqpoint{2.618850in}{1.233825in}}{\pgfqpoint{2.623240in}{1.223226in}}{\pgfqpoint{2.631054in}{1.215412in}}%
\pgfpathcurveto{\pgfqpoint{2.638867in}{1.207599in}}{\pgfqpoint{2.649466in}{1.203208in}}{\pgfqpoint{2.660516in}{1.203208in}}%
\pgfpathlineto{\pgfqpoint{2.660516in}{1.203208in}}%
\pgfpathclose%
\pgfusepath{stroke}%
\end{pgfscope}%
\begin{pgfscope}%
\pgfpathrectangle{\pgfqpoint{0.847223in}{0.554012in}}{\pgfqpoint{6.200000in}{4.530000in}}%
\pgfusepath{clip}%
\pgfsetbuttcap%
\pgfsetroundjoin%
\pgfsetlinewidth{1.003750pt}%
\definecolor{currentstroke}{rgb}{1.000000,0.000000,0.000000}%
\pgfsetstrokecolor{currentstroke}%
\pgfsetdash{}{0pt}%
\pgfpathmoveto{\pgfqpoint{2.665850in}{1.200685in}}%
\pgfpathcurveto{\pgfqpoint{2.676900in}{1.200685in}}{\pgfqpoint{2.687499in}{1.205075in}}{\pgfqpoint{2.695312in}{1.212889in}}%
\pgfpathcurveto{\pgfqpoint{2.703126in}{1.220703in}}{\pgfqpoint{2.707516in}{1.231302in}}{\pgfqpoint{2.707516in}{1.242352in}}%
\pgfpathcurveto{\pgfqpoint{2.707516in}{1.253402in}}{\pgfqpoint{2.703126in}{1.264001in}}{\pgfqpoint{2.695312in}{1.271814in}}%
\pgfpathcurveto{\pgfqpoint{2.687499in}{1.279628in}}{\pgfqpoint{2.676900in}{1.284018in}}{\pgfqpoint{2.665850in}{1.284018in}}%
\pgfpathcurveto{\pgfqpoint{2.654799in}{1.284018in}}{\pgfqpoint{2.644200in}{1.279628in}}{\pgfqpoint{2.636387in}{1.271814in}}%
\pgfpathcurveto{\pgfqpoint{2.628573in}{1.264001in}}{\pgfqpoint{2.624183in}{1.253402in}}{\pgfqpoint{2.624183in}{1.242352in}}%
\pgfpathcurveto{\pgfqpoint{2.624183in}{1.231302in}}{\pgfqpoint{2.628573in}{1.220703in}}{\pgfqpoint{2.636387in}{1.212889in}}%
\pgfpathcurveto{\pgfqpoint{2.644200in}{1.205075in}}{\pgfqpoint{2.654799in}{1.200685in}}{\pgfqpoint{2.665850in}{1.200685in}}%
\pgfpathlineto{\pgfqpoint{2.665850in}{1.200685in}}%
\pgfpathclose%
\pgfusepath{stroke}%
\end{pgfscope}%
\begin{pgfscope}%
\pgfpathrectangle{\pgfqpoint{0.847223in}{0.554012in}}{\pgfqpoint{6.200000in}{4.530000in}}%
\pgfusepath{clip}%
\pgfsetbuttcap%
\pgfsetroundjoin%
\pgfsetlinewidth{1.003750pt}%
\definecolor{currentstroke}{rgb}{1.000000,0.000000,0.000000}%
\pgfsetstrokecolor{currentstroke}%
\pgfsetdash{}{0pt}%
\pgfpathmoveto{\pgfqpoint{2.671183in}{1.198173in}}%
\pgfpathcurveto{\pgfqpoint{2.682233in}{1.198173in}}{\pgfqpoint{2.692832in}{1.202563in}}{\pgfqpoint{2.700646in}{1.210377in}}%
\pgfpathcurveto{\pgfqpoint{2.708459in}{1.218191in}}{\pgfqpoint{2.712849in}{1.228790in}}{\pgfqpoint{2.712849in}{1.239840in}}%
\pgfpathcurveto{\pgfqpoint{2.712849in}{1.250890in}}{\pgfqpoint{2.708459in}{1.261489in}}{\pgfqpoint{2.700646in}{1.269303in}}%
\pgfpathcurveto{\pgfqpoint{2.692832in}{1.277116in}}{\pgfqpoint{2.682233in}{1.281507in}}{\pgfqpoint{2.671183in}{1.281507in}}%
\pgfpathcurveto{\pgfqpoint{2.660133in}{1.281507in}}{\pgfqpoint{2.649534in}{1.277116in}}{\pgfqpoint{2.641720in}{1.269303in}}%
\pgfpathcurveto{\pgfqpoint{2.633906in}{1.261489in}}{\pgfqpoint{2.629516in}{1.250890in}}{\pgfqpoint{2.629516in}{1.239840in}}%
\pgfpathcurveto{\pgfqpoint{2.629516in}{1.228790in}}{\pgfqpoint{2.633906in}{1.218191in}}{\pgfqpoint{2.641720in}{1.210377in}}%
\pgfpathcurveto{\pgfqpoint{2.649534in}{1.202563in}}{\pgfqpoint{2.660133in}{1.198173in}}{\pgfqpoint{2.671183in}{1.198173in}}%
\pgfpathlineto{\pgfqpoint{2.671183in}{1.198173in}}%
\pgfpathclose%
\pgfusepath{stroke}%
\end{pgfscope}%
\begin{pgfscope}%
\pgfpathrectangle{\pgfqpoint{0.847223in}{0.554012in}}{\pgfqpoint{6.200000in}{4.530000in}}%
\pgfusepath{clip}%
\pgfsetbuttcap%
\pgfsetroundjoin%
\pgfsetlinewidth{1.003750pt}%
\definecolor{currentstroke}{rgb}{1.000000,0.000000,0.000000}%
\pgfsetstrokecolor{currentstroke}%
\pgfsetdash{}{0pt}%
\pgfpathmoveto{\pgfqpoint{2.676516in}{1.195673in}}%
\pgfpathcurveto{\pgfqpoint{2.687566in}{1.195673in}}{\pgfqpoint{2.698165in}{1.200063in}}{\pgfqpoint{2.705979in}{1.207877in}}%
\pgfpathcurveto{\pgfqpoint{2.713792in}{1.215690in}}{\pgfqpoint{2.718183in}{1.226289in}}{\pgfqpoint{2.718183in}{1.237339in}}%
\pgfpathcurveto{\pgfqpoint{2.718183in}{1.248390in}}{\pgfqpoint{2.713792in}{1.258989in}}{\pgfqpoint{2.705979in}{1.266802in}}%
\pgfpathcurveto{\pgfqpoint{2.698165in}{1.274616in}}{\pgfqpoint{2.687566in}{1.279006in}}{\pgfqpoint{2.676516in}{1.279006in}}%
\pgfpathcurveto{\pgfqpoint{2.665466in}{1.279006in}}{\pgfqpoint{2.654867in}{1.274616in}}{\pgfqpoint{2.647053in}{1.266802in}}%
\pgfpathcurveto{\pgfqpoint{2.639240in}{1.258989in}}{\pgfqpoint{2.634849in}{1.248390in}}{\pgfqpoint{2.634849in}{1.237339in}}%
\pgfpathcurveto{\pgfqpoint{2.634849in}{1.226289in}}{\pgfqpoint{2.639240in}{1.215690in}}{\pgfqpoint{2.647053in}{1.207877in}}%
\pgfpathcurveto{\pgfqpoint{2.654867in}{1.200063in}}{\pgfqpoint{2.665466in}{1.195673in}}{\pgfqpoint{2.676516in}{1.195673in}}%
\pgfpathlineto{\pgfqpoint{2.676516in}{1.195673in}}%
\pgfpathclose%
\pgfusepath{stroke}%
\end{pgfscope}%
\begin{pgfscope}%
\pgfpathrectangle{\pgfqpoint{0.847223in}{0.554012in}}{\pgfqpoint{6.200000in}{4.530000in}}%
\pgfusepath{clip}%
\pgfsetbuttcap%
\pgfsetroundjoin%
\pgfsetlinewidth{1.003750pt}%
\definecolor{currentstroke}{rgb}{1.000000,0.000000,0.000000}%
\pgfsetstrokecolor{currentstroke}%
\pgfsetdash{}{0pt}%
\pgfpathmoveto{\pgfqpoint{2.681849in}{1.193184in}}%
\pgfpathcurveto{\pgfqpoint{2.692899in}{1.193184in}}{\pgfqpoint{2.703498in}{1.197574in}}{\pgfqpoint{2.711312in}{1.205388in}}%
\pgfpathcurveto{\pgfqpoint{2.719126in}{1.213201in}}{\pgfqpoint{2.723516in}{1.223800in}}{\pgfqpoint{2.723516in}{1.234850in}}%
\pgfpathcurveto{\pgfqpoint{2.723516in}{1.245900in}}{\pgfqpoint{2.719126in}{1.256499in}}{\pgfqpoint{2.711312in}{1.264313in}}%
\pgfpathcurveto{\pgfqpoint{2.703498in}{1.272127in}}{\pgfqpoint{2.692899in}{1.276517in}}{\pgfqpoint{2.681849in}{1.276517in}}%
\pgfpathcurveto{\pgfqpoint{2.670799in}{1.276517in}}{\pgfqpoint{2.660200in}{1.272127in}}{\pgfqpoint{2.652386in}{1.264313in}}%
\pgfpathcurveto{\pgfqpoint{2.644573in}{1.256499in}}{\pgfqpoint{2.640183in}{1.245900in}}{\pgfqpoint{2.640183in}{1.234850in}}%
\pgfpathcurveto{\pgfqpoint{2.640183in}{1.223800in}}{\pgfqpoint{2.644573in}{1.213201in}}{\pgfqpoint{2.652386in}{1.205388in}}%
\pgfpathcurveto{\pgfqpoint{2.660200in}{1.197574in}}{\pgfqpoint{2.670799in}{1.193184in}}{\pgfqpoint{2.681849in}{1.193184in}}%
\pgfpathlineto{\pgfqpoint{2.681849in}{1.193184in}}%
\pgfpathclose%
\pgfusepath{stroke}%
\end{pgfscope}%
\begin{pgfscope}%
\pgfpathrectangle{\pgfqpoint{0.847223in}{0.554012in}}{\pgfqpoint{6.200000in}{4.530000in}}%
\pgfusepath{clip}%
\pgfsetbuttcap%
\pgfsetroundjoin%
\pgfsetlinewidth{1.003750pt}%
\definecolor{currentstroke}{rgb}{1.000000,0.000000,0.000000}%
\pgfsetstrokecolor{currentstroke}%
\pgfsetdash{}{0pt}%
\pgfpathmoveto{\pgfqpoint{2.687182in}{1.190706in}}%
\pgfpathcurveto{\pgfqpoint{2.698233in}{1.190706in}}{\pgfqpoint{2.708832in}{1.195096in}}{\pgfqpoint{2.716645in}{1.202910in}}%
\pgfpathcurveto{\pgfqpoint{2.724459in}{1.210723in}}{\pgfqpoint{2.728849in}{1.221322in}}{\pgfqpoint{2.728849in}{1.232372in}}%
\pgfpathcurveto{\pgfqpoint{2.728849in}{1.243423in}}{\pgfqpoint{2.724459in}{1.254022in}}{\pgfqpoint{2.716645in}{1.261835in}}%
\pgfpathcurveto{\pgfqpoint{2.708832in}{1.269649in}}{\pgfqpoint{2.698233in}{1.274039in}}{\pgfqpoint{2.687182in}{1.274039in}}%
\pgfpathcurveto{\pgfqpoint{2.676132in}{1.274039in}}{\pgfqpoint{2.665533in}{1.269649in}}{\pgfqpoint{2.657720in}{1.261835in}}%
\pgfpathcurveto{\pgfqpoint{2.649906in}{1.254022in}}{\pgfqpoint{2.645516in}{1.243423in}}{\pgfqpoint{2.645516in}{1.232372in}}%
\pgfpathcurveto{\pgfqpoint{2.645516in}{1.221322in}}{\pgfqpoint{2.649906in}{1.210723in}}{\pgfqpoint{2.657720in}{1.202910in}}%
\pgfpathcurveto{\pgfqpoint{2.665533in}{1.195096in}}{\pgfqpoint{2.676132in}{1.190706in}}{\pgfqpoint{2.687182in}{1.190706in}}%
\pgfpathlineto{\pgfqpoint{2.687182in}{1.190706in}}%
\pgfpathclose%
\pgfusepath{stroke}%
\end{pgfscope}%
\begin{pgfscope}%
\pgfpathrectangle{\pgfqpoint{0.847223in}{0.554012in}}{\pgfqpoint{6.200000in}{4.530000in}}%
\pgfusepath{clip}%
\pgfsetbuttcap%
\pgfsetroundjoin%
\pgfsetlinewidth{1.003750pt}%
\definecolor{currentstroke}{rgb}{1.000000,0.000000,0.000000}%
\pgfsetstrokecolor{currentstroke}%
\pgfsetdash{}{0pt}%
\pgfpathmoveto{\pgfqpoint{2.692516in}{1.188239in}}%
\pgfpathcurveto{\pgfqpoint{2.703566in}{1.188239in}}{\pgfqpoint{2.714165in}{1.192629in}}{\pgfqpoint{2.721978in}{1.200443in}}%
\pgfpathcurveto{\pgfqpoint{2.729792in}{1.208257in}}{\pgfqpoint{2.734182in}{1.218856in}}{\pgfqpoint{2.734182in}{1.229906in}}%
\pgfpathcurveto{\pgfqpoint{2.734182in}{1.240956in}}{\pgfqpoint{2.729792in}{1.251555in}}{\pgfqpoint{2.721978in}{1.259369in}}%
\pgfpathcurveto{\pgfqpoint{2.714165in}{1.267182in}}{\pgfqpoint{2.703566in}{1.271572in}}{\pgfqpoint{2.692516in}{1.271572in}}%
\pgfpathcurveto{\pgfqpoint{2.681466in}{1.271572in}}{\pgfqpoint{2.670867in}{1.267182in}}{\pgfqpoint{2.663053in}{1.259369in}}%
\pgfpathcurveto{\pgfqpoint{2.655239in}{1.251555in}}{\pgfqpoint{2.650849in}{1.240956in}}{\pgfqpoint{2.650849in}{1.229906in}}%
\pgfpathcurveto{\pgfqpoint{2.650849in}{1.218856in}}{\pgfqpoint{2.655239in}{1.208257in}}{\pgfqpoint{2.663053in}{1.200443in}}%
\pgfpathcurveto{\pgfqpoint{2.670867in}{1.192629in}}{\pgfqpoint{2.681466in}{1.188239in}}{\pgfqpoint{2.692516in}{1.188239in}}%
\pgfpathlineto{\pgfqpoint{2.692516in}{1.188239in}}%
\pgfpathclose%
\pgfusepath{stroke}%
\end{pgfscope}%
\begin{pgfscope}%
\pgfpathrectangle{\pgfqpoint{0.847223in}{0.554012in}}{\pgfqpoint{6.200000in}{4.530000in}}%
\pgfusepath{clip}%
\pgfsetbuttcap%
\pgfsetroundjoin%
\pgfsetlinewidth{1.003750pt}%
\definecolor{currentstroke}{rgb}{1.000000,0.000000,0.000000}%
\pgfsetstrokecolor{currentstroke}%
\pgfsetdash{}{0pt}%
\pgfpathmoveto{\pgfqpoint{2.697849in}{1.185783in}}%
\pgfpathcurveto{\pgfqpoint{2.708899in}{1.185783in}}{\pgfqpoint{2.719498in}{1.190174in}}{\pgfqpoint{2.727312in}{1.197987in}}%
\pgfpathcurveto{\pgfqpoint{2.735125in}{1.205801in}}{\pgfqpoint{2.739516in}{1.216400in}}{\pgfqpoint{2.739516in}{1.227450in}}%
\pgfpathcurveto{\pgfqpoint{2.739516in}{1.238500in}}{\pgfqpoint{2.735125in}{1.249099in}}{\pgfqpoint{2.727312in}{1.256913in}}%
\pgfpathcurveto{\pgfqpoint{2.719498in}{1.264727in}}{\pgfqpoint{2.708899in}{1.269117in}}{\pgfqpoint{2.697849in}{1.269117in}}%
\pgfpathcurveto{\pgfqpoint{2.686799in}{1.269117in}}{\pgfqpoint{2.676200in}{1.264727in}}{\pgfqpoint{2.668386in}{1.256913in}}%
\pgfpathcurveto{\pgfqpoint{2.660572in}{1.249099in}}{\pgfqpoint{2.656182in}{1.238500in}}{\pgfqpoint{2.656182in}{1.227450in}}%
\pgfpathcurveto{\pgfqpoint{2.656182in}{1.216400in}}{\pgfqpoint{2.660572in}{1.205801in}}{\pgfqpoint{2.668386in}{1.197987in}}%
\pgfpathcurveto{\pgfqpoint{2.676200in}{1.190174in}}{\pgfqpoint{2.686799in}{1.185783in}}{\pgfqpoint{2.697849in}{1.185783in}}%
\pgfpathlineto{\pgfqpoint{2.697849in}{1.185783in}}%
\pgfpathclose%
\pgfusepath{stroke}%
\end{pgfscope}%
\begin{pgfscope}%
\pgfpathrectangle{\pgfqpoint{0.847223in}{0.554012in}}{\pgfqpoint{6.200000in}{4.530000in}}%
\pgfusepath{clip}%
\pgfsetbuttcap%
\pgfsetroundjoin%
\pgfsetlinewidth{1.003750pt}%
\definecolor{currentstroke}{rgb}{1.000000,0.000000,0.000000}%
\pgfsetstrokecolor{currentstroke}%
\pgfsetdash{}{0pt}%
\pgfpathmoveto{\pgfqpoint{2.703182in}{1.183339in}}%
\pgfpathcurveto{\pgfqpoint{2.714232in}{1.183339in}}{\pgfqpoint{2.724831in}{1.187729in}}{\pgfqpoint{2.732645in}{1.195543in}}%
\pgfpathcurveto{\pgfqpoint{2.740459in}{1.203356in}}{\pgfqpoint{2.744849in}{1.213955in}}{\pgfqpoint{2.744849in}{1.225005in}}%
\pgfpathcurveto{\pgfqpoint{2.744849in}{1.236056in}}{\pgfqpoint{2.740459in}{1.246655in}}{\pgfqpoint{2.732645in}{1.254468in}}%
\pgfpathcurveto{\pgfqpoint{2.724831in}{1.262282in}}{\pgfqpoint{2.714232in}{1.266672in}}{\pgfqpoint{2.703182in}{1.266672in}}%
\pgfpathcurveto{\pgfqpoint{2.692132in}{1.266672in}}{\pgfqpoint{2.681533in}{1.262282in}}{\pgfqpoint{2.673719in}{1.254468in}}%
\pgfpathcurveto{\pgfqpoint{2.665906in}{1.246655in}}{\pgfqpoint{2.661515in}{1.236056in}}{\pgfqpoint{2.661515in}{1.225005in}}%
\pgfpathcurveto{\pgfqpoint{2.661515in}{1.213955in}}{\pgfqpoint{2.665906in}{1.203356in}}{\pgfqpoint{2.673719in}{1.195543in}}%
\pgfpathcurveto{\pgfqpoint{2.681533in}{1.187729in}}{\pgfqpoint{2.692132in}{1.183339in}}{\pgfqpoint{2.703182in}{1.183339in}}%
\pgfpathlineto{\pgfqpoint{2.703182in}{1.183339in}}%
\pgfpathclose%
\pgfusepath{stroke}%
\end{pgfscope}%
\begin{pgfscope}%
\pgfpathrectangle{\pgfqpoint{0.847223in}{0.554012in}}{\pgfqpoint{6.200000in}{4.530000in}}%
\pgfusepath{clip}%
\pgfsetbuttcap%
\pgfsetroundjoin%
\pgfsetlinewidth{1.003750pt}%
\definecolor{currentstroke}{rgb}{1.000000,0.000000,0.000000}%
\pgfsetstrokecolor{currentstroke}%
\pgfsetdash{}{0pt}%
\pgfpathmoveto{\pgfqpoint{2.708515in}{1.180905in}}%
\pgfpathcurveto{\pgfqpoint{2.719565in}{1.180905in}}{\pgfqpoint{2.730164in}{1.185295in}}{\pgfqpoint{2.737978in}{1.193109in}}%
\pgfpathcurveto{\pgfqpoint{2.745792in}{1.200923in}}{\pgfqpoint{2.750182in}{1.211522in}}{\pgfqpoint{2.750182in}{1.222572in}}%
\pgfpathcurveto{\pgfqpoint{2.750182in}{1.233622in}}{\pgfqpoint{2.745792in}{1.244221in}}{\pgfqpoint{2.737978in}{1.252035in}}%
\pgfpathcurveto{\pgfqpoint{2.730164in}{1.259848in}}{\pgfqpoint{2.719565in}{1.264238in}}{\pgfqpoint{2.708515in}{1.264238in}}%
\pgfpathcurveto{\pgfqpoint{2.697465in}{1.264238in}}{\pgfqpoint{2.686866in}{1.259848in}}{\pgfqpoint{2.679053in}{1.252035in}}%
\pgfpathcurveto{\pgfqpoint{2.671239in}{1.244221in}}{\pgfqpoint{2.666849in}{1.233622in}}{\pgfqpoint{2.666849in}{1.222572in}}%
\pgfpathcurveto{\pgfqpoint{2.666849in}{1.211522in}}{\pgfqpoint{2.671239in}{1.200923in}}{\pgfqpoint{2.679053in}{1.193109in}}%
\pgfpathcurveto{\pgfqpoint{2.686866in}{1.185295in}}{\pgfqpoint{2.697465in}{1.180905in}}{\pgfqpoint{2.708515in}{1.180905in}}%
\pgfpathlineto{\pgfqpoint{2.708515in}{1.180905in}}%
\pgfpathclose%
\pgfusepath{stroke}%
\end{pgfscope}%
\begin{pgfscope}%
\pgfpathrectangle{\pgfqpoint{0.847223in}{0.554012in}}{\pgfqpoint{6.200000in}{4.530000in}}%
\pgfusepath{clip}%
\pgfsetbuttcap%
\pgfsetroundjoin%
\pgfsetlinewidth{1.003750pt}%
\definecolor{currentstroke}{rgb}{1.000000,0.000000,0.000000}%
\pgfsetstrokecolor{currentstroke}%
\pgfsetdash{}{0pt}%
\pgfpathmoveto{\pgfqpoint{2.713849in}{1.178482in}}%
\pgfpathcurveto{\pgfqpoint{2.724899in}{1.178482in}}{\pgfqpoint{2.735498in}{1.182873in}}{\pgfqpoint{2.743311in}{1.190686in}}%
\pgfpathcurveto{\pgfqpoint{2.751125in}{1.198500in}}{\pgfqpoint{2.755515in}{1.209099in}}{\pgfqpoint{2.755515in}{1.220149in}}%
\pgfpathcurveto{\pgfqpoint{2.755515in}{1.231199in}}{\pgfqpoint{2.751125in}{1.241798in}}{\pgfqpoint{2.743311in}{1.249612in}}%
\pgfpathcurveto{\pgfqpoint{2.735498in}{1.257425in}}{\pgfqpoint{2.724899in}{1.261816in}}{\pgfqpoint{2.713849in}{1.261816in}}%
\pgfpathcurveto{\pgfqpoint{2.702798in}{1.261816in}}{\pgfqpoint{2.692199in}{1.257425in}}{\pgfqpoint{2.684386in}{1.249612in}}%
\pgfpathcurveto{\pgfqpoint{2.676572in}{1.241798in}}{\pgfqpoint{2.672182in}{1.231199in}}{\pgfqpoint{2.672182in}{1.220149in}}%
\pgfpathcurveto{\pgfqpoint{2.672182in}{1.209099in}}{\pgfqpoint{2.676572in}{1.198500in}}{\pgfqpoint{2.684386in}{1.190686in}}%
\pgfpathcurveto{\pgfqpoint{2.692199in}{1.182873in}}{\pgfqpoint{2.702798in}{1.178482in}}{\pgfqpoint{2.713849in}{1.178482in}}%
\pgfpathlineto{\pgfqpoint{2.713849in}{1.178482in}}%
\pgfpathclose%
\pgfusepath{stroke}%
\end{pgfscope}%
\begin{pgfscope}%
\pgfpathrectangle{\pgfqpoint{0.847223in}{0.554012in}}{\pgfqpoint{6.200000in}{4.530000in}}%
\pgfusepath{clip}%
\pgfsetbuttcap%
\pgfsetroundjoin%
\pgfsetlinewidth{1.003750pt}%
\definecolor{currentstroke}{rgb}{1.000000,0.000000,0.000000}%
\pgfsetstrokecolor{currentstroke}%
\pgfsetdash{}{0pt}%
\pgfpathmoveto{\pgfqpoint{2.719182in}{1.176070in}}%
\pgfpathcurveto{\pgfqpoint{2.730232in}{1.176070in}}{\pgfqpoint{2.740831in}{1.180461in}}{\pgfqpoint{2.748645in}{1.188274in}}%
\pgfpathcurveto{\pgfqpoint{2.756458in}{1.196088in}}{\pgfqpoint{2.760848in}{1.206687in}}{\pgfqpoint{2.760848in}{1.217737in}}%
\pgfpathcurveto{\pgfqpoint{2.760848in}{1.228787in}}{\pgfqpoint{2.756458in}{1.239386in}}{\pgfqpoint{2.748645in}{1.247200in}}%
\pgfpathcurveto{\pgfqpoint{2.740831in}{1.255013in}}{\pgfqpoint{2.730232in}{1.259404in}}{\pgfqpoint{2.719182in}{1.259404in}}%
\pgfpathcurveto{\pgfqpoint{2.708132in}{1.259404in}}{\pgfqpoint{2.697533in}{1.255013in}}{\pgfqpoint{2.689719in}{1.247200in}}%
\pgfpathcurveto{\pgfqpoint{2.681905in}{1.239386in}}{\pgfqpoint{2.677515in}{1.228787in}}{\pgfqpoint{2.677515in}{1.217737in}}%
\pgfpathcurveto{\pgfqpoint{2.677515in}{1.206687in}}{\pgfqpoint{2.681905in}{1.196088in}}{\pgfqpoint{2.689719in}{1.188274in}}%
\pgfpathcurveto{\pgfqpoint{2.697533in}{1.180461in}}{\pgfqpoint{2.708132in}{1.176070in}}{\pgfqpoint{2.719182in}{1.176070in}}%
\pgfpathlineto{\pgfqpoint{2.719182in}{1.176070in}}%
\pgfpathclose%
\pgfusepath{stroke}%
\end{pgfscope}%
\begin{pgfscope}%
\pgfpathrectangle{\pgfqpoint{0.847223in}{0.554012in}}{\pgfqpoint{6.200000in}{4.530000in}}%
\pgfusepath{clip}%
\pgfsetbuttcap%
\pgfsetroundjoin%
\pgfsetlinewidth{1.003750pt}%
\definecolor{currentstroke}{rgb}{1.000000,0.000000,0.000000}%
\pgfsetstrokecolor{currentstroke}%
\pgfsetdash{}{0pt}%
\pgfpathmoveto{\pgfqpoint{2.724515in}{1.173669in}}%
\pgfpathcurveto{\pgfqpoint{2.735565in}{1.173669in}}{\pgfqpoint{2.746164in}{1.178059in}}{\pgfqpoint{2.753978in}{1.185873in}}%
\pgfpathcurveto{\pgfqpoint{2.761791in}{1.193686in}}{\pgfqpoint{2.766182in}{1.204285in}}{\pgfqpoint{2.766182in}{1.215336in}}%
\pgfpathcurveto{\pgfqpoint{2.766182in}{1.226386in}}{\pgfqpoint{2.761791in}{1.236985in}}{\pgfqpoint{2.753978in}{1.244798in}}%
\pgfpathcurveto{\pgfqpoint{2.746164in}{1.252612in}}{\pgfqpoint{2.735565in}{1.257002in}}{\pgfqpoint{2.724515in}{1.257002in}}%
\pgfpathcurveto{\pgfqpoint{2.713465in}{1.257002in}}{\pgfqpoint{2.702866in}{1.252612in}}{\pgfqpoint{2.695052in}{1.244798in}}%
\pgfpathcurveto{\pgfqpoint{2.687239in}{1.236985in}}{\pgfqpoint{2.682848in}{1.226386in}}{\pgfqpoint{2.682848in}{1.215336in}}%
\pgfpathcurveto{\pgfqpoint{2.682848in}{1.204285in}}{\pgfqpoint{2.687239in}{1.193686in}}{\pgfqpoint{2.695052in}{1.185873in}}%
\pgfpathcurveto{\pgfqpoint{2.702866in}{1.178059in}}{\pgfqpoint{2.713465in}{1.173669in}}{\pgfqpoint{2.724515in}{1.173669in}}%
\pgfpathlineto{\pgfqpoint{2.724515in}{1.173669in}}%
\pgfpathclose%
\pgfusepath{stroke}%
\end{pgfscope}%
\begin{pgfscope}%
\pgfpathrectangle{\pgfqpoint{0.847223in}{0.554012in}}{\pgfqpoint{6.200000in}{4.530000in}}%
\pgfusepath{clip}%
\pgfsetbuttcap%
\pgfsetroundjoin%
\pgfsetlinewidth{1.003750pt}%
\definecolor{currentstroke}{rgb}{1.000000,0.000000,0.000000}%
\pgfsetstrokecolor{currentstroke}%
\pgfsetdash{}{0pt}%
\pgfpathmoveto{\pgfqpoint{2.729848in}{1.171278in}}%
\pgfpathcurveto{\pgfqpoint{2.740898in}{1.171278in}}{\pgfqpoint{2.751497in}{1.175669in}}{\pgfqpoint{2.759311in}{1.183482in}}%
\pgfpathcurveto{\pgfqpoint{2.767125in}{1.191296in}}{\pgfqpoint{2.771515in}{1.201895in}}{\pgfqpoint{2.771515in}{1.212945in}}%
\pgfpathcurveto{\pgfqpoint{2.771515in}{1.223995in}}{\pgfqpoint{2.767125in}{1.234594in}}{\pgfqpoint{2.759311in}{1.242408in}}%
\pgfpathcurveto{\pgfqpoint{2.751497in}{1.250221in}}{\pgfqpoint{2.740898in}{1.254612in}}{\pgfqpoint{2.729848in}{1.254612in}}%
\pgfpathcurveto{\pgfqpoint{2.718798in}{1.254612in}}{\pgfqpoint{2.708199in}{1.250221in}}{\pgfqpoint{2.700385in}{1.242408in}}%
\pgfpathcurveto{\pgfqpoint{2.692572in}{1.234594in}}{\pgfqpoint{2.688182in}{1.223995in}}{\pgfqpoint{2.688182in}{1.212945in}}%
\pgfpathcurveto{\pgfqpoint{2.688182in}{1.201895in}}{\pgfqpoint{2.692572in}{1.191296in}}{\pgfqpoint{2.700385in}{1.183482in}}%
\pgfpathcurveto{\pgfqpoint{2.708199in}{1.175669in}}{\pgfqpoint{2.718798in}{1.171278in}}{\pgfqpoint{2.729848in}{1.171278in}}%
\pgfpathlineto{\pgfqpoint{2.729848in}{1.171278in}}%
\pgfpathclose%
\pgfusepath{stroke}%
\end{pgfscope}%
\begin{pgfscope}%
\pgfpathrectangle{\pgfqpoint{0.847223in}{0.554012in}}{\pgfqpoint{6.200000in}{4.530000in}}%
\pgfusepath{clip}%
\pgfsetbuttcap%
\pgfsetroundjoin%
\pgfsetlinewidth{1.003750pt}%
\definecolor{currentstroke}{rgb}{1.000000,0.000000,0.000000}%
\pgfsetstrokecolor{currentstroke}%
\pgfsetdash{}{0pt}%
\pgfpathmoveto{\pgfqpoint{2.735181in}{1.168898in}}%
\pgfpathcurveto{\pgfqpoint{2.746232in}{1.168898in}}{\pgfqpoint{2.756831in}{1.173288in}}{\pgfqpoint{2.764644in}{1.181102in}}%
\pgfpathcurveto{\pgfqpoint{2.772458in}{1.188916in}}{\pgfqpoint{2.776848in}{1.199515in}}{\pgfqpoint{2.776848in}{1.210565in}}%
\pgfpathcurveto{\pgfqpoint{2.776848in}{1.221615in}}{\pgfqpoint{2.772458in}{1.232214in}}{\pgfqpoint{2.764644in}{1.240028in}}%
\pgfpathcurveto{\pgfqpoint{2.756831in}{1.247841in}}{\pgfqpoint{2.746232in}{1.252231in}}{\pgfqpoint{2.735181in}{1.252231in}}%
\pgfpathcurveto{\pgfqpoint{2.724131in}{1.252231in}}{\pgfqpoint{2.713532in}{1.247841in}}{\pgfqpoint{2.705719in}{1.240028in}}%
\pgfpathcurveto{\pgfqpoint{2.697905in}{1.232214in}}{\pgfqpoint{2.693515in}{1.221615in}}{\pgfqpoint{2.693515in}{1.210565in}}%
\pgfpathcurveto{\pgfqpoint{2.693515in}{1.199515in}}{\pgfqpoint{2.697905in}{1.188916in}}{\pgfqpoint{2.705719in}{1.181102in}}%
\pgfpathcurveto{\pgfqpoint{2.713532in}{1.173288in}}{\pgfqpoint{2.724131in}{1.168898in}}{\pgfqpoint{2.735181in}{1.168898in}}%
\pgfpathlineto{\pgfqpoint{2.735181in}{1.168898in}}%
\pgfpathclose%
\pgfusepath{stroke}%
\end{pgfscope}%
\begin{pgfscope}%
\pgfpathrectangle{\pgfqpoint{0.847223in}{0.554012in}}{\pgfqpoint{6.200000in}{4.530000in}}%
\pgfusepath{clip}%
\pgfsetbuttcap%
\pgfsetroundjoin%
\pgfsetlinewidth{1.003750pt}%
\definecolor{currentstroke}{rgb}{1.000000,0.000000,0.000000}%
\pgfsetstrokecolor{currentstroke}%
\pgfsetdash{}{0pt}%
\pgfpathmoveto{\pgfqpoint{2.740515in}{1.166529in}}%
\pgfpathcurveto{\pgfqpoint{2.751565in}{1.166529in}}{\pgfqpoint{2.762164in}{1.170919in}}{\pgfqpoint{2.769977in}{1.178732in}}%
\pgfpathcurveto{\pgfqpoint{2.777791in}{1.186546in}}{\pgfqpoint{2.782181in}{1.197145in}}{\pgfqpoint{2.782181in}{1.208195in}}%
\pgfpathcurveto{\pgfqpoint{2.782181in}{1.219245in}}{\pgfqpoint{2.777791in}{1.229844in}}{\pgfqpoint{2.769977in}{1.237658in}}%
\pgfpathcurveto{\pgfqpoint{2.762164in}{1.245472in}}{\pgfqpoint{2.751565in}{1.249862in}}{\pgfqpoint{2.740515in}{1.249862in}}%
\pgfpathcurveto{\pgfqpoint{2.729464in}{1.249862in}}{\pgfqpoint{2.718865in}{1.245472in}}{\pgfqpoint{2.711052in}{1.237658in}}%
\pgfpathcurveto{\pgfqpoint{2.703238in}{1.229844in}}{\pgfqpoint{2.698848in}{1.219245in}}{\pgfqpoint{2.698848in}{1.208195in}}%
\pgfpathcurveto{\pgfqpoint{2.698848in}{1.197145in}}{\pgfqpoint{2.703238in}{1.186546in}}{\pgfqpoint{2.711052in}{1.178732in}}%
\pgfpathcurveto{\pgfqpoint{2.718865in}{1.170919in}}{\pgfqpoint{2.729464in}{1.166529in}}{\pgfqpoint{2.740515in}{1.166529in}}%
\pgfpathlineto{\pgfqpoint{2.740515in}{1.166529in}}%
\pgfpathclose%
\pgfusepath{stroke}%
\end{pgfscope}%
\begin{pgfscope}%
\pgfpathrectangle{\pgfqpoint{0.847223in}{0.554012in}}{\pgfqpoint{6.200000in}{4.530000in}}%
\pgfusepath{clip}%
\pgfsetbuttcap%
\pgfsetroundjoin%
\pgfsetlinewidth{1.003750pt}%
\definecolor{currentstroke}{rgb}{1.000000,0.000000,0.000000}%
\pgfsetstrokecolor{currentstroke}%
\pgfsetdash{}{0pt}%
\pgfpathmoveto{\pgfqpoint{2.745848in}{1.164169in}}%
\pgfpathcurveto{\pgfqpoint{2.756898in}{1.164169in}}{\pgfqpoint{2.767497in}{1.168560in}}{\pgfqpoint{2.775311in}{1.176373in}}%
\pgfpathcurveto{\pgfqpoint{2.783124in}{1.184187in}}{\pgfqpoint{2.787514in}{1.194786in}}{\pgfqpoint{2.787514in}{1.205836in}}%
\pgfpathcurveto{\pgfqpoint{2.787514in}{1.216886in}}{\pgfqpoint{2.783124in}{1.227485in}}{\pgfqpoint{2.775311in}{1.235299in}}%
\pgfpathcurveto{\pgfqpoint{2.767497in}{1.243112in}}{\pgfqpoint{2.756898in}{1.247503in}}{\pgfqpoint{2.745848in}{1.247503in}}%
\pgfpathcurveto{\pgfqpoint{2.734798in}{1.247503in}}{\pgfqpoint{2.724199in}{1.243112in}}{\pgfqpoint{2.716385in}{1.235299in}}%
\pgfpathcurveto{\pgfqpoint{2.708571in}{1.227485in}}{\pgfqpoint{2.704181in}{1.216886in}}{\pgfqpoint{2.704181in}{1.205836in}}%
\pgfpathcurveto{\pgfqpoint{2.704181in}{1.194786in}}{\pgfqpoint{2.708571in}{1.184187in}}{\pgfqpoint{2.716385in}{1.176373in}}%
\pgfpathcurveto{\pgfqpoint{2.724199in}{1.168560in}}{\pgfqpoint{2.734798in}{1.164169in}}{\pgfqpoint{2.745848in}{1.164169in}}%
\pgfpathlineto{\pgfqpoint{2.745848in}{1.164169in}}%
\pgfpathclose%
\pgfusepath{stroke}%
\end{pgfscope}%
\begin{pgfscope}%
\pgfpathrectangle{\pgfqpoint{0.847223in}{0.554012in}}{\pgfqpoint{6.200000in}{4.530000in}}%
\pgfusepath{clip}%
\pgfsetbuttcap%
\pgfsetroundjoin%
\pgfsetlinewidth{1.003750pt}%
\definecolor{currentstroke}{rgb}{1.000000,0.000000,0.000000}%
\pgfsetstrokecolor{currentstroke}%
\pgfsetdash{}{0pt}%
\pgfpathmoveto{\pgfqpoint{2.751181in}{1.161821in}}%
\pgfpathcurveto{\pgfqpoint{2.762231in}{1.161821in}}{\pgfqpoint{2.772830in}{1.166211in}}{\pgfqpoint{2.780644in}{1.174024in}}%
\pgfpathcurveto{\pgfqpoint{2.788457in}{1.181838in}}{\pgfqpoint{2.792848in}{1.192437in}}{\pgfqpoint{2.792848in}{1.203487in}}%
\pgfpathcurveto{\pgfqpoint{2.792848in}{1.214537in}}{\pgfqpoint{2.788457in}{1.225136in}}{\pgfqpoint{2.780644in}{1.232950in}}%
\pgfpathcurveto{\pgfqpoint{2.772830in}{1.240764in}}{\pgfqpoint{2.762231in}{1.245154in}}{\pgfqpoint{2.751181in}{1.245154in}}%
\pgfpathcurveto{\pgfqpoint{2.740131in}{1.245154in}}{\pgfqpoint{2.729532in}{1.240764in}}{\pgfqpoint{2.721718in}{1.232950in}}%
\pgfpathcurveto{\pgfqpoint{2.713905in}{1.225136in}}{\pgfqpoint{2.709514in}{1.214537in}}{\pgfqpoint{2.709514in}{1.203487in}}%
\pgfpathcurveto{\pgfqpoint{2.709514in}{1.192437in}}{\pgfqpoint{2.713905in}{1.181838in}}{\pgfqpoint{2.721718in}{1.174024in}}%
\pgfpathcurveto{\pgfqpoint{2.729532in}{1.166211in}}{\pgfqpoint{2.740131in}{1.161821in}}{\pgfqpoint{2.751181in}{1.161821in}}%
\pgfpathlineto{\pgfqpoint{2.751181in}{1.161821in}}%
\pgfpathclose%
\pgfusepath{stroke}%
\end{pgfscope}%
\begin{pgfscope}%
\pgfpathrectangle{\pgfqpoint{0.847223in}{0.554012in}}{\pgfqpoint{6.200000in}{4.530000in}}%
\pgfusepath{clip}%
\pgfsetbuttcap%
\pgfsetroundjoin%
\pgfsetlinewidth{1.003750pt}%
\definecolor{currentstroke}{rgb}{1.000000,0.000000,0.000000}%
\pgfsetstrokecolor{currentstroke}%
\pgfsetdash{}{0pt}%
\pgfpathmoveto{\pgfqpoint{2.756514in}{1.159482in}}%
\pgfpathcurveto{\pgfqpoint{2.767564in}{1.159482in}}{\pgfqpoint{2.778163in}{1.163872in}}{\pgfqpoint{2.785977in}{1.171686in}}%
\pgfpathcurveto{\pgfqpoint{2.793791in}{1.179500in}}{\pgfqpoint{2.798181in}{1.190099in}}{\pgfqpoint{2.798181in}{1.201149in}}%
\pgfpathcurveto{\pgfqpoint{2.798181in}{1.212199in}}{\pgfqpoint{2.793791in}{1.222798in}}{\pgfqpoint{2.785977in}{1.230612in}}%
\pgfpathcurveto{\pgfqpoint{2.778163in}{1.238425in}}{\pgfqpoint{2.767564in}{1.242815in}}{\pgfqpoint{2.756514in}{1.242815in}}%
\pgfpathcurveto{\pgfqpoint{2.745464in}{1.242815in}}{\pgfqpoint{2.734865in}{1.238425in}}{\pgfqpoint{2.727051in}{1.230612in}}%
\pgfpathcurveto{\pgfqpoint{2.719238in}{1.222798in}}{\pgfqpoint{2.714848in}{1.212199in}}{\pgfqpoint{2.714848in}{1.201149in}}%
\pgfpathcurveto{\pgfqpoint{2.714848in}{1.190099in}}{\pgfqpoint{2.719238in}{1.179500in}}{\pgfqpoint{2.727051in}{1.171686in}}%
\pgfpathcurveto{\pgfqpoint{2.734865in}{1.163872in}}{\pgfqpoint{2.745464in}{1.159482in}}{\pgfqpoint{2.756514in}{1.159482in}}%
\pgfpathlineto{\pgfqpoint{2.756514in}{1.159482in}}%
\pgfpathclose%
\pgfusepath{stroke}%
\end{pgfscope}%
\begin{pgfscope}%
\pgfpathrectangle{\pgfqpoint{0.847223in}{0.554012in}}{\pgfqpoint{6.200000in}{4.530000in}}%
\pgfusepath{clip}%
\pgfsetbuttcap%
\pgfsetroundjoin%
\pgfsetlinewidth{1.003750pt}%
\definecolor{currentstroke}{rgb}{1.000000,0.000000,0.000000}%
\pgfsetstrokecolor{currentstroke}%
\pgfsetdash{}{0pt}%
\pgfpathmoveto{\pgfqpoint{2.761847in}{1.157154in}}%
\pgfpathcurveto{\pgfqpoint{2.772898in}{1.157154in}}{\pgfqpoint{2.783497in}{1.161544in}}{\pgfqpoint{2.791310in}{1.169358in}}%
\pgfpathcurveto{\pgfqpoint{2.799124in}{1.177171in}}{\pgfqpoint{2.803514in}{1.187770in}}{\pgfqpoint{2.803514in}{1.198821in}}%
\pgfpathcurveto{\pgfqpoint{2.803514in}{1.209871in}}{\pgfqpoint{2.799124in}{1.220470in}}{\pgfqpoint{2.791310in}{1.228283in}}%
\pgfpathcurveto{\pgfqpoint{2.783497in}{1.236097in}}{\pgfqpoint{2.772898in}{1.240487in}}{\pgfqpoint{2.761847in}{1.240487in}}%
\pgfpathcurveto{\pgfqpoint{2.750797in}{1.240487in}}{\pgfqpoint{2.740198in}{1.236097in}}{\pgfqpoint{2.732385in}{1.228283in}}%
\pgfpathcurveto{\pgfqpoint{2.724571in}{1.220470in}}{\pgfqpoint{2.720181in}{1.209871in}}{\pgfqpoint{2.720181in}{1.198821in}}%
\pgfpathcurveto{\pgfqpoint{2.720181in}{1.187770in}}{\pgfqpoint{2.724571in}{1.177171in}}{\pgfqpoint{2.732385in}{1.169358in}}%
\pgfpathcurveto{\pgfqpoint{2.740198in}{1.161544in}}{\pgfqpoint{2.750797in}{1.157154in}}{\pgfqpoint{2.761847in}{1.157154in}}%
\pgfpathlineto{\pgfqpoint{2.761847in}{1.157154in}}%
\pgfpathclose%
\pgfusepath{stroke}%
\end{pgfscope}%
\begin{pgfscope}%
\pgfpathrectangle{\pgfqpoint{0.847223in}{0.554012in}}{\pgfqpoint{6.200000in}{4.530000in}}%
\pgfusepath{clip}%
\pgfsetbuttcap%
\pgfsetroundjoin%
\pgfsetlinewidth{1.003750pt}%
\definecolor{currentstroke}{rgb}{1.000000,0.000000,0.000000}%
\pgfsetstrokecolor{currentstroke}%
\pgfsetdash{}{0pt}%
\pgfpathmoveto{\pgfqpoint{2.767181in}{1.154836in}}%
\pgfpathcurveto{\pgfqpoint{2.778231in}{1.154836in}}{\pgfqpoint{2.788830in}{1.159226in}}{\pgfqpoint{2.796643in}{1.167040in}}%
\pgfpathcurveto{\pgfqpoint{2.804457in}{1.174853in}}{\pgfqpoint{2.808847in}{1.185452in}}{\pgfqpoint{2.808847in}{1.196502in}}%
\pgfpathcurveto{\pgfqpoint{2.808847in}{1.207553in}}{\pgfqpoint{2.804457in}{1.218152in}}{\pgfqpoint{2.796643in}{1.225965in}}%
\pgfpathcurveto{\pgfqpoint{2.788830in}{1.233779in}}{\pgfqpoint{2.778231in}{1.238169in}}{\pgfqpoint{2.767181in}{1.238169in}}%
\pgfpathcurveto{\pgfqpoint{2.756131in}{1.238169in}}{\pgfqpoint{2.745532in}{1.233779in}}{\pgfqpoint{2.737718in}{1.225965in}}%
\pgfpathcurveto{\pgfqpoint{2.729904in}{1.218152in}}{\pgfqpoint{2.725514in}{1.207553in}}{\pgfqpoint{2.725514in}{1.196502in}}%
\pgfpathcurveto{\pgfqpoint{2.725514in}{1.185452in}}{\pgfqpoint{2.729904in}{1.174853in}}{\pgfqpoint{2.737718in}{1.167040in}}%
\pgfpathcurveto{\pgfqpoint{2.745532in}{1.159226in}}{\pgfqpoint{2.756131in}{1.154836in}}{\pgfqpoint{2.767181in}{1.154836in}}%
\pgfpathlineto{\pgfqpoint{2.767181in}{1.154836in}}%
\pgfpathclose%
\pgfusepath{stroke}%
\end{pgfscope}%
\begin{pgfscope}%
\pgfpathrectangle{\pgfqpoint{0.847223in}{0.554012in}}{\pgfqpoint{6.200000in}{4.530000in}}%
\pgfusepath{clip}%
\pgfsetbuttcap%
\pgfsetroundjoin%
\pgfsetlinewidth{1.003750pt}%
\definecolor{currentstroke}{rgb}{1.000000,0.000000,0.000000}%
\pgfsetstrokecolor{currentstroke}%
\pgfsetdash{}{0pt}%
\pgfpathmoveto{\pgfqpoint{2.772514in}{1.152528in}}%
\pgfpathcurveto{\pgfqpoint{2.783564in}{1.152528in}}{\pgfqpoint{2.794163in}{1.156918in}}{\pgfqpoint{2.801977in}{1.164732in}}%
\pgfpathcurveto{\pgfqpoint{2.809790in}{1.172545in}}{\pgfqpoint{2.814181in}{1.183144in}}{\pgfqpoint{2.814181in}{1.194194in}}%
\pgfpathcurveto{\pgfqpoint{2.814181in}{1.205245in}}{\pgfqpoint{2.809790in}{1.215844in}}{\pgfqpoint{2.801977in}{1.223657in}}%
\pgfpathcurveto{\pgfqpoint{2.794163in}{1.231471in}}{\pgfqpoint{2.783564in}{1.235861in}}{\pgfqpoint{2.772514in}{1.235861in}}%
\pgfpathcurveto{\pgfqpoint{2.761464in}{1.235861in}}{\pgfqpoint{2.750865in}{1.231471in}}{\pgfqpoint{2.743051in}{1.223657in}}%
\pgfpathcurveto{\pgfqpoint{2.735238in}{1.215844in}}{\pgfqpoint{2.730847in}{1.205245in}}{\pgfqpoint{2.730847in}{1.194194in}}%
\pgfpathcurveto{\pgfqpoint{2.730847in}{1.183144in}}{\pgfqpoint{2.735238in}{1.172545in}}{\pgfqpoint{2.743051in}{1.164732in}}%
\pgfpathcurveto{\pgfqpoint{2.750865in}{1.156918in}}{\pgfqpoint{2.761464in}{1.152528in}}{\pgfqpoint{2.772514in}{1.152528in}}%
\pgfpathlineto{\pgfqpoint{2.772514in}{1.152528in}}%
\pgfpathclose%
\pgfusepath{stroke}%
\end{pgfscope}%
\begin{pgfscope}%
\pgfpathrectangle{\pgfqpoint{0.847223in}{0.554012in}}{\pgfqpoint{6.200000in}{4.530000in}}%
\pgfusepath{clip}%
\pgfsetbuttcap%
\pgfsetroundjoin%
\pgfsetlinewidth{1.003750pt}%
\definecolor{currentstroke}{rgb}{1.000000,0.000000,0.000000}%
\pgfsetstrokecolor{currentstroke}%
\pgfsetdash{}{0pt}%
\pgfpathmoveto{\pgfqpoint{2.777847in}{1.150230in}}%
\pgfpathcurveto{\pgfqpoint{2.788897in}{1.150230in}}{\pgfqpoint{2.799496in}{1.154620in}}{\pgfqpoint{2.807310in}{1.162434in}}%
\pgfpathcurveto{\pgfqpoint{2.815124in}{1.170247in}}{\pgfqpoint{2.819514in}{1.180846in}}{\pgfqpoint{2.819514in}{1.191896in}}%
\pgfpathcurveto{\pgfqpoint{2.819514in}{1.202947in}}{\pgfqpoint{2.815124in}{1.213546in}}{\pgfqpoint{2.807310in}{1.221359in}}%
\pgfpathcurveto{\pgfqpoint{2.799496in}{1.229173in}}{\pgfqpoint{2.788897in}{1.233563in}}{\pgfqpoint{2.777847in}{1.233563in}}%
\pgfpathcurveto{\pgfqpoint{2.766797in}{1.233563in}}{\pgfqpoint{2.756198in}{1.229173in}}{\pgfqpoint{2.748384in}{1.221359in}}%
\pgfpathcurveto{\pgfqpoint{2.740571in}{1.213546in}}{\pgfqpoint{2.736180in}{1.202947in}}{\pgfqpoint{2.736180in}{1.191896in}}%
\pgfpathcurveto{\pgfqpoint{2.736180in}{1.180846in}}{\pgfqpoint{2.740571in}{1.170247in}}{\pgfqpoint{2.748384in}{1.162434in}}%
\pgfpathcurveto{\pgfqpoint{2.756198in}{1.154620in}}{\pgfqpoint{2.766797in}{1.150230in}}{\pgfqpoint{2.777847in}{1.150230in}}%
\pgfpathlineto{\pgfqpoint{2.777847in}{1.150230in}}%
\pgfpathclose%
\pgfusepath{stroke}%
\end{pgfscope}%
\begin{pgfscope}%
\pgfpathrectangle{\pgfqpoint{0.847223in}{0.554012in}}{\pgfqpoint{6.200000in}{4.530000in}}%
\pgfusepath{clip}%
\pgfsetbuttcap%
\pgfsetroundjoin%
\pgfsetlinewidth{1.003750pt}%
\definecolor{currentstroke}{rgb}{1.000000,0.000000,0.000000}%
\pgfsetstrokecolor{currentstroke}%
\pgfsetdash{}{0pt}%
\pgfpathmoveto{\pgfqpoint{2.783180in}{1.147942in}}%
\pgfpathcurveto{\pgfqpoint{2.794230in}{1.147942in}}{\pgfqpoint{2.804830in}{1.152332in}}{\pgfqpoint{2.812643in}{1.160146in}}%
\pgfpathcurveto{\pgfqpoint{2.820457in}{1.167959in}}{\pgfqpoint{2.824847in}{1.178558in}}{\pgfqpoint{2.824847in}{1.189608in}}%
\pgfpathcurveto{\pgfqpoint{2.824847in}{1.200659in}}{\pgfqpoint{2.820457in}{1.211258in}}{\pgfqpoint{2.812643in}{1.219071in}}%
\pgfpathcurveto{\pgfqpoint{2.804830in}{1.226885in}}{\pgfqpoint{2.794230in}{1.231275in}}{\pgfqpoint{2.783180in}{1.231275in}}%
\pgfpathcurveto{\pgfqpoint{2.772130in}{1.231275in}}{\pgfqpoint{2.761531in}{1.226885in}}{\pgfqpoint{2.753718in}{1.219071in}}%
\pgfpathcurveto{\pgfqpoint{2.745904in}{1.211258in}}{\pgfqpoint{2.741514in}{1.200659in}}{\pgfqpoint{2.741514in}{1.189608in}}%
\pgfpathcurveto{\pgfqpoint{2.741514in}{1.178558in}}{\pgfqpoint{2.745904in}{1.167959in}}{\pgfqpoint{2.753718in}{1.160146in}}%
\pgfpathcurveto{\pgfqpoint{2.761531in}{1.152332in}}{\pgfqpoint{2.772130in}{1.147942in}}{\pgfqpoint{2.783180in}{1.147942in}}%
\pgfpathlineto{\pgfqpoint{2.783180in}{1.147942in}}%
\pgfpathclose%
\pgfusepath{stroke}%
\end{pgfscope}%
\begin{pgfscope}%
\pgfpathrectangle{\pgfqpoint{0.847223in}{0.554012in}}{\pgfqpoint{6.200000in}{4.530000in}}%
\pgfusepath{clip}%
\pgfsetbuttcap%
\pgfsetroundjoin%
\pgfsetlinewidth{1.003750pt}%
\definecolor{currentstroke}{rgb}{1.000000,0.000000,0.000000}%
\pgfsetstrokecolor{currentstroke}%
\pgfsetdash{}{0pt}%
\pgfpathmoveto{\pgfqpoint{2.788514in}{1.145664in}}%
\pgfpathcurveto{\pgfqpoint{2.799564in}{1.145664in}}{\pgfqpoint{2.810163in}{1.150054in}}{\pgfqpoint{2.817976in}{1.157868in}}%
\pgfpathcurveto{\pgfqpoint{2.825790in}{1.165681in}}{\pgfqpoint{2.830180in}{1.176280in}}{\pgfqpoint{2.830180in}{1.187330in}}%
\pgfpathcurveto{\pgfqpoint{2.830180in}{1.198380in}}{\pgfqpoint{2.825790in}{1.208979in}}{\pgfqpoint{2.817976in}{1.216793in}}%
\pgfpathcurveto{\pgfqpoint{2.810163in}{1.224607in}}{\pgfqpoint{2.799564in}{1.228997in}}{\pgfqpoint{2.788514in}{1.228997in}}%
\pgfpathcurveto{\pgfqpoint{2.777463in}{1.228997in}}{\pgfqpoint{2.766864in}{1.224607in}}{\pgfqpoint{2.759051in}{1.216793in}}%
\pgfpathcurveto{\pgfqpoint{2.751237in}{1.208979in}}{\pgfqpoint{2.746847in}{1.198380in}}{\pgfqpoint{2.746847in}{1.187330in}}%
\pgfpathcurveto{\pgfqpoint{2.746847in}{1.176280in}}{\pgfqpoint{2.751237in}{1.165681in}}{\pgfqpoint{2.759051in}{1.157868in}}%
\pgfpathcurveto{\pgfqpoint{2.766864in}{1.150054in}}{\pgfqpoint{2.777463in}{1.145664in}}{\pgfqpoint{2.788514in}{1.145664in}}%
\pgfpathlineto{\pgfqpoint{2.788514in}{1.145664in}}%
\pgfpathclose%
\pgfusepath{stroke}%
\end{pgfscope}%
\begin{pgfscope}%
\pgfpathrectangle{\pgfqpoint{0.847223in}{0.554012in}}{\pgfqpoint{6.200000in}{4.530000in}}%
\pgfusepath{clip}%
\pgfsetbuttcap%
\pgfsetroundjoin%
\pgfsetlinewidth{1.003750pt}%
\definecolor{currentstroke}{rgb}{1.000000,0.000000,0.000000}%
\pgfsetstrokecolor{currentstroke}%
\pgfsetdash{}{0pt}%
\pgfpathmoveto{\pgfqpoint{2.793847in}{1.143395in}}%
\pgfpathcurveto{\pgfqpoint{2.804897in}{1.143395in}}{\pgfqpoint{2.815496in}{1.147786in}}{\pgfqpoint{2.823310in}{1.155599in}}%
\pgfpathcurveto{\pgfqpoint{2.831123in}{1.163413in}}{\pgfqpoint{2.835513in}{1.174012in}}{\pgfqpoint{2.835513in}{1.185062in}}%
\pgfpathcurveto{\pgfqpoint{2.835513in}{1.196112in}}{\pgfqpoint{2.831123in}{1.206711in}}{\pgfqpoint{2.823310in}{1.214525in}}%
\pgfpathcurveto{\pgfqpoint{2.815496in}{1.222338in}}{\pgfqpoint{2.804897in}{1.226729in}}{\pgfqpoint{2.793847in}{1.226729in}}%
\pgfpathcurveto{\pgfqpoint{2.782797in}{1.226729in}}{\pgfqpoint{2.772198in}{1.222338in}}{\pgfqpoint{2.764384in}{1.214525in}}%
\pgfpathcurveto{\pgfqpoint{2.756570in}{1.206711in}}{\pgfqpoint{2.752180in}{1.196112in}}{\pgfqpoint{2.752180in}{1.185062in}}%
\pgfpathcurveto{\pgfqpoint{2.752180in}{1.174012in}}{\pgfqpoint{2.756570in}{1.163413in}}{\pgfqpoint{2.764384in}{1.155599in}}%
\pgfpathcurveto{\pgfqpoint{2.772198in}{1.147786in}}{\pgfqpoint{2.782797in}{1.143395in}}{\pgfqpoint{2.793847in}{1.143395in}}%
\pgfpathlineto{\pgfqpoint{2.793847in}{1.143395in}}%
\pgfpathclose%
\pgfusepath{stroke}%
\end{pgfscope}%
\begin{pgfscope}%
\pgfpathrectangle{\pgfqpoint{0.847223in}{0.554012in}}{\pgfqpoint{6.200000in}{4.530000in}}%
\pgfusepath{clip}%
\pgfsetbuttcap%
\pgfsetroundjoin%
\pgfsetlinewidth{1.003750pt}%
\definecolor{currentstroke}{rgb}{1.000000,0.000000,0.000000}%
\pgfsetstrokecolor{currentstroke}%
\pgfsetdash{}{0pt}%
\pgfpathmoveto{\pgfqpoint{2.799180in}{1.141137in}}%
\pgfpathcurveto{\pgfqpoint{2.810230in}{1.141137in}}{\pgfqpoint{2.820829in}{1.145527in}}{\pgfqpoint{2.828643in}{1.153341in}}%
\pgfpathcurveto{\pgfqpoint{2.836456in}{1.161154in}}{\pgfqpoint{2.840847in}{1.171753in}}{\pgfqpoint{2.840847in}{1.182804in}}%
\pgfpathcurveto{\pgfqpoint{2.840847in}{1.193854in}}{\pgfqpoint{2.836456in}{1.204453in}}{\pgfqpoint{2.828643in}{1.212266in}}%
\pgfpathcurveto{\pgfqpoint{2.820829in}{1.220080in}}{\pgfqpoint{2.810230in}{1.224470in}}{\pgfqpoint{2.799180in}{1.224470in}}%
\pgfpathcurveto{\pgfqpoint{2.788130in}{1.224470in}}{\pgfqpoint{2.777531in}{1.220080in}}{\pgfqpoint{2.769717in}{1.212266in}}%
\pgfpathcurveto{\pgfqpoint{2.761904in}{1.204453in}}{\pgfqpoint{2.757513in}{1.193854in}}{\pgfqpoint{2.757513in}{1.182804in}}%
\pgfpathcurveto{\pgfqpoint{2.757513in}{1.171753in}}{\pgfqpoint{2.761904in}{1.161154in}}{\pgfqpoint{2.769717in}{1.153341in}}%
\pgfpathcurveto{\pgfqpoint{2.777531in}{1.145527in}}{\pgfqpoint{2.788130in}{1.141137in}}{\pgfqpoint{2.799180in}{1.141137in}}%
\pgfpathlineto{\pgfqpoint{2.799180in}{1.141137in}}%
\pgfpathclose%
\pgfusepath{stroke}%
\end{pgfscope}%
\begin{pgfscope}%
\pgfpathrectangle{\pgfqpoint{0.847223in}{0.554012in}}{\pgfqpoint{6.200000in}{4.530000in}}%
\pgfusepath{clip}%
\pgfsetbuttcap%
\pgfsetroundjoin%
\pgfsetlinewidth{1.003750pt}%
\definecolor{currentstroke}{rgb}{1.000000,0.000000,0.000000}%
\pgfsetstrokecolor{currentstroke}%
\pgfsetdash{}{0pt}%
\pgfpathmoveto{\pgfqpoint{2.804513in}{1.138888in}}%
\pgfpathcurveto{\pgfqpoint{2.815563in}{1.138888in}}{\pgfqpoint{2.826162in}{1.143278in}}{\pgfqpoint{2.833976in}{1.151092in}}%
\pgfpathcurveto{\pgfqpoint{2.841790in}{1.158906in}}{\pgfqpoint{2.846180in}{1.169505in}}{\pgfqpoint{2.846180in}{1.180555in}}%
\pgfpathcurveto{\pgfqpoint{2.846180in}{1.191605in}}{\pgfqpoint{2.841790in}{1.202204in}}{\pgfqpoint{2.833976in}{1.210018in}}%
\pgfpathcurveto{\pgfqpoint{2.826162in}{1.217831in}}{\pgfqpoint{2.815563in}{1.222221in}}{\pgfqpoint{2.804513in}{1.222221in}}%
\pgfpathcurveto{\pgfqpoint{2.793463in}{1.222221in}}{\pgfqpoint{2.782864in}{1.217831in}}{\pgfqpoint{2.775050in}{1.210018in}}%
\pgfpathcurveto{\pgfqpoint{2.767237in}{1.202204in}}{\pgfqpoint{2.762847in}{1.191605in}}{\pgfqpoint{2.762847in}{1.180555in}}%
\pgfpathcurveto{\pgfqpoint{2.762847in}{1.169505in}}{\pgfqpoint{2.767237in}{1.158906in}}{\pgfqpoint{2.775050in}{1.151092in}}%
\pgfpathcurveto{\pgfqpoint{2.782864in}{1.143278in}}{\pgfqpoint{2.793463in}{1.138888in}}{\pgfqpoint{2.804513in}{1.138888in}}%
\pgfpathlineto{\pgfqpoint{2.804513in}{1.138888in}}%
\pgfpathclose%
\pgfusepath{stroke}%
\end{pgfscope}%
\begin{pgfscope}%
\pgfpathrectangle{\pgfqpoint{0.847223in}{0.554012in}}{\pgfqpoint{6.200000in}{4.530000in}}%
\pgfusepath{clip}%
\pgfsetbuttcap%
\pgfsetroundjoin%
\pgfsetlinewidth{1.003750pt}%
\definecolor{currentstroke}{rgb}{1.000000,0.000000,0.000000}%
\pgfsetstrokecolor{currentstroke}%
\pgfsetdash{}{0pt}%
\pgfpathmoveto{\pgfqpoint{2.809846in}{1.136649in}}%
\pgfpathcurveto{\pgfqpoint{2.820897in}{1.136649in}}{\pgfqpoint{2.831496in}{1.141039in}}{\pgfqpoint{2.839309in}{1.148853in}}%
\pgfpathcurveto{\pgfqpoint{2.847123in}{1.156667in}}{\pgfqpoint{2.851513in}{1.167266in}}{\pgfqpoint{2.851513in}{1.178316in}}%
\pgfpathcurveto{\pgfqpoint{2.851513in}{1.189366in}}{\pgfqpoint{2.847123in}{1.199965in}}{\pgfqpoint{2.839309in}{1.207778in}}%
\pgfpathcurveto{\pgfqpoint{2.831496in}{1.215592in}}{\pgfqpoint{2.820897in}{1.219982in}}{\pgfqpoint{2.809846in}{1.219982in}}%
\pgfpathcurveto{\pgfqpoint{2.798796in}{1.219982in}}{\pgfqpoint{2.788197in}{1.215592in}}{\pgfqpoint{2.780384in}{1.207778in}}%
\pgfpathcurveto{\pgfqpoint{2.772570in}{1.199965in}}{\pgfqpoint{2.768180in}{1.189366in}}{\pgfqpoint{2.768180in}{1.178316in}}%
\pgfpathcurveto{\pgfqpoint{2.768180in}{1.167266in}}{\pgfqpoint{2.772570in}{1.156667in}}{\pgfqpoint{2.780384in}{1.148853in}}%
\pgfpathcurveto{\pgfqpoint{2.788197in}{1.141039in}}{\pgfqpoint{2.798796in}{1.136649in}}{\pgfqpoint{2.809846in}{1.136649in}}%
\pgfpathlineto{\pgfqpoint{2.809846in}{1.136649in}}%
\pgfpathclose%
\pgfusepath{stroke}%
\end{pgfscope}%
\begin{pgfscope}%
\pgfpathrectangle{\pgfqpoint{0.847223in}{0.554012in}}{\pgfqpoint{6.200000in}{4.530000in}}%
\pgfusepath{clip}%
\pgfsetbuttcap%
\pgfsetroundjoin%
\pgfsetlinewidth{1.003750pt}%
\definecolor{currentstroke}{rgb}{1.000000,0.000000,0.000000}%
\pgfsetstrokecolor{currentstroke}%
\pgfsetdash{}{0pt}%
\pgfpathmoveto{\pgfqpoint{2.815180in}{1.134419in}}%
\pgfpathcurveto{\pgfqpoint{2.826230in}{1.134419in}}{\pgfqpoint{2.836829in}{1.138810in}}{\pgfqpoint{2.844642in}{1.146623in}}%
\pgfpathcurveto{\pgfqpoint{2.852456in}{1.154437in}}{\pgfqpoint{2.856846in}{1.165036in}}{\pgfqpoint{2.856846in}{1.176086in}}%
\pgfpathcurveto{\pgfqpoint{2.856846in}{1.187136in}}{\pgfqpoint{2.852456in}{1.197735in}}{\pgfqpoint{2.844642in}{1.205549in}}%
\pgfpathcurveto{\pgfqpoint{2.836829in}{1.213363in}}{\pgfqpoint{2.826230in}{1.217753in}}{\pgfqpoint{2.815180in}{1.217753in}}%
\pgfpathcurveto{\pgfqpoint{2.804130in}{1.217753in}}{\pgfqpoint{2.793530in}{1.213363in}}{\pgfqpoint{2.785717in}{1.205549in}}%
\pgfpathcurveto{\pgfqpoint{2.777903in}{1.197735in}}{\pgfqpoint{2.773513in}{1.187136in}}{\pgfqpoint{2.773513in}{1.176086in}}%
\pgfpathcurveto{\pgfqpoint{2.773513in}{1.165036in}}{\pgfqpoint{2.777903in}{1.154437in}}{\pgfqpoint{2.785717in}{1.146623in}}%
\pgfpathcurveto{\pgfqpoint{2.793530in}{1.138810in}}{\pgfqpoint{2.804130in}{1.134419in}}{\pgfqpoint{2.815180in}{1.134419in}}%
\pgfpathlineto{\pgfqpoint{2.815180in}{1.134419in}}%
\pgfpathclose%
\pgfusepath{stroke}%
\end{pgfscope}%
\begin{pgfscope}%
\pgfpathrectangle{\pgfqpoint{0.847223in}{0.554012in}}{\pgfqpoint{6.200000in}{4.530000in}}%
\pgfusepath{clip}%
\pgfsetbuttcap%
\pgfsetroundjoin%
\pgfsetlinewidth{1.003750pt}%
\definecolor{currentstroke}{rgb}{1.000000,0.000000,0.000000}%
\pgfsetstrokecolor{currentstroke}%
\pgfsetdash{}{0pt}%
\pgfpathmoveto{\pgfqpoint{2.820513in}{1.132199in}}%
\pgfpathcurveto{\pgfqpoint{2.831563in}{1.132199in}}{\pgfqpoint{2.842162in}{1.136590in}}{\pgfqpoint{2.849976in}{1.144403in}}%
\pgfpathcurveto{\pgfqpoint{2.857789in}{1.152217in}}{\pgfqpoint{2.862180in}{1.162816in}}{\pgfqpoint{2.862180in}{1.173866in}}%
\pgfpathcurveto{\pgfqpoint{2.862180in}{1.184916in}}{\pgfqpoint{2.857789in}{1.195515in}}{\pgfqpoint{2.849976in}{1.203329in}}%
\pgfpathcurveto{\pgfqpoint{2.842162in}{1.211143in}}{\pgfqpoint{2.831563in}{1.215533in}}{\pgfqpoint{2.820513in}{1.215533in}}%
\pgfpathcurveto{\pgfqpoint{2.809463in}{1.215533in}}{\pgfqpoint{2.798864in}{1.211143in}}{\pgfqpoint{2.791050in}{1.203329in}}%
\pgfpathcurveto{\pgfqpoint{2.783236in}{1.195515in}}{\pgfqpoint{2.778846in}{1.184916in}}{\pgfqpoint{2.778846in}{1.173866in}}%
\pgfpathcurveto{\pgfqpoint{2.778846in}{1.162816in}}{\pgfqpoint{2.783236in}{1.152217in}}{\pgfqpoint{2.791050in}{1.144403in}}%
\pgfpathcurveto{\pgfqpoint{2.798864in}{1.136590in}}{\pgfqpoint{2.809463in}{1.132199in}}{\pgfqpoint{2.820513in}{1.132199in}}%
\pgfpathlineto{\pgfqpoint{2.820513in}{1.132199in}}%
\pgfpathclose%
\pgfusepath{stroke}%
\end{pgfscope}%
\begin{pgfscope}%
\pgfpathrectangle{\pgfqpoint{0.847223in}{0.554012in}}{\pgfqpoint{6.200000in}{4.530000in}}%
\pgfusepath{clip}%
\pgfsetbuttcap%
\pgfsetroundjoin%
\pgfsetlinewidth{1.003750pt}%
\definecolor{currentstroke}{rgb}{1.000000,0.000000,0.000000}%
\pgfsetstrokecolor{currentstroke}%
\pgfsetdash{}{0pt}%
\pgfpathmoveto{\pgfqpoint{2.825846in}{1.129989in}}%
\pgfpathcurveto{\pgfqpoint{2.836896in}{1.129989in}}{\pgfqpoint{2.847495in}{1.134379in}}{\pgfqpoint{2.855309in}{1.142193in}}%
\pgfpathcurveto{\pgfqpoint{2.863122in}{1.150006in}}{\pgfqpoint{2.867513in}{1.160605in}}{\pgfqpoint{2.867513in}{1.171656in}}%
\pgfpathcurveto{\pgfqpoint{2.867513in}{1.182706in}}{\pgfqpoint{2.863122in}{1.193305in}}{\pgfqpoint{2.855309in}{1.201118in}}%
\pgfpathcurveto{\pgfqpoint{2.847495in}{1.208932in}}{\pgfqpoint{2.836896in}{1.213322in}}{\pgfqpoint{2.825846in}{1.213322in}}%
\pgfpathcurveto{\pgfqpoint{2.814796in}{1.213322in}}{\pgfqpoint{2.804197in}{1.208932in}}{\pgfqpoint{2.796383in}{1.201118in}}%
\pgfpathcurveto{\pgfqpoint{2.788570in}{1.193305in}}{\pgfqpoint{2.784179in}{1.182706in}}{\pgfqpoint{2.784179in}{1.171656in}}%
\pgfpathcurveto{\pgfqpoint{2.784179in}{1.160605in}}{\pgfqpoint{2.788570in}{1.150006in}}{\pgfqpoint{2.796383in}{1.142193in}}%
\pgfpathcurveto{\pgfqpoint{2.804197in}{1.134379in}}{\pgfqpoint{2.814796in}{1.129989in}}{\pgfqpoint{2.825846in}{1.129989in}}%
\pgfpathlineto{\pgfqpoint{2.825846in}{1.129989in}}%
\pgfpathclose%
\pgfusepath{stroke}%
\end{pgfscope}%
\begin{pgfscope}%
\pgfpathrectangle{\pgfqpoint{0.847223in}{0.554012in}}{\pgfqpoint{6.200000in}{4.530000in}}%
\pgfusepath{clip}%
\pgfsetbuttcap%
\pgfsetroundjoin%
\pgfsetlinewidth{1.003750pt}%
\definecolor{currentstroke}{rgb}{1.000000,0.000000,0.000000}%
\pgfsetstrokecolor{currentstroke}%
\pgfsetdash{}{0pt}%
\pgfpathmoveto{\pgfqpoint{2.831179in}{1.127788in}}%
\pgfpathcurveto{\pgfqpoint{2.842229in}{1.127788in}}{\pgfqpoint{2.852828in}{1.132178in}}{\pgfqpoint{2.860642in}{1.139992in}}%
\pgfpathcurveto{\pgfqpoint{2.868456in}{1.147805in}}{\pgfqpoint{2.872846in}{1.158404in}}{\pgfqpoint{2.872846in}{1.169454in}}%
\pgfpathcurveto{\pgfqpoint{2.872846in}{1.180505in}}{\pgfqpoint{2.868456in}{1.191104in}}{\pgfqpoint{2.860642in}{1.198917in}}%
\pgfpathcurveto{\pgfqpoint{2.852828in}{1.206731in}}{\pgfqpoint{2.842229in}{1.211121in}}{\pgfqpoint{2.831179in}{1.211121in}}%
\pgfpathcurveto{\pgfqpoint{2.820129in}{1.211121in}}{\pgfqpoint{2.809530in}{1.206731in}}{\pgfqpoint{2.801716in}{1.198917in}}%
\pgfpathcurveto{\pgfqpoint{2.793903in}{1.191104in}}{\pgfqpoint{2.789513in}{1.180505in}}{\pgfqpoint{2.789513in}{1.169454in}}%
\pgfpathcurveto{\pgfqpoint{2.789513in}{1.158404in}}{\pgfqpoint{2.793903in}{1.147805in}}{\pgfqpoint{2.801716in}{1.139992in}}%
\pgfpathcurveto{\pgfqpoint{2.809530in}{1.132178in}}{\pgfqpoint{2.820129in}{1.127788in}}{\pgfqpoint{2.831179in}{1.127788in}}%
\pgfpathlineto{\pgfqpoint{2.831179in}{1.127788in}}%
\pgfpathclose%
\pgfusepath{stroke}%
\end{pgfscope}%
\begin{pgfscope}%
\pgfpathrectangle{\pgfqpoint{0.847223in}{0.554012in}}{\pgfqpoint{6.200000in}{4.530000in}}%
\pgfusepath{clip}%
\pgfsetbuttcap%
\pgfsetroundjoin%
\pgfsetlinewidth{1.003750pt}%
\definecolor{currentstroke}{rgb}{1.000000,0.000000,0.000000}%
\pgfsetstrokecolor{currentstroke}%
\pgfsetdash{}{0pt}%
\pgfpathmoveto{\pgfqpoint{2.836512in}{1.125596in}}%
\pgfpathcurveto{\pgfqpoint{2.847563in}{1.125596in}}{\pgfqpoint{2.858162in}{1.129986in}}{\pgfqpoint{2.865975in}{1.137800in}}%
\pgfpathcurveto{\pgfqpoint{2.873789in}{1.145613in}}{\pgfqpoint{2.878179in}{1.156212in}}{\pgfqpoint{2.878179in}{1.167263in}}%
\pgfpathcurveto{\pgfqpoint{2.878179in}{1.178313in}}{\pgfqpoint{2.873789in}{1.188912in}}{\pgfqpoint{2.865975in}{1.196725in}}%
\pgfpathcurveto{\pgfqpoint{2.858162in}{1.204539in}}{\pgfqpoint{2.847563in}{1.208929in}}{\pgfqpoint{2.836512in}{1.208929in}}%
\pgfpathcurveto{\pgfqpoint{2.825462in}{1.208929in}}{\pgfqpoint{2.814863in}{1.204539in}}{\pgfqpoint{2.807050in}{1.196725in}}%
\pgfpathcurveto{\pgfqpoint{2.799236in}{1.188912in}}{\pgfqpoint{2.794846in}{1.178313in}}{\pgfqpoint{2.794846in}{1.167263in}}%
\pgfpathcurveto{\pgfqpoint{2.794846in}{1.156212in}}{\pgfqpoint{2.799236in}{1.145613in}}{\pgfqpoint{2.807050in}{1.137800in}}%
\pgfpathcurveto{\pgfqpoint{2.814863in}{1.129986in}}{\pgfqpoint{2.825462in}{1.125596in}}{\pgfqpoint{2.836512in}{1.125596in}}%
\pgfpathlineto{\pgfqpoint{2.836512in}{1.125596in}}%
\pgfpathclose%
\pgfusepath{stroke}%
\end{pgfscope}%
\begin{pgfscope}%
\pgfpathrectangle{\pgfqpoint{0.847223in}{0.554012in}}{\pgfqpoint{6.200000in}{4.530000in}}%
\pgfusepath{clip}%
\pgfsetbuttcap%
\pgfsetroundjoin%
\pgfsetlinewidth{1.003750pt}%
\definecolor{currentstroke}{rgb}{1.000000,0.000000,0.000000}%
\pgfsetstrokecolor{currentstroke}%
\pgfsetdash{}{0pt}%
\pgfpathmoveto{\pgfqpoint{2.841846in}{1.123413in}}%
\pgfpathcurveto{\pgfqpoint{2.852896in}{1.123413in}}{\pgfqpoint{2.863495in}{1.127804in}}{\pgfqpoint{2.871308in}{1.135617in}}%
\pgfpathcurveto{\pgfqpoint{2.879122in}{1.143431in}}{\pgfqpoint{2.883512in}{1.154030in}}{\pgfqpoint{2.883512in}{1.165080in}}%
\pgfpathcurveto{\pgfqpoint{2.883512in}{1.176130in}}{\pgfqpoint{2.879122in}{1.186729in}}{\pgfqpoint{2.871308in}{1.194543in}}%
\pgfpathcurveto{\pgfqpoint{2.863495in}{1.202356in}}{\pgfqpoint{2.852896in}{1.206747in}}{\pgfqpoint{2.841846in}{1.206747in}}%
\pgfpathcurveto{\pgfqpoint{2.830796in}{1.206747in}}{\pgfqpoint{2.820197in}{1.202356in}}{\pgfqpoint{2.812383in}{1.194543in}}%
\pgfpathcurveto{\pgfqpoint{2.804569in}{1.186729in}}{\pgfqpoint{2.800179in}{1.176130in}}{\pgfqpoint{2.800179in}{1.165080in}}%
\pgfpathcurveto{\pgfqpoint{2.800179in}{1.154030in}}{\pgfqpoint{2.804569in}{1.143431in}}{\pgfqpoint{2.812383in}{1.135617in}}%
\pgfpathcurveto{\pgfqpoint{2.820197in}{1.127804in}}{\pgfqpoint{2.830796in}{1.123413in}}{\pgfqpoint{2.841846in}{1.123413in}}%
\pgfpathlineto{\pgfqpoint{2.841846in}{1.123413in}}%
\pgfpathclose%
\pgfusepath{stroke}%
\end{pgfscope}%
\begin{pgfscope}%
\pgfpathrectangle{\pgfqpoint{0.847223in}{0.554012in}}{\pgfqpoint{6.200000in}{4.530000in}}%
\pgfusepath{clip}%
\pgfsetbuttcap%
\pgfsetroundjoin%
\pgfsetlinewidth{1.003750pt}%
\definecolor{currentstroke}{rgb}{1.000000,0.000000,0.000000}%
\pgfsetstrokecolor{currentstroke}%
\pgfsetdash{}{0pt}%
\pgfpathmoveto{\pgfqpoint{2.847179in}{1.121240in}}%
\pgfpathcurveto{\pgfqpoint{2.858229in}{1.121240in}}{\pgfqpoint{2.868828in}{1.125630in}}{\pgfqpoint{2.876642in}{1.133444in}}%
\pgfpathcurveto{\pgfqpoint{2.884455in}{1.141258in}}{\pgfqpoint{2.888846in}{1.151857in}}{\pgfqpoint{2.888846in}{1.162907in}}%
\pgfpathcurveto{\pgfqpoint{2.888846in}{1.173957in}}{\pgfqpoint{2.884455in}{1.184556in}}{\pgfqpoint{2.876642in}{1.192370in}}%
\pgfpathcurveto{\pgfqpoint{2.868828in}{1.200183in}}{\pgfqpoint{2.858229in}{1.204573in}}{\pgfqpoint{2.847179in}{1.204573in}}%
\pgfpathcurveto{\pgfqpoint{2.836129in}{1.204573in}}{\pgfqpoint{2.825530in}{1.200183in}}{\pgfqpoint{2.817716in}{1.192370in}}%
\pgfpathcurveto{\pgfqpoint{2.809903in}{1.184556in}}{\pgfqpoint{2.805512in}{1.173957in}}{\pgfqpoint{2.805512in}{1.162907in}}%
\pgfpathcurveto{\pgfqpoint{2.805512in}{1.151857in}}{\pgfqpoint{2.809903in}{1.141258in}}{\pgfqpoint{2.817716in}{1.133444in}}%
\pgfpathcurveto{\pgfqpoint{2.825530in}{1.125630in}}{\pgfqpoint{2.836129in}{1.121240in}}{\pgfqpoint{2.847179in}{1.121240in}}%
\pgfpathlineto{\pgfqpoint{2.847179in}{1.121240in}}%
\pgfpathclose%
\pgfusepath{stroke}%
\end{pgfscope}%
\begin{pgfscope}%
\pgfpathrectangle{\pgfqpoint{0.847223in}{0.554012in}}{\pgfqpoint{6.200000in}{4.530000in}}%
\pgfusepath{clip}%
\pgfsetbuttcap%
\pgfsetroundjoin%
\pgfsetlinewidth{1.003750pt}%
\definecolor{currentstroke}{rgb}{1.000000,0.000000,0.000000}%
\pgfsetstrokecolor{currentstroke}%
\pgfsetdash{}{0pt}%
\pgfpathmoveto{\pgfqpoint{2.852512in}{1.119076in}}%
\pgfpathcurveto{\pgfqpoint{2.863562in}{1.119076in}}{\pgfqpoint{2.874161in}{1.123466in}}{\pgfqpoint{2.881975in}{1.131280in}}%
\pgfpathcurveto{\pgfqpoint{2.889789in}{1.139094in}}{\pgfqpoint{2.894179in}{1.149693in}}{\pgfqpoint{2.894179in}{1.160743in}}%
\pgfpathcurveto{\pgfqpoint{2.894179in}{1.171793in}}{\pgfqpoint{2.889789in}{1.182392in}}{\pgfqpoint{2.881975in}{1.190205in}}%
\pgfpathcurveto{\pgfqpoint{2.874161in}{1.198019in}}{\pgfqpoint{2.863562in}{1.202409in}}{\pgfqpoint{2.852512in}{1.202409in}}%
\pgfpathcurveto{\pgfqpoint{2.841462in}{1.202409in}}{\pgfqpoint{2.830863in}{1.198019in}}{\pgfqpoint{2.823049in}{1.190205in}}%
\pgfpathcurveto{\pgfqpoint{2.815236in}{1.182392in}}{\pgfqpoint{2.810845in}{1.171793in}}{\pgfqpoint{2.810845in}{1.160743in}}%
\pgfpathcurveto{\pgfqpoint{2.810845in}{1.149693in}}{\pgfqpoint{2.815236in}{1.139094in}}{\pgfqpoint{2.823049in}{1.131280in}}%
\pgfpathcurveto{\pgfqpoint{2.830863in}{1.123466in}}{\pgfqpoint{2.841462in}{1.119076in}}{\pgfqpoint{2.852512in}{1.119076in}}%
\pgfpathlineto{\pgfqpoint{2.852512in}{1.119076in}}%
\pgfpathclose%
\pgfusepath{stroke}%
\end{pgfscope}%
\begin{pgfscope}%
\pgfpathrectangle{\pgfqpoint{0.847223in}{0.554012in}}{\pgfqpoint{6.200000in}{4.530000in}}%
\pgfusepath{clip}%
\pgfsetbuttcap%
\pgfsetroundjoin%
\pgfsetlinewidth{1.003750pt}%
\definecolor{currentstroke}{rgb}{1.000000,0.000000,0.000000}%
\pgfsetstrokecolor{currentstroke}%
\pgfsetdash{}{0pt}%
\pgfpathmoveto{\pgfqpoint{2.857845in}{1.116921in}}%
\pgfpathcurveto{\pgfqpoint{2.868895in}{1.116921in}}{\pgfqpoint{2.879495in}{1.121311in}}{\pgfqpoint{2.887308in}{1.129125in}}%
\pgfpathcurveto{\pgfqpoint{2.895122in}{1.136938in}}{\pgfqpoint{2.899512in}{1.147538in}}{\pgfqpoint{2.899512in}{1.158588in}}%
\pgfpathcurveto{\pgfqpoint{2.899512in}{1.169638in}}{\pgfqpoint{2.895122in}{1.180237in}}{\pgfqpoint{2.887308in}{1.188050in}}%
\pgfpathcurveto{\pgfqpoint{2.879495in}{1.195864in}}{\pgfqpoint{2.868895in}{1.200254in}}{\pgfqpoint{2.857845in}{1.200254in}}%
\pgfpathcurveto{\pgfqpoint{2.846795in}{1.200254in}}{\pgfqpoint{2.836196in}{1.195864in}}{\pgfqpoint{2.828383in}{1.188050in}}%
\pgfpathcurveto{\pgfqpoint{2.820569in}{1.180237in}}{\pgfqpoint{2.816179in}{1.169638in}}{\pgfqpoint{2.816179in}{1.158588in}}%
\pgfpathcurveto{\pgfqpoint{2.816179in}{1.147538in}}{\pgfqpoint{2.820569in}{1.136938in}}{\pgfqpoint{2.828383in}{1.129125in}}%
\pgfpathcurveto{\pgfqpoint{2.836196in}{1.121311in}}{\pgfqpoint{2.846795in}{1.116921in}}{\pgfqpoint{2.857845in}{1.116921in}}%
\pgfpathlineto{\pgfqpoint{2.857845in}{1.116921in}}%
\pgfpathclose%
\pgfusepath{stroke}%
\end{pgfscope}%
\begin{pgfscope}%
\pgfpathrectangle{\pgfqpoint{0.847223in}{0.554012in}}{\pgfqpoint{6.200000in}{4.530000in}}%
\pgfusepath{clip}%
\pgfsetbuttcap%
\pgfsetroundjoin%
\pgfsetlinewidth{1.003750pt}%
\definecolor{currentstroke}{rgb}{1.000000,0.000000,0.000000}%
\pgfsetstrokecolor{currentstroke}%
\pgfsetdash{}{0pt}%
\pgfpathmoveto{\pgfqpoint{2.863179in}{1.114775in}}%
\pgfpathcurveto{\pgfqpoint{2.874229in}{1.114775in}}{\pgfqpoint{2.884828in}{1.119165in}}{\pgfqpoint{2.892641in}{1.126979in}}%
\pgfpathcurveto{\pgfqpoint{2.900455in}{1.134793in}}{\pgfqpoint{2.904845in}{1.145392in}}{\pgfqpoint{2.904845in}{1.156442in}}%
\pgfpathcurveto{\pgfqpoint{2.904845in}{1.167492in}}{\pgfqpoint{2.900455in}{1.178091in}}{\pgfqpoint{2.892641in}{1.185904in}}%
\pgfpathcurveto{\pgfqpoint{2.884828in}{1.193718in}}{\pgfqpoint{2.874229in}{1.198108in}}{\pgfqpoint{2.863179in}{1.198108in}}%
\pgfpathcurveto{\pgfqpoint{2.852128in}{1.198108in}}{\pgfqpoint{2.841529in}{1.193718in}}{\pgfqpoint{2.833716in}{1.185904in}}%
\pgfpathcurveto{\pgfqpoint{2.825902in}{1.178091in}}{\pgfqpoint{2.821512in}{1.167492in}}{\pgfqpoint{2.821512in}{1.156442in}}%
\pgfpathcurveto{\pgfqpoint{2.821512in}{1.145392in}}{\pgfqpoint{2.825902in}{1.134793in}}{\pgfqpoint{2.833716in}{1.126979in}}%
\pgfpathcurveto{\pgfqpoint{2.841529in}{1.119165in}}{\pgfqpoint{2.852128in}{1.114775in}}{\pgfqpoint{2.863179in}{1.114775in}}%
\pgfpathlineto{\pgfqpoint{2.863179in}{1.114775in}}%
\pgfpathclose%
\pgfusepath{stroke}%
\end{pgfscope}%
\begin{pgfscope}%
\pgfpathrectangle{\pgfqpoint{0.847223in}{0.554012in}}{\pgfqpoint{6.200000in}{4.530000in}}%
\pgfusepath{clip}%
\pgfsetbuttcap%
\pgfsetroundjoin%
\pgfsetlinewidth{1.003750pt}%
\definecolor{currentstroke}{rgb}{1.000000,0.000000,0.000000}%
\pgfsetstrokecolor{currentstroke}%
\pgfsetdash{}{0pt}%
\pgfpathmoveto{\pgfqpoint{2.868512in}{1.112638in}}%
\pgfpathcurveto{\pgfqpoint{2.879562in}{1.112638in}}{\pgfqpoint{2.890161in}{1.117028in}}{\pgfqpoint{2.897975in}{1.124842in}}%
\pgfpathcurveto{\pgfqpoint{2.905788in}{1.132656in}}{\pgfqpoint{2.910178in}{1.143255in}}{\pgfqpoint{2.910178in}{1.154305in}}%
\pgfpathcurveto{\pgfqpoint{2.910178in}{1.165355in}}{\pgfqpoint{2.905788in}{1.175954in}}{\pgfqpoint{2.897975in}{1.183767in}}%
\pgfpathcurveto{\pgfqpoint{2.890161in}{1.191581in}}{\pgfqpoint{2.879562in}{1.195971in}}{\pgfqpoint{2.868512in}{1.195971in}}%
\pgfpathcurveto{\pgfqpoint{2.857462in}{1.195971in}}{\pgfqpoint{2.846863in}{1.191581in}}{\pgfqpoint{2.839049in}{1.183767in}}%
\pgfpathcurveto{\pgfqpoint{2.831235in}{1.175954in}}{\pgfqpoint{2.826845in}{1.165355in}}{\pgfqpoint{2.826845in}{1.154305in}}%
\pgfpathcurveto{\pgfqpoint{2.826845in}{1.143255in}}{\pgfqpoint{2.831235in}{1.132656in}}{\pgfqpoint{2.839049in}{1.124842in}}%
\pgfpathcurveto{\pgfqpoint{2.846863in}{1.117028in}}{\pgfqpoint{2.857462in}{1.112638in}}{\pgfqpoint{2.868512in}{1.112638in}}%
\pgfpathlineto{\pgfqpoint{2.868512in}{1.112638in}}%
\pgfpathclose%
\pgfusepath{stroke}%
\end{pgfscope}%
\begin{pgfscope}%
\pgfpathrectangle{\pgfqpoint{0.847223in}{0.554012in}}{\pgfqpoint{6.200000in}{4.530000in}}%
\pgfusepath{clip}%
\pgfsetbuttcap%
\pgfsetroundjoin%
\pgfsetlinewidth{1.003750pt}%
\definecolor{currentstroke}{rgb}{1.000000,0.000000,0.000000}%
\pgfsetstrokecolor{currentstroke}%
\pgfsetdash{}{0pt}%
\pgfpathmoveto{\pgfqpoint{2.873845in}{1.110510in}}%
\pgfpathcurveto{\pgfqpoint{2.884895in}{1.110510in}}{\pgfqpoint{2.895494in}{1.114900in}}{\pgfqpoint{2.903308in}{1.122714in}}%
\pgfpathcurveto{\pgfqpoint{2.911121in}{1.130528in}}{\pgfqpoint{2.915512in}{1.141127in}}{\pgfqpoint{2.915512in}{1.152177in}}%
\pgfpathcurveto{\pgfqpoint{2.915512in}{1.163227in}}{\pgfqpoint{2.911121in}{1.173826in}}{\pgfqpoint{2.903308in}{1.181639in}}%
\pgfpathcurveto{\pgfqpoint{2.895494in}{1.189453in}}{\pgfqpoint{2.884895in}{1.193843in}}{\pgfqpoint{2.873845in}{1.193843in}}%
\pgfpathcurveto{\pgfqpoint{2.862795in}{1.193843in}}{\pgfqpoint{2.852196in}{1.189453in}}{\pgfqpoint{2.844382in}{1.181639in}}%
\pgfpathcurveto{\pgfqpoint{2.836569in}{1.173826in}}{\pgfqpoint{2.832178in}{1.163227in}}{\pgfqpoint{2.832178in}{1.152177in}}%
\pgfpathcurveto{\pgfqpoint{2.832178in}{1.141127in}}{\pgfqpoint{2.836569in}{1.130528in}}{\pgfqpoint{2.844382in}{1.122714in}}%
\pgfpathcurveto{\pgfqpoint{2.852196in}{1.114900in}}{\pgfqpoint{2.862795in}{1.110510in}}{\pgfqpoint{2.873845in}{1.110510in}}%
\pgfpathlineto{\pgfqpoint{2.873845in}{1.110510in}}%
\pgfpathclose%
\pgfusepath{stroke}%
\end{pgfscope}%
\begin{pgfscope}%
\pgfpathrectangle{\pgfqpoint{0.847223in}{0.554012in}}{\pgfqpoint{6.200000in}{4.530000in}}%
\pgfusepath{clip}%
\pgfsetbuttcap%
\pgfsetroundjoin%
\pgfsetlinewidth{1.003750pt}%
\definecolor{currentstroke}{rgb}{1.000000,0.000000,0.000000}%
\pgfsetstrokecolor{currentstroke}%
\pgfsetdash{}{0pt}%
\pgfpathmoveto{\pgfqpoint{2.879178in}{1.108391in}}%
\pgfpathcurveto{\pgfqpoint{2.890228in}{1.108391in}}{\pgfqpoint{2.900827in}{1.112781in}}{\pgfqpoint{2.908641in}{1.120595in}}%
\pgfpathcurveto{\pgfqpoint{2.916455in}{1.128408in}}{\pgfqpoint{2.920845in}{1.139007in}}{\pgfqpoint{2.920845in}{1.150058in}}%
\pgfpathcurveto{\pgfqpoint{2.920845in}{1.161108in}}{\pgfqpoint{2.916455in}{1.171707in}}{\pgfqpoint{2.908641in}{1.179520in}}%
\pgfpathcurveto{\pgfqpoint{2.900827in}{1.187334in}}{\pgfqpoint{2.890228in}{1.191724in}}{\pgfqpoint{2.879178in}{1.191724in}}%
\pgfpathcurveto{\pgfqpoint{2.868128in}{1.191724in}}{\pgfqpoint{2.857529in}{1.187334in}}{\pgfqpoint{2.849715in}{1.179520in}}%
\pgfpathcurveto{\pgfqpoint{2.841902in}{1.171707in}}{\pgfqpoint{2.837512in}{1.161108in}}{\pgfqpoint{2.837512in}{1.150058in}}%
\pgfpathcurveto{\pgfqpoint{2.837512in}{1.139007in}}{\pgfqpoint{2.841902in}{1.128408in}}{\pgfqpoint{2.849715in}{1.120595in}}%
\pgfpathcurveto{\pgfqpoint{2.857529in}{1.112781in}}{\pgfqpoint{2.868128in}{1.108391in}}{\pgfqpoint{2.879178in}{1.108391in}}%
\pgfpathlineto{\pgfqpoint{2.879178in}{1.108391in}}%
\pgfpathclose%
\pgfusepath{stroke}%
\end{pgfscope}%
\begin{pgfscope}%
\pgfpathrectangle{\pgfqpoint{0.847223in}{0.554012in}}{\pgfqpoint{6.200000in}{4.530000in}}%
\pgfusepath{clip}%
\pgfsetbuttcap%
\pgfsetroundjoin%
\pgfsetlinewidth{1.003750pt}%
\definecolor{currentstroke}{rgb}{1.000000,0.000000,0.000000}%
\pgfsetstrokecolor{currentstroke}%
\pgfsetdash{}{0pt}%
\pgfpathmoveto{\pgfqpoint{2.884511in}{1.106281in}}%
\pgfpathcurveto{\pgfqpoint{2.895562in}{1.106281in}}{\pgfqpoint{2.906161in}{1.110671in}}{\pgfqpoint{2.913974in}{1.118484in}}%
\pgfpathcurveto{\pgfqpoint{2.921788in}{1.126298in}}{\pgfqpoint{2.926178in}{1.136897in}}{\pgfqpoint{2.926178in}{1.147947in}}%
\pgfpathcurveto{\pgfqpoint{2.926178in}{1.158997in}}{\pgfqpoint{2.921788in}{1.169596in}}{\pgfqpoint{2.913974in}{1.177410in}}%
\pgfpathcurveto{\pgfqpoint{2.906161in}{1.185224in}}{\pgfqpoint{2.895562in}{1.189614in}}{\pgfqpoint{2.884511in}{1.189614in}}%
\pgfpathcurveto{\pgfqpoint{2.873461in}{1.189614in}}{\pgfqpoint{2.862862in}{1.185224in}}{\pgfqpoint{2.855049in}{1.177410in}}%
\pgfpathcurveto{\pgfqpoint{2.847235in}{1.169596in}}{\pgfqpoint{2.842845in}{1.158997in}}{\pgfqpoint{2.842845in}{1.147947in}}%
\pgfpathcurveto{\pgfqpoint{2.842845in}{1.136897in}}{\pgfqpoint{2.847235in}{1.126298in}}{\pgfqpoint{2.855049in}{1.118484in}}%
\pgfpathcurveto{\pgfqpoint{2.862862in}{1.110671in}}{\pgfqpoint{2.873461in}{1.106281in}}{\pgfqpoint{2.884511in}{1.106281in}}%
\pgfpathlineto{\pgfqpoint{2.884511in}{1.106281in}}%
\pgfpathclose%
\pgfusepath{stroke}%
\end{pgfscope}%
\begin{pgfscope}%
\pgfpathrectangle{\pgfqpoint{0.847223in}{0.554012in}}{\pgfqpoint{6.200000in}{4.530000in}}%
\pgfusepath{clip}%
\pgfsetbuttcap%
\pgfsetroundjoin%
\pgfsetlinewidth{1.003750pt}%
\definecolor{currentstroke}{rgb}{1.000000,0.000000,0.000000}%
\pgfsetstrokecolor{currentstroke}%
\pgfsetdash{}{0pt}%
\pgfpathmoveto{\pgfqpoint{2.889845in}{1.104179in}}%
\pgfpathcurveto{\pgfqpoint{2.900895in}{1.104179in}}{\pgfqpoint{2.911494in}{1.108569in}}{\pgfqpoint{2.919307in}{1.116383in}}%
\pgfpathcurveto{\pgfqpoint{2.927121in}{1.124197in}}{\pgfqpoint{2.931511in}{1.134796in}}{\pgfqpoint{2.931511in}{1.145846in}}%
\pgfpathcurveto{\pgfqpoint{2.931511in}{1.156896in}}{\pgfqpoint{2.927121in}{1.167495in}}{\pgfqpoint{2.919307in}{1.175308in}}%
\pgfpathcurveto{\pgfqpoint{2.911494in}{1.183122in}}{\pgfqpoint{2.900895in}{1.187512in}}{\pgfqpoint{2.889845in}{1.187512in}}%
\pgfpathcurveto{\pgfqpoint{2.878795in}{1.187512in}}{\pgfqpoint{2.868195in}{1.183122in}}{\pgfqpoint{2.860382in}{1.175308in}}%
\pgfpathcurveto{\pgfqpoint{2.852568in}{1.167495in}}{\pgfqpoint{2.848178in}{1.156896in}}{\pgfqpoint{2.848178in}{1.145846in}}%
\pgfpathcurveto{\pgfqpoint{2.848178in}{1.134796in}}{\pgfqpoint{2.852568in}{1.124197in}}{\pgfqpoint{2.860382in}{1.116383in}}%
\pgfpathcurveto{\pgfqpoint{2.868195in}{1.108569in}}{\pgfqpoint{2.878795in}{1.104179in}}{\pgfqpoint{2.889845in}{1.104179in}}%
\pgfpathlineto{\pgfqpoint{2.889845in}{1.104179in}}%
\pgfpathclose%
\pgfusepath{stroke}%
\end{pgfscope}%
\begin{pgfscope}%
\pgfpathrectangle{\pgfqpoint{0.847223in}{0.554012in}}{\pgfqpoint{6.200000in}{4.530000in}}%
\pgfusepath{clip}%
\pgfsetbuttcap%
\pgfsetroundjoin%
\pgfsetlinewidth{1.003750pt}%
\definecolor{currentstroke}{rgb}{1.000000,0.000000,0.000000}%
\pgfsetstrokecolor{currentstroke}%
\pgfsetdash{}{0pt}%
\pgfpathmoveto{\pgfqpoint{2.895178in}{1.102086in}}%
\pgfpathcurveto{\pgfqpoint{2.906228in}{1.102086in}}{\pgfqpoint{2.916827in}{1.106476in}}{\pgfqpoint{2.924641in}{1.114290in}}%
\pgfpathcurveto{\pgfqpoint{2.932454in}{1.122104in}}{\pgfqpoint{2.936845in}{1.132703in}}{\pgfqpoint{2.936845in}{1.143753in}}%
\pgfpathcurveto{\pgfqpoint{2.936845in}{1.154803in}}{\pgfqpoint{2.932454in}{1.165402in}}{\pgfqpoint{2.924641in}{1.173216in}}%
\pgfpathcurveto{\pgfqpoint{2.916827in}{1.181029in}}{\pgfqpoint{2.906228in}{1.185420in}}{\pgfqpoint{2.895178in}{1.185420in}}%
\pgfpathcurveto{\pgfqpoint{2.884128in}{1.185420in}}{\pgfqpoint{2.873529in}{1.181029in}}{\pgfqpoint{2.865715in}{1.173216in}}%
\pgfpathcurveto{\pgfqpoint{2.857901in}{1.165402in}}{\pgfqpoint{2.853511in}{1.154803in}}{\pgfqpoint{2.853511in}{1.143753in}}%
\pgfpathcurveto{\pgfqpoint{2.853511in}{1.132703in}}{\pgfqpoint{2.857901in}{1.122104in}}{\pgfqpoint{2.865715in}{1.114290in}}%
\pgfpathcurveto{\pgfqpoint{2.873529in}{1.106476in}}{\pgfqpoint{2.884128in}{1.102086in}}{\pgfqpoint{2.895178in}{1.102086in}}%
\pgfpathlineto{\pgfqpoint{2.895178in}{1.102086in}}%
\pgfpathclose%
\pgfusepath{stroke}%
\end{pgfscope}%
\begin{pgfscope}%
\pgfpathrectangle{\pgfqpoint{0.847223in}{0.554012in}}{\pgfqpoint{6.200000in}{4.530000in}}%
\pgfusepath{clip}%
\pgfsetbuttcap%
\pgfsetroundjoin%
\pgfsetlinewidth{1.003750pt}%
\definecolor{currentstroke}{rgb}{1.000000,0.000000,0.000000}%
\pgfsetstrokecolor{currentstroke}%
\pgfsetdash{}{0pt}%
\pgfpathmoveto{\pgfqpoint{2.900511in}{1.100002in}}%
\pgfpathcurveto{\pgfqpoint{2.911561in}{1.100002in}}{\pgfqpoint{2.922160in}{1.104392in}}{\pgfqpoint{2.929974in}{1.112206in}}%
\pgfpathcurveto{\pgfqpoint{2.937787in}{1.120020in}}{\pgfqpoint{2.942178in}{1.130619in}}{\pgfqpoint{2.942178in}{1.141669in}}%
\pgfpathcurveto{\pgfqpoint{2.942178in}{1.152719in}}{\pgfqpoint{2.937787in}{1.163318in}}{\pgfqpoint{2.929974in}{1.171131in}}%
\pgfpathcurveto{\pgfqpoint{2.922160in}{1.178945in}}{\pgfqpoint{2.911561in}{1.183335in}}{\pgfqpoint{2.900511in}{1.183335in}}%
\pgfpathcurveto{\pgfqpoint{2.889461in}{1.183335in}}{\pgfqpoint{2.878862in}{1.178945in}}{\pgfqpoint{2.871048in}{1.171131in}}%
\pgfpathcurveto{\pgfqpoint{2.863235in}{1.163318in}}{\pgfqpoint{2.858844in}{1.152719in}}{\pgfqpoint{2.858844in}{1.141669in}}%
\pgfpathcurveto{\pgfqpoint{2.858844in}{1.130619in}}{\pgfqpoint{2.863235in}{1.120020in}}{\pgfqpoint{2.871048in}{1.112206in}}%
\pgfpathcurveto{\pgfqpoint{2.878862in}{1.104392in}}{\pgfqpoint{2.889461in}{1.100002in}}{\pgfqpoint{2.900511in}{1.100002in}}%
\pgfpathlineto{\pgfqpoint{2.900511in}{1.100002in}}%
\pgfpathclose%
\pgfusepath{stroke}%
\end{pgfscope}%
\begin{pgfscope}%
\pgfpathrectangle{\pgfqpoint{0.847223in}{0.554012in}}{\pgfqpoint{6.200000in}{4.530000in}}%
\pgfusepath{clip}%
\pgfsetbuttcap%
\pgfsetroundjoin%
\pgfsetlinewidth{1.003750pt}%
\definecolor{currentstroke}{rgb}{1.000000,0.000000,0.000000}%
\pgfsetstrokecolor{currentstroke}%
\pgfsetdash{}{0pt}%
\pgfpathmoveto{\pgfqpoint{2.905844in}{1.097926in}}%
\pgfpathcurveto{\pgfqpoint{2.916894in}{1.097926in}}{\pgfqpoint{2.927493in}{1.102317in}}{\pgfqpoint{2.935307in}{1.110130in}}%
\pgfpathcurveto{\pgfqpoint{2.943121in}{1.117944in}}{\pgfqpoint{2.947511in}{1.128543in}}{\pgfqpoint{2.947511in}{1.139593in}}%
\pgfpathcurveto{\pgfqpoint{2.947511in}{1.150643in}}{\pgfqpoint{2.943121in}{1.161242in}}{\pgfqpoint{2.935307in}{1.169056in}}%
\pgfpathcurveto{\pgfqpoint{2.927493in}{1.176870in}}{\pgfqpoint{2.916894in}{1.181260in}}{\pgfqpoint{2.905844in}{1.181260in}}%
\pgfpathcurveto{\pgfqpoint{2.894794in}{1.181260in}}{\pgfqpoint{2.884195in}{1.176870in}}{\pgfqpoint{2.876382in}{1.169056in}}%
\pgfpathcurveto{\pgfqpoint{2.868568in}{1.161242in}}{\pgfqpoint{2.864178in}{1.150643in}}{\pgfqpoint{2.864178in}{1.139593in}}%
\pgfpathcurveto{\pgfqpoint{2.864178in}{1.128543in}}{\pgfqpoint{2.868568in}{1.117944in}}{\pgfqpoint{2.876382in}{1.110130in}}%
\pgfpathcurveto{\pgfqpoint{2.884195in}{1.102317in}}{\pgfqpoint{2.894794in}{1.097926in}}{\pgfqpoint{2.905844in}{1.097926in}}%
\pgfpathlineto{\pgfqpoint{2.905844in}{1.097926in}}%
\pgfpathclose%
\pgfusepath{stroke}%
\end{pgfscope}%
\begin{pgfscope}%
\pgfpathrectangle{\pgfqpoint{0.847223in}{0.554012in}}{\pgfqpoint{6.200000in}{4.530000in}}%
\pgfusepath{clip}%
\pgfsetbuttcap%
\pgfsetroundjoin%
\pgfsetlinewidth{1.003750pt}%
\definecolor{currentstroke}{rgb}{1.000000,0.000000,0.000000}%
\pgfsetstrokecolor{currentstroke}%
\pgfsetdash{}{0pt}%
\pgfpathmoveto{\pgfqpoint{2.911178in}{1.095859in}}%
\pgfpathcurveto{\pgfqpoint{2.922228in}{1.095859in}}{\pgfqpoint{2.932827in}{1.100250in}}{\pgfqpoint{2.940640in}{1.108063in}}%
\pgfpathcurveto{\pgfqpoint{2.948454in}{1.115877in}}{\pgfqpoint{2.952844in}{1.126476in}}{\pgfqpoint{2.952844in}{1.137526in}}%
\pgfpathcurveto{\pgfqpoint{2.952844in}{1.148576in}}{\pgfqpoint{2.948454in}{1.159175in}}{\pgfqpoint{2.940640in}{1.166989in}}%
\pgfpathcurveto{\pgfqpoint{2.932827in}{1.174803in}}{\pgfqpoint{2.922228in}{1.179193in}}{\pgfqpoint{2.911178in}{1.179193in}}%
\pgfpathcurveto{\pgfqpoint{2.900127in}{1.179193in}}{\pgfqpoint{2.889528in}{1.174803in}}{\pgfqpoint{2.881715in}{1.166989in}}%
\pgfpathcurveto{\pgfqpoint{2.873901in}{1.159175in}}{\pgfqpoint{2.869511in}{1.148576in}}{\pgfqpoint{2.869511in}{1.137526in}}%
\pgfpathcurveto{\pgfqpoint{2.869511in}{1.126476in}}{\pgfqpoint{2.873901in}{1.115877in}}{\pgfqpoint{2.881715in}{1.108063in}}%
\pgfpathcurveto{\pgfqpoint{2.889528in}{1.100250in}}{\pgfqpoint{2.900127in}{1.095859in}}{\pgfqpoint{2.911178in}{1.095859in}}%
\pgfpathlineto{\pgfqpoint{2.911178in}{1.095859in}}%
\pgfpathclose%
\pgfusepath{stroke}%
\end{pgfscope}%
\begin{pgfscope}%
\pgfpathrectangle{\pgfqpoint{0.847223in}{0.554012in}}{\pgfqpoint{6.200000in}{4.530000in}}%
\pgfusepath{clip}%
\pgfsetbuttcap%
\pgfsetroundjoin%
\pgfsetlinewidth{1.003750pt}%
\definecolor{currentstroke}{rgb}{1.000000,0.000000,0.000000}%
\pgfsetstrokecolor{currentstroke}%
\pgfsetdash{}{0pt}%
\pgfpathmoveto{\pgfqpoint{2.916511in}{1.093801in}}%
\pgfpathcurveto{\pgfqpoint{2.927561in}{1.093801in}}{\pgfqpoint{2.938160in}{1.098191in}}{\pgfqpoint{2.945974in}{1.106005in}}%
\pgfpathcurveto{\pgfqpoint{2.953787in}{1.113818in}}{\pgfqpoint{2.958177in}{1.124417in}}{\pgfqpoint{2.958177in}{1.135468in}}%
\pgfpathcurveto{\pgfqpoint{2.958177in}{1.146518in}}{\pgfqpoint{2.953787in}{1.157117in}}{\pgfqpoint{2.945974in}{1.164930in}}%
\pgfpathcurveto{\pgfqpoint{2.938160in}{1.172744in}}{\pgfqpoint{2.927561in}{1.177134in}}{\pgfqpoint{2.916511in}{1.177134in}}%
\pgfpathcurveto{\pgfqpoint{2.905461in}{1.177134in}}{\pgfqpoint{2.894862in}{1.172744in}}{\pgfqpoint{2.887048in}{1.164930in}}%
\pgfpathcurveto{\pgfqpoint{2.879234in}{1.157117in}}{\pgfqpoint{2.874844in}{1.146518in}}{\pgfqpoint{2.874844in}{1.135468in}}%
\pgfpathcurveto{\pgfqpoint{2.874844in}{1.124417in}}{\pgfqpoint{2.879234in}{1.113818in}}{\pgfqpoint{2.887048in}{1.106005in}}%
\pgfpathcurveto{\pgfqpoint{2.894862in}{1.098191in}}{\pgfqpoint{2.905461in}{1.093801in}}{\pgfqpoint{2.916511in}{1.093801in}}%
\pgfpathlineto{\pgfqpoint{2.916511in}{1.093801in}}%
\pgfpathclose%
\pgfusepath{stroke}%
\end{pgfscope}%
\begin{pgfscope}%
\pgfpathrectangle{\pgfqpoint{0.847223in}{0.554012in}}{\pgfqpoint{6.200000in}{4.530000in}}%
\pgfusepath{clip}%
\pgfsetbuttcap%
\pgfsetroundjoin%
\pgfsetlinewidth{1.003750pt}%
\definecolor{currentstroke}{rgb}{1.000000,0.000000,0.000000}%
\pgfsetstrokecolor{currentstroke}%
\pgfsetdash{}{0pt}%
\pgfpathmoveto{\pgfqpoint{2.921844in}{1.091751in}}%
\pgfpathcurveto{\pgfqpoint{2.932894in}{1.091751in}}{\pgfqpoint{2.943493in}{1.096141in}}{\pgfqpoint{2.951307in}{1.103955in}}%
\pgfpathcurveto{\pgfqpoint{2.959120in}{1.111768in}}{\pgfqpoint{2.963511in}{1.122367in}}{\pgfqpoint{2.963511in}{1.133418in}}%
\pgfpathcurveto{\pgfqpoint{2.963511in}{1.144468in}}{\pgfqpoint{2.959120in}{1.155067in}}{\pgfqpoint{2.951307in}{1.162880in}}%
\pgfpathcurveto{\pgfqpoint{2.943493in}{1.170694in}}{\pgfqpoint{2.932894in}{1.175084in}}{\pgfqpoint{2.921844in}{1.175084in}}%
\pgfpathcurveto{\pgfqpoint{2.910794in}{1.175084in}}{\pgfqpoint{2.900195in}{1.170694in}}{\pgfqpoint{2.892381in}{1.162880in}}%
\pgfpathcurveto{\pgfqpoint{2.884568in}{1.155067in}}{\pgfqpoint{2.880177in}{1.144468in}}{\pgfqpoint{2.880177in}{1.133418in}}%
\pgfpathcurveto{\pgfqpoint{2.880177in}{1.122367in}}{\pgfqpoint{2.884568in}{1.111768in}}{\pgfqpoint{2.892381in}{1.103955in}}%
\pgfpathcurveto{\pgfqpoint{2.900195in}{1.096141in}}{\pgfqpoint{2.910794in}{1.091751in}}{\pgfqpoint{2.921844in}{1.091751in}}%
\pgfpathlineto{\pgfqpoint{2.921844in}{1.091751in}}%
\pgfpathclose%
\pgfusepath{stroke}%
\end{pgfscope}%
\begin{pgfscope}%
\pgfpathrectangle{\pgfqpoint{0.847223in}{0.554012in}}{\pgfqpoint{6.200000in}{4.530000in}}%
\pgfusepath{clip}%
\pgfsetbuttcap%
\pgfsetroundjoin%
\pgfsetlinewidth{1.003750pt}%
\definecolor{currentstroke}{rgb}{1.000000,0.000000,0.000000}%
\pgfsetstrokecolor{currentstroke}%
\pgfsetdash{}{0pt}%
\pgfpathmoveto{\pgfqpoint{2.927177in}{1.089709in}}%
\pgfpathcurveto{\pgfqpoint{2.938227in}{1.089709in}}{\pgfqpoint{2.948826in}{1.094100in}}{\pgfqpoint{2.956640in}{1.101913in}}%
\pgfpathcurveto{\pgfqpoint{2.964454in}{1.109727in}}{\pgfqpoint{2.968844in}{1.120326in}}{\pgfqpoint{2.968844in}{1.131376in}}%
\pgfpathcurveto{\pgfqpoint{2.968844in}{1.142426in}}{\pgfqpoint{2.964454in}{1.153025in}}{\pgfqpoint{2.956640in}{1.160839in}}%
\pgfpathcurveto{\pgfqpoint{2.948826in}{1.168652in}}{\pgfqpoint{2.938227in}{1.173043in}}{\pgfqpoint{2.927177in}{1.173043in}}%
\pgfpathcurveto{\pgfqpoint{2.916127in}{1.173043in}}{\pgfqpoint{2.905528in}{1.168652in}}{\pgfqpoint{2.897714in}{1.160839in}}%
\pgfpathcurveto{\pgfqpoint{2.889901in}{1.153025in}}{\pgfqpoint{2.885510in}{1.142426in}}{\pgfqpoint{2.885510in}{1.131376in}}%
\pgfpathcurveto{\pgfqpoint{2.885510in}{1.120326in}}{\pgfqpoint{2.889901in}{1.109727in}}{\pgfqpoint{2.897714in}{1.101913in}}%
\pgfpathcurveto{\pgfqpoint{2.905528in}{1.094100in}}{\pgfqpoint{2.916127in}{1.089709in}}{\pgfqpoint{2.927177in}{1.089709in}}%
\pgfpathlineto{\pgfqpoint{2.927177in}{1.089709in}}%
\pgfpathclose%
\pgfusepath{stroke}%
\end{pgfscope}%
\begin{pgfscope}%
\pgfpathrectangle{\pgfqpoint{0.847223in}{0.554012in}}{\pgfqpoint{6.200000in}{4.530000in}}%
\pgfusepath{clip}%
\pgfsetbuttcap%
\pgfsetroundjoin%
\pgfsetlinewidth{1.003750pt}%
\definecolor{currentstroke}{rgb}{1.000000,0.000000,0.000000}%
\pgfsetstrokecolor{currentstroke}%
\pgfsetdash{}{0pt}%
\pgfpathmoveto{\pgfqpoint{2.932510in}{1.087676in}}%
\pgfpathcurveto{\pgfqpoint{2.943561in}{1.087676in}}{\pgfqpoint{2.954160in}{1.092066in}}{\pgfqpoint{2.961973in}{1.099880in}}%
\pgfpathcurveto{\pgfqpoint{2.969787in}{1.107693in}}{\pgfqpoint{2.974177in}{1.118292in}}{\pgfqpoint{2.974177in}{1.129343in}}%
\pgfpathcurveto{\pgfqpoint{2.974177in}{1.140393in}}{\pgfqpoint{2.969787in}{1.150992in}}{\pgfqpoint{2.961973in}{1.158805in}}%
\pgfpathcurveto{\pgfqpoint{2.954160in}{1.166619in}}{\pgfqpoint{2.943561in}{1.171009in}}{\pgfqpoint{2.932510in}{1.171009in}}%
\pgfpathcurveto{\pgfqpoint{2.921460in}{1.171009in}}{\pgfqpoint{2.910861in}{1.166619in}}{\pgfqpoint{2.903048in}{1.158805in}}%
\pgfpathcurveto{\pgfqpoint{2.895234in}{1.150992in}}{\pgfqpoint{2.890844in}{1.140393in}}{\pgfqpoint{2.890844in}{1.129343in}}%
\pgfpathcurveto{\pgfqpoint{2.890844in}{1.118292in}}{\pgfqpoint{2.895234in}{1.107693in}}{\pgfqpoint{2.903048in}{1.099880in}}%
\pgfpathcurveto{\pgfqpoint{2.910861in}{1.092066in}}{\pgfqpoint{2.921460in}{1.087676in}}{\pgfqpoint{2.932510in}{1.087676in}}%
\pgfpathlineto{\pgfqpoint{2.932510in}{1.087676in}}%
\pgfpathclose%
\pgfusepath{stroke}%
\end{pgfscope}%
\begin{pgfscope}%
\pgfpathrectangle{\pgfqpoint{0.847223in}{0.554012in}}{\pgfqpoint{6.200000in}{4.530000in}}%
\pgfusepath{clip}%
\pgfsetbuttcap%
\pgfsetroundjoin%
\pgfsetlinewidth{1.003750pt}%
\definecolor{currentstroke}{rgb}{1.000000,0.000000,0.000000}%
\pgfsetstrokecolor{currentstroke}%
\pgfsetdash{}{0pt}%
\pgfpathmoveto{\pgfqpoint{2.937844in}{1.085651in}}%
\pgfpathcurveto{\pgfqpoint{2.948894in}{1.085651in}}{\pgfqpoint{2.959493in}{1.090041in}}{\pgfqpoint{2.967306in}{1.097855in}}%
\pgfpathcurveto{\pgfqpoint{2.975120in}{1.105668in}}{\pgfqpoint{2.979510in}{1.116267in}}{\pgfqpoint{2.979510in}{1.127318in}}%
\pgfpathcurveto{\pgfqpoint{2.979510in}{1.138368in}}{\pgfqpoint{2.975120in}{1.148967in}}{\pgfqpoint{2.967306in}{1.156780in}}%
\pgfpathcurveto{\pgfqpoint{2.959493in}{1.164594in}}{\pgfqpoint{2.948894in}{1.168984in}}{\pgfqpoint{2.937844in}{1.168984in}}%
\pgfpathcurveto{\pgfqpoint{2.926793in}{1.168984in}}{\pgfqpoint{2.916194in}{1.164594in}}{\pgfqpoint{2.908381in}{1.156780in}}%
\pgfpathcurveto{\pgfqpoint{2.900567in}{1.148967in}}{\pgfqpoint{2.896177in}{1.138368in}}{\pgfqpoint{2.896177in}{1.127318in}}%
\pgfpathcurveto{\pgfqpoint{2.896177in}{1.116267in}}{\pgfqpoint{2.900567in}{1.105668in}}{\pgfqpoint{2.908381in}{1.097855in}}%
\pgfpathcurveto{\pgfqpoint{2.916194in}{1.090041in}}{\pgfqpoint{2.926793in}{1.085651in}}{\pgfqpoint{2.937844in}{1.085651in}}%
\pgfpathlineto{\pgfqpoint{2.937844in}{1.085651in}}%
\pgfpathclose%
\pgfusepath{stroke}%
\end{pgfscope}%
\begin{pgfscope}%
\pgfpathrectangle{\pgfqpoint{0.847223in}{0.554012in}}{\pgfqpoint{6.200000in}{4.530000in}}%
\pgfusepath{clip}%
\pgfsetbuttcap%
\pgfsetroundjoin%
\pgfsetlinewidth{1.003750pt}%
\definecolor{currentstroke}{rgb}{1.000000,0.000000,0.000000}%
\pgfsetstrokecolor{currentstroke}%
\pgfsetdash{}{0pt}%
\pgfpathmoveto{\pgfqpoint{2.943177in}{1.083634in}}%
\pgfpathcurveto{\pgfqpoint{2.954227in}{1.083634in}}{\pgfqpoint{2.964826in}{1.088024in}}{\pgfqpoint{2.972640in}{1.095838in}}%
\pgfpathcurveto{\pgfqpoint{2.980453in}{1.103652in}}{\pgfqpoint{2.984843in}{1.114251in}}{\pgfqpoint{2.984843in}{1.125301in}}%
\pgfpathcurveto{\pgfqpoint{2.984843in}{1.136351in}}{\pgfqpoint{2.980453in}{1.146950in}}{\pgfqpoint{2.972640in}{1.154764in}}%
\pgfpathcurveto{\pgfqpoint{2.964826in}{1.162577in}}{\pgfqpoint{2.954227in}{1.166968in}}{\pgfqpoint{2.943177in}{1.166968in}}%
\pgfpathcurveto{\pgfqpoint{2.932127in}{1.166968in}}{\pgfqpoint{2.921528in}{1.162577in}}{\pgfqpoint{2.913714in}{1.154764in}}%
\pgfpathcurveto{\pgfqpoint{2.905900in}{1.146950in}}{\pgfqpoint{2.901510in}{1.136351in}}{\pgfqpoint{2.901510in}{1.125301in}}%
\pgfpathcurveto{\pgfqpoint{2.901510in}{1.114251in}}{\pgfqpoint{2.905900in}{1.103652in}}{\pgfqpoint{2.913714in}{1.095838in}}%
\pgfpathcurveto{\pgfqpoint{2.921528in}{1.088024in}}{\pgfqpoint{2.932127in}{1.083634in}}{\pgfqpoint{2.943177in}{1.083634in}}%
\pgfpathlineto{\pgfqpoint{2.943177in}{1.083634in}}%
\pgfpathclose%
\pgfusepath{stroke}%
\end{pgfscope}%
\begin{pgfscope}%
\pgfpathrectangle{\pgfqpoint{0.847223in}{0.554012in}}{\pgfqpoint{6.200000in}{4.530000in}}%
\pgfusepath{clip}%
\pgfsetbuttcap%
\pgfsetroundjoin%
\pgfsetlinewidth{1.003750pt}%
\definecolor{currentstroke}{rgb}{1.000000,0.000000,0.000000}%
\pgfsetstrokecolor{currentstroke}%
\pgfsetdash{}{0pt}%
\pgfpathmoveto{\pgfqpoint{2.948510in}{1.081626in}}%
\pgfpathcurveto{\pgfqpoint{2.959560in}{1.081626in}}{\pgfqpoint{2.970159in}{1.086016in}}{\pgfqpoint{2.977973in}{1.093830in}}%
\pgfpathcurveto{\pgfqpoint{2.985786in}{1.101643in}}{\pgfqpoint{2.990177in}{1.112242in}}{\pgfqpoint{2.990177in}{1.123292in}}%
\pgfpathcurveto{\pgfqpoint{2.990177in}{1.134342in}}{\pgfqpoint{2.985786in}{1.144941in}}{\pgfqpoint{2.977973in}{1.152755in}}%
\pgfpathcurveto{\pgfqpoint{2.970159in}{1.160569in}}{\pgfqpoint{2.959560in}{1.164959in}}{\pgfqpoint{2.948510in}{1.164959in}}%
\pgfpathcurveto{\pgfqpoint{2.937460in}{1.164959in}}{\pgfqpoint{2.926861in}{1.160569in}}{\pgfqpoint{2.919047in}{1.152755in}}%
\pgfpathcurveto{\pgfqpoint{2.911234in}{1.144941in}}{\pgfqpoint{2.906843in}{1.134342in}}{\pgfqpoint{2.906843in}{1.123292in}}%
\pgfpathcurveto{\pgfqpoint{2.906843in}{1.112242in}}{\pgfqpoint{2.911234in}{1.101643in}}{\pgfqpoint{2.919047in}{1.093830in}}%
\pgfpathcurveto{\pgfqpoint{2.926861in}{1.086016in}}{\pgfqpoint{2.937460in}{1.081626in}}{\pgfqpoint{2.948510in}{1.081626in}}%
\pgfpathlineto{\pgfqpoint{2.948510in}{1.081626in}}%
\pgfpathclose%
\pgfusepath{stroke}%
\end{pgfscope}%
\begin{pgfscope}%
\pgfpathrectangle{\pgfqpoint{0.847223in}{0.554012in}}{\pgfqpoint{6.200000in}{4.530000in}}%
\pgfusepath{clip}%
\pgfsetbuttcap%
\pgfsetroundjoin%
\pgfsetlinewidth{1.003750pt}%
\definecolor{currentstroke}{rgb}{1.000000,0.000000,0.000000}%
\pgfsetstrokecolor{currentstroke}%
\pgfsetdash{}{0pt}%
\pgfpathmoveto{\pgfqpoint{2.953843in}{1.079625in}}%
\pgfpathcurveto{\pgfqpoint{2.964893in}{1.079625in}}{\pgfqpoint{2.975492in}{1.084016in}}{\pgfqpoint{2.983306in}{1.091829in}}%
\pgfpathcurveto{\pgfqpoint{2.991120in}{1.099643in}}{\pgfqpoint{2.995510in}{1.110242in}}{\pgfqpoint{2.995510in}{1.121292in}}%
\pgfpathcurveto{\pgfqpoint{2.995510in}{1.132342in}}{\pgfqpoint{2.991120in}{1.142941in}}{\pgfqpoint{2.983306in}{1.150755in}}%
\pgfpathcurveto{\pgfqpoint{2.975492in}{1.158568in}}{\pgfqpoint{2.964893in}{1.162959in}}{\pgfqpoint{2.953843in}{1.162959in}}%
\pgfpathcurveto{\pgfqpoint{2.942793in}{1.162959in}}{\pgfqpoint{2.932194in}{1.158568in}}{\pgfqpoint{2.924380in}{1.150755in}}%
\pgfpathcurveto{\pgfqpoint{2.916567in}{1.142941in}}{\pgfqpoint{2.912177in}{1.132342in}}{\pgfqpoint{2.912177in}{1.121292in}}%
\pgfpathcurveto{\pgfqpoint{2.912177in}{1.110242in}}{\pgfqpoint{2.916567in}{1.099643in}}{\pgfqpoint{2.924380in}{1.091829in}}%
\pgfpathcurveto{\pgfqpoint{2.932194in}{1.084016in}}{\pgfqpoint{2.942793in}{1.079625in}}{\pgfqpoint{2.953843in}{1.079625in}}%
\pgfpathlineto{\pgfqpoint{2.953843in}{1.079625in}}%
\pgfpathclose%
\pgfusepath{stroke}%
\end{pgfscope}%
\begin{pgfscope}%
\pgfpathrectangle{\pgfqpoint{0.847223in}{0.554012in}}{\pgfqpoint{6.200000in}{4.530000in}}%
\pgfusepath{clip}%
\pgfsetbuttcap%
\pgfsetroundjoin%
\pgfsetlinewidth{1.003750pt}%
\definecolor{currentstroke}{rgb}{1.000000,0.000000,0.000000}%
\pgfsetstrokecolor{currentstroke}%
\pgfsetdash{}{0pt}%
\pgfpathmoveto{\pgfqpoint{2.959176in}{1.077633in}}%
\pgfpathcurveto{\pgfqpoint{2.970227in}{1.077633in}}{\pgfqpoint{2.980826in}{1.082023in}}{\pgfqpoint{2.988639in}{1.089837in}}%
\pgfpathcurveto{\pgfqpoint{2.996453in}{1.097650in}}{\pgfqpoint{3.000843in}{1.108249in}}{\pgfqpoint{3.000843in}{1.119300in}}%
\pgfpathcurveto{\pgfqpoint{3.000843in}{1.130350in}}{\pgfqpoint{2.996453in}{1.140949in}}{\pgfqpoint{2.988639in}{1.148762in}}%
\pgfpathcurveto{\pgfqpoint{2.980826in}{1.156576in}}{\pgfqpoint{2.970227in}{1.160966in}}{\pgfqpoint{2.959176in}{1.160966in}}%
\pgfpathcurveto{\pgfqpoint{2.948126in}{1.160966in}}{\pgfqpoint{2.937527in}{1.156576in}}{\pgfqpoint{2.929714in}{1.148762in}}%
\pgfpathcurveto{\pgfqpoint{2.921900in}{1.140949in}}{\pgfqpoint{2.917510in}{1.130350in}}{\pgfqpoint{2.917510in}{1.119300in}}%
\pgfpathcurveto{\pgfqpoint{2.917510in}{1.108249in}}{\pgfqpoint{2.921900in}{1.097650in}}{\pgfqpoint{2.929714in}{1.089837in}}%
\pgfpathcurveto{\pgfqpoint{2.937527in}{1.082023in}}{\pgfqpoint{2.948126in}{1.077633in}}{\pgfqpoint{2.959176in}{1.077633in}}%
\pgfpathlineto{\pgfqpoint{2.959176in}{1.077633in}}%
\pgfpathclose%
\pgfusepath{stroke}%
\end{pgfscope}%
\begin{pgfscope}%
\pgfpathrectangle{\pgfqpoint{0.847223in}{0.554012in}}{\pgfqpoint{6.200000in}{4.530000in}}%
\pgfusepath{clip}%
\pgfsetbuttcap%
\pgfsetroundjoin%
\pgfsetlinewidth{1.003750pt}%
\definecolor{currentstroke}{rgb}{1.000000,0.000000,0.000000}%
\pgfsetstrokecolor{currentstroke}%
\pgfsetdash{}{0pt}%
\pgfpathmoveto{\pgfqpoint{2.964510in}{1.075649in}}%
\pgfpathcurveto{\pgfqpoint{2.975560in}{1.075649in}}{\pgfqpoint{2.986159in}{1.080039in}}{\pgfqpoint{2.993972in}{1.087853in}}%
\pgfpathcurveto{\pgfqpoint{3.001786in}{1.095666in}}{\pgfqpoint{3.006176in}{1.106265in}}{\pgfqpoint{3.006176in}{1.117315in}}%
\pgfpathcurveto{\pgfqpoint{3.006176in}{1.128365in}}{\pgfqpoint{3.001786in}{1.138964in}}{\pgfqpoint{2.993972in}{1.146778in}}%
\pgfpathcurveto{\pgfqpoint{2.986159in}{1.154592in}}{\pgfqpoint{2.975560in}{1.158982in}}{\pgfqpoint{2.964510in}{1.158982in}}%
\pgfpathcurveto{\pgfqpoint{2.953460in}{1.158982in}}{\pgfqpoint{2.942861in}{1.154592in}}{\pgfqpoint{2.935047in}{1.146778in}}%
\pgfpathcurveto{\pgfqpoint{2.927233in}{1.138964in}}{\pgfqpoint{2.922843in}{1.128365in}}{\pgfqpoint{2.922843in}{1.117315in}}%
\pgfpathcurveto{\pgfqpoint{2.922843in}{1.106265in}}{\pgfqpoint{2.927233in}{1.095666in}}{\pgfqpoint{2.935047in}{1.087853in}}%
\pgfpathcurveto{\pgfqpoint{2.942861in}{1.080039in}}{\pgfqpoint{2.953460in}{1.075649in}}{\pgfqpoint{2.964510in}{1.075649in}}%
\pgfpathlineto{\pgfqpoint{2.964510in}{1.075649in}}%
\pgfpathclose%
\pgfusepath{stroke}%
\end{pgfscope}%
\begin{pgfscope}%
\pgfpathrectangle{\pgfqpoint{0.847223in}{0.554012in}}{\pgfqpoint{6.200000in}{4.530000in}}%
\pgfusepath{clip}%
\pgfsetbuttcap%
\pgfsetroundjoin%
\pgfsetlinewidth{1.003750pt}%
\definecolor{currentstroke}{rgb}{1.000000,0.000000,0.000000}%
\pgfsetstrokecolor{currentstroke}%
\pgfsetdash{}{0pt}%
\pgfpathmoveto{\pgfqpoint{2.969843in}{1.073672in}}%
\pgfpathcurveto{\pgfqpoint{2.980893in}{1.073672in}}{\pgfqpoint{2.991492in}{1.078063in}}{\pgfqpoint{2.999306in}{1.085876in}}%
\pgfpathcurveto{\pgfqpoint{3.007119in}{1.093690in}}{\pgfqpoint{3.011510in}{1.104289in}}{\pgfqpoint{3.011510in}{1.115339in}}%
\pgfpathcurveto{\pgfqpoint{3.011510in}{1.126389in}}{\pgfqpoint{3.007119in}{1.136988in}}{\pgfqpoint{2.999306in}{1.144802in}}%
\pgfpathcurveto{\pgfqpoint{2.991492in}{1.152615in}}{\pgfqpoint{2.980893in}{1.157006in}}{\pgfqpoint{2.969843in}{1.157006in}}%
\pgfpathcurveto{\pgfqpoint{2.958793in}{1.157006in}}{\pgfqpoint{2.948194in}{1.152615in}}{\pgfqpoint{2.940380in}{1.144802in}}%
\pgfpathcurveto{\pgfqpoint{2.932566in}{1.136988in}}{\pgfqpoint{2.928176in}{1.126389in}}{\pgfqpoint{2.928176in}{1.115339in}}%
\pgfpathcurveto{\pgfqpoint{2.928176in}{1.104289in}}{\pgfqpoint{2.932566in}{1.093690in}}{\pgfqpoint{2.940380in}{1.085876in}}%
\pgfpathcurveto{\pgfqpoint{2.948194in}{1.078063in}}{\pgfqpoint{2.958793in}{1.073672in}}{\pgfqpoint{2.969843in}{1.073672in}}%
\pgfpathlineto{\pgfqpoint{2.969843in}{1.073672in}}%
\pgfpathclose%
\pgfusepath{stroke}%
\end{pgfscope}%
\begin{pgfscope}%
\pgfpathrectangle{\pgfqpoint{0.847223in}{0.554012in}}{\pgfqpoint{6.200000in}{4.530000in}}%
\pgfusepath{clip}%
\pgfsetbuttcap%
\pgfsetroundjoin%
\pgfsetlinewidth{1.003750pt}%
\definecolor{currentstroke}{rgb}{1.000000,0.000000,0.000000}%
\pgfsetstrokecolor{currentstroke}%
\pgfsetdash{}{0pt}%
\pgfpathmoveto{\pgfqpoint{2.975176in}{1.071704in}}%
\pgfpathcurveto{\pgfqpoint{2.986226in}{1.071704in}}{\pgfqpoint{2.996825in}{1.076094in}}{\pgfqpoint{3.004639in}{1.083908in}}%
\pgfpathcurveto{\pgfqpoint{3.012453in}{1.091722in}}{\pgfqpoint{3.016843in}{1.102321in}}{\pgfqpoint{3.016843in}{1.113371in}}%
\pgfpathcurveto{\pgfqpoint{3.016843in}{1.124421in}}{\pgfqpoint{3.012453in}{1.135020in}}{\pgfqpoint{3.004639in}{1.142834in}}%
\pgfpathcurveto{\pgfqpoint{2.996825in}{1.150647in}}{\pgfqpoint{2.986226in}{1.155037in}}{\pgfqpoint{2.975176in}{1.155037in}}%
\pgfpathcurveto{\pgfqpoint{2.964126in}{1.155037in}}{\pgfqpoint{2.953527in}{1.150647in}}{\pgfqpoint{2.945713in}{1.142834in}}%
\pgfpathcurveto{\pgfqpoint{2.937900in}{1.135020in}}{\pgfqpoint{2.933509in}{1.124421in}}{\pgfqpoint{2.933509in}{1.113371in}}%
\pgfpathcurveto{\pgfqpoint{2.933509in}{1.102321in}}{\pgfqpoint{2.937900in}{1.091722in}}{\pgfqpoint{2.945713in}{1.083908in}}%
\pgfpathcurveto{\pgfqpoint{2.953527in}{1.076094in}}{\pgfqpoint{2.964126in}{1.071704in}}{\pgfqpoint{2.975176in}{1.071704in}}%
\pgfpathlineto{\pgfqpoint{2.975176in}{1.071704in}}%
\pgfpathclose%
\pgfusepath{stroke}%
\end{pgfscope}%
\begin{pgfscope}%
\pgfpathrectangle{\pgfqpoint{0.847223in}{0.554012in}}{\pgfqpoint{6.200000in}{4.530000in}}%
\pgfusepath{clip}%
\pgfsetbuttcap%
\pgfsetroundjoin%
\pgfsetlinewidth{1.003750pt}%
\definecolor{currentstroke}{rgb}{1.000000,0.000000,0.000000}%
\pgfsetstrokecolor{currentstroke}%
\pgfsetdash{}{0pt}%
\pgfpathmoveto{\pgfqpoint{2.980509in}{1.069744in}}%
\pgfpathcurveto{\pgfqpoint{2.991559in}{1.069744in}}{\pgfqpoint{3.002158in}{1.074134in}}{\pgfqpoint{3.009972in}{1.081948in}}%
\pgfpathcurveto{\pgfqpoint{3.017786in}{1.089761in}}{\pgfqpoint{3.022176in}{1.100360in}}{\pgfqpoint{3.022176in}{1.111410in}}%
\pgfpathcurveto{\pgfqpoint{3.022176in}{1.122460in}}{\pgfqpoint{3.017786in}{1.133059in}}{\pgfqpoint{3.009972in}{1.140873in}}%
\pgfpathcurveto{\pgfqpoint{3.002158in}{1.148687in}}{\pgfqpoint{2.991559in}{1.153077in}}{\pgfqpoint{2.980509in}{1.153077in}}%
\pgfpathcurveto{\pgfqpoint{2.969459in}{1.153077in}}{\pgfqpoint{2.958860in}{1.148687in}}{\pgfqpoint{2.951047in}{1.140873in}}%
\pgfpathcurveto{\pgfqpoint{2.943233in}{1.133059in}}{\pgfqpoint{2.938843in}{1.122460in}}{\pgfqpoint{2.938843in}{1.111410in}}%
\pgfpathcurveto{\pgfqpoint{2.938843in}{1.100360in}}{\pgfqpoint{2.943233in}{1.089761in}}{\pgfqpoint{2.951047in}{1.081948in}}%
\pgfpathcurveto{\pgfqpoint{2.958860in}{1.074134in}}{\pgfqpoint{2.969459in}{1.069744in}}{\pgfqpoint{2.980509in}{1.069744in}}%
\pgfpathlineto{\pgfqpoint{2.980509in}{1.069744in}}%
\pgfpathclose%
\pgfusepath{stroke}%
\end{pgfscope}%
\begin{pgfscope}%
\pgfpathrectangle{\pgfqpoint{0.847223in}{0.554012in}}{\pgfqpoint{6.200000in}{4.530000in}}%
\pgfusepath{clip}%
\pgfsetbuttcap%
\pgfsetroundjoin%
\pgfsetlinewidth{1.003750pt}%
\definecolor{currentstroke}{rgb}{1.000000,0.000000,0.000000}%
\pgfsetstrokecolor{currentstroke}%
\pgfsetdash{}{0pt}%
\pgfpathmoveto{\pgfqpoint{2.985843in}{1.067791in}}%
\pgfpathcurveto{\pgfqpoint{2.996893in}{1.067791in}}{\pgfqpoint{3.007492in}{1.072181in}}{\pgfqpoint{3.015305in}{1.079995in}}%
\pgfpathcurveto{\pgfqpoint{3.023119in}{1.087809in}}{\pgfqpoint{3.027509in}{1.098408in}}{\pgfqpoint{3.027509in}{1.109458in}}%
\pgfpathcurveto{\pgfqpoint{3.027509in}{1.120508in}}{\pgfqpoint{3.023119in}{1.131107in}}{\pgfqpoint{3.015305in}{1.138921in}}%
\pgfpathcurveto{\pgfqpoint{3.007492in}{1.146734in}}{\pgfqpoint{2.996893in}{1.151124in}}{\pgfqpoint{2.985843in}{1.151124in}}%
\pgfpathcurveto{\pgfqpoint{2.974792in}{1.151124in}}{\pgfqpoint{2.964193in}{1.146734in}}{\pgfqpoint{2.956380in}{1.138921in}}%
\pgfpathcurveto{\pgfqpoint{2.948566in}{1.131107in}}{\pgfqpoint{2.944176in}{1.120508in}}{\pgfqpoint{2.944176in}{1.109458in}}%
\pgfpathcurveto{\pgfqpoint{2.944176in}{1.098408in}}{\pgfqpoint{2.948566in}{1.087809in}}{\pgfqpoint{2.956380in}{1.079995in}}%
\pgfpathcurveto{\pgfqpoint{2.964193in}{1.072181in}}{\pgfqpoint{2.974792in}{1.067791in}}{\pgfqpoint{2.985843in}{1.067791in}}%
\pgfpathlineto{\pgfqpoint{2.985843in}{1.067791in}}%
\pgfpathclose%
\pgfusepath{stroke}%
\end{pgfscope}%
\begin{pgfscope}%
\pgfpathrectangle{\pgfqpoint{0.847223in}{0.554012in}}{\pgfqpoint{6.200000in}{4.530000in}}%
\pgfusepath{clip}%
\pgfsetbuttcap%
\pgfsetroundjoin%
\pgfsetlinewidth{1.003750pt}%
\definecolor{currentstroke}{rgb}{1.000000,0.000000,0.000000}%
\pgfsetstrokecolor{currentstroke}%
\pgfsetdash{}{0pt}%
\pgfpathmoveto{\pgfqpoint{2.991176in}{1.065846in}}%
\pgfpathcurveto{\pgfqpoint{3.002226in}{1.065846in}}{\pgfqpoint{3.012825in}{1.070237in}}{\pgfqpoint{3.020639in}{1.078050in}}%
\pgfpathcurveto{\pgfqpoint{3.028452in}{1.085864in}}{\pgfqpoint{3.032842in}{1.096463in}}{\pgfqpoint{3.032842in}{1.107513in}}%
\pgfpathcurveto{\pgfqpoint{3.032842in}{1.118563in}}{\pgfqpoint{3.028452in}{1.129162in}}{\pgfqpoint{3.020639in}{1.136976in}}%
\pgfpathcurveto{\pgfqpoint{3.012825in}{1.144789in}}{\pgfqpoint{3.002226in}{1.149180in}}{\pgfqpoint{2.991176in}{1.149180in}}%
\pgfpathcurveto{\pgfqpoint{2.980126in}{1.149180in}}{\pgfqpoint{2.969527in}{1.144789in}}{\pgfqpoint{2.961713in}{1.136976in}}%
\pgfpathcurveto{\pgfqpoint{2.953899in}{1.129162in}}{\pgfqpoint{2.949509in}{1.118563in}}{\pgfqpoint{2.949509in}{1.107513in}}%
\pgfpathcurveto{\pgfqpoint{2.949509in}{1.096463in}}{\pgfqpoint{2.953899in}{1.085864in}}{\pgfqpoint{2.961713in}{1.078050in}}%
\pgfpathcurveto{\pgfqpoint{2.969527in}{1.070237in}}{\pgfqpoint{2.980126in}{1.065846in}}{\pgfqpoint{2.991176in}{1.065846in}}%
\pgfpathlineto{\pgfqpoint{2.991176in}{1.065846in}}%
\pgfpathclose%
\pgfusepath{stroke}%
\end{pgfscope}%
\begin{pgfscope}%
\pgfpathrectangle{\pgfqpoint{0.847223in}{0.554012in}}{\pgfqpoint{6.200000in}{4.530000in}}%
\pgfusepath{clip}%
\pgfsetbuttcap%
\pgfsetroundjoin%
\pgfsetlinewidth{1.003750pt}%
\definecolor{currentstroke}{rgb}{1.000000,0.000000,0.000000}%
\pgfsetstrokecolor{currentstroke}%
\pgfsetdash{}{0pt}%
\pgfpathmoveto{\pgfqpoint{2.996509in}{1.063909in}}%
\pgfpathcurveto{\pgfqpoint{3.007559in}{1.063909in}}{\pgfqpoint{3.018158in}{1.068300in}}{\pgfqpoint{3.025972in}{1.076113in}}%
\pgfpathcurveto{\pgfqpoint{3.033785in}{1.083927in}}{\pgfqpoint{3.038176in}{1.094526in}}{\pgfqpoint{3.038176in}{1.105576in}}%
\pgfpathcurveto{\pgfqpoint{3.038176in}{1.116626in}}{\pgfqpoint{3.033785in}{1.127225in}}{\pgfqpoint{3.025972in}{1.135039in}}%
\pgfpathcurveto{\pgfqpoint{3.018158in}{1.142852in}}{\pgfqpoint{3.007559in}{1.147243in}}{\pgfqpoint{2.996509in}{1.147243in}}%
\pgfpathcurveto{\pgfqpoint{2.985459in}{1.147243in}}{\pgfqpoint{2.974860in}{1.142852in}}{\pgfqpoint{2.967046in}{1.135039in}}%
\pgfpathcurveto{\pgfqpoint{2.959233in}{1.127225in}}{\pgfqpoint{2.954842in}{1.116626in}}{\pgfqpoint{2.954842in}{1.105576in}}%
\pgfpathcurveto{\pgfqpoint{2.954842in}{1.094526in}}{\pgfqpoint{2.959233in}{1.083927in}}{\pgfqpoint{2.967046in}{1.076113in}}%
\pgfpathcurveto{\pgfqpoint{2.974860in}{1.068300in}}{\pgfqpoint{2.985459in}{1.063909in}}{\pgfqpoint{2.996509in}{1.063909in}}%
\pgfpathlineto{\pgfqpoint{2.996509in}{1.063909in}}%
\pgfpathclose%
\pgfusepath{stroke}%
\end{pgfscope}%
\begin{pgfscope}%
\pgfpathrectangle{\pgfqpoint{0.847223in}{0.554012in}}{\pgfqpoint{6.200000in}{4.530000in}}%
\pgfusepath{clip}%
\pgfsetbuttcap%
\pgfsetroundjoin%
\pgfsetlinewidth{1.003750pt}%
\definecolor{currentstroke}{rgb}{1.000000,0.000000,0.000000}%
\pgfsetstrokecolor{currentstroke}%
\pgfsetdash{}{0pt}%
\pgfpathmoveto{\pgfqpoint{3.001842in}{1.061980in}}%
\pgfpathcurveto{\pgfqpoint{3.012892in}{1.061980in}}{\pgfqpoint{3.023491in}{1.066370in}}{\pgfqpoint{3.031305in}{1.074184in}}%
\pgfpathcurveto{\pgfqpoint{3.039119in}{1.081998in}}{\pgfqpoint{3.043509in}{1.092597in}}{\pgfqpoint{3.043509in}{1.103647in}}%
\pgfpathcurveto{\pgfqpoint{3.043509in}{1.114697in}}{\pgfqpoint{3.039119in}{1.125296in}}{\pgfqpoint{3.031305in}{1.133110in}}%
\pgfpathcurveto{\pgfqpoint{3.023491in}{1.140923in}}{\pgfqpoint{3.012892in}{1.145313in}}{\pgfqpoint{3.001842in}{1.145313in}}%
\pgfpathcurveto{\pgfqpoint{2.990792in}{1.145313in}}{\pgfqpoint{2.980193in}{1.140923in}}{\pgfqpoint{2.972379in}{1.133110in}}%
\pgfpathcurveto{\pgfqpoint{2.964566in}{1.125296in}}{\pgfqpoint{2.960176in}{1.114697in}}{\pgfqpoint{2.960176in}{1.103647in}}%
\pgfpathcurveto{\pgfqpoint{2.960176in}{1.092597in}}{\pgfqpoint{2.964566in}{1.081998in}}{\pgfqpoint{2.972379in}{1.074184in}}%
\pgfpathcurveto{\pgfqpoint{2.980193in}{1.066370in}}{\pgfqpoint{2.990792in}{1.061980in}}{\pgfqpoint{3.001842in}{1.061980in}}%
\pgfpathlineto{\pgfqpoint{3.001842in}{1.061980in}}%
\pgfpathclose%
\pgfusepath{stroke}%
\end{pgfscope}%
\begin{pgfscope}%
\pgfpathrectangle{\pgfqpoint{0.847223in}{0.554012in}}{\pgfqpoint{6.200000in}{4.530000in}}%
\pgfusepath{clip}%
\pgfsetbuttcap%
\pgfsetroundjoin%
\pgfsetlinewidth{1.003750pt}%
\definecolor{currentstroke}{rgb}{1.000000,0.000000,0.000000}%
\pgfsetstrokecolor{currentstroke}%
\pgfsetdash{}{0pt}%
\pgfpathmoveto{\pgfqpoint{3.007175in}{1.060058in}}%
\pgfpathcurveto{\pgfqpoint{3.018226in}{1.060058in}}{\pgfqpoint{3.028825in}{1.064449in}}{\pgfqpoint{3.036638in}{1.072262in}}%
\pgfpathcurveto{\pgfqpoint{3.044452in}{1.080076in}}{\pgfqpoint{3.048842in}{1.090675in}}{\pgfqpoint{3.048842in}{1.101725in}}%
\pgfpathcurveto{\pgfqpoint{3.048842in}{1.112775in}}{\pgfqpoint{3.044452in}{1.123374in}}{\pgfqpoint{3.036638in}{1.131188in}}%
\pgfpathcurveto{\pgfqpoint{3.028825in}{1.139002in}}{\pgfqpoint{3.018226in}{1.143392in}}{\pgfqpoint{3.007175in}{1.143392in}}%
\pgfpathcurveto{\pgfqpoint{2.996125in}{1.143392in}}{\pgfqpoint{2.985526in}{1.139002in}}{\pgfqpoint{2.977713in}{1.131188in}}%
\pgfpathcurveto{\pgfqpoint{2.969899in}{1.123374in}}{\pgfqpoint{2.965509in}{1.112775in}}{\pgfqpoint{2.965509in}{1.101725in}}%
\pgfpathcurveto{\pgfqpoint{2.965509in}{1.090675in}}{\pgfqpoint{2.969899in}{1.080076in}}{\pgfqpoint{2.977713in}{1.072262in}}%
\pgfpathcurveto{\pgfqpoint{2.985526in}{1.064449in}}{\pgfqpoint{2.996125in}{1.060058in}}{\pgfqpoint{3.007175in}{1.060058in}}%
\pgfpathlineto{\pgfqpoint{3.007175in}{1.060058in}}%
\pgfpathclose%
\pgfusepath{stroke}%
\end{pgfscope}%
\begin{pgfscope}%
\pgfpathrectangle{\pgfqpoint{0.847223in}{0.554012in}}{\pgfqpoint{6.200000in}{4.530000in}}%
\pgfusepath{clip}%
\pgfsetbuttcap%
\pgfsetroundjoin%
\pgfsetlinewidth{1.003750pt}%
\definecolor{currentstroke}{rgb}{1.000000,0.000000,0.000000}%
\pgfsetstrokecolor{currentstroke}%
\pgfsetdash{}{0pt}%
\pgfpathmoveto{\pgfqpoint{3.012509in}{1.058144in}}%
\pgfpathcurveto{\pgfqpoint{3.023559in}{1.058144in}}{\pgfqpoint{3.034158in}{1.062535in}}{\pgfqpoint{3.041971in}{1.070348in}}%
\pgfpathcurveto{\pgfqpoint{3.049785in}{1.078162in}}{\pgfqpoint{3.054175in}{1.088761in}}{\pgfqpoint{3.054175in}{1.099811in}}%
\pgfpathcurveto{\pgfqpoint{3.054175in}{1.110861in}}{\pgfqpoint{3.049785in}{1.121460in}}{\pgfqpoint{3.041971in}{1.129274in}}%
\pgfpathcurveto{\pgfqpoint{3.034158in}{1.137088in}}{\pgfqpoint{3.023559in}{1.141478in}}{\pgfqpoint{3.012509in}{1.141478in}}%
\pgfpathcurveto{\pgfqpoint{3.001458in}{1.141478in}}{\pgfqpoint{2.990859in}{1.137088in}}{\pgfqpoint{2.983046in}{1.129274in}}%
\pgfpathcurveto{\pgfqpoint{2.975232in}{1.121460in}}{\pgfqpoint{2.970842in}{1.110861in}}{\pgfqpoint{2.970842in}{1.099811in}}%
\pgfpathcurveto{\pgfqpoint{2.970842in}{1.088761in}}{\pgfqpoint{2.975232in}{1.078162in}}{\pgfqpoint{2.983046in}{1.070348in}}%
\pgfpathcurveto{\pgfqpoint{2.990859in}{1.062535in}}{\pgfqpoint{3.001458in}{1.058144in}}{\pgfqpoint{3.012509in}{1.058144in}}%
\pgfpathlineto{\pgfqpoint{3.012509in}{1.058144in}}%
\pgfpathclose%
\pgfusepath{stroke}%
\end{pgfscope}%
\begin{pgfscope}%
\pgfpathrectangle{\pgfqpoint{0.847223in}{0.554012in}}{\pgfqpoint{6.200000in}{4.530000in}}%
\pgfusepath{clip}%
\pgfsetbuttcap%
\pgfsetroundjoin%
\pgfsetlinewidth{1.003750pt}%
\definecolor{currentstroke}{rgb}{1.000000,0.000000,0.000000}%
\pgfsetstrokecolor{currentstroke}%
\pgfsetdash{}{0pt}%
\pgfpathmoveto{\pgfqpoint{3.017842in}{1.056238in}}%
\pgfpathcurveto{\pgfqpoint{3.028892in}{1.056238in}}{\pgfqpoint{3.039491in}{1.060628in}}{\pgfqpoint{3.047305in}{1.068442in}}%
\pgfpathcurveto{\pgfqpoint{3.055118in}{1.076256in}}{\pgfqpoint{3.059508in}{1.086855in}}{\pgfqpoint{3.059508in}{1.097905in}}%
\pgfpathcurveto{\pgfqpoint{3.059508in}{1.108955in}}{\pgfqpoint{3.055118in}{1.119554in}}{\pgfqpoint{3.047305in}{1.127368in}}%
\pgfpathcurveto{\pgfqpoint{3.039491in}{1.135181in}}{\pgfqpoint{3.028892in}{1.139571in}}{\pgfqpoint{3.017842in}{1.139571in}}%
\pgfpathcurveto{\pgfqpoint{3.006792in}{1.139571in}}{\pgfqpoint{2.996193in}{1.135181in}}{\pgfqpoint{2.988379in}{1.127368in}}%
\pgfpathcurveto{\pgfqpoint{2.980565in}{1.119554in}}{\pgfqpoint{2.976175in}{1.108955in}}{\pgfqpoint{2.976175in}{1.097905in}}%
\pgfpathcurveto{\pgfqpoint{2.976175in}{1.086855in}}{\pgfqpoint{2.980565in}{1.076256in}}{\pgfqpoint{2.988379in}{1.068442in}}%
\pgfpathcurveto{\pgfqpoint{2.996193in}{1.060628in}}{\pgfqpoint{3.006792in}{1.056238in}}{\pgfqpoint{3.017842in}{1.056238in}}%
\pgfpathlineto{\pgfqpoint{3.017842in}{1.056238in}}%
\pgfpathclose%
\pgfusepath{stroke}%
\end{pgfscope}%
\begin{pgfscope}%
\pgfpathrectangle{\pgfqpoint{0.847223in}{0.554012in}}{\pgfqpoint{6.200000in}{4.530000in}}%
\pgfusepath{clip}%
\pgfsetbuttcap%
\pgfsetroundjoin%
\pgfsetlinewidth{1.003750pt}%
\definecolor{currentstroke}{rgb}{1.000000,0.000000,0.000000}%
\pgfsetstrokecolor{currentstroke}%
\pgfsetdash{}{0pt}%
\pgfpathmoveto{\pgfqpoint{3.023175in}{1.054339in}}%
\pgfpathcurveto{\pgfqpoint{3.034225in}{1.054339in}}{\pgfqpoint{3.044824in}{1.058729in}}{\pgfqpoint{3.052638in}{1.066543in}}%
\pgfpathcurveto{\pgfqpoint{3.060451in}{1.074357in}}{\pgfqpoint{3.064842in}{1.084956in}}{\pgfqpoint{3.064842in}{1.096006in}}%
\pgfpathcurveto{\pgfqpoint{3.064842in}{1.107056in}}{\pgfqpoint{3.060451in}{1.117655in}}{\pgfqpoint{3.052638in}{1.125469in}}%
\pgfpathcurveto{\pgfqpoint{3.044824in}{1.133282in}}{\pgfqpoint{3.034225in}{1.137673in}}{\pgfqpoint{3.023175in}{1.137673in}}%
\pgfpathcurveto{\pgfqpoint{3.012125in}{1.137673in}}{\pgfqpoint{3.001526in}{1.133282in}}{\pgfqpoint{2.993712in}{1.125469in}}%
\pgfpathcurveto{\pgfqpoint{2.985899in}{1.117655in}}{\pgfqpoint{2.981508in}{1.107056in}}{\pgfqpoint{2.981508in}{1.096006in}}%
\pgfpathcurveto{\pgfqpoint{2.981508in}{1.084956in}}{\pgfqpoint{2.985899in}{1.074357in}}{\pgfqpoint{2.993712in}{1.066543in}}%
\pgfpathcurveto{\pgfqpoint{3.001526in}{1.058729in}}{\pgfqpoint{3.012125in}{1.054339in}}{\pgfqpoint{3.023175in}{1.054339in}}%
\pgfpathlineto{\pgfqpoint{3.023175in}{1.054339in}}%
\pgfpathclose%
\pgfusepath{stroke}%
\end{pgfscope}%
\begin{pgfscope}%
\pgfpathrectangle{\pgfqpoint{0.847223in}{0.554012in}}{\pgfqpoint{6.200000in}{4.530000in}}%
\pgfusepath{clip}%
\pgfsetbuttcap%
\pgfsetroundjoin%
\pgfsetlinewidth{1.003750pt}%
\definecolor{currentstroke}{rgb}{1.000000,0.000000,0.000000}%
\pgfsetstrokecolor{currentstroke}%
\pgfsetdash{}{0pt}%
\pgfpathmoveto{\pgfqpoint{3.028508in}{1.052448in}}%
\pgfpathcurveto{\pgfqpoint{3.039558in}{1.052448in}}{\pgfqpoint{3.050157in}{1.056838in}}{\pgfqpoint{3.057971in}{1.064652in}}%
\pgfpathcurveto{\pgfqpoint{3.065785in}{1.072465in}}{\pgfqpoint{3.070175in}{1.083064in}}{\pgfqpoint{3.070175in}{1.094115in}}%
\pgfpathcurveto{\pgfqpoint{3.070175in}{1.105165in}}{\pgfqpoint{3.065785in}{1.115764in}}{\pgfqpoint{3.057971in}{1.123577in}}%
\pgfpathcurveto{\pgfqpoint{3.050157in}{1.131391in}}{\pgfqpoint{3.039558in}{1.135781in}}{\pgfqpoint{3.028508in}{1.135781in}}%
\pgfpathcurveto{\pgfqpoint{3.017458in}{1.135781in}}{\pgfqpoint{3.006859in}{1.131391in}}{\pgfqpoint{2.999045in}{1.123577in}}%
\pgfpathcurveto{\pgfqpoint{2.991232in}{1.115764in}}{\pgfqpoint{2.986842in}{1.105165in}}{\pgfqpoint{2.986842in}{1.094115in}}%
\pgfpathcurveto{\pgfqpoint{2.986842in}{1.083064in}}{\pgfqpoint{2.991232in}{1.072465in}}{\pgfqpoint{2.999045in}{1.064652in}}%
\pgfpathcurveto{\pgfqpoint{3.006859in}{1.056838in}}{\pgfqpoint{3.017458in}{1.052448in}}{\pgfqpoint{3.028508in}{1.052448in}}%
\pgfpathlineto{\pgfqpoint{3.028508in}{1.052448in}}%
\pgfpathclose%
\pgfusepath{stroke}%
\end{pgfscope}%
\begin{pgfscope}%
\pgfpathrectangle{\pgfqpoint{0.847223in}{0.554012in}}{\pgfqpoint{6.200000in}{4.530000in}}%
\pgfusepath{clip}%
\pgfsetbuttcap%
\pgfsetroundjoin%
\pgfsetlinewidth{1.003750pt}%
\definecolor{currentstroke}{rgb}{1.000000,0.000000,0.000000}%
\pgfsetstrokecolor{currentstroke}%
\pgfsetdash{}{0pt}%
\pgfpathmoveto{\pgfqpoint{3.033841in}{1.050564in}}%
\pgfpathcurveto{\pgfqpoint{3.044892in}{1.050564in}}{\pgfqpoint{3.055491in}{1.054954in}}{\pgfqpoint{3.063304in}{1.062768in}}%
\pgfpathcurveto{\pgfqpoint{3.071118in}{1.070581in}}{\pgfqpoint{3.075508in}{1.081180in}}{\pgfqpoint{3.075508in}{1.092231in}}%
\pgfpathcurveto{\pgfqpoint{3.075508in}{1.103281in}}{\pgfqpoint{3.071118in}{1.113880in}}{\pgfqpoint{3.063304in}{1.121693in}}%
\pgfpathcurveto{\pgfqpoint{3.055491in}{1.129507in}}{\pgfqpoint{3.044892in}{1.133897in}}{\pgfqpoint{3.033841in}{1.133897in}}%
\pgfpathcurveto{\pgfqpoint{3.022791in}{1.133897in}}{\pgfqpoint{3.012192in}{1.129507in}}{\pgfqpoint{3.004379in}{1.121693in}}%
\pgfpathcurveto{\pgfqpoint{2.996565in}{1.113880in}}{\pgfqpoint{2.992175in}{1.103281in}}{\pgfqpoint{2.992175in}{1.092231in}}%
\pgfpathcurveto{\pgfqpoint{2.992175in}{1.081180in}}{\pgfqpoint{2.996565in}{1.070581in}}{\pgfqpoint{3.004379in}{1.062768in}}%
\pgfpathcurveto{\pgfqpoint{3.012192in}{1.054954in}}{\pgfqpoint{3.022791in}{1.050564in}}{\pgfqpoint{3.033841in}{1.050564in}}%
\pgfpathlineto{\pgfqpoint{3.033841in}{1.050564in}}%
\pgfpathclose%
\pgfusepath{stroke}%
\end{pgfscope}%
\begin{pgfscope}%
\pgfpathrectangle{\pgfqpoint{0.847223in}{0.554012in}}{\pgfqpoint{6.200000in}{4.530000in}}%
\pgfusepath{clip}%
\pgfsetbuttcap%
\pgfsetroundjoin%
\pgfsetlinewidth{1.003750pt}%
\definecolor{currentstroke}{rgb}{1.000000,0.000000,0.000000}%
\pgfsetstrokecolor{currentstroke}%
\pgfsetdash{}{0pt}%
\pgfpathmoveto{\pgfqpoint{3.039175in}{1.048687in}}%
\pgfpathcurveto{\pgfqpoint{3.050225in}{1.048687in}}{\pgfqpoint{3.060824in}{1.053078in}}{\pgfqpoint{3.068637in}{1.060891in}}%
\pgfpathcurveto{\pgfqpoint{3.076451in}{1.068705in}}{\pgfqpoint{3.080841in}{1.079304in}}{\pgfqpoint{3.080841in}{1.090354in}}%
\pgfpathcurveto{\pgfqpoint{3.080841in}{1.101404in}}{\pgfqpoint{3.076451in}{1.112003in}}{\pgfqpoint{3.068637in}{1.119817in}}%
\pgfpathcurveto{\pgfqpoint{3.060824in}{1.127630in}}{\pgfqpoint{3.050225in}{1.132021in}}{\pgfqpoint{3.039175in}{1.132021in}}%
\pgfpathcurveto{\pgfqpoint{3.028125in}{1.132021in}}{\pgfqpoint{3.017526in}{1.127630in}}{\pgfqpoint{3.009712in}{1.119817in}}%
\pgfpathcurveto{\pgfqpoint{3.001898in}{1.112003in}}{\pgfqpoint{2.997508in}{1.101404in}}{\pgfqpoint{2.997508in}{1.090354in}}%
\pgfpathcurveto{\pgfqpoint{2.997508in}{1.079304in}}{\pgfqpoint{3.001898in}{1.068705in}}{\pgfqpoint{3.009712in}{1.060891in}}%
\pgfpathcurveto{\pgfqpoint{3.017526in}{1.053078in}}{\pgfqpoint{3.028125in}{1.048687in}}{\pgfqpoint{3.039175in}{1.048687in}}%
\pgfpathlineto{\pgfqpoint{3.039175in}{1.048687in}}%
\pgfpathclose%
\pgfusepath{stroke}%
\end{pgfscope}%
\begin{pgfscope}%
\pgfpathrectangle{\pgfqpoint{0.847223in}{0.554012in}}{\pgfqpoint{6.200000in}{4.530000in}}%
\pgfusepath{clip}%
\pgfsetbuttcap%
\pgfsetroundjoin%
\pgfsetlinewidth{1.003750pt}%
\definecolor{currentstroke}{rgb}{1.000000,0.000000,0.000000}%
\pgfsetstrokecolor{currentstroke}%
\pgfsetdash{}{0pt}%
\pgfpathmoveto{\pgfqpoint{3.044508in}{1.046818in}}%
\pgfpathcurveto{\pgfqpoint{3.055558in}{1.046818in}}{\pgfqpoint{3.066157in}{1.051208in}}{\pgfqpoint{3.073971in}{1.059022in}}%
\pgfpathcurveto{\pgfqpoint{3.081784in}{1.066836in}}{\pgfqpoint{3.086175in}{1.077435in}}{\pgfqpoint{3.086175in}{1.088485in}}%
\pgfpathcurveto{\pgfqpoint{3.086175in}{1.099535in}}{\pgfqpoint{3.081784in}{1.110134in}}{\pgfqpoint{3.073971in}{1.117948in}}%
\pgfpathcurveto{\pgfqpoint{3.066157in}{1.125761in}}{\pgfqpoint{3.055558in}{1.130152in}}{\pgfqpoint{3.044508in}{1.130152in}}%
\pgfpathcurveto{\pgfqpoint{3.033458in}{1.130152in}}{\pgfqpoint{3.022859in}{1.125761in}}{\pgfqpoint{3.015045in}{1.117948in}}%
\pgfpathcurveto{\pgfqpoint{3.007232in}{1.110134in}}{\pgfqpoint{3.002841in}{1.099535in}}{\pgfqpoint{3.002841in}{1.088485in}}%
\pgfpathcurveto{\pgfqpoint{3.002841in}{1.077435in}}{\pgfqpoint{3.007232in}{1.066836in}}{\pgfqpoint{3.015045in}{1.059022in}}%
\pgfpathcurveto{\pgfqpoint{3.022859in}{1.051208in}}{\pgfqpoint{3.033458in}{1.046818in}}{\pgfqpoint{3.044508in}{1.046818in}}%
\pgfpathlineto{\pgfqpoint{3.044508in}{1.046818in}}%
\pgfpathclose%
\pgfusepath{stroke}%
\end{pgfscope}%
\begin{pgfscope}%
\pgfpathrectangle{\pgfqpoint{0.847223in}{0.554012in}}{\pgfqpoint{6.200000in}{4.530000in}}%
\pgfusepath{clip}%
\pgfsetbuttcap%
\pgfsetroundjoin%
\pgfsetlinewidth{1.003750pt}%
\definecolor{currentstroke}{rgb}{1.000000,0.000000,0.000000}%
\pgfsetstrokecolor{currentstroke}%
\pgfsetdash{}{0pt}%
\pgfpathmoveto{\pgfqpoint{3.049841in}{1.044956in}}%
\pgfpathcurveto{\pgfqpoint{3.060891in}{1.044956in}}{\pgfqpoint{3.071490in}{1.049347in}}{\pgfqpoint{3.079304in}{1.057160in}}%
\pgfpathcurveto{\pgfqpoint{3.087118in}{1.064974in}}{\pgfqpoint{3.091508in}{1.075573in}}{\pgfqpoint{3.091508in}{1.086623in}}%
\pgfpathcurveto{\pgfqpoint{3.091508in}{1.097673in}}{\pgfqpoint{3.087118in}{1.108272in}}{\pgfqpoint{3.079304in}{1.116086in}}%
\pgfpathcurveto{\pgfqpoint{3.071490in}{1.123899in}}{\pgfqpoint{3.060891in}{1.128290in}}{\pgfqpoint{3.049841in}{1.128290in}}%
\pgfpathcurveto{\pgfqpoint{3.038791in}{1.128290in}}{\pgfqpoint{3.028192in}{1.123899in}}{\pgfqpoint{3.020378in}{1.116086in}}%
\pgfpathcurveto{\pgfqpoint{3.012565in}{1.108272in}}{\pgfqpoint{3.008174in}{1.097673in}}{\pgfqpoint{3.008174in}{1.086623in}}%
\pgfpathcurveto{\pgfqpoint{3.008174in}{1.075573in}}{\pgfqpoint{3.012565in}{1.064974in}}{\pgfqpoint{3.020378in}{1.057160in}}%
\pgfpathcurveto{\pgfqpoint{3.028192in}{1.049347in}}{\pgfqpoint{3.038791in}{1.044956in}}{\pgfqpoint{3.049841in}{1.044956in}}%
\pgfpathlineto{\pgfqpoint{3.049841in}{1.044956in}}%
\pgfpathclose%
\pgfusepath{stroke}%
\end{pgfscope}%
\begin{pgfscope}%
\pgfpathrectangle{\pgfqpoint{0.847223in}{0.554012in}}{\pgfqpoint{6.200000in}{4.530000in}}%
\pgfusepath{clip}%
\pgfsetbuttcap%
\pgfsetroundjoin%
\pgfsetlinewidth{1.003750pt}%
\definecolor{currentstroke}{rgb}{1.000000,0.000000,0.000000}%
\pgfsetstrokecolor{currentstroke}%
\pgfsetdash{}{0pt}%
\pgfpathmoveto{\pgfqpoint{3.055174in}{1.043102in}}%
\pgfpathcurveto{\pgfqpoint{3.066224in}{1.043102in}}{\pgfqpoint{3.076824in}{1.047492in}}{\pgfqpoint{3.084637in}{1.055306in}}%
\pgfpathcurveto{\pgfqpoint{3.092451in}{1.063119in}}{\pgfqpoint{3.096841in}{1.073718in}}{\pgfqpoint{3.096841in}{1.084768in}}%
\pgfpathcurveto{\pgfqpoint{3.096841in}{1.095819in}}{\pgfqpoint{3.092451in}{1.106418in}}{\pgfqpoint{3.084637in}{1.114231in}}%
\pgfpathcurveto{\pgfqpoint{3.076824in}{1.122045in}}{\pgfqpoint{3.066224in}{1.126435in}}{\pgfqpoint{3.055174in}{1.126435in}}%
\pgfpathcurveto{\pgfqpoint{3.044124in}{1.126435in}}{\pgfqpoint{3.033525in}{1.122045in}}{\pgfqpoint{3.025712in}{1.114231in}}%
\pgfpathcurveto{\pgfqpoint{3.017898in}{1.106418in}}{\pgfqpoint{3.013508in}{1.095819in}}{\pgfqpoint{3.013508in}{1.084768in}}%
\pgfpathcurveto{\pgfqpoint{3.013508in}{1.073718in}}{\pgfqpoint{3.017898in}{1.063119in}}{\pgfqpoint{3.025712in}{1.055306in}}%
\pgfpathcurveto{\pgfqpoint{3.033525in}{1.047492in}}{\pgfqpoint{3.044124in}{1.043102in}}{\pgfqpoint{3.055174in}{1.043102in}}%
\pgfpathlineto{\pgfqpoint{3.055174in}{1.043102in}}%
\pgfpathclose%
\pgfusepath{stroke}%
\end{pgfscope}%
\begin{pgfscope}%
\pgfpathrectangle{\pgfqpoint{0.847223in}{0.554012in}}{\pgfqpoint{6.200000in}{4.530000in}}%
\pgfusepath{clip}%
\pgfsetbuttcap%
\pgfsetroundjoin%
\pgfsetlinewidth{1.003750pt}%
\definecolor{currentstroke}{rgb}{1.000000,0.000000,0.000000}%
\pgfsetstrokecolor{currentstroke}%
\pgfsetdash{}{0pt}%
\pgfpathmoveto{\pgfqpoint{3.060508in}{1.041254in}}%
\pgfpathcurveto{\pgfqpoint{3.071558in}{1.041254in}}{\pgfqpoint{3.082157in}{1.045645in}}{\pgfqpoint{3.089970in}{1.053458in}}%
\pgfpathcurveto{\pgfqpoint{3.097784in}{1.061272in}}{\pgfqpoint{3.102174in}{1.071871in}}{\pgfqpoint{3.102174in}{1.082921in}}%
\pgfpathcurveto{\pgfqpoint{3.102174in}{1.093971in}}{\pgfqpoint{3.097784in}{1.104570in}}{\pgfqpoint{3.089970in}{1.112384in}}%
\pgfpathcurveto{\pgfqpoint{3.082157in}{1.120197in}}{\pgfqpoint{3.071558in}{1.124588in}}{\pgfqpoint{3.060508in}{1.124588in}}%
\pgfpathcurveto{\pgfqpoint{3.049457in}{1.124588in}}{\pgfqpoint{3.038858in}{1.120197in}}{\pgfqpoint{3.031045in}{1.112384in}}%
\pgfpathcurveto{\pgfqpoint{3.023231in}{1.104570in}}{\pgfqpoint{3.018841in}{1.093971in}}{\pgfqpoint{3.018841in}{1.082921in}}%
\pgfpathcurveto{\pgfqpoint{3.018841in}{1.071871in}}{\pgfqpoint{3.023231in}{1.061272in}}{\pgfqpoint{3.031045in}{1.053458in}}%
\pgfpathcurveto{\pgfqpoint{3.038858in}{1.045645in}}{\pgfqpoint{3.049457in}{1.041254in}}{\pgfqpoint{3.060508in}{1.041254in}}%
\pgfpathlineto{\pgfqpoint{3.060508in}{1.041254in}}%
\pgfpathclose%
\pgfusepath{stroke}%
\end{pgfscope}%
\begin{pgfscope}%
\pgfpathrectangle{\pgfqpoint{0.847223in}{0.554012in}}{\pgfqpoint{6.200000in}{4.530000in}}%
\pgfusepath{clip}%
\pgfsetbuttcap%
\pgfsetroundjoin%
\pgfsetlinewidth{1.003750pt}%
\definecolor{currentstroke}{rgb}{1.000000,0.000000,0.000000}%
\pgfsetstrokecolor{currentstroke}%
\pgfsetdash{}{0pt}%
\pgfpathmoveto{\pgfqpoint{3.065841in}{1.039414in}}%
\pgfpathcurveto{\pgfqpoint{3.076891in}{1.039414in}}{\pgfqpoint{3.087490in}{1.043804in}}{\pgfqpoint{3.095304in}{1.051618in}}%
\pgfpathcurveto{\pgfqpoint{3.103117in}{1.059432in}}{\pgfqpoint{3.107507in}{1.070031in}}{\pgfqpoint{3.107507in}{1.081081in}}%
\pgfpathcurveto{\pgfqpoint{3.107507in}{1.092131in}}{\pgfqpoint{3.103117in}{1.102730in}}{\pgfqpoint{3.095304in}{1.110544in}}%
\pgfpathcurveto{\pgfqpoint{3.087490in}{1.118357in}}{\pgfqpoint{3.076891in}{1.122748in}}{\pgfqpoint{3.065841in}{1.122748in}}%
\pgfpathcurveto{\pgfqpoint{3.054791in}{1.122748in}}{\pgfqpoint{3.044192in}{1.118357in}}{\pgfqpoint{3.036378in}{1.110544in}}%
\pgfpathcurveto{\pgfqpoint{3.028564in}{1.102730in}}{\pgfqpoint{3.024174in}{1.092131in}}{\pgfqpoint{3.024174in}{1.081081in}}%
\pgfpathcurveto{\pgfqpoint{3.024174in}{1.070031in}}{\pgfqpoint{3.028564in}{1.059432in}}{\pgfqpoint{3.036378in}{1.051618in}}%
\pgfpathcurveto{\pgfqpoint{3.044192in}{1.043804in}}{\pgfqpoint{3.054791in}{1.039414in}}{\pgfqpoint{3.065841in}{1.039414in}}%
\pgfpathlineto{\pgfqpoint{3.065841in}{1.039414in}}%
\pgfpathclose%
\pgfusepath{stroke}%
\end{pgfscope}%
\begin{pgfscope}%
\pgfpathrectangle{\pgfqpoint{0.847223in}{0.554012in}}{\pgfqpoint{6.200000in}{4.530000in}}%
\pgfusepath{clip}%
\pgfsetbuttcap%
\pgfsetroundjoin%
\pgfsetlinewidth{1.003750pt}%
\definecolor{currentstroke}{rgb}{1.000000,0.000000,0.000000}%
\pgfsetstrokecolor{currentstroke}%
\pgfsetdash{}{0pt}%
\pgfpathmoveto{\pgfqpoint{3.071174in}{1.037581in}}%
\pgfpathcurveto{\pgfqpoint{3.082224in}{1.037581in}}{\pgfqpoint{3.092823in}{1.041971in}}{\pgfqpoint{3.100637in}{1.049785in}}%
\pgfpathcurveto{\pgfqpoint{3.108450in}{1.057599in}}{\pgfqpoint{3.112841in}{1.068198in}}{\pgfqpoint{3.112841in}{1.079248in}}%
\pgfpathcurveto{\pgfqpoint{3.112841in}{1.090298in}}{\pgfqpoint{3.108450in}{1.100897in}}{\pgfqpoint{3.100637in}{1.108711in}}%
\pgfpathcurveto{\pgfqpoint{3.092823in}{1.116524in}}{\pgfqpoint{3.082224in}{1.120914in}}{\pgfqpoint{3.071174in}{1.120914in}}%
\pgfpathcurveto{\pgfqpoint{3.060124in}{1.120914in}}{\pgfqpoint{3.049525in}{1.116524in}}{\pgfqpoint{3.041711in}{1.108711in}}%
\pgfpathcurveto{\pgfqpoint{3.033898in}{1.100897in}}{\pgfqpoint{3.029507in}{1.090298in}}{\pgfqpoint{3.029507in}{1.079248in}}%
\pgfpathcurveto{\pgfqpoint{3.029507in}{1.068198in}}{\pgfqpoint{3.033898in}{1.057599in}}{\pgfqpoint{3.041711in}{1.049785in}}%
\pgfpathcurveto{\pgfqpoint{3.049525in}{1.041971in}}{\pgfqpoint{3.060124in}{1.037581in}}{\pgfqpoint{3.071174in}{1.037581in}}%
\pgfpathlineto{\pgfqpoint{3.071174in}{1.037581in}}%
\pgfpathclose%
\pgfusepath{stroke}%
\end{pgfscope}%
\begin{pgfscope}%
\pgfpathrectangle{\pgfqpoint{0.847223in}{0.554012in}}{\pgfqpoint{6.200000in}{4.530000in}}%
\pgfusepath{clip}%
\pgfsetbuttcap%
\pgfsetroundjoin%
\pgfsetlinewidth{1.003750pt}%
\definecolor{currentstroke}{rgb}{1.000000,0.000000,0.000000}%
\pgfsetstrokecolor{currentstroke}%
\pgfsetdash{}{0pt}%
\pgfpathmoveto{\pgfqpoint{3.076507in}{1.035755in}}%
\pgfpathcurveto{\pgfqpoint{3.087557in}{1.035755in}}{\pgfqpoint{3.098156in}{1.040145in}}{\pgfqpoint{3.105970in}{1.047959in}}%
\pgfpathcurveto{\pgfqpoint{3.113784in}{1.055773in}}{\pgfqpoint{3.118174in}{1.066372in}}{\pgfqpoint{3.118174in}{1.077422in}}%
\pgfpathcurveto{\pgfqpoint{3.118174in}{1.088472in}}{\pgfqpoint{3.113784in}{1.099071in}}{\pgfqpoint{3.105970in}{1.106885in}}%
\pgfpathcurveto{\pgfqpoint{3.098156in}{1.114698in}}{\pgfqpoint{3.087557in}{1.119089in}}{\pgfqpoint{3.076507in}{1.119089in}}%
\pgfpathcurveto{\pgfqpoint{3.065457in}{1.119089in}}{\pgfqpoint{3.054858in}{1.114698in}}{\pgfqpoint{3.047044in}{1.106885in}}%
\pgfpathcurveto{\pgfqpoint{3.039231in}{1.099071in}}{\pgfqpoint{3.034841in}{1.088472in}}{\pgfqpoint{3.034841in}{1.077422in}}%
\pgfpathcurveto{\pgfqpoint{3.034841in}{1.066372in}}{\pgfqpoint{3.039231in}{1.055773in}}{\pgfqpoint{3.047044in}{1.047959in}}%
\pgfpathcurveto{\pgfqpoint{3.054858in}{1.040145in}}{\pgfqpoint{3.065457in}{1.035755in}}{\pgfqpoint{3.076507in}{1.035755in}}%
\pgfpathlineto{\pgfqpoint{3.076507in}{1.035755in}}%
\pgfpathclose%
\pgfusepath{stroke}%
\end{pgfscope}%
\begin{pgfscope}%
\pgfpathrectangle{\pgfqpoint{0.847223in}{0.554012in}}{\pgfqpoint{6.200000in}{4.530000in}}%
\pgfusepath{clip}%
\pgfsetbuttcap%
\pgfsetroundjoin%
\pgfsetlinewidth{1.003750pt}%
\definecolor{currentstroke}{rgb}{1.000000,0.000000,0.000000}%
\pgfsetstrokecolor{currentstroke}%
\pgfsetdash{}{0pt}%
\pgfpathmoveto{\pgfqpoint{3.081840in}{1.033936in}}%
\pgfpathcurveto{\pgfqpoint{3.092891in}{1.033936in}}{\pgfqpoint{3.103490in}{1.038327in}}{\pgfqpoint{3.111303in}{1.046140in}}%
\pgfpathcurveto{\pgfqpoint{3.119117in}{1.053954in}}{\pgfqpoint{3.123507in}{1.064553in}}{\pgfqpoint{3.123507in}{1.075603in}}%
\pgfpathcurveto{\pgfqpoint{3.123507in}{1.086653in}}{\pgfqpoint{3.119117in}{1.097252in}}{\pgfqpoint{3.111303in}{1.105066in}}%
\pgfpathcurveto{\pgfqpoint{3.103490in}{1.112879in}}{\pgfqpoint{3.092891in}{1.117270in}}{\pgfqpoint{3.081840in}{1.117270in}}%
\pgfpathcurveto{\pgfqpoint{3.070790in}{1.117270in}}{\pgfqpoint{3.060191in}{1.112879in}}{\pgfqpoint{3.052378in}{1.105066in}}%
\pgfpathcurveto{\pgfqpoint{3.044564in}{1.097252in}}{\pgfqpoint{3.040174in}{1.086653in}}{\pgfqpoint{3.040174in}{1.075603in}}%
\pgfpathcurveto{\pgfqpoint{3.040174in}{1.064553in}}{\pgfqpoint{3.044564in}{1.053954in}}{\pgfqpoint{3.052378in}{1.046140in}}%
\pgfpathcurveto{\pgfqpoint{3.060191in}{1.038327in}}{\pgfqpoint{3.070790in}{1.033936in}}{\pgfqpoint{3.081840in}{1.033936in}}%
\pgfpathlineto{\pgfqpoint{3.081840in}{1.033936in}}%
\pgfpathclose%
\pgfusepath{stroke}%
\end{pgfscope}%
\begin{pgfscope}%
\pgfpathrectangle{\pgfqpoint{0.847223in}{0.554012in}}{\pgfqpoint{6.200000in}{4.530000in}}%
\pgfusepath{clip}%
\pgfsetbuttcap%
\pgfsetroundjoin%
\pgfsetlinewidth{1.003750pt}%
\definecolor{currentstroke}{rgb}{1.000000,0.000000,0.000000}%
\pgfsetstrokecolor{currentstroke}%
\pgfsetdash{}{0pt}%
\pgfpathmoveto{\pgfqpoint{3.087174in}{1.032124in}}%
\pgfpathcurveto{\pgfqpoint{3.098224in}{1.032124in}}{\pgfqpoint{3.108823in}{1.036515in}}{\pgfqpoint{3.116636in}{1.044328in}}%
\pgfpathcurveto{\pgfqpoint{3.124450in}{1.052142in}}{\pgfqpoint{3.128840in}{1.062741in}}{\pgfqpoint{3.128840in}{1.073791in}}%
\pgfpathcurveto{\pgfqpoint{3.128840in}{1.084841in}}{\pgfqpoint{3.124450in}{1.095440in}}{\pgfqpoint{3.116636in}{1.103254in}}%
\pgfpathcurveto{\pgfqpoint{3.108823in}{1.111068in}}{\pgfqpoint{3.098224in}{1.115458in}}{\pgfqpoint{3.087174in}{1.115458in}}%
\pgfpathcurveto{\pgfqpoint{3.076124in}{1.115458in}}{\pgfqpoint{3.065524in}{1.111068in}}{\pgfqpoint{3.057711in}{1.103254in}}%
\pgfpathcurveto{\pgfqpoint{3.049897in}{1.095440in}}{\pgfqpoint{3.045507in}{1.084841in}}{\pgfqpoint{3.045507in}{1.073791in}}%
\pgfpathcurveto{\pgfqpoint{3.045507in}{1.062741in}}{\pgfqpoint{3.049897in}{1.052142in}}{\pgfqpoint{3.057711in}{1.044328in}}%
\pgfpathcurveto{\pgfqpoint{3.065524in}{1.036515in}}{\pgfqpoint{3.076124in}{1.032124in}}{\pgfqpoint{3.087174in}{1.032124in}}%
\pgfpathlineto{\pgfqpoint{3.087174in}{1.032124in}}%
\pgfpathclose%
\pgfusepath{stroke}%
\end{pgfscope}%
\begin{pgfscope}%
\pgfpathrectangle{\pgfqpoint{0.847223in}{0.554012in}}{\pgfqpoint{6.200000in}{4.530000in}}%
\pgfusepath{clip}%
\pgfsetbuttcap%
\pgfsetroundjoin%
\pgfsetlinewidth{1.003750pt}%
\definecolor{currentstroke}{rgb}{1.000000,0.000000,0.000000}%
\pgfsetstrokecolor{currentstroke}%
\pgfsetdash{}{0pt}%
\pgfpathmoveto{\pgfqpoint{3.092507in}{1.030320in}}%
\pgfpathcurveto{\pgfqpoint{3.103557in}{1.030320in}}{\pgfqpoint{3.114156in}{1.034710in}}{\pgfqpoint{3.121970in}{1.042523in}}%
\pgfpathcurveto{\pgfqpoint{3.129783in}{1.050337in}}{\pgfqpoint{3.134174in}{1.060936in}}{\pgfqpoint{3.134174in}{1.071986in}}%
\pgfpathcurveto{\pgfqpoint{3.134174in}{1.083036in}}{\pgfqpoint{3.129783in}{1.093635in}}{\pgfqpoint{3.121970in}{1.101449in}}%
\pgfpathcurveto{\pgfqpoint{3.114156in}{1.109263in}}{\pgfqpoint{3.103557in}{1.113653in}}{\pgfqpoint{3.092507in}{1.113653in}}%
\pgfpathcurveto{\pgfqpoint{3.081457in}{1.113653in}}{\pgfqpoint{3.070858in}{1.109263in}}{\pgfqpoint{3.063044in}{1.101449in}}%
\pgfpathcurveto{\pgfqpoint{3.055230in}{1.093635in}}{\pgfqpoint{3.050840in}{1.083036in}}{\pgfqpoint{3.050840in}{1.071986in}}%
\pgfpathcurveto{\pgfqpoint{3.050840in}{1.060936in}}{\pgfqpoint{3.055230in}{1.050337in}}{\pgfqpoint{3.063044in}{1.042523in}}%
\pgfpathcurveto{\pgfqpoint{3.070858in}{1.034710in}}{\pgfqpoint{3.081457in}{1.030320in}}{\pgfqpoint{3.092507in}{1.030320in}}%
\pgfpathlineto{\pgfqpoint{3.092507in}{1.030320in}}%
\pgfpathclose%
\pgfusepath{stroke}%
\end{pgfscope}%
\begin{pgfscope}%
\pgfpathrectangle{\pgfqpoint{0.847223in}{0.554012in}}{\pgfqpoint{6.200000in}{4.530000in}}%
\pgfusepath{clip}%
\pgfsetbuttcap%
\pgfsetroundjoin%
\pgfsetlinewidth{1.003750pt}%
\definecolor{currentstroke}{rgb}{1.000000,0.000000,0.000000}%
\pgfsetstrokecolor{currentstroke}%
\pgfsetdash{}{0pt}%
\pgfpathmoveto{\pgfqpoint{3.097840in}{1.028522in}}%
\pgfpathcurveto{\pgfqpoint{3.108890in}{1.028522in}}{\pgfqpoint{3.119489in}{1.032912in}}{\pgfqpoint{3.127303in}{1.040726in}}%
\pgfpathcurveto{\pgfqpoint{3.135116in}{1.048539in}}{\pgfqpoint{3.139507in}{1.059138in}}{\pgfqpoint{3.139507in}{1.070188in}}%
\pgfpathcurveto{\pgfqpoint{3.139507in}{1.081238in}}{\pgfqpoint{3.135116in}{1.091838in}}{\pgfqpoint{3.127303in}{1.099651in}}%
\pgfpathcurveto{\pgfqpoint{3.119489in}{1.107465in}}{\pgfqpoint{3.108890in}{1.111855in}}{\pgfqpoint{3.097840in}{1.111855in}}%
\pgfpathcurveto{\pgfqpoint{3.086790in}{1.111855in}}{\pgfqpoint{3.076191in}{1.107465in}}{\pgfqpoint{3.068377in}{1.099651in}}%
\pgfpathcurveto{\pgfqpoint{3.060564in}{1.091838in}}{\pgfqpoint{3.056173in}{1.081238in}}{\pgfqpoint{3.056173in}{1.070188in}}%
\pgfpathcurveto{\pgfqpoint{3.056173in}{1.059138in}}{\pgfqpoint{3.060564in}{1.048539in}}{\pgfqpoint{3.068377in}{1.040726in}}%
\pgfpathcurveto{\pgfqpoint{3.076191in}{1.032912in}}{\pgfqpoint{3.086790in}{1.028522in}}{\pgfqpoint{3.097840in}{1.028522in}}%
\pgfpathlineto{\pgfqpoint{3.097840in}{1.028522in}}%
\pgfpathclose%
\pgfusepath{stroke}%
\end{pgfscope}%
\begin{pgfscope}%
\pgfpathrectangle{\pgfqpoint{0.847223in}{0.554012in}}{\pgfqpoint{6.200000in}{4.530000in}}%
\pgfusepath{clip}%
\pgfsetbuttcap%
\pgfsetroundjoin%
\pgfsetlinewidth{1.003750pt}%
\definecolor{currentstroke}{rgb}{1.000000,0.000000,0.000000}%
\pgfsetstrokecolor{currentstroke}%
\pgfsetdash{}{0pt}%
\pgfpathmoveto{\pgfqpoint{3.103173in}{1.026731in}}%
\pgfpathcurveto{\pgfqpoint{3.114223in}{1.026731in}}{\pgfqpoint{3.124822in}{1.031121in}}{\pgfqpoint{3.132636in}{1.038935in}}%
\pgfpathcurveto{\pgfqpoint{3.140450in}{1.046748in}}{\pgfqpoint{3.144840in}{1.057347in}}{\pgfqpoint{3.144840in}{1.068397in}}%
\pgfpathcurveto{\pgfqpoint{3.144840in}{1.079447in}}{\pgfqpoint{3.140450in}{1.090046in}}{\pgfqpoint{3.132636in}{1.097860in}}%
\pgfpathcurveto{\pgfqpoint{3.124822in}{1.105674in}}{\pgfqpoint{3.114223in}{1.110064in}}{\pgfqpoint{3.103173in}{1.110064in}}%
\pgfpathcurveto{\pgfqpoint{3.092123in}{1.110064in}}{\pgfqpoint{3.081524in}{1.105674in}}{\pgfqpoint{3.073710in}{1.097860in}}%
\pgfpathcurveto{\pgfqpoint{3.065897in}{1.090046in}}{\pgfqpoint{3.061507in}{1.079447in}}{\pgfqpoint{3.061507in}{1.068397in}}%
\pgfpathcurveto{\pgfqpoint{3.061507in}{1.057347in}}{\pgfqpoint{3.065897in}{1.046748in}}{\pgfqpoint{3.073710in}{1.038935in}}%
\pgfpathcurveto{\pgfqpoint{3.081524in}{1.031121in}}{\pgfqpoint{3.092123in}{1.026731in}}{\pgfqpoint{3.103173in}{1.026731in}}%
\pgfpathlineto{\pgfqpoint{3.103173in}{1.026731in}}%
\pgfpathclose%
\pgfusepath{stroke}%
\end{pgfscope}%
\begin{pgfscope}%
\pgfpathrectangle{\pgfqpoint{0.847223in}{0.554012in}}{\pgfqpoint{6.200000in}{4.530000in}}%
\pgfusepath{clip}%
\pgfsetbuttcap%
\pgfsetroundjoin%
\pgfsetlinewidth{1.003750pt}%
\definecolor{currentstroke}{rgb}{1.000000,0.000000,0.000000}%
\pgfsetstrokecolor{currentstroke}%
\pgfsetdash{}{0pt}%
\pgfpathmoveto{\pgfqpoint{3.108506in}{1.024946in}}%
\pgfpathcurveto{\pgfqpoint{3.119557in}{1.024946in}}{\pgfqpoint{3.130156in}{1.029337in}}{\pgfqpoint{3.137969in}{1.037150in}}%
\pgfpathcurveto{\pgfqpoint{3.145783in}{1.044964in}}{\pgfqpoint{3.150173in}{1.055563in}}{\pgfqpoint{3.150173in}{1.066613in}}%
\pgfpathcurveto{\pgfqpoint{3.150173in}{1.077663in}}{\pgfqpoint{3.145783in}{1.088262in}}{\pgfqpoint{3.137969in}{1.096076in}}%
\pgfpathcurveto{\pgfqpoint{3.130156in}{1.103890in}}{\pgfqpoint{3.119557in}{1.108280in}}{\pgfqpoint{3.108506in}{1.108280in}}%
\pgfpathcurveto{\pgfqpoint{3.097456in}{1.108280in}}{\pgfqpoint{3.086857in}{1.103890in}}{\pgfqpoint{3.079044in}{1.096076in}}%
\pgfpathcurveto{\pgfqpoint{3.071230in}{1.088262in}}{\pgfqpoint{3.066840in}{1.077663in}}{\pgfqpoint{3.066840in}{1.066613in}}%
\pgfpathcurveto{\pgfqpoint{3.066840in}{1.055563in}}{\pgfqpoint{3.071230in}{1.044964in}}{\pgfqpoint{3.079044in}{1.037150in}}%
\pgfpathcurveto{\pgfqpoint{3.086857in}{1.029337in}}{\pgfqpoint{3.097456in}{1.024946in}}{\pgfqpoint{3.108506in}{1.024946in}}%
\pgfpathlineto{\pgfqpoint{3.108506in}{1.024946in}}%
\pgfpathclose%
\pgfusepath{stroke}%
\end{pgfscope}%
\begin{pgfscope}%
\pgfpathrectangle{\pgfqpoint{0.847223in}{0.554012in}}{\pgfqpoint{6.200000in}{4.530000in}}%
\pgfusepath{clip}%
\pgfsetbuttcap%
\pgfsetroundjoin%
\pgfsetlinewidth{1.003750pt}%
\definecolor{currentstroke}{rgb}{1.000000,0.000000,0.000000}%
\pgfsetstrokecolor{currentstroke}%
\pgfsetdash{}{0pt}%
\pgfpathmoveto{\pgfqpoint{3.113840in}{1.023169in}}%
\pgfpathcurveto{\pgfqpoint{3.124890in}{1.023169in}}{\pgfqpoint{3.135489in}{1.027559in}}{\pgfqpoint{3.143302in}{1.035373in}}%
\pgfpathcurveto{\pgfqpoint{3.151116in}{1.043187in}}{\pgfqpoint{3.155506in}{1.053786in}}{\pgfqpoint{3.155506in}{1.064836in}}%
\pgfpathcurveto{\pgfqpoint{3.155506in}{1.075886in}}{\pgfqpoint{3.151116in}{1.086485in}}{\pgfqpoint{3.143302in}{1.094299in}}%
\pgfpathcurveto{\pgfqpoint{3.135489in}{1.102112in}}{\pgfqpoint{3.124890in}{1.106502in}}{\pgfqpoint{3.113840in}{1.106502in}}%
\pgfpathcurveto{\pgfqpoint{3.102790in}{1.106502in}}{\pgfqpoint{3.092191in}{1.102112in}}{\pgfqpoint{3.084377in}{1.094299in}}%
\pgfpathcurveto{\pgfqpoint{3.076563in}{1.086485in}}{\pgfqpoint{3.072173in}{1.075886in}}{\pgfqpoint{3.072173in}{1.064836in}}%
\pgfpathcurveto{\pgfqpoint{3.072173in}{1.053786in}}{\pgfqpoint{3.076563in}{1.043187in}}{\pgfqpoint{3.084377in}{1.035373in}}%
\pgfpathcurveto{\pgfqpoint{3.092191in}{1.027559in}}{\pgfqpoint{3.102790in}{1.023169in}}{\pgfqpoint{3.113840in}{1.023169in}}%
\pgfpathlineto{\pgfqpoint{3.113840in}{1.023169in}}%
\pgfpathclose%
\pgfusepath{stroke}%
\end{pgfscope}%
\begin{pgfscope}%
\pgfpathrectangle{\pgfqpoint{0.847223in}{0.554012in}}{\pgfqpoint{6.200000in}{4.530000in}}%
\pgfusepath{clip}%
\pgfsetbuttcap%
\pgfsetroundjoin%
\pgfsetlinewidth{1.003750pt}%
\definecolor{currentstroke}{rgb}{1.000000,0.000000,0.000000}%
\pgfsetstrokecolor{currentstroke}%
\pgfsetdash{}{0pt}%
\pgfpathmoveto{\pgfqpoint{3.119173in}{1.021399in}}%
\pgfpathcurveto{\pgfqpoint{3.130223in}{1.021399in}}{\pgfqpoint{3.140822in}{1.025789in}}{\pgfqpoint{3.148636in}{1.033602in}}%
\pgfpathcurveto{\pgfqpoint{3.156449in}{1.041416in}}{\pgfqpoint{3.160840in}{1.052015in}}{\pgfqpoint{3.160840in}{1.063065in}}%
\pgfpathcurveto{\pgfqpoint{3.160840in}{1.074115in}}{\pgfqpoint{3.156449in}{1.084714in}}{\pgfqpoint{3.148636in}{1.092528in}}%
\pgfpathcurveto{\pgfqpoint{3.140822in}{1.100342in}}{\pgfqpoint{3.130223in}{1.104732in}}{\pgfqpoint{3.119173in}{1.104732in}}%
\pgfpathcurveto{\pgfqpoint{3.108123in}{1.104732in}}{\pgfqpoint{3.097524in}{1.100342in}}{\pgfqpoint{3.089710in}{1.092528in}}%
\pgfpathcurveto{\pgfqpoint{3.081897in}{1.084714in}}{\pgfqpoint{3.077506in}{1.074115in}}{\pgfqpoint{3.077506in}{1.063065in}}%
\pgfpathcurveto{\pgfqpoint{3.077506in}{1.052015in}}{\pgfqpoint{3.081897in}{1.041416in}}{\pgfqpoint{3.089710in}{1.033602in}}%
\pgfpathcurveto{\pgfqpoint{3.097524in}{1.025789in}}{\pgfqpoint{3.108123in}{1.021399in}}{\pgfqpoint{3.119173in}{1.021399in}}%
\pgfpathlineto{\pgfqpoint{3.119173in}{1.021399in}}%
\pgfpathclose%
\pgfusepath{stroke}%
\end{pgfscope}%
\begin{pgfscope}%
\pgfpathrectangle{\pgfqpoint{0.847223in}{0.554012in}}{\pgfqpoint{6.200000in}{4.530000in}}%
\pgfusepath{clip}%
\pgfsetbuttcap%
\pgfsetroundjoin%
\pgfsetlinewidth{1.003750pt}%
\definecolor{currentstroke}{rgb}{1.000000,0.000000,0.000000}%
\pgfsetstrokecolor{currentstroke}%
\pgfsetdash{}{0pt}%
\pgfpathmoveto{\pgfqpoint{3.124506in}{1.019635in}}%
\pgfpathcurveto{\pgfqpoint{3.135556in}{1.019635in}}{\pgfqpoint{3.146155in}{1.024025in}}{\pgfqpoint{3.153969in}{1.031839in}}%
\pgfpathcurveto{\pgfqpoint{3.161783in}{1.039652in}}{\pgfqpoint{3.166173in}{1.050251in}}{\pgfqpoint{3.166173in}{1.061301in}}%
\pgfpathcurveto{\pgfqpoint{3.166173in}{1.072352in}}{\pgfqpoint{3.161783in}{1.082951in}}{\pgfqpoint{3.153969in}{1.090764in}}%
\pgfpathcurveto{\pgfqpoint{3.146155in}{1.098578in}}{\pgfqpoint{3.135556in}{1.102968in}}{\pgfqpoint{3.124506in}{1.102968in}}%
\pgfpathcurveto{\pgfqpoint{3.113456in}{1.102968in}}{\pgfqpoint{3.102857in}{1.098578in}}{\pgfqpoint{3.095043in}{1.090764in}}%
\pgfpathcurveto{\pgfqpoint{3.087230in}{1.082951in}}{\pgfqpoint{3.082839in}{1.072352in}}{\pgfqpoint{3.082839in}{1.061301in}}%
\pgfpathcurveto{\pgfqpoint{3.082839in}{1.050251in}}{\pgfqpoint{3.087230in}{1.039652in}}{\pgfqpoint{3.095043in}{1.031839in}}%
\pgfpathcurveto{\pgfqpoint{3.102857in}{1.024025in}}{\pgfqpoint{3.113456in}{1.019635in}}{\pgfqpoint{3.124506in}{1.019635in}}%
\pgfpathlineto{\pgfqpoint{3.124506in}{1.019635in}}%
\pgfpathclose%
\pgfusepath{stroke}%
\end{pgfscope}%
\begin{pgfscope}%
\pgfpathrectangle{\pgfqpoint{0.847223in}{0.554012in}}{\pgfqpoint{6.200000in}{4.530000in}}%
\pgfusepath{clip}%
\pgfsetbuttcap%
\pgfsetroundjoin%
\pgfsetlinewidth{1.003750pt}%
\definecolor{currentstroke}{rgb}{1.000000,0.000000,0.000000}%
\pgfsetstrokecolor{currentstroke}%
\pgfsetdash{}{0pt}%
\pgfpathmoveto{\pgfqpoint{3.129839in}{1.017878in}}%
\pgfpathcurveto{\pgfqpoint{3.140889in}{1.017878in}}{\pgfqpoint{3.151489in}{1.022268in}}{\pgfqpoint{3.159302in}{1.030082in}}%
\pgfpathcurveto{\pgfqpoint{3.167116in}{1.037895in}}{\pgfqpoint{3.171506in}{1.048494in}}{\pgfqpoint{3.171506in}{1.059544in}}%
\pgfpathcurveto{\pgfqpoint{3.171506in}{1.070594in}}{\pgfqpoint{3.167116in}{1.081193in}}{\pgfqpoint{3.159302in}{1.089007in}}%
\pgfpathcurveto{\pgfqpoint{3.151489in}{1.096821in}}{\pgfqpoint{3.140889in}{1.101211in}}{\pgfqpoint{3.129839in}{1.101211in}}%
\pgfpathcurveto{\pgfqpoint{3.118789in}{1.101211in}}{\pgfqpoint{3.108190in}{1.096821in}}{\pgfqpoint{3.100377in}{1.089007in}}%
\pgfpathcurveto{\pgfqpoint{3.092563in}{1.081193in}}{\pgfqpoint{3.088173in}{1.070594in}}{\pgfqpoint{3.088173in}{1.059544in}}%
\pgfpathcurveto{\pgfqpoint{3.088173in}{1.048494in}}{\pgfqpoint{3.092563in}{1.037895in}}{\pgfqpoint{3.100377in}{1.030082in}}%
\pgfpathcurveto{\pgfqpoint{3.108190in}{1.022268in}}{\pgfqpoint{3.118789in}{1.017878in}}{\pgfqpoint{3.129839in}{1.017878in}}%
\pgfpathlineto{\pgfqpoint{3.129839in}{1.017878in}}%
\pgfpathclose%
\pgfusepath{stroke}%
\end{pgfscope}%
\begin{pgfscope}%
\pgfpathrectangle{\pgfqpoint{0.847223in}{0.554012in}}{\pgfqpoint{6.200000in}{4.530000in}}%
\pgfusepath{clip}%
\pgfsetbuttcap%
\pgfsetroundjoin%
\pgfsetlinewidth{1.003750pt}%
\definecolor{currentstroke}{rgb}{1.000000,0.000000,0.000000}%
\pgfsetstrokecolor{currentstroke}%
\pgfsetdash{}{0pt}%
\pgfpathmoveto{\pgfqpoint{3.135173in}{1.016127in}}%
\pgfpathcurveto{\pgfqpoint{3.146223in}{1.016127in}}{\pgfqpoint{3.156822in}{1.020517in}}{\pgfqpoint{3.164635in}{1.028331in}}%
\pgfpathcurveto{\pgfqpoint{3.172449in}{1.036145in}}{\pgfqpoint{3.176839in}{1.046744in}}{\pgfqpoint{3.176839in}{1.057794in}}%
\pgfpathcurveto{\pgfqpoint{3.176839in}{1.068844in}}{\pgfqpoint{3.172449in}{1.079443in}}{\pgfqpoint{3.164635in}{1.087257in}}%
\pgfpathcurveto{\pgfqpoint{3.156822in}{1.095070in}}{\pgfqpoint{3.146223in}{1.099461in}}{\pgfqpoint{3.135173in}{1.099461in}}%
\pgfpathcurveto{\pgfqpoint{3.124122in}{1.099461in}}{\pgfqpoint{3.113523in}{1.095070in}}{\pgfqpoint{3.105710in}{1.087257in}}%
\pgfpathcurveto{\pgfqpoint{3.097896in}{1.079443in}}{\pgfqpoint{3.093506in}{1.068844in}}{\pgfqpoint{3.093506in}{1.057794in}}%
\pgfpathcurveto{\pgfqpoint{3.093506in}{1.046744in}}{\pgfqpoint{3.097896in}{1.036145in}}{\pgfqpoint{3.105710in}{1.028331in}}%
\pgfpathcurveto{\pgfqpoint{3.113523in}{1.020517in}}{\pgfqpoint{3.124122in}{1.016127in}}{\pgfqpoint{3.135173in}{1.016127in}}%
\pgfpathlineto{\pgfqpoint{3.135173in}{1.016127in}}%
\pgfpathclose%
\pgfusepath{stroke}%
\end{pgfscope}%
\begin{pgfscope}%
\pgfpathrectangle{\pgfqpoint{0.847223in}{0.554012in}}{\pgfqpoint{6.200000in}{4.530000in}}%
\pgfusepath{clip}%
\pgfsetbuttcap%
\pgfsetroundjoin%
\pgfsetlinewidth{1.003750pt}%
\definecolor{currentstroke}{rgb}{1.000000,0.000000,0.000000}%
\pgfsetstrokecolor{currentstroke}%
\pgfsetdash{}{0pt}%
\pgfpathmoveto{\pgfqpoint{3.140506in}{1.014383in}}%
\pgfpathcurveto{\pgfqpoint{3.151556in}{1.014383in}}{\pgfqpoint{3.162155in}{1.018774in}}{\pgfqpoint{3.169969in}{1.026587in}}%
\pgfpathcurveto{\pgfqpoint{3.177782in}{1.034401in}}{\pgfqpoint{3.182172in}{1.045000in}}{\pgfqpoint{3.182172in}{1.056050in}}%
\pgfpathcurveto{\pgfqpoint{3.182172in}{1.067100in}}{\pgfqpoint{3.177782in}{1.077699in}}{\pgfqpoint{3.169969in}{1.085513in}}%
\pgfpathcurveto{\pgfqpoint{3.162155in}{1.093326in}}{\pgfqpoint{3.151556in}{1.097717in}}{\pgfqpoint{3.140506in}{1.097717in}}%
\pgfpathcurveto{\pgfqpoint{3.129456in}{1.097717in}}{\pgfqpoint{3.118857in}{1.093326in}}{\pgfqpoint{3.111043in}{1.085513in}}%
\pgfpathcurveto{\pgfqpoint{3.103229in}{1.077699in}}{\pgfqpoint{3.098839in}{1.067100in}}{\pgfqpoint{3.098839in}{1.056050in}}%
\pgfpathcurveto{\pgfqpoint{3.098839in}{1.045000in}}{\pgfqpoint{3.103229in}{1.034401in}}{\pgfqpoint{3.111043in}{1.026587in}}%
\pgfpathcurveto{\pgfqpoint{3.118857in}{1.018774in}}{\pgfqpoint{3.129456in}{1.014383in}}{\pgfqpoint{3.140506in}{1.014383in}}%
\pgfpathlineto{\pgfqpoint{3.140506in}{1.014383in}}%
\pgfpathclose%
\pgfusepath{stroke}%
\end{pgfscope}%
\begin{pgfscope}%
\pgfpathrectangle{\pgfqpoint{0.847223in}{0.554012in}}{\pgfqpoint{6.200000in}{4.530000in}}%
\pgfusepath{clip}%
\pgfsetbuttcap%
\pgfsetroundjoin%
\pgfsetlinewidth{1.003750pt}%
\definecolor{currentstroke}{rgb}{1.000000,0.000000,0.000000}%
\pgfsetstrokecolor{currentstroke}%
\pgfsetdash{}{0pt}%
\pgfpathmoveto{\pgfqpoint{3.145839in}{1.012646in}}%
\pgfpathcurveto{\pgfqpoint{3.156889in}{1.012646in}}{\pgfqpoint{3.167488in}{1.017036in}}{\pgfqpoint{3.175302in}{1.024850in}}%
\pgfpathcurveto{\pgfqpoint{3.183115in}{1.032664in}}{\pgfqpoint{3.187506in}{1.043263in}}{\pgfqpoint{3.187506in}{1.054313in}}%
\pgfpathcurveto{\pgfqpoint{3.187506in}{1.065363in}}{\pgfqpoint{3.183115in}{1.075962in}}{\pgfqpoint{3.175302in}{1.083776in}}%
\pgfpathcurveto{\pgfqpoint{3.167488in}{1.091589in}}{\pgfqpoint{3.156889in}{1.095979in}}{\pgfqpoint{3.145839in}{1.095979in}}%
\pgfpathcurveto{\pgfqpoint{3.134789in}{1.095979in}}{\pgfqpoint{3.124190in}{1.091589in}}{\pgfqpoint{3.116376in}{1.083776in}}%
\pgfpathcurveto{\pgfqpoint{3.108563in}{1.075962in}}{\pgfqpoint{3.104172in}{1.065363in}}{\pgfqpoint{3.104172in}{1.054313in}}%
\pgfpathcurveto{\pgfqpoint{3.104172in}{1.043263in}}{\pgfqpoint{3.108563in}{1.032664in}}{\pgfqpoint{3.116376in}{1.024850in}}%
\pgfpathcurveto{\pgfqpoint{3.124190in}{1.017036in}}{\pgfqpoint{3.134789in}{1.012646in}}{\pgfqpoint{3.145839in}{1.012646in}}%
\pgfpathlineto{\pgfqpoint{3.145839in}{1.012646in}}%
\pgfpathclose%
\pgfusepath{stroke}%
\end{pgfscope}%
\begin{pgfscope}%
\pgfpathrectangle{\pgfqpoint{0.847223in}{0.554012in}}{\pgfqpoint{6.200000in}{4.530000in}}%
\pgfusepath{clip}%
\pgfsetbuttcap%
\pgfsetroundjoin%
\pgfsetlinewidth{1.003750pt}%
\definecolor{currentstroke}{rgb}{1.000000,0.000000,0.000000}%
\pgfsetstrokecolor{currentstroke}%
\pgfsetdash{}{0pt}%
\pgfpathmoveto{\pgfqpoint{3.151172in}{1.010915in}}%
\pgfpathcurveto{\pgfqpoint{3.162222in}{1.010915in}}{\pgfqpoint{3.172821in}{1.015306in}}{\pgfqpoint{3.180635in}{1.023119in}}%
\pgfpathcurveto{\pgfqpoint{3.188449in}{1.030933in}}{\pgfqpoint{3.192839in}{1.041532in}}{\pgfqpoint{3.192839in}{1.052582in}}%
\pgfpathcurveto{\pgfqpoint{3.192839in}{1.063632in}}{\pgfqpoint{3.188449in}{1.074231in}}{\pgfqpoint{3.180635in}{1.082045in}}%
\pgfpathcurveto{\pgfqpoint{3.172821in}{1.089859in}}{\pgfqpoint{3.162222in}{1.094249in}}{\pgfqpoint{3.151172in}{1.094249in}}%
\pgfpathcurveto{\pgfqpoint{3.140122in}{1.094249in}}{\pgfqpoint{3.129523in}{1.089859in}}{\pgfqpoint{3.121709in}{1.082045in}}%
\pgfpathcurveto{\pgfqpoint{3.113896in}{1.074231in}}{\pgfqpoint{3.109506in}{1.063632in}}{\pgfqpoint{3.109506in}{1.052582in}}%
\pgfpathcurveto{\pgfqpoint{3.109506in}{1.041532in}}{\pgfqpoint{3.113896in}{1.030933in}}{\pgfqpoint{3.121709in}{1.023119in}}%
\pgfpathcurveto{\pgfqpoint{3.129523in}{1.015306in}}{\pgfqpoint{3.140122in}{1.010915in}}{\pgfqpoint{3.151172in}{1.010915in}}%
\pgfpathlineto{\pgfqpoint{3.151172in}{1.010915in}}%
\pgfpathclose%
\pgfusepath{stroke}%
\end{pgfscope}%
\begin{pgfscope}%
\pgfpathrectangle{\pgfqpoint{0.847223in}{0.554012in}}{\pgfqpoint{6.200000in}{4.530000in}}%
\pgfusepath{clip}%
\pgfsetbuttcap%
\pgfsetroundjoin%
\pgfsetlinewidth{1.003750pt}%
\definecolor{currentstroke}{rgb}{1.000000,0.000000,0.000000}%
\pgfsetstrokecolor{currentstroke}%
\pgfsetdash{}{0pt}%
\pgfpathmoveto{\pgfqpoint{3.156505in}{1.009191in}}%
\pgfpathcurveto{\pgfqpoint{3.167556in}{1.009191in}}{\pgfqpoint{3.178155in}{1.013582in}}{\pgfqpoint{3.185968in}{1.021395in}}%
\pgfpathcurveto{\pgfqpoint{3.193782in}{1.029209in}}{\pgfqpoint{3.198172in}{1.039808in}}{\pgfqpoint{3.198172in}{1.050858in}}%
\pgfpathcurveto{\pgfqpoint{3.198172in}{1.061908in}}{\pgfqpoint{3.193782in}{1.072507in}}{\pgfqpoint{3.185968in}{1.080321in}}%
\pgfpathcurveto{\pgfqpoint{3.178155in}{1.088134in}}{\pgfqpoint{3.167556in}{1.092525in}}{\pgfqpoint{3.156505in}{1.092525in}}%
\pgfpathcurveto{\pgfqpoint{3.145455in}{1.092525in}}{\pgfqpoint{3.134856in}{1.088134in}}{\pgfqpoint{3.127043in}{1.080321in}}%
\pgfpathcurveto{\pgfqpoint{3.119229in}{1.072507in}}{\pgfqpoint{3.114839in}{1.061908in}}{\pgfqpoint{3.114839in}{1.050858in}}%
\pgfpathcurveto{\pgfqpoint{3.114839in}{1.039808in}}{\pgfqpoint{3.119229in}{1.029209in}}{\pgfqpoint{3.127043in}{1.021395in}}%
\pgfpathcurveto{\pgfqpoint{3.134856in}{1.013582in}}{\pgfqpoint{3.145455in}{1.009191in}}{\pgfqpoint{3.156505in}{1.009191in}}%
\pgfpathlineto{\pgfqpoint{3.156505in}{1.009191in}}%
\pgfpathclose%
\pgfusepath{stroke}%
\end{pgfscope}%
\begin{pgfscope}%
\pgfpathrectangle{\pgfqpoint{0.847223in}{0.554012in}}{\pgfqpoint{6.200000in}{4.530000in}}%
\pgfusepath{clip}%
\pgfsetbuttcap%
\pgfsetroundjoin%
\pgfsetlinewidth{1.003750pt}%
\definecolor{currentstroke}{rgb}{1.000000,0.000000,0.000000}%
\pgfsetstrokecolor{currentstroke}%
\pgfsetdash{}{0pt}%
\pgfpathmoveto{\pgfqpoint{3.161839in}{1.007474in}}%
\pgfpathcurveto{\pgfqpoint{3.172889in}{1.007474in}}{\pgfqpoint{3.183488in}{1.011864in}}{\pgfqpoint{3.191301in}{1.019678in}}%
\pgfpathcurveto{\pgfqpoint{3.199115in}{1.027491in}}{\pgfqpoint{3.203505in}{1.038090in}}{\pgfqpoint{3.203505in}{1.049140in}}%
\pgfpathcurveto{\pgfqpoint{3.203505in}{1.060190in}}{\pgfqpoint{3.199115in}{1.070789in}}{\pgfqpoint{3.191301in}{1.078603in}}%
\pgfpathcurveto{\pgfqpoint{3.183488in}{1.086417in}}{\pgfqpoint{3.172889in}{1.090807in}}{\pgfqpoint{3.161839in}{1.090807in}}%
\pgfpathcurveto{\pgfqpoint{3.150789in}{1.090807in}}{\pgfqpoint{3.140189in}{1.086417in}}{\pgfqpoint{3.132376in}{1.078603in}}%
\pgfpathcurveto{\pgfqpoint{3.124562in}{1.070789in}}{\pgfqpoint{3.120172in}{1.060190in}}{\pgfqpoint{3.120172in}{1.049140in}}%
\pgfpathcurveto{\pgfqpoint{3.120172in}{1.038090in}}{\pgfqpoint{3.124562in}{1.027491in}}{\pgfqpoint{3.132376in}{1.019678in}}%
\pgfpathcurveto{\pgfqpoint{3.140189in}{1.011864in}}{\pgfqpoint{3.150789in}{1.007474in}}{\pgfqpoint{3.161839in}{1.007474in}}%
\pgfpathlineto{\pgfqpoint{3.161839in}{1.007474in}}%
\pgfpathclose%
\pgfusepath{stroke}%
\end{pgfscope}%
\begin{pgfscope}%
\pgfpathrectangle{\pgfqpoint{0.847223in}{0.554012in}}{\pgfqpoint{6.200000in}{4.530000in}}%
\pgfusepath{clip}%
\pgfsetbuttcap%
\pgfsetroundjoin%
\pgfsetlinewidth{1.003750pt}%
\definecolor{currentstroke}{rgb}{1.000000,0.000000,0.000000}%
\pgfsetstrokecolor{currentstroke}%
\pgfsetdash{}{0pt}%
\pgfpathmoveto{\pgfqpoint{3.167172in}{1.005762in}}%
\pgfpathcurveto{\pgfqpoint{3.178222in}{1.005762in}}{\pgfqpoint{3.188821in}{1.010153in}}{\pgfqpoint{3.196635in}{1.017966in}}%
\pgfpathcurveto{\pgfqpoint{3.204448in}{1.025780in}}{\pgfqpoint{3.208839in}{1.036379in}}{\pgfqpoint{3.208839in}{1.047429in}}%
\pgfpathcurveto{\pgfqpoint{3.208839in}{1.058479in}}{\pgfqpoint{3.204448in}{1.069078in}}{\pgfqpoint{3.196635in}{1.076892in}}%
\pgfpathcurveto{\pgfqpoint{3.188821in}{1.084705in}}{\pgfqpoint{3.178222in}{1.089096in}}{\pgfqpoint{3.167172in}{1.089096in}}%
\pgfpathcurveto{\pgfqpoint{3.156122in}{1.089096in}}{\pgfqpoint{3.145523in}{1.084705in}}{\pgfqpoint{3.137709in}{1.076892in}}%
\pgfpathcurveto{\pgfqpoint{3.129895in}{1.069078in}}{\pgfqpoint{3.125505in}{1.058479in}}{\pgfqpoint{3.125505in}{1.047429in}}%
\pgfpathcurveto{\pgfqpoint{3.125505in}{1.036379in}}{\pgfqpoint{3.129895in}{1.025780in}}{\pgfqpoint{3.137709in}{1.017966in}}%
\pgfpathcurveto{\pgfqpoint{3.145523in}{1.010153in}}{\pgfqpoint{3.156122in}{1.005762in}}{\pgfqpoint{3.167172in}{1.005762in}}%
\pgfpathlineto{\pgfqpoint{3.167172in}{1.005762in}}%
\pgfpathclose%
\pgfusepath{stroke}%
\end{pgfscope}%
\begin{pgfscope}%
\pgfpathrectangle{\pgfqpoint{0.847223in}{0.554012in}}{\pgfqpoint{6.200000in}{4.530000in}}%
\pgfusepath{clip}%
\pgfsetbuttcap%
\pgfsetroundjoin%
\pgfsetlinewidth{1.003750pt}%
\definecolor{currentstroke}{rgb}{1.000000,0.000000,0.000000}%
\pgfsetstrokecolor{currentstroke}%
\pgfsetdash{}{0pt}%
\pgfpathmoveto{\pgfqpoint{3.172505in}{1.004058in}}%
\pgfpathcurveto{\pgfqpoint{3.183555in}{1.004058in}}{\pgfqpoint{3.194154in}{1.008448in}}{\pgfqpoint{3.201968in}{1.016261in}}%
\pgfpathcurveto{\pgfqpoint{3.209781in}{1.024075in}}{\pgfqpoint{3.214172in}{1.034674in}}{\pgfqpoint{3.214172in}{1.045724in}}%
\pgfpathcurveto{\pgfqpoint{3.214172in}{1.056774in}}{\pgfqpoint{3.209781in}{1.067373in}}{\pgfqpoint{3.201968in}{1.075187in}}%
\pgfpathcurveto{\pgfqpoint{3.194154in}{1.083001in}}{\pgfqpoint{3.183555in}{1.087391in}}{\pgfqpoint{3.172505in}{1.087391in}}%
\pgfpathcurveto{\pgfqpoint{3.161455in}{1.087391in}}{\pgfqpoint{3.150856in}{1.083001in}}{\pgfqpoint{3.143042in}{1.075187in}}%
\pgfpathcurveto{\pgfqpoint{3.135229in}{1.067373in}}{\pgfqpoint{3.130838in}{1.056774in}}{\pgfqpoint{3.130838in}{1.045724in}}%
\pgfpathcurveto{\pgfqpoint{3.130838in}{1.034674in}}{\pgfqpoint{3.135229in}{1.024075in}}{\pgfqpoint{3.143042in}{1.016261in}}%
\pgfpathcurveto{\pgfqpoint{3.150856in}{1.008448in}}{\pgfqpoint{3.161455in}{1.004058in}}{\pgfqpoint{3.172505in}{1.004058in}}%
\pgfpathlineto{\pgfqpoint{3.172505in}{1.004058in}}%
\pgfpathclose%
\pgfusepath{stroke}%
\end{pgfscope}%
\begin{pgfscope}%
\pgfpathrectangle{\pgfqpoint{0.847223in}{0.554012in}}{\pgfqpoint{6.200000in}{4.530000in}}%
\pgfusepath{clip}%
\pgfsetbuttcap%
\pgfsetroundjoin%
\pgfsetlinewidth{1.003750pt}%
\definecolor{currentstroke}{rgb}{1.000000,0.000000,0.000000}%
\pgfsetstrokecolor{currentstroke}%
\pgfsetdash{}{0pt}%
\pgfpathmoveto{\pgfqpoint{3.177838in}{1.002359in}}%
\pgfpathcurveto{\pgfqpoint{3.188888in}{1.002359in}}{\pgfqpoint{3.199487in}{1.006749in}}{\pgfqpoint{3.207301in}{1.014563in}}%
\pgfpathcurveto{\pgfqpoint{3.215115in}{1.022377in}}{\pgfqpoint{3.219505in}{1.032976in}}{\pgfqpoint{3.219505in}{1.044026in}}%
\pgfpathcurveto{\pgfqpoint{3.219505in}{1.055076in}}{\pgfqpoint{3.215115in}{1.065675in}}{\pgfqpoint{3.207301in}{1.073489in}}%
\pgfpathcurveto{\pgfqpoint{3.199487in}{1.081302in}}{\pgfqpoint{3.188888in}{1.085692in}}{\pgfqpoint{3.177838in}{1.085692in}}%
\pgfpathcurveto{\pgfqpoint{3.166788in}{1.085692in}}{\pgfqpoint{3.156189in}{1.081302in}}{\pgfqpoint{3.148376in}{1.073489in}}%
\pgfpathcurveto{\pgfqpoint{3.140562in}{1.065675in}}{\pgfqpoint{3.136172in}{1.055076in}}{\pgfqpoint{3.136172in}{1.044026in}}%
\pgfpathcurveto{\pgfqpoint{3.136172in}{1.032976in}}{\pgfqpoint{3.140562in}{1.022377in}}{\pgfqpoint{3.148376in}{1.014563in}}%
\pgfpathcurveto{\pgfqpoint{3.156189in}{1.006749in}}{\pgfqpoint{3.166788in}{1.002359in}}{\pgfqpoint{3.177838in}{1.002359in}}%
\pgfpathlineto{\pgfqpoint{3.177838in}{1.002359in}}%
\pgfpathclose%
\pgfusepath{stroke}%
\end{pgfscope}%
\begin{pgfscope}%
\pgfpathrectangle{\pgfqpoint{0.847223in}{0.554012in}}{\pgfqpoint{6.200000in}{4.530000in}}%
\pgfusepath{clip}%
\pgfsetbuttcap%
\pgfsetroundjoin%
\pgfsetlinewidth{1.003750pt}%
\definecolor{currentstroke}{rgb}{1.000000,0.000000,0.000000}%
\pgfsetstrokecolor{currentstroke}%
\pgfsetdash{}{0pt}%
\pgfpathmoveto{\pgfqpoint{3.183172in}{1.000667in}}%
\pgfpathcurveto{\pgfqpoint{3.194222in}{1.000667in}}{\pgfqpoint{3.204821in}{1.005057in}}{\pgfqpoint{3.212634in}{1.012871in}}%
\pgfpathcurveto{\pgfqpoint{3.220448in}{1.020684in}}{\pgfqpoint{3.224838in}{1.031284in}}{\pgfqpoint{3.224838in}{1.042334in}}%
\pgfpathcurveto{\pgfqpoint{3.224838in}{1.053384in}}{\pgfqpoint{3.220448in}{1.063983in}}{\pgfqpoint{3.212634in}{1.071796in}}%
\pgfpathcurveto{\pgfqpoint{3.204821in}{1.079610in}}{\pgfqpoint{3.194222in}{1.084000in}}{\pgfqpoint{3.183172in}{1.084000in}}%
\pgfpathcurveto{\pgfqpoint{3.172121in}{1.084000in}}{\pgfqpoint{3.161522in}{1.079610in}}{\pgfqpoint{3.153709in}{1.071796in}}%
\pgfpathcurveto{\pgfqpoint{3.145895in}{1.063983in}}{\pgfqpoint{3.141505in}{1.053384in}}{\pgfqpoint{3.141505in}{1.042334in}}%
\pgfpathcurveto{\pgfqpoint{3.141505in}{1.031284in}}{\pgfqpoint{3.145895in}{1.020684in}}{\pgfqpoint{3.153709in}{1.012871in}}%
\pgfpathcurveto{\pgfqpoint{3.161522in}{1.005057in}}{\pgfqpoint{3.172121in}{1.000667in}}{\pgfqpoint{3.183172in}{1.000667in}}%
\pgfpathlineto{\pgfqpoint{3.183172in}{1.000667in}}%
\pgfpathclose%
\pgfusepath{stroke}%
\end{pgfscope}%
\begin{pgfscope}%
\pgfpathrectangle{\pgfqpoint{0.847223in}{0.554012in}}{\pgfqpoint{6.200000in}{4.530000in}}%
\pgfusepath{clip}%
\pgfsetbuttcap%
\pgfsetroundjoin%
\pgfsetlinewidth{1.003750pt}%
\definecolor{currentstroke}{rgb}{1.000000,0.000000,0.000000}%
\pgfsetstrokecolor{currentstroke}%
\pgfsetdash{}{0pt}%
\pgfpathmoveto{\pgfqpoint{3.188505in}{0.998981in}}%
\pgfpathcurveto{\pgfqpoint{3.199555in}{0.998981in}}{\pgfqpoint{3.210154in}{1.003371in}}{\pgfqpoint{3.217968in}{1.011185in}}%
\pgfpathcurveto{\pgfqpoint{3.225781in}{1.018999in}}{\pgfqpoint{3.230171in}{1.029598in}}{\pgfqpoint{3.230171in}{1.040648in}}%
\pgfpathcurveto{\pgfqpoint{3.230171in}{1.051698in}}{\pgfqpoint{3.225781in}{1.062297in}}{\pgfqpoint{3.217968in}{1.070111in}}%
\pgfpathcurveto{\pgfqpoint{3.210154in}{1.077924in}}{\pgfqpoint{3.199555in}{1.082314in}}{\pgfqpoint{3.188505in}{1.082314in}}%
\pgfpathcurveto{\pgfqpoint{3.177455in}{1.082314in}}{\pgfqpoint{3.166856in}{1.077924in}}{\pgfqpoint{3.159042in}{1.070111in}}%
\pgfpathcurveto{\pgfqpoint{3.151228in}{1.062297in}}{\pgfqpoint{3.146838in}{1.051698in}}{\pgfqpoint{3.146838in}{1.040648in}}%
\pgfpathcurveto{\pgfqpoint{3.146838in}{1.029598in}}{\pgfqpoint{3.151228in}{1.018999in}}{\pgfqpoint{3.159042in}{1.011185in}}%
\pgfpathcurveto{\pgfqpoint{3.166856in}{1.003371in}}{\pgfqpoint{3.177455in}{0.998981in}}{\pgfqpoint{3.188505in}{0.998981in}}%
\pgfpathlineto{\pgfqpoint{3.188505in}{0.998981in}}%
\pgfpathclose%
\pgfusepath{stroke}%
\end{pgfscope}%
\begin{pgfscope}%
\pgfpathrectangle{\pgfqpoint{0.847223in}{0.554012in}}{\pgfqpoint{6.200000in}{4.530000in}}%
\pgfusepath{clip}%
\pgfsetbuttcap%
\pgfsetroundjoin%
\pgfsetlinewidth{1.003750pt}%
\definecolor{currentstroke}{rgb}{1.000000,0.000000,0.000000}%
\pgfsetstrokecolor{currentstroke}%
\pgfsetdash{}{0pt}%
\pgfpathmoveto{\pgfqpoint{3.193838in}{0.997302in}}%
\pgfpathcurveto{\pgfqpoint{3.204888in}{0.997302in}}{\pgfqpoint{3.215487in}{1.001692in}}{\pgfqpoint{3.223301in}{1.009505in}}%
\pgfpathcurveto{\pgfqpoint{3.231114in}{1.017319in}}{\pgfqpoint{3.235505in}{1.027918in}}{\pgfqpoint{3.235505in}{1.038968in}}%
\pgfpathcurveto{\pgfqpoint{3.235505in}{1.050018in}}{\pgfqpoint{3.231114in}{1.060617in}}{\pgfqpoint{3.223301in}{1.068431in}}%
\pgfpathcurveto{\pgfqpoint{3.215487in}{1.076245in}}{\pgfqpoint{3.204888in}{1.080635in}}{\pgfqpoint{3.193838in}{1.080635in}}%
\pgfpathcurveto{\pgfqpoint{3.182788in}{1.080635in}}{\pgfqpoint{3.172189in}{1.076245in}}{\pgfqpoint{3.164375in}{1.068431in}}%
\pgfpathcurveto{\pgfqpoint{3.156562in}{1.060617in}}{\pgfqpoint{3.152171in}{1.050018in}}{\pgfqpoint{3.152171in}{1.038968in}}%
\pgfpathcurveto{\pgfqpoint{3.152171in}{1.027918in}}{\pgfqpoint{3.156562in}{1.017319in}}{\pgfqpoint{3.164375in}{1.009505in}}%
\pgfpathcurveto{\pgfqpoint{3.172189in}{1.001692in}}{\pgfqpoint{3.182788in}{0.997302in}}{\pgfqpoint{3.193838in}{0.997302in}}%
\pgfpathlineto{\pgfqpoint{3.193838in}{0.997302in}}%
\pgfpathclose%
\pgfusepath{stroke}%
\end{pgfscope}%
\begin{pgfscope}%
\pgfpathrectangle{\pgfqpoint{0.847223in}{0.554012in}}{\pgfqpoint{6.200000in}{4.530000in}}%
\pgfusepath{clip}%
\pgfsetbuttcap%
\pgfsetroundjoin%
\pgfsetlinewidth{1.003750pt}%
\definecolor{currentstroke}{rgb}{1.000000,0.000000,0.000000}%
\pgfsetstrokecolor{currentstroke}%
\pgfsetdash{}{0pt}%
\pgfpathmoveto{\pgfqpoint{3.199171in}{0.995628in}}%
\pgfpathcurveto{\pgfqpoint{3.210221in}{0.995628in}}{\pgfqpoint{3.220820in}{1.000019in}}{\pgfqpoint{3.228634in}{1.007832in}}%
\pgfpathcurveto{\pgfqpoint{3.236448in}{1.015646in}}{\pgfqpoint{3.240838in}{1.026245in}}{\pgfqpoint{3.240838in}{1.037295in}}%
\pgfpathcurveto{\pgfqpoint{3.240838in}{1.048345in}}{\pgfqpoint{3.236448in}{1.058944in}}{\pgfqpoint{3.228634in}{1.066758in}}%
\pgfpathcurveto{\pgfqpoint{3.220820in}{1.074571in}}{\pgfqpoint{3.210221in}{1.078962in}}{\pgfqpoint{3.199171in}{1.078962in}}%
\pgfpathcurveto{\pgfqpoint{3.188121in}{1.078962in}}{\pgfqpoint{3.177522in}{1.074571in}}{\pgfqpoint{3.169708in}{1.066758in}}%
\pgfpathcurveto{\pgfqpoint{3.161895in}{1.058944in}}{\pgfqpoint{3.157504in}{1.048345in}}{\pgfqpoint{3.157504in}{1.037295in}}%
\pgfpathcurveto{\pgfqpoint{3.157504in}{1.026245in}}{\pgfqpoint{3.161895in}{1.015646in}}{\pgfqpoint{3.169708in}{1.007832in}}%
\pgfpathcurveto{\pgfqpoint{3.177522in}{1.000019in}}{\pgfqpoint{3.188121in}{0.995628in}}{\pgfqpoint{3.199171in}{0.995628in}}%
\pgfpathlineto{\pgfqpoint{3.199171in}{0.995628in}}%
\pgfpathclose%
\pgfusepath{stroke}%
\end{pgfscope}%
\begin{pgfscope}%
\pgfpathrectangle{\pgfqpoint{0.847223in}{0.554012in}}{\pgfqpoint{6.200000in}{4.530000in}}%
\pgfusepath{clip}%
\pgfsetbuttcap%
\pgfsetroundjoin%
\pgfsetlinewidth{1.003750pt}%
\definecolor{currentstroke}{rgb}{1.000000,0.000000,0.000000}%
\pgfsetstrokecolor{currentstroke}%
\pgfsetdash{}{0pt}%
\pgfpathmoveto{\pgfqpoint{3.204504in}{0.993961in}}%
\pgfpathcurveto{\pgfqpoint{3.215555in}{0.993961in}}{\pgfqpoint{3.226154in}{0.998351in}}{\pgfqpoint{3.233967in}{1.006165in}}%
\pgfpathcurveto{\pgfqpoint{3.241781in}{1.013979in}}{\pgfqpoint{3.246171in}{1.024578in}}{\pgfqpoint{3.246171in}{1.035628in}}%
\pgfpathcurveto{\pgfqpoint{3.246171in}{1.046678in}}{\pgfqpoint{3.241781in}{1.057277in}}{\pgfqpoint{3.233967in}{1.065091in}}%
\pgfpathcurveto{\pgfqpoint{3.226154in}{1.072904in}}{\pgfqpoint{3.215555in}{1.077294in}}{\pgfqpoint{3.204504in}{1.077294in}}%
\pgfpathcurveto{\pgfqpoint{3.193454in}{1.077294in}}{\pgfqpoint{3.182855in}{1.072904in}}{\pgfqpoint{3.175042in}{1.065091in}}%
\pgfpathcurveto{\pgfqpoint{3.167228in}{1.057277in}}{\pgfqpoint{3.162838in}{1.046678in}}{\pgfqpoint{3.162838in}{1.035628in}}%
\pgfpathcurveto{\pgfqpoint{3.162838in}{1.024578in}}{\pgfqpoint{3.167228in}{1.013979in}}{\pgfqpoint{3.175042in}{1.006165in}}%
\pgfpathcurveto{\pgfqpoint{3.182855in}{0.998351in}}{\pgfqpoint{3.193454in}{0.993961in}}{\pgfqpoint{3.204504in}{0.993961in}}%
\pgfpathlineto{\pgfqpoint{3.204504in}{0.993961in}}%
\pgfpathclose%
\pgfusepath{stroke}%
\end{pgfscope}%
\begin{pgfscope}%
\pgfpathrectangle{\pgfqpoint{0.847223in}{0.554012in}}{\pgfqpoint{6.200000in}{4.530000in}}%
\pgfusepath{clip}%
\pgfsetbuttcap%
\pgfsetroundjoin%
\pgfsetlinewidth{1.003750pt}%
\definecolor{currentstroke}{rgb}{1.000000,0.000000,0.000000}%
\pgfsetstrokecolor{currentstroke}%
\pgfsetdash{}{0pt}%
\pgfpathmoveto{\pgfqpoint{3.209838in}{0.992300in}}%
\pgfpathcurveto{\pgfqpoint{3.220888in}{0.992300in}}{\pgfqpoint{3.231487in}{0.996690in}}{\pgfqpoint{3.239300in}{1.004504in}}%
\pgfpathcurveto{\pgfqpoint{3.247114in}{1.012318in}}{\pgfqpoint{3.251504in}{1.022917in}}{\pgfqpoint{3.251504in}{1.033967in}}%
\pgfpathcurveto{\pgfqpoint{3.251504in}{1.045017in}}{\pgfqpoint{3.247114in}{1.055616in}}{\pgfqpoint{3.239300in}{1.063430in}}%
\pgfpathcurveto{\pgfqpoint{3.231487in}{1.071243in}}{\pgfqpoint{3.220888in}{1.075634in}}{\pgfqpoint{3.209838in}{1.075634in}}%
\pgfpathcurveto{\pgfqpoint{3.198787in}{1.075634in}}{\pgfqpoint{3.188188in}{1.071243in}}{\pgfqpoint{3.180375in}{1.063430in}}%
\pgfpathcurveto{\pgfqpoint{3.172561in}{1.055616in}}{\pgfqpoint{3.168171in}{1.045017in}}{\pgfqpoint{3.168171in}{1.033967in}}%
\pgfpathcurveto{\pgfqpoint{3.168171in}{1.022917in}}{\pgfqpoint{3.172561in}{1.012318in}}{\pgfqpoint{3.180375in}{1.004504in}}%
\pgfpathcurveto{\pgfqpoint{3.188188in}{0.996690in}}{\pgfqpoint{3.198787in}{0.992300in}}{\pgfqpoint{3.209838in}{0.992300in}}%
\pgfpathlineto{\pgfqpoint{3.209838in}{0.992300in}}%
\pgfpathclose%
\pgfusepath{stroke}%
\end{pgfscope}%
\begin{pgfscope}%
\pgfpathrectangle{\pgfqpoint{0.847223in}{0.554012in}}{\pgfqpoint{6.200000in}{4.530000in}}%
\pgfusepath{clip}%
\pgfsetbuttcap%
\pgfsetroundjoin%
\pgfsetlinewidth{1.003750pt}%
\definecolor{currentstroke}{rgb}{1.000000,0.000000,0.000000}%
\pgfsetstrokecolor{currentstroke}%
\pgfsetdash{}{0pt}%
\pgfpathmoveto{\pgfqpoint{3.215171in}{0.990645in}}%
\pgfpathcurveto{\pgfqpoint{3.226221in}{0.990645in}}{\pgfqpoint{3.236820in}{0.995036in}}{\pgfqpoint{3.244634in}{1.002849in}}%
\pgfpathcurveto{\pgfqpoint{3.252447in}{1.010663in}}{\pgfqpoint{3.256837in}{1.021262in}}{\pgfqpoint{3.256837in}{1.032312in}}%
\pgfpathcurveto{\pgfqpoint{3.256837in}{1.043362in}}{\pgfqpoint{3.252447in}{1.053961in}}{\pgfqpoint{3.244634in}{1.061775in}}%
\pgfpathcurveto{\pgfqpoint{3.236820in}{1.069588in}}{\pgfqpoint{3.226221in}{1.073979in}}{\pgfqpoint{3.215171in}{1.073979in}}%
\pgfpathcurveto{\pgfqpoint{3.204121in}{1.073979in}}{\pgfqpoint{3.193522in}{1.069588in}}{\pgfqpoint{3.185708in}{1.061775in}}%
\pgfpathcurveto{\pgfqpoint{3.177894in}{1.053961in}}{\pgfqpoint{3.173504in}{1.043362in}}{\pgfqpoint{3.173504in}{1.032312in}}%
\pgfpathcurveto{\pgfqpoint{3.173504in}{1.021262in}}{\pgfqpoint{3.177894in}{1.010663in}}{\pgfqpoint{3.185708in}{1.002849in}}%
\pgfpathcurveto{\pgfqpoint{3.193522in}{0.995036in}}{\pgfqpoint{3.204121in}{0.990645in}}{\pgfqpoint{3.215171in}{0.990645in}}%
\pgfpathlineto{\pgfqpoint{3.215171in}{0.990645in}}%
\pgfpathclose%
\pgfusepath{stroke}%
\end{pgfscope}%
\begin{pgfscope}%
\pgfpathrectangle{\pgfqpoint{0.847223in}{0.554012in}}{\pgfqpoint{6.200000in}{4.530000in}}%
\pgfusepath{clip}%
\pgfsetbuttcap%
\pgfsetroundjoin%
\pgfsetlinewidth{1.003750pt}%
\definecolor{currentstroke}{rgb}{1.000000,0.000000,0.000000}%
\pgfsetstrokecolor{currentstroke}%
\pgfsetdash{}{0pt}%
\pgfpathmoveto{\pgfqpoint{3.220504in}{0.988997in}}%
\pgfpathcurveto{\pgfqpoint{3.231554in}{0.988997in}}{\pgfqpoint{3.242153in}{0.993387in}}{\pgfqpoint{3.249967in}{1.001200in}}%
\pgfpathcurveto{\pgfqpoint{3.257780in}{1.009014in}}{\pgfqpoint{3.262171in}{1.019613in}}{\pgfqpoint{3.262171in}{1.030663in}}%
\pgfpathcurveto{\pgfqpoint{3.262171in}{1.041713in}}{\pgfqpoint{3.257780in}{1.052312in}}{\pgfqpoint{3.249967in}{1.060126in}}%
\pgfpathcurveto{\pgfqpoint{3.242153in}{1.067940in}}{\pgfqpoint{3.231554in}{1.072330in}}{\pgfqpoint{3.220504in}{1.072330in}}%
\pgfpathcurveto{\pgfqpoint{3.209454in}{1.072330in}}{\pgfqpoint{3.198855in}{1.067940in}}{\pgfqpoint{3.191041in}{1.060126in}}%
\pgfpathcurveto{\pgfqpoint{3.183228in}{1.052312in}}{\pgfqpoint{3.178837in}{1.041713in}}{\pgfqpoint{3.178837in}{1.030663in}}%
\pgfpathcurveto{\pgfqpoint{3.178837in}{1.019613in}}{\pgfqpoint{3.183228in}{1.009014in}}{\pgfqpoint{3.191041in}{1.001200in}}%
\pgfpathcurveto{\pgfqpoint{3.198855in}{0.993387in}}{\pgfqpoint{3.209454in}{0.988997in}}{\pgfqpoint{3.220504in}{0.988997in}}%
\pgfpathlineto{\pgfqpoint{3.220504in}{0.988997in}}%
\pgfpathclose%
\pgfusepath{stroke}%
\end{pgfscope}%
\begin{pgfscope}%
\pgfpathrectangle{\pgfqpoint{0.847223in}{0.554012in}}{\pgfqpoint{6.200000in}{4.530000in}}%
\pgfusepath{clip}%
\pgfsetbuttcap%
\pgfsetroundjoin%
\pgfsetlinewidth{1.003750pt}%
\definecolor{currentstroke}{rgb}{1.000000,0.000000,0.000000}%
\pgfsetstrokecolor{currentstroke}%
\pgfsetdash{}{0pt}%
\pgfpathmoveto{\pgfqpoint{3.225837in}{0.987354in}}%
\pgfpathcurveto{\pgfqpoint{3.236887in}{0.987354in}}{\pgfqpoint{3.247486in}{0.991744in}}{\pgfqpoint{3.255300in}{0.999558in}}%
\pgfpathcurveto{\pgfqpoint{3.263114in}{1.007371in}}{\pgfqpoint{3.267504in}{1.017970in}}{\pgfqpoint{3.267504in}{1.029021in}}%
\pgfpathcurveto{\pgfqpoint{3.267504in}{1.040071in}}{\pgfqpoint{3.263114in}{1.050670in}}{\pgfqpoint{3.255300in}{1.058483in}}%
\pgfpathcurveto{\pgfqpoint{3.247486in}{1.066297in}}{\pgfqpoint{3.236887in}{1.070687in}}{\pgfqpoint{3.225837in}{1.070687in}}%
\pgfpathcurveto{\pgfqpoint{3.214787in}{1.070687in}}{\pgfqpoint{3.204188in}{1.066297in}}{\pgfqpoint{3.196374in}{1.058483in}}%
\pgfpathcurveto{\pgfqpoint{3.188561in}{1.050670in}}{\pgfqpoint{3.184171in}{1.040071in}}{\pgfqpoint{3.184171in}{1.029021in}}%
\pgfpathcurveto{\pgfqpoint{3.184171in}{1.017970in}}{\pgfqpoint{3.188561in}{1.007371in}}{\pgfqpoint{3.196374in}{0.999558in}}%
\pgfpathcurveto{\pgfqpoint{3.204188in}{0.991744in}}{\pgfqpoint{3.214787in}{0.987354in}}{\pgfqpoint{3.225837in}{0.987354in}}%
\pgfpathlineto{\pgfqpoint{3.225837in}{0.987354in}}%
\pgfpathclose%
\pgfusepath{stroke}%
\end{pgfscope}%
\begin{pgfscope}%
\pgfpathrectangle{\pgfqpoint{0.847223in}{0.554012in}}{\pgfqpoint{6.200000in}{4.530000in}}%
\pgfusepath{clip}%
\pgfsetbuttcap%
\pgfsetroundjoin%
\pgfsetlinewidth{1.003750pt}%
\definecolor{currentstroke}{rgb}{1.000000,0.000000,0.000000}%
\pgfsetstrokecolor{currentstroke}%
\pgfsetdash{}{0pt}%
\pgfpathmoveto{\pgfqpoint{3.231170in}{0.985717in}}%
\pgfpathcurveto{\pgfqpoint{3.242221in}{0.985717in}}{\pgfqpoint{3.252820in}{0.990108in}}{\pgfqpoint{3.260633in}{0.997921in}}%
\pgfpathcurveto{\pgfqpoint{3.268447in}{1.005735in}}{\pgfqpoint{3.272837in}{1.016334in}}{\pgfqpoint{3.272837in}{1.027384in}}%
\pgfpathcurveto{\pgfqpoint{3.272837in}{1.038434in}}{\pgfqpoint{3.268447in}{1.049033in}}{\pgfqpoint{3.260633in}{1.056847in}}%
\pgfpathcurveto{\pgfqpoint{3.252820in}{1.064660in}}{\pgfqpoint{3.242221in}{1.069051in}}{\pgfqpoint{3.231170in}{1.069051in}}%
\pgfpathcurveto{\pgfqpoint{3.220120in}{1.069051in}}{\pgfqpoint{3.209521in}{1.064660in}}{\pgfqpoint{3.201708in}{1.056847in}}%
\pgfpathcurveto{\pgfqpoint{3.193894in}{1.049033in}}{\pgfqpoint{3.189504in}{1.038434in}}{\pgfqpoint{3.189504in}{1.027384in}}%
\pgfpathcurveto{\pgfqpoint{3.189504in}{1.016334in}}{\pgfqpoint{3.193894in}{1.005735in}}{\pgfqpoint{3.201708in}{0.997921in}}%
\pgfpathcurveto{\pgfqpoint{3.209521in}{0.990108in}}{\pgfqpoint{3.220120in}{0.985717in}}{\pgfqpoint{3.231170in}{0.985717in}}%
\pgfpathlineto{\pgfqpoint{3.231170in}{0.985717in}}%
\pgfpathclose%
\pgfusepath{stroke}%
\end{pgfscope}%
\begin{pgfscope}%
\pgfpathrectangle{\pgfqpoint{0.847223in}{0.554012in}}{\pgfqpoint{6.200000in}{4.530000in}}%
\pgfusepath{clip}%
\pgfsetbuttcap%
\pgfsetroundjoin%
\pgfsetlinewidth{1.003750pt}%
\definecolor{currentstroke}{rgb}{1.000000,0.000000,0.000000}%
\pgfsetstrokecolor{currentstroke}%
\pgfsetdash{}{0pt}%
\pgfpathmoveto{\pgfqpoint{3.236504in}{0.984087in}}%
\pgfpathcurveto{\pgfqpoint{3.247554in}{0.984087in}}{\pgfqpoint{3.258153in}{0.988477in}}{\pgfqpoint{3.265966in}{0.996290in}}%
\pgfpathcurveto{\pgfqpoint{3.273780in}{1.004104in}}{\pgfqpoint{3.278170in}{1.014703in}}{\pgfqpoint{3.278170in}{1.025753in}}%
\pgfpathcurveto{\pgfqpoint{3.278170in}{1.036803in}}{\pgfqpoint{3.273780in}{1.047402in}}{\pgfqpoint{3.265966in}{1.055216in}}%
\pgfpathcurveto{\pgfqpoint{3.258153in}{1.063030in}}{\pgfqpoint{3.247554in}{1.067420in}}{\pgfqpoint{3.236504in}{1.067420in}}%
\pgfpathcurveto{\pgfqpoint{3.225454in}{1.067420in}}{\pgfqpoint{3.214855in}{1.063030in}}{\pgfqpoint{3.207041in}{1.055216in}}%
\pgfpathcurveto{\pgfqpoint{3.199227in}{1.047402in}}{\pgfqpoint{3.194837in}{1.036803in}}{\pgfqpoint{3.194837in}{1.025753in}}%
\pgfpathcurveto{\pgfqpoint{3.194837in}{1.014703in}}{\pgfqpoint{3.199227in}{1.004104in}}{\pgfqpoint{3.207041in}{0.996290in}}%
\pgfpathcurveto{\pgfqpoint{3.214855in}{0.988477in}}{\pgfqpoint{3.225454in}{0.984087in}}{\pgfqpoint{3.236504in}{0.984087in}}%
\pgfpathlineto{\pgfqpoint{3.236504in}{0.984087in}}%
\pgfpathclose%
\pgfusepath{stroke}%
\end{pgfscope}%
\begin{pgfscope}%
\pgfpathrectangle{\pgfqpoint{0.847223in}{0.554012in}}{\pgfqpoint{6.200000in}{4.530000in}}%
\pgfusepath{clip}%
\pgfsetbuttcap%
\pgfsetroundjoin%
\pgfsetlinewidth{1.003750pt}%
\definecolor{currentstroke}{rgb}{1.000000,0.000000,0.000000}%
\pgfsetstrokecolor{currentstroke}%
\pgfsetdash{}{0pt}%
\pgfpathmoveto{\pgfqpoint{3.241837in}{0.982462in}}%
\pgfpathcurveto{\pgfqpoint{3.252887in}{0.982462in}}{\pgfqpoint{3.263486in}{0.986852in}}{\pgfqpoint{3.271300in}{0.994666in}}%
\pgfpathcurveto{\pgfqpoint{3.279113in}{1.002479in}}{\pgfqpoint{3.283504in}{1.013078in}}{\pgfqpoint{3.283504in}{1.024129in}}%
\pgfpathcurveto{\pgfqpoint{3.283504in}{1.035179in}}{\pgfqpoint{3.279113in}{1.045778in}}{\pgfqpoint{3.271300in}{1.053591in}}%
\pgfpathcurveto{\pgfqpoint{3.263486in}{1.061405in}}{\pgfqpoint{3.252887in}{1.065795in}}{\pgfqpoint{3.241837in}{1.065795in}}%
\pgfpathcurveto{\pgfqpoint{3.230787in}{1.065795in}}{\pgfqpoint{3.220188in}{1.061405in}}{\pgfqpoint{3.212374in}{1.053591in}}%
\pgfpathcurveto{\pgfqpoint{3.204560in}{1.045778in}}{\pgfqpoint{3.200170in}{1.035179in}}{\pgfqpoint{3.200170in}{1.024129in}}%
\pgfpathcurveto{\pgfqpoint{3.200170in}{1.013078in}}{\pgfqpoint{3.204560in}{1.002479in}}{\pgfqpoint{3.212374in}{0.994666in}}%
\pgfpathcurveto{\pgfqpoint{3.220188in}{0.986852in}}{\pgfqpoint{3.230787in}{0.982462in}}{\pgfqpoint{3.241837in}{0.982462in}}%
\pgfpathlineto{\pgfqpoint{3.241837in}{0.982462in}}%
\pgfpathclose%
\pgfusepath{stroke}%
\end{pgfscope}%
\begin{pgfscope}%
\pgfpathrectangle{\pgfqpoint{0.847223in}{0.554012in}}{\pgfqpoint{6.200000in}{4.530000in}}%
\pgfusepath{clip}%
\pgfsetbuttcap%
\pgfsetroundjoin%
\pgfsetlinewidth{1.003750pt}%
\definecolor{currentstroke}{rgb}{1.000000,0.000000,0.000000}%
\pgfsetstrokecolor{currentstroke}%
\pgfsetdash{}{0pt}%
\pgfpathmoveto{\pgfqpoint{3.247170in}{0.980843in}}%
\pgfpathcurveto{\pgfqpoint{3.258220in}{0.980843in}}{\pgfqpoint{3.268819in}{0.985233in}}{\pgfqpoint{3.276633in}{0.993047in}}%
\pgfpathcurveto{\pgfqpoint{3.284447in}{1.000861in}}{\pgfqpoint{3.288837in}{1.011460in}}{\pgfqpoint{3.288837in}{1.022510in}}%
\pgfpathcurveto{\pgfqpoint{3.288837in}{1.033560in}}{\pgfqpoint{3.284447in}{1.044159in}}{\pgfqpoint{3.276633in}{1.051973in}}%
\pgfpathcurveto{\pgfqpoint{3.268819in}{1.059786in}}{\pgfqpoint{3.258220in}{1.064176in}}{\pgfqpoint{3.247170in}{1.064176in}}%
\pgfpathcurveto{\pgfqpoint{3.236120in}{1.064176in}}{\pgfqpoint{3.225521in}{1.059786in}}{\pgfqpoint{3.217707in}{1.051973in}}%
\pgfpathcurveto{\pgfqpoint{3.209894in}{1.044159in}}{\pgfqpoint{3.205503in}{1.033560in}}{\pgfqpoint{3.205503in}{1.022510in}}%
\pgfpathcurveto{\pgfqpoint{3.205503in}{1.011460in}}{\pgfqpoint{3.209894in}{1.000861in}}{\pgfqpoint{3.217707in}{0.993047in}}%
\pgfpathcurveto{\pgfqpoint{3.225521in}{0.985233in}}{\pgfqpoint{3.236120in}{0.980843in}}{\pgfqpoint{3.247170in}{0.980843in}}%
\pgfpathlineto{\pgfqpoint{3.247170in}{0.980843in}}%
\pgfpathclose%
\pgfusepath{stroke}%
\end{pgfscope}%
\begin{pgfscope}%
\pgfpathrectangle{\pgfqpoint{0.847223in}{0.554012in}}{\pgfqpoint{6.200000in}{4.530000in}}%
\pgfusepath{clip}%
\pgfsetbuttcap%
\pgfsetroundjoin%
\pgfsetlinewidth{1.003750pt}%
\definecolor{currentstroke}{rgb}{1.000000,0.000000,0.000000}%
\pgfsetstrokecolor{currentstroke}%
\pgfsetdash{}{0pt}%
\pgfpathmoveto{\pgfqpoint{3.252503in}{0.979230in}}%
\pgfpathcurveto{\pgfqpoint{3.263553in}{0.979230in}}{\pgfqpoint{3.274152in}{0.983620in}}{\pgfqpoint{3.281966in}{0.991434in}}%
\pgfpathcurveto{\pgfqpoint{3.289780in}{0.999248in}}{\pgfqpoint{3.294170in}{1.009847in}}{\pgfqpoint{3.294170in}{1.020897in}}%
\pgfpathcurveto{\pgfqpoint{3.294170in}{1.031947in}}{\pgfqpoint{3.289780in}{1.042546in}}{\pgfqpoint{3.281966in}{1.050360in}}%
\pgfpathcurveto{\pgfqpoint{3.274152in}{1.058173in}}{\pgfqpoint{3.263553in}{1.062564in}}{\pgfqpoint{3.252503in}{1.062564in}}%
\pgfpathcurveto{\pgfqpoint{3.241453in}{1.062564in}}{\pgfqpoint{3.230854in}{1.058173in}}{\pgfqpoint{3.223041in}{1.050360in}}%
\pgfpathcurveto{\pgfqpoint{3.215227in}{1.042546in}}{\pgfqpoint{3.210837in}{1.031947in}}{\pgfqpoint{3.210837in}{1.020897in}}%
\pgfpathcurveto{\pgfqpoint{3.210837in}{1.009847in}}{\pgfqpoint{3.215227in}{0.999248in}}{\pgfqpoint{3.223041in}{0.991434in}}%
\pgfpathcurveto{\pgfqpoint{3.230854in}{0.983620in}}{\pgfqpoint{3.241453in}{0.979230in}}{\pgfqpoint{3.252503in}{0.979230in}}%
\pgfpathlineto{\pgfqpoint{3.252503in}{0.979230in}}%
\pgfpathclose%
\pgfusepath{stroke}%
\end{pgfscope}%
\begin{pgfscope}%
\pgfpathrectangle{\pgfqpoint{0.847223in}{0.554012in}}{\pgfqpoint{6.200000in}{4.530000in}}%
\pgfusepath{clip}%
\pgfsetbuttcap%
\pgfsetroundjoin%
\pgfsetlinewidth{1.003750pt}%
\definecolor{currentstroke}{rgb}{1.000000,0.000000,0.000000}%
\pgfsetstrokecolor{currentstroke}%
\pgfsetdash{}{0pt}%
\pgfpathmoveto{\pgfqpoint{3.257837in}{0.977623in}}%
\pgfpathcurveto{\pgfqpoint{3.268887in}{0.977623in}}{\pgfqpoint{3.279486in}{0.982013in}}{\pgfqpoint{3.287299in}{0.989827in}}%
\pgfpathcurveto{\pgfqpoint{3.295113in}{0.997641in}}{\pgfqpoint{3.299503in}{1.008240in}}{\pgfqpoint{3.299503in}{1.019290in}}%
\pgfpathcurveto{\pgfqpoint{3.299503in}{1.030340in}}{\pgfqpoint{3.295113in}{1.040939in}}{\pgfqpoint{3.287299in}{1.048753in}}%
\pgfpathcurveto{\pgfqpoint{3.279486in}{1.056566in}}{\pgfqpoint{3.268887in}{1.060957in}}{\pgfqpoint{3.257837in}{1.060957in}}%
\pgfpathcurveto{\pgfqpoint{3.246786in}{1.060957in}}{\pgfqpoint{3.236187in}{1.056566in}}{\pgfqpoint{3.228374in}{1.048753in}}%
\pgfpathcurveto{\pgfqpoint{3.220560in}{1.040939in}}{\pgfqpoint{3.216170in}{1.030340in}}{\pgfqpoint{3.216170in}{1.019290in}}%
\pgfpathcurveto{\pgfqpoint{3.216170in}{1.008240in}}{\pgfqpoint{3.220560in}{0.997641in}}{\pgfqpoint{3.228374in}{0.989827in}}%
\pgfpathcurveto{\pgfqpoint{3.236187in}{0.982013in}}{\pgfqpoint{3.246786in}{0.977623in}}{\pgfqpoint{3.257837in}{0.977623in}}%
\pgfpathlineto{\pgfqpoint{3.257837in}{0.977623in}}%
\pgfpathclose%
\pgfusepath{stroke}%
\end{pgfscope}%
\begin{pgfscope}%
\pgfpathrectangle{\pgfqpoint{0.847223in}{0.554012in}}{\pgfqpoint{6.200000in}{4.530000in}}%
\pgfusepath{clip}%
\pgfsetbuttcap%
\pgfsetroundjoin%
\pgfsetlinewidth{1.003750pt}%
\definecolor{currentstroke}{rgb}{1.000000,0.000000,0.000000}%
\pgfsetstrokecolor{currentstroke}%
\pgfsetdash{}{0pt}%
\pgfpathmoveto{\pgfqpoint{3.263170in}{0.976022in}}%
\pgfpathcurveto{\pgfqpoint{3.274220in}{0.976022in}}{\pgfqpoint{3.284819in}{0.980412in}}{\pgfqpoint{3.292633in}{0.988226in}}%
\pgfpathcurveto{\pgfqpoint{3.300446in}{0.996040in}}{\pgfqpoint{3.304836in}{1.006639in}}{\pgfqpoint{3.304836in}{1.017689in}}%
\pgfpathcurveto{\pgfqpoint{3.304836in}{1.028739in}}{\pgfqpoint{3.300446in}{1.039338in}}{\pgfqpoint{3.292633in}{1.047151in}}%
\pgfpathcurveto{\pgfqpoint{3.284819in}{1.054965in}}{\pgfqpoint{3.274220in}{1.059355in}}{\pgfqpoint{3.263170in}{1.059355in}}%
\pgfpathcurveto{\pgfqpoint{3.252120in}{1.059355in}}{\pgfqpoint{3.241521in}{1.054965in}}{\pgfqpoint{3.233707in}{1.047151in}}%
\pgfpathcurveto{\pgfqpoint{3.225893in}{1.039338in}}{\pgfqpoint{3.221503in}{1.028739in}}{\pgfqpoint{3.221503in}{1.017689in}}%
\pgfpathcurveto{\pgfqpoint{3.221503in}{1.006639in}}{\pgfqpoint{3.225893in}{0.996040in}}{\pgfqpoint{3.233707in}{0.988226in}}%
\pgfpathcurveto{\pgfqpoint{3.241521in}{0.980412in}}{\pgfqpoint{3.252120in}{0.976022in}}{\pgfqpoint{3.263170in}{0.976022in}}%
\pgfpathlineto{\pgfqpoint{3.263170in}{0.976022in}}%
\pgfpathclose%
\pgfusepath{stroke}%
\end{pgfscope}%
\begin{pgfscope}%
\pgfpathrectangle{\pgfqpoint{0.847223in}{0.554012in}}{\pgfqpoint{6.200000in}{4.530000in}}%
\pgfusepath{clip}%
\pgfsetbuttcap%
\pgfsetroundjoin%
\pgfsetlinewidth{1.003750pt}%
\definecolor{currentstroke}{rgb}{1.000000,0.000000,0.000000}%
\pgfsetstrokecolor{currentstroke}%
\pgfsetdash{}{0pt}%
\pgfpathmoveto{\pgfqpoint{3.268503in}{0.974427in}}%
\pgfpathcurveto{\pgfqpoint{3.279553in}{0.974427in}}{\pgfqpoint{3.290152in}{0.978817in}}{\pgfqpoint{3.297966in}{0.986631in}}%
\pgfpathcurveto{\pgfqpoint{3.305779in}{0.994444in}}{\pgfqpoint{3.310170in}{1.005043in}}{\pgfqpoint{3.310170in}{1.016093in}}%
\pgfpathcurveto{\pgfqpoint{3.310170in}{1.027143in}}{\pgfqpoint{3.305779in}{1.037742in}}{\pgfqpoint{3.297966in}{1.045556in}}%
\pgfpathcurveto{\pgfqpoint{3.290152in}{1.053370in}}{\pgfqpoint{3.279553in}{1.057760in}}{\pgfqpoint{3.268503in}{1.057760in}}%
\pgfpathcurveto{\pgfqpoint{3.257453in}{1.057760in}}{\pgfqpoint{3.246854in}{1.053370in}}{\pgfqpoint{3.239040in}{1.045556in}}%
\pgfpathcurveto{\pgfqpoint{3.231227in}{1.037742in}}{\pgfqpoint{3.226836in}{1.027143in}}{\pgfqpoint{3.226836in}{1.016093in}}%
\pgfpathcurveto{\pgfqpoint{3.226836in}{1.005043in}}{\pgfqpoint{3.231227in}{0.994444in}}{\pgfqpoint{3.239040in}{0.986631in}}%
\pgfpathcurveto{\pgfqpoint{3.246854in}{0.978817in}}{\pgfqpoint{3.257453in}{0.974427in}}{\pgfqpoint{3.268503in}{0.974427in}}%
\pgfpathlineto{\pgfqpoint{3.268503in}{0.974427in}}%
\pgfpathclose%
\pgfusepath{stroke}%
\end{pgfscope}%
\begin{pgfscope}%
\pgfpathrectangle{\pgfqpoint{0.847223in}{0.554012in}}{\pgfqpoint{6.200000in}{4.530000in}}%
\pgfusepath{clip}%
\pgfsetbuttcap%
\pgfsetroundjoin%
\pgfsetlinewidth{1.003750pt}%
\definecolor{currentstroke}{rgb}{1.000000,0.000000,0.000000}%
\pgfsetstrokecolor{currentstroke}%
\pgfsetdash{}{0pt}%
\pgfpathmoveto{\pgfqpoint{3.273836in}{0.972837in}}%
\pgfpathcurveto{\pgfqpoint{3.284886in}{0.972837in}}{\pgfqpoint{3.295485in}{0.977227in}}{\pgfqpoint{3.303299in}{0.985041in}}%
\pgfpathcurveto{\pgfqpoint{3.311113in}{0.992855in}}{\pgfqpoint{3.315503in}{1.003454in}}{\pgfqpoint{3.315503in}{1.014504in}}%
\pgfpathcurveto{\pgfqpoint{3.315503in}{1.025554in}}{\pgfqpoint{3.311113in}{1.036153in}}{\pgfqpoint{3.303299in}{1.043966in}}%
\pgfpathcurveto{\pgfqpoint{3.295485in}{1.051780in}}{\pgfqpoint{3.284886in}{1.056170in}}{\pgfqpoint{3.273836in}{1.056170in}}%
\pgfpathcurveto{\pgfqpoint{3.262786in}{1.056170in}}{\pgfqpoint{3.252187in}{1.051780in}}{\pgfqpoint{3.244373in}{1.043966in}}%
\pgfpathcurveto{\pgfqpoint{3.236560in}{1.036153in}}{\pgfqpoint{3.232170in}{1.025554in}}{\pgfqpoint{3.232170in}{1.014504in}}%
\pgfpathcurveto{\pgfqpoint{3.232170in}{1.003454in}}{\pgfqpoint{3.236560in}{0.992855in}}{\pgfqpoint{3.244373in}{0.985041in}}%
\pgfpathcurveto{\pgfqpoint{3.252187in}{0.977227in}}{\pgfqpoint{3.262786in}{0.972837in}}{\pgfqpoint{3.273836in}{0.972837in}}%
\pgfpathlineto{\pgfqpoint{3.273836in}{0.972837in}}%
\pgfpathclose%
\pgfusepath{stroke}%
\end{pgfscope}%
\begin{pgfscope}%
\pgfpathrectangle{\pgfqpoint{0.847223in}{0.554012in}}{\pgfqpoint{6.200000in}{4.530000in}}%
\pgfusepath{clip}%
\pgfsetbuttcap%
\pgfsetroundjoin%
\pgfsetlinewidth{1.003750pt}%
\definecolor{currentstroke}{rgb}{1.000000,0.000000,0.000000}%
\pgfsetstrokecolor{currentstroke}%
\pgfsetdash{}{0pt}%
\pgfpathmoveto{\pgfqpoint{3.279169in}{0.971253in}}%
\pgfpathcurveto{\pgfqpoint{3.290220in}{0.971253in}}{\pgfqpoint{3.300819in}{0.975643in}}{\pgfqpoint{3.308632in}{0.983457in}}%
\pgfpathcurveto{\pgfqpoint{3.316446in}{0.991271in}}{\pgfqpoint{3.320836in}{1.001870in}}{\pgfqpoint{3.320836in}{1.012920in}}%
\pgfpathcurveto{\pgfqpoint{3.320836in}{1.023970in}}{\pgfqpoint{3.316446in}{1.034569in}}{\pgfqpoint{3.308632in}{1.042383in}}%
\pgfpathcurveto{\pgfqpoint{3.300819in}{1.050196in}}{\pgfqpoint{3.290220in}{1.054586in}}{\pgfqpoint{3.279169in}{1.054586in}}%
\pgfpathcurveto{\pgfqpoint{3.268119in}{1.054586in}}{\pgfqpoint{3.257520in}{1.050196in}}{\pgfqpoint{3.249707in}{1.042383in}}%
\pgfpathcurveto{\pgfqpoint{3.241893in}{1.034569in}}{\pgfqpoint{3.237503in}{1.023970in}}{\pgfqpoint{3.237503in}{1.012920in}}%
\pgfpathcurveto{\pgfqpoint{3.237503in}{1.001870in}}{\pgfqpoint{3.241893in}{0.991271in}}{\pgfqpoint{3.249707in}{0.983457in}}%
\pgfpathcurveto{\pgfqpoint{3.257520in}{0.975643in}}{\pgfqpoint{3.268119in}{0.971253in}}{\pgfqpoint{3.279169in}{0.971253in}}%
\pgfpathlineto{\pgfqpoint{3.279169in}{0.971253in}}%
\pgfpathclose%
\pgfusepath{stroke}%
\end{pgfscope}%
\begin{pgfscope}%
\pgfpathrectangle{\pgfqpoint{0.847223in}{0.554012in}}{\pgfqpoint{6.200000in}{4.530000in}}%
\pgfusepath{clip}%
\pgfsetbuttcap%
\pgfsetroundjoin%
\pgfsetlinewidth{1.003750pt}%
\definecolor{currentstroke}{rgb}{1.000000,0.000000,0.000000}%
\pgfsetstrokecolor{currentstroke}%
\pgfsetdash{}{0pt}%
\pgfpathmoveto{\pgfqpoint{3.284503in}{0.969675in}}%
\pgfpathcurveto{\pgfqpoint{3.295553in}{0.969675in}}{\pgfqpoint{3.306152in}{0.974065in}}{\pgfqpoint{3.313965in}{0.981879in}}%
\pgfpathcurveto{\pgfqpoint{3.321779in}{0.989692in}}{\pgfqpoint{3.326169in}{1.000292in}}{\pgfqpoint{3.326169in}{1.011342in}}%
\pgfpathcurveto{\pgfqpoint{3.326169in}{1.022392in}}{\pgfqpoint{3.321779in}{1.032991in}}{\pgfqpoint{3.313965in}{1.040804in}}%
\pgfpathcurveto{\pgfqpoint{3.306152in}{1.048618in}}{\pgfqpoint{3.295553in}{1.053008in}}{\pgfqpoint{3.284503in}{1.053008in}}%
\pgfpathcurveto{\pgfqpoint{3.273452in}{1.053008in}}{\pgfqpoint{3.262853in}{1.048618in}}{\pgfqpoint{3.255040in}{1.040804in}}%
\pgfpathcurveto{\pgfqpoint{3.247226in}{1.032991in}}{\pgfqpoint{3.242836in}{1.022392in}}{\pgfqpoint{3.242836in}{1.011342in}}%
\pgfpathcurveto{\pgfqpoint{3.242836in}{1.000292in}}{\pgfqpoint{3.247226in}{0.989692in}}{\pgfqpoint{3.255040in}{0.981879in}}%
\pgfpathcurveto{\pgfqpoint{3.262853in}{0.974065in}}{\pgfqpoint{3.273452in}{0.969675in}}{\pgfqpoint{3.284503in}{0.969675in}}%
\pgfpathlineto{\pgfqpoint{3.284503in}{0.969675in}}%
\pgfpathclose%
\pgfusepath{stroke}%
\end{pgfscope}%
\begin{pgfscope}%
\pgfpathrectangle{\pgfqpoint{0.847223in}{0.554012in}}{\pgfqpoint{6.200000in}{4.530000in}}%
\pgfusepath{clip}%
\pgfsetbuttcap%
\pgfsetroundjoin%
\pgfsetlinewidth{1.003750pt}%
\definecolor{currentstroke}{rgb}{1.000000,0.000000,0.000000}%
\pgfsetstrokecolor{currentstroke}%
\pgfsetdash{}{0pt}%
\pgfpathmoveto{\pgfqpoint{3.289836in}{0.968102in}}%
\pgfpathcurveto{\pgfqpoint{3.300886in}{0.968102in}}{\pgfqpoint{3.311485in}{0.972493in}}{\pgfqpoint{3.319299in}{0.980306in}}%
\pgfpathcurveto{\pgfqpoint{3.327112in}{0.988120in}}{\pgfqpoint{3.331502in}{0.998719in}}{\pgfqpoint{3.331502in}{1.009769in}}%
\pgfpathcurveto{\pgfqpoint{3.331502in}{1.020819in}}{\pgfqpoint{3.327112in}{1.031418in}}{\pgfqpoint{3.319299in}{1.039232in}}%
\pgfpathcurveto{\pgfqpoint{3.311485in}{1.047046in}}{\pgfqpoint{3.300886in}{1.051436in}}{\pgfqpoint{3.289836in}{1.051436in}}%
\pgfpathcurveto{\pgfqpoint{3.278786in}{1.051436in}}{\pgfqpoint{3.268187in}{1.047046in}}{\pgfqpoint{3.260373in}{1.039232in}}%
\pgfpathcurveto{\pgfqpoint{3.252559in}{1.031418in}}{\pgfqpoint{3.248169in}{1.020819in}}{\pgfqpoint{3.248169in}{1.009769in}}%
\pgfpathcurveto{\pgfqpoint{3.248169in}{0.998719in}}{\pgfqpoint{3.252559in}{0.988120in}}{\pgfqpoint{3.260373in}{0.980306in}}%
\pgfpathcurveto{\pgfqpoint{3.268187in}{0.972493in}}{\pgfqpoint{3.278786in}{0.968102in}}{\pgfqpoint{3.289836in}{0.968102in}}%
\pgfpathlineto{\pgfqpoint{3.289836in}{0.968102in}}%
\pgfpathclose%
\pgfusepath{stroke}%
\end{pgfscope}%
\begin{pgfscope}%
\pgfpathrectangle{\pgfqpoint{0.847223in}{0.554012in}}{\pgfqpoint{6.200000in}{4.530000in}}%
\pgfusepath{clip}%
\pgfsetbuttcap%
\pgfsetroundjoin%
\pgfsetlinewidth{1.003750pt}%
\definecolor{currentstroke}{rgb}{1.000000,0.000000,0.000000}%
\pgfsetstrokecolor{currentstroke}%
\pgfsetdash{}{0pt}%
\pgfpathmoveto{\pgfqpoint{3.295169in}{0.966536in}}%
\pgfpathcurveto{\pgfqpoint{3.306219in}{0.966536in}}{\pgfqpoint{3.316818in}{0.970926in}}{\pgfqpoint{3.324632in}{0.978740in}}%
\pgfpathcurveto{\pgfqpoint{3.332445in}{0.986553in}}{\pgfqpoint{3.336836in}{0.997152in}}{\pgfqpoint{3.336836in}{1.008202in}}%
\pgfpathcurveto{\pgfqpoint{3.336836in}{1.019252in}}{\pgfqpoint{3.332445in}{1.029851in}}{\pgfqpoint{3.324632in}{1.037665in}}%
\pgfpathcurveto{\pgfqpoint{3.316818in}{1.045479in}}{\pgfqpoint{3.306219in}{1.049869in}}{\pgfqpoint{3.295169in}{1.049869in}}%
\pgfpathcurveto{\pgfqpoint{3.284119in}{1.049869in}}{\pgfqpoint{3.273520in}{1.045479in}}{\pgfqpoint{3.265706in}{1.037665in}}%
\pgfpathcurveto{\pgfqpoint{3.257893in}{1.029851in}}{\pgfqpoint{3.253502in}{1.019252in}}{\pgfqpoint{3.253502in}{1.008202in}}%
\pgfpathcurveto{\pgfqpoint{3.253502in}{0.997152in}}{\pgfqpoint{3.257893in}{0.986553in}}{\pgfqpoint{3.265706in}{0.978740in}}%
\pgfpathcurveto{\pgfqpoint{3.273520in}{0.970926in}}{\pgfqpoint{3.284119in}{0.966536in}}{\pgfqpoint{3.295169in}{0.966536in}}%
\pgfpathlineto{\pgfqpoint{3.295169in}{0.966536in}}%
\pgfpathclose%
\pgfusepath{stroke}%
\end{pgfscope}%
\begin{pgfscope}%
\pgfpathrectangle{\pgfqpoint{0.847223in}{0.554012in}}{\pgfqpoint{6.200000in}{4.530000in}}%
\pgfusepath{clip}%
\pgfsetbuttcap%
\pgfsetroundjoin%
\pgfsetlinewidth{1.003750pt}%
\definecolor{currentstroke}{rgb}{1.000000,0.000000,0.000000}%
\pgfsetstrokecolor{currentstroke}%
\pgfsetdash{}{0pt}%
\pgfpathmoveto{\pgfqpoint{3.300502in}{0.964974in}}%
\pgfpathcurveto{\pgfqpoint{3.311552in}{0.964974in}}{\pgfqpoint{3.322151in}{0.969365in}}{\pgfqpoint{3.329965in}{0.977178in}}%
\pgfpathcurveto{\pgfqpoint{3.337779in}{0.984992in}}{\pgfqpoint{3.342169in}{0.995591in}}{\pgfqpoint{3.342169in}{1.006641in}}%
\pgfpathcurveto{\pgfqpoint{3.342169in}{1.017691in}}{\pgfqpoint{3.337779in}{1.028290in}}{\pgfqpoint{3.329965in}{1.036104in}}%
\pgfpathcurveto{\pgfqpoint{3.322151in}{1.043917in}}{\pgfqpoint{3.311552in}{1.048308in}}{\pgfqpoint{3.300502in}{1.048308in}}%
\pgfpathcurveto{\pgfqpoint{3.289452in}{1.048308in}}{\pgfqpoint{3.278853in}{1.043917in}}{\pgfqpoint{3.271039in}{1.036104in}}%
\pgfpathcurveto{\pgfqpoint{3.263226in}{1.028290in}}{\pgfqpoint{3.258836in}{1.017691in}}{\pgfqpoint{3.258836in}{1.006641in}}%
\pgfpathcurveto{\pgfqpoint{3.258836in}{0.995591in}}{\pgfqpoint{3.263226in}{0.984992in}}{\pgfqpoint{3.271039in}{0.977178in}}%
\pgfpathcurveto{\pgfqpoint{3.278853in}{0.969365in}}{\pgfqpoint{3.289452in}{0.964974in}}{\pgfqpoint{3.300502in}{0.964974in}}%
\pgfpathlineto{\pgfqpoint{3.300502in}{0.964974in}}%
\pgfpathclose%
\pgfusepath{stroke}%
\end{pgfscope}%
\begin{pgfscope}%
\pgfpathrectangle{\pgfqpoint{0.847223in}{0.554012in}}{\pgfqpoint{6.200000in}{4.530000in}}%
\pgfusepath{clip}%
\pgfsetbuttcap%
\pgfsetroundjoin%
\pgfsetlinewidth{1.003750pt}%
\definecolor{currentstroke}{rgb}{1.000000,0.000000,0.000000}%
\pgfsetstrokecolor{currentstroke}%
\pgfsetdash{}{0pt}%
\pgfpathmoveto{\pgfqpoint{3.305835in}{0.963419in}}%
\pgfpathcurveto{\pgfqpoint{3.316886in}{0.963419in}}{\pgfqpoint{3.327485in}{0.967809in}}{\pgfqpoint{3.335298in}{0.975623in}}%
\pgfpathcurveto{\pgfqpoint{3.343112in}{0.983436in}}{\pgfqpoint{3.347502in}{0.994035in}}{\pgfqpoint{3.347502in}{1.005085in}}%
\pgfpathcurveto{\pgfqpoint{3.347502in}{1.016136in}}{\pgfqpoint{3.343112in}{1.026735in}}{\pgfqpoint{3.335298in}{1.034548in}}%
\pgfpathcurveto{\pgfqpoint{3.327485in}{1.042362in}}{\pgfqpoint{3.316886in}{1.046752in}}{\pgfqpoint{3.305835in}{1.046752in}}%
\pgfpathcurveto{\pgfqpoint{3.294785in}{1.046752in}}{\pgfqpoint{3.284186in}{1.042362in}}{\pgfqpoint{3.276373in}{1.034548in}}%
\pgfpathcurveto{\pgfqpoint{3.268559in}{1.026735in}}{\pgfqpoint{3.264169in}{1.016136in}}{\pgfqpoint{3.264169in}{1.005085in}}%
\pgfpathcurveto{\pgfqpoint{3.264169in}{0.994035in}}{\pgfqpoint{3.268559in}{0.983436in}}{\pgfqpoint{3.276373in}{0.975623in}}%
\pgfpathcurveto{\pgfqpoint{3.284186in}{0.967809in}}{\pgfqpoint{3.294785in}{0.963419in}}{\pgfqpoint{3.305835in}{0.963419in}}%
\pgfpathlineto{\pgfqpoint{3.305835in}{0.963419in}}%
\pgfpathclose%
\pgfusepath{stroke}%
\end{pgfscope}%
\begin{pgfscope}%
\pgfpathrectangle{\pgfqpoint{0.847223in}{0.554012in}}{\pgfqpoint{6.200000in}{4.530000in}}%
\pgfusepath{clip}%
\pgfsetbuttcap%
\pgfsetroundjoin%
\pgfsetlinewidth{1.003750pt}%
\definecolor{currentstroke}{rgb}{1.000000,0.000000,0.000000}%
\pgfsetstrokecolor{currentstroke}%
\pgfsetdash{}{0pt}%
\pgfpathmoveto{\pgfqpoint{3.311169in}{0.961869in}}%
\pgfpathcurveto{\pgfqpoint{3.322219in}{0.961869in}}{\pgfqpoint{3.332818in}{0.966259in}}{\pgfqpoint{3.340631in}{0.974073in}}%
\pgfpathcurveto{\pgfqpoint{3.348445in}{0.981886in}}{\pgfqpoint{3.352835in}{0.992485in}}{\pgfqpoint{3.352835in}{1.003535in}}%
\pgfpathcurveto{\pgfqpoint{3.352835in}{1.014585in}}{\pgfqpoint{3.348445in}{1.025184in}}{\pgfqpoint{3.340631in}{1.032998in}}%
\pgfpathcurveto{\pgfqpoint{3.332818in}{1.040812in}}{\pgfqpoint{3.322219in}{1.045202in}}{\pgfqpoint{3.311169in}{1.045202in}}%
\pgfpathcurveto{\pgfqpoint{3.300119in}{1.045202in}}{\pgfqpoint{3.289520in}{1.040812in}}{\pgfqpoint{3.281706in}{1.032998in}}%
\pgfpathcurveto{\pgfqpoint{3.273892in}{1.025184in}}{\pgfqpoint{3.269502in}{1.014585in}}{\pgfqpoint{3.269502in}{1.003535in}}%
\pgfpathcurveto{\pgfqpoint{3.269502in}{0.992485in}}{\pgfqpoint{3.273892in}{0.981886in}}{\pgfqpoint{3.281706in}{0.974073in}}%
\pgfpathcurveto{\pgfqpoint{3.289520in}{0.966259in}}{\pgfqpoint{3.300119in}{0.961869in}}{\pgfqpoint{3.311169in}{0.961869in}}%
\pgfpathlineto{\pgfqpoint{3.311169in}{0.961869in}}%
\pgfpathclose%
\pgfusepath{stroke}%
\end{pgfscope}%
\begin{pgfscope}%
\pgfpathrectangle{\pgfqpoint{0.847223in}{0.554012in}}{\pgfqpoint{6.200000in}{4.530000in}}%
\pgfusepath{clip}%
\pgfsetbuttcap%
\pgfsetroundjoin%
\pgfsetlinewidth{1.003750pt}%
\definecolor{currentstroke}{rgb}{1.000000,0.000000,0.000000}%
\pgfsetstrokecolor{currentstroke}%
\pgfsetdash{}{0pt}%
\pgfpathmoveto{\pgfqpoint{3.316502in}{0.960324in}}%
\pgfpathcurveto{\pgfqpoint{3.327552in}{0.960324in}}{\pgfqpoint{3.338151in}{0.964714in}}{\pgfqpoint{3.345965in}{0.972528in}}%
\pgfpathcurveto{\pgfqpoint{3.353778in}{0.980342in}}{\pgfqpoint{3.358169in}{0.990941in}}{\pgfqpoint{3.358169in}{1.001991in}}%
\pgfpathcurveto{\pgfqpoint{3.358169in}{1.013041in}}{\pgfqpoint{3.353778in}{1.023640in}}{\pgfqpoint{3.345965in}{1.031454in}}%
\pgfpathcurveto{\pgfqpoint{3.338151in}{1.039267in}}{\pgfqpoint{3.327552in}{1.043657in}}{\pgfqpoint{3.316502in}{1.043657in}}%
\pgfpathcurveto{\pgfqpoint{3.305452in}{1.043657in}}{\pgfqpoint{3.294853in}{1.039267in}}{\pgfqpoint{3.287039in}{1.031454in}}%
\pgfpathcurveto{\pgfqpoint{3.279226in}{1.023640in}}{\pgfqpoint{3.274835in}{1.013041in}}{\pgfqpoint{3.274835in}{1.001991in}}%
\pgfpathcurveto{\pgfqpoint{3.274835in}{0.990941in}}{\pgfqpoint{3.279226in}{0.980342in}}{\pgfqpoint{3.287039in}{0.972528in}}%
\pgfpathcurveto{\pgfqpoint{3.294853in}{0.964714in}}{\pgfqpoint{3.305452in}{0.960324in}}{\pgfqpoint{3.316502in}{0.960324in}}%
\pgfpathlineto{\pgfqpoint{3.316502in}{0.960324in}}%
\pgfpathclose%
\pgfusepath{stroke}%
\end{pgfscope}%
\begin{pgfscope}%
\pgfpathrectangle{\pgfqpoint{0.847223in}{0.554012in}}{\pgfqpoint{6.200000in}{4.530000in}}%
\pgfusepath{clip}%
\pgfsetbuttcap%
\pgfsetroundjoin%
\pgfsetlinewidth{1.003750pt}%
\definecolor{currentstroke}{rgb}{1.000000,0.000000,0.000000}%
\pgfsetstrokecolor{currentstroke}%
\pgfsetdash{}{0pt}%
\pgfpathmoveto{\pgfqpoint{3.321835in}{0.958785in}}%
\pgfpathcurveto{\pgfqpoint{3.332885in}{0.958785in}}{\pgfqpoint{3.343484in}{0.963175in}}{\pgfqpoint{3.351298in}{0.970989in}}%
\pgfpathcurveto{\pgfqpoint{3.359112in}{0.978803in}}{\pgfqpoint{3.363502in}{0.989402in}}{\pgfqpoint{3.363502in}{1.000452in}}%
\pgfpathcurveto{\pgfqpoint{3.363502in}{1.011502in}}{\pgfqpoint{3.359112in}{1.022101in}}{\pgfqpoint{3.351298in}{1.029914in}}%
\pgfpathcurveto{\pgfqpoint{3.343484in}{1.037728in}}{\pgfqpoint{3.332885in}{1.042118in}}{\pgfqpoint{3.321835in}{1.042118in}}%
\pgfpathcurveto{\pgfqpoint{3.310785in}{1.042118in}}{\pgfqpoint{3.300186in}{1.037728in}}{\pgfqpoint{3.292372in}{1.029914in}}%
\pgfpathcurveto{\pgfqpoint{3.284559in}{1.022101in}}{\pgfqpoint{3.280168in}{1.011502in}}{\pgfqpoint{3.280168in}{1.000452in}}%
\pgfpathcurveto{\pgfqpoint{3.280168in}{0.989402in}}{\pgfqpoint{3.284559in}{0.978803in}}{\pgfqpoint{3.292372in}{0.970989in}}%
\pgfpathcurveto{\pgfqpoint{3.300186in}{0.963175in}}{\pgfqpoint{3.310785in}{0.958785in}}{\pgfqpoint{3.321835in}{0.958785in}}%
\pgfpathlineto{\pgfqpoint{3.321835in}{0.958785in}}%
\pgfpathclose%
\pgfusepath{stroke}%
\end{pgfscope}%
\begin{pgfscope}%
\pgfpathrectangle{\pgfqpoint{0.847223in}{0.554012in}}{\pgfqpoint{6.200000in}{4.530000in}}%
\pgfusepath{clip}%
\pgfsetbuttcap%
\pgfsetroundjoin%
\pgfsetlinewidth{1.003750pt}%
\definecolor{currentstroke}{rgb}{1.000000,0.000000,0.000000}%
\pgfsetstrokecolor{currentstroke}%
\pgfsetdash{}{0pt}%
\pgfpathmoveto{\pgfqpoint{3.327168in}{0.957251in}}%
\pgfpathcurveto{\pgfqpoint{3.338218in}{0.957251in}}{\pgfqpoint{3.348817in}{0.961642in}}{\pgfqpoint{3.356631in}{0.969455in}}%
\pgfpathcurveto{\pgfqpoint{3.364445in}{0.977269in}}{\pgfqpoint{3.368835in}{0.987868in}}{\pgfqpoint{3.368835in}{0.998918in}}%
\pgfpathcurveto{\pgfqpoint{3.368835in}{1.009968in}}{\pgfqpoint{3.364445in}{1.020567in}}{\pgfqpoint{3.356631in}{1.028381in}}%
\pgfpathcurveto{\pgfqpoint{3.348817in}{1.036194in}}{\pgfqpoint{3.338218in}{1.040585in}}{\pgfqpoint{3.327168in}{1.040585in}}%
\pgfpathcurveto{\pgfqpoint{3.316118in}{1.040585in}}{\pgfqpoint{3.305519in}{1.036194in}}{\pgfqpoint{3.297706in}{1.028381in}}%
\pgfpathcurveto{\pgfqpoint{3.289892in}{1.020567in}}{\pgfqpoint{3.285502in}{1.009968in}}{\pgfqpoint{3.285502in}{0.998918in}}%
\pgfpathcurveto{\pgfqpoint{3.285502in}{0.987868in}}{\pgfqpoint{3.289892in}{0.977269in}}{\pgfqpoint{3.297706in}{0.969455in}}%
\pgfpathcurveto{\pgfqpoint{3.305519in}{0.961642in}}{\pgfqpoint{3.316118in}{0.957251in}}{\pgfqpoint{3.327168in}{0.957251in}}%
\pgfpathlineto{\pgfqpoint{3.327168in}{0.957251in}}%
\pgfpathclose%
\pgfusepath{stroke}%
\end{pgfscope}%
\begin{pgfscope}%
\pgfpathrectangle{\pgfqpoint{0.847223in}{0.554012in}}{\pgfqpoint{6.200000in}{4.530000in}}%
\pgfusepath{clip}%
\pgfsetbuttcap%
\pgfsetroundjoin%
\pgfsetlinewidth{1.003750pt}%
\definecolor{currentstroke}{rgb}{1.000000,0.000000,0.000000}%
\pgfsetstrokecolor{currentstroke}%
\pgfsetdash{}{0pt}%
\pgfpathmoveto{\pgfqpoint{3.332502in}{0.955723in}}%
\pgfpathcurveto{\pgfqpoint{3.343552in}{0.955723in}}{\pgfqpoint{3.354151in}{0.960113in}}{\pgfqpoint{3.361964in}{0.967927in}}%
\pgfpathcurveto{\pgfqpoint{3.369778in}{0.975741in}}{\pgfqpoint{3.374168in}{0.986340in}}{\pgfqpoint{3.374168in}{0.997390in}}%
\pgfpathcurveto{\pgfqpoint{3.374168in}{1.008440in}}{\pgfqpoint{3.369778in}{1.019039in}}{\pgfqpoint{3.361964in}{1.026853in}}%
\pgfpathcurveto{\pgfqpoint{3.354151in}{1.034666in}}{\pgfqpoint{3.343552in}{1.039057in}}{\pgfqpoint{3.332502in}{1.039057in}}%
\pgfpathcurveto{\pgfqpoint{3.321451in}{1.039057in}}{\pgfqpoint{3.310852in}{1.034666in}}{\pgfqpoint{3.303039in}{1.026853in}}%
\pgfpathcurveto{\pgfqpoint{3.295225in}{1.019039in}}{\pgfqpoint{3.290835in}{1.008440in}}{\pgfqpoint{3.290835in}{0.997390in}}%
\pgfpathcurveto{\pgfqpoint{3.290835in}{0.986340in}}{\pgfqpoint{3.295225in}{0.975741in}}{\pgfqpoint{3.303039in}{0.967927in}}%
\pgfpathcurveto{\pgfqpoint{3.310852in}{0.960113in}}{\pgfqpoint{3.321451in}{0.955723in}}{\pgfqpoint{3.332502in}{0.955723in}}%
\pgfpathlineto{\pgfqpoint{3.332502in}{0.955723in}}%
\pgfpathclose%
\pgfusepath{stroke}%
\end{pgfscope}%
\begin{pgfscope}%
\pgfpathrectangle{\pgfqpoint{0.847223in}{0.554012in}}{\pgfqpoint{6.200000in}{4.530000in}}%
\pgfusepath{clip}%
\pgfsetbuttcap%
\pgfsetroundjoin%
\pgfsetlinewidth{1.003750pt}%
\definecolor{currentstroke}{rgb}{1.000000,0.000000,0.000000}%
\pgfsetstrokecolor{currentstroke}%
\pgfsetdash{}{0pt}%
\pgfpathmoveto{\pgfqpoint{3.337835in}{0.954200in}}%
\pgfpathcurveto{\pgfqpoint{3.348885in}{0.954200in}}{\pgfqpoint{3.359484in}{0.958591in}}{\pgfqpoint{3.367298in}{0.966404in}}%
\pgfpathcurveto{\pgfqpoint{3.375111in}{0.974218in}}{\pgfqpoint{3.379501in}{0.984817in}}{\pgfqpoint{3.379501in}{0.995867in}}%
\pgfpathcurveto{\pgfqpoint{3.379501in}{1.006917in}}{\pgfqpoint{3.375111in}{1.017516in}}{\pgfqpoint{3.367298in}{1.025330in}}%
\pgfpathcurveto{\pgfqpoint{3.359484in}{1.033144in}}{\pgfqpoint{3.348885in}{1.037534in}}{\pgfqpoint{3.337835in}{1.037534in}}%
\pgfpathcurveto{\pgfqpoint{3.326785in}{1.037534in}}{\pgfqpoint{3.316186in}{1.033144in}}{\pgfqpoint{3.308372in}{1.025330in}}%
\pgfpathcurveto{\pgfqpoint{3.300558in}{1.017516in}}{\pgfqpoint{3.296168in}{1.006917in}}{\pgfqpoint{3.296168in}{0.995867in}}%
\pgfpathcurveto{\pgfqpoint{3.296168in}{0.984817in}}{\pgfqpoint{3.300558in}{0.974218in}}{\pgfqpoint{3.308372in}{0.966404in}}%
\pgfpathcurveto{\pgfqpoint{3.316186in}{0.958591in}}{\pgfqpoint{3.326785in}{0.954200in}}{\pgfqpoint{3.337835in}{0.954200in}}%
\pgfpathlineto{\pgfqpoint{3.337835in}{0.954200in}}%
\pgfpathclose%
\pgfusepath{stroke}%
\end{pgfscope}%
\begin{pgfscope}%
\pgfpathrectangle{\pgfqpoint{0.847223in}{0.554012in}}{\pgfqpoint{6.200000in}{4.530000in}}%
\pgfusepath{clip}%
\pgfsetbuttcap%
\pgfsetroundjoin%
\pgfsetlinewidth{1.003750pt}%
\definecolor{currentstroke}{rgb}{1.000000,0.000000,0.000000}%
\pgfsetstrokecolor{currentstroke}%
\pgfsetdash{}{0pt}%
\pgfpathmoveto{\pgfqpoint{3.343168in}{0.952683in}}%
\pgfpathcurveto{\pgfqpoint{3.354218in}{0.952683in}}{\pgfqpoint{3.364817in}{0.957073in}}{\pgfqpoint{3.372631in}{0.964887in}}%
\pgfpathcurveto{\pgfqpoint{3.380444in}{0.972701in}}{\pgfqpoint{3.384835in}{0.983300in}}{\pgfqpoint{3.384835in}{0.994350in}}%
\pgfpathcurveto{\pgfqpoint{3.384835in}{1.005400in}}{\pgfqpoint{3.380444in}{1.015999in}}{\pgfqpoint{3.372631in}{1.023813in}}%
\pgfpathcurveto{\pgfqpoint{3.364817in}{1.031626in}}{\pgfqpoint{3.354218in}{1.036016in}}{\pgfqpoint{3.343168in}{1.036016in}}%
\pgfpathcurveto{\pgfqpoint{3.332118in}{1.036016in}}{\pgfqpoint{3.321519in}{1.031626in}}{\pgfqpoint{3.313705in}{1.023813in}}%
\pgfpathcurveto{\pgfqpoint{3.305892in}{1.015999in}}{\pgfqpoint{3.301501in}{1.005400in}}{\pgfqpoint{3.301501in}{0.994350in}}%
\pgfpathcurveto{\pgfqpoint{3.301501in}{0.983300in}}{\pgfqpoint{3.305892in}{0.972701in}}{\pgfqpoint{3.313705in}{0.964887in}}%
\pgfpathcurveto{\pgfqpoint{3.321519in}{0.957073in}}{\pgfqpoint{3.332118in}{0.952683in}}{\pgfqpoint{3.343168in}{0.952683in}}%
\pgfpathlineto{\pgfqpoint{3.343168in}{0.952683in}}%
\pgfpathclose%
\pgfusepath{stroke}%
\end{pgfscope}%
\begin{pgfscope}%
\pgfpathrectangle{\pgfqpoint{0.847223in}{0.554012in}}{\pgfqpoint{6.200000in}{4.530000in}}%
\pgfusepath{clip}%
\pgfsetbuttcap%
\pgfsetroundjoin%
\pgfsetlinewidth{1.003750pt}%
\definecolor{currentstroke}{rgb}{1.000000,0.000000,0.000000}%
\pgfsetstrokecolor{currentstroke}%
\pgfsetdash{}{0pt}%
\pgfpathmoveto{\pgfqpoint{3.348501in}{0.951171in}}%
\pgfpathcurveto{\pgfqpoint{3.359551in}{0.951171in}}{\pgfqpoint{3.370150in}{0.955561in}}{\pgfqpoint{3.377964in}{0.963375in}}%
\pgfpathcurveto{\pgfqpoint{3.385778in}{0.971189in}}{\pgfqpoint{3.390168in}{0.981788in}}{\pgfqpoint{3.390168in}{0.992838in}}%
\pgfpathcurveto{\pgfqpoint{3.390168in}{1.003888in}}{\pgfqpoint{3.385778in}{1.014487in}}{\pgfqpoint{3.377964in}{1.022300in}}%
\pgfpathcurveto{\pgfqpoint{3.370150in}{1.030114in}}{\pgfqpoint{3.359551in}{1.034504in}}{\pgfqpoint{3.348501in}{1.034504in}}%
\pgfpathcurveto{\pgfqpoint{3.337451in}{1.034504in}}{\pgfqpoint{3.326852in}{1.030114in}}{\pgfqpoint{3.319038in}{1.022300in}}%
\pgfpathcurveto{\pgfqpoint{3.311225in}{1.014487in}}{\pgfqpoint{3.306835in}{1.003888in}}{\pgfqpoint{3.306835in}{0.992838in}}%
\pgfpathcurveto{\pgfqpoint{3.306835in}{0.981788in}}{\pgfqpoint{3.311225in}{0.971189in}}{\pgfqpoint{3.319038in}{0.963375in}}%
\pgfpathcurveto{\pgfqpoint{3.326852in}{0.955561in}}{\pgfqpoint{3.337451in}{0.951171in}}{\pgfqpoint{3.348501in}{0.951171in}}%
\pgfpathlineto{\pgfqpoint{3.348501in}{0.951171in}}%
\pgfpathclose%
\pgfusepath{stroke}%
\end{pgfscope}%
\begin{pgfscope}%
\pgfpathrectangle{\pgfqpoint{0.847223in}{0.554012in}}{\pgfqpoint{6.200000in}{4.530000in}}%
\pgfusepath{clip}%
\pgfsetbuttcap%
\pgfsetroundjoin%
\pgfsetlinewidth{1.003750pt}%
\definecolor{currentstroke}{rgb}{1.000000,0.000000,0.000000}%
\pgfsetstrokecolor{currentstroke}%
\pgfsetdash{}{0pt}%
\pgfpathmoveto{\pgfqpoint{3.353834in}{0.949664in}}%
\pgfpathcurveto{\pgfqpoint{3.364885in}{0.949664in}}{\pgfqpoint{3.375484in}{0.954055in}}{\pgfqpoint{3.383297in}{0.961868in}}%
\pgfpathcurveto{\pgfqpoint{3.391111in}{0.969682in}}{\pgfqpoint{3.395501in}{0.980281in}}{\pgfqpoint{3.395501in}{0.991331in}}%
\pgfpathcurveto{\pgfqpoint{3.395501in}{1.002381in}}{\pgfqpoint{3.391111in}{1.012980in}}{\pgfqpoint{3.383297in}{1.020794in}}%
\pgfpathcurveto{\pgfqpoint{3.375484in}{1.028607in}}{\pgfqpoint{3.364885in}{1.032998in}}{\pgfqpoint{3.353834in}{1.032998in}}%
\pgfpathcurveto{\pgfqpoint{3.342784in}{1.032998in}}{\pgfqpoint{3.332185in}{1.028607in}}{\pgfqpoint{3.324372in}{1.020794in}}%
\pgfpathcurveto{\pgfqpoint{3.316558in}{1.012980in}}{\pgfqpoint{3.312168in}{1.002381in}}{\pgfqpoint{3.312168in}{0.991331in}}%
\pgfpathcurveto{\pgfqpoint{3.312168in}{0.980281in}}{\pgfqpoint{3.316558in}{0.969682in}}{\pgfqpoint{3.324372in}{0.961868in}}%
\pgfpathcurveto{\pgfqpoint{3.332185in}{0.954055in}}{\pgfqpoint{3.342784in}{0.949664in}}{\pgfqpoint{3.353834in}{0.949664in}}%
\pgfpathlineto{\pgfqpoint{3.353834in}{0.949664in}}%
\pgfpathclose%
\pgfusepath{stroke}%
\end{pgfscope}%
\begin{pgfscope}%
\pgfpathrectangle{\pgfqpoint{0.847223in}{0.554012in}}{\pgfqpoint{6.200000in}{4.530000in}}%
\pgfusepath{clip}%
\pgfsetbuttcap%
\pgfsetroundjoin%
\pgfsetlinewidth{1.003750pt}%
\definecolor{currentstroke}{rgb}{1.000000,0.000000,0.000000}%
\pgfsetstrokecolor{currentstroke}%
\pgfsetdash{}{0pt}%
\pgfpathmoveto{\pgfqpoint{3.359168in}{0.948163in}}%
\pgfpathcurveto{\pgfqpoint{3.370218in}{0.948163in}}{\pgfqpoint{3.380817in}{0.952553in}}{\pgfqpoint{3.388630in}{0.960367in}}%
\pgfpathcurveto{\pgfqpoint{3.396444in}{0.968180in}}{\pgfqpoint{3.400834in}{0.978779in}}{\pgfqpoint{3.400834in}{0.989830in}}%
\pgfpathcurveto{\pgfqpoint{3.400834in}{1.000880in}}{\pgfqpoint{3.396444in}{1.011479in}}{\pgfqpoint{3.388630in}{1.019292in}}%
\pgfpathcurveto{\pgfqpoint{3.380817in}{1.027106in}}{\pgfqpoint{3.370218in}{1.031496in}}{\pgfqpoint{3.359168in}{1.031496in}}%
\pgfpathcurveto{\pgfqpoint{3.348118in}{1.031496in}}{\pgfqpoint{3.337518in}{1.027106in}}{\pgfqpoint{3.329705in}{1.019292in}}%
\pgfpathcurveto{\pgfqpoint{3.321891in}{1.011479in}}{\pgfqpoint{3.317501in}{1.000880in}}{\pgfqpoint{3.317501in}{0.989830in}}%
\pgfpathcurveto{\pgfqpoint{3.317501in}{0.978779in}}{\pgfqpoint{3.321891in}{0.968180in}}{\pgfqpoint{3.329705in}{0.960367in}}%
\pgfpathcurveto{\pgfqpoint{3.337518in}{0.952553in}}{\pgfqpoint{3.348118in}{0.948163in}}{\pgfqpoint{3.359168in}{0.948163in}}%
\pgfpathlineto{\pgfqpoint{3.359168in}{0.948163in}}%
\pgfpathclose%
\pgfusepath{stroke}%
\end{pgfscope}%
\begin{pgfscope}%
\pgfpathrectangle{\pgfqpoint{0.847223in}{0.554012in}}{\pgfqpoint{6.200000in}{4.530000in}}%
\pgfusepath{clip}%
\pgfsetbuttcap%
\pgfsetroundjoin%
\pgfsetlinewidth{1.003750pt}%
\definecolor{currentstroke}{rgb}{1.000000,0.000000,0.000000}%
\pgfsetstrokecolor{currentstroke}%
\pgfsetdash{}{0pt}%
\pgfpathmoveto{\pgfqpoint{3.364501in}{0.946667in}}%
\pgfpathcurveto{\pgfqpoint{3.375551in}{0.946667in}}{\pgfqpoint{3.386150in}{0.951057in}}{\pgfqpoint{3.393964in}{0.958871in}}%
\pgfpathcurveto{\pgfqpoint{3.401777in}{0.966684in}}{\pgfqpoint{3.406168in}{0.977283in}}{\pgfqpoint{3.406168in}{0.988333in}}%
\pgfpathcurveto{\pgfqpoint{3.406168in}{0.999384in}}{\pgfqpoint{3.401777in}{1.009983in}}{\pgfqpoint{3.393964in}{1.017796in}}%
\pgfpathcurveto{\pgfqpoint{3.386150in}{1.025610in}}{\pgfqpoint{3.375551in}{1.030000in}}{\pgfqpoint{3.364501in}{1.030000in}}%
\pgfpathcurveto{\pgfqpoint{3.353451in}{1.030000in}}{\pgfqpoint{3.342852in}{1.025610in}}{\pgfqpoint{3.335038in}{1.017796in}}%
\pgfpathcurveto{\pgfqpoint{3.327224in}{1.009983in}}{\pgfqpoint{3.322834in}{0.999384in}}{\pgfqpoint{3.322834in}{0.988333in}}%
\pgfpathcurveto{\pgfqpoint{3.322834in}{0.977283in}}{\pgfqpoint{3.327224in}{0.966684in}}{\pgfqpoint{3.335038in}{0.958871in}}%
\pgfpathcurveto{\pgfqpoint{3.342852in}{0.951057in}}{\pgfqpoint{3.353451in}{0.946667in}}{\pgfqpoint{3.364501in}{0.946667in}}%
\pgfpathlineto{\pgfqpoint{3.364501in}{0.946667in}}%
\pgfpathclose%
\pgfusepath{stroke}%
\end{pgfscope}%
\begin{pgfscope}%
\pgfpathrectangle{\pgfqpoint{0.847223in}{0.554012in}}{\pgfqpoint{6.200000in}{4.530000in}}%
\pgfusepath{clip}%
\pgfsetbuttcap%
\pgfsetroundjoin%
\pgfsetlinewidth{1.003750pt}%
\definecolor{currentstroke}{rgb}{1.000000,0.000000,0.000000}%
\pgfsetstrokecolor{currentstroke}%
\pgfsetdash{}{0pt}%
\pgfpathmoveto{\pgfqpoint{3.369834in}{0.945176in}}%
\pgfpathcurveto{\pgfqpoint{3.380884in}{0.945176in}}{\pgfqpoint{3.391483in}{0.949566in}}{\pgfqpoint{3.399297in}{0.957380in}}%
\pgfpathcurveto{\pgfqpoint{3.407110in}{0.965193in}}{\pgfqpoint{3.411501in}{0.975792in}}{\pgfqpoint{3.411501in}{0.986843in}}%
\pgfpathcurveto{\pgfqpoint{3.411501in}{0.997893in}}{\pgfqpoint{3.407110in}{1.008492in}}{\pgfqpoint{3.399297in}{1.016305in}}%
\pgfpathcurveto{\pgfqpoint{3.391483in}{1.024119in}}{\pgfqpoint{3.380884in}{1.028509in}}{\pgfqpoint{3.369834in}{1.028509in}}%
\pgfpathcurveto{\pgfqpoint{3.358784in}{1.028509in}}{\pgfqpoint{3.348185in}{1.024119in}}{\pgfqpoint{3.340371in}{1.016305in}}%
\pgfpathcurveto{\pgfqpoint{3.332558in}{1.008492in}}{\pgfqpoint{3.328167in}{0.997893in}}{\pgfqpoint{3.328167in}{0.986843in}}%
\pgfpathcurveto{\pgfqpoint{3.328167in}{0.975792in}}{\pgfqpoint{3.332558in}{0.965193in}}{\pgfqpoint{3.340371in}{0.957380in}}%
\pgfpathcurveto{\pgfqpoint{3.348185in}{0.949566in}}{\pgfqpoint{3.358784in}{0.945176in}}{\pgfqpoint{3.369834in}{0.945176in}}%
\pgfpathlineto{\pgfqpoint{3.369834in}{0.945176in}}%
\pgfpathclose%
\pgfusepath{stroke}%
\end{pgfscope}%
\begin{pgfscope}%
\pgfpathrectangle{\pgfqpoint{0.847223in}{0.554012in}}{\pgfqpoint{6.200000in}{4.530000in}}%
\pgfusepath{clip}%
\pgfsetbuttcap%
\pgfsetroundjoin%
\pgfsetlinewidth{1.003750pt}%
\definecolor{currentstroke}{rgb}{1.000000,0.000000,0.000000}%
\pgfsetstrokecolor{currentstroke}%
\pgfsetdash{}{0pt}%
\pgfpathmoveto{\pgfqpoint{3.375167in}{0.943690in}}%
\pgfpathcurveto{\pgfqpoint{3.386217in}{0.943690in}}{\pgfqpoint{3.396816in}{0.948080in}}{\pgfqpoint{3.404630in}{0.955894in}}%
\pgfpathcurveto{\pgfqpoint{3.412444in}{0.963708in}}{\pgfqpoint{3.416834in}{0.974307in}}{\pgfqpoint{3.416834in}{0.985357in}}%
\pgfpathcurveto{\pgfqpoint{3.416834in}{0.996407in}}{\pgfqpoint{3.412444in}{1.007006in}}{\pgfqpoint{3.404630in}{1.014820in}}%
\pgfpathcurveto{\pgfqpoint{3.396816in}{1.022633in}}{\pgfqpoint{3.386217in}{1.027023in}}{\pgfqpoint{3.375167in}{1.027023in}}%
\pgfpathcurveto{\pgfqpoint{3.364117in}{1.027023in}}{\pgfqpoint{3.353518in}{1.022633in}}{\pgfqpoint{3.345704in}{1.014820in}}%
\pgfpathcurveto{\pgfqpoint{3.337891in}{1.007006in}}{\pgfqpoint{3.333501in}{0.996407in}}{\pgfqpoint{3.333501in}{0.985357in}}%
\pgfpathcurveto{\pgfqpoint{3.333501in}{0.974307in}}{\pgfqpoint{3.337891in}{0.963708in}}{\pgfqpoint{3.345704in}{0.955894in}}%
\pgfpathcurveto{\pgfqpoint{3.353518in}{0.948080in}}{\pgfqpoint{3.364117in}{0.943690in}}{\pgfqpoint{3.375167in}{0.943690in}}%
\pgfpathlineto{\pgfqpoint{3.375167in}{0.943690in}}%
\pgfpathclose%
\pgfusepath{stroke}%
\end{pgfscope}%
\begin{pgfscope}%
\pgfpathrectangle{\pgfqpoint{0.847223in}{0.554012in}}{\pgfqpoint{6.200000in}{4.530000in}}%
\pgfusepath{clip}%
\pgfsetbuttcap%
\pgfsetroundjoin%
\pgfsetlinewidth{1.003750pt}%
\definecolor{currentstroke}{rgb}{1.000000,0.000000,0.000000}%
\pgfsetstrokecolor{currentstroke}%
\pgfsetdash{}{0pt}%
\pgfpathmoveto{\pgfqpoint{3.380500in}{0.942210in}}%
\pgfpathcurveto{\pgfqpoint{3.391551in}{0.942210in}}{\pgfqpoint{3.402150in}{0.946600in}}{\pgfqpoint{3.409963in}{0.954414in}}%
\pgfpathcurveto{\pgfqpoint{3.417777in}{0.962227in}}{\pgfqpoint{3.422167in}{0.972826in}}{\pgfqpoint{3.422167in}{0.983876in}}%
\pgfpathcurveto{\pgfqpoint{3.422167in}{0.994926in}}{\pgfqpoint{3.417777in}{1.005525in}}{\pgfqpoint{3.409963in}{1.013339in}}%
\pgfpathcurveto{\pgfqpoint{3.402150in}{1.021153in}}{\pgfqpoint{3.391551in}{1.025543in}}{\pgfqpoint{3.380500in}{1.025543in}}%
\pgfpathcurveto{\pgfqpoint{3.369450in}{1.025543in}}{\pgfqpoint{3.358851in}{1.021153in}}{\pgfqpoint{3.351038in}{1.013339in}}%
\pgfpathcurveto{\pgfqpoint{3.343224in}{1.005525in}}{\pgfqpoint{3.338834in}{0.994926in}}{\pgfqpoint{3.338834in}{0.983876in}}%
\pgfpathcurveto{\pgfqpoint{3.338834in}{0.972826in}}{\pgfqpoint{3.343224in}{0.962227in}}{\pgfqpoint{3.351038in}{0.954414in}}%
\pgfpathcurveto{\pgfqpoint{3.358851in}{0.946600in}}{\pgfqpoint{3.369450in}{0.942210in}}{\pgfqpoint{3.380500in}{0.942210in}}%
\pgfpathlineto{\pgfqpoint{3.380500in}{0.942210in}}%
\pgfpathclose%
\pgfusepath{stroke}%
\end{pgfscope}%
\begin{pgfscope}%
\pgfpathrectangle{\pgfqpoint{0.847223in}{0.554012in}}{\pgfqpoint{6.200000in}{4.530000in}}%
\pgfusepath{clip}%
\pgfsetbuttcap%
\pgfsetroundjoin%
\pgfsetlinewidth{1.003750pt}%
\definecolor{currentstroke}{rgb}{1.000000,0.000000,0.000000}%
\pgfsetstrokecolor{currentstroke}%
\pgfsetdash{}{0pt}%
\pgfpathmoveto{\pgfqpoint{3.385834in}{0.940734in}}%
\pgfpathcurveto{\pgfqpoint{3.396884in}{0.940734in}}{\pgfqpoint{3.407483in}{0.945125in}}{\pgfqpoint{3.415296in}{0.952938in}}%
\pgfpathcurveto{\pgfqpoint{3.423110in}{0.960752in}}{\pgfqpoint{3.427500in}{0.971351in}}{\pgfqpoint{3.427500in}{0.982401in}}%
\pgfpathcurveto{\pgfqpoint{3.427500in}{0.993451in}}{\pgfqpoint{3.423110in}{1.004050in}}{\pgfqpoint{3.415296in}{1.011864in}}%
\pgfpathcurveto{\pgfqpoint{3.407483in}{1.019677in}}{\pgfqpoint{3.396884in}{1.024068in}}{\pgfqpoint{3.385834in}{1.024068in}}%
\pgfpathcurveto{\pgfqpoint{3.374784in}{1.024068in}}{\pgfqpoint{3.364185in}{1.019677in}}{\pgfqpoint{3.356371in}{1.011864in}}%
\pgfpathcurveto{\pgfqpoint{3.348557in}{1.004050in}}{\pgfqpoint{3.344167in}{0.993451in}}{\pgfqpoint{3.344167in}{0.982401in}}%
\pgfpathcurveto{\pgfqpoint{3.344167in}{0.971351in}}{\pgfqpoint{3.348557in}{0.960752in}}{\pgfqpoint{3.356371in}{0.952938in}}%
\pgfpathcurveto{\pgfqpoint{3.364185in}{0.945125in}}{\pgfqpoint{3.374784in}{0.940734in}}{\pgfqpoint{3.385834in}{0.940734in}}%
\pgfpathlineto{\pgfqpoint{3.385834in}{0.940734in}}%
\pgfpathclose%
\pgfusepath{stroke}%
\end{pgfscope}%
\begin{pgfscope}%
\pgfpathrectangle{\pgfqpoint{0.847223in}{0.554012in}}{\pgfqpoint{6.200000in}{4.530000in}}%
\pgfusepath{clip}%
\pgfsetbuttcap%
\pgfsetroundjoin%
\pgfsetlinewidth{1.003750pt}%
\definecolor{currentstroke}{rgb}{1.000000,0.000000,0.000000}%
\pgfsetstrokecolor{currentstroke}%
\pgfsetdash{}{0pt}%
\pgfpathmoveto{\pgfqpoint{3.391167in}{0.939264in}}%
\pgfpathcurveto{\pgfqpoint{3.402217in}{0.939264in}}{\pgfqpoint{3.412816in}{0.943654in}}{\pgfqpoint{3.420630in}{0.951468in}}%
\pgfpathcurveto{\pgfqpoint{3.428443in}{0.959282in}}{\pgfqpoint{3.432834in}{0.969881in}}{\pgfqpoint{3.432834in}{0.980931in}}%
\pgfpathcurveto{\pgfqpoint{3.432834in}{0.991981in}}{\pgfqpoint{3.428443in}{1.002580in}}{\pgfqpoint{3.420630in}{1.010393in}}%
\pgfpathcurveto{\pgfqpoint{3.412816in}{1.018207in}}{\pgfqpoint{3.402217in}{1.022597in}}{\pgfqpoint{3.391167in}{1.022597in}}%
\pgfpathcurveto{\pgfqpoint{3.380117in}{1.022597in}}{\pgfqpoint{3.369518in}{1.018207in}}{\pgfqpoint{3.361704in}{1.010393in}}%
\pgfpathcurveto{\pgfqpoint{3.353891in}{1.002580in}}{\pgfqpoint{3.349500in}{0.991981in}}{\pgfqpoint{3.349500in}{0.980931in}}%
\pgfpathcurveto{\pgfqpoint{3.349500in}{0.969881in}}{\pgfqpoint{3.353891in}{0.959282in}}{\pgfqpoint{3.361704in}{0.951468in}}%
\pgfpathcurveto{\pgfqpoint{3.369518in}{0.943654in}}{\pgfqpoint{3.380117in}{0.939264in}}{\pgfqpoint{3.391167in}{0.939264in}}%
\pgfpathlineto{\pgfqpoint{3.391167in}{0.939264in}}%
\pgfpathclose%
\pgfusepath{stroke}%
\end{pgfscope}%
\begin{pgfscope}%
\pgfpathrectangle{\pgfqpoint{0.847223in}{0.554012in}}{\pgfqpoint{6.200000in}{4.530000in}}%
\pgfusepath{clip}%
\pgfsetbuttcap%
\pgfsetroundjoin%
\pgfsetlinewidth{1.003750pt}%
\definecolor{currentstroke}{rgb}{1.000000,0.000000,0.000000}%
\pgfsetstrokecolor{currentstroke}%
\pgfsetdash{}{0pt}%
\pgfpathmoveto{\pgfqpoint{3.396500in}{0.937799in}}%
\pgfpathcurveto{\pgfqpoint{3.407550in}{0.937799in}}{\pgfqpoint{3.418149in}{0.942189in}}{\pgfqpoint{3.425963in}{0.950003in}}%
\pgfpathcurveto{\pgfqpoint{3.433777in}{0.957816in}}{\pgfqpoint{3.438167in}{0.968415in}}{\pgfqpoint{3.438167in}{0.979466in}}%
\pgfpathcurveto{\pgfqpoint{3.438167in}{0.990516in}}{\pgfqpoint{3.433777in}{1.001115in}}{\pgfqpoint{3.425963in}{1.008928in}}%
\pgfpathcurveto{\pgfqpoint{3.418149in}{1.016742in}}{\pgfqpoint{3.407550in}{1.021132in}}{\pgfqpoint{3.396500in}{1.021132in}}%
\pgfpathcurveto{\pgfqpoint{3.385450in}{1.021132in}}{\pgfqpoint{3.374851in}{1.016742in}}{\pgfqpoint{3.367037in}{1.008928in}}%
\pgfpathcurveto{\pgfqpoint{3.359224in}{1.001115in}}{\pgfqpoint{3.354833in}{0.990516in}}{\pgfqpoint{3.354833in}{0.979466in}}%
\pgfpathcurveto{\pgfqpoint{3.354833in}{0.968415in}}{\pgfqpoint{3.359224in}{0.957816in}}{\pgfqpoint{3.367037in}{0.950003in}}%
\pgfpathcurveto{\pgfqpoint{3.374851in}{0.942189in}}{\pgfqpoint{3.385450in}{0.937799in}}{\pgfqpoint{3.396500in}{0.937799in}}%
\pgfpathlineto{\pgfqpoint{3.396500in}{0.937799in}}%
\pgfpathclose%
\pgfusepath{stroke}%
\end{pgfscope}%
\begin{pgfscope}%
\pgfpathrectangle{\pgfqpoint{0.847223in}{0.554012in}}{\pgfqpoint{6.200000in}{4.530000in}}%
\pgfusepath{clip}%
\pgfsetbuttcap%
\pgfsetroundjoin%
\pgfsetlinewidth{1.003750pt}%
\definecolor{currentstroke}{rgb}{1.000000,0.000000,0.000000}%
\pgfsetstrokecolor{currentstroke}%
\pgfsetdash{}{0pt}%
\pgfpathmoveto{\pgfqpoint{3.401833in}{0.936339in}}%
\pgfpathcurveto{\pgfqpoint{3.412883in}{0.936339in}}{\pgfqpoint{3.423483in}{0.940729in}}{\pgfqpoint{3.431296in}{0.948543in}}%
\pgfpathcurveto{\pgfqpoint{3.439110in}{0.956356in}}{\pgfqpoint{3.443500in}{0.966955in}}{\pgfqpoint{3.443500in}{0.978006in}}%
\pgfpathcurveto{\pgfqpoint{3.443500in}{0.989056in}}{\pgfqpoint{3.439110in}{0.999655in}}{\pgfqpoint{3.431296in}{1.007468in}}%
\pgfpathcurveto{\pgfqpoint{3.423483in}{1.015282in}}{\pgfqpoint{3.412883in}{1.019672in}}{\pgfqpoint{3.401833in}{1.019672in}}%
\pgfpathcurveto{\pgfqpoint{3.390783in}{1.019672in}}{\pgfqpoint{3.380184in}{1.015282in}}{\pgfqpoint{3.372371in}{1.007468in}}%
\pgfpathcurveto{\pgfqpoint{3.364557in}{0.999655in}}{\pgfqpoint{3.360167in}{0.989056in}}{\pgfqpoint{3.360167in}{0.978006in}}%
\pgfpathcurveto{\pgfqpoint{3.360167in}{0.966955in}}{\pgfqpoint{3.364557in}{0.956356in}}{\pgfqpoint{3.372371in}{0.948543in}}%
\pgfpathcurveto{\pgfqpoint{3.380184in}{0.940729in}}{\pgfqpoint{3.390783in}{0.936339in}}{\pgfqpoint{3.401833in}{0.936339in}}%
\pgfpathlineto{\pgfqpoint{3.401833in}{0.936339in}}%
\pgfpathclose%
\pgfusepath{stroke}%
\end{pgfscope}%
\begin{pgfscope}%
\pgfpathrectangle{\pgfqpoint{0.847223in}{0.554012in}}{\pgfqpoint{6.200000in}{4.530000in}}%
\pgfusepath{clip}%
\pgfsetbuttcap%
\pgfsetroundjoin%
\pgfsetlinewidth{1.003750pt}%
\definecolor{currentstroke}{rgb}{1.000000,0.000000,0.000000}%
\pgfsetstrokecolor{currentstroke}%
\pgfsetdash{}{0pt}%
\pgfpathmoveto{\pgfqpoint{3.407167in}{0.934884in}}%
\pgfpathcurveto{\pgfqpoint{3.418217in}{0.934884in}}{\pgfqpoint{3.428816in}{0.939274in}}{\pgfqpoint{3.436629in}{0.947088in}}%
\pgfpathcurveto{\pgfqpoint{3.444443in}{0.954901in}}{\pgfqpoint{3.448833in}{0.965500in}}{\pgfqpoint{3.448833in}{0.976551in}}%
\pgfpathcurveto{\pgfqpoint{3.448833in}{0.987601in}}{\pgfqpoint{3.444443in}{0.998200in}}{\pgfqpoint{3.436629in}{1.006013in}}%
\pgfpathcurveto{\pgfqpoint{3.428816in}{1.013827in}}{\pgfqpoint{3.418217in}{1.018217in}}{\pgfqpoint{3.407167in}{1.018217in}}%
\pgfpathcurveto{\pgfqpoint{3.396116in}{1.018217in}}{\pgfqpoint{3.385517in}{1.013827in}}{\pgfqpoint{3.377704in}{1.006013in}}%
\pgfpathcurveto{\pgfqpoint{3.369890in}{0.998200in}}{\pgfqpoint{3.365500in}{0.987601in}}{\pgfqpoint{3.365500in}{0.976551in}}%
\pgfpathcurveto{\pgfqpoint{3.365500in}{0.965500in}}{\pgfqpoint{3.369890in}{0.954901in}}{\pgfqpoint{3.377704in}{0.947088in}}%
\pgfpathcurveto{\pgfqpoint{3.385517in}{0.939274in}}{\pgfqpoint{3.396116in}{0.934884in}}{\pgfqpoint{3.407167in}{0.934884in}}%
\pgfpathlineto{\pgfqpoint{3.407167in}{0.934884in}}%
\pgfpathclose%
\pgfusepath{stroke}%
\end{pgfscope}%
\begin{pgfscope}%
\pgfpathrectangle{\pgfqpoint{0.847223in}{0.554012in}}{\pgfqpoint{6.200000in}{4.530000in}}%
\pgfusepath{clip}%
\pgfsetbuttcap%
\pgfsetroundjoin%
\pgfsetlinewidth{1.003750pt}%
\definecolor{currentstroke}{rgb}{1.000000,0.000000,0.000000}%
\pgfsetstrokecolor{currentstroke}%
\pgfsetdash{}{0pt}%
\pgfpathmoveto{\pgfqpoint{3.412500in}{0.933434in}}%
\pgfpathcurveto{\pgfqpoint{3.423550in}{0.933434in}}{\pgfqpoint{3.434149in}{0.937824in}}{\pgfqpoint{3.441963in}{0.945638in}}%
\pgfpathcurveto{\pgfqpoint{3.449776in}{0.953451in}}{\pgfqpoint{3.454166in}{0.964050in}}{\pgfqpoint{3.454166in}{0.975101in}}%
\pgfpathcurveto{\pgfqpoint{3.454166in}{0.986151in}}{\pgfqpoint{3.449776in}{0.996750in}}{\pgfqpoint{3.441963in}{1.004563in}}%
\pgfpathcurveto{\pgfqpoint{3.434149in}{1.012377in}}{\pgfqpoint{3.423550in}{1.016767in}}{\pgfqpoint{3.412500in}{1.016767in}}%
\pgfpathcurveto{\pgfqpoint{3.401450in}{1.016767in}}{\pgfqpoint{3.390851in}{1.012377in}}{\pgfqpoint{3.383037in}{1.004563in}}%
\pgfpathcurveto{\pgfqpoint{3.375223in}{0.996750in}}{\pgfqpoint{3.370833in}{0.986151in}}{\pgfqpoint{3.370833in}{0.975101in}}%
\pgfpathcurveto{\pgfqpoint{3.370833in}{0.964050in}}{\pgfqpoint{3.375223in}{0.953451in}}{\pgfqpoint{3.383037in}{0.945638in}}%
\pgfpathcurveto{\pgfqpoint{3.390851in}{0.937824in}}{\pgfqpoint{3.401450in}{0.933434in}}{\pgfqpoint{3.412500in}{0.933434in}}%
\pgfpathlineto{\pgfqpoint{3.412500in}{0.933434in}}%
\pgfpathclose%
\pgfusepath{stroke}%
\end{pgfscope}%
\begin{pgfscope}%
\pgfpathrectangle{\pgfqpoint{0.847223in}{0.554012in}}{\pgfqpoint{6.200000in}{4.530000in}}%
\pgfusepath{clip}%
\pgfsetbuttcap%
\pgfsetroundjoin%
\pgfsetlinewidth{1.003750pt}%
\definecolor{currentstroke}{rgb}{1.000000,0.000000,0.000000}%
\pgfsetstrokecolor{currentstroke}%
\pgfsetdash{}{0pt}%
\pgfpathmoveto{\pgfqpoint{3.417833in}{0.931989in}}%
\pgfpathcurveto{\pgfqpoint{3.428883in}{0.931989in}}{\pgfqpoint{3.439482in}{0.936379in}}{\pgfqpoint{3.447296in}{0.944193in}}%
\pgfpathcurveto{\pgfqpoint{3.455109in}{0.952006in}}{\pgfqpoint{3.459500in}{0.962605in}}{\pgfqpoint{3.459500in}{0.973656in}}%
\pgfpathcurveto{\pgfqpoint{3.459500in}{0.984706in}}{\pgfqpoint{3.455109in}{0.995305in}}{\pgfqpoint{3.447296in}{1.003118in}}%
\pgfpathcurveto{\pgfqpoint{3.439482in}{1.010932in}}{\pgfqpoint{3.428883in}{1.015322in}}{\pgfqpoint{3.417833in}{1.015322in}}%
\pgfpathcurveto{\pgfqpoint{3.406783in}{1.015322in}}{\pgfqpoint{3.396184in}{1.010932in}}{\pgfqpoint{3.388370in}{1.003118in}}%
\pgfpathcurveto{\pgfqpoint{3.380557in}{0.995305in}}{\pgfqpoint{3.376166in}{0.984706in}}{\pgfqpoint{3.376166in}{0.973656in}}%
\pgfpathcurveto{\pgfqpoint{3.376166in}{0.962605in}}{\pgfqpoint{3.380557in}{0.952006in}}{\pgfqpoint{3.388370in}{0.944193in}}%
\pgfpathcurveto{\pgfqpoint{3.396184in}{0.936379in}}{\pgfqpoint{3.406783in}{0.931989in}}{\pgfqpoint{3.417833in}{0.931989in}}%
\pgfpathlineto{\pgfqpoint{3.417833in}{0.931989in}}%
\pgfpathclose%
\pgfusepath{stroke}%
\end{pgfscope}%
\begin{pgfscope}%
\pgfpathrectangle{\pgfqpoint{0.847223in}{0.554012in}}{\pgfqpoint{6.200000in}{4.530000in}}%
\pgfusepath{clip}%
\pgfsetbuttcap%
\pgfsetroundjoin%
\pgfsetlinewidth{1.003750pt}%
\definecolor{currentstroke}{rgb}{1.000000,0.000000,0.000000}%
\pgfsetstrokecolor{currentstroke}%
\pgfsetdash{}{0pt}%
\pgfpathmoveto{\pgfqpoint{3.423166in}{0.930549in}}%
\pgfpathcurveto{\pgfqpoint{3.434216in}{0.930549in}}{\pgfqpoint{3.444815in}{0.934939in}}{\pgfqpoint{3.452629in}{0.942753in}}%
\pgfpathcurveto{\pgfqpoint{3.460443in}{0.950566in}}{\pgfqpoint{3.464833in}{0.961165in}}{\pgfqpoint{3.464833in}{0.972216in}}%
\pgfpathcurveto{\pgfqpoint{3.464833in}{0.983266in}}{\pgfqpoint{3.460443in}{0.993865in}}{\pgfqpoint{3.452629in}{1.001678in}}%
\pgfpathcurveto{\pgfqpoint{3.444815in}{1.009492in}}{\pgfqpoint{3.434216in}{1.013882in}}{\pgfqpoint{3.423166in}{1.013882in}}%
\pgfpathcurveto{\pgfqpoint{3.412116in}{1.013882in}}{\pgfqpoint{3.401517in}{1.009492in}}{\pgfqpoint{3.393703in}{1.001678in}}%
\pgfpathcurveto{\pgfqpoint{3.385890in}{0.993865in}}{\pgfqpoint{3.381500in}{0.983266in}}{\pgfqpoint{3.381500in}{0.972216in}}%
\pgfpathcurveto{\pgfqpoint{3.381500in}{0.961165in}}{\pgfqpoint{3.385890in}{0.950566in}}{\pgfqpoint{3.393703in}{0.942753in}}%
\pgfpathcurveto{\pgfqpoint{3.401517in}{0.934939in}}{\pgfqpoint{3.412116in}{0.930549in}}{\pgfqpoint{3.423166in}{0.930549in}}%
\pgfpathlineto{\pgfqpoint{3.423166in}{0.930549in}}%
\pgfpathclose%
\pgfusepath{stroke}%
\end{pgfscope}%
\begin{pgfscope}%
\pgfpathrectangle{\pgfqpoint{0.847223in}{0.554012in}}{\pgfqpoint{6.200000in}{4.530000in}}%
\pgfusepath{clip}%
\pgfsetbuttcap%
\pgfsetroundjoin%
\pgfsetlinewidth{1.003750pt}%
\definecolor{currentstroke}{rgb}{1.000000,0.000000,0.000000}%
\pgfsetstrokecolor{currentstroke}%
\pgfsetdash{}{0pt}%
\pgfpathmoveto{\pgfqpoint{3.428499in}{0.929114in}}%
\pgfpathcurveto{\pgfqpoint{3.439550in}{0.929114in}}{\pgfqpoint{3.450149in}{0.933504in}}{\pgfqpoint{3.457962in}{0.941318in}}%
\pgfpathcurveto{\pgfqpoint{3.465776in}{0.949131in}}{\pgfqpoint{3.470166in}{0.959730in}}{\pgfqpoint{3.470166in}{0.970781in}}%
\pgfpathcurveto{\pgfqpoint{3.470166in}{0.981831in}}{\pgfqpoint{3.465776in}{0.992430in}}{\pgfqpoint{3.457962in}{1.000243in}}%
\pgfpathcurveto{\pgfqpoint{3.450149in}{1.008057in}}{\pgfqpoint{3.439550in}{1.012447in}}{\pgfqpoint{3.428499in}{1.012447in}}%
\pgfpathcurveto{\pgfqpoint{3.417449in}{1.012447in}}{\pgfqpoint{3.406850in}{1.008057in}}{\pgfqpoint{3.399037in}{1.000243in}}%
\pgfpathcurveto{\pgfqpoint{3.391223in}{0.992430in}}{\pgfqpoint{3.386833in}{0.981831in}}{\pgfqpoint{3.386833in}{0.970781in}}%
\pgfpathcurveto{\pgfqpoint{3.386833in}{0.959730in}}{\pgfqpoint{3.391223in}{0.949131in}}{\pgfqpoint{3.399037in}{0.941318in}}%
\pgfpathcurveto{\pgfqpoint{3.406850in}{0.933504in}}{\pgfqpoint{3.417449in}{0.929114in}}{\pgfqpoint{3.428499in}{0.929114in}}%
\pgfpathlineto{\pgfqpoint{3.428499in}{0.929114in}}%
\pgfpathclose%
\pgfusepath{stroke}%
\end{pgfscope}%
\begin{pgfscope}%
\pgfpathrectangle{\pgfqpoint{0.847223in}{0.554012in}}{\pgfqpoint{6.200000in}{4.530000in}}%
\pgfusepath{clip}%
\pgfsetbuttcap%
\pgfsetroundjoin%
\pgfsetlinewidth{1.003750pt}%
\definecolor{currentstroke}{rgb}{1.000000,0.000000,0.000000}%
\pgfsetstrokecolor{currentstroke}%
\pgfsetdash{}{0pt}%
\pgfpathmoveto{\pgfqpoint{3.433833in}{0.927684in}}%
\pgfpathcurveto{\pgfqpoint{3.444883in}{0.927684in}}{\pgfqpoint{3.455482in}{0.932074in}}{\pgfqpoint{3.463295in}{0.939888in}}%
\pgfpathcurveto{\pgfqpoint{3.471109in}{0.947701in}}{\pgfqpoint{3.475499in}{0.958300in}}{\pgfqpoint{3.475499in}{0.969350in}}%
\pgfpathcurveto{\pgfqpoint{3.475499in}{0.980401in}}{\pgfqpoint{3.471109in}{0.991000in}}{\pgfqpoint{3.463295in}{0.998813in}}%
\pgfpathcurveto{\pgfqpoint{3.455482in}{1.006627in}}{\pgfqpoint{3.444883in}{1.011017in}}{\pgfqpoint{3.433833in}{1.011017in}}%
\pgfpathcurveto{\pgfqpoint{3.422783in}{1.011017in}}{\pgfqpoint{3.412183in}{1.006627in}}{\pgfqpoint{3.404370in}{0.998813in}}%
\pgfpathcurveto{\pgfqpoint{3.396556in}{0.991000in}}{\pgfqpoint{3.392166in}{0.980401in}}{\pgfqpoint{3.392166in}{0.969350in}}%
\pgfpathcurveto{\pgfqpoint{3.392166in}{0.958300in}}{\pgfqpoint{3.396556in}{0.947701in}}{\pgfqpoint{3.404370in}{0.939888in}}%
\pgfpathcurveto{\pgfqpoint{3.412183in}{0.932074in}}{\pgfqpoint{3.422783in}{0.927684in}}{\pgfqpoint{3.433833in}{0.927684in}}%
\pgfpathlineto{\pgfqpoint{3.433833in}{0.927684in}}%
\pgfpathclose%
\pgfusepath{stroke}%
\end{pgfscope}%
\begin{pgfscope}%
\pgfpathrectangle{\pgfqpoint{0.847223in}{0.554012in}}{\pgfqpoint{6.200000in}{4.530000in}}%
\pgfusepath{clip}%
\pgfsetbuttcap%
\pgfsetroundjoin%
\pgfsetlinewidth{1.003750pt}%
\definecolor{currentstroke}{rgb}{1.000000,0.000000,0.000000}%
\pgfsetstrokecolor{currentstroke}%
\pgfsetdash{}{0pt}%
\pgfpathmoveto{\pgfqpoint{3.439166in}{0.926259in}}%
\pgfpathcurveto{\pgfqpoint{3.450216in}{0.926259in}}{\pgfqpoint{3.460815in}{0.930649in}}{\pgfqpoint{3.468629in}{0.938462in}}%
\pgfpathcurveto{\pgfqpoint{3.476442in}{0.946276in}}{\pgfqpoint{3.480833in}{0.956875in}}{\pgfqpoint{3.480833in}{0.967925in}}%
\pgfpathcurveto{\pgfqpoint{3.480833in}{0.978975in}}{\pgfqpoint{3.476442in}{0.989574in}}{\pgfqpoint{3.468629in}{0.997388in}}%
\pgfpathcurveto{\pgfqpoint{3.460815in}{1.005202in}}{\pgfqpoint{3.450216in}{1.009592in}}{\pgfqpoint{3.439166in}{1.009592in}}%
\pgfpathcurveto{\pgfqpoint{3.428116in}{1.009592in}}{\pgfqpoint{3.417517in}{1.005202in}}{\pgfqpoint{3.409703in}{0.997388in}}%
\pgfpathcurveto{\pgfqpoint{3.401889in}{0.989574in}}{\pgfqpoint{3.397499in}{0.978975in}}{\pgfqpoint{3.397499in}{0.967925in}}%
\pgfpathcurveto{\pgfqpoint{3.397499in}{0.956875in}}{\pgfqpoint{3.401889in}{0.946276in}}{\pgfqpoint{3.409703in}{0.938462in}}%
\pgfpathcurveto{\pgfqpoint{3.417517in}{0.930649in}}{\pgfqpoint{3.428116in}{0.926259in}}{\pgfqpoint{3.439166in}{0.926259in}}%
\pgfpathlineto{\pgfqpoint{3.439166in}{0.926259in}}%
\pgfpathclose%
\pgfusepath{stroke}%
\end{pgfscope}%
\begin{pgfscope}%
\pgfpathrectangle{\pgfqpoint{0.847223in}{0.554012in}}{\pgfqpoint{6.200000in}{4.530000in}}%
\pgfusepath{clip}%
\pgfsetbuttcap%
\pgfsetroundjoin%
\pgfsetlinewidth{1.003750pt}%
\definecolor{currentstroke}{rgb}{1.000000,0.000000,0.000000}%
\pgfsetstrokecolor{currentstroke}%
\pgfsetdash{}{0pt}%
\pgfpathmoveto{\pgfqpoint{3.444499in}{0.924838in}}%
\pgfpathcurveto{\pgfqpoint{3.455549in}{0.924838in}}{\pgfqpoint{3.466148in}{0.929228in}}{\pgfqpoint{3.473962in}{0.937042in}}%
\pgfpathcurveto{\pgfqpoint{3.481775in}{0.944856in}}{\pgfqpoint{3.486166in}{0.955455in}}{\pgfqpoint{3.486166in}{0.966505in}}%
\pgfpathcurveto{\pgfqpoint{3.486166in}{0.977555in}}{\pgfqpoint{3.481775in}{0.988154in}}{\pgfqpoint{3.473962in}{0.995968in}}%
\pgfpathcurveto{\pgfqpoint{3.466148in}{1.003781in}}{\pgfqpoint{3.455549in}{1.008171in}}{\pgfqpoint{3.444499in}{1.008171in}}%
\pgfpathcurveto{\pgfqpoint{3.433449in}{1.008171in}}{\pgfqpoint{3.422850in}{1.003781in}}{\pgfqpoint{3.415036in}{0.995968in}}%
\pgfpathcurveto{\pgfqpoint{3.407223in}{0.988154in}}{\pgfqpoint{3.402832in}{0.977555in}}{\pgfqpoint{3.402832in}{0.966505in}}%
\pgfpathcurveto{\pgfqpoint{3.402832in}{0.955455in}}{\pgfqpoint{3.407223in}{0.944856in}}{\pgfqpoint{3.415036in}{0.937042in}}%
\pgfpathcurveto{\pgfqpoint{3.422850in}{0.929228in}}{\pgfqpoint{3.433449in}{0.924838in}}{\pgfqpoint{3.444499in}{0.924838in}}%
\pgfpathlineto{\pgfqpoint{3.444499in}{0.924838in}}%
\pgfpathclose%
\pgfusepath{stroke}%
\end{pgfscope}%
\begin{pgfscope}%
\pgfpathrectangle{\pgfqpoint{0.847223in}{0.554012in}}{\pgfqpoint{6.200000in}{4.530000in}}%
\pgfusepath{clip}%
\pgfsetbuttcap%
\pgfsetroundjoin%
\pgfsetlinewidth{1.003750pt}%
\definecolor{currentstroke}{rgb}{1.000000,0.000000,0.000000}%
\pgfsetstrokecolor{currentstroke}%
\pgfsetdash{}{0pt}%
\pgfpathmoveto{\pgfqpoint{3.449832in}{0.923423in}}%
\pgfpathcurveto{\pgfqpoint{3.460882in}{0.923423in}}{\pgfqpoint{3.471481in}{0.927813in}}{\pgfqpoint{3.479295in}{0.935627in}}%
\pgfpathcurveto{\pgfqpoint{3.487109in}{0.943440in}}{\pgfqpoint{3.491499in}{0.954039in}}{\pgfqpoint{3.491499in}{0.965089in}}%
\pgfpathcurveto{\pgfqpoint{3.491499in}{0.976139in}}{\pgfqpoint{3.487109in}{0.986738in}}{\pgfqpoint{3.479295in}{0.994552in}}%
\pgfpathcurveto{\pgfqpoint{3.471481in}{1.002366in}}{\pgfqpoint{3.460882in}{1.006756in}}{\pgfqpoint{3.449832in}{1.006756in}}%
\pgfpathcurveto{\pgfqpoint{3.438782in}{1.006756in}}{\pgfqpoint{3.428183in}{1.002366in}}{\pgfqpoint{3.420370in}{0.994552in}}%
\pgfpathcurveto{\pgfqpoint{3.412556in}{0.986738in}}{\pgfqpoint{3.408166in}{0.976139in}}{\pgfqpoint{3.408166in}{0.965089in}}%
\pgfpathcurveto{\pgfqpoint{3.408166in}{0.954039in}}{\pgfqpoint{3.412556in}{0.943440in}}{\pgfqpoint{3.420370in}{0.935627in}}%
\pgfpathcurveto{\pgfqpoint{3.428183in}{0.927813in}}{\pgfqpoint{3.438782in}{0.923423in}}{\pgfqpoint{3.449832in}{0.923423in}}%
\pgfpathlineto{\pgfqpoint{3.449832in}{0.923423in}}%
\pgfpathclose%
\pgfusepath{stroke}%
\end{pgfscope}%
\begin{pgfscope}%
\pgfpathrectangle{\pgfqpoint{0.847223in}{0.554012in}}{\pgfqpoint{6.200000in}{4.530000in}}%
\pgfusepath{clip}%
\pgfsetbuttcap%
\pgfsetroundjoin%
\pgfsetlinewidth{1.003750pt}%
\definecolor{currentstroke}{rgb}{1.000000,0.000000,0.000000}%
\pgfsetstrokecolor{currentstroke}%
\pgfsetdash{}{0pt}%
\pgfpathmoveto{\pgfqpoint{3.455166in}{0.922012in}}%
\pgfpathcurveto{\pgfqpoint{3.466216in}{0.922012in}}{\pgfqpoint{3.476815in}{0.926402in}}{\pgfqpoint{3.484628in}{0.934216in}}%
\pgfpathcurveto{\pgfqpoint{3.492442in}{0.942029in}}{\pgfqpoint{3.496832in}{0.952628in}}{\pgfqpoint{3.496832in}{0.963679in}}%
\pgfpathcurveto{\pgfqpoint{3.496832in}{0.974729in}}{\pgfqpoint{3.492442in}{0.985328in}}{\pgfqpoint{3.484628in}{0.993141in}}%
\pgfpathcurveto{\pgfqpoint{3.476815in}{1.000955in}}{\pgfqpoint{3.466216in}{1.005345in}}{\pgfqpoint{3.455166in}{1.005345in}}%
\pgfpathcurveto{\pgfqpoint{3.444115in}{1.005345in}}{\pgfqpoint{3.433516in}{1.000955in}}{\pgfqpoint{3.425703in}{0.993141in}}%
\pgfpathcurveto{\pgfqpoint{3.417889in}{0.985328in}}{\pgfqpoint{3.413499in}{0.974729in}}{\pgfqpoint{3.413499in}{0.963679in}}%
\pgfpathcurveto{\pgfqpoint{3.413499in}{0.952628in}}{\pgfqpoint{3.417889in}{0.942029in}}{\pgfqpoint{3.425703in}{0.934216in}}%
\pgfpathcurveto{\pgfqpoint{3.433516in}{0.926402in}}{\pgfqpoint{3.444115in}{0.922012in}}{\pgfqpoint{3.455166in}{0.922012in}}%
\pgfpathlineto{\pgfqpoint{3.455166in}{0.922012in}}%
\pgfpathclose%
\pgfusepath{stroke}%
\end{pgfscope}%
\begin{pgfscope}%
\pgfpathrectangle{\pgfqpoint{0.847223in}{0.554012in}}{\pgfqpoint{6.200000in}{4.530000in}}%
\pgfusepath{clip}%
\pgfsetbuttcap%
\pgfsetroundjoin%
\pgfsetlinewidth{1.003750pt}%
\definecolor{currentstroke}{rgb}{1.000000,0.000000,0.000000}%
\pgfsetstrokecolor{currentstroke}%
\pgfsetdash{}{0pt}%
\pgfpathmoveto{\pgfqpoint{3.460499in}{0.920606in}}%
\pgfpathcurveto{\pgfqpoint{3.471549in}{0.920606in}}{\pgfqpoint{3.482148in}{0.924996in}}{\pgfqpoint{3.489962in}{0.932810in}}%
\pgfpathcurveto{\pgfqpoint{3.497775in}{0.940624in}}{\pgfqpoint{3.502165in}{0.951223in}}{\pgfqpoint{3.502165in}{0.962273in}}%
\pgfpathcurveto{\pgfqpoint{3.502165in}{0.973323in}}{\pgfqpoint{3.497775in}{0.983922in}}{\pgfqpoint{3.489962in}{0.991736in}}%
\pgfpathcurveto{\pgfqpoint{3.482148in}{0.999549in}}{\pgfqpoint{3.471549in}{1.003939in}}{\pgfqpoint{3.460499in}{1.003939in}}%
\pgfpathcurveto{\pgfqpoint{3.449449in}{1.003939in}}{\pgfqpoint{3.438850in}{0.999549in}}{\pgfqpoint{3.431036in}{0.991736in}}%
\pgfpathcurveto{\pgfqpoint{3.423222in}{0.983922in}}{\pgfqpoint{3.418832in}{0.973323in}}{\pgfqpoint{3.418832in}{0.962273in}}%
\pgfpathcurveto{\pgfqpoint{3.418832in}{0.951223in}}{\pgfqpoint{3.423222in}{0.940624in}}{\pgfqpoint{3.431036in}{0.932810in}}%
\pgfpathcurveto{\pgfqpoint{3.438850in}{0.924996in}}{\pgfqpoint{3.449449in}{0.920606in}}{\pgfqpoint{3.460499in}{0.920606in}}%
\pgfpathlineto{\pgfqpoint{3.460499in}{0.920606in}}%
\pgfpathclose%
\pgfusepath{stroke}%
\end{pgfscope}%
\begin{pgfscope}%
\pgfpathrectangle{\pgfqpoint{0.847223in}{0.554012in}}{\pgfqpoint{6.200000in}{4.530000in}}%
\pgfusepath{clip}%
\pgfsetbuttcap%
\pgfsetroundjoin%
\pgfsetlinewidth{1.003750pt}%
\definecolor{currentstroke}{rgb}{1.000000,0.000000,0.000000}%
\pgfsetstrokecolor{currentstroke}%
\pgfsetdash{}{0pt}%
\pgfpathmoveto{\pgfqpoint{3.465832in}{0.919205in}}%
\pgfpathcurveto{\pgfqpoint{3.476882in}{0.919205in}}{\pgfqpoint{3.487481in}{0.923595in}}{\pgfqpoint{3.495295in}{0.931409in}}%
\pgfpathcurveto{\pgfqpoint{3.503108in}{0.939222in}}{\pgfqpoint{3.507499in}{0.949821in}}{\pgfqpoint{3.507499in}{0.960872in}}%
\pgfpathcurveto{\pgfqpoint{3.507499in}{0.971922in}}{\pgfqpoint{3.503108in}{0.982521in}}{\pgfqpoint{3.495295in}{0.990334in}}%
\pgfpathcurveto{\pgfqpoint{3.487481in}{0.998148in}}{\pgfqpoint{3.476882in}{1.002538in}}{\pgfqpoint{3.465832in}{1.002538in}}%
\pgfpathcurveto{\pgfqpoint{3.454782in}{1.002538in}}{\pgfqpoint{3.444183in}{0.998148in}}{\pgfqpoint{3.436369in}{0.990334in}}%
\pgfpathcurveto{\pgfqpoint{3.428556in}{0.982521in}}{\pgfqpoint{3.424165in}{0.971922in}}{\pgfqpoint{3.424165in}{0.960872in}}%
\pgfpathcurveto{\pgfqpoint{3.424165in}{0.949821in}}{\pgfqpoint{3.428556in}{0.939222in}}{\pgfqpoint{3.436369in}{0.931409in}}%
\pgfpathcurveto{\pgfqpoint{3.444183in}{0.923595in}}{\pgfqpoint{3.454782in}{0.919205in}}{\pgfqpoint{3.465832in}{0.919205in}}%
\pgfpathlineto{\pgfqpoint{3.465832in}{0.919205in}}%
\pgfpathclose%
\pgfusepath{stroke}%
\end{pgfscope}%
\begin{pgfscope}%
\pgfpathrectangle{\pgfqpoint{0.847223in}{0.554012in}}{\pgfqpoint{6.200000in}{4.530000in}}%
\pgfusepath{clip}%
\pgfsetbuttcap%
\pgfsetroundjoin%
\pgfsetlinewidth{1.003750pt}%
\definecolor{currentstroke}{rgb}{1.000000,0.000000,0.000000}%
\pgfsetstrokecolor{currentstroke}%
\pgfsetdash{}{0pt}%
\pgfpathmoveto{\pgfqpoint{3.471165in}{0.917809in}}%
\pgfpathcurveto{\pgfqpoint{3.482215in}{0.917809in}}{\pgfqpoint{3.492814in}{0.922199in}}{\pgfqpoint{3.500628in}{0.930012in}}%
\pgfpathcurveto{\pgfqpoint{3.508442in}{0.937826in}}{\pgfqpoint{3.512832in}{0.948425in}}{\pgfqpoint{3.512832in}{0.959475in}}%
\pgfpathcurveto{\pgfqpoint{3.512832in}{0.970525in}}{\pgfqpoint{3.508442in}{0.981124in}}{\pgfqpoint{3.500628in}{0.988938in}}%
\pgfpathcurveto{\pgfqpoint{3.492814in}{0.996752in}}{\pgfqpoint{3.482215in}{1.001142in}}{\pgfqpoint{3.471165in}{1.001142in}}%
\pgfpathcurveto{\pgfqpoint{3.460115in}{1.001142in}}{\pgfqpoint{3.449516in}{0.996752in}}{\pgfqpoint{3.441702in}{0.988938in}}%
\pgfpathcurveto{\pgfqpoint{3.433889in}{0.981124in}}{\pgfqpoint{3.429498in}{0.970525in}}{\pgfqpoint{3.429498in}{0.959475in}}%
\pgfpathcurveto{\pgfqpoint{3.429498in}{0.948425in}}{\pgfqpoint{3.433889in}{0.937826in}}{\pgfqpoint{3.441702in}{0.930012in}}%
\pgfpathcurveto{\pgfqpoint{3.449516in}{0.922199in}}{\pgfqpoint{3.460115in}{0.917809in}}{\pgfqpoint{3.471165in}{0.917809in}}%
\pgfpathlineto{\pgfqpoint{3.471165in}{0.917809in}}%
\pgfpathclose%
\pgfusepath{stroke}%
\end{pgfscope}%
\begin{pgfscope}%
\pgfpathrectangle{\pgfqpoint{0.847223in}{0.554012in}}{\pgfqpoint{6.200000in}{4.530000in}}%
\pgfusepath{clip}%
\pgfsetbuttcap%
\pgfsetroundjoin%
\pgfsetlinewidth{1.003750pt}%
\definecolor{currentstroke}{rgb}{1.000000,0.000000,0.000000}%
\pgfsetstrokecolor{currentstroke}%
\pgfsetdash{}{0pt}%
\pgfpathmoveto{\pgfqpoint{3.476498in}{0.916417in}}%
\pgfpathcurveto{\pgfqpoint{3.487549in}{0.916417in}}{\pgfqpoint{3.498148in}{0.920807in}}{\pgfqpoint{3.505961in}{0.928621in}}%
\pgfpathcurveto{\pgfqpoint{3.513775in}{0.936434in}}{\pgfqpoint{3.518165in}{0.947033in}}{\pgfqpoint{3.518165in}{0.958084in}}%
\pgfpathcurveto{\pgfqpoint{3.518165in}{0.969134in}}{\pgfqpoint{3.513775in}{0.979733in}}{\pgfqpoint{3.505961in}{0.987546in}}%
\pgfpathcurveto{\pgfqpoint{3.498148in}{0.995360in}}{\pgfqpoint{3.487549in}{0.999750in}}{\pgfqpoint{3.476498in}{0.999750in}}%
\pgfpathcurveto{\pgfqpoint{3.465448in}{0.999750in}}{\pgfqpoint{3.454849in}{0.995360in}}{\pgfqpoint{3.447036in}{0.987546in}}%
\pgfpathcurveto{\pgfqpoint{3.439222in}{0.979733in}}{\pgfqpoint{3.434832in}{0.969134in}}{\pgfqpoint{3.434832in}{0.958084in}}%
\pgfpathcurveto{\pgfqpoint{3.434832in}{0.947033in}}{\pgfqpoint{3.439222in}{0.936434in}}{\pgfqpoint{3.447036in}{0.928621in}}%
\pgfpathcurveto{\pgfqpoint{3.454849in}{0.920807in}}{\pgfqpoint{3.465448in}{0.916417in}}{\pgfqpoint{3.476498in}{0.916417in}}%
\pgfpathlineto{\pgfqpoint{3.476498in}{0.916417in}}%
\pgfpathclose%
\pgfusepath{stroke}%
\end{pgfscope}%
\begin{pgfscope}%
\pgfpathrectangle{\pgfqpoint{0.847223in}{0.554012in}}{\pgfqpoint{6.200000in}{4.530000in}}%
\pgfusepath{clip}%
\pgfsetbuttcap%
\pgfsetroundjoin%
\pgfsetlinewidth{1.003750pt}%
\definecolor{currentstroke}{rgb}{1.000000,0.000000,0.000000}%
\pgfsetstrokecolor{currentstroke}%
\pgfsetdash{}{0pt}%
\pgfpathmoveto{\pgfqpoint{3.481832in}{0.915030in}}%
\pgfpathcurveto{\pgfqpoint{3.492882in}{0.915030in}}{\pgfqpoint{3.503481in}{0.919420in}}{\pgfqpoint{3.511294in}{0.927234in}}%
\pgfpathcurveto{\pgfqpoint{3.519108in}{0.935047in}}{\pgfqpoint{3.523498in}{0.945646in}}{\pgfqpoint{3.523498in}{0.956697in}}%
\pgfpathcurveto{\pgfqpoint{3.523498in}{0.967747in}}{\pgfqpoint{3.519108in}{0.978346in}}{\pgfqpoint{3.511294in}{0.986159in}}%
\pgfpathcurveto{\pgfqpoint{3.503481in}{0.993973in}}{\pgfqpoint{3.492882in}{0.998363in}}{\pgfqpoint{3.481832in}{0.998363in}}%
\pgfpathcurveto{\pgfqpoint{3.470781in}{0.998363in}}{\pgfqpoint{3.460182in}{0.993973in}}{\pgfqpoint{3.452369in}{0.986159in}}%
\pgfpathcurveto{\pgfqpoint{3.444555in}{0.978346in}}{\pgfqpoint{3.440165in}{0.967747in}}{\pgfqpoint{3.440165in}{0.956697in}}%
\pgfpathcurveto{\pgfqpoint{3.440165in}{0.945646in}}{\pgfqpoint{3.444555in}{0.935047in}}{\pgfqpoint{3.452369in}{0.927234in}}%
\pgfpathcurveto{\pgfqpoint{3.460182in}{0.919420in}}{\pgfqpoint{3.470781in}{0.915030in}}{\pgfqpoint{3.481832in}{0.915030in}}%
\pgfpathlineto{\pgfqpoint{3.481832in}{0.915030in}}%
\pgfpathclose%
\pgfusepath{stroke}%
\end{pgfscope}%
\begin{pgfscope}%
\pgfpathrectangle{\pgfqpoint{0.847223in}{0.554012in}}{\pgfqpoint{6.200000in}{4.530000in}}%
\pgfusepath{clip}%
\pgfsetbuttcap%
\pgfsetroundjoin%
\pgfsetlinewidth{1.003750pt}%
\definecolor{currentstroke}{rgb}{1.000000,0.000000,0.000000}%
\pgfsetstrokecolor{currentstroke}%
\pgfsetdash{}{0pt}%
\pgfpathmoveto{\pgfqpoint{3.487165in}{0.913648in}}%
\pgfpathcurveto{\pgfqpoint{3.498215in}{0.913648in}}{\pgfqpoint{3.508814in}{0.918038in}}{\pgfqpoint{3.516628in}{0.925852in}}%
\pgfpathcurveto{\pgfqpoint{3.524441in}{0.933665in}}{\pgfqpoint{3.528831in}{0.944264in}}{\pgfqpoint{3.528831in}{0.955314in}}%
\pgfpathcurveto{\pgfqpoint{3.528831in}{0.966364in}}{\pgfqpoint{3.524441in}{0.976964in}}{\pgfqpoint{3.516628in}{0.984777in}}%
\pgfpathcurveto{\pgfqpoint{3.508814in}{0.992591in}}{\pgfqpoint{3.498215in}{0.996981in}}{\pgfqpoint{3.487165in}{0.996981in}}%
\pgfpathcurveto{\pgfqpoint{3.476115in}{0.996981in}}{\pgfqpoint{3.465516in}{0.992591in}}{\pgfqpoint{3.457702in}{0.984777in}}%
\pgfpathcurveto{\pgfqpoint{3.449888in}{0.976964in}}{\pgfqpoint{3.445498in}{0.966364in}}{\pgfqpoint{3.445498in}{0.955314in}}%
\pgfpathcurveto{\pgfqpoint{3.445498in}{0.944264in}}{\pgfqpoint{3.449888in}{0.933665in}}{\pgfqpoint{3.457702in}{0.925852in}}%
\pgfpathcurveto{\pgfqpoint{3.465516in}{0.918038in}}{\pgfqpoint{3.476115in}{0.913648in}}{\pgfqpoint{3.487165in}{0.913648in}}%
\pgfpathlineto{\pgfqpoint{3.487165in}{0.913648in}}%
\pgfpathclose%
\pgfusepath{stroke}%
\end{pgfscope}%
\begin{pgfscope}%
\pgfpathrectangle{\pgfqpoint{0.847223in}{0.554012in}}{\pgfqpoint{6.200000in}{4.530000in}}%
\pgfusepath{clip}%
\pgfsetbuttcap%
\pgfsetroundjoin%
\pgfsetlinewidth{1.003750pt}%
\definecolor{currentstroke}{rgb}{1.000000,0.000000,0.000000}%
\pgfsetstrokecolor{currentstroke}%
\pgfsetdash{}{0pt}%
\pgfpathmoveto{\pgfqpoint{3.492498in}{0.912270in}}%
\pgfpathcurveto{\pgfqpoint{3.503548in}{0.912270in}}{\pgfqpoint{3.514147in}{0.916660in}}{\pgfqpoint{3.521961in}{0.924474in}}%
\pgfpathcurveto{\pgfqpoint{3.529774in}{0.932288in}}{\pgfqpoint{3.534165in}{0.942887in}}{\pgfqpoint{3.534165in}{0.953937in}}%
\pgfpathcurveto{\pgfqpoint{3.534165in}{0.964987in}}{\pgfqpoint{3.529774in}{0.975586in}}{\pgfqpoint{3.521961in}{0.983400in}}%
\pgfpathcurveto{\pgfqpoint{3.514147in}{0.991213in}}{\pgfqpoint{3.503548in}{0.995603in}}{\pgfqpoint{3.492498in}{0.995603in}}%
\pgfpathcurveto{\pgfqpoint{3.481448in}{0.995603in}}{\pgfqpoint{3.470849in}{0.991213in}}{\pgfqpoint{3.463035in}{0.983400in}}%
\pgfpathcurveto{\pgfqpoint{3.455222in}{0.975586in}}{\pgfqpoint{3.450831in}{0.964987in}}{\pgfqpoint{3.450831in}{0.953937in}}%
\pgfpathcurveto{\pgfqpoint{3.450831in}{0.942887in}}{\pgfqpoint{3.455222in}{0.932288in}}{\pgfqpoint{3.463035in}{0.924474in}}%
\pgfpathcurveto{\pgfqpoint{3.470849in}{0.916660in}}{\pgfqpoint{3.481448in}{0.912270in}}{\pgfqpoint{3.492498in}{0.912270in}}%
\pgfpathlineto{\pgfqpoint{3.492498in}{0.912270in}}%
\pgfpathclose%
\pgfusepath{stroke}%
\end{pgfscope}%
\begin{pgfscope}%
\pgfpathrectangle{\pgfqpoint{0.847223in}{0.554012in}}{\pgfqpoint{6.200000in}{4.530000in}}%
\pgfusepath{clip}%
\pgfsetbuttcap%
\pgfsetroundjoin%
\pgfsetlinewidth{1.003750pt}%
\definecolor{currentstroke}{rgb}{1.000000,0.000000,0.000000}%
\pgfsetstrokecolor{currentstroke}%
\pgfsetdash{}{0pt}%
\pgfpathmoveto{\pgfqpoint{3.497831in}{0.910897in}}%
\pgfpathcurveto{\pgfqpoint{3.508881in}{0.910897in}}{\pgfqpoint{3.519480in}{0.915287in}}{\pgfqpoint{3.527294in}{0.923101in}}%
\pgfpathcurveto{\pgfqpoint{3.535108in}{0.930915in}}{\pgfqpoint{3.539498in}{0.941514in}}{\pgfqpoint{3.539498in}{0.952564in}}%
\pgfpathcurveto{\pgfqpoint{3.539498in}{0.963614in}}{\pgfqpoint{3.535108in}{0.974213in}}{\pgfqpoint{3.527294in}{0.982027in}}%
\pgfpathcurveto{\pgfqpoint{3.519480in}{0.989840in}}{\pgfqpoint{3.508881in}{0.994230in}}{\pgfqpoint{3.497831in}{0.994230in}}%
\pgfpathcurveto{\pgfqpoint{3.486781in}{0.994230in}}{\pgfqpoint{3.476182in}{0.989840in}}{\pgfqpoint{3.468368in}{0.982027in}}%
\pgfpathcurveto{\pgfqpoint{3.460555in}{0.974213in}}{\pgfqpoint{3.456165in}{0.963614in}}{\pgfqpoint{3.456165in}{0.952564in}}%
\pgfpathcurveto{\pgfqpoint{3.456165in}{0.941514in}}{\pgfqpoint{3.460555in}{0.930915in}}{\pgfqpoint{3.468368in}{0.923101in}}%
\pgfpathcurveto{\pgfqpoint{3.476182in}{0.915287in}}{\pgfqpoint{3.486781in}{0.910897in}}{\pgfqpoint{3.497831in}{0.910897in}}%
\pgfpathlineto{\pgfqpoint{3.497831in}{0.910897in}}%
\pgfpathclose%
\pgfusepath{stroke}%
\end{pgfscope}%
\begin{pgfscope}%
\pgfpathrectangle{\pgfqpoint{0.847223in}{0.554012in}}{\pgfqpoint{6.200000in}{4.530000in}}%
\pgfusepath{clip}%
\pgfsetbuttcap%
\pgfsetroundjoin%
\pgfsetlinewidth{1.003750pt}%
\definecolor{currentstroke}{rgb}{1.000000,0.000000,0.000000}%
\pgfsetstrokecolor{currentstroke}%
\pgfsetdash{}{0pt}%
\pgfpathmoveto{\pgfqpoint{3.503164in}{0.909529in}}%
\pgfpathcurveto{\pgfqpoint{3.514215in}{0.909529in}}{\pgfqpoint{3.524814in}{0.913919in}}{\pgfqpoint{3.532627in}{0.921733in}}%
\pgfpathcurveto{\pgfqpoint{3.540441in}{0.929546in}}{\pgfqpoint{3.544831in}{0.940145in}}{\pgfqpoint{3.544831in}{0.951195in}}%
\pgfpathcurveto{\pgfqpoint{3.544831in}{0.962245in}}{\pgfqpoint{3.540441in}{0.972845in}}{\pgfqpoint{3.532627in}{0.980658in}}%
\pgfpathcurveto{\pgfqpoint{3.524814in}{0.988472in}}{\pgfqpoint{3.514215in}{0.992862in}}{\pgfqpoint{3.503164in}{0.992862in}}%
\pgfpathcurveto{\pgfqpoint{3.492114in}{0.992862in}}{\pgfqpoint{3.481515in}{0.988472in}}{\pgfqpoint{3.473702in}{0.980658in}}%
\pgfpathcurveto{\pgfqpoint{3.465888in}{0.972845in}}{\pgfqpoint{3.461498in}{0.962245in}}{\pgfqpoint{3.461498in}{0.951195in}}%
\pgfpathcurveto{\pgfqpoint{3.461498in}{0.940145in}}{\pgfqpoint{3.465888in}{0.929546in}}{\pgfqpoint{3.473702in}{0.921733in}}%
\pgfpathcurveto{\pgfqpoint{3.481515in}{0.913919in}}{\pgfqpoint{3.492114in}{0.909529in}}{\pgfqpoint{3.503164in}{0.909529in}}%
\pgfpathlineto{\pgfqpoint{3.503164in}{0.909529in}}%
\pgfpathclose%
\pgfusepath{stroke}%
\end{pgfscope}%
\begin{pgfscope}%
\pgfpathrectangle{\pgfqpoint{0.847223in}{0.554012in}}{\pgfqpoint{6.200000in}{4.530000in}}%
\pgfusepath{clip}%
\pgfsetbuttcap%
\pgfsetroundjoin%
\pgfsetlinewidth{1.003750pt}%
\definecolor{currentstroke}{rgb}{1.000000,0.000000,0.000000}%
\pgfsetstrokecolor{currentstroke}%
\pgfsetdash{}{0pt}%
\pgfpathmoveto{\pgfqpoint{3.508498in}{0.908165in}}%
\pgfpathcurveto{\pgfqpoint{3.519548in}{0.908165in}}{\pgfqpoint{3.530147in}{0.912555in}}{\pgfqpoint{3.537960in}{0.920369in}}%
\pgfpathcurveto{\pgfqpoint{3.545774in}{0.928182in}}{\pgfqpoint{3.550164in}{0.938781in}}{\pgfqpoint{3.550164in}{0.949832in}}%
\pgfpathcurveto{\pgfqpoint{3.550164in}{0.960882in}}{\pgfqpoint{3.545774in}{0.971481in}}{\pgfqpoint{3.537960in}{0.979294in}}%
\pgfpathcurveto{\pgfqpoint{3.530147in}{0.987108in}}{\pgfqpoint{3.519548in}{0.991498in}}{\pgfqpoint{3.508498in}{0.991498in}}%
\pgfpathcurveto{\pgfqpoint{3.497448in}{0.991498in}}{\pgfqpoint{3.486849in}{0.987108in}}{\pgfqpoint{3.479035in}{0.979294in}}%
\pgfpathcurveto{\pgfqpoint{3.471221in}{0.971481in}}{\pgfqpoint{3.466831in}{0.960882in}}{\pgfqpoint{3.466831in}{0.949832in}}%
\pgfpathcurveto{\pgfqpoint{3.466831in}{0.938781in}}{\pgfqpoint{3.471221in}{0.928182in}}{\pgfqpoint{3.479035in}{0.920369in}}%
\pgfpathcurveto{\pgfqpoint{3.486849in}{0.912555in}}{\pgfqpoint{3.497448in}{0.908165in}}{\pgfqpoint{3.508498in}{0.908165in}}%
\pgfpathlineto{\pgfqpoint{3.508498in}{0.908165in}}%
\pgfpathclose%
\pgfusepath{stroke}%
\end{pgfscope}%
\begin{pgfscope}%
\pgfpathrectangle{\pgfqpoint{0.847223in}{0.554012in}}{\pgfqpoint{6.200000in}{4.530000in}}%
\pgfusepath{clip}%
\pgfsetbuttcap%
\pgfsetroundjoin%
\pgfsetlinewidth{1.003750pt}%
\definecolor{currentstroke}{rgb}{1.000000,0.000000,0.000000}%
\pgfsetstrokecolor{currentstroke}%
\pgfsetdash{}{0pt}%
\pgfpathmoveto{\pgfqpoint{3.513831in}{0.906806in}}%
\pgfpathcurveto{\pgfqpoint{3.524881in}{0.906806in}}{\pgfqpoint{3.535480in}{0.911196in}}{\pgfqpoint{3.543294in}{0.919010in}}%
\pgfpathcurveto{\pgfqpoint{3.551107in}{0.926823in}}{\pgfqpoint{3.555498in}{0.937422in}}{\pgfqpoint{3.555498in}{0.948472in}}%
\pgfpathcurveto{\pgfqpoint{3.555498in}{0.959522in}}{\pgfqpoint{3.551107in}{0.970121in}}{\pgfqpoint{3.543294in}{0.977935in}}%
\pgfpathcurveto{\pgfqpoint{3.535480in}{0.985749in}}{\pgfqpoint{3.524881in}{0.990139in}}{\pgfqpoint{3.513831in}{0.990139in}}%
\pgfpathcurveto{\pgfqpoint{3.502781in}{0.990139in}}{\pgfqpoint{3.492182in}{0.985749in}}{\pgfqpoint{3.484368in}{0.977935in}}%
\pgfpathcurveto{\pgfqpoint{3.476554in}{0.970121in}}{\pgfqpoint{3.472164in}{0.959522in}}{\pgfqpoint{3.472164in}{0.948472in}}%
\pgfpathcurveto{\pgfqpoint{3.472164in}{0.937422in}}{\pgfqpoint{3.476554in}{0.926823in}}{\pgfqpoint{3.484368in}{0.919010in}}%
\pgfpathcurveto{\pgfqpoint{3.492182in}{0.911196in}}{\pgfqpoint{3.502781in}{0.906806in}}{\pgfqpoint{3.513831in}{0.906806in}}%
\pgfpathlineto{\pgfqpoint{3.513831in}{0.906806in}}%
\pgfpathclose%
\pgfusepath{stroke}%
\end{pgfscope}%
\begin{pgfscope}%
\pgfpathrectangle{\pgfqpoint{0.847223in}{0.554012in}}{\pgfqpoint{6.200000in}{4.530000in}}%
\pgfusepath{clip}%
\pgfsetbuttcap%
\pgfsetroundjoin%
\pgfsetlinewidth{1.003750pt}%
\definecolor{currentstroke}{rgb}{1.000000,0.000000,0.000000}%
\pgfsetstrokecolor{currentstroke}%
\pgfsetdash{}{0pt}%
\pgfpathmoveto{\pgfqpoint{3.519164in}{0.905451in}}%
\pgfpathcurveto{\pgfqpoint{3.530214in}{0.905451in}}{\pgfqpoint{3.540813in}{0.909841in}}{\pgfqpoint{3.548627in}{0.917655in}}%
\pgfpathcurveto{\pgfqpoint{3.556441in}{0.925468in}}{\pgfqpoint{3.560831in}{0.936067in}}{\pgfqpoint{3.560831in}{0.947118in}}%
\pgfpathcurveto{\pgfqpoint{3.560831in}{0.958168in}}{\pgfqpoint{3.556441in}{0.968767in}}{\pgfqpoint{3.548627in}{0.976580in}}%
\pgfpathcurveto{\pgfqpoint{3.540813in}{0.984394in}}{\pgfqpoint{3.530214in}{0.988784in}}{\pgfqpoint{3.519164in}{0.988784in}}%
\pgfpathcurveto{\pgfqpoint{3.508114in}{0.988784in}}{\pgfqpoint{3.497515in}{0.984394in}}{\pgfqpoint{3.489701in}{0.976580in}}%
\pgfpathcurveto{\pgfqpoint{3.481888in}{0.968767in}}{\pgfqpoint{3.477497in}{0.958168in}}{\pgfqpoint{3.477497in}{0.947118in}}%
\pgfpathcurveto{\pgfqpoint{3.477497in}{0.936067in}}{\pgfqpoint{3.481888in}{0.925468in}}{\pgfqpoint{3.489701in}{0.917655in}}%
\pgfpathcurveto{\pgfqpoint{3.497515in}{0.909841in}}{\pgfqpoint{3.508114in}{0.905451in}}{\pgfqpoint{3.519164in}{0.905451in}}%
\pgfpathlineto{\pgfqpoint{3.519164in}{0.905451in}}%
\pgfpathclose%
\pgfusepath{stroke}%
\end{pgfscope}%
\begin{pgfscope}%
\pgfpathrectangle{\pgfqpoint{0.847223in}{0.554012in}}{\pgfqpoint{6.200000in}{4.530000in}}%
\pgfusepath{clip}%
\pgfsetbuttcap%
\pgfsetroundjoin%
\pgfsetlinewidth{1.003750pt}%
\definecolor{currentstroke}{rgb}{1.000000,0.000000,0.000000}%
\pgfsetstrokecolor{currentstroke}%
\pgfsetdash{}{0pt}%
\pgfpathmoveto{\pgfqpoint{3.524497in}{0.904101in}}%
\pgfpathcurveto{\pgfqpoint{3.535547in}{0.904101in}}{\pgfqpoint{3.546146in}{0.908491in}}{\pgfqpoint{3.553960in}{0.916305in}}%
\pgfpathcurveto{\pgfqpoint{3.561774in}{0.924118in}}{\pgfqpoint{3.566164in}{0.934717in}}{\pgfqpoint{3.566164in}{0.945767in}}%
\pgfpathcurveto{\pgfqpoint{3.566164in}{0.956818in}}{\pgfqpoint{3.561774in}{0.967417in}}{\pgfqpoint{3.553960in}{0.975230in}}%
\pgfpathcurveto{\pgfqpoint{3.546146in}{0.983044in}}{\pgfqpoint{3.535547in}{0.987434in}}{\pgfqpoint{3.524497in}{0.987434in}}%
\pgfpathcurveto{\pgfqpoint{3.513447in}{0.987434in}}{\pgfqpoint{3.502848in}{0.983044in}}{\pgfqpoint{3.495035in}{0.975230in}}%
\pgfpathcurveto{\pgfqpoint{3.487221in}{0.967417in}}{\pgfqpoint{3.482831in}{0.956818in}}{\pgfqpoint{3.482831in}{0.945767in}}%
\pgfpathcurveto{\pgfqpoint{3.482831in}{0.934717in}}{\pgfqpoint{3.487221in}{0.924118in}}{\pgfqpoint{3.495035in}{0.916305in}}%
\pgfpathcurveto{\pgfqpoint{3.502848in}{0.908491in}}{\pgfqpoint{3.513447in}{0.904101in}}{\pgfqpoint{3.524497in}{0.904101in}}%
\pgfpathlineto{\pgfqpoint{3.524497in}{0.904101in}}%
\pgfpathclose%
\pgfusepath{stroke}%
\end{pgfscope}%
\begin{pgfscope}%
\pgfpathrectangle{\pgfqpoint{0.847223in}{0.554012in}}{\pgfqpoint{6.200000in}{4.530000in}}%
\pgfusepath{clip}%
\pgfsetbuttcap%
\pgfsetroundjoin%
\pgfsetlinewidth{1.003750pt}%
\definecolor{currentstroke}{rgb}{1.000000,0.000000,0.000000}%
\pgfsetstrokecolor{currentstroke}%
\pgfsetdash{}{0pt}%
\pgfpathmoveto{\pgfqpoint{3.529831in}{0.902755in}}%
\pgfpathcurveto{\pgfqpoint{3.540881in}{0.902755in}}{\pgfqpoint{3.551480in}{0.907145in}}{\pgfqpoint{3.559293in}{0.914959in}}%
\pgfpathcurveto{\pgfqpoint{3.567107in}{0.922773in}}{\pgfqpoint{3.571497in}{0.933372in}}{\pgfqpoint{3.571497in}{0.944422in}}%
\pgfpathcurveto{\pgfqpoint{3.571497in}{0.955472in}}{\pgfqpoint{3.567107in}{0.966071in}}{\pgfqpoint{3.559293in}{0.973885in}}%
\pgfpathcurveto{\pgfqpoint{3.551480in}{0.981698in}}{\pgfqpoint{3.540881in}{0.986088in}}{\pgfqpoint{3.529831in}{0.986088in}}%
\pgfpathcurveto{\pgfqpoint{3.518780in}{0.986088in}}{\pgfqpoint{3.508181in}{0.981698in}}{\pgfqpoint{3.500368in}{0.973885in}}%
\pgfpathcurveto{\pgfqpoint{3.492554in}{0.966071in}}{\pgfqpoint{3.488164in}{0.955472in}}{\pgfqpoint{3.488164in}{0.944422in}}%
\pgfpathcurveto{\pgfqpoint{3.488164in}{0.933372in}}{\pgfqpoint{3.492554in}{0.922773in}}{\pgfqpoint{3.500368in}{0.914959in}}%
\pgfpathcurveto{\pgfqpoint{3.508181in}{0.907145in}}{\pgfqpoint{3.518780in}{0.902755in}}{\pgfqpoint{3.529831in}{0.902755in}}%
\pgfpathlineto{\pgfqpoint{3.529831in}{0.902755in}}%
\pgfpathclose%
\pgfusepath{stroke}%
\end{pgfscope}%
\begin{pgfscope}%
\pgfpathrectangle{\pgfqpoint{0.847223in}{0.554012in}}{\pgfqpoint{6.200000in}{4.530000in}}%
\pgfusepath{clip}%
\pgfsetbuttcap%
\pgfsetroundjoin%
\pgfsetlinewidth{1.003750pt}%
\definecolor{currentstroke}{rgb}{1.000000,0.000000,0.000000}%
\pgfsetstrokecolor{currentstroke}%
\pgfsetdash{}{0pt}%
\pgfpathmoveto{\pgfqpoint{3.535164in}{0.901414in}}%
\pgfpathcurveto{\pgfqpoint{3.546214in}{0.901414in}}{\pgfqpoint{3.556813in}{0.905804in}}{\pgfqpoint{3.564627in}{0.913618in}}%
\pgfpathcurveto{\pgfqpoint{3.572440in}{0.921431in}}{\pgfqpoint{3.576830in}{0.932030in}}{\pgfqpoint{3.576830in}{0.943081in}}%
\pgfpathcurveto{\pgfqpoint{3.576830in}{0.954131in}}{\pgfqpoint{3.572440in}{0.964730in}}{\pgfqpoint{3.564627in}{0.972543in}}%
\pgfpathcurveto{\pgfqpoint{3.556813in}{0.980357in}}{\pgfqpoint{3.546214in}{0.984747in}}{\pgfqpoint{3.535164in}{0.984747in}}%
\pgfpathcurveto{\pgfqpoint{3.524114in}{0.984747in}}{\pgfqpoint{3.513515in}{0.980357in}}{\pgfqpoint{3.505701in}{0.972543in}}%
\pgfpathcurveto{\pgfqpoint{3.497887in}{0.964730in}}{\pgfqpoint{3.493497in}{0.954131in}}{\pgfqpoint{3.493497in}{0.943081in}}%
\pgfpathcurveto{\pgfqpoint{3.493497in}{0.932030in}}{\pgfqpoint{3.497887in}{0.921431in}}{\pgfqpoint{3.505701in}{0.913618in}}%
\pgfpathcurveto{\pgfqpoint{3.513515in}{0.905804in}}{\pgfqpoint{3.524114in}{0.901414in}}{\pgfqpoint{3.535164in}{0.901414in}}%
\pgfpathlineto{\pgfqpoint{3.535164in}{0.901414in}}%
\pgfpathclose%
\pgfusepath{stroke}%
\end{pgfscope}%
\begin{pgfscope}%
\pgfpathrectangle{\pgfqpoint{0.847223in}{0.554012in}}{\pgfqpoint{6.200000in}{4.530000in}}%
\pgfusepath{clip}%
\pgfsetbuttcap%
\pgfsetroundjoin%
\pgfsetlinewidth{1.003750pt}%
\definecolor{currentstroke}{rgb}{1.000000,0.000000,0.000000}%
\pgfsetstrokecolor{currentstroke}%
\pgfsetdash{}{0pt}%
\pgfpathmoveto{\pgfqpoint{3.540497in}{0.900077in}}%
\pgfpathcurveto{\pgfqpoint{3.551547in}{0.900077in}}{\pgfqpoint{3.562146in}{0.904467in}}{\pgfqpoint{3.569960in}{0.912281in}}%
\pgfpathcurveto{\pgfqpoint{3.577773in}{0.920095in}}{\pgfqpoint{3.582164in}{0.930694in}}{\pgfqpoint{3.582164in}{0.941744in}}%
\pgfpathcurveto{\pgfqpoint{3.582164in}{0.952794in}}{\pgfqpoint{3.577773in}{0.963393in}}{\pgfqpoint{3.569960in}{0.971207in}}%
\pgfpathcurveto{\pgfqpoint{3.562146in}{0.979020in}}{\pgfqpoint{3.551547in}{0.983410in}}{\pgfqpoint{3.540497in}{0.983410in}}%
\pgfpathcurveto{\pgfqpoint{3.529447in}{0.983410in}}{\pgfqpoint{3.518848in}{0.979020in}}{\pgfqpoint{3.511034in}{0.971207in}}%
\pgfpathcurveto{\pgfqpoint{3.503221in}{0.963393in}}{\pgfqpoint{3.498830in}{0.952794in}}{\pgfqpoint{3.498830in}{0.941744in}}%
\pgfpathcurveto{\pgfqpoint{3.498830in}{0.930694in}}{\pgfqpoint{3.503221in}{0.920095in}}{\pgfqpoint{3.511034in}{0.912281in}}%
\pgfpathcurveto{\pgfqpoint{3.518848in}{0.904467in}}{\pgfqpoint{3.529447in}{0.900077in}}{\pgfqpoint{3.540497in}{0.900077in}}%
\pgfpathlineto{\pgfqpoint{3.540497in}{0.900077in}}%
\pgfpathclose%
\pgfusepath{stroke}%
\end{pgfscope}%
\begin{pgfscope}%
\pgfpathrectangle{\pgfqpoint{0.847223in}{0.554012in}}{\pgfqpoint{6.200000in}{4.530000in}}%
\pgfusepath{clip}%
\pgfsetbuttcap%
\pgfsetroundjoin%
\pgfsetlinewidth{1.003750pt}%
\definecolor{currentstroke}{rgb}{1.000000,0.000000,0.000000}%
\pgfsetstrokecolor{currentstroke}%
\pgfsetdash{}{0pt}%
\pgfpathmoveto{\pgfqpoint{3.545830in}{0.898745in}}%
\pgfpathcurveto{\pgfqpoint{3.556880in}{0.898745in}}{\pgfqpoint{3.567479in}{0.903135in}}{\pgfqpoint{3.575293in}{0.910949in}}%
\pgfpathcurveto{\pgfqpoint{3.583107in}{0.918762in}}{\pgfqpoint{3.587497in}{0.929361in}}{\pgfqpoint{3.587497in}{0.940411in}}%
\pgfpathcurveto{\pgfqpoint{3.587497in}{0.951461in}}{\pgfqpoint{3.583107in}{0.962061in}}{\pgfqpoint{3.575293in}{0.969874in}}%
\pgfpathcurveto{\pgfqpoint{3.567479in}{0.977688in}}{\pgfqpoint{3.556880in}{0.982078in}}{\pgfqpoint{3.545830in}{0.982078in}}%
\pgfpathcurveto{\pgfqpoint{3.534780in}{0.982078in}}{\pgfqpoint{3.524181in}{0.977688in}}{\pgfqpoint{3.516367in}{0.969874in}}%
\pgfpathcurveto{\pgfqpoint{3.508554in}{0.962061in}}{\pgfqpoint{3.504164in}{0.951461in}}{\pgfqpoint{3.504164in}{0.940411in}}%
\pgfpathcurveto{\pgfqpoint{3.504164in}{0.929361in}}{\pgfqpoint{3.508554in}{0.918762in}}{\pgfqpoint{3.516367in}{0.910949in}}%
\pgfpathcurveto{\pgfqpoint{3.524181in}{0.903135in}}{\pgfqpoint{3.534780in}{0.898745in}}{\pgfqpoint{3.545830in}{0.898745in}}%
\pgfpathlineto{\pgfqpoint{3.545830in}{0.898745in}}%
\pgfpathclose%
\pgfusepath{stroke}%
\end{pgfscope}%
\begin{pgfscope}%
\pgfpathrectangle{\pgfqpoint{0.847223in}{0.554012in}}{\pgfqpoint{6.200000in}{4.530000in}}%
\pgfusepath{clip}%
\pgfsetbuttcap%
\pgfsetroundjoin%
\pgfsetlinewidth{1.003750pt}%
\definecolor{currentstroke}{rgb}{1.000000,0.000000,0.000000}%
\pgfsetstrokecolor{currentstroke}%
\pgfsetdash{}{0pt}%
\pgfpathmoveto{\pgfqpoint{3.551163in}{0.897417in}}%
\pgfpathcurveto{\pgfqpoint{3.562214in}{0.897417in}}{\pgfqpoint{3.572813in}{0.901807in}}{\pgfqpoint{3.580626in}{0.909621in}}%
\pgfpathcurveto{\pgfqpoint{3.588440in}{0.917434in}}{\pgfqpoint{3.592830in}{0.928033in}}{\pgfqpoint{3.592830in}{0.939083in}}%
\pgfpathcurveto{\pgfqpoint{3.592830in}{0.950134in}}{\pgfqpoint{3.588440in}{0.960733in}}{\pgfqpoint{3.580626in}{0.968546in}}%
\pgfpathcurveto{\pgfqpoint{3.572813in}{0.976360in}}{\pgfqpoint{3.562214in}{0.980750in}}{\pgfqpoint{3.551163in}{0.980750in}}%
\pgfpathcurveto{\pgfqpoint{3.540113in}{0.980750in}}{\pgfqpoint{3.529514in}{0.976360in}}{\pgfqpoint{3.521701in}{0.968546in}}%
\pgfpathcurveto{\pgfqpoint{3.513887in}{0.960733in}}{\pgfqpoint{3.509497in}{0.950134in}}{\pgfqpoint{3.509497in}{0.939083in}}%
\pgfpathcurveto{\pgfqpoint{3.509497in}{0.928033in}}{\pgfqpoint{3.513887in}{0.917434in}}{\pgfqpoint{3.521701in}{0.909621in}}%
\pgfpathcurveto{\pgfqpoint{3.529514in}{0.901807in}}{\pgfqpoint{3.540113in}{0.897417in}}{\pgfqpoint{3.551163in}{0.897417in}}%
\pgfpathlineto{\pgfqpoint{3.551163in}{0.897417in}}%
\pgfpathclose%
\pgfusepath{stroke}%
\end{pgfscope}%
\begin{pgfscope}%
\pgfpathrectangle{\pgfqpoint{0.847223in}{0.554012in}}{\pgfqpoint{6.200000in}{4.530000in}}%
\pgfusepath{clip}%
\pgfsetbuttcap%
\pgfsetroundjoin%
\pgfsetlinewidth{1.003750pt}%
\definecolor{currentstroke}{rgb}{1.000000,0.000000,0.000000}%
\pgfsetstrokecolor{currentstroke}%
\pgfsetdash{}{0pt}%
\pgfpathmoveto{\pgfqpoint{3.556497in}{0.896093in}}%
\pgfpathcurveto{\pgfqpoint{3.567547in}{0.896093in}}{\pgfqpoint{3.578146in}{0.900483in}}{\pgfqpoint{3.585959in}{0.908297in}}%
\pgfpathcurveto{\pgfqpoint{3.593773in}{0.916111in}}{\pgfqpoint{3.598163in}{0.926710in}}{\pgfqpoint{3.598163in}{0.937760in}}%
\pgfpathcurveto{\pgfqpoint{3.598163in}{0.948810in}}{\pgfqpoint{3.593773in}{0.959409in}}{\pgfqpoint{3.585959in}{0.967223in}}%
\pgfpathcurveto{\pgfqpoint{3.578146in}{0.975036in}}{\pgfqpoint{3.567547in}{0.979427in}}{\pgfqpoint{3.556497in}{0.979427in}}%
\pgfpathcurveto{\pgfqpoint{3.545446in}{0.979427in}}{\pgfqpoint{3.534847in}{0.975036in}}{\pgfqpoint{3.527034in}{0.967223in}}%
\pgfpathcurveto{\pgfqpoint{3.519220in}{0.959409in}}{\pgfqpoint{3.514830in}{0.948810in}}{\pgfqpoint{3.514830in}{0.937760in}}%
\pgfpathcurveto{\pgfqpoint{3.514830in}{0.926710in}}{\pgfqpoint{3.519220in}{0.916111in}}{\pgfqpoint{3.527034in}{0.908297in}}%
\pgfpathcurveto{\pgfqpoint{3.534847in}{0.900483in}}{\pgfqpoint{3.545446in}{0.896093in}}{\pgfqpoint{3.556497in}{0.896093in}}%
\pgfpathlineto{\pgfqpoint{3.556497in}{0.896093in}}%
\pgfpathclose%
\pgfusepath{stroke}%
\end{pgfscope}%
\begin{pgfscope}%
\pgfpathrectangle{\pgfqpoint{0.847223in}{0.554012in}}{\pgfqpoint{6.200000in}{4.530000in}}%
\pgfusepath{clip}%
\pgfsetbuttcap%
\pgfsetroundjoin%
\pgfsetlinewidth{1.003750pt}%
\definecolor{currentstroke}{rgb}{1.000000,0.000000,0.000000}%
\pgfsetstrokecolor{currentstroke}%
\pgfsetdash{}{0pt}%
\pgfpathmoveto{\pgfqpoint{3.561830in}{0.894774in}}%
\pgfpathcurveto{\pgfqpoint{3.572880in}{0.894774in}}{\pgfqpoint{3.583479in}{0.899164in}}{\pgfqpoint{3.591293in}{0.906978in}}%
\pgfpathcurveto{\pgfqpoint{3.599106in}{0.914791in}}{\pgfqpoint{3.603496in}{0.925391in}}{\pgfqpoint{3.603496in}{0.936441in}}%
\pgfpathcurveto{\pgfqpoint{3.603496in}{0.947491in}}{\pgfqpoint{3.599106in}{0.958090in}}{\pgfqpoint{3.591293in}{0.965903in}}%
\pgfpathcurveto{\pgfqpoint{3.583479in}{0.973717in}}{\pgfqpoint{3.572880in}{0.978107in}}{\pgfqpoint{3.561830in}{0.978107in}}%
\pgfpathcurveto{\pgfqpoint{3.550780in}{0.978107in}}{\pgfqpoint{3.540181in}{0.973717in}}{\pgfqpoint{3.532367in}{0.965903in}}%
\pgfpathcurveto{\pgfqpoint{3.524553in}{0.958090in}}{\pgfqpoint{3.520163in}{0.947491in}}{\pgfqpoint{3.520163in}{0.936441in}}%
\pgfpathcurveto{\pgfqpoint{3.520163in}{0.925391in}}{\pgfqpoint{3.524553in}{0.914791in}}{\pgfqpoint{3.532367in}{0.906978in}}%
\pgfpathcurveto{\pgfqpoint{3.540181in}{0.899164in}}{\pgfqpoint{3.550780in}{0.894774in}}{\pgfqpoint{3.561830in}{0.894774in}}%
\pgfpathlineto{\pgfqpoint{3.561830in}{0.894774in}}%
\pgfpathclose%
\pgfusepath{stroke}%
\end{pgfscope}%
\begin{pgfscope}%
\pgfpathrectangle{\pgfqpoint{0.847223in}{0.554012in}}{\pgfqpoint{6.200000in}{4.530000in}}%
\pgfusepath{clip}%
\pgfsetbuttcap%
\pgfsetroundjoin%
\pgfsetlinewidth{1.003750pt}%
\definecolor{currentstroke}{rgb}{1.000000,0.000000,0.000000}%
\pgfsetstrokecolor{currentstroke}%
\pgfsetdash{}{0pt}%
\pgfpathmoveto{\pgfqpoint{3.567163in}{0.893459in}}%
\pgfpathcurveto{\pgfqpoint{3.578213in}{0.893459in}}{\pgfqpoint{3.588812in}{0.897849in}}{\pgfqpoint{3.596626in}{0.905663in}}%
\pgfpathcurveto{\pgfqpoint{3.604439in}{0.913477in}}{\pgfqpoint{3.608830in}{0.924076in}}{\pgfqpoint{3.608830in}{0.935126in}}%
\pgfpathcurveto{\pgfqpoint{3.608830in}{0.946176in}}{\pgfqpoint{3.604439in}{0.956775in}}{\pgfqpoint{3.596626in}{0.964589in}}%
\pgfpathcurveto{\pgfqpoint{3.588812in}{0.972402in}}{\pgfqpoint{3.578213in}{0.976792in}}{\pgfqpoint{3.567163in}{0.976792in}}%
\pgfpathcurveto{\pgfqpoint{3.556113in}{0.976792in}}{\pgfqpoint{3.545514in}{0.972402in}}{\pgfqpoint{3.537700in}{0.964589in}}%
\pgfpathcurveto{\pgfqpoint{3.529887in}{0.956775in}}{\pgfqpoint{3.525496in}{0.946176in}}{\pgfqpoint{3.525496in}{0.935126in}}%
\pgfpathcurveto{\pgfqpoint{3.525496in}{0.924076in}}{\pgfqpoint{3.529887in}{0.913477in}}{\pgfqpoint{3.537700in}{0.905663in}}%
\pgfpathcurveto{\pgfqpoint{3.545514in}{0.897849in}}{\pgfqpoint{3.556113in}{0.893459in}}{\pgfqpoint{3.567163in}{0.893459in}}%
\pgfpathlineto{\pgfqpoint{3.567163in}{0.893459in}}%
\pgfpathclose%
\pgfusepath{stroke}%
\end{pgfscope}%
\begin{pgfscope}%
\pgfpathrectangle{\pgfqpoint{0.847223in}{0.554012in}}{\pgfqpoint{6.200000in}{4.530000in}}%
\pgfusepath{clip}%
\pgfsetbuttcap%
\pgfsetroundjoin%
\pgfsetlinewidth{1.003750pt}%
\definecolor{currentstroke}{rgb}{1.000000,0.000000,0.000000}%
\pgfsetstrokecolor{currentstroke}%
\pgfsetdash{}{0pt}%
\pgfpathmoveto{\pgfqpoint{3.572496in}{0.892149in}}%
\pgfpathcurveto{\pgfqpoint{3.583546in}{0.892149in}}{\pgfqpoint{3.594145in}{0.896539in}}{\pgfqpoint{3.601959in}{0.904352in}}%
\pgfpathcurveto{\pgfqpoint{3.609773in}{0.912166in}}{\pgfqpoint{3.614163in}{0.922765in}}{\pgfqpoint{3.614163in}{0.933815in}}%
\pgfpathcurveto{\pgfqpoint{3.614163in}{0.944865in}}{\pgfqpoint{3.609773in}{0.955464in}}{\pgfqpoint{3.601959in}{0.963278in}}%
\pgfpathcurveto{\pgfqpoint{3.594145in}{0.971092in}}{\pgfqpoint{3.583546in}{0.975482in}}{\pgfqpoint{3.572496in}{0.975482in}}%
\pgfpathcurveto{\pgfqpoint{3.561446in}{0.975482in}}{\pgfqpoint{3.550847in}{0.971092in}}{\pgfqpoint{3.543033in}{0.963278in}}%
\pgfpathcurveto{\pgfqpoint{3.535220in}{0.955464in}}{\pgfqpoint{3.530830in}{0.944865in}}{\pgfqpoint{3.530830in}{0.933815in}}%
\pgfpathcurveto{\pgfqpoint{3.530830in}{0.922765in}}{\pgfqpoint{3.535220in}{0.912166in}}{\pgfqpoint{3.543033in}{0.904352in}}%
\pgfpathcurveto{\pgfqpoint{3.550847in}{0.896539in}}{\pgfqpoint{3.561446in}{0.892149in}}{\pgfqpoint{3.572496in}{0.892149in}}%
\pgfpathlineto{\pgfqpoint{3.572496in}{0.892149in}}%
\pgfpathclose%
\pgfusepath{stroke}%
\end{pgfscope}%
\begin{pgfscope}%
\pgfpathrectangle{\pgfqpoint{0.847223in}{0.554012in}}{\pgfqpoint{6.200000in}{4.530000in}}%
\pgfusepath{clip}%
\pgfsetbuttcap%
\pgfsetroundjoin%
\pgfsetlinewidth{1.003750pt}%
\definecolor{currentstroke}{rgb}{1.000000,0.000000,0.000000}%
\pgfsetstrokecolor{currentstroke}%
\pgfsetdash{}{0pt}%
\pgfpathmoveto{\pgfqpoint{3.577829in}{0.890842in}}%
\pgfpathcurveto{\pgfqpoint{3.588880in}{0.890842in}}{\pgfqpoint{3.599479in}{0.895233in}}{\pgfqpoint{3.607292in}{0.903046in}}%
\pgfpathcurveto{\pgfqpoint{3.615106in}{0.910860in}}{\pgfqpoint{3.619496in}{0.921459in}}{\pgfqpoint{3.619496in}{0.932509in}}%
\pgfpathcurveto{\pgfqpoint{3.619496in}{0.943559in}}{\pgfqpoint{3.615106in}{0.954158in}}{\pgfqpoint{3.607292in}{0.961972in}}%
\pgfpathcurveto{\pgfqpoint{3.599479in}{0.969785in}}{\pgfqpoint{3.588880in}{0.974176in}}{\pgfqpoint{3.577829in}{0.974176in}}%
\pgfpathcurveto{\pgfqpoint{3.566779in}{0.974176in}}{\pgfqpoint{3.556180in}{0.969785in}}{\pgfqpoint{3.548367in}{0.961972in}}%
\pgfpathcurveto{\pgfqpoint{3.540553in}{0.954158in}}{\pgfqpoint{3.536163in}{0.943559in}}{\pgfqpoint{3.536163in}{0.932509in}}%
\pgfpathcurveto{\pgfqpoint{3.536163in}{0.921459in}}{\pgfqpoint{3.540553in}{0.910860in}}{\pgfqpoint{3.548367in}{0.903046in}}%
\pgfpathcurveto{\pgfqpoint{3.556180in}{0.895233in}}{\pgfqpoint{3.566779in}{0.890842in}}{\pgfqpoint{3.577829in}{0.890842in}}%
\pgfpathlineto{\pgfqpoint{3.577829in}{0.890842in}}%
\pgfpathclose%
\pgfusepath{stroke}%
\end{pgfscope}%
\begin{pgfscope}%
\pgfpathrectangle{\pgfqpoint{0.847223in}{0.554012in}}{\pgfqpoint{6.200000in}{4.530000in}}%
\pgfusepath{clip}%
\pgfsetbuttcap%
\pgfsetroundjoin%
\pgfsetlinewidth{1.003750pt}%
\definecolor{currentstroke}{rgb}{1.000000,0.000000,0.000000}%
\pgfsetstrokecolor{currentstroke}%
\pgfsetdash{}{0pt}%
\pgfpathmoveto{\pgfqpoint{3.583163in}{0.889540in}}%
\pgfpathcurveto{\pgfqpoint{3.594213in}{0.889540in}}{\pgfqpoint{3.604812in}{0.893931in}}{\pgfqpoint{3.612625in}{0.901744in}}%
\pgfpathcurveto{\pgfqpoint{3.620439in}{0.909558in}}{\pgfqpoint{3.624829in}{0.920157in}}{\pgfqpoint{3.624829in}{0.931207in}}%
\pgfpathcurveto{\pgfqpoint{3.624829in}{0.942257in}}{\pgfqpoint{3.620439in}{0.952856in}}{\pgfqpoint{3.612625in}{0.960670in}}%
\pgfpathcurveto{\pgfqpoint{3.604812in}{0.968483in}}{\pgfqpoint{3.594213in}{0.972874in}}{\pgfqpoint{3.583163in}{0.972874in}}%
\pgfpathcurveto{\pgfqpoint{3.572113in}{0.972874in}}{\pgfqpoint{3.561514in}{0.968483in}}{\pgfqpoint{3.553700in}{0.960670in}}%
\pgfpathcurveto{\pgfqpoint{3.545886in}{0.952856in}}{\pgfqpoint{3.541496in}{0.942257in}}{\pgfqpoint{3.541496in}{0.931207in}}%
\pgfpathcurveto{\pgfqpoint{3.541496in}{0.920157in}}{\pgfqpoint{3.545886in}{0.909558in}}{\pgfqpoint{3.553700in}{0.901744in}}%
\pgfpathcurveto{\pgfqpoint{3.561514in}{0.893931in}}{\pgfqpoint{3.572113in}{0.889540in}}{\pgfqpoint{3.583163in}{0.889540in}}%
\pgfpathlineto{\pgfqpoint{3.583163in}{0.889540in}}%
\pgfpathclose%
\pgfusepath{stroke}%
\end{pgfscope}%
\begin{pgfscope}%
\pgfpathrectangle{\pgfqpoint{0.847223in}{0.554012in}}{\pgfqpoint{6.200000in}{4.530000in}}%
\pgfusepath{clip}%
\pgfsetbuttcap%
\pgfsetroundjoin%
\pgfsetlinewidth{1.003750pt}%
\definecolor{currentstroke}{rgb}{1.000000,0.000000,0.000000}%
\pgfsetstrokecolor{currentstroke}%
\pgfsetdash{}{0pt}%
\pgfpathmoveto{\pgfqpoint{3.588496in}{0.888243in}}%
\pgfpathcurveto{\pgfqpoint{3.599546in}{0.888243in}}{\pgfqpoint{3.610145in}{0.892633in}}{\pgfqpoint{3.617959in}{0.900446in}}%
\pgfpathcurveto{\pgfqpoint{3.625772in}{0.908260in}}{\pgfqpoint{3.630163in}{0.918859in}}{\pgfqpoint{3.630163in}{0.929909in}}%
\pgfpathcurveto{\pgfqpoint{3.630163in}{0.940959in}}{\pgfqpoint{3.625772in}{0.951558in}}{\pgfqpoint{3.617959in}{0.959372in}}%
\pgfpathcurveto{\pgfqpoint{3.610145in}{0.967186in}}{\pgfqpoint{3.599546in}{0.971576in}}{\pgfqpoint{3.588496in}{0.971576in}}%
\pgfpathcurveto{\pgfqpoint{3.577446in}{0.971576in}}{\pgfqpoint{3.566847in}{0.967186in}}{\pgfqpoint{3.559033in}{0.959372in}}%
\pgfpathcurveto{\pgfqpoint{3.551220in}{0.951558in}}{\pgfqpoint{3.546829in}{0.940959in}}{\pgfqpoint{3.546829in}{0.929909in}}%
\pgfpathcurveto{\pgfqpoint{3.546829in}{0.918859in}}{\pgfqpoint{3.551220in}{0.908260in}}{\pgfqpoint{3.559033in}{0.900446in}}%
\pgfpathcurveto{\pgfqpoint{3.566847in}{0.892633in}}{\pgfqpoint{3.577446in}{0.888243in}}{\pgfqpoint{3.588496in}{0.888243in}}%
\pgfpathlineto{\pgfqpoint{3.588496in}{0.888243in}}%
\pgfpathclose%
\pgfusepath{stroke}%
\end{pgfscope}%
\begin{pgfscope}%
\pgfpathrectangle{\pgfqpoint{0.847223in}{0.554012in}}{\pgfqpoint{6.200000in}{4.530000in}}%
\pgfusepath{clip}%
\pgfsetbuttcap%
\pgfsetroundjoin%
\pgfsetlinewidth{1.003750pt}%
\definecolor{currentstroke}{rgb}{1.000000,0.000000,0.000000}%
\pgfsetstrokecolor{currentstroke}%
\pgfsetdash{}{0pt}%
\pgfpathmoveto{\pgfqpoint{3.593829in}{0.886949in}}%
\pgfpathcurveto{\pgfqpoint{3.604879in}{0.886949in}}{\pgfqpoint{3.615478in}{0.891339in}}{\pgfqpoint{3.623292in}{0.899153in}}%
\pgfpathcurveto{\pgfqpoint{3.631106in}{0.906967in}}{\pgfqpoint{3.635496in}{0.917566in}}{\pgfqpoint{3.635496in}{0.928616in}}%
\pgfpathcurveto{\pgfqpoint{3.635496in}{0.939666in}}{\pgfqpoint{3.631106in}{0.950265in}}{\pgfqpoint{3.623292in}{0.958079in}}%
\pgfpathcurveto{\pgfqpoint{3.615478in}{0.965892in}}{\pgfqpoint{3.604879in}{0.970282in}}{\pgfqpoint{3.593829in}{0.970282in}}%
\pgfpathcurveto{\pgfqpoint{3.582779in}{0.970282in}}{\pgfqpoint{3.572180in}{0.965892in}}{\pgfqpoint{3.564366in}{0.958079in}}%
\pgfpathcurveto{\pgfqpoint{3.556553in}{0.950265in}}{\pgfqpoint{3.552162in}{0.939666in}}{\pgfqpoint{3.552162in}{0.928616in}}%
\pgfpathcurveto{\pgfqpoint{3.552162in}{0.917566in}}{\pgfqpoint{3.556553in}{0.906967in}}{\pgfqpoint{3.564366in}{0.899153in}}%
\pgfpathcurveto{\pgfqpoint{3.572180in}{0.891339in}}{\pgfqpoint{3.582779in}{0.886949in}}{\pgfqpoint{3.593829in}{0.886949in}}%
\pgfpathlineto{\pgfqpoint{3.593829in}{0.886949in}}%
\pgfpathclose%
\pgfusepath{stroke}%
\end{pgfscope}%
\begin{pgfscope}%
\pgfpathrectangle{\pgfqpoint{0.847223in}{0.554012in}}{\pgfqpoint{6.200000in}{4.530000in}}%
\pgfusepath{clip}%
\pgfsetbuttcap%
\pgfsetroundjoin%
\pgfsetlinewidth{1.003750pt}%
\definecolor{currentstroke}{rgb}{1.000000,0.000000,0.000000}%
\pgfsetstrokecolor{currentstroke}%
\pgfsetdash{}{0pt}%
\pgfpathmoveto{\pgfqpoint{3.599162in}{0.885660in}}%
\pgfpathcurveto{\pgfqpoint{3.610212in}{0.885660in}}{\pgfqpoint{3.620811in}{0.890050in}}{\pgfqpoint{3.628625in}{0.897864in}}%
\pgfpathcurveto{\pgfqpoint{3.636439in}{0.905677in}}{\pgfqpoint{3.640829in}{0.916276in}}{\pgfqpoint{3.640829in}{0.927326in}}%
\pgfpathcurveto{\pgfqpoint{3.640829in}{0.938377in}}{\pgfqpoint{3.636439in}{0.948976in}}{\pgfqpoint{3.628625in}{0.956789in}}%
\pgfpathcurveto{\pgfqpoint{3.620811in}{0.964603in}}{\pgfqpoint{3.610212in}{0.968993in}}{\pgfqpoint{3.599162in}{0.968993in}}%
\pgfpathcurveto{\pgfqpoint{3.588112in}{0.968993in}}{\pgfqpoint{3.577513in}{0.964603in}}{\pgfqpoint{3.569700in}{0.956789in}}%
\pgfpathcurveto{\pgfqpoint{3.561886in}{0.948976in}}{\pgfqpoint{3.557496in}{0.938377in}}{\pgfqpoint{3.557496in}{0.927326in}}%
\pgfpathcurveto{\pgfqpoint{3.557496in}{0.916276in}}{\pgfqpoint{3.561886in}{0.905677in}}{\pgfqpoint{3.569700in}{0.897864in}}%
\pgfpathcurveto{\pgfqpoint{3.577513in}{0.890050in}}{\pgfqpoint{3.588112in}{0.885660in}}{\pgfqpoint{3.599162in}{0.885660in}}%
\pgfpathlineto{\pgfqpoint{3.599162in}{0.885660in}}%
\pgfpathclose%
\pgfusepath{stroke}%
\end{pgfscope}%
\begin{pgfscope}%
\pgfpathrectangle{\pgfqpoint{0.847223in}{0.554012in}}{\pgfqpoint{6.200000in}{4.530000in}}%
\pgfusepath{clip}%
\pgfsetbuttcap%
\pgfsetroundjoin%
\pgfsetlinewidth{1.003750pt}%
\definecolor{currentstroke}{rgb}{1.000000,0.000000,0.000000}%
\pgfsetstrokecolor{currentstroke}%
\pgfsetdash{}{0pt}%
\pgfpathmoveto{\pgfqpoint{3.604496in}{0.884375in}}%
\pgfpathcurveto{\pgfqpoint{3.615546in}{0.884375in}}{\pgfqpoint{3.626145in}{0.888765in}}{\pgfqpoint{3.633958in}{0.896579in}}%
\pgfpathcurveto{\pgfqpoint{3.641772in}{0.904392in}}{\pgfqpoint{3.646162in}{0.914991in}}{\pgfqpoint{3.646162in}{0.926041in}}%
\pgfpathcurveto{\pgfqpoint{3.646162in}{0.937092in}}{\pgfqpoint{3.641772in}{0.947691in}}{\pgfqpoint{3.633958in}{0.955504in}}%
\pgfpathcurveto{\pgfqpoint{3.626145in}{0.963318in}}{\pgfqpoint{3.615546in}{0.967708in}}{\pgfqpoint{3.604496in}{0.967708in}}%
\pgfpathcurveto{\pgfqpoint{3.593445in}{0.967708in}}{\pgfqpoint{3.582846in}{0.963318in}}{\pgfqpoint{3.575033in}{0.955504in}}%
\pgfpathcurveto{\pgfqpoint{3.567219in}{0.947691in}}{\pgfqpoint{3.562829in}{0.937092in}}{\pgfqpoint{3.562829in}{0.926041in}}%
\pgfpathcurveto{\pgfqpoint{3.562829in}{0.914991in}}{\pgfqpoint{3.567219in}{0.904392in}}{\pgfqpoint{3.575033in}{0.896579in}}%
\pgfpathcurveto{\pgfqpoint{3.582846in}{0.888765in}}{\pgfqpoint{3.593445in}{0.884375in}}{\pgfqpoint{3.604496in}{0.884375in}}%
\pgfpathlineto{\pgfqpoint{3.604496in}{0.884375in}}%
\pgfpathclose%
\pgfusepath{stroke}%
\end{pgfscope}%
\begin{pgfscope}%
\pgfpathrectangle{\pgfqpoint{0.847223in}{0.554012in}}{\pgfqpoint{6.200000in}{4.530000in}}%
\pgfusepath{clip}%
\pgfsetbuttcap%
\pgfsetroundjoin%
\pgfsetlinewidth{1.003750pt}%
\definecolor{currentstroke}{rgb}{1.000000,0.000000,0.000000}%
\pgfsetstrokecolor{currentstroke}%
\pgfsetdash{}{0pt}%
\pgfpathmoveto{\pgfqpoint{3.609829in}{0.883094in}}%
\pgfpathcurveto{\pgfqpoint{3.620879in}{0.883094in}}{\pgfqpoint{3.631478in}{0.887484in}}{\pgfqpoint{3.639292in}{0.895298in}}%
\pgfpathcurveto{\pgfqpoint{3.647105in}{0.903111in}}{\pgfqpoint{3.651495in}{0.913710in}}{\pgfqpoint{3.651495in}{0.924760in}}%
\pgfpathcurveto{\pgfqpoint{3.651495in}{0.935811in}}{\pgfqpoint{3.647105in}{0.946410in}}{\pgfqpoint{3.639292in}{0.954223in}}%
\pgfpathcurveto{\pgfqpoint{3.631478in}{0.962037in}}{\pgfqpoint{3.620879in}{0.966427in}}{\pgfqpoint{3.609829in}{0.966427in}}%
\pgfpathcurveto{\pgfqpoint{3.598779in}{0.966427in}}{\pgfqpoint{3.588180in}{0.962037in}}{\pgfqpoint{3.580366in}{0.954223in}}%
\pgfpathcurveto{\pgfqpoint{3.572552in}{0.946410in}}{\pgfqpoint{3.568162in}{0.935811in}}{\pgfqpoint{3.568162in}{0.924760in}}%
\pgfpathcurveto{\pgfqpoint{3.568162in}{0.913710in}}{\pgfqpoint{3.572552in}{0.903111in}}{\pgfqpoint{3.580366in}{0.895298in}}%
\pgfpathcurveto{\pgfqpoint{3.588180in}{0.887484in}}{\pgfqpoint{3.598779in}{0.883094in}}{\pgfqpoint{3.609829in}{0.883094in}}%
\pgfpathlineto{\pgfqpoint{3.609829in}{0.883094in}}%
\pgfpathclose%
\pgfusepath{stroke}%
\end{pgfscope}%
\begin{pgfscope}%
\pgfpathrectangle{\pgfqpoint{0.847223in}{0.554012in}}{\pgfqpoint{6.200000in}{4.530000in}}%
\pgfusepath{clip}%
\pgfsetbuttcap%
\pgfsetroundjoin%
\pgfsetlinewidth{1.003750pt}%
\definecolor{currentstroke}{rgb}{1.000000,0.000000,0.000000}%
\pgfsetstrokecolor{currentstroke}%
\pgfsetdash{}{0pt}%
\pgfpathmoveto{\pgfqpoint{3.615162in}{0.881817in}}%
\pgfpathcurveto{\pgfqpoint{3.626212in}{0.881817in}}{\pgfqpoint{3.636811in}{0.886207in}}{\pgfqpoint{3.644625in}{0.894021in}}%
\pgfpathcurveto{\pgfqpoint{3.652438in}{0.901835in}}{\pgfqpoint{3.656829in}{0.912434in}}{\pgfqpoint{3.656829in}{0.923484in}}%
\pgfpathcurveto{\pgfqpoint{3.656829in}{0.934534in}}{\pgfqpoint{3.652438in}{0.945133in}}{\pgfqpoint{3.644625in}{0.952946in}}%
\pgfpathcurveto{\pgfqpoint{3.636811in}{0.960760in}}{\pgfqpoint{3.626212in}{0.965150in}}{\pgfqpoint{3.615162in}{0.965150in}}%
\pgfpathcurveto{\pgfqpoint{3.604112in}{0.965150in}}{\pgfqpoint{3.593513in}{0.960760in}}{\pgfqpoint{3.585699in}{0.952946in}}%
\pgfpathcurveto{\pgfqpoint{3.577886in}{0.945133in}}{\pgfqpoint{3.573495in}{0.934534in}}{\pgfqpoint{3.573495in}{0.923484in}}%
\pgfpathcurveto{\pgfqpoint{3.573495in}{0.912434in}}{\pgfqpoint{3.577886in}{0.901835in}}{\pgfqpoint{3.585699in}{0.894021in}}%
\pgfpathcurveto{\pgfqpoint{3.593513in}{0.886207in}}{\pgfqpoint{3.604112in}{0.881817in}}{\pgfqpoint{3.615162in}{0.881817in}}%
\pgfpathlineto{\pgfqpoint{3.615162in}{0.881817in}}%
\pgfpathclose%
\pgfusepath{stroke}%
\end{pgfscope}%
\begin{pgfscope}%
\pgfpathrectangle{\pgfqpoint{0.847223in}{0.554012in}}{\pgfqpoint{6.200000in}{4.530000in}}%
\pgfusepath{clip}%
\pgfsetbuttcap%
\pgfsetroundjoin%
\pgfsetlinewidth{1.003750pt}%
\definecolor{currentstroke}{rgb}{1.000000,0.000000,0.000000}%
\pgfsetstrokecolor{currentstroke}%
\pgfsetdash{}{0pt}%
\pgfpathmoveto{\pgfqpoint{3.620495in}{0.880544in}}%
\pgfpathcurveto{\pgfqpoint{3.631545in}{0.880544in}}{\pgfqpoint{3.642144in}{0.884935in}}{\pgfqpoint{3.649958in}{0.892748in}}%
\pgfpathcurveto{\pgfqpoint{3.657772in}{0.900562in}}{\pgfqpoint{3.662162in}{0.911161in}}{\pgfqpoint{3.662162in}{0.922211in}}%
\pgfpathcurveto{\pgfqpoint{3.662162in}{0.933261in}}{\pgfqpoint{3.657772in}{0.943860in}}{\pgfqpoint{3.649958in}{0.951674in}}%
\pgfpathcurveto{\pgfqpoint{3.642144in}{0.959487in}}{\pgfqpoint{3.631545in}{0.963878in}}{\pgfqpoint{3.620495in}{0.963878in}}%
\pgfpathcurveto{\pgfqpoint{3.609445in}{0.963878in}}{\pgfqpoint{3.598846in}{0.959487in}}{\pgfqpoint{3.591032in}{0.951674in}}%
\pgfpathcurveto{\pgfqpoint{3.583219in}{0.943860in}}{\pgfqpoint{3.578829in}{0.933261in}}{\pgfqpoint{3.578829in}{0.922211in}}%
\pgfpathcurveto{\pgfqpoint{3.578829in}{0.911161in}}{\pgfqpoint{3.583219in}{0.900562in}}{\pgfqpoint{3.591032in}{0.892748in}}%
\pgfpathcurveto{\pgfqpoint{3.598846in}{0.884935in}}{\pgfqpoint{3.609445in}{0.880544in}}{\pgfqpoint{3.620495in}{0.880544in}}%
\pgfpathlineto{\pgfqpoint{3.620495in}{0.880544in}}%
\pgfpathclose%
\pgfusepath{stroke}%
\end{pgfscope}%
\begin{pgfscope}%
\pgfpathrectangle{\pgfqpoint{0.847223in}{0.554012in}}{\pgfqpoint{6.200000in}{4.530000in}}%
\pgfusepath{clip}%
\pgfsetbuttcap%
\pgfsetroundjoin%
\pgfsetlinewidth{1.003750pt}%
\definecolor{currentstroke}{rgb}{1.000000,0.000000,0.000000}%
\pgfsetstrokecolor{currentstroke}%
\pgfsetdash{}{0pt}%
\pgfpathmoveto{\pgfqpoint{3.625828in}{0.879276in}}%
\pgfpathcurveto{\pgfqpoint{3.636879in}{0.879276in}}{\pgfqpoint{3.647478in}{0.883666in}}{\pgfqpoint{3.655291in}{0.891480in}}%
\pgfpathcurveto{\pgfqpoint{3.663105in}{0.899293in}}{\pgfqpoint{3.667495in}{0.909892in}}{\pgfqpoint{3.667495in}{0.920943in}}%
\pgfpathcurveto{\pgfqpoint{3.667495in}{0.931993in}}{\pgfqpoint{3.663105in}{0.942592in}}{\pgfqpoint{3.655291in}{0.950405in}}%
\pgfpathcurveto{\pgfqpoint{3.647478in}{0.958219in}}{\pgfqpoint{3.636879in}{0.962609in}}{\pgfqpoint{3.625828in}{0.962609in}}%
\pgfpathcurveto{\pgfqpoint{3.614778in}{0.962609in}}{\pgfqpoint{3.604179in}{0.958219in}}{\pgfqpoint{3.596366in}{0.950405in}}%
\pgfpathcurveto{\pgfqpoint{3.588552in}{0.942592in}}{\pgfqpoint{3.584162in}{0.931993in}}{\pgfqpoint{3.584162in}{0.920943in}}%
\pgfpathcurveto{\pgfqpoint{3.584162in}{0.909892in}}{\pgfqpoint{3.588552in}{0.899293in}}{\pgfqpoint{3.596366in}{0.891480in}}%
\pgfpathcurveto{\pgfqpoint{3.604179in}{0.883666in}}{\pgfqpoint{3.614778in}{0.879276in}}{\pgfqpoint{3.625828in}{0.879276in}}%
\pgfpathlineto{\pgfqpoint{3.625828in}{0.879276in}}%
\pgfpathclose%
\pgfusepath{stroke}%
\end{pgfscope}%
\begin{pgfscope}%
\pgfpathrectangle{\pgfqpoint{0.847223in}{0.554012in}}{\pgfqpoint{6.200000in}{4.530000in}}%
\pgfusepath{clip}%
\pgfsetbuttcap%
\pgfsetroundjoin%
\pgfsetlinewidth{1.003750pt}%
\definecolor{currentstroke}{rgb}{1.000000,0.000000,0.000000}%
\pgfsetstrokecolor{currentstroke}%
\pgfsetdash{}{0pt}%
\pgfpathmoveto{\pgfqpoint{3.631162in}{0.878011in}}%
\pgfpathcurveto{\pgfqpoint{3.642212in}{0.878011in}}{\pgfqpoint{3.652811in}{0.882402in}}{\pgfqpoint{3.660624in}{0.890215in}}%
\pgfpathcurveto{\pgfqpoint{3.668438in}{0.898029in}}{\pgfqpoint{3.672828in}{0.908628in}}{\pgfqpoint{3.672828in}{0.919678in}}%
\pgfpathcurveto{\pgfqpoint{3.672828in}{0.930728in}}{\pgfqpoint{3.668438in}{0.941327in}}{\pgfqpoint{3.660624in}{0.949141in}}%
\pgfpathcurveto{\pgfqpoint{3.652811in}{0.956955in}}{\pgfqpoint{3.642212in}{0.961345in}}{\pgfqpoint{3.631162in}{0.961345in}}%
\pgfpathcurveto{\pgfqpoint{3.620112in}{0.961345in}}{\pgfqpoint{3.609512in}{0.956955in}}{\pgfqpoint{3.601699in}{0.949141in}}%
\pgfpathcurveto{\pgfqpoint{3.593885in}{0.941327in}}{\pgfqpoint{3.589495in}{0.930728in}}{\pgfqpoint{3.589495in}{0.919678in}}%
\pgfpathcurveto{\pgfqpoint{3.589495in}{0.908628in}}{\pgfqpoint{3.593885in}{0.898029in}}{\pgfqpoint{3.601699in}{0.890215in}}%
\pgfpathcurveto{\pgfqpoint{3.609512in}{0.882402in}}{\pgfqpoint{3.620112in}{0.878011in}}{\pgfqpoint{3.631162in}{0.878011in}}%
\pgfpathlineto{\pgfqpoint{3.631162in}{0.878011in}}%
\pgfpathclose%
\pgfusepath{stroke}%
\end{pgfscope}%
\begin{pgfscope}%
\pgfpathrectangle{\pgfqpoint{0.847223in}{0.554012in}}{\pgfqpoint{6.200000in}{4.530000in}}%
\pgfusepath{clip}%
\pgfsetbuttcap%
\pgfsetroundjoin%
\pgfsetlinewidth{1.003750pt}%
\definecolor{currentstroke}{rgb}{1.000000,0.000000,0.000000}%
\pgfsetstrokecolor{currentstroke}%
\pgfsetdash{}{0pt}%
\pgfpathmoveto{\pgfqpoint{3.636495in}{0.876751in}}%
\pgfpathcurveto{\pgfqpoint{3.647545in}{0.876751in}}{\pgfqpoint{3.658144in}{0.881141in}}{\pgfqpoint{3.665958in}{0.888955in}}%
\pgfpathcurveto{\pgfqpoint{3.673771in}{0.896769in}}{\pgfqpoint{3.678162in}{0.907368in}}{\pgfqpoint{3.678162in}{0.918418in}}%
\pgfpathcurveto{\pgfqpoint{3.678162in}{0.929468in}}{\pgfqpoint{3.673771in}{0.940067in}}{\pgfqpoint{3.665958in}{0.947881in}}%
\pgfpathcurveto{\pgfqpoint{3.658144in}{0.955694in}}{\pgfqpoint{3.647545in}{0.960084in}}{\pgfqpoint{3.636495in}{0.960084in}}%
\pgfpathcurveto{\pgfqpoint{3.625445in}{0.960084in}}{\pgfqpoint{3.614846in}{0.955694in}}{\pgfqpoint{3.607032in}{0.947881in}}%
\pgfpathcurveto{\pgfqpoint{3.599218in}{0.940067in}}{\pgfqpoint{3.594828in}{0.929468in}}{\pgfqpoint{3.594828in}{0.918418in}}%
\pgfpathcurveto{\pgfqpoint{3.594828in}{0.907368in}}{\pgfqpoint{3.599218in}{0.896769in}}{\pgfqpoint{3.607032in}{0.888955in}}%
\pgfpathcurveto{\pgfqpoint{3.614846in}{0.881141in}}{\pgfqpoint{3.625445in}{0.876751in}}{\pgfqpoint{3.636495in}{0.876751in}}%
\pgfpathlineto{\pgfqpoint{3.636495in}{0.876751in}}%
\pgfpathclose%
\pgfusepath{stroke}%
\end{pgfscope}%
\begin{pgfscope}%
\pgfpathrectangle{\pgfqpoint{0.847223in}{0.554012in}}{\pgfqpoint{6.200000in}{4.530000in}}%
\pgfusepath{clip}%
\pgfsetbuttcap%
\pgfsetroundjoin%
\pgfsetlinewidth{1.003750pt}%
\definecolor{currentstroke}{rgb}{1.000000,0.000000,0.000000}%
\pgfsetstrokecolor{currentstroke}%
\pgfsetdash{}{0pt}%
\pgfpathmoveto{\pgfqpoint{3.641828in}{0.875495in}}%
\pgfpathcurveto{\pgfqpoint{3.652878in}{0.875495in}}{\pgfqpoint{3.663477in}{0.879885in}}{\pgfqpoint{3.671291in}{0.887699in}}%
\pgfpathcurveto{\pgfqpoint{3.679104in}{0.895512in}}{\pgfqpoint{3.683495in}{0.906111in}}{\pgfqpoint{3.683495in}{0.917161in}}%
\pgfpathcurveto{\pgfqpoint{3.683495in}{0.928212in}}{\pgfqpoint{3.679104in}{0.938811in}}{\pgfqpoint{3.671291in}{0.946624in}}%
\pgfpathcurveto{\pgfqpoint{3.663477in}{0.954438in}}{\pgfqpoint{3.652878in}{0.958828in}}{\pgfqpoint{3.641828in}{0.958828in}}%
\pgfpathcurveto{\pgfqpoint{3.630778in}{0.958828in}}{\pgfqpoint{3.620179in}{0.954438in}}{\pgfqpoint{3.612365in}{0.946624in}}%
\pgfpathcurveto{\pgfqpoint{3.604552in}{0.938811in}}{\pgfqpoint{3.600161in}{0.928212in}}{\pgfqpoint{3.600161in}{0.917161in}}%
\pgfpathcurveto{\pgfqpoint{3.600161in}{0.906111in}}{\pgfqpoint{3.604552in}{0.895512in}}{\pgfqpoint{3.612365in}{0.887699in}}%
\pgfpathcurveto{\pgfqpoint{3.620179in}{0.879885in}}{\pgfqpoint{3.630778in}{0.875495in}}{\pgfqpoint{3.641828in}{0.875495in}}%
\pgfpathlineto{\pgfqpoint{3.641828in}{0.875495in}}%
\pgfpathclose%
\pgfusepath{stroke}%
\end{pgfscope}%
\begin{pgfscope}%
\pgfpathrectangle{\pgfqpoint{0.847223in}{0.554012in}}{\pgfqpoint{6.200000in}{4.530000in}}%
\pgfusepath{clip}%
\pgfsetbuttcap%
\pgfsetroundjoin%
\pgfsetlinewidth{1.003750pt}%
\definecolor{currentstroke}{rgb}{1.000000,0.000000,0.000000}%
\pgfsetstrokecolor{currentstroke}%
\pgfsetdash{}{0pt}%
\pgfpathmoveto{\pgfqpoint{3.647161in}{0.874242in}}%
\pgfpathcurveto{\pgfqpoint{3.658211in}{0.874242in}}{\pgfqpoint{3.668810in}{0.878633in}}{\pgfqpoint{3.676624in}{0.886446in}}%
\pgfpathcurveto{\pgfqpoint{3.684438in}{0.894260in}}{\pgfqpoint{3.688828in}{0.904859in}}{\pgfqpoint{3.688828in}{0.915909in}}%
\pgfpathcurveto{\pgfqpoint{3.688828in}{0.926959in}}{\pgfqpoint{3.684438in}{0.937558in}}{\pgfqpoint{3.676624in}{0.945372in}}%
\pgfpathcurveto{\pgfqpoint{3.668810in}{0.953186in}}{\pgfqpoint{3.658211in}{0.957576in}}{\pgfqpoint{3.647161in}{0.957576in}}%
\pgfpathcurveto{\pgfqpoint{3.636111in}{0.957576in}}{\pgfqpoint{3.625512in}{0.953186in}}{\pgfqpoint{3.617698in}{0.945372in}}%
\pgfpathcurveto{\pgfqpoint{3.609885in}{0.937558in}}{\pgfqpoint{3.605495in}{0.926959in}}{\pgfqpoint{3.605495in}{0.915909in}}%
\pgfpathcurveto{\pgfqpoint{3.605495in}{0.904859in}}{\pgfqpoint{3.609885in}{0.894260in}}{\pgfqpoint{3.617698in}{0.886446in}}%
\pgfpathcurveto{\pgfqpoint{3.625512in}{0.878633in}}{\pgfqpoint{3.636111in}{0.874242in}}{\pgfqpoint{3.647161in}{0.874242in}}%
\pgfpathlineto{\pgfqpoint{3.647161in}{0.874242in}}%
\pgfpathclose%
\pgfusepath{stroke}%
\end{pgfscope}%
\begin{pgfscope}%
\pgfpathrectangle{\pgfqpoint{0.847223in}{0.554012in}}{\pgfqpoint{6.200000in}{4.530000in}}%
\pgfusepath{clip}%
\pgfsetbuttcap%
\pgfsetroundjoin%
\pgfsetlinewidth{1.003750pt}%
\definecolor{currentstroke}{rgb}{1.000000,0.000000,0.000000}%
\pgfsetstrokecolor{currentstroke}%
\pgfsetdash{}{0pt}%
\pgfpathmoveto{\pgfqpoint{3.652494in}{0.872994in}}%
\pgfpathcurveto{\pgfqpoint{3.663545in}{0.872994in}}{\pgfqpoint{3.674144in}{0.877384in}}{\pgfqpoint{3.681957in}{0.885198in}}%
\pgfpathcurveto{\pgfqpoint{3.689771in}{0.893012in}}{\pgfqpoint{3.694161in}{0.903611in}}{\pgfqpoint{3.694161in}{0.914661in}}%
\pgfpathcurveto{\pgfqpoint{3.694161in}{0.925711in}}{\pgfqpoint{3.689771in}{0.936310in}}{\pgfqpoint{3.681957in}{0.944124in}}%
\pgfpathcurveto{\pgfqpoint{3.674144in}{0.951937in}}{\pgfqpoint{3.663545in}{0.956328in}}{\pgfqpoint{3.652494in}{0.956328in}}%
\pgfpathcurveto{\pgfqpoint{3.641444in}{0.956328in}}{\pgfqpoint{3.630845in}{0.951937in}}{\pgfqpoint{3.623032in}{0.944124in}}%
\pgfpathcurveto{\pgfqpoint{3.615218in}{0.936310in}}{\pgfqpoint{3.610828in}{0.925711in}}{\pgfqpoint{3.610828in}{0.914661in}}%
\pgfpathcurveto{\pgfqpoint{3.610828in}{0.903611in}}{\pgfqpoint{3.615218in}{0.893012in}}{\pgfqpoint{3.623032in}{0.885198in}}%
\pgfpathcurveto{\pgfqpoint{3.630845in}{0.877384in}}{\pgfqpoint{3.641444in}{0.872994in}}{\pgfqpoint{3.652494in}{0.872994in}}%
\pgfpathlineto{\pgfqpoint{3.652494in}{0.872994in}}%
\pgfpathclose%
\pgfusepath{stroke}%
\end{pgfscope}%
\begin{pgfscope}%
\pgfpathrectangle{\pgfqpoint{0.847223in}{0.554012in}}{\pgfqpoint{6.200000in}{4.530000in}}%
\pgfusepath{clip}%
\pgfsetbuttcap%
\pgfsetroundjoin%
\pgfsetlinewidth{1.003750pt}%
\definecolor{currentstroke}{rgb}{1.000000,0.000000,0.000000}%
\pgfsetstrokecolor{currentstroke}%
\pgfsetdash{}{0pt}%
\pgfpathmoveto{\pgfqpoint{3.657828in}{0.871750in}}%
\pgfpathcurveto{\pgfqpoint{3.668878in}{0.871750in}}{\pgfqpoint{3.679477in}{0.876140in}}{\pgfqpoint{3.687290in}{0.883954in}}%
\pgfpathcurveto{\pgfqpoint{3.695104in}{0.891767in}}{\pgfqpoint{3.699494in}{0.902366in}}{\pgfqpoint{3.699494in}{0.913417in}}%
\pgfpathcurveto{\pgfqpoint{3.699494in}{0.924467in}}{\pgfqpoint{3.695104in}{0.935066in}}{\pgfqpoint{3.687290in}{0.942879in}}%
\pgfpathcurveto{\pgfqpoint{3.679477in}{0.950693in}}{\pgfqpoint{3.668878in}{0.955083in}}{\pgfqpoint{3.657828in}{0.955083in}}%
\pgfpathcurveto{\pgfqpoint{3.646778in}{0.955083in}}{\pgfqpoint{3.636179in}{0.950693in}}{\pgfqpoint{3.628365in}{0.942879in}}%
\pgfpathcurveto{\pgfqpoint{3.620551in}{0.935066in}}{\pgfqpoint{3.616161in}{0.924467in}}{\pgfqpoint{3.616161in}{0.913417in}}%
\pgfpathcurveto{\pgfqpoint{3.616161in}{0.902366in}}{\pgfqpoint{3.620551in}{0.891767in}}{\pgfqpoint{3.628365in}{0.883954in}}%
\pgfpathcurveto{\pgfqpoint{3.636179in}{0.876140in}}{\pgfqpoint{3.646778in}{0.871750in}}{\pgfqpoint{3.657828in}{0.871750in}}%
\pgfpathlineto{\pgfqpoint{3.657828in}{0.871750in}}%
\pgfpathclose%
\pgfusepath{stroke}%
\end{pgfscope}%
\begin{pgfscope}%
\pgfpathrectangle{\pgfqpoint{0.847223in}{0.554012in}}{\pgfqpoint{6.200000in}{4.530000in}}%
\pgfusepath{clip}%
\pgfsetbuttcap%
\pgfsetroundjoin%
\pgfsetlinewidth{1.003750pt}%
\definecolor{currentstroke}{rgb}{1.000000,0.000000,0.000000}%
\pgfsetstrokecolor{currentstroke}%
\pgfsetdash{}{0pt}%
\pgfpathmoveto{\pgfqpoint{3.663161in}{0.870510in}}%
\pgfpathcurveto{\pgfqpoint{3.674211in}{0.870510in}}{\pgfqpoint{3.684810in}{0.874900in}}{\pgfqpoint{3.692624in}{0.882714in}}%
\pgfpathcurveto{\pgfqpoint{3.700437in}{0.890527in}}{\pgfqpoint{3.704828in}{0.901126in}}{\pgfqpoint{3.704828in}{0.912176in}}%
\pgfpathcurveto{\pgfqpoint{3.704828in}{0.923226in}}{\pgfqpoint{3.700437in}{0.933825in}}{\pgfqpoint{3.692624in}{0.941639in}}%
\pgfpathcurveto{\pgfqpoint{3.684810in}{0.949453in}}{\pgfqpoint{3.674211in}{0.953843in}}{\pgfqpoint{3.663161in}{0.953843in}}%
\pgfpathcurveto{\pgfqpoint{3.652111in}{0.953843in}}{\pgfqpoint{3.641512in}{0.949453in}}{\pgfqpoint{3.633698in}{0.941639in}}%
\pgfpathcurveto{\pgfqpoint{3.625885in}{0.933825in}}{\pgfqpoint{3.621494in}{0.923226in}}{\pgfqpoint{3.621494in}{0.912176in}}%
\pgfpathcurveto{\pgfqpoint{3.621494in}{0.901126in}}{\pgfqpoint{3.625885in}{0.890527in}}{\pgfqpoint{3.633698in}{0.882714in}}%
\pgfpathcurveto{\pgfqpoint{3.641512in}{0.874900in}}{\pgfqpoint{3.652111in}{0.870510in}}{\pgfqpoint{3.663161in}{0.870510in}}%
\pgfpathlineto{\pgfqpoint{3.663161in}{0.870510in}}%
\pgfpathclose%
\pgfusepath{stroke}%
\end{pgfscope}%
\begin{pgfscope}%
\pgfpathrectangle{\pgfqpoint{0.847223in}{0.554012in}}{\pgfqpoint{6.200000in}{4.530000in}}%
\pgfusepath{clip}%
\pgfsetbuttcap%
\pgfsetroundjoin%
\pgfsetlinewidth{1.003750pt}%
\definecolor{currentstroke}{rgb}{1.000000,0.000000,0.000000}%
\pgfsetstrokecolor{currentstroke}%
\pgfsetdash{}{0pt}%
\pgfpathmoveto{\pgfqpoint{3.668494in}{0.869273in}}%
\pgfpathcurveto{\pgfqpoint{3.679544in}{0.869273in}}{\pgfqpoint{3.690143in}{0.873664in}}{\pgfqpoint{3.697957in}{0.881477in}}%
\pgfpathcurveto{\pgfqpoint{3.705771in}{0.889291in}}{\pgfqpoint{3.710161in}{0.899890in}}{\pgfqpoint{3.710161in}{0.910940in}}%
\pgfpathcurveto{\pgfqpoint{3.710161in}{0.921990in}}{\pgfqpoint{3.705771in}{0.932589in}}{\pgfqpoint{3.697957in}{0.940403in}}%
\pgfpathcurveto{\pgfqpoint{3.690143in}{0.948216in}}{\pgfqpoint{3.679544in}{0.952607in}}{\pgfqpoint{3.668494in}{0.952607in}}%
\pgfpathcurveto{\pgfqpoint{3.657444in}{0.952607in}}{\pgfqpoint{3.646845in}{0.948216in}}{\pgfqpoint{3.639031in}{0.940403in}}%
\pgfpathcurveto{\pgfqpoint{3.631218in}{0.932589in}}{\pgfqpoint{3.626827in}{0.921990in}}{\pgfqpoint{3.626827in}{0.910940in}}%
\pgfpathcurveto{\pgfqpoint{3.626827in}{0.899890in}}{\pgfqpoint{3.631218in}{0.889291in}}{\pgfqpoint{3.639031in}{0.881477in}}%
\pgfpathcurveto{\pgfqpoint{3.646845in}{0.873664in}}{\pgfqpoint{3.657444in}{0.869273in}}{\pgfqpoint{3.668494in}{0.869273in}}%
\pgfpathlineto{\pgfqpoint{3.668494in}{0.869273in}}%
\pgfpathclose%
\pgfusepath{stroke}%
\end{pgfscope}%
\begin{pgfscope}%
\pgfpathrectangle{\pgfqpoint{0.847223in}{0.554012in}}{\pgfqpoint{6.200000in}{4.530000in}}%
\pgfusepath{clip}%
\pgfsetbuttcap%
\pgfsetroundjoin%
\pgfsetlinewidth{1.003750pt}%
\definecolor{currentstroke}{rgb}{1.000000,0.000000,0.000000}%
\pgfsetstrokecolor{currentstroke}%
\pgfsetdash{}{0pt}%
\pgfpathmoveto{\pgfqpoint{3.673827in}{0.868041in}}%
\pgfpathcurveto{\pgfqpoint{3.684877in}{0.868041in}}{\pgfqpoint{3.695477in}{0.872431in}}{\pgfqpoint{3.703290in}{0.880245in}}%
\pgfpathcurveto{\pgfqpoint{3.711104in}{0.888058in}}{\pgfqpoint{3.715494in}{0.898657in}}{\pgfqpoint{3.715494in}{0.909708in}}%
\pgfpathcurveto{\pgfqpoint{3.715494in}{0.920758in}}{\pgfqpoint{3.711104in}{0.931357in}}{\pgfqpoint{3.703290in}{0.939170in}}%
\pgfpathcurveto{\pgfqpoint{3.695477in}{0.946984in}}{\pgfqpoint{3.684877in}{0.951374in}}{\pgfqpoint{3.673827in}{0.951374in}}%
\pgfpathcurveto{\pgfqpoint{3.662777in}{0.951374in}}{\pgfqpoint{3.652178in}{0.946984in}}{\pgfqpoint{3.644365in}{0.939170in}}%
\pgfpathcurveto{\pgfqpoint{3.636551in}{0.931357in}}{\pgfqpoint{3.632161in}{0.920758in}}{\pgfqpoint{3.632161in}{0.909708in}}%
\pgfpathcurveto{\pgfqpoint{3.632161in}{0.898657in}}{\pgfqpoint{3.636551in}{0.888058in}}{\pgfqpoint{3.644365in}{0.880245in}}%
\pgfpathcurveto{\pgfqpoint{3.652178in}{0.872431in}}{\pgfqpoint{3.662777in}{0.868041in}}{\pgfqpoint{3.673827in}{0.868041in}}%
\pgfpathlineto{\pgfqpoint{3.673827in}{0.868041in}}%
\pgfpathclose%
\pgfusepath{stroke}%
\end{pgfscope}%
\begin{pgfscope}%
\pgfpathrectangle{\pgfqpoint{0.847223in}{0.554012in}}{\pgfqpoint{6.200000in}{4.530000in}}%
\pgfusepath{clip}%
\pgfsetbuttcap%
\pgfsetroundjoin%
\pgfsetlinewidth{1.003750pt}%
\definecolor{currentstroke}{rgb}{1.000000,0.000000,0.000000}%
\pgfsetstrokecolor{currentstroke}%
\pgfsetdash{}{0pt}%
\pgfpathmoveto{\pgfqpoint{3.679161in}{0.866812in}}%
\pgfpathcurveto{\pgfqpoint{3.690211in}{0.866812in}}{\pgfqpoint{3.700810in}{0.871203in}}{\pgfqpoint{3.708623in}{0.879016in}}%
\pgfpathcurveto{\pgfqpoint{3.716437in}{0.886830in}}{\pgfqpoint{3.720827in}{0.897429in}}{\pgfqpoint{3.720827in}{0.908479in}}%
\pgfpathcurveto{\pgfqpoint{3.720827in}{0.919529in}}{\pgfqpoint{3.716437in}{0.930128in}}{\pgfqpoint{3.708623in}{0.937942in}}%
\pgfpathcurveto{\pgfqpoint{3.700810in}{0.945755in}}{\pgfqpoint{3.690211in}{0.950146in}}{\pgfqpoint{3.679161in}{0.950146in}}%
\pgfpathcurveto{\pgfqpoint{3.668110in}{0.950146in}}{\pgfqpoint{3.657511in}{0.945755in}}{\pgfqpoint{3.649698in}{0.937942in}}%
\pgfpathcurveto{\pgfqpoint{3.641884in}{0.930128in}}{\pgfqpoint{3.637494in}{0.919529in}}{\pgfqpoint{3.637494in}{0.908479in}}%
\pgfpathcurveto{\pgfqpoint{3.637494in}{0.897429in}}{\pgfqpoint{3.641884in}{0.886830in}}{\pgfqpoint{3.649698in}{0.879016in}}%
\pgfpathcurveto{\pgfqpoint{3.657511in}{0.871203in}}{\pgfqpoint{3.668110in}{0.866812in}}{\pgfqpoint{3.679161in}{0.866812in}}%
\pgfpathlineto{\pgfqpoint{3.679161in}{0.866812in}}%
\pgfpathclose%
\pgfusepath{stroke}%
\end{pgfscope}%
\begin{pgfscope}%
\pgfpathrectangle{\pgfqpoint{0.847223in}{0.554012in}}{\pgfqpoint{6.200000in}{4.530000in}}%
\pgfusepath{clip}%
\pgfsetbuttcap%
\pgfsetroundjoin%
\pgfsetlinewidth{1.003750pt}%
\definecolor{currentstroke}{rgb}{1.000000,0.000000,0.000000}%
\pgfsetstrokecolor{currentstroke}%
\pgfsetdash{}{0pt}%
\pgfpathmoveto{\pgfqpoint{3.684494in}{0.865588in}}%
\pgfpathcurveto{\pgfqpoint{3.695544in}{0.865588in}}{\pgfqpoint{3.706143in}{0.869978in}}{\pgfqpoint{3.713957in}{0.877792in}}%
\pgfpathcurveto{\pgfqpoint{3.721770in}{0.885605in}}{\pgfqpoint{3.726160in}{0.896204in}}{\pgfqpoint{3.726160in}{0.907254in}}%
\pgfpathcurveto{\pgfqpoint{3.726160in}{0.918305in}}{\pgfqpoint{3.721770in}{0.928904in}}{\pgfqpoint{3.713957in}{0.936717in}}%
\pgfpathcurveto{\pgfqpoint{3.706143in}{0.944531in}}{\pgfqpoint{3.695544in}{0.948921in}}{\pgfqpoint{3.684494in}{0.948921in}}%
\pgfpathcurveto{\pgfqpoint{3.673444in}{0.948921in}}{\pgfqpoint{3.662845in}{0.944531in}}{\pgfqpoint{3.655031in}{0.936717in}}%
\pgfpathcurveto{\pgfqpoint{3.647217in}{0.928904in}}{\pgfqpoint{3.642827in}{0.918305in}}{\pgfqpoint{3.642827in}{0.907254in}}%
\pgfpathcurveto{\pgfqpoint{3.642827in}{0.896204in}}{\pgfqpoint{3.647217in}{0.885605in}}{\pgfqpoint{3.655031in}{0.877792in}}%
\pgfpathcurveto{\pgfqpoint{3.662845in}{0.869978in}}{\pgfqpoint{3.673444in}{0.865588in}}{\pgfqpoint{3.684494in}{0.865588in}}%
\pgfpathlineto{\pgfqpoint{3.684494in}{0.865588in}}%
\pgfpathclose%
\pgfusepath{stroke}%
\end{pgfscope}%
\begin{pgfscope}%
\pgfpathrectangle{\pgfqpoint{0.847223in}{0.554012in}}{\pgfqpoint{6.200000in}{4.530000in}}%
\pgfusepath{clip}%
\pgfsetbuttcap%
\pgfsetroundjoin%
\pgfsetlinewidth{1.003750pt}%
\definecolor{currentstroke}{rgb}{1.000000,0.000000,0.000000}%
\pgfsetstrokecolor{currentstroke}%
\pgfsetdash{}{0pt}%
\pgfpathmoveto{\pgfqpoint{3.689827in}{0.864367in}}%
\pgfpathcurveto{\pgfqpoint{3.700877in}{0.864367in}}{\pgfqpoint{3.711476in}{0.868757in}}{\pgfqpoint{3.719290in}{0.876571in}}%
\pgfpathcurveto{\pgfqpoint{3.727103in}{0.884385in}}{\pgfqpoint{3.731494in}{0.894984in}}{\pgfqpoint{3.731494in}{0.906034in}}%
\pgfpathcurveto{\pgfqpoint{3.731494in}{0.917084in}}{\pgfqpoint{3.727103in}{0.927683in}}{\pgfqpoint{3.719290in}{0.935497in}}%
\pgfpathcurveto{\pgfqpoint{3.711476in}{0.943310in}}{\pgfqpoint{3.700877in}{0.947700in}}{\pgfqpoint{3.689827in}{0.947700in}}%
\pgfpathcurveto{\pgfqpoint{3.678777in}{0.947700in}}{\pgfqpoint{3.668178in}{0.943310in}}{\pgfqpoint{3.660364in}{0.935497in}}%
\pgfpathcurveto{\pgfqpoint{3.652551in}{0.927683in}}{\pgfqpoint{3.648160in}{0.917084in}}{\pgfqpoint{3.648160in}{0.906034in}}%
\pgfpathcurveto{\pgfqpoint{3.648160in}{0.894984in}}{\pgfqpoint{3.652551in}{0.884385in}}{\pgfqpoint{3.660364in}{0.876571in}}%
\pgfpathcurveto{\pgfqpoint{3.668178in}{0.868757in}}{\pgfqpoint{3.678777in}{0.864367in}}{\pgfqpoint{3.689827in}{0.864367in}}%
\pgfpathlineto{\pgfqpoint{3.689827in}{0.864367in}}%
\pgfpathclose%
\pgfusepath{stroke}%
\end{pgfscope}%
\begin{pgfscope}%
\pgfpathrectangle{\pgfqpoint{0.847223in}{0.554012in}}{\pgfqpoint{6.200000in}{4.530000in}}%
\pgfusepath{clip}%
\pgfsetbuttcap%
\pgfsetroundjoin%
\pgfsetlinewidth{1.003750pt}%
\definecolor{currentstroke}{rgb}{1.000000,0.000000,0.000000}%
\pgfsetstrokecolor{currentstroke}%
\pgfsetdash{}{0pt}%
\pgfpathmoveto{\pgfqpoint{3.695160in}{0.863150in}}%
\pgfpathcurveto{\pgfqpoint{3.706210in}{0.863150in}}{\pgfqpoint{3.716809in}{0.867541in}}{\pgfqpoint{3.724623in}{0.875354in}}%
\pgfpathcurveto{\pgfqpoint{3.732437in}{0.883168in}}{\pgfqpoint{3.736827in}{0.893767in}}{\pgfqpoint{3.736827in}{0.904817in}}%
\pgfpathcurveto{\pgfqpoint{3.736827in}{0.915867in}}{\pgfqpoint{3.732437in}{0.926466in}}{\pgfqpoint{3.724623in}{0.934280in}}%
\pgfpathcurveto{\pgfqpoint{3.716809in}{0.942093in}}{\pgfqpoint{3.706210in}{0.946484in}}{\pgfqpoint{3.695160in}{0.946484in}}%
\pgfpathcurveto{\pgfqpoint{3.684110in}{0.946484in}}{\pgfqpoint{3.673511in}{0.942093in}}{\pgfqpoint{3.665697in}{0.934280in}}%
\pgfpathcurveto{\pgfqpoint{3.657884in}{0.926466in}}{\pgfqpoint{3.653494in}{0.915867in}}{\pgfqpoint{3.653494in}{0.904817in}}%
\pgfpathcurveto{\pgfqpoint{3.653494in}{0.893767in}}{\pgfqpoint{3.657884in}{0.883168in}}{\pgfqpoint{3.665697in}{0.875354in}}%
\pgfpathcurveto{\pgfqpoint{3.673511in}{0.867541in}}{\pgfqpoint{3.684110in}{0.863150in}}{\pgfqpoint{3.695160in}{0.863150in}}%
\pgfpathlineto{\pgfqpoint{3.695160in}{0.863150in}}%
\pgfpathclose%
\pgfusepath{stroke}%
\end{pgfscope}%
\begin{pgfscope}%
\pgfpathrectangle{\pgfqpoint{0.847223in}{0.554012in}}{\pgfqpoint{6.200000in}{4.530000in}}%
\pgfusepath{clip}%
\pgfsetbuttcap%
\pgfsetroundjoin%
\pgfsetlinewidth{1.003750pt}%
\definecolor{currentstroke}{rgb}{1.000000,0.000000,0.000000}%
\pgfsetstrokecolor{currentstroke}%
\pgfsetdash{}{0pt}%
\pgfpathmoveto{\pgfqpoint{3.700493in}{0.861937in}}%
\pgfpathcurveto{\pgfqpoint{3.711544in}{0.861937in}}{\pgfqpoint{3.722143in}{0.866328in}}{\pgfqpoint{3.729956in}{0.874141in}}%
\pgfpathcurveto{\pgfqpoint{3.737770in}{0.881955in}}{\pgfqpoint{3.742160in}{0.892554in}}{\pgfqpoint{3.742160in}{0.903604in}}%
\pgfpathcurveto{\pgfqpoint{3.742160in}{0.914654in}}{\pgfqpoint{3.737770in}{0.925253in}}{\pgfqpoint{3.729956in}{0.933067in}}%
\pgfpathcurveto{\pgfqpoint{3.722143in}{0.940880in}}{\pgfqpoint{3.711544in}{0.945271in}}{\pgfqpoint{3.700493in}{0.945271in}}%
\pgfpathcurveto{\pgfqpoint{3.689443in}{0.945271in}}{\pgfqpoint{3.678844in}{0.940880in}}{\pgfqpoint{3.671031in}{0.933067in}}%
\pgfpathcurveto{\pgfqpoint{3.663217in}{0.925253in}}{\pgfqpoint{3.658827in}{0.914654in}}{\pgfqpoint{3.658827in}{0.903604in}}%
\pgfpathcurveto{\pgfqpoint{3.658827in}{0.892554in}}{\pgfqpoint{3.663217in}{0.881955in}}{\pgfqpoint{3.671031in}{0.874141in}}%
\pgfpathcurveto{\pgfqpoint{3.678844in}{0.866328in}}{\pgfqpoint{3.689443in}{0.861937in}}{\pgfqpoint{3.700493in}{0.861937in}}%
\pgfpathlineto{\pgfqpoint{3.700493in}{0.861937in}}%
\pgfpathclose%
\pgfusepath{stroke}%
\end{pgfscope}%
\begin{pgfscope}%
\pgfpathrectangle{\pgfqpoint{0.847223in}{0.554012in}}{\pgfqpoint{6.200000in}{4.530000in}}%
\pgfusepath{clip}%
\pgfsetbuttcap%
\pgfsetroundjoin%
\pgfsetlinewidth{1.003750pt}%
\definecolor{currentstroke}{rgb}{1.000000,0.000000,0.000000}%
\pgfsetstrokecolor{currentstroke}%
\pgfsetdash{}{0pt}%
\pgfpathmoveto{\pgfqpoint{3.705827in}{0.860728in}}%
\pgfpathcurveto{\pgfqpoint{3.716877in}{0.860728in}}{\pgfqpoint{3.727476in}{0.865118in}}{\pgfqpoint{3.735289in}{0.872932in}}%
\pgfpathcurveto{\pgfqpoint{3.743103in}{0.880746in}}{\pgfqpoint{3.747493in}{0.891345in}}{\pgfqpoint{3.747493in}{0.902395in}}%
\pgfpathcurveto{\pgfqpoint{3.747493in}{0.913445in}}{\pgfqpoint{3.743103in}{0.924044in}}{\pgfqpoint{3.735289in}{0.931858in}}%
\pgfpathcurveto{\pgfqpoint{3.727476in}{0.939671in}}{\pgfqpoint{3.716877in}{0.944061in}}{\pgfqpoint{3.705827in}{0.944061in}}%
\pgfpathcurveto{\pgfqpoint{3.694777in}{0.944061in}}{\pgfqpoint{3.684177in}{0.939671in}}{\pgfqpoint{3.676364in}{0.931858in}}%
\pgfpathcurveto{\pgfqpoint{3.668550in}{0.924044in}}{\pgfqpoint{3.664160in}{0.913445in}}{\pgfqpoint{3.664160in}{0.902395in}}%
\pgfpathcurveto{\pgfqpoint{3.664160in}{0.891345in}}{\pgfqpoint{3.668550in}{0.880746in}}{\pgfqpoint{3.676364in}{0.872932in}}%
\pgfpathcurveto{\pgfqpoint{3.684177in}{0.865118in}}{\pgfqpoint{3.694777in}{0.860728in}}{\pgfqpoint{3.705827in}{0.860728in}}%
\pgfpathlineto{\pgfqpoint{3.705827in}{0.860728in}}%
\pgfpathclose%
\pgfusepath{stroke}%
\end{pgfscope}%
\begin{pgfscope}%
\pgfpathrectangle{\pgfqpoint{0.847223in}{0.554012in}}{\pgfqpoint{6.200000in}{4.530000in}}%
\pgfusepath{clip}%
\pgfsetbuttcap%
\pgfsetroundjoin%
\pgfsetlinewidth{1.003750pt}%
\definecolor{currentstroke}{rgb}{1.000000,0.000000,0.000000}%
\pgfsetstrokecolor{currentstroke}%
\pgfsetdash{}{0pt}%
\pgfpathmoveto{\pgfqpoint{3.711160in}{0.859523in}}%
\pgfpathcurveto{\pgfqpoint{3.722210in}{0.859523in}}{\pgfqpoint{3.732809in}{0.863913in}}{\pgfqpoint{3.740623in}{0.871727in}}%
\pgfpathcurveto{\pgfqpoint{3.748436in}{0.879540in}}{\pgfqpoint{3.752827in}{0.890139in}}{\pgfqpoint{3.752827in}{0.901189in}}%
\pgfpathcurveto{\pgfqpoint{3.752827in}{0.912240in}}{\pgfqpoint{3.748436in}{0.922839in}}{\pgfqpoint{3.740623in}{0.930652in}}%
\pgfpathcurveto{\pgfqpoint{3.732809in}{0.938466in}}{\pgfqpoint{3.722210in}{0.942856in}}{\pgfqpoint{3.711160in}{0.942856in}}%
\pgfpathcurveto{\pgfqpoint{3.700110in}{0.942856in}}{\pgfqpoint{3.689511in}{0.938466in}}{\pgfqpoint{3.681697in}{0.930652in}}%
\pgfpathcurveto{\pgfqpoint{3.673883in}{0.922839in}}{\pgfqpoint{3.669493in}{0.912240in}}{\pgfqpoint{3.669493in}{0.901189in}}%
\pgfpathcurveto{\pgfqpoint{3.669493in}{0.890139in}}{\pgfqpoint{3.673883in}{0.879540in}}{\pgfqpoint{3.681697in}{0.871727in}}%
\pgfpathcurveto{\pgfqpoint{3.689511in}{0.863913in}}{\pgfqpoint{3.700110in}{0.859523in}}{\pgfqpoint{3.711160in}{0.859523in}}%
\pgfpathlineto{\pgfqpoint{3.711160in}{0.859523in}}%
\pgfpathclose%
\pgfusepath{stroke}%
\end{pgfscope}%
\begin{pgfscope}%
\pgfpathrectangle{\pgfqpoint{0.847223in}{0.554012in}}{\pgfqpoint{6.200000in}{4.530000in}}%
\pgfusepath{clip}%
\pgfsetbuttcap%
\pgfsetroundjoin%
\pgfsetlinewidth{1.003750pt}%
\definecolor{currentstroke}{rgb}{1.000000,0.000000,0.000000}%
\pgfsetstrokecolor{currentstroke}%
\pgfsetdash{}{0pt}%
\pgfpathmoveto{\pgfqpoint{3.716493in}{0.858321in}}%
\pgfpathcurveto{\pgfqpoint{3.727543in}{0.858321in}}{\pgfqpoint{3.738142in}{0.862711in}}{\pgfqpoint{3.745956in}{0.870525in}}%
\pgfpathcurveto{\pgfqpoint{3.753769in}{0.878339in}}{\pgfqpoint{3.758160in}{0.888938in}}{\pgfqpoint{3.758160in}{0.899988in}}%
\pgfpathcurveto{\pgfqpoint{3.758160in}{0.911038in}}{\pgfqpoint{3.753769in}{0.921637in}}{\pgfqpoint{3.745956in}{0.929451in}}%
\pgfpathcurveto{\pgfqpoint{3.738142in}{0.937264in}}{\pgfqpoint{3.727543in}{0.941654in}}{\pgfqpoint{3.716493in}{0.941654in}}%
\pgfpathcurveto{\pgfqpoint{3.705443in}{0.941654in}}{\pgfqpoint{3.694844in}{0.937264in}}{\pgfqpoint{3.687030in}{0.929451in}}%
\pgfpathcurveto{\pgfqpoint{3.679217in}{0.921637in}}{\pgfqpoint{3.674826in}{0.911038in}}{\pgfqpoint{3.674826in}{0.899988in}}%
\pgfpathcurveto{\pgfqpoint{3.674826in}{0.888938in}}{\pgfqpoint{3.679217in}{0.878339in}}{\pgfqpoint{3.687030in}{0.870525in}}%
\pgfpathcurveto{\pgfqpoint{3.694844in}{0.862711in}}{\pgfqpoint{3.705443in}{0.858321in}}{\pgfqpoint{3.716493in}{0.858321in}}%
\pgfpathlineto{\pgfqpoint{3.716493in}{0.858321in}}%
\pgfpathclose%
\pgfusepath{stroke}%
\end{pgfscope}%
\begin{pgfscope}%
\pgfpathrectangle{\pgfqpoint{0.847223in}{0.554012in}}{\pgfqpoint{6.200000in}{4.530000in}}%
\pgfusepath{clip}%
\pgfsetbuttcap%
\pgfsetroundjoin%
\pgfsetlinewidth{1.003750pt}%
\definecolor{currentstroke}{rgb}{1.000000,0.000000,0.000000}%
\pgfsetstrokecolor{currentstroke}%
\pgfsetdash{}{0pt}%
\pgfpathmoveto{\pgfqpoint{3.721826in}{0.857123in}}%
\pgfpathcurveto{\pgfqpoint{3.732876in}{0.857123in}}{\pgfqpoint{3.743475in}{0.861514in}}{\pgfqpoint{3.751289in}{0.869327in}}%
\pgfpathcurveto{\pgfqpoint{3.759103in}{0.877141in}}{\pgfqpoint{3.763493in}{0.887740in}}{\pgfqpoint{3.763493in}{0.898790in}}%
\pgfpathcurveto{\pgfqpoint{3.763493in}{0.909840in}}{\pgfqpoint{3.759103in}{0.920439in}}{\pgfqpoint{3.751289in}{0.928253in}}%
\pgfpathcurveto{\pgfqpoint{3.743475in}{0.936066in}}{\pgfqpoint{3.732876in}{0.940457in}}{\pgfqpoint{3.721826in}{0.940457in}}%
\pgfpathcurveto{\pgfqpoint{3.710776in}{0.940457in}}{\pgfqpoint{3.700177in}{0.936066in}}{\pgfqpoint{3.692364in}{0.928253in}}%
\pgfpathcurveto{\pgfqpoint{3.684550in}{0.920439in}}{\pgfqpoint{3.680160in}{0.909840in}}{\pgfqpoint{3.680160in}{0.898790in}}%
\pgfpathcurveto{\pgfqpoint{3.680160in}{0.887740in}}{\pgfqpoint{3.684550in}{0.877141in}}{\pgfqpoint{3.692364in}{0.869327in}}%
\pgfpathcurveto{\pgfqpoint{3.700177in}{0.861514in}}{\pgfqpoint{3.710776in}{0.857123in}}{\pgfqpoint{3.721826in}{0.857123in}}%
\pgfpathlineto{\pgfqpoint{3.721826in}{0.857123in}}%
\pgfpathclose%
\pgfusepath{stroke}%
\end{pgfscope}%
\begin{pgfscope}%
\pgfpathrectangle{\pgfqpoint{0.847223in}{0.554012in}}{\pgfqpoint{6.200000in}{4.530000in}}%
\pgfusepath{clip}%
\pgfsetbuttcap%
\pgfsetroundjoin%
\pgfsetlinewidth{1.003750pt}%
\definecolor{currentstroke}{rgb}{1.000000,0.000000,0.000000}%
\pgfsetstrokecolor{currentstroke}%
\pgfsetdash{}{0pt}%
\pgfpathmoveto{\pgfqpoint{3.727160in}{0.855929in}}%
\pgfpathcurveto{\pgfqpoint{3.738210in}{0.855929in}}{\pgfqpoint{3.748809in}{0.860320in}}{\pgfqpoint{3.756622in}{0.868133in}}%
\pgfpathcurveto{\pgfqpoint{3.764436in}{0.875947in}}{\pgfqpoint{3.768826in}{0.886546in}}{\pgfqpoint{3.768826in}{0.897596in}}%
\pgfpathcurveto{\pgfqpoint{3.768826in}{0.908646in}}{\pgfqpoint{3.764436in}{0.919245in}}{\pgfqpoint{3.756622in}{0.927059in}}%
\pgfpathcurveto{\pgfqpoint{3.748809in}{0.934872in}}{\pgfqpoint{3.738210in}{0.939263in}}{\pgfqpoint{3.727160in}{0.939263in}}%
\pgfpathcurveto{\pgfqpoint{3.716109in}{0.939263in}}{\pgfqpoint{3.705510in}{0.934872in}}{\pgfqpoint{3.697697in}{0.927059in}}%
\pgfpathcurveto{\pgfqpoint{3.689883in}{0.919245in}}{\pgfqpoint{3.685493in}{0.908646in}}{\pgfqpoint{3.685493in}{0.897596in}}%
\pgfpathcurveto{\pgfqpoint{3.685493in}{0.886546in}}{\pgfqpoint{3.689883in}{0.875947in}}{\pgfqpoint{3.697697in}{0.868133in}}%
\pgfpathcurveto{\pgfqpoint{3.705510in}{0.860320in}}{\pgfqpoint{3.716109in}{0.855929in}}{\pgfqpoint{3.727160in}{0.855929in}}%
\pgfpathlineto{\pgfqpoint{3.727160in}{0.855929in}}%
\pgfpathclose%
\pgfusepath{stroke}%
\end{pgfscope}%
\begin{pgfscope}%
\pgfpathrectangle{\pgfqpoint{0.847223in}{0.554012in}}{\pgfqpoint{6.200000in}{4.530000in}}%
\pgfusepath{clip}%
\pgfsetbuttcap%
\pgfsetroundjoin%
\pgfsetlinewidth{1.003750pt}%
\definecolor{currentstroke}{rgb}{1.000000,0.000000,0.000000}%
\pgfsetstrokecolor{currentstroke}%
\pgfsetdash{}{0pt}%
\pgfpathmoveto{\pgfqpoint{3.732493in}{0.854739in}}%
\pgfpathcurveto{\pgfqpoint{3.743543in}{0.854739in}}{\pgfqpoint{3.754142in}{0.859129in}}{\pgfqpoint{3.761956in}{0.866943in}}%
\pgfpathcurveto{\pgfqpoint{3.769769in}{0.874756in}}{\pgfqpoint{3.774159in}{0.885356in}}{\pgfqpoint{3.774159in}{0.896406in}}%
\pgfpathcurveto{\pgfqpoint{3.774159in}{0.907456in}}{\pgfqpoint{3.769769in}{0.918055in}}{\pgfqpoint{3.761956in}{0.925868in}}%
\pgfpathcurveto{\pgfqpoint{3.754142in}{0.933682in}}{\pgfqpoint{3.743543in}{0.938072in}}{\pgfqpoint{3.732493in}{0.938072in}}%
\pgfpathcurveto{\pgfqpoint{3.721443in}{0.938072in}}{\pgfqpoint{3.710844in}{0.933682in}}{\pgfqpoint{3.703030in}{0.925868in}}%
\pgfpathcurveto{\pgfqpoint{3.695216in}{0.918055in}}{\pgfqpoint{3.690826in}{0.907456in}}{\pgfqpoint{3.690826in}{0.896406in}}%
\pgfpathcurveto{\pgfqpoint{3.690826in}{0.885356in}}{\pgfqpoint{3.695216in}{0.874756in}}{\pgfqpoint{3.703030in}{0.866943in}}%
\pgfpathcurveto{\pgfqpoint{3.710844in}{0.859129in}}{\pgfqpoint{3.721443in}{0.854739in}}{\pgfqpoint{3.732493in}{0.854739in}}%
\pgfpathlineto{\pgfqpoint{3.732493in}{0.854739in}}%
\pgfpathclose%
\pgfusepath{stroke}%
\end{pgfscope}%
\begin{pgfscope}%
\pgfpathrectangle{\pgfqpoint{0.847223in}{0.554012in}}{\pgfqpoint{6.200000in}{4.530000in}}%
\pgfusepath{clip}%
\pgfsetbuttcap%
\pgfsetroundjoin%
\pgfsetlinewidth{1.003750pt}%
\definecolor{currentstroke}{rgb}{1.000000,0.000000,0.000000}%
\pgfsetstrokecolor{currentstroke}%
\pgfsetdash{}{0pt}%
\pgfpathmoveto{\pgfqpoint{3.737826in}{0.853552in}}%
\pgfpathcurveto{\pgfqpoint{3.748876in}{0.853552in}}{\pgfqpoint{3.759475in}{0.857943in}}{\pgfqpoint{3.767289in}{0.865756in}}%
\pgfpathcurveto{\pgfqpoint{3.775102in}{0.873570in}}{\pgfqpoint{3.779493in}{0.884169in}}{\pgfqpoint{3.779493in}{0.895219in}}%
\pgfpathcurveto{\pgfqpoint{3.779493in}{0.906269in}}{\pgfqpoint{3.775102in}{0.916868in}}{\pgfqpoint{3.767289in}{0.924682in}}%
\pgfpathcurveto{\pgfqpoint{3.759475in}{0.932495in}}{\pgfqpoint{3.748876in}{0.936886in}}{\pgfqpoint{3.737826in}{0.936886in}}%
\pgfpathcurveto{\pgfqpoint{3.726776in}{0.936886in}}{\pgfqpoint{3.716177in}{0.932495in}}{\pgfqpoint{3.708363in}{0.924682in}}%
\pgfpathcurveto{\pgfqpoint{3.700550in}{0.916868in}}{\pgfqpoint{3.696159in}{0.906269in}}{\pgfqpoint{3.696159in}{0.895219in}}%
\pgfpathcurveto{\pgfqpoint{3.696159in}{0.884169in}}{\pgfqpoint{3.700550in}{0.873570in}}{\pgfqpoint{3.708363in}{0.865756in}}%
\pgfpathcurveto{\pgfqpoint{3.716177in}{0.857943in}}{\pgfqpoint{3.726776in}{0.853552in}}{\pgfqpoint{3.737826in}{0.853552in}}%
\pgfpathlineto{\pgfqpoint{3.737826in}{0.853552in}}%
\pgfpathclose%
\pgfusepath{stroke}%
\end{pgfscope}%
\begin{pgfscope}%
\pgfpathrectangle{\pgfqpoint{0.847223in}{0.554012in}}{\pgfqpoint{6.200000in}{4.530000in}}%
\pgfusepath{clip}%
\pgfsetbuttcap%
\pgfsetroundjoin%
\pgfsetlinewidth{1.003750pt}%
\definecolor{currentstroke}{rgb}{1.000000,0.000000,0.000000}%
\pgfsetstrokecolor{currentstroke}%
\pgfsetdash{}{0pt}%
\pgfpathmoveto{\pgfqpoint{3.743159in}{0.852369in}}%
\pgfpathcurveto{\pgfqpoint{3.754209in}{0.852369in}}{\pgfqpoint{3.764808in}{0.856760in}}{\pgfqpoint{3.772622in}{0.864573in}}%
\pgfpathcurveto{\pgfqpoint{3.780436in}{0.872387in}}{\pgfqpoint{3.784826in}{0.882986in}}{\pgfqpoint{3.784826in}{0.894036in}}%
\pgfpathcurveto{\pgfqpoint{3.784826in}{0.905086in}}{\pgfqpoint{3.780436in}{0.915685in}}{\pgfqpoint{3.772622in}{0.923499in}}%
\pgfpathcurveto{\pgfqpoint{3.764808in}{0.931313in}}{\pgfqpoint{3.754209in}{0.935703in}}{\pgfqpoint{3.743159in}{0.935703in}}%
\pgfpathcurveto{\pgfqpoint{3.732109in}{0.935703in}}{\pgfqpoint{3.721510in}{0.931313in}}{\pgfqpoint{3.713696in}{0.923499in}}%
\pgfpathcurveto{\pgfqpoint{3.705883in}{0.915685in}}{\pgfqpoint{3.701492in}{0.905086in}}{\pgfqpoint{3.701492in}{0.894036in}}%
\pgfpathcurveto{\pgfqpoint{3.701492in}{0.882986in}}{\pgfqpoint{3.705883in}{0.872387in}}{\pgfqpoint{3.713696in}{0.864573in}}%
\pgfpathcurveto{\pgfqpoint{3.721510in}{0.856760in}}{\pgfqpoint{3.732109in}{0.852369in}}{\pgfqpoint{3.743159in}{0.852369in}}%
\pgfpathlineto{\pgfqpoint{3.743159in}{0.852369in}}%
\pgfpathclose%
\pgfusepath{stroke}%
\end{pgfscope}%
\begin{pgfscope}%
\pgfpathrectangle{\pgfqpoint{0.847223in}{0.554012in}}{\pgfqpoint{6.200000in}{4.530000in}}%
\pgfusepath{clip}%
\pgfsetbuttcap%
\pgfsetroundjoin%
\pgfsetlinewidth{1.003750pt}%
\definecolor{currentstroke}{rgb}{1.000000,0.000000,0.000000}%
\pgfsetstrokecolor{currentstroke}%
\pgfsetdash{}{0pt}%
\pgfpathmoveto{\pgfqpoint{3.748492in}{0.851190in}}%
\pgfpathcurveto{\pgfqpoint{3.759543in}{0.851190in}}{\pgfqpoint{3.770142in}{0.855581in}}{\pgfqpoint{3.777955in}{0.863394in}}%
\pgfpathcurveto{\pgfqpoint{3.785769in}{0.871208in}}{\pgfqpoint{3.790159in}{0.881807in}}{\pgfqpoint{3.790159in}{0.892857in}}%
\pgfpathcurveto{\pgfqpoint{3.790159in}{0.903907in}}{\pgfqpoint{3.785769in}{0.914506in}}{\pgfqpoint{3.777955in}{0.922320in}}%
\pgfpathcurveto{\pgfqpoint{3.770142in}{0.930133in}}{\pgfqpoint{3.759543in}{0.934524in}}{\pgfqpoint{3.748492in}{0.934524in}}%
\pgfpathcurveto{\pgfqpoint{3.737442in}{0.934524in}}{\pgfqpoint{3.726843in}{0.930133in}}{\pgfqpoint{3.719030in}{0.922320in}}%
\pgfpathcurveto{\pgfqpoint{3.711216in}{0.914506in}}{\pgfqpoint{3.706826in}{0.903907in}}{\pgfqpoint{3.706826in}{0.892857in}}%
\pgfpathcurveto{\pgfqpoint{3.706826in}{0.881807in}}{\pgfqpoint{3.711216in}{0.871208in}}{\pgfqpoint{3.719030in}{0.863394in}}%
\pgfpathcurveto{\pgfqpoint{3.726843in}{0.855581in}}{\pgfqpoint{3.737442in}{0.851190in}}{\pgfqpoint{3.748492in}{0.851190in}}%
\pgfpathlineto{\pgfqpoint{3.748492in}{0.851190in}}%
\pgfpathclose%
\pgfusepath{stroke}%
\end{pgfscope}%
\begin{pgfscope}%
\pgfpathrectangle{\pgfqpoint{0.847223in}{0.554012in}}{\pgfqpoint{6.200000in}{4.530000in}}%
\pgfusepath{clip}%
\pgfsetbuttcap%
\pgfsetroundjoin%
\pgfsetlinewidth{1.003750pt}%
\definecolor{currentstroke}{rgb}{1.000000,0.000000,0.000000}%
\pgfsetstrokecolor{currentstroke}%
\pgfsetdash{}{0pt}%
\pgfpathmoveto{\pgfqpoint{3.753826in}{0.850015in}}%
\pgfpathcurveto{\pgfqpoint{3.764876in}{0.850015in}}{\pgfqpoint{3.775475in}{0.854405in}}{\pgfqpoint{3.783288in}{0.862219in}}%
\pgfpathcurveto{\pgfqpoint{3.791102in}{0.870032in}}{\pgfqpoint{3.795492in}{0.880631in}}{\pgfqpoint{3.795492in}{0.891681in}}%
\pgfpathcurveto{\pgfqpoint{3.795492in}{0.902732in}}{\pgfqpoint{3.791102in}{0.913331in}}{\pgfqpoint{3.783288in}{0.921144in}}%
\pgfpathcurveto{\pgfqpoint{3.775475in}{0.928958in}}{\pgfqpoint{3.764876in}{0.933348in}}{\pgfqpoint{3.753826in}{0.933348in}}%
\pgfpathcurveto{\pgfqpoint{3.742775in}{0.933348in}}{\pgfqpoint{3.732176in}{0.928958in}}{\pgfqpoint{3.724363in}{0.921144in}}%
\pgfpathcurveto{\pgfqpoint{3.716549in}{0.913331in}}{\pgfqpoint{3.712159in}{0.902732in}}{\pgfqpoint{3.712159in}{0.891681in}}%
\pgfpathcurveto{\pgfqpoint{3.712159in}{0.880631in}}{\pgfqpoint{3.716549in}{0.870032in}}{\pgfqpoint{3.724363in}{0.862219in}}%
\pgfpathcurveto{\pgfqpoint{3.732176in}{0.854405in}}{\pgfqpoint{3.742775in}{0.850015in}}{\pgfqpoint{3.753826in}{0.850015in}}%
\pgfpathlineto{\pgfqpoint{3.753826in}{0.850015in}}%
\pgfpathclose%
\pgfusepath{stroke}%
\end{pgfscope}%
\begin{pgfscope}%
\pgfpathrectangle{\pgfqpoint{0.847223in}{0.554012in}}{\pgfqpoint{6.200000in}{4.530000in}}%
\pgfusepath{clip}%
\pgfsetbuttcap%
\pgfsetroundjoin%
\pgfsetlinewidth{1.003750pt}%
\definecolor{currentstroke}{rgb}{1.000000,0.000000,0.000000}%
\pgfsetstrokecolor{currentstroke}%
\pgfsetdash{}{0pt}%
\pgfpathmoveto{\pgfqpoint{3.759159in}{0.848843in}}%
\pgfpathcurveto{\pgfqpoint{3.770209in}{0.848843in}}{\pgfqpoint{3.780808in}{0.853233in}}{\pgfqpoint{3.788622in}{0.861047in}}%
\pgfpathcurveto{\pgfqpoint{3.796435in}{0.868860in}}{\pgfqpoint{3.800825in}{0.879459in}}{\pgfqpoint{3.800825in}{0.890509in}}%
\pgfpathcurveto{\pgfqpoint{3.800825in}{0.901560in}}{\pgfqpoint{3.796435in}{0.912159in}}{\pgfqpoint{3.788622in}{0.919972in}}%
\pgfpathcurveto{\pgfqpoint{3.780808in}{0.927786in}}{\pgfqpoint{3.770209in}{0.932176in}}{\pgfqpoint{3.759159in}{0.932176in}}%
\pgfpathcurveto{\pgfqpoint{3.748109in}{0.932176in}}{\pgfqpoint{3.737510in}{0.927786in}}{\pgfqpoint{3.729696in}{0.919972in}}%
\pgfpathcurveto{\pgfqpoint{3.721882in}{0.912159in}}{\pgfqpoint{3.717492in}{0.901560in}}{\pgfqpoint{3.717492in}{0.890509in}}%
\pgfpathcurveto{\pgfqpoint{3.717492in}{0.879459in}}{\pgfqpoint{3.721882in}{0.868860in}}{\pgfqpoint{3.729696in}{0.861047in}}%
\pgfpathcurveto{\pgfqpoint{3.737510in}{0.853233in}}{\pgfqpoint{3.748109in}{0.848843in}}{\pgfqpoint{3.759159in}{0.848843in}}%
\pgfpathlineto{\pgfqpoint{3.759159in}{0.848843in}}%
\pgfpathclose%
\pgfusepath{stroke}%
\end{pgfscope}%
\begin{pgfscope}%
\pgfpathrectangle{\pgfqpoint{0.847223in}{0.554012in}}{\pgfqpoint{6.200000in}{4.530000in}}%
\pgfusepath{clip}%
\pgfsetbuttcap%
\pgfsetroundjoin%
\pgfsetlinewidth{1.003750pt}%
\definecolor{currentstroke}{rgb}{1.000000,0.000000,0.000000}%
\pgfsetstrokecolor{currentstroke}%
\pgfsetdash{}{0pt}%
\pgfpathmoveto{\pgfqpoint{3.764492in}{0.847675in}}%
\pgfpathcurveto{\pgfqpoint{3.775542in}{0.847675in}}{\pgfqpoint{3.786141in}{0.852065in}}{\pgfqpoint{3.793955in}{0.859878in}}%
\pgfpathcurveto{\pgfqpoint{3.801768in}{0.867692in}}{\pgfqpoint{3.806159in}{0.878291in}}{\pgfqpoint{3.806159in}{0.889341in}}%
\pgfpathcurveto{\pgfqpoint{3.806159in}{0.900391in}}{\pgfqpoint{3.801768in}{0.910990in}}{\pgfqpoint{3.793955in}{0.918804in}}%
\pgfpathcurveto{\pgfqpoint{3.786141in}{0.926618in}}{\pgfqpoint{3.775542in}{0.931008in}}{\pgfqpoint{3.764492in}{0.931008in}}%
\pgfpathcurveto{\pgfqpoint{3.753442in}{0.931008in}}{\pgfqpoint{3.742843in}{0.926618in}}{\pgfqpoint{3.735029in}{0.918804in}}%
\pgfpathcurveto{\pgfqpoint{3.727216in}{0.910990in}}{\pgfqpoint{3.722825in}{0.900391in}}{\pgfqpoint{3.722825in}{0.889341in}}%
\pgfpathcurveto{\pgfqpoint{3.722825in}{0.878291in}}{\pgfqpoint{3.727216in}{0.867692in}}{\pgfqpoint{3.735029in}{0.859878in}}%
\pgfpathcurveto{\pgfqpoint{3.742843in}{0.852065in}}{\pgfqpoint{3.753442in}{0.847675in}}{\pgfqpoint{3.764492in}{0.847675in}}%
\pgfpathlineto{\pgfqpoint{3.764492in}{0.847675in}}%
\pgfpathclose%
\pgfusepath{stroke}%
\end{pgfscope}%
\begin{pgfscope}%
\pgfpathrectangle{\pgfqpoint{0.847223in}{0.554012in}}{\pgfqpoint{6.200000in}{4.530000in}}%
\pgfusepath{clip}%
\pgfsetbuttcap%
\pgfsetroundjoin%
\pgfsetlinewidth{1.003750pt}%
\definecolor{currentstroke}{rgb}{1.000000,0.000000,0.000000}%
\pgfsetstrokecolor{currentstroke}%
\pgfsetdash{}{0pt}%
\pgfpathmoveto{\pgfqpoint{3.769825in}{0.846510in}}%
\pgfpathcurveto{\pgfqpoint{3.780875in}{0.846510in}}{\pgfqpoint{3.791474in}{0.850900in}}{\pgfqpoint{3.799288in}{0.858714in}}%
\pgfpathcurveto{\pgfqpoint{3.807102in}{0.866527in}}{\pgfqpoint{3.811492in}{0.877126in}}{\pgfqpoint{3.811492in}{0.888177in}}%
\pgfpathcurveto{\pgfqpoint{3.811492in}{0.899227in}}{\pgfqpoint{3.807102in}{0.909826in}}{\pgfqpoint{3.799288in}{0.917639in}}%
\pgfpathcurveto{\pgfqpoint{3.791474in}{0.925453in}}{\pgfqpoint{3.780875in}{0.929843in}}{\pgfqpoint{3.769825in}{0.929843in}}%
\pgfpathcurveto{\pgfqpoint{3.758775in}{0.929843in}}{\pgfqpoint{3.748176in}{0.925453in}}{\pgfqpoint{3.740362in}{0.917639in}}%
\pgfpathcurveto{\pgfqpoint{3.732549in}{0.909826in}}{\pgfqpoint{3.728159in}{0.899227in}}{\pgfqpoint{3.728159in}{0.888177in}}%
\pgfpathcurveto{\pgfqpoint{3.728159in}{0.877126in}}{\pgfqpoint{3.732549in}{0.866527in}}{\pgfqpoint{3.740362in}{0.858714in}}%
\pgfpathcurveto{\pgfqpoint{3.748176in}{0.850900in}}{\pgfqpoint{3.758775in}{0.846510in}}{\pgfqpoint{3.769825in}{0.846510in}}%
\pgfpathlineto{\pgfqpoint{3.769825in}{0.846510in}}%
\pgfpathclose%
\pgfusepath{stroke}%
\end{pgfscope}%
\begin{pgfscope}%
\pgfpathrectangle{\pgfqpoint{0.847223in}{0.554012in}}{\pgfqpoint{6.200000in}{4.530000in}}%
\pgfusepath{clip}%
\pgfsetbuttcap%
\pgfsetroundjoin%
\pgfsetlinewidth{1.003750pt}%
\definecolor{currentstroke}{rgb}{1.000000,0.000000,0.000000}%
\pgfsetstrokecolor{currentstroke}%
\pgfsetdash{}{0pt}%
\pgfpathmoveto{\pgfqpoint{3.775158in}{0.845349in}}%
\pgfpathcurveto{\pgfqpoint{3.786209in}{0.845349in}}{\pgfqpoint{3.796808in}{0.849739in}}{\pgfqpoint{3.804621in}{0.857553in}}%
\pgfpathcurveto{\pgfqpoint{3.812435in}{0.865366in}}{\pgfqpoint{3.816825in}{0.875965in}}{\pgfqpoint{3.816825in}{0.887016in}}%
\pgfpathcurveto{\pgfqpoint{3.816825in}{0.898066in}}{\pgfqpoint{3.812435in}{0.908665in}}{\pgfqpoint{3.804621in}{0.916478in}}%
\pgfpathcurveto{\pgfqpoint{3.796808in}{0.924292in}}{\pgfqpoint{3.786209in}{0.928682in}}{\pgfqpoint{3.775158in}{0.928682in}}%
\pgfpathcurveto{\pgfqpoint{3.764108in}{0.928682in}}{\pgfqpoint{3.753509in}{0.924292in}}{\pgfqpoint{3.745696in}{0.916478in}}%
\pgfpathcurveto{\pgfqpoint{3.737882in}{0.908665in}}{\pgfqpoint{3.733492in}{0.898066in}}{\pgfqpoint{3.733492in}{0.887016in}}%
\pgfpathcurveto{\pgfqpoint{3.733492in}{0.875965in}}{\pgfqpoint{3.737882in}{0.865366in}}{\pgfqpoint{3.745696in}{0.857553in}}%
\pgfpathcurveto{\pgfqpoint{3.753509in}{0.849739in}}{\pgfqpoint{3.764108in}{0.845349in}}{\pgfqpoint{3.775158in}{0.845349in}}%
\pgfpathlineto{\pgfqpoint{3.775158in}{0.845349in}}%
\pgfpathclose%
\pgfusepath{stroke}%
\end{pgfscope}%
\begin{pgfscope}%
\pgfpathrectangle{\pgfqpoint{0.847223in}{0.554012in}}{\pgfqpoint{6.200000in}{4.530000in}}%
\pgfusepath{clip}%
\pgfsetbuttcap%
\pgfsetroundjoin%
\pgfsetlinewidth{1.003750pt}%
\definecolor{currentstroke}{rgb}{1.000000,0.000000,0.000000}%
\pgfsetstrokecolor{currentstroke}%
\pgfsetdash{}{0pt}%
\pgfpathmoveto{\pgfqpoint{3.780492in}{0.844191in}}%
\pgfpathcurveto{\pgfqpoint{3.791542in}{0.844191in}}{\pgfqpoint{3.802141in}{0.848582in}}{\pgfqpoint{3.809954in}{0.856395in}}%
\pgfpathcurveto{\pgfqpoint{3.817768in}{0.864209in}}{\pgfqpoint{3.822158in}{0.874808in}}{\pgfqpoint{3.822158in}{0.885858in}}%
\pgfpathcurveto{\pgfqpoint{3.822158in}{0.896908in}}{\pgfqpoint{3.817768in}{0.907507in}}{\pgfqpoint{3.809954in}{0.915321in}}%
\pgfpathcurveto{\pgfqpoint{3.802141in}{0.923134in}}{\pgfqpoint{3.791542in}{0.927525in}}{\pgfqpoint{3.780492in}{0.927525in}}%
\pgfpathcurveto{\pgfqpoint{3.769442in}{0.927525in}}{\pgfqpoint{3.758843in}{0.923134in}}{\pgfqpoint{3.751029in}{0.915321in}}%
\pgfpathcurveto{\pgfqpoint{3.743215in}{0.907507in}}{\pgfqpoint{3.738825in}{0.896908in}}{\pgfqpoint{3.738825in}{0.885858in}}%
\pgfpathcurveto{\pgfqpoint{3.738825in}{0.874808in}}{\pgfqpoint{3.743215in}{0.864209in}}{\pgfqpoint{3.751029in}{0.856395in}}%
\pgfpathcurveto{\pgfqpoint{3.758843in}{0.848582in}}{\pgfqpoint{3.769442in}{0.844191in}}{\pgfqpoint{3.780492in}{0.844191in}}%
\pgfpathlineto{\pgfqpoint{3.780492in}{0.844191in}}%
\pgfpathclose%
\pgfusepath{stroke}%
\end{pgfscope}%
\begin{pgfscope}%
\pgfpathrectangle{\pgfqpoint{0.847223in}{0.554012in}}{\pgfqpoint{6.200000in}{4.530000in}}%
\pgfusepath{clip}%
\pgfsetbuttcap%
\pgfsetroundjoin%
\pgfsetlinewidth{1.003750pt}%
\definecolor{currentstroke}{rgb}{1.000000,0.000000,0.000000}%
\pgfsetstrokecolor{currentstroke}%
\pgfsetdash{}{0pt}%
\pgfpathmoveto{\pgfqpoint{3.785825in}{0.843038in}}%
\pgfpathcurveto{\pgfqpoint{3.796875in}{0.843038in}}{\pgfqpoint{3.807474in}{0.847428in}}{\pgfqpoint{3.815288in}{0.855241in}}%
\pgfpathcurveto{\pgfqpoint{3.823101in}{0.863055in}}{\pgfqpoint{3.827492in}{0.873654in}}{\pgfqpoint{3.827492in}{0.884704in}}%
\pgfpathcurveto{\pgfqpoint{3.827492in}{0.895754in}}{\pgfqpoint{3.823101in}{0.906353in}}{\pgfqpoint{3.815288in}{0.914167in}}%
\pgfpathcurveto{\pgfqpoint{3.807474in}{0.921981in}}{\pgfqpoint{3.796875in}{0.926371in}}{\pgfqpoint{3.785825in}{0.926371in}}%
\pgfpathcurveto{\pgfqpoint{3.774775in}{0.926371in}}{\pgfqpoint{3.764176in}{0.921981in}}{\pgfqpoint{3.756362in}{0.914167in}}%
\pgfpathcurveto{\pgfqpoint{3.748548in}{0.906353in}}{\pgfqpoint{3.744158in}{0.895754in}}{\pgfqpoint{3.744158in}{0.884704in}}%
\pgfpathcurveto{\pgfqpoint{3.744158in}{0.873654in}}{\pgfqpoint{3.748548in}{0.863055in}}{\pgfqpoint{3.756362in}{0.855241in}}%
\pgfpathcurveto{\pgfqpoint{3.764176in}{0.847428in}}{\pgfqpoint{3.774775in}{0.843038in}}{\pgfqpoint{3.785825in}{0.843038in}}%
\pgfpathlineto{\pgfqpoint{3.785825in}{0.843038in}}%
\pgfpathclose%
\pgfusepath{stroke}%
\end{pgfscope}%
\begin{pgfscope}%
\pgfpathrectangle{\pgfqpoint{0.847223in}{0.554012in}}{\pgfqpoint{6.200000in}{4.530000in}}%
\pgfusepath{clip}%
\pgfsetbuttcap%
\pgfsetroundjoin%
\pgfsetlinewidth{1.003750pt}%
\definecolor{currentstroke}{rgb}{1.000000,0.000000,0.000000}%
\pgfsetstrokecolor{currentstroke}%
\pgfsetdash{}{0pt}%
\pgfpathmoveto{\pgfqpoint{3.791158in}{0.841887in}}%
\pgfpathcurveto{\pgfqpoint{3.802208in}{0.841887in}}{\pgfqpoint{3.812807in}{0.846277in}}{\pgfqpoint{3.820621in}{0.854091in}}%
\pgfpathcurveto{\pgfqpoint{3.828435in}{0.861905in}}{\pgfqpoint{3.832825in}{0.872504in}}{\pgfqpoint{3.832825in}{0.883554in}}%
\pgfpathcurveto{\pgfqpoint{3.832825in}{0.894604in}}{\pgfqpoint{3.828435in}{0.905203in}}{\pgfqpoint{3.820621in}{0.913017in}}%
\pgfpathcurveto{\pgfqpoint{3.812807in}{0.920830in}}{\pgfqpoint{3.802208in}{0.925221in}}{\pgfqpoint{3.791158in}{0.925221in}}%
\pgfpathcurveto{\pgfqpoint{3.780108in}{0.925221in}}{\pgfqpoint{3.769509in}{0.920830in}}{\pgfqpoint{3.761695in}{0.913017in}}%
\pgfpathcurveto{\pgfqpoint{3.753882in}{0.905203in}}{\pgfqpoint{3.749491in}{0.894604in}}{\pgfqpoint{3.749491in}{0.883554in}}%
\pgfpathcurveto{\pgfqpoint{3.749491in}{0.872504in}}{\pgfqpoint{3.753882in}{0.861905in}}{\pgfqpoint{3.761695in}{0.854091in}}%
\pgfpathcurveto{\pgfqpoint{3.769509in}{0.846277in}}{\pgfqpoint{3.780108in}{0.841887in}}{\pgfqpoint{3.791158in}{0.841887in}}%
\pgfpathlineto{\pgfqpoint{3.791158in}{0.841887in}}%
\pgfpathclose%
\pgfusepath{stroke}%
\end{pgfscope}%
\begin{pgfscope}%
\pgfpathrectangle{\pgfqpoint{0.847223in}{0.554012in}}{\pgfqpoint{6.200000in}{4.530000in}}%
\pgfusepath{clip}%
\pgfsetbuttcap%
\pgfsetroundjoin%
\pgfsetlinewidth{1.003750pt}%
\definecolor{currentstroke}{rgb}{1.000000,0.000000,0.000000}%
\pgfsetstrokecolor{currentstroke}%
\pgfsetdash{}{0pt}%
\pgfpathmoveto{\pgfqpoint{3.796491in}{0.840740in}}%
\pgfpathcurveto{\pgfqpoint{3.807541in}{0.840740in}}{\pgfqpoint{3.818140in}{0.845131in}}{\pgfqpoint{3.825954in}{0.852944in}}%
\pgfpathcurveto{\pgfqpoint{3.833768in}{0.860758in}}{\pgfqpoint{3.838158in}{0.871357in}}{\pgfqpoint{3.838158in}{0.882407in}}%
\pgfpathcurveto{\pgfqpoint{3.838158in}{0.893457in}}{\pgfqpoint{3.833768in}{0.904056in}}{\pgfqpoint{3.825954in}{0.911870in}}%
\pgfpathcurveto{\pgfqpoint{3.818140in}{0.919683in}}{\pgfqpoint{3.807541in}{0.924074in}}{\pgfqpoint{3.796491in}{0.924074in}}%
\pgfpathcurveto{\pgfqpoint{3.785441in}{0.924074in}}{\pgfqpoint{3.774842in}{0.919683in}}{\pgfqpoint{3.767029in}{0.911870in}}%
\pgfpathcurveto{\pgfqpoint{3.759215in}{0.904056in}}{\pgfqpoint{3.754825in}{0.893457in}}{\pgfqpoint{3.754825in}{0.882407in}}%
\pgfpathcurveto{\pgfqpoint{3.754825in}{0.871357in}}{\pgfqpoint{3.759215in}{0.860758in}}{\pgfqpoint{3.767029in}{0.852944in}}%
\pgfpathcurveto{\pgfqpoint{3.774842in}{0.845131in}}{\pgfqpoint{3.785441in}{0.840740in}}{\pgfqpoint{3.796491in}{0.840740in}}%
\pgfpathlineto{\pgfqpoint{3.796491in}{0.840740in}}%
\pgfpathclose%
\pgfusepath{stroke}%
\end{pgfscope}%
\begin{pgfscope}%
\pgfpathrectangle{\pgfqpoint{0.847223in}{0.554012in}}{\pgfqpoint{6.200000in}{4.530000in}}%
\pgfusepath{clip}%
\pgfsetbuttcap%
\pgfsetroundjoin%
\pgfsetlinewidth{1.003750pt}%
\definecolor{currentstroke}{rgb}{1.000000,0.000000,0.000000}%
\pgfsetstrokecolor{currentstroke}%
\pgfsetdash{}{0pt}%
\pgfpathmoveto{\pgfqpoint{3.801825in}{0.839597in}}%
\pgfpathcurveto{\pgfqpoint{3.812875in}{0.839597in}}{\pgfqpoint{3.823474in}{0.843987in}}{\pgfqpoint{3.831287in}{0.851801in}}%
\pgfpathcurveto{\pgfqpoint{3.839101in}{0.859615in}}{\pgfqpoint{3.843491in}{0.870214in}}{\pgfqpoint{3.843491in}{0.881264in}}%
\pgfpathcurveto{\pgfqpoint{3.843491in}{0.892314in}}{\pgfqpoint{3.839101in}{0.902913in}}{\pgfqpoint{3.831287in}{0.910727in}}%
\pgfpathcurveto{\pgfqpoint{3.823474in}{0.918540in}}{\pgfqpoint{3.812875in}{0.922930in}}{\pgfqpoint{3.801825in}{0.922930in}}%
\pgfpathcurveto{\pgfqpoint{3.790774in}{0.922930in}}{\pgfqpoint{3.780175in}{0.918540in}}{\pgfqpoint{3.772362in}{0.910727in}}%
\pgfpathcurveto{\pgfqpoint{3.764548in}{0.902913in}}{\pgfqpoint{3.760158in}{0.892314in}}{\pgfqpoint{3.760158in}{0.881264in}}%
\pgfpathcurveto{\pgfqpoint{3.760158in}{0.870214in}}{\pgfqpoint{3.764548in}{0.859615in}}{\pgfqpoint{3.772362in}{0.851801in}}%
\pgfpathcurveto{\pgfqpoint{3.780175in}{0.843987in}}{\pgfqpoint{3.790774in}{0.839597in}}{\pgfqpoint{3.801825in}{0.839597in}}%
\pgfpathlineto{\pgfqpoint{3.801825in}{0.839597in}}%
\pgfpathclose%
\pgfusepath{stroke}%
\end{pgfscope}%
\begin{pgfscope}%
\pgfpathrectangle{\pgfqpoint{0.847223in}{0.554012in}}{\pgfqpoint{6.200000in}{4.530000in}}%
\pgfusepath{clip}%
\pgfsetbuttcap%
\pgfsetroundjoin%
\pgfsetlinewidth{1.003750pt}%
\definecolor{currentstroke}{rgb}{1.000000,0.000000,0.000000}%
\pgfsetstrokecolor{currentstroke}%
\pgfsetdash{}{0pt}%
\pgfpathmoveto{\pgfqpoint{3.807158in}{0.838457in}}%
\pgfpathcurveto{\pgfqpoint{3.818208in}{0.838457in}}{\pgfqpoint{3.828807in}{0.842848in}}{\pgfqpoint{3.836621in}{0.850661in}}%
\pgfpathcurveto{\pgfqpoint{3.844434in}{0.858475in}}{\pgfqpoint{3.848824in}{0.869074in}}{\pgfqpoint{3.848824in}{0.880124in}}%
\pgfpathcurveto{\pgfqpoint{3.848824in}{0.891174in}}{\pgfqpoint{3.844434in}{0.901773in}}{\pgfqpoint{3.836621in}{0.909587in}}%
\pgfpathcurveto{\pgfqpoint{3.828807in}{0.917400in}}{\pgfqpoint{3.818208in}{0.921791in}}{\pgfqpoint{3.807158in}{0.921791in}}%
\pgfpathcurveto{\pgfqpoint{3.796108in}{0.921791in}}{\pgfqpoint{3.785509in}{0.917400in}}{\pgfqpoint{3.777695in}{0.909587in}}%
\pgfpathcurveto{\pgfqpoint{3.769881in}{0.901773in}}{\pgfqpoint{3.765491in}{0.891174in}}{\pgfqpoint{3.765491in}{0.880124in}}%
\pgfpathcurveto{\pgfqpoint{3.765491in}{0.869074in}}{\pgfqpoint{3.769881in}{0.858475in}}{\pgfqpoint{3.777695in}{0.850661in}}%
\pgfpathcurveto{\pgfqpoint{3.785509in}{0.842848in}}{\pgfqpoint{3.796108in}{0.838457in}}{\pgfqpoint{3.807158in}{0.838457in}}%
\pgfpathlineto{\pgfqpoint{3.807158in}{0.838457in}}%
\pgfpathclose%
\pgfusepath{stroke}%
\end{pgfscope}%
\begin{pgfscope}%
\pgfpathrectangle{\pgfqpoint{0.847223in}{0.554012in}}{\pgfqpoint{6.200000in}{4.530000in}}%
\pgfusepath{clip}%
\pgfsetbuttcap%
\pgfsetroundjoin%
\pgfsetlinewidth{1.003750pt}%
\definecolor{currentstroke}{rgb}{1.000000,0.000000,0.000000}%
\pgfsetstrokecolor{currentstroke}%
\pgfsetdash{}{0pt}%
\pgfpathmoveto{\pgfqpoint{3.812491in}{0.837321in}}%
\pgfpathcurveto{\pgfqpoint{3.823541in}{0.837321in}}{\pgfqpoint{3.834140in}{0.841711in}}{\pgfqpoint{3.841954in}{0.849525in}}%
\pgfpathcurveto{\pgfqpoint{3.849767in}{0.857338in}}{\pgfqpoint{3.854158in}{0.867937in}}{\pgfqpoint{3.854158in}{0.878988in}}%
\pgfpathcurveto{\pgfqpoint{3.854158in}{0.890038in}}{\pgfqpoint{3.849767in}{0.900637in}}{\pgfqpoint{3.841954in}{0.908450in}}%
\pgfpathcurveto{\pgfqpoint{3.834140in}{0.916264in}}{\pgfqpoint{3.823541in}{0.920654in}}{\pgfqpoint{3.812491in}{0.920654in}}%
\pgfpathcurveto{\pgfqpoint{3.801441in}{0.920654in}}{\pgfqpoint{3.790842in}{0.916264in}}{\pgfqpoint{3.783028in}{0.908450in}}%
\pgfpathcurveto{\pgfqpoint{3.775215in}{0.900637in}}{\pgfqpoint{3.770824in}{0.890038in}}{\pgfqpoint{3.770824in}{0.878988in}}%
\pgfpathcurveto{\pgfqpoint{3.770824in}{0.867937in}}{\pgfqpoint{3.775215in}{0.857338in}}{\pgfqpoint{3.783028in}{0.849525in}}%
\pgfpathcurveto{\pgfqpoint{3.790842in}{0.841711in}}{\pgfqpoint{3.801441in}{0.837321in}}{\pgfqpoint{3.812491in}{0.837321in}}%
\pgfpathlineto{\pgfqpoint{3.812491in}{0.837321in}}%
\pgfpathclose%
\pgfusepath{stroke}%
\end{pgfscope}%
\begin{pgfscope}%
\pgfpathrectangle{\pgfqpoint{0.847223in}{0.554012in}}{\pgfqpoint{6.200000in}{4.530000in}}%
\pgfusepath{clip}%
\pgfsetbuttcap%
\pgfsetroundjoin%
\pgfsetlinewidth{1.003750pt}%
\definecolor{currentstroke}{rgb}{1.000000,0.000000,0.000000}%
\pgfsetstrokecolor{currentstroke}%
\pgfsetdash{}{0pt}%
\pgfpathmoveto{\pgfqpoint{3.817824in}{0.836188in}}%
\pgfpathcurveto{\pgfqpoint{3.828874in}{0.836188in}}{\pgfqpoint{3.839473in}{0.840578in}}{\pgfqpoint{3.847287in}{0.848392in}}%
\pgfpathcurveto{\pgfqpoint{3.855101in}{0.856206in}}{\pgfqpoint{3.859491in}{0.866805in}}{\pgfqpoint{3.859491in}{0.877855in}}%
\pgfpathcurveto{\pgfqpoint{3.859491in}{0.888905in}}{\pgfqpoint{3.855101in}{0.899504in}}{\pgfqpoint{3.847287in}{0.907318in}}%
\pgfpathcurveto{\pgfqpoint{3.839473in}{0.915131in}}{\pgfqpoint{3.828874in}{0.919521in}}{\pgfqpoint{3.817824in}{0.919521in}}%
\pgfpathcurveto{\pgfqpoint{3.806774in}{0.919521in}}{\pgfqpoint{3.796175in}{0.915131in}}{\pgfqpoint{3.788361in}{0.907318in}}%
\pgfpathcurveto{\pgfqpoint{3.780548in}{0.899504in}}{\pgfqpoint{3.776158in}{0.888905in}}{\pgfqpoint{3.776158in}{0.877855in}}%
\pgfpathcurveto{\pgfqpoint{3.776158in}{0.866805in}}{\pgfqpoint{3.780548in}{0.856206in}}{\pgfqpoint{3.788361in}{0.848392in}}%
\pgfpathcurveto{\pgfqpoint{3.796175in}{0.840578in}}{\pgfqpoint{3.806774in}{0.836188in}}{\pgfqpoint{3.817824in}{0.836188in}}%
\pgfpathlineto{\pgfqpoint{3.817824in}{0.836188in}}%
\pgfpathclose%
\pgfusepath{stroke}%
\end{pgfscope}%
\begin{pgfscope}%
\pgfpathrectangle{\pgfqpoint{0.847223in}{0.554012in}}{\pgfqpoint{6.200000in}{4.530000in}}%
\pgfusepath{clip}%
\pgfsetbuttcap%
\pgfsetroundjoin%
\pgfsetlinewidth{1.003750pt}%
\definecolor{currentstroke}{rgb}{1.000000,0.000000,0.000000}%
\pgfsetstrokecolor{currentstroke}%
\pgfsetdash{}{0pt}%
\pgfpathmoveto{\pgfqpoint{3.823157in}{0.835059in}}%
\pgfpathcurveto{\pgfqpoint{3.834208in}{0.835059in}}{\pgfqpoint{3.844807in}{0.839449in}}{\pgfqpoint{3.852620in}{0.847263in}}%
\pgfpathcurveto{\pgfqpoint{3.860434in}{0.855076in}}{\pgfqpoint{3.864824in}{0.865675in}}{\pgfqpoint{3.864824in}{0.876725in}}%
\pgfpathcurveto{\pgfqpoint{3.864824in}{0.887776in}}{\pgfqpoint{3.860434in}{0.898375in}}{\pgfqpoint{3.852620in}{0.906188in}}%
\pgfpathcurveto{\pgfqpoint{3.844807in}{0.914002in}}{\pgfqpoint{3.834208in}{0.918392in}}{\pgfqpoint{3.823157in}{0.918392in}}%
\pgfpathcurveto{\pgfqpoint{3.812107in}{0.918392in}}{\pgfqpoint{3.801508in}{0.914002in}}{\pgfqpoint{3.793695in}{0.906188in}}%
\pgfpathcurveto{\pgfqpoint{3.785881in}{0.898375in}}{\pgfqpoint{3.781491in}{0.887776in}}{\pgfqpoint{3.781491in}{0.876725in}}%
\pgfpathcurveto{\pgfqpoint{3.781491in}{0.865675in}}{\pgfqpoint{3.785881in}{0.855076in}}{\pgfqpoint{3.793695in}{0.847263in}}%
\pgfpathcurveto{\pgfqpoint{3.801508in}{0.839449in}}{\pgfqpoint{3.812107in}{0.835059in}}{\pgfqpoint{3.823157in}{0.835059in}}%
\pgfpathlineto{\pgfqpoint{3.823157in}{0.835059in}}%
\pgfpathclose%
\pgfusepath{stroke}%
\end{pgfscope}%
\begin{pgfscope}%
\pgfpathrectangle{\pgfqpoint{0.847223in}{0.554012in}}{\pgfqpoint{6.200000in}{4.530000in}}%
\pgfusepath{clip}%
\pgfsetbuttcap%
\pgfsetroundjoin%
\pgfsetlinewidth{1.003750pt}%
\definecolor{currentstroke}{rgb}{1.000000,0.000000,0.000000}%
\pgfsetstrokecolor{currentstroke}%
\pgfsetdash{}{0pt}%
\pgfpathmoveto{\pgfqpoint{3.828491in}{0.833933in}}%
\pgfpathcurveto{\pgfqpoint{3.839541in}{0.833933in}}{\pgfqpoint{3.850140in}{0.838323in}}{\pgfqpoint{3.857953in}{0.846137in}}%
\pgfpathcurveto{\pgfqpoint{3.865767in}{0.853950in}}{\pgfqpoint{3.870157in}{0.864549in}}{\pgfqpoint{3.870157in}{0.875599in}}%
\pgfpathcurveto{\pgfqpoint{3.870157in}{0.886650in}}{\pgfqpoint{3.865767in}{0.897249in}}{\pgfqpoint{3.857953in}{0.905062in}}%
\pgfpathcurveto{\pgfqpoint{3.850140in}{0.912876in}}{\pgfqpoint{3.839541in}{0.917266in}}{\pgfqpoint{3.828491in}{0.917266in}}%
\pgfpathcurveto{\pgfqpoint{3.817440in}{0.917266in}}{\pgfqpoint{3.806841in}{0.912876in}}{\pgfqpoint{3.799028in}{0.905062in}}%
\pgfpathcurveto{\pgfqpoint{3.791214in}{0.897249in}}{\pgfqpoint{3.786824in}{0.886650in}}{\pgfqpoint{3.786824in}{0.875599in}}%
\pgfpathcurveto{\pgfqpoint{3.786824in}{0.864549in}}{\pgfqpoint{3.791214in}{0.853950in}}{\pgfqpoint{3.799028in}{0.846137in}}%
\pgfpathcurveto{\pgfqpoint{3.806841in}{0.838323in}}{\pgfqpoint{3.817440in}{0.833933in}}{\pgfqpoint{3.828491in}{0.833933in}}%
\pgfpathlineto{\pgfqpoint{3.828491in}{0.833933in}}%
\pgfpathclose%
\pgfusepath{stroke}%
\end{pgfscope}%
\begin{pgfscope}%
\pgfpathrectangle{\pgfqpoint{0.847223in}{0.554012in}}{\pgfqpoint{6.200000in}{4.530000in}}%
\pgfusepath{clip}%
\pgfsetbuttcap%
\pgfsetroundjoin%
\pgfsetlinewidth{1.003750pt}%
\definecolor{currentstroke}{rgb}{1.000000,0.000000,0.000000}%
\pgfsetstrokecolor{currentstroke}%
\pgfsetdash{}{0pt}%
\pgfpathmoveto{\pgfqpoint{3.833824in}{0.832810in}}%
\pgfpathcurveto{\pgfqpoint{3.844874in}{0.832810in}}{\pgfqpoint{3.855473in}{0.837201in}}{\pgfqpoint{3.863287in}{0.845014in}}%
\pgfpathcurveto{\pgfqpoint{3.871100in}{0.852828in}}{\pgfqpoint{3.875490in}{0.863427in}}{\pgfqpoint{3.875490in}{0.874477in}}%
\pgfpathcurveto{\pgfqpoint{3.875490in}{0.885527in}}{\pgfqpoint{3.871100in}{0.896126in}}{\pgfqpoint{3.863287in}{0.903940in}}%
\pgfpathcurveto{\pgfqpoint{3.855473in}{0.911753in}}{\pgfqpoint{3.844874in}{0.916144in}}{\pgfqpoint{3.833824in}{0.916144in}}%
\pgfpathcurveto{\pgfqpoint{3.822774in}{0.916144in}}{\pgfqpoint{3.812175in}{0.911753in}}{\pgfqpoint{3.804361in}{0.903940in}}%
\pgfpathcurveto{\pgfqpoint{3.796547in}{0.896126in}}{\pgfqpoint{3.792157in}{0.885527in}}{\pgfqpoint{3.792157in}{0.874477in}}%
\pgfpathcurveto{\pgfqpoint{3.792157in}{0.863427in}}{\pgfqpoint{3.796547in}{0.852828in}}{\pgfqpoint{3.804361in}{0.845014in}}%
\pgfpathcurveto{\pgfqpoint{3.812175in}{0.837201in}}{\pgfqpoint{3.822774in}{0.832810in}}{\pgfqpoint{3.833824in}{0.832810in}}%
\pgfpathlineto{\pgfqpoint{3.833824in}{0.832810in}}%
\pgfpathclose%
\pgfusepath{stroke}%
\end{pgfscope}%
\begin{pgfscope}%
\pgfpathrectangle{\pgfqpoint{0.847223in}{0.554012in}}{\pgfqpoint{6.200000in}{4.530000in}}%
\pgfusepath{clip}%
\pgfsetbuttcap%
\pgfsetroundjoin%
\pgfsetlinewidth{1.003750pt}%
\definecolor{currentstroke}{rgb}{1.000000,0.000000,0.000000}%
\pgfsetstrokecolor{currentstroke}%
\pgfsetdash{}{0pt}%
\pgfpathmoveto{\pgfqpoint{3.839157in}{0.831691in}}%
\pgfpathcurveto{\pgfqpoint{3.850207in}{0.831691in}}{\pgfqpoint{3.860806in}{0.836081in}}{\pgfqpoint{3.868620in}{0.843895in}}%
\pgfpathcurveto{\pgfqpoint{3.876433in}{0.851709in}}{\pgfqpoint{3.880824in}{0.862308in}}{\pgfqpoint{3.880824in}{0.873358in}}%
\pgfpathcurveto{\pgfqpoint{3.880824in}{0.884408in}}{\pgfqpoint{3.876433in}{0.895007in}}{\pgfqpoint{3.868620in}{0.902821in}}%
\pgfpathcurveto{\pgfqpoint{3.860806in}{0.910634in}}{\pgfqpoint{3.850207in}{0.915024in}}{\pgfqpoint{3.839157in}{0.915024in}}%
\pgfpathcurveto{\pgfqpoint{3.828107in}{0.915024in}}{\pgfqpoint{3.817508in}{0.910634in}}{\pgfqpoint{3.809694in}{0.902821in}}%
\pgfpathcurveto{\pgfqpoint{3.801881in}{0.895007in}}{\pgfqpoint{3.797490in}{0.884408in}}{\pgfqpoint{3.797490in}{0.873358in}}%
\pgfpathcurveto{\pgfqpoint{3.797490in}{0.862308in}}{\pgfqpoint{3.801881in}{0.851709in}}{\pgfqpoint{3.809694in}{0.843895in}}%
\pgfpathcurveto{\pgfqpoint{3.817508in}{0.836081in}}{\pgfqpoint{3.828107in}{0.831691in}}{\pgfqpoint{3.839157in}{0.831691in}}%
\pgfpathlineto{\pgfqpoint{3.839157in}{0.831691in}}%
\pgfpathclose%
\pgfusepath{stroke}%
\end{pgfscope}%
\begin{pgfscope}%
\pgfpathrectangle{\pgfqpoint{0.847223in}{0.554012in}}{\pgfqpoint{6.200000in}{4.530000in}}%
\pgfusepath{clip}%
\pgfsetbuttcap%
\pgfsetroundjoin%
\pgfsetlinewidth{1.003750pt}%
\definecolor{currentstroke}{rgb}{1.000000,0.000000,0.000000}%
\pgfsetstrokecolor{currentstroke}%
\pgfsetdash{}{0pt}%
\pgfpathmoveto{\pgfqpoint{3.844490in}{0.830575in}}%
\pgfpathcurveto{\pgfqpoint{3.855540in}{0.830575in}}{\pgfqpoint{3.866139in}{0.834966in}}{\pgfqpoint{3.873953in}{0.842779in}}%
\pgfpathcurveto{\pgfqpoint{3.881767in}{0.850593in}}{\pgfqpoint{3.886157in}{0.861192in}}{\pgfqpoint{3.886157in}{0.872242in}}%
\pgfpathcurveto{\pgfqpoint{3.886157in}{0.883292in}}{\pgfqpoint{3.881767in}{0.893891in}}{\pgfqpoint{3.873953in}{0.901705in}}%
\pgfpathcurveto{\pgfqpoint{3.866139in}{0.909518in}}{\pgfqpoint{3.855540in}{0.913909in}}{\pgfqpoint{3.844490in}{0.913909in}}%
\pgfpathcurveto{\pgfqpoint{3.833440in}{0.913909in}}{\pgfqpoint{3.822841in}{0.909518in}}{\pgfqpoint{3.815027in}{0.901705in}}%
\pgfpathcurveto{\pgfqpoint{3.807214in}{0.893891in}}{\pgfqpoint{3.802824in}{0.883292in}}{\pgfqpoint{3.802824in}{0.872242in}}%
\pgfpathcurveto{\pgfqpoint{3.802824in}{0.861192in}}{\pgfqpoint{3.807214in}{0.850593in}}{\pgfqpoint{3.815027in}{0.842779in}}%
\pgfpathcurveto{\pgfqpoint{3.822841in}{0.834966in}}{\pgfqpoint{3.833440in}{0.830575in}}{\pgfqpoint{3.844490in}{0.830575in}}%
\pgfpathlineto{\pgfqpoint{3.844490in}{0.830575in}}%
\pgfpathclose%
\pgfusepath{stroke}%
\end{pgfscope}%
\begin{pgfscope}%
\pgfpathrectangle{\pgfqpoint{0.847223in}{0.554012in}}{\pgfqpoint{6.200000in}{4.530000in}}%
\pgfusepath{clip}%
\pgfsetbuttcap%
\pgfsetroundjoin%
\pgfsetlinewidth{1.003750pt}%
\definecolor{currentstroke}{rgb}{1.000000,0.000000,0.000000}%
\pgfsetstrokecolor{currentstroke}%
\pgfsetdash{}{0pt}%
\pgfpathmoveto{\pgfqpoint{3.849823in}{0.829463in}}%
\pgfpathcurveto{\pgfqpoint{3.860874in}{0.829463in}}{\pgfqpoint{3.871473in}{0.833853in}}{\pgfqpoint{3.879286in}{0.841667in}}%
\pgfpathcurveto{\pgfqpoint{3.887100in}{0.849481in}}{\pgfqpoint{3.891490in}{0.860080in}}{\pgfqpoint{3.891490in}{0.871130in}}%
\pgfpathcurveto{\pgfqpoint{3.891490in}{0.882180in}}{\pgfqpoint{3.887100in}{0.892779in}}{\pgfqpoint{3.879286in}{0.900592in}}%
\pgfpathcurveto{\pgfqpoint{3.871473in}{0.908406in}}{\pgfqpoint{3.860874in}{0.912796in}}{\pgfqpoint{3.849823in}{0.912796in}}%
\pgfpathcurveto{\pgfqpoint{3.838773in}{0.912796in}}{\pgfqpoint{3.828174in}{0.908406in}}{\pgfqpoint{3.820361in}{0.900592in}}%
\pgfpathcurveto{\pgfqpoint{3.812547in}{0.892779in}}{\pgfqpoint{3.808157in}{0.882180in}}{\pgfqpoint{3.808157in}{0.871130in}}%
\pgfpathcurveto{\pgfqpoint{3.808157in}{0.860080in}}{\pgfqpoint{3.812547in}{0.849481in}}{\pgfqpoint{3.820361in}{0.841667in}}%
\pgfpathcurveto{\pgfqpoint{3.828174in}{0.833853in}}{\pgfqpoint{3.838773in}{0.829463in}}{\pgfqpoint{3.849823in}{0.829463in}}%
\pgfpathlineto{\pgfqpoint{3.849823in}{0.829463in}}%
\pgfpathclose%
\pgfusepath{stroke}%
\end{pgfscope}%
\begin{pgfscope}%
\pgfpathrectangle{\pgfqpoint{0.847223in}{0.554012in}}{\pgfqpoint{6.200000in}{4.530000in}}%
\pgfusepath{clip}%
\pgfsetbuttcap%
\pgfsetroundjoin%
\pgfsetlinewidth{1.003750pt}%
\definecolor{currentstroke}{rgb}{1.000000,0.000000,0.000000}%
\pgfsetstrokecolor{currentstroke}%
\pgfsetdash{}{0pt}%
\pgfpathmoveto{\pgfqpoint{3.855157in}{0.828354in}}%
\pgfpathcurveto{\pgfqpoint{3.866207in}{0.828354in}}{\pgfqpoint{3.876806in}{0.832744in}}{\pgfqpoint{3.884619in}{0.840558in}}%
\pgfpathcurveto{\pgfqpoint{3.892433in}{0.848372in}}{\pgfqpoint{3.896823in}{0.858971in}}{\pgfqpoint{3.896823in}{0.870021in}}%
\pgfpathcurveto{\pgfqpoint{3.896823in}{0.881071in}}{\pgfqpoint{3.892433in}{0.891670in}}{\pgfqpoint{3.884619in}{0.899483in}}%
\pgfpathcurveto{\pgfqpoint{3.876806in}{0.907297in}}{\pgfqpoint{3.866207in}{0.911687in}}{\pgfqpoint{3.855157in}{0.911687in}}%
\pgfpathcurveto{\pgfqpoint{3.844107in}{0.911687in}}{\pgfqpoint{3.833508in}{0.907297in}}{\pgfqpoint{3.825694in}{0.899483in}}%
\pgfpathcurveto{\pgfqpoint{3.817880in}{0.891670in}}{\pgfqpoint{3.813490in}{0.881071in}}{\pgfqpoint{3.813490in}{0.870021in}}%
\pgfpathcurveto{\pgfqpoint{3.813490in}{0.858971in}}{\pgfqpoint{3.817880in}{0.848372in}}{\pgfqpoint{3.825694in}{0.840558in}}%
\pgfpathcurveto{\pgfqpoint{3.833508in}{0.832744in}}{\pgfqpoint{3.844107in}{0.828354in}}{\pgfqpoint{3.855157in}{0.828354in}}%
\pgfpathlineto{\pgfqpoint{3.855157in}{0.828354in}}%
\pgfpathclose%
\pgfusepath{stroke}%
\end{pgfscope}%
\begin{pgfscope}%
\pgfpathrectangle{\pgfqpoint{0.847223in}{0.554012in}}{\pgfqpoint{6.200000in}{4.530000in}}%
\pgfusepath{clip}%
\pgfsetbuttcap%
\pgfsetroundjoin%
\pgfsetlinewidth{1.003750pt}%
\definecolor{currentstroke}{rgb}{1.000000,0.000000,0.000000}%
\pgfsetstrokecolor{currentstroke}%
\pgfsetdash{}{0pt}%
\pgfpathmoveto{\pgfqpoint{3.860490in}{0.827248in}}%
\pgfpathcurveto{\pgfqpoint{3.871540in}{0.827248in}}{\pgfqpoint{3.882139in}{0.831639in}}{\pgfqpoint{3.889953in}{0.839452in}}%
\pgfpathcurveto{\pgfqpoint{3.897766in}{0.847266in}}{\pgfqpoint{3.902157in}{0.857865in}}{\pgfqpoint{3.902157in}{0.868915in}}%
\pgfpathcurveto{\pgfqpoint{3.902157in}{0.879965in}}{\pgfqpoint{3.897766in}{0.890564in}}{\pgfqpoint{3.889953in}{0.898378in}}%
\pgfpathcurveto{\pgfqpoint{3.882139in}{0.906191in}}{\pgfqpoint{3.871540in}{0.910582in}}{\pgfqpoint{3.860490in}{0.910582in}}%
\pgfpathcurveto{\pgfqpoint{3.849440in}{0.910582in}}{\pgfqpoint{3.838841in}{0.906191in}}{\pgfqpoint{3.831027in}{0.898378in}}%
\pgfpathcurveto{\pgfqpoint{3.823213in}{0.890564in}}{\pgfqpoint{3.818823in}{0.879965in}}{\pgfqpoint{3.818823in}{0.868915in}}%
\pgfpathcurveto{\pgfqpoint{3.818823in}{0.857865in}}{\pgfqpoint{3.823213in}{0.847266in}}{\pgfqpoint{3.831027in}{0.839452in}}%
\pgfpathcurveto{\pgfqpoint{3.838841in}{0.831639in}}{\pgfqpoint{3.849440in}{0.827248in}}{\pgfqpoint{3.860490in}{0.827248in}}%
\pgfpathlineto{\pgfqpoint{3.860490in}{0.827248in}}%
\pgfpathclose%
\pgfusepath{stroke}%
\end{pgfscope}%
\begin{pgfscope}%
\pgfpathrectangle{\pgfqpoint{0.847223in}{0.554012in}}{\pgfqpoint{6.200000in}{4.530000in}}%
\pgfusepath{clip}%
\pgfsetbuttcap%
\pgfsetroundjoin%
\pgfsetlinewidth{1.003750pt}%
\definecolor{currentstroke}{rgb}{1.000000,0.000000,0.000000}%
\pgfsetstrokecolor{currentstroke}%
\pgfsetdash{}{0pt}%
\pgfpathmoveto{\pgfqpoint{3.865823in}{0.826146in}}%
\pgfpathcurveto{\pgfqpoint{3.876873in}{0.826146in}}{\pgfqpoint{3.887472in}{0.830536in}}{\pgfqpoint{3.895286in}{0.838350in}}%
\pgfpathcurveto{\pgfqpoint{3.903100in}{0.846164in}}{\pgfqpoint{3.907490in}{0.856763in}}{\pgfqpoint{3.907490in}{0.867813in}}%
\pgfpathcurveto{\pgfqpoint{3.907490in}{0.878863in}}{\pgfqpoint{3.903100in}{0.889462in}}{\pgfqpoint{3.895286in}{0.897276in}}%
\pgfpathcurveto{\pgfqpoint{3.887472in}{0.905089in}}{\pgfqpoint{3.876873in}{0.909479in}}{\pgfqpoint{3.865823in}{0.909479in}}%
\pgfpathcurveto{\pgfqpoint{3.854773in}{0.909479in}}{\pgfqpoint{3.844174in}{0.905089in}}{\pgfqpoint{3.836360in}{0.897276in}}%
\pgfpathcurveto{\pgfqpoint{3.828547in}{0.889462in}}{\pgfqpoint{3.824156in}{0.878863in}}{\pgfqpoint{3.824156in}{0.867813in}}%
\pgfpathcurveto{\pgfqpoint{3.824156in}{0.856763in}}{\pgfqpoint{3.828547in}{0.846164in}}{\pgfqpoint{3.836360in}{0.838350in}}%
\pgfpathcurveto{\pgfqpoint{3.844174in}{0.830536in}}{\pgfqpoint{3.854773in}{0.826146in}}{\pgfqpoint{3.865823in}{0.826146in}}%
\pgfpathlineto{\pgfqpoint{3.865823in}{0.826146in}}%
\pgfpathclose%
\pgfusepath{stroke}%
\end{pgfscope}%
\begin{pgfscope}%
\pgfpathrectangle{\pgfqpoint{0.847223in}{0.554012in}}{\pgfqpoint{6.200000in}{4.530000in}}%
\pgfusepath{clip}%
\pgfsetbuttcap%
\pgfsetroundjoin%
\pgfsetlinewidth{1.003750pt}%
\definecolor{currentstroke}{rgb}{1.000000,0.000000,0.000000}%
\pgfsetstrokecolor{currentstroke}%
\pgfsetdash{}{0pt}%
\pgfpathmoveto{\pgfqpoint{3.871156in}{0.825047in}}%
\pgfpathcurveto{\pgfqpoint{3.882206in}{0.825047in}}{\pgfqpoint{3.892805in}{0.829437in}}{\pgfqpoint{3.900619in}{0.837251in}}%
\pgfpathcurveto{\pgfqpoint{3.908433in}{0.845065in}}{\pgfqpoint{3.912823in}{0.855664in}}{\pgfqpoint{3.912823in}{0.866714in}}%
\pgfpathcurveto{\pgfqpoint{3.912823in}{0.877764in}}{\pgfqpoint{3.908433in}{0.888363in}}{\pgfqpoint{3.900619in}{0.896176in}}%
\pgfpathcurveto{\pgfqpoint{3.892805in}{0.903990in}}{\pgfqpoint{3.882206in}{0.908380in}}{\pgfqpoint{3.871156in}{0.908380in}}%
\pgfpathcurveto{\pgfqpoint{3.860106in}{0.908380in}}{\pgfqpoint{3.849507in}{0.903990in}}{\pgfqpoint{3.841694in}{0.896176in}}%
\pgfpathcurveto{\pgfqpoint{3.833880in}{0.888363in}}{\pgfqpoint{3.829490in}{0.877764in}}{\pgfqpoint{3.829490in}{0.866714in}}%
\pgfpathcurveto{\pgfqpoint{3.829490in}{0.855664in}}{\pgfqpoint{3.833880in}{0.845065in}}{\pgfqpoint{3.841694in}{0.837251in}}%
\pgfpathcurveto{\pgfqpoint{3.849507in}{0.829437in}}{\pgfqpoint{3.860106in}{0.825047in}}{\pgfqpoint{3.871156in}{0.825047in}}%
\pgfpathlineto{\pgfqpoint{3.871156in}{0.825047in}}%
\pgfpathclose%
\pgfusepath{stroke}%
\end{pgfscope}%
\begin{pgfscope}%
\pgfpathrectangle{\pgfqpoint{0.847223in}{0.554012in}}{\pgfqpoint{6.200000in}{4.530000in}}%
\pgfusepath{clip}%
\pgfsetbuttcap%
\pgfsetroundjoin%
\pgfsetlinewidth{1.003750pt}%
\definecolor{currentstroke}{rgb}{1.000000,0.000000,0.000000}%
\pgfsetstrokecolor{currentstroke}%
\pgfsetdash{}{0pt}%
\pgfpathmoveto{\pgfqpoint{3.876490in}{0.823951in}}%
\pgfpathcurveto{\pgfqpoint{3.887540in}{0.823951in}}{\pgfqpoint{3.898139in}{0.828342in}}{\pgfqpoint{3.905952in}{0.836155in}}%
\pgfpathcurveto{\pgfqpoint{3.913766in}{0.843969in}}{\pgfqpoint{3.918156in}{0.854568in}}{\pgfqpoint{3.918156in}{0.865618in}}%
\pgfpathcurveto{\pgfqpoint{3.918156in}{0.876668in}}{\pgfqpoint{3.913766in}{0.887267in}}{\pgfqpoint{3.905952in}{0.895081in}}%
\pgfpathcurveto{\pgfqpoint{3.898139in}{0.902894in}}{\pgfqpoint{3.887540in}{0.907285in}}{\pgfqpoint{3.876490in}{0.907285in}}%
\pgfpathcurveto{\pgfqpoint{3.865439in}{0.907285in}}{\pgfqpoint{3.854840in}{0.902894in}}{\pgfqpoint{3.847027in}{0.895081in}}%
\pgfpathcurveto{\pgfqpoint{3.839213in}{0.887267in}}{\pgfqpoint{3.834823in}{0.876668in}}{\pgfqpoint{3.834823in}{0.865618in}}%
\pgfpathcurveto{\pgfqpoint{3.834823in}{0.854568in}}{\pgfqpoint{3.839213in}{0.843969in}}{\pgfqpoint{3.847027in}{0.836155in}}%
\pgfpathcurveto{\pgfqpoint{3.854840in}{0.828342in}}{\pgfqpoint{3.865439in}{0.823951in}}{\pgfqpoint{3.876490in}{0.823951in}}%
\pgfpathlineto{\pgfqpoint{3.876490in}{0.823951in}}%
\pgfpathclose%
\pgfusepath{stroke}%
\end{pgfscope}%
\begin{pgfscope}%
\pgfpathrectangle{\pgfqpoint{0.847223in}{0.554012in}}{\pgfqpoint{6.200000in}{4.530000in}}%
\pgfusepath{clip}%
\pgfsetbuttcap%
\pgfsetroundjoin%
\pgfsetlinewidth{1.003750pt}%
\definecolor{currentstroke}{rgb}{1.000000,0.000000,0.000000}%
\pgfsetstrokecolor{currentstroke}%
\pgfsetdash{}{0pt}%
\pgfpathmoveto{\pgfqpoint{3.881823in}{0.822859in}}%
\pgfpathcurveto{\pgfqpoint{3.892873in}{0.822859in}}{\pgfqpoint{3.903472in}{0.827249in}}{\pgfqpoint{3.911286in}{0.835063in}}%
\pgfpathcurveto{\pgfqpoint{3.919099in}{0.842876in}}{\pgfqpoint{3.923489in}{0.853475in}}{\pgfqpoint{3.923489in}{0.864526in}}%
\pgfpathcurveto{\pgfqpoint{3.923489in}{0.875576in}}{\pgfqpoint{3.919099in}{0.886175in}}{\pgfqpoint{3.911286in}{0.893988in}}%
\pgfpathcurveto{\pgfqpoint{3.903472in}{0.901802in}}{\pgfqpoint{3.892873in}{0.906192in}}{\pgfqpoint{3.881823in}{0.906192in}}%
\pgfpathcurveto{\pgfqpoint{3.870773in}{0.906192in}}{\pgfqpoint{3.860174in}{0.901802in}}{\pgfqpoint{3.852360in}{0.893988in}}%
\pgfpathcurveto{\pgfqpoint{3.844546in}{0.886175in}}{\pgfqpoint{3.840156in}{0.875576in}}{\pgfqpoint{3.840156in}{0.864526in}}%
\pgfpathcurveto{\pgfqpoint{3.840156in}{0.853475in}}{\pgfqpoint{3.844546in}{0.842876in}}{\pgfqpoint{3.852360in}{0.835063in}}%
\pgfpathcurveto{\pgfqpoint{3.860174in}{0.827249in}}{\pgfqpoint{3.870773in}{0.822859in}}{\pgfqpoint{3.881823in}{0.822859in}}%
\pgfpathlineto{\pgfqpoint{3.881823in}{0.822859in}}%
\pgfpathclose%
\pgfusepath{stroke}%
\end{pgfscope}%
\begin{pgfscope}%
\pgfpathrectangle{\pgfqpoint{0.847223in}{0.554012in}}{\pgfqpoint{6.200000in}{4.530000in}}%
\pgfusepath{clip}%
\pgfsetbuttcap%
\pgfsetroundjoin%
\pgfsetlinewidth{1.003750pt}%
\definecolor{currentstroke}{rgb}{1.000000,0.000000,0.000000}%
\pgfsetstrokecolor{currentstroke}%
\pgfsetdash{}{0pt}%
\pgfpathmoveto{\pgfqpoint{3.887156in}{0.821770in}}%
\pgfpathcurveto{\pgfqpoint{3.898206in}{0.821770in}}{\pgfqpoint{3.908805in}{0.826160in}}{\pgfqpoint{3.916619in}{0.833974in}}%
\pgfpathcurveto{\pgfqpoint{3.924432in}{0.841787in}}{\pgfqpoint{3.928823in}{0.852386in}}{\pgfqpoint{3.928823in}{0.863436in}}%
\pgfpathcurveto{\pgfqpoint{3.928823in}{0.874487in}}{\pgfqpoint{3.924432in}{0.885086in}}{\pgfqpoint{3.916619in}{0.892899in}}%
\pgfpathcurveto{\pgfqpoint{3.908805in}{0.900713in}}{\pgfqpoint{3.898206in}{0.905103in}}{\pgfqpoint{3.887156in}{0.905103in}}%
\pgfpathcurveto{\pgfqpoint{3.876106in}{0.905103in}}{\pgfqpoint{3.865507in}{0.900713in}}{\pgfqpoint{3.857693in}{0.892899in}}%
\pgfpathcurveto{\pgfqpoint{3.849880in}{0.885086in}}{\pgfqpoint{3.845489in}{0.874487in}}{\pgfqpoint{3.845489in}{0.863436in}}%
\pgfpathcurveto{\pgfqpoint{3.845489in}{0.852386in}}{\pgfqpoint{3.849880in}{0.841787in}}{\pgfqpoint{3.857693in}{0.833974in}}%
\pgfpathcurveto{\pgfqpoint{3.865507in}{0.826160in}}{\pgfqpoint{3.876106in}{0.821770in}}{\pgfqpoint{3.887156in}{0.821770in}}%
\pgfpathlineto{\pgfqpoint{3.887156in}{0.821770in}}%
\pgfpathclose%
\pgfusepath{stroke}%
\end{pgfscope}%
\begin{pgfscope}%
\pgfpathrectangle{\pgfqpoint{0.847223in}{0.554012in}}{\pgfqpoint{6.200000in}{4.530000in}}%
\pgfusepath{clip}%
\pgfsetbuttcap%
\pgfsetroundjoin%
\pgfsetlinewidth{1.003750pt}%
\definecolor{currentstroke}{rgb}{1.000000,0.000000,0.000000}%
\pgfsetstrokecolor{currentstroke}%
\pgfsetdash{}{0pt}%
\pgfpathmoveto{\pgfqpoint{3.892489in}{0.820684in}}%
\pgfpathcurveto{\pgfqpoint{3.903539in}{0.820684in}}{\pgfqpoint{3.914138in}{0.825074in}}{\pgfqpoint{3.921952in}{0.832888in}}%
\pgfpathcurveto{\pgfqpoint{3.929766in}{0.840701in}}{\pgfqpoint{3.934156in}{0.851300in}}{\pgfqpoint{3.934156in}{0.862351in}}%
\pgfpathcurveto{\pgfqpoint{3.934156in}{0.873401in}}{\pgfqpoint{3.929766in}{0.884000in}}{\pgfqpoint{3.921952in}{0.891813in}}%
\pgfpathcurveto{\pgfqpoint{3.914138in}{0.899627in}}{\pgfqpoint{3.903539in}{0.904017in}}{\pgfqpoint{3.892489in}{0.904017in}}%
\pgfpathcurveto{\pgfqpoint{3.881439in}{0.904017in}}{\pgfqpoint{3.870840in}{0.899627in}}{\pgfqpoint{3.863026in}{0.891813in}}%
\pgfpathcurveto{\pgfqpoint{3.855213in}{0.884000in}}{\pgfqpoint{3.850823in}{0.873401in}}{\pgfqpoint{3.850823in}{0.862351in}}%
\pgfpathcurveto{\pgfqpoint{3.850823in}{0.851300in}}{\pgfqpoint{3.855213in}{0.840701in}}{\pgfqpoint{3.863026in}{0.832888in}}%
\pgfpathcurveto{\pgfqpoint{3.870840in}{0.825074in}}{\pgfqpoint{3.881439in}{0.820684in}}{\pgfqpoint{3.892489in}{0.820684in}}%
\pgfpathlineto{\pgfqpoint{3.892489in}{0.820684in}}%
\pgfpathclose%
\pgfusepath{stroke}%
\end{pgfscope}%
\begin{pgfscope}%
\pgfpathrectangle{\pgfqpoint{0.847223in}{0.554012in}}{\pgfqpoint{6.200000in}{4.530000in}}%
\pgfusepath{clip}%
\pgfsetbuttcap%
\pgfsetroundjoin%
\pgfsetlinewidth{1.003750pt}%
\definecolor{currentstroke}{rgb}{1.000000,0.000000,0.000000}%
\pgfsetstrokecolor{currentstroke}%
\pgfsetdash{}{0pt}%
\pgfpathmoveto{\pgfqpoint{3.897822in}{0.819601in}}%
\pgfpathcurveto{\pgfqpoint{3.908873in}{0.819601in}}{\pgfqpoint{3.919472in}{0.823991in}}{\pgfqpoint{3.927285in}{0.831805in}}%
\pgfpathcurveto{\pgfqpoint{3.935099in}{0.839619in}}{\pgfqpoint{3.939489in}{0.850218in}}{\pgfqpoint{3.939489in}{0.861268in}}%
\pgfpathcurveto{\pgfqpoint{3.939489in}{0.872318in}}{\pgfqpoint{3.935099in}{0.882917in}}{\pgfqpoint{3.927285in}{0.890731in}}%
\pgfpathcurveto{\pgfqpoint{3.919472in}{0.898544in}}{\pgfqpoint{3.908873in}{0.902935in}}{\pgfqpoint{3.897822in}{0.902935in}}%
\pgfpathcurveto{\pgfqpoint{3.886772in}{0.902935in}}{\pgfqpoint{3.876173in}{0.898544in}}{\pgfqpoint{3.868360in}{0.890731in}}%
\pgfpathcurveto{\pgfqpoint{3.860546in}{0.882917in}}{\pgfqpoint{3.856156in}{0.872318in}}{\pgfqpoint{3.856156in}{0.861268in}}%
\pgfpathcurveto{\pgfqpoint{3.856156in}{0.850218in}}{\pgfqpoint{3.860546in}{0.839619in}}{\pgfqpoint{3.868360in}{0.831805in}}%
\pgfpathcurveto{\pgfqpoint{3.876173in}{0.823991in}}{\pgfqpoint{3.886772in}{0.819601in}}{\pgfqpoint{3.897822in}{0.819601in}}%
\pgfpathlineto{\pgfqpoint{3.897822in}{0.819601in}}%
\pgfpathclose%
\pgfusepath{stroke}%
\end{pgfscope}%
\begin{pgfscope}%
\pgfpathrectangle{\pgfqpoint{0.847223in}{0.554012in}}{\pgfqpoint{6.200000in}{4.530000in}}%
\pgfusepath{clip}%
\pgfsetbuttcap%
\pgfsetroundjoin%
\pgfsetlinewidth{1.003750pt}%
\definecolor{currentstroke}{rgb}{1.000000,0.000000,0.000000}%
\pgfsetstrokecolor{currentstroke}%
\pgfsetdash{}{0pt}%
\pgfpathmoveto{\pgfqpoint{3.903156in}{0.818522in}}%
\pgfpathcurveto{\pgfqpoint{3.914206in}{0.818522in}}{\pgfqpoint{3.924805in}{0.822912in}}{\pgfqpoint{3.932618in}{0.830726in}}%
\pgfpathcurveto{\pgfqpoint{3.940432in}{0.838539in}}{\pgfqpoint{3.944822in}{0.849138in}}{\pgfqpoint{3.944822in}{0.860188in}}%
\pgfpathcurveto{\pgfqpoint{3.944822in}{0.871239in}}{\pgfqpoint{3.940432in}{0.881838in}}{\pgfqpoint{3.932618in}{0.889651in}}%
\pgfpathcurveto{\pgfqpoint{3.924805in}{0.897465in}}{\pgfqpoint{3.914206in}{0.901855in}}{\pgfqpoint{3.903156in}{0.901855in}}%
\pgfpathcurveto{\pgfqpoint{3.892105in}{0.901855in}}{\pgfqpoint{3.881506in}{0.897465in}}{\pgfqpoint{3.873693in}{0.889651in}}%
\pgfpathcurveto{\pgfqpoint{3.865879in}{0.881838in}}{\pgfqpoint{3.861489in}{0.871239in}}{\pgfqpoint{3.861489in}{0.860188in}}%
\pgfpathcurveto{\pgfqpoint{3.861489in}{0.849138in}}{\pgfqpoint{3.865879in}{0.838539in}}{\pgfqpoint{3.873693in}{0.830726in}}%
\pgfpathcurveto{\pgfqpoint{3.881506in}{0.822912in}}{\pgfqpoint{3.892105in}{0.818522in}}{\pgfqpoint{3.903156in}{0.818522in}}%
\pgfpathlineto{\pgfqpoint{3.903156in}{0.818522in}}%
\pgfpathclose%
\pgfusepath{stroke}%
\end{pgfscope}%
\begin{pgfscope}%
\pgfpathrectangle{\pgfqpoint{0.847223in}{0.554012in}}{\pgfqpoint{6.200000in}{4.530000in}}%
\pgfusepath{clip}%
\pgfsetbuttcap%
\pgfsetroundjoin%
\pgfsetlinewidth{1.003750pt}%
\definecolor{currentstroke}{rgb}{1.000000,0.000000,0.000000}%
\pgfsetstrokecolor{currentstroke}%
\pgfsetdash{}{0pt}%
\pgfpathmoveto{\pgfqpoint{3.908489in}{0.817446in}}%
\pgfpathcurveto{\pgfqpoint{3.919539in}{0.817446in}}{\pgfqpoint{3.930138in}{0.821836in}}{\pgfqpoint{3.937952in}{0.829649in}}%
\pgfpathcurveto{\pgfqpoint{3.945765in}{0.837463in}}{\pgfqpoint{3.950156in}{0.848062in}}{\pgfqpoint{3.950156in}{0.859112in}}%
\pgfpathcurveto{\pgfqpoint{3.950156in}{0.870162in}}{\pgfqpoint{3.945765in}{0.880761in}}{\pgfqpoint{3.937952in}{0.888575in}}%
\pgfpathcurveto{\pgfqpoint{3.930138in}{0.896389in}}{\pgfqpoint{3.919539in}{0.900779in}}{\pgfqpoint{3.908489in}{0.900779in}}%
\pgfpathcurveto{\pgfqpoint{3.897439in}{0.900779in}}{\pgfqpoint{3.886840in}{0.896389in}}{\pgfqpoint{3.879026in}{0.888575in}}%
\pgfpathcurveto{\pgfqpoint{3.871212in}{0.880761in}}{\pgfqpoint{3.866822in}{0.870162in}}{\pgfqpoint{3.866822in}{0.859112in}}%
\pgfpathcurveto{\pgfqpoint{3.866822in}{0.848062in}}{\pgfqpoint{3.871212in}{0.837463in}}{\pgfqpoint{3.879026in}{0.829649in}}%
\pgfpathcurveto{\pgfqpoint{3.886840in}{0.821836in}}{\pgfqpoint{3.897439in}{0.817446in}}{\pgfqpoint{3.908489in}{0.817446in}}%
\pgfpathlineto{\pgfqpoint{3.908489in}{0.817446in}}%
\pgfpathclose%
\pgfusepath{stroke}%
\end{pgfscope}%
\begin{pgfscope}%
\pgfpathrectangle{\pgfqpoint{0.847223in}{0.554012in}}{\pgfqpoint{6.200000in}{4.530000in}}%
\pgfusepath{clip}%
\pgfsetbuttcap%
\pgfsetroundjoin%
\pgfsetlinewidth{1.003750pt}%
\definecolor{currentstroke}{rgb}{1.000000,0.000000,0.000000}%
\pgfsetstrokecolor{currentstroke}%
\pgfsetdash{}{0pt}%
\pgfpathmoveto{\pgfqpoint{3.913822in}{0.816373in}}%
\pgfpathcurveto{\pgfqpoint{3.924872in}{0.816373in}}{\pgfqpoint{3.935471in}{0.820763in}}{\pgfqpoint{3.943285in}{0.828576in}}%
\pgfpathcurveto{\pgfqpoint{3.951098in}{0.836390in}}{\pgfqpoint{3.955489in}{0.846989in}}{\pgfqpoint{3.955489in}{0.858039in}}%
\pgfpathcurveto{\pgfqpoint{3.955489in}{0.869089in}}{\pgfqpoint{3.951098in}{0.879688in}}{\pgfqpoint{3.943285in}{0.887502in}}%
\pgfpathcurveto{\pgfqpoint{3.935471in}{0.895316in}}{\pgfqpoint{3.924872in}{0.899706in}}{\pgfqpoint{3.913822in}{0.899706in}}%
\pgfpathcurveto{\pgfqpoint{3.902772in}{0.899706in}}{\pgfqpoint{3.892173in}{0.895316in}}{\pgfqpoint{3.884359in}{0.887502in}}%
\pgfpathcurveto{\pgfqpoint{3.876546in}{0.879688in}}{\pgfqpoint{3.872155in}{0.869089in}}{\pgfqpoint{3.872155in}{0.858039in}}%
\pgfpathcurveto{\pgfqpoint{3.872155in}{0.846989in}}{\pgfqpoint{3.876546in}{0.836390in}}{\pgfqpoint{3.884359in}{0.828576in}}%
\pgfpathcurveto{\pgfqpoint{3.892173in}{0.820763in}}{\pgfqpoint{3.902772in}{0.816373in}}{\pgfqpoint{3.913822in}{0.816373in}}%
\pgfpathlineto{\pgfqpoint{3.913822in}{0.816373in}}%
\pgfpathclose%
\pgfusepath{stroke}%
\end{pgfscope}%
\begin{pgfscope}%
\pgfpathrectangle{\pgfqpoint{0.847223in}{0.554012in}}{\pgfqpoint{6.200000in}{4.530000in}}%
\pgfusepath{clip}%
\pgfsetbuttcap%
\pgfsetroundjoin%
\pgfsetlinewidth{1.003750pt}%
\definecolor{currentstroke}{rgb}{1.000000,0.000000,0.000000}%
\pgfsetstrokecolor{currentstroke}%
\pgfsetdash{}{0pt}%
\pgfpathmoveto{\pgfqpoint{3.919155in}{0.815303in}}%
\pgfpathcurveto{\pgfqpoint{3.930205in}{0.815303in}}{\pgfqpoint{3.940804in}{0.819693in}}{\pgfqpoint{3.948618in}{0.827507in}}%
\pgfpathcurveto{\pgfqpoint{3.956432in}{0.835320in}}{\pgfqpoint{3.960822in}{0.845919in}}{\pgfqpoint{3.960822in}{0.856969in}}%
\pgfpathcurveto{\pgfqpoint{3.960822in}{0.868019in}}{\pgfqpoint{3.956432in}{0.878619in}}{\pgfqpoint{3.948618in}{0.886432in}}%
\pgfpathcurveto{\pgfqpoint{3.940804in}{0.894246in}}{\pgfqpoint{3.930205in}{0.898636in}}{\pgfqpoint{3.919155in}{0.898636in}}%
\pgfpathcurveto{\pgfqpoint{3.908105in}{0.898636in}}{\pgfqpoint{3.897506in}{0.894246in}}{\pgfqpoint{3.889692in}{0.886432in}}%
\pgfpathcurveto{\pgfqpoint{3.881879in}{0.878619in}}{\pgfqpoint{3.877489in}{0.868019in}}{\pgfqpoint{3.877489in}{0.856969in}}%
\pgfpathcurveto{\pgfqpoint{3.877489in}{0.845919in}}{\pgfqpoint{3.881879in}{0.835320in}}{\pgfqpoint{3.889692in}{0.827507in}}%
\pgfpathcurveto{\pgfqpoint{3.897506in}{0.819693in}}{\pgfqpoint{3.908105in}{0.815303in}}{\pgfqpoint{3.919155in}{0.815303in}}%
\pgfpathlineto{\pgfqpoint{3.919155in}{0.815303in}}%
\pgfpathclose%
\pgfusepath{stroke}%
\end{pgfscope}%
\begin{pgfscope}%
\pgfpathrectangle{\pgfqpoint{0.847223in}{0.554012in}}{\pgfqpoint{6.200000in}{4.530000in}}%
\pgfusepath{clip}%
\pgfsetbuttcap%
\pgfsetroundjoin%
\pgfsetlinewidth{1.003750pt}%
\definecolor{currentstroke}{rgb}{1.000000,0.000000,0.000000}%
\pgfsetstrokecolor{currentstroke}%
\pgfsetdash{}{0pt}%
\pgfpathmoveto{\pgfqpoint{3.924488in}{0.814236in}}%
\pgfpathcurveto{\pgfqpoint{3.935539in}{0.814236in}}{\pgfqpoint{3.946138in}{0.818626in}}{\pgfqpoint{3.953951in}{0.826440in}}%
\pgfpathcurveto{\pgfqpoint{3.961765in}{0.834254in}}{\pgfqpoint{3.966155in}{0.844853in}}{\pgfqpoint{3.966155in}{0.855903in}}%
\pgfpathcurveto{\pgfqpoint{3.966155in}{0.866953in}}{\pgfqpoint{3.961765in}{0.877552in}}{\pgfqpoint{3.953951in}{0.885365in}}%
\pgfpathcurveto{\pgfqpoint{3.946138in}{0.893179in}}{\pgfqpoint{3.935539in}{0.897569in}}{\pgfqpoint{3.924488in}{0.897569in}}%
\pgfpathcurveto{\pgfqpoint{3.913438in}{0.897569in}}{\pgfqpoint{3.902839in}{0.893179in}}{\pgfqpoint{3.895026in}{0.885365in}}%
\pgfpathcurveto{\pgfqpoint{3.887212in}{0.877552in}}{\pgfqpoint{3.882822in}{0.866953in}}{\pgfqpoint{3.882822in}{0.855903in}}%
\pgfpathcurveto{\pgfqpoint{3.882822in}{0.844853in}}{\pgfqpoint{3.887212in}{0.834254in}}{\pgfqpoint{3.895026in}{0.826440in}}%
\pgfpathcurveto{\pgfqpoint{3.902839in}{0.818626in}}{\pgfqpoint{3.913438in}{0.814236in}}{\pgfqpoint{3.924488in}{0.814236in}}%
\pgfpathlineto{\pgfqpoint{3.924488in}{0.814236in}}%
\pgfpathclose%
\pgfusepath{stroke}%
\end{pgfscope}%
\begin{pgfscope}%
\pgfpathrectangle{\pgfqpoint{0.847223in}{0.554012in}}{\pgfqpoint{6.200000in}{4.530000in}}%
\pgfusepath{clip}%
\pgfsetbuttcap%
\pgfsetroundjoin%
\pgfsetlinewidth{1.003750pt}%
\definecolor{currentstroke}{rgb}{1.000000,0.000000,0.000000}%
\pgfsetstrokecolor{currentstroke}%
\pgfsetdash{}{0pt}%
\pgfpathmoveto{\pgfqpoint{3.929822in}{0.813173in}}%
\pgfpathcurveto{\pgfqpoint{3.940872in}{0.813173in}}{\pgfqpoint{3.951471in}{0.817563in}}{\pgfqpoint{3.959284in}{0.825376in}}%
\pgfpathcurveto{\pgfqpoint{3.967098in}{0.833190in}}{\pgfqpoint{3.971488in}{0.843789in}}{\pgfqpoint{3.971488in}{0.854839in}}%
\pgfpathcurveto{\pgfqpoint{3.971488in}{0.865889in}}{\pgfqpoint{3.967098in}{0.876488in}}{\pgfqpoint{3.959284in}{0.884302in}}%
\pgfpathcurveto{\pgfqpoint{3.951471in}{0.892116in}}{\pgfqpoint{3.940872in}{0.896506in}}{\pgfqpoint{3.929822in}{0.896506in}}%
\pgfpathcurveto{\pgfqpoint{3.918772in}{0.896506in}}{\pgfqpoint{3.908173in}{0.892116in}}{\pgfqpoint{3.900359in}{0.884302in}}%
\pgfpathcurveto{\pgfqpoint{3.892545in}{0.876488in}}{\pgfqpoint{3.888155in}{0.865889in}}{\pgfqpoint{3.888155in}{0.854839in}}%
\pgfpathcurveto{\pgfqpoint{3.888155in}{0.843789in}}{\pgfqpoint{3.892545in}{0.833190in}}{\pgfqpoint{3.900359in}{0.825376in}}%
\pgfpathcurveto{\pgfqpoint{3.908173in}{0.817563in}}{\pgfqpoint{3.918772in}{0.813173in}}{\pgfqpoint{3.929822in}{0.813173in}}%
\pgfpathlineto{\pgfqpoint{3.929822in}{0.813173in}}%
\pgfpathclose%
\pgfusepath{stroke}%
\end{pgfscope}%
\begin{pgfscope}%
\pgfpathrectangle{\pgfqpoint{0.847223in}{0.554012in}}{\pgfqpoint{6.200000in}{4.530000in}}%
\pgfusepath{clip}%
\pgfsetbuttcap%
\pgfsetroundjoin%
\pgfsetlinewidth{1.003750pt}%
\definecolor{currentstroke}{rgb}{1.000000,0.000000,0.000000}%
\pgfsetstrokecolor{currentstroke}%
\pgfsetdash{}{0pt}%
\pgfpathmoveto{\pgfqpoint{3.935155in}{0.812112in}}%
\pgfpathcurveto{\pgfqpoint{3.946205in}{0.812112in}}{\pgfqpoint{3.956804in}{0.816502in}}{\pgfqpoint{3.964618in}{0.824316in}}%
\pgfpathcurveto{\pgfqpoint{3.972431in}{0.832130in}}{\pgfqpoint{3.976822in}{0.842729in}}{\pgfqpoint{3.976822in}{0.853779in}}%
\pgfpathcurveto{\pgfqpoint{3.976822in}{0.864829in}}{\pgfqpoint{3.972431in}{0.875428in}}{\pgfqpoint{3.964618in}{0.883242in}}%
\pgfpathcurveto{\pgfqpoint{3.956804in}{0.891055in}}{\pgfqpoint{3.946205in}{0.895445in}}{\pgfqpoint{3.935155in}{0.895445in}}%
\pgfpathcurveto{\pgfqpoint{3.924105in}{0.895445in}}{\pgfqpoint{3.913506in}{0.891055in}}{\pgfqpoint{3.905692in}{0.883242in}}%
\pgfpathcurveto{\pgfqpoint{3.897879in}{0.875428in}}{\pgfqpoint{3.893488in}{0.864829in}}{\pgfqpoint{3.893488in}{0.853779in}}%
\pgfpathcurveto{\pgfqpoint{3.893488in}{0.842729in}}{\pgfqpoint{3.897879in}{0.832130in}}{\pgfqpoint{3.905692in}{0.824316in}}%
\pgfpathcurveto{\pgfqpoint{3.913506in}{0.816502in}}{\pgfqpoint{3.924105in}{0.812112in}}{\pgfqpoint{3.935155in}{0.812112in}}%
\pgfpathlineto{\pgfqpoint{3.935155in}{0.812112in}}%
\pgfpathclose%
\pgfusepath{stroke}%
\end{pgfscope}%
\begin{pgfscope}%
\pgfpathrectangle{\pgfqpoint{0.847223in}{0.554012in}}{\pgfqpoint{6.200000in}{4.530000in}}%
\pgfusepath{clip}%
\pgfsetbuttcap%
\pgfsetroundjoin%
\pgfsetlinewidth{1.003750pt}%
\definecolor{currentstroke}{rgb}{1.000000,0.000000,0.000000}%
\pgfsetstrokecolor{currentstroke}%
\pgfsetdash{}{0pt}%
\pgfpathmoveto{\pgfqpoint{3.940488in}{0.811055in}}%
\pgfpathcurveto{\pgfqpoint{3.951538in}{0.811055in}}{\pgfqpoint{3.962137in}{0.815445in}}{\pgfqpoint{3.969951in}{0.823259in}}%
\pgfpathcurveto{\pgfqpoint{3.977765in}{0.831072in}}{\pgfqpoint{3.982155in}{0.841671in}}{\pgfqpoint{3.982155in}{0.852722in}}%
\pgfpathcurveto{\pgfqpoint{3.982155in}{0.863772in}}{\pgfqpoint{3.977765in}{0.874371in}}{\pgfqpoint{3.969951in}{0.882184in}}%
\pgfpathcurveto{\pgfqpoint{3.962137in}{0.889998in}}{\pgfqpoint{3.951538in}{0.894388in}}{\pgfqpoint{3.940488in}{0.894388in}}%
\pgfpathcurveto{\pgfqpoint{3.929438in}{0.894388in}}{\pgfqpoint{3.918839in}{0.889998in}}{\pgfqpoint{3.911025in}{0.882184in}}%
\pgfpathcurveto{\pgfqpoint{3.903212in}{0.874371in}}{\pgfqpoint{3.898821in}{0.863772in}}{\pgfqpoint{3.898821in}{0.852722in}}%
\pgfpathcurveto{\pgfqpoint{3.898821in}{0.841671in}}{\pgfqpoint{3.903212in}{0.831072in}}{\pgfqpoint{3.911025in}{0.823259in}}%
\pgfpathcurveto{\pgfqpoint{3.918839in}{0.815445in}}{\pgfqpoint{3.929438in}{0.811055in}}{\pgfqpoint{3.940488in}{0.811055in}}%
\pgfpathlineto{\pgfqpoint{3.940488in}{0.811055in}}%
\pgfpathclose%
\pgfusepath{stroke}%
\end{pgfscope}%
\begin{pgfscope}%
\pgfpathrectangle{\pgfqpoint{0.847223in}{0.554012in}}{\pgfqpoint{6.200000in}{4.530000in}}%
\pgfusepath{clip}%
\pgfsetbuttcap%
\pgfsetroundjoin%
\pgfsetlinewidth{1.003750pt}%
\definecolor{currentstroke}{rgb}{1.000000,0.000000,0.000000}%
\pgfsetstrokecolor{currentstroke}%
\pgfsetdash{}{0pt}%
\pgfpathmoveto{\pgfqpoint{3.945821in}{0.810001in}}%
\pgfpathcurveto{\pgfqpoint{3.956871in}{0.810001in}}{\pgfqpoint{3.967471in}{0.814391in}}{\pgfqpoint{3.975284in}{0.822205in}}%
\pgfpathcurveto{\pgfqpoint{3.983098in}{0.830018in}}{\pgfqpoint{3.987488in}{0.840617in}}{\pgfqpoint{3.987488in}{0.851667in}}%
\pgfpathcurveto{\pgfqpoint{3.987488in}{0.862718in}}{\pgfqpoint{3.983098in}{0.873317in}}{\pgfqpoint{3.975284in}{0.881130in}}%
\pgfpathcurveto{\pgfqpoint{3.967471in}{0.888944in}}{\pgfqpoint{3.956871in}{0.893334in}}{\pgfqpoint{3.945821in}{0.893334in}}%
\pgfpathcurveto{\pgfqpoint{3.934771in}{0.893334in}}{\pgfqpoint{3.924172in}{0.888944in}}{\pgfqpoint{3.916359in}{0.881130in}}%
\pgfpathcurveto{\pgfqpoint{3.908545in}{0.873317in}}{\pgfqpoint{3.904155in}{0.862718in}}{\pgfqpoint{3.904155in}{0.851667in}}%
\pgfpathcurveto{\pgfqpoint{3.904155in}{0.840617in}}{\pgfqpoint{3.908545in}{0.830018in}}{\pgfqpoint{3.916359in}{0.822205in}}%
\pgfpathcurveto{\pgfqpoint{3.924172in}{0.814391in}}{\pgfqpoint{3.934771in}{0.810001in}}{\pgfqpoint{3.945821in}{0.810001in}}%
\pgfpathlineto{\pgfqpoint{3.945821in}{0.810001in}}%
\pgfpathclose%
\pgfusepath{stroke}%
\end{pgfscope}%
\begin{pgfscope}%
\pgfpathrectangle{\pgfqpoint{0.847223in}{0.554012in}}{\pgfqpoint{6.200000in}{4.530000in}}%
\pgfusepath{clip}%
\pgfsetbuttcap%
\pgfsetroundjoin%
\pgfsetlinewidth{1.003750pt}%
\definecolor{currentstroke}{rgb}{1.000000,0.000000,0.000000}%
\pgfsetstrokecolor{currentstroke}%
\pgfsetdash{}{0pt}%
\pgfpathmoveto{\pgfqpoint{3.951155in}{0.808950in}}%
\pgfpathcurveto{\pgfqpoint{3.962205in}{0.808950in}}{\pgfqpoint{3.972804in}{0.813340in}}{\pgfqpoint{3.980617in}{0.821154in}}%
\pgfpathcurveto{\pgfqpoint{3.988431in}{0.828967in}}{\pgfqpoint{3.992821in}{0.839566in}}{\pgfqpoint{3.992821in}{0.850616in}}%
\pgfpathcurveto{\pgfqpoint{3.992821in}{0.861667in}}{\pgfqpoint{3.988431in}{0.872266in}}{\pgfqpoint{3.980617in}{0.880079in}}%
\pgfpathcurveto{\pgfqpoint{3.972804in}{0.887893in}}{\pgfqpoint{3.962205in}{0.892283in}}{\pgfqpoint{3.951155in}{0.892283in}}%
\pgfpathcurveto{\pgfqpoint{3.940104in}{0.892283in}}{\pgfqpoint{3.929505in}{0.887893in}}{\pgfqpoint{3.921692in}{0.880079in}}%
\pgfpathcurveto{\pgfqpoint{3.913878in}{0.872266in}}{\pgfqpoint{3.909488in}{0.861667in}}{\pgfqpoint{3.909488in}{0.850616in}}%
\pgfpathcurveto{\pgfqpoint{3.909488in}{0.839566in}}{\pgfqpoint{3.913878in}{0.828967in}}{\pgfqpoint{3.921692in}{0.821154in}}%
\pgfpathcurveto{\pgfqpoint{3.929505in}{0.813340in}}{\pgfqpoint{3.940104in}{0.808950in}}{\pgfqpoint{3.951155in}{0.808950in}}%
\pgfpathlineto{\pgfqpoint{3.951155in}{0.808950in}}%
\pgfpathclose%
\pgfusepath{stroke}%
\end{pgfscope}%
\begin{pgfscope}%
\pgfpathrectangle{\pgfqpoint{0.847223in}{0.554012in}}{\pgfqpoint{6.200000in}{4.530000in}}%
\pgfusepath{clip}%
\pgfsetbuttcap%
\pgfsetroundjoin%
\pgfsetlinewidth{1.003750pt}%
\definecolor{currentstroke}{rgb}{1.000000,0.000000,0.000000}%
\pgfsetstrokecolor{currentstroke}%
\pgfsetdash{}{0pt}%
\pgfpathmoveto{\pgfqpoint{3.956488in}{0.807902in}}%
\pgfpathcurveto{\pgfqpoint{3.967538in}{0.807902in}}{\pgfqpoint{3.978137in}{0.812292in}}{\pgfqpoint{3.985951in}{0.820106in}}%
\pgfpathcurveto{\pgfqpoint{3.993764in}{0.827919in}}{\pgfqpoint{3.998154in}{0.838518in}}{\pgfqpoint{3.998154in}{0.849568in}}%
\pgfpathcurveto{\pgfqpoint{3.998154in}{0.860619in}}{\pgfqpoint{3.993764in}{0.871218in}}{\pgfqpoint{3.985951in}{0.879031in}}%
\pgfpathcurveto{\pgfqpoint{3.978137in}{0.886845in}}{\pgfqpoint{3.967538in}{0.891235in}}{\pgfqpoint{3.956488in}{0.891235in}}%
\pgfpathcurveto{\pgfqpoint{3.945438in}{0.891235in}}{\pgfqpoint{3.934839in}{0.886845in}}{\pgfqpoint{3.927025in}{0.879031in}}%
\pgfpathcurveto{\pgfqpoint{3.919211in}{0.871218in}}{\pgfqpoint{3.914821in}{0.860619in}}{\pgfqpoint{3.914821in}{0.849568in}}%
\pgfpathcurveto{\pgfqpoint{3.914821in}{0.838518in}}{\pgfqpoint{3.919211in}{0.827919in}}{\pgfqpoint{3.927025in}{0.820106in}}%
\pgfpathcurveto{\pgfqpoint{3.934839in}{0.812292in}}{\pgfqpoint{3.945438in}{0.807902in}}{\pgfqpoint{3.956488in}{0.807902in}}%
\pgfpathlineto{\pgfqpoint{3.956488in}{0.807902in}}%
\pgfpathclose%
\pgfusepath{stroke}%
\end{pgfscope}%
\begin{pgfscope}%
\pgfpathrectangle{\pgfqpoint{0.847223in}{0.554012in}}{\pgfqpoint{6.200000in}{4.530000in}}%
\pgfusepath{clip}%
\pgfsetbuttcap%
\pgfsetroundjoin%
\pgfsetlinewidth{1.003750pt}%
\definecolor{currentstroke}{rgb}{1.000000,0.000000,0.000000}%
\pgfsetstrokecolor{currentstroke}%
\pgfsetdash{}{0pt}%
\pgfpathmoveto{\pgfqpoint{3.961821in}{0.806857in}}%
\pgfpathcurveto{\pgfqpoint{3.972871in}{0.806857in}}{\pgfqpoint{3.983470in}{0.811247in}}{\pgfqpoint{3.991284in}{0.819061in}}%
\pgfpathcurveto{\pgfqpoint{3.999097in}{0.826874in}}{\pgfqpoint{4.003488in}{0.837473in}}{\pgfqpoint{4.003488in}{0.848524in}}%
\pgfpathcurveto{\pgfqpoint{4.003488in}{0.859574in}}{\pgfqpoint{3.999097in}{0.870173in}}{\pgfqpoint{3.991284in}{0.877986in}}%
\pgfpathcurveto{\pgfqpoint{3.983470in}{0.885800in}}{\pgfqpoint{3.972871in}{0.890190in}}{\pgfqpoint{3.961821in}{0.890190in}}%
\pgfpathcurveto{\pgfqpoint{3.950771in}{0.890190in}}{\pgfqpoint{3.940172in}{0.885800in}}{\pgfqpoint{3.932358in}{0.877986in}}%
\pgfpathcurveto{\pgfqpoint{3.924545in}{0.870173in}}{\pgfqpoint{3.920154in}{0.859574in}}{\pgfqpoint{3.920154in}{0.848524in}}%
\pgfpathcurveto{\pgfqpoint{3.920154in}{0.837473in}}{\pgfqpoint{3.924545in}{0.826874in}}{\pgfqpoint{3.932358in}{0.819061in}}%
\pgfpathcurveto{\pgfqpoint{3.940172in}{0.811247in}}{\pgfqpoint{3.950771in}{0.806857in}}{\pgfqpoint{3.961821in}{0.806857in}}%
\pgfpathlineto{\pgfqpoint{3.961821in}{0.806857in}}%
\pgfpathclose%
\pgfusepath{stroke}%
\end{pgfscope}%
\begin{pgfscope}%
\pgfpathrectangle{\pgfqpoint{0.847223in}{0.554012in}}{\pgfqpoint{6.200000in}{4.530000in}}%
\pgfusepath{clip}%
\pgfsetbuttcap%
\pgfsetroundjoin%
\pgfsetlinewidth{1.003750pt}%
\definecolor{currentstroke}{rgb}{1.000000,0.000000,0.000000}%
\pgfsetstrokecolor{currentstroke}%
\pgfsetdash{}{0pt}%
\pgfpathmoveto{\pgfqpoint{3.967154in}{0.805815in}}%
\pgfpathcurveto{\pgfqpoint{3.978204in}{0.805815in}}{\pgfqpoint{3.988803in}{0.810205in}}{\pgfqpoint{3.996617in}{0.818019in}}%
\pgfpathcurveto{\pgfqpoint{4.004431in}{0.825833in}}{\pgfqpoint{4.008821in}{0.836432in}}{\pgfqpoint{4.008821in}{0.847482in}}%
\pgfpathcurveto{\pgfqpoint{4.008821in}{0.858532in}}{\pgfqpoint{4.004431in}{0.869131in}}{\pgfqpoint{3.996617in}{0.876945in}}%
\pgfpathcurveto{\pgfqpoint{3.988803in}{0.884758in}}{\pgfqpoint{3.978204in}{0.889148in}}{\pgfqpoint{3.967154in}{0.889148in}}%
\pgfpathcurveto{\pgfqpoint{3.956104in}{0.889148in}}{\pgfqpoint{3.945505in}{0.884758in}}{\pgfqpoint{3.937691in}{0.876945in}}%
\pgfpathcurveto{\pgfqpoint{3.929878in}{0.869131in}}{\pgfqpoint{3.925488in}{0.858532in}}{\pgfqpoint{3.925488in}{0.847482in}}%
\pgfpathcurveto{\pgfqpoint{3.925488in}{0.836432in}}{\pgfqpoint{3.929878in}{0.825833in}}{\pgfqpoint{3.937691in}{0.818019in}}%
\pgfpathcurveto{\pgfqpoint{3.945505in}{0.810205in}}{\pgfqpoint{3.956104in}{0.805815in}}{\pgfqpoint{3.967154in}{0.805815in}}%
\pgfpathlineto{\pgfqpoint{3.967154in}{0.805815in}}%
\pgfpathclose%
\pgfusepath{stroke}%
\end{pgfscope}%
\begin{pgfscope}%
\pgfpathrectangle{\pgfqpoint{0.847223in}{0.554012in}}{\pgfqpoint{6.200000in}{4.530000in}}%
\pgfusepath{clip}%
\pgfsetbuttcap%
\pgfsetroundjoin%
\pgfsetlinewidth{1.003750pt}%
\definecolor{currentstroke}{rgb}{1.000000,0.000000,0.000000}%
\pgfsetstrokecolor{currentstroke}%
\pgfsetdash{}{0pt}%
\pgfpathmoveto{\pgfqpoint{3.972487in}{0.804776in}}%
\pgfpathcurveto{\pgfqpoint{3.983538in}{0.804776in}}{\pgfqpoint{3.994137in}{0.809167in}}{\pgfqpoint{4.001950in}{0.816980in}}%
\pgfpathcurveto{\pgfqpoint{4.009764in}{0.824794in}}{\pgfqpoint{4.014154in}{0.835393in}}{\pgfqpoint{4.014154in}{0.846443in}}%
\pgfpathcurveto{\pgfqpoint{4.014154in}{0.857493in}}{\pgfqpoint{4.009764in}{0.868092in}}{\pgfqpoint{4.001950in}{0.875906in}}%
\pgfpathcurveto{\pgfqpoint{3.994137in}{0.883719in}}{\pgfqpoint{3.983538in}{0.888110in}}{\pgfqpoint{3.972487in}{0.888110in}}%
\pgfpathcurveto{\pgfqpoint{3.961437in}{0.888110in}}{\pgfqpoint{3.950838in}{0.883719in}}{\pgfqpoint{3.943025in}{0.875906in}}%
\pgfpathcurveto{\pgfqpoint{3.935211in}{0.868092in}}{\pgfqpoint{3.930821in}{0.857493in}}{\pgfqpoint{3.930821in}{0.846443in}}%
\pgfpathcurveto{\pgfqpoint{3.930821in}{0.835393in}}{\pgfqpoint{3.935211in}{0.824794in}}{\pgfqpoint{3.943025in}{0.816980in}}%
\pgfpathcurveto{\pgfqpoint{3.950838in}{0.809167in}}{\pgfqpoint{3.961437in}{0.804776in}}{\pgfqpoint{3.972487in}{0.804776in}}%
\pgfpathlineto{\pgfqpoint{3.972487in}{0.804776in}}%
\pgfpathclose%
\pgfusepath{stroke}%
\end{pgfscope}%
\begin{pgfscope}%
\pgfpathrectangle{\pgfqpoint{0.847223in}{0.554012in}}{\pgfqpoint{6.200000in}{4.530000in}}%
\pgfusepath{clip}%
\pgfsetbuttcap%
\pgfsetroundjoin%
\pgfsetlinewidth{1.003750pt}%
\definecolor{currentstroke}{rgb}{1.000000,0.000000,0.000000}%
\pgfsetstrokecolor{currentstroke}%
\pgfsetdash{}{0pt}%
\pgfpathmoveto{\pgfqpoint{3.977821in}{0.803741in}}%
\pgfpathcurveto{\pgfqpoint{3.988871in}{0.803741in}}{\pgfqpoint{3.999470in}{0.808131in}}{\pgfqpoint{4.007283in}{0.815944in}}%
\pgfpathcurveto{\pgfqpoint{4.015097in}{0.823758in}}{\pgfqpoint{4.019487in}{0.834357in}}{\pgfqpoint{4.019487in}{0.845407in}}%
\pgfpathcurveto{\pgfqpoint{4.019487in}{0.856457in}}{\pgfqpoint{4.015097in}{0.867056in}}{\pgfqpoint{4.007283in}{0.874870in}}%
\pgfpathcurveto{\pgfqpoint{3.999470in}{0.882684in}}{\pgfqpoint{3.988871in}{0.887074in}}{\pgfqpoint{3.977821in}{0.887074in}}%
\pgfpathcurveto{\pgfqpoint{3.966771in}{0.887074in}}{\pgfqpoint{3.956171in}{0.882684in}}{\pgfqpoint{3.948358in}{0.874870in}}%
\pgfpathcurveto{\pgfqpoint{3.940544in}{0.867056in}}{\pgfqpoint{3.936154in}{0.856457in}}{\pgfqpoint{3.936154in}{0.845407in}}%
\pgfpathcurveto{\pgfqpoint{3.936154in}{0.834357in}}{\pgfqpoint{3.940544in}{0.823758in}}{\pgfqpoint{3.948358in}{0.815944in}}%
\pgfpathcurveto{\pgfqpoint{3.956171in}{0.808131in}}{\pgfqpoint{3.966771in}{0.803741in}}{\pgfqpoint{3.977821in}{0.803741in}}%
\pgfpathlineto{\pgfqpoint{3.977821in}{0.803741in}}%
\pgfpathclose%
\pgfusepath{stroke}%
\end{pgfscope}%
\begin{pgfscope}%
\pgfpathrectangle{\pgfqpoint{0.847223in}{0.554012in}}{\pgfqpoint{6.200000in}{4.530000in}}%
\pgfusepath{clip}%
\pgfsetbuttcap%
\pgfsetroundjoin%
\pgfsetlinewidth{1.003750pt}%
\definecolor{currentstroke}{rgb}{1.000000,0.000000,0.000000}%
\pgfsetstrokecolor{currentstroke}%
\pgfsetdash{}{0pt}%
\pgfpathmoveto{\pgfqpoint{3.983154in}{0.802708in}}%
\pgfpathcurveto{\pgfqpoint{3.994204in}{0.802708in}}{\pgfqpoint{4.004803in}{0.807098in}}{\pgfqpoint{4.012617in}{0.814912in}}%
\pgfpathcurveto{\pgfqpoint{4.020430in}{0.822725in}}{\pgfqpoint{4.024821in}{0.833324in}}{\pgfqpoint{4.024821in}{0.844375in}}%
\pgfpathcurveto{\pgfqpoint{4.024821in}{0.855425in}}{\pgfqpoint{4.020430in}{0.866024in}}{\pgfqpoint{4.012617in}{0.873837in}}%
\pgfpathcurveto{\pgfqpoint{4.004803in}{0.881651in}}{\pgfqpoint{3.994204in}{0.886041in}}{\pgfqpoint{3.983154in}{0.886041in}}%
\pgfpathcurveto{\pgfqpoint{3.972104in}{0.886041in}}{\pgfqpoint{3.961505in}{0.881651in}}{\pgfqpoint{3.953691in}{0.873837in}}%
\pgfpathcurveto{\pgfqpoint{3.945877in}{0.866024in}}{\pgfqpoint{3.941487in}{0.855425in}}{\pgfqpoint{3.941487in}{0.844375in}}%
\pgfpathcurveto{\pgfqpoint{3.941487in}{0.833324in}}{\pgfqpoint{3.945877in}{0.822725in}}{\pgfqpoint{3.953691in}{0.814912in}}%
\pgfpathcurveto{\pgfqpoint{3.961505in}{0.807098in}}{\pgfqpoint{3.972104in}{0.802708in}}{\pgfqpoint{3.983154in}{0.802708in}}%
\pgfpathlineto{\pgfqpoint{3.983154in}{0.802708in}}%
\pgfpathclose%
\pgfusepath{stroke}%
\end{pgfscope}%
\begin{pgfscope}%
\pgfpathrectangle{\pgfqpoint{0.847223in}{0.554012in}}{\pgfqpoint{6.200000in}{4.530000in}}%
\pgfusepath{clip}%
\pgfsetbuttcap%
\pgfsetroundjoin%
\pgfsetlinewidth{1.003750pt}%
\definecolor{currentstroke}{rgb}{1.000000,0.000000,0.000000}%
\pgfsetstrokecolor{currentstroke}%
\pgfsetdash{}{0pt}%
\pgfpathmoveto{\pgfqpoint{3.988487in}{0.801678in}}%
\pgfpathcurveto{\pgfqpoint{3.999537in}{0.801678in}}{\pgfqpoint{4.010136in}{0.806068in}}{\pgfqpoint{4.017950in}{0.813882in}}%
\pgfpathcurveto{\pgfqpoint{4.025763in}{0.821696in}}{\pgfqpoint{4.030154in}{0.832295in}}{\pgfqpoint{4.030154in}{0.843345in}}%
\pgfpathcurveto{\pgfqpoint{4.030154in}{0.854395in}}{\pgfqpoint{4.025763in}{0.864994in}}{\pgfqpoint{4.017950in}{0.872808in}}%
\pgfpathcurveto{\pgfqpoint{4.010136in}{0.880621in}}{\pgfqpoint{3.999537in}{0.885011in}}{\pgfqpoint{3.988487in}{0.885011in}}%
\pgfpathcurveto{\pgfqpoint{3.977437in}{0.885011in}}{\pgfqpoint{3.966838in}{0.880621in}}{\pgfqpoint{3.959024in}{0.872808in}}%
\pgfpathcurveto{\pgfqpoint{3.951211in}{0.864994in}}{\pgfqpoint{3.946820in}{0.854395in}}{\pgfqpoint{3.946820in}{0.843345in}}%
\pgfpathcurveto{\pgfqpoint{3.946820in}{0.832295in}}{\pgfqpoint{3.951211in}{0.821696in}}{\pgfqpoint{3.959024in}{0.813882in}}%
\pgfpathcurveto{\pgfqpoint{3.966838in}{0.806068in}}{\pgfqpoint{3.977437in}{0.801678in}}{\pgfqpoint{3.988487in}{0.801678in}}%
\pgfpathlineto{\pgfqpoint{3.988487in}{0.801678in}}%
\pgfpathclose%
\pgfusepath{stroke}%
\end{pgfscope}%
\begin{pgfscope}%
\pgfpathrectangle{\pgfqpoint{0.847223in}{0.554012in}}{\pgfqpoint{6.200000in}{4.530000in}}%
\pgfusepath{clip}%
\pgfsetbuttcap%
\pgfsetroundjoin%
\pgfsetlinewidth{1.003750pt}%
\definecolor{currentstroke}{rgb}{1.000000,0.000000,0.000000}%
\pgfsetstrokecolor{currentstroke}%
\pgfsetdash{}{0pt}%
\pgfpathmoveto{\pgfqpoint{3.993820in}{0.800651in}}%
\pgfpathcurveto{\pgfqpoint{4.004870in}{0.800651in}}{\pgfqpoint{4.015469in}{0.805042in}}{\pgfqpoint{4.023283in}{0.812855in}}%
\pgfpathcurveto{\pgfqpoint{4.031097in}{0.820669in}}{\pgfqpoint{4.035487in}{0.831268in}}{\pgfqpoint{4.035487in}{0.842318in}}%
\pgfpathcurveto{\pgfqpoint{4.035487in}{0.853368in}}{\pgfqpoint{4.031097in}{0.863967in}}{\pgfqpoint{4.023283in}{0.871781in}}%
\pgfpathcurveto{\pgfqpoint{4.015469in}{0.879595in}}{\pgfqpoint{4.004870in}{0.883985in}}{\pgfqpoint{3.993820in}{0.883985in}}%
\pgfpathcurveto{\pgfqpoint{3.982770in}{0.883985in}}{\pgfqpoint{3.972171in}{0.879595in}}{\pgfqpoint{3.964358in}{0.871781in}}%
\pgfpathcurveto{\pgfqpoint{3.956544in}{0.863967in}}{\pgfqpoint{3.952154in}{0.853368in}}{\pgfqpoint{3.952154in}{0.842318in}}%
\pgfpathcurveto{\pgfqpoint{3.952154in}{0.831268in}}{\pgfqpoint{3.956544in}{0.820669in}}{\pgfqpoint{3.964358in}{0.812855in}}%
\pgfpathcurveto{\pgfqpoint{3.972171in}{0.805042in}}{\pgfqpoint{3.982770in}{0.800651in}}{\pgfqpoint{3.993820in}{0.800651in}}%
\pgfpathlineto{\pgfqpoint{3.993820in}{0.800651in}}%
\pgfpathclose%
\pgfusepath{stroke}%
\end{pgfscope}%
\begin{pgfscope}%
\pgfpathrectangle{\pgfqpoint{0.847223in}{0.554012in}}{\pgfqpoint{6.200000in}{4.530000in}}%
\pgfusepath{clip}%
\pgfsetbuttcap%
\pgfsetroundjoin%
\pgfsetlinewidth{1.003750pt}%
\definecolor{currentstroke}{rgb}{1.000000,0.000000,0.000000}%
\pgfsetstrokecolor{currentstroke}%
\pgfsetdash{}{0pt}%
\pgfpathmoveto{\pgfqpoint{3.999154in}{0.799628in}}%
\pgfpathcurveto{\pgfqpoint{4.010204in}{0.799628in}}{\pgfqpoint{4.020803in}{0.804018in}}{\pgfqpoint{4.028616in}{0.811832in}}%
\pgfpathcurveto{\pgfqpoint{4.036430in}{0.819645in}}{\pgfqpoint{4.040820in}{0.830244in}}{\pgfqpoint{4.040820in}{0.841294in}}%
\pgfpathcurveto{\pgfqpoint{4.040820in}{0.852345in}}{\pgfqpoint{4.036430in}{0.862944in}}{\pgfqpoint{4.028616in}{0.870757in}}%
\pgfpathcurveto{\pgfqpoint{4.020803in}{0.878571in}}{\pgfqpoint{4.010204in}{0.882961in}}{\pgfqpoint{3.999154in}{0.882961in}}%
\pgfpathcurveto{\pgfqpoint{3.988103in}{0.882961in}}{\pgfqpoint{3.977504in}{0.878571in}}{\pgfqpoint{3.969691in}{0.870757in}}%
\pgfpathcurveto{\pgfqpoint{3.961877in}{0.862944in}}{\pgfqpoint{3.957487in}{0.852345in}}{\pgfqpoint{3.957487in}{0.841294in}}%
\pgfpathcurveto{\pgfqpoint{3.957487in}{0.830244in}}{\pgfqpoint{3.961877in}{0.819645in}}{\pgfqpoint{3.969691in}{0.811832in}}%
\pgfpathcurveto{\pgfqpoint{3.977504in}{0.804018in}}{\pgfqpoint{3.988103in}{0.799628in}}{\pgfqpoint{3.999154in}{0.799628in}}%
\pgfpathlineto{\pgfqpoint{3.999154in}{0.799628in}}%
\pgfpathclose%
\pgfusepath{stroke}%
\end{pgfscope}%
\begin{pgfscope}%
\pgfpathrectangle{\pgfqpoint{0.847223in}{0.554012in}}{\pgfqpoint{6.200000in}{4.530000in}}%
\pgfusepath{clip}%
\pgfsetbuttcap%
\pgfsetroundjoin%
\pgfsetlinewidth{1.003750pt}%
\definecolor{currentstroke}{rgb}{1.000000,0.000000,0.000000}%
\pgfsetstrokecolor{currentstroke}%
\pgfsetdash{}{0pt}%
\pgfpathmoveto{\pgfqpoint{4.004487in}{0.798607in}}%
\pgfpathcurveto{\pgfqpoint{4.015537in}{0.798607in}}{\pgfqpoint{4.026136in}{0.802997in}}{\pgfqpoint{4.033950in}{0.810811in}}%
\pgfpathcurveto{\pgfqpoint{4.041763in}{0.818624in}}{\pgfqpoint{4.046153in}{0.829223in}}{\pgfqpoint{4.046153in}{0.840274in}}%
\pgfpathcurveto{\pgfqpoint{4.046153in}{0.851324in}}{\pgfqpoint{4.041763in}{0.861923in}}{\pgfqpoint{4.033950in}{0.869736in}}%
\pgfpathcurveto{\pgfqpoint{4.026136in}{0.877550in}}{\pgfqpoint{4.015537in}{0.881940in}}{\pgfqpoint{4.004487in}{0.881940in}}%
\pgfpathcurveto{\pgfqpoint{3.993437in}{0.881940in}}{\pgfqpoint{3.982838in}{0.877550in}}{\pgfqpoint{3.975024in}{0.869736in}}%
\pgfpathcurveto{\pgfqpoint{3.967210in}{0.861923in}}{\pgfqpoint{3.962820in}{0.851324in}}{\pgfqpoint{3.962820in}{0.840274in}}%
\pgfpathcurveto{\pgfqpoint{3.962820in}{0.829223in}}{\pgfqpoint{3.967210in}{0.818624in}}{\pgfqpoint{3.975024in}{0.810811in}}%
\pgfpathcurveto{\pgfqpoint{3.982838in}{0.802997in}}{\pgfqpoint{3.993437in}{0.798607in}}{\pgfqpoint{4.004487in}{0.798607in}}%
\pgfpathlineto{\pgfqpoint{4.004487in}{0.798607in}}%
\pgfpathclose%
\pgfusepath{stroke}%
\end{pgfscope}%
\begin{pgfscope}%
\pgfpathrectangle{\pgfqpoint{0.847223in}{0.554012in}}{\pgfqpoint{6.200000in}{4.530000in}}%
\pgfusepath{clip}%
\pgfsetbuttcap%
\pgfsetroundjoin%
\pgfsetlinewidth{1.003750pt}%
\definecolor{currentstroke}{rgb}{1.000000,0.000000,0.000000}%
\pgfsetstrokecolor{currentstroke}%
\pgfsetdash{}{0pt}%
\pgfpathmoveto{\pgfqpoint{4.009820in}{0.797589in}}%
\pgfpathcurveto{\pgfqpoint{4.020870in}{0.797589in}}{\pgfqpoint{4.031469in}{0.801979in}}{\pgfqpoint{4.039283in}{0.809793in}}%
\pgfpathcurveto{\pgfqpoint{4.047096in}{0.817607in}}{\pgfqpoint{4.051487in}{0.828206in}}{\pgfqpoint{4.051487in}{0.839256in}}%
\pgfpathcurveto{\pgfqpoint{4.051487in}{0.850306in}}{\pgfqpoint{4.047096in}{0.860905in}}{\pgfqpoint{4.039283in}{0.868719in}}%
\pgfpathcurveto{\pgfqpoint{4.031469in}{0.876532in}}{\pgfqpoint{4.020870in}{0.880922in}}{\pgfqpoint{4.009820in}{0.880922in}}%
\pgfpathcurveto{\pgfqpoint{3.998770in}{0.880922in}}{\pgfqpoint{3.988171in}{0.876532in}}{\pgfqpoint{3.980357in}{0.868719in}}%
\pgfpathcurveto{\pgfqpoint{3.972544in}{0.860905in}}{\pgfqpoint{3.968153in}{0.850306in}}{\pgfqpoint{3.968153in}{0.839256in}}%
\pgfpathcurveto{\pgfqpoint{3.968153in}{0.828206in}}{\pgfqpoint{3.972544in}{0.817607in}}{\pgfqpoint{3.980357in}{0.809793in}}%
\pgfpathcurveto{\pgfqpoint{3.988171in}{0.801979in}}{\pgfqpoint{3.998770in}{0.797589in}}{\pgfqpoint{4.009820in}{0.797589in}}%
\pgfpathlineto{\pgfqpoint{4.009820in}{0.797589in}}%
\pgfpathclose%
\pgfusepath{stroke}%
\end{pgfscope}%
\begin{pgfscope}%
\pgfpathrectangle{\pgfqpoint{0.847223in}{0.554012in}}{\pgfqpoint{6.200000in}{4.530000in}}%
\pgfusepath{clip}%
\pgfsetbuttcap%
\pgfsetroundjoin%
\pgfsetlinewidth{1.003750pt}%
\definecolor{currentstroke}{rgb}{1.000000,0.000000,0.000000}%
\pgfsetstrokecolor{currentstroke}%
\pgfsetdash{}{0pt}%
\pgfpathmoveto{\pgfqpoint{4.015153in}{0.796574in}}%
\pgfpathcurveto{\pgfqpoint{4.026203in}{0.796574in}}{\pgfqpoint{4.036802in}{0.800965in}}{\pgfqpoint{4.044616in}{0.808778in}}%
\pgfpathcurveto{\pgfqpoint{4.052430in}{0.816592in}}{\pgfqpoint{4.056820in}{0.827191in}}{\pgfqpoint{4.056820in}{0.838241in}}%
\pgfpathcurveto{\pgfqpoint{4.056820in}{0.849291in}}{\pgfqpoint{4.052430in}{0.859890in}}{\pgfqpoint{4.044616in}{0.867704in}}%
\pgfpathcurveto{\pgfqpoint{4.036802in}{0.875517in}}{\pgfqpoint{4.026203in}{0.879908in}}{\pgfqpoint{4.015153in}{0.879908in}}%
\pgfpathcurveto{\pgfqpoint{4.004103in}{0.879908in}}{\pgfqpoint{3.993504in}{0.875517in}}{\pgfqpoint{3.985690in}{0.867704in}}%
\pgfpathcurveto{\pgfqpoint{3.977877in}{0.859890in}}{\pgfqpoint{3.973486in}{0.849291in}}{\pgfqpoint{3.973486in}{0.838241in}}%
\pgfpathcurveto{\pgfqpoint{3.973486in}{0.827191in}}{\pgfqpoint{3.977877in}{0.816592in}}{\pgfqpoint{3.985690in}{0.808778in}}%
\pgfpathcurveto{\pgfqpoint{3.993504in}{0.800965in}}{\pgfqpoint{4.004103in}{0.796574in}}{\pgfqpoint{4.015153in}{0.796574in}}%
\pgfpathlineto{\pgfqpoint{4.015153in}{0.796574in}}%
\pgfpathclose%
\pgfusepath{stroke}%
\end{pgfscope}%
\begin{pgfscope}%
\pgfpathrectangle{\pgfqpoint{0.847223in}{0.554012in}}{\pgfqpoint{6.200000in}{4.530000in}}%
\pgfusepath{clip}%
\pgfsetbuttcap%
\pgfsetroundjoin%
\pgfsetlinewidth{1.003750pt}%
\definecolor{currentstroke}{rgb}{1.000000,0.000000,0.000000}%
\pgfsetstrokecolor{currentstroke}%
\pgfsetdash{}{0pt}%
\pgfpathmoveto{\pgfqpoint{4.020486in}{0.795562in}}%
\pgfpathcurveto{\pgfqpoint{4.031537in}{0.795562in}}{\pgfqpoint{4.042136in}{0.799953in}}{\pgfqpoint{4.049949in}{0.807766in}}%
\pgfpathcurveto{\pgfqpoint{4.057763in}{0.815580in}}{\pgfqpoint{4.062153in}{0.826179in}}{\pgfqpoint{4.062153in}{0.837229in}}%
\pgfpathcurveto{\pgfqpoint{4.062153in}{0.848279in}}{\pgfqpoint{4.057763in}{0.858878in}}{\pgfqpoint{4.049949in}{0.866692in}}%
\pgfpathcurveto{\pgfqpoint{4.042136in}{0.874505in}}{\pgfqpoint{4.031537in}{0.878896in}}{\pgfqpoint{4.020486in}{0.878896in}}%
\pgfpathcurveto{\pgfqpoint{4.009436in}{0.878896in}}{\pgfqpoint{3.998837in}{0.874505in}}{\pgfqpoint{3.991024in}{0.866692in}}%
\pgfpathcurveto{\pgfqpoint{3.983210in}{0.858878in}}{\pgfqpoint{3.978820in}{0.848279in}}{\pgfqpoint{3.978820in}{0.837229in}}%
\pgfpathcurveto{\pgfqpoint{3.978820in}{0.826179in}}{\pgfqpoint{3.983210in}{0.815580in}}{\pgfqpoint{3.991024in}{0.807766in}}%
\pgfpathcurveto{\pgfqpoint{3.998837in}{0.799953in}}{\pgfqpoint{4.009436in}{0.795562in}}{\pgfqpoint{4.020486in}{0.795562in}}%
\pgfpathlineto{\pgfqpoint{4.020486in}{0.795562in}}%
\pgfpathclose%
\pgfusepath{stroke}%
\end{pgfscope}%
\begin{pgfscope}%
\pgfpathrectangle{\pgfqpoint{0.847223in}{0.554012in}}{\pgfqpoint{6.200000in}{4.530000in}}%
\pgfusepath{clip}%
\pgfsetbuttcap%
\pgfsetroundjoin%
\pgfsetlinewidth{1.003750pt}%
\definecolor{currentstroke}{rgb}{1.000000,0.000000,0.000000}%
\pgfsetstrokecolor{currentstroke}%
\pgfsetdash{}{0pt}%
\pgfpathmoveto{\pgfqpoint{4.025820in}{0.794553in}}%
\pgfpathcurveto{\pgfqpoint{4.036870in}{0.794553in}}{\pgfqpoint{4.047469in}{0.798944in}}{\pgfqpoint{4.055282in}{0.806757in}}%
\pgfpathcurveto{\pgfqpoint{4.063096in}{0.814571in}}{\pgfqpoint{4.067486in}{0.825170in}}{\pgfqpoint{4.067486in}{0.836220in}}%
\pgfpathcurveto{\pgfqpoint{4.067486in}{0.847270in}}{\pgfqpoint{4.063096in}{0.857869in}}{\pgfqpoint{4.055282in}{0.865683in}}%
\pgfpathcurveto{\pgfqpoint{4.047469in}{0.873496in}}{\pgfqpoint{4.036870in}{0.877887in}}{\pgfqpoint{4.025820in}{0.877887in}}%
\pgfpathcurveto{\pgfqpoint{4.014769in}{0.877887in}}{\pgfqpoint{4.004170in}{0.873496in}}{\pgfqpoint{3.996357in}{0.865683in}}%
\pgfpathcurveto{\pgfqpoint{3.988543in}{0.857869in}}{\pgfqpoint{3.984153in}{0.847270in}}{\pgfqpoint{3.984153in}{0.836220in}}%
\pgfpathcurveto{\pgfqpoint{3.984153in}{0.825170in}}{\pgfqpoint{3.988543in}{0.814571in}}{\pgfqpoint{3.996357in}{0.806757in}}%
\pgfpathcurveto{\pgfqpoint{4.004170in}{0.798944in}}{\pgfqpoint{4.014769in}{0.794553in}}{\pgfqpoint{4.025820in}{0.794553in}}%
\pgfpathlineto{\pgfqpoint{4.025820in}{0.794553in}}%
\pgfpathclose%
\pgfusepath{stroke}%
\end{pgfscope}%
\begin{pgfscope}%
\pgfpathrectangle{\pgfqpoint{0.847223in}{0.554012in}}{\pgfqpoint{6.200000in}{4.530000in}}%
\pgfusepath{clip}%
\pgfsetbuttcap%
\pgfsetroundjoin%
\pgfsetlinewidth{1.003750pt}%
\definecolor{currentstroke}{rgb}{1.000000,0.000000,0.000000}%
\pgfsetstrokecolor{currentstroke}%
\pgfsetdash{}{0pt}%
\pgfpathmoveto{\pgfqpoint{4.031153in}{0.793547in}}%
\pgfpathcurveto{\pgfqpoint{4.042203in}{0.793547in}}{\pgfqpoint{4.052802in}{0.797937in}}{\pgfqpoint{4.060616in}{0.805751in}}%
\pgfpathcurveto{\pgfqpoint{4.068429in}{0.813565in}}{\pgfqpoint{4.072819in}{0.824164in}}{\pgfqpoint{4.072819in}{0.835214in}}%
\pgfpathcurveto{\pgfqpoint{4.072819in}{0.846264in}}{\pgfqpoint{4.068429in}{0.856863in}}{\pgfqpoint{4.060616in}{0.864677in}}%
\pgfpathcurveto{\pgfqpoint{4.052802in}{0.872490in}}{\pgfqpoint{4.042203in}{0.876880in}}{\pgfqpoint{4.031153in}{0.876880in}}%
\pgfpathcurveto{\pgfqpoint{4.020103in}{0.876880in}}{\pgfqpoint{4.009504in}{0.872490in}}{\pgfqpoint{4.001690in}{0.864677in}}%
\pgfpathcurveto{\pgfqpoint{3.993876in}{0.856863in}}{\pgfqpoint{3.989486in}{0.846264in}}{\pgfqpoint{3.989486in}{0.835214in}}%
\pgfpathcurveto{\pgfqpoint{3.989486in}{0.824164in}}{\pgfqpoint{3.993876in}{0.813565in}}{\pgfqpoint{4.001690in}{0.805751in}}%
\pgfpathcurveto{\pgfqpoint{4.009504in}{0.797937in}}{\pgfqpoint{4.020103in}{0.793547in}}{\pgfqpoint{4.031153in}{0.793547in}}%
\pgfpathlineto{\pgfqpoint{4.031153in}{0.793547in}}%
\pgfpathclose%
\pgfusepath{stroke}%
\end{pgfscope}%
\begin{pgfscope}%
\pgfpathrectangle{\pgfqpoint{0.847223in}{0.554012in}}{\pgfqpoint{6.200000in}{4.530000in}}%
\pgfusepath{clip}%
\pgfsetbuttcap%
\pgfsetroundjoin%
\pgfsetlinewidth{1.003750pt}%
\definecolor{currentstroke}{rgb}{1.000000,0.000000,0.000000}%
\pgfsetstrokecolor{currentstroke}%
\pgfsetdash{}{0pt}%
\pgfpathmoveto{\pgfqpoint{4.036486in}{0.792544in}}%
\pgfpathcurveto{\pgfqpoint{4.047536in}{0.792544in}}{\pgfqpoint{4.058135in}{0.796934in}}{\pgfqpoint{4.065949in}{0.804748in}}%
\pgfpathcurveto{\pgfqpoint{4.073762in}{0.812561in}}{\pgfqpoint{4.078153in}{0.823160in}}{\pgfqpoint{4.078153in}{0.834211in}}%
\pgfpathcurveto{\pgfqpoint{4.078153in}{0.845261in}}{\pgfqpoint{4.073762in}{0.855860in}}{\pgfqpoint{4.065949in}{0.863673in}}%
\pgfpathcurveto{\pgfqpoint{4.058135in}{0.871487in}}{\pgfqpoint{4.047536in}{0.875877in}}{\pgfqpoint{4.036486in}{0.875877in}}%
\pgfpathcurveto{\pgfqpoint{4.025436in}{0.875877in}}{\pgfqpoint{4.014837in}{0.871487in}}{\pgfqpoint{4.007023in}{0.863673in}}%
\pgfpathcurveto{\pgfqpoint{3.999210in}{0.855860in}}{\pgfqpoint{3.994819in}{0.845261in}}{\pgfqpoint{3.994819in}{0.834211in}}%
\pgfpathcurveto{\pgfqpoint{3.994819in}{0.823160in}}{\pgfqpoint{3.999210in}{0.812561in}}{\pgfqpoint{4.007023in}{0.804748in}}%
\pgfpathcurveto{\pgfqpoint{4.014837in}{0.796934in}}{\pgfqpoint{4.025436in}{0.792544in}}{\pgfqpoint{4.036486in}{0.792544in}}%
\pgfpathlineto{\pgfqpoint{4.036486in}{0.792544in}}%
\pgfpathclose%
\pgfusepath{stroke}%
\end{pgfscope}%
\begin{pgfscope}%
\pgfpathrectangle{\pgfqpoint{0.847223in}{0.554012in}}{\pgfqpoint{6.200000in}{4.530000in}}%
\pgfusepath{clip}%
\pgfsetbuttcap%
\pgfsetroundjoin%
\pgfsetlinewidth{1.003750pt}%
\definecolor{currentstroke}{rgb}{1.000000,0.000000,0.000000}%
\pgfsetstrokecolor{currentstroke}%
\pgfsetdash{}{0pt}%
\pgfpathmoveto{\pgfqpoint{4.041819in}{0.791544in}}%
\pgfpathcurveto{\pgfqpoint{4.052869in}{0.791544in}}{\pgfqpoint{4.063468in}{0.795934in}}{\pgfqpoint{4.071282in}{0.803747in}}%
\pgfpathcurveto{\pgfqpoint{4.079096in}{0.811561in}}{\pgfqpoint{4.083486in}{0.822160in}}{\pgfqpoint{4.083486in}{0.833210in}}%
\pgfpathcurveto{\pgfqpoint{4.083486in}{0.844260in}}{\pgfqpoint{4.079096in}{0.854859in}}{\pgfqpoint{4.071282in}{0.862673in}}%
\pgfpathcurveto{\pgfqpoint{4.063468in}{0.870487in}}{\pgfqpoint{4.052869in}{0.874877in}}{\pgfqpoint{4.041819in}{0.874877in}}%
\pgfpathcurveto{\pgfqpoint{4.030769in}{0.874877in}}{\pgfqpoint{4.020170in}{0.870487in}}{\pgfqpoint{4.012356in}{0.862673in}}%
\pgfpathcurveto{\pgfqpoint{4.004543in}{0.854859in}}{\pgfqpoint{4.000153in}{0.844260in}}{\pgfqpoint{4.000153in}{0.833210in}}%
\pgfpathcurveto{\pgfqpoint{4.000153in}{0.822160in}}{\pgfqpoint{4.004543in}{0.811561in}}{\pgfqpoint{4.012356in}{0.803747in}}%
\pgfpathcurveto{\pgfqpoint{4.020170in}{0.795934in}}{\pgfqpoint{4.030769in}{0.791544in}}{\pgfqpoint{4.041819in}{0.791544in}}%
\pgfpathlineto{\pgfqpoint{4.041819in}{0.791544in}}%
\pgfpathclose%
\pgfusepath{stroke}%
\end{pgfscope}%
\begin{pgfscope}%
\pgfpathrectangle{\pgfqpoint{0.847223in}{0.554012in}}{\pgfqpoint{6.200000in}{4.530000in}}%
\pgfusepath{clip}%
\pgfsetbuttcap%
\pgfsetroundjoin%
\pgfsetlinewidth{1.003750pt}%
\definecolor{currentstroke}{rgb}{1.000000,0.000000,0.000000}%
\pgfsetstrokecolor{currentstroke}%
\pgfsetdash{}{0pt}%
\pgfpathmoveto{\pgfqpoint{4.047152in}{0.790546in}}%
\pgfpathcurveto{\pgfqpoint{4.058203in}{0.790546in}}{\pgfqpoint{4.068802in}{0.794936in}}{\pgfqpoint{4.076615in}{0.802750in}}%
\pgfpathcurveto{\pgfqpoint{4.084429in}{0.810564in}}{\pgfqpoint{4.088819in}{0.821163in}}{\pgfqpoint{4.088819in}{0.832213in}}%
\pgfpathcurveto{\pgfqpoint{4.088819in}{0.843263in}}{\pgfqpoint{4.084429in}{0.853862in}}{\pgfqpoint{4.076615in}{0.861675in}}%
\pgfpathcurveto{\pgfqpoint{4.068802in}{0.869489in}}{\pgfqpoint{4.058203in}{0.873879in}}{\pgfqpoint{4.047152in}{0.873879in}}%
\pgfpathcurveto{\pgfqpoint{4.036102in}{0.873879in}}{\pgfqpoint{4.025503in}{0.869489in}}{\pgfqpoint{4.017690in}{0.861675in}}%
\pgfpathcurveto{\pgfqpoint{4.009876in}{0.853862in}}{\pgfqpoint{4.005486in}{0.843263in}}{\pgfqpoint{4.005486in}{0.832213in}}%
\pgfpathcurveto{\pgfqpoint{4.005486in}{0.821163in}}{\pgfqpoint{4.009876in}{0.810564in}}{\pgfqpoint{4.017690in}{0.802750in}}%
\pgfpathcurveto{\pgfqpoint{4.025503in}{0.794936in}}{\pgfqpoint{4.036102in}{0.790546in}}{\pgfqpoint{4.047152in}{0.790546in}}%
\pgfpathlineto{\pgfqpoint{4.047152in}{0.790546in}}%
\pgfpathclose%
\pgfusepath{stroke}%
\end{pgfscope}%
\begin{pgfscope}%
\pgfpathrectangle{\pgfqpoint{0.847223in}{0.554012in}}{\pgfqpoint{6.200000in}{4.530000in}}%
\pgfusepath{clip}%
\pgfsetbuttcap%
\pgfsetroundjoin%
\pgfsetlinewidth{1.003750pt}%
\definecolor{currentstroke}{rgb}{1.000000,0.000000,0.000000}%
\pgfsetstrokecolor{currentstroke}%
\pgfsetdash{}{0pt}%
\pgfpathmoveto{\pgfqpoint{4.052486in}{0.789551in}}%
\pgfpathcurveto{\pgfqpoint{4.063536in}{0.789551in}}{\pgfqpoint{4.074135in}{0.793942in}}{\pgfqpoint{4.081948in}{0.801755in}}%
\pgfpathcurveto{\pgfqpoint{4.089762in}{0.809569in}}{\pgfqpoint{4.094152in}{0.820168in}}{\pgfqpoint{4.094152in}{0.831218in}}%
\pgfpathcurveto{\pgfqpoint{4.094152in}{0.842268in}}{\pgfqpoint{4.089762in}{0.852867in}}{\pgfqpoint{4.081948in}{0.860681in}}%
\pgfpathcurveto{\pgfqpoint{4.074135in}{0.868494in}}{\pgfqpoint{4.063536in}{0.872885in}}{\pgfqpoint{4.052486in}{0.872885in}}%
\pgfpathcurveto{\pgfqpoint{4.041436in}{0.872885in}}{\pgfqpoint{4.030837in}{0.868494in}}{\pgfqpoint{4.023023in}{0.860681in}}%
\pgfpathcurveto{\pgfqpoint{4.015209in}{0.852867in}}{\pgfqpoint{4.010819in}{0.842268in}}{\pgfqpoint{4.010819in}{0.831218in}}%
\pgfpathcurveto{\pgfqpoint{4.010819in}{0.820168in}}{\pgfqpoint{4.015209in}{0.809569in}}{\pgfqpoint{4.023023in}{0.801755in}}%
\pgfpathcurveto{\pgfqpoint{4.030837in}{0.793942in}}{\pgfqpoint{4.041436in}{0.789551in}}{\pgfqpoint{4.052486in}{0.789551in}}%
\pgfpathlineto{\pgfqpoint{4.052486in}{0.789551in}}%
\pgfpathclose%
\pgfusepath{stroke}%
\end{pgfscope}%
\begin{pgfscope}%
\pgfpathrectangle{\pgfqpoint{0.847223in}{0.554012in}}{\pgfqpoint{6.200000in}{4.530000in}}%
\pgfusepath{clip}%
\pgfsetbuttcap%
\pgfsetroundjoin%
\pgfsetlinewidth{1.003750pt}%
\definecolor{currentstroke}{rgb}{1.000000,0.000000,0.000000}%
\pgfsetstrokecolor{currentstroke}%
\pgfsetdash{}{0pt}%
\pgfpathmoveto{\pgfqpoint{4.057819in}{0.788560in}}%
\pgfpathcurveto{\pgfqpoint{4.068869in}{0.788560in}}{\pgfqpoint{4.079468in}{0.792950in}}{\pgfqpoint{4.087282in}{0.800763in}}%
\pgfpathcurveto{\pgfqpoint{4.095095in}{0.808577in}}{\pgfqpoint{4.099486in}{0.819176in}}{\pgfqpoint{4.099486in}{0.830226in}}%
\pgfpathcurveto{\pgfqpoint{4.099486in}{0.841276in}}{\pgfqpoint{4.095095in}{0.851875in}}{\pgfqpoint{4.087282in}{0.859689in}}%
\pgfpathcurveto{\pgfqpoint{4.079468in}{0.867503in}}{\pgfqpoint{4.068869in}{0.871893in}}{\pgfqpoint{4.057819in}{0.871893in}}%
\pgfpathcurveto{\pgfqpoint{4.046769in}{0.871893in}}{\pgfqpoint{4.036170in}{0.867503in}}{\pgfqpoint{4.028356in}{0.859689in}}%
\pgfpathcurveto{\pgfqpoint{4.020542in}{0.851875in}}{\pgfqpoint{4.016152in}{0.841276in}}{\pgfqpoint{4.016152in}{0.830226in}}%
\pgfpathcurveto{\pgfqpoint{4.016152in}{0.819176in}}{\pgfqpoint{4.020542in}{0.808577in}}{\pgfqpoint{4.028356in}{0.800763in}}%
\pgfpathcurveto{\pgfqpoint{4.036170in}{0.792950in}}{\pgfqpoint{4.046769in}{0.788560in}}{\pgfqpoint{4.057819in}{0.788560in}}%
\pgfpathlineto{\pgfqpoint{4.057819in}{0.788560in}}%
\pgfpathclose%
\pgfusepath{stroke}%
\end{pgfscope}%
\begin{pgfscope}%
\pgfpathrectangle{\pgfqpoint{0.847223in}{0.554012in}}{\pgfqpoint{6.200000in}{4.530000in}}%
\pgfusepath{clip}%
\pgfsetbuttcap%
\pgfsetroundjoin%
\pgfsetlinewidth{1.003750pt}%
\definecolor{currentstroke}{rgb}{1.000000,0.000000,0.000000}%
\pgfsetstrokecolor{currentstroke}%
\pgfsetdash{}{0pt}%
\pgfpathmoveto{\pgfqpoint{4.063152in}{0.787571in}}%
\pgfpathcurveto{\pgfqpoint{4.074202in}{0.787571in}}{\pgfqpoint{4.084801in}{0.791961in}}{\pgfqpoint{4.092615in}{0.799775in}}%
\pgfpathcurveto{\pgfqpoint{4.100429in}{0.807588in}}{\pgfqpoint{4.104819in}{0.818187in}}{\pgfqpoint{4.104819in}{0.829237in}}%
\pgfpathcurveto{\pgfqpoint{4.104819in}{0.840287in}}{\pgfqpoint{4.100429in}{0.850886in}}{\pgfqpoint{4.092615in}{0.858700in}}%
\pgfpathcurveto{\pgfqpoint{4.084801in}{0.866514in}}{\pgfqpoint{4.074202in}{0.870904in}}{\pgfqpoint{4.063152in}{0.870904in}}%
\pgfpathcurveto{\pgfqpoint{4.052102in}{0.870904in}}{\pgfqpoint{4.041503in}{0.866514in}}{\pgfqpoint{4.033689in}{0.858700in}}%
\pgfpathcurveto{\pgfqpoint{4.025876in}{0.850886in}}{\pgfqpoint{4.021485in}{0.840287in}}{\pgfqpoint{4.021485in}{0.829237in}}%
\pgfpathcurveto{\pgfqpoint{4.021485in}{0.818187in}}{\pgfqpoint{4.025876in}{0.807588in}}{\pgfqpoint{4.033689in}{0.799775in}}%
\pgfpathcurveto{\pgfqpoint{4.041503in}{0.791961in}}{\pgfqpoint{4.052102in}{0.787571in}}{\pgfqpoint{4.063152in}{0.787571in}}%
\pgfpathlineto{\pgfqpoint{4.063152in}{0.787571in}}%
\pgfpathclose%
\pgfusepath{stroke}%
\end{pgfscope}%
\begin{pgfscope}%
\pgfpathrectangle{\pgfqpoint{0.847223in}{0.554012in}}{\pgfqpoint{6.200000in}{4.530000in}}%
\pgfusepath{clip}%
\pgfsetbuttcap%
\pgfsetroundjoin%
\pgfsetlinewidth{1.003750pt}%
\definecolor{currentstroke}{rgb}{1.000000,0.000000,0.000000}%
\pgfsetstrokecolor{currentstroke}%
\pgfsetdash{}{0pt}%
\pgfpathmoveto{\pgfqpoint{4.068485in}{0.786584in}}%
\pgfpathcurveto{\pgfqpoint{4.079535in}{0.786584in}}{\pgfqpoint{4.090134in}{0.790975in}}{\pgfqpoint{4.097948in}{0.798788in}}%
\pgfpathcurveto{\pgfqpoint{4.105762in}{0.806602in}}{\pgfqpoint{4.110152in}{0.817201in}}{\pgfqpoint{4.110152in}{0.828251in}}%
\pgfpathcurveto{\pgfqpoint{4.110152in}{0.839301in}}{\pgfqpoint{4.105762in}{0.849900in}}{\pgfqpoint{4.097948in}{0.857714in}}%
\pgfpathcurveto{\pgfqpoint{4.090134in}{0.865528in}}{\pgfqpoint{4.079535in}{0.869918in}}{\pgfqpoint{4.068485in}{0.869918in}}%
\pgfpathcurveto{\pgfqpoint{4.057435in}{0.869918in}}{\pgfqpoint{4.046836in}{0.865528in}}{\pgfqpoint{4.039023in}{0.857714in}}%
\pgfpathcurveto{\pgfqpoint{4.031209in}{0.849900in}}{\pgfqpoint{4.026819in}{0.839301in}}{\pgfqpoint{4.026819in}{0.828251in}}%
\pgfpathcurveto{\pgfqpoint{4.026819in}{0.817201in}}{\pgfqpoint{4.031209in}{0.806602in}}{\pgfqpoint{4.039023in}{0.798788in}}%
\pgfpathcurveto{\pgfqpoint{4.046836in}{0.790975in}}{\pgfqpoint{4.057435in}{0.786584in}}{\pgfqpoint{4.068485in}{0.786584in}}%
\pgfpathlineto{\pgfqpoint{4.068485in}{0.786584in}}%
\pgfpathclose%
\pgfusepath{stroke}%
\end{pgfscope}%
\begin{pgfscope}%
\pgfpathrectangle{\pgfqpoint{0.847223in}{0.554012in}}{\pgfqpoint{6.200000in}{4.530000in}}%
\pgfusepath{clip}%
\pgfsetbuttcap%
\pgfsetroundjoin%
\pgfsetlinewidth{1.003750pt}%
\definecolor{currentstroke}{rgb}{1.000000,0.000000,0.000000}%
\pgfsetstrokecolor{currentstroke}%
\pgfsetdash{}{0pt}%
\pgfpathmoveto{\pgfqpoint{4.073819in}{0.785601in}}%
\pgfpathcurveto{\pgfqpoint{4.084869in}{0.785601in}}{\pgfqpoint{4.095468in}{0.789991in}}{\pgfqpoint{4.103281in}{0.797805in}}%
\pgfpathcurveto{\pgfqpoint{4.111095in}{0.805619in}}{\pgfqpoint{4.115485in}{0.816218in}}{\pgfqpoint{4.115485in}{0.827268in}}%
\pgfpathcurveto{\pgfqpoint{4.115485in}{0.838318in}}{\pgfqpoint{4.111095in}{0.848917in}}{\pgfqpoint{4.103281in}{0.856731in}}%
\pgfpathcurveto{\pgfqpoint{4.095468in}{0.864544in}}{\pgfqpoint{4.084869in}{0.868934in}}{\pgfqpoint{4.073819in}{0.868934in}}%
\pgfpathcurveto{\pgfqpoint{4.062768in}{0.868934in}}{\pgfqpoint{4.052169in}{0.864544in}}{\pgfqpoint{4.044356in}{0.856731in}}%
\pgfpathcurveto{\pgfqpoint{4.036542in}{0.848917in}}{\pgfqpoint{4.032152in}{0.838318in}}{\pgfqpoint{4.032152in}{0.827268in}}%
\pgfpathcurveto{\pgfqpoint{4.032152in}{0.816218in}}{\pgfqpoint{4.036542in}{0.805619in}}{\pgfqpoint{4.044356in}{0.797805in}}%
\pgfpathcurveto{\pgfqpoint{4.052169in}{0.789991in}}{\pgfqpoint{4.062768in}{0.785601in}}{\pgfqpoint{4.073819in}{0.785601in}}%
\pgfpathlineto{\pgfqpoint{4.073819in}{0.785601in}}%
\pgfpathclose%
\pgfusepath{stroke}%
\end{pgfscope}%
\begin{pgfscope}%
\pgfpathrectangle{\pgfqpoint{0.847223in}{0.554012in}}{\pgfqpoint{6.200000in}{4.530000in}}%
\pgfusepath{clip}%
\pgfsetbuttcap%
\pgfsetroundjoin%
\pgfsetlinewidth{1.003750pt}%
\definecolor{currentstroke}{rgb}{1.000000,0.000000,0.000000}%
\pgfsetstrokecolor{currentstroke}%
\pgfsetdash{}{0pt}%
\pgfpathmoveto{\pgfqpoint{4.079152in}{0.784621in}}%
\pgfpathcurveto{\pgfqpoint{4.090202in}{0.784621in}}{\pgfqpoint{4.100801in}{0.789011in}}{\pgfqpoint{4.108615in}{0.796824in}}%
\pgfpathcurveto{\pgfqpoint{4.116428in}{0.804638in}}{\pgfqpoint{4.120818in}{0.815237in}}{\pgfqpoint{4.120818in}{0.826287in}}%
\pgfpathcurveto{\pgfqpoint{4.120818in}{0.837337in}}{\pgfqpoint{4.116428in}{0.847936in}}{\pgfqpoint{4.108615in}{0.855750in}}%
\pgfpathcurveto{\pgfqpoint{4.100801in}{0.863564in}}{\pgfqpoint{4.090202in}{0.867954in}}{\pgfqpoint{4.079152in}{0.867954in}}%
\pgfpathcurveto{\pgfqpoint{4.068102in}{0.867954in}}{\pgfqpoint{4.057503in}{0.863564in}}{\pgfqpoint{4.049689in}{0.855750in}}%
\pgfpathcurveto{\pgfqpoint{4.041875in}{0.847936in}}{\pgfqpoint{4.037485in}{0.837337in}}{\pgfqpoint{4.037485in}{0.826287in}}%
\pgfpathcurveto{\pgfqpoint{4.037485in}{0.815237in}}{\pgfqpoint{4.041875in}{0.804638in}}{\pgfqpoint{4.049689in}{0.796824in}}%
\pgfpathcurveto{\pgfqpoint{4.057503in}{0.789011in}}{\pgfqpoint{4.068102in}{0.784621in}}{\pgfqpoint{4.079152in}{0.784621in}}%
\pgfpathlineto{\pgfqpoint{4.079152in}{0.784621in}}%
\pgfpathclose%
\pgfusepath{stroke}%
\end{pgfscope}%
\begin{pgfscope}%
\pgfpathrectangle{\pgfqpoint{0.847223in}{0.554012in}}{\pgfqpoint{6.200000in}{4.530000in}}%
\pgfusepath{clip}%
\pgfsetbuttcap%
\pgfsetroundjoin%
\pgfsetlinewidth{1.003750pt}%
\definecolor{currentstroke}{rgb}{1.000000,0.000000,0.000000}%
\pgfsetstrokecolor{currentstroke}%
\pgfsetdash{}{0pt}%
\pgfpathmoveto{\pgfqpoint{4.084485in}{0.783643in}}%
\pgfpathcurveto{\pgfqpoint{4.095535in}{0.783643in}}{\pgfqpoint{4.106134in}{0.788033in}}{\pgfqpoint{4.113948in}{0.795847in}}%
\pgfpathcurveto{\pgfqpoint{4.121761in}{0.803660in}}{\pgfqpoint{4.126152in}{0.814259in}}{\pgfqpoint{4.126152in}{0.825309in}}%
\pgfpathcurveto{\pgfqpoint{4.126152in}{0.836360in}}{\pgfqpoint{4.121761in}{0.846959in}}{\pgfqpoint{4.113948in}{0.854772in}}%
\pgfpathcurveto{\pgfqpoint{4.106134in}{0.862586in}}{\pgfqpoint{4.095535in}{0.866976in}}{\pgfqpoint{4.084485in}{0.866976in}}%
\pgfpathcurveto{\pgfqpoint{4.073435in}{0.866976in}}{\pgfqpoint{4.062836in}{0.862586in}}{\pgfqpoint{4.055022in}{0.854772in}}%
\pgfpathcurveto{\pgfqpoint{4.047209in}{0.846959in}}{\pgfqpoint{4.042818in}{0.836360in}}{\pgfqpoint{4.042818in}{0.825309in}}%
\pgfpathcurveto{\pgfqpoint{4.042818in}{0.814259in}}{\pgfqpoint{4.047209in}{0.803660in}}{\pgfqpoint{4.055022in}{0.795847in}}%
\pgfpathcurveto{\pgfqpoint{4.062836in}{0.788033in}}{\pgfqpoint{4.073435in}{0.783643in}}{\pgfqpoint{4.084485in}{0.783643in}}%
\pgfpathlineto{\pgfqpoint{4.084485in}{0.783643in}}%
\pgfpathclose%
\pgfusepath{stroke}%
\end{pgfscope}%
\begin{pgfscope}%
\pgfpathrectangle{\pgfqpoint{0.847223in}{0.554012in}}{\pgfqpoint{6.200000in}{4.530000in}}%
\pgfusepath{clip}%
\pgfsetbuttcap%
\pgfsetroundjoin%
\pgfsetlinewidth{1.003750pt}%
\definecolor{currentstroke}{rgb}{1.000000,0.000000,0.000000}%
\pgfsetstrokecolor{currentstroke}%
\pgfsetdash{}{0pt}%
\pgfpathmoveto{\pgfqpoint{4.089818in}{0.782668in}}%
\pgfpathcurveto{\pgfqpoint{4.100868in}{0.782668in}}{\pgfqpoint{4.111467in}{0.787058in}}{\pgfqpoint{4.119281in}{0.794872in}}%
\pgfpathcurveto{\pgfqpoint{4.127095in}{0.802685in}}{\pgfqpoint{4.131485in}{0.813284in}}{\pgfqpoint{4.131485in}{0.824334in}}%
\pgfpathcurveto{\pgfqpoint{4.131485in}{0.835385in}}{\pgfqpoint{4.127095in}{0.845984in}}{\pgfqpoint{4.119281in}{0.853797in}}%
\pgfpathcurveto{\pgfqpoint{4.111467in}{0.861611in}}{\pgfqpoint{4.100868in}{0.866001in}}{\pgfqpoint{4.089818in}{0.866001in}}%
\pgfpathcurveto{\pgfqpoint{4.078768in}{0.866001in}}{\pgfqpoint{4.068169in}{0.861611in}}{\pgfqpoint{4.060355in}{0.853797in}}%
\pgfpathcurveto{\pgfqpoint{4.052542in}{0.845984in}}{\pgfqpoint{4.048152in}{0.835385in}}{\pgfqpoint{4.048152in}{0.824334in}}%
\pgfpathcurveto{\pgfqpoint{4.048152in}{0.813284in}}{\pgfqpoint{4.052542in}{0.802685in}}{\pgfqpoint{4.060355in}{0.794872in}}%
\pgfpathcurveto{\pgfqpoint{4.068169in}{0.787058in}}{\pgfqpoint{4.078768in}{0.782668in}}{\pgfqpoint{4.089818in}{0.782668in}}%
\pgfpathlineto{\pgfqpoint{4.089818in}{0.782668in}}%
\pgfpathclose%
\pgfusepath{stroke}%
\end{pgfscope}%
\begin{pgfscope}%
\pgfpathrectangle{\pgfqpoint{0.847223in}{0.554012in}}{\pgfqpoint{6.200000in}{4.530000in}}%
\pgfusepath{clip}%
\pgfsetbuttcap%
\pgfsetroundjoin%
\pgfsetlinewidth{1.003750pt}%
\definecolor{currentstroke}{rgb}{1.000000,0.000000,0.000000}%
\pgfsetstrokecolor{currentstroke}%
\pgfsetdash{}{0pt}%
\pgfpathmoveto{\pgfqpoint{4.095151in}{0.781696in}}%
\pgfpathcurveto{\pgfqpoint{4.106202in}{0.781696in}}{\pgfqpoint{4.116801in}{0.786086in}}{\pgfqpoint{4.124614in}{0.793899in}}%
\pgfpathcurveto{\pgfqpoint{4.132428in}{0.801713in}}{\pgfqpoint{4.136818in}{0.812312in}}{\pgfqpoint{4.136818in}{0.823362in}}%
\pgfpathcurveto{\pgfqpoint{4.136818in}{0.834412in}}{\pgfqpoint{4.132428in}{0.845011in}}{\pgfqpoint{4.124614in}{0.852825in}}%
\pgfpathcurveto{\pgfqpoint{4.116801in}{0.860639in}}{\pgfqpoint{4.106202in}{0.865029in}}{\pgfqpoint{4.095151in}{0.865029in}}%
\pgfpathcurveto{\pgfqpoint{4.084101in}{0.865029in}}{\pgfqpoint{4.073502in}{0.860639in}}{\pgfqpoint{4.065689in}{0.852825in}}%
\pgfpathcurveto{\pgfqpoint{4.057875in}{0.845011in}}{\pgfqpoint{4.053485in}{0.834412in}}{\pgfqpoint{4.053485in}{0.823362in}}%
\pgfpathcurveto{\pgfqpoint{4.053485in}{0.812312in}}{\pgfqpoint{4.057875in}{0.801713in}}{\pgfqpoint{4.065689in}{0.793899in}}%
\pgfpathcurveto{\pgfqpoint{4.073502in}{0.786086in}}{\pgfqpoint{4.084101in}{0.781696in}}{\pgfqpoint{4.095151in}{0.781696in}}%
\pgfpathlineto{\pgfqpoint{4.095151in}{0.781696in}}%
\pgfpathclose%
\pgfusepath{stroke}%
\end{pgfscope}%
\begin{pgfscope}%
\pgfpathrectangle{\pgfqpoint{0.847223in}{0.554012in}}{\pgfqpoint{6.200000in}{4.530000in}}%
\pgfusepath{clip}%
\pgfsetbuttcap%
\pgfsetroundjoin%
\pgfsetlinewidth{1.003750pt}%
\definecolor{currentstroke}{rgb}{1.000000,0.000000,0.000000}%
\pgfsetstrokecolor{currentstroke}%
\pgfsetdash{}{0pt}%
\pgfpathmoveto{\pgfqpoint{4.100485in}{0.780726in}}%
\pgfpathcurveto{\pgfqpoint{4.111535in}{0.780726in}}{\pgfqpoint{4.122134in}{0.785116in}}{\pgfqpoint{4.129947in}{0.792930in}}%
\pgfpathcurveto{\pgfqpoint{4.137761in}{0.800744in}}{\pgfqpoint{4.142151in}{0.811343in}}{\pgfqpoint{4.142151in}{0.822393in}}%
\pgfpathcurveto{\pgfqpoint{4.142151in}{0.833443in}}{\pgfqpoint{4.137761in}{0.844042in}}{\pgfqpoint{4.129947in}{0.851855in}}%
\pgfpathcurveto{\pgfqpoint{4.122134in}{0.859669in}}{\pgfqpoint{4.111535in}{0.864059in}}{\pgfqpoint{4.100485in}{0.864059in}}%
\pgfpathcurveto{\pgfqpoint{4.089434in}{0.864059in}}{\pgfqpoint{4.078835in}{0.859669in}}{\pgfqpoint{4.071022in}{0.851855in}}%
\pgfpathcurveto{\pgfqpoint{4.063208in}{0.844042in}}{\pgfqpoint{4.058818in}{0.833443in}}{\pgfqpoint{4.058818in}{0.822393in}}%
\pgfpathcurveto{\pgfqpoint{4.058818in}{0.811343in}}{\pgfqpoint{4.063208in}{0.800744in}}{\pgfqpoint{4.071022in}{0.792930in}}%
\pgfpathcurveto{\pgfqpoint{4.078835in}{0.785116in}}{\pgfqpoint{4.089434in}{0.780726in}}{\pgfqpoint{4.100485in}{0.780726in}}%
\pgfpathlineto{\pgfqpoint{4.100485in}{0.780726in}}%
\pgfpathclose%
\pgfusepath{stroke}%
\end{pgfscope}%
\begin{pgfscope}%
\pgfpathrectangle{\pgfqpoint{0.847223in}{0.554012in}}{\pgfqpoint{6.200000in}{4.530000in}}%
\pgfusepath{clip}%
\pgfsetbuttcap%
\pgfsetroundjoin%
\pgfsetlinewidth{1.003750pt}%
\definecolor{currentstroke}{rgb}{1.000000,0.000000,0.000000}%
\pgfsetstrokecolor{currentstroke}%
\pgfsetdash{}{0pt}%
\pgfpathmoveto{\pgfqpoint{4.105818in}{0.779759in}}%
\pgfpathcurveto{\pgfqpoint{4.116868in}{0.779759in}}{\pgfqpoint{4.127467in}{0.784150in}}{\pgfqpoint{4.135281in}{0.791963in}}%
\pgfpathcurveto{\pgfqpoint{4.143094in}{0.799777in}}{\pgfqpoint{4.147484in}{0.810376in}}{\pgfqpoint{4.147484in}{0.821426in}}%
\pgfpathcurveto{\pgfqpoint{4.147484in}{0.832476in}}{\pgfqpoint{4.143094in}{0.843075in}}{\pgfqpoint{4.135281in}{0.850889in}}%
\pgfpathcurveto{\pgfqpoint{4.127467in}{0.858702in}}{\pgfqpoint{4.116868in}{0.863093in}}{\pgfqpoint{4.105818in}{0.863093in}}%
\pgfpathcurveto{\pgfqpoint{4.094768in}{0.863093in}}{\pgfqpoint{4.084169in}{0.858702in}}{\pgfqpoint{4.076355in}{0.850889in}}%
\pgfpathcurveto{\pgfqpoint{4.068541in}{0.843075in}}{\pgfqpoint{4.064151in}{0.832476in}}{\pgfqpoint{4.064151in}{0.821426in}}%
\pgfpathcurveto{\pgfqpoint{4.064151in}{0.810376in}}{\pgfqpoint{4.068541in}{0.799777in}}{\pgfqpoint{4.076355in}{0.791963in}}%
\pgfpathcurveto{\pgfqpoint{4.084169in}{0.784150in}}{\pgfqpoint{4.094768in}{0.779759in}}{\pgfqpoint{4.105818in}{0.779759in}}%
\pgfpathlineto{\pgfqpoint{4.105818in}{0.779759in}}%
\pgfpathclose%
\pgfusepath{stroke}%
\end{pgfscope}%
\begin{pgfscope}%
\pgfpathrectangle{\pgfqpoint{0.847223in}{0.554012in}}{\pgfqpoint{6.200000in}{4.530000in}}%
\pgfusepath{clip}%
\pgfsetbuttcap%
\pgfsetroundjoin%
\pgfsetlinewidth{1.003750pt}%
\definecolor{currentstroke}{rgb}{1.000000,0.000000,0.000000}%
\pgfsetstrokecolor{currentstroke}%
\pgfsetdash{}{0pt}%
\pgfpathmoveto{\pgfqpoint{4.111151in}{0.778795in}}%
\pgfpathcurveto{\pgfqpoint{4.122201in}{0.778795in}}{\pgfqpoint{4.132800in}{0.783185in}}{\pgfqpoint{4.140614in}{0.790999in}}%
\pgfpathcurveto{\pgfqpoint{4.148427in}{0.798813in}}{\pgfqpoint{4.152818in}{0.809412in}}{\pgfqpoint{4.152818in}{0.820462in}}%
\pgfpathcurveto{\pgfqpoint{4.152818in}{0.831512in}}{\pgfqpoint{4.148427in}{0.842111in}}{\pgfqpoint{4.140614in}{0.849925in}}%
\pgfpathcurveto{\pgfqpoint{4.132800in}{0.857738in}}{\pgfqpoint{4.122201in}{0.862129in}}{\pgfqpoint{4.111151in}{0.862129in}}%
\pgfpathcurveto{\pgfqpoint{4.100101in}{0.862129in}}{\pgfqpoint{4.089502in}{0.857738in}}{\pgfqpoint{4.081688in}{0.849925in}}%
\pgfpathcurveto{\pgfqpoint{4.073875in}{0.842111in}}{\pgfqpoint{4.069484in}{0.831512in}}{\pgfqpoint{4.069484in}{0.820462in}}%
\pgfpathcurveto{\pgfqpoint{4.069484in}{0.809412in}}{\pgfqpoint{4.073875in}{0.798813in}}{\pgfqpoint{4.081688in}{0.790999in}}%
\pgfpathcurveto{\pgfqpoint{4.089502in}{0.783185in}}{\pgfqpoint{4.100101in}{0.778795in}}{\pgfqpoint{4.111151in}{0.778795in}}%
\pgfpathlineto{\pgfqpoint{4.111151in}{0.778795in}}%
\pgfpathclose%
\pgfusepath{stroke}%
\end{pgfscope}%
\begin{pgfscope}%
\pgfpathrectangle{\pgfqpoint{0.847223in}{0.554012in}}{\pgfqpoint{6.200000in}{4.530000in}}%
\pgfusepath{clip}%
\pgfsetbuttcap%
\pgfsetroundjoin%
\pgfsetlinewidth{1.003750pt}%
\definecolor{currentstroke}{rgb}{1.000000,0.000000,0.000000}%
\pgfsetstrokecolor{currentstroke}%
\pgfsetdash{}{0pt}%
\pgfpathmoveto{\pgfqpoint{4.116484in}{0.777834in}}%
\pgfpathcurveto{\pgfqpoint{4.127534in}{0.777834in}}{\pgfqpoint{4.138133in}{0.782224in}}{\pgfqpoint{4.145947in}{0.790038in}}%
\pgfpathcurveto{\pgfqpoint{4.153761in}{0.797851in}}{\pgfqpoint{4.158151in}{0.808450in}}{\pgfqpoint{4.158151in}{0.819501in}}%
\pgfpathcurveto{\pgfqpoint{4.158151in}{0.830551in}}{\pgfqpoint{4.153761in}{0.841150in}}{\pgfqpoint{4.145947in}{0.848963in}}%
\pgfpathcurveto{\pgfqpoint{4.138133in}{0.856777in}}{\pgfqpoint{4.127534in}{0.861167in}}{\pgfqpoint{4.116484in}{0.861167in}}%
\pgfpathcurveto{\pgfqpoint{4.105434in}{0.861167in}}{\pgfqpoint{4.094835in}{0.856777in}}{\pgfqpoint{4.087021in}{0.848963in}}%
\pgfpathcurveto{\pgfqpoint{4.079208in}{0.841150in}}{\pgfqpoint{4.074818in}{0.830551in}}{\pgfqpoint{4.074818in}{0.819501in}}%
\pgfpathcurveto{\pgfqpoint{4.074818in}{0.808450in}}{\pgfqpoint{4.079208in}{0.797851in}}{\pgfqpoint{4.087021in}{0.790038in}}%
\pgfpathcurveto{\pgfqpoint{4.094835in}{0.782224in}}{\pgfqpoint{4.105434in}{0.777834in}}{\pgfqpoint{4.116484in}{0.777834in}}%
\pgfpathlineto{\pgfqpoint{4.116484in}{0.777834in}}%
\pgfpathclose%
\pgfusepath{stroke}%
\end{pgfscope}%
\begin{pgfscope}%
\pgfpathrectangle{\pgfqpoint{0.847223in}{0.554012in}}{\pgfqpoint{6.200000in}{4.530000in}}%
\pgfusepath{clip}%
\pgfsetbuttcap%
\pgfsetroundjoin%
\pgfsetlinewidth{1.003750pt}%
\definecolor{currentstroke}{rgb}{1.000000,0.000000,0.000000}%
\pgfsetstrokecolor{currentstroke}%
\pgfsetdash{}{0pt}%
\pgfpathmoveto{\pgfqpoint{4.121817in}{0.776875in}}%
\pgfpathcurveto{\pgfqpoint{4.132868in}{0.776875in}}{\pgfqpoint{4.143467in}{0.781266in}}{\pgfqpoint{4.151280in}{0.789079in}}%
\pgfpathcurveto{\pgfqpoint{4.159094in}{0.796893in}}{\pgfqpoint{4.163484in}{0.807492in}}{\pgfqpoint{4.163484in}{0.818542in}}%
\pgfpathcurveto{\pgfqpoint{4.163484in}{0.829592in}}{\pgfqpoint{4.159094in}{0.840191in}}{\pgfqpoint{4.151280in}{0.848005in}}%
\pgfpathcurveto{\pgfqpoint{4.143467in}{0.855818in}}{\pgfqpoint{4.132868in}{0.860209in}}{\pgfqpoint{4.121817in}{0.860209in}}%
\pgfpathcurveto{\pgfqpoint{4.110767in}{0.860209in}}{\pgfqpoint{4.100168in}{0.855818in}}{\pgfqpoint{4.092355in}{0.848005in}}%
\pgfpathcurveto{\pgfqpoint{4.084541in}{0.840191in}}{\pgfqpoint{4.080151in}{0.829592in}}{\pgfqpoint{4.080151in}{0.818542in}}%
\pgfpathcurveto{\pgfqpoint{4.080151in}{0.807492in}}{\pgfqpoint{4.084541in}{0.796893in}}{\pgfqpoint{4.092355in}{0.789079in}}%
\pgfpathcurveto{\pgfqpoint{4.100168in}{0.781266in}}{\pgfqpoint{4.110767in}{0.776875in}}{\pgfqpoint{4.121817in}{0.776875in}}%
\pgfpathlineto{\pgfqpoint{4.121817in}{0.776875in}}%
\pgfpathclose%
\pgfusepath{stroke}%
\end{pgfscope}%
\begin{pgfscope}%
\pgfpathrectangle{\pgfqpoint{0.847223in}{0.554012in}}{\pgfqpoint{6.200000in}{4.530000in}}%
\pgfusepath{clip}%
\pgfsetbuttcap%
\pgfsetroundjoin%
\pgfsetlinewidth{1.003750pt}%
\definecolor{currentstroke}{rgb}{1.000000,0.000000,0.000000}%
\pgfsetstrokecolor{currentstroke}%
\pgfsetdash{}{0pt}%
\pgfpathmoveto{\pgfqpoint{4.127151in}{0.775919in}}%
\pgfpathcurveto{\pgfqpoint{4.138201in}{0.775919in}}{\pgfqpoint{4.148800in}{0.780310in}}{\pgfqpoint{4.156613in}{0.788123in}}%
\pgfpathcurveto{\pgfqpoint{4.164427in}{0.795937in}}{\pgfqpoint{4.168817in}{0.806536in}}{\pgfqpoint{4.168817in}{0.817586in}}%
\pgfpathcurveto{\pgfqpoint{4.168817in}{0.828636in}}{\pgfqpoint{4.164427in}{0.839235in}}{\pgfqpoint{4.156613in}{0.847049in}}%
\pgfpathcurveto{\pgfqpoint{4.148800in}{0.854862in}}{\pgfqpoint{4.138201in}{0.859253in}}{\pgfqpoint{4.127151in}{0.859253in}}%
\pgfpathcurveto{\pgfqpoint{4.116101in}{0.859253in}}{\pgfqpoint{4.105502in}{0.854862in}}{\pgfqpoint{4.097688in}{0.847049in}}%
\pgfpathcurveto{\pgfqpoint{4.089874in}{0.839235in}}{\pgfqpoint{4.085484in}{0.828636in}}{\pgfqpoint{4.085484in}{0.817586in}}%
\pgfpathcurveto{\pgfqpoint{4.085484in}{0.806536in}}{\pgfqpoint{4.089874in}{0.795937in}}{\pgfqpoint{4.097688in}{0.788123in}}%
\pgfpathcurveto{\pgfqpoint{4.105502in}{0.780310in}}{\pgfqpoint{4.116101in}{0.775919in}}{\pgfqpoint{4.127151in}{0.775919in}}%
\pgfpathlineto{\pgfqpoint{4.127151in}{0.775919in}}%
\pgfpathclose%
\pgfusepath{stroke}%
\end{pgfscope}%
\begin{pgfscope}%
\pgfpathrectangle{\pgfqpoint{0.847223in}{0.554012in}}{\pgfqpoint{6.200000in}{4.530000in}}%
\pgfusepath{clip}%
\pgfsetbuttcap%
\pgfsetroundjoin%
\pgfsetlinewidth{1.003750pt}%
\definecolor{currentstroke}{rgb}{1.000000,0.000000,0.000000}%
\pgfsetstrokecolor{currentstroke}%
\pgfsetdash{}{0pt}%
\pgfpathmoveto{\pgfqpoint{4.132484in}{0.774966in}}%
\pgfpathcurveto{\pgfqpoint{4.143534in}{0.774966in}}{\pgfqpoint{4.154133in}{0.779356in}}{\pgfqpoint{4.161947in}{0.787170in}}%
\pgfpathcurveto{\pgfqpoint{4.169760in}{0.794984in}}{\pgfqpoint{4.174151in}{0.805583in}}{\pgfqpoint{4.174151in}{0.816633in}}%
\pgfpathcurveto{\pgfqpoint{4.174151in}{0.827683in}}{\pgfqpoint{4.169760in}{0.838282in}}{\pgfqpoint{4.161947in}{0.846095in}}%
\pgfpathcurveto{\pgfqpoint{4.154133in}{0.853909in}}{\pgfqpoint{4.143534in}{0.858299in}}{\pgfqpoint{4.132484in}{0.858299in}}%
\pgfpathcurveto{\pgfqpoint{4.121434in}{0.858299in}}{\pgfqpoint{4.110835in}{0.853909in}}{\pgfqpoint{4.103021in}{0.846095in}}%
\pgfpathcurveto{\pgfqpoint{4.095207in}{0.838282in}}{\pgfqpoint{4.090817in}{0.827683in}}{\pgfqpoint{4.090817in}{0.816633in}}%
\pgfpathcurveto{\pgfqpoint{4.090817in}{0.805583in}}{\pgfqpoint{4.095207in}{0.794984in}}{\pgfqpoint{4.103021in}{0.787170in}}%
\pgfpathcurveto{\pgfqpoint{4.110835in}{0.779356in}}{\pgfqpoint{4.121434in}{0.774966in}}{\pgfqpoint{4.132484in}{0.774966in}}%
\pgfpathlineto{\pgfqpoint{4.132484in}{0.774966in}}%
\pgfpathclose%
\pgfusepath{stroke}%
\end{pgfscope}%
\begin{pgfscope}%
\pgfpathrectangle{\pgfqpoint{0.847223in}{0.554012in}}{\pgfqpoint{6.200000in}{4.530000in}}%
\pgfusepath{clip}%
\pgfsetbuttcap%
\pgfsetroundjoin%
\pgfsetlinewidth{1.003750pt}%
\definecolor{currentstroke}{rgb}{1.000000,0.000000,0.000000}%
\pgfsetstrokecolor{currentstroke}%
\pgfsetdash{}{0pt}%
\pgfpathmoveto{\pgfqpoint{4.137817in}{0.774015in}}%
\pgfpathcurveto{\pgfqpoint{4.148867in}{0.774015in}}{\pgfqpoint{4.159466in}{0.778406in}}{\pgfqpoint{4.167280in}{0.786219in}}%
\pgfpathcurveto{\pgfqpoint{4.175094in}{0.794033in}}{\pgfqpoint{4.179484in}{0.804632in}}{\pgfqpoint{4.179484in}{0.815682in}}%
\pgfpathcurveto{\pgfqpoint{4.179484in}{0.826732in}}{\pgfqpoint{4.175094in}{0.837331in}}{\pgfqpoint{4.167280in}{0.845145in}}%
\pgfpathcurveto{\pgfqpoint{4.159466in}{0.852959in}}{\pgfqpoint{4.148867in}{0.857349in}}{\pgfqpoint{4.137817in}{0.857349in}}%
\pgfpathcurveto{\pgfqpoint{4.126767in}{0.857349in}}{\pgfqpoint{4.116168in}{0.852959in}}{\pgfqpoint{4.108354in}{0.845145in}}%
\pgfpathcurveto{\pgfqpoint{4.100541in}{0.837331in}}{\pgfqpoint{4.096150in}{0.826732in}}{\pgfqpoint{4.096150in}{0.815682in}}%
\pgfpathcurveto{\pgfqpoint{4.096150in}{0.804632in}}{\pgfqpoint{4.100541in}{0.794033in}}{\pgfqpoint{4.108354in}{0.786219in}}%
\pgfpathcurveto{\pgfqpoint{4.116168in}{0.778406in}}{\pgfqpoint{4.126767in}{0.774015in}}{\pgfqpoint{4.137817in}{0.774015in}}%
\pgfpathlineto{\pgfqpoint{4.137817in}{0.774015in}}%
\pgfpathclose%
\pgfusepath{stroke}%
\end{pgfscope}%
\begin{pgfscope}%
\pgfpathrectangle{\pgfqpoint{0.847223in}{0.554012in}}{\pgfqpoint{6.200000in}{4.530000in}}%
\pgfusepath{clip}%
\pgfsetbuttcap%
\pgfsetroundjoin%
\pgfsetlinewidth{1.003750pt}%
\definecolor{currentstroke}{rgb}{1.000000,0.000000,0.000000}%
\pgfsetstrokecolor{currentstroke}%
\pgfsetdash{}{0pt}%
\pgfpathmoveto{\pgfqpoint{4.143150in}{0.773067in}}%
\pgfpathcurveto{\pgfqpoint{4.154200in}{0.773067in}}{\pgfqpoint{4.164799in}{0.777458in}}{\pgfqpoint{4.172613in}{0.785271in}}%
\pgfpathcurveto{\pgfqpoint{4.180427in}{0.793085in}}{\pgfqpoint{4.184817in}{0.803684in}}{\pgfqpoint{4.184817in}{0.814734in}}%
\pgfpathcurveto{\pgfqpoint{4.184817in}{0.825784in}}{\pgfqpoint{4.180427in}{0.836383in}}{\pgfqpoint{4.172613in}{0.844197in}}%
\pgfpathcurveto{\pgfqpoint{4.164799in}{0.852011in}}{\pgfqpoint{4.154200in}{0.856401in}}{\pgfqpoint{4.143150in}{0.856401in}}%
\pgfpathcurveto{\pgfqpoint{4.132100in}{0.856401in}}{\pgfqpoint{4.121501in}{0.852011in}}{\pgfqpoint{4.113688in}{0.844197in}}%
\pgfpathcurveto{\pgfqpoint{4.105874in}{0.836383in}}{\pgfqpoint{4.101484in}{0.825784in}}{\pgfqpoint{4.101484in}{0.814734in}}%
\pgfpathcurveto{\pgfqpoint{4.101484in}{0.803684in}}{\pgfqpoint{4.105874in}{0.793085in}}{\pgfqpoint{4.113688in}{0.785271in}}%
\pgfpathcurveto{\pgfqpoint{4.121501in}{0.777458in}}{\pgfqpoint{4.132100in}{0.773067in}}{\pgfqpoint{4.143150in}{0.773067in}}%
\pgfpathlineto{\pgfqpoint{4.143150in}{0.773067in}}%
\pgfpathclose%
\pgfusepath{stroke}%
\end{pgfscope}%
\begin{pgfscope}%
\pgfpathrectangle{\pgfqpoint{0.847223in}{0.554012in}}{\pgfqpoint{6.200000in}{4.530000in}}%
\pgfusepath{clip}%
\pgfsetbuttcap%
\pgfsetroundjoin%
\pgfsetlinewidth{1.003750pt}%
\definecolor{currentstroke}{rgb}{1.000000,0.000000,0.000000}%
\pgfsetstrokecolor{currentstroke}%
\pgfsetdash{}{0pt}%
\pgfpathmoveto{\pgfqpoint{4.148484in}{0.772122in}}%
\pgfpathcurveto{\pgfqpoint{4.159534in}{0.772122in}}{\pgfqpoint{4.170133in}{0.776512in}}{\pgfqpoint{4.177946in}{0.784326in}}%
\pgfpathcurveto{\pgfqpoint{4.185760in}{0.792140in}}{\pgfqpoint{4.190150in}{0.802739in}}{\pgfqpoint{4.190150in}{0.813789in}}%
\pgfpathcurveto{\pgfqpoint{4.190150in}{0.824839in}}{\pgfqpoint{4.185760in}{0.835438in}}{\pgfqpoint{4.177946in}{0.843252in}}%
\pgfpathcurveto{\pgfqpoint{4.170133in}{0.851065in}}{\pgfqpoint{4.159534in}{0.855456in}}{\pgfqpoint{4.148484in}{0.855456in}}%
\pgfpathcurveto{\pgfqpoint{4.137433in}{0.855456in}}{\pgfqpoint{4.126834in}{0.851065in}}{\pgfqpoint{4.119021in}{0.843252in}}%
\pgfpathcurveto{\pgfqpoint{4.111207in}{0.835438in}}{\pgfqpoint{4.106817in}{0.824839in}}{\pgfqpoint{4.106817in}{0.813789in}}%
\pgfpathcurveto{\pgfqpoint{4.106817in}{0.802739in}}{\pgfqpoint{4.111207in}{0.792140in}}{\pgfqpoint{4.119021in}{0.784326in}}%
\pgfpathcurveto{\pgfqpoint{4.126834in}{0.776512in}}{\pgfqpoint{4.137433in}{0.772122in}}{\pgfqpoint{4.148484in}{0.772122in}}%
\pgfpathlineto{\pgfqpoint{4.148484in}{0.772122in}}%
\pgfpathclose%
\pgfusepath{stroke}%
\end{pgfscope}%
\begin{pgfscope}%
\pgfpathrectangle{\pgfqpoint{0.847223in}{0.554012in}}{\pgfqpoint{6.200000in}{4.530000in}}%
\pgfusepath{clip}%
\pgfsetbuttcap%
\pgfsetroundjoin%
\pgfsetlinewidth{1.003750pt}%
\definecolor{currentstroke}{rgb}{1.000000,0.000000,0.000000}%
\pgfsetstrokecolor{currentstroke}%
\pgfsetdash{}{0pt}%
\pgfpathmoveto{\pgfqpoint{4.153817in}{0.771180in}}%
\pgfpathcurveto{\pgfqpoint{4.164867in}{0.771180in}}{\pgfqpoint{4.175466in}{0.775570in}}{\pgfqpoint{4.183280in}{0.783383in}}%
\pgfpathcurveto{\pgfqpoint{4.191093in}{0.791197in}}{\pgfqpoint{4.195483in}{0.801796in}}{\pgfqpoint{4.195483in}{0.812846in}}%
\pgfpathcurveto{\pgfqpoint{4.195483in}{0.823896in}}{\pgfqpoint{4.191093in}{0.834495in}}{\pgfqpoint{4.183280in}{0.842309in}}%
\pgfpathcurveto{\pgfqpoint{4.175466in}{0.850123in}}{\pgfqpoint{4.164867in}{0.854513in}}{\pgfqpoint{4.153817in}{0.854513in}}%
\pgfpathcurveto{\pgfqpoint{4.142767in}{0.854513in}}{\pgfqpoint{4.132168in}{0.850123in}}{\pgfqpoint{4.124354in}{0.842309in}}%
\pgfpathcurveto{\pgfqpoint{4.116540in}{0.834495in}}{\pgfqpoint{4.112150in}{0.823896in}}{\pgfqpoint{4.112150in}{0.812846in}}%
\pgfpathcurveto{\pgfqpoint{4.112150in}{0.801796in}}{\pgfqpoint{4.116540in}{0.791197in}}{\pgfqpoint{4.124354in}{0.783383in}}%
\pgfpathcurveto{\pgfqpoint{4.132168in}{0.775570in}}{\pgfqpoint{4.142767in}{0.771180in}}{\pgfqpoint{4.153817in}{0.771180in}}%
\pgfpathlineto{\pgfqpoint{4.153817in}{0.771180in}}%
\pgfpathclose%
\pgfusepath{stroke}%
\end{pgfscope}%
\begin{pgfscope}%
\pgfpathrectangle{\pgfqpoint{0.847223in}{0.554012in}}{\pgfqpoint{6.200000in}{4.530000in}}%
\pgfusepath{clip}%
\pgfsetbuttcap%
\pgfsetroundjoin%
\pgfsetlinewidth{1.003750pt}%
\definecolor{currentstroke}{rgb}{1.000000,0.000000,0.000000}%
\pgfsetstrokecolor{currentstroke}%
\pgfsetdash{}{0pt}%
\pgfpathmoveto{\pgfqpoint{4.159150in}{0.770239in}}%
\pgfpathcurveto{\pgfqpoint{4.170200in}{0.770239in}}{\pgfqpoint{4.180799in}{0.774630in}}{\pgfqpoint{4.188613in}{0.782443in}}%
\pgfpathcurveto{\pgfqpoint{4.196426in}{0.790257in}}{\pgfqpoint{4.200817in}{0.800856in}}{\pgfqpoint{4.200817in}{0.811906in}}%
\pgfpathcurveto{\pgfqpoint{4.200817in}{0.822956in}}{\pgfqpoint{4.196426in}{0.833555in}}{\pgfqpoint{4.188613in}{0.841369in}}%
\pgfpathcurveto{\pgfqpoint{4.180799in}{0.849183in}}{\pgfqpoint{4.170200in}{0.853573in}}{\pgfqpoint{4.159150in}{0.853573in}}%
\pgfpathcurveto{\pgfqpoint{4.148100in}{0.853573in}}{\pgfqpoint{4.137501in}{0.849183in}}{\pgfqpoint{4.129687in}{0.841369in}}%
\pgfpathcurveto{\pgfqpoint{4.121874in}{0.833555in}}{\pgfqpoint{4.117483in}{0.822956in}}{\pgfqpoint{4.117483in}{0.811906in}}%
\pgfpathcurveto{\pgfqpoint{4.117483in}{0.800856in}}{\pgfqpoint{4.121874in}{0.790257in}}{\pgfqpoint{4.129687in}{0.782443in}}%
\pgfpathcurveto{\pgfqpoint{4.137501in}{0.774630in}}{\pgfqpoint{4.148100in}{0.770239in}}{\pgfqpoint{4.159150in}{0.770239in}}%
\pgfpathlineto{\pgfqpoint{4.159150in}{0.770239in}}%
\pgfpathclose%
\pgfusepath{stroke}%
\end{pgfscope}%
\begin{pgfscope}%
\pgfpathrectangle{\pgfqpoint{0.847223in}{0.554012in}}{\pgfqpoint{6.200000in}{4.530000in}}%
\pgfusepath{clip}%
\pgfsetbuttcap%
\pgfsetroundjoin%
\pgfsetlinewidth{1.003750pt}%
\definecolor{currentstroke}{rgb}{1.000000,0.000000,0.000000}%
\pgfsetstrokecolor{currentstroke}%
\pgfsetdash{}{0pt}%
\pgfpathmoveto{\pgfqpoint{4.164483in}{0.769302in}}%
\pgfpathcurveto{\pgfqpoint{4.175533in}{0.769302in}}{\pgfqpoint{4.186132in}{0.773692in}}{\pgfqpoint{4.193946in}{0.781506in}}%
\pgfpathcurveto{\pgfqpoint{4.201760in}{0.789320in}}{\pgfqpoint{4.206150in}{0.799919in}}{\pgfqpoint{4.206150in}{0.810969in}}%
\pgfpathcurveto{\pgfqpoint{4.206150in}{0.822019in}}{\pgfqpoint{4.201760in}{0.832618in}}{\pgfqpoint{4.193946in}{0.840431in}}%
\pgfpathcurveto{\pgfqpoint{4.186132in}{0.848245in}}{\pgfqpoint{4.175533in}{0.852635in}}{\pgfqpoint{4.164483in}{0.852635in}}%
\pgfpathcurveto{\pgfqpoint{4.153433in}{0.852635in}}{\pgfqpoint{4.142834in}{0.848245in}}{\pgfqpoint{4.135020in}{0.840431in}}%
\pgfpathcurveto{\pgfqpoint{4.127207in}{0.832618in}}{\pgfqpoint{4.122817in}{0.822019in}}{\pgfqpoint{4.122817in}{0.810969in}}%
\pgfpathcurveto{\pgfqpoint{4.122817in}{0.799919in}}{\pgfqpoint{4.127207in}{0.789320in}}{\pgfqpoint{4.135020in}{0.781506in}}%
\pgfpathcurveto{\pgfqpoint{4.142834in}{0.773692in}}{\pgfqpoint{4.153433in}{0.769302in}}{\pgfqpoint{4.164483in}{0.769302in}}%
\pgfpathlineto{\pgfqpoint{4.164483in}{0.769302in}}%
\pgfpathclose%
\pgfusepath{stroke}%
\end{pgfscope}%
\begin{pgfscope}%
\pgfpathrectangle{\pgfqpoint{0.847223in}{0.554012in}}{\pgfqpoint{6.200000in}{4.530000in}}%
\pgfusepath{clip}%
\pgfsetbuttcap%
\pgfsetroundjoin%
\pgfsetlinewidth{1.003750pt}%
\definecolor{currentstroke}{rgb}{1.000000,0.000000,0.000000}%
\pgfsetstrokecolor{currentstroke}%
\pgfsetdash{}{0pt}%
\pgfpathmoveto{\pgfqpoint{4.169816in}{0.768367in}}%
\pgfpathcurveto{\pgfqpoint{4.180867in}{0.768367in}}{\pgfqpoint{4.191466in}{0.772757in}}{\pgfqpoint{4.199279in}{0.780571in}}%
\pgfpathcurveto{\pgfqpoint{4.207093in}{0.788385in}}{\pgfqpoint{4.211483in}{0.798984in}}{\pgfqpoint{4.211483in}{0.810034in}}%
\pgfpathcurveto{\pgfqpoint{4.211483in}{0.821084in}}{\pgfqpoint{4.207093in}{0.831683in}}{\pgfqpoint{4.199279in}{0.839497in}}%
\pgfpathcurveto{\pgfqpoint{4.191466in}{0.847310in}}{\pgfqpoint{4.180867in}{0.851701in}}{\pgfqpoint{4.169816in}{0.851701in}}%
\pgfpathcurveto{\pgfqpoint{4.158766in}{0.851701in}}{\pgfqpoint{4.148167in}{0.847310in}}{\pgfqpoint{4.140354in}{0.839497in}}%
\pgfpathcurveto{\pgfqpoint{4.132540in}{0.831683in}}{\pgfqpoint{4.128150in}{0.821084in}}{\pgfqpoint{4.128150in}{0.810034in}}%
\pgfpathcurveto{\pgfqpoint{4.128150in}{0.798984in}}{\pgfqpoint{4.132540in}{0.788385in}}{\pgfqpoint{4.140354in}{0.780571in}}%
\pgfpathcurveto{\pgfqpoint{4.148167in}{0.772757in}}{\pgfqpoint{4.158766in}{0.768367in}}{\pgfqpoint{4.169816in}{0.768367in}}%
\pgfpathlineto{\pgfqpoint{4.169816in}{0.768367in}}%
\pgfpathclose%
\pgfusepath{stroke}%
\end{pgfscope}%
\begin{pgfscope}%
\pgfpathrectangle{\pgfqpoint{0.847223in}{0.554012in}}{\pgfqpoint{6.200000in}{4.530000in}}%
\pgfusepath{clip}%
\pgfsetbuttcap%
\pgfsetroundjoin%
\pgfsetlinewidth{1.003750pt}%
\definecolor{currentstroke}{rgb}{1.000000,0.000000,0.000000}%
\pgfsetstrokecolor{currentstroke}%
\pgfsetdash{}{0pt}%
\pgfpathmoveto{\pgfqpoint{4.175150in}{0.767435in}}%
\pgfpathcurveto{\pgfqpoint{4.186200in}{0.767435in}}{\pgfqpoint{4.196799in}{0.771825in}}{\pgfqpoint{4.204612in}{0.779639in}}%
\pgfpathcurveto{\pgfqpoint{4.212426in}{0.787452in}}{\pgfqpoint{4.216816in}{0.798051in}}{\pgfqpoint{4.216816in}{0.809102in}}%
\pgfpathcurveto{\pgfqpoint{4.216816in}{0.820152in}}{\pgfqpoint{4.212426in}{0.830751in}}{\pgfqpoint{4.204612in}{0.838564in}}%
\pgfpathcurveto{\pgfqpoint{4.196799in}{0.846378in}}{\pgfqpoint{4.186200in}{0.850768in}}{\pgfqpoint{4.175150in}{0.850768in}}%
\pgfpathcurveto{\pgfqpoint{4.164099in}{0.850768in}}{\pgfqpoint{4.153500in}{0.846378in}}{\pgfqpoint{4.145687in}{0.838564in}}%
\pgfpathcurveto{\pgfqpoint{4.137873in}{0.830751in}}{\pgfqpoint{4.133483in}{0.820152in}}{\pgfqpoint{4.133483in}{0.809102in}}%
\pgfpathcurveto{\pgfqpoint{4.133483in}{0.798051in}}{\pgfqpoint{4.137873in}{0.787452in}}{\pgfqpoint{4.145687in}{0.779639in}}%
\pgfpathcurveto{\pgfqpoint{4.153500in}{0.771825in}}{\pgfqpoint{4.164099in}{0.767435in}}{\pgfqpoint{4.175150in}{0.767435in}}%
\pgfpathlineto{\pgfqpoint{4.175150in}{0.767435in}}%
\pgfpathclose%
\pgfusepath{stroke}%
\end{pgfscope}%
\begin{pgfscope}%
\pgfpathrectangle{\pgfqpoint{0.847223in}{0.554012in}}{\pgfqpoint{6.200000in}{4.530000in}}%
\pgfusepath{clip}%
\pgfsetbuttcap%
\pgfsetroundjoin%
\pgfsetlinewidth{1.003750pt}%
\definecolor{currentstroke}{rgb}{1.000000,0.000000,0.000000}%
\pgfsetstrokecolor{currentstroke}%
\pgfsetdash{}{0pt}%
\pgfpathmoveto{\pgfqpoint{4.180483in}{0.766505in}}%
\pgfpathcurveto{\pgfqpoint{4.191533in}{0.766505in}}{\pgfqpoint{4.202132in}{0.770896in}}{\pgfqpoint{4.209946in}{0.778709in}}%
\pgfpathcurveto{\pgfqpoint{4.217759in}{0.786523in}}{\pgfqpoint{4.222150in}{0.797122in}}{\pgfqpoint{4.222150in}{0.808172in}}%
\pgfpathcurveto{\pgfqpoint{4.222150in}{0.819222in}}{\pgfqpoint{4.217759in}{0.829821in}}{\pgfqpoint{4.209946in}{0.837635in}}%
\pgfpathcurveto{\pgfqpoint{4.202132in}{0.845448in}}{\pgfqpoint{4.191533in}{0.849839in}}{\pgfqpoint{4.180483in}{0.849839in}}%
\pgfpathcurveto{\pgfqpoint{4.169433in}{0.849839in}}{\pgfqpoint{4.158834in}{0.845448in}}{\pgfqpoint{4.151020in}{0.837635in}}%
\pgfpathcurveto{\pgfqpoint{4.143206in}{0.829821in}}{\pgfqpoint{4.138816in}{0.819222in}}{\pgfqpoint{4.138816in}{0.808172in}}%
\pgfpathcurveto{\pgfqpoint{4.138816in}{0.797122in}}{\pgfqpoint{4.143206in}{0.786523in}}{\pgfqpoint{4.151020in}{0.778709in}}%
\pgfpathcurveto{\pgfqpoint{4.158834in}{0.770896in}}{\pgfqpoint{4.169433in}{0.766505in}}{\pgfqpoint{4.180483in}{0.766505in}}%
\pgfpathlineto{\pgfqpoint{4.180483in}{0.766505in}}%
\pgfpathclose%
\pgfusepath{stroke}%
\end{pgfscope}%
\begin{pgfscope}%
\pgfpathrectangle{\pgfqpoint{0.847223in}{0.554012in}}{\pgfqpoint{6.200000in}{4.530000in}}%
\pgfusepath{clip}%
\pgfsetbuttcap%
\pgfsetroundjoin%
\pgfsetlinewidth{1.003750pt}%
\definecolor{currentstroke}{rgb}{1.000000,0.000000,0.000000}%
\pgfsetstrokecolor{currentstroke}%
\pgfsetdash{}{0pt}%
\pgfpathmoveto{\pgfqpoint{4.185816in}{0.765578in}}%
\pgfpathcurveto{\pgfqpoint{4.196866in}{0.765578in}}{\pgfqpoint{4.207465in}{0.769968in}}{\pgfqpoint{4.215279in}{0.777782in}}%
\pgfpathcurveto{\pgfqpoint{4.223092in}{0.785596in}}{\pgfqpoint{4.227483in}{0.796195in}}{\pgfqpoint{4.227483in}{0.807245in}}%
\pgfpathcurveto{\pgfqpoint{4.227483in}{0.818295in}}{\pgfqpoint{4.223092in}{0.828894in}}{\pgfqpoint{4.215279in}{0.836708in}}%
\pgfpathcurveto{\pgfqpoint{4.207465in}{0.844521in}}{\pgfqpoint{4.196866in}{0.848912in}}{\pgfqpoint{4.185816in}{0.848912in}}%
\pgfpathcurveto{\pgfqpoint{4.174766in}{0.848912in}}{\pgfqpoint{4.164167in}{0.844521in}}{\pgfqpoint{4.156353in}{0.836708in}}%
\pgfpathcurveto{\pgfqpoint{4.148540in}{0.828894in}}{\pgfqpoint{4.144149in}{0.818295in}}{\pgfqpoint{4.144149in}{0.807245in}}%
\pgfpathcurveto{\pgfqpoint{4.144149in}{0.796195in}}{\pgfqpoint{4.148540in}{0.785596in}}{\pgfqpoint{4.156353in}{0.777782in}}%
\pgfpathcurveto{\pgfqpoint{4.164167in}{0.769968in}}{\pgfqpoint{4.174766in}{0.765578in}}{\pgfqpoint{4.185816in}{0.765578in}}%
\pgfpathlineto{\pgfqpoint{4.185816in}{0.765578in}}%
\pgfpathclose%
\pgfusepath{stroke}%
\end{pgfscope}%
\begin{pgfscope}%
\pgfpathrectangle{\pgfqpoint{0.847223in}{0.554012in}}{\pgfqpoint{6.200000in}{4.530000in}}%
\pgfusepath{clip}%
\pgfsetbuttcap%
\pgfsetroundjoin%
\pgfsetlinewidth{1.003750pt}%
\definecolor{currentstroke}{rgb}{1.000000,0.000000,0.000000}%
\pgfsetstrokecolor{currentstroke}%
\pgfsetdash{}{0pt}%
\pgfpathmoveto{\pgfqpoint{4.191149in}{0.764654in}}%
\pgfpathcurveto{\pgfqpoint{4.202199in}{0.764654in}}{\pgfqpoint{4.212798in}{0.769044in}}{\pgfqpoint{4.220612in}{0.776858in}}%
\pgfpathcurveto{\pgfqpoint{4.228426in}{0.784671in}}{\pgfqpoint{4.232816in}{0.795270in}}{\pgfqpoint{4.232816in}{0.806320in}}%
\pgfpathcurveto{\pgfqpoint{4.232816in}{0.817370in}}{\pgfqpoint{4.228426in}{0.827969in}}{\pgfqpoint{4.220612in}{0.835783in}}%
\pgfpathcurveto{\pgfqpoint{4.212798in}{0.843597in}}{\pgfqpoint{4.202199in}{0.847987in}}{\pgfqpoint{4.191149in}{0.847987in}}%
\pgfpathcurveto{\pgfqpoint{4.180099in}{0.847987in}}{\pgfqpoint{4.169500in}{0.843597in}}{\pgfqpoint{4.161686in}{0.835783in}}%
\pgfpathcurveto{\pgfqpoint{4.153873in}{0.827969in}}{\pgfqpoint{4.149483in}{0.817370in}}{\pgfqpoint{4.149483in}{0.806320in}}%
\pgfpathcurveto{\pgfqpoint{4.149483in}{0.795270in}}{\pgfqpoint{4.153873in}{0.784671in}}{\pgfqpoint{4.161686in}{0.776858in}}%
\pgfpathcurveto{\pgfqpoint{4.169500in}{0.769044in}}{\pgfqpoint{4.180099in}{0.764654in}}{\pgfqpoint{4.191149in}{0.764654in}}%
\pgfpathlineto{\pgfqpoint{4.191149in}{0.764654in}}%
\pgfpathclose%
\pgfusepath{stroke}%
\end{pgfscope}%
\begin{pgfscope}%
\pgfpathrectangle{\pgfqpoint{0.847223in}{0.554012in}}{\pgfqpoint{6.200000in}{4.530000in}}%
\pgfusepath{clip}%
\pgfsetbuttcap%
\pgfsetroundjoin%
\pgfsetlinewidth{1.003750pt}%
\definecolor{currentstroke}{rgb}{1.000000,0.000000,0.000000}%
\pgfsetstrokecolor{currentstroke}%
\pgfsetdash{}{0pt}%
\pgfpathmoveto{\pgfqpoint{4.196482in}{0.763732in}}%
\pgfpathcurveto{\pgfqpoint{4.207533in}{0.763732in}}{\pgfqpoint{4.218132in}{0.768122in}}{\pgfqpoint{4.225945in}{0.775936in}}%
\pgfpathcurveto{\pgfqpoint{4.233759in}{0.783749in}}{\pgfqpoint{4.238149in}{0.794348in}}{\pgfqpoint{4.238149in}{0.805398in}}%
\pgfpathcurveto{\pgfqpoint{4.238149in}{0.816448in}}{\pgfqpoint{4.233759in}{0.827047in}}{\pgfqpoint{4.225945in}{0.834861in}}%
\pgfpathcurveto{\pgfqpoint{4.218132in}{0.842675in}}{\pgfqpoint{4.207533in}{0.847065in}}{\pgfqpoint{4.196482in}{0.847065in}}%
\pgfpathcurveto{\pgfqpoint{4.185432in}{0.847065in}}{\pgfqpoint{4.174833in}{0.842675in}}{\pgfqpoint{4.167020in}{0.834861in}}%
\pgfpathcurveto{\pgfqpoint{4.159206in}{0.827047in}}{\pgfqpoint{4.154816in}{0.816448in}}{\pgfqpoint{4.154816in}{0.805398in}}%
\pgfpathcurveto{\pgfqpoint{4.154816in}{0.794348in}}{\pgfqpoint{4.159206in}{0.783749in}}{\pgfqpoint{4.167020in}{0.775936in}}%
\pgfpathcurveto{\pgfqpoint{4.174833in}{0.768122in}}{\pgfqpoint{4.185432in}{0.763732in}}{\pgfqpoint{4.196482in}{0.763732in}}%
\pgfpathlineto{\pgfqpoint{4.196482in}{0.763732in}}%
\pgfpathclose%
\pgfusepath{stroke}%
\end{pgfscope}%
\begin{pgfscope}%
\pgfpathrectangle{\pgfqpoint{0.847223in}{0.554012in}}{\pgfqpoint{6.200000in}{4.530000in}}%
\pgfusepath{clip}%
\pgfsetbuttcap%
\pgfsetroundjoin%
\pgfsetlinewidth{1.003750pt}%
\definecolor{currentstroke}{rgb}{1.000000,0.000000,0.000000}%
\pgfsetstrokecolor{currentstroke}%
\pgfsetdash{}{0pt}%
\pgfpathmoveto{\pgfqpoint{4.201816in}{0.762812in}}%
\pgfpathcurveto{\pgfqpoint{4.212866in}{0.762812in}}{\pgfqpoint{4.223465in}{0.767202in}}{\pgfqpoint{4.231278in}{0.775016in}}%
\pgfpathcurveto{\pgfqpoint{4.239092in}{0.782830in}}{\pgfqpoint{4.243482in}{0.793429in}}{\pgfqpoint{4.243482in}{0.804479in}}%
\pgfpathcurveto{\pgfqpoint{4.243482in}{0.815529in}}{\pgfqpoint{4.239092in}{0.826128in}}{\pgfqpoint{4.231278in}{0.833942in}}%
\pgfpathcurveto{\pgfqpoint{4.223465in}{0.841755in}}{\pgfqpoint{4.212866in}{0.846145in}}{\pgfqpoint{4.201816in}{0.846145in}}%
\pgfpathcurveto{\pgfqpoint{4.190766in}{0.846145in}}{\pgfqpoint{4.180167in}{0.841755in}}{\pgfqpoint{4.172353in}{0.833942in}}%
\pgfpathcurveto{\pgfqpoint{4.164539in}{0.826128in}}{\pgfqpoint{4.160149in}{0.815529in}}{\pgfqpoint{4.160149in}{0.804479in}}%
\pgfpathcurveto{\pgfqpoint{4.160149in}{0.793429in}}{\pgfqpoint{4.164539in}{0.782830in}}{\pgfqpoint{4.172353in}{0.775016in}}%
\pgfpathcurveto{\pgfqpoint{4.180167in}{0.767202in}}{\pgfqpoint{4.190766in}{0.762812in}}{\pgfqpoint{4.201816in}{0.762812in}}%
\pgfpathlineto{\pgfqpoint{4.201816in}{0.762812in}}%
\pgfpathclose%
\pgfusepath{stroke}%
\end{pgfscope}%
\begin{pgfscope}%
\pgfpathrectangle{\pgfqpoint{0.847223in}{0.554012in}}{\pgfqpoint{6.200000in}{4.530000in}}%
\pgfusepath{clip}%
\pgfsetbuttcap%
\pgfsetroundjoin%
\pgfsetlinewidth{1.003750pt}%
\definecolor{currentstroke}{rgb}{1.000000,0.000000,0.000000}%
\pgfsetstrokecolor{currentstroke}%
\pgfsetdash{}{0pt}%
\pgfpathmoveto{\pgfqpoint{4.207149in}{0.761895in}}%
\pgfpathcurveto{\pgfqpoint{4.218199in}{0.761895in}}{\pgfqpoint{4.228798in}{0.766285in}}{\pgfqpoint{4.236612in}{0.774099in}}%
\pgfpathcurveto{\pgfqpoint{4.244425in}{0.781913in}}{\pgfqpoint{4.248816in}{0.792512in}}{\pgfqpoint{4.248816in}{0.803562in}}%
\pgfpathcurveto{\pgfqpoint{4.248816in}{0.814612in}}{\pgfqpoint{4.244425in}{0.825211in}}{\pgfqpoint{4.236612in}{0.833025in}}%
\pgfpathcurveto{\pgfqpoint{4.228798in}{0.840838in}}{\pgfqpoint{4.218199in}{0.845229in}}{\pgfqpoint{4.207149in}{0.845229in}}%
\pgfpathcurveto{\pgfqpoint{4.196099in}{0.845229in}}{\pgfqpoint{4.185500in}{0.840838in}}{\pgfqpoint{4.177686in}{0.833025in}}%
\pgfpathcurveto{\pgfqpoint{4.169873in}{0.825211in}}{\pgfqpoint{4.165482in}{0.814612in}}{\pgfqpoint{4.165482in}{0.803562in}}%
\pgfpathcurveto{\pgfqpoint{4.165482in}{0.792512in}}{\pgfqpoint{4.169873in}{0.781913in}}{\pgfqpoint{4.177686in}{0.774099in}}%
\pgfpathcurveto{\pgfqpoint{4.185500in}{0.766285in}}{\pgfqpoint{4.196099in}{0.761895in}}{\pgfqpoint{4.207149in}{0.761895in}}%
\pgfpathlineto{\pgfqpoint{4.207149in}{0.761895in}}%
\pgfpathclose%
\pgfusepath{stroke}%
\end{pgfscope}%
\begin{pgfscope}%
\pgfpathrectangle{\pgfqpoint{0.847223in}{0.554012in}}{\pgfqpoint{6.200000in}{4.530000in}}%
\pgfusepath{clip}%
\pgfsetbuttcap%
\pgfsetroundjoin%
\pgfsetlinewidth{1.003750pt}%
\definecolor{currentstroke}{rgb}{1.000000,0.000000,0.000000}%
\pgfsetstrokecolor{currentstroke}%
\pgfsetdash{}{0pt}%
\pgfpathmoveto{\pgfqpoint{4.212482in}{0.760981in}}%
\pgfpathcurveto{\pgfqpoint{4.223532in}{0.760981in}}{\pgfqpoint{4.234131in}{0.765371in}}{\pgfqpoint{4.241945in}{0.773185in}}%
\pgfpathcurveto{\pgfqpoint{4.249759in}{0.780998in}}{\pgfqpoint{4.254149in}{0.791597in}}{\pgfqpoint{4.254149in}{0.802647in}}%
\pgfpathcurveto{\pgfqpoint{4.254149in}{0.813698in}}{\pgfqpoint{4.249759in}{0.824297in}}{\pgfqpoint{4.241945in}{0.832110in}}%
\pgfpathcurveto{\pgfqpoint{4.234131in}{0.839924in}}{\pgfqpoint{4.223532in}{0.844314in}}{\pgfqpoint{4.212482in}{0.844314in}}%
\pgfpathcurveto{\pgfqpoint{4.201432in}{0.844314in}}{\pgfqpoint{4.190833in}{0.839924in}}{\pgfqpoint{4.183019in}{0.832110in}}%
\pgfpathcurveto{\pgfqpoint{4.175206in}{0.824297in}}{\pgfqpoint{4.170815in}{0.813698in}}{\pgfqpoint{4.170815in}{0.802647in}}%
\pgfpathcurveto{\pgfqpoint{4.170815in}{0.791597in}}{\pgfqpoint{4.175206in}{0.780998in}}{\pgfqpoint{4.183019in}{0.773185in}}%
\pgfpathcurveto{\pgfqpoint{4.190833in}{0.765371in}}{\pgfqpoint{4.201432in}{0.760981in}}{\pgfqpoint{4.212482in}{0.760981in}}%
\pgfpathlineto{\pgfqpoint{4.212482in}{0.760981in}}%
\pgfpathclose%
\pgfusepath{stroke}%
\end{pgfscope}%
\begin{pgfscope}%
\pgfpathrectangle{\pgfqpoint{0.847223in}{0.554012in}}{\pgfqpoint{6.200000in}{4.530000in}}%
\pgfusepath{clip}%
\pgfsetbuttcap%
\pgfsetroundjoin%
\pgfsetlinewidth{1.003750pt}%
\definecolor{currentstroke}{rgb}{1.000000,0.000000,0.000000}%
\pgfsetstrokecolor{currentstroke}%
\pgfsetdash{}{0pt}%
\pgfpathmoveto{\pgfqpoint{4.217815in}{0.760069in}}%
\pgfpathcurveto{\pgfqpoint{4.228865in}{0.760069in}}{\pgfqpoint{4.239465in}{0.764459in}}{\pgfqpoint{4.247278in}{0.772273in}}%
\pgfpathcurveto{\pgfqpoint{4.255092in}{0.780086in}}{\pgfqpoint{4.259482in}{0.790685in}}{\pgfqpoint{4.259482in}{0.801736in}}%
\pgfpathcurveto{\pgfqpoint{4.259482in}{0.812786in}}{\pgfqpoint{4.255092in}{0.823385in}}{\pgfqpoint{4.247278in}{0.831198in}}%
\pgfpathcurveto{\pgfqpoint{4.239465in}{0.839012in}}{\pgfqpoint{4.228865in}{0.843402in}}{\pgfqpoint{4.217815in}{0.843402in}}%
\pgfpathcurveto{\pgfqpoint{4.206765in}{0.843402in}}{\pgfqpoint{4.196166in}{0.839012in}}{\pgfqpoint{4.188353in}{0.831198in}}%
\pgfpathcurveto{\pgfqpoint{4.180539in}{0.823385in}}{\pgfqpoint{4.176149in}{0.812786in}}{\pgfqpoint{4.176149in}{0.801736in}}%
\pgfpathcurveto{\pgfqpoint{4.176149in}{0.790685in}}{\pgfqpoint{4.180539in}{0.780086in}}{\pgfqpoint{4.188353in}{0.772273in}}%
\pgfpathcurveto{\pgfqpoint{4.196166in}{0.764459in}}{\pgfqpoint{4.206765in}{0.760069in}}{\pgfqpoint{4.217815in}{0.760069in}}%
\pgfpathlineto{\pgfqpoint{4.217815in}{0.760069in}}%
\pgfpathclose%
\pgfusepath{stroke}%
\end{pgfscope}%
\begin{pgfscope}%
\pgfpathrectangle{\pgfqpoint{0.847223in}{0.554012in}}{\pgfqpoint{6.200000in}{4.530000in}}%
\pgfusepath{clip}%
\pgfsetbuttcap%
\pgfsetroundjoin%
\pgfsetlinewidth{1.003750pt}%
\definecolor{currentstroke}{rgb}{1.000000,0.000000,0.000000}%
\pgfsetstrokecolor{currentstroke}%
\pgfsetdash{}{0pt}%
\pgfpathmoveto{\pgfqpoint{4.223149in}{0.759159in}}%
\pgfpathcurveto{\pgfqpoint{4.234199in}{0.759159in}}{\pgfqpoint{4.244798in}{0.763550in}}{\pgfqpoint{4.252611in}{0.771363in}}%
\pgfpathcurveto{\pgfqpoint{4.260425in}{0.779177in}}{\pgfqpoint{4.264815in}{0.789776in}}{\pgfqpoint{4.264815in}{0.800826in}}%
\pgfpathcurveto{\pgfqpoint{4.264815in}{0.811876in}}{\pgfqpoint{4.260425in}{0.822475in}}{\pgfqpoint{4.252611in}{0.830289in}}%
\pgfpathcurveto{\pgfqpoint{4.244798in}{0.838103in}}{\pgfqpoint{4.234199in}{0.842493in}}{\pgfqpoint{4.223149in}{0.842493in}}%
\pgfpathcurveto{\pgfqpoint{4.212098in}{0.842493in}}{\pgfqpoint{4.201499in}{0.838103in}}{\pgfqpoint{4.193686in}{0.830289in}}%
\pgfpathcurveto{\pgfqpoint{4.185872in}{0.822475in}}{\pgfqpoint{4.181482in}{0.811876in}}{\pgfqpoint{4.181482in}{0.800826in}}%
\pgfpathcurveto{\pgfqpoint{4.181482in}{0.789776in}}{\pgfqpoint{4.185872in}{0.779177in}}{\pgfqpoint{4.193686in}{0.771363in}}%
\pgfpathcurveto{\pgfqpoint{4.201499in}{0.763550in}}{\pgfqpoint{4.212098in}{0.759159in}}{\pgfqpoint{4.223149in}{0.759159in}}%
\pgfpathlineto{\pgfqpoint{4.223149in}{0.759159in}}%
\pgfpathclose%
\pgfusepath{stroke}%
\end{pgfscope}%
\begin{pgfscope}%
\pgfpathrectangle{\pgfqpoint{0.847223in}{0.554012in}}{\pgfqpoint{6.200000in}{4.530000in}}%
\pgfusepath{clip}%
\pgfsetbuttcap%
\pgfsetroundjoin%
\pgfsetlinewidth{1.003750pt}%
\definecolor{currentstroke}{rgb}{1.000000,0.000000,0.000000}%
\pgfsetstrokecolor{currentstroke}%
\pgfsetdash{}{0pt}%
\pgfpathmoveto{\pgfqpoint{4.228482in}{0.758253in}}%
\pgfpathcurveto{\pgfqpoint{4.239532in}{0.758253in}}{\pgfqpoint{4.250131in}{0.762643in}}{\pgfqpoint{4.257945in}{0.770456in}}%
\pgfpathcurveto{\pgfqpoint{4.265758in}{0.778270in}}{\pgfqpoint{4.270148in}{0.788869in}}{\pgfqpoint{4.270148in}{0.799919in}}%
\pgfpathcurveto{\pgfqpoint{4.270148in}{0.810969in}}{\pgfqpoint{4.265758in}{0.821568in}}{\pgfqpoint{4.257945in}{0.829382in}}%
\pgfpathcurveto{\pgfqpoint{4.250131in}{0.837196in}}{\pgfqpoint{4.239532in}{0.841586in}}{\pgfqpoint{4.228482in}{0.841586in}}%
\pgfpathcurveto{\pgfqpoint{4.217432in}{0.841586in}}{\pgfqpoint{4.206833in}{0.837196in}}{\pgfqpoint{4.199019in}{0.829382in}}%
\pgfpathcurveto{\pgfqpoint{4.191205in}{0.821568in}}{\pgfqpoint{4.186815in}{0.810969in}}{\pgfqpoint{4.186815in}{0.799919in}}%
\pgfpathcurveto{\pgfqpoint{4.186815in}{0.788869in}}{\pgfqpoint{4.191205in}{0.778270in}}{\pgfqpoint{4.199019in}{0.770456in}}%
\pgfpathcurveto{\pgfqpoint{4.206833in}{0.762643in}}{\pgfqpoint{4.217432in}{0.758253in}}{\pgfqpoint{4.228482in}{0.758253in}}%
\pgfpathlineto{\pgfqpoint{4.228482in}{0.758253in}}%
\pgfpathclose%
\pgfusepath{stroke}%
\end{pgfscope}%
\begin{pgfscope}%
\pgfpathrectangle{\pgfqpoint{0.847223in}{0.554012in}}{\pgfqpoint{6.200000in}{4.530000in}}%
\pgfusepath{clip}%
\pgfsetbuttcap%
\pgfsetroundjoin%
\pgfsetlinewidth{1.003750pt}%
\definecolor{currentstroke}{rgb}{1.000000,0.000000,0.000000}%
\pgfsetstrokecolor{currentstroke}%
\pgfsetdash{}{0pt}%
\pgfpathmoveto{\pgfqpoint{4.233815in}{0.757348in}}%
\pgfpathcurveto{\pgfqpoint{4.244865in}{0.757348in}}{\pgfqpoint{4.255464in}{0.761738in}}{\pgfqpoint{4.263278in}{0.769552in}}%
\pgfpathcurveto{\pgfqpoint{4.271091in}{0.777366in}}{\pgfqpoint{4.275482in}{0.787965in}}{\pgfqpoint{4.275482in}{0.799015in}}%
\pgfpathcurveto{\pgfqpoint{4.275482in}{0.810065in}}{\pgfqpoint{4.271091in}{0.820664in}}{\pgfqpoint{4.263278in}{0.828478in}}%
\pgfpathcurveto{\pgfqpoint{4.255464in}{0.836291in}}{\pgfqpoint{4.244865in}{0.840681in}}{\pgfqpoint{4.233815in}{0.840681in}}%
\pgfpathcurveto{\pgfqpoint{4.222765in}{0.840681in}}{\pgfqpoint{4.212166in}{0.836291in}}{\pgfqpoint{4.204352in}{0.828478in}}%
\pgfpathcurveto{\pgfqpoint{4.196539in}{0.820664in}}{\pgfqpoint{4.192148in}{0.810065in}}{\pgfqpoint{4.192148in}{0.799015in}}%
\pgfpathcurveto{\pgfqpoint{4.192148in}{0.787965in}}{\pgfqpoint{4.196539in}{0.777366in}}{\pgfqpoint{4.204352in}{0.769552in}}%
\pgfpathcurveto{\pgfqpoint{4.212166in}{0.761738in}}{\pgfqpoint{4.222765in}{0.757348in}}{\pgfqpoint{4.233815in}{0.757348in}}%
\pgfpathlineto{\pgfqpoint{4.233815in}{0.757348in}}%
\pgfpathclose%
\pgfusepath{stroke}%
\end{pgfscope}%
\begin{pgfscope}%
\pgfpathrectangle{\pgfqpoint{0.847223in}{0.554012in}}{\pgfqpoint{6.200000in}{4.530000in}}%
\pgfusepath{clip}%
\pgfsetbuttcap%
\pgfsetroundjoin%
\pgfsetlinewidth{1.003750pt}%
\definecolor{currentstroke}{rgb}{1.000000,0.000000,0.000000}%
\pgfsetstrokecolor{currentstroke}%
\pgfsetdash{}{0pt}%
\pgfpathmoveto{\pgfqpoint{4.239148in}{0.756446in}}%
\pgfpathcurveto{\pgfqpoint{4.250198in}{0.756446in}}{\pgfqpoint{4.260797in}{0.760836in}}{\pgfqpoint{4.268611in}{0.768650in}}%
\pgfpathcurveto{\pgfqpoint{4.276425in}{0.776464in}}{\pgfqpoint{4.280815in}{0.787063in}}{\pgfqpoint{4.280815in}{0.798113in}}%
\pgfpathcurveto{\pgfqpoint{4.280815in}{0.809163in}}{\pgfqpoint{4.276425in}{0.819762in}}{\pgfqpoint{4.268611in}{0.827576in}}%
\pgfpathcurveto{\pgfqpoint{4.260797in}{0.835389in}}{\pgfqpoint{4.250198in}{0.839779in}}{\pgfqpoint{4.239148in}{0.839779in}}%
\pgfpathcurveto{\pgfqpoint{4.228098in}{0.839779in}}{\pgfqpoint{4.217499in}{0.835389in}}{\pgfqpoint{4.209685in}{0.827576in}}%
\pgfpathcurveto{\pgfqpoint{4.201872in}{0.819762in}}{\pgfqpoint{4.197482in}{0.809163in}}{\pgfqpoint{4.197482in}{0.798113in}}%
\pgfpathcurveto{\pgfqpoint{4.197482in}{0.787063in}}{\pgfqpoint{4.201872in}{0.776464in}}{\pgfqpoint{4.209685in}{0.768650in}}%
\pgfpathcurveto{\pgfqpoint{4.217499in}{0.760836in}}{\pgfqpoint{4.228098in}{0.756446in}}{\pgfqpoint{4.239148in}{0.756446in}}%
\pgfpathlineto{\pgfqpoint{4.239148in}{0.756446in}}%
\pgfpathclose%
\pgfusepath{stroke}%
\end{pgfscope}%
\begin{pgfscope}%
\pgfpathrectangle{\pgfqpoint{0.847223in}{0.554012in}}{\pgfqpoint{6.200000in}{4.530000in}}%
\pgfusepath{clip}%
\pgfsetbuttcap%
\pgfsetroundjoin%
\pgfsetlinewidth{1.003750pt}%
\definecolor{currentstroke}{rgb}{1.000000,0.000000,0.000000}%
\pgfsetstrokecolor{currentstroke}%
\pgfsetdash{}{0pt}%
\pgfpathmoveto{\pgfqpoint{4.244481in}{0.755547in}}%
\pgfpathcurveto{\pgfqpoint{4.255532in}{0.755547in}}{\pgfqpoint{4.266131in}{0.759937in}}{\pgfqpoint{4.273944in}{0.767750in}}%
\pgfpathcurveto{\pgfqpoint{4.281758in}{0.775564in}}{\pgfqpoint{4.286148in}{0.786163in}}{\pgfqpoint{4.286148in}{0.797213in}}%
\pgfpathcurveto{\pgfqpoint{4.286148in}{0.808263in}}{\pgfqpoint{4.281758in}{0.818862in}}{\pgfqpoint{4.273944in}{0.826676in}}%
\pgfpathcurveto{\pgfqpoint{4.266131in}{0.834490in}}{\pgfqpoint{4.255532in}{0.838880in}}{\pgfqpoint{4.244481in}{0.838880in}}%
\pgfpathcurveto{\pgfqpoint{4.233431in}{0.838880in}}{\pgfqpoint{4.222832in}{0.834490in}}{\pgfqpoint{4.215019in}{0.826676in}}%
\pgfpathcurveto{\pgfqpoint{4.207205in}{0.818862in}}{\pgfqpoint{4.202815in}{0.808263in}}{\pgfqpoint{4.202815in}{0.797213in}}%
\pgfpathcurveto{\pgfqpoint{4.202815in}{0.786163in}}{\pgfqpoint{4.207205in}{0.775564in}}{\pgfqpoint{4.215019in}{0.767750in}}%
\pgfpathcurveto{\pgfqpoint{4.222832in}{0.759937in}}{\pgfqpoint{4.233431in}{0.755547in}}{\pgfqpoint{4.244481in}{0.755547in}}%
\pgfpathlineto{\pgfqpoint{4.244481in}{0.755547in}}%
\pgfpathclose%
\pgfusepath{stroke}%
\end{pgfscope}%
\begin{pgfscope}%
\pgfpathrectangle{\pgfqpoint{0.847223in}{0.554012in}}{\pgfqpoint{6.200000in}{4.530000in}}%
\pgfusepath{clip}%
\pgfsetbuttcap%
\pgfsetroundjoin%
\pgfsetlinewidth{1.003750pt}%
\definecolor{currentstroke}{rgb}{1.000000,0.000000,0.000000}%
\pgfsetstrokecolor{currentstroke}%
\pgfsetdash{}{0pt}%
\pgfpathmoveto{\pgfqpoint{4.249815in}{0.754649in}}%
\pgfpathcurveto{\pgfqpoint{4.260865in}{0.754649in}}{\pgfqpoint{4.271464in}{0.759040in}}{\pgfqpoint{4.279277in}{0.766853in}}%
\pgfpathcurveto{\pgfqpoint{4.287091in}{0.774667in}}{\pgfqpoint{4.291481in}{0.785266in}}{\pgfqpoint{4.291481in}{0.796316in}}%
\pgfpathcurveto{\pgfqpoint{4.291481in}{0.807366in}}{\pgfqpoint{4.287091in}{0.817965in}}{\pgfqpoint{4.279277in}{0.825779in}}%
\pgfpathcurveto{\pgfqpoint{4.271464in}{0.833592in}}{\pgfqpoint{4.260865in}{0.837983in}}{\pgfqpoint{4.249815in}{0.837983in}}%
\pgfpathcurveto{\pgfqpoint{4.238765in}{0.837983in}}{\pgfqpoint{4.228165in}{0.833592in}}{\pgfqpoint{4.220352in}{0.825779in}}%
\pgfpathcurveto{\pgfqpoint{4.212538in}{0.817965in}}{\pgfqpoint{4.208148in}{0.807366in}}{\pgfqpoint{4.208148in}{0.796316in}}%
\pgfpathcurveto{\pgfqpoint{4.208148in}{0.785266in}}{\pgfqpoint{4.212538in}{0.774667in}}{\pgfqpoint{4.220352in}{0.766853in}}%
\pgfpathcurveto{\pgfqpoint{4.228165in}{0.759040in}}{\pgfqpoint{4.238765in}{0.754649in}}{\pgfqpoint{4.249815in}{0.754649in}}%
\pgfpathlineto{\pgfqpoint{4.249815in}{0.754649in}}%
\pgfpathclose%
\pgfusepath{stroke}%
\end{pgfscope}%
\begin{pgfscope}%
\pgfpathrectangle{\pgfqpoint{0.847223in}{0.554012in}}{\pgfqpoint{6.200000in}{4.530000in}}%
\pgfusepath{clip}%
\pgfsetbuttcap%
\pgfsetroundjoin%
\pgfsetlinewidth{1.003750pt}%
\definecolor{currentstroke}{rgb}{1.000000,0.000000,0.000000}%
\pgfsetstrokecolor{currentstroke}%
\pgfsetdash{}{0pt}%
\pgfpathmoveto{\pgfqpoint{4.255148in}{0.753755in}}%
\pgfpathcurveto{\pgfqpoint{4.266198in}{0.753755in}}{\pgfqpoint{4.276797in}{0.758145in}}{\pgfqpoint{4.284611in}{0.765959in}}%
\pgfpathcurveto{\pgfqpoint{4.292424in}{0.773772in}}{\pgfqpoint{4.296815in}{0.784371in}}{\pgfqpoint{4.296815in}{0.795421in}}%
\pgfpathcurveto{\pgfqpoint{4.296815in}{0.806472in}}{\pgfqpoint{4.292424in}{0.817071in}}{\pgfqpoint{4.284611in}{0.824884in}}%
\pgfpathcurveto{\pgfqpoint{4.276797in}{0.832698in}}{\pgfqpoint{4.266198in}{0.837088in}}{\pgfqpoint{4.255148in}{0.837088in}}%
\pgfpathcurveto{\pgfqpoint{4.244098in}{0.837088in}}{\pgfqpoint{4.233499in}{0.832698in}}{\pgfqpoint{4.225685in}{0.824884in}}%
\pgfpathcurveto{\pgfqpoint{4.217871in}{0.817071in}}{\pgfqpoint{4.213481in}{0.806472in}}{\pgfqpoint{4.213481in}{0.795421in}}%
\pgfpathcurveto{\pgfqpoint{4.213481in}{0.784371in}}{\pgfqpoint{4.217871in}{0.773772in}}{\pgfqpoint{4.225685in}{0.765959in}}%
\pgfpathcurveto{\pgfqpoint{4.233499in}{0.758145in}}{\pgfqpoint{4.244098in}{0.753755in}}{\pgfqpoint{4.255148in}{0.753755in}}%
\pgfpathlineto{\pgfqpoint{4.255148in}{0.753755in}}%
\pgfpathclose%
\pgfusepath{stroke}%
\end{pgfscope}%
\begin{pgfscope}%
\pgfpathrectangle{\pgfqpoint{0.847223in}{0.554012in}}{\pgfqpoint{6.200000in}{4.530000in}}%
\pgfusepath{clip}%
\pgfsetbuttcap%
\pgfsetroundjoin%
\pgfsetlinewidth{1.003750pt}%
\definecolor{currentstroke}{rgb}{1.000000,0.000000,0.000000}%
\pgfsetstrokecolor{currentstroke}%
\pgfsetdash{}{0pt}%
\pgfpathmoveto{\pgfqpoint{4.260481in}{0.752862in}}%
\pgfpathcurveto{\pgfqpoint{4.271531in}{0.752862in}}{\pgfqpoint{4.282130in}{0.757253in}}{\pgfqpoint{4.289944in}{0.765066in}}%
\pgfpathcurveto{\pgfqpoint{4.297757in}{0.772880in}}{\pgfqpoint{4.302148in}{0.783479in}}{\pgfqpoint{4.302148in}{0.794529in}}%
\pgfpathcurveto{\pgfqpoint{4.302148in}{0.805579in}}{\pgfqpoint{4.297757in}{0.816178in}}{\pgfqpoint{4.289944in}{0.823992in}}%
\pgfpathcurveto{\pgfqpoint{4.282130in}{0.831806in}}{\pgfqpoint{4.271531in}{0.836196in}}{\pgfqpoint{4.260481in}{0.836196in}}%
\pgfpathcurveto{\pgfqpoint{4.249431in}{0.836196in}}{\pgfqpoint{4.238832in}{0.831806in}}{\pgfqpoint{4.231018in}{0.823992in}}%
\pgfpathcurveto{\pgfqpoint{4.223205in}{0.816178in}}{\pgfqpoint{4.218814in}{0.805579in}}{\pgfqpoint{4.218814in}{0.794529in}}%
\pgfpathcurveto{\pgfqpoint{4.218814in}{0.783479in}}{\pgfqpoint{4.223205in}{0.772880in}}{\pgfqpoint{4.231018in}{0.765066in}}%
\pgfpathcurveto{\pgfqpoint{4.238832in}{0.757253in}}{\pgfqpoint{4.249431in}{0.752862in}}{\pgfqpoint{4.260481in}{0.752862in}}%
\pgfpathlineto{\pgfqpoint{4.260481in}{0.752862in}}%
\pgfpathclose%
\pgfusepath{stroke}%
\end{pgfscope}%
\begin{pgfscope}%
\pgfpathrectangle{\pgfqpoint{0.847223in}{0.554012in}}{\pgfqpoint{6.200000in}{4.530000in}}%
\pgfusepath{clip}%
\pgfsetbuttcap%
\pgfsetroundjoin%
\pgfsetlinewidth{1.003750pt}%
\definecolor{currentstroke}{rgb}{1.000000,0.000000,0.000000}%
\pgfsetstrokecolor{currentstroke}%
\pgfsetdash{}{0pt}%
\pgfpathmoveto{\pgfqpoint{4.265814in}{0.751973in}}%
\pgfpathcurveto{\pgfqpoint{4.276864in}{0.751973in}}{\pgfqpoint{4.287463in}{0.756363in}}{\pgfqpoint{4.295277in}{0.764177in}}%
\pgfpathcurveto{\pgfqpoint{4.303091in}{0.771990in}}{\pgfqpoint{4.307481in}{0.782589in}}{\pgfqpoint{4.307481in}{0.793639in}}%
\pgfpathcurveto{\pgfqpoint{4.307481in}{0.804689in}}{\pgfqpoint{4.303091in}{0.815288in}}{\pgfqpoint{4.295277in}{0.823102in}}%
\pgfpathcurveto{\pgfqpoint{4.287463in}{0.830916in}}{\pgfqpoint{4.276864in}{0.835306in}}{\pgfqpoint{4.265814in}{0.835306in}}%
\pgfpathcurveto{\pgfqpoint{4.254764in}{0.835306in}}{\pgfqpoint{4.244165in}{0.830916in}}{\pgfqpoint{4.236352in}{0.823102in}}%
\pgfpathcurveto{\pgfqpoint{4.228538in}{0.815288in}}{\pgfqpoint{4.224148in}{0.804689in}}{\pgfqpoint{4.224148in}{0.793639in}}%
\pgfpathcurveto{\pgfqpoint{4.224148in}{0.782589in}}{\pgfqpoint{4.228538in}{0.771990in}}{\pgfqpoint{4.236352in}{0.764177in}}%
\pgfpathcurveto{\pgfqpoint{4.244165in}{0.756363in}}{\pgfqpoint{4.254764in}{0.751973in}}{\pgfqpoint{4.265814in}{0.751973in}}%
\pgfpathlineto{\pgfqpoint{4.265814in}{0.751973in}}%
\pgfpathclose%
\pgfusepath{stroke}%
\end{pgfscope}%
\begin{pgfscope}%
\pgfpathrectangle{\pgfqpoint{0.847223in}{0.554012in}}{\pgfqpoint{6.200000in}{4.530000in}}%
\pgfusepath{clip}%
\pgfsetbuttcap%
\pgfsetroundjoin%
\pgfsetlinewidth{1.003750pt}%
\definecolor{currentstroke}{rgb}{1.000000,0.000000,0.000000}%
\pgfsetstrokecolor{currentstroke}%
\pgfsetdash{}{0pt}%
\pgfpathmoveto{\pgfqpoint{4.271148in}{0.751085in}}%
\pgfpathcurveto{\pgfqpoint{4.282198in}{0.751085in}}{\pgfqpoint{4.292797in}{0.755476in}}{\pgfqpoint{4.300610in}{0.763289in}}%
\pgfpathcurveto{\pgfqpoint{4.308424in}{0.771103in}}{\pgfqpoint{4.312814in}{0.781702in}}{\pgfqpoint{4.312814in}{0.792752in}}%
\pgfpathcurveto{\pgfqpoint{4.312814in}{0.803802in}}{\pgfqpoint{4.308424in}{0.814401in}}{\pgfqpoint{4.300610in}{0.822215in}}%
\pgfpathcurveto{\pgfqpoint{4.292797in}{0.830028in}}{\pgfqpoint{4.282198in}{0.834419in}}{\pgfqpoint{4.271148in}{0.834419in}}%
\pgfpathcurveto{\pgfqpoint{4.260097in}{0.834419in}}{\pgfqpoint{4.249498in}{0.830028in}}{\pgfqpoint{4.241685in}{0.822215in}}%
\pgfpathcurveto{\pgfqpoint{4.233871in}{0.814401in}}{\pgfqpoint{4.229481in}{0.803802in}}{\pgfqpoint{4.229481in}{0.792752in}}%
\pgfpathcurveto{\pgfqpoint{4.229481in}{0.781702in}}{\pgfqpoint{4.233871in}{0.771103in}}{\pgfqpoint{4.241685in}{0.763289in}}%
\pgfpathcurveto{\pgfqpoint{4.249498in}{0.755476in}}{\pgfqpoint{4.260097in}{0.751085in}}{\pgfqpoint{4.271148in}{0.751085in}}%
\pgfpathlineto{\pgfqpoint{4.271148in}{0.751085in}}%
\pgfpathclose%
\pgfusepath{stroke}%
\end{pgfscope}%
\begin{pgfscope}%
\pgfpathrectangle{\pgfqpoint{0.847223in}{0.554012in}}{\pgfqpoint{6.200000in}{4.530000in}}%
\pgfusepath{clip}%
\pgfsetbuttcap%
\pgfsetroundjoin%
\pgfsetlinewidth{1.003750pt}%
\definecolor{currentstroke}{rgb}{1.000000,0.000000,0.000000}%
\pgfsetstrokecolor{currentstroke}%
\pgfsetdash{}{0pt}%
\pgfpathmoveto{\pgfqpoint{4.276481in}{0.750200in}}%
\pgfpathcurveto{\pgfqpoint{4.287531in}{0.750200in}}{\pgfqpoint{4.298130in}{0.754590in}}{\pgfqpoint{4.305944in}{0.762404in}}%
\pgfpathcurveto{\pgfqpoint{4.313757in}{0.770218in}}{\pgfqpoint{4.318147in}{0.780817in}}{\pgfqpoint{4.318147in}{0.791867in}}%
\pgfpathcurveto{\pgfqpoint{4.318147in}{0.802917in}}{\pgfqpoint{4.313757in}{0.813516in}}{\pgfqpoint{4.305944in}{0.821330in}}%
\pgfpathcurveto{\pgfqpoint{4.298130in}{0.829143in}}{\pgfqpoint{4.287531in}{0.833534in}}{\pgfqpoint{4.276481in}{0.833534in}}%
\pgfpathcurveto{\pgfqpoint{4.265431in}{0.833534in}}{\pgfqpoint{4.254832in}{0.829143in}}{\pgfqpoint{4.247018in}{0.821330in}}%
\pgfpathcurveto{\pgfqpoint{4.239204in}{0.813516in}}{\pgfqpoint{4.234814in}{0.802917in}}{\pgfqpoint{4.234814in}{0.791867in}}%
\pgfpathcurveto{\pgfqpoint{4.234814in}{0.780817in}}{\pgfqpoint{4.239204in}{0.770218in}}{\pgfqpoint{4.247018in}{0.762404in}}%
\pgfpathcurveto{\pgfqpoint{4.254832in}{0.754590in}}{\pgfqpoint{4.265431in}{0.750200in}}{\pgfqpoint{4.276481in}{0.750200in}}%
\pgfpathlineto{\pgfqpoint{4.276481in}{0.750200in}}%
\pgfpathclose%
\pgfusepath{stroke}%
\end{pgfscope}%
\begin{pgfscope}%
\pgfpathrectangle{\pgfqpoint{0.847223in}{0.554012in}}{\pgfqpoint{6.200000in}{4.530000in}}%
\pgfusepath{clip}%
\pgfsetbuttcap%
\pgfsetroundjoin%
\pgfsetlinewidth{1.003750pt}%
\definecolor{currentstroke}{rgb}{1.000000,0.000000,0.000000}%
\pgfsetstrokecolor{currentstroke}%
\pgfsetdash{}{0pt}%
\pgfpathmoveto{\pgfqpoint{4.281814in}{0.749318in}}%
\pgfpathcurveto{\pgfqpoint{4.292864in}{0.749318in}}{\pgfqpoint{4.303463in}{0.753708in}}{\pgfqpoint{4.311277in}{0.761521in}}%
\pgfpathcurveto{\pgfqpoint{4.319090in}{0.769335in}}{\pgfqpoint{4.323481in}{0.779934in}}{\pgfqpoint{4.323481in}{0.790984in}}%
\pgfpathcurveto{\pgfqpoint{4.323481in}{0.802034in}}{\pgfqpoint{4.319090in}{0.812633in}}{\pgfqpoint{4.311277in}{0.820447in}}%
\pgfpathcurveto{\pgfqpoint{4.303463in}{0.828261in}}{\pgfqpoint{4.292864in}{0.832651in}}{\pgfqpoint{4.281814in}{0.832651in}}%
\pgfpathcurveto{\pgfqpoint{4.270764in}{0.832651in}}{\pgfqpoint{4.260165in}{0.828261in}}{\pgfqpoint{4.252351in}{0.820447in}}%
\pgfpathcurveto{\pgfqpoint{4.244538in}{0.812633in}}{\pgfqpoint{4.240147in}{0.802034in}}{\pgfqpoint{4.240147in}{0.790984in}}%
\pgfpathcurveto{\pgfqpoint{4.240147in}{0.779934in}}{\pgfqpoint{4.244538in}{0.769335in}}{\pgfqpoint{4.252351in}{0.761521in}}%
\pgfpathcurveto{\pgfqpoint{4.260165in}{0.753708in}}{\pgfqpoint{4.270764in}{0.749318in}}{\pgfqpoint{4.281814in}{0.749318in}}%
\pgfpathlineto{\pgfqpoint{4.281814in}{0.749318in}}%
\pgfpathclose%
\pgfusepath{stroke}%
\end{pgfscope}%
\begin{pgfscope}%
\pgfpathrectangle{\pgfqpoint{0.847223in}{0.554012in}}{\pgfqpoint{6.200000in}{4.530000in}}%
\pgfusepath{clip}%
\pgfsetbuttcap%
\pgfsetroundjoin%
\pgfsetlinewidth{1.003750pt}%
\definecolor{currentstroke}{rgb}{1.000000,0.000000,0.000000}%
\pgfsetstrokecolor{currentstroke}%
\pgfsetdash{}{0pt}%
\pgfpathmoveto{\pgfqpoint{4.287147in}{0.748437in}}%
\pgfpathcurveto{\pgfqpoint{4.298197in}{0.748437in}}{\pgfqpoint{4.308796in}{0.752828in}}{\pgfqpoint{4.316610in}{0.760641in}}%
\pgfpathcurveto{\pgfqpoint{4.324424in}{0.768455in}}{\pgfqpoint{4.328814in}{0.779054in}}{\pgfqpoint{4.328814in}{0.790104in}}%
\pgfpathcurveto{\pgfqpoint{4.328814in}{0.801154in}}{\pgfqpoint{4.324424in}{0.811753in}}{\pgfqpoint{4.316610in}{0.819567in}}%
\pgfpathcurveto{\pgfqpoint{4.308796in}{0.827380in}}{\pgfqpoint{4.298197in}{0.831771in}}{\pgfqpoint{4.287147in}{0.831771in}}%
\pgfpathcurveto{\pgfqpoint{4.276097in}{0.831771in}}{\pgfqpoint{4.265498in}{0.827380in}}{\pgfqpoint{4.257684in}{0.819567in}}%
\pgfpathcurveto{\pgfqpoint{4.249871in}{0.811753in}}{\pgfqpoint{4.245480in}{0.801154in}}{\pgfqpoint{4.245480in}{0.790104in}}%
\pgfpathcurveto{\pgfqpoint{4.245480in}{0.779054in}}{\pgfqpoint{4.249871in}{0.768455in}}{\pgfqpoint{4.257684in}{0.760641in}}%
\pgfpathcurveto{\pgfqpoint{4.265498in}{0.752828in}}{\pgfqpoint{4.276097in}{0.748437in}}{\pgfqpoint{4.287147in}{0.748437in}}%
\pgfpathlineto{\pgfqpoint{4.287147in}{0.748437in}}%
\pgfpathclose%
\pgfusepath{stroke}%
\end{pgfscope}%
\begin{pgfscope}%
\pgfpathrectangle{\pgfqpoint{0.847223in}{0.554012in}}{\pgfqpoint{6.200000in}{4.530000in}}%
\pgfusepath{clip}%
\pgfsetbuttcap%
\pgfsetroundjoin%
\pgfsetlinewidth{1.003750pt}%
\definecolor{currentstroke}{rgb}{1.000000,0.000000,0.000000}%
\pgfsetstrokecolor{currentstroke}%
\pgfsetdash{}{0pt}%
\pgfpathmoveto{\pgfqpoint{4.292480in}{0.747559in}}%
\pgfpathcurveto{\pgfqpoint{4.303531in}{0.747559in}}{\pgfqpoint{4.314130in}{0.751950in}}{\pgfqpoint{4.321943in}{0.759763in}}%
\pgfpathcurveto{\pgfqpoint{4.329757in}{0.767577in}}{\pgfqpoint{4.334147in}{0.778176in}}{\pgfqpoint{4.334147in}{0.789226in}}%
\pgfpathcurveto{\pgfqpoint{4.334147in}{0.800276in}}{\pgfqpoint{4.329757in}{0.810875in}}{\pgfqpoint{4.321943in}{0.818689in}}%
\pgfpathcurveto{\pgfqpoint{4.314130in}{0.826502in}}{\pgfqpoint{4.303531in}{0.830893in}}{\pgfqpoint{4.292480in}{0.830893in}}%
\pgfpathcurveto{\pgfqpoint{4.281430in}{0.830893in}}{\pgfqpoint{4.270831in}{0.826502in}}{\pgfqpoint{4.263018in}{0.818689in}}%
\pgfpathcurveto{\pgfqpoint{4.255204in}{0.810875in}}{\pgfqpoint{4.250814in}{0.800276in}}{\pgfqpoint{4.250814in}{0.789226in}}%
\pgfpathcurveto{\pgfqpoint{4.250814in}{0.778176in}}{\pgfqpoint{4.255204in}{0.767577in}}{\pgfqpoint{4.263018in}{0.759763in}}%
\pgfpathcurveto{\pgfqpoint{4.270831in}{0.751950in}}{\pgfqpoint{4.281430in}{0.747559in}}{\pgfqpoint{4.292480in}{0.747559in}}%
\pgfpathlineto{\pgfqpoint{4.292480in}{0.747559in}}%
\pgfpathclose%
\pgfusepath{stroke}%
\end{pgfscope}%
\begin{pgfscope}%
\pgfpathrectangle{\pgfqpoint{0.847223in}{0.554012in}}{\pgfqpoint{6.200000in}{4.530000in}}%
\pgfusepath{clip}%
\pgfsetbuttcap%
\pgfsetroundjoin%
\pgfsetlinewidth{1.003750pt}%
\definecolor{currentstroke}{rgb}{1.000000,0.000000,0.000000}%
\pgfsetstrokecolor{currentstroke}%
\pgfsetdash{}{0pt}%
\pgfpathmoveto{\pgfqpoint{4.297814in}{0.746684in}}%
\pgfpathcurveto{\pgfqpoint{4.308864in}{0.746684in}}{\pgfqpoint{4.319463in}{0.751074in}}{\pgfqpoint{4.327276in}{0.758888in}}%
\pgfpathcurveto{\pgfqpoint{4.335090in}{0.766701in}}{\pgfqpoint{4.339480in}{0.777300in}}{\pgfqpoint{4.339480in}{0.788351in}}%
\pgfpathcurveto{\pgfqpoint{4.339480in}{0.799401in}}{\pgfqpoint{4.335090in}{0.810000in}}{\pgfqpoint{4.327276in}{0.817813in}}%
\pgfpathcurveto{\pgfqpoint{4.319463in}{0.825627in}}{\pgfqpoint{4.308864in}{0.830017in}}{\pgfqpoint{4.297814in}{0.830017in}}%
\pgfpathcurveto{\pgfqpoint{4.286763in}{0.830017in}}{\pgfqpoint{4.276164in}{0.825627in}}{\pgfqpoint{4.268351in}{0.817813in}}%
\pgfpathcurveto{\pgfqpoint{4.260537in}{0.810000in}}{\pgfqpoint{4.256147in}{0.799401in}}{\pgfqpoint{4.256147in}{0.788351in}}%
\pgfpathcurveto{\pgfqpoint{4.256147in}{0.777300in}}{\pgfqpoint{4.260537in}{0.766701in}}{\pgfqpoint{4.268351in}{0.758888in}}%
\pgfpathcurveto{\pgfqpoint{4.276164in}{0.751074in}}{\pgfqpoint{4.286763in}{0.746684in}}{\pgfqpoint{4.297814in}{0.746684in}}%
\pgfpathlineto{\pgfqpoint{4.297814in}{0.746684in}}%
\pgfpathclose%
\pgfusepath{stroke}%
\end{pgfscope}%
\begin{pgfscope}%
\pgfpathrectangle{\pgfqpoint{0.847223in}{0.554012in}}{\pgfqpoint{6.200000in}{4.530000in}}%
\pgfusepath{clip}%
\pgfsetbuttcap%
\pgfsetroundjoin%
\pgfsetlinewidth{1.003750pt}%
\definecolor{currentstroke}{rgb}{1.000000,0.000000,0.000000}%
\pgfsetstrokecolor{currentstroke}%
\pgfsetdash{}{0pt}%
\pgfpathmoveto{\pgfqpoint{4.303147in}{0.745811in}}%
\pgfpathcurveto{\pgfqpoint{4.314197in}{0.745811in}}{\pgfqpoint{4.324796in}{0.750201in}}{\pgfqpoint{4.332610in}{0.758015in}}%
\pgfpathcurveto{\pgfqpoint{4.340423in}{0.765828in}}{\pgfqpoint{4.344813in}{0.776427in}}{\pgfqpoint{4.344813in}{0.787477in}}%
\pgfpathcurveto{\pgfqpoint{4.344813in}{0.798527in}}{\pgfqpoint{4.340423in}{0.809126in}}{\pgfqpoint{4.332610in}{0.816940in}}%
\pgfpathcurveto{\pgfqpoint{4.324796in}{0.824754in}}{\pgfqpoint{4.314197in}{0.829144in}}{\pgfqpoint{4.303147in}{0.829144in}}%
\pgfpathcurveto{\pgfqpoint{4.292097in}{0.829144in}}{\pgfqpoint{4.281498in}{0.824754in}}{\pgfqpoint{4.273684in}{0.816940in}}%
\pgfpathcurveto{\pgfqpoint{4.265870in}{0.809126in}}{\pgfqpoint{4.261480in}{0.798527in}}{\pgfqpoint{4.261480in}{0.787477in}}%
\pgfpathcurveto{\pgfqpoint{4.261480in}{0.776427in}}{\pgfqpoint{4.265870in}{0.765828in}}{\pgfqpoint{4.273684in}{0.758015in}}%
\pgfpathcurveto{\pgfqpoint{4.281498in}{0.750201in}}{\pgfqpoint{4.292097in}{0.745811in}}{\pgfqpoint{4.303147in}{0.745811in}}%
\pgfpathlineto{\pgfqpoint{4.303147in}{0.745811in}}%
\pgfpathclose%
\pgfusepath{stroke}%
\end{pgfscope}%
\begin{pgfscope}%
\pgfpathrectangle{\pgfqpoint{0.847223in}{0.554012in}}{\pgfqpoint{6.200000in}{4.530000in}}%
\pgfusepath{clip}%
\pgfsetbuttcap%
\pgfsetroundjoin%
\pgfsetlinewidth{1.003750pt}%
\definecolor{currentstroke}{rgb}{1.000000,0.000000,0.000000}%
\pgfsetstrokecolor{currentstroke}%
\pgfsetdash{}{0pt}%
\pgfpathmoveto{\pgfqpoint{4.308480in}{0.744940in}}%
\pgfpathcurveto{\pgfqpoint{4.319530in}{0.744940in}}{\pgfqpoint{4.330129in}{0.749330in}}{\pgfqpoint{4.337943in}{0.757144in}}%
\pgfpathcurveto{\pgfqpoint{4.345756in}{0.764957in}}{\pgfqpoint{4.350147in}{0.775556in}}{\pgfqpoint{4.350147in}{0.786606in}}%
\pgfpathcurveto{\pgfqpoint{4.350147in}{0.797657in}}{\pgfqpoint{4.345756in}{0.808256in}}{\pgfqpoint{4.337943in}{0.816069in}}%
\pgfpathcurveto{\pgfqpoint{4.330129in}{0.823883in}}{\pgfqpoint{4.319530in}{0.828273in}}{\pgfqpoint{4.308480in}{0.828273in}}%
\pgfpathcurveto{\pgfqpoint{4.297430in}{0.828273in}}{\pgfqpoint{4.286831in}{0.823883in}}{\pgfqpoint{4.279017in}{0.816069in}}%
\pgfpathcurveto{\pgfqpoint{4.271204in}{0.808256in}}{\pgfqpoint{4.266813in}{0.797657in}}{\pgfqpoint{4.266813in}{0.786606in}}%
\pgfpathcurveto{\pgfqpoint{4.266813in}{0.775556in}}{\pgfqpoint{4.271204in}{0.764957in}}{\pgfqpoint{4.279017in}{0.757144in}}%
\pgfpathcurveto{\pgfqpoint{4.286831in}{0.749330in}}{\pgfqpoint{4.297430in}{0.744940in}}{\pgfqpoint{4.308480in}{0.744940in}}%
\pgfpathlineto{\pgfqpoint{4.308480in}{0.744940in}}%
\pgfpathclose%
\pgfusepath{stroke}%
\end{pgfscope}%
\begin{pgfscope}%
\pgfpathrectangle{\pgfqpoint{0.847223in}{0.554012in}}{\pgfqpoint{6.200000in}{4.530000in}}%
\pgfusepath{clip}%
\pgfsetbuttcap%
\pgfsetroundjoin%
\pgfsetlinewidth{1.003750pt}%
\definecolor{currentstroke}{rgb}{1.000000,0.000000,0.000000}%
\pgfsetstrokecolor{currentstroke}%
\pgfsetdash{}{0pt}%
\pgfpathmoveto{\pgfqpoint{4.313813in}{0.744071in}}%
\pgfpathcurveto{\pgfqpoint{4.324863in}{0.744071in}}{\pgfqpoint{4.335462in}{0.748462in}}{\pgfqpoint{4.343276in}{0.756275in}}%
\pgfpathcurveto{\pgfqpoint{4.351090in}{0.764089in}}{\pgfqpoint{4.355480in}{0.774688in}}{\pgfqpoint{4.355480in}{0.785738in}}%
\pgfpathcurveto{\pgfqpoint{4.355480in}{0.796788in}}{\pgfqpoint{4.351090in}{0.807387in}}{\pgfqpoint{4.343276in}{0.815201in}}%
\pgfpathcurveto{\pgfqpoint{4.335462in}{0.823014in}}{\pgfqpoint{4.324863in}{0.827405in}}{\pgfqpoint{4.313813in}{0.827405in}}%
\pgfpathcurveto{\pgfqpoint{4.302763in}{0.827405in}}{\pgfqpoint{4.292164in}{0.823014in}}{\pgfqpoint{4.284350in}{0.815201in}}%
\pgfpathcurveto{\pgfqpoint{4.276537in}{0.807387in}}{\pgfqpoint{4.272147in}{0.796788in}}{\pgfqpoint{4.272147in}{0.785738in}}%
\pgfpathcurveto{\pgfqpoint{4.272147in}{0.774688in}}{\pgfqpoint{4.276537in}{0.764089in}}{\pgfqpoint{4.284350in}{0.756275in}}%
\pgfpathcurveto{\pgfqpoint{4.292164in}{0.748462in}}{\pgfqpoint{4.302763in}{0.744071in}}{\pgfqpoint{4.313813in}{0.744071in}}%
\pgfpathlineto{\pgfqpoint{4.313813in}{0.744071in}}%
\pgfpathclose%
\pgfusepath{stroke}%
\end{pgfscope}%
\begin{pgfscope}%
\pgfpathrectangle{\pgfqpoint{0.847223in}{0.554012in}}{\pgfqpoint{6.200000in}{4.530000in}}%
\pgfusepath{clip}%
\pgfsetbuttcap%
\pgfsetroundjoin%
\pgfsetlinewidth{1.003750pt}%
\definecolor{currentstroke}{rgb}{1.000000,0.000000,0.000000}%
\pgfsetstrokecolor{currentstroke}%
\pgfsetdash{}{0pt}%
\pgfpathmoveto{\pgfqpoint{4.319146in}{0.743205in}}%
\pgfpathcurveto{\pgfqpoint{4.330197in}{0.743205in}}{\pgfqpoint{4.340796in}{0.747595in}}{\pgfqpoint{4.348609in}{0.755409in}}%
\pgfpathcurveto{\pgfqpoint{4.356423in}{0.763223in}}{\pgfqpoint{4.360813in}{0.773822in}}{\pgfqpoint{4.360813in}{0.784872in}}%
\pgfpathcurveto{\pgfqpoint{4.360813in}{0.795922in}}{\pgfqpoint{4.356423in}{0.806521in}}{\pgfqpoint{4.348609in}{0.814334in}}%
\pgfpathcurveto{\pgfqpoint{4.340796in}{0.822148in}}{\pgfqpoint{4.330197in}{0.826538in}}{\pgfqpoint{4.319146in}{0.826538in}}%
\pgfpathcurveto{\pgfqpoint{4.308096in}{0.826538in}}{\pgfqpoint{4.297497in}{0.822148in}}{\pgfqpoint{4.289684in}{0.814334in}}%
\pgfpathcurveto{\pgfqpoint{4.281870in}{0.806521in}}{\pgfqpoint{4.277480in}{0.795922in}}{\pgfqpoint{4.277480in}{0.784872in}}%
\pgfpathcurveto{\pgfqpoint{4.277480in}{0.773822in}}{\pgfqpoint{4.281870in}{0.763223in}}{\pgfqpoint{4.289684in}{0.755409in}}%
\pgfpathcurveto{\pgfqpoint{4.297497in}{0.747595in}}{\pgfqpoint{4.308096in}{0.743205in}}{\pgfqpoint{4.319146in}{0.743205in}}%
\pgfpathlineto{\pgfqpoint{4.319146in}{0.743205in}}%
\pgfpathclose%
\pgfusepath{stroke}%
\end{pgfscope}%
\begin{pgfscope}%
\pgfpathrectangle{\pgfqpoint{0.847223in}{0.554012in}}{\pgfqpoint{6.200000in}{4.530000in}}%
\pgfusepath{clip}%
\pgfsetbuttcap%
\pgfsetroundjoin%
\pgfsetlinewidth{1.003750pt}%
\definecolor{currentstroke}{rgb}{1.000000,0.000000,0.000000}%
\pgfsetstrokecolor{currentstroke}%
\pgfsetdash{}{0pt}%
\pgfpathmoveto{\pgfqpoint{4.324480in}{0.742341in}}%
\pgfpathcurveto{\pgfqpoint{4.335530in}{0.742341in}}{\pgfqpoint{4.346129in}{0.746731in}}{\pgfqpoint{4.353942in}{0.754545in}}%
\pgfpathcurveto{\pgfqpoint{4.361756in}{0.762359in}}{\pgfqpoint{4.366146in}{0.772958in}}{\pgfqpoint{4.366146in}{0.784008in}}%
\pgfpathcurveto{\pgfqpoint{4.366146in}{0.795058in}}{\pgfqpoint{4.361756in}{0.805657in}}{\pgfqpoint{4.353942in}{0.813471in}}%
\pgfpathcurveto{\pgfqpoint{4.346129in}{0.821284in}}{\pgfqpoint{4.335530in}{0.825674in}}{\pgfqpoint{4.324480in}{0.825674in}}%
\pgfpathcurveto{\pgfqpoint{4.313430in}{0.825674in}}{\pgfqpoint{4.302831in}{0.821284in}}{\pgfqpoint{4.295017in}{0.813471in}}%
\pgfpathcurveto{\pgfqpoint{4.287203in}{0.805657in}}{\pgfqpoint{4.282813in}{0.795058in}}{\pgfqpoint{4.282813in}{0.784008in}}%
\pgfpathcurveto{\pgfqpoint{4.282813in}{0.772958in}}{\pgfqpoint{4.287203in}{0.762359in}}{\pgfqpoint{4.295017in}{0.754545in}}%
\pgfpathcurveto{\pgfqpoint{4.302831in}{0.746731in}}{\pgfqpoint{4.313430in}{0.742341in}}{\pgfqpoint{4.324480in}{0.742341in}}%
\pgfpathlineto{\pgfqpoint{4.324480in}{0.742341in}}%
\pgfpathclose%
\pgfusepath{stroke}%
\end{pgfscope}%
\begin{pgfscope}%
\pgfpathrectangle{\pgfqpoint{0.847223in}{0.554012in}}{\pgfqpoint{6.200000in}{4.530000in}}%
\pgfusepath{clip}%
\pgfsetbuttcap%
\pgfsetroundjoin%
\pgfsetlinewidth{1.003750pt}%
\definecolor{currentstroke}{rgb}{1.000000,0.000000,0.000000}%
\pgfsetstrokecolor{currentstroke}%
\pgfsetdash{}{0pt}%
\pgfpathmoveto{\pgfqpoint{4.329813in}{0.741480in}}%
\pgfpathcurveto{\pgfqpoint{4.340863in}{0.741480in}}{\pgfqpoint{4.351462in}{0.745870in}}{\pgfqpoint{4.359276in}{0.753683in}}%
\pgfpathcurveto{\pgfqpoint{4.367089in}{0.761497in}}{\pgfqpoint{4.371480in}{0.772096in}}{\pgfqpoint{4.371480in}{0.783146in}}%
\pgfpathcurveto{\pgfqpoint{4.371480in}{0.794196in}}{\pgfqpoint{4.367089in}{0.804795in}}{\pgfqpoint{4.359276in}{0.812609in}}%
\pgfpathcurveto{\pgfqpoint{4.351462in}{0.820423in}}{\pgfqpoint{4.340863in}{0.824813in}}{\pgfqpoint{4.329813in}{0.824813in}}%
\pgfpathcurveto{\pgfqpoint{4.318763in}{0.824813in}}{\pgfqpoint{4.308164in}{0.820423in}}{\pgfqpoint{4.300350in}{0.812609in}}%
\pgfpathcurveto{\pgfqpoint{4.292536in}{0.804795in}}{\pgfqpoint{4.288146in}{0.794196in}}{\pgfqpoint{4.288146in}{0.783146in}}%
\pgfpathcurveto{\pgfqpoint{4.288146in}{0.772096in}}{\pgfqpoint{4.292536in}{0.761497in}}{\pgfqpoint{4.300350in}{0.753683in}}%
\pgfpathcurveto{\pgfqpoint{4.308164in}{0.745870in}}{\pgfqpoint{4.318763in}{0.741480in}}{\pgfqpoint{4.329813in}{0.741480in}}%
\pgfpathlineto{\pgfqpoint{4.329813in}{0.741480in}}%
\pgfpathclose%
\pgfusepath{stroke}%
\end{pgfscope}%
\begin{pgfscope}%
\pgfpathrectangle{\pgfqpoint{0.847223in}{0.554012in}}{\pgfqpoint{6.200000in}{4.530000in}}%
\pgfusepath{clip}%
\pgfsetbuttcap%
\pgfsetroundjoin%
\pgfsetlinewidth{1.003750pt}%
\definecolor{currentstroke}{rgb}{1.000000,0.000000,0.000000}%
\pgfsetstrokecolor{currentstroke}%
\pgfsetdash{}{0pt}%
\pgfpathmoveto{\pgfqpoint{4.335146in}{0.740620in}}%
\pgfpathcurveto{\pgfqpoint{4.346196in}{0.740620in}}{\pgfqpoint{4.356795in}{0.745010in}}{\pgfqpoint{4.364609in}{0.752824in}}%
\pgfpathcurveto{\pgfqpoint{4.372423in}{0.760638in}}{\pgfqpoint{4.376813in}{0.771237in}}{\pgfqpoint{4.376813in}{0.782287in}}%
\pgfpathcurveto{\pgfqpoint{4.376813in}{0.793337in}}{\pgfqpoint{4.372423in}{0.803936in}}{\pgfqpoint{4.364609in}{0.811750in}}%
\pgfpathcurveto{\pgfqpoint{4.356795in}{0.819563in}}{\pgfqpoint{4.346196in}{0.823954in}}{\pgfqpoint{4.335146in}{0.823954in}}%
\pgfpathcurveto{\pgfqpoint{4.324096in}{0.823954in}}{\pgfqpoint{4.313497in}{0.819563in}}{\pgfqpoint{4.305683in}{0.811750in}}%
\pgfpathcurveto{\pgfqpoint{4.297870in}{0.803936in}}{\pgfqpoint{4.293479in}{0.793337in}}{\pgfqpoint{4.293479in}{0.782287in}}%
\pgfpathcurveto{\pgfqpoint{4.293479in}{0.771237in}}{\pgfqpoint{4.297870in}{0.760638in}}{\pgfqpoint{4.305683in}{0.752824in}}%
\pgfpathcurveto{\pgfqpoint{4.313497in}{0.745010in}}{\pgfqpoint{4.324096in}{0.740620in}}{\pgfqpoint{4.335146in}{0.740620in}}%
\pgfpathlineto{\pgfqpoint{4.335146in}{0.740620in}}%
\pgfpathclose%
\pgfusepath{stroke}%
\end{pgfscope}%
\begin{pgfscope}%
\pgfpathrectangle{\pgfqpoint{0.847223in}{0.554012in}}{\pgfqpoint{6.200000in}{4.530000in}}%
\pgfusepath{clip}%
\pgfsetbuttcap%
\pgfsetroundjoin%
\pgfsetlinewidth{1.003750pt}%
\definecolor{currentstroke}{rgb}{1.000000,0.000000,0.000000}%
\pgfsetstrokecolor{currentstroke}%
\pgfsetdash{}{0pt}%
\pgfpathmoveto{\pgfqpoint{4.340479in}{0.739763in}}%
\pgfpathcurveto{\pgfqpoint{4.351529in}{0.739763in}}{\pgfqpoint{4.362128in}{0.744153in}}{\pgfqpoint{4.369942in}{0.751967in}}%
\pgfpathcurveto{\pgfqpoint{4.377756in}{0.759781in}}{\pgfqpoint{4.382146in}{0.770380in}}{\pgfqpoint{4.382146in}{0.781430in}}%
\pgfpathcurveto{\pgfqpoint{4.382146in}{0.792480in}}{\pgfqpoint{4.377756in}{0.803079in}}{\pgfqpoint{4.369942in}{0.810893in}}%
\pgfpathcurveto{\pgfqpoint{4.362128in}{0.818706in}}{\pgfqpoint{4.351529in}{0.823096in}}{\pgfqpoint{4.340479in}{0.823096in}}%
\pgfpathcurveto{\pgfqpoint{4.329429in}{0.823096in}}{\pgfqpoint{4.318830in}{0.818706in}}{\pgfqpoint{4.311017in}{0.810893in}}%
\pgfpathcurveto{\pgfqpoint{4.303203in}{0.803079in}}{\pgfqpoint{4.298813in}{0.792480in}}{\pgfqpoint{4.298813in}{0.781430in}}%
\pgfpathcurveto{\pgfqpoint{4.298813in}{0.770380in}}{\pgfqpoint{4.303203in}{0.759781in}}{\pgfqpoint{4.311017in}{0.751967in}}%
\pgfpathcurveto{\pgfqpoint{4.318830in}{0.744153in}}{\pgfqpoint{4.329429in}{0.739763in}}{\pgfqpoint{4.340479in}{0.739763in}}%
\pgfpathlineto{\pgfqpoint{4.340479in}{0.739763in}}%
\pgfpathclose%
\pgfusepath{stroke}%
\end{pgfscope}%
\begin{pgfscope}%
\pgfpathrectangle{\pgfqpoint{0.847223in}{0.554012in}}{\pgfqpoint{6.200000in}{4.530000in}}%
\pgfusepath{clip}%
\pgfsetbuttcap%
\pgfsetroundjoin%
\pgfsetlinewidth{1.003750pt}%
\definecolor{currentstroke}{rgb}{1.000000,0.000000,0.000000}%
\pgfsetstrokecolor{currentstroke}%
\pgfsetdash{}{0pt}%
\pgfpathmoveto{\pgfqpoint{4.345813in}{0.738908in}}%
\pgfpathcurveto{\pgfqpoint{4.356863in}{0.738908in}}{\pgfqpoint{4.367462in}{0.743299in}}{\pgfqpoint{4.375275in}{0.751112in}}%
\pgfpathcurveto{\pgfqpoint{4.383089in}{0.758926in}}{\pgfqpoint{4.387479in}{0.769525in}}{\pgfqpoint{4.387479in}{0.780575in}}%
\pgfpathcurveto{\pgfqpoint{4.387479in}{0.791625in}}{\pgfqpoint{4.383089in}{0.802224in}}{\pgfqpoint{4.375275in}{0.810038in}}%
\pgfpathcurveto{\pgfqpoint{4.367462in}{0.817851in}}{\pgfqpoint{4.356863in}{0.822242in}}{\pgfqpoint{4.345813in}{0.822242in}}%
\pgfpathcurveto{\pgfqpoint{4.334762in}{0.822242in}}{\pgfqpoint{4.324163in}{0.817851in}}{\pgfqpoint{4.316350in}{0.810038in}}%
\pgfpathcurveto{\pgfqpoint{4.308536in}{0.802224in}}{\pgfqpoint{4.304146in}{0.791625in}}{\pgfqpoint{4.304146in}{0.780575in}}%
\pgfpathcurveto{\pgfqpoint{4.304146in}{0.769525in}}{\pgfqpoint{4.308536in}{0.758926in}}{\pgfqpoint{4.316350in}{0.751112in}}%
\pgfpathcurveto{\pgfqpoint{4.324163in}{0.743299in}}{\pgfqpoint{4.334762in}{0.738908in}}{\pgfqpoint{4.345813in}{0.738908in}}%
\pgfpathlineto{\pgfqpoint{4.345813in}{0.738908in}}%
\pgfpathclose%
\pgfusepath{stroke}%
\end{pgfscope}%
\begin{pgfscope}%
\pgfpathrectangle{\pgfqpoint{0.847223in}{0.554012in}}{\pgfqpoint{6.200000in}{4.530000in}}%
\pgfusepath{clip}%
\pgfsetbuttcap%
\pgfsetroundjoin%
\pgfsetlinewidth{1.003750pt}%
\definecolor{currentstroke}{rgb}{1.000000,0.000000,0.000000}%
\pgfsetstrokecolor{currentstroke}%
\pgfsetdash{}{0pt}%
\pgfpathmoveto{\pgfqpoint{4.351146in}{0.738056in}}%
\pgfpathcurveto{\pgfqpoint{4.362196in}{0.738056in}}{\pgfqpoint{4.372795in}{0.742446in}}{\pgfqpoint{4.380609in}{0.750260in}}%
\pgfpathcurveto{\pgfqpoint{4.388422in}{0.758073in}}{\pgfqpoint{4.392812in}{0.768672in}}{\pgfqpoint{4.392812in}{0.779723in}}%
\pgfpathcurveto{\pgfqpoint{4.392812in}{0.790773in}}{\pgfqpoint{4.388422in}{0.801372in}}{\pgfqpoint{4.380609in}{0.809185in}}%
\pgfpathcurveto{\pgfqpoint{4.372795in}{0.816999in}}{\pgfqpoint{4.362196in}{0.821389in}}{\pgfqpoint{4.351146in}{0.821389in}}%
\pgfpathcurveto{\pgfqpoint{4.340096in}{0.821389in}}{\pgfqpoint{4.329497in}{0.816999in}}{\pgfqpoint{4.321683in}{0.809185in}}%
\pgfpathcurveto{\pgfqpoint{4.313869in}{0.801372in}}{\pgfqpoint{4.309479in}{0.790773in}}{\pgfqpoint{4.309479in}{0.779723in}}%
\pgfpathcurveto{\pgfqpoint{4.309479in}{0.768672in}}{\pgfqpoint{4.313869in}{0.758073in}}{\pgfqpoint{4.321683in}{0.750260in}}%
\pgfpathcurveto{\pgfqpoint{4.329497in}{0.742446in}}{\pgfqpoint{4.340096in}{0.738056in}}{\pgfqpoint{4.351146in}{0.738056in}}%
\pgfpathlineto{\pgfqpoint{4.351146in}{0.738056in}}%
\pgfpathclose%
\pgfusepath{stroke}%
\end{pgfscope}%
\begin{pgfscope}%
\pgfpathrectangle{\pgfqpoint{0.847223in}{0.554012in}}{\pgfqpoint{6.200000in}{4.530000in}}%
\pgfusepath{clip}%
\pgfsetbuttcap%
\pgfsetroundjoin%
\pgfsetlinewidth{1.003750pt}%
\definecolor{currentstroke}{rgb}{1.000000,0.000000,0.000000}%
\pgfsetstrokecolor{currentstroke}%
\pgfsetdash{}{0pt}%
\pgfpathmoveto{\pgfqpoint{4.356479in}{0.737206in}}%
\pgfpathcurveto{\pgfqpoint{4.367529in}{0.737206in}}{\pgfqpoint{4.378128in}{0.741596in}}{\pgfqpoint{4.385942in}{0.749410in}}%
\pgfpathcurveto{\pgfqpoint{4.393755in}{0.757223in}}{\pgfqpoint{4.398146in}{0.767822in}}{\pgfqpoint{4.398146in}{0.778872in}}%
\pgfpathcurveto{\pgfqpoint{4.398146in}{0.789922in}}{\pgfqpoint{4.393755in}{0.800521in}}{\pgfqpoint{4.385942in}{0.808335in}}%
\pgfpathcurveto{\pgfqpoint{4.378128in}{0.816149in}}{\pgfqpoint{4.367529in}{0.820539in}}{\pgfqpoint{4.356479in}{0.820539in}}%
\pgfpathcurveto{\pgfqpoint{4.345429in}{0.820539in}}{\pgfqpoint{4.334830in}{0.816149in}}{\pgfqpoint{4.327016in}{0.808335in}}%
\pgfpathcurveto{\pgfqpoint{4.319203in}{0.800521in}}{\pgfqpoint{4.314812in}{0.789922in}}{\pgfqpoint{4.314812in}{0.778872in}}%
\pgfpathcurveto{\pgfqpoint{4.314812in}{0.767822in}}{\pgfqpoint{4.319203in}{0.757223in}}{\pgfqpoint{4.327016in}{0.749410in}}%
\pgfpathcurveto{\pgfqpoint{4.334830in}{0.741596in}}{\pgfqpoint{4.345429in}{0.737206in}}{\pgfqpoint{4.356479in}{0.737206in}}%
\pgfpathlineto{\pgfqpoint{4.356479in}{0.737206in}}%
\pgfpathclose%
\pgfusepath{stroke}%
\end{pgfscope}%
\begin{pgfscope}%
\pgfpathrectangle{\pgfqpoint{0.847223in}{0.554012in}}{\pgfqpoint{6.200000in}{4.530000in}}%
\pgfusepath{clip}%
\pgfsetbuttcap%
\pgfsetroundjoin%
\pgfsetlinewidth{1.003750pt}%
\definecolor{currentstroke}{rgb}{1.000000,0.000000,0.000000}%
\pgfsetstrokecolor{currentstroke}%
\pgfsetdash{}{0pt}%
\pgfpathmoveto{\pgfqpoint{4.361812in}{0.736358in}}%
\pgfpathcurveto{\pgfqpoint{4.372862in}{0.736358in}}{\pgfqpoint{4.383461in}{0.740748in}}{\pgfqpoint{4.391275in}{0.748562in}}%
\pgfpathcurveto{\pgfqpoint{4.399089in}{0.756375in}}{\pgfqpoint{4.403479in}{0.766974in}}{\pgfqpoint{4.403479in}{0.778024in}}%
\pgfpathcurveto{\pgfqpoint{4.403479in}{0.789074in}}{\pgfqpoint{4.399089in}{0.799673in}}{\pgfqpoint{4.391275in}{0.807487in}}%
\pgfpathcurveto{\pgfqpoint{4.383461in}{0.815301in}}{\pgfqpoint{4.372862in}{0.819691in}}{\pgfqpoint{4.361812in}{0.819691in}}%
\pgfpathcurveto{\pgfqpoint{4.350762in}{0.819691in}}{\pgfqpoint{4.340163in}{0.815301in}}{\pgfqpoint{4.332349in}{0.807487in}}%
\pgfpathcurveto{\pgfqpoint{4.324536in}{0.799673in}}{\pgfqpoint{4.320146in}{0.789074in}}{\pgfqpoint{4.320146in}{0.778024in}}%
\pgfpathcurveto{\pgfqpoint{4.320146in}{0.766974in}}{\pgfqpoint{4.324536in}{0.756375in}}{\pgfqpoint{4.332349in}{0.748562in}}%
\pgfpathcurveto{\pgfqpoint{4.340163in}{0.740748in}}{\pgfqpoint{4.350762in}{0.736358in}}{\pgfqpoint{4.361812in}{0.736358in}}%
\pgfpathlineto{\pgfqpoint{4.361812in}{0.736358in}}%
\pgfpathclose%
\pgfusepath{stroke}%
\end{pgfscope}%
\begin{pgfscope}%
\pgfpathrectangle{\pgfqpoint{0.847223in}{0.554012in}}{\pgfqpoint{6.200000in}{4.530000in}}%
\pgfusepath{clip}%
\pgfsetbuttcap%
\pgfsetroundjoin%
\pgfsetlinewidth{1.003750pt}%
\definecolor{currentstroke}{rgb}{1.000000,0.000000,0.000000}%
\pgfsetstrokecolor{currentstroke}%
\pgfsetdash{}{0pt}%
\pgfpathmoveto{\pgfqpoint{4.367145in}{0.735512in}}%
\pgfpathcurveto{\pgfqpoint{4.378196in}{0.735512in}}{\pgfqpoint{4.388795in}{0.739902in}}{\pgfqpoint{4.396608in}{0.747716in}}%
\pgfpathcurveto{\pgfqpoint{4.404422in}{0.755529in}}{\pgfqpoint{4.408812in}{0.766128in}}{\pgfqpoint{4.408812in}{0.777179in}}%
\pgfpathcurveto{\pgfqpoint{4.408812in}{0.788229in}}{\pgfqpoint{4.404422in}{0.798828in}}{\pgfqpoint{4.396608in}{0.806641in}}%
\pgfpathcurveto{\pgfqpoint{4.388795in}{0.814455in}}{\pgfqpoint{4.378196in}{0.818845in}}{\pgfqpoint{4.367145in}{0.818845in}}%
\pgfpathcurveto{\pgfqpoint{4.356095in}{0.818845in}}{\pgfqpoint{4.345496in}{0.814455in}}{\pgfqpoint{4.337683in}{0.806641in}}%
\pgfpathcurveto{\pgfqpoint{4.329869in}{0.798828in}}{\pgfqpoint{4.325479in}{0.788229in}}{\pgfqpoint{4.325479in}{0.777179in}}%
\pgfpathcurveto{\pgfqpoint{4.325479in}{0.766128in}}{\pgfqpoint{4.329869in}{0.755529in}}{\pgfqpoint{4.337683in}{0.747716in}}%
\pgfpathcurveto{\pgfqpoint{4.345496in}{0.739902in}}{\pgfqpoint{4.356095in}{0.735512in}}{\pgfqpoint{4.367145in}{0.735512in}}%
\pgfpathlineto{\pgfqpoint{4.367145in}{0.735512in}}%
\pgfpathclose%
\pgfusepath{stroke}%
\end{pgfscope}%
\begin{pgfscope}%
\pgfpathrectangle{\pgfqpoint{0.847223in}{0.554012in}}{\pgfqpoint{6.200000in}{4.530000in}}%
\pgfusepath{clip}%
\pgfsetbuttcap%
\pgfsetroundjoin%
\pgfsetlinewidth{1.003750pt}%
\definecolor{currentstroke}{rgb}{1.000000,0.000000,0.000000}%
\pgfsetstrokecolor{currentstroke}%
\pgfsetdash{}{0pt}%
\pgfpathmoveto{\pgfqpoint{4.372479in}{0.734668in}}%
\pgfpathcurveto{\pgfqpoint{4.383529in}{0.734668in}}{\pgfqpoint{4.394128in}{0.739059in}}{\pgfqpoint{4.401941in}{0.746872in}}%
\pgfpathcurveto{\pgfqpoint{4.409755in}{0.754686in}}{\pgfqpoint{4.414145in}{0.765285in}}{\pgfqpoint{4.414145in}{0.776335in}}%
\pgfpathcurveto{\pgfqpoint{4.414145in}{0.787385in}}{\pgfqpoint{4.409755in}{0.797984in}}{\pgfqpoint{4.401941in}{0.805798in}}%
\pgfpathcurveto{\pgfqpoint{4.394128in}{0.813611in}}{\pgfqpoint{4.383529in}{0.818002in}}{\pgfqpoint{4.372479in}{0.818002in}}%
\pgfpathcurveto{\pgfqpoint{4.361428in}{0.818002in}}{\pgfqpoint{4.350829in}{0.813611in}}{\pgfqpoint{4.343016in}{0.805798in}}%
\pgfpathcurveto{\pgfqpoint{4.335202in}{0.797984in}}{\pgfqpoint{4.330812in}{0.787385in}}{\pgfqpoint{4.330812in}{0.776335in}}%
\pgfpathcurveto{\pgfqpoint{4.330812in}{0.765285in}}{\pgfqpoint{4.335202in}{0.754686in}}{\pgfqpoint{4.343016in}{0.746872in}}%
\pgfpathcurveto{\pgfqpoint{4.350829in}{0.739059in}}{\pgfqpoint{4.361428in}{0.734668in}}{\pgfqpoint{4.372479in}{0.734668in}}%
\pgfpathlineto{\pgfqpoint{4.372479in}{0.734668in}}%
\pgfpathclose%
\pgfusepath{stroke}%
\end{pgfscope}%
\begin{pgfscope}%
\pgfpathrectangle{\pgfqpoint{0.847223in}{0.554012in}}{\pgfqpoint{6.200000in}{4.530000in}}%
\pgfusepath{clip}%
\pgfsetbuttcap%
\pgfsetroundjoin%
\pgfsetlinewidth{1.003750pt}%
\definecolor{currentstroke}{rgb}{1.000000,0.000000,0.000000}%
\pgfsetstrokecolor{currentstroke}%
\pgfsetdash{}{0pt}%
\pgfpathmoveto{\pgfqpoint{4.377812in}{0.733827in}}%
\pgfpathcurveto{\pgfqpoint{4.388862in}{0.733827in}}{\pgfqpoint{4.399461in}{0.738217in}}{\pgfqpoint{4.407275in}{0.746031in}}%
\pgfpathcurveto{\pgfqpoint{4.415088in}{0.753845in}}{\pgfqpoint{4.419478in}{0.764444in}}{\pgfqpoint{4.419478in}{0.775494in}}%
\pgfpathcurveto{\pgfqpoint{4.419478in}{0.786544in}}{\pgfqpoint{4.415088in}{0.797143in}}{\pgfqpoint{4.407275in}{0.804957in}}%
\pgfpathcurveto{\pgfqpoint{4.399461in}{0.812770in}}{\pgfqpoint{4.388862in}{0.817160in}}{\pgfqpoint{4.377812in}{0.817160in}}%
\pgfpathcurveto{\pgfqpoint{4.366762in}{0.817160in}}{\pgfqpoint{4.356163in}{0.812770in}}{\pgfqpoint{4.348349in}{0.804957in}}%
\pgfpathcurveto{\pgfqpoint{4.340535in}{0.797143in}}{\pgfqpoint{4.336145in}{0.786544in}}{\pgfqpoint{4.336145in}{0.775494in}}%
\pgfpathcurveto{\pgfqpoint{4.336145in}{0.764444in}}{\pgfqpoint{4.340535in}{0.753845in}}{\pgfqpoint{4.348349in}{0.746031in}}%
\pgfpathcurveto{\pgfqpoint{4.356163in}{0.738217in}}{\pgfqpoint{4.366762in}{0.733827in}}{\pgfqpoint{4.377812in}{0.733827in}}%
\pgfpathlineto{\pgfqpoint{4.377812in}{0.733827in}}%
\pgfpathclose%
\pgfusepath{stroke}%
\end{pgfscope}%
\begin{pgfscope}%
\pgfpathrectangle{\pgfqpoint{0.847223in}{0.554012in}}{\pgfqpoint{6.200000in}{4.530000in}}%
\pgfusepath{clip}%
\pgfsetbuttcap%
\pgfsetroundjoin%
\pgfsetlinewidth{1.003750pt}%
\definecolor{currentstroke}{rgb}{1.000000,0.000000,0.000000}%
\pgfsetstrokecolor{currentstroke}%
\pgfsetdash{}{0pt}%
\pgfpathmoveto{\pgfqpoint{4.383145in}{0.732988in}}%
\pgfpathcurveto{\pgfqpoint{4.394195in}{0.732988in}}{\pgfqpoint{4.404794in}{0.737378in}}{\pgfqpoint{4.412608in}{0.745192in}}%
\pgfpathcurveto{\pgfqpoint{4.420421in}{0.753005in}}{\pgfqpoint{4.424812in}{0.763605in}}{\pgfqpoint{4.424812in}{0.774655in}}%
\pgfpathcurveto{\pgfqpoint{4.424812in}{0.785705in}}{\pgfqpoint{4.420421in}{0.796304in}}{\pgfqpoint{4.412608in}{0.804117in}}%
\pgfpathcurveto{\pgfqpoint{4.404794in}{0.811931in}}{\pgfqpoint{4.394195in}{0.816321in}}{\pgfqpoint{4.383145in}{0.816321in}}%
\pgfpathcurveto{\pgfqpoint{4.372095in}{0.816321in}}{\pgfqpoint{4.361496in}{0.811931in}}{\pgfqpoint{4.353682in}{0.804117in}}%
\pgfpathcurveto{\pgfqpoint{4.345869in}{0.796304in}}{\pgfqpoint{4.341478in}{0.785705in}}{\pgfqpoint{4.341478in}{0.774655in}}%
\pgfpathcurveto{\pgfqpoint{4.341478in}{0.763605in}}{\pgfqpoint{4.345869in}{0.753005in}}{\pgfqpoint{4.353682in}{0.745192in}}%
\pgfpathcurveto{\pgfqpoint{4.361496in}{0.737378in}}{\pgfqpoint{4.372095in}{0.732988in}}{\pgfqpoint{4.383145in}{0.732988in}}%
\pgfpathlineto{\pgfqpoint{4.383145in}{0.732988in}}%
\pgfpathclose%
\pgfusepath{stroke}%
\end{pgfscope}%
\begin{pgfscope}%
\pgfpathrectangle{\pgfqpoint{0.847223in}{0.554012in}}{\pgfqpoint{6.200000in}{4.530000in}}%
\pgfusepath{clip}%
\pgfsetbuttcap%
\pgfsetroundjoin%
\pgfsetlinewidth{1.003750pt}%
\definecolor{currentstroke}{rgb}{1.000000,0.000000,0.000000}%
\pgfsetstrokecolor{currentstroke}%
\pgfsetdash{}{0pt}%
\pgfpathmoveto{\pgfqpoint{4.388478in}{0.732151in}}%
\pgfpathcurveto{\pgfqpoint{4.399528in}{0.732151in}}{\pgfqpoint{4.410127in}{0.736541in}}{\pgfqpoint{4.417941in}{0.744355in}}%
\pgfpathcurveto{\pgfqpoint{4.425755in}{0.752169in}}{\pgfqpoint{4.430145in}{0.762768in}}{\pgfqpoint{4.430145in}{0.773818in}}%
\pgfpathcurveto{\pgfqpoint{4.430145in}{0.784868in}}{\pgfqpoint{4.425755in}{0.795467in}}{\pgfqpoint{4.417941in}{0.803281in}}%
\pgfpathcurveto{\pgfqpoint{4.410127in}{0.811094in}}{\pgfqpoint{4.399528in}{0.815484in}}{\pgfqpoint{4.388478in}{0.815484in}}%
\pgfpathcurveto{\pgfqpoint{4.377428in}{0.815484in}}{\pgfqpoint{4.366829in}{0.811094in}}{\pgfqpoint{4.359015in}{0.803281in}}%
\pgfpathcurveto{\pgfqpoint{4.351202in}{0.795467in}}{\pgfqpoint{4.346812in}{0.784868in}}{\pgfqpoint{4.346812in}{0.773818in}}%
\pgfpathcurveto{\pgfqpoint{4.346812in}{0.762768in}}{\pgfqpoint{4.351202in}{0.752169in}}{\pgfqpoint{4.359015in}{0.744355in}}%
\pgfpathcurveto{\pgfqpoint{4.366829in}{0.736541in}}{\pgfqpoint{4.377428in}{0.732151in}}{\pgfqpoint{4.388478in}{0.732151in}}%
\pgfpathlineto{\pgfqpoint{4.388478in}{0.732151in}}%
\pgfpathclose%
\pgfusepath{stroke}%
\end{pgfscope}%
\begin{pgfscope}%
\pgfpathrectangle{\pgfqpoint{0.847223in}{0.554012in}}{\pgfqpoint{6.200000in}{4.530000in}}%
\pgfusepath{clip}%
\pgfsetbuttcap%
\pgfsetroundjoin%
\pgfsetlinewidth{1.003750pt}%
\definecolor{currentstroke}{rgb}{1.000000,0.000000,0.000000}%
\pgfsetstrokecolor{currentstroke}%
\pgfsetdash{}{0pt}%
\pgfpathmoveto{\pgfqpoint{4.393811in}{0.731316in}}%
\pgfpathcurveto{\pgfqpoint{4.404862in}{0.731316in}}{\pgfqpoint{4.415461in}{0.735707in}}{\pgfqpoint{4.423274in}{0.743520in}}%
\pgfpathcurveto{\pgfqpoint{4.431088in}{0.751334in}}{\pgfqpoint{4.435478in}{0.761933in}}{\pgfqpoint{4.435478in}{0.772983in}}%
\pgfpathcurveto{\pgfqpoint{4.435478in}{0.784033in}}{\pgfqpoint{4.431088in}{0.794632in}}{\pgfqpoint{4.423274in}{0.802446in}}%
\pgfpathcurveto{\pgfqpoint{4.415461in}{0.810259in}}{\pgfqpoint{4.404862in}{0.814650in}}{\pgfqpoint{4.393811in}{0.814650in}}%
\pgfpathcurveto{\pgfqpoint{4.382761in}{0.814650in}}{\pgfqpoint{4.372162in}{0.810259in}}{\pgfqpoint{4.364349in}{0.802446in}}%
\pgfpathcurveto{\pgfqpoint{4.356535in}{0.794632in}}{\pgfqpoint{4.352145in}{0.784033in}}{\pgfqpoint{4.352145in}{0.772983in}}%
\pgfpathcurveto{\pgfqpoint{4.352145in}{0.761933in}}{\pgfqpoint{4.356535in}{0.751334in}}{\pgfqpoint{4.364349in}{0.743520in}}%
\pgfpathcurveto{\pgfqpoint{4.372162in}{0.735707in}}{\pgfqpoint{4.382761in}{0.731316in}}{\pgfqpoint{4.393811in}{0.731316in}}%
\pgfpathlineto{\pgfqpoint{4.393811in}{0.731316in}}%
\pgfpathclose%
\pgfusepath{stroke}%
\end{pgfscope}%
\begin{pgfscope}%
\pgfpathrectangle{\pgfqpoint{0.847223in}{0.554012in}}{\pgfqpoint{6.200000in}{4.530000in}}%
\pgfusepath{clip}%
\pgfsetbuttcap%
\pgfsetroundjoin%
\pgfsetlinewidth{1.003750pt}%
\definecolor{currentstroke}{rgb}{1.000000,0.000000,0.000000}%
\pgfsetstrokecolor{currentstroke}%
\pgfsetdash{}{0pt}%
\pgfpathmoveto{\pgfqpoint{4.399145in}{0.730484in}}%
\pgfpathcurveto{\pgfqpoint{4.410195in}{0.730484in}}{\pgfqpoint{4.420794in}{0.734874in}}{\pgfqpoint{4.428607in}{0.742688in}}%
\pgfpathcurveto{\pgfqpoint{4.436421in}{0.750501in}}{\pgfqpoint{4.440811in}{0.761100in}}{\pgfqpoint{4.440811in}{0.772151in}}%
\pgfpathcurveto{\pgfqpoint{4.440811in}{0.783201in}}{\pgfqpoint{4.436421in}{0.793800in}}{\pgfqpoint{4.428607in}{0.801613in}}%
\pgfpathcurveto{\pgfqpoint{4.420794in}{0.809427in}}{\pgfqpoint{4.410195in}{0.813817in}}{\pgfqpoint{4.399145in}{0.813817in}}%
\pgfpathcurveto{\pgfqpoint{4.388095in}{0.813817in}}{\pgfqpoint{4.377496in}{0.809427in}}{\pgfqpoint{4.369682in}{0.801613in}}%
\pgfpathcurveto{\pgfqpoint{4.361868in}{0.793800in}}{\pgfqpoint{4.357478in}{0.783201in}}{\pgfqpoint{4.357478in}{0.772151in}}%
\pgfpathcurveto{\pgfqpoint{4.357478in}{0.761100in}}{\pgfqpoint{4.361868in}{0.750501in}}{\pgfqpoint{4.369682in}{0.742688in}}%
\pgfpathcurveto{\pgfqpoint{4.377496in}{0.734874in}}{\pgfqpoint{4.388095in}{0.730484in}}{\pgfqpoint{4.399145in}{0.730484in}}%
\pgfpathlineto{\pgfqpoint{4.399145in}{0.730484in}}%
\pgfpathclose%
\pgfusepath{stroke}%
\end{pgfscope}%
\begin{pgfscope}%
\pgfpathrectangle{\pgfqpoint{0.847223in}{0.554012in}}{\pgfqpoint{6.200000in}{4.530000in}}%
\pgfusepath{clip}%
\pgfsetbuttcap%
\pgfsetroundjoin%
\pgfsetlinewidth{1.003750pt}%
\definecolor{currentstroke}{rgb}{1.000000,0.000000,0.000000}%
\pgfsetstrokecolor{currentstroke}%
\pgfsetdash{}{0pt}%
\pgfpathmoveto{\pgfqpoint{4.404478in}{0.729653in}}%
\pgfpathcurveto{\pgfqpoint{4.415528in}{0.729653in}}{\pgfqpoint{4.426127in}{0.734044in}}{\pgfqpoint{4.433941in}{0.741857in}}%
\pgfpathcurveto{\pgfqpoint{4.441754in}{0.749671in}}{\pgfqpoint{4.446145in}{0.760270in}}{\pgfqpoint{4.446145in}{0.771320in}}%
\pgfpathcurveto{\pgfqpoint{4.446145in}{0.782370in}}{\pgfqpoint{4.441754in}{0.792969in}}{\pgfqpoint{4.433941in}{0.800783in}}%
\pgfpathcurveto{\pgfqpoint{4.426127in}{0.808597in}}{\pgfqpoint{4.415528in}{0.812987in}}{\pgfqpoint{4.404478in}{0.812987in}}%
\pgfpathcurveto{\pgfqpoint{4.393428in}{0.812987in}}{\pgfqpoint{4.382829in}{0.808597in}}{\pgfqpoint{4.375015in}{0.800783in}}%
\pgfpathcurveto{\pgfqpoint{4.367201in}{0.792969in}}{\pgfqpoint{4.362811in}{0.782370in}}{\pgfqpoint{4.362811in}{0.771320in}}%
\pgfpathcurveto{\pgfqpoint{4.362811in}{0.760270in}}{\pgfqpoint{4.367201in}{0.749671in}}{\pgfqpoint{4.375015in}{0.741857in}}%
\pgfpathcurveto{\pgfqpoint{4.382829in}{0.734044in}}{\pgfqpoint{4.393428in}{0.729653in}}{\pgfqpoint{4.404478in}{0.729653in}}%
\pgfpathlineto{\pgfqpoint{4.404478in}{0.729653in}}%
\pgfpathclose%
\pgfusepath{stroke}%
\end{pgfscope}%
\begin{pgfscope}%
\pgfpathrectangle{\pgfqpoint{0.847223in}{0.554012in}}{\pgfqpoint{6.200000in}{4.530000in}}%
\pgfusepath{clip}%
\pgfsetbuttcap%
\pgfsetroundjoin%
\pgfsetlinewidth{1.003750pt}%
\definecolor{currentstroke}{rgb}{1.000000,0.000000,0.000000}%
\pgfsetstrokecolor{currentstroke}%
\pgfsetdash{}{0pt}%
\pgfpathmoveto{\pgfqpoint{4.409811in}{0.728825in}}%
\pgfpathcurveto{\pgfqpoint{4.420861in}{0.728825in}}{\pgfqpoint{4.431460in}{0.733216in}}{\pgfqpoint{4.439274in}{0.741029in}}%
\pgfpathcurveto{\pgfqpoint{4.447088in}{0.748843in}}{\pgfqpoint{4.451478in}{0.759442in}}{\pgfqpoint{4.451478in}{0.770492in}}%
\pgfpathcurveto{\pgfqpoint{4.451478in}{0.781542in}}{\pgfqpoint{4.447088in}{0.792141in}}{\pgfqpoint{4.439274in}{0.799955in}}%
\pgfpathcurveto{\pgfqpoint{4.431460in}{0.807768in}}{\pgfqpoint{4.420861in}{0.812159in}}{\pgfqpoint{4.409811in}{0.812159in}}%
\pgfpathcurveto{\pgfqpoint{4.398761in}{0.812159in}}{\pgfqpoint{4.388162in}{0.807768in}}{\pgfqpoint{4.380348in}{0.799955in}}%
\pgfpathcurveto{\pgfqpoint{4.372535in}{0.792141in}}{\pgfqpoint{4.368144in}{0.781542in}}{\pgfqpoint{4.368144in}{0.770492in}}%
\pgfpathcurveto{\pgfqpoint{4.368144in}{0.759442in}}{\pgfqpoint{4.372535in}{0.748843in}}{\pgfqpoint{4.380348in}{0.741029in}}%
\pgfpathcurveto{\pgfqpoint{4.388162in}{0.733216in}}{\pgfqpoint{4.398761in}{0.728825in}}{\pgfqpoint{4.409811in}{0.728825in}}%
\pgfpathlineto{\pgfqpoint{4.409811in}{0.728825in}}%
\pgfpathclose%
\pgfusepath{stroke}%
\end{pgfscope}%
\begin{pgfscope}%
\pgfpathrectangle{\pgfqpoint{0.847223in}{0.554012in}}{\pgfqpoint{6.200000in}{4.530000in}}%
\pgfusepath{clip}%
\pgfsetbuttcap%
\pgfsetroundjoin%
\pgfsetlinewidth{1.003750pt}%
\definecolor{currentstroke}{rgb}{1.000000,0.000000,0.000000}%
\pgfsetstrokecolor{currentstroke}%
\pgfsetdash{}{0pt}%
\pgfpathmoveto{\pgfqpoint{4.415144in}{0.727999in}}%
\pgfpathcurveto{\pgfqpoint{4.426194in}{0.727999in}}{\pgfqpoint{4.436793in}{0.732390in}}{\pgfqpoint{4.444607in}{0.740203in}}%
\pgfpathcurveto{\pgfqpoint{4.452421in}{0.748017in}}{\pgfqpoint{4.456811in}{0.758616in}}{\pgfqpoint{4.456811in}{0.769666in}}%
\pgfpathcurveto{\pgfqpoint{4.456811in}{0.780716in}}{\pgfqpoint{4.452421in}{0.791315in}}{\pgfqpoint{4.444607in}{0.799129in}}%
\pgfpathcurveto{\pgfqpoint{4.436793in}{0.806942in}}{\pgfqpoint{4.426194in}{0.811333in}}{\pgfqpoint{4.415144in}{0.811333in}}%
\pgfpathcurveto{\pgfqpoint{4.404094in}{0.811333in}}{\pgfqpoint{4.393495in}{0.806942in}}{\pgfqpoint{4.385682in}{0.799129in}}%
\pgfpathcurveto{\pgfqpoint{4.377868in}{0.791315in}}{\pgfqpoint{4.373478in}{0.780716in}}{\pgfqpoint{4.373478in}{0.769666in}}%
\pgfpathcurveto{\pgfqpoint{4.373478in}{0.758616in}}{\pgfqpoint{4.377868in}{0.748017in}}{\pgfqpoint{4.385682in}{0.740203in}}%
\pgfpathcurveto{\pgfqpoint{4.393495in}{0.732390in}}{\pgfqpoint{4.404094in}{0.727999in}}{\pgfqpoint{4.415144in}{0.727999in}}%
\pgfpathlineto{\pgfqpoint{4.415144in}{0.727999in}}%
\pgfpathclose%
\pgfusepath{stroke}%
\end{pgfscope}%
\begin{pgfscope}%
\pgfpathrectangle{\pgfqpoint{0.847223in}{0.554012in}}{\pgfqpoint{6.200000in}{4.530000in}}%
\pgfusepath{clip}%
\pgfsetbuttcap%
\pgfsetroundjoin%
\pgfsetlinewidth{1.003750pt}%
\definecolor{currentstroke}{rgb}{1.000000,0.000000,0.000000}%
\pgfsetstrokecolor{currentstroke}%
\pgfsetdash{}{0pt}%
\pgfpathmoveto{\pgfqpoint{4.420478in}{0.727175in}}%
\pgfpathcurveto{\pgfqpoint{4.431528in}{0.727175in}}{\pgfqpoint{4.442127in}{0.731566in}}{\pgfqpoint{4.449940in}{0.739379in}}%
\pgfpathcurveto{\pgfqpoint{4.457754in}{0.747193in}}{\pgfqpoint{4.462144in}{0.757792in}}{\pgfqpoint{4.462144in}{0.768842in}}%
\pgfpathcurveto{\pgfqpoint{4.462144in}{0.779892in}}{\pgfqpoint{4.457754in}{0.790491in}}{\pgfqpoint{4.449940in}{0.798305in}}%
\pgfpathcurveto{\pgfqpoint{4.442127in}{0.806119in}}{\pgfqpoint{4.431528in}{0.810509in}}{\pgfqpoint{4.420478in}{0.810509in}}%
\pgfpathcurveto{\pgfqpoint{4.409427in}{0.810509in}}{\pgfqpoint{4.398828in}{0.806119in}}{\pgfqpoint{4.391015in}{0.798305in}}%
\pgfpathcurveto{\pgfqpoint{4.383201in}{0.790491in}}{\pgfqpoint{4.378811in}{0.779892in}}{\pgfqpoint{4.378811in}{0.768842in}}%
\pgfpathcurveto{\pgfqpoint{4.378811in}{0.757792in}}{\pgfqpoint{4.383201in}{0.747193in}}{\pgfqpoint{4.391015in}{0.739379in}}%
\pgfpathcurveto{\pgfqpoint{4.398828in}{0.731566in}}{\pgfqpoint{4.409427in}{0.727175in}}{\pgfqpoint{4.420478in}{0.727175in}}%
\pgfpathlineto{\pgfqpoint{4.420478in}{0.727175in}}%
\pgfpathclose%
\pgfusepath{stroke}%
\end{pgfscope}%
\begin{pgfscope}%
\pgfpathrectangle{\pgfqpoint{0.847223in}{0.554012in}}{\pgfqpoint{6.200000in}{4.530000in}}%
\pgfusepath{clip}%
\pgfsetbuttcap%
\pgfsetroundjoin%
\pgfsetlinewidth{1.003750pt}%
\definecolor{currentstroke}{rgb}{1.000000,0.000000,0.000000}%
\pgfsetstrokecolor{currentstroke}%
\pgfsetdash{}{0pt}%
\pgfpathmoveto{\pgfqpoint{4.425811in}{0.726354in}}%
\pgfpathcurveto{\pgfqpoint{4.436861in}{0.726354in}}{\pgfqpoint{4.447460in}{0.730744in}}{\pgfqpoint{4.455274in}{0.738558in}}%
\pgfpathcurveto{\pgfqpoint{4.463087in}{0.746371in}}{\pgfqpoint{4.467477in}{0.756970in}}{\pgfqpoint{4.467477in}{0.768020in}}%
\pgfpathcurveto{\pgfqpoint{4.467477in}{0.779071in}}{\pgfqpoint{4.463087in}{0.789670in}}{\pgfqpoint{4.455274in}{0.797483in}}%
\pgfpathcurveto{\pgfqpoint{4.447460in}{0.805297in}}{\pgfqpoint{4.436861in}{0.809687in}}{\pgfqpoint{4.425811in}{0.809687in}}%
\pgfpathcurveto{\pgfqpoint{4.414761in}{0.809687in}}{\pgfqpoint{4.404162in}{0.805297in}}{\pgfqpoint{4.396348in}{0.797483in}}%
\pgfpathcurveto{\pgfqpoint{4.388534in}{0.789670in}}{\pgfqpoint{4.384144in}{0.779071in}}{\pgfqpoint{4.384144in}{0.768020in}}%
\pgfpathcurveto{\pgfqpoint{4.384144in}{0.756970in}}{\pgfqpoint{4.388534in}{0.746371in}}{\pgfqpoint{4.396348in}{0.738558in}}%
\pgfpathcurveto{\pgfqpoint{4.404162in}{0.730744in}}{\pgfqpoint{4.414761in}{0.726354in}}{\pgfqpoint{4.425811in}{0.726354in}}%
\pgfpathlineto{\pgfqpoint{4.425811in}{0.726354in}}%
\pgfpathclose%
\pgfusepath{stroke}%
\end{pgfscope}%
\begin{pgfscope}%
\pgfpathrectangle{\pgfqpoint{0.847223in}{0.554012in}}{\pgfqpoint{6.200000in}{4.530000in}}%
\pgfusepath{clip}%
\pgfsetbuttcap%
\pgfsetroundjoin%
\pgfsetlinewidth{1.003750pt}%
\definecolor{currentstroke}{rgb}{1.000000,0.000000,0.000000}%
\pgfsetstrokecolor{currentstroke}%
\pgfsetdash{}{0pt}%
\pgfpathmoveto{\pgfqpoint{4.431144in}{0.725534in}}%
\pgfpathcurveto{\pgfqpoint{4.442194in}{0.725534in}}{\pgfqpoint{4.452793in}{0.729924in}}{\pgfqpoint{4.460607in}{0.737738in}}%
\pgfpathcurveto{\pgfqpoint{4.468420in}{0.745552in}}{\pgfqpoint{4.472811in}{0.756151in}}{\pgfqpoint{4.472811in}{0.767201in}}%
\pgfpathcurveto{\pgfqpoint{4.472811in}{0.778251in}}{\pgfqpoint{4.468420in}{0.788850in}}{\pgfqpoint{4.460607in}{0.796664in}}%
\pgfpathcurveto{\pgfqpoint{4.452793in}{0.804477in}}{\pgfqpoint{4.442194in}{0.808867in}}{\pgfqpoint{4.431144in}{0.808867in}}%
\pgfpathcurveto{\pgfqpoint{4.420094in}{0.808867in}}{\pgfqpoint{4.409495in}{0.804477in}}{\pgfqpoint{4.401681in}{0.796664in}}%
\pgfpathcurveto{\pgfqpoint{4.393868in}{0.788850in}}{\pgfqpoint{4.389477in}{0.778251in}}{\pgfqpoint{4.389477in}{0.767201in}}%
\pgfpathcurveto{\pgfqpoint{4.389477in}{0.756151in}}{\pgfqpoint{4.393868in}{0.745552in}}{\pgfqpoint{4.401681in}{0.737738in}}%
\pgfpathcurveto{\pgfqpoint{4.409495in}{0.729924in}}{\pgfqpoint{4.420094in}{0.725534in}}{\pgfqpoint{4.431144in}{0.725534in}}%
\pgfpathlineto{\pgfqpoint{4.431144in}{0.725534in}}%
\pgfpathclose%
\pgfusepath{stroke}%
\end{pgfscope}%
\begin{pgfscope}%
\pgfpathrectangle{\pgfqpoint{0.847223in}{0.554012in}}{\pgfqpoint{6.200000in}{4.530000in}}%
\pgfusepath{clip}%
\pgfsetbuttcap%
\pgfsetroundjoin%
\pgfsetlinewidth{1.003750pt}%
\definecolor{currentstroke}{rgb}{1.000000,0.000000,0.000000}%
\pgfsetstrokecolor{currentstroke}%
\pgfsetdash{}{0pt}%
\pgfpathmoveto{\pgfqpoint{4.436477in}{0.724717in}}%
\pgfpathcurveto{\pgfqpoint{4.447527in}{0.724717in}}{\pgfqpoint{4.458126in}{0.729107in}}{\pgfqpoint{4.465940in}{0.736921in}}%
\pgfpathcurveto{\pgfqpoint{4.473754in}{0.744734in}}{\pgfqpoint{4.478144in}{0.755333in}}{\pgfqpoint{4.478144in}{0.766383in}}%
\pgfpathcurveto{\pgfqpoint{4.478144in}{0.777433in}}{\pgfqpoint{4.473754in}{0.788032in}}{\pgfqpoint{4.465940in}{0.795846in}}%
\pgfpathcurveto{\pgfqpoint{4.458126in}{0.803660in}}{\pgfqpoint{4.447527in}{0.808050in}}{\pgfqpoint{4.436477in}{0.808050in}}%
\pgfpathcurveto{\pgfqpoint{4.425427in}{0.808050in}}{\pgfqpoint{4.414828in}{0.803660in}}{\pgfqpoint{4.407014in}{0.795846in}}%
\pgfpathcurveto{\pgfqpoint{4.399201in}{0.788032in}}{\pgfqpoint{4.394811in}{0.777433in}}{\pgfqpoint{4.394811in}{0.766383in}}%
\pgfpathcurveto{\pgfqpoint{4.394811in}{0.755333in}}{\pgfqpoint{4.399201in}{0.744734in}}{\pgfqpoint{4.407014in}{0.736921in}}%
\pgfpathcurveto{\pgfqpoint{4.414828in}{0.729107in}}{\pgfqpoint{4.425427in}{0.724717in}}{\pgfqpoint{4.436477in}{0.724717in}}%
\pgfpathlineto{\pgfqpoint{4.436477in}{0.724717in}}%
\pgfpathclose%
\pgfusepath{stroke}%
\end{pgfscope}%
\begin{pgfscope}%
\pgfpathrectangle{\pgfqpoint{0.847223in}{0.554012in}}{\pgfqpoint{6.200000in}{4.530000in}}%
\pgfusepath{clip}%
\pgfsetbuttcap%
\pgfsetroundjoin%
\pgfsetlinewidth{1.003750pt}%
\definecolor{currentstroke}{rgb}{1.000000,0.000000,0.000000}%
\pgfsetstrokecolor{currentstroke}%
\pgfsetdash{}{0pt}%
\pgfpathmoveto{\pgfqpoint{4.441810in}{0.723901in}}%
\pgfpathcurveto{\pgfqpoint{4.452861in}{0.723901in}}{\pgfqpoint{4.463460in}{0.728292in}}{\pgfqpoint{4.471273in}{0.736105in}}%
\pgfpathcurveto{\pgfqpoint{4.479087in}{0.743919in}}{\pgfqpoint{4.483477in}{0.754518in}}{\pgfqpoint{4.483477in}{0.765568in}}%
\pgfpathcurveto{\pgfqpoint{4.483477in}{0.776618in}}{\pgfqpoint{4.479087in}{0.787217in}}{\pgfqpoint{4.471273in}{0.795031in}}%
\pgfpathcurveto{\pgfqpoint{4.463460in}{0.802844in}}{\pgfqpoint{4.452861in}{0.807235in}}{\pgfqpoint{4.441810in}{0.807235in}}%
\pgfpathcurveto{\pgfqpoint{4.430760in}{0.807235in}}{\pgfqpoint{4.420161in}{0.802844in}}{\pgfqpoint{4.412348in}{0.795031in}}%
\pgfpathcurveto{\pgfqpoint{4.404534in}{0.787217in}}{\pgfqpoint{4.400144in}{0.776618in}}{\pgfqpoint{4.400144in}{0.765568in}}%
\pgfpathcurveto{\pgfqpoint{4.400144in}{0.754518in}}{\pgfqpoint{4.404534in}{0.743919in}}{\pgfqpoint{4.412348in}{0.736105in}}%
\pgfpathcurveto{\pgfqpoint{4.420161in}{0.728292in}}{\pgfqpoint{4.430760in}{0.723901in}}{\pgfqpoint{4.441810in}{0.723901in}}%
\pgfpathlineto{\pgfqpoint{4.441810in}{0.723901in}}%
\pgfpathclose%
\pgfusepath{stroke}%
\end{pgfscope}%
\begin{pgfscope}%
\pgfpathrectangle{\pgfqpoint{0.847223in}{0.554012in}}{\pgfqpoint{6.200000in}{4.530000in}}%
\pgfusepath{clip}%
\pgfsetbuttcap%
\pgfsetroundjoin%
\pgfsetlinewidth{1.003750pt}%
\definecolor{currentstroke}{rgb}{1.000000,0.000000,0.000000}%
\pgfsetstrokecolor{currentstroke}%
\pgfsetdash{}{0pt}%
\pgfpathmoveto{\pgfqpoint{4.447144in}{0.723088in}}%
\pgfpathcurveto{\pgfqpoint{4.458194in}{0.723088in}}{\pgfqpoint{4.468793in}{0.727478in}}{\pgfqpoint{4.476606in}{0.735292in}}%
\pgfpathcurveto{\pgfqpoint{4.484420in}{0.743106in}}{\pgfqpoint{4.488810in}{0.753705in}}{\pgfqpoint{4.488810in}{0.764755in}}%
\pgfpathcurveto{\pgfqpoint{4.488810in}{0.775805in}}{\pgfqpoint{4.484420in}{0.786404in}}{\pgfqpoint{4.476606in}{0.794218in}}%
\pgfpathcurveto{\pgfqpoint{4.468793in}{0.802031in}}{\pgfqpoint{4.458194in}{0.806421in}}{\pgfqpoint{4.447144in}{0.806421in}}%
\pgfpathcurveto{\pgfqpoint{4.436093in}{0.806421in}}{\pgfqpoint{4.425494in}{0.802031in}}{\pgfqpoint{4.417681in}{0.794218in}}%
\pgfpathcurveto{\pgfqpoint{4.409867in}{0.786404in}}{\pgfqpoint{4.405477in}{0.775805in}}{\pgfqpoint{4.405477in}{0.764755in}}%
\pgfpathcurveto{\pgfqpoint{4.405477in}{0.753705in}}{\pgfqpoint{4.409867in}{0.743106in}}{\pgfqpoint{4.417681in}{0.735292in}}%
\pgfpathcurveto{\pgfqpoint{4.425494in}{0.727478in}}{\pgfqpoint{4.436093in}{0.723088in}}{\pgfqpoint{4.447144in}{0.723088in}}%
\pgfpathlineto{\pgfqpoint{4.447144in}{0.723088in}}%
\pgfpathclose%
\pgfusepath{stroke}%
\end{pgfscope}%
\begin{pgfscope}%
\pgfpathrectangle{\pgfqpoint{0.847223in}{0.554012in}}{\pgfqpoint{6.200000in}{4.530000in}}%
\pgfusepath{clip}%
\pgfsetbuttcap%
\pgfsetroundjoin%
\pgfsetlinewidth{1.003750pt}%
\definecolor{currentstroke}{rgb}{1.000000,0.000000,0.000000}%
\pgfsetstrokecolor{currentstroke}%
\pgfsetdash{}{0pt}%
\pgfpathmoveto{\pgfqpoint{4.452477in}{0.722277in}}%
\pgfpathcurveto{\pgfqpoint{4.463527in}{0.722277in}}{\pgfqpoint{4.474126in}{0.726667in}}{\pgfqpoint{4.481940in}{0.734481in}}%
\pgfpathcurveto{\pgfqpoint{4.489753in}{0.742294in}}{\pgfqpoint{4.494144in}{0.752893in}}{\pgfqpoint{4.494144in}{0.763944in}}%
\pgfpathcurveto{\pgfqpoint{4.494144in}{0.774994in}}{\pgfqpoint{4.489753in}{0.785593in}}{\pgfqpoint{4.481940in}{0.793406in}}%
\pgfpathcurveto{\pgfqpoint{4.474126in}{0.801220in}}{\pgfqpoint{4.463527in}{0.805610in}}{\pgfqpoint{4.452477in}{0.805610in}}%
\pgfpathcurveto{\pgfqpoint{4.441427in}{0.805610in}}{\pgfqpoint{4.430828in}{0.801220in}}{\pgfqpoint{4.423014in}{0.793406in}}%
\pgfpathcurveto{\pgfqpoint{4.415200in}{0.785593in}}{\pgfqpoint{4.410810in}{0.774994in}}{\pgfqpoint{4.410810in}{0.763944in}}%
\pgfpathcurveto{\pgfqpoint{4.410810in}{0.752893in}}{\pgfqpoint{4.415200in}{0.742294in}}{\pgfqpoint{4.423014in}{0.734481in}}%
\pgfpathcurveto{\pgfqpoint{4.430828in}{0.726667in}}{\pgfqpoint{4.441427in}{0.722277in}}{\pgfqpoint{4.452477in}{0.722277in}}%
\pgfpathlineto{\pgfqpoint{4.452477in}{0.722277in}}%
\pgfpathclose%
\pgfusepath{stroke}%
\end{pgfscope}%
\begin{pgfscope}%
\pgfpathrectangle{\pgfqpoint{0.847223in}{0.554012in}}{\pgfqpoint{6.200000in}{4.530000in}}%
\pgfusepath{clip}%
\pgfsetbuttcap%
\pgfsetroundjoin%
\pgfsetlinewidth{1.003750pt}%
\definecolor{currentstroke}{rgb}{1.000000,0.000000,0.000000}%
\pgfsetstrokecolor{currentstroke}%
\pgfsetdash{}{0pt}%
\pgfpathmoveto{\pgfqpoint{4.457810in}{0.721468in}}%
\pgfpathcurveto{\pgfqpoint{4.468860in}{0.721468in}}{\pgfqpoint{4.479459in}{0.725858in}}{\pgfqpoint{4.487273in}{0.733672in}}%
\pgfpathcurveto{\pgfqpoint{4.495086in}{0.741485in}}{\pgfqpoint{4.499477in}{0.752084in}}{\pgfqpoint{4.499477in}{0.763135in}}%
\pgfpathcurveto{\pgfqpoint{4.499477in}{0.774185in}}{\pgfqpoint{4.495086in}{0.784784in}}{\pgfqpoint{4.487273in}{0.792597in}}%
\pgfpathcurveto{\pgfqpoint{4.479459in}{0.800411in}}{\pgfqpoint{4.468860in}{0.804801in}}{\pgfqpoint{4.457810in}{0.804801in}}%
\pgfpathcurveto{\pgfqpoint{4.446760in}{0.804801in}}{\pgfqpoint{4.436161in}{0.800411in}}{\pgfqpoint{4.428347in}{0.792597in}}%
\pgfpathcurveto{\pgfqpoint{4.420534in}{0.784784in}}{\pgfqpoint{4.416143in}{0.774185in}}{\pgfqpoint{4.416143in}{0.763135in}}%
\pgfpathcurveto{\pgfqpoint{4.416143in}{0.752084in}}{\pgfqpoint{4.420534in}{0.741485in}}{\pgfqpoint{4.428347in}{0.733672in}}%
\pgfpathcurveto{\pgfqpoint{4.436161in}{0.725858in}}{\pgfqpoint{4.446760in}{0.721468in}}{\pgfqpoint{4.457810in}{0.721468in}}%
\pgfpathlineto{\pgfqpoint{4.457810in}{0.721468in}}%
\pgfpathclose%
\pgfusepath{stroke}%
\end{pgfscope}%
\begin{pgfscope}%
\pgfpathrectangle{\pgfqpoint{0.847223in}{0.554012in}}{\pgfqpoint{6.200000in}{4.530000in}}%
\pgfusepath{clip}%
\pgfsetbuttcap%
\pgfsetroundjoin%
\pgfsetlinewidth{1.003750pt}%
\definecolor{currentstroke}{rgb}{1.000000,0.000000,0.000000}%
\pgfsetstrokecolor{currentstroke}%
\pgfsetdash{}{0pt}%
\pgfpathmoveto{\pgfqpoint{4.463143in}{0.720661in}}%
\pgfpathcurveto{\pgfqpoint{4.474193in}{0.720661in}}{\pgfqpoint{4.484792in}{0.725051in}}{\pgfqpoint{4.492606in}{0.732865in}}%
\pgfpathcurveto{\pgfqpoint{4.500420in}{0.740678in}}{\pgfqpoint{4.504810in}{0.751277in}}{\pgfqpoint{4.504810in}{0.762328in}}%
\pgfpathcurveto{\pgfqpoint{4.504810in}{0.773378in}}{\pgfqpoint{4.500420in}{0.783977in}}{\pgfqpoint{4.492606in}{0.791790in}}%
\pgfpathcurveto{\pgfqpoint{4.484792in}{0.799604in}}{\pgfqpoint{4.474193in}{0.803994in}}{\pgfqpoint{4.463143in}{0.803994in}}%
\pgfpathcurveto{\pgfqpoint{4.452093in}{0.803994in}}{\pgfqpoint{4.441494in}{0.799604in}}{\pgfqpoint{4.433680in}{0.791790in}}%
\pgfpathcurveto{\pgfqpoint{4.425867in}{0.783977in}}{\pgfqpoint{4.421477in}{0.773378in}}{\pgfqpoint{4.421477in}{0.762328in}}%
\pgfpathcurveto{\pgfqpoint{4.421477in}{0.751277in}}{\pgfqpoint{4.425867in}{0.740678in}}{\pgfqpoint{4.433680in}{0.732865in}}%
\pgfpathcurveto{\pgfqpoint{4.441494in}{0.725051in}}{\pgfqpoint{4.452093in}{0.720661in}}{\pgfqpoint{4.463143in}{0.720661in}}%
\pgfpathlineto{\pgfqpoint{4.463143in}{0.720661in}}%
\pgfpathclose%
\pgfusepath{stroke}%
\end{pgfscope}%
\begin{pgfscope}%
\pgfpathrectangle{\pgfqpoint{0.847223in}{0.554012in}}{\pgfqpoint{6.200000in}{4.530000in}}%
\pgfusepath{clip}%
\pgfsetbuttcap%
\pgfsetroundjoin%
\pgfsetlinewidth{1.003750pt}%
\definecolor{currentstroke}{rgb}{1.000000,0.000000,0.000000}%
\pgfsetstrokecolor{currentstroke}%
\pgfsetdash{}{0pt}%
\pgfpathmoveto{\pgfqpoint{4.468476in}{0.719856in}}%
\pgfpathcurveto{\pgfqpoint{4.479527in}{0.719856in}}{\pgfqpoint{4.490126in}{0.724246in}}{\pgfqpoint{4.497939in}{0.732060in}}%
\pgfpathcurveto{\pgfqpoint{4.505753in}{0.739874in}}{\pgfqpoint{4.510143in}{0.750473in}}{\pgfqpoint{4.510143in}{0.761523in}}%
\pgfpathcurveto{\pgfqpoint{4.510143in}{0.772573in}}{\pgfqpoint{4.505753in}{0.783172in}}{\pgfqpoint{4.497939in}{0.790985in}}%
\pgfpathcurveto{\pgfqpoint{4.490126in}{0.798799in}}{\pgfqpoint{4.479527in}{0.803189in}}{\pgfqpoint{4.468476in}{0.803189in}}%
\pgfpathcurveto{\pgfqpoint{4.457426in}{0.803189in}}{\pgfqpoint{4.446827in}{0.798799in}}{\pgfqpoint{4.439014in}{0.790985in}}%
\pgfpathcurveto{\pgfqpoint{4.431200in}{0.783172in}}{\pgfqpoint{4.426810in}{0.772573in}}{\pgfqpoint{4.426810in}{0.761523in}}%
\pgfpathcurveto{\pgfqpoint{4.426810in}{0.750473in}}{\pgfqpoint{4.431200in}{0.739874in}}{\pgfqpoint{4.439014in}{0.732060in}}%
\pgfpathcurveto{\pgfqpoint{4.446827in}{0.724246in}}{\pgfqpoint{4.457426in}{0.719856in}}{\pgfqpoint{4.468476in}{0.719856in}}%
\pgfpathlineto{\pgfqpoint{4.468476in}{0.719856in}}%
\pgfpathclose%
\pgfusepath{stroke}%
\end{pgfscope}%
\begin{pgfscope}%
\pgfpathrectangle{\pgfqpoint{0.847223in}{0.554012in}}{\pgfqpoint{6.200000in}{4.530000in}}%
\pgfusepath{clip}%
\pgfsetbuttcap%
\pgfsetroundjoin%
\pgfsetlinewidth{1.003750pt}%
\definecolor{currentstroke}{rgb}{1.000000,0.000000,0.000000}%
\pgfsetstrokecolor{currentstroke}%
\pgfsetdash{}{0pt}%
\pgfpathmoveto{\pgfqpoint{4.473810in}{0.719053in}}%
\pgfpathcurveto{\pgfqpoint{4.484860in}{0.719053in}}{\pgfqpoint{4.495459in}{0.723443in}}{\pgfqpoint{4.503272in}{0.731257in}}%
\pgfpathcurveto{\pgfqpoint{4.511086in}{0.739071in}}{\pgfqpoint{4.515476in}{0.749670in}}{\pgfqpoint{4.515476in}{0.760720in}}%
\pgfpathcurveto{\pgfqpoint{4.515476in}{0.771770in}}{\pgfqpoint{4.511086in}{0.782369in}}{\pgfqpoint{4.503272in}{0.790183in}}%
\pgfpathcurveto{\pgfqpoint{4.495459in}{0.797996in}}{\pgfqpoint{4.484860in}{0.802387in}}{\pgfqpoint{4.473810in}{0.802387in}}%
\pgfpathcurveto{\pgfqpoint{4.462760in}{0.802387in}}{\pgfqpoint{4.452161in}{0.797996in}}{\pgfqpoint{4.444347in}{0.790183in}}%
\pgfpathcurveto{\pgfqpoint{4.436533in}{0.782369in}}{\pgfqpoint{4.432143in}{0.771770in}}{\pgfqpoint{4.432143in}{0.760720in}}%
\pgfpathcurveto{\pgfqpoint{4.432143in}{0.749670in}}{\pgfqpoint{4.436533in}{0.739071in}}{\pgfqpoint{4.444347in}{0.731257in}}%
\pgfpathcurveto{\pgfqpoint{4.452161in}{0.723443in}}{\pgfqpoint{4.462760in}{0.719053in}}{\pgfqpoint{4.473810in}{0.719053in}}%
\pgfpathlineto{\pgfqpoint{4.473810in}{0.719053in}}%
\pgfpathclose%
\pgfusepath{stroke}%
\end{pgfscope}%
\begin{pgfscope}%
\pgfpathrectangle{\pgfqpoint{0.847223in}{0.554012in}}{\pgfqpoint{6.200000in}{4.530000in}}%
\pgfusepath{clip}%
\pgfsetbuttcap%
\pgfsetroundjoin%
\pgfsetlinewidth{1.003750pt}%
\definecolor{currentstroke}{rgb}{1.000000,0.000000,0.000000}%
\pgfsetstrokecolor{currentstroke}%
\pgfsetdash{}{0pt}%
\pgfpathmoveto{\pgfqpoint{4.479143in}{0.718252in}}%
\pgfpathcurveto{\pgfqpoint{4.490193in}{0.718252in}}{\pgfqpoint{4.500792in}{0.722643in}}{\pgfqpoint{4.508606in}{0.730456in}}%
\pgfpathcurveto{\pgfqpoint{4.516419in}{0.738270in}}{\pgfqpoint{4.520810in}{0.748869in}}{\pgfqpoint{4.520810in}{0.759919in}}%
\pgfpathcurveto{\pgfqpoint{4.520810in}{0.770969in}}{\pgfqpoint{4.516419in}{0.781568in}}{\pgfqpoint{4.508606in}{0.789382in}}%
\pgfpathcurveto{\pgfqpoint{4.500792in}{0.797196in}}{\pgfqpoint{4.490193in}{0.801586in}}{\pgfqpoint{4.479143in}{0.801586in}}%
\pgfpathcurveto{\pgfqpoint{4.468093in}{0.801586in}}{\pgfqpoint{4.457494in}{0.797196in}}{\pgfqpoint{4.449680in}{0.789382in}}%
\pgfpathcurveto{\pgfqpoint{4.441867in}{0.781568in}}{\pgfqpoint{4.437476in}{0.770969in}}{\pgfqpoint{4.437476in}{0.759919in}}%
\pgfpathcurveto{\pgfqpoint{4.437476in}{0.748869in}}{\pgfqpoint{4.441867in}{0.738270in}}{\pgfqpoint{4.449680in}{0.730456in}}%
\pgfpathcurveto{\pgfqpoint{4.457494in}{0.722643in}}{\pgfqpoint{4.468093in}{0.718252in}}{\pgfqpoint{4.479143in}{0.718252in}}%
\pgfpathlineto{\pgfqpoint{4.479143in}{0.718252in}}%
\pgfpathclose%
\pgfusepath{stroke}%
\end{pgfscope}%
\begin{pgfscope}%
\pgfpathrectangle{\pgfqpoint{0.847223in}{0.554012in}}{\pgfqpoint{6.200000in}{4.530000in}}%
\pgfusepath{clip}%
\pgfsetbuttcap%
\pgfsetroundjoin%
\pgfsetlinewidth{1.003750pt}%
\definecolor{currentstroke}{rgb}{1.000000,0.000000,0.000000}%
\pgfsetstrokecolor{currentstroke}%
\pgfsetdash{}{0pt}%
\pgfpathmoveto{\pgfqpoint{4.484476in}{0.717454in}}%
\pgfpathcurveto{\pgfqpoint{4.495526in}{0.717454in}}{\pgfqpoint{4.506125in}{0.721844in}}{\pgfqpoint{4.513939in}{0.729658in}}%
\pgfpathcurveto{\pgfqpoint{4.521753in}{0.737471in}}{\pgfqpoint{4.526143in}{0.748070in}}{\pgfqpoint{4.526143in}{0.759120in}}%
\pgfpathcurveto{\pgfqpoint{4.526143in}{0.770171in}}{\pgfqpoint{4.521753in}{0.780770in}}{\pgfqpoint{4.513939in}{0.788583in}}%
\pgfpathcurveto{\pgfqpoint{4.506125in}{0.796397in}}{\pgfqpoint{4.495526in}{0.800787in}}{\pgfqpoint{4.484476in}{0.800787in}}%
\pgfpathcurveto{\pgfqpoint{4.473426in}{0.800787in}}{\pgfqpoint{4.462827in}{0.796397in}}{\pgfqpoint{4.455013in}{0.788583in}}%
\pgfpathcurveto{\pgfqpoint{4.447200in}{0.780770in}}{\pgfqpoint{4.442809in}{0.770171in}}{\pgfqpoint{4.442809in}{0.759120in}}%
\pgfpathcurveto{\pgfqpoint{4.442809in}{0.748070in}}{\pgfqpoint{4.447200in}{0.737471in}}{\pgfqpoint{4.455013in}{0.729658in}}%
\pgfpathcurveto{\pgfqpoint{4.462827in}{0.721844in}}{\pgfqpoint{4.473426in}{0.717454in}}{\pgfqpoint{4.484476in}{0.717454in}}%
\pgfpathlineto{\pgfqpoint{4.484476in}{0.717454in}}%
\pgfpathclose%
\pgfusepath{stroke}%
\end{pgfscope}%
\begin{pgfscope}%
\pgfpathrectangle{\pgfqpoint{0.847223in}{0.554012in}}{\pgfqpoint{6.200000in}{4.530000in}}%
\pgfusepath{clip}%
\pgfsetbuttcap%
\pgfsetroundjoin%
\pgfsetlinewidth{1.003750pt}%
\definecolor{currentstroke}{rgb}{1.000000,0.000000,0.000000}%
\pgfsetstrokecolor{currentstroke}%
\pgfsetdash{}{0pt}%
\pgfpathmoveto{\pgfqpoint{4.489809in}{0.716657in}}%
\pgfpathcurveto{\pgfqpoint{4.500859in}{0.716657in}}{\pgfqpoint{4.511459in}{0.721047in}}{\pgfqpoint{4.519272in}{0.728861in}}%
\pgfpathcurveto{\pgfqpoint{4.527086in}{0.736675in}}{\pgfqpoint{4.531476in}{0.747274in}}{\pgfqpoint{4.531476in}{0.758324in}}%
\pgfpathcurveto{\pgfqpoint{4.531476in}{0.769374in}}{\pgfqpoint{4.527086in}{0.779973in}}{\pgfqpoint{4.519272in}{0.787787in}}%
\pgfpathcurveto{\pgfqpoint{4.511459in}{0.795600in}}{\pgfqpoint{4.500859in}{0.799990in}}{\pgfqpoint{4.489809in}{0.799990in}}%
\pgfpathcurveto{\pgfqpoint{4.478759in}{0.799990in}}{\pgfqpoint{4.468160in}{0.795600in}}{\pgfqpoint{4.460347in}{0.787787in}}%
\pgfpathcurveto{\pgfqpoint{4.452533in}{0.779973in}}{\pgfqpoint{4.448143in}{0.769374in}}{\pgfqpoint{4.448143in}{0.758324in}}%
\pgfpathcurveto{\pgfqpoint{4.448143in}{0.747274in}}{\pgfqpoint{4.452533in}{0.736675in}}{\pgfqpoint{4.460347in}{0.728861in}}%
\pgfpathcurveto{\pgfqpoint{4.468160in}{0.721047in}}{\pgfqpoint{4.478759in}{0.716657in}}{\pgfqpoint{4.489809in}{0.716657in}}%
\pgfpathlineto{\pgfqpoint{4.489809in}{0.716657in}}%
\pgfpathclose%
\pgfusepath{stroke}%
\end{pgfscope}%
\begin{pgfscope}%
\pgfpathrectangle{\pgfqpoint{0.847223in}{0.554012in}}{\pgfqpoint{6.200000in}{4.530000in}}%
\pgfusepath{clip}%
\pgfsetbuttcap%
\pgfsetroundjoin%
\pgfsetlinewidth{1.003750pt}%
\definecolor{currentstroke}{rgb}{1.000000,0.000000,0.000000}%
\pgfsetstrokecolor{currentstroke}%
\pgfsetdash{}{0pt}%
\pgfpathmoveto{\pgfqpoint{4.495143in}{0.715862in}}%
\pgfpathcurveto{\pgfqpoint{4.506193in}{0.715862in}}{\pgfqpoint{4.516792in}{0.720253in}}{\pgfqpoint{4.524605in}{0.728066in}}%
\pgfpathcurveto{\pgfqpoint{4.532419in}{0.735880in}}{\pgfqpoint{4.536809in}{0.746479in}}{\pgfqpoint{4.536809in}{0.757529in}}%
\pgfpathcurveto{\pgfqpoint{4.536809in}{0.768579in}}{\pgfqpoint{4.532419in}{0.779178in}}{\pgfqpoint{4.524605in}{0.786992in}}%
\pgfpathcurveto{\pgfqpoint{4.516792in}{0.794806in}}{\pgfqpoint{4.506193in}{0.799196in}}{\pgfqpoint{4.495143in}{0.799196in}}%
\pgfpathcurveto{\pgfqpoint{4.484092in}{0.799196in}}{\pgfqpoint{4.473493in}{0.794806in}}{\pgfqpoint{4.465680in}{0.786992in}}%
\pgfpathcurveto{\pgfqpoint{4.457866in}{0.779178in}}{\pgfqpoint{4.453476in}{0.768579in}}{\pgfqpoint{4.453476in}{0.757529in}}%
\pgfpathcurveto{\pgfqpoint{4.453476in}{0.746479in}}{\pgfqpoint{4.457866in}{0.735880in}}{\pgfqpoint{4.465680in}{0.728066in}}%
\pgfpathcurveto{\pgfqpoint{4.473493in}{0.720253in}}{\pgfqpoint{4.484092in}{0.715862in}}{\pgfqpoint{4.495143in}{0.715862in}}%
\pgfpathlineto{\pgfqpoint{4.495143in}{0.715862in}}%
\pgfpathclose%
\pgfusepath{stroke}%
\end{pgfscope}%
\begin{pgfscope}%
\pgfpathrectangle{\pgfqpoint{0.847223in}{0.554012in}}{\pgfqpoint{6.200000in}{4.530000in}}%
\pgfusepath{clip}%
\pgfsetbuttcap%
\pgfsetroundjoin%
\pgfsetlinewidth{1.003750pt}%
\definecolor{currentstroke}{rgb}{1.000000,0.000000,0.000000}%
\pgfsetstrokecolor{currentstroke}%
\pgfsetdash{}{0pt}%
\pgfpathmoveto{\pgfqpoint{4.500476in}{0.715070in}}%
\pgfpathcurveto{\pgfqpoint{4.511526in}{0.715070in}}{\pgfqpoint{4.522125in}{0.719460in}}{\pgfqpoint{4.529939in}{0.727274in}}%
\pgfpathcurveto{\pgfqpoint{4.537752in}{0.735087in}}{\pgfqpoint{4.542142in}{0.745686in}}{\pgfqpoint{4.542142in}{0.756737in}}%
\pgfpathcurveto{\pgfqpoint{4.542142in}{0.767787in}}{\pgfqpoint{4.537752in}{0.778386in}}{\pgfqpoint{4.529939in}{0.786199in}}%
\pgfpathcurveto{\pgfqpoint{4.522125in}{0.794013in}}{\pgfqpoint{4.511526in}{0.798403in}}{\pgfqpoint{4.500476in}{0.798403in}}%
\pgfpathcurveto{\pgfqpoint{4.489426in}{0.798403in}}{\pgfqpoint{4.478827in}{0.794013in}}{\pgfqpoint{4.471013in}{0.786199in}}%
\pgfpathcurveto{\pgfqpoint{4.463199in}{0.778386in}}{\pgfqpoint{4.458809in}{0.767787in}}{\pgfqpoint{4.458809in}{0.756737in}}%
\pgfpathcurveto{\pgfqpoint{4.458809in}{0.745686in}}{\pgfqpoint{4.463199in}{0.735087in}}{\pgfqpoint{4.471013in}{0.727274in}}%
\pgfpathcurveto{\pgfqpoint{4.478827in}{0.719460in}}{\pgfqpoint{4.489426in}{0.715070in}}{\pgfqpoint{4.500476in}{0.715070in}}%
\pgfpathlineto{\pgfqpoint{4.500476in}{0.715070in}}%
\pgfpathclose%
\pgfusepath{stroke}%
\end{pgfscope}%
\begin{pgfscope}%
\pgfpathrectangle{\pgfqpoint{0.847223in}{0.554012in}}{\pgfqpoint{6.200000in}{4.530000in}}%
\pgfusepath{clip}%
\pgfsetbuttcap%
\pgfsetroundjoin%
\pgfsetlinewidth{1.003750pt}%
\definecolor{currentstroke}{rgb}{1.000000,0.000000,0.000000}%
\pgfsetstrokecolor{currentstroke}%
\pgfsetdash{}{0pt}%
\pgfpathmoveto{\pgfqpoint{4.505809in}{0.714279in}}%
\pgfpathcurveto{\pgfqpoint{4.516859in}{0.714279in}}{\pgfqpoint{4.527458in}{0.718670in}}{\pgfqpoint{4.535272in}{0.726483in}}%
\pgfpathcurveto{\pgfqpoint{4.543085in}{0.734297in}}{\pgfqpoint{4.547476in}{0.744896in}}{\pgfqpoint{4.547476in}{0.755946in}}%
\pgfpathcurveto{\pgfqpoint{4.547476in}{0.766996in}}{\pgfqpoint{4.543085in}{0.777595in}}{\pgfqpoint{4.535272in}{0.785409in}}%
\pgfpathcurveto{\pgfqpoint{4.527458in}{0.793222in}}{\pgfqpoint{4.516859in}{0.797613in}}{\pgfqpoint{4.505809in}{0.797613in}}%
\pgfpathcurveto{\pgfqpoint{4.494759in}{0.797613in}}{\pgfqpoint{4.484160in}{0.793222in}}{\pgfqpoint{4.476346in}{0.785409in}}%
\pgfpathcurveto{\pgfqpoint{4.468533in}{0.777595in}}{\pgfqpoint{4.464142in}{0.766996in}}{\pgfqpoint{4.464142in}{0.755946in}}%
\pgfpathcurveto{\pgfqpoint{4.464142in}{0.744896in}}{\pgfqpoint{4.468533in}{0.734297in}}{\pgfqpoint{4.476346in}{0.726483in}}%
\pgfpathcurveto{\pgfqpoint{4.484160in}{0.718670in}}{\pgfqpoint{4.494759in}{0.714279in}}{\pgfqpoint{4.505809in}{0.714279in}}%
\pgfpathlineto{\pgfqpoint{4.505809in}{0.714279in}}%
\pgfpathclose%
\pgfusepath{stroke}%
\end{pgfscope}%
\begin{pgfscope}%
\pgfpathrectangle{\pgfqpoint{0.847223in}{0.554012in}}{\pgfqpoint{6.200000in}{4.530000in}}%
\pgfusepath{clip}%
\pgfsetbuttcap%
\pgfsetroundjoin%
\pgfsetlinewidth{1.003750pt}%
\definecolor{currentstroke}{rgb}{1.000000,0.000000,0.000000}%
\pgfsetstrokecolor{currentstroke}%
\pgfsetdash{}{0pt}%
\pgfpathmoveto{\pgfqpoint{4.511142in}{0.713491in}}%
\pgfpathcurveto{\pgfqpoint{4.522192in}{0.713491in}}{\pgfqpoint{4.532791in}{0.717881in}}{\pgfqpoint{4.540605in}{0.725695in}}%
\pgfpathcurveto{\pgfqpoint{4.548419in}{0.733508in}}{\pgfqpoint{4.552809in}{0.744107in}}{\pgfqpoint{4.552809in}{0.755157in}}%
\pgfpathcurveto{\pgfqpoint{4.552809in}{0.766208in}}{\pgfqpoint{4.548419in}{0.776807in}}{\pgfqpoint{4.540605in}{0.784620in}}%
\pgfpathcurveto{\pgfqpoint{4.532791in}{0.792434in}}{\pgfqpoint{4.522192in}{0.796824in}}{\pgfqpoint{4.511142in}{0.796824in}}%
\pgfpathcurveto{\pgfqpoint{4.500092in}{0.796824in}}{\pgfqpoint{4.489493in}{0.792434in}}{\pgfqpoint{4.481679in}{0.784620in}}%
\pgfpathcurveto{\pgfqpoint{4.473866in}{0.776807in}}{\pgfqpoint{4.469476in}{0.766208in}}{\pgfqpoint{4.469476in}{0.755157in}}%
\pgfpathcurveto{\pgfqpoint{4.469476in}{0.744107in}}{\pgfqpoint{4.473866in}{0.733508in}}{\pgfqpoint{4.481679in}{0.725695in}}%
\pgfpathcurveto{\pgfqpoint{4.489493in}{0.717881in}}{\pgfqpoint{4.500092in}{0.713491in}}{\pgfqpoint{4.511142in}{0.713491in}}%
\pgfpathlineto{\pgfqpoint{4.511142in}{0.713491in}}%
\pgfpathclose%
\pgfusepath{stroke}%
\end{pgfscope}%
\begin{pgfscope}%
\pgfpathrectangle{\pgfqpoint{0.847223in}{0.554012in}}{\pgfqpoint{6.200000in}{4.530000in}}%
\pgfusepath{clip}%
\pgfsetbuttcap%
\pgfsetroundjoin%
\pgfsetlinewidth{1.003750pt}%
\definecolor{currentstroke}{rgb}{1.000000,0.000000,0.000000}%
\pgfsetstrokecolor{currentstroke}%
\pgfsetdash{}{0pt}%
\pgfpathmoveto{\pgfqpoint{4.516475in}{0.712704in}}%
\pgfpathcurveto{\pgfqpoint{4.527526in}{0.712704in}}{\pgfqpoint{4.538125in}{0.717095in}}{\pgfqpoint{4.545938in}{0.724908in}}%
\pgfpathcurveto{\pgfqpoint{4.553752in}{0.732722in}}{\pgfqpoint{4.558142in}{0.743321in}}{\pgfqpoint{4.558142in}{0.754371in}}%
\pgfpathcurveto{\pgfqpoint{4.558142in}{0.765421in}}{\pgfqpoint{4.553752in}{0.776020in}}{\pgfqpoint{4.545938in}{0.783834in}}%
\pgfpathcurveto{\pgfqpoint{4.538125in}{0.791647in}}{\pgfqpoint{4.527526in}{0.796038in}}{\pgfqpoint{4.516475in}{0.796038in}}%
\pgfpathcurveto{\pgfqpoint{4.505425in}{0.796038in}}{\pgfqpoint{4.494826in}{0.791647in}}{\pgfqpoint{4.487013in}{0.783834in}}%
\pgfpathcurveto{\pgfqpoint{4.479199in}{0.776020in}}{\pgfqpoint{4.474809in}{0.765421in}}{\pgfqpoint{4.474809in}{0.754371in}}%
\pgfpathcurveto{\pgfqpoint{4.474809in}{0.743321in}}{\pgfqpoint{4.479199in}{0.732722in}}{\pgfqpoint{4.487013in}{0.724908in}}%
\pgfpathcurveto{\pgfqpoint{4.494826in}{0.717095in}}{\pgfqpoint{4.505425in}{0.712704in}}{\pgfqpoint{4.516475in}{0.712704in}}%
\pgfpathlineto{\pgfqpoint{4.516475in}{0.712704in}}%
\pgfpathclose%
\pgfusepath{stroke}%
\end{pgfscope}%
\begin{pgfscope}%
\pgfpathrectangle{\pgfqpoint{0.847223in}{0.554012in}}{\pgfqpoint{6.200000in}{4.530000in}}%
\pgfusepath{clip}%
\pgfsetbuttcap%
\pgfsetroundjoin%
\pgfsetlinewidth{1.003750pt}%
\definecolor{currentstroke}{rgb}{1.000000,0.000000,0.000000}%
\pgfsetstrokecolor{currentstroke}%
\pgfsetdash{}{0pt}%
\pgfpathmoveto{\pgfqpoint{4.521809in}{0.711920in}}%
\pgfpathcurveto{\pgfqpoint{4.532859in}{0.711920in}}{\pgfqpoint{4.543458in}{0.716310in}}{\pgfqpoint{4.551271in}{0.724124in}}%
\pgfpathcurveto{\pgfqpoint{4.559085in}{0.731937in}}{\pgfqpoint{4.563475in}{0.742536in}}{\pgfqpoint{4.563475in}{0.753586in}}%
\pgfpathcurveto{\pgfqpoint{4.563475in}{0.764636in}}{\pgfqpoint{4.559085in}{0.775236in}}{\pgfqpoint{4.551271in}{0.783049in}}%
\pgfpathcurveto{\pgfqpoint{4.543458in}{0.790863in}}{\pgfqpoint{4.532859in}{0.795253in}}{\pgfqpoint{4.521809in}{0.795253in}}%
\pgfpathcurveto{\pgfqpoint{4.510759in}{0.795253in}}{\pgfqpoint{4.500159in}{0.790863in}}{\pgfqpoint{4.492346in}{0.783049in}}%
\pgfpathcurveto{\pgfqpoint{4.484532in}{0.775236in}}{\pgfqpoint{4.480142in}{0.764636in}}{\pgfqpoint{4.480142in}{0.753586in}}%
\pgfpathcurveto{\pgfqpoint{4.480142in}{0.742536in}}{\pgfqpoint{4.484532in}{0.731937in}}{\pgfqpoint{4.492346in}{0.724124in}}%
\pgfpathcurveto{\pgfqpoint{4.500159in}{0.716310in}}{\pgfqpoint{4.510759in}{0.711920in}}{\pgfqpoint{4.521809in}{0.711920in}}%
\pgfpathlineto{\pgfqpoint{4.521809in}{0.711920in}}%
\pgfpathclose%
\pgfusepath{stroke}%
\end{pgfscope}%
\begin{pgfscope}%
\pgfpathrectangle{\pgfqpoint{0.847223in}{0.554012in}}{\pgfqpoint{6.200000in}{4.530000in}}%
\pgfusepath{clip}%
\pgfsetbuttcap%
\pgfsetroundjoin%
\pgfsetlinewidth{1.003750pt}%
\definecolor{currentstroke}{rgb}{1.000000,0.000000,0.000000}%
\pgfsetstrokecolor{currentstroke}%
\pgfsetdash{}{0pt}%
\pgfpathmoveto{\pgfqpoint{4.527142in}{0.711137in}}%
\pgfpathcurveto{\pgfqpoint{4.538192in}{0.711137in}}{\pgfqpoint{4.548791in}{0.715527in}}{\pgfqpoint{4.556605in}{0.723341in}}%
\pgfpathcurveto{\pgfqpoint{4.564418in}{0.731155in}}{\pgfqpoint{4.568809in}{0.741754in}}{\pgfqpoint{4.568809in}{0.752804in}}%
\pgfpathcurveto{\pgfqpoint{4.568809in}{0.763854in}}{\pgfqpoint{4.564418in}{0.774453in}}{\pgfqpoint{4.556605in}{0.782267in}}%
\pgfpathcurveto{\pgfqpoint{4.548791in}{0.790080in}}{\pgfqpoint{4.538192in}{0.794470in}}{\pgfqpoint{4.527142in}{0.794470in}}%
\pgfpathcurveto{\pgfqpoint{4.516092in}{0.794470in}}{\pgfqpoint{4.505493in}{0.790080in}}{\pgfqpoint{4.497679in}{0.782267in}}%
\pgfpathcurveto{\pgfqpoint{4.489865in}{0.774453in}}{\pgfqpoint{4.485475in}{0.763854in}}{\pgfqpoint{4.485475in}{0.752804in}}%
\pgfpathcurveto{\pgfqpoint{4.485475in}{0.741754in}}{\pgfqpoint{4.489865in}{0.731155in}}{\pgfqpoint{4.497679in}{0.723341in}}%
\pgfpathcurveto{\pgfqpoint{4.505493in}{0.715527in}}{\pgfqpoint{4.516092in}{0.711137in}}{\pgfqpoint{4.527142in}{0.711137in}}%
\pgfpathlineto{\pgfqpoint{4.527142in}{0.711137in}}%
\pgfpathclose%
\pgfusepath{stroke}%
\end{pgfscope}%
\begin{pgfscope}%
\pgfpathrectangle{\pgfqpoint{0.847223in}{0.554012in}}{\pgfqpoint{6.200000in}{4.530000in}}%
\pgfusepath{clip}%
\pgfsetbuttcap%
\pgfsetroundjoin%
\pgfsetlinewidth{1.003750pt}%
\definecolor{currentstroke}{rgb}{1.000000,0.000000,0.000000}%
\pgfsetstrokecolor{currentstroke}%
\pgfsetdash{}{0pt}%
\pgfpathmoveto{\pgfqpoint{4.532475in}{0.710357in}}%
\pgfpathcurveto{\pgfqpoint{4.543525in}{0.710357in}}{\pgfqpoint{4.554124in}{0.714747in}}{\pgfqpoint{4.561938in}{0.722560in}}%
\pgfpathcurveto{\pgfqpoint{4.569751in}{0.730374in}}{\pgfqpoint{4.574142in}{0.740973in}}{\pgfqpoint{4.574142in}{0.752023in}}%
\pgfpathcurveto{\pgfqpoint{4.574142in}{0.763073in}}{\pgfqpoint{4.569751in}{0.773672in}}{\pgfqpoint{4.561938in}{0.781486in}}%
\pgfpathcurveto{\pgfqpoint{4.554124in}{0.789300in}}{\pgfqpoint{4.543525in}{0.793690in}}{\pgfqpoint{4.532475in}{0.793690in}}%
\pgfpathcurveto{\pgfqpoint{4.521425in}{0.793690in}}{\pgfqpoint{4.510826in}{0.789300in}}{\pgfqpoint{4.503012in}{0.781486in}}%
\pgfpathcurveto{\pgfqpoint{4.495199in}{0.773672in}}{\pgfqpoint{4.490808in}{0.763073in}}{\pgfqpoint{4.490808in}{0.752023in}}%
\pgfpathcurveto{\pgfqpoint{4.490808in}{0.740973in}}{\pgfqpoint{4.495199in}{0.730374in}}{\pgfqpoint{4.503012in}{0.722560in}}%
\pgfpathcurveto{\pgfqpoint{4.510826in}{0.714747in}}{\pgfqpoint{4.521425in}{0.710357in}}{\pgfqpoint{4.532475in}{0.710357in}}%
\pgfpathlineto{\pgfqpoint{4.532475in}{0.710357in}}%
\pgfpathclose%
\pgfusepath{stroke}%
\end{pgfscope}%
\begin{pgfscope}%
\pgfpathrectangle{\pgfqpoint{0.847223in}{0.554012in}}{\pgfqpoint{6.200000in}{4.530000in}}%
\pgfusepath{clip}%
\pgfsetbuttcap%
\pgfsetroundjoin%
\pgfsetlinewidth{1.003750pt}%
\definecolor{currentstroke}{rgb}{1.000000,0.000000,0.000000}%
\pgfsetstrokecolor{currentstroke}%
\pgfsetdash{}{0pt}%
\pgfpathmoveto{\pgfqpoint{4.537808in}{0.709578in}}%
\pgfpathcurveto{\pgfqpoint{4.548858in}{0.709578in}}{\pgfqpoint{4.559457in}{0.713968in}}{\pgfqpoint{4.567271in}{0.721782in}}%
\pgfpathcurveto{\pgfqpoint{4.575085in}{0.729595in}}{\pgfqpoint{4.579475in}{0.740194in}}{\pgfqpoint{4.579475in}{0.751245in}}%
\pgfpathcurveto{\pgfqpoint{4.579475in}{0.762295in}}{\pgfqpoint{4.575085in}{0.772894in}}{\pgfqpoint{4.567271in}{0.780707in}}%
\pgfpathcurveto{\pgfqpoint{4.559457in}{0.788521in}}{\pgfqpoint{4.548858in}{0.792911in}}{\pgfqpoint{4.537808in}{0.792911in}}%
\pgfpathcurveto{\pgfqpoint{4.526758in}{0.792911in}}{\pgfqpoint{4.516159in}{0.788521in}}{\pgfqpoint{4.508346in}{0.780707in}}%
\pgfpathcurveto{\pgfqpoint{4.500532in}{0.772894in}}{\pgfqpoint{4.496142in}{0.762295in}}{\pgfqpoint{4.496142in}{0.751245in}}%
\pgfpathcurveto{\pgfqpoint{4.496142in}{0.740194in}}{\pgfqpoint{4.500532in}{0.729595in}}{\pgfqpoint{4.508346in}{0.721782in}}%
\pgfpathcurveto{\pgfqpoint{4.516159in}{0.713968in}}{\pgfqpoint{4.526758in}{0.709578in}}{\pgfqpoint{4.537808in}{0.709578in}}%
\pgfpathlineto{\pgfqpoint{4.537808in}{0.709578in}}%
\pgfpathclose%
\pgfusepath{stroke}%
\end{pgfscope}%
\begin{pgfscope}%
\pgfpathrectangle{\pgfqpoint{0.847223in}{0.554012in}}{\pgfqpoint{6.200000in}{4.530000in}}%
\pgfusepath{clip}%
\pgfsetbuttcap%
\pgfsetroundjoin%
\pgfsetlinewidth{1.003750pt}%
\definecolor{currentstroke}{rgb}{1.000000,0.000000,0.000000}%
\pgfsetstrokecolor{currentstroke}%
\pgfsetdash{}{0pt}%
\pgfpathmoveto{\pgfqpoint{4.543142in}{0.708801in}}%
\pgfpathcurveto{\pgfqpoint{4.554192in}{0.708801in}}{\pgfqpoint{4.564791in}{0.713192in}}{\pgfqpoint{4.572604in}{0.721005in}}%
\pgfpathcurveto{\pgfqpoint{4.580418in}{0.728819in}}{\pgfqpoint{4.584808in}{0.739418in}}{\pgfqpoint{4.584808in}{0.750468in}}%
\pgfpathcurveto{\pgfqpoint{4.584808in}{0.761518in}}{\pgfqpoint{4.580418in}{0.772117in}}{\pgfqpoint{4.572604in}{0.779931in}}%
\pgfpathcurveto{\pgfqpoint{4.564791in}{0.787744in}}{\pgfqpoint{4.554192in}{0.792135in}}{\pgfqpoint{4.543142in}{0.792135in}}%
\pgfpathcurveto{\pgfqpoint{4.532091in}{0.792135in}}{\pgfqpoint{4.521492in}{0.787744in}}{\pgfqpoint{4.513679in}{0.779931in}}%
\pgfpathcurveto{\pgfqpoint{4.505865in}{0.772117in}}{\pgfqpoint{4.501475in}{0.761518in}}{\pgfqpoint{4.501475in}{0.750468in}}%
\pgfpathcurveto{\pgfqpoint{4.501475in}{0.739418in}}{\pgfqpoint{4.505865in}{0.728819in}}{\pgfqpoint{4.513679in}{0.721005in}}%
\pgfpathcurveto{\pgfqpoint{4.521492in}{0.713192in}}{\pgfqpoint{4.532091in}{0.708801in}}{\pgfqpoint{4.543142in}{0.708801in}}%
\pgfpathlineto{\pgfqpoint{4.543142in}{0.708801in}}%
\pgfpathclose%
\pgfusepath{stroke}%
\end{pgfscope}%
\begin{pgfscope}%
\pgfpathrectangle{\pgfqpoint{0.847223in}{0.554012in}}{\pgfqpoint{6.200000in}{4.530000in}}%
\pgfusepath{clip}%
\pgfsetbuttcap%
\pgfsetroundjoin%
\pgfsetlinewidth{1.003750pt}%
\definecolor{currentstroke}{rgb}{1.000000,0.000000,0.000000}%
\pgfsetstrokecolor{currentstroke}%
\pgfsetdash{}{0pt}%
\pgfpathmoveto{\pgfqpoint{4.548475in}{0.708027in}}%
\pgfpathcurveto{\pgfqpoint{4.559525in}{0.708027in}}{\pgfqpoint{4.570124in}{0.712417in}}{\pgfqpoint{4.577938in}{0.720231in}}%
\pgfpathcurveto{\pgfqpoint{4.585751in}{0.728044in}}{\pgfqpoint{4.590141in}{0.738643in}}{\pgfqpoint{4.590141in}{0.749693in}}%
\pgfpathcurveto{\pgfqpoint{4.590141in}{0.760743in}}{\pgfqpoint{4.585751in}{0.771342in}}{\pgfqpoint{4.577938in}{0.779156in}}%
\pgfpathcurveto{\pgfqpoint{4.570124in}{0.786970in}}{\pgfqpoint{4.559525in}{0.791360in}}{\pgfqpoint{4.548475in}{0.791360in}}%
\pgfpathcurveto{\pgfqpoint{4.537425in}{0.791360in}}{\pgfqpoint{4.526826in}{0.786970in}}{\pgfqpoint{4.519012in}{0.779156in}}%
\pgfpathcurveto{\pgfqpoint{4.511198in}{0.771342in}}{\pgfqpoint{4.506808in}{0.760743in}}{\pgfqpoint{4.506808in}{0.749693in}}%
\pgfpathcurveto{\pgfqpoint{4.506808in}{0.738643in}}{\pgfqpoint{4.511198in}{0.728044in}}{\pgfqpoint{4.519012in}{0.720231in}}%
\pgfpathcurveto{\pgfqpoint{4.526826in}{0.712417in}}{\pgfqpoint{4.537425in}{0.708027in}}{\pgfqpoint{4.548475in}{0.708027in}}%
\pgfpathlineto{\pgfqpoint{4.548475in}{0.708027in}}%
\pgfpathclose%
\pgfusepath{stroke}%
\end{pgfscope}%
\begin{pgfscope}%
\pgfpathrectangle{\pgfqpoint{0.847223in}{0.554012in}}{\pgfqpoint{6.200000in}{4.530000in}}%
\pgfusepath{clip}%
\pgfsetbuttcap%
\pgfsetroundjoin%
\pgfsetlinewidth{1.003750pt}%
\definecolor{currentstroke}{rgb}{1.000000,0.000000,0.000000}%
\pgfsetstrokecolor{currentstroke}%
\pgfsetdash{}{0pt}%
\pgfpathmoveto{\pgfqpoint{4.553808in}{0.707254in}}%
\pgfpathcurveto{\pgfqpoint{4.564858in}{0.707254in}}{\pgfqpoint{4.575457in}{0.711644in}}{\pgfqpoint{4.583271in}{0.719458in}}%
\pgfpathcurveto{\pgfqpoint{4.591084in}{0.727271in}}{\pgfqpoint{4.595475in}{0.737870in}}{\pgfqpoint{4.595475in}{0.748921in}}%
\pgfpathcurveto{\pgfqpoint{4.595475in}{0.759971in}}{\pgfqpoint{4.591084in}{0.770570in}}{\pgfqpoint{4.583271in}{0.778383in}}%
\pgfpathcurveto{\pgfqpoint{4.575457in}{0.786197in}}{\pgfqpoint{4.564858in}{0.790587in}}{\pgfqpoint{4.553808in}{0.790587in}}%
\pgfpathcurveto{\pgfqpoint{4.542758in}{0.790587in}}{\pgfqpoint{4.532159in}{0.786197in}}{\pgfqpoint{4.524345in}{0.778383in}}%
\pgfpathcurveto{\pgfqpoint{4.516532in}{0.770570in}}{\pgfqpoint{4.512141in}{0.759971in}}{\pgfqpoint{4.512141in}{0.748921in}}%
\pgfpathcurveto{\pgfqpoint{4.512141in}{0.737870in}}{\pgfqpoint{4.516532in}{0.727271in}}{\pgfqpoint{4.524345in}{0.719458in}}%
\pgfpathcurveto{\pgfqpoint{4.532159in}{0.711644in}}{\pgfqpoint{4.542758in}{0.707254in}}{\pgfqpoint{4.553808in}{0.707254in}}%
\pgfpathlineto{\pgfqpoint{4.553808in}{0.707254in}}%
\pgfpathclose%
\pgfusepath{stroke}%
\end{pgfscope}%
\begin{pgfscope}%
\pgfpathrectangle{\pgfqpoint{0.847223in}{0.554012in}}{\pgfqpoint{6.200000in}{4.530000in}}%
\pgfusepath{clip}%
\pgfsetbuttcap%
\pgfsetroundjoin%
\pgfsetlinewidth{1.003750pt}%
\definecolor{currentstroke}{rgb}{1.000000,0.000000,0.000000}%
\pgfsetstrokecolor{currentstroke}%
\pgfsetdash{}{0pt}%
\pgfpathmoveto{\pgfqpoint{4.559141in}{0.706483in}}%
\pgfpathcurveto{\pgfqpoint{4.570191in}{0.706483in}}{\pgfqpoint{4.580790in}{0.710873in}}{\pgfqpoint{4.588604in}{0.718687in}}%
\pgfpathcurveto{\pgfqpoint{4.596418in}{0.726501in}}{\pgfqpoint{4.600808in}{0.737100in}}{\pgfqpoint{4.600808in}{0.748150in}}%
\pgfpathcurveto{\pgfqpoint{4.600808in}{0.759200in}}{\pgfqpoint{4.596418in}{0.769799in}}{\pgfqpoint{4.588604in}{0.777613in}}%
\pgfpathcurveto{\pgfqpoint{4.580790in}{0.785426in}}{\pgfqpoint{4.570191in}{0.789817in}}{\pgfqpoint{4.559141in}{0.789817in}}%
\pgfpathcurveto{\pgfqpoint{4.548091in}{0.789817in}}{\pgfqpoint{4.537492in}{0.785426in}}{\pgfqpoint{4.529678in}{0.777613in}}%
\pgfpathcurveto{\pgfqpoint{4.521865in}{0.769799in}}{\pgfqpoint{4.517474in}{0.759200in}}{\pgfqpoint{4.517474in}{0.748150in}}%
\pgfpathcurveto{\pgfqpoint{4.517474in}{0.737100in}}{\pgfqpoint{4.521865in}{0.726501in}}{\pgfqpoint{4.529678in}{0.718687in}}%
\pgfpathcurveto{\pgfqpoint{4.537492in}{0.710873in}}{\pgfqpoint{4.548091in}{0.706483in}}{\pgfqpoint{4.559141in}{0.706483in}}%
\pgfpathlineto{\pgfqpoint{4.559141in}{0.706483in}}%
\pgfpathclose%
\pgfusepath{stroke}%
\end{pgfscope}%
\begin{pgfscope}%
\pgfpathrectangle{\pgfqpoint{0.847223in}{0.554012in}}{\pgfqpoint{6.200000in}{4.530000in}}%
\pgfusepath{clip}%
\pgfsetbuttcap%
\pgfsetroundjoin%
\pgfsetlinewidth{1.003750pt}%
\definecolor{currentstroke}{rgb}{1.000000,0.000000,0.000000}%
\pgfsetstrokecolor{currentstroke}%
\pgfsetdash{}{0pt}%
\pgfpathmoveto{\pgfqpoint{4.564474in}{0.705714in}}%
\pgfpathcurveto{\pgfqpoint{4.575524in}{0.705714in}}{\pgfqpoint{4.586124in}{0.710105in}}{\pgfqpoint{4.593937in}{0.717918in}}%
\pgfpathcurveto{\pgfqpoint{4.601751in}{0.725732in}}{\pgfqpoint{4.606141in}{0.736331in}}{\pgfqpoint{4.606141in}{0.747381in}}%
\pgfpathcurveto{\pgfqpoint{4.606141in}{0.758431in}}{\pgfqpoint{4.601751in}{0.769030in}}{\pgfqpoint{4.593937in}{0.776844in}}%
\pgfpathcurveto{\pgfqpoint{4.586124in}{0.784657in}}{\pgfqpoint{4.575524in}{0.789048in}}{\pgfqpoint{4.564474in}{0.789048in}}%
\pgfpathcurveto{\pgfqpoint{4.553424in}{0.789048in}}{\pgfqpoint{4.542825in}{0.784657in}}{\pgfqpoint{4.535012in}{0.776844in}}%
\pgfpathcurveto{\pgfqpoint{4.527198in}{0.769030in}}{\pgfqpoint{4.522808in}{0.758431in}}{\pgfqpoint{4.522808in}{0.747381in}}%
\pgfpathcurveto{\pgfqpoint{4.522808in}{0.736331in}}{\pgfqpoint{4.527198in}{0.725732in}}{\pgfqpoint{4.535012in}{0.717918in}}%
\pgfpathcurveto{\pgfqpoint{4.542825in}{0.710105in}}{\pgfqpoint{4.553424in}{0.705714in}}{\pgfqpoint{4.564474in}{0.705714in}}%
\pgfpathlineto{\pgfqpoint{4.564474in}{0.705714in}}%
\pgfpathclose%
\pgfusepath{stroke}%
\end{pgfscope}%
\begin{pgfscope}%
\pgfpathrectangle{\pgfqpoint{0.847223in}{0.554012in}}{\pgfqpoint{6.200000in}{4.530000in}}%
\pgfusepath{clip}%
\pgfsetbuttcap%
\pgfsetroundjoin%
\pgfsetlinewidth{1.003750pt}%
\definecolor{currentstroke}{rgb}{1.000000,0.000000,0.000000}%
\pgfsetstrokecolor{currentstroke}%
\pgfsetdash{}{0pt}%
\pgfpathmoveto{\pgfqpoint{4.569808in}{0.704947in}}%
\pgfpathcurveto{\pgfqpoint{4.580858in}{0.704947in}}{\pgfqpoint{4.591457in}{0.709338in}}{\pgfqpoint{4.599270in}{0.717151in}}%
\pgfpathcurveto{\pgfqpoint{4.607084in}{0.724965in}}{\pgfqpoint{4.611474in}{0.735564in}}{\pgfqpoint{4.611474in}{0.746614in}}%
\pgfpathcurveto{\pgfqpoint{4.611474in}{0.757664in}}{\pgfqpoint{4.607084in}{0.768263in}}{\pgfqpoint{4.599270in}{0.776077in}}%
\pgfpathcurveto{\pgfqpoint{4.591457in}{0.783891in}}{\pgfqpoint{4.580858in}{0.788281in}}{\pgfqpoint{4.569808in}{0.788281in}}%
\pgfpathcurveto{\pgfqpoint{4.558757in}{0.788281in}}{\pgfqpoint{4.548158in}{0.783891in}}{\pgfqpoint{4.540345in}{0.776077in}}%
\pgfpathcurveto{\pgfqpoint{4.532531in}{0.768263in}}{\pgfqpoint{4.528141in}{0.757664in}}{\pgfqpoint{4.528141in}{0.746614in}}%
\pgfpathcurveto{\pgfqpoint{4.528141in}{0.735564in}}{\pgfqpoint{4.532531in}{0.724965in}}{\pgfqpoint{4.540345in}{0.717151in}}%
\pgfpathcurveto{\pgfqpoint{4.548158in}{0.709338in}}{\pgfqpoint{4.558757in}{0.704947in}}{\pgfqpoint{4.569808in}{0.704947in}}%
\pgfpathlineto{\pgfqpoint{4.569808in}{0.704947in}}%
\pgfpathclose%
\pgfusepath{stroke}%
\end{pgfscope}%
\begin{pgfscope}%
\pgfpathrectangle{\pgfqpoint{0.847223in}{0.554012in}}{\pgfqpoint{6.200000in}{4.530000in}}%
\pgfusepath{clip}%
\pgfsetbuttcap%
\pgfsetroundjoin%
\pgfsetlinewidth{1.003750pt}%
\definecolor{currentstroke}{rgb}{1.000000,0.000000,0.000000}%
\pgfsetstrokecolor{currentstroke}%
\pgfsetdash{}{0pt}%
\pgfpathmoveto{\pgfqpoint{4.575141in}{0.704182in}}%
\pgfpathcurveto{\pgfqpoint{4.586191in}{0.704182in}}{\pgfqpoint{4.596790in}{0.708573in}}{\pgfqpoint{4.604604in}{0.716386in}}%
\pgfpathcurveto{\pgfqpoint{4.612417in}{0.724200in}}{\pgfqpoint{4.616807in}{0.734799in}}{\pgfqpoint{4.616807in}{0.745849in}}%
\pgfpathcurveto{\pgfqpoint{4.616807in}{0.756899in}}{\pgfqpoint{4.612417in}{0.767498in}}{\pgfqpoint{4.604604in}{0.775312in}}%
\pgfpathcurveto{\pgfqpoint{4.596790in}{0.783126in}}{\pgfqpoint{4.586191in}{0.787516in}}{\pgfqpoint{4.575141in}{0.787516in}}%
\pgfpathcurveto{\pgfqpoint{4.564091in}{0.787516in}}{\pgfqpoint{4.553492in}{0.783126in}}{\pgfqpoint{4.545678in}{0.775312in}}%
\pgfpathcurveto{\pgfqpoint{4.537864in}{0.767498in}}{\pgfqpoint{4.533474in}{0.756899in}}{\pgfqpoint{4.533474in}{0.745849in}}%
\pgfpathcurveto{\pgfqpoint{4.533474in}{0.734799in}}{\pgfqpoint{4.537864in}{0.724200in}}{\pgfqpoint{4.545678in}{0.716386in}}%
\pgfpathcurveto{\pgfqpoint{4.553492in}{0.708573in}}{\pgfqpoint{4.564091in}{0.704182in}}{\pgfqpoint{4.575141in}{0.704182in}}%
\pgfpathlineto{\pgfqpoint{4.575141in}{0.704182in}}%
\pgfpathclose%
\pgfusepath{stroke}%
\end{pgfscope}%
\begin{pgfscope}%
\pgfpathrectangle{\pgfqpoint{0.847223in}{0.554012in}}{\pgfqpoint{6.200000in}{4.530000in}}%
\pgfusepath{clip}%
\pgfsetbuttcap%
\pgfsetroundjoin%
\pgfsetlinewidth{1.003750pt}%
\definecolor{currentstroke}{rgb}{1.000000,0.000000,0.000000}%
\pgfsetstrokecolor{currentstroke}%
\pgfsetdash{}{0pt}%
\pgfpathmoveto{\pgfqpoint{4.580474in}{0.703419in}}%
\pgfpathcurveto{\pgfqpoint{4.591524in}{0.703419in}}{\pgfqpoint{4.602123in}{0.707810in}}{\pgfqpoint{4.609937in}{0.715623in}}%
\pgfpathcurveto{\pgfqpoint{4.617750in}{0.723437in}}{\pgfqpoint{4.622141in}{0.734036in}}{\pgfqpoint{4.622141in}{0.745086in}}%
\pgfpathcurveto{\pgfqpoint{4.622141in}{0.756136in}}{\pgfqpoint{4.617750in}{0.766735in}}{\pgfqpoint{4.609937in}{0.774549in}}%
\pgfpathcurveto{\pgfqpoint{4.602123in}{0.782362in}}{\pgfqpoint{4.591524in}{0.786753in}}{\pgfqpoint{4.580474in}{0.786753in}}%
\pgfpathcurveto{\pgfqpoint{4.569424in}{0.786753in}}{\pgfqpoint{4.558825in}{0.782362in}}{\pgfqpoint{4.551011in}{0.774549in}}%
\pgfpathcurveto{\pgfqpoint{4.543198in}{0.766735in}}{\pgfqpoint{4.538807in}{0.756136in}}{\pgfqpoint{4.538807in}{0.745086in}}%
\pgfpathcurveto{\pgfqpoint{4.538807in}{0.734036in}}{\pgfqpoint{4.543198in}{0.723437in}}{\pgfqpoint{4.551011in}{0.715623in}}%
\pgfpathcurveto{\pgfqpoint{4.558825in}{0.707810in}}{\pgfqpoint{4.569424in}{0.703419in}}{\pgfqpoint{4.580474in}{0.703419in}}%
\pgfpathlineto{\pgfqpoint{4.580474in}{0.703419in}}%
\pgfpathclose%
\pgfusepath{stroke}%
\end{pgfscope}%
\begin{pgfscope}%
\pgfpathrectangle{\pgfqpoint{0.847223in}{0.554012in}}{\pgfqpoint{6.200000in}{4.530000in}}%
\pgfusepath{clip}%
\pgfsetbuttcap%
\pgfsetroundjoin%
\pgfsetlinewidth{1.003750pt}%
\definecolor{currentstroke}{rgb}{1.000000,0.000000,0.000000}%
\pgfsetstrokecolor{currentstroke}%
\pgfsetdash{}{0pt}%
\pgfpathmoveto{\pgfqpoint{4.585807in}{0.702658in}}%
\pgfpathcurveto{\pgfqpoint{4.596857in}{0.702658in}}{\pgfqpoint{4.607456in}{0.707049in}}{\pgfqpoint{4.615270in}{0.714862in}}%
\pgfpathcurveto{\pgfqpoint{4.623084in}{0.722676in}}{\pgfqpoint{4.627474in}{0.733275in}}{\pgfqpoint{4.627474in}{0.744325in}}%
\pgfpathcurveto{\pgfqpoint{4.627474in}{0.755375in}}{\pgfqpoint{4.623084in}{0.765974in}}{\pgfqpoint{4.615270in}{0.773788in}}%
\pgfpathcurveto{\pgfqpoint{4.607456in}{0.781601in}}{\pgfqpoint{4.596857in}{0.785992in}}{\pgfqpoint{4.585807in}{0.785992in}}%
\pgfpathcurveto{\pgfqpoint{4.574757in}{0.785992in}}{\pgfqpoint{4.564158in}{0.781601in}}{\pgfqpoint{4.556344in}{0.773788in}}%
\pgfpathcurveto{\pgfqpoint{4.548531in}{0.765974in}}{\pgfqpoint{4.544141in}{0.755375in}}{\pgfqpoint{4.544141in}{0.744325in}}%
\pgfpathcurveto{\pgfqpoint{4.544141in}{0.733275in}}{\pgfqpoint{4.548531in}{0.722676in}}{\pgfqpoint{4.556344in}{0.714862in}}%
\pgfpathcurveto{\pgfqpoint{4.564158in}{0.707049in}}{\pgfqpoint{4.574757in}{0.702658in}}{\pgfqpoint{4.585807in}{0.702658in}}%
\pgfpathlineto{\pgfqpoint{4.585807in}{0.702658in}}%
\pgfpathclose%
\pgfusepath{stroke}%
\end{pgfscope}%
\begin{pgfscope}%
\pgfpathrectangle{\pgfqpoint{0.847223in}{0.554012in}}{\pgfqpoint{6.200000in}{4.530000in}}%
\pgfusepath{clip}%
\pgfsetbuttcap%
\pgfsetroundjoin%
\pgfsetlinewidth{1.003750pt}%
\definecolor{currentstroke}{rgb}{1.000000,0.000000,0.000000}%
\pgfsetstrokecolor{currentstroke}%
\pgfsetdash{}{0pt}%
\pgfpathmoveto{\pgfqpoint{4.591140in}{0.701899in}}%
\pgfpathcurveto{\pgfqpoint{4.602191in}{0.701899in}}{\pgfqpoint{4.612790in}{0.706289in}}{\pgfqpoint{4.620603in}{0.714103in}}%
\pgfpathcurveto{\pgfqpoint{4.628417in}{0.721917in}}{\pgfqpoint{4.632807in}{0.732516in}}{\pgfqpoint{4.632807in}{0.743566in}}%
\pgfpathcurveto{\pgfqpoint{4.632807in}{0.754616in}}{\pgfqpoint{4.628417in}{0.765215in}}{\pgfqpoint{4.620603in}{0.773028in}}%
\pgfpathcurveto{\pgfqpoint{4.612790in}{0.780842in}}{\pgfqpoint{4.602191in}{0.785232in}}{\pgfqpoint{4.591140in}{0.785232in}}%
\pgfpathcurveto{\pgfqpoint{4.580090in}{0.785232in}}{\pgfqpoint{4.569491in}{0.780842in}}{\pgfqpoint{4.561678in}{0.773028in}}%
\pgfpathcurveto{\pgfqpoint{4.553864in}{0.765215in}}{\pgfqpoint{4.549474in}{0.754616in}}{\pgfqpoint{4.549474in}{0.743566in}}%
\pgfpathcurveto{\pgfqpoint{4.549474in}{0.732516in}}{\pgfqpoint{4.553864in}{0.721917in}}{\pgfqpoint{4.561678in}{0.714103in}}%
\pgfpathcurveto{\pgfqpoint{4.569491in}{0.706289in}}{\pgfqpoint{4.580090in}{0.701899in}}{\pgfqpoint{4.591140in}{0.701899in}}%
\pgfpathlineto{\pgfqpoint{4.591140in}{0.701899in}}%
\pgfpathclose%
\pgfusepath{stroke}%
\end{pgfscope}%
\begin{pgfscope}%
\pgfpathrectangle{\pgfqpoint{0.847223in}{0.554012in}}{\pgfqpoint{6.200000in}{4.530000in}}%
\pgfusepath{clip}%
\pgfsetbuttcap%
\pgfsetroundjoin%
\pgfsetlinewidth{1.003750pt}%
\definecolor{currentstroke}{rgb}{1.000000,0.000000,0.000000}%
\pgfsetstrokecolor{currentstroke}%
\pgfsetdash{}{0pt}%
\pgfpathmoveto{\pgfqpoint{4.596474in}{0.701142in}}%
\pgfpathcurveto{\pgfqpoint{4.607524in}{0.701142in}}{\pgfqpoint{4.618123in}{0.705532in}}{\pgfqpoint{4.625936in}{0.713346in}}%
\pgfpathcurveto{\pgfqpoint{4.633750in}{0.721159in}}{\pgfqpoint{4.638140in}{0.731758in}}{\pgfqpoint{4.638140in}{0.742808in}}%
\pgfpathcurveto{\pgfqpoint{4.638140in}{0.753858in}}{\pgfqpoint{4.633750in}{0.764457in}}{\pgfqpoint{4.625936in}{0.772271in}}%
\pgfpathcurveto{\pgfqpoint{4.618123in}{0.780085in}}{\pgfqpoint{4.607524in}{0.784475in}}{\pgfqpoint{4.596474in}{0.784475in}}%
\pgfpathcurveto{\pgfqpoint{4.585424in}{0.784475in}}{\pgfqpoint{4.574825in}{0.780085in}}{\pgfqpoint{4.567011in}{0.772271in}}%
\pgfpathcurveto{\pgfqpoint{4.559197in}{0.764457in}}{\pgfqpoint{4.554807in}{0.753858in}}{\pgfqpoint{4.554807in}{0.742808in}}%
\pgfpathcurveto{\pgfqpoint{4.554807in}{0.731758in}}{\pgfqpoint{4.559197in}{0.721159in}}{\pgfqpoint{4.567011in}{0.713346in}}%
\pgfpathcurveto{\pgfqpoint{4.574825in}{0.705532in}}{\pgfqpoint{4.585424in}{0.701142in}}{\pgfqpoint{4.596474in}{0.701142in}}%
\pgfpathlineto{\pgfqpoint{4.596474in}{0.701142in}}%
\pgfpathclose%
\pgfusepath{stroke}%
\end{pgfscope}%
\begin{pgfscope}%
\pgfpathrectangle{\pgfqpoint{0.847223in}{0.554012in}}{\pgfqpoint{6.200000in}{4.530000in}}%
\pgfusepath{clip}%
\pgfsetbuttcap%
\pgfsetroundjoin%
\pgfsetlinewidth{1.003750pt}%
\definecolor{currentstroke}{rgb}{1.000000,0.000000,0.000000}%
\pgfsetstrokecolor{currentstroke}%
\pgfsetdash{}{0pt}%
\pgfpathmoveto{\pgfqpoint{4.601807in}{0.700386in}}%
\pgfpathcurveto{\pgfqpoint{4.612857in}{0.700386in}}{\pgfqpoint{4.623456in}{0.704776in}}{\pgfqpoint{4.631270in}{0.712590in}}%
\pgfpathcurveto{\pgfqpoint{4.639083in}{0.720404in}}{\pgfqpoint{4.643474in}{0.731003in}}{\pgfqpoint{4.643474in}{0.742053in}}%
\pgfpathcurveto{\pgfqpoint{4.643474in}{0.753103in}}{\pgfqpoint{4.639083in}{0.763702in}}{\pgfqpoint{4.631270in}{0.771516in}}%
\pgfpathcurveto{\pgfqpoint{4.623456in}{0.779329in}}{\pgfqpoint{4.612857in}{0.783720in}}{\pgfqpoint{4.601807in}{0.783720in}}%
\pgfpathcurveto{\pgfqpoint{4.590757in}{0.783720in}}{\pgfqpoint{4.580158in}{0.779329in}}{\pgfqpoint{4.572344in}{0.771516in}}%
\pgfpathcurveto{\pgfqpoint{4.564530in}{0.763702in}}{\pgfqpoint{4.560140in}{0.753103in}}{\pgfqpoint{4.560140in}{0.742053in}}%
\pgfpathcurveto{\pgfqpoint{4.560140in}{0.731003in}}{\pgfqpoint{4.564530in}{0.720404in}}{\pgfqpoint{4.572344in}{0.712590in}}%
\pgfpathcurveto{\pgfqpoint{4.580158in}{0.704776in}}{\pgfqpoint{4.590757in}{0.700386in}}{\pgfqpoint{4.601807in}{0.700386in}}%
\pgfpathlineto{\pgfqpoint{4.601807in}{0.700386in}}%
\pgfpathclose%
\pgfusepath{stroke}%
\end{pgfscope}%
\begin{pgfscope}%
\pgfpathrectangle{\pgfqpoint{0.847223in}{0.554012in}}{\pgfqpoint{6.200000in}{4.530000in}}%
\pgfusepath{clip}%
\pgfsetbuttcap%
\pgfsetroundjoin%
\pgfsetlinewidth{1.003750pt}%
\definecolor{currentstroke}{rgb}{1.000000,0.000000,0.000000}%
\pgfsetstrokecolor{currentstroke}%
\pgfsetdash{}{0pt}%
\pgfpathmoveto{\pgfqpoint{4.607140in}{0.699633in}}%
\pgfpathcurveto{\pgfqpoint{4.618190in}{0.699633in}}{\pgfqpoint{4.628789in}{0.704023in}}{\pgfqpoint{4.636603in}{0.711837in}}%
\pgfpathcurveto{\pgfqpoint{4.644416in}{0.719650in}}{\pgfqpoint{4.648807in}{0.730249in}}{\pgfqpoint{4.648807in}{0.741299in}}%
\pgfpathcurveto{\pgfqpoint{4.648807in}{0.752349in}}{\pgfqpoint{4.644416in}{0.762948in}}{\pgfqpoint{4.636603in}{0.770762in}}%
\pgfpathcurveto{\pgfqpoint{4.628789in}{0.778576in}}{\pgfqpoint{4.618190in}{0.782966in}}{\pgfqpoint{4.607140in}{0.782966in}}%
\pgfpathcurveto{\pgfqpoint{4.596090in}{0.782966in}}{\pgfqpoint{4.585491in}{0.778576in}}{\pgfqpoint{4.577677in}{0.770762in}}%
\pgfpathcurveto{\pgfqpoint{4.569864in}{0.762948in}}{\pgfqpoint{4.565473in}{0.752349in}}{\pgfqpoint{4.565473in}{0.741299in}}%
\pgfpathcurveto{\pgfqpoint{4.565473in}{0.730249in}}{\pgfqpoint{4.569864in}{0.719650in}}{\pgfqpoint{4.577677in}{0.711837in}}%
\pgfpathcurveto{\pgfqpoint{4.585491in}{0.704023in}}{\pgfqpoint{4.596090in}{0.699633in}}{\pgfqpoint{4.607140in}{0.699633in}}%
\pgfpathlineto{\pgfqpoint{4.607140in}{0.699633in}}%
\pgfpathclose%
\pgfusepath{stroke}%
\end{pgfscope}%
\begin{pgfscope}%
\pgfpathrectangle{\pgfqpoint{0.847223in}{0.554012in}}{\pgfqpoint{6.200000in}{4.530000in}}%
\pgfusepath{clip}%
\pgfsetbuttcap%
\pgfsetroundjoin%
\pgfsetlinewidth{1.003750pt}%
\definecolor{currentstroke}{rgb}{1.000000,0.000000,0.000000}%
\pgfsetstrokecolor{currentstroke}%
\pgfsetdash{}{0pt}%
\pgfpathmoveto{\pgfqpoint{4.612473in}{0.698881in}}%
\pgfpathcurveto{\pgfqpoint{4.623523in}{0.698881in}}{\pgfqpoint{4.634122in}{0.703271in}}{\pgfqpoint{4.641936in}{0.711085in}}%
\pgfpathcurveto{\pgfqpoint{4.649750in}{0.718898in}}{\pgfqpoint{4.654140in}{0.729497in}}{\pgfqpoint{4.654140in}{0.740548in}}%
\pgfpathcurveto{\pgfqpoint{4.654140in}{0.751598in}}{\pgfqpoint{4.649750in}{0.762197in}}{\pgfqpoint{4.641936in}{0.770010in}}%
\pgfpathcurveto{\pgfqpoint{4.634122in}{0.777824in}}{\pgfqpoint{4.623523in}{0.782214in}}{\pgfqpoint{4.612473in}{0.782214in}}%
\pgfpathcurveto{\pgfqpoint{4.601423in}{0.782214in}}{\pgfqpoint{4.590824in}{0.777824in}}{\pgfqpoint{4.583011in}{0.770010in}}%
\pgfpathcurveto{\pgfqpoint{4.575197in}{0.762197in}}{\pgfqpoint{4.570807in}{0.751598in}}{\pgfqpoint{4.570807in}{0.740548in}}%
\pgfpathcurveto{\pgfqpoint{4.570807in}{0.729497in}}{\pgfqpoint{4.575197in}{0.718898in}}{\pgfqpoint{4.583011in}{0.711085in}}%
\pgfpathcurveto{\pgfqpoint{4.590824in}{0.703271in}}{\pgfqpoint{4.601423in}{0.698881in}}{\pgfqpoint{4.612473in}{0.698881in}}%
\pgfpathlineto{\pgfqpoint{4.612473in}{0.698881in}}%
\pgfpathclose%
\pgfusepath{stroke}%
\end{pgfscope}%
\begin{pgfscope}%
\pgfpathrectangle{\pgfqpoint{0.847223in}{0.554012in}}{\pgfqpoint{6.200000in}{4.530000in}}%
\pgfusepath{clip}%
\pgfsetbuttcap%
\pgfsetroundjoin%
\pgfsetlinewidth{1.003750pt}%
\definecolor{currentstroke}{rgb}{1.000000,0.000000,0.000000}%
\pgfsetstrokecolor{currentstroke}%
\pgfsetdash{}{0pt}%
\pgfpathmoveto{\pgfqpoint{4.617807in}{0.698131in}}%
\pgfpathcurveto{\pgfqpoint{4.628857in}{0.698131in}}{\pgfqpoint{4.639456in}{0.702521in}}{\pgfqpoint{4.647269in}{0.710335in}}%
\pgfpathcurveto{\pgfqpoint{4.655083in}{0.718149in}}{\pgfqpoint{4.659473in}{0.728748in}}{\pgfqpoint{4.659473in}{0.739798in}}%
\pgfpathcurveto{\pgfqpoint{4.659473in}{0.750848in}}{\pgfqpoint{4.655083in}{0.761447in}}{\pgfqpoint{4.647269in}{0.769261in}}%
\pgfpathcurveto{\pgfqpoint{4.639456in}{0.777074in}}{\pgfqpoint{4.628857in}{0.781464in}}{\pgfqpoint{4.617807in}{0.781464in}}%
\pgfpathcurveto{\pgfqpoint{4.606756in}{0.781464in}}{\pgfqpoint{4.596157in}{0.777074in}}{\pgfqpoint{4.588344in}{0.769261in}}%
\pgfpathcurveto{\pgfqpoint{4.580530in}{0.761447in}}{\pgfqpoint{4.576140in}{0.750848in}}{\pgfqpoint{4.576140in}{0.739798in}}%
\pgfpathcurveto{\pgfqpoint{4.576140in}{0.728748in}}{\pgfqpoint{4.580530in}{0.718149in}}{\pgfqpoint{4.588344in}{0.710335in}}%
\pgfpathcurveto{\pgfqpoint{4.596157in}{0.702521in}}{\pgfqpoint{4.606756in}{0.698131in}}{\pgfqpoint{4.617807in}{0.698131in}}%
\pgfpathlineto{\pgfqpoint{4.617807in}{0.698131in}}%
\pgfpathclose%
\pgfusepath{stroke}%
\end{pgfscope}%
\begin{pgfscope}%
\pgfpathrectangle{\pgfqpoint{0.847223in}{0.554012in}}{\pgfqpoint{6.200000in}{4.530000in}}%
\pgfusepath{clip}%
\pgfsetbuttcap%
\pgfsetroundjoin%
\pgfsetlinewidth{1.003750pt}%
\definecolor{currentstroke}{rgb}{1.000000,0.000000,0.000000}%
\pgfsetstrokecolor{currentstroke}%
\pgfsetdash{}{0pt}%
\pgfpathmoveto{\pgfqpoint{4.623140in}{0.697383in}}%
\pgfpathcurveto{\pgfqpoint{4.634190in}{0.697383in}}{\pgfqpoint{4.644789in}{0.701773in}}{\pgfqpoint{4.652603in}{0.709587in}}%
\pgfpathcurveto{\pgfqpoint{4.660416in}{0.717401in}}{\pgfqpoint{4.664806in}{0.728000in}}{\pgfqpoint{4.664806in}{0.739050in}}%
\pgfpathcurveto{\pgfqpoint{4.664806in}{0.750100in}}{\pgfqpoint{4.660416in}{0.760699in}}{\pgfqpoint{4.652603in}{0.768513in}}%
\pgfpathcurveto{\pgfqpoint{4.644789in}{0.776326in}}{\pgfqpoint{4.634190in}{0.780716in}}{\pgfqpoint{4.623140in}{0.780716in}}%
\pgfpathcurveto{\pgfqpoint{4.612090in}{0.780716in}}{\pgfqpoint{4.601491in}{0.776326in}}{\pgfqpoint{4.593677in}{0.768513in}}%
\pgfpathcurveto{\pgfqpoint{4.585863in}{0.760699in}}{\pgfqpoint{4.581473in}{0.750100in}}{\pgfqpoint{4.581473in}{0.739050in}}%
\pgfpathcurveto{\pgfqpoint{4.581473in}{0.728000in}}{\pgfqpoint{4.585863in}{0.717401in}}{\pgfqpoint{4.593677in}{0.709587in}}%
\pgfpathcurveto{\pgfqpoint{4.601491in}{0.701773in}}{\pgfqpoint{4.612090in}{0.697383in}}{\pgfqpoint{4.623140in}{0.697383in}}%
\pgfpathlineto{\pgfqpoint{4.623140in}{0.697383in}}%
\pgfpathclose%
\pgfusepath{stroke}%
\end{pgfscope}%
\begin{pgfscope}%
\pgfpathrectangle{\pgfqpoint{0.847223in}{0.554012in}}{\pgfqpoint{6.200000in}{4.530000in}}%
\pgfusepath{clip}%
\pgfsetbuttcap%
\pgfsetroundjoin%
\pgfsetlinewidth{1.003750pt}%
\definecolor{currentstroke}{rgb}{1.000000,0.000000,0.000000}%
\pgfsetstrokecolor{currentstroke}%
\pgfsetdash{}{0pt}%
\pgfpathmoveto{\pgfqpoint{4.628473in}{0.696637in}}%
\pgfpathcurveto{\pgfqpoint{4.639523in}{0.696637in}}{\pgfqpoint{4.650122in}{0.701027in}}{\pgfqpoint{4.657936in}{0.708841in}}%
\pgfpathcurveto{\pgfqpoint{4.665749in}{0.716654in}}{\pgfqpoint{4.670140in}{0.727254in}}{\pgfqpoint{4.670140in}{0.738304in}}%
\pgfpathcurveto{\pgfqpoint{4.670140in}{0.749354in}}{\pgfqpoint{4.665749in}{0.759953in}}{\pgfqpoint{4.657936in}{0.767766in}}%
\pgfpathcurveto{\pgfqpoint{4.650122in}{0.775580in}}{\pgfqpoint{4.639523in}{0.779970in}}{\pgfqpoint{4.628473in}{0.779970in}}%
\pgfpathcurveto{\pgfqpoint{4.617423in}{0.779970in}}{\pgfqpoint{4.606824in}{0.775580in}}{\pgfqpoint{4.599010in}{0.767766in}}%
\pgfpathcurveto{\pgfqpoint{4.591197in}{0.759953in}}{\pgfqpoint{4.586806in}{0.749354in}}{\pgfqpoint{4.586806in}{0.738304in}}%
\pgfpathcurveto{\pgfqpoint{4.586806in}{0.727254in}}{\pgfqpoint{4.591197in}{0.716654in}}{\pgfqpoint{4.599010in}{0.708841in}}%
\pgfpathcurveto{\pgfqpoint{4.606824in}{0.701027in}}{\pgfqpoint{4.617423in}{0.696637in}}{\pgfqpoint{4.628473in}{0.696637in}}%
\pgfpathlineto{\pgfqpoint{4.628473in}{0.696637in}}%
\pgfpathclose%
\pgfusepath{stroke}%
\end{pgfscope}%
\begin{pgfscope}%
\pgfpathrectangle{\pgfqpoint{0.847223in}{0.554012in}}{\pgfqpoint{6.200000in}{4.530000in}}%
\pgfusepath{clip}%
\pgfsetbuttcap%
\pgfsetroundjoin%
\pgfsetlinewidth{1.003750pt}%
\definecolor{currentstroke}{rgb}{1.000000,0.000000,0.000000}%
\pgfsetstrokecolor{currentstroke}%
\pgfsetdash{}{0pt}%
\pgfpathmoveto{\pgfqpoint{4.633806in}{0.695893in}}%
\pgfpathcurveto{\pgfqpoint{4.644856in}{0.695893in}}{\pgfqpoint{4.655455in}{0.700283in}}{\pgfqpoint{4.663269in}{0.708097in}}%
\pgfpathcurveto{\pgfqpoint{4.671083in}{0.715910in}}{\pgfqpoint{4.675473in}{0.726509in}}{\pgfqpoint{4.675473in}{0.737559in}}%
\pgfpathcurveto{\pgfqpoint{4.675473in}{0.748609in}}{\pgfqpoint{4.671083in}{0.759209in}}{\pgfqpoint{4.663269in}{0.767022in}}%
\pgfpathcurveto{\pgfqpoint{4.655455in}{0.774836in}}{\pgfqpoint{4.644856in}{0.779226in}}{\pgfqpoint{4.633806in}{0.779226in}}%
\pgfpathcurveto{\pgfqpoint{4.622756in}{0.779226in}}{\pgfqpoint{4.612157in}{0.774836in}}{\pgfqpoint{4.604343in}{0.767022in}}%
\pgfpathcurveto{\pgfqpoint{4.596530in}{0.759209in}}{\pgfqpoint{4.592140in}{0.748609in}}{\pgfqpoint{4.592140in}{0.737559in}}%
\pgfpathcurveto{\pgfqpoint{4.592140in}{0.726509in}}{\pgfqpoint{4.596530in}{0.715910in}}{\pgfqpoint{4.604343in}{0.708097in}}%
\pgfpathcurveto{\pgfqpoint{4.612157in}{0.700283in}}{\pgfqpoint{4.622756in}{0.695893in}}{\pgfqpoint{4.633806in}{0.695893in}}%
\pgfpathlineto{\pgfqpoint{4.633806in}{0.695893in}}%
\pgfpathclose%
\pgfusepath{stroke}%
\end{pgfscope}%
\begin{pgfscope}%
\pgfpathrectangle{\pgfqpoint{0.847223in}{0.554012in}}{\pgfqpoint{6.200000in}{4.530000in}}%
\pgfusepath{clip}%
\pgfsetbuttcap%
\pgfsetroundjoin%
\pgfsetlinewidth{1.003750pt}%
\definecolor{currentstroke}{rgb}{1.000000,0.000000,0.000000}%
\pgfsetstrokecolor{currentstroke}%
\pgfsetdash{}{0pt}%
\pgfpathmoveto{\pgfqpoint{4.639139in}{0.695150in}}%
\pgfpathcurveto{\pgfqpoint{4.650190in}{0.695150in}}{\pgfqpoint{4.660789in}{0.699541in}}{\pgfqpoint{4.668602in}{0.707354in}}%
\pgfpathcurveto{\pgfqpoint{4.676416in}{0.715168in}}{\pgfqpoint{4.680806in}{0.725767in}}{\pgfqpoint{4.680806in}{0.736817in}}%
\pgfpathcurveto{\pgfqpoint{4.680806in}{0.747867in}}{\pgfqpoint{4.676416in}{0.758466in}}{\pgfqpoint{4.668602in}{0.766280in}}%
\pgfpathcurveto{\pgfqpoint{4.660789in}{0.774093in}}{\pgfqpoint{4.650190in}{0.778484in}}{\pgfqpoint{4.639139in}{0.778484in}}%
\pgfpathcurveto{\pgfqpoint{4.628089in}{0.778484in}}{\pgfqpoint{4.617490in}{0.774093in}}{\pgfqpoint{4.609677in}{0.766280in}}%
\pgfpathcurveto{\pgfqpoint{4.601863in}{0.758466in}}{\pgfqpoint{4.597473in}{0.747867in}}{\pgfqpoint{4.597473in}{0.736817in}}%
\pgfpathcurveto{\pgfqpoint{4.597473in}{0.725767in}}{\pgfqpoint{4.601863in}{0.715168in}}{\pgfqpoint{4.609677in}{0.707354in}}%
\pgfpathcurveto{\pgfqpoint{4.617490in}{0.699541in}}{\pgfqpoint{4.628089in}{0.695150in}}{\pgfqpoint{4.639139in}{0.695150in}}%
\pgfpathlineto{\pgfqpoint{4.639139in}{0.695150in}}%
\pgfpathclose%
\pgfusepath{stroke}%
\end{pgfscope}%
\begin{pgfscope}%
\pgfpathrectangle{\pgfqpoint{0.847223in}{0.554012in}}{\pgfqpoint{6.200000in}{4.530000in}}%
\pgfusepath{clip}%
\pgfsetbuttcap%
\pgfsetroundjoin%
\pgfsetlinewidth{1.003750pt}%
\definecolor{currentstroke}{rgb}{1.000000,0.000000,0.000000}%
\pgfsetstrokecolor{currentstroke}%
\pgfsetdash{}{0pt}%
\pgfpathmoveto{\pgfqpoint{4.644473in}{0.694410in}}%
\pgfpathcurveto{\pgfqpoint{4.655523in}{0.694410in}}{\pgfqpoint{4.666122in}{0.698800in}}{\pgfqpoint{4.673935in}{0.706614in}}%
\pgfpathcurveto{\pgfqpoint{4.681749in}{0.714427in}}{\pgfqpoint{4.686139in}{0.725026in}}{\pgfqpoint{4.686139in}{0.736076in}}%
\pgfpathcurveto{\pgfqpoint{4.686139in}{0.747126in}}{\pgfqpoint{4.681749in}{0.757725in}}{\pgfqpoint{4.673935in}{0.765539in}}%
\pgfpathcurveto{\pgfqpoint{4.666122in}{0.773353in}}{\pgfqpoint{4.655523in}{0.777743in}}{\pgfqpoint{4.644473in}{0.777743in}}%
\pgfpathcurveto{\pgfqpoint{4.633422in}{0.777743in}}{\pgfqpoint{4.622823in}{0.773353in}}{\pgfqpoint{4.615010in}{0.765539in}}%
\pgfpathcurveto{\pgfqpoint{4.607196in}{0.757725in}}{\pgfqpoint{4.602806in}{0.747126in}}{\pgfqpoint{4.602806in}{0.736076in}}%
\pgfpathcurveto{\pgfqpoint{4.602806in}{0.725026in}}{\pgfqpoint{4.607196in}{0.714427in}}{\pgfqpoint{4.615010in}{0.706614in}}%
\pgfpathcurveto{\pgfqpoint{4.622823in}{0.698800in}}{\pgfqpoint{4.633422in}{0.694410in}}{\pgfqpoint{4.644473in}{0.694410in}}%
\pgfpathlineto{\pgfqpoint{4.644473in}{0.694410in}}%
\pgfpathclose%
\pgfusepath{stroke}%
\end{pgfscope}%
\begin{pgfscope}%
\pgfpathrectangle{\pgfqpoint{0.847223in}{0.554012in}}{\pgfqpoint{6.200000in}{4.530000in}}%
\pgfusepath{clip}%
\pgfsetbuttcap%
\pgfsetroundjoin%
\pgfsetlinewidth{1.003750pt}%
\definecolor{currentstroke}{rgb}{1.000000,0.000000,0.000000}%
\pgfsetstrokecolor{currentstroke}%
\pgfsetdash{}{0pt}%
\pgfpathmoveto{\pgfqpoint{4.649806in}{0.693671in}}%
\pgfpathcurveto{\pgfqpoint{4.660856in}{0.693671in}}{\pgfqpoint{4.671455in}{0.698061in}}{\pgfqpoint{4.679269in}{0.705875in}}%
\pgfpathcurveto{\pgfqpoint{4.687082in}{0.713688in}}{\pgfqpoint{4.691472in}{0.724287in}}{\pgfqpoint{4.691472in}{0.735338in}}%
\pgfpathcurveto{\pgfqpoint{4.691472in}{0.746388in}}{\pgfqpoint{4.687082in}{0.756987in}}{\pgfqpoint{4.679269in}{0.764800in}}%
\pgfpathcurveto{\pgfqpoint{4.671455in}{0.772614in}}{\pgfqpoint{4.660856in}{0.777004in}}{\pgfqpoint{4.649806in}{0.777004in}}%
\pgfpathcurveto{\pgfqpoint{4.638756in}{0.777004in}}{\pgfqpoint{4.628157in}{0.772614in}}{\pgfqpoint{4.620343in}{0.764800in}}%
\pgfpathcurveto{\pgfqpoint{4.612529in}{0.756987in}}{\pgfqpoint{4.608139in}{0.746388in}}{\pgfqpoint{4.608139in}{0.735338in}}%
\pgfpathcurveto{\pgfqpoint{4.608139in}{0.724287in}}{\pgfqpoint{4.612529in}{0.713688in}}{\pgfqpoint{4.620343in}{0.705875in}}%
\pgfpathcurveto{\pgfqpoint{4.628157in}{0.698061in}}{\pgfqpoint{4.638756in}{0.693671in}}{\pgfqpoint{4.649806in}{0.693671in}}%
\pgfpathlineto{\pgfqpoint{4.649806in}{0.693671in}}%
\pgfpathclose%
\pgfusepath{stroke}%
\end{pgfscope}%
\begin{pgfscope}%
\pgfpathrectangle{\pgfqpoint{0.847223in}{0.554012in}}{\pgfqpoint{6.200000in}{4.530000in}}%
\pgfusepath{clip}%
\pgfsetbuttcap%
\pgfsetroundjoin%
\pgfsetlinewidth{1.003750pt}%
\definecolor{currentstroke}{rgb}{1.000000,0.000000,0.000000}%
\pgfsetstrokecolor{currentstroke}%
\pgfsetdash{}{0pt}%
\pgfpathmoveto{\pgfqpoint{4.655139in}{0.692934in}}%
\pgfpathcurveto{\pgfqpoint{4.666189in}{0.692934in}}{\pgfqpoint{4.676788in}{0.697324in}}{\pgfqpoint{4.684602in}{0.705138in}}%
\pgfpathcurveto{\pgfqpoint{4.692415in}{0.712951in}}{\pgfqpoint{4.696806in}{0.723550in}}{\pgfqpoint{4.696806in}{0.734601in}}%
\pgfpathcurveto{\pgfqpoint{4.696806in}{0.745651in}}{\pgfqpoint{4.692415in}{0.756250in}}{\pgfqpoint{4.684602in}{0.764063in}}%
\pgfpathcurveto{\pgfqpoint{4.676788in}{0.771877in}}{\pgfqpoint{4.666189in}{0.776267in}}{\pgfqpoint{4.655139in}{0.776267in}}%
\pgfpathcurveto{\pgfqpoint{4.644089in}{0.776267in}}{\pgfqpoint{4.633490in}{0.771877in}}{\pgfqpoint{4.625676in}{0.764063in}}%
\pgfpathcurveto{\pgfqpoint{4.617863in}{0.756250in}}{\pgfqpoint{4.613472in}{0.745651in}}{\pgfqpoint{4.613472in}{0.734601in}}%
\pgfpathcurveto{\pgfqpoint{4.613472in}{0.723550in}}{\pgfqpoint{4.617863in}{0.712951in}}{\pgfqpoint{4.625676in}{0.705138in}}%
\pgfpathcurveto{\pgfqpoint{4.633490in}{0.697324in}}{\pgfqpoint{4.644089in}{0.692934in}}{\pgfqpoint{4.655139in}{0.692934in}}%
\pgfpathlineto{\pgfqpoint{4.655139in}{0.692934in}}%
\pgfpathclose%
\pgfusepath{stroke}%
\end{pgfscope}%
\begin{pgfscope}%
\pgfpathrectangle{\pgfqpoint{0.847223in}{0.554012in}}{\pgfqpoint{6.200000in}{4.530000in}}%
\pgfusepath{clip}%
\pgfsetbuttcap%
\pgfsetroundjoin%
\pgfsetlinewidth{1.003750pt}%
\definecolor{currentstroke}{rgb}{1.000000,0.000000,0.000000}%
\pgfsetstrokecolor{currentstroke}%
\pgfsetdash{}{0pt}%
\pgfpathmoveto{\pgfqpoint{4.660472in}{0.692199in}}%
\pgfpathcurveto{\pgfqpoint{4.671522in}{0.692199in}}{\pgfqpoint{4.682121in}{0.696589in}}{\pgfqpoint{4.689935in}{0.704403in}}%
\pgfpathcurveto{\pgfqpoint{4.697749in}{0.712216in}}{\pgfqpoint{4.702139in}{0.722815in}}{\pgfqpoint{4.702139in}{0.733865in}}%
\pgfpathcurveto{\pgfqpoint{4.702139in}{0.744916in}}{\pgfqpoint{4.697749in}{0.755515in}}{\pgfqpoint{4.689935in}{0.763328in}}%
\pgfpathcurveto{\pgfqpoint{4.682121in}{0.771142in}}{\pgfqpoint{4.671522in}{0.775532in}}{\pgfqpoint{4.660472in}{0.775532in}}%
\pgfpathcurveto{\pgfqpoint{4.649422in}{0.775532in}}{\pgfqpoint{4.638823in}{0.771142in}}{\pgfqpoint{4.631009in}{0.763328in}}%
\pgfpathcurveto{\pgfqpoint{4.623196in}{0.755515in}}{\pgfqpoint{4.618806in}{0.744916in}}{\pgfqpoint{4.618806in}{0.733865in}}%
\pgfpathcurveto{\pgfqpoint{4.618806in}{0.722815in}}{\pgfqpoint{4.623196in}{0.712216in}}{\pgfqpoint{4.631009in}{0.704403in}}%
\pgfpathcurveto{\pgfqpoint{4.638823in}{0.696589in}}{\pgfqpoint{4.649422in}{0.692199in}}{\pgfqpoint{4.660472in}{0.692199in}}%
\pgfpathlineto{\pgfqpoint{4.660472in}{0.692199in}}%
\pgfpathclose%
\pgfusepath{stroke}%
\end{pgfscope}%
\begin{pgfscope}%
\pgfpathrectangle{\pgfqpoint{0.847223in}{0.554012in}}{\pgfqpoint{6.200000in}{4.530000in}}%
\pgfusepath{clip}%
\pgfsetbuttcap%
\pgfsetroundjoin%
\pgfsetlinewidth{1.003750pt}%
\definecolor{currentstroke}{rgb}{1.000000,0.000000,0.000000}%
\pgfsetstrokecolor{currentstroke}%
\pgfsetdash{}{0pt}%
\pgfpathmoveto{\pgfqpoint{4.665805in}{0.691465in}}%
\pgfpathcurveto{\pgfqpoint{4.676856in}{0.691465in}}{\pgfqpoint{4.687455in}{0.695856in}}{\pgfqpoint{4.695268in}{0.703669in}}%
\pgfpathcurveto{\pgfqpoint{4.703082in}{0.711483in}}{\pgfqpoint{4.707472in}{0.722082in}}{\pgfqpoint{4.707472in}{0.733132in}}%
\pgfpathcurveto{\pgfqpoint{4.707472in}{0.744182in}}{\pgfqpoint{4.703082in}{0.754781in}}{\pgfqpoint{4.695268in}{0.762595in}}%
\pgfpathcurveto{\pgfqpoint{4.687455in}{0.770408in}}{\pgfqpoint{4.676856in}{0.774799in}}{\pgfqpoint{4.665805in}{0.774799in}}%
\pgfpathcurveto{\pgfqpoint{4.654755in}{0.774799in}}{\pgfqpoint{4.644156in}{0.770408in}}{\pgfqpoint{4.636343in}{0.762595in}}%
\pgfpathcurveto{\pgfqpoint{4.628529in}{0.754781in}}{\pgfqpoint{4.624139in}{0.744182in}}{\pgfqpoint{4.624139in}{0.733132in}}%
\pgfpathcurveto{\pgfqpoint{4.624139in}{0.722082in}}{\pgfqpoint{4.628529in}{0.711483in}}{\pgfqpoint{4.636343in}{0.703669in}}%
\pgfpathcurveto{\pgfqpoint{4.644156in}{0.695856in}}{\pgfqpoint{4.654755in}{0.691465in}}{\pgfqpoint{4.665805in}{0.691465in}}%
\pgfpathlineto{\pgfqpoint{4.665805in}{0.691465in}}%
\pgfpathclose%
\pgfusepath{stroke}%
\end{pgfscope}%
\begin{pgfscope}%
\pgfpathrectangle{\pgfqpoint{0.847223in}{0.554012in}}{\pgfqpoint{6.200000in}{4.530000in}}%
\pgfusepath{clip}%
\pgfsetbuttcap%
\pgfsetroundjoin%
\pgfsetlinewidth{1.003750pt}%
\definecolor{currentstroke}{rgb}{1.000000,0.000000,0.000000}%
\pgfsetstrokecolor{currentstroke}%
\pgfsetdash{}{0pt}%
\pgfpathmoveto{\pgfqpoint{4.671139in}{0.690734in}}%
\pgfpathcurveto{\pgfqpoint{4.682189in}{0.690734in}}{\pgfqpoint{4.692788in}{0.695124in}}{\pgfqpoint{4.700601in}{0.702938in}}%
\pgfpathcurveto{\pgfqpoint{4.708415in}{0.710751in}}{\pgfqpoint{4.712805in}{0.721350in}}{\pgfqpoint{4.712805in}{0.732401in}}%
\pgfpathcurveto{\pgfqpoint{4.712805in}{0.743451in}}{\pgfqpoint{4.708415in}{0.754050in}}{\pgfqpoint{4.700601in}{0.761863in}}%
\pgfpathcurveto{\pgfqpoint{4.692788in}{0.769677in}}{\pgfqpoint{4.682189in}{0.774067in}}{\pgfqpoint{4.671139in}{0.774067in}}%
\pgfpathcurveto{\pgfqpoint{4.660089in}{0.774067in}}{\pgfqpoint{4.649490in}{0.769677in}}{\pgfqpoint{4.641676in}{0.761863in}}%
\pgfpathcurveto{\pgfqpoint{4.633862in}{0.754050in}}{\pgfqpoint{4.629472in}{0.743451in}}{\pgfqpoint{4.629472in}{0.732401in}}%
\pgfpathcurveto{\pgfqpoint{4.629472in}{0.721350in}}{\pgfqpoint{4.633862in}{0.710751in}}{\pgfqpoint{4.641676in}{0.702938in}}%
\pgfpathcurveto{\pgfqpoint{4.649490in}{0.695124in}}{\pgfqpoint{4.660089in}{0.690734in}}{\pgfqpoint{4.671139in}{0.690734in}}%
\pgfpathlineto{\pgfqpoint{4.671139in}{0.690734in}}%
\pgfpathclose%
\pgfusepath{stroke}%
\end{pgfscope}%
\begin{pgfscope}%
\pgfpathrectangle{\pgfqpoint{0.847223in}{0.554012in}}{\pgfqpoint{6.200000in}{4.530000in}}%
\pgfusepath{clip}%
\pgfsetbuttcap%
\pgfsetroundjoin%
\pgfsetlinewidth{1.003750pt}%
\definecolor{currentstroke}{rgb}{1.000000,0.000000,0.000000}%
\pgfsetstrokecolor{currentstroke}%
\pgfsetdash{}{0pt}%
\pgfpathmoveto{\pgfqpoint{4.676472in}{0.690004in}}%
\pgfpathcurveto{\pgfqpoint{4.687522in}{0.690004in}}{\pgfqpoint{4.698121in}{0.694394in}}{\pgfqpoint{4.705935in}{0.702208in}}%
\pgfpathcurveto{\pgfqpoint{4.713748in}{0.710022in}}{\pgfqpoint{4.718139in}{0.720621in}}{\pgfqpoint{4.718139in}{0.731671in}}%
\pgfpathcurveto{\pgfqpoint{4.718139in}{0.742721in}}{\pgfqpoint{4.713748in}{0.753320in}}{\pgfqpoint{4.705935in}{0.761134in}}%
\pgfpathcurveto{\pgfqpoint{4.698121in}{0.768947in}}{\pgfqpoint{4.687522in}{0.773337in}}{\pgfqpoint{4.676472in}{0.773337in}}%
\pgfpathcurveto{\pgfqpoint{4.665422in}{0.773337in}}{\pgfqpoint{4.654823in}{0.768947in}}{\pgfqpoint{4.647009in}{0.761134in}}%
\pgfpathcurveto{\pgfqpoint{4.639195in}{0.753320in}}{\pgfqpoint{4.634805in}{0.742721in}}{\pgfqpoint{4.634805in}{0.731671in}}%
\pgfpathcurveto{\pgfqpoint{4.634805in}{0.720621in}}{\pgfqpoint{4.639195in}{0.710022in}}{\pgfqpoint{4.647009in}{0.702208in}}%
\pgfpathcurveto{\pgfqpoint{4.654823in}{0.694394in}}{\pgfqpoint{4.665422in}{0.690004in}}{\pgfqpoint{4.676472in}{0.690004in}}%
\pgfpathlineto{\pgfqpoint{4.676472in}{0.690004in}}%
\pgfpathclose%
\pgfusepath{stroke}%
\end{pgfscope}%
\begin{pgfscope}%
\pgfpathrectangle{\pgfqpoint{0.847223in}{0.554012in}}{\pgfqpoint{6.200000in}{4.530000in}}%
\pgfusepath{clip}%
\pgfsetbuttcap%
\pgfsetroundjoin%
\pgfsetlinewidth{1.003750pt}%
\definecolor{currentstroke}{rgb}{1.000000,0.000000,0.000000}%
\pgfsetstrokecolor{currentstroke}%
\pgfsetdash{}{0pt}%
\pgfpathmoveto{\pgfqpoint{4.681805in}{0.689276in}}%
\pgfpathcurveto{\pgfqpoint{4.692855in}{0.689276in}}{\pgfqpoint{4.703454in}{0.693666in}}{\pgfqpoint{4.711268in}{0.701480in}}%
\pgfpathcurveto{\pgfqpoint{4.719082in}{0.709294in}}{\pgfqpoint{4.723472in}{0.719893in}}{\pgfqpoint{4.723472in}{0.730943in}}%
\pgfpathcurveto{\pgfqpoint{4.723472in}{0.741993in}}{\pgfqpoint{4.719082in}{0.752592in}}{\pgfqpoint{4.711268in}{0.760406in}}%
\pgfpathcurveto{\pgfqpoint{4.703454in}{0.768219in}}{\pgfqpoint{4.692855in}{0.772609in}}{\pgfqpoint{4.681805in}{0.772609in}}%
\pgfpathcurveto{\pgfqpoint{4.670755in}{0.772609in}}{\pgfqpoint{4.660156in}{0.768219in}}{\pgfqpoint{4.652342in}{0.760406in}}%
\pgfpathcurveto{\pgfqpoint{4.644529in}{0.752592in}}{\pgfqpoint{4.640138in}{0.741993in}}{\pgfqpoint{4.640138in}{0.730943in}}%
\pgfpathcurveto{\pgfqpoint{4.640138in}{0.719893in}}{\pgfqpoint{4.644529in}{0.709294in}}{\pgfqpoint{4.652342in}{0.701480in}}%
\pgfpathcurveto{\pgfqpoint{4.660156in}{0.693666in}}{\pgfqpoint{4.670755in}{0.689276in}}{\pgfqpoint{4.681805in}{0.689276in}}%
\pgfpathlineto{\pgfqpoint{4.681805in}{0.689276in}}%
\pgfpathclose%
\pgfusepath{stroke}%
\end{pgfscope}%
\begin{pgfscope}%
\pgfpathrectangle{\pgfqpoint{0.847223in}{0.554012in}}{\pgfqpoint{6.200000in}{4.530000in}}%
\pgfusepath{clip}%
\pgfsetbuttcap%
\pgfsetroundjoin%
\pgfsetlinewidth{1.003750pt}%
\definecolor{currentstroke}{rgb}{1.000000,0.000000,0.000000}%
\pgfsetstrokecolor{currentstroke}%
\pgfsetdash{}{0pt}%
\pgfpathmoveto{\pgfqpoint{4.687138in}{0.688550in}}%
\pgfpathcurveto{\pgfqpoint{4.698188in}{0.688550in}}{\pgfqpoint{4.708787in}{0.692940in}}{\pgfqpoint{4.716601in}{0.700754in}}%
\pgfpathcurveto{\pgfqpoint{4.724415in}{0.708567in}}{\pgfqpoint{4.728805in}{0.719166in}}{\pgfqpoint{4.728805in}{0.730217in}}%
\pgfpathcurveto{\pgfqpoint{4.728805in}{0.741267in}}{\pgfqpoint{4.724415in}{0.751866in}}{\pgfqpoint{4.716601in}{0.759679in}}%
\pgfpathcurveto{\pgfqpoint{4.708787in}{0.767493in}}{\pgfqpoint{4.698188in}{0.771883in}}{\pgfqpoint{4.687138in}{0.771883in}}%
\pgfpathcurveto{\pgfqpoint{4.676088in}{0.771883in}}{\pgfqpoint{4.665489in}{0.767493in}}{\pgfqpoint{4.657676in}{0.759679in}}%
\pgfpathcurveto{\pgfqpoint{4.649862in}{0.751866in}}{\pgfqpoint{4.645472in}{0.741267in}}{\pgfqpoint{4.645472in}{0.730217in}}%
\pgfpathcurveto{\pgfqpoint{4.645472in}{0.719166in}}{\pgfqpoint{4.649862in}{0.708567in}}{\pgfqpoint{4.657676in}{0.700754in}}%
\pgfpathcurveto{\pgfqpoint{4.665489in}{0.692940in}}{\pgfqpoint{4.676088in}{0.688550in}}{\pgfqpoint{4.687138in}{0.688550in}}%
\pgfpathlineto{\pgfqpoint{4.687138in}{0.688550in}}%
\pgfpathclose%
\pgfusepath{stroke}%
\end{pgfscope}%
\begin{pgfscope}%
\pgfpathrectangle{\pgfqpoint{0.847223in}{0.554012in}}{\pgfqpoint{6.200000in}{4.530000in}}%
\pgfusepath{clip}%
\pgfsetbuttcap%
\pgfsetroundjoin%
\pgfsetlinewidth{1.003750pt}%
\definecolor{currentstroke}{rgb}{1.000000,0.000000,0.000000}%
\pgfsetstrokecolor{currentstroke}%
\pgfsetdash{}{0pt}%
\pgfpathmoveto{\pgfqpoint{4.692472in}{0.687826in}}%
\pgfpathcurveto{\pgfqpoint{4.703522in}{0.687826in}}{\pgfqpoint{4.714121in}{0.692216in}}{\pgfqpoint{4.721934in}{0.700029in}}%
\pgfpathcurveto{\pgfqpoint{4.729748in}{0.707843in}}{\pgfqpoint{4.734138in}{0.718442in}}{\pgfqpoint{4.734138in}{0.729492in}}%
\pgfpathcurveto{\pgfqpoint{4.734138in}{0.740542in}}{\pgfqpoint{4.729748in}{0.751141in}}{\pgfqpoint{4.721934in}{0.758955in}}%
\pgfpathcurveto{\pgfqpoint{4.714121in}{0.766769in}}{\pgfqpoint{4.703522in}{0.771159in}}{\pgfqpoint{4.692472in}{0.771159in}}%
\pgfpathcurveto{\pgfqpoint{4.681421in}{0.771159in}}{\pgfqpoint{4.670822in}{0.766769in}}{\pgfqpoint{4.663009in}{0.758955in}}%
\pgfpathcurveto{\pgfqpoint{4.655195in}{0.751141in}}{\pgfqpoint{4.650805in}{0.740542in}}{\pgfqpoint{4.650805in}{0.729492in}}%
\pgfpathcurveto{\pgfqpoint{4.650805in}{0.718442in}}{\pgfqpoint{4.655195in}{0.707843in}}{\pgfqpoint{4.663009in}{0.700029in}}%
\pgfpathcurveto{\pgfqpoint{4.670822in}{0.692216in}}{\pgfqpoint{4.681421in}{0.687826in}}{\pgfqpoint{4.692472in}{0.687826in}}%
\pgfpathlineto{\pgfqpoint{4.692472in}{0.687826in}}%
\pgfpathclose%
\pgfusepath{stroke}%
\end{pgfscope}%
\begin{pgfscope}%
\pgfpathrectangle{\pgfqpoint{0.847223in}{0.554012in}}{\pgfqpoint{6.200000in}{4.530000in}}%
\pgfusepath{clip}%
\pgfsetbuttcap%
\pgfsetroundjoin%
\pgfsetlinewidth{1.003750pt}%
\definecolor{currentstroke}{rgb}{1.000000,0.000000,0.000000}%
\pgfsetstrokecolor{currentstroke}%
\pgfsetdash{}{0pt}%
\pgfpathmoveto{\pgfqpoint{4.697805in}{0.687103in}}%
\pgfpathcurveto{\pgfqpoint{4.708855in}{0.687103in}}{\pgfqpoint{4.719454in}{0.691493in}}{\pgfqpoint{4.727268in}{0.699307in}}%
\pgfpathcurveto{\pgfqpoint{4.735081in}{0.707120in}}{\pgfqpoint{4.739471in}{0.717719in}}{\pgfqpoint{4.739471in}{0.728770in}}%
\pgfpathcurveto{\pgfqpoint{4.739471in}{0.739820in}}{\pgfqpoint{4.735081in}{0.750419in}}{\pgfqpoint{4.727268in}{0.758232in}}%
\pgfpathcurveto{\pgfqpoint{4.719454in}{0.766046in}}{\pgfqpoint{4.708855in}{0.770436in}}{\pgfqpoint{4.697805in}{0.770436in}}%
\pgfpathcurveto{\pgfqpoint{4.686755in}{0.770436in}}{\pgfqpoint{4.676156in}{0.766046in}}{\pgfqpoint{4.668342in}{0.758232in}}%
\pgfpathcurveto{\pgfqpoint{4.660528in}{0.750419in}}{\pgfqpoint{4.656138in}{0.739820in}}{\pgfqpoint{4.656138in}{0.728770in}}%
\pgfpathcurveto{\pgfqpoint{4.656138in}{0.717719in}}{\pgfqpoint{4.660528in}{0.707120in}}{\pgfqpoint{4.668342in}{0.699307in}}%
\pgfpathcurveto{\pgfqpoint{4.676156in}{0.691493in}}{\pgfqpoint{4.686755in}{0.687103in}}{\pgfqpoint{4.697805in}{0.687103in}}%
\pgfpathlineto{\pgfqpoint{4.697805in}{0.687103in}}%
\pgfpathclose%
\pgfusepath{stroke}%
\end{pgfscope}%
\begin{pgfscope}%
\pgfpathrectangle{\pgfqpoint{0.847223in}{0.554012in}}{\pgfqpoint{6.200000in}{4.530000in}}%
\pgfusepath{clip}%
\pgfsetbuttcap%
\pgfsetroundjoin%
\pgfsetlinewidth{1.003750pt}%
\definecolor{currentstroke}{rgb}{1.000000,0.000000,0.000000}%
\pgfsetstrokecolor{currentstroke}%
\pgfsetdash{}{0pt}%
\pgfpathmoveto{\pgfqpoint{4.703138in}{0.686382in}}%
\pgfpathcurveto{\pgfqpoint{4.714188in}{0.686382in}}{\pgfqpoint{4.724787in}{0.690772in}}{\pgfqpoint{4.732601in}{0.698586in}}%
\pgfpathcurveto{\pgfqpoint{4.740414in}{0.706400in}}{\pgfqpoint{4.744805in}{0.716999in}}{\pgfqpoint{4.744805in}{0.728049in}}%
\pgfpathcurveto{\pgfqpoint{4.744805in}{0.739099in}}{\pgfqpoint{4.740414in}{0.749698in}}{\pgfqpoint{4.732601in}{0.757511in}}%
\pgfpathcurveto{\pgfqpoint{4.724787in}{0.765325in}}{\pgfqpoint{4.714188in}{0.769715in}}{\pgfqpoint{4.703138in}{0.769715in}}%
\pgfpathcurveto{\pgfqpoint{4.692088in}{0.769715in}}{\pgfqpoint{4.681489in}{0.765325in}}{\pgfqpoint{4.673675in}{0.757511in}}%
\pgfpathcurveto{\pgfqpoint{4.665862in}{0.749698in}}{\pgfqpoint{4.661471in}{0.739099in}}{\pgfqpoint{4.661471in}{0.728049in}}%
\pgfpathcurveto{\pgfqpoint{4.661471in}{0.716999in}}{\pgfqpoint{4.665862in}{0.706400in}}{\pgfqpoint{4.673675in}{0.698586in}}%
\pgfpathcurveto{\pgfqpoint{4.681489in}{0.690772in}}{\pgfqpoint{4.692088in}{0.686382in}}{\pgfqpoint{4.703138in}{0.686382in}}%
\pgfpathlineto{\pgfqpoint{4.703138in}{0.686382in}}%
\pgfpathclose%
\pgfusepath{stroke}%
\end{pgfscope}%
\begin{pgfscope}%
\pgfpathrectangle{\pgfqpoint{0.847223in}{0.554012in}}{\pgfqpoint{6.200000in}{4.530000in}}%
\pgfusepath{clip}%
\pgfsetbuttcap%
\pgfsetroundjoin%
\pgfsetlinewidth{1.003750pt}%
\definecolor{currentstroke}{rgb}{1.000000,0.000000,0.000000}%
\pgfsetstrokecolor{currentstroke}%
\pgfsetdash{}{0pt}%
\pgfpathmoveto{\pgfqpoint{4.708471in}{0.685663in}}%
\pgfpathcurveto{\pgfqpoint{4.719521in}{0.685663in}}{\pgfqpoint{4.730120in}{0.690053in}}{\pgfqpoint{4.737934in}{0.697867in}}%
\pgfpathcurveto{\pgfqpoint{4.745748in}{0.705680in}}{\pgfqpoint{4.750138in}{0.716279in}}{\pgfqpoint{4.750138in}{0.727330in}}%
\pgfpathcurveto{\pgfqpoint{4.750138in}{0.738380in}}{\pgfqpoint{4.745748in}{0.748979in}}{\pgfqpoint{4.737934in}{0.756792in}}%
\pgfpathcurveto{\pgfqpoint{4.730120in}{0.764606in}}{\pgfqpoint{4.719521in}{0.768996in}}{\pgfqpoint{4.708471in}{0.768996in}}%
\pgfpathcurveto{\pgfqpoint{4.697421in}{0.768996in}}{\pgfqpoint{4.686822in}{0.764606in}}{\pgfqpoint{4.679008in}{0.756792in}}%
\pgfpathcurveto{\pgfqpoint{4.671195in}{0.748979in}}{\pgfqpoint{4.666805in}{0.738380in}}{\pgfqpoint{4.666805in}{0.727330in}}%
\pgfpathcurveto{\pgfqpoint{4.666805in}{0.716279in}}{\pgfqpoint{4.671195in}{0.705680in}}{\pgfqpoint{4.679008in}{0.697867in}}%
\pgfpathcurveto{\pgfqpoint{4.686822in}{0.690053in}}{\pgfqpoint{4.697421in}{0.685663in}}{\pgfqpoint{4.708471in}{0.685663in}}%
\pgfpathlineto{\pgfqpoint{4.708471in}{0.685663in}}%
\pgfpathclose%
\pgfusepath{stroke}%
\end{pgfscope}%
\begin{pgfscope}%
\pgfpathrectangle{\pgfqpoint{0.847223in}{0.554012in}}{\pgfqpoint{6.200000in}{4.530000in}}%
\pgfusepath{clip}%
\pgfsetbuttcap%
\pgfsetroundjoin%
\pgfsetlinewidth{1.003750pt}%
\definecolor{currentstroke}{rgb}{1.000000,0.000000,0.000000}%
\pgfsetstrokecolor{currentstroke}%
\pgfsetdash{}{0pt}%
\pgfpathmoveto{\pgfqpoint{4.713804in}{0.684945in}}%
\pgfpathcurveto{\pgfqpoint{4.724855in}{0.684945in}}{\pgfqpoint{4.735454in}{0.689336in}}{\pgfqpoint{4.743267in}{0.697149in}}%
\pgfpathcurveto{\pgfqpoint{4.751081in}{0.704963in}}{\pgfqpoint{4.755471in}{0.715562in}}{\pgfqpoint{4.755471in}{0.726612in}}%
\pgfpathcurveto{\pgfqpoint{4.755471in}{0.737662in}}{\pgfqpoint{4.751081in}{0.748261in}}{\pgfqpoint{4.743267in}{0.756075in}}%
\pgfpathcurveto{\pgfqpoint{4.735454in}{0.763889in}}{\pgfqpoint{4.724855in}{0.768279in}}{\pgfqpoint{4.713804in}{0.768279in}}%
\pgfpathcurveto{\pgfqpoint{4.702754in}{0.768279in}}{\pgfqpoint{4.692155in}{0.763889in}}{\pgfqpoint{4.684342in}{0.756075in}}%
\pgfpathcurveto{\pgfqpoint{4.676528in}{0.748261in}}{\pgfqpoint{4.672138in}{0.737662in}}{\pgfqpoint{4.672138in}{0.726612in}}%
\pgfpathcurveto{\pgfqpoint{4.672138in}{0.715562in}}{\pgfqpoint{4.676528in}{0.704963in}}{\pgfqpoint{4.684342in}{0.697149in}}%
\pgfpathcurveto{\pgfqpoint{4.692155in}{0.689336in}}{\pgfqpoint{4.702754in}{0.684945in}}{\pgfqpoint{4.713804in}{0.684945in}}%
\pgfpathlineto{\pgfqpoint{4.713804in}{0.684945in}}%
\pgfpathclose%
\pgfusepath{stroke}%
\end{pgfscope}%
\begin{pgfscope}%
\pgfpathrectangle{\pgfqpoint{0.847223in}{0.554012in}}{\pgfqpoint{6.200000in}{4.530000in}}%
\pgfusepath{clip}%
\pgfsetbuttcap%
\pgfsetroundjoin%
\pgfsetlinewidth{1.003750pt}%
\definecolor{currentstroke}{rgb}{1.000000,0.000000,0.000000}%
\pgfsetstrokecolor{currentstroke}%
\pgfsetdash{}{0pt}%
\pgfpathmoveto{\pgfqpoint{4.719138in}{0.684230in}}%
\pgfpathcurveto{\pgfqpoint{4.730188in}{0.684230in}}{\pgfqpoint{4.740787in}{0.688620in}}{\pgfqpoint{4.748600in}{0.696434in}}%
\pgfpathcurveto{\pgfqpoint{4.756414in}{0.704247in}}{\pgfqpoint{4.760804in}{0.714846in}}{\pgfqpoint{4.760804in}{0.725897in}}%
\pgfpathcurveto{\pgfqpoint{4.760804in}{0.736947in}}{\pgfqpoint{4.756414in}{0.747546in}}{\pgfqpoint{4.748600in}{0.755359in}}%
\pgfpathcurveto{\pgfqpoint{4.740787in}{0.763173in}}{\pgfqpoint{4.730188in}{0.767563in}}{\pgfqpoint{4.719138in}{0.767563in}}%
\pgfpathcurveto{\pgfqpoint{4.708087in}{0.767563in}}{\pgfqpoint{4.697488in}{0.763173in}}{\pgfqpoint{4.689675in}{0.755359in}}%
\pgfpathcurveto{\pgfqpoint{4.681861in}{0.747546in}}{\pgfqpoint{4.677471in}{0.736947in}}{\pgfqpoint{4.677471in}{0.725897in}}%
\pgfpathcurveto{\pgfqpoint{4.677471in}{0.714846in}}{\pgfqpoint{4.681861in}{0.704247in}}{\pgfqpoint{4.689675in}{0.696434in}}%
\pgfpathcurveto{\pgfqpoint{4.697488in}{0.688620in}}{\pgfqpoint{4.708087in}{0.684230in}}{\pgfqpoint{4.719138in}{0.684230in}}%
\pgfpathlineto{\pgfqpoint{4.719138in}{0.684230in}}%
\pgfpathclose%
\pgfusepath{stroke}%
\end{pgfscope}%
\begin{pgfscope}%
\pgfpathrectangle{\pgfqpoint{0.847223in}{0.554012in}}{\pgfqpoint{6.200000in}{4.530000in}}%
\pgfusepath{clip}%
\pgfsetbuttcap%
\pgfsetroundjoin%
\pgfsetlinewidth{1.003750pt}%
\definecolor{currentstroke}{rgb}{1.000000,0.000000,0.000000}%
\pgfsetstrokecolor{currentstroke}%
\pgfsetdash{}{0pt}%
\pgfpathmoveto{\pgfqpoint{4.724471in}{0.683516in}}%
\pgfpathcurveto{\pgfqpoint{4.735521in}{0.683516in}}{\pgfqpoint{4.746120in}{0.687906in}}{\pgfqpoint{4.753934in}{0.695720in}}%
\pgfpathcurveto{\pgfqpoint{4.761747in}{0.703533in}}{\pgfqpoint{4.766138in}{0.714132in}}{\pgfqpoint{4.766138in}{0.725183in}}%
\pgfpathcurveto{\pgfqpoint{4.766138in}{0.736233in}}{\pgfqpoint{4.761747in}{0.746832in}}{\pgfqpoint{4.753934in}{0.754645in}}%
\pgfpathcurveto{\pgfqpoint{4.746120in}{0.762459in}}{\pgfqpoint{4.735521in}{0.766849in}}{\pgfqpoint{4.724471in}{0.766849in}}%
\pgfpathcurveto{\pgfqpoint{4.713421in}{0.766849in}}{\pgfqpoint{4.702822in}{0.762459in}}{\pgfqpoint{4.695008in}{0.754645in}}%
\pgfpathcurveto{\pgfqpoint{4.687194in}{0.746832in}}{\pgfqpoint{4.682804in}{0.736233in}}{\pgfqpoint{4.682804in}{0.725183in}}%
\pgfpathcurveto{\pgfqpoint{4.682804in}{0.714132in}}{\pgfqpoint{4.687194in}{0.703533in}}{\pgfqpoint{4.695008in}{0.695720in}}%
\pgfpathcurveto{\pgfqpoint{4.702822in}{0.687906in}}{\pgfqpoint{4.713421in}{0.683516in}}{\pgfqpoint{4.724471in}{0.683516in}}%
\pgfpathlineto{\pgfqpoint{4.724471in}{0.683516in}}%
\pgfpathclose%
\pgfusepath{stroke}%
\end{pgfscope}%
\begin{pgfscope}%
\pgfpathrectangle{\pgfqpoint{0.847223in}{0.554012in}}{\pgfqpoint{6.200000in}{4.530000in}}%
\pgfusepath{clip}%
\pgfsetbuttcap%
\pgfsetroundjoin%
\pgfsetlinewidth{1.003750pt}%
\definecolor{currentstroke}{rgb}{1.000000,0.000000,0.000000}%
\pgfsetstrokecolor{currentstroke}%
\pgfsetdash{}{0pt}%
\pgfpathmoveto{\pgfqpoint{4.729804in}{0.682804in}}%
\pgfpathcurveto{\pgfqpoint{4.740854in}{0.682804in}}{\pgfqpoint{4.751453in}{0.687194in}}{\pgfqpoint{4.759267in}{0.695008in}}%
\pgfpathcurveto{\pgfqpoint{4.767080in}{0.702821in}}{\pgfqpoint{4.771471in}{0.713420in}}{\pgfqpoint{4.771471in}{0.724470in}}%
\pgfpathcurveto{\pgfqpoint{4.771471in}{0.735521in}}{\pgfqpoint{4.767080in}{0.746120in}}{\pgfqpoint{4.759267in}{0.753933in}}%
\pgfpathcurveto{\pgfqpoint{4.751453in}{0.761747in}}{\pgfqpoint{4.740854in}{0.766137in}}{\pgfqpoint{4.729804in}{0.766137in}}%
\pgfpathcurveto{\pgfqpoint{4.718754in}{0.766137in}}{\pgfqpoint{4.708155in}{0.761747in}}{\pgfqpoint{4.700341in}{0.753933in}}%
\pgfpathcurveto{\pgfqpoint{4.692528in}{0.746120in}}{\pgfqpoint{4.688137in}{0.735521in}}{\pgfqpoint{4.688137in}{0.724470in}}%
\pgfpathcurveto{\pgfqpoint{4.688137in}{0.713420in}}{\pgfqpoint{4.692528in}{0.702821in}}{\pgfqpoint{4.700341in}{0.695008in}}%
\pgfpathcurveto{\pgfqpoint{4.708155in}{0.687194in}}{\pgfqpoint{4.718754in}{0.682804in}}{\pgfqpoint{4.729804in}{0.682804in}}%
\pgfpathlineto{\pgfqpoint{4.729804in}{0.682804in}}%
\pgfpathclose%
\pgfusepath{stroke}%
\end{pgfscope}%
\begin{pgfscope}%
\pgfpathrectangle{\pgfqpoint{0.847223in}{0.554012in}}{\pgfqpoint{6.200000in}{4.530000in}}%
\pgfusepath{clip}%
\pgfsetbuttcap%
\pgfsetroundjoin%
\pgfsetlinewidth{1.003750pt}%
\definecolor{currentstroke}{rgb}{1.000000,0.000000,0.000000}%
\pgfsetstrokecolor{currentstroke}%
\pgfsetdash{}{0pt}%
\pgfpathmoveto{\pgfqpoint{4.735137in}{0.682093in}}%
\pgfpathcurveto{\pgfqpoint{4.746187in}{0.682093in}}{\pgfqpoint{4.756786in}{0.686484in}}{\pgfqpoint{4.764600in}{0.694297in}}%
\pgfpathcurveto{\pgfqpoint{4.772414in}{0.702111in}}{\pgfqpoint{4.776804in}{0.712710in}}{\pgfqpoint{4.776804in}{0.723760in}}%
\pgfpathcurveto{\pgfqpoint{4.776804in}{0.734810in}}{\pgfqpoint{4.772414in}{0.745409in}}{\pgfqpoint{4.764600in}{0.753223in}}%
\pgfpathcurveto{\pgfqpoint{4.756786in}{0.761036in}}{\pgfqpoint{4.746187in}{0.765427in}}{\pgfqpoint{4.735137in}{0.765427in}}%
\pgfpathcurveto{\pgfqpoint{4.724087in}{0.765427in}}{\pgfqpoint{4.713488in}{0.761036in}}{\pgfqpoint{4.705674in}{0.753223in}}%
\pgfpathcurveto{\pgfqpoint{4.697861in}{0.745409in}}{\pgfqpoint{4.693471in}{0.734810in}}{\pgfqpoint{4.693471in}{0.723760in}}%
\pgfpathcurveto{\pgfqpoint{4.693471in}{0.712710in}}{\pgfqpoint{4.697861in}{0.702111in}}{\pgfqpoint{4.705674in}{0.694297in}}%
\pgfpathcurveto{\pgfqpoint{4.713488in}{0.686484in}}{\pgfqpoint{4.724087in}{0.682093in}}{\pgfqpoint{4.735137in}{0.682093in}}%
\pgfpathlineto{\pgfqpoint{4.735137in}{0.682093in}}%
\pgfpathclose%
\pgfusepath{stroke}%
\end{pgfscope}%
\begin{pgfscope}%
\pgfpathrectangle{\pgfqpoint{0.847223in}{0.554012in}}{\pgfqpoint{6.200000in}{4.530000in}}%
\pgfusepath{clip}%
\pgfsetbuttcap%
\pgfsetroundjoin%
\pgfsetlinewidth{1.003750pt}%
\definecolor{currentstroke}{rgb}{1.000000,0.000000,0.000000}%
\pgfsetstrokecolor{currentstroke}%
\pgfsetdash{}{0pt}%
\pgfpathmoveto{\pgfqpoint{4.740470in}{0.681385in}}%
\pgfpathcurveto{\pgfqpoint{4.751521in}{0.681385in}}{\pgfqpoint{4.762120in}{0.685775in}}{\pgfqpoint{4.769933in}{0.693588in}}%
\pgfpathcurveto{\pgfqpoint{4.777747in}{0.701402in}}{\pgfqpoint{4.782137in}{0.712001in}}{\pgfqpoint{4.782137in}{0.723051in}}%
\pgfpathcurveto{\pgfqpoint{4.782137in}{0.734101in}}{\pgfqpoint{4.777747in}{0.744700in}}{\pgfqpoint{4.769933in}{0.752514in}}%
\pgfpathcurveto{\pgfqpoint{4.762120in}{0.760328in}}{\pgfqpoint{4.751521in}{0.764718in}}{\pgfqpoint{4.740470in}{0.764718in}}%
\pgfpathcurveto{\pgfqpoint{4.729420in}{0.764718in}}{\pgfqpoint{4.718821in}{0.760328in}}{\pgfqpoint{4.711008in}{0.752514in}}%
\pgfpathcurveto{\pgfqpoint{4.703194in}{0.744700in}}{\pgfqpoint{4.698804in}{0.734101in}}{\pgfqpoint{4.698804in}{0.723051in}}%
\pgfpathcurveto{\pgfqpoint{4.698804in}{0.712001in}}{\pgfqpoint{4.703194in}{0.701402in}}{\pgfqpoint{4.711008in}{0.693588in}}%
\pgfpathcurveto{\pgfqpoint{4.718821in}{0.685775in}}{\pgfqpoint{4.729420in}{0.681385in}}{\pgfqpoint{4.740470in}{0.681385in}}%
\pgfpathlineto{\pgfqpoint{4.740470in}{0.681385in}}%
\pgfpathclose%
\pgfusepath{stroke}%
\end{pgfscope}%
\begin{pgfscope}%
\pgfpathrectangle{\pgfqpoint{0.847223in}{0.554012in}}{\pgfqpoint{6.200000in}{4.530000in}}%
\pgfusepath{clip}%
\pgfsetbuttcap%
\pgfsetroundjoin%
\pgfsetlinewidth{1.003750pt}%
\definecolor{currentstroke}{rgb}{1.000000,0.000000,0.000000}%
\pgfsetstrokecolor{currentstroke}%
\pgfsetdash{}{0pt}%
\pgfpathmoveto{\pgfqpoint{4.745804in}{0.680678in}}%
\pgfpathcurveto{\pgfqpoint{4.756854in}{0.680678in}}{\pgfqpoint{4.767453in}{0.685068in}}{\pgfqpoint{4.775266in}{0.692881in}}%
\pgfpathcurveto{\pgfqpoint{4.783080in}{0.700695in}}{\pgfqpoint{4.787470in}{0.711294in}}{\pgfqpoint{4.787470in}{0.722344in}}%
\pgfpathcurveto{\pgfqpoint{4.787470in}{0.733394in}}{\pgfqpoint{4.783080in}{0.743993in}}{\pgfqpoint{4.775266in}{0.751807in}}%
\pgfpathcurveto{\pgfqpoint{4.767453in}{0.759621in}}{\pgfqpoint{4.756854in}{0.764011in}}{\pgfqpoint{4.745804in}{0.764011in}}%
\pgfpathcurveto{\pgfqpoint{4.734754in}{0.764011in}}{\pgfqpoint{4.724155in}{0.759621in}}{\pgfqpoint{4.716341in}{0.751807in}}%
\pgfpathcurveto{\pgfqpoint{4.708527in}{0.743993in}}{\pgfqpoint{4.704137in}{0.733394in}}{\pgfqpoint{4.704137in}{0.722344in}}%
\pgfpathcurveto{\pgfqpoint{4.704137in}{0.711294in}}{\pgfqpoint{4.708527in}{0.700695in}}{\pgfqpoint{4.716341in}{0.692881in}}%
\pgfpathcurveto{\pgfqpoint{4.724155in}{0.685068in}}{\pgfqpoint{4.734754in}{0.680678in}}{\pgfqpoint{4.745804in}{0.680678in}}%
\pgfpathlineto{\pgfqpoint{4.745804in}{0.680678in}}%
\pgfpathclose%
\pgfusepath{stroke}%
\end{pgfscope}%
\begin{pgfscope}%
\pgfpathrectangle{\pgfqpoint{0.847223in}{0.554012in}}{\pgfqpoint{6.200000in}{4.530000in}}%
\pgfusepath{clip}%
\pgfsetbuttcap%
\pgfsetroundjoin%
\pgfsetlinewidth{1.003750pt}%
\definecolor{currentstroke}{rgb}{1.000000,0.000000,0.000000}%
\pgfsetstrokecolor{currentstroke}%
\pgfsetdash{}{0pt}%
\pgfpathmoveto{\pgfqpoint{4.751137in}{0.679972in}}%
\pgfpathcurveto{\pgfqpoint{4.762187in}{0.679972in}}{\pgfqpoint{4.772786in}{0.684362in}}{\pgfqpoint{4.780600in}{0.692176in}}%
\pgfpathcurveto{\pgfqpoint{4.788413in}{0.699990in}}{\pgfqpoint{4.792804in}{0.710589in}}{\pgfqpoint{4.792804in}{0.721639in}}%
\pgfpathcurveto{\pgfqpoint{4.792804in}{0.732689in}}{\pgfqpoint{4.788413in}{0.743288in}}{\pgfqpoint{4.780600in}{0.751102in}}%
\pgfpathcurveto{\pgfqpoint{4.772786in}{0.758915in}}{\pgfqpoint{4.762187in}{0.763306in}}{\pgfqpoint{4.751137in}{0.763306in}}%
\pgfpathcurveto{\pgfqpoint{4.740087in}{0.763306in}}{\pgfqpoint{4.729488in}{0.758915in}}{\pgfqpoint{4.721674in}{0.751102in}}%
\pgfpathcurveto{\pgfqpoint{4.713861in}{0.743288in}}{\pgfqpoint{4.709470in}{0.732689in}}{\pgfqpoint{4.709470in}{0.721639in}}%
\pgfpathcurveto{\pgfqpoint{4.709470in}{0.710589in}}{\pgfqpoint{4.713861in}{0.699990in}}{\pgfqpoint{4.721674in}{0.692176in}}%
\pgfpathcurveto{\pgfqpoint{4.729488in}{0.684362in}}{\pgfqpoint{4.740087in}{0.679972in}}{\pgfqpoint{4.751137in}{0.679972in}}%
\pgfpathlineto{\pgfqpoint{4.751137in}{0.679972in}}%
\pgfpathclose%
\pgfusepath{stroke}%
\end{pgfscope}%
\begin{pgfscope}%
\pgfpathrectangle{\pgfqpoint{0.847223in}{0.554012in}}{\pgfqpoint{6.200000in}{4.530000in}}%
\pgfusepath{clip}%
\pgfsetbuttcap%
\pgfsetroundjoin%
\pgfsetlinewidth{1.003750pt}%
\definecolor{currentstroke}{rgb}{1.000000,0.000000,0.000000}%
\pgfsetstrokecolor{currentstroke}%
\pgfsetdash{}{0pt}%
\pgfpathmoveto{\pgfqpoint{4.756470in}{0.679269in}}%
\pgfpathcurveto{\pgfqpoint{4.767520in}{0.679269in}}{\pgfqpoint{4.778119in}{0.683659in}}{\pgfqpoint{4.785933in}{0.691472in}}%
\pgfpathcurveto{\pgfqpoint{4.793747in}{0.699286in}}{\pgfqpoint{4.798137in}{0.709885in}}{\pgfqpoint{4.798137in}{0.720935in}}%
\pgfpathcurveto{\pgfqpoint{4.798137in}{0.731985in}}{\pgfqpoint{4.793747in}{0.742584in}}{\pgfqpoint{4.785933in}{0.750398in}}%
\pgfpathcurveto{\pgfqpoint{4.778119in}{0.758212in}}{\pgfqpoint{4.767520in}{0.762602in}}{\pgfqpoint{4.756470in}{0.762602in}}%
\pgfpathcurveto{\pgfqpoint{4.745420in}{0.762602in}}{\pgfqpoint{4.734821in}{0.758212in}}{\pgfqpoint{4.727007in}{0.750398in}}%
\pgfpathcurveto{\pgfqpoint{4.719194in}{0.742584in}}{\pgfqpoint{4.714803in}{0.731985in}}{\pgfqpoint{4.714803in}{0.720935in}}%
\pgfpathcurveto{\pgfqpoint{4.714803in}{0.709885in}}{\pgfqpoint{4.719194in}{0.699286in}}{\pgfqpoint{4.727007in}{0.691472in}}%
\pgfpathcurveto{\pgfqpoint{4.734821in}{0.683659in}}{\pgfqpoint{4.745420in}{0.679269in}}{\pgfqpoint{4.756470in}{0.679269in}}%
\pgfpathlineto{\pgfqpoint{4.756470in}{0.679269in}}%
\pgfpathclose%
\pgfusepath{stroke}%
\end{pgfscope}%
\begin{pgfscope}%
\pgfpathrectangle{\pgfqpoint{0.847223in}{0.554012in}}{\pgfqpoint{6.200000in}{4.530000in}}%
\pgfusepath{clip}%
\pgfsetbuttcap%
\pgfsetroundjoin%
\pgfsetlinewidth{1.003750pt}%
\definecolor{currentstroke}{rgb}{1.000000,0.000000,0.000000}%
\pgfsetstrokecolor{currentstroke}%
\pgfsetdash{}{0pt}%
\pgfpathmoveto{\pgfqpoint{4.761803in}{0.678567in}}%
\pgfpathcurveto{\pgfqpoint{4.772853in}{0.678567in}}{\pgfqpoint{4.783453in}{0.682957in}}{\pgfqpoint{4.791266in}{0.690771in}}%
\pgfpathcurveto{\pgfqpoint{4.799080in}{0.698584in}}{\pgfqpoint{4.803470in}{0.709183in}}{\pgfqpoint{4.803470in}{0.720233in}}%
\pgfpathcurveto{\pgfqpoint{4.803470in}{0.731283in}}{\pgfqpoint{4.799080in}{0.741882in}}{\pgfqpoint{4.791266in}{0.749696in}}%
\pgfpathcurveto{\pgfqpoint{4.783453in}{0.757510in}}{\pgfqpoint{4.772853in}{0.761900in}}{\pgfqpoint{4.761803in}{0.761900in}}%
\pgfpathcurveto{\pgfqpoint{4.750753in}{0.761900in}}{\pgfqpoint{4.740154in}{0.757510in}}{\pgfqpoint{4.732341in}{0.749696in}}%
\pgfpathcurveto{\pgfqpoint{4.724527in}{0.741882in}}{\pgfqpoint{4.720137in}{0.731283in}}{\pgfqpoint{4.720137in}{0.720233in}}%
\pgfpathcurveto{\pgfqpoint{4.720137in}{0.709183in}}{\pgfqpoint{4.724527in}{0.698584in}}{\pgfqpoint{4.732341in}{0.690771in}}%
\pgfpathcurveto{\pgfqpoint{4.740154in}{0.682957in}}{\pgfqpoint{4.750753in}{0.678567in}}{\pgfqpoint{4.761803in}{0.678567in}}%
\pgfpathlineto{\pgfqpoint{4.761803in}{0.678567in}}%
\pgfpathclose%
\pgfusepath{stroke}%
\end{pgfscope}%
\begin{pgfscope}%
\pgfpathrectangle{\pgfqpoint{0.847223in}{0.554012in}}{\pgfqpoint{6.200000in}{4.530000in}}%
\pgfusepath{clip}%
\pgfsetbuttcap%
\pgfsetroundjoin%
\pgfsetlinewidth{1.003750pt}%
\definecolor{currentstroke}{rgb}{1.000000,0.000000,0.000000}%
\pgfsetstrokecolor{currentstroke}%
\pgfsetdash{}{0pt}%
\pgfpathmoveto{\pgfqpoint{4.767137in}{0.677866in}}%
\pgfpathcurveto{\pgfqpoint{4.778187in}{0.677866in}}{\pgfqpoint{4.788786in}{0.682257in}}{\pgfqpoint{4.796599in}{0.690070in}}%
\pgfpathcurveto{\pgfqpoint{4.804413in}{0.697884in}}{\pgfqpoint{4.808803in}{0.708483in}}{\pgfqpoint{4.808803in}{0.719533in}}%
\pgfpathcurveto{\pgfqpoint{4.808803in}{0.730583in}}{\pgfqpoint{4.804413in}{0.741182in}}{\pgfqpoint{4.796599in}{0.748996in}}%
\pgfpathcurveto{\pgfqpoint{4.788786in}{0.756809in}}{\pgfqpoint{4.778187in}{0.761200in}}{\pgfqpoint{4.767137in}{0.761200in}}%
\pgfpathcurveto{\pgfqpoint{4.756086in}{0.761200in}}{\pgfqpoint{4.745487in}{0.756809in}}{\pgfqpoint{4.737674in}{0.748996in}}%
\pgfpathcurveto{\pgfqpoint{4.729860in}{0.741182in}}{\pgfqpoint{4.725470in}{0.730583in}}{\pgfqpoint{4.725470in}{0.719533in}}%
\pgfpathcurveto{\pgfqpoint{4.725470in}{0.708483in}}{\pgfqpoint{4.729860in}{0.697884in}}{\pgfqpoint{4.737674in}{0.690070in}}%
\pgfpathcurveto{\pgfqpoint{4.745487in}{0.682257in}}{\pgfqpoint{4.756086in}{0.677866in}}{\pgfqpoint{4.767137in}{0.677866in}}%
\pgfpathlineto{\pgfqpoint{4.767137in}{0.677866in}}%
\pgfpathclose%
\pgfusepath{stroke}%
\end{pgfscope}%
\begin{pgfscope}%
\pgfpathrectangle{\pgfqpoint{0.847223in}{0.554012in}}{\pgfqpoint{6.200000in}{4.530000in}}%
\pgfusepath{clip}%
\pgfsetbuttcap%
\pgfsetroundjoin%
\pgfsetlinewidth{1.003750pt}%
\definecolor{currentstroke}{rgb}{1.000000,0.000000,0.000000}%
\pgfsetstrokecolor{currentstroke}%
\pgfsetdash{}{0pt}%
\pgfpathmoveto{\pgfqpoint{4.772470in}{0.677168in}}%
\pgfpathcurveto{\pgfqpoint{4.783520in}{0.677168in}}{\pgfqpoint{4.794119in}{0.681558in}}{\pgfqpoint{4.801933in}{0.689372in}}%
\pgfpathcurveto{\pgfqpoint{4.809746in}{0.697185in}}{\pgfqpoint{4.814136in}{0.707784in}}{\pgfqpoint{4.814136in}{0.718834in}}%
\pgfpathcurveto{\pgfqpoint{4.814136in}{0.729885in}}{\pgfqpoint{4.809746in}{0.740484in}}{\pgfqpoint{4.801933in}{0.748297in}}%
\pgfpathcurveto{\pgfqpoint{4.794119in}{0.756111in}}{\pgfqpoint{4.783520in}{0.760501in}}{\pgfqpoint{4.772470in}{0.760501in}}%
\pgfpathcurveto{\pgfqpoint{4.761420in}{0.760501in}}{\pgfqpoint{4.750821in}{0.756111in}}{\pgfqpoint{4.743007in}{0.748297in}}%
\pgfpathcurveto{\pgfqpoint{4.735193in}{0.740484in}}{\pgfqpoint{4.730803in}{0.729885in}}{\pgfqpoint{4.730803in}{0.718834in}}%
\pgfpathcurveto{\pgfqpoint{4.730803in}{0.707784in}}{\pgfqpoint{4.735193in}{0.697185in}}{\pgfqpoint{4.743007in}{0.689372in}}%
\pgfpathcurveto{\pgfqpoint{4.750821in}{0.681558in}}{\pgfqpoint{4.761420in}{0.677168in}}{\pgfqpoint{4.772470in}{0.677168in}}%
\pgfpathlineto{\pgfqpoint{4.772470in}{0.677168in}}%
\pgfpathclose%
\pgfusepath{stroke}%
\end{pgfscope}%
\begin{pgfscope}%
\pgfpathrectangle{\pgfqpoint{0.847223in}{0.554012in}}{\pgfqpoint{6.200000in}{4.530000in}}%
\pgfusepath{clip}%
\pgfsetbuttcap%
\pgfsetroundjoin%
\pgfsetlinewidth{1.003750pt}%
\definecolor{currentstroke}{rgb}{1.000000,0.000000,0.000000}%
\pgfsetstrokecolor{currentstroke}%
\pgfsetdash{}{0pt}%
\pgfpathmoveto{\pgfqpoint{4.777803in}{0.676471in}}%
\pgfpathcurveto{\pgfqpoint{4.788853in}{0.676471in}}{\pgfqpoint{4.799452in}{0.680861in}}{\pgfqpoint{4.807266in}{0.688675in}}%
\pgfpathcurveto{\pgfqpoint{4.815079in}{0.696488in}}{\pgfqpoint{4.819470in}{0.707087in}}{\pgfqpoint{4.819470in}{0.718138in}}%
\pgfpathcurveto{\pgfqpoint{4.819470in}{0.729188in}}{\pgfqpoint{4.815079in}{0.739787in}}{\pgfqpoint{4.807266in}{0.747600in}}%
\pgfpathcurveto{\pgfqpoint{4.799452in}{0.755414in}}{\pgfqpoint{4.788853in}{0.759804in}}{\pgfqpoint{4.777803in}{0.759804in}}%
\pgfpathcurveto{\pgfqpoint{4.766753in}{0.759804in}}{\pgfqpoint{4.756154in}{0.755414in}}{\pgfqpoint{4.748340in}{0.747600in}}%
\pgfpathcurveto{\pgfqpoint{4.740527in}{0.739787in}}{\pgfqpoint{4.736136in}{0.729188in}}{\pgfqpoint{4.736136in}{0.718138in}}%
\pgfpathcurveto{\pgfqpoint{4.736136in}{0.707087in}}{\pgfqpoint{4.740527in}{0.696488in}}{\pgfqpoint{4.748340in}{0.688675in}}%
\pgfpathcurveto{\pgfqpoint{4.756154in}{0.680861in}}{\pgfqpoint{4.766753in}{0.676471in}}{\pgfqpoint{4.777803in}{0.676471in}}%
\pgfpathlineto{\pgfqpoint{4.777803in}{0.676471in}}%
\pgfpathclose%
\pgfusepath{stroke}%
\end{pgfscope}%
\begin{pgfscope}%
\pgfpathrectangle{\pgfqpoint{0.847223in}{0.554012in}}{\pgfqpoint{6.200000in}{4.530000in}}%
\pgfusepath{clip}%
\pgfsetbuttcap%
\pgfsetroundjoin%
\pgfsetlinewidth{1.003750pt}%
\definecolor{currentstroke}{rgb}{1.000000,0.000000,0.000000}%
\pgfsetstrokecolor{currentstroke}%
\pgfsetdash{}{0pt}%
\pgfpathmoveto{\pgfqpoint{4.783136in}{0.675776in}}%
\pgfpathcurveto{\pgfqpoint{4.794186in}{0.675776in}}{\pgfqpoint{4.804785in}{0.680166in}}{\pgfqpoint{4.812599in}{0.687980in}}%
\pgfpathcurveto{\pgfqpoint{4.820413in}{0.695793in}}{\pgfqpoint{4.824803in}{0.706392in}}{\pgfqpoint{4.824803in}{0.717442in}}%
\pgfpathcurveto{\pgfqpoint{4.824803in}{0.728492in}}{\pgfqpoint{4.820413in}{0.739091in}}{\pgfqpoint{4.812599in}{0.746905in}}%
\pgfpathcurveto{\pgfqpoint{4.804785in}{0.754719in}}{\pgfqpoint{4.794186in}{0.759109in}}{\pgfqpoint{4.783136in}{0.759109in}}%
\pgfpathcurveto{\pgfqpoint{4.772086in}{0.759109in}}{\pgfqpoint{4.761487in}{0.754719in}}{\pgfqpoint{4.753673in}{0.746905in}}%
\pgfpathcurveto{\pgfqpoint{4.745860in}{0.739091in}}{\pgfqpoint{4.741470in}{0.728492in}}{\pgfqpoint{4.741470in}{0.717442in}}%
\pgfpathcurveto{\pgfqpoint{4.741470in}{0.706392in}}{\pgfqpoint{4.745860in}{0.695793in}}{\pgfqpoint{4.753673in}{0.687980in}}%
\pgfpathcurveto{\pgfqpoint{4.761487in}{0.680166in}}{\pgfqpoint{4.772086in}{0.675776in}}{\pgfqpoint{4.783136in}{0.675776in}}%
\pgfpathlineto{\pgfqpoint{4.783136in}{0.675776in}}%
\pgfpathclose%
\pgfusepath{stroke}%
\end{pgfscope}%
\begin{pgfscope}%
\pgfpathrectangle{\pgfqpoint{0.847223in}{0.554012in}}{\pgfqpoint{6.200000in}{4.530000in}}%
\pgfusepath{clip}%
\pgfsetbuttcap%
\pgfsetroundjoin%
\pgfsetlinewidth{1.003750pt}%
\definecolor{currentstroke}{rgb}{1.000000,0.000000,0.000000}%
\pgfsetstrokecolor{currentstroke}%
\pgfsetdash{}{0pt}%
\pgfpathmoveto{\pgfqpoint{4.788469in}{0.675082in}}%
\pgfpathcurveto{\pgfqpoint{4.799520in}{0.675082in}}{\pgfqpoint{4.810119in}{0.679472in}}{\pgfqpoint{4.817932in}{0.687286in}}%
\pgfpathcurveto{\pgfqpoint{4.825746in}{0.695100in}}{\pgfqpoint{4.830136in}{0.705699in}}{\pgfqpoint{4.830136in}{0.716749in}}%
\pgfpathcurveto{\pgfqpoint{4.830136in}{0.727799in}}{\pgfqpoint{4.825746in}{0.738398in}}{\pgfqpoint{4.817932in}{0.746212in}}%
\pgfpathcurveto{\pgfqpoint{4.810119in}{0.754025in}}{\pgfqpoint{4.799520in}{0.758415in}}{\pgfqpoint{4.788469in}{0.758415in}}%
\pgfpathcurveto{\pgfqpoint{4.777419in}{0.758415in}}{\pgfqpoint{4.766820in}{0.754025in}}{\pgfqpoint{4.759007in}{0.746212in}}%
\pgfpathcurveto{\pgfqpoint{4.751193in}{0.738398in}}{\pgfqpoint{4.746803in}{0.727799in}}{\pgfqpoint{4.746803in}{0.716749in}}%
\pgfpathcurveto{\pgfqpoint{4.746803in}{0.705699in}}{\pgfqpoint{4.751193in}{0.695100in}}{\pgfqpoint{4.759007in}{0.687286in}}%
\pgfpathcurveto{\pgfqpoint{4.766820in}{0.679472in}}{\pgfqpoint{4.777419in}{0.675082in}}{\pgfqpoint{4.788469in}{0.675082in}}%
\pgfpathlineto{\pgfqpoint{4.788469in}{0.675082in}}%
\pgfpathclose%
\pgfusepath{stroke}%
\end{pgfscope}%
\begin{pgfscope}%
\pgfpathrectangle{\pgfqpoint{0.847223in}{0.554012in}}{\pgfqpoint{6.200000in}{4.530000in}}%
\pgfusepath{clip}%
\pgfsetbuttcap%
\pgfsetroundjoin%
\pgfsetlinewidth{1.003750pt}%
\definecolor{currentstroke}{rgb}{1.000000,0.000000,0.000000}%
\pgfsetstrokecolor{currentstroke}%
\pgfsetdash{}{0pt}%
\pgfpathmoveto{\pgfqpoint{4.793803in}{0.674390in}}%
\pgfpathcurveto{\pgfqpoint{4.804853in}{0.674390in}}{\pgfqpoint{4.815452in}{0.678780in}}{\pgfqpoint{4.823265in}{0.686594in}}%
\pgfpathcurveto{\pgfqpoint{4.831079in}{0.694408in}}{\pgfqpoint{4.835469in}{0.705007in}}{\pgfqpoint{4.835469in}{0.716057in}}%
\pgfpathcurveto{\pgfqpoint{4.835469in}{0.727107in}}{\pgfqpoint{4.831079in}{0.737706in}}{\pgfqpoint{4.823265in}{0.745520in}}%
\pgfpathcurveto{\pgfqpoint{4.815452in}{0.753333in}}{\pgfqpoint{4.804853in}{0.757723in}}{\pgfqpoint{4.793803in}{0.757723in}}%
\pgfpathcurveto{\pgfqpoint{4.782753in}{0.757723in}}{\pgfqpoint{4.772153in}{0.753333in}}{\pgfqpoint{4.764340in}{0.745520in}}%
\pgfpathcurveto{\pgfqpoint{4.756526in}{0.737706in}}{\pgfqpoint{4.752136in}{0.727107in}}{\pgfqpoint{4.752136in}{0.716057in}}%
\pgfpathcurveto{\pgfqpoint{4.752136in}{0.705007in}}{\pgfqpoint{4.756526in}{0.694408in}}{\pgfqpoint{4.764340in}{0.686594in}}%
\pgfpathcurveto{\pgfqpoint{4.772153in}{0.678780in}}{\pgfqpoint{4.782753in}{0.674390in}}{\pgfqpoint{4.793803in}{0.674390in}}%
\pgfpathlineto{\pgfqpoint{4.793803in}{0.674390in}}%
\pgfpathclose%
\pgfusepath{stroke}%
\end{pgfscope}%
\begin{pgfscope}%
\pgfpathrectangle{\pgfqpoint{0.847223in}{0.554012in}}{\pgfqpoint{6.200000in}{4.530000in}}%
\pgfusepath{clip}%
\pgfsetbuttcap%
\pgfsetroundjoin%
\pgfsetlinewidth{1.003750pt}%
\definecolor{currentstroke}{rgb}{1.000000,0.000000,0.000000}%
\pgfsetstrokecolor{currentstroke}%
\pgfsetdash{}{0pt}%
\pgfpathmoveto{\pgfqpoint{4.799136in}{0.673700in}}%
\pgfpathcurveto{\pgfqpoint{4.810186in}{0.673700in}}{\pgfqpoint{4.820785in}{0.678090in}}{\pgfqpoint{4.828599in}{0.685904in}}%
\pgfpathcurveto{\pgfqpoint{4.836412in}{0.693717in}}{\pgfqpoint{4.840803in}{0.704316in}}{\pgfqpoint{4.840803in}{0.715367in}}%
\pgfpathcurveto{\pgfqpoint{4.840803in}{0.726417in}}{\pgfqpoint{4.836412in}{0.737016in}}{\pgfqpoint{4.828599in}{0.744829in}}%
\pgfpathcurveto{\pgfqpoint{4.820785in}{0.752643in}}{\pgfqpoint{4.810186in}{0.757033in}}{\pgfqpoint{4.799136in}{0.757033in}}%
\pgfpathcurveto{\pgfqpoint{4.788086in}{0.757033in}}{\pgfqpoint{4.777487in}{0.752643in}}{\pgfqpoint{4.769673in}{0.744829in}}%
\pgfpathcurveto{\pgfqpoint{4.761859in}{0.737016in}}{\pgfqpoint{4.757469in}{0.726417in}}{\pgfqpoint{4.757469in}{0.715367in}}%
\pgfpathcurveto{\pgfqpoint{4.757469in}{0.704316in}}{\pgfqpoint{4.761859in}{0.693717in}}{\pgfqpoint{4.769673in}{0.685904in}}%
\pgfpathcurveto{\pgfqpoint{4.777487in}{0.678090in}}{\pgfqpoint{4.788086in}{0.673700in}}{\pgfqpoint{4.799136in}{0.673700in}}%
\pgfpathlineto{\pgfqpoint{4.799136in}{0.673700in}}%
\pgfpathclose%
\pgfusepath{stroke}%
\end{pgfscope}%
\begin{pgfscope}%
\pgfpathrectangle{\pgfqpoint{0.847223in}{0.554012in}}{\pgfqpoint{6.200000in}{4.530000in}}%
\pgfusepath{clip}%
\pgfsetbuttcap%
\pgfsetroundjoin%
\pgfsetlinewidth{1.003750pt}%
\definecolor{currentstroke}{rgb}{1.000000,0.000000,0.000000}%
\pgfsetstrokecolor{currentstroke}%
\pgfsetdash{}{0pt}%
\pgfpathmoveto{\pgfqpoint{4.804469in}{0.673011in}}%
\pgfpathcurveto{\pgfqpoint{4.815519in}{0.673011in}}{\pgfqpoint{4.826118in}{0.677402in}}{\pgfqpoint{4.833932in}{0.685215in}}%
\pgfpathcurveto{\pgfqpoint{4.841745in}{0.693029in}}{\pgfqpoint{4.846136in}{0.703628in}}{\pgfqpoint{4.846136in}{0.714678in}}%
\pgfpathcurveto{\pgfqpoint{4.846136in}{0.725728in}}{\pgfqpoint{4.841745in}{0.736327in}}{\pgfqpoint{4.833932in}{0.744141in}}%
\pgfpathcurveto{\pgfqpoint{4.826118in}{0.751954in}}{\pgfqpoint{4.815519in}{0.756345in}}{\pgfqpoint{4.804469in}{0.756345in}}%
\pgfpathcurveto{\pgfqpoint{4.793419in}{0.756345in}}{\pgfqpoint{4.782820in}{0.751954in}}{\pgfqpoint{4.775006in}{0.744141in}}%
\pgfpathcurveto{\pgfqpoint{4.767193in}{0.736327in}}{\pgfqpoint{4.762802in}{0.725728in}}{\pgfqpoint{4.762802in}{0.714678in}}%
\pgfpathcurveto{\pgfqpoint{4.762802in}{0.703628in}}{\pgfqpoint{4.767193in}{0.693029in}}{\pgfqpoint{4.775006in}{0.685215in}}%
\pgfpathcurveto{\pgfqpoint{4.782820in}{0.677402in}}{\pgfqpoint{4.793419in}{0.673011in}}{\pgfqpoint{4.804469in}{0.673011in}}%
\pgfpathlineto{\pgfqpoint{4.804469in}{0.673011in}}%
\pgfpathclose%
\pgfusepath{stroke}%
\end{pgfscope}%
\begin{pgfscope}%
\pgfpathrectangle{\pgfqpoint{0.847223in}{0.554012in}}{\pgfqpoint{6.200000in}{4.530000in}}%
\pgfusepath{clip}%
\pgfsetbuttcap%
\pgfsetroundjoin%
\pgfsetlinewidth{1.003750pt}%
\definecolor{currentstroke}{rgb}{1.000000,0.000000,0.000000}%
\pgfsetstrokecolor{currentstroke}%
\pgfsetdash{}{0pt}%
\pgfpathmoveto{\pgfqpoint{4.809802in}{0.672324in}}%
\pgfpathcurveto{\pgfqpoint{4.820852in}{0.672324in}}{\pgfqpoint{4.831451in}{0.676715in}}{\pgfqpoint{4.839265in}{0.684528in}}%
\pgfpathcurveto{\pgfqpoint{4.847079in}{0.692342in}}{\pgfqpoint{4.851469in}{0.702941in}}{\pgfqpoint{4.851469in}{0.713991in}}%
\pgfpathcurveto{\pgfqpoint{4.851469in}{0.725041in}}{\pgfqpoint{4.847079in}{0.735640in}}{\pgfqpoint{4.839265in}{0.743454in}}%
\pgfpathcurveto{\pgfqpoint{4.831451in}{0.751267in}}{\pgfqpoint{4.820852in}{0.755658in}}{\pgfqpoint{4.809802in}{0.755658in}}%
\pgfpathcurveto{\pgfqpoint{4.798752in}{0.755658in}}{\pgfqpoint{4.788153in}{0.751267in}}{\pgfqpoint{4.780340in}{0.743454in}}%
\pgfpathcurveto{\pgfqpoint{4.772526in}{0.735640in}}{\pgfqpoint{4.768136in}{0.725041in}}{\pgfqpoint{4.768136in}{0.713991in}}%
\pgfpathcurveto{\pgfqpoint{4.768136in}{0.702941in}}{\pgfqpoint{4.772526in}{0.692342in}}{\pgfqpoint{4.780340in}{0.684528in}}%
\pgfpathcurveto{\pgfqpoint{4.788153in}{0.676715in}}{\pgfqpoint{4.798752in}{0.672324in}}{\pgfqpoint{4.809802in}{0.672324in}}%
\pgfpathlineto{\pgfqpoint{4.809802in}{0.672324in}}%
\pgfpathclose%
\pgfusepath{stroke}%
\end{pgfscope}%
\begin{pgfscope}%
\pgfpathrectangle{\pgfqpoint{0.847223in}{0.554012in}}{\pgfqpoint{6.200000in}{4.530000in}}%
\pgfusepath{clip}%
\pgfsetbuttcap%
\pgfsetroundjoin%
\pgfsetlinewidth{1.003750pt}%
\definecolor{currentstroke}{rgb}{1.000000,0.000000,0.000000}%
\pgfsetstrokecolor{currentstroke}%
\pgfsetdash{}{0pt}%
\pgfpathmoveto{\pgfqpoint{4.815136in}{0.671639in}}%
\pgfpathcurveto{\pgfqpoint{4.826186in}{0.671639in}}{\pgfqpoint{4.836785in}{0.676029in}}{\pgfqpoint{4.844598in}{0.683843in}}%
\pgfpathcurveto{\pgfqpoint{4.852412in}{0.691656in}}{\pgfqpoint{4.856802in}{0.702255in}}{\pgfqpoint{4.856802in}{0.713306in}}%
\pgfpathcurveto{\pgfqpoint{4.856802in}{0.724356in}}{\pgfqpoint{4.852412in}{0.734955in}}{\pgfqpoint{4.844598in}{0.742768in}}%
\pgfpathcurveto{\pgfqpoint{4.836785in}{0.750582in}}{\pgfqpoint{4.826186in}{0.754972in}}{\pgfqpoint{4.815136in}{0.754972in}}%
\pgfpathcurveto{\pgfqpoint{4.804085in}{0.754972in}}{\pgfqpoint{4.793486in}{0.750582in}}{\pgfqpoint{4.785673in}{0.742768in}}%
\pgfpathcurveto{\pgfqpoint{4.777859in}{0.734955in}}{\pgfqpoint{4.773469in}{0.724356in}}{\pgfqpoint{4.773469in}{0.713306in}}%
\pgfpathcurveto{\pgfqpoint{4.773469in}{0.702255in}}{\pgfqpoint{4.777859in}{0.691656in}}{\pgfqpoint{4.785673in}{0.683843in}}%
\pgfpathcurveto{\pgfqpoint{4.793486in}{0.676029in}}{\pgfqpoint{4.804085in}{0.671639in}}{\pgfqpoint{4.815136in}{0.671639in}}%
\pgfpathlineto{\pgfqpoint{4.815136in}{0.671639in}}%
\pgfpathclose%
\pgfusepath{stroke}%
\end{pgfscope}%
\begin{pgfscope}%
\pgfpathrectangle{\pgfqpoint{0.847223in}{0.554012in}}{\pgfqpoint{6.200000in}{4.530000in}}%
\pgfusepath{clip}%
\pgfsetbuttcap%
\pgfsetroundjoin%
\pgfsetlinewidth{1.003750pt}%
\definecolor{currentstroke}{rgb}{1.000000,0.000000,0.000000}%
\pgfsetstrokecolor{currentstroke}%
\pgfsetdash{}{0pt}%
\pgfpathmoveto{\pgfqpoint{4.820469in}{0.670955in}}%
\pgfpathcurveto{\pgfqpoint{4.831519in}{0.670955in}}{\pgfqpoint{4.842118in}{0.675345in}}{\pgfqpoint{4.849932in}{0.683159in}}%
\pgfpathcurveto{\pgfqpoint{4.857745in}{0.690973in}}{\pgfqpoint{4.862135in}{0.701572in}}{\pgfqpoint{4.862135in}{0.712622in}}%
\pgfpathcurveto{\pgfqpoint{4.862135in}{0.723672in}}{\pgfqpoint{4.857745in}{0.734271in}}{\pgfqpoint{4.849932in}{0.742085in}}%
\pgfpathcurveto{\pgfqpoint{4.842118in}{0.749898in}}{\pgfqpoint{4.831519in}{0.754289in}}{\pgfqpoint{4.820469in}{0.754289in}}%
\pgfpathcurveto{\pgfqpoint{4.809419in}{0.754289in}}{\pgfqpoint{4.798820in}{0.749898in}}{\pgfqpoint{4.791006in}{0.742085in}}%
\pgfpathcurveto{\pgfqpoint{4.783192in}{0.734271in}}{\pgfqpoint{4.778802in}{0.723672in}}{\pgfqpoint{4.778802in}{0.712622in}}%
\pgfpathcurveto{\pgfqpoint{4.778802in}{0.701572in}}{\pgfqpoint{4.783192in}{0.690973in}}{\pgfqpoint{4.791006in}{0.683159in}}%
\pgfpathcurveto{\pgfqpoint{4.798820in}{0.675345in}}{\pgfqpoint{4.809419in}{0.670955in}}{\pgfqpoint{4.820469in}{0.670955in}}%
\pgfpathlineto{\pgfqpoint{4.820469in}{0.670955in}}%
\pgfpathclose%
\pgfusepath{stroke}%
\end{pgfscope}%
\begin{pgfscope}%
\pgfpathrectangle{\pgfqpoint{0.847223in}{0.554012in}}{\pgfqpoint{6.200000in}{4.530000in}}%
\pgfusepath{clip}%
\pgfsetbuttcap%
\pgfsetroundjoin%
\pgfsetlinewidth{1.003750pt}%
\definecolor{currentstroke}{rgb}{1.000000,0.000000,0.000000}%
\pgfsetstrokecolor{currentstroke}%
\pgfsetdash{}{0pt}%
\pgfpathmoveto{\pgfqpoint{4.825802in}{0.670273in}}%
\pgfpathcurveto{\pgfqpoint{4.836852in}{0.670273in}}{\pgfqpoint{4.847451in}{0.674663in}}{\pgfqpoint{4.855265in}{0.682477in}}%
\pgfpathcurveto{\pgfqpoint{4.863078in}{0.690291in}}{\pgfqpoint{4.867469in}{0.700890in}}{\pgfqpoint{4.867469in}{0.711940in}}%
\pgfpathcurveto{\pgfqpoint{4.867469in}{0.722990in}}{\pgfqpoint{4.863078in}{0.733589in}}{\pgfqpoint{4.855265in}{0.741403in}}%
\pgfpathcurveto{\pgfqpoint{4.847451in}{0.749216in}}{\pgfqpoint{4.836852in}{0.753606in}}{\pgfqpoint{4.825802in}{0.753606in}}%
\pgfpathcurveto{\pgfqpoint{4.814752in}{0.753606in}}{\pgfqpoint{4.804153in}{0.749216in}}{\pgfqpoint{4.796339in}{0.741403in}}%
\pgfpathcurveto{\pgfqpoint{4.788526in}{0.733589in}}{\pgfqpoint{4.784135in}{0.722990in}}{\pgfqpoint{4.784135in}{0.711940in}}%
\pgfpathcurveto{\pgfqpoint{4.784135in}{0.700890in}}{\pgfqpoint{4.788526in}{0.690291in}}{\pgfqpoint{4.796339in}{0.682477in}}%
\pgfpathcurveto{\pgfqpoint{4.804153in}{0.674663in}}{\pgfqpoint{4.814752in}{0.670273in}}{\pgfqpoint{4.825802in}{0.670273in}}%
\pgfpathlineto{\pgfqpoint{4.825802in}{0.670273in}}%
\pgfpathclose%
\pgfusepath{stroke}%
\end{pgfscope}%
\begin{pgfscope}%
\pgfpathrectangle{\pgfqpoint{0.847223in}{0.554012in}}{\pgfqpoint{6.200000in}{4.530000in}}%
\pgfusepath{clip}%
\pgfsetbuttcap%
\pgfsetroundjoin%
\pgfsetlinewidth{1.003750pt}%
\definecolor{currentstroke}{rgb}{1.000000,0.000000,0.000000}%
\pgfsetstrokecolor{currentstroke}%
\pgfsetdash{}{0pt}%
\pgfpathmoveto{\pgfqpoint{4.831135in}{0.669593in}}%
\pgfpathcurveto{\pgfqpoint{4.842185in}{0.669593in}}{\pgfqpoint{4.852784in}{0.673983in}}{\pgfqpoint{4.860598in}{0.681796in}}%
\pgfpathcurveto{\pgfqpoint{4.868412in}{0.689610in}}{\pgfqpoint{4.872802in}{0.700209in}}{\pgfqpoint{4.872802in}{0.711259in}}%
\pgfpathcurveto{\pgfqpoint{4.872802in}{0.722309in}}{\pgfqpoint{4.868412in}{0.732908in}}{\pgfqpoint{4.860598in}{0.740722in}}%
\pgfpathcurveto{\pgfqpoint{4.852784in}{0.748536in}}{\pgfqpoint{4.842185in}{0.752926in}}{\pgfqpoint{4.831135in}{0.752926in}}%
\pgfpathcurveto{\pgfqpoint{4.820085in}{0.752926in}}{\pgfqpoint{4.809486in}{0.748536in}}{\pgfqpoint{4.801672in}{0.740722in}}%
\pgfpathcurveto{\pgfqpoint{4.793859in}{0.732908in}}{\pgfqpoint{4.789468in}{0.722309in}}{\pgfqpoint{4.789468in}{0.711259in}}%
\pgfpathcurveto{\pgfqpoint{4.789468in}{0.700209in}}{\pgfqpoint{4.793859in}{0.689610in}}{\pgfqpoint{4.801672in}{0.681796in}}%
\pgfpathcurveto{\pgfqpoint{4.809486in}{0.673983in}}{\pgfqpoint{4.820085in}{0.669593in}}{\pgfqpoint{4.831135in}{0.669593in}}%
\pgfpathlineto{\pgfqpoint{4.831135in}{0.669593in}}%
\pgfpathclose%
\pgfusepath{stroke}%
\end{pgfscope}%
\begin{pgfscope}%
\pgfpathrectangle{\pgfqpoint{0.847223in}{0.554012in}}{\pgfqpoint{6.200000in}{4.530000in}}%
\pgfusepath{clip}%
\pgfsetbuttcap%
\pgfsetroundjoin%
\pgfsetlinewidth{1.003750pt}%
\definecolor{currentstroke}{rgb}{1.000000,0.000000,0.000000}%
\pgfsetstrokecolor{currentstroke}%
\pgfsetdash{}{0pt}%
\pgfpathmoveto{\pgfqpoint{4.836468in}{0.668914in}}%
\pgfpathcurveto{\pgfqpoint{4.847518in}{0.668914in}}{\pgfqpoint{4.858118in}{0.673304in}}{\pgfqpoint{4.865931in}{0.681118in}}%
\pgfpathcurveto{\pgfqpoint{4.873745in}{0.688931in}}{\pgfqpoint{4.878135in}{0.699530in}}{\pgfqpoint{4.878135in}{0.710580in}}%
\pgfpathcurveto{\pgfqpoint{4.878135in}{0.721630in}}{\pgfqpoint{4.873745in}{0.732229in}}{\pgfqpoint{4.865931in}{0.740043in}}%
\pgfpathcurveto{\pgfqpoint{4.858118in}{0.747857in}}{\pgfqpoint{4.847518in}{0.752247in}}{\pgfqpoint{4.836468in}{0.752247in}}%
\pgfpathcurveto{\pgfqpoint{4.825418in}{0.752247in}}{\pgfqpoint{4.814819in}{0.747857in}}{\pgfqpoint{4.807006in}{0.740043in}}%
\pgfpathcurveto{\pgfqpoint{4.799192in}{0.732229in}}{\pgfqpoint{4.794802in}{0.721630in}}{\pgfqpoint{4.794802in}{0.710580in}}%
\pgfpathcurveto{\pgfqpoint{4.794802in}{0.699530in}}{\pgfqpoint{4.799192in}{0.688931in}}{\pgfqpoint{4.807006in}{0.681118in}}%
\pgfpathcurveto{\pgfqpoint{4.814819in}{0.673304in}}{\pgfqpoint{4.825418in}{0.668914in}}{\pgfqpoint{4.836468in}{0.668914in}}%
\pgfpathlineto{\pgfqpoint{4.836468in}{0.668914in}}%
\pgfpathclose%
\pgfusepath{stroke}%
\end{pgfscope}%
\begin{pgfscope}%
\pgfpathrectangle{\pgfqpoint{0.847223in}{0.554012in}}{\pgfqpoint{6.200000in}{4.530000in}}%
\pgfusepath{clip}%
\pgfsetbuttcap%
\pgfsetroundjoin%
\pgfsetlinewidth{1.003750pt}%
\definecolor{currentstroke}{rgb}{1.000000,0.000000,0.000000}%
\pgfsetstrokecolor{currentstroke}%
\pgfsetdash{}{0pt}%
\pgfpathmoveto{\pgfqpoint{4.841802in}{0.668236in}}%
\pgfpathcurveto{\pgfqpoint{4.852852in}{0.668236in}}{\pgfqpoint{4.863451in}{0.672627in}}{\pgfqpoint{4.871264in}{0.680440in}}%
\pgfpathcurveto{\pgfqpoint{4.879078in}{0.688254in}}{\pgfqpoint{4.883468in}{0.698853in}}{\pgfqpoint{4.883468in}{0.709903in}}%
\pgfpathcurveto{\pgfqpoint{4.883468in}{0.720953in}}{\pgfqpoint{4.879078in}{0.731552in}}{\pgfqpoint{4.871264in}{0.739366in}}%
\pgfpathcurveto{\pgfqpoint{4.863451in}{0.747179in}}{\pgfqpoint{4.852852in}{0.751570in}}{\pgfqpoint{4.841802in}{0.751570in}}%
\pgfpathcurveto{\pgfqpoint{4.830751in}{0.751570in}}{\pgfqpoint{4.820152in}{0.747179in}}{\pgfqpoint{4.812339in}{0.739366in}}%
\pgfpathcurveto{\pgfqpoint{4.804525in}{0.731552in}}{\pgfqpoint{4.800135in}{0.720953in}}{\pgfqpoint{4.800135in}{0.709903in}}%
\pgfpathcurveto{\pgfqpoint{4.800135in}{0.698853in}}{\pgfqpoint{4.804525in}{0.688254in}}{\pgfqpoint{4.812339in}{0.680440in}}%
\pgfpathcurveto{\pgfqpoint{4.820152in}{0.672627in}}{\pgfqpoint{4.830751in}{0.668236in}}{\pgfqpoint{4.841802in}{0.668236in}}%
\pgfpathlineto{\pgfqpoint{4.841802in}{0.668236in}}%
\pgfpathclose%
\pgfusepath{stroke}%
\end{pgfscope}%
\begin{pgfscope}%
\pgfpathrectangle{\pgfqpoint{0.847223in}{0.554012in}}{\pgfqpoint{6.200000in}{4.530000in}}%
\pgfusepath{clip}%
\pgfsetbuttcap%
\pgfsetroundjoin%
\pgfsetlinewidth{1.003750pt}%
\definecolor{currentstroke}{rgb}{1.000000,0.000000,0.000000}%
\pgfsetstrokecolor{currentstroke}%
\pgfsetdash{}{0pt}%
\pgfpathmoveto{\pgfqpoint{4.847135in}{0.667561in}}%
\pgfpathcurveto{\pgfqpoint{4.858185in}{0.667561in}}{\pgfqpoint{4.868784in}{0.671951in}}{\pgfqpoint{4.876598in}{0.679765in}}%
\pgfpathcurveto{\pgfqpoint{4.884411in}{0.687578in}}{\pgfqpoint{4.888801in}{0.698177in}}{\pgfqpoint{4.888801in}{0.709227in}}%
\pgfpathcurveto{\pgfqpoint{4.888801in}{0.720277in}}{\pgfqpoint{4.884411in}{0.730877in}}{\pgfqpoint{4.876598in}{0.738690in}}%
\pgfpathcurveto{\pgfqpoint{4.868784in}{0.746504in}}{\pgfqpoint{4.858185in}{0.750894in}}{\pgfqpoint{4.847135in}{0.750894in}}%
\pgfpathcurveto{\pgfqpoint{4.836085in}{0.750894in}}{\pgfqpoint{4.825486in}{0.746504in}}{\pgfqpoint{4.817672in}{0.738690in}}%
\pgfpathcurveto{\pgfqpoint{4.809858in}{0.730877in}}{\pgfqpoint{4.805468in}{0.720277in}}{\pgfqpoint{4.805468in}{0.709227in}}%
\pgfpathcurveto{\pgfqpoint{4.805468in}{0.698177in}}{\pgfqpoint{4.809858in}{0.687578in}}{\pgfqpoint{4.817672in}{0.679765in}}%
\pgfpathcurveto{\pgfqpoint{4.825486in}{0.671951in}}{\pgfqpoint{4.836085in}{0.667561in}}{\pgfqpoint{4.847135in}{0.667561in}}%
\pgfpathlineto{\pgfqpoint{4.847135in}{0.667561in}}%
\pgfpathclose%
\pgfusepath{stroke}%
\end{pgfscope}%
\begin{pgfscope}%
\pgfpathrectangle{\pgfqpoint{0.847223in}{0.554012in}}{\pgfqpoint{6.200000in}{4.530000in}}%
\pgfusepath{clip}%
\pgfsetbuttcap%
\pgfsetroundjoin%
\pgfsetlinewidth{1.003750pt}%
\definecolor{currentstroke}{rgb}{1.000000,0.000000,0.000000}%
\pgfsetstrokecolor{currentstroke}%
\pgfsetdash{}{0pt}%
\pgfpathmoveto{\pgfqpoint{4.852468in}{0.666887in}}%
\pgfpathcurveto{\pgfqpoint{4.863518in}{0.666887in}}{\pgfqpoint{4.874117in}{0.671277in}}{\pgfqpoint{4.881931in}{0.679090in}}%
\pgfpathcurveto{\pgfqpoint{4.889744in}{0.686904in}}{\pgfqpoint{4.894135in}{0.697503in}}{\pgfqpoint{4.894135in}{0.708553in}}%
\pgfpathcurveto{\pgfqpoint{4.894135in}{0.719603in}}{\pgfqpoint{4.889744in}{0.730202in}}{\pgfqpoint{4.881931in}{0.738016in}}%
\pgfpathcurveto{\pgfqpoint{4.874117in}{0.745830in}}{\pgfqpoint{4.863518in}{0.750220in}}{\pgfqpoint{4.852468in}{0.750220in}}%
\pgfpathcurveto{\pgfqpoint{4.841418in}{0.750220in}}{\pgfqpoint{4.830819in}{0.745830in}}{\pgfqpoint{4.823005in}{0.738016in}}%
\pgfpathcurveto{\pgfqpoint{4.815192in}{0.730202in}}{\pgfqpoint{4.810801in}{0.719603in}}{\pgfqpoint{4.810801in}{0.708553in}}%
\pgfpathcurveto{\pgfqpoint{4.810801in}{0.697503in}}{\pgfqpoint{4.815192in}{0.686904in}}{\pgfqpoint{4.823005in}{0.679090in}}%
\pgfpathcurveto{\pgfqpoint{4.830819in}{0.671277in}}{\pgfqpoint{4.841418in}{0.666887in}}{\pgfqpoint{4.852468in}{0.666887in}}%
\pgfpathlineto{\pgfqpoint{4.852468in}{0.666887in}}%
\pgfpathclose%
\pgfusepath{stroke}%
\end{pgfscope}%
\begin{pgfscope}%
\pgfpathrectangle{\pgfqpoint{0.847223in}{0.554012in}}{\pgfqpoint{6.200000in}{4.530000in}}%
\pgfusepath{clip}%
\pgfsetbuttcap%
\pgfsetroundjoin%
\pgfsetlinewidth{1.003750pt}%
\definecolor{currentstroke}{rgb}{1.000000,0.000000,0.000000}%
\pgfsetstrokecolor{currentstroke}%
\pgfsetdash{}{0pt}%
\pgfpathmoveto{\pgfqpoint{4.857801in}{0.666214in}}%
\pgfpathcurveto{\pgfqpoint{4.868851in}{0.666214in}}{\pgfqpoint{4.879450in}{0.670604in}}{\pgfqpoint{4.887264in}{0.678418in}}%
\pgfpathcurveto{\pgfqpoint{4.895078in}{0.686232in}}{\pgfqpoint{4.899468in}{0.696831in}}{\pgfqpoint{4.899468in}{0.707881in}}%
\pgfpathcurveto{\pgfqpoint{4.899468in}{0.718931in}}{\pgfqpoint{4.895078in}{0.729530in}}{\pgfqpoint{4.887264in}{0.737344in}}%
\pgfpathcurveto{\pgfqpoint{4.879450in}{0.745157in}}{\pgfqpoint{4.868851in}{0.749547in}}{\pgfqpoint{4.857801in}{0.749547in}}%
\pgfpathcurveto{\pgfqpoint{4.846751in}{0.749547in}}{\pgfqpoint{4.836152in}{0.745157in}}{\pgfqpoint{4.828338in}{0.737344in}}%
\pgfpathcurveto{\pgfqpoint{4.820525in}{0.729530in}}{\pgfqpoint{4.816135in}{0.718931in}}{\pgfqpoint{4.816135in}{0.707881in}}%
\pgfpathcurveto{\pgfqpoint{4.816135in}{0.696831in}}{\pgfqpoint{4.820525in}{0.686232in}}{\pgfqpoint{4.828338in}{0.678418in}}%
\pgfpathcurveto{\pgfqpoint{4.836152in}{0.670604in}}{\pgfqpoint{4.846751in}{0.666214in}}{\pgfqpoint{4.857801in}{0.666214in}}%
\pgfpathlineto{\pgfqpoint{4.857801in}{0.666214in}}%
\pgfpathclose%
\pgfusepath{stroke}%
\end{pgfscope}%
\begin{pgfscope}%
\pgfpathrectangle{\pgfqpoint{0.847223in}{0.554012in}}{\pgfqpoint{6.200000in}{4.530000in}}%
\pgfusepath{clip}%
\pgfsetbuttcap%
\pgfsetroundjoin%
\pgfsetlinewidth{1.003750pt}%
\definecolor{currentstroke}{rgb}{1.000000,0.000000,0.000000}%
\pgfsetstrokecolor{currentstroke}%
\pgfsetdash{}{0pt}%
\pgfpathmoveto{\pgfqpoint{4.863134in}{0.665543in}}%
\pgfpathcurveto{\pgfqpoint{4.874185in}{0.665543in}}{\pgfqpoint{4.884784in}{0.669933in}}{\pgfqpoint{4.892597in}{0.677747in}}%
\pgfpathcurveto{\pgfqpoint{4.900411in}{0.685561in}}{\pgfqpoint{4.904801in}{0.696160in}}{\pgfqpoint{4.904801in}{0.707210in}}%
\pgfpathcurveto{\pgfqpoint{4.904801in}{0.718260in}}{\pgfqpoint{4.900411in}{0.728859in}}{\pgfqpoint{4.892597in}{0.736673in}}%
\pgfpathcurveto{\pgfqpoint{4.884784in}{0.744486in}}{\pgfqpoint{4.874185in}{0.748876in}}{\pgfqpoint{4.863134in}{0.748876in}}%
\pgfpathcurveto{\pgfqpoint{4.852084in}{0.748876in}}{\pgfqpoint{4.841485in}{0.744486in}}{\pgfqpoint{4.833672in}{0.736673in}}%
\pgfpathcurveto{\pgfqpoint{4.825858in}{0.728859in}}{\pgfqpoint{4.821468in}{0.718260in}}{\pgfqpoint{4.821468in}{0.707210in}}%
\pgfpathcurveto{\pgfqpoint{4.821468in}{0.696160in}}{\pgfqpoint{4.825858in}{0.685561in}}{\pgfqpoint{4.833672in}{0.677747in}}%
\pgfpathcurveto{\pgfqpoint{4.841485in}{0.669933in}}{\pgfqpoint{4.852084in}{0.665543in}}{\pgfqpoint{4.863134in}{0.665543in}}%
\pgfpathlineto{\pgfqpoint{4.863134in}{0.665543in}}%
\pgfpathclose%
\pgfusepath{stroke}%
\end{pgfscope}%
\begin{pgfscope}%
\pgfpathrectangle{\pgfqpoint{0.847223in}{0.554012in}}{\pgfqpoint{6.200000in}{4.530000in}}%
\pgfusepath{clip}%
\pgfsetbuttcap%
\pgfsetroundjoin%
\pgfsetlinewidth{1.003750pt}%
\definecolor{currentstroke}{rgb}{1.000000,0.000000,0.000000}%
\pgfsetstrokecolor{currentstroke}%
\pgfsetdash{}{0pt}%
\pgfpathmoveto{\pgfqpoint{4.868468in}{0.664874in}}%
\pgfpathcurveto{\pgfqpoint{4.879518in}{0.664874in}}{\pgfqpoint{4.890117in}{0.669264in}}{\pgfqpoint{4.897930in}{0.677078in}}%
\pgfpathcurveto{\pgfqpoint{4.905744in}{0.684891in}}{\pgfqpoint{4.910134in}{0.695490in}}{\pgfqpoint{4.910134in}{0.706540in}}%
\pgfpathcurveto{\pgfqpoint{4.910134in}{0.717591in}}{\pgfqpoint{4.905744in}{0.728190in}}{\pgfqpoint{4.897930in}{0.736003in}}%
\pgfpathcurveto{\pgfqpoint{4.890117in}{0.743817in}}{\pgfqpoint{4.879518in}{0.748207in}}{\pgfqpoint{4.868468in}{0.748207in}}%
\pgfpathcurveto{\pgfqpoint{4.857418in}{0.748207in}}{\pgfqpoint{4.846819in}{0.743817in}}{\pgfqpoint{4.839005in}{0.736003in}}%
\pgfpathcurveto{\pgfqpoint{4.831191in}{0.728190in}}{\pgfqpoint{4.826801in}{0.717591in}}{\pgfqpoint{4.826801in}{0.706540in}}%
\pgfpathcurveto{\pgfqpoint{4.826801in}{0.695490in}}{\pgfqpoint{4.831191in}{0.684891in}}{\pgfqpoint{4.839005in}{0.677078in}}%
\pgfpathcurveto{\pgfqpoint{4.846819in}{0.669264in}}{\pgfqpoint{4.857418in}{0.664874in}}{\pgfqpoint{4.868468in}{0.664874in}}%
\pgfpathlineto{\pgfqpoint{4.868468in}{0.664874in}}%
\pgfpathclose%
\pgfusepath{stroke}%
\end{pgfscope}%
\begin{pgfscope}%
\pgfpathrectangle{\pgfqpoint{0.847223in}{0.554012in}}{\pgfqpoint{6.200000in}{4.530000in}}%
\pgfusepath{clip}%
\pgfsetbuttcap%
\pgfsetroundjoin%
\pgfsetlinewidth{1.003750pt}%
\definecolor{currentstroke}{rgb}{1.000000,0.000000,0.000000}%
\pgfsetstrokecolor{currentstroke}%
\pgfsetdash{}{0pt}%
\pgfpathmoveto{\pgfqpoint{4.873801in}{0.664206in}}%
\pgfpathcurveto{\pgfqpoint{4.884851in}{0.664206in}}{\pgfqpoint{4.895450in}{0.668596in}}{\pgfqpoint{4.903264in}{0.676410in}}%
\pgfpathcurveto{\pgfqpoint{4.911077in}{0.684223in}}{\pgfqpoint{4.915468in}{0.694823in}}{\pgfqpoint{4.915468in}{0.705873in}}%
\pgfpathcurveto{\pgfqpoint{4.915468in}{0.716923in}}{\pgfqpoint{4.911077in}{0.727522in}}{\pgfqpoint{4.903264in}{0.735335in}}%
\pgfpathcurveto{\pgfqpoint{4.895450in}{0.743149in}}{\pgfqpoint{4.884851in}{0.747539in}}{\pgfqpoint{4.873801in}{0.747539in}}%
\pgfpathcurveto{\pgfqpoint{4.862751in}{0.747539in}}{\pgfqpoint{4.852152in}{0.743149in}}{\pgfqpoint{4.844338in}{0.735335in}}%
\pgfpathcurveto{\pgfqpoint{4.836524in}{0.727522in}}{\pgfqpoint{4.832134in}{0.716923in}}{\pgfqpoint{4.832134in}{0.705873in}}%
\pgfpathcurveto{\pgfqpoint{4.832134in}{0.694823in}}{\pgfqpoint{4.836524in}{0.684223in}}{\pgfqpoint{4.844338in}{0.676410in}}%
\pgfpathcurveto{\pgfqpoint{4.852152in}{0.668596in}}{\pgfqpoint{4.862751in}{0.664206in}}{\pgfqpoint{4.873801in}{0.664206in}}%
\pgfpathlineto{\pgfqpoint{4.873801in}{0.664206in}}%
\pgfpathclose%
\pgfusepath{stroke}%
\end{pgfscope}%
\begin{pgfscope}%
\pgfpathrectangle{\pgfqpoint{0.847223in}{0.554012in}}{\pgfqpoint{6.200000in}{4.530000in}}%
\pgfusepath{clip}%
\pgfsetbuttcap%
\pgfsetroundjoin%
\pgfsetlinewidth{1.003750pt}%
\definecolor{currentstroke}{rgb}{1.000000,0.000000,0.000000}%
\pgfsetstrokecolor{currentstroke}%
\pgfsetdash{}{0pt}%
\pgfpathmoveto{\pgfqpoint{4.879134in}{0.663540in}}%
\pgfpathcurveto{\pgfqpoint{4.890184in}{0.663540in}}{\pgfqpoint{4.900783in}{0.667930in}}{\pgfqpoint{4.908597in}{0.675744in}}%
\pgfpathcurveto{\pgfqpoint{4.916410in}{0.683557in}}{\pgfqpoint{4.920801in}{0.694156in}}{\pgfqpoint{4.920801in}{0.705206in}}%
\pgfpathcurveto{\pgfqpoint{4.920801in}{0.716257in}}{\pgfqpoint{4.916410in}{0.726856in}}{\pgfqpoint{4.908597in}{0.734669in}}%
\pgfpathcurveto{\pgfqpoint{4.900783in}{0.742483in}}{\pgfqpoint{4.890184in}{0.746873in}}{\pgfqpoint{4.879134in}{0.746873in}}%
\pgfpathcurveto{\pgfqpoint{4.868084in}{0.746873in}}{\pgfqpoint{4.857485in}{0.742483in}}{\pgfqpoint{4.849671in}{0.734669in}}%
\pgfpathcurveto{\pgfqpoint{4.841858in}{0.726856in}}{\pgfqpoint{4.837467in}{0.716257in}}{\pgfqpoint{4.837467in}{0.705206in}}%
\pgfpathcurveto{\pgfqpoint{4.837467in}{0.694156in}}{\pgfqpoint{4.841858in}{0.683557in}}{\pgfqpoint{4.849671in}{0.675744in}}%
\pgfpathcurveto{\pgfqpoint{4.857485in}{0.667930in}}{\pgfqpoint{4.868084in}{0.663540in}}{\pgfqpoint{4.879134in}{0.663540in}}%
\pgfpathlineto{\pgfqpoint{4.879134in}{0.663540in}}%
\pgfpathclose%
\pgfusepath{stroke}%
\end{pgfscope}%
\begin{pgfscope}%
\pgfpathrectangle{\pgfqpoint{0.847223in}{0.554012in}}{\pgfqpoint{6.200000in}{4.530000in}}%
\pgfusepath{clip}%
\pgfsetbuttcap%
\pgfsetroundjoin%
\pgfsetlinewidth{1.003750pt}%
\definecolor{currentstroke}{rgb}{1.000000,0.000000,0.000000}%
\pgfsetstrokecolor{currentstroke}%
\pgfsetdash{}{0pt}%
\pgfpathmoveto{\pgfqpoint{4.884467in}{0.662875in}}%
\pgfpathcurveto{\pgfqpoint{4.895517in}{0.662875in}}{\pgfqpoint{4.906116in}{0.667265in}}{\pgfqpoint{4.913930in}{0.675079in}}%
\pgfpathcurveto{\pgfqpoint{4.921744in}{0.682893in}}{\pgfqpoint{4.926134in}{0.693492in}}{\pgfqpoint{4.926134in}{0.704542in}}%
\pgfpathcurveto{\pgfqpoint{4.926134in}{0.715592in}}{\pgfqpoint{4.921744in}{0.726191in}}{\pgfqpoint{4.913930in}{0.734005in}}%
\pgfpathcurveto{\pgfqpoint{4.906116in}{0.741818in}}{\pgfqpoint{4.895517in}{0.746208in}}{\pgfqpoint{4.884467in}{0.746208in}}%
\pgfpathcurveto{\pgfqpoint{4.873417in}{0.746208in}}{\pgfqpoint{4.862818in}{0.741818in}}{\pgfqpoint{4.855005in}{0.734005in}}%
\pgfpathcurveto{\pgfqpoint{4.847191in}{0.726191in}}{\pgfqpoint{4.842801in}{0.715592in}}{\pgfqpoint{4.842801in}{0.704542in}}%
\pgfpathcurveto{\pgfqpoint{4.842801in}{0.693492in}}{\pgfqpoint{4.847191in}{0.682893in}}{\pgfqpoint{4.855005in}{0.675079in}}%
\pgfpathcurveto{\pgfqpoint{4.862818in}{0.667265in}}{\pgfqpoint{4.873417in}{0.662875in}}{\pgfqpoint{4.884467in}{0.662875in}}%
\pgfpathlineto{\pgfqpoint{4.884467in}{0.662875in}}%
\pgfpathclose%
\pgfusepath{stroke}%
\end{pgfscope}%
\begin{pgfscope}%
\pgfpathrectangle{\pgfqpoint{0.847223in}{0.554012in}}{\pgfqpoint{6.200000in}{4.530000in}}%
\pgfusepath{clip}%
\pgfsetbuttcap%
\pgfsetroundjoin%
\pgfsetlinewidth{1.003750pt}%
\definecolor{currentstroke}{rgb}{1.000000,0.000000,0.000000}%
\pgfsetstrokecolor{currentstroke}%
\pgfsetdash{}{0pt}%
\pgfpathmoveto{\pgfqpoint{4.889801in}{0.662212in}}%
\pgfpathcurveto{\pgfqpoint{4.900851in}{0.662212in}}{\pgfqpoint{4.911450in}{0.666602in}}{\pgfqpoint{4.919263in}{0.674416in}}%
\pgfpathcurveto{\pgfqpoint{4.927077in}{0.682229in}}{\pgfqpoint{4.931467in}{0.692829in}}{\pgfqpoint{4.931467in}{0.703879in}}%
\pgfpathcurveto{\pgfqpoint{4.931467in}{0.714929in}}{\pgfqpoint{4.927077in}{0.725528in}}{\pgfqpoint{4.919263in}{0.733341in}}%
\pgfpathcurveto{\pgfqpoint{4.911450in}{0.741155in}}{\pgfqpoint{4.900851in}{0.745545in}}{\pgfqpoint{4.889801in}{0.745545in}}%
\pgfpathcurveto{\pgfqpoint{4.878750in}{0.745545in}}{\pgfqpoint{4.868151in}{0.741155in}}{\pgfqpoint{4.860338in}{0.733341in}}%
\pgfpathcurveto{\pgfqpoint{4.852524in}{0.725528in}}{\pgfqpoint{4.848134in}{0.714929in}}{\pgfqpoint{4.848134in}{0.703879in}}%
\pgfpathcurveto{\pgfqpoint{4.848134in}{0.692829in}}{\pgfqpoint{4.852524in}{0.682229in}}{\pgfqpoint{4.860338in}{0.674416in}}%
\pgfpathcurveto{\pgfqpoint{4.868151in}{0.666602in}}{\pgfqpoint{4.878750in}{0.662212in}}{\pgfqpoint{4.889801in}{0.662212in}}%
\pgfpathlineto{\pgfqpoint{4.889801in}{0.662212in}}%
\pgfpathclose%
\pgfusepath{stroke}%
\end{pgfscope}%
\begin{pgfscope}%
\pgfpathrectangle{\pgfqpoint{0.847223in}{0.554012in}}{\pgfqpoint{6.200000in}{4.530000in}}%
\pgfusepath{clip}%
\pgfsetbuttcap%
\pgfsetroundjoin%
\pgfsetlinewidth{1.003750pt}%
\definecolor{currentstroke}{rgb}{1.000000,0.000000,0.000000}%
\pgfsetstrokecolor{currentstroke}%
\pgfsetdash{}{0pt}%
\pgfpathmoveto{\pgfqpoint{4.895134in}{0.661550in}}%
\pgfpathcurveto{\pgfqpoint{4.906184in}{0.661550in}}{\pgfqpoint{4.916783in}{0.665941in}}{\pgfqpoint{4.924597in}{0.673754in}}%
\pgfpathcurveto{\pgfqpoint{4.932410in}{0.681568in}}{\pgfqpoint{4.936800in}{0.692167in}}{\pgfqpoint{4.936800in}{0.703217in}}%
\pgfpathcurveto{\pgfqpoint{4.936800in}{0.714267in}}{\pgfqpoint{4.932410in}{0.724866in}}{\pgfqpoint{4.924597in}{0.732680in}}%
\pgfpathcurveto{\pgfqpoint{4.916783in}{0.740493in}}{\pgfqpoint{4.906184in}{0.744884in}}{\pgfqpoint{4.895134in}{0.744884in}}%
\pgfpathcurveto{\pgfqpoint{4.884084in}{0.744884in}}{\pgfqpoint{4.873485in}{0.740493in}}{\pgfqpoint{4.865671in}{0.732680in}}%
\pgfpathcurveto{\pgfqpoint{4.857857in}{0.724866in}}{\pgfqpoint{4.853467in}{0.714267in}}{\pgfqpoint{4.853467in}{0.703217in}}%
\pgfpathcurveto{\pgfqpoint{4.853467in}{0.692167in}}{\pgfqpoint{4.857857in}{0.681568in}}{\pgfqpoint{4.865671in}{0.673754in}}%
\pgfpathcurveto{\pgfqpoint{4.873485in}{0.665941in}}{\pgfqpoint{4.884084in}{0.661550in}}{\pgfqpoint{4.895134in}{0.661550in}}%
\pgfpathlineto{\pgfqpoint{4.895134in}{0.661550in}}%
\pgfpathclose%
\pgfusepath{stroke}%
\end{pgfscope}%
\begin{pgfscope}%
\pgfpathrectangle{\pgfqpoint{0.847223in}{0.554012in}}{\pgfqpoint{6.200000in}{4.530000in}}%
\pgfusepath{clip}%
\pgfsetbuttcap%
\pgfsetroundjoin%
\pgfsetlinewidth{1.003750pt}%
\definecolor{currentstroke}{rgb}{1.000000,0.000000,0.000000}%
\pgfsetstrokecolor{currentstroke}%
\pgfsetdash{}{0pt}%
\pgfpathmoveto{\pgfqpoint{4.900467in}{0.660890in}}%
\pgfpathcurveto{\pgfqpoint{4.911517in}{0.660890in}}{\pgfqpoint{4.922116in}{0.665281in}}{\pgfqpoint{4.929930in}{0.673094in}}%
\pgfpathcurveto{\pgfqpoint{4.937743in}{0.680908in}}{\pgfqpoint{4.942134in}{0.691507in}}{\pgfqpoint{4.942134in}{0.702557in}}%
\pgfpathcurveto{\pgfqpoint{4.942134in}{0.713607in}}{\pgfqpoint{4.937743in}{0.724206in}}{\pgfqpoint{4.929930in}{0.732020in}}%
\pgfpathcurveto{\pgfqpoint{4.922116in}{0.739833in}}{\pgfqpoint{4.911517in}{0.744224in}}{\pgfqpoint{4.900467in}{0.744224in}}%
\pgfpathcurveto{\pgfqpoint{4.889417in}{0.744224in}}{\pgfqpoint{4.878818in}{0.739833in}}{\pgfqpoint{4.871004in}{0.732020in}}%
\pgfpathcurveto{\pgfqpoint{4.863191in}{0.724206in}}{\pgfqpoint{4.858800in}{0.713607in}}{\pgfqpoint{4.858800in}{0.702557in}}%
\pgfpathcurveto{\pgfqpoint{4.858800in}{0.691507in}}{\pgfqpoint{4.863191in}{0.680908in}}{\pgfqpoint{4.871004in}{0.673094in}}%
\pgfpathcurveto{\pgfqpoint{4.878818in}{0.665281in}}{\pgfqpoint{4.889417in}{0.660890in}}{\pgfqpoint{4.900467in}{0.660890in}}%
\pgfpathlineto{\pgfqpoint{4.900467in}{0.660890in}}%
\pgfpathclose%
\pgfusepath{stroke}%
\end{pgfscope}%
\begin{pgfscope}%
\pgfpathrectangle{\pgfqpoint{0.847223in}{0.554012in}}{\pgfqpoint{6.200000in}{4.530000in}}%
\pgfusepath{clip}%
\pgfsetbuttcap%
\pgfsetroundjoin%
\pgfsetlinewidth{1.003750pt}%
\definecolor{currentstroke}{rgb}{1.000000,0.000000,0.000000}%
\pgfsetstrokecolor{currentstroke}%
\pgfsetdash{}{0pt}%
\pgfpathmoveto{\pgfqpoint{4.905800in}{0.660232in}}%
\pgfpathcurveto{\pgfqpoint{4.916850in}{0.660232in}}{\pgfqpoint{4.927449in}{0.664622in}}{\pgfqpoint{4.935263in}{0.672436in}}%
\pgfpathcurveto{\pgfqpoint{4.943077in}{0.680249in}}{\pgfqpoint{4.947467in}{0.690848in}}{\pgfqpoint{4.947467in}{0.701899in}}%
\pgfpathcurveto{\pgfqpoint{4.947467in}{0.712949in}}{\pgfqpoint{4.943077in}{0.723548in}}{\pgfqpoint{4.935263in}{0.731361in}}%
\pgfpathcurveto{\pgfqpoint{4.927449in}{0.739175in}}{\pgfqpoint{4.916850in}{0.743565in}}{\pgfqpoint{4.905800in}{0.743565in}}%
\pgfpathcurveto{\pgfqpoint{4.894750in}{0.743565in}}{\pgfqpoint{4.884151in}{0.739175in}}{\pgfqpoint{4.876337in}{0.731361in}}%
\pgfpathcurveto{\pgfqpoint{4.868524in}{0.723548in}}{\pgfqpoint{4.864134in}{0.712949in}}{\pgfqpoint{4.864134in}{0.701899in}}%
\pgfpathcurveto{\pgfqpoint{4.864134in}{0.690848in}}{\pgfqpoint{4.868524in}{0.680249in}}{\pgfqpoint{4.876337in}{0.672436in}}%
\pgfpathcurveto{\pgfqpoint{4.884151in}{0.664622in}}{\pgfqpoint{4.894750in}{0.660232in}}{\pgfqpoint{4.905800in}{0.660232in}}%
\pgfpathlineto{\pgfqpoint{4.905800in}{0.660232in}}%
\pgfpathclose%
\pgfusepath{stroke}%
\end{pgfscope}%
\begin{pgfscope}%
\pgfpathrectangle{\pgfqpoint{0.847223in}{0.554012in}}{\pgfqpoint{6.200000in}{4.530000in}}%
\pgfusepath{clip}%
\pgfsetbuttcap%
\pgfsetroundjoin%
\pgfsetlinewidth{1.003750pt}%
\definecolor{currentstroke}{rgb}{1.000000,0.000000,0.000000}%
\pgfsetstrokecolor{currentstroke}%
\pgfsetdash{}{0pt}%
\pgfpathmoveto{\pgfqpoint{4.911133in}{0.659575in}}%
\pgfpathcurveto{\pgfqpoint{4.922184in}{0.659575in}}{\pgfqpoint{4.932783in}{0.663965in}}{\pgfqpoint{4.940596in}{0.671779in}}%
\pgfpathcurveto{\pgfqpoint{4.948410in}{0.679592in}}{\pgfqpoint{4.952800in}{0.690191in}}{\pgfqpoint{4.952800in}{0.701242in}}%
\pgfpathcurveto{\pgfqpoint{4.952800in}{0.712292in}}{\pgfqpoint{4.948410in}{0.722891in}}{\pgfqpoint{4.940596in}{0.730704in}}%
\pgfpathcurveto{\pgfqpoint{4.932783in}{0.738518in}}{\pgfqpoint{4.922184in}{0.742908in}}{\pgfqpoint{4.911133in}{0.742908in}}%
\pgfpathcurveto{\pgfqpoint{4.900083in}{0.742908in}}{\pgfqpoint{4.889484in}{0.738518in}}{\pgfqpoint{4.881671in}{0.730704in}}%
\pgfpathcurveto{\pgfqpoint{4.873857in}{0.722891in}}{\pgfqpoint{4.869467in}{0.712292in}}{\pgfqpoint{4.869467in}{0.701242in}}%
\pgfpathcurveto{\pgfqpoint{4.869467in}{0.690191in}}{\pgfqpoint{4.873857in}{0.679592in}}{\pgfqpoint{4.881671in}{0.671779in}}%
\pgfpathcurveto{\pgfqpoint{4.889484in}{0.663965in}}{\pgfqpoint{4.900083in}{0.659575in}}{\pgfqpoint{4.911133in}{0.659575in}}%
\pgfpathlineto{\pgfqpoint{4.911133in}{0.659575in}}%
\pgfpathclose%
\pgfusepath{stroke}%
\end{pgfscope}%
\begin{pgfscope}%
\pgfpathrectangle{\pgfqpoint{0.847223in}{0.554012in}}{\pgfqpoint{6.200000in}{4.530000in}}%
\pgfusepath{clip}%
\pgfsetbuttcap%
\pgfsetroundjoin%
\pgfsetlinewidth{1.003750pt}%
\definecolor{currentstroke}{rgb}{1.000000,0.000000,0.000000}%
\pgfsetstrokecolor{currentstroke}%
\pgfsetdash{}{0pt}%
\pgfpathmoveto{\pgfqpoint{4.916467in}{0.658920in}}%
\pgfpathcurveto{\pgfqpoint{4.927517in}{0.658920in}}{\pgfqpoint{4.938116in}{0.663310in}}{\pgfqpoint{4.945929in}{0.671123in}}%
\pgfpathcurveto{\pgfqpoint{4.953743in}{0.678937in}}{\pgfqpoint{4.958133in}{0.689536in}}{\pgfqpoint{4.958133in}{0.700586in}}%
\pgfpathcurveto{\pgfqpoint{4.958133in}{0.711636in}}{\pgfqpoint{4.953743in}{0.722235in}}{\pgfqpoint{4.945929in}{0.730049in}}%
\pgfpathcurveto{\pgfqpoint{4.938116in}{0.737863in}}{\pgfqpoint{4.927517in}{0.742253in}}{\pgfqpoint{4.916467in}{0.742253in}}%
\pgfpathcurveto{\pgfqpoint{4.905416in}{0.742253in}}{\pgfqpoint{4.894817in}{0.737863in}}{\pgfqpoint{4.887004in}{0.730049in}}%
\pgfpathcurveto{\pgfqpoint{4.879190in}{0.722235in}}{\pgfqpoint{4.874800in}{0.711636in}}{\pgfqpoint{4.874800in}{0.700586in}}%
\pgfpathcurveto{\pgfqpoint{4.874800in}{0.689536in}}{\pgfqpoint{4.879190in}{0.678937in}}{\pgfqpoint{4.887004in}{0.671123in}}%
\pgfpathcurveto{\pgfqpoint{4.894817in}{0.663310in}}{\pgfqpoint{4.905416in}{0.658920in}}{\pgfqpoint{4.916467in}{0.658920in}}%
\pgfpathlineto{\pgfqpoint{4.916467in}{0.658920in}}%
\pgfpathclose%
\pgfusepath{stroke}%
\end{pgfscope}%
\begin{pgfscope}%
\pgfpathrectangle{\pgfqpoint{0.847223in}{0.554012in}}{\pgfqpoint{6.200000in}{4.530000in}}%
\pgfusepath{clip}%
\pgfsetbuttcap%
\pgfsetroundjoin%
\pgfsetlinewidth{1.003750pt}%
\definecolor{currentstroke}{rgb}{1.000000,0.000000,0.000000}%
\pgfsetstrokecolor{currentstroke}%
\pgfsetdash{}{0pt}%
\pgfpathmoveto{\pgfqpoint{4.921800in}{0.658266in}}%
\pgfpathcurveto{\pgfqpoint{4.932850in}{0.658266in}}{\pgfqpoint{4.943449in}{0.662656in}}{\pgfqpoint{4.951263in}{0.670469in}}%
\pgfpathcurveto{\pgfqpoint{4.959076in}{0.678283in}}{\pgfqpoint{4.963466in}{0.688882in}}{\pgfqpoint{4.963466in}{0.699932in}}%
\pgfpathcurveto{\pgfqpoint{4.963466in}{0.710982in}}{\pgfqpoint{4.959076in}{0.721581in}}{\pgfqpoint{4.951263in}{0.729395in}}%
\pgfpathcurveto{\pgfqpoint{4.943449in}{0.737209in}}{\pgfqpoint{4.932850in}{0.741599in}}{\pgfqpoint{4.921800in}{0.741599in}}%
\pgfpathcurveto{\pgfqpoint{4.910750in}{0.741599in}}{\pgfqpoint{4.900151in}{0.737209in}}{\pgfqpoint{4.892337in}{0.729395in}}%
\pgfpathcurveto{\pgfqpoint{4.884523in}{0.721581in}}{\pgfqpoint{4.880133in}{0.710982in}}{\pgfqpoint{4.880133in}{0.699932in}}%
\pgfpathcurveto{\pgfqpoint{4.880133in}{0.688882in}}{\pgfqpoint{4.884523in}{0.678283in}}{\pgfqpoint{4.892337in}{0.670469in}}%
\pgfpathcurveto{\pgfqpoint{4.900151in}{0.662656in}}{\pgfqpoint{4.910750in}{0.658266in}}{\pgfqpoint{4.921800in}{0.658266in}}%
\pgfpathlineto{\pgfqpoint{4.921800in}{0.658266in}}%
\pgfpathclose%
\pgfusepath{stroke}%
\end{pgfscope}%
\begin{pgfscope}%
\pgfpathrectangle{\pgfqpoint{0.847223in}{0.554012in}}{\pgfqpoint{6.200000in}{4.530000in}}%
\pgfusepath{clip}%
\pgfsetbuttcap%
\pgfsetroundjoin%
\pgfsetlinewidth{1.003750pt}%
\definecolor{currentstroke}{rgb}{1.000000,0.000000,0.000000}%
\pgfsetstrokecolor{currentstroke}%
\pgfsetdash{}{0pt}%
\pgfpathmoveto{\pgfqpoint{4.927133in}{0.657613in}}%
\pgfpathcurveto{\pgfqpoint{4.938183in}{0.657613in}}{\pgfqpoint{4.948782in}{0.662003in}}{\pgfqpoint{4.956596in}{0.669817in}}%
\pgfpathcurveto{\pgfqpoint{4.964409in}{0.677631in}}{\pgfqpoint{4.968800in}{0.688230in}}{\pgfqpoint{4.968800in}{0.699280in}}%
\pgfpathcurveto{\pgfqpoint{4.968800in}{0.710330in}}{\pgfqpoint{4.964409in}{0.720929in}}{\pgfqpoint{4.956596in}{0.728743in}}%
\pgfpathcurveto{\pgfqpoint{4.948782in}{0.736556in}}{\pgfqpoint{4.938183in}{0.740947in}}{\pgfqpoint{4.927133in}{0.740947in}}%
\pgfpathcurveto{\pgfqpoint{4.916083in}{0.740947in}}{\pgfqpoint{4.905484in}{0.736556in}}{\pgfqpoint{4.897670in}{0.728743in}}%
\pgfpathcurveto{\pgfqpoint{4.889857in}{0.720929in}}{\pgfqpoint{4.885466in}{0.710330in}}{\pgfqpoint{4.885466in}{0.699280in}}%
\pgfpathcurveto{\pgfqpoint{4.885466in}{0.688230in}}{\pgfqpoint{4.889857in}{0.677631in}}{\pgfqpoint{4.897670in}{0.669817in}}%
\pgfpathcurveto{\pgfqpoint{4.905484in}{0.662003in}}{\pgfqpoint{4.916083in}{0.657613in}}{\pgfqpoint{4.927133in}{0.657613in}}%
\pgfpathlineto{\pgfqpoint{4.927133in}{0.657613in}}%
\pgfpathclose%
\pgfusepath{stroke}%
\end{pgfscope}%
\begin{pgfscope}%
\pgfpathrectangle{\pgfqpoint{0.847223in}{0.554012in}}{\pgfqpoint{6.200000in}{4.530000in}}%
\pgfusepath{clip}%
\pgfsetbuttcap%
\pgfsetroundjoin%
\pgfsetlinewidth{1.003750pt}%
\definecolor{currentstroke}{rgb}{1.000000,0.000000,0.000000}%
\pgfsetstrokecolor{currentstroke}%
\pgfsetdash{}{0pt}%
\pgfpathmoveto{\pgfqpoint{4.932466in}{0.656962in}}%
\pgfpathcurveto{\pgfqpoint{4.943516in}{0.656962in}}{\pgfqpoint{4.954115in}{0.661353in}}{\pgfqpoint{4.961929in}{0.669166in}}%
\pgfpathcurveto{\pgfqpoint{4.969743in}{0.676980in}}{\pgfqpoint{4.974133in}{0.687579in}}{\pgfqpoint{4.974133in}{0.698629in}}%
\pgfpathcurveto{\pgfqpoint{4.974133in}{0.709679in}}{\pgfqpoint{4.969743in}{0.720278in}}{\pgfqpoint{4.961929in}{0.728092in}}%
\pgfpathcurveto{\pgfqpoint{4.954115in}{0.735905in}}{\pgfqpoint{4.943516in}{0.740296in}}{\pgfqpoint{4.932466in}{0.740296in}}%
\pgfpathcurveto{\pgfqpoint{4.921416in}{0.740296in}}{\pgfqpoint{4.910817in}{0.735905in}}{\pgfqpoint{4.903003in}{0.728092in}}%
\pgfpathcurveto{\pgfqpoint{4.895190in}{0.720278in}}{\pgfqpoint{4.890800in}{0.709679in}}{\pgfqpoint{4.890800in}{0.698629in}}%
\pgfpathcurveto{\pgfqpoint{4.890800in}{0.687579in}}{\pgfqpoint{4.895190in}{0.676980in}}{\pgfqpoint{4.903003in}{0.669166in}}%
\pgfpathcurveto{\pgfqpoint{4.910817in}{0.661353in}}{\pgfqpoint{4.921416in}{0.656962in}}{\pgfqpoint{4.932466in}{0.656962in}}%
\pgfpathlineto{\pgfqpoint{4.932466in}{0.656962in}}%
\pgfpathclose%
\pgfusepath{stroke}%
\end{pgfscope}%
\begin{pgfscope}%
\pgfpathrectangle{\pgfqpoint{0.847223in}{0.554012in}}{\pgfqpoint{6.200000in}{4.530000in}}%
\pgfusepath{clip}%
\pgfsetbuttcap%
\pgfsetroundjoin%
\pgfsetlinewidth{1.003750pt}%
\definecolor{currentstroke}{rgb}{1.000000,0.000000,0.000000}%
\pgfsetstrokecolor{currentstroke}%
\pgfsetdash{}{0pt}%
\pgfpathmoveto{\pgfqpoint{4.937799in}{0.656313in}}%
\pgfpathcurveto{\pgfqpoint{4.948850in}{0.656313in}}{\pgfqpoint{4.959449in}{0.660703in}}{\pgfqpoint{4.967262in}{0.668517in}}%
\pgfpathcurveto{\pgfqpoint{4.975076in}{0.676330in}}{\pgfqpoint{4.979466in}{0.686929in}}{\pgfqpoint{4.979466in}{0.697980in}}%
\pgfpathcurveto{\pgfqpoint{4.979466in}{0.709030in}}{\pgfqpoint{4.975076in}{0.719629in}}{\pgfqpoint{4.967262in}{0.727442in}}%
\pgfpathcurveto{\pgfqpoint{4.959449in}{0.735256in}}{\pgfqpoint{4.948850in}{0.739646in}}{\pgfqpoint{4.937799in}{0.739646in}}%
\pgfpathcurveto{\pgfqpoint{4.926749in}{0.739646in}}{\pgfqpoint{4.916150in}{0.735256in}}{\pgfqpoint{4.908337in}{0.727442in}}%
\pgfpathcurveto{\pgfqpoint{4.900523in}{0.719629in}}{\pgfqpoint{4.896133in}{0.709030in}}{\pgfqpoint{4.896133in}{0.697980in}}%
\pgfpathcurveto{\pgfqpoint{4.896133in}{0.686929in}}{\pgfqpoint{4.900523in}{0.676330in}}{\pgfqpoint{4.908337in}{0.668517in}}%
\pgfpathcurveto{\pgfqpoint{4.916150in}{0.660703in}}{\pgfqpoint{4.926749in}{0.656313in}}{\pgfqpoint{4.937799in}{0.656313in}}%
\pgfpathlineto{\pgfqpoint{4.937799in}{0.656313in}}%
\pgfpathclose%
\pgfusepath{stroke}%
\end{pgfscope}%
\begin{pgfscope}%
\pgfpathrectangle{\pgfqpoint{0.847223in}{0.554012in}}{\pgfqpoint{6.200000in}{4.530000in}}%
\pgfusepath{clip}%
\pgfsetbuttcap%
\pgfsetroundjoin%
\pgfsetlinewidth{1.003750pt}%
\definecolor{currentstroke}{rgb}{1.000000,0.000000,0.000000}%
\pgfsetstrokecolor{currentstroke}%
\pgfsetdash{}{0pt}%
\pgfpathmoveto{\pgfqpoint{4.943133in}{0.655665in}}%
\pgfpathcurveto{\pgfqpoint{4.954183in}{0.655665in}}{\pgfqpoint{4.964782in}{0.660055in}}{\pgfqpoint{4.972595in}{0.667869in}}%
\pgfpathcurveto{\pgfqpoint{4.980409in}{0.675683in}}{\pgfqpoint{4.984799in}{0.686282in}}{\pgfqpoint{4.984799in}{0.697332in}}%
\pgfpathcurveto{\pgfqpoint{4.984799in}{0.708382in}}{\pgfqpoint{4.980409in}{0.718981in}}{\pgfqpoint{4.972595in}{0.726794in}}%
\pgfpathcurveto{\pgfqpoint{4.964782in}{0.734608in}}{\pgfqpoint{4.954183in}{0.738998in}}{\pgfqpoint{4.943133in}{0.738998in}}%
\pgfpathcurveto{\pgfqpoint{4.932083in}{0.738998in}}{\pgfqpoint{4.921484in}{0.734608in}}{\pgfqpoint{4.913670in}{0.726794in}}%
\pgfpathcurveto{\pgfqpoint{4.905856in}{0.718981in}}{\pgfqpoint{4.901466in}{0.708382in}}{\pgfqpoint{4.901466in}{0.697332in}}%
\pgfpathcurveto{\pgfqpoint{4.901466in}{0.686282in}}{\pgfqpoint{4.905856in}{0.675683in}}{\pgfqpoint{4.913670in}{0.667869in}}%
\pgfpathcurveto{\pgfqpoint{4.921484in}{0.660055in}}{\pgfqpoint{4.932083in}{0.655665in}}{\pgfqpoint{4.943133in}{0.655665in}}%
\pgfpathlineto{\pgfqpoint{4.943133in}{0.655665in}}%
\pgfpathclose%
\pgfusepath{stroke}%
\end{pgfscope}%
\begin{pgfscope}%
\pgfpathrectangle{\pgfqpoint{0.847223in}{0.554012in}}{\pgfqpoint{6.200000in}{4.530000in}}%
\pgfusepath{clip}%
\pgfsetbuttcap%
\pgfsetroundjoin%
\pgfsetlinewidth{1.003750pt}%
\definecolor{currentstroke}{rgb}{1.000000,0.000000,0.000000}%
\pgfsetstrokecolor{currentstroke}%
\pgfsetdash{}{0pt}%
\pgfpathmoveto{\pgfqpoint{4.948466in}{0.655019in}}%
\pgfpathcurveto{\pgfqpoint{4.959516in}{0.655019in}}{\pgfqpoint{4.970115in}{0.659409in}}{\pgfqpoint{4.977929in}{0.667223in}}%
\pgfpathcurveto{\pgfqpoint{4.985742in}{0.675036in}}{\pgfqpoint{4.990133in}{0.685635in}}{\pgfqpoint{4.990133in}{0.696685in}}%
\pgfpathcurveto{\pgfqpoint{4.990133in}{0.707735in}}{\pgfqpoint{4.985742in}{0.718334in}}{\pgfqpoint{4.977929in}{0.726148in}}%
\pgfpathcurveto{\pgfqpoint{4.970115in}{0.733962in}}{\pgfqpoint{4.959516in}{0.738352in}}{\pgfqpoint{4.948466in}{0.738352in}}%
\pgfpathcurveto{\pgfqpoint{4.937416in}{0.738352in}}{\pgfqpoint{4.926817in}{0.733962in}}{\pgfqpoint{4.919003in}{0.726148in}}%
\pgfpathcurveto{\pgfqpoint{4.911189in}{0.718334in}}{\pgfqpoint{4.906799in}{0.707735in}}{\pgfqpoint{4.906799in}{0.696685in}}%
\pgfpathcurveto{\pgfqpoint{4.906799in}{0.685635in}}{\pgfqpoint{4.911189in}{0.675036in}}{\pgfqpoint{4.919003in}{0.667223in}}%
\pgfpathcurveto{\pgfqpoint{4.926817in}{0.659409in}}{\pgfqpoint{4.937416in}{0.655019in}}{\pgfqpoint{4.948466in}{0.655019in}}%
\pgfpathlineto{\pgfqpoint{4.948466in}{0.655019in}}%
\pgfpathclose%
\pgfusepath{stroke}%
\end{pgfscope}%
\begin{pgfscope}%
\pgfpathrectangle{\pgfqpoint{0.847223in}{0.554012in}}{\pgfqpoint{6.200000in}{4.530000in}}%
\pgfusepath{clip}%
\pgfsetbuttcap%
\pgfsetroundjoin%
\pgfsetlinewidth{1.003750pt}%
\definecolor{currentstroke}{rgb}{1.000000,0.000000,0.000000}%
\pgfsetstrokecolor{currentstroke}%
\pgfsetdash{}{0pt}%
\pgfpathmoveto{\pgfqpoint{4.953799in}{0.654374in}}%
\pgfpathcurveto{\pgfqpoint{4.964849in}{0.654374in}}{\pgfqpoint{4.975448in}{0.658764in}}{\pgfqpoint{4.983262in}{0.666578in}}%
\pgfpathcurveto{\pgfqpoint{4.991076in}{0.674391in}}{\pgfqpoint{4.995466in}{0.684990in}}{\pgfqpoint{4.995466in}{0.696040in}}%
\pgfpathcurveto{\pgfqpoint{4.995466in}{0.707091in}}{\pgfqpoint{4.991076in}{0.717690in}}{\pgfqpoint{4.983262in}{0.725503in}}%
\pgfpathcurveto{\pgfqpoint{4.975448in}{0.733317in}}{\pgfqpoint{4.964849in}{0.737707in}}{\pgfqpoint{4.953799in}{0.737707in}}%
\pgfpathcurveto{\pgfqpoint{4.942749in}{0.737707in}}{\pgfqpoint{4.932150in}{0.733317in}}{\pgfqpoint{4.924336in}{0.725503in}}%
\pgfpathcurveto{\pgfqpoint{4.916523in}{0.717690in}}{\pgfqpoint{4.912132in}{0.707091in}}{\pgfqpoint{4.912132in}{0.696040in}}%
\pgfpathcurveto{\pgfqpoint{4.912132in}{0.684990in}}{\pgfqpoint{4.916523in}{0.674391in}}{\pgfqpoint{4.924336in}{0.666578in}}%
\pgfpathcurveto{\pgfqpoint{4.932150in}{0.658764in}}{\pgfqpoint{4.942749in}{0.654374in}}{\pgfqpoint{4.953799in}{0.654374in}}%
\pgfpathlineto{\pgfqpoint{4.953799in}{0.654374in}}%
\pgfpathclose%
\pgfusepath{stroke}%
\end{pgfscope}%
\begin{pgfscope}%
\pgfpathrectangle{\pgfqpoint{0.847223in}{0.554012in}}{\pgfqpoint{6.200000in}{4.530000in}}%
\pgfusepath{clip}%
\pgfsetbuttcap%
\pgfsetroundjoin%
\pgfsetlinewidth{1.003750pt}%
\definecolor{currentstroke}{rgb}{1.000000,0.000000,0.000000}%
\pgfsetstrokecolor{currentstroke}%
\pgfsetdash{}{0pt}%
\pgfpathmoveto{\pgfqpoint{4.959132in}{0.653730in}}%
\pgfpathcurveto{\pgfqpoint{4.970182in}{0.653730in}}{\pgfqpoint{4.980781in}{0.658121in}}{\pgfqpoint{4.988595in}{0.665934in}}%
\pgfpathcurveto{\pgfqpoint{4.996409in}{0.673748in}}{\pgfqpoint{5.000799in}{0.684347in}}{\pgfqpoint{5.000799in}{0.695397in}}%
\pgfpathcurveto{\pgfqpoint{5.000799in}{0.706447in}}{\pgfqpoint{4.996409in}{0.717046in}}{\pgfqpoint{4.988595in}{0.724860in}}%
\pgfpathcurveto{\pgfqpoint{4.980781in}{0.732673in}}{\pgfqpoint{4.970182in}{0.737064in}}{\pgfqpoint{4.959132in}{0.737064in}}%
\pgfpathcurveto{\pgfqpoint{4.948082in}{0.737064in}}{\pgfqpoint{4.937483in}{0.732673in}}{\pgfqpoint{4.929670in}{0.724860in}}%
\pgfpathcurveto{\pgfqpoint{4.921856in}{0.717046in}}{\pgfqpoint{4.917466in}{0.706447in}}{\pgfqpoint{4.917466in}{0.695397in}}%
\pgfpathcurveto{\pgfqpoint{4.917466in}{0.684347in}}{\pgfqpoint{4.921856in}{0.673748in}}{\pgfqpoint{4.929670in}{0.665934in}}%
\pgfpathcurveto{\pgfqpoint{4.937483in}{0.658121in}}{\pgfqpoint{4.948082in}{0.653730in}}{\pgfqpoint{4.959132in}{0.653730in}}%
\pgfpathlineto{\pgfqpoint{4.959132in}{0.653730in}}%
\pgfpathclose%
\pgfusepath{stroke}%
\end{pgfscope}%
\begin{pgfscope}%
\pgfpathrectangle{\pgfqpoint{0.847223in}{0.554012in}}{\pgfqpoint{6.200000in}{4.530000in}}%
\pgfusepath{clip}%
\pgfsetbuttcap%
\pgfsetroundjoin%
\pgfsetlinewidth{1.003750pt}%
\definecolor{currentstroke}{rgb}{1.000000,0.000000,0.000000}%
\pgfsetstrokecolor{currentstroke}%
\pgfsetdash{}{0pt}%
\pgfpathmoveto{\pgfqpoint{4.964466in}{0.653088in}}%
\pgfpathcurveto{\pgfqpoint{4.975516in}{0.653088in}}{\pgfqpoint{4.986115in}{0.657479in}}{\pgfqpoint{4.993928in}{0.665292in}}%
\pgfpathcurveto{\pgfqpoint{5.001742in}{0.673106in}}{\pgfqpoint{5.006132in}{0.683705in}}{\pgfqpoint{5.006132in}{0.694755in}}%
\pgfpathcurveto{\pgfqpoint{5.006132in}{0.705805in}}{\pgfqpoint{5.001742in}{0.716404in}}{\pgfqpoint{4.993928in}{0.724218in}}%
\pgfpathcurveto{\pgfqpoint{4.986115in}{0.732031in}}{\pgfqpoint{4.975516in}{0.736422in}}{\pgfqpoint{4.964466in}{0.736422in}}%
\pgfpathcurveto{\pgfqpoint{4.953415in}{0.736422in}}{\pgfqpoint{4.942816in}{0.732031in}}{\pgfqpoint{4.935003in}{0.724218in}}%
\pgfpathcurveto{\pgfqpoint{4.927189in}{0.716404in}}{\pgfqpoint{4.922799in}{0.705805in}}{\pgfqpoint{4.922799in}{0.694755in}}%
\pgfpathcurveto{\pgfqpoint{4.922799in}{0.683705in}}{\pgfqpoint{4.927189in}{0.673106in}}{\pgfqpoint{4.935003in}{0.665292in}}%
\pgfpathcurveto{\pgfqpoint{4.942816in}{0.657479in}}{\pgfqpoint{4.953415in}{0.653088in}}{\pgfqpoint{4.964466in}{0.653088in}}%
\pgfpathlineto{\pgfqpoint{4.964466in}{0.653088in}}%
\pgfpathclose%
\pgfusepath{stroke}%
\end{pgfscope}%
\begin{pgfscope}%
\pgfpathrectangle{\pgfqpoint{0.847223in}{0.554012in}}{\pgfqpoint{6.200000in}{4.530000in}}%
\pgfusepath{clip}%
\pgfsetbuttcap%
\pgfsetroundjoin%
\pgfsetlinewidth{1.003750pt}%
\definecolor{currentstroke}{rgb}{1.000000,0.000000,0.000000}%
\pgfsetstrokecolor{currentstroke}%
\pgfsetdash{}{0pt}%
\pgfpathmoveto{\pgfqpoint{4.969799in}{0.652448in}}%
\pgfpathcurveto{\pgfqpoint{4.980849in}{0.652448in}}{\pgfqpoint{4.991448in}{0.656838in}}{\pgfqpoint{4.999262in}{0.664652in}}%
\pgfpathcurveto{\pgfqpoint{5.007075in}{0.672465in}}{\pgfqpoint{5.011465in}{0.683064in}}{\pgfqpoint{5.011465in}{0.694115in}}%
\pgfpathcurveto{\pgfqpoint{5.011465in}{0.705165in}}{\pgfqpoint{5.007075in}{0.715764in}}{\pgfqpoint{4.999262in}{0.723577in}}%
\pgfpathcurveto{\pgfqpoint{4.991448in}{0.731391in}}{\pgfqpoint{4.980849in}{0.735781in}}{\pgfqpoint{4.969799in}{0.735781in}}%
\pgfpathcurveto{\pgfqpoint{4.958749in}{0.735781in}}{\pgfqpoint{4.948150in}{0.731391in}}{\pgfqpoint{4.940336in}{0.723577in}}%
\pgfpathcurveto{\pgfqpoint{4.932522in}{0.715764in}}{\pgfqpoint{4.928132in}{0.705165in}}{\pgfqpoint{4.928132in}{0.694115in}}%
\pgfpathcurveto{\pgfqpoint{4.928132in}{0.683064in}}{\pgfqpoint{4.932522in}{0.672465in}}{\pgfqpoint{4.940336in}{0.664652in}}%
\pgfpathcurveto{\pgfqpoint{4.948150in}{0.656838in}}{\pgfqpoint{4.958749in}{0.652448in}}{\pgfqpoint{4.969799in}{0.652448in}}%
\pgfpathlineto{\pgfqpoint{4.969799in}{0.652448in}}%
\pgfpathclose%
\pgfusepath{stroke}%
\end{pgfscope}%
\begin{pgfscope}%
\pgfpathrectangle{\pgfqpoint{0.847223in}{0.554012in}}{\pgfqpoint{6.200000in}{4.530000in}}%
\pgfusepath{clip}%
\pgfsetbuttcap%
\pgfsetroundjoin%
\pgfsetlinewidth{1.003750pt}%
\definecolor{currentstroke}{rgb}{1.000000,0.000000,0.000000}%
\pgfsetstrokecolor{currentstroke}%
\pgfsetdash{}{0pt}%
\pgfpathmoveto{\pgfqpoint{4.975132in}{0.651809in}}%
\pgfpathcurveto{\pgfqpoint{4.986182in}{0.651809in}}{\pgfqpoint{4.996781in}{0.656199in}}{\pgfqpoint{5.004595in}{0.664013in}}%
\pgfpathcurveto{\pgfqpoint{5.012408in}{0.671826in}}{\pgfqpoint{5.016799in}{0.682425in}}{\pgfqpoint{5.016799in}{0.693476in}}%
\pgfpathcurveto{\pgfqpoint{5.016799in}{0.704526in}}{\pgfqpoint{5.012408in}{0.715125in}}{\pgfqpoint{5.004595in}{0.722938in}}%
\pgfpathcurveto{\pgfqpoint{4.996781in}{0.730752in}}{\pgfqpoint{4.986182in}{0.735142in}}{\pgfqpoint{4.975132in}{0.735142in}}%
\pgfpathcurveto{\pgfqpoint{4.964082in}{0.735142in}}{\pgfqpoint{4.953483in}{0.730752in}}{\pgfqpoint{4.945669in}{0.722938in}}%
\pgfpathcurveto{\pgfqpoint{4.937856in}{0.715125in}}{\pgfqpoint{4.933465in}{0.704526in}}{\pgfqpoint{4.933465in}{0.693476in}}%
\pgfpathcurveto{\pgfqpoint{4.933465in}{0.682425in}}{\pgfqpoint{4.937856in}{0.671826in}}{\pgfqpoint{4.945669in}{0.664013in}}%
\pgfpathcurveto{\pgfqpoint{4.953483in}{0.656199in}}{\pgfqpoint{4.964082in}{0.651809in}}{\pgfqpoint{4.975132in}{0.651809in}}%
\pgfpathlineto{\pgfqpoint{4.975132in}{0.651809in}}%
\pgfpathclose%
\pgfusepath{stroke}%
\end{pgfscope}%
\begin{pgfscope}%
\pgfpathrectangle{\pgfqpoint{0.847223in}{0.554012in}}{\pgfqpoint{6.200000in}{4.530000in}}%
\pgfusepath{clip}%
\pgfsetbuttcap%
\pgfsetroundjoin%
\pgfsetlinewidth{1.003750pt}%
\definecolor{currentstroke}{rgb}{1.000000,0.000000,0.000000}%
\pgfsetstrokecolor{currentstroke}%
\pgfsetdash{}{0pt}%
\pgfpathmoveto{\pgfqpoint{4.980465in}{0.651171in}}%
\pgfpathcurveto{\pgfqpoint{4.991515in}{0.651171in}}{\pgfqpoint{5.002114in}{0.655562in}}{\pgfqpoint{5.009928in}{0.663375in}}%
\pgfpathcurveto{\pgfqpoint{5.017742in}{0.671189in}}{\pgfqpoint{5.022132in}{0.681788in}}{\pgfqpoint{5.022132in}{0.692838in}}%
\pgfpathcurveto{\pgfqpoint{5.022132in}{0.703888in}}{\pgfqpoint{5.017742in}{0.714487in}}{\pgfqpoint{5.009928in}{0.722301in}}%
\pgfpathcurveto{\pgfqpoint{5.002114in}{0.730114in}}{\pgfqpoint{4.991515in}{0.734505in}}{\pgfqpoint{4.980465in}{0.734505in}}%
\pgfpathcurveto{\pgfqpoint{4.969415in}{0.734505in}}{\pgfqpoint{4.958816in}{0.730114in}}{\pgfqpoint{4.951002in}{0.722301in}}%
\pgfpathcurveto{\pgfqpoint{4.943189in}{0.714487in}}{\pgfqpoint{4.938799in}{0.703888in}}{\pgfqpoint{4.938799in}{0.692838in}}%
\pgfpathcurveto{\pgfqpoint{4.938799in}{0.681788in}}{\pgfqpoint{4.943189in}{0.671189in}}{\pgfqpoint{4.951002in}{0.663375in}}%
\pgfpathcurveto{\pgfqpoint{4.958816in}{0.655562in}}{\pgfqpoint{4.969415in}{0.651171in}}{\pgfqpoint{4.980465in}{0.651171in}}%
\pgfpathlineto{\pgfqpoint{4.980465in}{0.651171in}}%
\pgfpathclose%
\pgfusepath{stroke}%
\end{pgfscope}%
\begin{pgfscope}%
\pgfpathrectangle{\pgfqpoint{0.847223in}{0.554012in}}{\pgfqpoint{6.200000in}{4.530000in}}%
\pgfusepath{clip}%
\pgfsetbuttcap%
\pgfsetroundjoin%
\pgfsetlinewidth{1.003750pt}%
\definecolor{currentstroke}{rgb}{1.000000,0.000000,0.000000}%
\pgfsetstrokecolor{currentstroke}%
\pgfsetdash{}{0pt}%
\pgfpathmoveto{\pgfqpoint{4.985798in}{0.650535in}}%
\pgfpathcurveto{\pgfqpoint{4.996849in}{0.650535in}}{\pgfqpoint{5.007448in}{0.654926in}}{\pgfqpoint{5.015261in}{0.662739in}}%
\pgfpathcurveto{\pgfqpoint{5.023075in}{0.670553in}}{\pgfqpoint{5.027465in}{0.681152in}}{\pgfqpoint{5.027465in}{0.692202in}}%
\pgfpathcurveto{\pgfqpoint{5.027465in}{0.703252in}}{\pgfqpoint{5.023075in}{0.713851in}}{\pgfqpoint{5.015261in}{0.721665in}}%
\pgfpathcurveto{\pgfqpoint{5.007448in}{0.729478in}}{\pgfqpoint{4.996849in}{0.733869in}}{\pgfqpoint{4.985798in}{0.733869in}}%
\pgfpathcurveto{\pgfqpoint{4.974748in}{0.733869in}}{\pgfqpoint{4.964149in}{0.729478in}}{\pgfqpoint{4.956336in}{0.721665in}}%
\pgfpathcurveto{\pgfqpoint{4.948522in}{0.713851in}}{\pgfqpoint{4.944132in}{0.703252in}}{\pgfqpoint{4.944132in}{0.692202in}}%
\pgfpathcurveto{\pgfqpoint{4.944132in}{0.681152in}}{\pgfqpoint{4.948522in}{0.670553in}}{\pgfqpoint{4.956336in}{0.662739in}}%
\pgfpathcurveto{\pgfqpoint{4.964149in}{0.654926in}}{\pgfqpoint{4.974748in}{0.650535in}}{\pgfqpoint{4.985798in}{0.650535in}}%
\pgfpathlineto{\pgfqpoint{4.985798in}{0.650535in}}%
\pgfpathclose%
\pgfusepath{stroke}%
\end{pgfscope}%
\begin{pgfscope}%
\pgfpathrectangle{\pgfqpoint{0.847223in}{0.554012in}}{\pgfqpoint{6.200000in}{4.530000in}}%
\pgfusepath{clip}%
\pgfsetbuttcap%
\pgfsetroundjoin%
\pgfsetlinewidth{1.003750pt}%
\definecolor{currentstroke}{rgb}{1.000000,0.000000,0.000000}%
\pgfsetstrokecolor{currentstroke}%
\pgfsetdash{}{0pt}%
\pgfpathmoveto{\pgfqpoint{4.991132in}{0.649901in}}%
\pgfpathcurveto{\pgfqpoint{5.002182in}{0.649901in}}{\pgfqpoint{5.012781in}{0.654291in}}{\pgfqpoint{5.020594in}{0.662105in}}%
\pgfpathcurveto{\pgfqpoint{5.028408in}{0.669918in}}{\pgfqpoint{5.032798in}{0.680517in}}{\pgfqpoint{5.032798in}{0.691567in}}%
\pgfpathcurveto{\pgfqpoint{5.032798in}{0.702618in}}{\pgfqpoint{5.028408in}{0.713217in}}{\pgfqpoint{5.020594in}{0.721030in}}%
\pgfpathcurveto{\pgfqpoint{5.012781in}{0.728844in}}{\pgfqpoint{5.002182in}{0.733234in}}{\pgfqpoint{4.991132in}{0.733234in}}%
\pgfpathcurveto{\pgfqpoint{4.980081in}{0.733234in}}{\pgfqpoint{4.969482in}{0.728844in}}{\pgfqpoint{4.961669in}{0.721030in}}%
\pgfpathcurveto{\pgfqpoint{4.953855in}{0.713217in}}{\pgfqpoint{4.949465in}{0.702618in}}{\pgfqpoint{4.949465in}{0.691567in}}%
\pgfpathcurveto{\pgfqpoint{4.949465in}{0.680517in}}{\pgfqpoint{4.953855in}{0.669918in}}{\pgfqpoint{4.961669in}{0.662105in}}%
\pgfpathcurveto{\pgfqpoint{4.969482in}{0.654291in}}{\pgfqpoint{4.980081in}{0.649901in}}{\pgfqpoint{4.991132in}{0.649901in}}%
\pgfpathlineto{\pgfqpoint{4.991132in}{0.649901in}}%
\pgfpathclose%
\pgfusepath{stroke}%
\end{pgfscope}%
\begin{pgfscope}%
\pgfpathrectangle{\pgfqpoint{0.847223in}{0.554012in}}{\pgfqpoint{6.200000in}{4.530000in}}%
\pgfusepath{clip}%
\pgfsetbuttcap%
\pgfsetroundjoin%
\pgfsetlinewidth{1.003750pt}%
\definecolor{currentstroke}{rgb}{1.000000,0.000000,0.000000}%
\pgfsetstrokecolor{currentstroke}%
\pgfsetdash{}{0pt}%
\pgfpathmoveto{\pgfqpoint{4.996465in}{0.649268in}}%
\pgfpathcurveto{\pgfqpoint{5.007515in}{0.649268in}}{\pgfqpoint{5.018114in}{0.653658in}}{\pgfqpoint{5.025928in}{0.661471in}}%
\pgfpathcurveto{\pgfqpoint{5.033741in}{0.669285in}}{\pgfqpoint{5.038132in}{0.679884in}}{\pgfqpoint{5.038132in}{0.690934in}}%
\pgfpathcurveto{\pgfqpoint{5.038132in}{0.701984in}}{\pgfqpoint{5.033741in}{0.712583in}}{\pgfqpoint{5.025928in}{0.720397in}}%
\pgfpathcurveto{\pgfqpoint{5.018114in}{0.728211in}}{\pgfqpoint{5.007515in}{0.732601in}}{\pgfqpoint{4.996465in}{0.732601in}}%
\pgfpathcurveto{\pgfqpoint{4.985415in}{0.732601in}}{\pgfqpoint{4.974816in}{0.728211in}}{\pgfqpoint{4.967002in}{0.720397in}}%
\pgfpathcurveto{\pgfqpoint{4.959188in}{0.712583in}}{\pgfqpoint{4.954798in}{0.701984in}}{\pgfqpoint{4.954798in}{0.690934in}}%
\pgfpathcurveto{\pgfqpoint{4.954798in}{0.679884in}}{\pgfqpoint{4.959188in}{0.669285in}}{\pgfqpoint{4.967002in}{0.661471in}}%
\pgfpathcurveto{\pgfqpoint{4.974816in}{0.653658in}}{\pgfqpoint{4.985415in}{0.649268in}}{\pgfqpoint{4.996465in}{0.649268in}}%
\pgfpathlineto{\pgfqpoint{4.996465in}{0.649268in}}%
\pgfpathclose%
\pgfusepath{stroke}%
\end{pgfscope}%
\begin{pgfscope}%
\pgfpathrectangle{\pgfqpoint{0.847223in}{0.554012in}}{\pgfqpoint{6.200000in}{4.530000in}}%
\pgfusepath{clip}%
\pgfsetbuttcap%
\pgfsetroundjoin%
\pgfsetlinewidth{1.003750pt}%
\definecolor{currentstroke}{rgb}{1.000000,0.000000,0.000000}%
\pgfsetstrokecolor{currentstroke}%
\pgfsetdash{}{0pt}%
\pgfpathmoveto{\pgfqpoint{5.001798in}{0.648636in}}%
\pgfpathcurveto{\pgfqpoint{5.012848in}{0.648636in}}{\pgfqpoint{5.023447in}{0.653026in}}{\pgfqpoint{5.031261in}{0.660840in}}%
\pgfpathcurveto{\pgfqpoint{5.039074in}{0.668653in}}{\pgfqpoint{5.043465in}{0.679252in}}{\pgfqpoint{5.043465in}{0.690303in}}%
\pgfpathcurveto{\pgfqpoint{5.043465in}{0.701353in}}{\pgfqpoint{5.039074in}{0.711952in}}{\pgfqpoint{5.031261in}{0.719765in}}%
\pgfpathcurveto{\pgfqpoint{5.023447in}{0.727579in}}{\pgfqpoint{5.012848in}{0.731969in}}{\pgfqpoint{5.001798in}{0.731969in}}%
\pgfpathcurveto{\pgfqpoint{4.990748in}{0.731969in}}{\pgfqpoint{4.980149in}{0.727579in}}{\pgfqpoint{4.972335in}{0.719765in}}%
\pgfpathcurveto{\pgfqpoint{4.964522in}{0.711952in}}{\pgfqpoint{4.960131in}{0.701353in}}{\pgfqpoint{4.960131in}{0.690303in}}%
\pgfpathcurveto{\pgfqpoint{4.960131in}{0.679252in}}{\pgfqpoint{4.964522in}{0.668653in}}{\pgfqpoint{4.972335in}{0.660840in}}%
\pgfpathcurveto{\pgfqpoint{4.980149in}{0.653026in}}{\pgfqpoint{4.990748in}{0.648636in}}{\pgfqpoint{5.001798in}{0.648636in}}%
\pgfpathlineto{\pgfqpoint{5.001798in}{0.648636in}}%
\pgfpathclose%
\pgfusepath{stroke}%
\end{pgfscope}%
\begin{pgfscope}%
\pgfpathrectangle{\pgfqpoint{0.847223in}{0.554012in}}{\pgfqpoint{6.200000in}{4.530000in}}%
\pgfusepath{clip}%
\pgfsetbuttcap%
\pgfsetroundjoin%
\pgfsetlinewidth{1.003750pt}%
\definecolor{currentstroke}{rgb}{1.000000,0.000000,0.000000}%
\pgfsetstrokecolor{currentstroke}%
\pgfsetdash{}{0pt}%
\pgfpathmoveto{\pgfqpoint{5.007131in}{0.648006in}}%
\pgfpathcurveto{\pgfqpoint{5.018181in}{0.648006in}}{\pgfqpoint{5.028780in}{0.652396in}}{\pgfqpoint{5.036594in}{0.660209in}}%
\pgfpathcurveto{\pgfqpoint{5.044408in}{0.668023in}}{\pgfqpoint{5.048798in}{0.678622in}}{\pgfqpoint{5.048798in}{0.689672in}}%
\pgfpathcurveto{\pgfqpoint{5.048798in}{0.700722in}}{\pgfqpoint{5.044408in}{0.711321in}}{\pgfqpoint{5.036594in}{0.719135in}}%
\pgfpathcurveto{\pgfqpoint{5.028780in}{0.726949in}}{\pgfqpoint{5.018181in}{0.731339in}}{\pgfqpoint{5.007131in}{0.731339in}}%
\pgfpathcurveto{\pgfqpoint{4.996081in}{0.731339in}}{\pgfqpoint{4.985482in}{0.726949in}}{\pgfqpoint{4.977668in}{0.719135in}}%
\pgfpathcurveto{\pgfqpoint{4.969855in}{0.711321in}}{\pgfqpoint{4.965465in}{0.700722in}}{\pgfqpoint{4.965465in}{0.689672in}}%
\pgfpathcurveto{\pgfqpoint{4.965465in}{0.678622in}}{\pgfqpoint{4.969855in}{0.668023in}}{\pgfqpoint{4.977668in}{0.660209in}}%
\pgfpathcurveto{\pgfqpoint{4.985482in}{0.652396in}}{\pgfqpoint{4.996081in}{0.648006in}}{\pgfqpoint{5.007131in}{0.648006in}}%
\pgfpathlineto{\pgfqpoint{5.007131in}{0.648006in}}%
\pgfpathclose%
\pgfusepath{stroke}%
\end{pgfscope}%
\begin{pgfscope}%
\pgfpathrectangle{\pgfqpoint{0.847223in}{0.554012in}}{\pgfqpoint{6.200000in}{4.530000in}}%
\pgfusepath{clip}%
\pgfsetbuttcap%
\pgfsetroundjoin%
\pgfsetlinewidth{1.003750pt}%
\definecolor{currentstroke}{rgb}{1.000000,0.000000,0.000000}%
\pgfsetstrokecolor{currentstroke}%
\pgfsetdash{}{0pt}%
\pgfpathmoveto{\pgfqpoint{5.012464in}{0.647377in}}%
\pgfpathcurveto{\pgfqpoint{5.023515in}{0.647377in}}{\pgfqpoint{5.034114in}{0.651767in}}{\pgfqpoint{5.041927in}{0.659581in}}%
\pgfpathcurveto{\pgfqpoint{5.049741in}{0.667394in}}{\pgfqpoint{5.054131in}{0.677993in}}{\pgfqpoint{5.054131in}{0.689043in}}%
\pgfpathcurveto{\pgfqpoint{5.054131in}{0.700093in}}{\pgfqpoint{5.049741in}{0.710693in}}{\pgfqpoint{5.041927in}{0.718506in}}%
\pgfpathcurveto{\pgfqpoint{5.034114in}{0.726320in}}{\pgfqpoint{5.023515in}{0.730710in}}{\pgfqpoint{5.012464in}{0.730710in}}%
\pgfpathcurveto{\pgfqpoint{5.001414in}{0.730710in}}{\pgfqpoint{4.990815in}{0.726320in}}{\pgfqpoint{4.983002in}{0.718506in}}%
\pgfpathcurveto{\pgfqpoint{4.975188in}{0.710693in}}{\pgfqpoint{4.970798in}{0.700093in}}{\pgfqpoint{4.970798in}{0.689043in}}%
\pgfpathcurveto{\pgfqpoint{4.970798in}{0.677993in}}{\pgfqpoint{4.975188in}{0.667394in}}{\pgfqpoint{4.983002in}{0.659581in}}%
\pgfpathcurveto{\pgfqpoint{4.990815in}{0.651767in}}{\pgfqpoint{5.001414in}{0.647377in}}{\pgfqpoint{5.012464in}{0.647377in}}%
\pgfpathlineto{\pgfqpoint{5.012464in}{0.647377in}}%
\pgfpathclose%
\pgfusepath{stroke}%
\end{pgfscope}%
\begin{pgfscope}%
\pgfpathrectangle{\pgfqpoint{0.847223in}{0.554012in}}{\pgfqpoint{6.200000in}{4.530000in}}%
\pgfusepath{clip}%
\pgfsetbuttcap%
\pgfsetroundjoin%
\pgfsetlinewidth{1.003750pt}%
\definecolor{currentstroke}{rgb}{1.000000,0.000000,0.000000}%
\pgfsetstrokecolor{currentstroke}%
\pgfsetdash{}{0pt}%
\pgfpathmoveto{\pgfqpoint{5.017798in}{0.646749in}}%
\pgfpathcurveto{\pgfqpoint{5.028848in}{0.646749in}}{\pgfqpoint{5.039447in}{0.651140in}}{\pgfqpoint{5.047260in}{0.658953in}}%
\pgfpathcurveto{\pgfqpoint{5.055074in}{0.666767in}}{\pgfqpoint{5.059464in}{0.677366in}}{\pgfqpoint{5.059464in}{0.688416in}}%
\pgfpathcurveto{\pgfqpoint{5.059464in}{0.699466in}}{\pgfqpoint{5.055074in}{0.710065in}}{\pgfqpoint{5.047260in}{0.717879in}}%
\pgfpathcurveto{\pgfqpoint{5.039447in}{0.725692in}}{\pgfqpoint{5.028848in}{0.730083in}}{\pgfqpoint{5.017798in}{0.730083in}}%
\pgfpathcurveto{\pgfqpoint{5.006748in}{0.730083in}}{\pgfqpoint{4.996149in}{0.725692in}}{\pgfqpoint{4.988335in}{0.717879in}}%
\pgfpathcurveto{\pgfqpoint{4.980521in}{0.710065in}}{\pgfqpoint{4.976131in}{0.699466in}}{\pgfqpoint{4.976131in}{0.688416in}}%
\pgfpathcurveto{\pgfqpoint{4.976131in}{0.677366in}}{\pgfqpoint{4.980521in}{0.666767in}}{\pgfqpoint{4.988335in}{0.658953in}}%
\pgfpathcurveto{\pgfqpoint{4.996149in}{0.651140in}}{\pgfqpoint{5.006748in}{0.646749in}}{\pgfqpoint{5.017798in}{0.646749in}}%
\pgfpathlineto{\pgfqpoint{5.017798in}{0.646749in}}%
\pgfpathclose%
\pgfusepath{stroke}%
\end{pgfscope}%
\begin{pgfscope}%
\pgfpathrectangle{\pgfqpoint{0.847223in}{0.554012in}}{\pgfqpoint{6.200000in}{4.530000in}}%
\pgfusepath{clip}%
\pgfsetbuttcap%
\pgfsetroundjoin%
\pgfsetlinewidth{1.003750pt}%
\definecolor{currentstroke}{rgb}{1.000000,0.000000,0.000000}%
\pgfsetstrokecolor{currentstroke}%
\pgfsetdash{}{0pt}%
\pgfpathmoveto{\pgfqpoint{5.023131in}{0.646123in}}%
\pgfpathcurveto{\pgfqpoint{5.034181in}{0.646123in}}{\pgfqpoint{5.044780in}{0.650514in}}{\pgfqpoint{5.052594in}{0.658327in}}%
\pgfpathcurveto{\pgfqpoint{5.060407in}{0.666141in}}{\pgfqpoint{5.064798in}{0.676740in}}{\pgfqpoint{5.064798in}{0.687790in}}%
\pgfpathcurveto{\pgfqpoint{5.064798in}{0.698840in}}{\pgfqpoint{5.060407in}{0.709439in}}{\pgfqpoint{5.052594in}{0.717253in}}%
\pgfpathcurveto{\pgfqpoint{5.044780in}{0.725066in}}{\pgfqpoint{5.034181in}{0.729457in}}{\pgfqpoint{5.023131in}{0.729457in}}%
\pgfpathcurveto{\pgfqpoint{5.012081in}{0.729457in}}{\pgfqpoint{5.001482in}{0.725066in}}{\pgfqpoint{4.993668in}{0.717253in}}%
\pgfpathcurveto{\pgfqpoint{4.985855in}{0.709439in}}{\pgfqpoint{4.981464in}{0.698840in}}{\pgfqpoint{4.981464in}{0.687790in}}%
\pgfpathcurveto{\pgfqpoint{4.981464in}{0.676740in}}{\pgfqpoint{4.985855in}{0.666141in}}{\pgfqpoint{4.993668in}{0.658327in}}%
\pgfpathcurveto{\pgfqpoint{5.001482in}{0.650514in}}{\pgfqpoint{5.012081in}{0.646123in}}{\pgfqpoint{5.023131in}{0.646123in}}%
\pgfpathlineto{\pgfqpoint{5.023131in}{0.646123in}}%
\pgfpathclose%
\pgfusepath{stroke}%
\end{pgfscope}%
\begin{pgfscope}%
\pgfpathrectangle{\pgfqpoint{0.847223in}{0.554012in}}{\pgfqpoint{6.200000in}{4.530000in}}%
\pgfusepath{clip}%
\pgfsetbuttcap%
\pgfsetroundjoin%
\pgfsetlinewidth{1.003750pt}%
\definecolor{currentstroke}{rgb}{1.000000,0.000000,0.000000}%
\pgfsetstrokecolor{currentstroke}%
\pgfsetdash{}{0pt}%
\pgfpathmoveto{\pgfqpoint{5.028464in}{0.645499in}}%
\pgfpathcurveto{\pgfqpoint{5.039514in}{0.645499in}}{\pgfqpoint{5.050113in}{0.649889in}}{\pgfqpoint{5.057927in}{0.657703in}}%
\pgfpathcurveto{\pgfqpoint{5.065741in}{0.665516in}}{\pgfqpoint{5.070131in}{0.676115in}}{\pgfqpoint{5.070131in}{0.687165in}}%
\pgfpathcurveto{\pgfqpoint{5.070131in}{0.698215in}}{\pgfqpoint{5.065741in}{0.708815in}}{\pgfqpoint{5.057927in}{0.716628in}}%
\pgfpathcurveto{\pgfqpoint{5.050113in}{0.724442in}}{\pgfqpoint{5.039514in}{0.728832in}}{\pgfqpoint{5.028464in}{0.728832in}}%
\pgfpathcurveto{\pgfqpoint{5.017414in}{0.728832in}}{\pgfqpoint{5.006815in}{0.724442in}}{\pgfqpoint{4.999001in}{0.716628in}}%
\pgfpathcurveto{\pgfqpoint{4.991188in}{0.708815in}}{\pgfqpoint{4.986797in}{0.698215in}}{\pgfqpoint{4.986797in}{0.687165in}}%
\pgfpathcurveto{\pgfqpoint{4.986797in}{0.676115in}}{\pgfqpoint{4.991188in}{0.665516in}}{\pgfqpoint{4.999001in}{0.657703in}}%
\pgfpathcurveto{\pgfqpoint{5.006815in}{0.649889in}}{\pgfqpoint{5.017414in}{0.645499in}}{\pgfqpoint{5.028464in}{0.645499in}}%
\pgfpathlineto{\pgfqpoint{5.028464in}{0.645499in}}%
\pgfpathclose%
\pgfusepath{stroke}%
\end{pgfscope}%
\begin{pgfscope}%
\pgfpathrectangle{\pgfqpoint{0.847223in}{0.554012in}}{\pgfqpoint{6.200000in}{4.530000in}}%
\pgfusepath{clip}%
\pgfsetbuttcap%
\pgfsetroundjoin%
\pgfsetlinewidth{1.003750pt}%
\definecolor{currentstroke}{rgb}{1.000000,0.000000,0.000000}%
\pgfsetstrokecolor{currentstroke}%
\pgfsetdash{}{0pt}%
\pgfpathmoveto{\pgfqpoint{5.033797in}{0.644876in}}%
\pgfpathcurveto{\pgfqpoint{5.044847in}{0.644876in}}{\pgfqpoint{5.055447in}{0.649266in}}{\pgfqpoint{5.063260in}{0.657079in}}%
\pgfpathcurveto{\pgfqpoint{5.071074in}{0.664893in}}{\pgfqpoint{5.075464in}{0.675492in}}{\pgfqpoint{5.075464in}{0.686542in}}%
\pgfpathcurveto{\pgfqpoint{5.075464in}{0.697592in}}{\pgfqpoint{5.071074in}{0.708191in}}{\pgfqpoint{5.063260in}{0.716005in}}%
\pgfpathcurveto{\pgfqpoint{5.055447in}{0.723819in}}{\pgfqpoint{5.044847in}{0.728209in}}{\pgfqpoint{5.033797in}{0.728209in}}%
\pgfpathcurveto{\pgfqpoint{5.022747in}{0.728209in}}{\pgfqpoint{5.012148in}{0.723819in}}{\pgfqpoint{5.004335in}{0.716005in}}%
\pgfpathcurveto{\pgfqpoint{4.996521in}{0.708191in}}{\pgfqpoint{4.992131in}{0.697592in}}{\pgfqpoint{4.992131in}{0.686542in}}%
\pgfpathcurveto{\pgfqpoint{4.992131in}{0.675492in}}{\pgfqpoint{4.996521in}{0.664893in}}{\pgfqpoint{5.004335in}{0.657079in}}%
\pgfpathcurveto{\pgfqpoint{5.012148in}{0.649266in}}{\pgfqpoint{5.022747in}{0.644876in}}{\pgfqpoint{5.033797in}{0.644876in}}%
\pgfpathlineto{\pgfqpoint{5.033797in}{0.644876in}}%
\pgfpathclose%
\pgfusepath{stroke}%
\end{pgfscope}%
\begin{pgfscope}%
\pgfpathrectangle{\pgfqpoint{0.847223in}{0.554012in}}{\pgfqpoint{6.200000in}{4.530000in}}%
\pgfusepath{clip}%
\pgfsetbuttcap%
\pgfsetroundjoin%
\pgfsetlinewidth{1.003750pt}%
\definecolor{currentstroke}{rgb}{1.000000,0.000000,0.000000}%
\pgfsetstrokecolor{currentstroke}%
\pgfsetdash{}{0pt}%
\pgfpathmoveto{\pgfqpoint{5.039131in}{0.644254in}}%
\pgfpathcurveto{\pgfqpoint{5.050181in}{0.644254in}}{\pgfqpoint{5.060780in}{0.648644in}}{\pgfqpoint{5.068593in}{0.656458in}}%
\pgfpathcurveto{\pgfqpoint{5.076407in}{0.664271in}}{\pgfqpoint{5.080797in}{0.674870in}}{\pgfqpoint{5.080797in}{0.685920in}}%
\pgfpathcurveto{\pgfqpoint{5.080797in}{0.696971in}}{\pgfqpoint{5.076407in}{0.707570in}}{\pgfqpoint{5.068593in}{0.715383in}}%
\pgfpathcurveto{\pgfqpoint{5.060780in}{0.723197in}}{\pgfqpoint{5.050181in}{0.727587in}}{\pgfqpoint{5.039131in}{0.727587in}}%
\pgfpathcurveto{\pgfqpoint{5.028080in}{0.727587in}}{\pgfqpoint{5.017481in}{0.723197in}}{\pgfqpoint{5.009668in}{0.715383in}}%
\pgfpathcurveto{\pgfqpoint{5.001854in}{0.707570in}}{\pgfqpoint{4.997464in}{0.696971in}}{\pgfqpoint{4.997464in}{0.685920in}}%
\pgfpathcurveto{\pgfqpoint{4.997464in}{0.674870in}}{\pgfqpoint{5.001854in}{0.664271in}}{\pgfqpoint{5.009668in}{0.656458in}}%
\pgfpathcurveto{\pgfqpoint{5.017481in}{0.648644in}}{\pgfqpoint{5.028080in}{0.644254in}}{\pgfqpoint{5.039131in}{0.644254in}}%
\pgfpathlineto{\pgfqpoint{5.039131in}{0.644254in}}%
\pgfpathclose%
\pgfusepath{stroke}%
\end{pgfscope}%
\begin{pgfscope}%
\pgfpathrectangle{\pgfqpoint{0.847223in}{0.554012in}}{\pgfqpoint{6.200000in}{4.530000in}}%
\pgfusepath{clip}%
\pgfsetbuttcap%
\pgfsetroundjoin%
\pgfsetlinewidth{1.003750pt}%
\definecolor{currentstroke}{rgb}{1.000000,0.000000,0.000000}%
\pgfsetstrokecolor{currentstroke}%
\pgfsetdash{}{0pt}%
\pgfpathmoveto{\pgfqpoint{5.044464in}{0.643633in}}%
\pgfpathcurveto{\pgfqpoint{5.055514in}{0.643633in}}{\pgfqpoint{5.066113in}{0.648024in}}{\pgfqpoint{5.073927in}{0.655837in}}%
\pgfpathcurveto{\pgfqpoint{5.081740in}{0.663651in}}{\pgfqpoint{5.086130in}{0.674250in}}{\pgfqpoint{5.086130in}{0.685300in}}%
\pgfpathcurveto{\pgfqpoint{5.086130in}{0.696350in}}{\pgfqpoint{5.081740in}{0.706949in}}{\pgfqpoint{5.073927in}{0.714763in}}%
\pgfpathcurveto{\pgfqpoint{5.066113in}{0.722576in}}{\pgfqpoint{5.055514in}{0.726967in}}{\pgfqpoint{5.044464in}{0.726967in}}%
\pgfpathcurveto{\pgfqpoint{5.033414in}{0.726967in}}{\pgfqpoint{5.022815in}{0.722576in}}{\pgfqpoint{5.015001in}{0.714763in}}%
\pgfpathcurveto{\pgfqpoint{5.007187in}{0.706949in}}{\pgfqpoint{5.002797in}{0.696350in}}{\pgfqpoint{5.002797in}{0.685300in}}%
\pgfpathcurveto{\pgfqpoint{5.002797in}{0.674250in}}{\pgfqpoint{5.007187in}{0.663651in}}{\pgfqpoint{5.015001in}{0.655837in}}%
\pgfpathcurveto{\pgfqpoint{5.022815in}{0.648024in}}{\pgfqpoint{5.033414in}{0.643633in}}{\pgfqpoint{5.044464in}{0.643633in}}%
\pgfpathlineto{\pgfqpoint{5.044464in}{0.643633in}}%
\pgfpathclose%
\pgfusepath{stroke}%
\end{pgfscope}%
\begin{pgfscope}%
\pgfpathrectangle{\pgfqpoint{0.847223in}{0.554012in}}{\pgfqpoint{6.200000in}{4.530000in}}%
\pgfusepath{clip}%
\pgfsetbuttcap%
\pgfsetroundjoin%
\pgfsetlinewidth{1.003750pt}%
\definecolor{currentstroke}{rgb}{1.000000,0.000000,0.000000}%
\pgfsetstrokecolor{currentstroke}%
\pgfsetdash{}{0pt}%
\pgfpathmoveto{\pgfqpoint{5.049797in}{0.643014in}}%
\pgfpathcurveto{\pgfqpoint{5.060847in}{0.643014in}}{\pgfqpoint{5.071446in}{0.647405in}}{\pgfqpoint{5.079260in}{0.655218in}}%
\pgfpathcurveto{\pgfqpoint{5.087073in}{0.663032in}}{\pgfqpoint{5.091464in}{0.673631in}}{\pgfqpoint{5.091464in}{0.684681in}}%
\pgfpathcurveto{\pgfqpoint{5.091464in}{0.695731in}}{\pgfqpoint{5.087073in}{0.706330in}}{\pgfqpoint{5.079260in}{0.714144in}}%
\pgfpathcurveto{\pgfqpoint{5.071446in}{0.721958in}}{\pgfqpoint{5.060847in}{0.726348in}}{\pgfqpoint{5.049797in}{0.726348in}}%
\pgfpathcurveto{\pgfqpoint{5.038747in}{0.726348in}}{\pgfqpoint{5.028148in}{0.721958in}}{\pgfqpoint{5.020334in}{0.714144in}}%
\pgfpathcurveto{\pgfqpoint{5.012521in}{0.706330in}}{\pgfqpoint{5.008130in}{0.695731in}}{\pgfqpoint{5.008130in}{0.684681in}}%
\pgfpathcurveto{\pgfqpoint{5.008130in}{0.673631in}}{\pgfqpoint{5.012521in}{0.663032in}}{\pgfqpoint{5.020334in}{0.655218in}}%
\pgfpathcurveto{\pgfqpoint{5.028148in}{0.647405in}}{\pgfqpoint{5.038747in}{0.643014in}}{\pgfqpoint{5.049797in}{0.643014in}}%
\pgfpathlineto{\pgfqpoint{5.049797in}{0.643014in}}%
\pgfpathclose%
\pgfusepath{stroke}%
\end{pgfscope}%
\begin{pgfscope}%
\pgfpathrectangle{\pgfqpoint{0.847223in}{0.554012in}}{\pgfqpoint{6.200000in}{4.530000in}}%
\pgfusepath{clip}%
\pgfsetbuttcap%
\pgfsetroundjoin%
\pgfsetlinewidth{1.003750pt}%
\definecolor{currentstroke}{rgb}{1.000000,0.000000,0.000000}%
\pgfsetstrokecolor{currentstroke}%
\pgfsetdash{}{0pt}%
\pgfpathmoveto{\pgfqpoint{5.055130in}{0.642397in}}%
\pgfpathcurveto{\pgfqpoint{5.066180in}{0.642397in}}{\pgfqpoint{5.076779in}{0.646787in}}{\pgfqpoint{5.084593in}{0.654601in}}%
\pgfpathcurveto{\pgfqpoint{5.092407in}{0.662414in}}{\pgfqpoint{5.096797in}{0.673013in}}{\pgfqpoint{5.096797in}{0.684064in}}%
\pgfpathcurveto{\pgfqpoint{5.096797in}{0.695114in}}{\pgfqpoint{5.092407in}{0.705713in}}{\pgfqpoint{5.084593in}{0.713526in}}%
\pgfpathcurveto{\pgfqpoint{5.076779in}{0.721340in}}{\pgfqpoint{5.066180in}{0.725730in}}{\pgfqpoint{5.055130in}{0.725730in}}%
\pgfpathcurveto{\pgfqpoint{5.044080in}{0.725730in}}{\pgfqpoint{5.033481in}{0.721340in}}{\pgfqpoint{5.025667in}{0.713526in}}%
\pgfpathcurveto{\pgfqpoint{5.017854in}{0.705713in}}{\pgfqpoint{5.013464in}{0.695114in}}{\pgfqpoint{5.013464in}{0.684064in}}%
\pgfpathcurveto{\pgfqpoint{5.013464in}{0.673013in}}{\pgfqpoint{5.017854in}{0.662414in}}{\pgfqpoint{5.025667in}{0.654601in}}%
\pgfpathcurveto{\pgfqpoint{5.033481in}{0.646787in}}{\pgfqpoint{5.044080in}{0.642397in}}{\pgfqpoint{5.055130in}{0.642397in}}%
\pgfpathlineto{\pgfqpoint{5.055130in}{0.642397in}}%
\pgfpathclose%
\pgfusepath{stroke}%
\end{pgfscope}%
\begin{pgfscope}%
\pgfpathrectangle{\pgfqpoint{0.847223in}{0.554012in}}{\pgfqpoint{6.200000in}{4.530000in}}%
\pgfusepath{clip}%
\pgfsetbuttcap%
\pgfsetroundjoin%
\pgfsetlinewidth{1.003750pt}%
\definecolor{currentstroke}{rgb}{1.000000,0.000000,0.000000}%
\pgfsetstrokecolor{currentstroke}%
\pgfsetdash{}{0pt}%
\pgfpathmoveto{\pgfqpoint{5.060463in}{0.641781in}}%
\pgfpathcurveto{\pgfqpoint{5.071514in}{0.641781in}}{\pgfqpoint{5.082113in}{0.646171in}}{\pgfqpoint{5.089926in}{0.653985in}}%
\pgfpathcurveto{\pgfqpoint{5.097740in}{0.661798in}}{\pgfqpoint{5.102130in}{0.672397in}}{\pgfqpoint{5.102130in}{0.683447in}}%
\pgfpathcurveto{\pgfqpoint{5.102130in}{0.694498in}}{\pgfqpoint{5.097740in}{0.705097in}}{\pgfqpoint{5.089926in}{0.712910in}}%
\pgfpathcurveto{\pgfqpoint{5.082113in}{0.720724in}}{\pgfqpoint{5.071514in}{0.725114in}}{\pgfqpoint{5.060463in}{0.725114in}}%
\pgfpathcurveto{\pgfqpoint{5.049413in}{0.725114in}}{\pgfqpoint{5.038814in}{0.720724in}}{\pgfqpoint{5.031001in}{0.712910in}}%
\pgfpathcurveto{\pgfqpoint{5.023187in}{0.705097in}}{\pgfqpoint{5.018797in}{0.694498in}}{\pgfqpoint{5.018797in}{0.683447in}}%
\pgfpathcurveto{\pgfqpoint{5.018797in}{0.672397in}}{\pgfqpoint{5.023187in}{0.661798in}}{\pgfqpoint{5.031001in}{0.653985in}}%
\pgfpathcurveto{\pgfqpoint{5.038814in}{0.646171in}}{\pgfqpoint{5.049413in}{0.641781in}}{\pgfqpoint{5.060463in}{0.641781in}}%
\pgfpathlineto{\pgfqpoint{5.060463in}{0.641781in}}%
\pgfpathclose%
\pgfusepath{stroke}%
\end{pgfscope}%
\begin{pgfscope}%
\pgfpathrectangle{\pgfqpoint{0.847223in}{0.554012in}}{\pgfqpoint{6.200000in}{4.530000in}}%
\pgfusepath{clip}%
\pgfsetbuttcap%
\pgfsetroundjoin%
\pgfsetlinewidth{1.003750pt}%
\definecolor{currentstroke}{rgb}{1.000000,0.000000,0.000000}%
\pgfsetstrokecolor{currentstroke}%
\pgfsetdash{}{0pt}%
\pgfpathmoveto{\pgfqpoint{5.065797in}{0.641166in}}%
\pgfpathcurveto{\pgfqpoint{5.076847in}{0.641166in}}{\pgfqpoint{5.087446in}{0.645556in}}{\pgfqpoint{5.095259in}{0.653370in}}%
\pgfpathcurveto{\pgfqpoint{5.103073in}{0.661183in}}{\pgfqpoint{5.107463in}{0.671783in}}{\pgfqpoint{5.107463in}{0.682833in}}%
\pgfpathcurveto{\pgfqpoint{5.107463in}{0.693883in}}{\pgfqpoint{5.103073in}{0.704482in}}{\pgfqpoint{5.095259in}{0.712295in}}%
\pgfpathcurveto{\pgfqpoint{5.087446in}{0.720109in}}{\pgfqpoint{5.076847in}{0.724499in}}{\pgfqpoint{5.065797in}{0.724499in}}%
\pgfpathcurveto{\pgfqpoint{5.054747in}{0.724499in}}{\pgfqpoint{5.044147in}{0.720109in}}{\pgfqpoint{5.036334in}{0.712295in}}%
\pgfpathcurveto{\pgfqpoint{5.028520in}{0.704482in}}{\pgfqpoint{5.024130in}{0.693883in}}{\pgfqpoint{5.024130in}{0.682833in}}%
\pgfpathcurveto{\pgfqpoint{5.024130in}{0.671783in}}{\pgfqpoint{5.028520in}{0.661183in}}{\pgfqpoint{5.036334in}{0.653370in}}%
\pgfpathcurveto{\pgfqpoint{5.044147in}{0.645556in}}{\pgfqpoint{5.054747in}{0.641166in}}{\pgfqpoint{5.065797in}{0.641166in}}%
\pgfpathlineto{\pgfqpoint{5.065797in}{0.641166in}}%
\pgfpathclose%
\pgfusepath{stroke}%
\end{pgfscope}%
\begin{pgfscope}%
\pgfpathrectangle{\pgfqpoint{0.847223in}{0.554012in}}{\pgfqpoint{6.200000in}{4.530000in}}%
\pgfusepath{clip}%
\pgfsetbuttcap%
\pgfsetroundjoin%
\pgfsetlinewidth{1.003750pt}%
\definecolor{currentstroke}{rgb}{1.000000,0.000000,0.000000}%
\pgfsetstrokecolor{currentstroke}%
\pgfsetdash{}{0pt}%
\pgfpathmoveto{\pgfqpoint{5.071130in}{0.640553in}}%
\pgfpathcurveto{\pgfqpoint{5.082180in}{0.640553in}}{\pgfqpoint{5.092779in}{0.644943in}}{\pgfqpoint{5.100593in}{0.652756in}}%
\pgfpathcurveto{\pgfqpoint{5.108406in}{0.660570in}}{\pgfqpoint{5.112797in}{0.671169in}}{\pgfqpoint{5.112797in}{0.682219in}}%
\pgfpathcurveto{\pgfqpoint{5.112797in}{0.693269in}}{\pgfqpoint{5.108406in}{0.703868in}}{\pgfqpoint{5.100593in}{0.711682in}}%
\pgfpathcurveto{\pgfqpoint{5.092779in}{0.719496in}}{\pgfqpoint{5.082180in}{0.723886in}}{\pgfqpoint{5.071130in}{0.723886in}}%
\pgfpathcurveto{\pgfqpoint{5.060080in}{0.723886in}}{\pgfqpoint{5.049481in}{0.719496in}}{\pgfqpoint{5.041667in}{0.711682in}}%
\pgfpathcurveto{\pgfqpoint{5.033853in}{0.703868in}}{\pgfqpoint{5.029463in}{0.693269in}}{\pgfqpoint{5.029463in}{0.682219in}}%
\pgfpathcurveto{\pgfqpoint{5.029463in}{0.671169in}}{\pgfqpoint{5.033853in}{0.660570in}}{\pgfqpoint{5.041667in}{0.652756in}}%
\pgfpathcurveto{\pgfqpoint{5.049481in}{0.644943in}}{\pgfqpoint{5.060080in}{0.640553in}}{\pgfqpoint{5.071130in}{0.640553in}}%
\pgfpathlineto{\pgfqpoint{5.071130in}{0.640553in}}%
\pgfpathclose%
\pgfusepath{stroke}%
\end{pgfscope}%
\begin{pgfscope}%
\pgfpathrectangle{\pgfqpoint{0.847223in}{0.554012in}}{\pgfqpoint{6.200000in}{4.530000in}}%
\pgfusepath{clip}%
\pgfsetbuttcap%
\pgfsetroundjoin%
\pgfsetlinewidth{1.003750pt}%
\definecolor{currentstroke}{rgb}{1.000000,0.000000,0.000000}%
\pgfsetstrokecolor{currentstroke}%
\pgfsetdash{}{0pt}%
\pgfpathmoveto{\pgfqpoint{5.076463in}{0.639941in}}%
\pgfpathcurveto{\pgfqpoint{5.087513in}{0.639941in}}{\pgfqpoint{5.098112in}{0.644331in}}{\pgfqpoint{5.105926in}{0.652144in}}%
\pgfpathcurveto{\pgfqpoint{5.113739in}{0.659958in}}{\pgfqpoint{5.118130in}{0.670557in}}{\pgfqpoint{5.118130in}{0.681607in}}%
\pgfpathcurveto{\pgfqpoint{5.118130in}{0.692657in}}{\pgfqpoint{5.113739in}{0.703256in}}{\pgfqpoint{5.105926in}{0.711070in}}%
\pgfpathcurveto{\pgfqpoint{5.098112in}{0.718884in}}{\pgfqpoint{5.087513in}{0.723274in}}{\pgfqpoint{5.076463in}{0.723274in}}%
\pgfpathcurveto{\pgfqpoint{5.065413in}{0.723274in}}{\pgfqpoint{5.054814in}{0.718884in}}{\pgfqpoint{5.047000in}{0.711070in}}%
\pgfpathcurveto{\pgfqpoint{5.039187in}{0.703256in}}{\pgfqpoint{5.034796in}{0.692657in}}{\pgfqpoint{5.034796in}{0.681607in}}%
\pgfpathcurveto{\pgfqpoint{5.034796in}{0.670557in}}{\pgfqpoint{5.039187in}{0.659958in}}{\pgfqpoint{5.047000in}{0.652144in}}%
\pgfpathcurveto{\pgfqpoint{5.054814in}{0.644331in}}{\pgfqpoint{5.065413in}{0.639941in}}{\pgfqpoint{5.076463in}{0.639941in}}%
\pgfpathlineto{\pgfqpoint{5.076463in}{0.639941in}}%
\pgfpathclose%
\pgfusepath{stroke}%
\end{pgfscope}%
\begin{pgfscope}%
\pgfpathrectangle{\pgfqpoint{0.847223in}{0.554012in}}{\pgfqpoint{6.200000in}{4.530000in}}%
\pgfusepath{clip}%
\pgfsetbuttcap%
\pgfsetroundjoin%
\pgfsetlinewidth{1.003750pt}%
\definecolor{currentstroke}{rgb}{1.000000,0.000000,0.000000}%
\pgfsetstrokecolor{currentstroke}%
\pgfsetdash{}{0pt}%
\pgfpathmoveto{\pgfqpoint{5.081796in}{0.639330in}}%
\pgfpathcurveto{\pgfqpoint{5.092846in}{0.639330in}}{\pgfqpoint{5.103445in}{0.643720in}}{\pgfqpoint{5.111259in}{0.651534in}}%
\pgfpathcurveto{\pgfqpoint{5.119073in}{0.659347in}}{\pgfqpoint{5.123463in}{0.669946in}}{\pgfqpoint{5.123463in}{0.680997in}}%
\pgfpathcurveto{\pgfqpoint{5.123463in}{0.692047in}}{\pgfqpoint{5.119073in}{0.702646in}}{\pgfqpoint{5.111259in}{0.710459in}}%
\pgfpathcurveto{\pgfqpoint{5.103445in}{0.718273in}}{\pgfqpoint{5.092846in}{0.722663in}}{\pgfqpoint{5.081796in}{0.722663in}}%
\pgfpathcurveto{\pgfqpoint{5.070746in}{0.722663in}}{\pgfqpoint{5.060147in}{0.718273in}}{\pgfqpoint{5.052334in}{0.710459in}}%
\pgfpathcurveto{\pgfqpoint{5.044520in}{0.702646in}}{\pgfqpoint{5.040130in}{0.692047in}}{\pgfqpoint{5.040130in}{0.680997in}}%
\pgfpathcurveto{\pgfqpoint{5.040130in}{0.669946in}}{\pgfqpoint{5.044520in}{0.659347in}}{\pgfqpoint{5.052334in}{0.651534in}}%
\pgfpathcurveto{\pgfqpoint{5.060147in}{0.643720in}}{\pgfqpoint{5.070746in}{0.639330in}}{\pgfqpoint{5.081796in}{0.639330in}}%
\pgfpathlineto{\pgfqpoint{5.081796in}{0.639330in}}%
\pgfpathclose%
\pgfusepath{stroke}%
\end{pgfscope}%
\begin{pgfscope}%
\pgfpathrectangle{\pgfqpoint{0.847223in}{0.554012in}}{\pgfqpoint{6.200000in}{4.530000in}}%
\pgfusepath{clip}%
\pgfsetbuttcap%
\pgfsetroundjoin%
\pgfsetlinewidth{1.003750pt}%
\definecolor{currentstroke}{rgb}{1.000000,0.000000,0.000000}%
\pgfsetstrokecolor{currentstroke}%
\pgfsetdash{}{0pt}%
\pgfpathmoveto{\pgfqpoint{5.087130in}{0.638721in}}%
\pgfpathcurveto{\pgfqpoint{5.098180in}{0.638721in}}{\pgfqpoint{5.108779in}{0.643111in}}{\pgfqpoint{5.116592in}{0.650925in}}%
\pgfpathcurveto{\pgfqpoint{5.124406in}{0.658738in}}{\pgfqpoint{5.128796in}{0.669337in}}{\pgfqpoint{5.128796in}{0.680387in}}%
\pgfpathcurveto{\pgfqpoint{5.128796in}{0.691437in}}{\pgfqpoint{5.124406in}{0.702036in}}{\pgfqpoint{5.116592in}{0.709850in}}%
\pgfpathcurveto{\pgfqpoint{5.108779in}{0.717664in}}{\pgfqpoint{5.098180in}{0.722054in}}{\pgfqpoint{5.087130in}{0.722054in}}%
\pgfpathcurveto{\pgfqpoint{5.076079in}{0.722054in}}{\pgfqpoint{5.065480in}{0.717664in}}{\pgfqpoint{5.057667in}{0.709850in}}%
\pgfpathcurveto{\pgfqpoint{5.049853in}{0.702036in}}{\pgfqpoint{5.045463in}{0.691437in}}{\pgfqpoint{5.045463in}{0.680387in}}%
\pgfpathcurveto{\pgfqpoint{5.045463in}{0.669337in}}{\pgfqpoint{5.049853in}{0.658738in}}{\pgfqpoint{5.057667in}{0.650925in}}%
\pgfpathcurveto{\pgfqpoint{5.065480in}{0.643111in}}{\pgfqpoint{5.076079in}{0.638721in}}{\pgfqpoint{5.087130in}{0.638721in}}%
\pgfpathlineto{\pgfqpoint{5.087130in}{0.638721in}}%
\pgfpathclose%
\pgfusepath{stroke}%
\end{pgfscope}%
\begin{pgfscope}%
\pgfpathrectangle{\pgfqpoint{0.847223in}{0.554012in}}{\pgfqpoint{6.200000in}{4.530000in}}%
\pgfusepath{clip}%
\pgfsetbuttcap%
\pgfsetroundjoin%
\pgfsetlinewidth{1.003750pt}%
\definecolor{currentstroke}{rgb}{1.000000,0.000000,0.000000}%
\pgfsetstrokecolor{currentstroke}%
\pgfsetdash{}{0pt}%
\pgfpathmoveto{\pgfqpoint{5.092463in}{0.638113in}}%
\pgfpathcurveto{\pgfqpoint{5.103513in}{0.638113in}}{\pgfqpoint{5.114112in}{0.642503in}}{\pgfqpoint{5.121926in}{0.650317in}}%
\pgfpathcurveto{\pgfqpoint{5.129739in}{0.658130in}}{\pgfqpoint{5.134129in}{0.668729in}}{\pgfqpoint{5.134129in}{0.679779in}}%
\pgfpathcurveto{\pgfqpoint{5.134129in}{0.690829in}}{\pgfqpoint{5.129739in}{0.701429in}}{\pgfqpoint{5.121926in}{0.709242in}}%
\pgfpathcurveto{\pgfqpoint{5.114112in}{0.717056in}}{\pgfqpoint{5.103513in}{0.721446in}}{\pgfqpoint{5.092463in}{0.721446in}}%
\pgfpathcurveto{\pgfqpoint{5.081413in}{0.721446in}}{\pgfqpoint{5.070814in}{0.717056in}}{\pgfqpoint{5.063000in}{0.709242in}}%
\pgfpathcurveto{\pgfqpoint{5.055186in}{0.701429in}}{\pgfqpoint{5.050796in}{0.690829in}}{\pgfqpoint{5.050796in}{0.679779in}}%
\pgfpathcurveto{\pgfqpoint{5.050796in}{0.668729in}}{\pgfqpoint{5.055186in}{0.658130in}}{\pgfqpoint{5.063000in}{0.650317in}}%
\pgfpathcurveto{\pgfqpoint{5.070814in}{0.642503in}}{\pgfqpoint{5.081413in}{0.638113in}}{\pgfqpoint{5.092463in}{0.638113in}}%
\pgfpathlineto{\pgfqpoint{5.092463in}{0.638113in}}%
\pgfpathclose%
\pgfusepath{stroke}%
\end{pgfscope}%
\begin{pgfscope}%
\pgfpathrectangle{\pgfqpoint{0.847223in}{0.554012in}}{\pgfqpoint{6.200000in}{4.530000in}}%
\pgfusepath{clip}%
\pgfsetbuttcap%
\pgfsetroundjoin%
\pgfsetlinewidth{1.003750pt}%
\definecolor{currentstroke}{rgb}{1.000000,0.000000,0.000000}%
\pgfsetstrokecolor{currentstroke}%
\pgfsetdash{}{0pt}%
\pgfpathmoveto{\pgfqpoint{5.097796in}{0.637506in}}%
\pgfpathcurveto{\pgfqpoint{5.108846in}{0.637506in}}{\pgfqpoint{5.119445in}{0.641896in}}{\pgfqpoint{5.127259in}{0.649710in}}%
\pgfpathcurveto{\pgfqpoint{5.135072in}{0.657524in}}{\pgfqpoint{5.139463in}{0.668123in}}{\pgfqpoint{5.139463in}{0.679173in}}%
\pgfpathcurveto{\pgfqpoint{5.139463in}{0.690223in}}{\pgfqpoint{5.135072in}{0.700822in}}{\pgfqpoint{5.127259in}{0.708636in}}%
\pgfpathcurveto{\pgfqpoint{5.119445in}{0.716449in}}{\pgfqpoint{5.108846in}{0.720839in}}{\pgfqpoint{5.097796in}{0.720839in}}%
\pgfpathcurveto{\pgfqpoint{5.086746in}{0.720839in}}{\pgfqpoint{5.076147in}{0.716449in}}{\pgfqpoint{5.068333in}{0.708636in}}%
\pgfpathcurveto{\pgfqpoint{5.060520in}{0.700822in}}{\pgfqpoint{5.056129in}{0.690223in}}{\pgfqpoint{5.056129in}{0.679173in}}%
\pgfpathcurveto{\pgfqpoint{5.056129in}{0.668123in}}{\pgfqpoint{5.060520in}{0.657524in}}{\pgfqpoint{5.068333in}{0.649710in}}%
\pgfpathcurveto{\pgfqpoint{5.076147in}{0.641896in}}{\pgfqpoint{5.086746in}{0.637506in}}{\pgfqpoint{5.097796in}{0.637506in}}%
\pgfpathlineto{\pgfqpoint{5.097796in}{0.637506in}}%
\pgfpathclose%
\pgfusepath{stroke}%
\end{pgfscope}%
\begin{pgfscope}%
\pgfpathrectangle{\pgfqpoint{0.847223in}{0.554012in}}{\pgfqpoint{6.200000in}{4.530000in}}%
\pgfusepath{clip}%
\pgfsetbuttcap%
\pgfsetroundjoin%
\pgfsetlinewidth{1.003750pt}%
\definecolor{currentstroke}{rgb}{1.000000,0.000000,0.000000}%
\pgfsetstrokecolor{currentstroke}%
\pgfsetdash{}{0pt}%
\pgfpathmoveto{\pgfqpoint{5.103129in}{0.636901in}}%
\pgfpathcurveto{\pgfqpoint{5.114179in}{0.636901in}}{\pgfqpoint{5.124778in}{0.641291in}}{\pgfqpoint{5.132592in}{0.649105in}}%
\pgfpathcurveto{\pgfqpoint{5.140406in}{0.656918in}}{\pgfqpoint{5.144796in}{0.667517in}}{\pgfqpoint{5.144796in}{0.678568in}}%
\pgfpathcurveto{\pgfqpoint{5.144796in}{0.689618in}}{\pgfqpoint{5.140406in}{0.700217in}}{\pgfqpoint{5.132592in}{0.708030in}}%
\pgfpathcurveto{\pgfqpoint{5.124778in}{0.715844in}}{\pgfqpoint{5.114179in}{0.720234in}}{\pgfqpoint{5.103129in}{0.720234in}}%
\pgfpathcurveto{\pgfqpoint{5.092079in}{0.720234in}}{\pgfqpoint{5.081480in}{0.715844in}}{\pgfqpoint{5.073666in}{0.708030in}}%
\pgfpathcurveto{\pgfqpoint{5.065853in}{0.700217in}}{\pgfqpoint{5.061462in}{0.689618in}}{\pgfqpoint{5.061462in}{0.678568in}}%
\pgfpathcurveto{\pgfqpoint{5.061462in}{0.667517in}}{\pgfqpoint{5.065853in}{0.656918in}}{\pgfqpoint{5.073666in}{0.649105in}}%
\pgfpathcurveto{\pgfqpoint{5.081480in}{0.641291in}}{\pgfqpoint{5.092079in}{0.636901in}}{\pgfqpoint{5.103129in}{0.636901in}}%
\pgfpathlineto{\pgfqpoint{5.103129in}{0.636901in}}%
\pgfpathclose%
\pgfusepath{stroke}%
\end{pgfscope}%
\begin{pgfscope}%
\pgfpathrectangle{\pgfqpoint{0.847223in}{0.554012in}}{\pgfqpoint{6.200000in}{4.530000in}}%
\pgfusepath{clip}%
\pgfsetbuttcap%
\pgfsetroundjoin%
\pgfsetlinewidth{1.003750pt}%
\definecolor{currentstroke}{rgb}{1.000000,0.000000,0.000000}%
\pgfsetstrokecolor{currentstroke}%
\pgfsetdash{}{0pt}%
\pgfpathmoveto{\pgfqpoint{5.108462in}{0.636297in}}%
\pgfpathcurveto{\pgfqpoint{5.119512in}{0.636297in}}{\pgfqpoint{5.130112in}{0.640687in}}{\pgfqpoint{5.137925in}{0.648501in}}%
\pgfpathcurveto{\pgfqpoint{5.145739in}{0.656315in}}{\pgfqpoint{5.150129in}{0.666914in}}{\pgfqpoint{5.150129in}{0.677964in}}%
\pgfpathcurveto{\pgfqpoint{5.150129in}{0.689014in}}{\pgfqpoint{5.145739in}{0.699613in}}{\pgfqpoint{5.137925in}{0.707427in}}%
\pgfpathcurveto{\pgfqpoint{5.130112in}{0.715240in}}{\pgfqpoint{5.119512in}{0.719630in}}{\pgfqpoint{5.108462in}{0.719630in}}%
\pgfpathcurveto{\pgfqpoint{5.097412in}{0.719630in}}{\pgfqpoint{5.086813in}{0.715240in}}{\pgfqpoint{5.079000in}{0.707427in}}%
\pgfpathcurveto{\pgfqpoint{5.071186in}{0.699613in}}{\pgfqpoint{5.066796in}{0.689014in}}{\pgfqpoint{5.066796in}{0.677964in}}%
\pgfpathcurveto{\pgfqpoint{5.066796in}{0.666914in}}{\pgfqpoint{5.071186in}{0.656315in}}{\pgfqpoint{5.079000in}{0.648501in}}%
\pgfpathcurveto{\pgfqpoint{5.086813in}{0.640687in}}{\pgfqpoint{5.097412in}{0.636297in}}{\pgfqpoint{5.108462in}{0.636297in}}%
\pgfpathlineto{\pgfqpoint{5.108462in}{0.636297in}}%
\pgfpathclose%
\pgfusepath{stroke}%
\end{pgfscope}%
\begin{pgfscope}%
\pgfpathrectangle{\pgfqpoint{0.847223in}{0.554012in}}{\pgfqpoint{6.200000in}{4.530000in}}%
\pgfusepath{clip}%
\pgfsetbuttcap%
\pgfsetroundjoin%
\pgfsetlinewidth{1.003750pt}%
\definecolor{currentstroke}{rgb}{1.000000,0.000000,0.000000}%
\pgfsetstrokecolor{currentstroke}%
\pgfsetdash{}{0pt}%
\pgfpathmoveto{\pgfqpoint{5.113796in}{0.635695in}}%
\pgfpathcurveto{\pgfqpoint{5.124846in}{0.635695in}}{\pgfqpoint{5.135445in}{0.640085in}}{\pgfqpoint{5.143258in}{0.647898in}}%
\pgfpathcurveto{\pgfqpoint{5.151072in}{0.655712in}}{\pgfqpoint{5.155462in}{0.666311in}}{\pgfqpoint{5.155462in}{0.677361in}}%
\pgfpathcurveto{\pgfqpoint{5.155462in}{0.688411in}}{\pgfqpoint{5.151072in}{0.699010in}}{\pgfqpoint{5.143258in}{0.706824in}}%
\pgfpathcurveto{\pgfqpoint{5.135445in}{0.714638in}}{\pgfqpoint{5.124846in}{0.719028in}}{\pgfqpoint{5.113796in}{0.719028in}}%
\pgfpathcurveto{\pgfqpoint{5.102745in}{0.719028in}}{\pgfqpoint{5.092146in}{0.714638in}}{\pgfqpoint{5.084333in}{0.706824in}}%
\pgfpathcurveto{\pgfqpoint{5.076519in}{0.699010in}}{\pgfqpoint{5.072129in}{0.688411in}}{\pgfqpoint{5.072129in}{0.677361in}}%
\pgfpathcurveto{\pgfqpoint{5.072129in}{0.666311in}}{\pgfqpoint{5.076519in}{0.655712in}}{\pgfqpoint{5.084333in}{0.647898in}}%
\pgfpathcurveto{\pgfqpoint{5.092146in}{0.640085in}}{\pgfqpoint{5.102745in}{0.635695in}}{\pgfqpoint{5.113796in}{0.635695in}}%
\pgfpathlineto{\pgfqpoint{5.113796in}{0.635695in}}%
\pgfpathclose%
\pgfusepath{stroke}%
\end{pgfscope}%
\begin{pgfscope}%
\pgfpathrectangle{\pgfqpoint{0.847223in}{0.554012in}}{\pgfqpoint{6.200000in}{4.530000in}}%
\pgfusepath{clip}%
\pgfsetbuttcap%
\pgfsetroundjoin%
\pgfsetlinewidth{1.003750pt}%
\definecolor{currentstroke}{rgb}{1.000000,0.000000,0.000000}%
\pgfsetstrokecolor{currentstroke}%
\pgfsetdash{}{0pt}%
\pgfpathmoveto{\pgfqpoint{5.119129in}{0.635093in}}%
\pgfpathcurveto{\pgfqpoint{5.130179in}{0.635093in}}{\pgfqpoint{5.140778in}{0.639484in}}{\pgfqpoint{5.148592in}{0.647297in}}%
\pgfpathcurveto{\pgfqpoint{5.156405in}{0.655111in}}{\pgfqpoint{5.160795in}{0.665710in}}{\pgfqpoint{5.160795in}{0.676760in}}%
\pgfpathcurveto{\pgfqpoint{5.160795in}{0.687810in}}{\pgfqpoint{5.156405in}{0.698409in}}{\pgfqpoint{5.148592in}{0.706223in}}%
\pgfpathcurveto{\pgfqpoint{5.140778in}{0.714036in}}{\pgfqpoint{5.130179in}{0.718427in}}{\pgfqpoint{5.119129in}{0.718427in}}%
\pgfpathcurveto{\pgfqpoint{5.108079in}{0.718427in}}{\pgfqpoint{5.097480in}{0.714036in}}{\pgfqpoint{5.089666in}{0.706223in}}%
\pgfpathcurveto{\pgfqpoint{5.081852in}{0.698409in}}{\pgfqpoint{5.077462in}{0.687810in}}{\pgfqpoint{5.077462in}{0.676760in}}%
\pgfpathcurveto{\pgfqpoint{5.077462in}{0.665710in}}{\pgfqpoint{5.081852in}{0.655111in}}{\pgfqpoint{5.089666in}{0.647297in}}%
\pgfpathcurveto{\pgfqpoint{5.097480in}{0.639484in}}{\pgfqpoint{5.108079in}{0.635093in}}{\pgfqpoint{5.119129in}{0.635093in}}%
\pgfpathlineto{\pgfqpoint{5.119129in}{0.635093in}}%
\pgfpathclose%
\pgfusepath{stroke}%
\end{pgfscope}%
\begin{pgfscope}%
\pgfpathrectangle{\pgfqpoint{0.847223in}{0.554012in}}{\pgfqpoint{6.200000in}{4.530000in}}%
\pgfusepath{clip}%
\pgfsetbuttcap%
\pgfsetroundjoin%
\pgfsetlinewidth{1.003750pt}%
\definecolor{currentstroke}{rgb}{1.000000,0.000000,0.000000}%
\pgfsetstrokecolor{currentstroke}%
\pgfsetdash{}{0pt}%
\pgfpathmoveto{\pgfqpoint{5.124462in}{0.634494in}}%
\pgfpathcurveto{\pgfqpoint{5.135512in}{0.634494in}}{\pgfqpoint{5.146111in}{0.638884in}}{\pgfqpoint{5.153925in}{0.646697in}}%
\pgfpathcurveto{\pgfqpoint{5.161738in}{0.654511in}}{\pgfqpoint{5.166129in}{0.665110in}}{\pgfqpoint{5.166129in}{0.676160in}}%
\pgfpathcurveto{\pgfqpoint{5.166129in}{0.687210in}}{\pgfqpoint{5.161738in}{0.697809in}}{\pgfqpoint{5.153925in}{0.705623in}}%
\pgfpathcurveto{\pgfqpoint{5.146111in}{0.713437in}}{\pgfqpoint{5.135512in}{0.717827in}}{\pgfqpoint{5.124462in}{0.717827in}}%
\pgfpathcurveto{\pgfqpoint{5.113412in}{0.717827in}}{\pgfqpoint{5.102813in}{0.713437in}}{\pgfqpoint{5.094999in}{0.705623in}}%
\pgfpathcurveto{\pgfqpoint{5.087186in}{0.697809in}}{\pgfqpoint{5.082795in}{0.687210in}}{\pgfqpoint{5.082795in}{0.676160in}}%
\pgfpathcurveto{\pgfqpoint{5.082795in}{0.665110in}}{\pgfqpoint{5.087186in}{0.654511in}}{\pgfqpoint{5.094999in}{0.646697in}}%
\pgfpathcurveto{\pgfqpoint{5.102813in}{0.638884in}}{\pgfqpoint{5.113412in}{0.634494in}}{\pgfqpoint{5.124462in}{0.634494in}}%
\pgfpathlineto{\pgfqpoint{5.124462in}{0.634494in}}%
\pgfpathclose%
\pgfusepath{stroke}%
\end{pgfscope}%
\begin{pgfscope}%
\pgfpathrectangle{\pgfqpoint{0.847223in}{0.554012in}}{\pgfqpoint{6.200000in}{4.530000in}}%
\pgfusepath{clip}%
\pgfsetbuttcap%
\pgfsetroundjoin%
\pgfsetlinewidth{1.003750pt}%
\definecolor{currentstroke}{rgb}{1.000000,0.000000,0.000000}%
\pgfsetstrokecolor{currentstroke}%
\pgfsetdash{}{0pt}%
\pgfpathmoveto{\pgfqpoint{5.129795in}{0.633895in}}%
\pgfpathcurveto{\pgfqpoint{5.140845in}{0.633895in}}{\pgfqpoint{5.151444in}{0.638285in}}{\pgfqpoint{5.159258in}{0.646099in}}%
\pgfpathcurveto{\pgfqpoint{5.167072in}{0.653913in}}{\pgfqpoint{5.171462in}{0.664512in}}{\pgfqpoint{5.171462in}{0.675562in}}%
\pgfpathcurveto{\pgfqpoint{5.171462in}{0.686612in}}{\pgfqpoint{5.167072in}{0.697211in}}{\pgfqpoint{5.159258in}{0.705024in}}%
\pgfpathcurveto{\pgfqpoint{5.151444in}{0.712838in}}{\pgfqpoint{5.140845in}{0.717228in}}{\pgfqpoint{5.129795in}{0.717228in}}%
\pgfpathcurveto{\pgfqpoint{5.118745in}{0.717228in}}{\pgfqpoint{5.108146in}{0.712838in}}{\pgfqpoint{5.100332in}{0.705024in}}%
\pgfpathcurveto{\pgfqpoint{5.092519in}{0.697211in}}{\pgfqpoint{5.088129in}{0.686612in}}{\pgfqpoint{5.088129in}{0.675562in}}%
\pgfpathcurveto{\pgfqpoint{5.088129in}{0.664512in}}{\pgfqpoint{5.092519in}{0.653913in}}{\pgfqpoint{5.100332in}{0.646099in}}%
\pgfpathcurveto{\pgfqpoint{5.108146in}{0.638285in}}{\pgfqpoint{5.118745in}{0.633895in}}{\pgfqpoint{5.129795in}{0.633895in}}%
\pgfpathlineto{\pgfqpoint{5.129795in}{0.633895in}}%
\pgfpathclose%
\pgfusepath{stroke}%
\end{pgfscope}%
\begin{pgfscope}%
\pgfpathrectangle{\pgfqpoint{0.847223in}{0.554012in}}{\pgfqpoint{6.200000in}{4.530000in}}%
\pgfusepath{clip}%
\pgfsetbuttcap%
\pgfsetroundjoin%
\pgfsetlinewidth{1.003750pt}%
\definecolor{currentstroke}{rgb}{1.000000,0.000000,0.000000}%
\pgfsetstrokecolor{currentstroke}%
\pgfsetdash{}{0pt}%
\pgfpathmoveto{\pgfqpoint{5.135128in}{0.633298in}}%
\pgfpathcurveto{\pgfqpoint{5.146179in}{0.633298in}}{\pgfqpoint{5.156778in}{0.637688in}}{\pgfqpoint{5.164591in}{0.645502in}}%
\pgfpathcurveto{\pgfqpoint{5.172405in}{0.653315in}}{\pgfqpoint{5.176795in}{0.663914in}}{\pgfqpoint{5.176795in}{0.674965in}}%
\pgfpathcurveto{\pgfqpoint{5.176795in}{0.686015in}}{\pgfqpoint{5.172405in}{0.696614in}}{\pgfqpoint{5.164591in}{0.704427in}}%
\pgfpathcurveto{\pgfqpoint{5.156778in}{0.712241in}}{\pgfqpoint{5.146179in}{0.716631in}}{\pgfqpoint{5.135128in}{0.716631in}}%
\pgfpathcurveto{\pgfqpoint{5.124078in}{0.716631in}}{\pgfqpoint{5.113479in}{0.712241in}}{\pgfqpoint{5.105666in}{0.704427in}}%
\pgfpathcurveto{\pgfqpoint{5.097852in}{0.696614in}}{\pgfqpoint{5.093462in}{0.686015in}}{\pgfqpoint{5.093462in}{0.674965in}}%
\pgfpathcurveto{\pgfqpoint{5.093462in}{0.663914in}}{\pgfqpoint{5.097852in}{0.653315in}}{\pgfqpoint{5.105666in}{0.645502in}}%
\pgfpathcurveto{\pgfqpoint{5.113479in}{0.637688in}}{\pgfqpoint{5.124078in}{0.633298in}}{\pgfqpoint{5.135128in}{0.633298in}}%
\pgfpathlineto{\pgfqpoint{5.135128in}{0.633298in}}%
\pgfpathclose%
\pgfusepath{stroke}%
\end{pgfscope}%
\begin{pgfscope}%
\pgfpathrectangle{\pgfqpoint{0.847223in}{0.554012in}}{\pgfqpoint{6.200000in}{4.530000in}}%
\pgfusepath{clip}%
\pgfsetbuttcap%
\pgfsetroundjoin%
\pgfsetlinewidth{1.003750pt}%
\definecolor{currentstroke}{rgb}{1.000000,0.000000,0.000000}%
\pgfsetstrokecolor{currentstroke}%
\pgfsetdash{}{0pt}%
\pgfpathmoveto{\pgfqpoint{5.140462in}{0.632702in}}%
\pgfpathcurveto{\pgfqpoint{5.151512in}{0.632702in}}{\pgfqpoint{5.162111in}{0.637092in}}{\pgfqpoint{5.169924in}{0.644906in}}%
\pgfpathcurveto{\pgfqpoint{5.177738in}{0.652720in}}{\pgfqpoint{5.182128in}{0.663319in}}{\pgfqpoint{5.182128in}{0.674369in}}%
\pgfpathcurveto{\pgfqpoint{5.182128in}{0.685419in}}{\pgfqpoint{5.177738in}{0.696018in}}{\pgfqpoint{5.169924in}{0.703831in}}%
\pgfpathcurveto{\pgfqpoint{5.162111in}{0.711645in}}{\pgfqpoint{5.151512in}{0.716035in}}{\pgfqpoint{5.140462in}{0.716035in}}%
\pgfpathcurveto{\pgfqpoint{5.129412in}{0.716035in}}{\pgfqpoint{5.118813in}{0.711645in}}{\pgfqpoint{5.110999in}{0.703831in}}%
\pgfpathcurveto{\pgfqpoint{5.103185in}{0.696018in}}{\pgfqpoint{5.098795in}{0.685419in}}{\pgfqpoint{5.098795in}{0.674369in}}%
\pgfpathcurveto{\pgfqpoint{5.098795in}{0.663319in}}{\pgfqpoint{5.103185in}{0.652720in}}{\pgfqpoint{5.110999in}{0.644906in}}%
\pgfpathcurveto{\pgfqpoint{5.118813in}{0.637092in}}{\pgfqpoint{5.129412in}{0.632702in}}{\pgfqpoint{5.140462in}{0.632702in}}%
\pgfpathlineto{\pgfqpoint{5.140462in}{0.632702in}}%
\pgfpathclose%
\pgfusepath{stroke}%
\end{pgfscope}%
\begin{pgfscope}%
\pgfpathrectangle{\pgfqpoint{0.847223in}{0.554012in}}{\pgfqpoint{6.200000in}{4.530000in}}%
\pgfusepath{clip}%
\pgfsetbuttcap%
\pgfsetroundjoin%
\pgfsetlinewidth{1.003750pt}%
\definecolor{currentstroke}{rgb}{1.000000,0.000000,0.000000}%
\pgfsetstrokecolor{currentstroke}%
\pgfsetdash{}{0pt}%
\pgfpathmoveto{\pgfqpoint{5.145795in}{0.632107in}}%
\pgfpathcurveto{\pgfqpoint{5.156845in}{0.632107in}}{\pgfqpoint{5.167444in}{0.636498in}}{\pgfqpoint{5.175258in}{0.644311in}}%
\pgfpathcurveto{\pgfqpoint{5.183071in}{0.652125in}}{\pgfqpoint{5.187462in}{0.662724in}}{\pgfqpoint{5.187462in}{0.673774in}}%
\pgfpathcurveto{\pgfqpoint{5.187462in}{0.684824in}}{\pgfqpoint{5.183071in}{0.695423in}}{\pgfqpoint{5.175258in}{0.703237in}}%
\pgfpathcurveto{\pgfqpoint{5.167444in}{0.711051in}}{\pgfqpoint{5.156845in}{0.715441in}}{\pgfqpoint{5.145795in}{0.715441in}}%
\pgfpathcurveto{\pgfqpoint{5.134745in}{0.715441in}}{\pgfqpoint{5.124146in}{0.711051in}}{\pgfqpoint{5.116332in}{0.703237in}}%
\pgfpathcurveto{\pgfqpoint{5.108518in}{0.695423in}}{\pgfqpoint{5.104128in}{0.684824in}}{\pgfqpoint{5.104128in}{0.673774in}}%
\pgfpathcurveto{\pgfqpoint{5.104128in}{0.662724in}}{\pgfqpoint{5.108518in}{0.652125in}}{\pgfqpoint{5.116332in}{0.644311in}}%
\pgfpathcurveto{\pgfqpoint{5.124146in}{0.636498in}}{\pgfqpoint{5.134745in}{0.632107in}}{\pgfqpoint{5.145795in}{0.632107in}}%
\pgfpathlineto{\pgfqpoint{5.145795in}{0.632107in}}%
\pgfpathclose%
\pgfusepath{stroke}%
\end{pgfscope}%
\begin{pgfscope}%
\pgfpathrectangle{\pgfqpoint{0.847223in}{0.554012in}}{\pgfqpoint{6.200000in}{4.530000in}}%
\pgfusepath{clip}%
\pgfsetbuttcap%
\pgfsetroundjoin%
\pgfsetlinewidth{1.003750pt}%
\definecolor{currentstroke}{rgb}{1.000000,0.000000,0.000000}%
\pgfsetstrokecolor{currentstroke}%
\pgfsetdash{}{0pt}%
\pgfpathmoveto{\pgfqpoint{5.151128in}{0.631514in}}%
\pgfpathcurveto{\pgfqpoint{5.162178in}{0.631514in}}{\pgfqpoint{5.172777in}{0.635905in}}{\pgfqpoint{5.180591in}{0.643718in}}%
\pgfpathcurveto{\pgfqpoint{5.188404in}{0.651532in}}{\pgfqpoint{5.192795in}{0.662131in}}{\pgfqpoint{5.192795in}{0.673181in}}%
\pgfpathcurveto{\pgfqpoint{5.192795in}{0.684231in}}{\pgfqpoint{5.188404in}{0.694830in}}{\pgfqpoint{5.180591in}{0.702644in}}%
\pgfpathcurveto{\pgfqpoint{5.172777in}{0.710457in}}{\pgfqpoint{5.162178in}{0.714848in}}{\pgfqpoint{5.151128in}{0.714848in}}%
\pgfpathcurveto{\pgfqpoint{5.140078in}{0.714848in}}{\pgfqpoint{5.129479in}{0.710457in}}{\pgfqpoint{5.121665in}{0.702644in}}%
\pgfpathcurveto{\pgfqpoint{5.113852in}{0.694830in}}{\pgfqpoint{5.109461in}{0.684231in}}{\pgfqpoint{5.109461in}{0.673181in}}%
\pgfpathcurveto{\pgfqpoint{5.109461in}{0.662131in}}{\pgfqpoint{5.113852in}{0.651532in}}{\pgfqpoint{5.121665in}{0.643718in}}%
\pgfpathcurveto{\pgfqpoint{5.129479in}{0.635905in}}{\pgfqpoint{5.140078in}{0.631514in}}{\pgfqpoint{5.151128in}{0.631514in}}%
\pgfpathlineto{\pgfqpoint{5.151128in}{0.631514in}}%
\pgfpathclose%
\pgfusepath{stroke}%
\end{pgfscope}%
\begin{pgfscope}%
\pgfpathrectangle{\pgfqpoint{0.847223in}{0.554012in}}{\pgfqpoint{6.200000in}{4.530000in}}%
\pgfusepath{clip}%
\pgfsetbuttcap%
\pgfsetroundjoin%
\pgfsetlinewidth{1.003750pt}%
\definecolor{currentstroke}{rgb}{1.000000,0.000000,0.000000}%
\pgfsetstrokecolor{currentstroke}%
\pgfsetdash{}{0pt}%
\pgfpathmoveto{\pgfqpoint{5.156461in}{0.630922in}}%
\pgfpathcurveto{\pgfqpoint{5.167511in}{0.630922in}}{\pgfqpoint{5.178110in}{0.635313in}}{\pgfqpoint{5.185924in}{0.643126in}}%
\pgfpathcurveto{\pgfqpoint{5.193738in}{0.650940in}}{\pgfqpoint{5.198128in}{0.661539in}}{\pgfqpoint{5.198128in}{0.672589in}}%
\pgfpathcurveto{\pgfqpoint{5.198128in}{0.683639in}}{\pgfqpoint{5.193738in}{0.694238in}}{\pgfqpoint{5.185924in}{0.702052in}}%
\pgfpathcurveto{\pgfqpoint{5.178110in}{0.709865in}}{\pgfqpoint{5.167511in}{0.714256in}}{\pgfqpoint{5.156461in}{0.714256in}}%
\pgfpathcurveto{\pgfqpoint{5.145411in}{0.714256in}}{\pgfqpoint{5.134812in}{0.709865in}}{\pgfqpoint{5.126999in}{0.702052in}}%
\pgfpathcurveto{\pgfqpoint{5.119185in}{0.694238in}}{\pgfqpoint{5.114795in}{0.683639in}}{\pgfqpoint{5.114795in}{0.672589in}}%
\pgfpathcurveto{\pgfqpoint{5.114795in}{0.661539in}}{\pgfqpoint{5.119185in}{0.650940in}}{\pgfqpoint{5.126999in}{0.643126in}}%
\pgfpathcurveto{\pgfqpoint{5.134812in}{0.635313in}}{\pgfqpoint{5.145411in}{0.630922in}}{\pgfqpoint{5.156461in}{0.630922in}}%
\pgfpathlineto{\pgfqpoint{5.156461in}{0.630922in}}%
\pgfpathclose%
\pgfusepath{stroke}%
\end{pgfscope}%
\begin{pgfscope}%
\pgfpathrectangle{\pgfqpoint{0.847223in}{0.554012in}}{\pgfqpoint{6.200000in}{4.530000in}}%
\pgfusepath{clip}%
\pgfsetbuttcap%
\pgfsetroundjoin%
\pgfsetlinewidth{1.003750pt}%
\definecolor{currentstroke}{rgb}{1.000000,0.000000,0.000000}%
\pgfsetstrokecolor{currentstroke}%
\pgfsetdash{}{0pt}%
\pgfpathmoveto{\pgfqpoint{5.161795in}{0.630332in}}%
\pgfpathcurveto{\pgfqpoint{5.172845in}{0.630332in}}{\pgfqpoint{5.183444in}{0.634722in}}{\pgfqpoint{5.191257in}{0.642536in}}%
\pgfpathcurveto{\pgfqpoint{5.199071in}{0.650349in}}{\pgfqpoint{5.203461in}{0.660948in}}{\pgfqpoint{5.203461in}{0.671998in}}%
\pgfpathcurveto{\pgfqpoint{5.203461in}{0.683049in}}{\pgfqpoint{5.199071in}{0.693648in}}{\pgfqpoint{5.191257in}{0.701461in}}%
\pgfpathcurveto{\pgfqpoint{5.183444in}{0.709275in}}{\pgfqpoint{5.172845in}{0.713665in}}{\pgfqpoint{5.161795in}{0.713665in}}%
\pgfpathcurveto{\pgfqpoint{5.150744in}{0.713665in}}{\pgfqpoint{5.140145in}{0.709275in}}{\pgfqpoint{5.132332in}{0.701461in}}%
\pgfpathcurveto{\pgfqpoint{5.124518in}{0.693648in}}{\pgfqpoint{5.120128in}{0.683049in}}{\pgfqpoint{5.120128in}{0.671998in}}%
\pgfpathcurveto{\pgfqpoint{5.120128in}{0.660948in}}{\pgfqpoint{5.124518in}{0.650349in}}{\pgfqpoint{5.132332in}{0.642536in}}%
\pgfpathcurveto{\pgfqpoint{5.140145in}{0.634722in}}{\pgfqpoint{5.150744in}{0.630332in}}{\pgfqpoint{5.161795in}{0.630332in}}%
\pgfpathlineto{\pgfqpoint{5.161795in}{0.630332in}}%
\pgfpathclose%
\pgfusepath{stroke}%
\end{pgfscope}%
\begin{pgfscope}%
\pgfpathrectangle{\pgfqpoint{0.847223in}{0.554012in}}{\pgfqpoint{6.200000in}{4.530000in}}%
\pgfusepath{clip}%
\pgfsetbuttcap%
\pgfsetroundjoin%
\pgfsetlinewidth{1.003750pt}%
\definecolor{currentstroke}{rgb}{1.000000,0.000000,0.000000}%
\pgfsetstrokecolor{currentstroke}%
\pgfsetdash{}{0pt}%
\pgfpathmoveto{\pgfqpoint{5.167128in}{0.629742in}}%
\pgfpathcurveto{\pgfqpoint{5.178178in}{0.629742in}}{\pgfqpoint{5.188777in}{0.634133in}}{\pgfqpoint{5.196591in}{0.641946in}}%
\pgfpathcurveto{\pgfqpoint{5.204404in}{0.649760in}}{\pgfqpoint{5.208794in}{0.660359in}}{\pgfqpoint{5.208794in}{0.671409in}}%
\pgfpathcurveto{\pgfqpoint{5.208794in}{0.682459in}}{\pgfqpoint{5.204404in}{0.693058in}}{\pgfqpoint{5.196591in}{0.700872in}}%
\pgfpathcurveto{\pgfqpoint{5.188777in}{0.708685in}}{\pgfqpoint{5.178178in}{0.713076in}}{\pgfqpoint{5.167128in}{0.713076in}}%
\pgfpathcurveto{\pgfqpoint{5.156078in}{0.713076in}}{\pgfqpoint{5.145479in}{0.708685in}}{\pgfqpoint{5.137665in}{0.700872in}}%
\pgfpathcurveto{\pgfqpoint{5.129851in}{0.693058in}}{\pgfqpoint{5.125461in}{0.682459in}}{\pgfqpoint{5.125461in}{0.671409in}}%
\pgfpathcurveto{\pgfqpoint{5.125461in}{0.660359in}}{\pgfqpoint{5.129851in}{0.649760in}}{\pgfqpoint{5.137665in}{0.641946in}}%
\pgfpathcurveto{\pgfqpoint{5.145479in}{0.634133in}}{\pgfqpoint{5.156078in}{0.629742in}}{\pgfqpoint{5.167128in}{0.629742in}}%
\pgfpathlineto{\pgfqpoint{5.167128in}{0.629742in}}%
\pgfpathclose%
\pgfusepath{stroke}%
\end{pgfscope}%
\begin{pgfscope}%
\pgfpathrectangle{\pgfqpoint{0.847223in}{0.554012in}}{\pgfqpoint{6.200000in}{4.530000in}}%
\pgfusepath{clip}%
\pgfsetbuttcap%
\pgfsetroundjoin%
\pgfsetlinewidth{1.003750pt}%
\definecolor{currentstroke}{rgb}{1.000000,0.000000,0.000000}%
\pgfsetstrokecolor{currentstroke}%
\pgfsetdash{}{0pt}%
\pgfpathmoveto{\pgfqpoint{5.172461in}{0.629154in}}%
\pgfpathcurveto{\pgfqpoint{5.183511in}{0.629154in}}{\pgfqpoint{5.194110in}{0.633545in}}{\pgfqpoint{5.201924in}{0.641358in}}%
\pgfpathcurveto{\pgfqpoint{5.209737in}{0.649172in}}{\pgfqpoint{5.214128in}{0.659771in}}{\pgfqpoint{5.214128in}{0.670821in}}%
\pgfpathcurveto{\pgfqpoint{5.214128in}{0.681871in}}{\pgfqpoint{5.209737in}{0.692470in}}{\pgfqpoint{5.201924in}{0.700284in}}%
\pgfpathcurveto{\pgfqpoint{5.194110in}{0.708097in}}{\pgfqpoint{5.183511in}{0.712488in}}{\pgfqpoint{5.172461in}{0.712488in}}%
\pgfpathcurveto{\pgfqpoint{5.161411in}{0.712488in}}{\pgfqpoint{5.150812in}{0.708097in}}{\pgfqpoint{5.142998in}{0.700284in}}%
\pgfpathcurveto{\pgfqpoint{5.135185in}{0.692470in}}{\pgfqpoint{5.130794in}{0.681871in}}{\pgfqpoint{5.130794in}{0.670821in}}%
\pgfpathcurveto{\pgfqpoint{5.130794in}{0.659771in}}{\pgfqpoint{5.135185in}{0.649172in}}{\pgfqpoint{5.142998in}{0.641358in}}%
\pgfpathcurveto{\pgfqpoint{5.150812in}{0.633545in}}{\pgfqpoint{5.161411in}{0.629154in}}{\pgfqpoint{5.172461in}{0.629154in}}%
\pgfpathlineto{\pgfqpoint{5.172461in}{0.629154in}}%
\pgfpathclose%
\pgfusepath{stroke}%
\end{pgfscope}%
\begin{pgfscope}%
\pgfpathrectangle{\pgfqpoint{0.847223in}{0.554012in}}{\pgfqpoint{6.200000in}{4.530000in}}%
\pgfusepath{clip}%
\pgfsetbuttcap%
\pgfsetroundjoin%
\pgfsetlinewidth{1.003750pt}%
\definecolor{currentstroke}{rgb}{1.000000,0.000000,0.000000}%
\pgfsetstrokecolor{currentstroke}%
\pgfsetdash{}{0pt}%
\pgfpathmoveto{\pgfqpoint{5.177794in}{0.628568in}}%
\pgfpathcurveto{\pgfqpoint{5.188844in}{0.628568in}}{\pgfqpoint{5.199443in}{0.632958in}}{\pgfqpoint{5.207257in}{0.640772in}}%
\pgfpathcurveto{\pgfqpoint{5.215071in}{0.648585in}}{\pgfqpoint{5.219461in}{0.659184in}}{\pgfqpoint{5.219461in}{0.670234in}}%
\pgfpathcurveto{\pgfqpoint{5.219461in}{0.681284in}}{\pgfqpoint{5.215071in}{0.691884in}}{\pgfqpoint{5.207257in}{0.699697in}}%
\pgfpathcurveto{\pgfqpoint{5.199443in}{0.707511in}}{\pgfqpoint{5.188844in}{0.711901in}}{\pgfqpoint{5.177794in}{0.711901in}}%
\pgfpathcurveto{\pgfqpoint{5.166744in}{0.711901in}}{\pgfqpoint{5.156145in}{0.707511in}}{\pgfqpoint{5.148331in}{0.699697in}}%
\pgfpathcurveto{\pgfqpoint{5.140518in}{0.691884in}}{\pgfqpoint{5.136128in}{0.681284in}}{\pgfqpoint{5.136128in}{0.670234in}}%
\pgfpathcurveto{\pgfqpoint{5.136128in}{0.659184in}}{\pgfqpoint{5.140518in}{0.648585in}}{\pgfqpoint{5.148331in}{0.640772in}}%
\pgfpathcurveto{\pgfqpoint{5.156145in}{0.632958in}}{\pgfqpoint{5.166744in}{0.628568in}}{\pgfqpoint{5.177794in}{0.628568in}}%
\pgfpathlineto{\pgfqpoint{5.177794in}{0.628568in}}%
\pgfpathclose%
\pgfusepath{stroke}%
\end{pgfscope}%
\begin{pgfscope}%
\pgfpathrectangle{\pgfqpoint{0.847223in}{0.554012in}}{\pgfqpoint{6.200000in}{4.530000in}}%
\pgfusepath{clip}%
\pgfsetbuttcap%
\pgfsetroundjoin%
\pgfsetlinewidth{1.003750pt}%
\definecolor{currentstroke}{rgb}{1.000000,0.000000,0.000000}%
\pgfsetstrokecolor{currentstroke}%
\pgfsetdash{}{0pt}%
\pgfpathmoveto{\pgfqpoint{5.183127in}{0.627982in}}%
\pgfpathcurveto{\pgfqpoint{5.194178in}{0.627982in}}{\pgfqpoint{5.204777in}{0.632373in}}{\pgfqpoint{5.212590in}{0.640186in}}%
\pgfpathcurveto{\pgfqpoint{5.220404in}{0.648000in}}{\pgfqpoint{5.224794in}{0.658599in}}{\pgfqpoint{5.224794in}{0.669649in}}%
\pgfpathcurveto{\pgfqpoint{5.224794in}{0.680699in}}{\pgfqpoint{5.220404in}{0.691298in}}{\pgfqpoint{5.212590in}{0.699112in}}%
\pgfpathcurveto{\pgfqpoint{5.204777in}{0.706925in}}{\pgfqpoint{5.194178in}{0.711316in}}{\pgfqpoint{5.183127in}{0.711316in}}%
\pgfpathcurveto{\pgfqpoint{5.172077in}{0.711316in}}{\pgfqpoint{5.161478in}{0.706925in}}{\pgfqpoint{5.153665in}{0.699112in}}%
\pgfpathcurveto{\pgfqpoint{5.145851in}{0.691298in}}{\pgfqpoint{5.141461in}{0.680699in}}{\pgfqpoint{5.141461in}{0.669649in}}%
\pgfpathcurveto{\pgfqpoint{5.141461in}{0.658599in}}{\pgfqpoint{5.145851in}{0.648000in}}{\pgfqpoint{5.153665in}{0.640186in}}%
\pgfpathcurveto{\pgfqpoint{5.161478in}{0.632373in}}{\pgfqpoint{5.172077in}{0.627982in}}{\pgfqpoint{5.183127in}{0.627982in}}%
\pgfpathlineto{\pgfqpoint{5.183127in}{0.627982in}}%
\pgfpathclose%
\pgfusepath{stroke}%
\end{pgfscope}%
\begin{pgfscope}%
\pgfpathrectangle{\pgfqpoint{0.847223in}{0.554012in}}{\pgfqpoint{6.200000in}{4.530000in}}%
\pgfusepath{clip}%
\pgfsetbuttcap%
\pgfsetroundjoin%
\pgfsetlinewidth{1.003750pt}%
\definecolor{currentstroke}{rgb}{1.000000,0.000000,0.000000}%
\pgfsetstrokecolor{currentstroke}%
\pgfsetdash{}{0pt}%
\pgfpathmoveto{\pgfqpoint{5.188461in}{0.627398in}}%
\pgfpathcurveto{\pgfqpoint{5.199511in}{0.627398in}}{\pgfqpoint{5.210110in}{0.631788in}}{\pgfqpoint{5.217923in}{0.639602in}}%
\pgfpathcurveto{\pgfqpoint{5.225737in}{0.647416in}}{\pgfqpoint{5.230127in}{0.658015in}}{\pgfqpoint{5.230127in}{0.669065in}}%
\pgfpathcurveto{\pgfqpoint{5.230127in}{0.680115in}}{\pgfqpoint{5.225737in}{0.690714in}}{\pgfqpoint{5.217923in}{0.698528in}}%
\pgfpathcurveto{\pgfqpoint{5.210110in}{0.706341in}}{\pgfqpoint{5.199511in}{0.710731in}}{\pgfqpoint{5.188461in}{0.710731in}}%
\pgfpathcurveto{\pgfqpoint{5.177410in}{0.710731in}}{\pgfqpoint{5.166811in}{0.706341in}}{\pgfqpoint{5.158998in}{0.698528in}}%
\pgfpathcurveto{\pgfqpoint{5.151184in}{0.690714in}}{\pgfqpoint{5.146794in}{0.680115in}}{\pgfqpoint{5.146794in}{0.669065in}}%
\pgfpathcurveto{\pgfqpoint{5.146794in}{0.658015in}}{\pgfqpoint{5.151184in}{0.647416in}}{\pgfqpoint{5.158998in}{0.639602in}}%
\pgfpathcurveto{\pgfqpoint{5.166811in}{0.631788in}}{\pgfqpoint{5.177410in}{0.627398in}}{\pgfqpoint{5.188461in}{0.627398in}}%
\pgfpathlineto{\pgfqpoint{5.188461in}{0.627398in}}%
\pgfpathclose%
\pgfusepath{stroke}%
\end{pgfscope}%
\begin{pgfscope}%
\pgfpathrectangle{\pgfqpoint{0.847223in}{0.554012in}}{\pgfqpoint{6.200000in}{4.530000in}}%
\pgfusepath{clip}%
\pgfsetbuttcap%
\pgfsetroundjoin%
\pgfsetlinewidth{1.003750pt}%
\definecolor{currentstroke}{rgb}{1.000000,0.000000,0.000000}%
\pgfsetstrokecolor{currentstroke}%
\pgfsetdash{}{0pt}%
\pgfpathmoveto{\pgfqpoint{5.193794in}{0.626815in}}%
\pgfpathcurveto{\pgfqpoint{5.204844in}{0.626815in}}{\pgfqpoint{5.215443in}{0.631205in}}{\pgfqpoint{5.223257in}{0.639019in}}%
\pgfpathcurveto{\pgfqpoint{5.231070in}{0.646833in}}{\pgfqpoint{5.235460in}{0.657432in}}{\pgfqpoint{5.235460in}{0.668482in}}%
\pgfpathcurveto{\pgfqpoint{5.235460in}{0.679532in}}{\pgfqpoint{5.231070in}{0.690131in}}{\pgfqpoint{5.223257in}{0.697945in}}%
\pgfpathcurveto{\pgfqpoint{5.215443in}{0.705758in}}{\pgfqpoint{5.204844in}{0.710149in}}{\pgfqpoint{5.193794in}{0.710149in}}%
\pgfpathcurveto{\pgfqpoint{5.182744in}{0.710149in}}{\pgfqpoint{5.172145in}{0.705758in}}{\pgfqpoint{5.164331in}{0.697945in}}%
\pgfpathcurveto{\pgfqpoint{5.156517in}{0.690131in}}{\pgfqpoint{5.152127in}{0.679532in}}{\pgfqpoint{5.152127in}{0.668482in}}%
\pgfpathcurveto{\pgfqpoint{5.152127in}{0.657432in}}{\pgfqpoint{5.156517in}{0.646833in}}{\pgfqpoint{5.164331in}{0.639019in}}%
\pgfpathcurveto{\pgfqpoint{5.172145in}{0.631205in}}{\pgfqpoint{5.182744in}{0.626815in}}{\pgfqpoint{5.193794in}{0.626815in}}%
\pgfpathlineto{\pgfqpoint{5.193794in}{0.626815in}}%
\pgfpathclose%
\pgfusepath{stroke}%
\end{pgfscope}%
\begin{pgfscope}%
\pgfpathrectangle{\pgfqpoint{0.847223in}{0.554012in}}{\pgfqpoint{6.200000in}{4.530000in}}%
\pgfusepath{clip}%
\pgfsetbuttcap%
\pgfsetroundjoin%
\pgfsetlinewidth{1.003750pt}%
\definecolor{currentstroke}{rgb}{1.000000,0.000000,0.000000}%
\pgfsetstrokecolor{currentstroke}%
\pgfsetdash{}{0pt}%
\pgfpathmoveto{\pgfqpoint{5.199127in}{0.626234in}}%
\pgfpathcurveto{\pgfqpoint{5.210177in}{0.626234in}}{\pgfqpoint{5.220776in}{0.630624in}}{\pgfqpoint{5.228590in}{0.638438in}}%
\pgfpathcurveto{\pgfqpoint{5.236403in}{0.646251in}}{\pgfqpoint{5.240794in}{0.656850in}}{\pgfqpoint{5.240794in}{0.667900in}}%
\pgfpathcurveto{\pgfqpoint{5.240794in}{0.678950in}}{\pgfqpoint{5.236403in}{0.689549in}}{\pgfqpoint{5.228590in}{0.697363in}}%
\pgfpathcurveto{\pgfqpoint{5.220776in}{0.705177in}}{\pgfqpoint{5.210177in}{0.709567in}}{\pgfqpoint{5.199127in}{0.709567in}}%
\pgfpathcurveto{\pgfqpoint{5.188077in}{0.709567in}}{\pgfqpoint{5.177478in}{0.705177in}}{\pgfqpoint{5.169664in}{0.697363in}}%
\pgfpathcurveto{\pgfqpoint{5.161851in}{0.689549in}}{\pgfqpoint{5.157460in}{0.678950in}}{\pgfqpoint{5.157460in}{0.667900in}}%
\pgfpathcurveto{\pgfqpoint{5.157460in}{0.656850in}}{\pgfqpoint{5.161851in}{0.646251in}}{\pgfqpoint{5.169664in}{0.638438in}}%
\pgfpathcurveto{\pgfqpoint{5.177478in}{0.630624in}}{\pgfqpoint{5.188077in}{0.626234in}}{\pgfqpoint{5.199127in}{0.626234in}}%
\pgfpathlineto{\pgfqpoint{5.199127in}{0.626234in}}%
\pgfpathclose%
\pgfusepath{stroke}%
\end{pgfscope}%
\begin{pgfscope}%
\pgfpathrectangle{\pgfqpoint{0.847223in}{0.554012in}}{\pgfqpoint{6.200000in}{4.530000in}}%
\pgfusepath{clip}%
\pgfsetbuttcap%
\pgfsetroundjoin%
\pgfsetlinewidth{1.003750pt}%
\definecolor{currentstroke}{rgb}{1.000000,0.000000,0.000000}%
\pgfsetstrokecolor{currentstroke}%
\pgfsetdash{}{0pt}%
\pgfpathmoveto{\pgfqpoint{5.204460in}{0.625653in}}%
\pgfpathcurveto{\pgfqpoint{5.215510in}{0.625653in}}{\pgfqpoint{5.226109in}{0.630044in}}{\pgfqpoint{5.233923in}{0.637857in}}%
\pgfpathcurveto{\pgfqpoint{5.241737in}{0.645671in}}{\pgfqpoint{5.246127in}{0.656270in}}{\pgfqpoint{5.246127in}{0.667320in}}%
\pgfpathcurveto{\pgfqpoint{5.246127in}{0.678370in}}{\pgfqpoint{5.241737in}{0.688969in}}{\pgfqpoint{5.233923in}{0.696783in}}%
\pgfpathcurveto{\pgfqpoint{5.226109in}{0.704596in}}{\pgfqpoint{5.215510in}{0.708987in}}{\pgfqpoint{5.204460in}{0.708987in}}%
\pgfpathcurveto{\pgfqpoint{5.193410in}{0.708987in}}{\pgfqpoint{5.182811in}{0.704596in}}{\pgfqpoint{5.174997in}{0.696783in}}%
\pgfpathcurveto{\pgfqpoint{5.167184in}{0.688969in}}{\pgfqpoint{5.162794in}{0.678370in}}{\pgfqpoint{5.162794in}{0.667320in}}%
\pgfpathcurveto{\pgfqpoint{5.162794in}{0.656270in}}{\pgfqpoint{5.167184in}{0.645671in}}{\pgfqpoint{5.174997in}{0.637857in}}%
\pgfpathcurveto{\pgfqpoint{5.182811in}{0.630044in}}{\pgfqpoint{5.193410in}{0.625653in}}{\pgfqpoint{5.204460in}{0.625653in}}%
\pgfpathlineto{\pgfqpoint{5.204460in}{0.625653in}}%
\pgfpathclose%
\pgfusepath{stroke}%
\end{pgfscope}%
\begin{pgfscope}%
\pgfpathrectangle{\pgfqpoint{0.847223in}{0.554012in}}{\pgfqpoint{6.200000in}{4.530000in}}%
\pgfusepath{clip}%
\pgfsetbuttcap%
\pgfsetroundjoin%
\pgfsetlinewidth{1.003750pt}%
\definecolor{currentstroke}{rgb}{1.000000,0.000000,0.000000}%
\pgfsetstrokecolor{currentstroke}%
\pgfsetdash{}{0pt}%
\pgfpathmoveto{\pgfqpoint{5.209793in}{0.625074in}}%
\pgfpathcurveto{\pgfqpoint{5.220844in}{0.625074in}}{\pgfqpoint{5.231443in}{0.629465in}}{\pgfqpoint{5.239256in}{0.637278in}}%
\pgfpathcurveto{\pgfqpoint{5.247070in}{0.645092in}}{\pgfqpoint{5.251460in}{0.655691in}}{\pgfqpoint{5.251460in}{0.666741in}}%
\pgfpathcurveto{\pgfqpoint{5.251460in}{0.677791in}}{\pgfqpoint{5.247070in}{0.688390in}}{\pgfqpoint{5.239256in}{0.696204in}}%
\pgfpathcurveto{\pgfqpoint{5.231443in}{0.704017in}}{\pgfqpoint{5.220844in}{0.708408in}}{\pgfqpoint{5.209793in}{0.708408in}}%
\pgfpathcurveto{\pgfqpoint{5.198743in}{0.708408in}}{\pgfqpoint{5.188144in}{0.704017in}}{\pgfqpoint{5.180331in}{0.696204in}}%
\pgfpathcurveto{\pgfqpoint{5.172517in}{0.688390in}}{\pgfqpoint{5.168127in}{0.677791in}}{\pgfqpoint{5.168127in}{0.666741in}}%
\pgfpathcurveto{\pgfqpoint{5.168127in}{0.655691in}}{\pgfqpoint{5.172517in}{0.645092in}}{\pgfqpoint{5.180331in}{0.637278in}}%
\pgfpathcurveto{\pgfqpoint{5.188144in}{0.629465in}}{\pgfqpoint{5.198743in}{0.625074in}}{\pgfqpoint{5.209793in}{0.625074in}}%
\pgfpathlineto{\pgfqpoint{5.209793in}{0.625074in}}%
\pgfpathclose%
\pgfusepath{stroke}%
\end{pgfscope}%
\begin{pgfscope}%
\pgfpathrectangle{\pgfqpoint{0.847223in}{0.554012in}}{\pgfqpoint{6.200000in}{4.530000in}}%
\pgfusepath{clip}%
\pgfsetbuttcap%
\pgfsetroundjoin%
\pgfsetlinewidth{1.003750pt}%
\definecolor{currentstroke}{rgb}{1.000000,0.000000,0.000000}%
\pgfsetstrokecolor{currentstroke}%
\pgfsetdash{}{0pt}%
\pgfpathmoveto{\pgfqpoint{5.215127in}{0.624496in}}%
\pgfpathcurveto{\pgfqpoint{5.226177in}{0.624496in}}{\pgfqpoint{5.236776in}{0.628887in}}{\pgfqpoint{5.244589in}{0.636700in}}%
\pgfpathcurveto{\pgfqpoint{5.252403in}{0.644514in}}{\pgfqpoint{5.256793in}{0.655113in}}{\pgfqpoint{5.256793in}{0.666163in}}%
\pgfpathcurveto{\pgfqpoint{5.256793in}{0.677213in}}{\pgfqpoint{5.252403in}{0.687812in}}{\pgfqpoint{5.244589in}{0.695626in}}%
\pgfpathcurveto{\pgfqpoint{5.236776in}{0.703440in}}{\pgfqpoint{5.226177in}{0.707830in}}{\pgfqpoint{5.215127in}{0.707830in}}%
\pgfpathcurveto{\pgfqpoint{5.204077in}{0.707830in}}{\pgfqpoint{5.193478in}{0.703440in}}{\pgfqpoint{5.185664in}{0.695626in}}%
\pgfpathcurveto{\pgfqpoint{5.177850in}{0.687812in}}{\pgfqpoint{5.173460in}{0.677213in}}{\pgfqpoint{5.173460in}{0.666163in}}%
\pgfpathcurveto{\pgfqpoint{5.173460in}{0.655113in}}{\pgfqpoint{5.177850in}{0.644514in}}{\pgfqpoint{5.185664in}{0.636700in}}%
\pgfpathcurveto{\pgfqpoint{5.193478in}{0.628887in}}{\pgfqpoint{5.204077in}{0.624496in}}{\pgfqpoint{5.215127in}{0.624496in}}%
\pgfpathlineto{\pgfqpoint{5.215127in}{0.624496in}}%
\pgfpathclose%
\pgfusepath{stroke}%
\end{pgfscope}%
\begin{pgfscope}%
\pgfpathrectangle{\pgfqpoint{0.847223in}{0.554012in}}{\pgfqpoint{6.200000in}{4.530000in}}%
\pgfusepath{clip}%
\pgfsetbuttcap%
\pgfsetroundjoin%
\pgfsetlinewidth{1.003750pt}%
\definecolor{currentstroke}{rgb}{1.000000,0.000000,0.000000}%
\pgfsetstrokecolor{currentstroke}%
\pgfsetdash{}{0pt}%
\pgfpathmoveto{\pgfqpoint{5.220460in}{0.623920in}}%
\pgfpathcurveto{\pgfqpoint{5.231510in}{0.623920in}}{\pgfqpoint{5.242109in}{0.628310in}}{\pgfqpoint{5.249923in}{0.636124in}}%
\pgfpathcurveto{\pgfqpoint{5.257736in}{0.643937in}}{\pgfqpoint{5.262127in}{0.654536in}}{\pgfqpoint{5.262127in}{0.665587in}}%
\pgfpathcurveto{\pgfqpoint{5.262127in}{0.676637in}}{\pgfqpoint{5.257736in}{0.687236in}}{\pgfqpoint{5.249923in}{0.695049in}}%
\pgfpathcurveto{\pgfqpoint{5.242109in}{0.702863in}}{\pgfqpoint{5.231510in}{0.707253in}}{\pgfqpoint{5.220460in}{0.707253in}}%
\pgfpathcurveto{\pgfqpoint{5.209410in}{0.707253in}}{\pgfqpoint{5.198811in}{0.702863in}}{\pgfqpoint{5.190997in}{0.695049in}}%
\pgfpathcurveto{\pgfqpoint{5.183183in}{0.687236in}}{\pgfqpoint{5.178793in}{0.676637in}}{\pgfqpoint{5.178793in}{0.665587in}}%
\pgfpathcurveto{\pgfqpoint{5.178793in}{0.654536in}}{\pgfqpoint{5.183183in}{0.643937in}}{\pgfqpoint{5.190997in}{0.636124in}}%
\pgfpathcurveto{\pgfqpoint{5.198811in}{0.628310in}}{\pgfqpoint{5.209410in}{0.623920in}}{\pgfqpoint{5.220460in}{0.623920in}}%
\pgfpathlineto{\pgfqpoint{5.220460in}{0.623920in}}%
\pgfpathclose%
\pgfusepath{stroke}%
\end{pgfscope}%
\begin{pgfscope}%
\pgfpathrectangle{\pgfqpoint{0.847223in}{0.554012in}}{\pgfqpoint{6.200000in}{4.530000in}}%
\pgfusepath{clip}%
\pgfsetbuttcap%
\pgfsetroundjoin%
\pgfsetlinewidth{1.003750pt}%
\definecolor{currentstroke}{rgb}{1.000000,0.000000,0.000000}%
\pgfsetstrokecolor{currentstroke}%
\pgfsetdash{}{0pt}%
\pgfpathmoveto{\pgfqpoint{5.225793in}{0.623345in}}%
\pgfpathcurveto{\pgfqpoint{5.236843in}{0.623345in}}{\pgfqpoint{5.247442in}{0.627735in}}{\pgfqpoint{5.255256in}{0.635549in}}%
\pgfpathcurveto{\pgfqpoint{5.263070in}{0.643362in}}{\pgfqpoint{5.267460in}{0.653961in}}{\pgfqpoint{5.267460in}{0.665011in}}%
\pgfpathcurveto{\pgfqpoint{5.267460in}{0.676061in}}{\pgfqpoint{5.263070in}{0.686660in}}{\pgfqpoint{5.255256in}{0.694474in}}%
\pgfpathcurveto{\pgfqpoint{5.247442in}{0.702288in}}{\pgfqpoint{5.236843in}{0.706678in}}{\pgfqpoint{5.225793in}{0.706678in}}%
\pgfpathcurveto{\pgfqpoint{5.214743in}{0.706678in}}{\pgfqpoint{5.204144in}{0.702288in}}{\pgfqpoint{5.196330in}{0.694474in}}%
\pgfpathcurveto{\pgfqpoint{5.188517in}{0.686660in}}{\pgfqpoint{5.184126in}{0.676061in}}{\pgfqpoint{5.184126in}{0.665011in}}%
\pgfpathcurveto{\pgfqpoint{5.184126in}{0.653961in}}{\pgfqpoint{5.188517in}{0.643362in}}{\pgfqpoint{5.196330in}{0.635549in}}%
\pgfpathcurveto{\pgfqpoint{5.204144in}{0.627735in}}{\pgfqpoint{5.214743in}{0.623345in}}{\pgfqpoint{5.225793in}{0.623345in}}%
\pgfpathlineto{\pgfqpoint{5.225793in}{0.623345in}}%
\pgfpathclose%
\pgfusepath{stroke}%
\end{pgfscope}%
\begin{pgfscope}%
\pgfpathrectangle{\pgfqpoint{0.847223in}{0.554012in}}{\pgfqpoint{6.200000in}{4.530000in}}%
\pgfusepath{clip}%
\pgfsetbuttcap%
\pgfsetroundjoin%
\pgfsetlinewidth{1.003750pt}%
\definecolor{currentstroke}{rgb}{1.000000,0.000000,0.000000}%
\pgfsetstrokecolor{currentstroke}%
\pgfsetdash{}{0pt}%
\pgfpathmoveto{\pgfqpoint{5.231126in}{0.622771in}}%
\pgfpathcurveto{\pgfqpoint{5.242176in}{0.622771in}}{\pgfqpoint{5.252775in}{0.627161in}}{\pgfqpoint{5.260589in}{0.634974in}}%
\pgfpathcurveto{\pgfqpoint{5.268403in}{0.642788in}}{\pgfqpoint{5.272793in}{0.653387in}}{\pgfqpoint{5.272793in}{0.664437in}}%
\pgfpathcurveto{\pgfqpoint{5.272793in}{0.675487in}}{\pgfqpoint{5.268403in}{0.686086in}}{\pgfqpoint{5.260589in}{0.693900in}}%
\pgfpathcurveto{\pgfqpoint{5.252775in}{0.701714in}}{\pgfqpoint{5.242176in}{0.706104in}}{\pgfqpoint{5.231126in}{0.706104in}}%
\pgfpathcurveto{\pgfqpoint{5.220076in}{0.706104in}}{\pgfqpoint{5.209477in}{0.701714in}}{\pgfqpoint{5.201664in}{0.693900in}}%
\pgfpathcurveto{\pgfqpoint{5.193850in}{0.686086in}}{\pgfqpoint{5.189460in}{0.675487in}}{\pgfqpoint{5.189460in}{0.664437in}}%
\pgfpathcurveto{\pgfqpoint{5.189460in}{0.653387in}}{\pgfqpoint{5.193850in}{0.642788in}}{\pgfqpoint{5.201664in}{0.634974in}}%
\pgfpathcurveto{\pgfqpoint{5.209477in}{0.627161in}}{\pgfqpoint{5.220076in}{0.622771in}}{\pgfqpoint{5.231126in}{0.622771in}}%
\pgfpathlineto{\pgfqpoint{5.231126in}{0.622771in}}%
\pgfpathclose%
\pgfusepath{stroke}%
\end{pgfscope}%
\begin{pgfscope}%
\pgfpathrectangle{\pgfqpoint{0.847223in}{0.554012in}}{\pgfqpoint{6.200000in}{4.530000in}}%
\pgfusepath{clip}%
\pgfsetbuttcap%
\pgfsetroundjoin%
\pgfsetlinewidth{1.003750pt}%
\definecolor{currentstroke}{rgb}{1.000000,0.000000,0.000000}%
\pgfsetstrokecolor{currentstroke}%
\pgfsetdash{}{0pt}%
\pgfpathmoveto{\pgfqpoint{5.236460in}{0.622198in}}%
\pgfpathcurveto{\pgfqpoint{5.247510in}{0.622198in}}{\pgfqpoint{5.258109in}{0.626588in}}{\pgfqpoint{5.265922in}{0.634402in}}%
\pgfpathcurveto{\pgfqpoint{5.273736in}{0.642215in}}{\pgfqpoint{5.278126in}{0.652814in}}{\pgfqpoint{5.278126in}{0.663864in}}%
\pgfpathcurveto{\pgfqpoint{5.278126in}{0.674915in}}{\pgfqpoint{5.273736in}{0.685514in}}{\pgfqpoint{5.265922in}{0.693327in}}%
\pgfpathcurveto{\pgfqpoint{5.258109in}{0.701141in}}{\pgfqpoint{5.247510in}{0.705531in}}{\pgfqpoint{5.236460in}{0.705531in}}%
\pgfpathcurveto{\pgfqpoint{5.225409in}{0.705531in}}{\pgfqpoint{5.214810in}{0.701141in}}{\pgfqpoint{5.206997in}{0.693327in}}%
\pgfpathcurveto{\pgfqpoint{5.199183in}{0.685514in}}{\pgfqpoint{5.194793in}{0.674915in}}{\pgfqpoint{5.194793in}{0.663864in}}%
\pgfpathcurveto{\pgfqpoint{5.194793in}{0.652814in}}{\pgfqpoint{5.199183in}{0.642215in}}{\pgfqpoint{5.206997in}{0.634402in}}%
\pgfpathcurveto{\pgfqpoint{5.214810in}{0.626588in}}{\pgfqpoint{5.225409in}{0.622198in}}{\pgfqpoint{5.236460in}{0.622198in}}%
\pgfpathlineto{\pgfqpoint{5.236460in}{0.622198in}}%
\pgfpathclose%
\pgfusepath{stroke}%
\end{pgfscope}%
\begin{pgfscope}%
\pgfpathrectangle{\pgfqpoint{0.847223in}{0.554012in}}{\pgfqpoint{6.200000in}{4.530000in}}%
\pgfusepath{clip}%
\pgfsetbuttcap%
\pgfsetroundjoin%
\pgfsetlinewidth{1.003750pt}%
\definecolor{currentstroke}{rgb}{1.000000,0.000000,0.000000}%
\pgfsetstrokecolor{currentstroke}%
\pgfsetdash{}{0pt}%
\pgfpathmoveto{\pgfqpoint{5.241793in}{0.621626in}}%
\pgfpathcurveto{\pgfqpoint{5.252843in}{0.621626in}}{\pgfqpoint{5.263442in}{0.626017in}}{\pgfqpoint{5.271256in}{0.633830in}}%
\pgfpathcurveto{\pgfqpoint{5.279069in}{0.641644in}}{\pgfqpoint{5.283459in}{0.652243in}}{\pgfqpoint{5.283459in}{0.663293in}}%
\pgfpathcurveto{\pgfqpoint{5.283459in}{0.674343in}}{\pgfqpoint{5.279069in}{0.684942in}}{\pgfqpoint{5.271256in}{0.692756in}}%
\pgfpathcurveto{\pgfqpoint{5.263442in}{0.700569in}}{\pgfqpoint{5.252843in}{0.704960in}}{\pgfqpoint{5.241793in}{0.704960in}}%
\pgfpathcurveto{\pgfqpoint{5.230743in}{0.704960in}}{\pgfqpoint{5.220144in}{0.700569in}}{\pgfqpoint{5.212330in}{0.692756in}}%
\pgfpathcurveto{\pgfqpoint{5.204516in}{0.684942in}}{\pgfqpoint{5.200126in}{0.674343in}}{\pgfqpoint{5.200126in}{0.663293in}}%
\pgfpathcurveto{\pgfqpoint{5.200126in}{0.652243in}}{\pgfqpoint{5.204516in}{0.641644in}}{\pgfqpoint{5.212330in}{0.633830in}}%
\pgfpathcurveto{\pgfqpoint{5.220144in}{0.626017in}}{\pgfqpoint{5.230743in}{0.621626in}}{\pgfqpoint{5.241793in}{0.621626in}}%
\pgfpathlineto{\pgfqpoint{5.241793in}{0.621626in}}%
\pgfpathclose%
\pgfusepath{stroke}%
\end{pgfscope}%
\begin{pgfscope}%
\pgfpathrectangle{\pgfqpoint{0.847223in}{0.554012in}}{\pgfqpoint{6.200000in}{4.530000in}}%
\pgfusepath{clip}%
\pgfsetbuttcap%
\pgfsetroundjoin%
\pgfsetlinewidth{1.003750pt}%
\definecolor{currentstroke}{rgb}{1.000000,0.000000,0.000000}%
\pgfsetstrokecolor{currentstroke}%
\pgfsetdash{}{0pt}%
\pgfpathmoveto{\pgfqpoint{5.247126in}{0.621056in}}%
\pgfpathcurveto{\pgfqpoint{5.258176in}{0.621056in}}{\pgfqpoint{5.268775in}{0.625446in}}{\pgfqpoint{5.276589in}{0.633260in}}%
\pgfpathcurveto{\pgfqpoint{5.284402in}{0.641073in}}{\pgfqpoint{5.288793in}{0.651673in}}{\pgfqpoint{5.288793in}{0.662723in}}%
\pgfpathcurveto{\pgfqpoint{5.288793in}{0.673773in}}{\pgfqpoint{5.284402in}{0.684372in}}{\pgfqpoint{5.276589in}{0.692185in}}%
\pgfpathcurveto{\pgfqpoint{5.268775in}{0.699999in}}{\pgfqpoint{5.258176in}{0.704389in}}{\pgfqpoint{5.247126in}{0.704389in}}%
\pgfpathcurveto{\pgfqpoint{5.236076in}{0.704389in}}{\pgfqpoint{5.225477in}{0.699999in}}{\pgfqpoint{5.217663in}{0.692185in}}%
\pgfpathcurveto{\pgfqpoint{5.209850in}{0.684372in}}{\pgfqpoint{5.205459in}{0.673773in}}{\pgfqpoint{5.205459in}{0.662723in}}%
\pgfpathcurveto{\pgfqpoint{5.205459in}{0.651673in}}{\pgfqpoint{5.209850in}{0.641073in}}{\pgfqpoint{5.217663in}{0.633260in}}%
\pgfpathcurveto{\pgfqpoint{5.225477in}{0.625446in}}{\pgfqpoint{5.236076in}{0.621056in}}{\pgfqpoint{5.247126in}{0.621056in}}%
\pgfpathlineto{\pgfqpoint{5.247126in}{0.621056in}}%
\pgfpathclose%
\pgfusepath{stroke}%
\end{pgfscope}%
\begin{pgfscope}%
\pgfpathrectangle{\pgfqpoint{0.847223in}{0.554012in}}{\pgfqpoint{6.200000in}{4.530000in}}%
\pgfusepath{clip}%
\pgfsetbuttcap%
\pgfsetroundjoin%
\pgfsetlinewidth{1.003750pt}%
\definecolor{currentstroke}{rgb}{1.000000,0.000000,0.000000}%
\pgfsetstrokecolor{currentstroke}%
\pgfsetdash{}{0pt}%
\pgfpathmoveto{\pgfqpoint{5.252459in}{0.620487in}}%
\pgfpathcurveto{\pgfqpoint{5.263509in}{0.620487in}}{\pgfqpoint{5.274108in}{0.624877in}}{\pgfqpoint{5.281922in}{0.632691in}}%
\pgfpathcurveto{\pgfqpoint{5.289736in}{0.640504in}}{\pgfqpoint{5.294126in}{0.651103in}}{\pgfqpoint{5.294126in}{0.662154in}}%
\pgfpathcurveto{\pgfqpoint{5.294126in}{0.673204in}}{\pgfqpoint{5.289736in}{0.683803in}}{\pgfqpoint{5.281922in}{0.691616in}}%
\pgfpathcurveto{\pgfqpoint{5.274108in}{0.699430in}}{\pgfqpoint{5.263509in}{0.703820in}}{\pgfqpoint{5.252459in}{0.703820in}}%
\pgfpathcurveto{\pgfqpoint{5.241409in}{0.703820in}}{\pgfqpoint{5.230810in}{0.699430in}}{\pgfqpoint{5.222996in}{0.691616in}}%
\pgfpathcurveto{\pgfqpoint{5.215183in}{0.683803in}}{\pgfqpoint{5.210793in}{0.673204in}}{\pgfqpoint{5.210793in}{0.662154in}}%
\pgfpathcurveto{\pgfqpoint{5.210793in}{0.651103in}}{\pgfqpoint{5.215183in}{0.640504in}}{\pgfqpoint{5.222996in}{0.632691in}}%
\pgfpathcurveto{\pgfqpoint{5.230810in}{0.624877in}}{\pgfqpoint{5.241409in}{0.620487in}}{\pgfqpoint{5.252459in}{0.620487in}}%
\pgfpathlineto{\pgfqpoint{5.252459in}{0.620487in}}%
\pgfpathclose%
\pgfusepath{stroke}%
\end{pgfscope}%
\begin{pgfscope}%
\pgfpathrectangle{\pgfqpoint{0.847223in}{0.554012in}}{\pgfqpoint{6.200000in}{4.530000in}}%
\pgfusepath{clip}%
\pgfsetbuttcap%
\pgfsetroundjoin%
\pgfsetlinewidth{1.003750pt}%
\definecolor{currentstroke}{rgb}{1.000000,0.000000,0.000000}%
\pgfsetstrokecolor{currentstroke}%
\pgfsetdash{}{0pt}%
\pgfpathmoveto{\pgfqpoint{5.257792in}{0.619919in}}%
\pgfpathcurveto{\pgfqpoint{5.268843in}{0.619919in}}{\pgfqpoint{5.279442in}{0.624309in}}{\pgfqpoint{5.287255in}{0.632123in}}%
\pgfpathcurveto{\pgfqpoint{5.295069in}{0.639937in}}{\pgfqpoint{5.299459in}{0.650536in}}{\pgfqpoint{5.299459in}{0.661586in}}%
\pgfpathcurveto{\pgfqpoint{5.299459in}{0.672636in}}{\pgfqpoint{5.295069in}{0.683235in}}{\pgfqpoint{5.287255in}{0.691049in}}%
\pgfpathcurveto{\pgfqpoint{5.279442in}{0.698862in}}{\pgfqpoint{5.268843in}{0.703252in}}{\pgfqpoint{5.257792in}{0.703252in}}%
\pgfpathcurveto{\pgfqpoint{5.246742in}{0.703252in}}{\pgfqpoint{5.236143in}{0.698862in}}{\pgfqpoint{5.228330in}{0.691049in}}%
\pgfpathcurveto{\pgfqpoint{5.220516in}{0.683235in}}{\pgfqpoint{5.216126in}{0.672636in}}{\pgfqpoint{5.216126in}{0.661586in}}%
\pgfpathcurveto{\pgfqpoint{5.216126in}{0.650536in}}{\pgfqpoint{5.220516in}{0.639937in}}{\pgfqpoint{5.228330in}{0.632123in}}%
\pgfpathcurveto{\pgfqpoint{5.236143in}{0.624309in}}{\pgfqpoint{5.246742in}{0.619919in}}{\pgfqpoint{5.257792in}{0.619919in}}%
\pgfpathlineto{\pgfqpoint{5.257792in}{0.619919in}}%
\pgfpathclose%
\pgfusepath{stroke}%
\end{pgfscope}%
\begin{pgfscope}%
\pgfpathrectangle{\pgfqpoint{0.847223in}{0.554012in}}{\pgfqpoint{6.200000in}{4.530000in}}%
\pgfusepath{clip}%
\pgfsetbuttcap%
\pgfsetroundjoin%
\pgfsetlinewidth{1.003750pt}%
\definecolor{currentstroke}{rgb}{1.000000,0.000000,0.000000}%
\pgfsetstrokecolor{currentstroke}%
\pgfsetdash{}{0pt}%
\pgfpathmoveto{\pgfqpoint{5.263126in}{0.619352in}}%
\pgfpathcurveto{\pgfqpoint{5.274176in}{0.619352in}}{\pgfqpoint{5.284775in}{0.623743in}}{\pgfqpoint{5.292588in}{0.631556in}}%
\pgfpathcurveto{\pgfqpoint{5.300402in}{0.639370in}}{\pgfqpoint{5.304792in}{0.649969in}}{\pgfqpoint{5.304792in}{0.661019in}}%
\pgfpathcurveto{\pgfqpoint{5.304792in}{0.672069in}}{\pgfqpoint{5.300402in}{0.682668in}}{\pgfqpoint{5.292588in}{0.690482in}}%
\pgfpathcurveto{\pgfqpoint{5.284775in}{0.698296in}}{\pgfqpoint{5.274176in}{0.702686in}}{\pgfqpoint{5.263126in}{0.702686in}}%
\pgfpathcurveto{\pgfqpoint{5.252075in}{0.702686in}}{\pgfqpoint{5.241476in}{0.698296in}}{\pgfqpoint{5.233663in}{0.690482in}}%
\pgfpathcurveto{\pgfqpoint{5.225849in}{0.682668in}}{\pgfqpoint{5.221459in}{0.672069in}}{\pgfqpoint{5.221459in}{0.661019in}}%
\pgfpathcurveto{\pgfqpoint{5.221459in}{0.649969in}}{\pgfqpoint{5.225849in}{0.639370in}}{\pgfqpoint{5.233663in}{0.631556in}}%
\pgfpathcurveto{\pgfqpoint{5.241476in}{0.623743in}}{\pgfqpoint{5.252075in}{0.619352in}}{\pgfqpoint{5.263126in}{0.619352in}}%
\pgfpathlineto{\pgfqpoint{5.263126in}{0.619352in}}%
\pgfpathclose%
\pgfusepath{stroke}%
\end{pgfscope}%
\begin{pgfscope}%
\pgfpathrectangle{\pgfqpoint{0.847223in}{0.554012in}}{\pgfqpoint{6.200000in}{4.530000in}}%
\pgfusepath{clip}%
\pgfsetbuttcap%
\pgfsetroundjoin%
\pgfsetlinewidth{1.003750pt}%
\definecolor{currentstroke}{rgb}{1.000000,0.000000,0.000000}%
\pgfsetstrokecolor{currentstroke}%
\pgfsetdash{}{0pt}%
\pgfpathmoveto{\pgfqpoint{5.268459in}{0.618787in}}%
\pgfpathcurveto{\pgfqpoint{5.279509in}{0.618787in}}{\pgfqpoint{5.290108in}{0.623177in}}{\pgfqpoint{5.297922in}{0.630991in}}%
\pgfpathcurveto{\pgfqpoint{5.305735in}{0.638805in}}{\pgfqpoint{5.310126in}{0.649404in}}{\pgfqpoint{5.310126in}{0.660454in}}%
\pgfpathcurveto{\pgfqpoint{5.310126in}{0.671504in}}{\pgfqpoint{5.305735in}{0.682103in}}{\pgfqpoint{5.297922in}{0.689917in}}%
\pgfpathcurveto{\pgfqpoint{5.290108in}{0.697730in}}{\pgfqpoint{5.279509in}{0.702120in}}{\pgfqpoint{5.268459in}{0.702120in}}%
\pgfpathcurveto{\pgfqpoint{5.257409in}{0.702120in}}{\pgfqpoint{5.246810in}{0.697730in}}{\pgfqpoint{5.238996in}{0.689917in}}%
\pgfpathcurveto{\pgfqpoint{5.231182in}{0.682103in}}{\pgfqpoint{5.226792in}{0.671504in}}{\pgfqpoint{5.226792in}{0.660454in}}%
\pgfpathcurveto{\pgfqpoint{5.226792in}{0.649404in}}{\pgfqpoint{5.231182in}{0.638805in}}{\pgfqpoint{5.238996in}{0.630991in}}%
\pgfpathcurveto{\pgfqpoint{5.246810in}{0.623177in}}{\pgfqpoint{5.257409in}{0.618787in}}{\pgfqpoint{5.268459in}{0.618787in}}%
\pgfpathlineto{\pgfqpoint{5.268459in}{0.618787in}}%
\pgfpathclose%
\pgfusepath{stroke}%
\end{pgfscope}%
\begin{pgfscope}%
\pgfpathrectangle{\pgfqpoint{0.847223in}{0.554012in}}{\pgfqpoint{6.200000in}{4.530000in}}%
\pgfusepath{clip}%
\pgfsetbuttcap%
\pgfsetroundjoin%
\pgfsetlinewidth{1.003750pt}%
\definecolor{currentstroke}{rgb}{1.000000,0.000000,0.000000}%
\pgfsetstrokecolor{currentstroke}%
\pgfsetdash{}{0pt}%
\pgfpathmoveto{\pgfqpoint{5.273792in}{0.618223in}}%
\pgfpathcurveto{\pgfqpoint{5.284842in}{0.618223in}}{\pgfqpoint{5.295441in}{0.622613in}}{\pgfqpoint{5.303255in}{0.630427in}}%
\pgfpathcurveto{\pgfqpoint{5.311068in}{0.638240in}}{\pgfqpoint{5.315459in}{0.648839in}}{\pgfqpoint{5.315459in}{0.659890in}}%
\pgfpathcurveto{\pgfqpoint{5.315459in}{0.670940in}}{\pgfqpoint{5.311068in}{0.681539in}}{\pgfqpoint{5.303255in}{0.689352in}}%
\pgfpathcurveto{\pgfqpoint{5.295441in}{0.697166in}}{\pgfqpoint{5.284842in}{0.701556in}}{\pgfqpoint{5.273792in}{0.701556in}}%
\pgfpathcurveto{\pgfqpoint{5.262742in}{0.701556in}}{\pgfqpoint{5.252143in}{0.697166in}}{\pgfqpoint{5.244329in}{0.689352in}}%
\pgfpathcurveto{\pgfqpoint{5.236516in}{0.681539in}}{\pgfqpoint{5.232125in}{0.670940in}}{\pgfqpoint{5.232125in}{0.659890in}}%
\pgfpathcurveto{\pgfqpoint{5.232125in}{0.648839in}}{\pgfqpoint{5.236516in}{0.638240in}}{\pgfqpoint{5.244329in}{0.630427in}}%
\pgfpathcurveto{\pgfqpoint{5.252143in}{0.622613in}}{\pgfqpoint{5.262742in}{0.618223in}}{\pgfqpoint{5.273792in}{0.618223in}}%
\pgfpathlineto{\pgfqpoint{5.273792in}{0.618223in}}%
\pgfpathclose%
\pgfusepath{stroke}%
\end{pgfscope}%
\begin{pgfscope}%
\pgfpathrectangle{\pgfqpoint{0.847223in}{0.554012in}}{\pgfqpoint{6.200000in}{4.530000in}}%
\pgfusepath{clip}%
\pgfsetbuttcap%
\pgfsetroundjoin%
\pgfsetlinewidth{1.003750pt}%
\definecolor{currentstroke}{rgb}{1.000000,0.000000,0.000000}%
\pgfsetstrokecolor{currentstroke}%
\pgfsetdash{}{0pt}%
\pgfpathmoveto{\pgfqpoint{5.279125in}{0.617660in}}%
\pgfpathcurveto{\pgfqpoint{5.290175in}{0.617660in}}{\pgfqpoint{5.300774in}{0.622050in}}{\pgfqpoint{5.308588in}{0.629864in}}%
\pgfpathcurveto{\pgfqpoint{5.316402in}{0.637677in}}{\pgfqpoint{5.320792in}{0.648276in}}{\pgfqpoint{5.320792in}{0.659327in}}%
\pgfpathcurveto{\pgfqpoint{5.320792in}{0.670377in}}{\pgfqpoint{5.316402in}{0.680976in}}{\pgfqpoint{5.308588in}{0.688789in}}%
\pgfpathcurveto{\pgfqpoint{5.300774in}{0.696603in}}{\pgfqpoint{5.290175in}{0.700993in}}{\pgfqpoint{5.279125in}{0.700993in}}%
\pgfpathcurveto{\pgfqpoint{5.268075in}{0.700993in}}{\pgfqpoint{5.257476in}{0.696603in}}{\pgfqpoint{5.249662in}{0.688789in}}%
\pgfpathcurveto{\pgfqpoint{5.241849in}{0.680976in}}{\pgfqpoint{5.237459in}{0.670377in}}{\pgfqpoint{5.237459in}{0.659327in}}%
\pgfpathcurveto{\pgfqpoint{5.237459in}{0.648276in}}{\pgfqpoint{5.241849in}{0.637677in}}{\pgfqpoint{5.249662in}{0.629864in}}%
\pgfpathcurveto{\pgfqpoint{5.257476in}{0.622050in}}{\pgfqpoint{5.268075in}{0.617660in}}{\pgfqpoint{5.279125in}{0.617660in}}%
\pgfpathlineto{\pgfqpoint{5.279125in}{0.617660in}}%
\pgfpathclose%
\pgfusepath{stroke}%
\end{pgfscope}%
\begin{pgfscope}%
\pgfpathrectangle{\pgfqpoint{0.847223in}{0.554012in}}{\pgfqpoint{6.200000in}{4.530000in}}%
\pgfusepath{clip}%
\pgfsetbuttcap%
\pgfsetroundjoin%
\pgfsetlinewidth{1.003750pt}%
\definecolor{currentstroke}{rgb}{1.000000,0.000000,0.000000}%
\pgfsetstrokecolor{currentstroke}%
\pgfsetdash{}{0pt}%
\pgfpathmoveto{\pgfqpoint{5.284458in}{0.617098in}}%
\pgfpathcurveto{\pgfqpoint{5.295509in}{0.617098in}}{\pgfqpoint{5.306108in}{0.621488in}}{\pgfqpoint{5.313921in}{0.629302in}}%
\pgfpathcurveto{\pgfqpoint{5.321735in}{0.637116in}}{\pgfqpoint{5.326125in}{0.647715in}}{\pgfqpoint{5.326125in}{0.658765in}}%
\pgfpathcurveto{\pgfqpoint{5.326125in}{0.669815in}}{\pgfqpoint{5.321735in}{0.680414in}}{\pgfqpoint{5.313921in}{0.688228in}}%
\pgfpathcurveto{\pgfqpoint{5.306108in}{0.696041in}}{\pgfqpoint{5.295509in}{0.700432in}}{\pgfqpoint{5.284458in}{0.700432in}}%
\pgfpathcurveto{\pgfqpoint{5.273408in}{0.700432in}}{\pgfqpoint{5.262809in}{0.696041in}}{\pgfqpoint{5.254996in}{0.688228in}}%
\pgfpathcurveto{\pgfqpoint{5.247182in}{0.680414in}}{\pgfqpoint{5.242792in}{0.669815in}}{\pgfqpoint{5.242792in}{0.658765in}}%
\pgfpathcurveto{\pgfqpoint{5.242792in}{0.647715in}}{\pgfqpoint{5.247182in}{0.637116in}}{\pgfqpoint{5.254996in}{0.629302in}}%
\pgfpathcurveto{\pgfqpoint{5.262809in}{0.621488in}}{\pgfqpoint{5.273408in}{0.617098in}}{\pgfqpoint{5.284458in}{0.617098in}}%
\pgfpathlineto{\pgfqpoint{5.284458in}{0.617098in}}%
\pgfpathclose%
\pgfusepath{stroke}%
\end{pgfscope}%
\begin{pgfscope}%
\pgfpathrectangle{\pgfqpoint{0.847223in}{0.554012in}}{\pgfqpoint{6.200000in}{4.530000in}}%
\pgfusepath{clip}%
\pgfsetbuttcap%
\pgfsetroundjoin%
\pgfsetlinewidth{1.003750pt}%
\definecolor{currentstroke}{rgb}{1.000000,0.000000,0.000000}%
\pgfsetstrokecolor{currentstroke}%
\pgfsetdash{}{0pt}%
\pgfpathmoveto{\pgfqpoint{5.289792in}{0.616538in}}%
\pgfpathcurveto{\pgfqpoint{5.300842in}{0.616538in}}{\pgfqpoint{5.311441in}{0.620928in}}{\pgfqpoint{5.319254in}{0.628742in}}%
\pgfpathcurveto{\pgfqpoint{5.327068in}{0.636555in}}{\pgfqpoint{5.331458in}{0.647154in}}{\pgfqpoint{5.331458in}{0.658204in}}%
\pgfpathcurveto{\pgfqpoint{5.331458in}{0.669254in}}{\pgfqpoint{5.327068in}{0.679853in}}{\pgfqpoint{5.319254in}{0.687667in}}%
\pgfpathcurveto{\pgfqpoint{5.311441in}{0.695481in}}{\pgfqpoint{5.300842in}{0.699871in}}{\pgfqpoint{5.289792in}{0.699871in}}%
\pgfpathcurveto{\pgfqpoint{5.278742in}{0.699871in}}{\pgfqpoint{5.268143in}{0.695481in}}{\pgfqpoint{5.260329in}{0.687667in}}%
\pgfpathcurveto{\pgfqpoint{5.252515in}{0.679853in}}{\pgfqpoint{5.248125in}{0.669254in}}{\pgfqpoint{5.248125in}{0.658204in}}%
\pgfpathcurveto{\pgfqpoint{5.248125in}{0.647154in}}{\pgfqpoint{5.252515in}{0.636555in}}{\pgfqpoint{5.260329in}{0.628742in}}%
\pgfpathcurveto{\pgfqpoint{5.268143in}{0.620928in}}{\pgfqpoint{5.278742in}{0.616538in}}{\pgfqpoint{5.289792in}{0.616538in}}%
\pgfpathlineto{\pgfqpoint{5.289792in}{0.616538in}}%
\pgfpathclose%
\pgfusepath{stroke}%
\end{pgfscope}%
\begin{pgfscope}%
\pgfpathrectangle{\pgfqpoint{0.847223in}{0.554012in}}{\pgfqpoint{6.200000in}{4.530000in}}%
\pgfusepath{clip}%
\pgfsetbuttcap%
\pgfsetroundjoin%
\pgfsetlinewidth{1.003750pt}%
\definecolor{currentstroke}{rgb}{1.000000,0.000000,0.000000}%
\pgfsetstrokecolor{currentstroke}%
\pgfsetdash{}{0pt}%
\pgfpathmoveto{\pgfqpoint{5.295125in}{0.615978in}}%
\pgfpathcurveto{\pgfqpoint{5.306175in}{0.615978in}}{\pgfqpoint{5.316774in}{0.620369in}}{\pgfqpoint{5.324588in}{0.628182in}}%
\pgfpathcurveto{\pgfqpoint{5.332401in}{0.635996in}}{\pgfqpoint{5.336792in}{0.646595in}}{\pgfqpoint{5.336792in}{0.657645in}}%
\pgfpathcurveto{\pgfqpoint{5.336792in}{0.668695in}}{\pgfqpoint{5.332401in}{0.679294in}}{\pgfqpoint{5.324588in}{0.687108in}}%
\pgfpathcurveto{\pgfqpoint{5.316774in}{0.694921in}}{\pgfqpoint{5.306175in}{0.699312in}}{\pgfqpoint{5.295125in}{0.699312in}}%
\pgfpathcurveto{\pgfqpoint{5.284075in}{0.699312in}}{\pgfqpoint{5.273476in}{0.694921in}}{\pgfqpoint{5.265662in}{0.687108in}}%
\pgfpathcurveto{\pgfqpoint{5.257849in}{0.679294in}}{\pgfqpoint{5.253458in}{0.668695in}}{\pgfqpoint{5.253458in}{0.657645in}}%
\pgfpathcurveto{\pgfqpoint{5.253458in}{0.646595in}}{\pgfqpoint{5.257849in}{0.635996in}}{\pgfqpoint{5.265662in}{0.628182in}}%
\pgfpathcurveto{\pgfqpoint{5.273476in}{0.620369in}}{\pgfqpoint{5.284075in}{0.615978in}}{\pgfqpoint{5.295125in}{0.615978in}}%
\pgfpathlineto{\pgfqpoint{5.295125in}{0.615978in}}%
\pgfpathclose%
\pgfusepath{stroke}%
\end{pgfscope}%
\begin{pgfscope}%
\pgfpathrectangle{\pgfqpoint{0.847223in}{0.554012in}}{\pgfqpoint{6.200000in}{4.530000in}}%
\pgfusepath{clip}%
\pgfsetbuttcap%
\pgfsetroundjoin%
\pgfsetlinewidth{1.003750pt}%
\definecolor{currentstroke}{rgb}{1.000000,0.000000,0.000000}%
\pgfsetstrokecolor{currentstroke}%
\pgfsetdash{}{0pt}%
\pgfpathmoveto{\pgfqpoint{5.300458in}{0.615420in}}%
\pgfpathcurveto{\pgfqpoint{5.311508in}{0.615420in}}{\pgfqpoint{5.322107in}{0.619810in}}{\pgfqpoint{5.329921in}{0.627624in}}%
\pgfpathcurveto{\pgfqpoint{5.337735in}{0.635438in}}{\pgfqpoint{5.342125in}{0.646037in}}{\pgfqpoint{5.342125in}{0.657087in}}%
\pgfpathcurveto{\pgfqpoint{5.342125in}{0.668137in}}{\pgfqpoint{5.337735in}{0.678736in}}{\pgfqpoint{5.329921in}{0.686550in}}%
\pgfpathcurveto{\pgfqpoint{5.322107in}{0.694363in}}{\pgfqpoint{5.311508in}{0.698754in}}{\pgfqpoint{5.300458in}{0.698754in}}%
\pgfpathcurveto{\pgfqpoint{5.289408in}{0.698754in}}{\pgfqpoint{5.278809in}{0.694363in}}{\pgfqpoint{5.270995in}{0.686550in}}%
\pgfpathcurveto{\pgfqpoint{5.263182in}{0.678736in}}{\pgfqpoint{5.258791in}{0.668137in}}{\pgfqpoint{5.258791in}{0.657087in}}%
\pgfpathcurveto{\pgfqpoint{5.258791in}{0.646037in}}{\pgfqpoint{5.263182in}{0.635438in}}{\pgfqpoint{5.270995in}{0.627624in}}%
\pgfpathcurveto{\pgfqpoint{5.278809in}{0.619810in}}{\pgfqpoint{5.289408in}{0.615420in}}{\pgfqpoint{5.300458in}{0.615420in}}%
\pgfpathlineto{\pgfqpoint{5.300458in}{0.615420in}}%
\pgfpathclose%
\pgfusepath{stroke}%
\end{pgfscope}%
\begin{pgfscope}%
\pgfpathrectangle{\pgfqpoint{0.847223in}{0.554012in}}{\pgfqpoint{6.200000in}{4.530000in}}%
\pgfusepath{clip}%
\pgfsetbuttcap%
\pgfsetroundjoin%
\pgfsetlinewidth{1.003750pt}%
\definecolor{currentstroke}{rgb}{1.000000,0.000000,0.000000}%
\pgfsetstrokecolor{currentstroke}%
\pgfsetdash{}{0pt}%
\pgfpathmoveto{\pgfqpoint{5.305791in}{0.614863in}}%
\pgfpathcurveto{\pgfqpoint{5.316841in}{0.614863in}}{\pgfqpoint{5.327441in}{0.619254in}}{\pgfqpoint{5.335254in}{0.627067in}}%
\pgfpathcurveto{\pgfqpoint{5.343068in}{0.634881in}}{\pgfqpoint{5.347458in}{0.645480in}}{\pgfqpoint{5.347458in}{0.656530in}}%
\pgfpathcurveto{\pgfqpoint{5.347458in}{0.667580in}}{\pgfqpoint{5.343068in}{0.678179in}}{\pgfqpoint{5.335254in}{0.685993in}}%
\pgfpathcurveto{\pgfqpoint{5.327441in}{0.693806in}}{\pgfqpoint{5.316841in}{0.698197in}}{\pgfqpoint{5.305791in}{0.698197in}}%
\pgfpathcurveto{\pgfqpoint{5.294741in}{0.698197in}}{\pgfqpoint{5.284142in}{0.693806in}}{\pgfqpoint{5.276329in}{0.685993in}}%
\pgfpathcurveto{\pgfqpoint{5.268515in}{0.678179in}}{\pgfqpoint{5.264125in}{0.667580in}}{\pgfqpoint{5.264125in}{0.656530in}}%
\pgfpathcurveto{\pgfqpoint{5.264125in}{0.645480in}}{\pgfqpoint{5.268515in}{0.634881in}}{\pgfqpoint{5.276329in}{0.627067in}}%
\pgfpathcurveto{\pgfqpoint{5.284142in}{0.619254in}}{\pgfqpoint{5.294741in}{0.614863in}}{\pgfqpoint{5.305791in}{0.614863in}}%
\pgfpathlineto{\pgfqpoint{5.305791in}{0.614863in}}%
\pgfpathclose%
\pgfusepath{stroke}%
\end{pgfscope}%
\begin{pgfscope}%
\pgfpathrectangle{\pgfqpoint{0.847223in}{0.554012in}}{\pgfqpoint{6.200000in}{4.530000in}}%
\pgfusepath{clip}%
\pgfsetbuttcap%
\pgfsetroundjoin%
\pgfsetlinewidth{1.003750pt}%
\definecolor{currentstroke}{rgb}{1.000000,0.000000,0.000000}%
\pgfsetstrokecolor{currentstroke}%
\pgfsetdash{}{0pt}%
\pgfpathmoveto{\pgfqpoint{5.311125in}{0.614307in}}%
\pgfpathcurveto{\pgfqpoint{5.322175in}{0.614307in}}{\pgfqpoint{5.332774in}{0.618698in}}{\pgfqpoint{5.340587in}{0.626511in}}%
\pgfpathcurveto{\pgfqpoint{5.348401in}{0.634325in}}{\pgfqpoint{5.352791in}{0.644924in}}{\pgfqpoint{5.352791in}{0.655974in}}%
\pgfpathcurveto{\pgfqpoint{5.352791in}{0.667024in}}{\pgfqpoint{5.348401in}{0.677623in}}{\pgfqpoint{5.340587in}{0.685437in}}%
\pgfpathcurveto{\pgfqpoint{5.332774in}{0.693251in}}{\pgfqpoint{5.322175in}{0.697641in}}{\pgfqpoint{5.311125in}{0.697641in}}%
\pgfpathcurveto{\pgfqpoint{5.300074in}{0.697641in}}{\pgfqpoint{5.289475in}{0.693251in}}{\pgfqpoint{5.281662in}{0.685437in}}%
\pgfpathcurveto{\pgfqpoint{5.273848in}{0.677623in}}{\pgfqpoint{5.269458in}{0.667024in}}{\pgfqpoint{5.269458in}{0.655974in}}%
\pgfpathcurveto{\pgfqpoint{5.269458in}{0.644924in}}{\pgfqpoint{5.273848in}{0.634325in}}{\pgfqpoint{5.281662in}{0.626511in}}%
\pgfpathcurveto{\pgfqpoint{5.289475in}{0.618698in}}{\pgfqpoint{5.300074in}{0.614307in}}{\pgfqpoint{5.311125in}{0.614307in}}%
\pgfpathlineto{\pgfqpoint{5.311125in}{0.614307in}}%
\pgfpathclose%
\pgfusepath{stroke}%
\end{pgfscope}%
\begin{pgfscope}%
\pgfpathrectangle{\pgfqpoint{0.847223in}{0.554012in}}{\pgfqpoint{6.200000in}{4.530000in}}%
\pgfusepath{clip}%
\pgfsetbuttcap%
\pgfsetroundjoin%
\pgfsetlinewidth{1.003750pt}%
\definecolor{currentstroke}{rgb}{1.000000,0.000000,0.000000}%
\pgfsetstrokecolor{currentstroke}%
\pgfsetdash{}{0pt}%
\pgfpathmoveto{\pgfqpoint{5.316458in}{0.613753in}}%
\pgfpathcurveto{\pgfqpoint{5.327508in}{0.613753in}}{\pgfqpoint{5.338107in}{0.618143in}}{\pgfqpoint{5.345921in}{0.625957in}}%
\pgfpathcurveto{\pgfqpoint{5.353734in}{0.633770in}}{\pgfqpoint{5.358124in}{0.644369in}}{\pgfqpoint{5.358124in}{0.655420in}}%
\pgfpathcurveto{\pgfqpoint{5.358124in}{0.666470in}}{\pgfqpoint{5.353734in}{0.677069in}}{\pgfqpoint{5.345921in}{0.684882in}}%
\pgfpathcurveto{\pgfqpoint{5.338107in}{0.692696in}}{\pgfqpoint{5.327508in}{0.697086in}}{\pgfqpoint{5.316458in}{0.697086in}}%
\pgfpathcurveto{\pgfqpoint{5.305408in}{0.697086in}}{\pgfqpoint{5.294809in}{0.692696in}}{\pgfqpoint{5.286995in}{0.684882in}}%
\pgfpathcurveto{\pgfqpoint{5.279181in}{0.677069in}}{\pgfqpoint{5.274791in}{0.666470in}}{\pgfqpoint{5.274791in}{0.655420in}}%
\pgfpathcurveto{\pgfqpoint{5.274791in}{0.644369in}}{\pgfqpoint{5.279181in}{0.633770in}}{\pgfqpoint{5.286995in}{0.625957in}}%
\pgfpathcurveto{\pgfqpoint{5.294809in}{0.618143in}}{\pgfqpoint{5.305408in}{0.613753in}}{\pgfqpoint{5.316458in}{0.613753in}}%
\pgfpathlineto{\pgfqpoint{5.316458in}{0.613753in}}%
\pgfpathclose%
\pgfusepath{stroke}%
\end{pgfscope}%
\begin{pgfscope}%
\pgfpathrectangle{\pgfqpoint{0.847223in}{0.554012in}}{\pgfqpoint{6.200000in}{4.530000in}}%
\pgfusepath{clip}%
\pgfsetbuttcap%
\pgfsetroundjoin%
\pgfsetlinewidth{1.003750pt}%
\definecolor{currentstroke}{rgb}{1.000000,0.000000,0.000000}%
\pgfsetstrokecolor{currentstroke}%
\pgfsetdash{}{0pt}%
\pgfpathmoveto{\pgfqpoint{5.321791in}{0.613200in}}%
\pgfpathcurveto{\pgfqpoint{5.332841in}{0.613200in}}{\pgfqpoint{5.343440in}{0.617590in}}{\pgfqpoint{5.351254in}{0.625403in}}%
\pgfpathcurveto{\pgfqpoint{5.359067in}{0.633217in}}{\pgfqpoint{5.363458in}{0.643816in}}{\pgfqpoint{5.363458in}{0.654866in}}%
\pgfpathcurveto{\pgfqpoint{5.363458in}{0.665916in}}{\pgfqpoint{5.359067in}{0.676515in}}{\pgfqpoint{5.351254in}{0.684329in}}%
\pgfpathcurveto{\pgfqpoint{5.343440in}{0.692143in}}{\pgfqpoint{5.332841in}{0.696533in}}{\pgfqpoint{5.321791in}{0.696533in}}%
\pgfpathcurveto{\pgfqpoint{5.310741in}{0.696533in}}{\pgfqpoint{5.300142in}{0.692143in}}{\pgfqpoint{5.292328in}{0.684329in}}%
\pgfpathcurveto{\pgfqpoint{5.284515in}{0.676515in}}{\pgfqpoint{5.280124in}{0.665916in}}{\pgfqpoint{5.280124in}{0.654866in}}%
\pgfpathcurveto{\pgfqpoint{5.280124in}{0.643816in}}{\pgfqpoint{5.284515in}{0.633217in}}{\pgfqpoint{5.292328in}{0.625403in}}%
\pgfpathcurveto{\pgfqpoint{5.300142in}{0.617590in}}{\pgfqpoint{5.310741in}{0.613200in}}{\pgfqpoint{5.321791in}{0.613200in}}%
\pgfpathlineto{\pgfqpoint{5.321791in}{0.613200in}}%
\pgfpathclose%
\pgfusepath{stroke}%
\end{pgfscope}%
\begin{pgfscope}%
\pgfpathrectangle{\pgfqpoint{0.847223in}{0.554012in}}{\pgfqpoint{6.200000in}{4.530000in}}%
\pgfusepath{clip}%
\pgfsetbuttcap%
\pgfsetroundjoin%
\pgfsetlinewidth{1.003750pt}%
\definecolor{currentstroke}{rgb}{1.000000,0.000000,0.000000}%
\pgfsetstrokecolor{currentstroke}%
\pgfsetdash{}{0pt}%
\pgfpathmoveto{\pgfqpoint{5.327124in}{0.612647in}}%
\pgfpathcurveto{\pgfqpoint{5.338174in}{0.612647in}}{\pgfqpoint{5.348773in}{0.617038in}}{\pgfqpoint{5.356587in}{0.624851in}}%
\pgfpathcurveto{\pgfqpoint{5.364401in}{0.632665in}}{\pgfqpoint{5.368791in}{0.643264in}}{\pgfqpoint{5.368791in}{0.654314in}}%
\pgfpathcurveto{\pgfqpoint{5.368791in}{0.665364in}}{\pgfqpoint{5.364401in}{0.675963in}}{\pgfqpoint{5.356587in}{0.683777in}}%
\pgfpathcurveto{\pgfqpoint{5.348773in}{0.691590in}}{\pgfqpoint{5.338174in}{0.695981in}}{\pgfqpoint{5.327124in}{0.695981in}}%
\pgfpathcurveto{\pgfqpoint{5.316074in}{0.695981in}}{\pgfqpoint{5.305475in}{0.691590in}}{\pgfqpoint{5.297661in}{0.683777in}}%
\pgfpathcurveto{\pgfqpoint{5.289848in}{0.675963in}}{\pgfqpoint{5.285458in}{0.665364in}}{\pgfqpoint{5.285458in}{0.654314in}}%
\pgfpathcurveto{\pgfqpoint{5.285458in}{0.643264in}}{\pgfqpoint{5.289848in}{0.632665in}}{\pgfqpoint{5.297661in}{0.624851in}}%
\pgfpathcurveto{\pgfqpoint{5.305475in}{0.617038in}}{\pgfqpoint{5.316074in}{0.612647in}}{\pgfqpoint{5.327124in}{0.612647in}}%
\pgfpathlineto{\pgfqpoint{5.327124in}{0.612647in}}%
\pgfpathclose%
\pgfusepath{stroke}%
\end{pgfscope}%
\begin{pgfscope}%
\pgfpathrectangle{\pgfqpoint{0.847223in}{0.554012in}}{\pgfqpoint{6.200000in}{4.530000in}}%
\pgfusepath{clip}%
\pgfsetbuttcap%
\pgfsetroundjoin%
\pgfsetlinewidth{1.003750pt}%
\definecolor{currentstroke}{rgb}{1.000000,0.000000,0.000000}%
\pgfsetstrokecolor{currentstroke}%
\pgfsetdash{}{0pt}%
\pgfpathmoveto{\pgfqpoint{5.332457in}{0.612096in}}%
\pgfpathcurveto{\pgfqpoint{5.343508in}{0.612096in}}{\pgfqpoint{5.354107in}{0.616487in}}{\pgfqpoint{5.361920in}{0.624300in}}%
\pgfpathcurveto{\pgfqpoint{5.369734in}{0.632114in}}{\pgfqpoint{5.374124in}{0.642713in}}{\pgfqpoint{5.374124in}{0.653763in}}%
\pgfpathcurveto{\pgfqpoint{5.374124in}{0.664813in}}{\pgfqpoint{5.369734in}{0.675412in}}{\pgfqpoint{5.361920in}{0.683226in}}%
\pgfpathcurveto{\pgfqpoint{5.354107in}{0.691039in}}{\pgfqpoint{5.343508in}{0.695430in}}{\pgfqpoint{5.332457in}{0.695430in}}%
\pgfpathcurveto{\pgfqpoint{5.321407in}{0.695430in}}{\pgfqpoint{5.310808in}{0.691039in}}{\pgfqpoint{5.302995in}{0.683226in}}%
\pgfpathcurveto{\pgfqpoint{5.295181in}{0.675412in}}{\pgfqpoint{5.290791in}{0.664813in}}{\pgfqpoint{5.290791in}{0.653763in}}%
\pgfpathcurveto{\pgfqpoint{5.290791in}{0.642713in}}{\pgfqpoint{5.295181in}{0.632114in}}{\pgfqpoint{5.302995in}{0.624300in}}%
\pgfpathcurveto{\pgfqpoint{5.310808in}{0.616487in}}{\pgfqpoint{5.321407in}{0.612096in}}{\pgfqpoint{5.332457in}{0.612096in}}%
\pgfpathlineto{\pgfqpoint{5.332457in}{0.612096in}}%
\pgfpathclose%
\pgfusepath{stroke}%
\end{pgfscope}%
\begin{pgfscope}%
\pgfpathrectangle{\pgfqpoint{0.847223in}{0.554012in}}{\pgfqpoint{6.200000in}{4.530000in}}%
\pgfusepath{clip}%
\pgfsetbuttcap%
\pgfsetroundjoin%
\pgfsetlinewidth{1.003750pt}%
\definecolor{currentstroke}{rgb}{1.000000,0.000000,0.000000}%
\pgfsetstrokecolor{currentstroke}%
\pgfsetdash{}{0pt}%
\pgfpathmoveto{\pgfqpoint{5.337791in}{0.611546in}}%
\pgfpathcurveto{\pgfqpoint{5.348841in}{0.611546in}}{\pgfqpoint{5.359440in}{0.615937in}}{\pgfqpoint{5.367253in}{0.623750in}}%
\pgfpathcurveto{\pgfqpoint{5.375067in}{0.631564in}}{\pgfqpoint{5.379457in}{0.642163in}}{\pgfqpoint{5.379457in}{0.653213in}}%
\pgfpathcurveto{\pgfqpoint{5.379457in}{0.664263in}}{\pgfqpoint{5.375067in}{0.674862in}}{\pgfqpoint{5.367253in}{0.682676in}}%
\pgfpathcurveto{\pgfqpoint{5.359440in}{0.690490in}}{\pgfqpoint{5.348841in}{0.694880in}}{\pgfqpoint{5.337791in}{0.694880in}}%
\pgfpathcurveto{\pgfqpoint{5.326741in}{0.694880in}}{\pgfqpoint{5.316141in}{0.690490in}}{\pgfqpoint{5.308328in}{0.682676in}}%
\pgfpathcurveto{\pgfqpoint{5.300514in}{0.674862in}}{\pgfqpoint{5.296124in}{0.664263in}}{\pgfqpoint{5.296124in}{0.653213in}}%
\pgfpathcurveto{\pgfqpoint{5.296124in}{0.642163in}}{\pgfqpoint{5.300514in}{0.631564in}}{\pgfqpoint{5.308328in}{0.623750in}}%
\pgfpathcurveto{\pgfqpoint{5.316141in}{0.615937in}}{\pgfqpoint{5.326741in}{0.611546in}}{\pgfqpoint{5.337791in}{0.611546in}}%
\pgfpathlineto{\pgfqpoint{5.337791in}{0.611546in}}%
\pgfpathclose%
\pgfusepath{stroke}%
\end{pgfscope}%
\begin{pgfscope}%
\pgfpathrectangle{\pgfqpoint{0.847223in}{0.554012in}}{\pgfqpoint{6.200000in}{4.530000in}}%
\pgfusepath{clip}%
\pgfsetbuttcap%
\pgfsetroundjoin%
\pgfsetlinewidth{1.003750pt}%
\definecolor{currentstroke}{rgb}{1.000000,0.000000,0.000000}%
\pgfsetstrokecolor{currentstroke}%
\pgfsetdash{}{0pt}%
\pgfpathmoveto{\pgfqpoint{5.343124in}{0.610998in}}%
\pgfpathcurveto{\pgfqpoint{5.354174in}{0.610998in}}{\pgfqpoint{5.364773in}{0.615388in}}{\pgfqpoint{5.372587in}{0.623202in}}%
\pgfpathcurveto{\pgfqpoint{5.380400in}{0.631015in}}{\pgfqpoint{5.384791in}{0.641614in}}{\pgfqpoint{5.384791in}{0.652664in}}%
\pgfpathcurveto{\pgfqpoint{5.384791in}{0.663715in}}{\pgfqpoint{5.380400in}{0.674314in}}{\pgfqpoint{5.372587in}{0.682127in}}%
\pgfpathcurveto{\pgfqpoint{5.364773in}{0.689941in}}{\pgfqpoint{5.354174in}{0.694331in}}{\pgfqpoint{5.343124in}{0.694331in}}%
\pgfpathcurveto{\pgfqpoint{5.332074in}{0.694331in}}{\pgfqpoint{5.321475in}{0.689941in}}{\pgfqpoint{5.313661in}{0.682127in}}%
\pgfpathcurveto{\pgfqpoint{5.305847in}{0.674314in}}{\pgfqpoint{5.301457in}{0.663715in}}{\pgfqpoint{5.301457in}{0.652664in}}%
\pgfpathcurveto{\pgfqpoint{5.301457in}{0.641614in}}{\pgfqpoint{5.305847in}{0.631015in}}{\pgfqpoint{5.313661in}{0.623202in}}%
\pgfpathcurveto{\pgfqpoint{5.321475in}{0.615388in}}{\pgfqpoint{5.332074in}{0.610998in}}{\pgfqpoint{5.343124in}{0.610998in}}%
\pgfpathlineto{\pgfqpoint{5.343124in}{0.610998in}}%
\pgfpathclose%
\pgfusepath{stroke}%
\end{pgfscope}%
\begin{pgfscope}%
\pgfpathrectangle{\pgfqpoint{0.847223in}{0.554012in}}{\pgfqpoint{6.200000in}{4.530000in}}%
\pgfusepath{clip}%
\pgfsetbuttcap%
\pgfsetroundjoin%
\pgfsetlinewidth{1.003750pt}%
\definecolor{currentstroke}{rgb}{1.000000,0.000000,0.000000}%
\pgfsetstrokecolor{currentstroke}%
\pgfsetdash{}{0pt}%
\pgfpathmoveto{\pgfqpoint{5.348457in}{0.610450in}}%
\pgfpathcurveto{\pgfqpoint{5.359507in}{0.610450in}}{\pgfqpoint{5.370106in}{0.614841in}}{\pgfqpoint{5.377920in}{0.622654in}}%
\pgfpathcurveto{\pgfqpoint{5.385733in}{0.630468in}}{\pgfqpoint{5.390124in}{0.641067in}}{\pgfqpoint{5.390124in}{0.652117in}}%
\pgfpathcurveto{\pgfqpoint{5.390124in}{0.663167in}}{\pgfqpoint{5.385733in}{0.673766in}}{\pgfqpoint{5.377920in}{0.681580in}}%
\pgfpathcurveto{\pgfqpoint{5.370106in}{0.689393in}}{\pgfqpoint{5.359507in}{0.693784in}}{\pgfqpoint{5.348457in}{0.693784in}}%
\pgfpathcurveto{\pgfqpoint{5.337407in}{0.693784in}}{\pgfqpoint{5.326808in}{0.689393in}}{\pgfqpoint{5.318994in}{0.681580in}}%
\pgfpathcurveto{\pgfqpoint{5.311181in}{0.673766in}}{\pgfqpoint{5.306790in}{0.663167in}}{\pgfqpoint{5.306790in}{0.652117in}}%
\pgfpathcurveto{\pgfqpoint{5.306790in}{0.641067in}}{\pgfqpoint{5.311181in}{0.630468in}}{\pgfqpoint{5.318994in}{0.622654in}}%
\pgfpathcurveto{\pgfqpoint{5.326808in}{0.614841in}}{\pgfqpoint{5.337407in}{0.610450in}}{\pgfqpoint{5.348457in}{0.610450in}}%
\pgfpathlineto{\pgfqpoint{5.348457in}{0.610450in}}%
\pgfpathclose%
\pgfusepath{stroke}%
\end{pgfscope}%
\begin{pgfscope}%
\pgfpathrectangle{\pgfqpoint{0.847223in}{0.554012in}}{\pgfqpoint{6.200000in}{4.530000in}}%
\pgfusepath{clip}%
\pgfsetbuttcap%
\pgfsetroundjoin%
\pgfsetlinewidth{1.003750pt}%
\definecolor{currentstroke}{rgb}{1.000000,0.000000,0.000000}%
\pgfsetstrokecolor{currentstroke}%
\pgfsetdash{}{0pt}%
\pgfpathmoveto{\pgfqpoint{5.353790in}{0.609904in}}%
\pgfpathcurveto{\pgfqpoint{5.364840in}{0.609904in}}{\pgfqpoint{5.375439in}{0.614294in}}{\pgfqpoint{5.383253in}{0.622108in}}%
\pgfpathcurveto{\pgfqpoint{5.391067in}{0.629921in}}{\pgfqpoint{5.395457in}{0.640520in}}{\pgfqpoint{5.395457in}{0.651571in}}%
\pgfpathcurveto{\pgfqpoint{5.395457in}{0.662621in}}{\pgfqpoint{5.391067in}{0.673220in}}{\pgfqpoint{5.383253in}{0.681033in}}%
\pgfpathcurveto{\pgfqpoint{5.375439in}{0.688847in}}{\pgfqpoint{5.364840in}{0.693237in}}{\pgfqpoint{5.353790in}{0.693237in}}%
\pgfpathcurveto{\pgfqpoint{5.342740in}{0.693237in}}{\pgfqpoint{5.332141in}{0.688847in}}{\pgfqpoint{5.324328in}{0.681033in}}%
\pgfpathcurveto{\pgfqpoint{5.316514in}{0.673220in}}{\pgfqpoint{5.312124in}{0.662621in}}{\pgfqpoint{5.312124in}{0.651571in}}%
\pgfpathcurveto{\pgfqpoint{5.312124in}{0.640520in}}{\pgfqpoint{5.316514in}{0.629921in}}{\pgfqpoint{5.324328in}{0.622108in}}%
\pgfpathcurveto{\pgfqpoint{5.332141in}{0.614294in}}{\pgfqpoint{5.342740in}{0.609904in}}{\pgfqpoint{5.353790in}{0.609904in}}%
\pgfpathlineto{\pgfqpoint{5.353790in}{0.609904in}}%
\pgfpathclose%
\pgfusepath{stroke}%
\end{pgfscope}%
\begin{pgfscope}%
\pgfpathrectangle{\pgfqpoint{0.847223in}{0.554012in}}{\pgfqpoint{6.200000in}{4.530000in}}%
\pgfusepath{clip}%
\pgfsetbuttcap%
\pgfsetroundjoin%
\pgfsetlinewidth{1.003750pt}%
\definecolor{currentstroke}{rgb}{1.000000,0.000000,0.000000}%
\pgfsetstrokecolor{currentstroke}%
\pgfsetdash{}{0pt}%
\pgfpathmoveto{\pgfqpoint{5.359124in}{0.609359in}}%
\pgfpathcurveto{\pgfqpoint{5.370174in}{0.609359in}}{\pgfqpoint{5.380773in}{0.613749in}}{\pgfqpoint{5.388586in}{0.621563in}}%
\pgfpathcurveto{\pgfqpoint{5.396400in}{0.629376in}}{\pgfqpoint{5.400790in}{0.639975in}}{\pgfqpoint{5.400790in}{0.651025in}}%
\pgfpathcurveto{\pgfqpoint{5.400790in}{0.662075in}}{\pgfqpoint{5.396400in}{0.672675in}}{\pgfqpoint{5.388586in}{0.680488in}}%
\pgfpathcurveto{\pgfqpoint{5.380773in}{0.688302in}}{\pgfqpoint{5.370174in}{0.692692in}}{\pgfqpoint{5.359124in}{0.692692in}}%
\pgfpathcurveto{\pgfqpoint{5.348073in}{0.692692in}}{\pgfqpoint{5.337474in}{0.688302in}}{\pgfqpoint{5.329661in}{0.680488in}}%
\pgfpathcurveto{\pgfqpoint{5.321847in}{0.672675in}}{\pgfqpoint{5.317457in}{0.662075in}}{\pgfqpoint{5.317457in}{0.651025in}}%
\pgfpathcurveto{\pgfqpoint{5.317457in}{0.639975in}}{\pgfqpoint{5.321847in}{0.629376in}}{\pgfqpoint{5.329661in}{0.621563in}}%
\pgfpathcurveto{\pgfqpoint{5.337474in}{0.613749in}}{\pgfqpoint{5.348073in}{0.609359in}}{\pgfqpoint{5.359124in}{0.609359in}}%
\pgfpathlineto{\pgfqpoint{5.359124in}{0.609359in}}%
\pgfpathclose%
\pgfusepath{stroke}%
\end{pgfscope}%
\begin{pgfscope}%
\pgfpathrectangle{\pgfqpoint{0.847223in}{0.554012in}}{\pgfqpoint{6.200000in}{4.530000in}}%
\pgfusepath{clip}%
\pgfsetbuttcap%
\pgfsetroundjoin%
\pgfsetlinewidth{1.003750pt}%
\definecolor{currentstroke}{rgb}{1.000000,0.000000,0.000000}%
\pgfsetstrokecolor{currentstroke}%
\pgfsetdash{}{0pt}%
\pgfpathmoveto{\pgfqpoint{5.364457in}{0.608815in}}%
\pgfpathcurveto{\pgfqpoint{5.375507in}{0.608815in}}{\pgfqpoint{5.386106in}{0.613205in}}{\pgfqpoint{5.393920in}{0.621019in}}%
\pgfpathcurveto{\pgfqpoint{5.401733in}{0.628832in}}{\pgfqpoint{5.406123in}{0.639431in}}{\pgfqpoint{5.406123in}{0.650481in}}%
\pgfpathcurveto{\pgfqpoint{5.406123in}{0.661531in}}{\pgfqpoint{5.401733in}{0.672130in}}{\pgfqpoint{5.393920in}{0.679944in}}%
\pgfpathcurveto{\pgfqpoint{5.386106in}{0.687758in}}{\pgfqpoint{5.375507in}{0.692148in}}{\pgfqpoint{5.364457in}{0.692148in}}%
\pgfpathcurveto{\pgfqpoint{5.353407in}{0.692148in}}{\pgfqpoint{5.342808in}{0.687758in}}{\pgfqpoint{5.334994in}{0.679944in}}%
\pgfpathcurveto{\pgfqpoint{5.327180in}{0.672130in}}{\pgfqpoint{5.322790in}{0.661531in}}{\pgfqpoint{5.322790in}{0.650481in}}%
\pgfpathcurveto{\pgfqpoint{5.322790in}{0.639431in}}{\pgfqpoint{5.327180in}{0.628832in}}{\pgfqpoint{5.334994in}{0.621019in}}%
\pgfpathcurveto{\pgfqpoint{5.342808in}{0.613205in}}{\pgfqpoint{5.353407in}{0.608815in}}{\pgfqpoint{5.364457in}{0.608815in}}%
\pgfpathlineto{\pgfqpoint{5.364457in}{0.608815in}}%
\pgfpathclose%
\pgfusepath{stroke}%
\end{pgfscope}%
\begin{pgfscope}%
\pgfpathrectangle{\pgfqpoint{0.847223in}{0.554012in}}{\pgfqpoint{6.200000in}{4.530000in}}%
\pgfusepath{clip}%
\pgfsetbuttcap%
\pgfsetroundjoin%
\pgfsetlinewidth{1.003750pt}%
\definecolor{currentstroke}{rgb}{1.000000,0.000000,0.000000}%
\pgfsetstrokecolor{currentstroke}%
\pgfsetdash{}{0pt}%
\pgfpathmoveto{\pgfqpoint{5.369790in}{0.608272in}}%
\pgfpathcurveto{\pgfqpoint{5.380840in}{0.608272in}}{\pgfqpoint{5.391439in}{0.612662in}}{\pgfqpoint{5.399253in}{0.620476in}}%
\pgfpathcurveto{\pgfqpoint{5.407066in}{0.628289in}}{\pgfqpoint{5.411457in}{0.638888in}}{\pgfqpoint{5.411457in}{0.649938in}}%
\pgfpathcurveto{\pgfqpoint{5.411457in}{0.660989in}}{\pgfqpoint{5.407066in}{0.671588in}}{\pgfqpoint{5.399253in}{0.679401in}}%
\pgfpathcurveto{\pgfqpoint{5.391439in}{0.687215in}}{\pgfqpoint{5.380840in}{0.691605in}}{\pgfqpoint{5.369790in}{0.691605in}}%
\pgfpathcurveto{\pgfqpoint{5.358740in}{0.691605in}}{\pgfqpoint{5.348141in}{0.687215in}}{\pgfqpoint{5.340327in}{0.679401in}}%
\pgfpathcurveto{\pgfqpoint{5.332514in}{0.671588in}}{\pgfqpoint{5.328123in}{0.660989in}}{\pgfqpoint{5.328123in}{0.649938in}}%
\pgfpathcurveto{\pgfqpoint{5.328123in}{0.638888in}}{\pgfqpoint{5.332514in}{0.628289in}}{\pgfqpoint{5.340327in}{0.620476in}}%
\pgfpathcurveto{\pgfqpoint{5.348141in}{0.612662in}}{\pgfqpoint{5.358740in}{0.608272in}}{\pgfqpoint{5.369790in}{0.608272in}}%
\pgfpathlineto{\pgfqpoint{5.369790in}{0.608272in}}%
\pgfpathclose%
\pgfusepath{stroke}%
\end{pgfscope}%
\begin{pgfscope}%
\pgfpathrectangle{\pgfqpoint{0.847223in}{0.554012in}}{\pgfqpoint{6.200000in}{4.530000in}}%
\pgfusepath{clip}%
\pgfsetbuttcap%
\pgfsetroundjoin%
\pgfsetlinewidth{1.003750pt}%
\definecolor{currentstroke}{rgb}{1.000000,0.000000,0.000000}%
\pgfsetstrokecolor{currentstroke}%
\pgfsetdash{}{0pt}%
\pgfpathmoveto{\pgfqpoint{5.375123in}{0.607730in}}%
\pgfpathcurveto{\pgfqpoint{5.386173in}{0.607730in}}{\pgfqpoint{5.396772in}{0.612120in}}{\pgfqpoint{5.404586in}{0.619934in}}%
\pgfpathcurveto{\pgfqpoint{5.412400in}{0.627747in}}{\pgfqpoint{5.416790in}{0.638347in}}{\pgfqpoint{5.416790in}{0.649397in}}%
\pgfpathcurveto{\pgfqpoint{5.416790in}{0.660447in}}{\pgfqpoint{5.412400in}{0.671046in}}{\pgfqpoint{5.404586in}{0.678859in}}%
\pgfpathcurveto{\pgfqpoint{5.396772in}{0.686673in}}{\pgfqpoint{5.386173in}{0.691063in}}{\pgfqpoint{5.375123in}{0.691063in}}%
\pgfpathcurveto{\pgfqpoint{5.364073in}{0.691063in}}{\pgfqpoint{5.353474in}{0.686673in}}{\pgfqpoint{5.345660in}{0.678859in}}%
\pgfpathcurveto{\pgfqpoint{5.337847in}{0.671046in}}{\pgfqpoint{5.333456in}{0.660447in}}{\pgfqpoint{5.333456in}{0.649397in}}%
\pgfpathcurveto{\pgfqpoint{5.333456in}{0.638347in}}{\pgfqpoint{5.337847in}{0.627747in}}{\pgfqpoint{5.345660in}{0.619934in}}%
\pgfpathcurveto{\pgfqpoint{5.353474in}{0.612120in}}{\pgfqpoint{5.364073in}{0.607730in}}{\pgfqpoint{5.375123in}{0.607730in}}%
\pgfpathlineto{\pgfqpoint{5.375123in}{0.607730in}}%
\pgfpathclose%
\pgfusepath{stroke}%
\end{pgfscope}%
\begin{pgfscope}%
\pgfpathrectangle{\pgfqpoint{0.847223in}{0.554012in}}{\pgfqpoint{6.200000in}{4.530000in}}%
\pgfusepath{clip}%
\pgfsetbuttcap%
\pgfsetroundjoin%
\pgfsetlinewidth{1.003750pt}%
\definecolor{currentstroke}{rgb}{1.000000,0.000000,0.000000}%
\pgfsetstrokecolor{currentstroke}%
\pgfsetdash{}{0pt}%
\pgfpathmoveto{\pgfqpoint{5.380456in}{0.607189in}}%
\pgfpathcurveto{\pgfqpoint{5.391506in}{0.607189in}}{\pgfqpoint{5.402106in}{0.611580in}}{\pgfqpoint{5.409919in}{0.619393in}}%
\pgfpathcurveto{\pgfqpoint{5.417733in}{0.627207in}}{\pgfqpoint{5.422123in}{0.637806in}}{\pgfqpoint{5.422123in}{0.648856in}}%
\pgfpathcurveto{\pgfqpoint{5.422123in}{0.659906in}}{\pgfqpoint{5.417733in}{0.670505in}}{\pgfqpoint{5.409919in}{0.678319in}}%
\pgfpathcurveto{\pgfqpoint{5.402106in}{0.686132in}}{\pgfqpoint{5.391506in}{0.690523in}}{\pgfqpoint{5.380456in}{0.690523in}}%
\pgfpathcurveto{\pgfqpoint{5.369406in}{0.690523in}}{\pgfqpoint{5.358807in}{0.686132in}}{\pgfqpoint{5.350994in}{0.678319in}}%
\pgfpathcurveto{\pgfqpoint{5.343180in}{0.670505in}}{\pgfqpoint{5.338790in}{0.659906in}}{\pgfqpoint{5.338790in}{0.648856in}}%
\pgfpathcurveto{\pgfqpoint{5.338790in}{0.637806in}}{\pgfqpoint{5.343180in}{0.627207in}}{\pgfqpoint{5.350994in}{0.619393in}}%
\pgfpathcurveto{\pgfqpoint{5.358807in}{0.611580in}}{\pgfqpoint{5.369406in}{0.607189in}}{\pgfqpoint{5.380456in}{0.607189in}}%
\pgfpathlineto{\pgfqpoint{5.380456in}{0.607189in}}%
\pgfpathclose%
\pgfusepath{stroke}%
\end{pgfscope}%
\begin{pgfscope}%
\pgfpathrectangle{\pgfqpoint{0.847223in}{0.554012in}}{\pgfqpoint{6.200000in}{4.530000in}}%
\pgfusepath{clip}%
\pgfsetbuttcap%
\pgfsetroundjoin%
\pgfsetlinewidth{1.003750pt}%
\definecolor{currentstroke}{rgb}{1.000000,0.000000,0.000000}%
\pgfsetstrokecolor{currentstroke}%
\pgfsetdash{}{0pt}%
\pgfpathmoveto{\pgfqpoint{5.385790in}{0.606650in}}%
\pgfpathcurveto{\pgfqpoint{5.396840in}{0.606650in}}{\pgfqpoint{5.407439in}{0.611040in}}{\pgfqpoint{5.415252in}{0.618854in}}%
\pgfpathcurveto{\pgfqpoint{5.423066in}{0.626667in}}{\pgfqpoint{5.427456in}{0.637266in}}{\pgfqpoint{5.427456in}{0.648317in}}%
\pgfpathcurveto{\pgfqpoint{5.427456in}{0.659367in}}{\pgfqpoint{5.423066in}{0.669966in}}{\pgfqpoint{5.415252in}{0.677779in}}%
\pgfpathcurveto{\pgfqpoint{5.407439in}{0.685593in}}{\pgfqpoint{5.396840in}{0.689983in}}{\pgfqpoint{5.385790in}{0.689983in}}%
\pgfpathcurveto{\pgfqpoint{5.374739in}{0.689983in}}{\pgfqpoint{5.364140in}{0.685593in}}{\pgfqpoint{5.356327in}{0.677779in}}%
\pgfpathcurveto{\pgfqpoint{5.348513in}{0.669966in}}{\pgfqpoint{5.344123in}{0.659367in}}{\pgfqpoint{5.344123in}{0.648317in}}%
\pgfpathcurveto{\pgfqpoint{5.344123in}{0.637266in}}{\pgfqpoint{5.348513in}{0.626667in}}{\pgfqpoint{5.356327in}{0.618854in}}%
\pgfpathcurveto{\pgfqpoint{5.364140in}{0.611040in}}{\pgfqpoint{5.374739in}{0.606650in}}{\pgfqpoint{5.385790in}{0.606650in}}%
\pgfpathlineto{\pgfqpoint{5.385790in}{0.606650in}}%
\pgfpathclose%
\pgfusepath{stroke}%
\end{pgfscope}%
\begin{pgfscope}%
\pgfpathrectangle{\pgfqpoint{0.847223in}{0.554012in}}{\pgfqpoint{6.200000in}{4.530000in}}%
\pgfusepath{clip}%
\pgfsetbuttcap%
\pgfsetroundjoin%
\pgfsetlinewidth{1.003750pt}%
\definecolor{currentstroke}{rgb}{1.000000,0.000000,0.000000}%
\pgfsetstrokecolor{currentstroke}%
\pgfsetdash{}{0pt}%
\pgfpathmoveto{\pgfqpoint{5.391123in}{0.606112in}}%
\pgfpathcurveto{\pgfqpoint{5.402173in}{0.606112in}}{\pgfqpoint{5.412772in}{0.610502in}}{\pgfqpoint{5.420586in}{0.618315in}}%
\pgfpathcurveto{\pgfqpoint{5.428399in}{0.626129in}}{\pgfqpoint{5.432789in}{0.636728in}}{\pgfqpoint{5.432789in}{0.647778in}}%
\pgfpathcurveto{\pgfqpoint{5.432789in}{0.658828in}}{\pgfqpoint{5.428399in}{0.669427in}}{\pgfqpoint{5.420586in}{0.677241in}}%
\pgfpathcurveto{\pgfqpoint{5.412772in}{0.685055in}}{\pgfqpoint{5.402173in}{0.689445in}}{\pgfqpoint{5.391123in}{0.689445in}}%
\pgfpathcurveto{\pgfqpoint{5.380073in}{0.689445in}}{\pgfqpoint{5.369474in}{0.685055in}}{\pgfqpoint{5.361660in}{0.677241in}}%
\pgfpathcurveto{\pgfqpoint{5.353846in}{0.669427in}}{\pgfqpoint{5.349456in}{0.658828in}}{\pgfqpoint{5.349456in}{0.647778in}}%
\pgfpathcurveto{\pgfqpoint{5.349456in}{0.636728in}}{\pgfqpoint{5.353846in}{0.626129in}}{\pgfqpoint{5.361660in}{0.618315in}}%
\pgfpathcurveto{\pgfqpoint{5.369474in}{0.610502in}}{\pgfqpoint{5.380073in}{0.606112in}}{\pgfqpoint{5.391123in}{0.606112in}}%
\pgfpathlineto{\pgfqpoint{5.391123in}{0.606112in}}%
\pgfpathclose%
\pgfusepath{stroke}%
\end{pgfscope}%
\begin{pgfscope}%
\pgfpathrectangle{\pgfqpoint{0.847223in}{0.554012in}}{\pgfqpoint{6.200000in}{4.530000in}}%
\pgfusepath{clip}%
\pgfsetbuttcap%
\pgfsetroundjoin%
\pgfsetlinewidth{1.003750pt}%
\definecolor{currentstroke}{rgb}{1.000000,0.000000,0.000000}%
\pgfsetstrokecolor{currentstroke}%
\pgfsetdash{}{0pt}%
\pgfpathmoveto{\pgfqpoint{5.396456in}{0.605574in}}%
\pgfpathcurveto{\pgfqpoint{5.407506in}{0.605574in}}{\pgfqpoint{5.418105in}{0.609965in}}{\pgfqpoint{5.425919in}{0.617778in}}%
\pgfpathcurveto{\pgfqpoint{5.433732in}{0.625592in}}{\pgfqpoint{5.438123in}{0.636191in}}{\pgfqpoint{5.438123in}{0.647241in}}%
\pgfpathcurveto{\pgfqpoint{5.438123in}{0.658291in}}{\pgfqpoint{5.433732in}{0.668890in}}{\pgfqpoint{5.425919in}{0.676704in}}%
\pgfpathcurveto{\pgfqpoint{5.418105in}{0.684517in}}{\pgfqpoint{5.407506in}{0.688908in}}{\pgfqpoint{5.396456in}{0.688908in}}%
\pgfpathcurveto{\pgfqpoint{5.385406in}{0.688908in}}{\pgfqpoint{5.374807in}{0.684517in}}{\pgfqpoint{5.366993in}{0.676704in}}%
\pgfpathcurveto{\pgfqpoint{5.359180in}{0.668890in}}{\pgfqpoint{5.354789in}{0.658291in}}{\pgfqpoint{5.354789in}{0.647241in}}%
\pgfpathcurveto{\pgfqpoint{5.354789in}{0.636191in}}{\pgfqpoint{5.359180in}{0.625592in}}{\pgfqpoint{5.366993in}{0.617778in}}%
\pgfpathcurveto{\pgfqpoint{5.374807in}{0.609965in}}{\pgfqpoint{5.385406in}{0.605574in}}{\pgfqpoint{5.396456in}{0.605574in}}%
\pgfpathlineto{\pgfqpoint{5.396456in}{0.605574in}}%
\pgfpathclose%
\pgfusepath{stroke}%
\end{pgfscope}%
\begin{pgfscope}%
\pgfpathrectangle{\pgfqpoint{0.847223in}{0.554012in}}{\pgfqpoint{6.200000in}{4.530000in}}%
\pgfusepath{clip}%
\pgfsetbuttcap%
\pgfsetroundjoin%
\pgfsetlinewidth{1.003750pt}%
\definecolor{currentstroke}{rgb}{1.000000,0.000000,0.000000}%
\pgfsetstrokecolor{currentstroke}%
\pgfsetdash{}{0pt}%
\pgfpathmoveto{\pgfqpoint{5.401789in}{0.605038in}}%
\pgfpathcurveto{\pgfqpoint{5.412839in}{0.605038in}}{\pgfqpoint{5.423438in}{0.609429in}}{\pgfqpoint{5.431252in}{0.617242in}}%
\pgfpathcurveto{\pgfqpoint{5.439066in}{0.625056in}}{\pgfqpoint{5.443456in}{0.635655in}}{\pgfqpoint{5.443456in}{0.646705in}}%
\pgfpathcurveto{\pgfqpoint{5.443456in}{0.657755in}}{\pgfqpoint{5.439066in}{0.668354in}}{\pgfqpoint{5.431252in}{0.676168in}}%
\pgfpathcurveto{\pgfqpoint{5.423438in}{0.683981in}}{\pgfqpoint{5.412839in}{0.688372in}}{\pgfqpoint{5.401789in}{0.688372in}}%
\pgfpathcurveto{\pgfqpoint{5.390739in}{0.688372in}}{\pgfqpoint{5.380140in}{0.683981in}}{\pgfqpoint{5.372326in}{0.676168in}}%
\pgfpathcurveto{\pgfqpoint{5.364513in}{0.668354in}}{\pgfqpoint{5.360123in}{0.657755in}}{\pgfqpoint{5.360123in}{0.646705in}}%
\pgfpathcurveto{\pgfqpoint{5.360123in}{0.635655in}}{\pgfqpoint{5.364513in}{0.625056in}}{\pgfqpoint{5.372326in}{0.617242in}}%
\pgfpathcurveto{\pgfqpoint{5.380140in}{0.609429in}}{\pgfqpoint{5.390739in}{0.605038in}}{\pgfqpoint{5.401789in}{0.605038in}}%
\pgfpathlineto{\pgfqpoint{5.401789in}{0.605038in}}%
\pgfpathclose%
\pgfusepath{stroke}%
\end{pgfscope}%
\begin{pgfscope}%
\pgfpathrectangle{\pgfqpoint{0.847223in}{0.554012in}}{\pgfqpoint{6.200000in}{4.530000in}}%
\pgfusepath{clip}%
\pgfsetbuttcap%
\pgfsetroundjoin%
\pgfsetlinewidth{1.003750pt}%
\definecolor{currentstroke}{rgb}{1.000000,0.000000,0.000000}%
\pgfsetstrokecolor{currentstroke}%
\pgfsetdash{}{0pt}%
\pgfpathmoveto{\pgfqpoint{5.407122in}{0.604503in}}%
\pgfpathcurveto{\pgfqpoint{5.418173in}{0.604503in}}{\pgfqpoint{5.428772in}{0.608894in}}{\pgfqpoint{5.436585in}{0.616707in}}%
\pgfpathcurveto{\pgfqpoint{5.444399in}{0.624521in}}{\pgfqpoint{5.448789in}{0.635120in}}{\pgfqpoint{5.448789in}{0.646170in}}%
\pgfpathcurveto{\pgfqpoint{5.448789in}{0.657220in}}{\pgfqpoint{5.444399in}{0.667819in}}{\pgfqpoint{5.436585in}{0.675633in}}%
\pgfpathcurveto{\pgfqpoint{5.428772in}{0.683446in}}{\pgfqpoint{5.418173in}{0.687837in}}{\pgfqpoint{5.407122in}{0.687837in}}%
\pgfpathcurveto{\pgfqpoint{5.396072in}{0.687837in}}{\pgfqpoint{5.385473in}{0.683446in}}{\pgfqpoint{5.377660in}{0.675633in}}%
\pgfpathcurveto{\pgfqpoint{5.369846in}{0.667819in}}{\pgfqpoint{5.365456in}{0.657220in}}{\pgfqpoint{5.365456in}{0.646170in}}%
\pgfpathcurveto{\pgfqpoint{5.365456in}{0.635120in}}{\pgfqpoint{5.369846in}{0.624521in}}{\pgfqpoint{5.377660in}{0.616707in}}%
\pgfpathcurveto{\pgfqpoint{5.385473in}{0.608894in}}{\pgfqpoint{5.396072in}{0.604503in}}{\pgfqpoint{5.407122in}{0.604503in}}%
\pgfpathlineto{\pgfqpoint{5.407122in}{0.604503in}}%
\pgfpathclose%
\pgfusepath{stroke}%
\end{pgfscope}%
\begin{pgfscope}%
\pgfpathrectangle{\pgfqpoint{0.847223in}{0.554012in}}{\pgfqpoint{6.200000in}{4.530000in}}%
\pgfusepath{clip}%
\pgfsetbuttcap%
\pgfsetroundjoin%
\pgfsetlinewidth{1.003750pt}%
\definecolor{currentstroke}{rgb}{1.000000,0.000000,0.000000}%
\pgfsetstrokecolor{currentstroke}%
\pgfsetdash{}{0pt}%
\pgfpathmoveto{\pgfqpoint{5.412456in}{0.603970in}}%
\pgfpathcurveto{\pgfqpoint{5.423506in}{0.603970in}}{\pgfqpoint{5.434105in}{0.608360in}}{\pgfqpoint{5.441918in}{0.616173in}}%
\pgfpathcurveto{\pgfqpoint{5.449732in}{0.623987in}}{\pgfqpoint{5.454122in}{0.634586in}}{\pgfqpoint{5.454122in}{0.645636in}}%
\pgfpathcurveto{\pgfqpoint{5.454122in}{0.656686in}}{\pgfqpoint{5.449732in}{0.667285in}}{\pgfqpoint{5.441918in}{0.675099in}}%
\pgfpathcurveto{\pgfqpoint{5.434105in}{0.682913in}}{\pgfqpoint{5.423506in}{0.687303in}}{\pgfqpoint{5.412456in}{0.687303in}}%
\pgfpathcurveto{\pgfqpoint{5.401406in}{0.687303in}}{\pgfqpoint{5.390806in}{0.682913in}}{\pgfqpoint{5.382993in}{0.675099in}}%
\pgfpathcurveto{\pgfqpoint{5.375179in}{0.667285in}}{\pgfqpoint{5.370789in}{0.656686in}}{\pgfqpoint{5.370789in}{0.645636in}}%
\pgfpathcurveto{\pgfqpoint{5.370789in}{0.634586in}}{\pgfqpoint{5.375179in}{0.623987in}}{\pgfqpoint{5.382993in}{0.616173in}}%
\pgfpathcurveto{\pgfqpoint{5.390806in}{0.608360in}}{\pgfqpoint{5.401406in}{0.603970in}}{\pgfqpoint{5.412456in}{0.603970in}}%
\pgfpathlineto{\pgfqpoint{5.412456in}{0.603970in}}%
\pgfpathclose%
\pgfusepath{stroke}%
\end{pgfscope}%
\begin{pgfscope}%
\pgfpathrectangle{\pgfqpoint{0.847223in}{0.554012in}}{\pgfqpoint{6.200000in}{4.530000in}}%
\pgfusepath{clip}%
\pgfsetbuttcap%
\pgfsetroundjoin%
\pgfsetlinewidth{1.003750pt}%
\definecolor{currentstroke}{rgb}{1.000000,0.000000,0.000000}%
\pgfsetstrokecolor{currentstroke}%
\pgfsetdash{}{0pt}%
\pgfpathmoveto{\pgfqpoint{5.417789in}{0.603437in}}%
\pgfpathcurveto{\pgfqpoint{5.428839in}{0.603437in}}{\pgfqpoint{5.439438in}{0.607827in}}{\pgfqpoint{5.447252in}{0.615641in}}%
\pgfpathcurveto{\pgfqpoint{5.455065in}{0.623454in}}{\pgfqpoint{5.459456in}{0.634053in}}{\pgfqpoint{5.459456in}{0.645103in}}%
\pgfpathcurveto{\pgfqpoint{5.459456in}{0.656154in}}{\pgfqpoint{5.455065in}{0.666753in}}{\pgfqpoint{5.447252in}{0.674566in}}%
\pgfpathcurveto{\pgfqpoint{5.439438in}{0.682380in}}{\pgfqpoint{5.428839in}{0.686770in}}{\pgfqpoint{5.417789in}{0.686770in}}%
\pgfpathcurveto{\pgfqpoint{5.406739in}{0.686770in}}{\pgfqpoint{5.396140in}{0.682380in}}{\pgfqpoint{5.388326in}{0.674566in}}%
\pgfpathcurveto{\pgfqpoint{5.380512in}{0.666753in}}{\pgfqpoint{5.376122in}{0.656154in}}{\pgfqpoint{5.376122in}{0.645103in}}%
\pgfpathcurveto{\pgfqpoint{5.376122in}{0.634053in}}{\pgfqpoint{5.380512in}{0.623454in}}{\pgfqpoint{5.388326in}{0.615641in}}%
\pgfpathcurveto{\pgfqpoint{5.396140in}{0.607827in}}{\pgfqpoint{5.406739in}{0.603437in}}{\pgfqpoint{5.417789in}{0.603437in}}%
\pgfpathlineto{\pgfqpoint{5.417789in}{0.603437in}}%
\pgfpathclose%
\pgfusepath{stroke}%
\end{pgfscope}%
\begin{pgfscope}%
\pgfpathrectangle{\pgfqpoint{0.847223in}{0.554012in}}{\pgfqpoint{6.200000in}{4.530000in}}%
\pgfusepath{clip}%
\pgfsetbuttcap%
\pgfsetroundjoin%
\pgfsetlinewidth{1.003750pt}%
\definecolor{currentstroke}{rgb}{1.000000,0.000000,0.000000}%
\pgfsetstrokecolor{currentstroke}%
\pgfsetdash{}{0pt}%
\pgfpathmoveto{\pgfqpoint{5.423122in}{0.602905in}}%
\pgfpathcurveto{\pgfqpoint{5.434172in}{0.602905in}}{\pgfqpoint{5.444771in}{0.607295in}}{\pgfqpoint{5.452585in}{0.615109in}}%
\pgfpathcurveto{\pgfqpoint{5.460398in}{0.622923in}}{\pgfqpoint{5.464789in}{0.633522in}}{\pgfqpoint{5.464789in}{0.644572in}}%
\pgfpathcurveto{\pgfqpoint{5.464789in}{0.655622in}}{\pgfqpoint{5.460398in}{0.666221in}}{\pgfqpoint{5.452585in}{0.674035in}}%
\pgfpathcurveto{\pgfqpoint{5.444771in}{0.681848in}}{\pgfqpoint{5.434172in}{0.686239in}}{\pgfqpoint{5.423122in}{0.686239in}}%
\pgfpathcurveto{\pgfqpoint{5.412072in}{0.686239in}}{\pgfqpoint{5.401473in}{0.681848in}}{\pgfqpoint{5.393659in}{0.674035in}}%
\pgfpathcurveto{\pgfqpoint{5.385846in}{0.666221in}}{\pgfqpoint{5.381455in}{0.655622in}}{\pgfqpoint{5.381455in}{0.644572in}}%
\pgfpathcurveto{\pgfqpoint{5.381455in}{0.633522in}}{\pgfqpoint{5.385846in}{0.622923in}}{\pgfqpoint{5.393659in}{0.615109in}}%
\pgfpathcurveto{\pgfqpoint{5.401473in}{0.607295in}}{\pgfqpoint{5.412072in}{0.602905in}}{\pgfqpoint{5.423122in}{0.602905in}}%
\pgfpathlineto{\pgfqpoint{5.423122in}{0.602905in}}%
\pgfpathclose%
\pgfusepath{stroke}%
\end{pgfscope}%
\begin{pgfscope}%
\pgfpathrectangle{\pgfqpoint{0.847223in}{0.554012in}}{\pgfqpoint{6.200000in}{4.530000in}}%
\pgfusepath{clip}%
\pgfsetbuttcap%
\pgfsetroundjoin%
\pgfsetlinewidth{1.003750pt}%
\definecolor{currentstroke}{rgb}{1.000000,0.000000,0.000000}%
\pgfsetstrokecolor{currentstroke}%
\pgfsetdash{}{0pt}%
\pgfpathmoveto{\pgfqpoint{5.428455in}{0.602375in}}%
\pgfpathcurveto{\pgfqpoint{5.439505in}{0.602375in}}{\pgfqpoint{5.450104in}{0.606765in}}{\pgfqpoint{5.457918in}{0.614579in}}%
\pgfpathcurveto{\pgfqpoint{5.465732in}{0.622392in}}{\pgfqpoint{5.470122in}{0.632991in}}{\pgfqpoint{5.470122in}{0.644041in}}%
\pgfpathcurveto{\pgfqpoint{5.470122in}{0.655092in}}{\pgfqpoint{5.465732in}{0.665691in}}{\pgfqpoint{5.457918in}{0.673504in}}%
\pgfpathcurveto{\pgfqpoint{5.450104in}{0.681318in}}{\pgfqpoint{5.439505in}{0.685708in}}{\pgfqpoint{5.428455in}{0.685708in}}%
\pgfpathcurveto{\pgfqpoint{5.417405in}{0.685708in}}{\pgfqpoint{5.406806in}{0.681318in}}{\pgfqpoint{5.398993in}{0.673504in}}%
\pgfpathcurveto{\pgfqpoint{5.391179in}{0.665691in}}{\pgfqpoint{5.386789in}{0.655092in}}{\pgfqpoint{5.386789in}{0.644041in}}%
\pgfpathcurveto{\pgfqpoint{5.386789in}{0.632991in}}{\pgfqpoint{5.391179in}{0.622392in}}{\pgfqpoint{5.398993in}{0.614579in}}%
\pgfpathcurveto{\pgfqpoint{5.406806in}{0.606765in}}{\pgfqpoint{5.417405in}{0.602375in}}{\pgfqpoint{5.428455in}{0.602375in}}%
\pgfpathlineto{\pgfqpoint{5.428455in}{0.602375in}}%
\pgfpathclose%
\pgfusepath{stroke}%
\end{pgfscope}%
\begin{pgfscope}%
\pgfpathrectangle{\pgfqpoint{0.847223in}{0.554012in}}{\pgfqpoint{6.200000in}{4.530000in}}%
\pgfusepath{clip}%
\pgfsetbuttcap%
\pgfsetroundjoin%
\pgfsetlinewidth{1.003750pt}%
\definecolor{currentstroke}{rgb}{1.000000,0.000000,0.000000}%
\pgfsetstrokecolor{currentstroke}%
\pgfsetdash{}{0pt}%
\pgfpathmoveto{\pgfqpoint{5.433789in}{0.601845in}}%
\pgfpathcurveto{\pgfqpoint{5.444839in}{0.601845in}}{\pgfqpoint{5.455438in}{0.606236in}}{\pgfqpoint{5.463251in}{0.614049in}}%
\pgfpathcurveto{\pgfqpoint{5.471065in}{0.621863in}}{\pgfqpoint{5.475455in}{0.632462in}}{\pgfqpoint{5.475455in}{0.643512in}}%
\pgfpathcurveto{\pgfqpoint{5.475455in}{0.654562in}}{\pgfqpoint{5.471065in}{0.665161in}}{\pgfqpoint{5.463251in}{0.672975in}}%
\pgfpathcurveto{\pgfqpoint{5.455438in}{0.680788in}}{\pgfqpoint{5.444839in}{0.685179in}}{\pgfqpoint{5.433789in}{0.685179in}}%
\pgfpathcurveto{\pgfqpoint{5.422738in}{0.685179in}}{\pgfqpoint{5.412139in}{0.680788in}}{\pgfqpoint{5.404326in}{0.672975in}}%
\pgfpathcurveto{\pgfqpoint{5.396512in}{0.665161in}}{\pgfqpoint{5.392122in}{0.654562in}}{\pgfqpoint{5.392122in}{0.643512in}}%
\pgfpathcurveto{\pgfqpoint{5.392122in}{0.632462in}}{\pgfqpoint{5.396512in}{0.621863in}}{\pgfqpoint{5.404326in}{0.614049in}}%
\pgfpathcurveto{\pgfqpoint{5.412139in}{0.606236in}}{\pgfqpoint{5.422738in}{0.601845in}}{\pgfqpoint{5.433789in}{0.601845in}}%
\pgfpathlineto{\pgfqpoint{5.433789in}{0.601845in}}%
\pgfpathclose%
\pgfusepath{stroke}%
\end{pgfscope}%
\begin{pgfscope}%
\pgfpathrectangle{\pgfqpoint{0.847223in}{0.554012in}}{\pgfqpoint{6.200000in}{4.530000in}}%
\pgfusepath{clip}%
\pgfsetbuttcap%
\pgfsetroundjoin%
\pgfsetlinewidth{1.003750pt}%
\definecolor{currentstroke}{rgb}{1.000000,0.000000,0.000000}%
\pgfsetstrokecolor{currentstroke}%
\pgfsetdash{}{0pt}%
\pgfpathmoveto{\pgfqpoint{5.439122in}{0.601317in}}%
\pgfpathcurveto{\pgfqpoint{5.450172in}{0.601317in}}{\pgfqpoint{5.460771in}{0.605707in}}{\pgfqpoint{5.468585in}{0.613521in}}%
\pgfpathcurveto{\pgfqpoint{5.476398in}{0.621335in}}{\pgfqpoint{5.480788in}{0.631934in}}{\pgfqpoint{5.480788in}{0.642984in}}%
\pgfpathcurveto{\pgfqpoint{5.480788in}{0.654034in}}{\pgfqpoint{5.476398in}{0.664633in}}{\pgfqpoint{5.468585in}{0.672447in}}%
\pgfpathcurveto{\pgfqpoint{5.460771in}{0.680260in}}{\pgfqpoint{5.450172in}{0.684650in}}{\pgfqpoint{5.439122in}{0.684650in}}%
\pgfpathcurveto{\pgfqpoint{5.428072in}{0.684650in}}{\pgfqpoint{5.417473in}{0.680260in}}{\pgfqpoint{5.409659in}{0.672447in}}%
\pgfpathcurveto{\pgfqpoint{5.401845in}{0.664633in}}{\pgfqpoint{5.397455in}{0.654034in}}{\pgfqpoint{5.397455in}{0.642984in}}%
\pgfpathcurveto{\pgfqpoint{5.397455in}{0.631934in}}{\pgfqpoint{5.401845in}{0.621335in}}{\pgfqpoint{5.409659in}{0.613521in}}%
\pgfpathcurveto{\pgfqpoint{5.417473in}{0.605707in}}{\pgfqpoint{5.428072in}{0.601317in}}{\pgfqpoint{5.439122in}{0.601317in}}%
\pgfpathlineto{\pgfqpoint{5.439122in}{0.601317in}}%
\pgfpathclose%
\pgfusepath{stroke}%
\end{pgfscope}%
\begin{pgfscope}%
\pgfpathrectangle{\pgfqpoint{0.847223in}{0.554012in}}{\pgfqpoint{6.200000in}{4.530000in}}%
\pgfusepath{clip}%
\pgfsetbuttcap%
\pgfsetroundjoin%
\pgfsetlinewidth{1.003750pt}%
\definecolor{currentstroke}{rgb}{1.000000,0.000000,0.000000}%
\pgfsetstrokecolor{currentstroke}%
\pgfsetdash{}{0pt}%
\pgfpathmoveto{\pgfqpoint{5.444455in}{0.600790in}}%
\pgfpathcurveto{\pgfqpoint{5.455505in}{0.600790in}}{\pgfqpoint{5.466104in}{0.605180in}}{\pgfqpoint{5.473918in}{0.612994in}}%
\pgfpathcurveto{\pgfqpoint{5.481731in}{0.620807in}}{\pgfqpoint{5.486122in}{0.631406in}}{\pgfqpoint{5.486122in}{0.642457in}}%
\pgfpathcurveto{\pgfqpoint{5.486122in}{0.653507in}}{\pgfqpoint{5.481731in}{0.664106in}}{\pgfqpoint{5.473918in}{0.671919in}}%
\pgfpathcurveto{\pgfqpoint{5.466104in}{0.679733in}}{\pgfqpoint{5.455505in}{0.684123in}}{\pgfqpoint{5.444455in}{0.684123in}}%
\pgfpathcurveto{\pgfqpoint{5.433405in}{0.684123in}}{\pgfqpoint{5.422806in}{0.679733in}}{\pgfqpoint{5.414992in}{0.671919in}}%
\pgfpathcurveto{\pgfqpoint{5.407179in}{0.664106in}}{\pgfqpoint{5.402788in}{0.653507in}}{\pgfqpoint{5.402788in}{0.642457in}}%
\pgfpathcurveto{\pgfqpoint{5.402788in}{0.631406in}}{\pgfqpoint{5.407179in}{0.620807in}}{\pgfqpoint{5.414992in}{0.612994in}}%
\pgfpathcurveto{\pgfqpoint{5.422806in}{0.605180in}}{\pgfqpoint{5.433405in}{0.600790in}}{\pgfqpoint{5.444455in}{0.600790in}}%
\pgfpathlineto{\pgfqpoint{5.444455in}{0.600790in}}%
\pgfpathclose%
\pgfusepath{stroke}%
\end{pgfscope}%
\begin{pgfscope}%
\pgfpathrectangle{\pgfqpoint{0.847223in}{0.554012in}}{\pgfqpoint{6.200000in}{4.530000in}}%
\pgfusepath{clip}%
\pgfsetbuttcap%
\pgfsetroundjoin%
\pgfsetlinewidth{1.003750pt}%
\definecolor{currentstroke}{rgb}{1.000000,0.000000,0.000000}%
\pgfsetstrokecolor{currentstroke}%
\pgfsetdash{}{0pt}%
\pgfpathmoveto{\pgfqpoint{5.449788in}{0.600264in}}%
\pgfpathcurveto{\pgfqpoint{5.460838in}{0.600264in}}{\pgfqpoint{5.471437in}{0.604654in}}{\pgfqpoint{5.479251in}{0.612468in}}%
\pgfpathcurveto{\pgfqpoint{5.487065in}{0.620281in}}{\pgfqpoint{5.491455in}{0.630880in}}{\pgfqpoint{5.491455in}{0.641930in}}%
\pgfpathcurveto{\pgfqpoint{5.491455in}{0.652981in}}{\pgfqpoint{5.487065in}{0.663580in}}{\pgfqpoint{5.479251in}{0.671393in}}%
\pgfpathcurveto{\pgfqpoint{5.471437in}{0.679207in}}{\pgfqpoint{5.460838in}{0.683597in}}{\pgfqpoint{5.449788in}{0.683597in}}%
\pgfpathcurveto{\pgfqpoint{5.438738in}{0.683597in}}{\pgfqpoint{5.428139in}{0.679207in}}{\pgfqpoint{5.420325in}{0.671393in}}%
\pgfpathcurveto{\pgfqpoint{5.412512in}{0.663580in}}{\pgfqpoint{5.408122in}{0.652981in}}{\pgfqpoint{5.408122in}{0.641930in}}%
\pgfpathcurveto{\pgfqpoint{5.408122in}{0.630880in}}{\pgfqpoint{5.412512in}{0.620281in}}{\pgfqpoint{5.420325in}{0.612468in}}%
\pgfpathcurveto{\pgfqpoint{5.428139in}{0.604654in}}{\pgfqpoint{5.438738in}{0.600264in}}{\pgfqpoint{5.449788in}{0.600264in}}%
\pgfpathlineto{\pgfqpoint{5.449788in}{0.600264in}}%
\pgfpathclose%
\pgfusepath{stroke}%
\end{pgfscope}%
\begin{pgfscope}%
\pgfpathrectangle{\pgfqpoint{0.847223in}{0.554012in}}{\pgfqpoint{6.200000in}{4.530000in}}%
\pgfusepath{clip}%
\pgfsetbuttcap%
\pgfsetroundjoin%
\pgfsetlinewidth{1.003750pt}%
\definecolor{currentstroke}{rgb}{1.000000,0.000000,0.000000}%
\pgfsetstrokecolor{currentstroke}%
\pgfsetdash{}{0pt}%
\pgfpathmoveto{\pgfqpoint{5.455121in}{0.599739in}}%
\pgfpathcurveto{\pgfqpoint{5.466172in}{0.599739in}}{\pgfqpoint{5.476771in}{0.604129in}}{\pgfqpoint{5.484584in}{0.611943in}}%
\pgfpathcurveto{\pgfqpoint{5.492398in}{0.619756in}}{\pgfqpoint{5.496788in}{0.630355in}}{\pgfqpoint{5.496788in}{0.641406in}}%
\pgfpathcurveto{\pgfqpoint{5.496788in}{0.652456in}}{\pgfqpoint{5.492398in}{0.663055in}}{\pgfqpoint{5.484584in}{0.670868in}}%
\pgfpathcurveto{\pgfqpoint{5.476771in}{0.678682in}}{\pgfqpoint{5.466172in}{0.683072in}}{\pgfqpoint{5.455121in}{0.683072in}}%
\pgfpathcurveto{\pgfqpoint{5.444071in}{0.683072in}}{\pgfqpoint{5.433472in}{0.678682in}}{\pgfqpoint{5.425659in}{0.670868in}}%
\pgfpathcurveto{\pgfqpoint{5.417845in}{0.663055in}}{\pgfqpoint{5.413455in}{0.652456in}}{\pgfqpoint{5.413455in}{0.641406in}}%
\pgfpathcurveto{\pgfqpoint{5.413455in}{0.630355in}}{\pgfqpoint{5.417845in}{0.619756in}}{\pgfqpoint{5.425659in}{0.611943in}}%
\pgfpathcurveto{\pgfqpoint{5.433472in}{0.604129in}}{\pgfqpoint{5.444071in}{0.599739in}}{\pgfqpoint{5.455121in}{0.599739in}}%
\pgfpathlineto{\pgfqpoint{5.455121in}{0.599739in}}%
\pgfpathclose%
\pgfusepath{stroke}%
\end{pgfscope}%
\begin{pgfscope}%
\pgfpathrectangle{\pgfqpoint{0.847223in}{0.554012in}}{\pgfqpoint{6.200000in}{4.530000in}}%
\pgfusepath{clip}%
\pgfsetbuttcap%
\pgfsetroundjoin%
\pgfsetlinewidth{1.003750pt}%
\definecolor{currentstroke}{rgb}{1.000000,0.000000,0.000000}%
\pgfsetstrokecolor{currentstroke}%
\pgfsetdash{}{0pt}%
\pgfpathmoveto{\pgfqpoint{5.460455in}{0.599215in}}%
\pgfpathcurveto{\pgfqpoint{5.471505in}{0.599215in}}{\pgfqpoint{5.482104in}{0.603605in}}{\pgfqpoint{5.489917in}{0.611419in}}%
\pgfpathcurveto{\pgfqpoint{5.497731in}{0.619232in}}{\pgfqpoint{5.502121in}{0.629831in}}{\pgfqpoint{5.502121in}{0.640882in}}%
\pgfpathcurveto{\pgfqpoint{5.502121in}{0.651932in}}{\pgfqpoint{5.497731in}{0.662531in}}{\pgfqpoint{5.489917in}{0.670344in}}%
\pgfpathcurveto{\pgfqpoint{5.482104in}{0.678158in}}{\pgfqpoint{5.471505in}{0.682548in}}{\pgfqpoint{5.460455in}{0.682548in}}%
\pgfpathcurveto{\pgfqpoint{5.449404in}{0.682548in}}{\pgfqpoint{5.438805in}{0.678158in}}{\pgfqpoint{5.430992in}{0.670344in}}%
\pgfpathcurveto{\pgfqpoint{5.423178in}{0.662531in}}{\pgfqpoint{5.418788in}{0.651932in}}{\pgfqpoint{5.418788in}{0.640882in}}%
\pgfpathcurveto{\pgfqpoint{5.418788in}{0.629831in}}{\pgfqpoint{5.423178in}{0.619232in}}{\pgfqpoint{5.430992in}{0.611419in}}%
\pgfpathcurveto{\pgfqpoint{5.438805in}{0.603605in}}{\pgfqpoint{5.449404in}{0.599215in}}{\pgfqpoint{5.460455in}{0.599215in}}%
\pgfpathlineto{\pgfqpoint{5.460455in}{0.599215in}}%
\pgfpathclose%
\pgfusepath{stroke}%
\end{pgfscope}%
\begin{pgfscope}%
\pgfpathrectangle{\pgfqpoint{0.847223in}{0.554012in}}{\pgfqpoint{6.200000in}{4.530000in}}%
\pgfusepath{clip}%
\pgfsetbuttcap%
\pgfsetroundjoin%
\pgfsetlinewidth{1.003750pt}%
\definecolor{currentstroke}{rgb}{1.000000,0.000000,0.000000}%
\pgfsetstrokecolor{currentstroke}%
\pgfsetdash{}{0pt}%
\pgfpathmoveto{\pgfqpoint{5.465788in}{0.598692in}}%
\pgfpathcurveto{\pgfqpoint{5.476838in}{0.598692in}}{\pgfqpoint{5.487437in}{0.603082in}}{\pgfqpoint{5.495251in}{0.610896in}}%
\pgfpathcurveto{\pgfqpoint{5.503064in}{0.618710in}}{\pgfqpoint{5.507454in}{0.629309in}}{\pgfqpoint{5.507454in}{0.640359in}}%
\pgfpathcurveto{\pgfqpoint{5.507454in}{0.651409in}}{\pgfqpoint{5.503064in}{0.662008in}}{\pgfqpoint{5.495251in}{0.669822in}}%
\pgfpathcurveto{\pgfqpoint{5.487437in}{0.677635in}}{\pgfqpoint{5.476838in}{0.682025in}}{\pgfqpoint{5.465788in}{0.682025in}}%
\pgfpathcurveto{\pgfqpoint{5.454738in}{0.682025in}}{\pgfqpoint{5.444139in}{0.677635in}}{\pgfqpoint{5.436325in}{0.669822in}}%
\pgfpathcurveto{\pgfqpoint{5.428511in}{0.662008in}}{\pgfqpoint{5.424121in}{0.651409in}}{\pgfqpoint{5.424121in}{0.640359in}}%
\pgfpathcurveto{\pgfqpoint{5.424121in}{0.629309in}}{\pgfqpoint{5.428511in}{0.618710in}}{\pgfqpoint{5.436325in}{0.610896in}}%
\pgfpathcurveto{\pgfqpoint{5.444139in}{0.603082in}}{\pgfqpoint{5.454738in}{0.598692in}}{\pgfqpoint{5.465788in}{0.598692in}}%
\pgfpathlineto{\pgfqpoint{5.465788in}{0.598692in}}%
\pgfpathclose%
\pgfusepath{stroke}%
\end{pgfscope}%
\begin{pgfscope}%
\pgfpathrectangle{\pgfqpoint{0.847223in}{0.554012in}}{\pgfqpoint{6.200000in}{4.530000in}}%
\pgfusepath{clip}%
\pgfsetbuttcap%
\pgfsetroundjoin%
\pgfsetlinewidth{1.003750pt}%
\definecolor{currentstroke}{rgb}{1.000000,0.000000,0.000000}%
\pgfsetstrokecolor{currentstroke}%
\pgfsetdash{}{0pt}%
\pgfpathmoveto{\pgfqpoint{5.471121in}{0.598170in}}%
\pgfpathcurveto{\pgfqpoint{5.482171in}{0.598170in}}{\pgfqpoint{5.492770in}{0.602561in}}{\pgfqpoint{5.500584in}{0.610374in}}%
\pgfpathcurveto{\pgfqpoint{5.508397in}{0.618188in}}{\pgfqpoint{5.512788in}{0.628787in}}{\pgfqpoint{5.512788in}{0.639837in}}%
\pgfpathcurveto{\pgfqpoint{5.512788in}{0.650887in}}{\pgfqpoint{5.508397in}{0.661486in}}{\pgfqpoint{5.500584in}{0.669300in}}%
\pgfpathcurveto{\pgfqpoint{5.492770in}{0.677113in}}{\pgfqpoint{5.482171in}{0.681504in}}{\pgfqpoint{5.471121in}{0.681504in}}%
\pgfpathcurveto{\pgfqpoint{5.460071in}{0.681504in}}{\pgfqpoint{5.449472in}{0.677113in}}{\pgfqpoint{5.441658in}{0.669300in}}%
\pgfpathcurveto{\pgfqpoint{5.433845in}{0.661486in}}{\pgfqpoint{5.429454in}{0.650887in}}{\pgfqpoint{5.429454in}{0.639837in}}%
\pgfpathcurveto{\pgfqpoint{5.429454in}{0.628787in}}{\pgfqpoint{5.433845in}{0.618188in}}{\pgfqpoint{5.441658in}{0.610374in}}%
\pgfpathcurveto{\pgfqpoint{5.449472in}{0.602561in}}{\pgfqpoint{5.460071in}{0.598170in}}{\pgfqpoint{5.471121in}{0.598170in}}%
\pgfpathlineto{\pgfqpoint{5.471121in}{0.598170in}}%
\pgfpathclose%
\pgfusepath{stroke}%
\end{pgfscope}%
\begin{pgfscope}%
\pgfpathrectangle{\pgfqpoint{0.847223in}{0.554012in}}{\pgfqpoint{6.200000in}{4.530000in}}%
\pgfusepath{clip}%
\pgfsetbuttcap%
\pgfsetroundjoin%
\pgfsetlinewidth{1.003750pt}%
\definecolor{currentstroke}{rgb}{1.000000,0.000000,0.000000}%
\pgfsetstrokecolor{currentstroke}%
\pgfsetdash{}{0pt}%
\pgfpathmoveto{\pgfqpoint{5.476454in}{0.597650in}}%
\pgfpathcurveto{\pgfqpoint{5.487504in}{0.597650in}}{\pgfqpoint{5.498103in}{0.602040in}}{\pgfqpoint{5.505917in}{0.609854in}}%
\pgfpathcurveto{\pgfqpoint{5.513731in}{0.617667in}}{\pgfqpoint{5.518121in}{0.628266in}}{\pgfqpoint{5.518121in}{0.639316in}}%
\pgfpathcurveto{\pgfqpoint{5.518121in}{0.650367in}}{\pgfqpoint{5.513731in}{0.660966in}}{\pgfqpoint{5.505917in}{0.668779in}}%
\pgfpathcurveto{\pgfqpoint{5.498103in}{0.676593in}}{\pgfqpoint{5.487504in}{0.680983in}}{\pgfqpoint{5.476454in}{0.680983in}}%
\pgfpathcurveto{\pgfqpoint{5.465404in}{0.680983in}}{\pgfqpoint{5.454805in}{0.676593in}}{\pgfqpoint{5.446991in}{0.668779in}}%
\pgfpathcurveto{\pgfqpoint{5.439178in}{0.660966in}}{\pgfqpoint{5.434788in}{0.650367in}}{\pgfqpoint{5.434788in}{0.639316in}}%
\pgfpathcurveto{\pgfqpoint{5.434788in}{0.628266in}}{\pgfqpoint{5.439178in}{0.617667in}}{\pgfqpoint{5.446991in}{0.609854in}}%
\pgfpathcurveto{\pgfqpoint{5.454805in}{0.602040in}}{\pgfqpoint{5.465404in}{0.597650in}}{\pgfqpoint{5.476454in}{0.597650in}}%
\pgfpathlineto{\pgfqpoint{5.476454in}{0.597650in}}%
\pgfpathclose%
\pgfusepath{stroke}%
\end{pgfscope}%
\begin{pgfscope}%
\pgfpathrectangle{\pgfqpoint{0.847223in}{0.554012in}}{\pgfqpoint{6.200000in}{4.530000in}}%
\pgfusepath{clip}%
\pgfsetbuttcap%
\pgfsetroundjoin%
\pgfsetlinewidth{1.003750pt}%
\definecolor{currentstroke}{rgb}{1.000000,0.000000,0.000000}%
\pgfsetstrokecolor{currentstroke}%
\pgfsetdash{}{0pt}%
\pgfpathmoveto{\pgfqpoint{5.481787in}{0.597130in}}%
\pgfpathcurveto{\pgfqpoint{5.492838in}{0.597130in}}{\pgfqpoint{5.503437in}{0.601520in}}{\pgfqpoint{5.511250in}{0.609334in}}%
\pgfpathcurveto{\pgfqpoint{5.519064in}{0.617148in}}{\pgfqpoint{5.523454in}{0.627747in}}{\pgfqpoint{5.523454in}{0.638797in}}%
\pgfpathcurveto{\pgfqpoint{5.523454in}{0.649847in}}{\pgfqpoint{5.519064in}{0.660446in}}{\pgfqpoint{5.511250in}{0.668260in}}%
\pgfpathcurveto{\pgfqpoint{5.503437in}{0.676073in}}{\pgfqpoint{5.492838in}{0.680464in}}{\pgfqpoint{5.481787in}{0.680464in}}%
\pgfpathcurveto{\pgfqpoint{5.470737in}{0.680464in}}{\pgfqpoint{5.460138in}{0.676073in}}{\pgfqpoint{5.452325in}{0.668260in}}%
\pgfpathcurveto{\pgfqpoint{5.444511in}{0.660446in}}{\pgfqpoint{5.440121in}{0.649847in}}{\pgfqpoint{5.440121in}{0.638797in}}%
\pgfpathcurveto{\pgfqpoint{5.440121in}{0.627747in}}{\pgfqpoint{5.444511in}{0.617148in}}{\pgfqpoint{5.452325in}{0.609334in}}%
\pgfpathcurveto{\pgfqpoint{5.460138in}{0.601520in}}{\pgfqpoint{5.470737in}{0.597130in}}{\pgfqpoint{5.481787in}{0.597130in}}%
\pgfpathlineto{\pgfqpoint{5.481787in}{0.597130in}}%
\pgfpathclose%
\pgfusepath{stroke}%
\end{pgfscope}%
\begin{pgfscope}%
\pgfpathrectangle{\pgfqpoint{0.847223in}{0.554012in}}{\pgfqpoint{6.200000in}{4.530000in}}%
\pgfusepath{clip}%
\pgfsetbuttcap%
\pgfsetroundjoin%
\pgfsetlinewidth{1.003750pt}%
\definecolor{currentstroke}{rgb}{1.000000,0.000000,0.000000}%
\pgfsetstrokecolor{currentstroke}%
\pgfsetdash{}{0pt}%
\pgfpathmoveto{\pgfqpoint{5.487121in}{0.596612in}}%
\pgfpathcurveto{\pgfqpoint{5.498171in}{0.596612in}}{\pgfqpoint{5.508770in}{0.601002in}}{\pgfqpoint{5.516583in}{0.608816in}}%
\pgfpathcurveto{\pgfqpoint{5.524397in}{0.616629in}}{\pgfqpoint{5.528787in}{0.627228in}}{\pgfqpoint{5.528787in}{0.638278in}}%
\pgfpathcurveto{\pgfqpoint{5.528787in}{0.649329in}}{\pgfqpoint{5.524397in}{0.659928in}}{\pgfqpoint{5.516583in}{0.667741in}}%
\pgfpathcurveto{\pgfqpoint{5.508770in}{0.675555in}}{\pgfqpoint{5.498171in}{0.679945in}}{\pgfqpoint{5.487121in}{0.679945in}}%
\pgfpathcurveto{\pgfqpoint{5.476071in}{0.679945in}}{\pgfqpoint{5.465472in}{0.675555in}}{\pgfqpoint{5.457658in}{0.667741in}}%
\pgfpathcurveto{\pgfqpoint{5.449844in}{0.659928in}}{\pgfqpoint{5.445454in}{0.649329in}}{\pgfqpoint{5.445454in}{0.638278in}}%
\pgfpathcurveto{\pgfqpoint{5.445454in}{0.627228in}}{\pgfqpoint{5.449844in}{0.616629in}}{\pgfqpoint{5.457658in}{0.608816in}}%
\pgfpathcurveto{\pgfqpoint{5.465472in}{0.601002in}}{\pgfqpoint{5.476071in}{0.596612in}}{\pgfqpoint{5.487121in}{0.596612in}}%
\pgfpathlineto{\pgfqpoint{5.487121in}{0.596612in}}%
\pgfpathclose%
\pgfusepath{stroke}%
\end{pgfscope}%
\begin{pgfscope}%
\pgfpathrectangle{\pgfqpoint{0.847223in}{0.554012in}}{\pgfqpoint{6.200000in}{4.530000in}}%
\pgfusepath{clip}%
\pgfsetbuttcap%
\pgfsetroundjoin%
\pgfsetlinewidth{1.003750pt}%
\definecolor{currentstroke}{rgb}{1.000000,0.000000,0.000000}%
\pgfsetstrokecolor{currentstroke}%
\pgfsetdash{}{0pt}%
\pgfpathmoveto{\pgfqpoint{5.492454in}{0.596094in}}%
\pgfpathcurveto{\pgfqpoint{5.503504in}{0.596094in}}{\pgfqpoint{5.514103in}{0.600485in}}{\pgfqpoint{5.521917in}{0.608298in}}%
\pgfpathcurveto{\pgfqpoint{5.529730in}{0.616112in}}{\pgfqpoint{5.534121in}{0.626711in}}{\pgfqpoint{5.534121in}{0.637761in}}%
\pgfpathcurveto{\pgfqpoint{5.534121in}{0.648811in}}{\pgfqpoint{5.529730in}{0.659410in}}{\pgfqpoint{5.521917in}{0.667224in}}%
\pgfpathcurveto{\pgfqpoint{5.514103in}{0.675037in}}{\pgfqpoint{5.503504in}{0.679428in}}{\pgfqpoint{5.492454in}{0.679428in}}%
\pgfpathcurveto{\pgfqpoint{5.481404in}{0.679428in}}{\pgfqpoint{5.470805in}{0.675037in}}{\pgfqpoint{5.462991in}{0.667224in}}%
\pgfpathcurveto{\pgfqpoint{5.455177in}{0.659410in}}{\pgfqpoint{5.450787in}{0.648811in}}{\pgfqpoint{5.450787in}{0.637761in}}%
\pgfpathcurveto{\pgfqpoint{5.450787in}{0.626711in}}{\pgfqpoint{5.455177in}{0.616112in}}{\pgfqpoint{5.462991in}{0.608298in}}%
\pgfpathcurveto{\pgfqpoint{5.470805in}{0.600485in}}{\pgfqpoint{5.481404in}{0.596094in}}{\pgfqpoint{5.492454in}{0.596094in}}%
\pgfpathlineto{\pgfqpoint{5.492454in}{0.596094in}}%
\pgfpathclose%
\pgfusepath{stroke}%
\end{pgfscope}%
\begin{pgfscope}%
\pgfpathrectangle{\pgfqpoint{0.847223in}{0.554012in}}{\pgfqpoint{6.200000in}{4.530000in}}%
\pgfusepath{clip}%
\pgfsetbuttcap%
\pgfsetroundjoin%
\pgfsetlinewidth{1.003750pt}%
\definecolor{currentstroke}{rgb}{1.000000,0.000000,0.000000}%
\pgfsetstrokecolor{currentstroke}%
\pgfsetdash{}{0pt}%
\pgfpathmoveto{\pgfqpoint{5.497787in}{0.595578in}}%
\pgfpathcurveto{\pgfqpoint{5.508837in}{0.595578in}}{\pgfqpoint{5.519436in}{0.599968in}}{\pgfqpoint{5.527250in}{0.607782in}}%
\pgfpathcurveto{\pgfqpoint{5.535064in}{0.615595in}}{\pgfqpoint{5.539454in}{0.626194in}}{\pgfqpoint{5.539454in}{0.637245in}}%
\pgfpathcurveto{\pgfqpoint{5.539454in}{0.648295in}}{\pgfqpoint{5.535064in}{0.658894in}}{\pgfqpoint{5.527250in}{0.666707in}}%
\pgfpathcurveto{\pgfqpoint{5.519436in}{0.674521in}}{\pgfqpoint{5.508837in}{0.678911in}}{\pgfqpoint{5.497787in}{0.678911in}}%
\pgfpathcurveto{\pgfqpoint{5.486737in}{0.678911in}}{\pgfqpoint{5.476138in}{0.674521in}}{\pgfqpoint{5.468324in}{0.666707in}}%
\pgfpathcurveto{\pgfqpoint{5.460511in}{0.658894in}}{\pgfqpoint{5.456120in}{0.648295in}}{\pgfqpoint{5.456120in}{0.637245in}}%
\pgfpathcurveto{\pgfqpoint{5.456120in}{0.626194in}}{\pgfqpoint{5.460511in}{0.615595in}}{\pgfqpoint{5.468324in}{0.607782in}}%
\pgfpathcurveto{\pgfqpoint{5.476138in}{0.599968in}}{\pgfqpoint{5.486737in}{0.595578in}}{\pgfqpoint{5.497787in}{0.595578in}}%
\pgfpathlineto{\pgfqpoint{5.497787in}{0.595578in}}%
\pgfpathclose%
\pgfusepath{stroke}%
\end{pgfscope}%
\begin{pgfscope}%
\pgfpathrectangle{\pgfqpoint{0.847223in}{0.554012in}}{\pgfqpoint{6.200000in}{4.530000in}}%
\pgfusepath{clip}%
\pgfsetbuttcap%
\pgfsetroundjoin%
\pgfsetlinewidth{1.003750pt}%
\definecolor{currentstroke}{rgb}{1.000000,0.000000,0.000000}%
\pgfsetstrokecolor{currentstroke}%
\pgfsetdash{}{0pt}%
\pgfpathmoveto{\pgfqpoint{5.503120in}{0.595063in}}%
\pgfpathcurveto{\pgfqpoint{5.514170in}{0.595063in}}{\pgfqpoint{5.524769in}{0.599453in}}{\pgfqpoint{5.532583in}{0.607267in}}%
\pgfpathcurveto{\pgfqpoint{5.540397in}{0.615080in}}{\pgfqpoint{5.544787in}{0.625679in}}{\pgfqpoint{5.544787in}{0.636729in}}%
\pgfpathcurveto{\pgfqpoint{5.544787in}{0.647779in}}{\pgfqpoint{5.540397in}{0.658378in}}{\pgfqpoint{5.532583in}{0.666192in}}%
\pgfpathcurveto{\pgfqpoint{5.524769in}{0.674006in}}{\pgfqpoint{5.514170in}{0.678396in}}{\pgfqpoint{5.503120in}{0.678396in}}%
\pgfpathcurveto{\pgfqpoint{5.492070in}{0.678396in}}{\pgfqpoint{5.481471in}{0.674006in}}{\pgfqpoint{5.473658in}{0.666192in}}%
\pgfpathcurveto{\pgfqpoint{5.465844in}{0.658378in}}{\pgfqpoint{5.461454in}{0.647779in}}{\pgfqpoint{5.461454in}{0.636729in}}%
\pgfpathcurveto{\pgfqpoint{5.461454in}{0.625679in}}{\pgfqpoint{5.465844in}{0.615080in}}{\pgfqpoint{5.473658in}{0.607267in}}%
\pgfpathcurveto{\pgfqpoint{5.481471in}{0.599453in}}{\pgfqpoint{5.492070in}{0.595063in}}{\pgfqpoint{5.503120in}{0.595063in}}%
\pgfpathlineto{\pgfqpoint{5.503120in}{0.595063in}}%
\pgfpathclose%
\pgfusepath{stroke}%
\end{pgfscope}%
\begin{pgfscope}%
\pgfpathrectangle{\pgfqpoint{0.847223in}{0.554012in}}{\pgfqpoint{6.200000in}{4.530000in}}%
\pgfusepath{clip}%
\pgfsetbuttcap%
\pgfsetroundjoin%
\pgfsetlinewidth{1.003750pt}%
\definecolor{currentstroke}{rgb}{1.000000,0.000000,0.000000}%
\pgfsetstrokecolor{currentstroke}%
\pgfsetdash{}{0pt}%
\pgfpathmoveto{\pgfqpoint{5.508454in}{0.594548in}}%
\pgfpathcurveto{\pgfqpoint{5.519504in}{0.594548in}}{\pgfqpoint{5.530103in}{0.598939in}}{\pgfqpoint{5.537916in}{0.606752in}}%
\pgfpathcurveto{\pgfqpoint{5.545730in}{0.614566in}}{\pgfqpoint{5.550120in}{0.625165in}}{\pgfqpoint{5.550120in}{0.636215in}}%
\pgfpathcurveto{\pgfqpoint{5.550120in}{0.647265in}}{\pgfqpoint{5.545730in}{0.657864in}}{\pgfqpoint{5.537916in}{0.665678in}}%
\pgfpathcurveto{\pgfqpoint{5.530103in}{0.673491in}}{\pgfqpoint{5.519504in}{0.677882in}}{\pgfqpoint{5.508454in}{0.677882in}}%
\pgfpathcurveto{\pgfqpoint{5.497403in}{0.677882in}}{\pgfqpoint{5.486804in}{0.673491in}}{\pgfqpoint{5.478991in}{0.665678in}}%
\pgfpathcurveto{\pgfqpoint{5.471177in}{0.657864in}}{\pgfqpoint{5.466787in}{0.647265in}}{\pgfqpoint{5.466787in}{0.636215in}}%
\pgfpathcurveto{\pgfqpoint{5.466787in}{0.625165in}}{\pgfqpoint{5.471177in}{0.614566in}}{\pgfqpoint{5.478991in}{0.606752in}}%
\pgfpathcurveto{\pgfqpoint{5.486804in}{0.598939in}}{\pgfqpoint{5.497403in}{0.594548in}}{\pgfqpoint{5.508454in}{0.594548in}}%
\pgfpathlineto{\pgfqpoint{5.508454in}{0.594548in}}%
\pgfpathclose%
\pgfusepath{stroke}%
\end{pgfscope}%
\begin{pgfscope}%
\pgfpathrectangle{\pgfqpoint{0.847223in}{0.554012in}}{\pgfqpoint{6.200000in}{4.530000in}}%
\pgfusepath{clip}%
\pgfsetbuttcap%
\pgfsetroundjoin%
\pgfsetlinewidth{1.003750pt}%
\definecolor{currentstroke}{rgb}{1.000000,0.000000,0.000000}%
\pgfsetstrokecolor{currentstroke}%
\pgfsetdash{}{0pt}%
\pgfpathmoveto{\pgfqpoint{5.513787in}{0.594035in}}%
\pgfpathcurveto{\pgfqpoint{5.524837in}{0.594035in}}{\pgfqpoint{5.535436in}{0.598426in}}{\pgfqpoint{5.543250in}{0.606239in}}%
\pgfpathcurveto{\pgfqpoint{5.551063in}{0.614053in}}{\pgfqpoint{5.555453in}{0.624652in}}{\pgfqpoint{5.555453in}{0.635702in}}%
\pgfpathcurveto{\pgfqpoint{5.555453in}{0.646752in}}{\pgfqpoint{5.551063in}{0.657351in}}{\pgfqpoint{5.543250in}{0.665165in}}%
\pgfpathcurveto{\pgfqpoint{5.535436in}{0.672978in}}{\pgfqpoint{5.524837in}{0.677369in}}{\pgfqpoint{5.513787in}{0.677369in}}%
\pgfpathcurveto{\pgfqpoint{5.502737in}{0.677369in}}{\pgfqpoint{5.492138in}{0.672978in}}{\pgfqpoint{5.484324in}{0.665165in}}%
\pgfpathcurveto{\pgfqpoint{5.476510in}{0.657351in}}{\pgfqpoint{5.472120in}{0.646752in}}{\pgfqpoint{5.472120in}{0.635702in}}%
\pgfpathcurveto{\pgfqpoint{5.472120in}{0.624652in}}{\pgfqpoint{5.476510in}{0.614053in}}{\pgfqpoint{5.484324in}{0.606239in}}%
\pgfpathcurveto{\pgfqpoint{5.492138in}{0.598426in}}{\pgfqpoint{5.502737in}{0.594035in}}{\pgfqpoint{5.513787in}{0.594035in}}%
\pgfpathlineto{\pgfqpoint{5.513787in}{0.594035in}}%
\pgfpathclose%
\pgfusepath{stroke}%
\end{pgfscope}%
\begin{pgfscope}%
\pgfpathrectangle{\pgfqpoint{0.847223in}{0.554012in}}{\pgfqpoint{6.200000in}{4.530000in}}%
\pgfusepath{clip}%
\pgfsetbuttcap%
\pgfsetroundjoin%
\pgfsetlinewidth{1.003750pt}%
\definecolor{currentstroke}{rgb}{1.000000,0.000000,0.000000}%
\pgfsetstrokecolor{currentstroke}%
\pgfsetdash{}{0pt}%
\pgfpathmoveto{\pgfqpoint{5.519120in}{0.593523in}}%
\pgfpathcurveto{\pgfqpoint{5.530170in}{0.593523in}}{\pgfqpoint{5.540769in}{0.597913in}}{\pgfqpoint{5.548583in}{0.605727in}}%
\pgfpathcurveto{\pgfqpoint{5.556396in}{0.613541in}}{\pgfqpoint{5.560787in}{0.624140in}}{\pgfqpoint{5.560787in}{0.635190in}}%
\pgfpathcurveto{\pgfqpoint{5.560787in}{0.646240in}}{\pgfqpoint{5.556396in}{0.656839in}}{\pgfqpoint{5.548583in}{0.664653in}}%
\pgfpathcurveto{\pgfqpoint{5.540769in}{0.672466in}}{\pgfqpoint{5.530170in}{0.676856in}}{\pgfqpoint{5.519120in}{0.676856in}}%
\pgfpathcurveto{\pgfqpoint{5.508070in}{0.676856in}}{\pgfqpoint{5.497471in}{0.672466in}}{\pgfqpoint{5.489657in}{0.664653in}}%
\pgfpathcurveto{\pgfqpoint{5.481844in}{0.656839in}}{\pgfqpoint{5.477453in}{0.646240in}}{\pgfqpoint{5.477453in}{0.635190in}}%
\pgfpathcurveto{\pgfqpoint{5.477453in}{0.624140in}}{\pgfqpoint{5.481844in}{0.613541in}}{\pgfqpoint{5.489657in}{0.605727in}}%
\pgfpathcurveto{\pgfqpoint{5.497471in}{0.597913in}}{\pgfqpoint{5.508070in}{0.593523in}}{\pgfqpoint{5.519120in}{0.593523in}}%
\pgfpathlineto{\pgfqpoint{5.519120in}{0.593523in}}%
\pgfpathclose%
\pgfusepath{stroke}%
\end{pgfscope}%
\begin{pgfscope}%
\pgfpathrectangle{\pgfqpoint{0.847223in}{0.554012in}}{\pgfqpoint{6.200000in}{4.530000in}}%
\pgfusepath{clip}%
\pgfsetbuttcap%
\pgfsetroundjoin%
\pgfsetlinewidth{1.003750pt}%
\definecolor{currentstroke}{rgb}{1.000000,0.000000,0.000000}%
\pgfsetstrokecolor{currentstroke}%
\pgfsetdash{}{0pt}%
\pgfpathmoveto{\pgfqpoint{5.524453in}{0.593012in}}%
\pgfpathcurveto{\pgfqpoint{5.535503in}{0.593012in}}{\pgfqpoint{5.546102in}{0.597402in}}{\pgfqpoint{5.553916in}{0.605216in}}%
\pgfpathcurveto{\pgfqpoint{5.561730in}{0.613030in}}{\pgfqpoint{5.566120in}{0.623629in}}{\pgfqpoint{5.566120in}{0.634679in}}%
\pgfpathcurveto{\pgfqpoint{5.566120in}{0.645729in}}{\pgfqpoint{5.561730in}{0.656328in}}{\pgfqpoint{5.553916in}{0.664142in}}%
\pgfpathcurveto{\pgfqpoint{5.546102in}{0.671955in}}{\pgfqpoint{5.535503in}{0.676345in}}{\pgfqpoint{5.524453in}{0.676345in}}%
\pgfpathcurveto{\pgfqpoint{5.513403in}{0.676345in}}{\pgfqpoint{5.502804in}{0.671955in}}{\pgfqpoint{5.494990in}{0.664142in}}%
\pgfpathcurveto{\pgfqpoint{5.487177in}{0.656328in}}{\pgfqpoint{5.482787in}{0.645729in}}{\pgfqpoint{5.482787in}{0.634679in}}%
\pgfpathcurveto{\pgfqpoint{5.482787in}{0.623629in}}{\pgfqpoint{5.487177in}{0.613030in}}{\pgfqpoint{5.494990in}{0.605216in}}%
\pgfpathcurveto{\pgfqpoint{5.502804in}{0.597402in}}{\pgfqpoint{5.513403in}{0.593012in}}{\pgfqpoint{5.524453in}{0.593012in}}%
\pgfpathlineto{\pgfqpoint{5.524453in}{0.593012in}}%
\pgfpathclose%
\pgfusepath{stroke}%
\end{pgfscope}%
\begin{pgfscope}%
\pgfpathrectangle{\pgfqpoint{0.847223in}{0.554012in}}{\pgfqpoint{6.200000in}{4.530000in}}%
\pgfusepath{clip}%
\pgfsetbuttcap%
\pgfsetroundjoin%
\pgfsetlinewidth{1.003750pt}%
\definecolor{currentstroke}{rgb}{1.000000,0.000000,0.000000}%
\pgfsetstrokecolor{currentstroke}%
\pgfsetdash{}{0pt}%
\pgfpathmoveto{\pgfqpoint{5.529786in}{0.592502in}}%
\pgfpathcurveto{\pgfqpoint{5.540837in}{0.592502in}}{\pgfqpoint{5.551436in}{0.596892in}}{\pgfqpoint{5.559249in}{0.604706in}}%
\pgfpathcurveto{\pgfqpoint{5.567063in}{0.612520in}}{\pgfqpoint{5.571453in}{0.623119in}}{\pgfqpoint{5.571453in}{0.634169in}}%
\pgfpathcurveto{\pgfqpoint{5.571453in}{0.645219in}}{\pgfqpoint{5.567063in}{0.655818in}}{\pgfqpoint{5.559249in}{0.663631in}}%
\pgfpathcurveto{\pgfqpoint{5.551436in}{0.671445in}}{\pgfqpoint{5.540837in}{0.675835in}}{\pgfqpoint{5.529786in}{0.675835in}}%
\pgfpathcurveto{\pgfqpoint{5.518736in}{0.675835in}}{\pgfqpoint{5.508137in}{0.671445in}}{\pgfqpoint{5.500324in}{0.663631in}}%
\pgfpathcurveto{\pgfqpoint{5.492510in}{0.655818in}}{\pgfqpoint{5.488120in}{0.645219in}}{\pgfqpoint{5.488120in}{0.634169in}}%
\pgfpathcurveto{\pgfqpoint{5.488120in}{0.623119in}}{\pgfqpoint{5.492510in}{0.612520in}}{\pgfqpoint{5.500324in}{0.604706in}}%
\pgfpathcurveto{\pgfqpoint{5.508137in}{0.596892in}}{\pgfqpoint{5.518736in}{0.592502in}}{\pgfqpoint{5.529786in}{0.592502in}}%
\pgfpathlineto{\pgfqpoint{5.529786in}{0.592502in}}%
\pgfpathclose%
\pgfusepath{stroke}%
\end{pgfscope}%
\begin{pgfscope}%
\pgfpathrectangle{\pgfqpoint{0.847223in}{0.554012in}}{\pgfqpoint{6.200000in}{4.530000in}}%
\pgfusepath{clip}%
\pgfsetbuttcap%
\pgfsetroundjoin%
\pgfsetlinewidth{1.003750pt}%
\definecolor{currentstroke}{rgb}{1.000000,0.000000,0.000000}%
\pgfsetstrokecolor{currentstroke}%
\pgfsetdash{}{0pt}%
\pgfpathmoveto{\pgfqpoint{5.535120in}{0.591993in}}%
\pgfpathcurveto{\pgfqpoint{5.546170in}{0.591993in}}{\pgfqpoint{5.556769in}{0.596383in}}{\pgfqpoint{5.564582in}{0.604197in}}%
\pgfpathcurveto{\pgfqpoint{5.572396in}{0.612011in}}{\pgfqpoint{5.576786in}{0.622610in}}{\pgfqpoint{5.576786in}{0.633660in}}%
\pgfpathcurveto{\pgfqpoint{5.576786in}{0.644710in}}{\pgfqpoint{5.572396in}{0.655309in}}{\pgfqpoint{5.564582in}{0.663123in}}%
\pgfpathcurveto{\pgfqpoint{5.556769in}{0.670936in}}{\pgfqpoint{5.546170in}{0.675326in}}{\pgfqpoint{5.535120in}{0.675326in}}%
\pgfpathcurveto{\pgfqpoint{5.524069in}{0.675326in}}{\pgfqpoint{5.513470in}{0.670936in}}{\pgfqpoint{5.505657in}{0.663123in}}%
\pgfpathcurveto{\pgfqpoint{5.497843in}{0.655309in}}{\pgfqpoint{5.493453in}{0.644710in}}{\pgfqpoint{5.493453in}{0.633660in}}%
\pgfpathcurveto{\pgfqpoint{5.493453in}{0.622610in}}{\pgfqpoint{5.497843in}{0.612011in}}{\pgfqpoint{5.505657in}{0.604197in}}%
\pgfpathcurveto{\pgfqpoint{5.513470in}{0.596383in}}{\pgfqpoint{5.524069in}{0.591993in}}{\pgfqpoint{5.535120in}{0.591993in}}%
\pgfpathlineto{\pgfqpoint{5.535120in}{0.591993in}}%
\pgfpathclose%
\pgfusepath{stroke}%
\end{pgfscope}%
\begin{pgfscope}%
\pgfpathrectangle{\pgfqpoint{0.847223in}{0.554012in}}{\pgfqpoint{6.200000in}{4.530000in}}%
\pgfusepath{clip}%
\pgfsetbuttcap%
\pgfsetroundjoin%
\pgfsetlinewidth{1.003750pt}%
\definecolor{currentstroke}{rgb}{1.000000,0.000000,0.000000}%
\pgfsetstrokecolor{currentstroke}%
\pgfsetdash{}{0pt}%
\pgfpathmoveto{\pgfqpoint{5.540453in}{0.591485in}}%
\pgfpathcurveto{\pgfqpoint{5.551503in}{0.591485in}}{\pgfqpoint{5.562102in}{0.595875in}}{\pgfqpoint{5.569916in}{0.603689in}}%
\pgfpathcurveto{\pgfqpoint{5.577729in}{0.611503in}}{\pgfqpoint{5.582120in}{0.622102in}}{\pgfqpoint{5.582120in}{0.633152in}}%
\pgfpathcurveto{\pgfqpoint{5.582120in}{0.644202in}}{\pgfqpoint{5.577729in}{0.654801in}}{\pgfqpoint{5.569916in}{0.662615in}}%
\pgfpathcurveto{\pgfqpoint{5.562102in}{0.670428in}}{\pgfqpoint{5.551503in}{0.674818in}}{\pgfqpoint{5.540453in}{0.674818in}}%
\pgfpathcurveto{\pgfqpoint{5.529403in}{0.674818in}}{\pgfqpoint{5.518804in}{0.670428in}}{\pgfqpoint{5.510990in}{0.662615in}}%
\pgfpathcurveto{\pgfqpoint{5.503176in}{0.654801in}}{\pgfqpoint{5.498786in}{0.644202in}}{\pgfqpoint{5.498786in}{0.633152in}}%
\pgfpathcurveto{\pgfqpoint{5.498786in}{0.622102in}}{\pgfqpoint{5.503176in}{0.611503in}}{\pgfqpoint{5.510990in}{0.603689in}}%
\pgfpathcurveto{\pgfqpoint{5.518804in}{0.595875in}}{\pgfqpoint{5.529403in}{0.591485in}}{\pgfqpoint{5.540453in}{0.591485in}}%
\pgfpathlineto{\pgfqpoint{5.540453in}{0.591485in}}%
\pgfpathclose%
\pgfusepath{stroke}%
\end{pgfscope}%
\begin{pgfscope}%
\pgfpathrectangle{\pgfqpoint{0.847223in}{0.554012in}}{\pgfqpoint{6.200000in}{4.530000in}}%
\pgfusepath{clip}%
\pgfsetbuttcap%
\pgfsetroundjoin%
\pgfsetlinewidth{1.003750pt}%
\definecolor{currentstroke}{rgb}{1.000000,0.000000,0.000000}%
\pgfsetstrokecolor{currentstroke}%
\pgfsetdash{}{0pt}%
\pgfpathmoveto{\pgfqpoint{5.545786in}{0.590978in}}%
\pgfpathcurveto{\pgfqpoint{5.556836in}{0.590978in}}{\pgfqpoint{5.567435in}{0.595368in}}{\pgfqpoint{5.575249in}{0.603182in}}%
\pgfpathcurveto{\pgfqpoint{5.583062in}{0.610996in}}{\pgfqpoint{5.587453in}{0.621595in}}{\pgfqpoint{5.587453in}{0.632645in}}%
\pgfpathcurveto{\pgfqpoint{5.587453in}{0.643695in}}{\pgfqpoint{5.583062in}{0.654294in}}{\pgfqpoint{5.575249in}{0.662108in}}%
\pgfpathcurveto{\pgfqpoint{5.567435in}{0.669921in}}{\pgfqpoint{5.556836in}{0.674312in}}{\pgfqpoint{5.545786in}{0.674312in}}%
\pgfpathcurveto{\pgfqpoint{5.534736in}{0.674312in}}{\pgfqpoint{5.524137in}{0.669921in}}{\pgfqpoint{5.516323in}{0.662108in}}%
\pgfpathcurveto{\pgfqpoint{5.508510in}{0.654294in}}{\pgfqpoint{5.504119in}{0.643695in}}{\pgfqpoint{5.504119in}{0.632645in}}%
\pgfpathcurveto{\pgfqpoint{5.504119in}{0.621595in}}{\pgfqpoint{5.508510in}{0.610996in}}{\pgfqpoint{5.516323in}{0.603182in}}%
\pgfpathcurveto{\pgfqpoint{5.524137in}{0.595368in}}{\pgfqpoint{5.534736in}{0.590978in}}{\pgfqpoint{5.545786in}{0.590978in}}%
\pgfpathlineto{\pgfqpoint{5.545786in}{0.590978in}}%
\pgfpathclose%
\pgfusepath{stroke}%
\end{pgfscope}%
\begin{pgfscope}%
\pgfpathrectangle{\pgfqpoint{0.847223in}{0.554012in}}{\pgfqpoint{6.200000in}{4.530000in}}%
\pgfusepath{clip}%
\pgfsetbuttcap%
\pgfsetroundjoin%
\pgfsetlinewidth{1.003750pt}%
\definecolor{currentstroke}{rgb}{1.000000,0.000000,0.000000}%
\pgfsetstrokecolor{currentstroke}%
\pgfsetdash{}{0pt}%
\pgfpathmoveto{\pgfqpoint{5.551119in}{0.590472in}}%
\pgfpathcurveto{\pgfqpoint{5.562169in}{0.590472in}}{\pgfqpoint{5.572768in}{0.594863in}}{\pgfqpoint{5.580582in}{0.602676in}}%
\pgfpathcurveto{\pgfqpoint{5.588396in}{0.610490in}}{\pgfqpoint{5.592786in}{0.621089in}}{\pgfqpoint{5.592786in}{0.632139in}}%
\pgfpathcurveto{\pgfqpoint{5.592786in}{0.643189in}}{\pgfqpoint{5.588396in}{0.653788in}}{\pgfqpoint{5.580582in}{0.661602in}}%
\pgfpathcurveto{\pgfqpoint{5.572768in}{0.669415in}}{\pgfqpoint{5.562169in}{0.673806in}}{\pgfqpoint{5.551119in}{0.673806in}}%
\pgfpathcurveto{\pgfqpoint{5.540069in}{0.673806in}}{\pgfqpoint{5.529470in}{0.669415in}}{\pgfqpoint{5.521656in}{0.661602in}}%
\pgfpathcurveto{\pgfqpoint{5.513843in}{0.653788in}}{\pgfqpoint{5.509453in}{0.643189in}}{\pgfqpoint{5.509453in}{0.632139in}}%
\pgfpathcurveto{\pgfqpoint{5.509453in}{0.621089in}}{\pgfqpoint{5.513843in}{0.610490in}}{\pgfqpoint{5.521656in}{0.602676in}}%
\pgfpathcurveto{\pgfqpoint{5.529470in}{0.594863in}}{\pgfqpoint{5.540069in}{0.590472in}}{\pgfqpoint{5.551119in}{0.590472in}}%
\pgfpathlineto{\pgfqpoint{5.551119in}{0.590472in}}%
\pgfpathclose%
\pgfusepath{stroke}%
\end{pgfscope}%
\begin{pgfscope}%
\pgfpathrectangle{\pgfqpoint{0.847223in}{0.554012in}}{\pgfqpoint{6.200000in}{4.530000in}}%
\pgfusepath{clip}%
\pgfsetbuttcap%
\pgfsetroundjoin%
\pgfsetlinewidth{1.003750pt}%
\definecolor{currentstroke}{rgb}{1.000000,0.000000,0.000000}%
\pgfsetstrokecolor{currentstroke}%
\pgfsetdash{}{0pt}%
\pgfpathmoveto{\pgfqpoint{5.556452in}{0.589968in}}%
\pgfpathcurveto{\pgfqpoint{5.567503in}{0.589968in}}{\pgfqpoint{5.578102in}{0.594358in}}{\pgfqpoint{5.585915in}{0.602171in}}%
\pgfpathcurveto{\pgfqpoint{5.593729in}{0.609985in}}{\pgfqpoint{5.598119in}{0.620584in}}{\pgfqpoint{5.598119in}{0.631634in}}%
\pgfpathcurveto{\pgfqpoint{5.598119in}{0.642684in}}{\pgfqpoint{5.593729in}{0.653283in}}{\pgfqpoint{5.585915in}{0.661097in}}%
\pgfpathcurveto{\pgfqpoint{5.578102in}{0.668911in}}{\pgfqpoint{5.567503in}{0.673301in}}{\pgfqpoint{5.556452in}{0.673301in}}%
\pgfpathcurveto{\pgfqpoint{5.545402in}{0.673301in}}{\pgfqpoint{5.534803in}{0.668911in}}{\pgfqpoint{5.526990in}{0.661097in}}%
\pgfpathcurveto{\pgfqpoint{5.519176in}{0.653283in}}{\pgfqpoint{5.514786in}{0.642684in}}{\pgfqpoint{5.514786in}{0.631634in}}%
\pgfpathcurveto{\pgfqpoint{5.514786in}{0.620584in}}{\pgfqpoint{5.519176in}{0.609985in}}{\pgfqpoint{5.526990in}{0.602171in}}%
\pgfpathcurveto{\pgfqpoint{5.534803in}{0.594358in}}{\pgfqpoint{5.545402in}{0.589968in}}{\pgfqpoint{5.556452in}{0.589968in}}%
\pgfpathlineto{\pgfqpoint{5.556452in}{0.589968in}}%
\pgfpathclose%
\pgfusepath{stroke}%
\end{pgfscope}%
\begin{pgfscope}%
\pgfpathrectangle{\pgfqpoint{0.847223in}{0.554012in}}{\pgfqpoint{6.200000in}{4.530000in}}%
\pgfusepath{clip}%
\pgfsetbuttcap%
\pgfsetroundjoin%
\pgfsetlinewidth{1.003750pt}%
\definecolor{currentstroke}{rgb}{1.000000,0.000000,0.000000}%
\pgfsetstrokecolor{currentstroke}%
\pgfsetdash{}{0pt}%
\pgfpathmoveto{\pgfqpoint{5.561786in}{0.589464in}}%
\pgfpathcurveto{\pgfqpoint{5.572836in}{0.589464in}}{\pgfqpoint{5.583435in}{0.593854in}}{\pgfqpoint{5.591248in}{0.601668in}}%
\pgfpathcurveto{\pgfqpoint{5.599062in}{0.609481in}}{\pgfqpoint{5.603452in}{0.620080in}}{\pgfqpoint{5.603452in}{0.631130in}}%
\pgfpathcurveto{\pgfqpoint{5.603452in}{0.642180in}}{\pgfqpoint{5.599062in}{0.652780in}}{\pgfqpoint{5.591248in}{0.660593in}}%
\pgfpathcurveto{\pgfqpoint{5.583435in}{0.668407in}}{\pgfqpoint{5.572836in}{0.672797in}}{\pgfqpoint{5.561786in}{0.672797in}}%
\pgfpathcurveto{\pgfqpoint{5.550736in}{0.672797in}}{\pgfqpoint{5.540137in}{0.668407in}}{\pgfqpoint{5.532323in}{0.660593in}}%
\pgfpathcurveto{\pgfqpoint{5.524509in}{0.652780in}}{\pgfqpoint{5.520119in}{0.642180in}}{\pgfqpoint{5.520119in}{0.631130in}}%
\pgfpathcurveto{\pgfqpoint{5.520119in}{0.620080in}}{\pgfqpoint{5.524509in}{0.609481in}}{\pgfqpoint{5.532323in}{0.601668in}}%
\pgfpathcurveto{\pgfqpoint{5.540137in}{0.593854in}}{\pgfqpoint{5.550736in}{0.589464in}}{\pgfqpoint{5.561786in}{0.589464in}}%
\pgfpathlineto{\pgfqpoint{5.561786in}{0.589464in}}%
\pgfpathclose%
\pgfusepath{stroke}%
\end{pgfscope}%
\begin{pgfscope}%
\pgfpathrectangle{\pgfqpoint{0.847223in}{0.554012in}}{\pgfqpoint{6.200000in}{4.530000in}}%
\pgfusepath{clip}%
\pgfsetbuttcap%
\pgfsetroundjoin%
\pgfsetlinewidth{1.003750pt}%
\definecolor{currentstroke}{rgb}{1.000000,0.000000,0.000000}%
\pgfsetstrokecolor{currentstroke}%
\pgfsetdash{}{0pt}%
\pgfpathmoveto{\pgfqpoint{5.567119in}{0.588961in}}%
\pgfpathcurveto{\pgfqpoint{5.578169in}{0.588961in}}{\pgfqpoint{5.588768in}{0.593351in}}{\pgfqpoint{5.596582in}{0.601165in}}%
\pgfpathcurveto{\pgfqpoint{5.604395in}{0.608978in}}{\pgfqpoint{5.608786in}{0.619577in}}{\pgfqpoint{5.608786in}{0.630628in}}%
\pgfpathcurveto{\pgfqpoint{5.608786in}{0.641678in}}{\pgfqpoint{5.604395in}{0.652277in}}{\pgfqpoint{5.596582in}{0.660090in}}%
\pgfpathcurveto{\pgfqpoint{5.588768in}{0.667904in}}{\pgfqpoint{5.578169in}{0.672294in}}{\pgfqpoint{5.567119in}{0.672294in}}%
\pgfpathcurveto{\pgfqpoint{5.556069in}{0.672294in}}{\pgfqpoint{5.545470in}{0.667904in}}{\pgfqpoint{5.537656in}{0.660090in}}%
\pgfpathcurveto{\pgfqpoint{5.529843in}{0.652277in}}{\pgfqpoint{5.525452in}{0.641678in}}{\pgfqpoint{5.525452in}{0.630628in}}%
\pgfpathcurveto{\pgfqpoint{5.525452in}{0.619577in}}{\pgfqpoint{5.529843in}{0.608978in}}{\pgfqpoint{5.537656in}{0.601165in}}%
\pgfpathcurveto{\pgfqpoint{5.545470in}{0.593351in}}{\pgfqpoint{5.556069in}{0.588961in}}{\pgfqpoint{5.567119in}{0.588961in}}%
\pgfpathlineto{\pgfqpoint{5.567119in}{0.588961in}}%
\pgfpathclose%
\pgfusepath{stroke}%
\end{pgfscope}%
\begin{pgfscope}%
\pgfpathrectangle{\pgfqpoint{0.847223in}{0.554012in}}{\pgfqpoint{6.200000in}{4.530000in}}%
\pgfusepath{clip}%
\pgfsetbuttcap%
\pgfsetroundjoin%
\pgfsetlinewidth{1.003750pt}%
\definecolor{currentstroke}{rgb}{1.000000,0.000000,0.000000}%
\pgfsetstrokecolor{currentstroke}%
\pgfsetdash{}{0pt}%
\pgfpathmoveto{\pgfqpoint{5.572452in}{0.588459in}}%
\pgfpathcurveto{\pgfqpoint{5.583502in}{0.588459in}}{\pgfqpoint{5.594101in}{0.592849in}}{\pgfqpoint{5.601915in}{0.600663in}}%
\pgfpathcurveto{\pgfqpoint{5.609729in}{0.608477in}}{\pgfqpoint{5.614119in}{0.619076in}}{\pgfqpoint{5.614119in}{0.630126in}}%
\pgfpathcurveto{\pgfqpoint{5.614119in}{0.641176in}}{\pgfqpoint{5.609729in}{0.651775in}}{\pgfqpoint{5.601915in}{0.659589in}}%
\pgfpathcurveto{\pgfqpoint{5.594101in}{0.667402in}}{\pgfqpoint{5.583502in}{0.671792in}}{\pgfqpoint{5.572452in}{0.671792in}}%
\pgfpathcurveto{\pgfqpoint{5.561402in}{0.671792in}}{\pgfqpoint{5.550803in}{0.667402in}}{\pgfqpoint{5.542989in}{0.659589in}}%
\pgfpathcurveto{\pgfqpoint{5.535176in}{0.651775in}}{\pgfqpoint{5.530785in}{0.641176in}}{\pgfqpoint{5.530785in}{0.630126in}}%
\pgfpathcurveto{\pgfqpoint{5.530785in}{0.619076in}}{\pgfqpoint{5.535176in}{0.608477in}}{\pgfqpoint{5.542989in}{0.600663in}}%
\pgfpathcurveto{\pgfqpoint{5.550803in}{0.592849in}}{\pgfqpoint{5.561402in}{0.588459in}}{\pgfqpoint{5.572452in}{0.588459in}}%
\pgfpathlineto{\pgfqpoint{5.572452in}{0.588459in}}%
\pgfpathclose%
\pgfusepath{stroke}%
\end{pgfscope}%
\begin{pgfscope}%
\pgfpathrectangle{\pgfqpoint{0.847223in}{0.554012in}}{\pgfqpoint{6.200000in}{4.530000in}}%
\pgfusepath{clip}%
\pgfsetbuttcap%
\pgfsetroundjoin%
\pgfsetlinewidth{1.003750pt}%
\definecolor{currentstroke}{rgb}{1.000000,0.000000,0.000000}%
\pgfsetstrokecolor{currentstroke}%
\pgfsetdash{}{0pt}%
\pgfpathmoveto{\pgfqpoint{5.577785in}{0.587958in}}%
\pgfpathcurveto{\pgfqpoint{5.588835in}{0.587958in}}{\pgfqpoint{5.599435in}{0.592349in}}{\pgfqpoint{5.607248in}{0.600162in}}%
\pgfpathcurveto{\pgfqpoint{5.615062in}{0.607976in}}{\pgfqpoint{5.619452in}{0.618575in}}{\pgfqpoint{5.619452in}{0.629625in}}%
\pgfpathcurveto{\pgfqpoint{5.619452in}{0.640675in}}{\pgfqpoint{5.615062in}{0.651274in}}{\pgfqpoint{5.607248in}{0.659088in}}%
\pgfpathcurveto{\pgfqpoint{5.599435in}{0.666901in}}{\pgfqpoint{5.588835in}{0.671292in}}{\pgfqpoint{5.577785in}{0.671292in}}%
\pgfpathcurveto{\pgfqpoint{5.566735in}{0.671292in}}{\pgfqpoint{5.556136in}{0.666901in}}{\pgfqpoint{5.548323in}{0.659088in}}%
\pgfpathcurveto{\pgfqpoint{5.540509in}{0.651274in}}{\pgfqpoint{5.536119in}{0.640675in}}{\pgfqpoint{5.536119in}{0.629625in}}%
\pgfpathcurveto{\pgfqpoint{5.536119in}{0.618575in}}{\pgfqpoint{5.540509in}{0.607976in}}{\pgfqpoint{5.548323in}{0.600162in}}%
\pgfpathcurveto{\pgfqpoint{5.556136in}{0.592349in}}{\pgfqpoint{5.566735in}{0.587958in}}{\pgfqpoint{5.577785in}{0.587958in}}%
\pgfpathlineto{\pgfqpoint{5.577785in}{0.587958in}}%
\pgfpathclose%
\pgfusepath{stroke}%
\end{pgfscope}%
\begin{pgfscope}%
\pgfpathrectangle{\pgfqpoint{0.847223in}{0.554012in}}{\pgfqpoint{6.200000in}{4.530000in}}%
\pgfusepath{clip}%
\pgfsetbuttcap%
\pgfsetroundjoin%
\pgfsetlinewidth{1.003750pt}%
\definecolor{currentstroke}{rgb}{1.000000,0.000000,0.000000}%
\pgfsetstrokecolor{currentstroke}%
\pgfsetdash{}{0pt}%
\pgfpathmoveto{\pgfqpoint{5.583119in}{0.587459in}}%
\pgfpathcurveto{\pgfqpoint{5.594169in}{0.587459in}}{\pgfqpoint{5.604768in}{0.591849in}}{\pgfqpoint{5.612581in}{0.599663in}}%
\pgfpathcurveto{\pgfqpoint{5.620395in}{0.607476in}}{\pgfqpoint{5.624785in}{0.618075in}}{\pgfqpoint{5.624785in}{0.629125in}}%
\pgfpathcurveto{\pgfqpoint{5.624785in}{0.640175in}}{\pgfqpoint{5.620395in}{0.650775in}}{\pgfqpoint{5.612581in}{0.658588in}}%
\pgfpathcurveto{\pgfqpoint{5.604768in}{0.666402in}}{\pgfqpoint{5.594169in}{0.670792in}}{\pgfqpoint{5.583119in}{0.670792in}}%
\pgfpathcurveto{\pgfqpoint{5.572068in}{0.670792in}}{\pgfqpoint{5.561469in}{0.666402in}}{\pgfqpoint{5.553656in}{0.658588in}}%
\pgfpathcurveto{\pgfqpoint{5.545842in}{0.650775in}}{\pgfqpoint{5.541452in}{0.640175in}}{\pgfqpoint{5.541452in}{0.629125in}}%
\pgfpathcurveto{\pgfqpoint{5.541452in}{0.618075in}}{\pgfqpoint{5.545842in}{0.607476in}}{\pgfqpoint{5.553656in}{0.599663in}}%
\pgfpathcurveto{\pgfqpoint{5.561469in}{0.591849in}}{\pgfqpoint{5.572068in}{0.587459in}}{\pgfqpoint{5.583119in}{0.587459in}}%
\pgfpathlineto{\pgfqpoint{5.583119in}{0.587459in}}%
\pgfpathclose%
\pgfusepath{stroke}%
\end{pgfscope}%
\begin{pgfscope}%
\pgfpathrectangle{\pgfqpoint{0.847223in}{0.554012in}}{\pgfqpoint{6.200000in}{4.530000in}}%
\pgfusepath{clip}%
\pgfsetbuttcap%
\pgfsetroundjoin%
\pgfsetlinewidth{1.003750pt}%
\definecolor{currentstroke}{rgb}{1.000000,0.000000,0.000000}%
\pgfsetstrokecolor{currentstroke}%
\pgfsetdash{}{0pt}%
\pgfpathmoveto{\pgfqpoint{5.588452in}{0.586960in}}%
\pgfpathcurveto{\pgfqpoint{5.599502in}{0.586960in}}{\pgfqpoint{5.610101in}{0.591350in}}{\pgfqpoint{5.617915in}{0.599164in}}%
\pgfpathcurveto{\pgfqpoint{5.625728in}{0.606977in}}{\pgfqpoint{5.630118in}{0.617577in}}{\pgfqpoint{5.630118in}{0.628627in}}%
\pgfpathcurveto{\pgfqpoint{5.630118in}{0.639677in}}{\pgfqpoint{5.625728in}{0.650276in}}{\pgfqpoint{5.617915in}{0.658089in}}%
\pgfpathcurveto{\pgfqpoint{5.610101in}{0.665903in}}{\pgfqpoint{5.599502in}{0.670293in}}{\pgfqpoint{5.588452in}{0.670293in}}%
\pgfpathcurveto{\pgfqpoint{5.577402in}{0.670293in}}{\pgfqpoint{5.566803in}{0.665903in}}{\pgfqpoint{5.558989in}{0.658089in}}%
\pgfpathcurveto{\pgfqpoint{5.551175in}{0.650276in}}{\pgfqpoint{5.546785in}{0.639677in}}{\pgfqpoint{5.546785in}{0.628627in}}%
\pgfpathcurveto{\pgfqpoint{5.546785in}{0.617577in}}{\pgfqpoint{5.551175in}{0.606977in}}{\pgfqpoint{5.558989in}{0.599164in}}%
\pgfpathcurveto{\pgfqpoint{5.566803in}{0.591350in}}{\pgfqpoint{5.577402in}{0.586960in}}{\pgfqpoint{5.588452in}{0.586960in}}%
\pgfpathlineto{\pgfqpoint{5.588452in}{0.586960in}}%
\pgfpathclose%
\pgfusepath{stroke}%
\end{pgfscope}%
\begin{pgfscope}%
\pgfpathrectangle{\pgfqpoint{0.847223in}{0.554012in}}{\pgfqpoint{6.200000in}{4.530000in}}%
\pgfusepath{clip}%
\pgfsetbuttcap%
\pgfsetroundjoin%
\pgfsetlinewidth{1.003750pt}%
\definecolor{currentstroke}{rgb}{1.000000,0.000000,0.000000}%
\pgfsetstrokecolor{currentstroke}%
\pgfsetdash{}{0pt}%
\pgfpathmoveto{\pgfqpoint{5.593785in}{0.586462in}}%
\pgfpathcurveto{\pgfqpoint{5.604835in}{0.586462in}}{\pgfqpoint{5.615434in}{0.590853in}}{\pgfqpoint{5.623248in}{0.598666in}}%
\pgfpathcurveto{\pgfqpoint{5.631061in}{0.606480in}}{\pgfqpoint{5.635452in}{0.617079in}}{\pgfqpoint{5.635452in}{0.628129in}}%
\pgfpathcurveto{\pgfqpoint{5.635452in}{0.639179in}}{\pgfqpoint{5.631061in}{0.649778in}}{\pgfqpoint{5.623248in}{0.657592in}}%
\pgfpathcurveto{\pgfqpoint{5.615434in}{0.665405in}}{\pgfqpoint{5.604835in}{0.669796in}}{\pgfqpoint{5.593785in}{0.669796in}}%
\pgfpathcurveto{\pgfqpoint{5.582735in}{0.669796in}}{\pgfqpoint{5.572136in}{0.665405in}}{\pgfqpoint{5.564322in}{0.657592in}}%
\pgfpathcurveto{\pgfqpoint{5.556509in}{0.649778in}}{\pgfqpoint{5.552118in}{0.639179in}}{\pgfqpoint{5.552118in}{0.628129in}}%
\pgfpathcurveto{\pgfqpoint{5.552118in}{0.617079in}}{\pgfqpoint{5.556509in}{0.606480in}}{\pgfqpoint{5.564322in}{0.598666in}}%
\pgfpathcurveto{\pgfqpoint{5.572136in}{0.590853in}}{\pgfqpoint{5.582735in}{0.586462in}}{\pgfqpoint{5.593785in}{0.586462in}}%
\pgfpathlineto{\pgfqpoint{5.593785in}{0.586462in}}%
\pgfpathclose%
\pgfusepath{stroke}%
\end{pgfscope}%
\begin{pgfscope}%
\pgfpathrectangle{\pgfqpoint{0.847223in}{0.554012in}}{\pgfqpoint{6.200000in}{4.530000in}}%
\pgfusepath{clip}%
\pgfsetbuttcap%
\pgfsetroundjoin%
\pgfsetlinewidth{1.003750pt}%
\definecolor{currentstroke}{rgb}{1.000000,0.000000,0.000000}%
\pgfsetstrokecolor{currentstroke}%
\pgfsetdash{}{0pt}%
\pgfpathmoveto{\pgfqpoint{5.599118in}{0.585966in}}%
\pgfpathcurveto{\pgfqpoint{5.610168in}{0.585966in}}{\pgfqpoint{5.620767in}{0.590356in}}{\pgfqpoint{5.628581in}{0.598169in}}%
\pgfpathcurveto{\pgfqpoint{5.636395in}{0.605983in}}{\pgfqpoint{5.640785in}{0.616582in}}{\pgfqpoint{5.640785in}{0.627632in}}%
\pgfpathcurveto{\pgfqpoint{5.640785in}{0.638682in}}{\pgfqpoint{5.636395in}{0.649281in}}{\pgfqpoint{5.628581in}{0.657095in}}%
\pgfpathcurveto{\pgfqpoint{5.620767in}{0.664909in}}{\pgfqpoint{5.610168in}{0.669299in}}{\pgfqpoint{5.599118in}{0.669299in}}%
\pgfpathcurveto{\pgfqpoint{5.588068in}{0.669299in}}{\pgfqpoint{5.577469in}{0.664909in}}{\pgfqpoint{5.569655in}{0.657095in}}%
\pgfpathcurveto{\pgfqpoint{5.561842in}{0.649281in}}{\pgfqpoint{5.557452in}{0.638682in}}{\pgfqpoint{5.557452in}{0.627632in}}%
\pgfpathcurveto{\pgfqpoint{5.557452in}{0.616582in}}{\pgfqpoint{5.561842in}{0.605983in}}{\pgfqpoint{5.569655in}{0.598169in}}%
\pgfpathcurveto{\pgfqpoint{5.577469in}{0.590356in}}{\pgfqpoint{5.588068in}{0.585966in}}{\pgfqpoint{5.599118in}{0.585966in}}%
\pgfpathlineto{\pgfqpoint{5.599118in}{0.585966in}}%
\pgfpathclose%
\pgfusepath{stroke}%
\end{pgfscope}%
\begin{pgfscope}%
\pgfpathrectangle{\pgfqpoint{0.847223in}{0.554012in}}{\pgfqpoint{6.200000in}{4.530000in}}%
\pgfusepath{clip}%
\pgfsetbuttcap%
\pgfsetroundjoin%
\pgfsetlinewidth{1.003750pt}%
\definecolor{currentstroke}{rgb}{1.000000,0.000000,0.000000}%
\pgfsetstrokecolor{currentstroke}%
\pgfsetdash{}{0pt}%
\pgfpathmoveto{\pgfqpoint{5.604451in}{0.585470in}}%
\pgfpathcurveto{\pgfqpoint{5.615502in}{0.585470in}}{\pgfqpoint{5.626101in}{0.589860in}}{\pgfqpoint{5.633914in}{0.597674in}}%
\pgfpathcurveto{\pgfqpoint{5.641728in}{0.605487in}}{\pgfqpoint{5.646118in}{0.616086in}}{\pgfqpoint{5.646118in}{0.627137in}}%
\pgfpathcurveto{\pgfqpoint{5.646118in}{0.638187in}}{\pgfqpoint{5.641728in}{0.648786in}}{\pgfqpoint{5.633914in}{0.656599in}}%
\pgfpathcurveto{\pgfqpoint{5.626101in}{0.664413in}}{\pgfqpoint{5.615502in}{0.668803in}}{\pgfqpoint{5.604451in}{0.668803in}}%
\pgfpathcurveto{\pgfqpoint{5.593401in}{0.668803in}}{\pgfqpoint{5.582802in}{0.664413in}}{\pgfqpoint{5.574989in}{0.656599in}}%
\pgfpathcurveto{\pgfqpoint{5.567175in}{0.648786in}}{\pgfqpoint{5.562785in}{0.638187in}}{\pgfqpoint{5.562785in}{0.627137in}}%
\pgfpathcurveto{\pgfqpoint{5.562785in}{0.616086in}}{\pgfqpoint{5.567175in}{0.605487in}}{\pgfqpoint{5.574989in}{0.597674in}}%
\pgfpathcurveto{\pgfqpoint{5.582802in}{0.589860in}}{\pgfqpoint{5.593401in}{0.585470in}}{\pgfqpoint{5.604451in}{0.585470in}}%
\pgfpathlineto{\pgfqpoint{5.604451in}{0.585470in}}%
\pgfpathclose%
\pgfusepath{stroke}%
\end{pgfscope}%
\begin{pgfscope}%
\pgfpathrectangle{\pgfqpoint{0.847223in}{0.554012in}}{\pgfqpoint{6.200000in}{4.530000in}}%
\pgfusepath{clip}%
\pgfsetbuttcap%
\pgfsetroundjoin%
\pgfsetlinewidth{1.003750pt}%
\definecolor{currentstroke}{rgb}{1.000000,0.000000,0.000000}%
\pgfsetstrokecolor{currentstroke}%
\pgfsetdash{}{0pt}%
\pgfpathmoveto{\pgfqpoint{5.609785in}{0.584975in}}%
\pgfpathcurveto{\pgfqpoint{5.620835in}{0.584975in}}{\pgfqpoint{5.631434in}{0.589365in}}{\pgfqpoint{5.639247in}{0.597179in}}%
\pgfpathcurveto{\pgfqpoint{5.647061in}{0.604993in}}{\pgfqpoint{5.651451in}{0.615592in}}{\pgfqpoint{5.651451in}{0.626642in}}%
\pgfpathcurveto{\pgfqpoint{5.651451in}{0.637692in}}{\pgfqpoint{5.647061in}{0.648291in}}{\pgfqpoint{5.639247in}{0.656105in}}%
\pgfpathcurveto{\pgfqpoint{5.631434in}{0.663918in}}{\pgfqpoint{5.620835in}{0.668309in}}{\pgfqpoint{5.609785in}{0.668309in}}%
\pgfpathcurveto{\pgfqpoint{5.598735in}{0.668309in}}{\pgfqpoint{5.588135in}{0.663918in}}{\pgfqpoint{5.580322in}{0.656105in}}%
\pgfpathcurveto{\pgfqpoint{5.572508in}{0.648291in}}{\pgfqpoint{5.568118in}{0.637692in}}{\pgfqpoint{5.568118in}{0.626642in}}%
\pgfpathcurveto{\pgfqpoint{5.568118in}{0.615592in}}{\pgfqpoint{5.572508in}{0.604993in}}{\pgfqpoint{5.580322in}{0.597179in}}%
\pgfpathcurveto{\pgfqpoint{5.588135in}{0.589365in}}{\pgfqpoint{5.598735in}{0.584975in}}{\pgfqpoint{5.609785in}{0.584975in}}%
\pgfpathlineto{\pgfqpoint{5.609785in}{0.584975in}}%
\pgfpathclose%
\pgfusepath{stroke}%
\end{pgfscope}%
\begin{pgfscope}%
\pgfpathrectangle{\pgfqpoint{0.847223in}{0.554012in}}{\pgfqpoint{6.200000in}{4.530000in}}%
\pgfusepath{clip}%
\pgfsetbuttcap%
\pgfsetroundjoin%
\pgfsetlinewidth{1.003750pt}%
\definecolor{currentstroke}{rgb}{1.000000,0.000000,0.000000}%
\pgfsetstrokecolor{currentstroke}%
\pgfsetdash{}{0pt}%
\pgfpathmoveto{\pgfqpoint{5.615118in}{0.584481in}}%
\pgfpathcurveto{\pgfqpoint{5.626168in}{0.584481in}}{\pgfqpoint{5.636767in}{0.588872in}}{\pgfqpoint{5.644581in}{0.596685in}}%
\pgfpathcurveto{\pgfqpoint{5.652394in}{0.604499in}}{\pgfqpoint{5.656785in}{0.615098in}}{\pgfqpoint{5.656785in}{0.626148in}}%
\pgfpathcurveto{\pgfqpoint{5.656785in}{0.637198in}}{\pgfqpoint{5.652394in}{0.647797in}}{\pgfqpoint{5.644581in}{0.655611in}}%
\pgfpathcurveto{\pgfqpoint{5.636767in}{0.663425in}}{\pgfqpoint{5.626168in}{0.667815in}}{\pgfqpoint{5.615118in}{0.667815in}}%
\pgfpathcurveto{\pgfqpoint{5.604068in}{0.667815in}}{\pgfqpoint{5.593469in}{0.663425in}}{\pgfqpoint{5.585655in}{0.655611in}}%
\pgfpathcurveto{\pgfqpoint{5.577841in}{0.647797in}}{\pgfqpoint{5.573451in}{0.637198in}}{\pgfqpoint{5.573451in}{0.626148in}}%
\pgfpathcurveto{\pgfqpoint{5.573451in}{0.615098in}}{\pgfqpoint{5.577841in}{0.604499in}}{\pgfqpoint{5.585655in}{0.596685in}}%
\pgfpathcurveto{\pgfqpoint{5.593469in}{0.588872in}}{\pgfqpoint{5.604068in}{0.584481in}}{\pgfqpoint{5.615118in}{0.584481in}}%
\pgfpathlineto{\pgfqpoint{5.615118in}{0.584481in}}%
\pgfpathclose%
\pgfusepath{stroke}%
\end{pgfscope}%
\begin{pgfscope}%
\pgfpathrectangle{\pgfqpoint{0.847223in}{0.554012in}}{\pgfqpoint{6.200000in}{4.530000in}}%
\pgfusepath{clip}%
\pgfsetbuttcap%
\pgfsetroundjoin%
\pgfsetlinewidth{1.003750pt}%
\definecolor{currentstroke}{rgb}{1.000000,0.000000,0.000000}%
\pgfsetstrokecolor{currentstroke}%
\pgfsetdash{}{0pt}%
\pgfpathmoveto{\pgfqpoint{5.620451in}{0.583989in}}%
\pgfpathcurveto{\pgfqpoint{5.631501in}{0.583989in}}{\pgfqpoint{5.642100in}{0.588379in}}{\pgfqpoint{5.649914in}{0.596193in}}%
\pgfpathcurveto{\pgfqpoint{5.657727in}{0.604006in}}{\pgfqpoint{5.662118in}{0.614605in}}{\pgfqpoint{5.662118in}{0.625655in}}%
\pgfpathcurveto{\pgfqpoint{5.662118in}{0.636706in}}{\pgfqpoint{5.657727in}{0.647305in}}{\pgfqpoint{5.649914in}{0.655118in}}%
\pgfpathcurveto{\pgfqpoint{5.642100in}{0.662932in}}{\pgfqpoint{5.631501in}{0.667322in}}{\pgfqpoint{5.620451in}{0.667322in}}%
\pgfpathcurveto{\pgfqpoint{5.609401in}{0.667322in}}{\pgfqpoint{5.598802in}{0.662932in}}{\pgfqpoint{5.590988in}{0.655118in}}%
\pgfpathcurveto{\pgfqpoint{5.583175in}{0.647305in}}{\pgfqpoint{5.578784in}{0.636706in}}{\pgfqpoint{5.578784in}{0.625655in}}%
\pgfpathcurveto{\pgfqpoint{5.578784in}{0.614605in}}{\pgfqpoint{5.583175in}{0.604006in}}{\pgfqpoint{5.590988in}{0.596193in}}%
\pgfpathcurveto{\pgfqpoint{5.598802in}{0.588379in}}{\pgfqpoint{5.609401in}{0.583989in}}{\pgfqpoint{5.620451in}{0.583989in}}%
\pgfpathlineto{\pgfqpoint{5.620451in}{0.583989in}}%
\pgfpathclose%
\pgfusepath{stroke}%
\end{pgfscope}%
\begin{pgfscope}%
\pgfpathrectangle{\pgfqpoint{0.847223in}{0.554012in}}{\pgfqpoint{6.200000in}{4.530000in}}%
\pgfusepath{clip}%
\pgfsetbuttcap%
\pgfsetroundjoin%
\pgfsetlinewidth{1.003750pt}%
\definecolor{currentstroke}{rgb}{1.000000,0.000000,0.000000}%
\pgfsetstrokecolor{currentstroke}%
\pgfsetdash{}{0pt}%
\pgfpathmoveto{\pgfqpoint{5.625784in}{0.583497in}}%
\pgfpathcurveto{\pgfqpoint{5.636834in}{0.583497in}}{\pgfqpoint{5.647433in}{0.587887in}}{\pgfqpoint{5.655247in}{0.595701in}}%
\pgfpathcurveto{\pgfqpoint{5.663061in}{0.603515in}}{\pgfqpoint{5.667451in}{0.614114in}}{\pgfqpoint{5.667451in}{0.625164in}}%
\pgfpathcurveto{\pgfqpoint{5.667451in}{0.636214in}}{\pgfqpoint{5.663061in}{0.646813in}}{\pgfqpoint{5.655247in}{0.654627in}}%
\pgfpathcurveto{\pgfqpoint{5.647433in}{0.662440in}}{\pgfqpoint{5.636834in}{0.666830in}}{\pgfqpoint{5.625784in}{0.666830in}}%
\pgfpathcurveto{\pgfqpoint{5.614734in}{0.666830in}}{\pgfqpoint{5.604135in}{0.662440in}}{\pgfqpoint{5.596322in}{0.654627in}}%
\pgfpathcurveto{\pgfqpoint{5.588508in}{0.646813in}}{\pgfqpoint{5.584118in}{0.636214in}}{\pgfqpoint{5.584118in}{0.625164in}}%
\pgfpathcurveto{\pgfqpoint{5.584118in}{0.614114in}}{\pgfqpoint{5.588508in}{0.603515in}}{\pgfqpoint{5.596322in}{0.595701in}}%
\pgfpathcurveto{\pgfqpoint{5.604135in}{0.587887in}}{\pgfqpoint{5.614734in}{0.583497in}}{\pgfqpoint{5.625784in}{0.583497in}}%
\pgfpathlineto{\pgfqpoint{5.625784in}{0.583497in}}%
\pgfpathclose%
\pgfusepath{stroke}%
\end{pgfscope}%
\begin{pgfscope}%
\pgfpathrectangle{\pgfqpoint{0.847223in}{0.554012in}}{\pgfqpoint{6.200000in}{4.530000in}}%
\pgfusepath{clip}%
\pgfsetbuttcap%
\pgfsetroundjoin%
\pgfsetlinewidth{1.003750pt}%
\definecolor{currentstroke}{rgb}{1.000000,0.000000,0.000000}%
\pgfsetstrokecolor{currentstroke}%
\pgfsetdash{}{0pt}%
\pgfpathmoveto{\pgfqpoint{5.631118in}{0.583006in}}%
\pgfpathcurveto{\pgfqpoint{5.642168in}{0.583006in}}{\pgfqpoint{5.652767in}{0.587397in}}{\pgfqpoint{5.660580in}{0.595210in}}%
\pgfpathcurveto{\pgfqpoint{5.668394in}{0.603024in}}{\pgfqpoint{5.672784in}{0.613623in}}{\pgfqpoint{5.672784in}{0.624673in}}%
\pgfpathcurveto{\pgfqpoint{5.672784in}{0.635723in}}{\pgfqpoint{5.668394in}{0.646322in}}{\pgfqpoint{5.660580in}{0.654136in}}%
\pgfpathcurveto{\pgfqpoint{5.652767in}{0.661949in}}{\pgfqpoint{5.642168in}{0.666340in}}{\pgfqpoint{5.631118in}{0.666340in}}%
\pgfpathcurveto{\pgfqpoint{5.620067in}{0.666340in}}{\pgfqpoint{5.609468in}{0.661949in}}{\pgfqpoint{5.601655in}{0.654136in}}%
\pgfpathcurveto{\pgfqpoint{5.593841in}{0.646322in}}{\pgfqpoint{5.589451in}{0.635723in}}{\pgfqpoint{5.589451in}{0.624673in}}%
\pgfpathcurveto{\pgfqpoint{5.589451in}{0.613623in}}{\pgfqpoint{5.593841in}{0.603024in}}{\pgfqpoint{5.601655in}{0.595210in}}%
\pgfpathcurveto{\pgfqpoint{5.609468in}{0.587397in}}{\pgfqpoint{5.620067in}{0.583006in}}{\pgfqpoint{5.631118in}{0.583006in}}%
\pgfpathlineto{\pgfqpoint{5.631118in}{0.583006in}}%
\pgfpathclose%
\pgfusepath{stroke}%
\end{pgfscope}%
\begin{pgfscope}%
\pgfpathrectangle{\pgfqpoint{0.847223in}{0.554012in}}{\pgfqpoint{6.200000in}{4.530000in}}%
\pgfusepath{clip}%
\pgfsetbuttcap%
\pgfsetroundjoin%
\pgfsetlinewidth{1.003750pt}%
\definecolor{currentstroke}{rgb}{1.000000,0.000000,0.000000}%
\pgfsetstrokecolor{currentstroke}%
\pgfsetdash{}{0pt}%
\pgfpathmoveto{\pgfqpoint{5.636451in}{0.582517in}}%
\pgfpathcurveto{\pgfqpoint{5.647501in}{0.582517in}}{\pgfqpoint{5.658100in}{0.586907in}}{\pgfqpoint{5.665914in}{0.594720in}}%
\pgfpathcurveto{\pgfqpoint{5.673727in}{0.602534in}}{\pgfqpoint{5.678117in}{0.613133in}}{\pgfqpoint{5.678117in}{0.624183in}}%
\pgfpathcurveto{\pgfqpoint{5.678117in}{0.635233in}}{\pgfqpoint{5.673727in}{0.645832in}}{\pgfqpoint{5.665914in}{0.653646in}}%
\pgfpathcurveto{\pgfqpoint{5.658100in}{0.661460in}}{\pgfqpoint{5.647501in}{0.665850in}}{\pgfqpoint{5.636451in}{0.665850in}}%
\pgfpathcurveto{\pgfqpoint{5.625401in}{0.665850in}}{\pgfqpoint{5.614802in}{0.661460in}}{\pgfqpoint{5.606988in}{0.653646in}}%
\pgfpathcurveto{\pgfqpoint{5.599174in}{0.645832in}}{\pgfqpoint{5.594784in}{0.635233in}}{\pgfqpoint{5.594784in}{0.624183in}}%
\pgfpathcurveto{\pgfqpoint{5.594784in}{0.613133in}}{\pgfqpoint{5.599174in}{0.602534in}}{\pgfqpoint{5.606988in}{0.594720in}}%
\pgfpathcurveto{\pgfqpoint{5.614802in}{0.586907in}}{\pgfqpoint{5.625401in}{0.582517in}}{\pgfqpoint{5.636451in}{0.582517in}}%
\pgfpathlineto{\pgfqpoint{5.636451in}{0.582517in}}%
\pgfpathclose%
\pgfusepath{stroke}%
\end{pgfscope}%
\begin{pgfscope}%
\pgfpathrectangle{\pgfqpoint{0.847223in}{0.554012in}}{\pgfqpoint{6.200000in}{4.530000in}}%
\pgfusepath{clip}%
\pgfsetbuttcap%
\pgfsetroundjoin%
\pgfsetlinewidth{1.003750pt}%
\definecolor{currentstroke}{rgb}{1.000000,0.000000,0.000000}%
\pgfsetstrokecolor{currentstroke}%
\pgfsetdash{}{0pt}%
\pgfpathmoveto{\pgfqpoint{5.641784in}{0.582028in}}%
\pgfpathcurveto{\pgfqpoint{5.652834in}{0.582028in}}{\pgfqpoint{5.663433in}{0.586418in}}{\pgfqpoint{5.671247in}{0.594232in}}%
\pgfpathcurveto{\pgfqpoint{5.679060in}{0.602045in}}{\pgfqpoint{5.683451in}{0.612644in}}{\pgfqpoint{5.683451in}{0.623695in}}%
\pgfpathcurveto{\pgfqpoint{5.683451in}{0.634745in}}{\pgfqpoint{5.679060in}{0.645344in}}{\pgfqpoint{5.671247in}{0.653157in}}%
\pgfpathcurveto{\pgfqpoint{5.663433in}{0.660971in}}{\pgfqpoint{5.652834in}{0.665361in}}{\pgfqpoint{5.641784in}{0.665361in}}%
\pgfpathcurveto{\pgfqpoint{5.630734in}{0.665361in}}{\pgfqpoint{5.620135in}{0.660971in}}{\pgfqpoint{5.612321in}{0.653157in}}%
\pgfpathcurveto{\pgfqpoint{5.604508in}{0.645344in}}{\pgfqpoint{5.600117in}{0.634745in}}{\pgfqpoint{5.600117in}{0.623695in}}%
\pgfpathcurveto{\pgfqpoint{5.600117in}{0.612644in}}{\pgfqpoint{5.604508in}{0.602045in}}{\pgfqpoint{5.612321in}{0.594232in}}%
\pgfpathcurveto{\pgfqpoint{5.620135in}{0.586418in}}{\pgfqpoint{5.630734in}{0.582028in}}{\pgfqpoint{5.641784in}{0.582028in}}%
\pgfpathlineto{\pgfqpoint{5.641784in}{0.582028in}}%
\pgfpathclose%
\pgfusepath{stroke}%
\end{pgfscope}%
\begin{pgfscope}%
\pgfpathrectangle{\pgfqpoint{0.847223in}{0.554012in}}{\pgfqpoint{6.200000in}{4.530000in}}%
\pgfusepath{clip}%
\pgfsetbuttcap%
\pgfsetroundjoin%
\pgfsetlinewidth{1.003750pt}%
\definecolor{currentstroke}{rgb}{1.000000,0.000000,0.000000}%
\pgfsetstrokecolor{currentstroke}%
\pgfsetdash{}{0pt}%
\pgfpathmoveto{\pgfqpoint{5.647117in}{0.581540in}}%
\pgfpathcurveto{\pgfqpoint{5.658167in}{0.581540in}}{\pgfqpoint{5.668766in}{0.585930in}}{\pgfqpoint{5.676580in}{0.593744in}}%
\pgfpathcurveto{\pgfqpoint{5.684394in}{0.601558in}}{\pgfqpoint{5.688784in}{0.612157in}}{\pgfqpoint{5.688784in}{0.623207in}}%
\pgfpathcurveto{\pgfqpoint{5.688784in}{0.634257in}}{\pgfqpoint{5.684394in}{0.644856in}}{\pgfqpoint{5.676580in}{0.652670in}}%
\pgfpathcurveto{\pgfqpoint{5.668766in}{0.660483in}}{\pgfqpoint{5.658167in}{0.664873in}}{\pgfqpoint{5.647117in}{0.664873in}}%
\pgfpathcurveto{\pgfqpoint{5.636067in}{0.664873in}}{\pgfqpoint{5.625468in}{0.660483in}}{\pgfqpoint{5.617654in}{0.652670in}}%
\pgfpathcurveto{\pgfqpoint{5.609841in}{0.644856in}}{\pgfqpoint{5.605450in}{0.634257in}}{\pgfqpoint{5.605450in}{0.623207in}}%
\pgfpathcurveto{\pgfqpoint{5.605450in}{0.612157in}}{\pgfqpoint{5.609841in}{0.601558in}}{\pgfqpoint{5.617654in}{0.593744in}}%
\pgfpathcurveto{\pgfqpoint{5.625468in}{0.585930in}}{\pgfqpoint{5.636067in}{0.581540in}}{\pgfqpoint{5.647117in}{0.581540in}}%
\pgfpathlineto{\pgfqpoint{5.647117in}{0.581540in}}%
\pgfpathclose%
\pgfusepath{stroke}%
\end{pgfscope}%
\begin{pgfscope}%
\pgfpathrectangle{\pgfqpoint{0.847223in}{0.554012in}}{\pgfqpoint{6.200000in}{4.530000in}}%
\pgfusepath{clip}%
\pgfsetbuttcap%
\pgfsetroundjoin%
\pgfsetlinewidth{1.003750pt}%
\definecolor{currentstroke}{rgb}{1.000000,0.000000,0.000000}%
\pgfsetstrokecolor{currentstroke}%
\pgfsetdash{}{0pt}%
\pgfpathmoveto{\pgfqpoint{5.652450in}{0.581053in}}%
\pgfpathcurveto{\pgfqpoint{5.663500in}{0.581053in}}{\pgfqpoint{5.674100in}{0.585444in}}{\pgfqpoint{5.681913in}{0.593257in}}%
\pgfpathcurveto{\pgfqpoint{5.689727in}{0.601071in}}{\pgfqpoint{5.694117in}{0.611670in}}{\pgfqpoint{5.694117in}{0.622720in}}%
\pgfpathcurveto{\pgfqpoint{5.694117in}{0.633770in}}{\pgfqpoint{5.689727in}{0.644369in}}{\pgfqpoint{5.681913in}{0.652183in}}%
\pgfpathcurveto{\pgfqpoint{5.674100in}{0.659996in}}{\pgfqpoint{5.663500in}{0.664387in}}{\pgfqpoint{5.652450in}{0.664387in}}%
\pgfpathcurveto{\pgfqpoint{5.641400in}{0.664387in}}{\pgfqpoint{5.630801in}{0.659996in}}{\pgfqpoint{5.622988in}{0.652183in}}%
\pgfpathcurveto{\pgfqpoint{5.615174in}{0.644369in}}{\pgfqpoint{5.610784in}{0.633770in}}{\pgfqpoint{5.610784in}{0.622720in}}%
\pgfpathcurveto{\pgfqpoint{5.610784in}{0.611670in}}{\pgfqpoint{5.615174in}{0.601071in}}{\pgfqpoint{5.622988in}{0.593257in}}%
\pgfpathcurveto{\pgfqpoint{5.630801in}{0.585444in}}{\pgfqpoint{5.641400in}{0.581053in}}{\pgfqpoint{5.652450in}{0.581053in}}%
\pgfpathlineto{\pgfqpoint{5.652450in}{0.581053in}}%
\pgfpathclose%
\pgfusepath{stroke}%
\end{pgfscope}%
\begin{pgfscope}%
\pgfpathrectangle{\pgfqpoint{0.847223in}{0.554012in}}{\pgfqpoint{6.200000in}{4.530000in}}%
\pgfusepath{clip}%
\pgfsetbuttcap%
\pgfsetroundjoin%
\pgfsetlinewidth{1.003750pt}%
\definecolor{currentstroke}{rgb}{1.000000,0.000000,0.000000}%
\pgfsetstrokecolor{currentstroke}%
\pgfsetdash{}{0pt}%
\pgfpathmoveto{\pgfqpoint{5.657784in}{0.580567in}}%
\pgfpathcurveto{\pgfqpoint{5.668834in}{0.580567in}}{\pgfqpoint{5.679433in}{0.584958in}}{\pgfqpoint{5.687246in}{0.592771in}}%
\pgfpathcurveto{\pgfqpoint{5.695060in}{0.600585in}}{\pgfqpoint{5.699450in}{0.611184in}}{\pgfqpoint{5.699450in}{0.622234in}}%
\pgfpathcurveto{\pgfqpoint{5.699450in}{0.633284in}}{\pgfqpoint{5.695060in}{0.643883in}}{\pgfqpoint{5.687246in}{0.651697in}}%
\pgfpathcurveto{\pgfqpoint{5.679433in}{0.659510in}}{\pgfqpoint{5.668834in}{0.663901in}}{\pgfqpoint{5.657784in}{0.663901in}}%
\pgfpathcurveto{\pgfqpoint{5.646733in}{0.663901in}}{\pgfqpoint{5.636134in}{0.659510in}}{\pgfqpoint{5.628321in}{0.651697in}}%
\pgfpathcurveto{\pgfqpoint{5.620507in}{0.643883in}}{\pgfqpoint{5.616117in}{0.633284in}}{\pgfqpoint{5.616117in}{0.622234in}}%
\pgfpathcurveto{\pgfqpoint{5.616117in}{0.611184in}}{\pgfqpoint{5.620507in}{0.600585in}}{\pgfqpoint{5.628321in}{0.592771in}}%
\pgfpathcurveto{\pgfqpoint{5.636134in}{0.584958in}}{\pgfqpoint{5.646733in}{0.580567in}}{\pgfqpoint{5.657784in}{0.580567in}}%
\pgfpathlineto{\pgfqpoint{5.657784in}{0.580567in}}%
\pgfpathclose%
\pgfusepath{stroke}%
\end{pgfscope}%
\begin{pgfscope}%
\pgfpathrectangle{\pgfqpoint{0.847223in}{0.554012in}}{\pgfqpoint{6.200000in}{4.530000in}}%
\pgfusepath{clip}%
\pgfsetbuttcap%
\pgfsetroundjoin%
\pgfsetlinewidth{1.003750pt}%
\definecolor{currentstroke}{rgb}{1.000000,0.000000,0.000000}%
\pgfsetstrokecolor{currentstroke}%
\pgfsetdash{}{0pt}%
\pgfpathmoveto{\pgfqpoint{5.663117in}{0.580083in}}%
\pgfpathcurveto{\pgfqpoint{5.674167in}{0.580083in}}{\pgfqpoint{5.684766in}{0.584473in}}{\pgfqpoint{5.692580in}{0.592286in}}%
\pgfpathcurveto{\pgfqpoint{5.700393in}{0.600100in}}{\pgfqpoint{5.704783in}{0.610699in}}{\pgfqpoint{5.704783in}{0.621749in}}%
\pgfpathcurveto{\pgfqpoint{5.704783in}{0.632799in}}{\pgfqpoint{5.700393in}{0.643398in}}{\pgfqpoint{5.692580in}{0.651212in}}%
\pgfpathcurveto{\pgfqpoint{5.684766in}{0.659026in}}{\pgfqpoint{5.674167in}{0.663416in}}{\pgfqpoint{5.663117in}{0.663416in}}%
\pgfpathcurveto{\pgfqpoint{5.652067in}{0.663416in}}{\pgfqpoint{5.641468in}{0.659026in}}{\pgfqpoint{5.633654in}{0.651212in}}%
\pgfpathcurveto{\pgfqpoint{5.625840in}{0.643398in}}{\pgfqpoint{5.621450in}{0.632799in}}{\pgfqpoint{5.621450in}{0.621749in}}%
\pgfpathcurveto{\pgfqpoint{5.621450in}{0.610699in}}{\pgfqpoint{5.625840in}{0.600100in}}{\pgfqpoint{5.633654in}{0.592286in}}%
\pgfpathcurveto{\pgfqpoint{5.641468in}{0.584473in}}{\pgfqpoint{5.652067in}{0.580083in}}{\pgfqpoint{5.663117in}{0.580083in}}%
\pgfpathlineto{\pgfqpoint{5.663117in}{0.580083in}}%
\pgfpathclose%
\pgfusepath{stroke}%
\end{pgfscope}%
\begin{pgfscope}%
\pgfpathrectangle{\pgfqpoint{0.847223in}{0.554012in}}{\pgfqpoint{6.200000in}{4.530000in}}%
\pgfusepath{clip}%
\pgfsetbuttcap%
\pgfsetroundjoin%
\pgfsetlinewidth{1.003750pt}%
\definecolor{currentstroke}{rgb}{1.000000,0.000000,0.000000}%
\pgfsetstrokecolor{currentstroke}%
\pgfsetdash{}{0pt}%
\pgfpathmoveto{\pgfqpoint{5.668450in}{0.579599in}}%
\pgfpathcurveto{\pgfqpoint{5.679500in}{0.579599in}}{\pgfqpoint{5.690099in}{0.583989in}}{\pgfqpoint{5.697913in}{0.591803in}}%
\pgfpathcurveto{\pgfqpoint{5.705726in}{0.599616in}}{\pgfqpoint{5.710117in}{0.610215in}}{\pgfqpoint{5.710117in}{0.621265in}}%
\pgfpathcurveto{\pgfqpoint{5.710117in}{0.632315in}}{\pgfqpoint{5.705726in}{0.642915in}}{\pgfqpoint{5.697913in}{0.650728in}}%
\pgfpathcurveto{\pgfqpoint{5.690099in}{0.658542in}}{\pgfqpoint{5.679500in}{0.662932in}}{\pgfqpoint{5.668450in}{0.662932in}}%
\pgfpathcurveto{\pgfqpoint{5.657400in}{0.662932in}}{\pgfqpoint{5.646801in}{0.658542in}}{\pgfqpoint{5.638987in}{0.650728in}}%
\pgfpathcurveto{\pgfqpoint{5.631174in}{0.642915in}}{\pgfqpoint{5.626783in}{0.632315in}}{\pgfqpoint{5.626783in}{0.621265in}}%
\pgfpathcurveto{\pgfqpoint{5.626783in}{0.610215in}}{\pgfqpoint{5.631174in}{0.599616in}}{\pgfqpoint{5.638987in}{0.591803in}}%
\pgfpathcurveto{\pgfqpoint{5.646801in}{0.583989in}}{\pgfqpoint{5.657400in}{0.579599in}}{\pgfqpoint{5.668450in}{0.579599in}}%
\pgfpathlineto{\pgfqpoint{5.668450in}{0.579599in}}%
\pgfpathclose%
\pgfusepath{stroke}%
\end{pgfscope}%
\begin{pgfscope}%
\pgfpathrectangle{\pgfqpoint{0.847223in}{0.554012in}}{\pgfqpoint{6.200000in}{4.530000in}}%
\pgfusepath{clip}%
\pgfsetbuttcap%
\pgfsetroundjoin%
\pgfsetlinewidth{1.003750pt}%
\definecolor{currentstroke}{rgb}{1.000000,0.000000,0.000000}%
\pgfsetstrokecolor{currentstroke}%
\pgfsetdash{}{0pt}%
\pgfpathmoveto{\pgfqpoint{5.673783in}{0.579116in}}%
\pgfpathcurveto{\pgfqpoint{5.684833in}{0.579116in}}{\pgfqpoint{5.695432in}{0.583506in}}{\pgfqpoint{5.703246in}{0.591320in}}%
\pgfpathcurveto{\pgfqpoint{5.711060in}{0.599133in}}{\pgfqpoint{5.715450in}{0.609732in}}{\pgfqpoint{5.715450in}{0.620782in}}%
\pgfpathcurveto{\pgfqpoint{5.715450in}{0.631833in}}{\pgfqpoint{5.711060in}{0.642432in}}{\pgfqpoint{5.703246in}{0.650245in}}%
\pgfpathcurveto{\pgfqpoint{5.695432in}{0.658059in}}{\pgfqpoint{5.684833in}{0.662449in}}{\pgfqpoint{5.673783in}{0.662449in}}%
\pgfpathcurveto{\pgfqpoint{5.662733in}{0.662449in}}{\pgfqpoint{5.652134in}{0.658059in}}{\pgfqpoint{5.644320in}{0.650245in}}%
\pgfpathcurveto{\pgfqpoint{5.636507in}{0.642432in}}{\pgfqpoint{5.632117in}{0.631833in}}{\pgfqpoint{5.632117in}{0.620782in}}%
\pgfpathcurveto{\pgfqpoint{5.632117in}{0.609732in}}{\pgfqpoint{5.636507in}{0.599133in}}{\pgfqpoint{5.644320in}{0.591320in}}%
\pgfpathcurveto{\pgfqpoint{5.652134in}{0.583506in}}{\pgfqpoint{5.662733in}{0.579116in}}{\pgfqpoint{5.673783in}{0.579116in}}%
\pgfpathlineto{\pgfqpoint{5.673783in}{0.579116in}}%
\pgfpathclose%
\pgfusepath{stroke}%
\end{pgfscope}%
\begin{pgfscope}%
\pgfpathrectangle{\pgfqpoint{0.847223in}{0.554012in}}{\pgfqpoint{6.200000in}{4.530000in}}%
\pgfusepath{clip}%
\pgfsetbuttcap%
\pgfsetroundjoin%
\pgfsetlinewidth{1.003750pt}%
\definecolor{currentstroke}{rgb}{1.000000,0.000000,0.000000}%
\pgfsetstrokecolor{currentstroke}%
\pgfsetdash{}{0pt}%
\pgfpathmoveto{\pgfqpoint{5.679116in}{0.578634in}}%
\pgfpathcurveto{\pgfqpoint{5.690167in}{0.578634in}}{\pgfqpoint{5.700766in}{0.583024in}}{\pgfqpoint{5.708579in}{0.590838in}}%
\pgfpathcurveto{\pgfqpoint{5.716393in}{0.598651in}}{\pgfqpoint{5.720783in}{0.609250in}}{\pgfqpoint{5.720783in}{0.620300in}}%
\pgfpathcurveto{\pgfqpoint{5.720783in}{0.631351in}}{\pgfqpoint{5.716393in}{0.641950in}}{\pgfqpoint{5.708579in}{0.649763in}}%
\pgfpathcurveto{\pgfqpoint{5.700766in}{0.657577in}}{\pgfqpoint{5.690167in}{0.661967in}}{\pgfqpoint{5.679116in}{0.661967in}}%
\pgfpathcurveto{\pgfqpoint{5.668066in}{0.661967in}}{\pgfqpoint{5.657467in}{0.657577in}}{\pgfqpoint{5.649654in}{0.649763in}}%
\pgfpathcurveto{\pgfqpoint{5.641840in}{0.641950in}}{\pgfqpoint{5.637450in}{0.631351in}}{\pgfqpoint{5.637450in}{0.620300in}}%
\pgfpathcurveto{\pgfqpoint{5.637450in}{0.609250in}}{\pgfqpoint{5.641840in}{0.598651in}}{\pgfqpoint{5.649654in}{0.590838in}}%
\pgfpathcurveto{\pgfqpoint{5.657467in}{0.583024in}}{\pgfqpoint{5.668066in}{0.578634in}}{\pgfqpoint{5.679116in}{0.578634in}}%
\pgfpathlineto{\pgfqpoint{5.679116in}{0.578634in}}%
\pgfpathclose%
\pgfusepath{stroke}%
\end{pgfscope}%
\begin{pgfscope}%
\pgfpathrectangle{\pgfqpoint{0.847223in}{0.554012in}}{\pgfqpoint{6.200000in}{4.530000in}}%
\pgfusepath{clip}%
\pgfsetbuttcap%
\pgfsetroundjoin%
\pgfsetlinewidth{1.003750pt}%
\definecolor{currentstroke}{rgb}{1.000000,0.000000,0.000000}%
\pgfsetstrokecolor{currentstroke}%
\pgfsetdash{}{0pt}%
\pgfpathmoveto{\pgfqpoint{5.684450in}{0.578153in}}%
\pgfpathcurveto{\pgfqpoint{5.695500in}{0.578153in}}{\pgfqpoint{5.706099in}{0.582543in}}{\pgfqpoint{5.713912in}{0.590357in}}%
\pgfpathcurveto{\pgfqpoint{5.721726in}{0.598170in}}{\pgfqpoint{5.726116in}{0.608769in}}{\pgfqpoint{5.726116in}{0.619819in}}%
\pgfpathcurveto{\pgfqpoint{5.726116in}{0.630870in}}{\pgfqpoint{5.721726in}{0.641469in}}{\pgfqpoint{5.713912in}{0.649282in}}%
\pgfpathcurveto{\pgfqpoint{5.706099in}{0.657096in}}{\pgfqpoint{5.695500in}{0.661486in}}{\pgfqpoint{5.684450in}{0.661486in}}%
\pgfpathcurveto{\pgfqpoint{5.673400in}{0.661486in}}{\pgfqpoint{5.662800in}{0.657096in}}{\pgfqpoint{5.654987in}{0.649282in}}%
\pgfpathcurveto{\pgfqpoint{5.647173in}{0.641469in}}{\pgfqpoint{5.642783in}{0.630870in}}{\pgfqpoint{5.642783in}{0.619819in}}%
\pgfpathcurveto{\pgfqpoint{5.642783in}{0.608769in}}{\pgfqpoint{5.647173in}{0.598170in}}{\pgfqpoint{5.654987in}{0.590357in}}%
\pgfpathcurveto{\pgfqpoint{5.662800in}{0.582543in}}{\pgfqpoint{5.673400in}{0.578153in}}{\pgfqpoint{5.684450in}{0.578153in}}%
\pgfpathlineto{\pgfqpoint{5.684450in}{0.578153in}}%
\pgfpathclose%
\pgfusepath{stroke}%
\end{pgfscope}%
\begin{pgfscope}%
\pgfpathrectangle{\pgfqpoint{0.847223in}{0.554012in}}{\pgfqpoint{6.200000in}{4.530000in}}%
\pgfusepath{clip}%
\pgfsetbuttcap%
\pgfsetroundjoin%
\pgfsetlinewidth{1.003750pt}%
\definecolor{currentstroke}{rgb}{1.000000,0.000000,0.000000}%
\pgfsetstrokecolor{currentstroke}%
\pgfsetdash{}{0pt}%
\pgfpathmoveto{\pgfqpoint{5.689783in}{0.577673in}}%
\pgfpathcurveto{\pgfqpoint{5.700833in}{0.577673in}}{\pgfqpoint{5.711432in}{0.582063in}}{\pgfqpoint{5.719246in}{0.589877in}}%
\pgfpathcurveto{\pgfqpoint{5.727059in}{0.597690in}}{\pgfqpoint{5.731450in}{0.608289in}}{\pgfqpoint{5.731450in}{0.619339in}}%
\pgfpathcurveto{\pgfqpoint{5.731450in}{0.630390in}}{\pgfqpoint{5.727059in}{0.640989in}}{\pgfqpoint{5.719246in}{0.648802in}}%
\pgfpathcurveto{\pgfqpoint{5.711432in}{0.656616in}}{\pgfqpoint{5.700833in}{0.661006in}}{\pgfqpoint{5.689783in}{0.661006in}}%
\pgfpathcurveto{\pgfqpoint{5.678733in}{0.661006in}}{\pgfqpoint{5.668134in}{0.656616in}}{\pgfqpoint{5.660320in}{0.648802in}}%
\pgfpathcurveto{\pgfqpoint{5.652506in}{0.640989in}}{\pgfqpoint{5.648116in}{0.630390in}}{\pgfqpoint{5.648116in}{0.619339in}}%
\pgfpathcurveto{\pgfqpoint{5.648116in}{0.608289in}}{\pgfqpoint{5.652506in}{0.597690in}}{\pgfqpoint{5.660320in}{0.589877in}}%
\pgfpathcurveto{\pgfqpoint{5.668134in}{0.582063in}}{\pgfqpoint{5.678733in}{0.577673in}}{\pgfqpoint{5.689783in}{0.577673in}}%
\pgfpathlineto{\pgfqpoint{5.689783in}{0.577673in}}%
\pgfpathclose%
\pgfusepath{stroke}%
\end{pgfscope}%
\begin{pgfscope}%
\pgfpathrectangle{\pgfqpoint{0.847223in}{0.554012in}}{\pgfqpoint{6.200000in}{4.530000in}}%
\pgfusepath{clip}%
\pgfsetbuttcap%
\pgfsetroundjoin%
\pgfsetlinewidth{1.003750pt}%
\definecolor{currentstroke}{rgb}{1.000000,0.000000,0.000000}%
\pgfsetstrokecolor{currentstroke}%
\pgfsetdash{}{0pt}%
\pgfpathmoveto{\pgfqpoint{5.695116in}{0.577194in}}%
\pgfpathcurveto{\pgfqpoint{5.706166in}{0.577194in}}{\pgfqpoint{5.716765in}{0.581584in}}{\pgfqpoint{5.724579in}{0.589398in}}%
\pgfpathcurveto{\pgfqpoint{5.732392in}{0.597211in}}{\pgfqpoint{5.736783in}{0.607810in}}{\pgfqpoint{5.736783in}{0.618860in}}%
\pgfpathcurveto{\pgfqpoint{5.736783in}{0.629910in}}{\pgfqpoint{5.732392in}{0.640509in}}{\pgfqpoint{5.724579in}{0.648323in}}%
\pgfpathcurveto{\pgfqpoint{5.716765in}{0.656137in}}{\pgfqpoint{5.706166in}{0.660527in}}{\pgfqpoint{5.695116in}{0.660527in}}%
\pgfpathcurveto{\pgfqpoint{5.684066in}{0.660527in}}{\pgfqpoint{5.673467in}{0.656137in}}{\pgfqpoint{5.665653in}{0.648323in}}%
\pgfpathcurveto{\pgfqpoint{5.657840in}{0.640509in}}{\pgfqpoint{5.653449in}{0.629910in}}{\pgfqpoint{5.653449in}{0.618860in}}%
\pgfpathcurveto{\pgfqpoint{5.653449in}{0.607810in}}{\pgfqpoint{5.657840in}{0.597211in}}{\pgfqpoint{5.665653in}{0.589398in}}%
\pgfpathcurveto{\pgfqpoint{5.673467in}{0.581584in}}{\pgfqpoint{5.684066in}{0.577194in}}{\pgfqpoint{5.695116in}{0.577194in}}%
\pgfpathlineto{\pgfqpoint{5.695116in}{0.577194in}}%
\pgfpathclose%
\pgfusepath{stroke}%
\end{pgfscope}%
\begin{pgfscope}%
\pgfpathrectangle{\pgfqpoint{0.847223in}{0.554012in}}{\pgfqpoint{6.200000in}{4.530000in}}%
\pgfusepath{clip}%
\pgfsetbuttcap%
\pgfsetroundjoin%
\pgfsetlinewidth{1.003750pt}%
\definecolor{currentstroke}{rgb}{1.000000,0.000000,0.000000}%
\pgfsetstrokecolor{currentstroke}%
\pgfsetdash{}{0pt}%
\pgfpathmoveto{\pgfqpoint{5.700449in}{0.576716in}}%
\pgfpathcurveto{\pgfqpoint{5.711499in}{0.576716in}}{\pgfqpoint{5.722098in}{0.581106in}}{\pgfqpoint{5.729912in}{0.588919in}}%
\pgfpathcurveto{\pgfqpoint{5.737726in}{0.596733in}}{\pgfqpoint{5.742116in}{0.607332in}}{\pgfqpoint{5.742116in}{0.618382in}}%
\pgfpathcurveto{\pgfqpoint{5.742116in}{0.629432in}}{\pgfqpoint{5.737726in}{0.640031in}}{\pgfqpoint{5.729912in}{0.647845in}}%
\pgfpathcurveto{\pgfqpoint{5.722098in}{0.655659in}}{\pgfqpoint{5.711499in}{0.660049in}}{\pgfqpoint{5.700449in}{0.660049in}}%
\pgfpathcurveto{\pgfqpoint{5.689399in}{0.660049in}}{\pgfqpoint{5.678800in}{0.655659in}}{\pgfqpoint{5.670987in}{0.647845in}}%
\pgfpathcurveto{\pgfqpoint{5.663173in}{0.640031in}}{\pgfqpoint{5.658783in}{0.629432in}}{\pgfqpoint{5.658783in}{0.618382in}}%
\pgfpathcurveto{\pgfqpoint{5.658783in}{0.607332in}}{\pgfqpoint{5.663173in}{0.596733in}}{\pgfqpoint{5.670987in}{0.588919in}}%
\pgfpathcurveto{\pgfqpoint{5.678800in}{0.581106in}}{\pgfqpoint{5.689399in}{0.576716in}}{\pgfqpoint{5.700449in}{0.576716in}}%
\pgfpathlineto{\pgfqpoint{5.700449in}{0.576716in}}%
\pgfpathclose%
\pgfusepath{stroke}%
\end{pgfscope}%
\begin{pgfscope}%
\pgfpathrectangle{\pgfqpoint{0.847223in}{0.554012in}}{\pgfqpoint{6.200000in}{4.530000in}}%
\pgfusepath{clip}%
\pgfsetbuttcap%
\pgfsetroundjoin%
\pgfsetlinewidth{1.003750pt}%
\definecolor{currentstroke}{rgb}{1.000000,0.000000,0.000000}%
\pgfsetstrokecolor{currentstroke}%
\pgfsetdash{}{0pt}%
\pgfpathmoveto{\pgfqpoint{5.705783in}{0.576238in}}%
\pgfpathcurveto{\pgfqpoint{5.716833in}{0.576238in}}{\pgfqpoint{5.727432in}{0.580629in}}{\pgfqpoint{5.735245in}{0.588442in}}%
\pgfpathcurveto{\pgfqpoint{5.743059in}{0.596256in}}{\pgfqpoint{5.747449in}{0.606855in}}{\pgfqpoint{5.747449in}{0.617905in}}%
\pgfpathcurveto{\pgfqpoint{5.747449in}{0.628955in}}{\pgfqpoint{5.743059in}{0.639554in}}{\pgfqpoint{5.735245in}{0.647368in}}%
\pgfpathcurveto{\pgfqpoint{5.727432in}{0.655181in}}{\pgfqpoint{5.716833in}{0.659572in}}{\pgfqpoint{5.705783in}{0.659572in}}%
\pgfpathcurveto{\pgfqpoint{5.694732in}{0.659572in}}{\pgfqpoint{5.684133in}{0.655181in}}{\pgfqpoint{5.676320in}{0.647368in}}%
\pgfpathcurveto{\pgfqpoint{5.668506in}{0.639554in}}{\pgfqpoint{5.664116in}{0.628955in}}{\pgfqpoint{5.664116in}{0.617905in}}%
\pgfpathcurveto{\pgfqpoint{5.664116in}{0.606855in}}{\pgfqpoint{5.668506in}{0.596256in}}{\pgfqpoint{5.676320in}{0.588442in}}%
\pgfpathcurveto{\pgfqpoint{5.684133in}{0.580629in}}{\pgfqpoint{5.694732in}{0.576238in}}{\pgfqpoint{5.705783in}{0.576238in}}%
\pgfpathlineto{\pgfqpoint{5.705783in}{0.576238in}}%
\pgfpathclose%
\pgfusepath{stroke}%
\end{pgfscope}%
\begin{pgfscope}%
\pgfpathrectangle{\pgfqpoint{0.847223in}{0.554012in}}{\pgfqpoint{6.200000in}{4.530000in}}%
\pgfusepath{clip}%
\pgfsetbuttcap%
\pgfsetroundjoin%
\pgfsetlinewidth{1.003750pt}%
\definecolor{currentstroke}{rgb}{1.000000,0.000000,0.000000}%
\pgfsetstrokecolor{currentstroke}%
\pgfsetdash{}{0pt}%
\pgfpathmoveto{\pgfqpoint{5.711116in}{0.575762in}}%
\pgfpathcurveto{\pgfqpoint{5.722166in}{0.575762in}}{\pgfqpoint{5.732765in}{0.580152in}}{\pgfqpoint{5.740579in}{0.587966in}}%
\pgfpathcurveto{\pgfqpoint{5.748392in}{0.595780in}}{\pgfqpoint{5.752782in}{0.606379in}}{\pgfqpoint{5.752782in}{0.617429in}}%
\pgfpathcurveto{\pgfqpoint{5.752782in}{0.628479in}}{\pgfqpoint{5.748392in}{0.639078in}}{\pgfqpoint{5.740579in}{0.646892in}}%
\pgfpathcurveto{\pgfqpoint{5.732765in}{0.654705in}}{\pgfqpoint{5.722166in}{0.659095in}}{\pgfqpoint{5.711116in}{0.659095in}}%
\pgfpathcurveto{\pgfqpoint{5.700066in}{0.659095in}}{\pgfqpoint{5.689467in}{0.654705in}}{\pgfqpoint{5.681653in}{0.646892in}}%
\pgfpathcurveto{\pgfqpoint{5.673839in}{0.639078in}}{\pgfqpoint{5.669449in}{0.628479in}}{\pgfqpoint{5.669449in}{0.617429in}}%
\pgfpathcurveto{\pgfqpoint{5.669449in}{0.606379in}}{\pgfqpoint{5.673839in}{0.595780in}}{\pgfqpoint{5.681653in}{0.587966in}}%
\pgfpathcurveto{\pgfqpoint{5.689467in}{0.580152in}}{\pgfqpoint{5.700066in}{0.575762in}}{\pgfqpoint{5.711116in}{0.575762in}}%
\pgfpathlineto{\pgfqpoint{5.711116in}{0.575762in}}%
\pgfpathclose%
\pgfusepath{stroke}%
\end{pgfscope}%
\begin{pgfscope}%
\pgfpathrectangle{\pgfqpoint{0.847223in}{0.554012in}}{\pgfqpoint{6.200000in}{4.530000in}}%
\pgfusepath{clip}%
\pgfsetbuttcap%
\pgfsetroundjoin%
\pgfsetlinewidth{1.003750pt}%
\definecolor{currentstroke}{rgb}{1.000000,0.000000,0.000000}%
\pgfsetstrokecolor{currentstroke}%
\pgfsetdash{}{0pt}%
\pgfpathmoveto{\pgfqpoint{5.716449in}{0.575287in}}%
\pgfpathcurveto{\pgfqpoint{5.727499in}{0.575287in}}{\pgfqpoint{5.738098in}{0.579677in}}{\pgfqpoint{5.745912in}{0.587491in}}%
\pgfpathcurveto{\pgfqpoint{5.753725in}{0.595304in}}{\pgfqpoint{5.758116in}{0.605903in}}{\pgfqpoint{5.758116in}{0.616953in}}%
\pgfpathcurveto{\pgfqpoint{5.758116in}{0.628004in}}{\pgfqpoint{5.753725in}{0.638603in}}{\pgfqpoint{5.745912in}{0.646416in}}%
\pgfpathcurveto{\pgfqpoint{5.738098in}{0.654230in}}{\pgfqpoint{5.727499in}{0.658620in}}{\pgfqpoint{5.716449in}{0.658620in}}%
\pgfpathcurveto{\pgfqpoint{5.705399in}{0.658620in}}{\pgfqpoint{5.694800in}{0.654230in}}{\pgfqpoint{5.686986in}{0.646416in}}%
\pgfpathcurveto{\pgfqpoint{5.679173in}{0.638603in}}{\pgfqpoint{5.674782in}{0.628004in}}{\pgfqpoint{5.674782in}{0.616953in}}%
\pgfpathcurveto{\pgfqpoint{5.674782in}{0.605903in}}{\pgfqpoint{5.679173in}{0.595304in}}{\pgfqpoint{5.686986in}{0.587491in}}%
\pgfpathcurveto{\pgfqpoint{5.694800in}{0.579677in}}{\pgfqpoint{5.705399in}{0.575287in}}{\pgfqpoint{5.716449in}{0.575287in}}%
\pgfpathlineto{\pgfqpoint{5.716449in}{0.575287in}}%
\pgfpathclose%
\pgfusepath{stroke}%
\end{pgfscope}%
\begin{pgfscope}%
\pgfpathrectangle{\pgfqpoint{0.847223in}{0.554012in}}{\pgfqpoint{6.200000in}{4.530000in}}%
\pgfusepath{clip}%
\pgfsetbuttcap%
\pgfsetroundjoin%
\pgfsetlinewidth{1.003750pt}%
\definecolor{currentstroke}{rgb}{1.000000,0.000000,0.000000}%
\pgfsetstrokecolor{currentstroke}%
\pgfsetdash{}{0pt}%
\pgfpathmoveto{\pgfqpoint{5.721782in}{0.574812in}}%
\pgfpathcurveto{\pgfqpoint{5.732832in}{0.574812in}}{\pgfqpoint{5.743431in}{0.579203in}}{\pgfqpoint{5.751245in}{0.587016in}}%
\pgfpathcurveto{\pgfqpoint{5.759059in}{0.594830in}}{\pgfqpoint{5.763449in}{0.605429in}}{\pgfqpoint{5.763449in}{0.616479in}}%
\pgfpathcurveto{\pgfqpoint{5.763449in}{0.627529in}}{\pgfqpoint{5.759059in}{0.638128in}}{\pgfqpoint{5.751245in}{0.645942in}}%
\pgfpathcurveto{\pgfqpoint{5.743431in}{0.653755in}}{\pgfqpoint{5.732832in}{0.658146in}}{\pgfqpoint{5.721782in}{0.658146in}}%
\pgfpathcurveto{\pgfqpoint{5.710732in}{0.658146in}}{\pgfqpoint{5.700133in}{0.653755in}}{\pgfqpoint{5.692319in}{0.645942in}}%
\pgfpathcurveto{\pgfqpoint{5.684506in}{0.638128in}}{\pgfqpoint{5.680116in}{0.627529in}}{\pgfqpoint{5.680116in}{0.616479in}}%
\pgfpathcurveto{\pgfqpoint{5.680116in}{0.605429in}}{\pgfqpoint{5.684506in}{0.594830in}}{\pgfqpoint{5.692319in}{0.587016in}}%
\pgfpathcurveto{\pgfqpoint{5.700133in}{0.579203in}}{\pgfqpoint{5.710732in}{0.574812in}}{\pgfqpoint{5.721782in}{0.574812in}}%
\pgfpathlineto{\pgfqpoint{5.721782in}{0.574812in}}%
\pgfpathclose%
\pgfusepath{stroke}%
\end{pgfscope}%
\begin{pgfscope}%
\pgfpathrectangle{\pgfqpoint{0.847223in}{0.554012in}}{\pgfqpoint{6.200000in}{4.530000in}}%
\pgfusepath{clip}%
\pgfsetbuttcap%
\pgfsetroundjoin%
\pgfsetlinewidth{1.003750pt}%
\definecolor{currentstroke}{rgb}{1.000000,0.000000,0.000000}%
\pgfsetstrokecolor{currentstroke}%
\pgfsetdash{}{0pt}%
\pgfpathmoveto{\pgfqpoint{5.727115in}{0.574339in}}%
\pgfpathcurveto{\pgfqpoint{5.738166in}{0.574339in}}{\pgfqpoint{5.748765in}{0.578729in}}{\pgfqpoint{5.756578in}{0.586543in}}%
\pgfpathcurveto{\pgfqpoint{5.764392in}{0.594356in}}{\pgfqpoint{5.768782in}{0.604955in}}{\pgfqpoint{5.768782in}{0.616006in}}%
\pgfpathcurveto{\pgfqpoint{5.768782in}{0.627056in}}{\pgfqpoint{5.764392in}{0.637655in}}{\pgfqpoint{5.756578in}{0.645468in}}%
\pgfpathcurveto{\pgfqpoint{5.748765in}{0.653282in}}{\pgfqpoint{5.738166in}{0.657672in}}{\pgfqpoint{5.727115in}{0.657672in}}%
\pgfpathcurveto{\pgfqpoint{5.716065in}{0.657672in}}{\pgfqpoint{5.705466in}{0.653282in}}{\pgfqpoint{5.697653in}{0.645468in}}%
\pgfpathcurveto{\pgfqpoint{5.689839in}{0.637655in}}{\pgfqpoint{5.685449in}{0.627056in}}{\pgfqpoint{5.685449in}{0.616006in}}%
\pgfpathcurveto{\pgfqpoint{5.685449in}{0.604955in}}{\pgfqpoint{5.689839in}{0.594356in}}{\pgfqpoint{5.697653in}{0.586543in}}%
\pgfpathcurveto{\pgfqpoint{5.705466in}{0.578729in}}{\pgfqpoint{5.716065in}{0.574339in}}{\pgfqpoint{5.727115in}{0.574339in}}%
\pgfpathlineto{\pgfqpoint{5.727115in}{0.574339in}}%
\pgfpathclose%
\pgfusepath{stroke}%
\end{pgfscope}%
\begin{pgfscope}%
\pgfpathrectangle{\pgfqpoint{0.847223in}{0.554012in}}{\pgfqpoint{6.200000in}{4.530000in}}%
\pgfusepath{clip}%
\pgfsetbuttcap%
\pgfsetroundjoin%
\pgfsetlinewidth{1.003750pt}%
\definecolor{currentstroke}{rgb}{1.000000,0.000000,0.000000}%
\pgfsetstrokecolor{currentstroke}%
\pgfsetdash{}{0pt}%
\pgfpathmoveto{\pgfqpoint{5.732449in}{0.573866in}}%
\pgfpathcurveto{\pgfqpoint{5.743499in}{0.573866in}}{\pgfqpoint{5.754098in}{0.578257in}}{\pgfqpoint{5.761911in}{0.586070in}}%
\pgfpathcurveto{\pgfqpoint{5.769725in}{0.593884in}}{\pgfqpoint{5.774115in}{0.604483in}}{\pgfqpoint{5.774115in}{0.615533in}}%
\pgfpathcurveto{\pgfqpoint{5.774115in}{0.626583in}}{\pgfqpoint{5.769725in}{0.637182in}}{\pgfqpoint{5.761911in}{0.644996in}}%
\pgfpathcurveto{\pgfqpoint{5.754098in}{0.652810in}}{\pgfqpoint{5.743499in}{0.657200in}}{\pgfqpoint{5.732449in}{0.657200in}}%
\pgfpathcurveto{\pgfqpoint{5.721398in}{0.657200in}}{\pgfqpoint{5.710799in}{0.652810in}}{\pgfqpoint{5.702986in}{0.644996in}}%
\pgfpathcurveto{\pgfqpoint{5.695172in}{0.637182in}}{\pgfqpoint{5.690782in}{0.626583in}}{\pgfqpoint{5.690782in}{0.615533in}}%
\pgfpathcurveto{\pgfqpoint{5.690782in}{0.604483in}}{\pgfqpoint{5.695172in}{0.593884in}}{\pgfqpoint{5.702986in}{0.586070in}}%
\pgfpathcurveto{\pgfqpoint{5.710799in}{0.578257in}}{\pgfqpoint{5.721398in}{0.573866in}}{\pgfqpoint{5.732449in}{0.573866in}}%
\pgfpathlineto{\pgfqpoint{5.732449in}{0.573866in}}%
\pgfpathclose%
\pgfusepath{stroke}%
\end{pgfscope}%
\begin{pgfscope}%
\pgfpathrectangle{\pgfqpoint{0.847223in}{0.554012in}}{\pgfqpoint{6.200000in}{4.530000in}}%
\pgfusepath{clip}%
\pgfsetbuttcap%
\pgfsetroundjoin%
\pgfsetlinewidth{1.003750pt}%
\definecolor{currentstroke}{rgb}{1.000000,0.000000,0.000000}%
\pgfsetstrokecolor{currentstroke}%
\pgfsetdash{}{0pt}%
\pgfpathmoveto{\pgfqpoint{5.737782in}{0.573395in}}%
\pgfpathcurveto{\pgfqpoint{5.748832in}{0.573395in}}{\pgfqpoint{5.759431in}{0.577785in}}{\pgfqpoint{5.767245in}{0.585599in}}%
\pgfpathcurveto{\pgfqpoint{5.775058in}{0.593412in}}{\pgfqpoint{5.779448in}{0.604011in}}{\pgfqpoint{5.779448in}{0.615062in}}%
\pgfpathcurveto{\pgfqpoint{5.779448in}{0.626112in}}{\pgfqpoint{5.775058in}{0.636711in}}{\pgfqpoint{5.767245in}{0.644524in}}%
\pgfpathcurveto{\pgfqpoint{5.759431in}{0.652338in}}{\pgfqpoint{5.748832in}{0.656728in}}{\pgfqpoint{5.737782in}{0.656728in}}%
\pgfpathcurveto{\pgfqpoint{5.726732in}{0.656728in}}{\pgfqpoint{5.716133in}{0.652338in}}{\pgfqpoint{5.708319in}{0.644524in}}%
\pgfpathcurveto{\pgfqpoint{5.700505in}{0.636711in}}{\pgfqpoint{5.696115in}{0.626112in}}{\pgfqpoint{5.696115in}{0.615062in}}%
\pgfpathcurveto{\pgfqpoint{5.696115in}{0.604011in}}{\pgfqpoint{5.700505in}{0.593412in}}{\pgfqpoint{5.708319in}{0.585599in}}%
\pgfpathcurveto{\pgfqpoint{5.716133in}{0.577785in}}{\pgfqpoint{5.726732in}{0.573395in}}{\pgfqpoint{5.737782in}{0.573395in}}%
\pgfpathlineto{\pgfqpoint{5.737782in}{0.573395in}}%
\pgfpathclose%
\pgfusepath{stroke}%
\end{pgfscope}%
\begin{pgfscope}%
\pgfpathrectangle{\pgfqpoint{0.847223in}{0.554012in}}{\pgfqpoint{6.200000in}{4.530000in}}%
\pgfusepath{clip}%
\pgfsetbuttcap%
\pgfsetroundjoin%
\pgfsetlinewidth{1.003750pt}%
\definecolor{currentstroke}{rgb}{1.000000,0.000000,0.000000}%
\pgfsetstrokecolor{currentstroke}%
\pgfsetdash{}{0pt}%
\pgfpathmoveto{\pgfqpoint{5.743115in}{0.572924in}}%
\pgfpathcurveto{\pgfqpoint{5.754165in}{0.572924in}}{\pgfqpoint{5.764764in}{0.577314in}}{\pgfqpoint{5.772578in}{0.585128in}}%
\pgfpathcurveto{\pgfqpoint{5.780391in}{0.592942in}}{\pgfqpoint{5.784782in}{0.603541in}}{\pgfqpoint{5.784782in}{0.614591in}}%
\pgfpathcurveto{\pgfqpoint{5.784782in}{0.625641in}}{\pgfqpoint{5.780391in}{0.636240in}}{\pgfqpoint{5.772578in}{0.644054in}}%
\pgfpathcurveto{\pgfqpoint{5.764764in}{0.651867in}}{\pgfqpoint{5.754165in}{0.656258in}}{\pgfqpoint{5.743115in}{0.656258in}}%
\pgfpathcurveto{\pgfqpoint{5.732065in}{0.656258in}}{\pgfqpoint{5.721466in}{0.651867in}}{\pgfqpoint{5.713652in}{0.644054in}}%
\pgfpathcurveto{\pgfqpoint{5.705839in}{0.636240in}}{\pgfqpoint{5.701448in}{0.625641in}}{\pgfqpoint{5.701448in}{0.614591in}}%
\pgfpathcurveto{\pgfqpoint{5.701448in}{0.603541in}}{\pgfqpoint{5.705839in}{0.592942in}}{\pgfqpoint{5.713652in}{0.585128in}}%
\pgfpathcurveto{\pgfqpoint{5.721466in}{0.577314in}}{\pgfqpoint{5.732065in}{0.572924in}}{\pgfqpoint{5.743115in}{0.572924in}}%
\pgfpathlineto{\pgfqpoint{5.743115in}{0.572924in}}%
\pgfpathclose%
\pgfusepath{stroke}%
\end{pgfscope}%
\begin{pgfscope}%
\pgfpathrectangle{\pgfqpoint{0.847223in}{0.554012in}}{\pgfqpoint{6.200000in}{4.530000in}}%
\pgfusepath{clip}%
\pgfsetbuttcap%
\pgfsetroundjoin%
\pgfsetlinewidth{1.003750pt}%
\definecolor{currentstroke}{rgb}{1.000000,0.000000,0.000000}%
\pgfsetstrokecolor{currentstroke}%
\pgfsetdash{}{0pt}%
\pgfpathmoveto{\pgfqpoint{5.748448in}{0.572455in}}%
\pgfpathcurveto{\pgfqpoint{5.759498in}{0.572455in}}{\pgfqpoint{5.770097in}{0.576845in}}{\pgfqpoint{5.777911in}{0.584658in}}%
\pgfpathcurveto{\pgfqpoint{5.785725in}{0.592472in}}{\pgfqpoint{5.790115in}{0.603071in}}{\pgfqpoint{5.790115in}{0.614121in}}%
\pgfpathcurveto{\pgfqpoint{5.790115in}{0.625171in}}{\pgfqpoint{5.785725in}{0.635770in}}{\pgfqpoint{5.777911in}{0.643584in}}%
\pgfpathcurveto{\pgfqpoint{5.770097in}{0.651398in}}{\pgfqpoint{5.759498in}{0.655788in}}{\pgfqpoint{5.748448in}{0.655788in}}%
\pgfpathcurveto{\pgfqpoint{5.737398in}{0.655788in}}{\pgfqpoint{5.726799in}{0.651398in}}{\pgfqpoint{5.718985in}{0.643584in}}%
\pgfpathcurveto{\pgfqpoint{5.711172in}{0.635770in}}{\pgfqpoint{5.706782in}{0.625171in}}{\pgfqpoint{5.706782in}{0.614121in}}%
\pgfpathcurveto{\pgfqpoint{5.706782in}{0.603071in}}{\pgfqpoint{5.711172in}{0.592472in}}{\pgfqpoint{5.718985in}{0.584658in}}%
\pgfpathcurveto{\pgfqpoint{5.726799in}{0.576845in}}{\pgfqpoint{5.737398in}{0.572455in}}{\pgfqpoint{5.748448in}{0.572455in}}%
\pgfpathlineto{\pgfqpoint{5.748448in}{0.572455in}}%
\pgfpathclose%
\pgfusepath{stroke}%
\end{pgfscope}%
\begin{pgfscope}%
\pgfpathrectangle{\pgfqpoint{0.847223in}{0.554012in}}{\pgfqpoint{6.200000in}{4.530000in}}%
\pgfusepath{clip}%
\pgfsetbuttcap%
\pgfsetroundjoin%
\pgfsetlinewidth{1.003750pt}%
\definecolor{currentstroke}{rgb}{1.000000,0.000000,0.000000}%
\pgfsetstrokecolor{currentstroke}%
\pgfsetdash{}{0pt}%
\pgfpathmoveto{\pgfqpoint{5.753781in}{0.571986in}}%
\pgfpathcurveto{\pgfqpoint{5.764832in}{0.571986in}}{\pgfqpoint{5.775431in}{0.576376in}}{\pgfqpoint{5.783244in}{0.584190in}}%
\pgfpathcurveto{\pgfqpoint{5.791058in}{0.592003in}}{\pgfqpoint{5.795448in}{0.602602in}}{\pgfqpoint{5.795448in}{0.613652in}}%
\pgfpathcurveto{\pgfqpoint{5.795448in}{0.624703in}}{\pgfqpoint{5.791058in}{0.635302in}}{\pgfqpoint{5.783244in}{0.643115in}}%
\pgfpathcurveto{\pgfqpoint{5.775431in}{0.650929in}}{\pgfqpoint{5.764832in}{0.655319in}}{\pgfqpoint{5.753781in}{0.655319in}}%
\pgfpathcurveto{\pgfqpoint{5.742731in}{0.655319in}}{\pgfqpoint{5.732132in}{0.650929in}}{\pgfqpoint{5.724319in}{0.643115in}}%
\pgfpathcurveto{\pgfqpoint{5.716505in}{0.635302in}}{\pgfqpoint{5.712115in}{0.624703in}}{\pgfqpoint{5.712115in}{0.613652in}}%
\pgfpathcurveto{\pgfqpoint{5.712115in}{0.602602in}}{\pgfqpoint{5.716505in}{0.592003in}}{\pgfqpoint{5.724319in}{0.584190in}}%
\pgfpathcurveto{\pgfqpoint{5.732132in}{0.576376in}}{\pgfqpoint{5.742731in}{0.571986in}}{\pgfqpoint{5.753781in}{0.571986in}}%
\pgfpathlineto{\pgfqpoint{5.753781in}{0.571986in}}%
\pgfpathclose%
\pgfusepath{stroke}%
\end{pgfscope}%
\begin{pgfscope}%
\pgfpathrectangle{\pgfqpoint{0.847223in}{0.554012in}}{\pgfqpoint{6.200000in}{4.530000in}}%
\pgfusepath{clip}%
\pgfsetbuttcap%
\pgfsetroundjoin%
\pgfsetlinewidth{1.003750pt}%
\definecolor{currentstroke}{rgb}{1.000000,0.000000,0.000000}%
\pgfsetstrokecolor{currentstroke}%
\pgfsetdash{}{0pt}%
\pgfpathmoveto{\pgfqpoint{5.759115in}{0.571518in}}%
\pgfpathcurveto{\pgfqpoint{5.770165in}{0.571518in}}{\pgfqpoint{5.780764in}{0.575908in}}{\pgfqpoint{5.788577in}{0.583722in}}%
\pgfpathcurveto{\pgfqpoint{5.796391in}{0.591535in}}{\pgfqpoint{5.800781in}{0.602134in}}{\pgfqpoint{5.800781in}{0.613185in}}%
\pgfpathcurveto{\pgfqpoint{5.800781in}{0.624235in}}{\pgfqpoint{5.796391in}{0.634834in}}{\pgfqpoint{5.788577in}{0.642647in}}%
\pgfpathcurveto{\pgfqpoint{5.780764in}{0.650461in}}{\pgfqpoint{5.770165in}{0.654851in}}{\pgfqpoint{5.759115in}{0.654851in}}%
\pgfpathcurveto{\pgfqpoint{5.748065in}{0.654851in}}{\pgfqpoint{5.737466in}{0.650461in}}{\pgfqpoint{5.729652in}{0.642647in}}%
\pgfpathcurveto{\pgfqpoint{5.721838in}{0.634834in}}{\pgfqpoint{5.717448in}{0.624235in}}{\pgfqpoint{5.717448in}{0.613185in}}%
\pgfpathcurveto{\pgfqpoint{5.717448in}{0.602134in}}{\pgfqpoint{5.721838in}{0.591535in}}{\pgfqpoint{5.729652in}{0.583722in}}%
\pgfpathcurveto{\pgfqpoint{5.737466in}{0.575908in}}{\pgfqpoint{5.748065in}{0.571518in}}{\pgfqpoint{5.759115in}{0.571518in}}%
\pgfpathlineto{\pgfqpoint{5.759115in}{0.571518in}}%
\pgfpathclose%
\pgfusepath{stroke}%
\end{pgfscope}%
\begin{pgfscope}%
\pgfpathrectangle{\pgfqpoint{0.847223in}{0.554012in}}{\pgfqpoint{6.200000in}{4.530000in}}%
\pgfusepath{clip}%
\pgfsetbuttcap%
\pgfsetroundjoin%
\pgfsetlinewidth{1.003750pt}%
\definecolor{currentstroke}{rgb}{1.000000,0.000000,0.000000}%
\pgfsetstrokecolor{currentstroke}%
\pgfsetdash{}{0pt}%
\pgfpathmoveto{\pgfqpoint{5.764448in}{0.571051in}}%
\pgfpathcurveto{\pgfqpoint{5.775498in}{0.571051in}}{\pgfqpoint{5.786097in}{0.575441in}}{\pgfqpoint{5.793911in}{0.583255in}}%
\pgfpathcurveto{\pgfqpoint{5.801724in}{0.591068in}}{\pgfqpoint{5.806115in}{0.601667in}}{\pgfqpoint{5.806115in}{0.612718in}}%
\pgfpathcurveto{\pgfqpoint{5.806115in}{0.623768in}}{\pgfqpoint{5.801724in}{0.634367in}}{\pgfqpoint{5.793911in}{0.642180in}}%
\pgfpathcurveto{\pgfqpoint{5.786097in}{0.649994in}}{\pgfqpoint{5.775498in}{0.654384in}}{\pgfqpoint{5.764448in}{0.654384in}}%
\pgfpathcurveto{\pgfqpoint{5.753398in}{0.654384in}}{\pgfqpoint{5.742799in}{0.649994in}}{\pgfqpoint{5.734985in}{0.642180in}}%
\pgfpathcurveto{\pgfqpoint{5.727171in}{0.634367in}}{\pgfqpoint{5.722781in}{0.623768in}}{\pgfqpoint{5.722781in}{0.612718in}}%
\pgfpathcurveto{\pgfqpoint{5.722781in}{0.601667in}}{\pgfqpoint{5.727171in}{0.591068in}}{\pgfqpoint{5.734985in}{0.583255in}}%
\pgfpathcurveto{\pgfqpoint{5.742799in}{0.575441in}}{\pgfqpoint{5.753398in}{0.571051in}}{\pgfqpoint{5.764448in}{0.571051in}}%
\pgfpathlineto{\pgfqpoint{5.764448in}{0.571051in}}%
\pgfpathclose%
\pgfusepath{stroke}%
\end{pgfscope}%
\begin{pgfscope}%
\pgfpathrectangle{\pgfqpoint{0.847223in}{0.554012in}}{\pgfqpoint{6.200000in}{4.530000in}}%
\pgfusepath{clip}%
\pgfsetbuttcap%
\pgfsetroundjoin%
\pgfsetlinewidth{1.003750pt}%
\definecolor{currentstroke}{rgb}{1.000000,0.000000,0.000000}%
\pgfsetstrokecolor{currentstroke}%
\pgfsetdash{}{0pt}%
\pgfpathmoveto{\pgfqpoint{5.769781in}{0.570585in}}%
\pgfpathcurveto{\pgfqpoint{5.780831in}{0.570585in}}{\pgfqpoint{5.791430in}{0.574975in}}{\pgfqpoint{5.799244in}{0.582789in}}%
\pgfpathcurveto{\pgfqpoint{5.807058in}{0.590602in}}{\pgfqpoint{5.811448in}{0.601201in}}{\pgfqpoint{5.811448in}{0.612252in}}%
\pgfpathcurveto{\pgfqpoint{5.811448in}{0.623302in}}{\pgfqpoint{5.807058in}{0.633901in}}{\pgfqpoint{5.799244in}{0.641714in}}%
\pgfpathcurveto{\pgfqpoint{5.791430in}{0.649528in}}{\pgfqpoint{5.780831in}{0.653918in}}{\pgfqpoint{5.769781in}{0.653918in}}%
\pgfpathcurveto{\pgfqpoint{5.758731in}{0.653918in}}{\pgfqpoint{5.748132in}{0.649528in}}{\pgfqpoint{5.740318in}{0.641714in}}%
\pgfpathcurveto{\pgfqpoint{5.732505in}{0.633901in}}{\pgfqpoint{5.728114in}{0.623302in}}{\pgfqpoint{5.728114in}{0.612252in}}%
\pgfpathcurveto{\pgfqpoint{5.728114in}{0.601201in}}{\pgfqpoint{5.732505in}{0.590602in}}{\pgfqpoint{5.740318in}{0.582789in}}%
\pgfpathcurveto{\pgfqpoint{5.748132in}{0.574975in}}{\pgfqpoint{5.758731in}{0.570585in}}{\pgfqpoint{5.769781in}{0.570585in}}%
\pgfpathlineto{\pgfqpoint{5.769781in}{0.570585in}}%
\pgfpathclose%
\pgfusepath{stroke}%
\end{pgfscope}%
\begin{pgfscope}%
\pgfpathrectangle{\pgfqpoint{0.847223in}{0.554012in}}{\pgfqpoint{6.200000in}{4.530000in}}%
\pgfusepath{clip}%
\pgfsetbuttcap%
\pgfsetroundjoin%
\pgfsetlinewidth{1.003750pt}%
\definecolor{currentstroke}{rgb}{1.000000,0.000000,0.000000}%
\pgfsetstrokecolor{currentstroke}%
\pgfsetdash{}{0pt}%
\pgfpathmoveto{\pgfqpoint{5.775114in}{0.570120in}}%
\pgfpathcurveto{\pgfqpoint{5.786164in}{0.570120in}}{\pgfqpoint{5.796763in}{0.574510in}}{\pgfqpoint{5.804577in}{0.582324in}}%
\pgfpathcurveto{\pgfqpoint{5.812391in}{0.590137in}}{\pgfqpoint{5.816781in}{0.600736in}}{\pgfqpoint{5.816781in}{0.611786in}}%
\pgfpathcurveto{\pgfqpoint{5.816781in}{0.622837in}}{\pgfqpoint{5.812391in}{0.633436in}}{\pgfqpoint{5.804577in}{0.641249in}}%
\pgfpathcurveto{\pgfqpoint{5.796763in}{0.649063in}}{\pgfqpoint{5.786164in}{0.653453in}}{\pgfqpoint{5.775114in}{0.653453in}}%
\pgfpathcurveto{\pgfqpoint{5.764064in}{0.653453in}}{\pgfqpoint{5.753465in}{0.649063in}}{\pgfqpoint{5.745652in}{0.641249in}}%
\pgfpathcurveto{\pgfqpoint{5.737838in}{0.633436in}}{\pgfqpoint{5.733448in}{0.622837in}}{\pgfqpoint{5.733448in}{0.611786in}}%
\pgfpathcurveto{\pgfqpoint{5.733448in}{0.600736in}}{\pgfqpoint{5.737838in}{0.590137in}}{\pgfqpoint{5.745652in}{0.582324in}}%
\pgfpathcurveto{\pgfqpoint{5.753465in}{0.574510in}}{\pgfqpoint{5.764064in}{0.570120in}}{\pgfqpoint{5.775114in}{0.570120in}}%
\pgfpathlineto{\pgfqpoint{5.775114in}{0.570120in}}%
\pgfpathclose%
\pgfusepath{stroke}%
\end{pgfscope}%
\begin{pgfscope}%
\pgfpathrectangle{\pgfqpoint{0.847223in}{0.554012in}}{\pgfqpoint{6.200000in}{4.530000in}}%
\pgfusepath{clip}%
\pgfsetbuttcap%
\pgfsetroundjoin%
\pgfsetlinewidth{1.003750pt}%
\definecolor{currentstroke}{rgb}{1.000000,0.000000,0.000000}%
\pgfsetstrokecolor{currentstroke}%
\pgfsetdash{}{0pt}%
\pgfpathmoveto{\pgfqpoint{5.780448in}{0.569655in}}%
\pgfpathcurveto{\pgfqpoint{5.791498in}{0.569655in}}{\pgfqpoint{5.802097in}{0.574046in}}{\pgfqpoint{5.809910in}{0.581859in}}%
\pgfpathcurveto{\pgfqpoint{5.817724in}{0.589673in}}{\pgfqpoint{5.822114in}{0.600272in}}{\pgfqpoint{5.822114in}{0.611322in}}%
\pgfpathcurveto{\pgfqpoint{5.822114in}{0.622372in}}{\pgfqpoint{5.817724in}{0.632971in}}{\pgfqpoint{5.809910in}{0.640785in}}%
\pgfpathcurveto{\pgfqpoint{5.802097in}{0.648599in}}{\pgfqpoint{5.791498in}{0.652989in}}{\pgfqpoint{5.780448in}{0.652989in}}%
\pgfpathcurveto{\pgfqpoint{5.769397in}{0.652989in}}{\pgfqpoint{5.758798in}{0.648599in}}{\pgfqpoint{5.750985in}{0.640785in}}%
\pgfpathcurveto{\pgfqpoint{5.743171in}{0.632971in}}{\pgfqpoint{5.738781in}{0.622372in}}{\pgfqpoint{5.738781in}{0.611322in}}%
\pgfpathcurveto{\pgfqpoint{5.738781in}{0.600272in}}{\pgfqpoint{5.743171in}{0.589673in}}{\pgfqpoint{5.750985in}{0.581859in}}%
\pgfpathcurveto{\pgfqpoint{5.758798in}{0.574046in}}{\pgfqpoint{5.769397in}{0.569655in}}{\pgfqpoint{5.780448in}{0.569655in}}%
\pgfpathlineto{\pgfqpoint{5.780448in}{0.569655in}}%
\pgfpathclose%
\pgfusepath{stroke}%
\end{pgfscope}%
\begin{pgfscope}%
\pgfpathrectangle{\pgfqpoint{0.847223in}{0.554012in}}{\pgfqpoint{6.200000in}{4.530000in}}%
\pgfusepath{clip}%
\pgfsetbuttcap%
\pgfsetroundjoin%
\pgfsetlinewidth{1.003750pt}%
\definecolor{currentstroke}{rgb}{1.000000,0.000000,0.000000}%
\pgfsetstrokecolor{currentstroke}%
\pgfsetdash{}{0pt}%
\pgfpathmoveto{\pgfqpoint{5.785781in}{0.569192in}}%
\pgfpathcurveto{\pgfqpoint{5.796831in}{0.569192in}}{\pgfqpoint{5.807430in}{0.573582in}}{\pgfqpoint{5.815244in}{0.581396in}}%
\pgfpathcurveto{\pgfqpoint{5.823057in}{0.589210in}}{\pgfqpoint{5.827447in}{0.599809in}}{\pgfqpoint{5.827447in}{0.610859in}}%
\pgfpathcurveto{\pgfqpoint{5.827447in}{0.621909in}}{\pgfqpoint{5.823057in}{0.632508in}}{\pgfqpoint{5.815244in}{0.640322in}}%
\pgfpathcurveto{\pgfqpoint{5.807430in}{0.648135in}}{\pgfqpoint{5.796831in}{0.652526in}}{\pgfqpoint{5.785781in}{0.652526in}}%
\pgfpathcurveto{\pgfqpoint{5.774731in}{0.652526in}}{\pgfqpoint{5.764132in}{0.648135in}}{\pgfqpoint{5.756318in}{0.640322in}}%
\pgfpathcurveto{\pgfqpoint{5.748504in}{0.632508in}}{\pgfqpoint{5.744114in}{0.621909in}}{\pgfqpoint{5.744114in}{0.610859in}}%
\pgfpathcurveto{\pgfqpoint{5.744114in}{0.599809in}}{\pgfqpoint{5.748504in}{0.589210in}}{\pgfqpoint{5.756318in}{0.581396in}}%
\pgfpathcurveto{\pgfqpoint{5.764132in}{0.573582in}}{\pgfqpoint{5.774731in}{0.569192in}}{\pgfqpoint{5.785781in}{0.569192in}}%
\pgfpathlineto{\pgfqpoint{5.785781in}{0.569192in}}%
\pgfpathclose%
\pgfusepath{stroke}%
\end{pgfscope}%
\begin{pgfscope}%
\pgfpathrectangle{\pgfqpoint{0.847223in}{0.554012in}}{\pgfqpoint{6.200000in}{4.530000in}}%
\pgfusepath{clip}%
\pgfsetbuttcap%
\pgfsetroundjoin%
\pgfsetlinewidth{1.003750pt}%
\definecolor{currentstroke}{rgb}{1.000000,0.000000,0.000000}%
\pgfsetstrokecolor{currentstroke}%
\pgfsetdash{}{0pt}%
\pgfpathmoveto{\pgfqpoint{5.791114in}{0.568730in}}%
\pgfpathcurveto{\pgfqpoint{5.802164in}{0.568730in}}{\pgfqpoint{5.812763in}{0.573120in}}{\pgfqpoint{5.820577in}{0.580934in}}%
\pgfpathcurveto{\pgfqpoint{5.828390in}{0.588747in}}{\pgfqpoint{5.832781in}{0.599346in}}{\pgfqpoint{5.832781in}{0.610396in}}%
\pgfpathcurveto{\pgfqpoint{5.832781in}{0.621447in}}{\pgfqpoint{5.828390in}{0.632046in}}{\pgfqpoint{5.820577in}{0.639859in}}%
\pgfpathcurveto{\pgfqpoint{5.812763in}{0.647673in}}{\pgfqpoint{5.802164in}{0.652063in}}{\pgfqpoint{5.791114in}{0.652063in}}%
\pgfpathcurveto{\pgfqpoint{5.780064in}{0.652063in}}{\pgfqpoint{5.769465in}{0.647673in}}{\pgfqpoint{5.761651in}{0.639859in}}%
\pgfpathcurveto{\pgfqpoint{5.753838in}{0.632046in}}{\pgfqpoint{5.749447in}{0.621447in}}{\pgfqpoint{5.749447in}{0.610396in}}%
\pgfpathcurveto{\pgfqpoint{5.749447in}{0.599346in}}{\pgfqpoint{5.753838in}{0.588747in}}{\pgfqpoint{5.761651in}{0.580934in}}%
\pgfpathcurveto{\pgfqpoint{5.769465in}{0.573120in}}{\pgfqpoint{5.780064in}{0.568730in}}{\pgfqpoint{5.791114in}{0.568730in}}%
\pgfpathlineto{\pgfqpoint{5.791114in}{0.568730in}}%
\pgfpathclose%
\pgfusepath{stroke}%
\end{pgfscope}%
\begin{pgfscope}%
\pgfpathrectangle{\pgfqpoint{0.847223in}{0.554012in}}{\pgfqpoint{6.200000in}{4.530000in}}%
\pgfusepath{clip}%
\pgfsetbuttcap%
\pgfsetroundjoin%
\pgfsetlinewidth{1.003750pt}%
\definecolor{currentstroke}{rgb}{1.000000,0.000000,0.000000}%
\pgfsetstrokecolor{currentstroke}%
\pgfsetdash{}{0pt}%
\pgfpathmoveto{\pgfqpoint{5.796447in}{0.568268in}}%
\pgfpathcurveto{\pgfqpoint{5.807497in}{0.568268in}}{\pgfqpoint{5.818096in}{0.572659in}}{\pgfqpoint{5.825910in}{0.580472in}}%
\pgfpathcurveto{\pgfqpoint{5.833724in}{0.588286in}}{\pgfqpoint{5.838114in}{0.598885in}}{\pgfqpoint{5.838114in}{0.609935in}}%
\pgfpathcurveto{\pgfqpoint{5.838114in}{0.620985in}}{\pgfqpoint{5.833724in}{0.631584in}}{\pgfqpoint{5.825910in}{0.639398in}}%
\pgfpathcurveto{\pgfqpoint{5.818096in}{0.647211in}}{\pgfqpoint{5.807497in}{0.651602in}}{\pgfqpoint{5.796447in}{0.651602in}}%
\pgfpathcurveto{\pgfqpoint{5.785397in}{0.651602in}}{\pgfqpoint{5.774798in}{0.647211in}}{\pgfqpoint{5.766984in}{0.639398in}}%
\pgfpathcurveto{\pgfqpoint{5.759171in}{0.631584in}}{\pgfqpoint{5.754781in}{0.620985in}}{\pgfqpoint{5.754781in}{0.609935in}}%
\pgfpathcurveto{\pgfqpoint{5.754781in}{0.598885in}}{\pgfqpoint{5.759171in}{0.588286in}}{\pgfqpoint{5.766984in}{0.580472in}}%
\pgfpathcurveto{\pgfqpoint{5.774798in}{0.572659in}}{\pgfqpoint{5.785397in}{0.568268in}}{\pgfqpoint{5.796447in}{0.568268in}}%
\pgfpathlineto{\pgfqpoint{5.796447in}{0.568268in}}%
\pgfpathclose%
\pgfusepath{stroke}%
\end{pgfscope}%
\begin{pgfscope}%
\pgfpathrectangle{\pgfqpoint{0.847223in}{0.554012in}}{\pgfqpoint{6.200000in}{4.530000in}}%
\pgfusepath{clip}%
\pgfsetbuttcap%
\pgfsetroundjoin%
\pgfsetlinewidth{1.003750pt}%
\definecolor{currentstroke}{rgb}{1.000000,0.000000,0.000000}%
\pgfsetstrokecolor{currentstroke}%
\pgfsetdash{}{0pt}%
\pgfpathmoveto{\pgfqpoint{5.801780in}{0.567808in}}%
\pgfpathcurveto{\pgfqpoint{5.812831in}{0.567808in}}{\pgfqpoint{5.823430in}{0.572198in}}{\pgfqpoint{5.831243in}{0.580012in}}%
\pgfpathcurveto{\pgfqpoint{5.839057in}{0.587825in}}{\pgfqpoint{5.843447in}{0.598424in}}{\pgfqpoint{5.843447in}{0.609474in}}%
\pgfpathcurveto{\pgfqpoint{5.843447in}{0.620524in}}{\pgfqpoint{5.839057in}{0.631123in}}{\pgfqpoint{5.831243in}{0.638937in}}%
\pgfpathcurveto{\pgfqpoint{5.823430in}{0.646751in}}{\pgfqpoint{5.812831in}{0.651141in}}{\pgfqpoint{5.801780in}{0.651141in}}%
\pgfpathcurveto{\pgfqpoint{5.790730in}{0.651141in}}{\pgfqpoint{5.780131in}{0.646751in}}{\pgfqpoint{5.772318in}{0.638937in}}%
\pgfpathcurveto{\pgfqpoint{5.764504in}{0.631123in}}{\pgfqpoint{5.760114in}{0.620524in}}{\pgfqpoint{5.760114in}{0.609474in}}%
\pgfpathcurveto{\pgfqpoint{5.760114in}{0.598424in}}{\pgfqpoint{5.764504in}{0.587825in}}{\pgfqpoint{5.772318in}{0.580012in}}%
\pgfpathcurveto{\pgfqpoint{5.780131in}{0.572198in}}{\pgfqpoint{5.790730in}{0.567808in}}{\pgfqpoint{5.801780in}{0.567808in}}%
\pgfpathlineto{\pgfqpoint{5.801780in}{0.567808in}}%
\pgfpathclose%
\pgfusepath{stroke}%
\end{pgfscope}%
\begin{pgfscope}%
\pgfpathrectangle{\pgfqpoint{0.847223in}{0.554012in}}{\pgfqpoint{6.200000in}{4.530000in}}%
\pgfusepath{clip}%
\pgfsetbuttcap%
\pgfsetroundjoin%
\pgfsetlinewidth{1.003750pt}%
\definecolor{currentstroke}{rgb}{1.000000,0.000000,0.000000}%
\pgfsetstrokecolor{currentstroke}%
\pgfsetdash{}{0pt}%
\pgfpathmoveto{\pgfqpoint{5.807114in}{0.567348in}}%
\pgfpathcurveto{\pgfqpoint{5.818164in}{0.567348in}}{\pgfqpoint{5.828763in}{0.571738in}}{\pgfqpoint{5.836576in}{0.579552in}}%
\pgfpathcurveto{\pgfqpoint{5.844390in}{0.587365in}}{\pgfqpoint{5.848780in}{0.597964in}}{\pgfqpoint{5.848780in}{0.609015in}}%
\pgfpathcurveto{\pgfqpoint{5.848780in}{0.620065in}}{\pgfqpoint{5.844390in}{0.630664in}}{\pgfqpoint{5.836576in}{0.638477in}}%
\pgfpathcurveto{\pgfqpoint{5.828763in}{0.646291in}}{\pgfqpoint{5.818164in}{0.650681in}}{\pgfqpoint{5.807114in}{0.650681in}}%
\pgfpathcurveto{\pgfqpoint{5.796063in}{0.650681in}}{\pgfqpoint{5.785464in}{0.646291in}}{\pgfqpoint{5.777651in}{0.638477in}}%
\pgfpathcurveto{\pgfqpoint{5.769837in}{0.630664in}}{\pgfqpoint{5.765447in}{0.620065in}}{\pgfqpoint{5.765447in}{0.609015in}}%
\pgfpathcurveto{\pgfqpoint{5.765447in}{0.597964in}}{\pgfqpoint{5.769837in}{0.587365in}}{\pgfqpoint{5.777651in}{0.579552in}}%
\pgfpathcurveto{\pgfqpoint{5.785464in}{0.571738in}}{\pgfqpoint{5.796063in}{0.567348in}}{\pgfqpoint{5.807114in}{0.567348in}}%
\pgfpathlineto{\pgfqpoint{5.807114in}{0.567348in}}%
\pgfpathclose%
\pgfusepath{stroke}%
\end{pgfscope}%
\begin{pgfscope}%
\pgfpathrectangle{\pgfqpoint{0.847223in}{0.554012in}}{\pgfqpoint{6.200000in}{4.530000in}}%
\pgfusepath{clip}%
\pgfsetbuttcap%
\pgfsetroundjoin%
\pgfsetlinewidth{1.003750pt}%
\definecolor{currentstroke}{rgb}{1.000000,0.000000,0.000000}%
\pgfsetstrokecolor{currentstroke}%
\pgfsetdash{}{0pt}%
\pgfpathmoveto{\pgfqpoint{5.812447in}{0.566889in}}%
\pgfpathcurveto{\pgfqpoint{5.823497in}{0.566889in}}{\pgfqpoint{5.834096in}{0.571279in}}{\pgfqpoint{5.841910in}{0.579093in}}%
\pgfpathcurveto{\pgfqpoint{5.849723in}{0.586907in}}{\pgfqpoint{5.854114in}{0.597506in}}{\pgfqpoint{5.854114in}{0.608556in}}%
\pgfpathcurveto{\pgfqpoint{5.854114in}{0.619606in}}{\pgfqpoint{5.849723in}{0.630205in}}{\pgfqpoint{5.841910in}{0.638019in}}%
\pgfpathcurveto{\pgfqpoint{5.834096in}{0.645832in}}{\pgfqpoint{5.823497in}{0.650222in}}{\pgfqpoint{5.812447in}{0.650222in}}%
\pgfpathcurveto{\pgfqpoint{5.801397in}{0.650222in}}{\pgfqpoint{5.790798in}{0.645832in}}{\pgfqpoint{5.782984in}{0.638019in}}%
\pgfpathcurveto{\pgfqpoint{5.775170in}{0.630205in}}{\pgfqpoint{5.770780in}{0.619606in}}{\pgfqpoint{5.770780in}{0.608556in}}%
\pgfpathcurveto{\pgfqpoint{5.770780in}{0.597506in}}{\pgfqpoint{5.775170in}{0.586907in}}{\pgfqpoint{5.782984in}{0.579093in}}%
\pgfpathcurveto{\pgfqpoint{5.790798in}{0.571279in}}{\pgfqpoint{5.801397in}{0.566889in}}{\pgfqpoint{5.812447in}{0.566889in}}%
\pgfpathlineto{\pgfqpoint{5.812447in}{0.566889in}}%
\pgfpathclose%
\pgfusepath{stroke}%
\end{pgfscope}%
\begin{pgfscope}%
\pgfpathrectangle{\pgfqpoint{0.847223in}{0.554012in}}{\pgfqpoint{6.200000in}{4.530000in}}%
\pgfusepath{clip}%
\pgfsetbuttcap%
\pgfsetroundjoin%
\pgfsetlinewidth{1.003750pt}%
\definecolor{currentstroke}{rgb}{1.000000,0.000000,0.000000}%
\pgfsetstrokecolor{currentstroke}%
\pgfsetdash{}{0pt}%
\pgfpathmoveto{\pgfqpoint{5.817780in}{0.566431in}}%
\pgfpathcurveto{\pgfqpoint{5.828830in}{0.566431in}}{\pgfqpoint{5.839429in}{0.570821in}}{\pgfqpoint{5.847243in}{0.578635in}}%
\pgfpathcurveto{\pgfqpoint{5.855056in}{0.586449in}}{\pgfqpoint{5.859447in}{0.597048in}}{\pgfqpoint{5.859447in}{0.608098in}}%
\pgfpathcurveto{\pgfqpoint{5.859447in}{0.619148in}}{\pgfqpoint{5.855056in}{0.629747in}}{\pgfqpoint{5.847243in}{0.637561in}}%
\pgfpathcurveto{\pgfqpoint{5.839429in}{0.645374in}}{\pgfqpoint{5.828830in}{0.649765in}}{\pgfqpoint{5.817780in}{0.649765in}}%
\pgfpathcurveto{\pgfqpoint{5.806730in}{0.649765in}}{\pgfqpoint{5.796131in}{0.645374in}}{\pgfqpoint{5.788317in}{0.637561in}}%
\pgfpathcurveto{\pgfqpoint{5.780504in}{0.629747in}}{\pgfqpoint{5.776113in}{0.619148in}}{\pgfqpoint{5.776113in}{0.608098in}}%
\pgfpathcurveto{\pgfqpoint{5.776113in}{0.597048in}}{\pgfqpoint{5.780504in}{0.586449in}}{\pgfqpoint{5.788317in}{0.578635in}}%
\pgfpathcurveto{\pgfqpoint{5.796131in}{0.570821in}}{\pgfqpoint{5.806730in}{0.566431in}}{\pgfqpoint{5.817780in}{0.566431in}}%
\pgfpathlineto{\pgfqpoint{5.817780in}{0.566431in}}%
\pgfpathclose%
\pgfusepath{stroke}%
\end{pgfscope}%
\begin{pgfscope}%
\pgfpathrectangle{\pgfqpoint{0.847223in}{0.554012in}}{\pgfqpoint{6.200000in}{4.530000in}}%
\pgfusepath{clip}%
\pgfsetbuttcap%
\pgfsetroundjoin%
\pgfsetlinewidth{1.003750pt}%
\definecolor{currentstroke}{rgb}{1.000000,0.000000,0.000000}%
\pgfsetstrokecolor{currentstroke}%
\pgfsetdash{}{0pt}%
\pgfpathmoveto{\pgfqpoint{5.823113in}{0.565974in}}%
\pgfpathcurveto{\pgfqpoint{5.834163in}{0.565974in}}{\pgfqpoint{5.844762in}{0.570364in}}{\pgfqpoint{5.852576in}{0.578178in}}%
\pgfpathcurveto{\pgfqpoint{5.860390in}{0.585992in}}{\pgfqpoint{5.864780in}{0.596591in}}{\pgfqpoint{5.864780in}{0.607641in}}%
\pgfpathcurveto{\pgfqpoint{5.864780in}{0.618691in}}{\pgfqpoint{5.860390in}{0.629290in}}{\pgfqpoint{5.852576in}{0.637104in}}%
\pgfpathcurveto{\pgfqpoint{5.844762in}{0.644917in}}{\pgfqpoint{5.834163in}{0.649307in}}{\pgfqpoint{5.823113in}{0.649307in}}%
\pgfpathcurveto{\pgfqpoint{5.812063in}{0.649307in}}{\pgfqpoint{5.801464in}{0.644917in}}{\pgfqpoint{5.793650in}{0.637104in}}%
\pgfpathcurveto{\pgfqpoint{5.785837in}{0.629290in}}{\pgfqpoint{5.781447in}{0.618691in}}{\pgfqpoint{5.781447in}{0.607641in}}%
\pgfpathcurveto{\pgfqpoint{5.781447in}{0.596591in}}{\pgfqpoint{5.785837in}{0.585992in}}{\pgfqpoint{5.793650in}{0.578178in}}%
\pgfpathcurveto{\pgfqpoint{5.801464in}{0.570364in}}{\pgfqpoint{5.812063in}{0.565974in}}{\pgfqpoint{5.823113in}{0.565974in}}%
\pgfpathlineto{\pgfqpoint{5.823113in}{0.565974in}}%
\pgfpathclose%
\pgfusepath{stroke}%
\end{pgfscope}%
\begin{pgfscope}%
\pgfpathrectangle{\pgfqpoint{0.847223in}{0.554012in}}{\pgfqpoint{6.200000in}{4.530000in}}%
\pgfusepath{clip}%
\pgfsetbuttcap%
\pgfsetroundjoin%
\pgfsetlinewidth{1.003750pt}%
\definecolor{currentstroke}{rgb}{1.000000,0.000000,0.000000}%
\pgfsetstrokecolor{currentstroke}%
\pgfsetdash{}{0pt}%
\pgfpathmoveto{\pgfqpoint{5.828446in}{0.565518in}}%
\pgfpathcurveto{\pgfqpoint{5.839497in}{0.565518in}}{\pgfqpoint{5.850096in}{0.569908in}}{\pgfqpoint{5.857909in}{0.577722in}}%
\pgfpathcurveto{\pgfqpoint{5.865723in}{0.585536in}}{\pgfqpoint{5.870113in}{0.596135in}}{\pgfqpoint{5.870113in}{0.607185in}}%
\pgfpathcurveto{\pgfqpoint{5.870113in}{0.618235in}}{\pgfqpoint{5.865723in}{0.628834in}}{\pgfqpoint{5.857909in}{0.636647in}}%
\pgfpathcurveto{\pgfqpoint{5.850096in}{0.644461in}}{\pgfqpoint{5.839497in}{0.648851in}}{\pgfqpoint{5.828446in}{0.648851in}}%
\pgfpathcurveto{\pgfqpoint{5.817396in}{0.648851in}}{\pgfqpoint{5.806797in}{0.644461in}}{\pgfqpoint{5.798984in}{0.636647in}}%
\pgfpathcurveto{\pgfqpoint{5.791170in}{0.628834in}}{\pgfqpoint{5.786780in}{0.618235in}}{\pgfqpoint{5.786780in}{0.607185in}}%
\pgfpathcurveto{\pgfqpoint{5.786780in}{0.596135in}}{\pgfqpoint{5.791170in}{0.585536in}}{\pgfqpoint{5.798984in}{0.577722in}}%
\pgfpathcurveto{\pgfqpoint{5.806797in}{0.569908in}}{\pgfqpoint{5.817396in}{0.565518in}}{\pgfqpoint{5.828446in}{0.565518in}}%
\pgfpathlineto{\pgfqpoint{5.828446in}{0.565518in}}%
\pgfpathclose%
\pgfusepath{stroke}%
\end{pgfscope}%
\begin{pgfscope}%
\pgfpathrectangle{\pgfqpoint{0.847223in}{0.554012in}}{\pgfqpoint{6.200000in}{4.530000in}}%
\pgfusepath{clip}%
\pgfsetbuttcap%
\pgfsetroundjoin%
\pgfsetlinewidth{1.003750pt}%
\definecolor{currentstroke}{rgb}{1.000000,0.000000,0.000000}%
\pgfsetstrokecolor{currentstroke}%
\pgfsetdash{}{0pt}%
\pgfpathmoveto{\pgfqpoint{5.833780in}{0.565063in}}%
\pgfpathcurveto{\pgfqpoint{5.844830in}{0.565063in}}{\pgfqpoint{5.855429in}{0.569453in}}{\pgfqpoint{5.863242in}{0.577267in}}%
\pgfpathcurveto{\pgfqpoint{5.871056in}{0.585080in}}{\pgfqpoint{5.875446in}{0.595679in}}{\pgfqpoint{5.875446in}{0.606729in}}%
\pgfpathcurveto{\pgfqpoint{5.875446in}{0.617780in}}{\pgfqpoint{5.871056in}{0.628379in}}{\pgfqpoint{5.863242in}{0.636192in}}%
\pgfpathcurveto{\pgfqpoint{5.855429in}{0.644006in}}{\pgfqpoint{5.844830in}{0.648396in}}{\pgfqpoint{5.833780in}{0.648396in}}%
\pgfpathcurveto{\pgfqpoint{5.822730in}{0.648396in}}{\pgfqpoint{5.812131in}{0.644006in}}{\pgfqpoint{5.804317in}{0.636192in}}%
\pgfpathcurveto{\pgfqpoint{5.796503in}{0.628379in}}{\pgfqpoint{5.792113in}{0.617780in}}{\pgfqpoint{5.792113in}{0.606729in}}%
\pgfpathcurveto{\pgfqpoint{5.792113in}{0.595679in}}{\pgfqpoint{5.796503in}{0.585080in}}{\pgfqpoint{5.804317in}{0.577267in}}%
\pgfpathcurveto{\pgfqpoint{5.812131in}{0.569453in}}{\pgfqpoint{5.822730in}{0.565063in}}{\pgfqpoint{5.833780in}{0.565063in}}%
\pgfpathlineto{\pgfqpoint{5.833780in}{0.565063in}}%
\pgfpathclose%
\pgfusepath{stroke}%
\end{pgfscope}%
\begin{pgfscope}%
\pgfpathrectangle{\pgfqpoint{0.847223in}{0.554012in}}{\pgfqpoint{6.200000in}{4.530000in}}%
\pgfusepath{clip}%
\pgfsetbuttcap%
\pgfsetroundjoin%
\pgfsetlinewidth{1.003750pt}%
\definecolor{currentstroke}{rgb}{1.000000,0.000000,0.000000}%
\pgfsetstrokecolor{currentstroke}%
\pgfsetdash{}{0pt}%
\pgfpathmoveto{\pgfqpoint{5.839113in}{0.564608in}}%
\pgfpathcurveto{\pgfqpoint{5.850163in}{0.564608in}}{\pgfqpoint{5.860762in}{0.568999in}}{\pgfqpoint{5.868576in}{0.576812in}}%
\pgfpathcurveto{\pgfqpoint{5.876389in}{0.584626in}}{\pgfqpoint{5.880780in}{0.595225in}}{\pgfqpoint{5.880780in}{0.606275in}}%
\pgfpathcurveto{\pgfqpoint{5.880780in}{0.617325in}}{\pgfqpoint{5.876389in}{0.627924in}}{\pgfqpoint{5.868576in}{0.635738in}}%
\pgfpathcurveto{\pgfqpoint{5.860762in}{0.643551in}}{\pgfqpoint{5.850163in}{0.647942in}}{\pgfqpoint{5.839113in}{0.647942in}}%
\pgfpathcurveto{\pgfqpoint{5.828063in}{0.647942in}}{\pgfqpoint{5.817464in}{0.643551in}}{\pgfqpoint{5.809650in}{0.635738in}}%
\pgfpathcurveto{\pgfqpoint{5.801837in}{0.627924in}}{\pgfqpoint{5.797446in}{0.617325in}}{\pgfqpoint{5.797446in}{0.606275in}}%
\pgfpathcurveto{\pgfqpoint{5.797446in}{0.595225in}}{\pgfqpoint{5.801837in}{0.584626in}}{\pgfqpoint{5.809650in}{0.576812in}}%
\pgfpathcurveto{\pgfqpoint{5.817464in}{0.568999in}}{\pgfqpoint{5.828063in}{0.564608in}}{\pgfqpoint{5.839113in}{0.564608in}}%
\pgfpathlineto{\pgfqpoint{5.839113in}{0.564608in}}%
\pgfpathclose%
\pgfusepath{stroke}%
\end{pgfscope}%
\begin{pgfscope}%
\pgfpathrectangle{\pgfqpoint{0.847223in}{0.554012in}}{\pgfqpoint{6.200000in}{4.530000in}}%
\pgfusepath{clip}%
\pgfsetbuttcap%
\pgfsetroundjoin%
\pgfsetlinewidth{1.003750pt}%
\definecolor{currentstroke}{rgb}{1.000000,0.000000,0.000000}%
\pgfsetstrokecolor{currentstroke}%
\pgfsetdash{}{0pt}%
\pgfpathmoveto{\pgfqpoint{5.844446in}{0.564155in}}%
\pgfpathcurveto{\pgfqpoint{5.855496in}{0.564155in}}{\pgfqpoint{5.866095in}{0.568545in}}{\pgfqpoint{5.873909in}{0.576359in}}%
\pgfpathcurveto{\pgfqpoint{5.881723in}{0.584172in}}{\pgfqpoint{5.886113in}{0.594771in}}{\pgfqpoint{5.886113in}{0.605821in}}%
\pgfpathcurveto{\pgfqpoint{5.886113in}{0.616872in}}{\pgfqpoint{5.881723in}{0.627471in}}{\pgfqpoint{5.873909in}{0.635284in}}%
\pgfpathcurveto{\pgfqpoint{5.866095in}{0.643098in}}{\pgfqpoint{5.855496in}{0.647488in}}{\pgfqpoint{5.844446in}{0.647488in}}%
\pgfpathcurveto{\pgfqpoint{5.833396in}{0.647488in}}{\pgfqpoint{5.822797in}{0.643098in}}{\pgfqpoint{5.814983in}{0.635284in}}%
\pgfpathcurveto{\pgfqpoint{5.807170in}{0.627471in}}{\pgfqpoint{5.802779in}{0.616872in}}{\pgfqpoint{5.802779in}{0.605821in}}%
\pgfpathcurveto{\pgfqpoint{5.802779in}{0.594771in}}{\pgfqpoint{5.807170in}{0.584172in}}{\pgfqpoint{5.814983in}{0.576359in}}%
\pgfpathcurveto{\pgfqpoint{5.822797in}{0.568545in}}{\pgfqpoint{5.833396in}{0.564155in}}{\pgfqpoint{5.844446in}{0.564155in}}%
\pgfpathlineto{\pgfqpoint{5.844446in}{0.564155in}}%
\pgfpathclose%
\pgfusepath{stroke}%
\end{pgfscope}%
\begin{pgfscope}%
\pgfpathrectangle{\pgfqpoint{0.847223in}{0.554012in}}{\pgfqpoint{6.200000in}{4.530000in}}%
\pgfusepath{clip}%
\pgfsetbuttcap%
\pgfsetroundjoin%
\pgfsetlinewidth{1.003750pt}%
\definecolor{currentstroke}{rgb}{1.000000,0.000000,0.000000}%
\pgfsetstrokecolor{currentstroke}%
\pgfsetdash{}{0pt}%
\pgfpathmoveto{\pgfqpoint{5.849779in}{0.563702in}}%
\pgfpathcurveto{\pgfqpoint{5.860829in}{0.563702in}}{\pgfqpoint{5.871429in}{0.568092in}}{\pgfqpoint{5.879242in}{0.575906in}}%
\pgfpathcurveto{\pgfqpoint{5.887056in}{0.583720in}}{\pgfqpoint{5.891446in}{0.594319in}}{\pgfqpoint{5.891446in}{0.605369in}}%
\pgfpathcurveto{\pgfqpoint{5.891446in}{0.616419in}}{\pgfqpoint{5.887056in}{0.627018in}}{\pgfqpoint{5.879242in}{0.634832in}}%
\pgfpathcurveto{\pgfqpoint{5.871429in}{0.642645in}}{\pgfqpoint{5.860829in}{0.647036in}}{\pgfqpoint{5.849779in}{0.647036in}}%
\pgfpathcurveto{\pgfqpoint{5.838729in}{0.647036in}}{\pgfqpoint{5.828130in}{0.642645in}}{\pgfqpoint{5.820317in}{0.634832in}}%
\pgfpathcurveto{\pgfqpoint{5.812503in}{0.627018in}}{\pgfqpoint{5.808113in}{0.616419in}}{\pgfqpoint{5.808113in}{0.605369in}}%
\pgfpathcurveto{\pgfqpoint{5.808113in}{0.594319in}}{\pgfqpoint{5.812503in}{0.583720in}}{\pgfqpoint{5.820317in}{0.575906in}}%
\pgfpathcurveto{\pgfqpoint{5.828130in}{0.568092in}}{\pgfqpoint{5.838729in}{0.563702in}}{\pgfqpoint{5.849779in}{0.563702in}}%
\pgfpathlineto{\pgfqpoint{5.849779in}{0.563702in}}%
\pgfpathclose%
\pgfusepath{stroke}%
\end{pgfscope}%
\begin{pgfscope}%
\pgfpathrectangle{\pgfqpoint{0.847223in}{0.554012in}}{\pgfqpoint{6.200000in}{4.530000in}}%
\pgfusepath{clip}%
\pgfsetbuttcap%
\pgfsetroundjoin%
\pgfsetlinewidth{1.003750pt}%
\definecolor{currentstroke}{rgb}{1.000000,0.000000,0.000000}%
\pgfsetstrokecolor{currentstroke}%
\pgfsetdash{}{0pt}%
\pgfpathmoveto{\pgfqpoint{5.855113in}{0.563250in}}%
\pgfpathcurveto{\pgfqpoint{5.866163in}{0.563250in}}{\pgfqpoint{5.876762in}{0.567641in}}{\pgfqpoint{5.884575in}{0.575454in}}%
\pgfpathcurveto{\pgfqpoint{5.892389in}{0.583268in}}{\pgfqpoint{5.896779in}{0.593867in}}{\pgfqpoint{5.896779in}{0.604917in}}%
\pgfpathcurveto{\pgfqpoint{5.896779in}{0.615967in}}{\pgfqpoint{5.892389in}{0.626566in}}{\pgfqpoint{5.884575in}{0.634380in}}%
\pgfpathcurveto{\pgfqpoint{5.876762in}{0.642193in}}{\pgfqpoint{5.866163in}{0.646584in}}{\pgfqpoint{5.855113in}{0.646584in}}%
\pgfpathcurveto{\pgfqpoint{5.844062in}{0.646584in}}{\pgfqpoint{5.833463in}{0.642193in}}{\pgfqpoint{5.825650in}{0.634380in}}%
\pgfpathcurveto{\pgfqpoint{5.817836in}{0.626566in}}{\pgfqpoint{5.813446in}{0.615967in}}{\pgfqpoint{5.813446in}{0.604917in}}%
\pgfpathcurveto{\pgfqpoint{5.813446in}{0.593867in}}{\pgfqpoint{5.817836in}{0.583268in}}{\pgfqpoint{5.825650in}{0.575454in}}%
\pgfpathcurveto{\pgfqpoint{5.833463in}{0.567641in}}{\pgfqpoint{5.844062in}{0.563250in}}{\pgfqpoint{5.855113in}{0.563250in}}%
\pgfpathlineto{\pgfqpoint{5.855113in}{0.563250in}}%
\pgfpathclose%
\pgfusepath{stroke}%
\end{pgfscope}%
\begin{pgfscope}%
\pgfpathrectangle{\pgfqpoint{0.847223in}{0.554012in}}{\pgfqpoint{6.200000in}{4.530000in}}%
\pgfusepath{clip}%
\pgfsetbuttcap%
\pgfsetroundjoin%
\pgfsetlinewidth{1.003750pt}%
\definecolor{currentstroke}{rgb}{1.000000,0.000000,0.000000}%
\pgfsetstrokecolor{currentstroke}%
\pgfsetdash{}{0pt}%
\pgfpathmoveto{\pgfqpoint{5.860446in}{0.562800in}}%
\pgfpathcurveto{\pgfqpoint{5.871496in}{0.562800in}}{\pgfqpoint{5.882095in}{0.567190in}}{\pgfqpoint{5.889909in}{0.575003in}}%
\pgfpathcurveto{\pgfqpoint{5.897722in}{0.582817in}}{\pgfqpoint{5.902112in}{0.593416in}}{\pgfqpoint{5.902112in}{0.604466in}}%
\pgfpathcurveto{\pgfqpoint{5.902112in}{0.615516in}}{\pgfqpoint{5.897722in}{0.626115in}}{\pgfqpoint{5.889909in}{0.633929in}}%
\pgfpathcurveto{\pgfqpoint{5.882095in}{0.641743in}}{\pgfqpoint{5.871496in}{0.646133in}}{\pgfqpoint{5.860446in}{0.646133in}}%
\pgfpathcurveto{\pgfqpoint{5.849396in}{0.646133in}}{\pgfqpoint{5.838797in}{0.641743in}}{\pgfqpoint{5.830983in}{0.633929in}}%
\pgfpathcurveto{\pgfqpoint{5.823169in}{0.626115in}}{\pgfqpoint{5.818779in}{0.615516in}}{\pgfqpoint{5.818779in}{0.604466in}}%
\pgfpathcurveto{\pgfqpoint{5.818779in}{0.593416in}}{\pgfqpoint{5.823169in}{0.582817in}}{\pgfqpoint{5.830983in}{0.575003in}}%
\pgfpathcurveto{\pgfqpoint{5.838797in}{0.567190in}}{\pgfqpoint{5.849396in}{0.562800in}}{\pgfqpoint{5.860446in}{0.562800in}}%
\pgfpathlineto{\pgfqpoint{5.860446in}{0.562800in}}%
\pgfpathclose%
\pgfusepath{stroke}%
\end{pgfscope}%
\begin{pgfscope}%
\pgfpathrectangle{\pgfqpoint{0.847223in}{0.554012in}}{\pgfqpoint{6.200000in}{4.530000in}}%
\pgfusepath{clip}%
\pgfsetbuttcap%
\pgfsetroundjoin%
\pgfsetlinewidth{1.003750pt}%
\definecolor{currentstroke}{rgb}{1.000000,0.000000,0.000000}%
\pgfsetstrokecolor{currentstroke}%
\pgfsetdash{}{0pt}%
\pgfpathmoveto{\pgfqpoint{5.865779in}{0.562349in}}%
\pgfpathcurveto{\pgfqpoint{5.876829in}{0.562349in}}{\pgfqpoint{5.887428in}{0.566740in}}{\pgfqpoint{5.895242in}{0.574553in}}%
\pgfpathcurveto{\pgfqpoint{5.903055in}{0.582367in}}{\pgfqpoint{5.907446in}{0.592966in}}{\pgfqpoint{5.907446in}{0.604016in}}%
\pgfpathcurveto{\pgfqpoint{5.907446in}{0.615066in}}{\pgfqpoint{5.903055in}{0.625665in}}{\pgfqpoint{5.895242in}{0.633479in}}%
\pgfpathcurveto{\pgfqpoint{5.887428in}{0.641293in}}{\pgfqpoint{5.876829in}{0.645683in}}{\pgfqpoint{5.865779in}{0.645683in}}%
\pgfpathcurveto{\pgfqpoint{5.854729in}{0.645683in}}{\pgfqpoint{5.844130in}{0.641293in}}{\pgfqpoint{5.836316in}{0.633479in}}%
\pgfpathcurveto{\pgfqpoint{5.828503in}{0.625665in}}{\pgfqpoint{5.824112in}{0.615066in}}{\pgfqpoint{5.824112in}{0.604016in}}%
\pgfpathcurveto{\pgfqpoint{5.824112in}{0.592966in}}{\pgfqpoint{5.828503in}{0.582367in}}{\pgfqpoint{5.836316in}{0.574553in}}%
\pgfpathcurveto{\pgfqpoint{5.844130in}{0.566740in}}{\pgfqpoint{5.854729in}{0.562349in}}{\pgfqpoint{5.865779in}{0.562349in}}%
\pgfpathlineto{\pgfqpoint{5.865779in}{0.562349in}}%
\pgfpathclose%
\pgfusepath{stroke}%
\end{pgfscope}%
\begin{pgfscope}%
\pgfpathrectangle{\pgfqpoint{0.847223in}{0.554012in}}{\pgfqpoint{6.200000in}{4.530000in}}%
\pgfusepath{clip}%
\pgfsetbuttcap%
\pgfsetroundjoin%
\pgfsetlinewidth{1.003750pt}%
\definecolor{currentstroke}{rgb}{1.000000,0.000000,0.000000}%
\pgfsetstrokecolor{currentstroke}%
\pgfsetdash{}{0pt}%
\pgfpathmoveto{\pgfqpoint{5.871112in}{0.561900in}}%
\pgfpathcurveto{\pgfqpoint{5.882162in}{0.561900in}}{\pgfqpoint{5.892761in}{0.566291in}}{\pgfqpoint{5.900575in}{0.574104in}}%
\pgfpathcurveto{\pgfqpoint{5.908389in}{0.581918in}}{\pgfqpoint{5.912779in}{0.592517in}}{\pgfqpoint{5.912779in}{0.603567in}}%
\pgfpathcurveto{\pgfqpoint{5.912779in}{0.614617in}}{\pgfqpoint{5.908389in}{0.625216in}}{\pgfqpoint{5.900575in}{0.633030in}}%
\pgfpathcurveto{\pgfqpoint{5.892761in}{0.640843in}}{\pgfqpoint{5.882162in}{0.645234in}}{\pgfqpoint{5.871112in}{0.645234in}}%
\pgfpathcurveto{\pgfqpoint{5.860062in}{0.645234in}}{\pgfqpoint{5.849463in}{0.640843in}}{\pgfqpoint{5.841649in}{0.633030in}}%
\pgfpathcurveto{\pgfqpoint{5.833836in}{0.625216in}}{\pgfqpoint{5.829446in}{0.614617in}}{\pgfqpoint{5.829446in}{0.603567in}}%
\pgfpathcurveto{\pgfqpoint{5.829446in}{0.592517in}}{\pgfqpoint{5.833836in}{0.581918in}}{\pgfqpoint{5.841649in}{0.574104in}}%
\pgfpathcurveto{\pgfqpoint{5.849463in}{0.566291in}}{\pgfqpoint{5.860062in}{0.561900in}}{\pgfqpoint{5.871112in}{0.561900in}}%
\pgfpathlineto{\pgfqpoint{5.871112in}{0.561900in}}%
\pgfpathclose%
\pgfusepath{stroke}%
\end{pgfscope}%
\begin{pgfscope}%
\pgfpathrectangle{\pgfqpoint{0.847223in}{0.554012in}}{\pgfqpoint{6.200000in}{4.530000in}}%
\pgfusepath{clip}%
\pgfsetbuttcap%
\pgfsetroundjoin%
\pgfsetlinewidth{1.003750pt}%
\definecolor{currentstroke}{rgb}{1.000000,0.000000,0.000000}%
\pgfsetstrokecolor{currentstroke}%
\pgfsetdash{}{0pt}%
\pgfpathmoveto{\pgfqpoint{5.876445in}{0.561452in}}%
\pgfpathcurveto{\pgfqpoint{5.887496in}{0.561452in}}{\pgfqpoint{5.898095in}{0.565842in}}{\pgfqpoint{5.905908in}{0.573656in}}%
\pgfpathcurveto{\pgfqpoint{5.913722in}{0.581470in}}{\pgfqpoint{5.918112in}{0.592069in}}{\pgfqpoint{5.918112in}{0.603119in}}%
\pgfpathcurveto{\pgfqpoint{5.918112in}{0.614169in}}{\pgfqpoint{5.913722in}{0.624768in}}{\pgfqpoint{5.905908in}{0.632581in}}%
\pgfpathcurveto{\pgfqpoint{5.898095in}{0.640395in}}{\pgfqpoint{5.887496in}{0.644785in}}{\pgfqpoint{5.876445in}{0.644785in}}%
\pgfpathcurveto{\pgfqpoint{5.865395in}{0.644785in}}{\pgfqpoint{5.854796in}{0.640395in}}{\pgfqpoint{5.846983in}{0.632581in}}%
\pgfpathcurveto{\pgfqpoint{5.839169in}{0.624768in}}{\pgfqpoint{5.834779in}{0.614169in}}{\pgfqpoint{5.834779in}{0.603119in}}%
\pgfpathcurveto{\pgfqpoint{5.834779in}{0.592069in}}{\pgfqpoint{5.839169in}{0.581470in}}{\pgfqpoint{5.846983in}{0.573656in}}%
\pgfpathcurveto{\pgfqpoint{5.854796in}{0.565842in}}{\pgfqpoint{5.865395in}{0.561452in}}{\pgfqpoint{5.876445in}{0.561452in}}%
\pgfpathlineto{\pgfqpoint{5.876445in}{0.561452in}}%
\pgfpathclose%
\pgfusepath{stroke}%
\end{pgfscope}%
\begin{pgfscope}%
\pgfpathrectangle{\pgfqpoint{0.847223in}{0.554012in}}{\pgfqpoint{6.200000in}{4.530000in}}%
\pgfusepath{clip}%
\pgfsetbuttcap%
\pgfsetroundjoin%
\pgfsetlinewidth{1.003750pt}%
\definecolor{currentstroke}{rgb}{1.000000,0.000000,0.000000}%
\pgfsetstrokecolor{currentstroke}%
\pgfsetdash{}{0pt}%
\pgfpathmoveto{\pgfqpoint{5.881779in}{0.561005in}}%
\pgfpathcurveto{\pgfqpoint{5.892829in}{0.561005in}}{\pgfqpoint{5.903428in}{0.565395in}}{\pgfqpoint{5.911241in}{0.573208in}}%
\pgfpathcurveto{\pgfqpoint{5.919055in}{0.581022in}}{\pgfqpoint{5.923445in}{0.591621in}}{\pgfqpoint{5.923445in}{0.602671in}}%
\pgfpathcurveto{\pgfqpoint{5.923445in}{0.613721in}}{\pgfqpoint{5.919055in}{0.624320in}}{\pgfqpoint{5.911241in}{0.632134in}}%
\pgfpathcurveto{\pgfqpoint{5.903428in}{0.639948in}}{\pgfqpoint{5.892829in}{0.644338in}}{\pgfqpoint{5.881779in}{0.644338in}}%
\pgfpathcurveto{\pgfqpoint{5.870729in}{0.644338in}}{\pgfqpoint{5.860129in}{0.639948in}}{\pgfqpoint{5.852316in}{0.632134in}}%
\pgfpathcurveto{\pgfqpoint{5.844502in}{0.624320in}}{\pgfqpoint{5.840112in}{0.613721in}}{\pgfqpoint{5.840112in}{0.602671in}}%
\pgfpathcurveto{\pgfqpoint{5.840112in}{0.591621in}}{\pgfqpoint{5.844502in}{0.581022in}}{\pgfqpoint{5.852316in}{0.573208in}}%
\pgfpathcurveto{\pgfqpoint{5.860129in}{0.565395in}}{\pgfqpoint{5.870729in}{0.561005in}}{\pgfqpoint{5.881779in}{0.561005in}}%
\pgfpathlineto{\pgfqpoint{5.881779in}{0.561005in}}%
\pgfpathclose%
\pgfusepath{stroke}%
\end{pgfscope}%
\begin{pgfscope}%
\pgfpathrectangle{\pgfqpoint{0.847223in}{0.554012in}}{\pgfqpoint{6.200000in}{4.530000in}}%
\pgfusepath{clip}%
\pgfsetbuttcap%
\pgfsetroundjoin%
\pgfsetlinewidth{1.003750pt}%
\definecolor{currentstroke}{rgb}{1.000000,0.000000,0.000000}%
\pgfsetstrokecolor{currentstroke}%
\pgfsetdash{}{0pt}%
\pgfpathmoveto{\pgfqpoint{5.887112in}{0.560558in}}%
\pgfpathcurveto{\pgfqpoint{5.898162in}{0.560558in}}{\pgfqpoint{5.908761in}{0.564948in}}{\pgfqpoint{5.916575in}{0.572762in}}%
\pgfpathcurveto{\pgfqpoint{5.924388in}{0.580576in}}{\pgfqpoint{5.928779in}{0.591175in}}{\pgfqpoint{5.928779in}{0.602225in}}%
\pgfpathcurveto{\pgfqpoint{5.928779in}{0.613275in}}{\pgfqpoint{5.924388in}{0.623874in}}{\pgfqpoint{5.916575in}{0.631687in}}%
\pgfpathcurveto{\pgfqpoint{5.908761in}{0.639501in}}{\pgfqpoint{5.898162in}{0.643891in}}{\pgfqpoint{5.887112in}{0.643891in}}%
\pgfpathcurveto{\pgfqpoint{5.876062in}{0.643891in}}{\pgfqpoint{5.865463in}{0.639501in}}{\pgfqpoint{5.857649in}{0.631687in}}%
\pgfpathcurveto{\pgfqpoint{5.849835in}{0.623874in}}{\pgfqpoint{5.845445in}{0.613275in}}{\pgfqpoint{5.845445in}{0.602225in}}%
\pgfpathcurveto{\pgfqpoint{5.845445in}{0.591175in}}{\pgfqpoint{5.849835in}{0.580576in}}{\pgfqpoint{5.857649in}{0.572762in}}%
\pgfpathcurveto{\pgfqpoint{5.865463in}{0.564948in}}{\pgfqpoint{5.876062in}{0.560558in}}{\pgfqpoint{5.887112in}{0.560558in}}%
\pgfpathlineto{\pgfqpoint{5.887112in}{0.560558in}}%
\pgfpathclose%
\pgfusepath{stroke}%
\end{pgfscope}%
\begin{pgfscope}%
\pgfpathrectangle{\pgfqpoint{0.847223in}{0.554012in}}{\pgfqpoint{6.200000in}{4.530000in}}%
\pgfusepath{clip}%
\pgfsetbuttcap%
\pgfsetroundjoin%
\pgfsetlinewidth{1.003750pt}%
\definecolor{currentstroke}{rgb}{1.000000,0.000000,0.000000}%
\pgfsetstrokecolor{currentstroke}%
\pgfsetdash{}{0pt}%
\pgfpathmoveto{\pgfqpoint{5.892445in}{0.560112in}}%
\pgfpathcurveto{\pgfqpoint{5.903495in}{0.560112in}}{\pgfqpoint{5.914094in}{0.564503in}}{\pgfqpoint{5.921908in}{0.572316in}}%
\pgfpathcurveto{\pgfqpoint{5.929721in}{0.580130in}}{\pgfqpoint{5.934112in}{0.590729in}}{\pgfqpoint{5.934112in}{0.601779in}}%
\pgfpathcurveto{\pgfqpoint{5.934112in}{0.612829in}}{\pgfqpoint{5.929721in}{0.623428in}}{\pgfqpoint{5.921908in}{0.631242in}}%
\pgfpathcurveto{\pgfqpoint{5.914094in}{0.639055in}}{\pgfqpoint{5.903495in}{0.643446in}}{\pgfqpoint{5.892445in}{0.643446in}}%
\pgfpathcurveto{\pgfqpoint{5.881395in}{0.643446in}}{\pgfqpoint{5.870796in}{0.639055in}}{\pgfqpoint{5.862982in}{0.631242in}}%
\pgfpathcurveto{\pgfqpoint{5.855169in}{0.623428in}}{\pgfqpoint{5.850778in}{0.612829in}}{\pgfqpoint{5.850778in}{0.601779in}}%
\pgfpathcurveto{\pgfqpoint{5.850778in}{0.590729in}}{\pgfqpoint{5.855169in}{0.580130in}}{\pgfqpoint{5.862982in}{0.572316in}}%
\pgfpathcurveto{\pgfqpoint{5.870796in}{0.564503in}}{\pgfqpoint{5.881395in}{0.560112in}}{\pgfqpoint{5.892445in}{0.560112in}}%
\pgfpathlineto{\pgfqpoint{5.892445in}{0.560112in}}%
\pgfpathclose%
\pgfusepath{stroke}%
\end{pgfscope}%
\begin{pgfscope}%
\pgfpathrectangle{\pgfqpoint{0.847223in}{0.554012in}}{\pgfqpoint{6.200000in}{4.530000in}}%
\pgfusepath{clip}%
\pgfsetbuttcap%
\pgfsetroundjoin%
\pgfsetlinewidth{1.003750pt}%
\definecolor{currentstroke}{rgb}{1.000000,0.000000,0.000000}%
\pgfsetstrokecolor{currentstroke}%
\pgfsetdash{}{0pt}%
\pgfpathmoveto{\pgfqpoint{5.897778in}{0.559667in}}%
\pgfpathcurveto{\pgfqpoint{5.908828in}{0.559667in}}{\pgfqpoint{5.919427in}{0.564058in}}{\pgfqpoint{5.927241in}{0.571871in}}%
\pgfpathcurveto{\pgfqpoint{5.935055in}{0.579685in}}{\pgfqpoint{5.939445in}{0.590284in}}{\pgfqpoint{5.939445in}{0.601334in}}%
\pgfpathcurveto{\pgfqpoint{5.939445in}{0.612384in}}{\pgfqpoint{5.935055in}{0.622983in}}{\pgfqpoint{5.927241in}{0.630797in}}%
\pgfpathcurveto{\pgfqpoint{5.919427in}{0.638610in}}{\pgfqpoint{5.908828in}{0.643001in}}{\pgfqpoint{5.897778in}{0.643001in}}%
\pgfpathcurveto{\pgfqpoint{5.886728in}{0.643001in}}{\pgfqpoint{5.876129in}{0.638610in}}{\pgfqpoint{5.868316in}{0.630797in}}%
\pgfpathcurveto{\pgfqpoint{5.860502in}{0.622983in}}{\pgfqpoint{5.856112in}{0.612384in}}{\pgfqpoint{5.856112in}{0.601334in}}%
\pgfpathcurveto{\pgfqpoint{5.856112in}{0.590284in}}{\pgfqpoint{5.860502in}{0.579685in}}{\pgfqpoint{5.868316in}{0.571871in}}%
\pgfpathcurveto{\pgfqpoint{5.876129in}{0.564058in}}{\pgfqpoint{5.886728in}{0.559667in}}{\pgfqpoint{5.897778in}{0.559667in}}%
\pgfpathlineto{\pgfqpoint{5.897778in}{0.559667in}}%
\pgfpathclose%
\pgfusepath{stroke}%
\end{pgfscope}%
\begin{pgfscope}%
\pgfpathrectangle{\pgfqpoint{0.847223in}{0.554012in}}{\pgfqpoint{6.200000in}{4.530000in}}%
\pgfusepath{clip}%
\pgfsetbuttcap%
\pgfsetroundjoin%
\pgfsetlinewidth{1.003750pt}%
\definecolor{currentstroke}{rgb}{1.000000,0.000000,0.000000}%
\pgfsetstrokecolor{currentstroke}%
\pgfsetdash{}{0pt}%
\pgfpathmoveto{\pgfqpoint{5.903112in}{0.559223in}}%
\pgfpathcurveto{\pgfqpoint{5.914162in}{0.559223in}}{\pgfqpoint{5.924761in}{0.563614in}}{\pgfqpoint{5.932574in}{0.571427in}}%
\pgfpathcurveto{\pgfqpoint{5.940388in}{0.579241in}}{\pgfqpoint{5.944778in}{0.589840in}}{\pgfqpoint{5.944778in}{0.600890in}}%
\pgfpathcurveto{\pgfqpoint{5.944778in}{0.611940in}}{\pgfqpoint{5.940388in}{0.622539in}}{\pgfqpoint{5.932574in}{0.630353in}}%
\pgfpathcurveto{\pgfqpoint{5.924761in}{0.638166in}}{\pgfqpoint{5.914162in}{0.642557in}}{\pgfqpoint{5.903112in}{0.642557in}}%
\pgfpathcurveto{\pgfqpoint{5.892061in}{0.642557in}}{\pgfqpoint{5.881462in}{0.638166in}}{\pgfqpoint{5.873649in}{0.630353in}}%
\pgfpathcurveto{\pgfqpoint{5.865835in}{0.622539in}}{\pgfqpoint{5.861445in}{0.611940in}}{\pgfqpoint{5.861445in}{0.600890in}}%
\pgfpathcurveto{\pgfqpoint{5.861445in}{0.589840in}}{\pgfqpoint{5.865835in}{0.579241in}}{\pgfqpoint{5.873649in}{0.571427in}}%
\pgfpathcurveto{\pgfqpoint{5.881462in}{0.563614in}}{\pgfqpoint{5.892061in}{0.559223in}}{\pgfqpoint{5.903112in}{0.559223in}}%
\pgfpathlineto{\pgfqpoint{5.903112in}{0.559223in}}%
\pgfpathclose%
\pgfusepath{stroke}%
\end{pgfscope}%
\begin{pgfscope}%
\pgfpathrectangle{\pgfqpoint{0.847223in}{0.554012in}}{\pgfqpoint{6.200000in}{4.530000in}}%
\pgfusepath{clip}%
\pgfsetbuttcap%
\pgfsetroundjoin%
\pgfsetlinewidth{1.003750pt}%
\definecolor{currentstroke}{rgb}{1.000000,0.000000,0.000000}%
\pgfsetstrokecolor{currentstroke}%
\pgfsetdash{}{0pt}%
\pgfpathmoveto{\pgfqpoint{5.908445in}{0.558780in}}%
\pgfpathcurveto{\pgfqpoint{5.919495in}{0.558780in}}{\pgfqpoint{5.930094in}{0.563170in}}{\pgfqpoint{5.937907in}{0.570984in}}%
\pgfpathcurveto{\pgfqpoint{5.945721in}{0.578798in}}{\pgfqpoint{5.950111in}{0.589397in}}{\pgfqpoint{5.950111in}{0.600447in}}%
\pgfpathcurveto{\pgfqpoint{5.950111in}{0.611497in}}{\pgfqpoint{5.945721in}{0.622096in}}{\pgfqpoint{5.937907in}{0.629910in}}%
\pgfpathcurveto{\pgfqpoint{5.930094in}{0.637723in}}{\pgfqpoint{5.919495in}{0.642114in}}{\pgfqpoint{5.908445in}{0.642114in}}%
\pgfpathcurveto{\pgfqpoint{5.897395in}{0.642114in}}{\pgfqpoint{5.886796in}{0.637723in}}{\pgfqpoint{5.878982in}{0.629910in}}%
\pgfpathcurveto{\pgfqpoint{5.871168in}{0.622096in}}{\pgfqpoint{5.866778in}{0.611497in}}{\pgfqpoint{5.866778in}{0.600447in}}%
\pgfpathcurveto{\pgfqpoint{5.866778in}{0.589397in}}{\pgfqpoint{5.871168in}{0.578798in}}{\pgfqpoint{5.878982in}{0.570984in}}%
\pgfpathcurveto{\pgfqpoint{5.886796in}{0.563170in}}{\pgfqpoint{5.897395in}{0.558780in}}{\pgfqpoint{5.908445in}{0.558780in}}%
\pgfpathlineto{\pgfqpoint{5.908445in}{0.558780in}}%
\pgfpathclose%
\pgfusepath{stroke}%
\end{pgfscope}%
\begin{pgfscope}%
\pgfpathrectangle{\pgfqpoint{0.847223in}{0.554012in}}{\pgfqpoint{6.200000in}{4.530000in}}%
\pgfusepath{clip}%
\pgfsetbuttcap%
\pgfsetroundjoin%
\pgfsetlinewidth{1.003750pt}%
\definecolor{currentstroke}{rgb}{1.000000,0.000000,0.000000}%
\pgfsetstrokecolor{currentstroke}%
\pgfsetdash{}{0pt}%
\pgfpathmoveto{\pgfqpoint{5.913778in}{0.558338in}}%
\pgfpathcurveto{\pgfqpoint{5.924828in}{0.558338in}}{\pgfqpoint{5.935427in}{0.562728in}}{\pgfqpoint{5.943241in}{0.570542in}}%
\pgfpathcurveto{\pgfqpoint{5.951054in}{0.578355in}}{\pgfqpoint{5.955445in}{0.588954in}}{\pgfqpoint{5.955445in}{0.600005in}}%
\pgfpathcurveto{\pgfqpoint{5.955445in}{0.611055in}}{\pgfqpoint{5.951054in}{0.621654in}}{\pgfqpoint{5.943241in}{0.629467in}}%
\pgfpathcurveto{\pgfqpoint{5.935427in}{0.637281in}}{\pgfqpoint{5.924828in}{0.641671in}}{\pgfqpoint{5.913778in}{0.641671in}}%
\pgfpathcurveto{\pgfqpoint{5.902728in}{0.641671in}}{\pgfqpoint{5.892129in}{0.637281in}}{\pgfqpoint{5.884315in}{0.629467in}}%
\pgfpathcurveto{\pgfqpoint{5.876502in}{0.621654in}}{\pgfqpoint{5.872111in}{0.611055in}}{\pgfqpoint{5.872111in}{0.600005in}}%
\pgfpathcurveto{\pgfqpoint{5.872111in}{0.588954in}}{\pgfqpoint{5.876502in}{0.578355in}}{\pgfqpoint{5.884315in}{0.570542in}}%
\pgfpathcurveto{\pgfqpoint{5.892129in}{0.562728in}}{\pgfqpoint{5.902728in}{0.558338in}}{\pgfqpoint{5.913778in}{0.558338in}}%
\pgfpathlineto{\pgfqpoint{5.913778in}{0.558338in}}%
\pgfpathclose%
\pgfusepath{stroke}%
\end{pgfscope}%
\begin{pgfscope}%
\pgfpathrectangle{\pgfqpoint{0.847223in}{0.554012in}}{\pgfqpoint{6.200000in}{4.530000in}}%
\pgfusepath{clip}%
\pgfsetbuttcap%
\pgfsetroundjoin%
\pgfsetlinewidth{1.003750pt}%
\definecolor{currentstroke}{rgb}{1.000000,0.000000,0.000000}%
\pgfsetstrokecolor{currentstroke}%
\pgfsetdash{}{0pt}%
\pgfpathmoveto{\pgfqpoint{5.919111in}{0.557896in}}%
\pgfpathcurveto{\pgfqpoint{5.930161in}{0.557896in}}{\pgfqpoint{5.940760in}{0.562287in}}{\pgfqpoint{5.948574in}{0.570100in}}%
\pgfpathcurveto{\pgfqpoint{5.956388in}{0.577914in}}{\pgfqpoint{5.960778in}{0.588513in}}{\pgfqpoint{5.960778in}{0.599563in}}%
\pgfpathcurveto{\pgfqpoint{5.960778in}{0.610613in}}{\pgfqpoint{5.956388in}{0.621212in}}{\pgfqpoint{5.948574in}{0.629026in}}%
\pgfpathcurveto{\pgfqpoint{5.940760in}{0.636839in}}{\pgfqpoint{5.930161in}{0.641230in}}{\pgfqpoint{5.919111in}{0.641230in}}%
\pgfpathcurveto{\pgfqpoint{5.908061in}{0.641230in}}{\pgfqpoint{5.897462in}{0.636839in}}{\pgfqpoint{5.889648in}{0.629026in}}%
\pgfpathcurveto{\pgfqpoint{5.881835in}{0.621212in}}{\pgfqpoint{5.877444in}{0.610613in}}{\pgfqpoint{5.877444in}{0.599563in}}%
\pgfpathcurveto{\pgfqpoint{5.877444in}{0.588513in}}{\pgfqpoint{5.881835in}{0.577914in}}{\pgfqpoint{5.889648in}{0.570100in}}%
\pgfpathcurveto{\pgfqpoint{5.897462in}{0.562287in}}{\pgfqpoint{5.908061in}{0.557896in}}{\pgfqpoint{5.919111in}{0.557896in}}%
\pgfpathlineto{\pgfqpoint{5.919111in}{0.557896in}}%
\pgfpathclose%
\pgfusepath{stroke}%
\end{pgfscope}%
\begin{pgfscope}%
\pgfpathrectangle{\pgfqpoint{0.847223in}{0.554012in}}{\pgfqpoint{6.200000in}{4.530000in}}%
\pgfusepath{clip}%
\pgfsetbuttcap%
\pgfsetroundjoin%
\pgfsetlinewidth{1.003750pt}%
\definecolor{currentstroke}{rgb}{1.000000,0.000000,0.000000}%
\pgfsetstrokecolor{currentstroke}%
\pgfsetdash{}{0pt}%
\pgfpathmoveto{\pgfqpoint{5.924444in}{0.557456in}}%
\pgfpathcurveto{\pgfqpoint{5.935494in}{0.557456in}}{\pgfqpoint{5.946094in}{0.561846in}}{\pgfqpoint{5.953907in}{0.569660in}}%
\pgfpathcurveto{\pgfqpoint{5.961721in}{0.577473in}}{\pgfqpoint{5.966111in}{0.588072in}}{\pgfqpoint{5.966111in}{0.599122in}}%
\pgfpathcurveto{\pgfqpoint{5.966111in}{0.610173in}}{\pgfqpoint{5.961721in}{0.620772in}}{\pgfqpoint{5.953907in}{0.628585in}}%
\pgfpathcurveto{\pgfqpoint{5.946094in}{0.636399in}}{\pgfqpoint{5.935494in}{0.640789in}}{\pgfqpoint{5.924444in}{0.640789in}}%
\pgfpathcurveto{\pgfqpoint{5.913394in}{0.640789in}}{\pgfqpoint{5.902795in}{0.636399in}}{\pgfqpoint{5.894982in}{0.628585in}}%
\pgfpathcurveto{\pgfqpoint{5.887168in}{0.620772in}}{\pgfqpoint{5.882778in}{0.610173in}}{\pgfqpoint{5.882778in}{0.599122in}}%
\pgfpathcurveto{\pgfqpoint{5.882778in}{0.588072in}}{\pgfqpoint{5.887168in}{0.577473in}}{\pgfqpoint{5.894982in}{0.569660in}}%
\pgfpathcurveto{\pgfqpoint{5.902795in}{0.561846in}}{\pgfqpoint{5.913394in}{0.557456in}}{\pgfqpoint{5.924444in}{0.557456in}}%
\pgfpathlineto{\pgfqpoint{5.924444in}{0.557456in}}%
\pgfpathclose%
\pgfusepath{stroke}%
\end{pgfscope}%
\begin{pgfscope}%
\pgfpathrectangle{\pgfqpoint{0.847223in}{0.554012in}}{\pgfqpoint{6.200000in}{4.530000in}}%
\pgfusepath{clip}%
\pgfsetbuttcap%
\pgfsetroundjoin%
\pgfsetlinewidth{1.003750pt}%
\definecolor{currentstroke}{rgb}{1.000000,0.000000,0.000000}%
\pgfsetstrokecolor{currentstroke}%
\pgfsetdash{}{0pt}%
\pgfpathmoveto{\pgfqpoint{5.929778in}{0.557016in}}%
\pgfpathcurveto{\pgfqpoint{5.940828in}{0.557016in}}{\pgfqpoint{5.951427in}{0.561406in}}{\pgfqpoint{5.959240in}{0.569220in}}%
\pgfpathcurveto{\pgfqpoint{5.967054in}{0.577033in}}{\pgfqpoint{5.971444in}{0.587632in}}{\pgfqpoint{5.971444in}{0.598683in}}%
\pgfpathcurveto{\pgfqpoint{5.971444in}{0.609733in}}{\pgfqpoint{5.967054in}{0.620332in}}{\pgfqpoint{5.959240in}{0.628145in}}%
\pgfpathcurveto{\pgfqpoint{5.951427in}{0.635959in}}{\pgfqpoint{5.940828in}{0.640349in}}{\pgfqpoint{5.929778in}{0.640349in}}%
\pgfpathcurveto{\pgfqpoint{5.918727in}{0.640349in}}{\pgfqpoint{5.908128in}{0.635959in}}{\pgfqpoint{5.900315in}{0.628145in}}%
\pgfpathcurveto{\pgfqpoint{5.892501in}{0.620332in}}{\pgfqpoint{5.888111in}{0.609733in}}{\pgfqpoint{5.888111in}{0.598683in}}%
\pgfpathcurveto{\pgfqpoint{5.888111in}{0.587632in}}{\pgfqpoint{5.892501in}{0.577033in}}{\pgfqpoint{5.900315in}{0.569220in}}%
\pgfpathcurveto{\pgfqpoint{5.908128in}{0.561406in}}{\pgfqpoint{5.918727in}{0.557016in}}{\pgfqpoint{5.929778in}{0.557016in}}%
\pgfpathlineto{\pgfqpoint{5.929778in}{0.557016in}}%
\pgfpathclose%
\pgfusepath{stroke}%
\end{pgfscope}%
\begin{pgfscope}%
\pgfpathrectangle{\pgfqpoint{0.847223in}{0.554012in}}{\pgfqpoint{6.200000in}{4.530000in}}%
\pgfusepath{clip}%
\pgfsetbuttcap%
\pgfsetroundjoin%
\pgfsetlinewidth{1.003750pt}%
\definecolor{currentstroke}{rgb}{1.000000,0.000000,0.000000}%
\pgfsetstrokecolor{currentstroke}%
\pgfsetdash{}{0pt}%
\pgfpathmoveto{\pgfqpoint{5.935111in}{0.556577in}}%
\pgfpathcurveto{\pgfqpoint{5.946161in}{0.556577in}}{\pgfqpoint{5.956760in}{0.560967in}}{\pgfqpoint{5.964574in}{0.568781in}}%
\pgfpathcurveto{\pgfqpoint{5.972387in}{0.576594in}}{\pgfqpoint{5.976777in}{0.587194in}}{\pgfqpoint{5.976777in}{0.598244in}}%
\pgfpathcurveto{\pgfqpoint{5.976777in}{0.609294in}}{\pgfqpoint{5.972387in}{0.619893in}}{\pgfqpoint{5.964574in}{0.627706in}}%
\pgfpathcurveto{\pgfqpoint{5.956760in}{0.635520in}}{\pgfqpoint{5.946161in}{0.639910in}}{\pgfqpoint{5.935111in}{0.639910in}}%
\pgfpathcurveto{\pgfqpoint{5.924061in}{0.639910in}}{\pgfqpoint{5.913462in}{0.635520in}}{\pgfqpoint{5.905648in}{0.627706in}}%
\pgfpathcurveto{\pgfqpoint{5.897834in}{0.619893in}}{\pgfqpoint{5.893444in}{0.609294in}}{\pgfqpoint{5.893444in}{0.598244in}}%
\pgfpathcurveto{\pgfqpoint{5.893444in}{0.587194in}}{\pgfqpoint{5.897834in}{0.576594in}}{\pgfqpoint{5.905648in}{0.568781in}}%
\pgfpathcurveto{\pgfqpoint{5.913462in}{0.560967in}}{\pgfqpoint{5.924061in}{0.556577in}}{\pgfqpoint{5.935111in}{0.556577in}}%
\pgfpathlineto{\pgfqpoint{5.935111in}{0.556577in}}%
\pgfpathclose%
\pgfusepath{stroke}%
\end{pgfscope}%
\begin{pgfscope}%
\pgfpathrectangle{\pgfqpoint{0.847223in}{0.554012in}}{\pgfqpoint{6.200000in}{4.530000in}}%
\pgfusepath{clip}%
\pgfsetbuttcap%
\pgfsetroundjoin%
\pgfsetlinewidth{1.003750pt}%
\definecolor{currentstroke}{rgb}{1.000000,0.000000,0.000000}%
\pgfsetstrokecolor{currentstroke}%
\pgfsetdash{}{0pt}%
\pgfpathmoveto{\pgfqpoint{5.940444in}{0.556139in}}%
\pgfpathcurveto{\pgfqpoint{5.951494in}{0.556139in}}{\pgfqpoint{5.962093in}{0.560529in}}{\pgfqpoint{5.969907in}{0.568343in}}%
\pgfpathcurveto{\pgfqpoint{5.977720in}{0.576156in}}{\pgfqpoint{5.982111in}{0.586755in}}{\pgfqpoint{5.982111in}{0.597806in}}%
\pgfpathcurveto{\pgfqpoint{5.982111in}{0.608856in}}{\pgfqpoint{5.977720in}{0.619455in}}{\pgfqpoint{5.969907in}{0.627268in}}%
\pgfpathcurveto{\pgfqpoint{5.962093in}{0.635082in}}{\pgfqpoint{5.951494in}{0.639472in}}{\pgfqpoint{5.940444in}{0.639472in}}%
\pgfpathcurveto{\pgfqpoint{5.929394in}{0.639472in}}{\pgfqpoint{5.918795in}{0.635082in}}{\pgfqpoint{5.910981in}{0.627268in}}%
\pgfpathcurveto{\pgfqpoint{5.903168in}{0.619455in}}{\pgfqpoint{5.898777in}{0.608856in}}{\pgfqpoint{5.898777in}{0.597806in}}%
\pgfpathcurveto{\pgfqpoint{5.898777in}{0.586755in}}{\pgfqpoint{5.903168in}{0.576156in}}{\pgfqpoint{5.910981in}{0.568343in}}%
\pgfpathcurveto{\pgfqpoint{5.918795in}{0.560529in}}{\pgfqpoint{5.929394in}{0.556139in}}{\pgfqpoint{5.940444in}{0.556139in}}%
\pgfpathlineto{\pgfqpoint{5.940444in}{0.556139in}}%
\pgfpathclose%
\pgfusepath{stroke}%
\end{pgfscope}%
\begin{pgfscope}%
\pgfpathrectangle{\pgfqpoint{0.847223in}{0.554012in}}{\pgfqpoint{6.200000in}{4.530000in}}%
\pgfusepath{clip}%
\pgfsetbuttcap%
\pgfsetroundjoin%
\pgfsetlinewidth{1.003750pt}%
\definecolor{currentstroke}{rgb}{1.000000,0.000000,0.000000}%
\pgfsetstrokecolor{currentstroke}%
\pgfsetdash{}{0pt}%
\pgfpathmoveto{\pgfqpoint{5.945777in}{0.555702in}}%
\pgfpathcurveto{\pgfqpoint{5.956827in}{0.555702in}}{\pgfqpoint{5.967426in}{0.560092in}}{\pgfqpoint{5.975240in}{0.567905in}}%
\pgfpathcurveto{\pgfqpoint{5.983054in}{0.575719in}}{\pgfqpoint{5.987444in}{0.586318in}}{\pgfqpoint{5.987444in}{0.597368in}}%
\pgfpathcurveto{\pgfqpoint{5.987444in}{0.608418in}}{\pgfqpoint{5.983054in}{0.619017in}}{\pgfqpoint{5.975240in}{0.626831in}}%
\pgfpathcurveto{\pgfqpoint{5.967426in}{0.634645in}}{\pgfqpoint{5.956827in}{0.639035in}}{\pgfqpoint{5.945777in}{0.639035in}}%
\pgfpathcurveto{\pgfqpoint{5.934727in}{0.639035in}}{\pgfqpoint{5.924128in}{0.634645in}}{\pgfqpoint{5.916314in}{0.626831in}}%
\pgfpathcurveto{\pgfqpoint{5.908501in}{0.619017in}}{\pgfqpoint{5.904111in}{0.608418in}}{\pgfqpoint{5.904111in}{0.597368in}}%
\pgfpathcurveto{\pgfqpoint{5.904111in}{0.586318in}}{\pgfqpoint{5.908501in}{0.575719in}}{\pgfqpoint{5.916314in}{0.567905in}}%
\pgfpathcurveto{\pgfqpoint{5.924128in}{0.560092in}}{\pgfqpoint{5.934727in}{0.555702in}}{\pgfqpoint{5.945777in}{0.555702in}}%
\pgfpathlineto{\pgfqpoint{5.945777in}{0.555702in}}%
\pgfpathclose%
\pgfusepath{stroke}%
\end{pgfscope}%
\begin{pgfscope}%
\pgfpathrectangle{\pgfqpoint{0.847223in}{0.554012in}}{\pgfqpoint{6.200000in}{4.530000in}}%
\pgfusepath{clip}%
\pgfsetbuttcap%
\pgfsetroundjoin%
\pgfsetlinewidth{1.003750pt}%
\definecolor{currentstroke}{rgb}{1.000000,0.000000,0.000000}%
\pgfsetstrokecolor{currentstroke}%
\pgfsetdash{}{0pt}%
\pgfpathmoveto{\pgfqpoint{5.951110in}{0.555265in}}%
\pgfpathcurveto{\pgfqpoint{5.962161in}{0.555265in}}{\pgfqpoint{5.972760in}{0.559655in}}{\pgfqpoint{5.980573in}{0.567469in}}%
\pgfpathcurveto{\pgfqpoint{5.988387in}{0.575283in}}{\pgfqpoint{5.992777in}{0.585882in}}{\pgfqpoint{5.992777in}{0.596932in}}%
\pgfpathcurveto{\pgfqpoint{5.992777in}{0.607982in}}{\pgfqpoint{5.988387in}{0.618581in}}{\pgfqpoint{5.980573in}{0.626395in}}%
\pgfpathcurveto{\pgfqpoint{5.972760in}{0.634208in}}{\pgfqpoint{5.962161in}{0.638598in}}{\pgfqpoint{5.951110in}{0.638598in}}%
\pgfpathcurveto{\pgfqpoint{5.940060in}{0.638598in}}{\pgfqpoint{5.929461in}{0.634208in}}{\pgfqpoint{5.921648in}{0.626395in}}%
\pgfpathcurveto{\pgfqpoint{5.913834in}{0.618581in}}{\pgfqpoint{5.909444in}{0.607982in}}{\pgfqpoint{5.909444in}{0.596932in}}%
\pgfpathcurveto{\pgfqpoint{5.909444in}{0.585882in}}{\pgfqpoint{5.913834in}{0.575283in}}{\pgfqpoint{5.921648in}{0.567469in}}%
\pgfpathcurveto{\pgfqpoint{5.929461in}{0.559655in}}{\pgfqpoint{5.940060in}{0.555265in}}{\pgfqpoint{5.951110in}{0.555265in}}%
\pgfpathlineto{\pgfqpoint{5.951110in}{0.555265in}}%
\pgfpathclose%
\pgfusepath{stroke}%
\end{pgfscope}%
\begin{pgfscope}%
\pgfpathrectangle{\pgfqpoint{0.847223in}{0.554012in}}{\pgfqpoint{6.200000in}{4.530000in}}%
\pgfusepath{clip}%
\pgfsetbuttcap%
\pgfsetroundjoin%
\pgfsetlinewidth{1.003750pt}%
\definecolor{currentstroke}{rgb}{1.000000,0.000000,0.000000}%
\pgfsetstrokecolor{currentstroke}%
\pgfsetdash{}{0pt}%
\pgfpathmoveto{\pgfqpoint{5.956444in}{0.554829in}}%
\pgfpathcurveto{\pgfqpoint{5.967494in}{0.554829in}}{\pgfqpoint{5.978093in}{0.559220in}}{\pgfqpoint{5.985906in}{0.567033in}}%
\pgfpathcurveto{\pgfqpoint{5.993720in}{0.574847in}}{\pgfqpoint{5.998110in}{0.585446in}}{\pgfqpoint{5.998110in}{0.596496in}}%
\pgfpathcurveto{\pgfqpoint{5.998110in}{0.607546in}}{\pgfqpoint{5.993720in}{0.618145in}}{\pgfqpoint{5.985906in}{0.625959in}}%
\pgfpathcurveto{\pgfqpoint{5.978093in}{0.633772in}}{\pgfqpoint{5.967494in}{0.638163in}}{\pgfqpoint{5.956444in}{0.638163in}}%
\pgfpathcurveto{\pgfqpoint{5.945394in}{0.638163in}}{\pgfqpoint{5.934794in}{0.633772in}}{\pgfqpoint{5.926981in}{0.625959in}}%
\pgfpathcurveto{\pgfqpoint{5.919167in}{0.618145in}}{\pgfqpoint{5.914777in}{0.607546in}}{\pgfqpoint{5.914777in}{0.596496in}}%
\pgfpathcurveto{\pgfqpoint{5.914777in}{0.585446in}}{\pgfqpoint{5.919167in}{0.574847in}}{\pgfqpoint{5.926981in}{0.567033in}}%
\pgfpathcurveto{\pgfqpoint{5.934794in}{0.559220in}}{\pgfqpoint{5.945394in}{0.554829in}}{\pgfqpoint{5.956444in}{0.554829in}}%
\pgfpathlineto{\pgfqpoint{5.956444in}{0.554829in}}%
\pgfpathclose%
\pgfusepath{stroke}%
\end{pgfscope}%
\begin{pgfscope}%
\pgfpathrectangle{\pgfqpoint{0.847223in}{0.554012in}}{\pgfqpoint{6.200000in}{4.530000in}}%
\pgfusepath{clip}%
\pgfsetbuttcap%
\pgfsetroundjoin%
\pgfsetlinewidth{1.003750pt}%
\definecolor{currentstroke}{rgb}{1.000000,0.000000,0.000000}%
\pgfsetstrokecolor{currentstroke}%
\pgfsetdash{}{0pt}%
\pgfpathmoveto{\pgfqpoint{5.961777in}{0.554395in}}%
\pgfpathcurveto{\pgfqpoint{5.972827in}{0.554395in}}{\pgfqpoint{5.983426in}{0.558785in}}{\pgfqpoint{5.991240in}{0.566598in}}%
\pgfpathcurveto{\pgfqpoint{5.999053in}{0.574412in}}{\pgfqpoint{6.003444in}{0.585011in}}{\pgfqpoint{6.003444in}{0.596061in}}%
\pgfpathcurveto{\pgfqpoint{6.003444in}{0.607111in}}{\pgfqpoint{5.999053in}{0.617710in}}{\pgfqpoint{5.991240in}{0.625524in}}%
\pgfpathcurveto{\pgfqpoint{5.983426in}{0.633338in}}{\pgfqpoint{5.972827in}{0.637728in}}{\pgfqpoint{5.961777in}{0.637728in}}%
\pgfpathcurveto{\pgfqpoint{5.950727in}{0.637728in}}{\pgfqpoint{5.940128in}{0.633338in}}{\pgfqpoint{5.932314in}{0.625524in}}%
\pgfpathcurveto{\pgfqpoint{5.924500in}{0.617710in}}{\pgfqpoint{5.920110in}{0.607111in}}{\pgfqpoint{5.920110in}{0.596061in}}%
\pgfpathcurveto{\pgfqpoint{5.920110in}{0.585011in}}{\pgfqpoint{5.924500in}{0.574412in}}{\pgfqpoint{5.932314in}{0.566598in}}%
\pgfpathcurveto{\pgfqpoint{5.940128in}{0.558785in}}{\pgfqpoint{5.950727in}{0.554395in}}{\pgfqpoint{5.961777in}{0.554395in}}%
\pgfpathlineto{\pgfqpoint{5.961777in}{0.554395in}}%
\pgfpathclose%
\pgfusepath{stroke}%
\end{pgfscope}%
\begin{pgfscope}%
\pgfpathrectangle{\pgfqpoint{0.847223in}{0.554012in}}{\pgfqpoint{6.200000in}{4.530000in}}%
\pgfusepath{clip}%
\pgfsetbuttcap%
\pgfsetroundjoin%
\pgfsetlinewidth{1.003750pt}%
\definecolor{currentstroke}{rgb}{1.000000,0.000000,0.000000}%
\pgfsetstrokecolor{currentstroke}%
\pgfsetdash{}{0pt}%
\pgfpathmoveto{\pgfqpoint{5.967110in}{0.553961in}}%
\pgfpathcurveto{\pgfqpoint{5.978160in}{0.553961in}}{\pgfqpoint{5.988759in}{0.558351in}}{\pgfqpoint{5.996573in}{0.566164in}}%
\pgfpathcurveto{\pgfqpoint{6.004386in}{0.573978in}}{\pgfqpoint{6.008777in}{0.584577in}}{\pgfqpoint{6.008777in}{0.595627in}}%
\pgfpathcurveto{\pgfqpoint{6.008777in}{0.606677in}}{\pgfqpoint{6.004386in}{0.617276in}}{\pgfqpoint{5.996573in}{0.625090in}}%
\pgfpathcurveto{\pgfqpoint{5.988759in}{0.632904in}}{\pgfqpoint{5.978160in}{0.637294in}}{\pgfqpoint{5.967110in}{0.637294in}}%
\pgfpathcurveto{\pgfqpoint{5.956060in}{0.637294in}}{\pgfqpoint{5.945461in}{0.632904in}}{\pgfqpoint{5.937647in}{0.625090in}}%
\pgfpathcurveto{\pgfqpoint{5.929834in}{0.617276in}}{\pgfqpoint{5.925443in}{0.606677in}}{\pgfqpoint{5.925443in}{0.595627in}}%
\pgfpathcurveto{\pgfqpoint{5.925443in}{0.584577in}}{\pgfqpoint{5.929834in}{0.573978in}}{\pgfqpoint{5.937647in}{0.566164in}}%
\pgfpathcurveto{\pgfqpoint{5.945461in}{0.558351in}}{\pgfqpoint{5.956060in}{0.553961in}}{\pgfqpoint{5.967110in}{0.553961in}}%
\pgfpathlineto{\pgfqpoint{5.967110in}{0.553961in}}%
\pgfpathclose%
\pgfusepath{stroke}%
\end{pgfscope}%
\begin{pgfscope}%
\pgfpathrectangle{\pgfqpoint{0.847223in}{0.554012in}}{\pgfqpoint{6.200000in}{4.530000in}}%
\pgfusepath{clip}%
\pgfsetbuttcap%
\pgfsetroundjoin%
\pgfsetlinewidth{1.003750pt}%
\definecolor{currentstroke}{rgb}{1.000000,0.000000,0.000000}%
\pgfsetstrokecolor{currentstroke}%
\pgfsetdash{}{0pt}%
\pgfpathmoveto{\pgfqpoint{5.972443in}{0.553527in}}%
\pgfpathcurveto{\pgfqpoint{5.983493in}{0.553527in}}{\pgfqpoint{5.994092in}{0.557918in}}{\pgfqpoint{6.001906in}{0.565731in}}%
\pgfpathcurveto{\pgfqpoint{6.009720in}{0.573545in}}{\pgfqpoint{6.014110in}{0.584144in}}{\pgfqpoint{6.014110in}{0.595194in}}%
\pgfpathcurveto{\pgfqpoint{6.014110in}{0.606244in}}{\pgfqpoint{6.009720in}{0.616843in}}{\pgfqpoint{6.001906in}{0.624657in}}%
\pgfpathcurveto{\pgfqpoint{5.994092in}{0.632470in}}{\pgfqpoint{5.983493in}{0.636861in}}{\pgfqpoint{5.972443in}{0.636861in}}%
\pgfpathcurveto{\pgfqpoint{5.961393in}{0.636861in}}{\pgfqpoint{5.950794in}{0.632470in}}{\pgfqpoint{5.942981in}{0.624657in}}%
\pgfpathcurveto{\pgfqpoint{5.935167in}{0.616843in}}{\pgfqpoint{5.930777in}{0.606244in}}{\pgfqpoint{5.930777in}{0.595194in}}%
\pgfpathcurveto{\pgfqpoint{5.930777in}{0.584144in}}{\pgfqpoint{5.935167in}{0.573545in}}{\pgfqpoint{5.942981in}{0.565731in}}%
\pgfpathcurveto{\pgfqpoint{5.950794in}{0.557918in}}{\pgfqpoint{5.961393in}{0.553527in}}{\pgfqpoint{5.972443in}{0.553527in}}%
\pgfpathlineto{\pgfqpoint{5.972443in}{0.553527in}}%
\pgfpathclose%
\pgfusepath{stroke}%
\end{pgfscope}%
\begin{pgfscope}%
\pgfpathrectangle{\pgfqpoint{0.847223in}{0.554012in}}{\pgfqpoint{6.200000in}{4.530000in}}%
\pgfusepath{clip}%
\pgfsetbuttcap%
\pgfsetroundjoin%
\pgfsetlinewidth{1.003750pt}%
\definecolor{currentstroke}{rgb}{1.000000,0.000000,0.000000}%
\pgfsetstrokecolor{currentstroke}%
\pgfsetdash{}{0pt}%
\pgfpathmoveto{\pgfqpoint{5.977777in}{0.553095in}}%
\pgfpathcurveto{\pgfqpoint{5.988827in}{0.553095in}}{\pgfqpoint{5.999426in}{0.557485in}}{\pgfqpoint{6.007239in}{0.565299in}}%
\pgfpathcurveto{\pgfqpoint{6.015053in}{0.573113in}}{\pgfqpoint{6.019443in}{0.583712in}}{\pgfqpoint{6.019443in}{0.594762in}}%
\pgfpathcurveto{\pgfqpoint{6.019443in}{0.605812in}}{\pgfqpoint{6.015053in}{0.616411in}}{\pgfqpoint{6.007239in}{0.624224in}}%
\pgfpathcurveto{\pgfqpoint{5.999426in}{0.632038in}}{\pgfqpoint{5.988827in}{0.636428in}}{\pgfqpoint{5.977777in}{0.636428in}}%
\pgfpathcurveto{\pgfqpoint{5.966726in}{0.636428in}}{\pgfqpoint{5.956127in}{0.632038in}}{\pgfqpoint{5.948314in}{0.624224in}}%
\pgfpathcurveto{\pgfqpoint{5.940500in}{0.616411in}}{\pgfqpoint{5.936110in}{0.605812in}}{\pgfqpoint{5.936110in}{0.594762in}}%
\pgfpathcurveto{\pgfqpoint{5.936110in}{0.583712in}}{\pgfqpoint{5.940500in}{0.573113in}}{\pgfqpoint{5.948314in}{0.565299in}}%
\pgfpathcurveto{\pgfqpoint{5.956127in}{0.557485in}}{\pgfqpoint{5.966726in}{0.553095in}}{\pgfqpoint{5.977777in}{0.553095in}}%
\pgfpathlineto{\pgfqpoint{5.977777in}{0.553095in}}%
\pgfpathclose%
\pgfusepath{stroke}%
\end{pgfscope}%
\begin{pgfscope}%
\pgfpathrectangle{\pgfqpoint{0.847223in}{0.554012in}}{\pgfqpoint{6.200000in}{4.530000in}}%
\pgfusepath{clip}%
\pgfsetbuttcap%
\pgfsetroundjoin%
\pgfsetlinewidth{1.003750pt}%
\definecolor{currentstroke}{rgb}{1.000000,0.000000,0.000000}%
\pgfsetstrokecolor{currentstroke}%
\pgfsetdash{}{0pt}%
\pgfpathmoveto{\pgfqpoint{5.983110in}{0.552663in}}%
\pgfpathcurveto{\pgfqpoint{5.994160in}{0.552663in}}{\pgfqpoint{6.004759in}{0.557054in}}{\pgfqpoint{6.012573in}{0.564867in}}%
\pgfpathcurveto{\pgfqpoint{6.020386in}{0.572681in}}{\pgfqpoint{6.024776in}{0.583280in}}{\pgfqpoint{6.024776in}{0.594330in}}%
\pgfpathcurveto{\pgfqpoint{6.024776in}{0.605380in}}{\pgfqpoint{6.020386in}{0.615979in}}{\pgfqpoint{6.012573in}{0.623793in}}%
\pgfpathcurveto{\pgfqpoint{6.004759in}{0.631607in}}{\pgfqpoint{5.994160in}{0.635997in}}{\pgfqpoint{5.983110in}{0.635997in}}%
\pgfpathcurveto{\pgfqpoint{5.972060in}{0.635997in}}{\pgfqpoint{5.961461in}{0.631607in}}{\pgfqpoint{5.953647in}{0.623793in}}%
\pgfpathcurveto{\pgfqpoint{5.945833in}{0.615979in}}{\pgfqpoint{5.941443in}{0.605380in}}{\pgfqpoint{5.941443in}{0.594330in}}%
\pgfpathcurveto{\pgfqpoint{5.941443in}{0.583280in}}{\pgfqpoint{5.945833in}{0.572681in}}{\pgfqpoint{5.953647in}{0.564867in}}%
\pgfpathcurveto{\pgfqpoint{5.961461in}{0.557054in}}{\pgfqpoint{5.972060in}{0.552663in}}{\pgfqpoint{5.983110in}{0.552663in}}%
\pgfpathlineto{\pgfqpoint{5.983110in}{0.552663in}}%
\pgfpathclose%
\pgfusepath{stroke}%
\end{pgfscope}%
\begin{pgfscope}%
\pgfpathrectangle{\pgfqpoint{0.847223in}{0.554012in}}{\pgfqpoint{6.200000in}{4.530000in}}%
\pgfusepath{clip}%
\pgfsetbuttcap%
\pgfsetroundjoin%
\pgfsetlinewidth{1.003750pt}%
\definecolor{currentstroke}{rgb}{1.000000,0.000000,0.000000}%
\pgfsetstrokecolor{currentstroke}%
\pgfsetdash{}{0pt}%
\pgfpathmoveto{\pgfqpoint{5.988443in}{0.552233in}}%
\pgfpathcurveto{\pgfqpoint{5.999493in}{0.552233in}}{\pgfqpoint{6.010092in}{0.556623in}}{\pgfqpoint{6.017906in}{0.564437in}}%
\pgfpathcurveto{\pgfqpoint{6.025719in}{0.572250in}}{\pgfqpoint{6.030110in}{0.582849in}}{\pgfqpoint{6.030110in}{0.593899in}}%
\pgfpathcurveto{\pgfqpoint{6.030110in}{0.604950in}}{\pgfqpoint{6.025719in}{0.615549in}}{\pgfqpoint{6.017906in}{0.623362in}}%
\pgfpathcurveto{\pgfqpoint{6.010092in}{0.631176in}}{\pgfqpoint{5.999493in}{0.635566in}}{\pgfqpoint{5.988443in}{0.635566in}}%
\pgfpathcurveto{\pgfqpoint{5.977393in}{0.635566in}}{\pgfqpoint{5.966794in}{0.631176in}}{\pgfqpoint{5.958980in}{0.623362in}}%
\pgfpathcurveto{\pgfqpoint{5.951167in}{0.615549in}}{\pgfqpoint{5.946776in}{0.604950in}}{\pgfqpoint{5.946776in}{0.593899in}}%
\pgfpathcurveto{\pgfqpoint{5.946776in}{0.582849in}}{\pgfqpoint{5.951167in}{0.572250in}}{\pgfqpoint{5.958980in}{0.564437in}}%
\pgfpathcurveto{\pgfqpoint{5.966794in}{0.556623in}}{\pgfqpoint{5.977393in}{0.552233in}}{\pgfqpoint{5.988443in}{0.552233in}}%
\pgfusepath{stroke}%
\end{pgfscope}%
\begin{pgfscope}%
\pgfpathrectangle{\pgfqpoint{0.847223in}{0.554012in}}{\pgfqpoint{6.200000in}{4.530000in}}%
\pgfusepath{clip}%
\pgfsetbuttcap%
\pgfsetroundjoin%
\pgfsetlinewidth{1.003750pt}%
\definecolor{currentstroke}{rgb}{1.000000,0.000000,0.000000}%
\pgfsetstrokecolor{currentstroke}%
\pgfsetdash{}{0pt}%
\pgfpathmoveto{\pgfqpoint{5.993776in}{0.551803in}}%
\pgfpathcurveto{\pgfqpoint{6.004826in}{0.551803in}}{\pgfqpoint{6.015425in}{0.556193in}}{\pgfqpoint{6.023239in}{0.564007in}}%
\pgfpathcurveto{\pgfqpoint{6.031053in}{0.571820in}}{\pgfqpoint{6.035443in}{0.582419in}}{\pgfqpoint{6.035443in}{0.593469in}}%
\pgfpathcurveto{\pgfqpoint{6.035443in}{0.604520in}}{\pgfqpoint{6.031053in}{0.615119in}}{\pgfqpoint{6.023239in}{0.622932in}}%
\pgfpathcurveto{\pgfqpoint{6.015425in}{0.630746in}}{\pgfqpoint{6.004826in}{0.635136in}}{\pgfqpoint{5.993776in}{0.635136in}}%
\pgfpathcurveto{\pgfqpoint{5.982726in}{0.635136in}}{\pgfqpoint{5.972127in}{0.630746in}}{\pgfqpoint{5.964313in}{0.622932in}}%
\pgfpathcurveto{\pgfqpoint{5.956500in}{0.615119in}}{\pgfqpoint{5.952110in}{0.604520in}}{\pgfqpoint{5.952110in}{0.593469in}}%
\pgfpathcurveto{\pgfqpoint{5.952110in}{0.582419in}}{\pgfqpoint{5.956500in}{0.571820in}}{\pgfqpoint{5.964313in}{0.564007in}}%
\pgfpathcurveto{\pgfqpoint{5.972127in}{0.556193in}}{\pgfqpoint{5.982726in}{0.551803in}}{\pgfqpoint{5.993776in}{0.551803in}}%
\pgfusepath{stroke}%
\end{pgfscope}%
\begin{pgfscope}%
\pgfpathrectangle{\pgfqpoint{0.847223in}{0.554012in}}{\pgfqpoint{6.200000in}{4.530000in}}%
\pgfusepath{clip}%
\pgfsetbuttcap%
\pgfsetroundjoin%
\pgfsetlinewidth{1.003750pt}%
\definecolor{currentstroke}{rgb}{1.000000,0.000000,0.000000}%
\pgfsetstrokecolor{currentstroke}%
\pgfsetdash{}{0pt}%
\pgfpathmoveto{\pgfqpoint{5.999109in}{0.551374in}}%
\pgfpathcurveto{\pgfqpoint{6.010160in}{0.551374in}}{\pgfqpoint{6.020759in}{0.555764in}}{\pgfqpoint{6.028572in}{0.563578in}}%
\pgfpathcurveto{\pgfqpoint{6.036386in}{0.571391in}}{\pgfqpoint{6.040776in}{0.581990in}}{\pgfqpoint{6.040776in}{0.593040in}}%
\pgfpathcurveto{\pgfqpoint{6.040776in}{0.604090in}}{\pgfqpoint{6.036386in}{0.614690in}}{\pgfqpoint{6.028572in}{0.622503in}}%
\pgfpathcurveto{\pgfqpoint{6.020759in}{0.630317in}}{\pgfqpoint{6.010160in}{0.634707in}}{\pgfqpoint{5.999109in}{0.634707in}}%
\pgfpathcurveto{\pgfqpoint{5.988059in}{0.634707in}}{\pgfqpoint{5.977460in}{0.630317in}}{\pgfqpoint{5.969647in}{0.622503in}}%
\pgfpathcurveto{\pgfqpoint{5.961833in}{0.614690in}}{\pgfqpoint{5.957443in}{0.604090in}}{\pgfqpoint{5.957443in}{0.593040in}}%
\pgfpathcurveto{\pgfqpoint{5.957443in}{0.581990in}}{\pgfqpoint{5.961833in}{0.571391in}}{\pgfqpoint{5.969647in}{0.563578in}}%
\pgfpathcurveto{\pgfqpoint{5.977460in}{0.555764in}}{\pgfqpoint{5.988059in}{0.551374in}}{\pgfqpoint{5.999109in}{0.551374in}}%
\pgfusepath{stroke}%
\end{pgfscope}%
\begin{pgfscope}%
\pgfpathrectangle{\pgfqpoint{0.847223in}{0.554012in}}{\pgfqpoint{6.200000in}{4.530000in}}%
\pgfusepath{clip}%
\pgfsetbuttcap%
\pgfsetroundjoin%
\pgfsetlinewidth{1.003750pt}%
\definecolor{currentstroke}{rgb}{1.000000,0.000000,0.000000}%
\pgfsetstrokecolor{currentstroke}%
\pgfsetdash{}{0pt}%
\pgfpathmoveto{\pgfqpoint{6.004443in}{0.550945in}}%
\pgfpathcurveto{\pgfqpoint{6.015493in}{0.550945in}}{\pgfqpoint{6.026092in}{0.555336in}}{\pgfqpoint{6.033905in}{0.563149in}}%
\pgfpathcurveto{\pgfqpoint{6.041719in}{0.570963in}}{\pgfqpoint{6.046109in}{0.581562in}}{\pgfqpoint{6.046109in}{0.592612in}}%
\pgfpathcurveto{\pgfqpoint{6.046109in}{0.603662in}}{\pgfqpoint{6.041719in}{0.614261in}}{\pgfqpoint{6.033905in}{0.622075in}}%
\pgfpathcurveto{\pgfqpoint{6.026092in}{0.629888in}}{\pgfqpoint{6.015493in}{0.634279in}}{\pgfqpoint{6.004443in}{0.634279in}}%
\pgfpathcurveto{\pgfqpoint{5.993392in}{0.634279in}}{\pgfqpoint{5.982793in}{0.629888in}}{\pgfqpoint{5.974980in}{0.622075in}}%
\pgfpathcurveto{\pgfqpoint{5.967166in}{0.614261in}}{\pgfqpoint{5.962776in}{0.603662in}}{\pgfqpoint{5.962776in}{0.592612in}}%
\pgfpathcurveto{\pgfqpoint{5.962776in}{0.581562in}}{\pgfqpoint{5.967166in}{0.570963in}}{\pgfqpoint{5.974980in}{0.563149in}}%
\pgfpathcurveto{\pgfqpoint{5.982793in}{0.555336in}}{\pgfqpoint{5.993392in}{0.550945in}}{\pgfqpoint{6.004443in}{0.550945in}}%
\pgfusepath{stroke}%
\end{pgfscope}%
\begin{pgfscope}%
\pgfpathrectangle{\pgfqpoint{0.847223in}{0.554012in}}{\pgfqpoint{6.200000in}{4.530000in}}%
\pgfusepath{clip}%
\pgfsetbuttcap%
\pgfsetroundjoin%
\pgfsetlinewidth{1.003750pt}%
\definecolor{currentstroke}{rgb}{1.000000,0.000000,0.000000}%
\pgfsetstrokecolor{currentstroke}%
\pgfsetdash{}{0pt}%
\pgfpathmoveto{\pgfqpoint{6.009776in}{0.550518in}}%
\pgfpathcurveto{\pgfqpoint{6.020826in}{0.550518in}}{\pgfqpoint{6.031425in}{0.554908in}}{\pgfqpoint{6.039239in}{0.562722in}}%
\pgfpathcurveto{\pgfqpoint{6.047052in}{0.570535in}}{\pgfqpoint{6.051442in}{0.581134in}}{\pgfqpoint{6.051442in}{0.592185in}}%
\pgfpathcurveto{\pgfqpoint{6.051442in}{0.603235in}}{\pgfqpoint{6.047052in}{0.613834in}}{\pgfqpoint{6.039239in}{0.621647in}}%
\pgfpathcurveto{\pgfqpoint{6.031425in}{0.629461in}}{\pgfqpoint{6.020826in}{0.633851in}}{\pgfqpoint{6.009776in}{0.633851in}}%
\pgfpathcurveto{\pgfqpoint{5.998726in}{0.633851in}}{\pgfqpoint{5.988127in}{0.629461in}}{\pgfqpoint{5.980313in}{0.621647in}}%
\pgfpathcurveto{\pgfqpoint{5.972499in}{0.613834in}}{\pgfqpoint{5.968109in}{0.603235in}}{\pgfqpoint{5.968109in}{0.592185in}}%
\pgfpathcurveto{\pgfqpoint{5.968109in}{0.581134in}}{\pgfqpoint{5.972499in}{0.570535in}}{\pgfqpoint{5.980313in}{0.562722in}}%
\pgfpathcurveto{\pgfqpoint{5.988127in}{0.554908in}}{\pgfqpoint{5.998726in}{0.550518in}}{\pgfqpoint{6.009776in}{0.550518in}}%
\pgfusepath{stroke}%
\end{pgfscope}%
\begin{pgfscope}%
\pgfpathrectangle{\pgfqpoint{0.847223in}{0.554012in}}{\pgfqpoint{6.200000in}{4.530000in}}%
\pgfusepath{clip}%
\pgfsetbuttcap%
\pgfsetroundjoin%
\pgfsetlinewidth{1.003750pt}%
\definecolor{currentstroke}{rgb}{1.000000,0.000000,0.000000}%
\pgfsetstrokecolor{currentstroke}%
\pgfsetdash{}{0pt}%
\pgfpathmoveto{\pgfqpoint{6.015109in}{0.550091in}}%
\pgfpathcurveto{\pgfqpoint{6.026159in}{0.550091in}}{\pgfqpoint{6.036758in}{0.554481in}}{\pgfqpoint{6.044572in}{0.562295in}}%
\pgfpathcurveto{\pgfqpoint{6.052385in}{0.570109in}}{\pgfqpoint{6.056776in}{0.580708in}}{\pgfqpoint{6.056776in}{0.591758in}}%
\pgfpathcurveto{\pgfqpoint{6.056776in}{0.602808in}}{\pgfqpoint{6.052385in}{0.613407in}}{\pgfqpoint{6.044572in}{0.621221in}}%
\pgfpathcurveto{\pgfqpoint{6.036758in}{0.629034in}}{\pgfqpoint{6.026159in}{0.633424in}}{\pgfqpoint{6.015109in}{0.633424in}}%
\pgfpathcurveto{\pgfqpoint{6.004059in}{0.633424in}}{\pgfqpoint{5.993460in}{0.629034in}}{\pgfqpoint{5.985646in}{0.621221in}}%
\pgfpathcurveto{\pgfqpoint{5.977833in}{0.613407in}}{\pgfqpoint{5.973442in}{0.602808in}}{\pgfqpoint{5.973442in}{0.591758in}}%
\pgfpathcurveto{\pgfqpoint{5.973442in}{0.580708in}}{\pgfqpoint{5.977833in}{0.570109in}}{\pgfqpoint{5.985646in}{0.562295in}}%
\pgfpathcurveto{\pgfqpoint{5.993460in}{0.554481in}}{\pgfqpoint{6.004059in}{0.550091in}}{\pgfqpoint{6.015109in}{0.550091in}}%
\pgfusepath{stroke}%
\end{pgfscope}%
\begin{pgfscope}%
\pgfpathrectangle{\pgfqpoint{0.847223in}{0.554012in}}{\pgfqpoint{6.200000in}{4.530000in}}%
\pgfusepath{clip}%
\pgfsetbuttcap%
\pgfsetroundjoin%
\pgfsetlinewidth{1.003750pt}%
\definecolor{currentstroke}{rgb}{1.000000,0.000000,0.000000}%
\pgfsetstrokecolor{currentstroke}%
\pgfsetdash{}{0pt}%
\pgfpathmoveto{\pgfqpoint{6.020442in}{0.549665in}}%
\pgfpathcurveto{\pgfqpoint{6.031492in}{0.549665in}}{\pgfqpoint{6.042091in}{0.554055in}}{\pgfqpoint{6.049905in}{0.561869in}}%
\pgfpathcurveto{\pgfqpoint{6.057719in}{0.569683in}}{\pgfqpoint{6.062109in}{0.580282in}}{\pgfqpoint{6.062109in}{0.591332in}}%
\pgfpathcurveto{\pgfqpoint{6.062109in}{0.602382in}}{\pgfqpoint{6.057719in}{0.612981in}}{\pgfqpoint{6.049905in}{0.620795in}}%
\pgfpathcurveto{\pgfqpoint{6.042091in}{0.628608in}}{\pgfqpoint{6.031492in}{0.632999in}}{\pgfqpoint{6.020442in}{0.632999in}}%
\pgfpathcurveto{\pgfqpoint{6.009392in}{0.632999in}}{\pgfqpoint{5.998793in}{0.628608in}}{\pgfqpoint{5.990979in}{0.620795in}}%
\pgfpathcurveto{\pgfqpoint{5.983166in}{0.612981in}}{\pgfqpoint{5.978776in}{0.602382in}}{\pgfqpoint{5.978776in}{0.591332in}}%
\pgfpathcurveto{\pgfqpoint{5.978776in}{0.580282in}}{\pgfqpoint{5.983166in}{0.569683in}}{\pgfqpoint{5.990979in}{0.561869in}}%
\pgfpathcurveto{\pgfqpoint{5.998793in}{0.554055in}}{\pgfqpoint{6.009392in}{0.549665in}}{\pgfqpoint{6.020442in}{0.549665in}}%
\pgfusepath{stroke}%
\end{pgfscope}%
\begin{pgfscope}%
\pgfpathrectangle{\pgfqpoint{0.847223in}{0.554012in}}{\pgfqpoint{6.200000in}{4.530000in}}%
\pgfusepath{clip}%
\pgfsetbuttcap%
\pgfsetroundjoin%
\pgfsetlinewidth{1.003750pt}%
\definecolor{currentstroke}{rgb}{1.000000,0.000000,0.000000}%
\pgfsetstrokecolor{currentstroke}%
\pgfsetdash{}{0pt}%
\pgfpathmoveto{\pgfqpoint{6.025775in}{0.549240in}}%
\pgfpathcurveto{\pgfqpoint{6.036826in}{0.549240in}}{\pgfqpoint{6.047425in}{0.553630in}}{\pgfqpoint{6.055238in}{0.561444in}}%
\pgfpathcurveto{\pgfqpoint{6.063052in}{0.569258in}}{\pgfqpoint{6.067442in}{0.579857in}}{\pgfqpoint{6.067442in}{0.590907in}}%
\pgfpathcurveto{\pgfqpoint{6.067442in}{0.601957in}}{\pgfqpoint{6.063052in}{0.612556in}}{\pgfqpoint{6.055238in}{0.620370in}}%
\pgfpathcurveto{\pgfqpoint{6.047425in}{0.628183in}}{\pgfqpoint{6.036826in}{0.632573in}}{\pgfqpoint{6.025775in}{0.632573in}}%
\pgfpathcurveto{\pgfqpoint{6.014725in}{0.632573in}}{\pgfqpoint{6.004126in}{0.628183in}}{\pgfqpoint{5.996313in}{0.620370in}}%
\pgfpathcurveto{\pgfqpoint{5.988499in}{0.612556in}}{\pgfqpoint{5.984109in}{0.601957in}}{\pgfqpoint{5.984109in}{0.590907in}}%
\pgfpathcurveto{\pgfqpoint{5.984109in}{0.579857in}}{\pgfqpoint{5.988499in}{0.569258in}}{\pgfqpoint{5.996313in}{0.561444in}}%
\pgfpathcurveto{\pgfqpoint{6.004126in}{0.553630in}}{\pgfqpoint{6.014725in}{0.549240in}}{\pgfqpoint{6.025775in}{0.549240in}}%
\pgfusepath{stroke}%
\end{pgfscope}%
\begin{pgfscope}%
\pgfpathrectangle{\pgfqpoint{0.847223in}{0.554012in}}{\pgfqpoint{6.200000in}{4.530000in}}%
\pgfusepath{clip}%
\pgfsetbuttcap%
\pgfsetroundjoin%
\pgfsetlinewidth{1.003750pt}%
\definecolor{currentstroke}{rgb}{1.000000,0.000000,0.000000}%
\pgfsetstrokecolor{currentstroke}%
\pgfsetdash{}{0pt}%
\pgfpathmoveto{\pgfqpoint{6.031109in}{0.548816in}}%
\pgfpathcurveto{\pgfqpoint{6.042159in}{0.548816in}}{\pgfqpoint{6.052758in}{0.553206in}}{\pgfqpoint{6.060571in}{0.561020in}}%
\pgfpathcurveto{\pgfqpoint{6.068385in}{0.568833in}}{\pgfqpoint{6.072775in}{0.579432in}}{\pgfqpoint{6.072775in}{0.590482in}}%
\pgfpathcurveto{\pgfqpoint{6.072775in}{0.601533in}}{\pgfqpoint{6.068385in}{0.612132in}}{\pgfqpoint{6.060571in}{0.619945in}}%
\pgfpathcurveto{\pgfqpoint{6.052758in}{0.627759in}}{\pgfqpoint{6.042159in}{0.632149in}}{\pgfqpoint{6.031109in}{0.632149in}}%
\pgfpathcurveto{\pgfqpoint{6.020059in}{0.632149in}}{\pgfqpoint{6.009460in}{0.627759in}}{\pgfqpoint{6.001646in}{0.619945in}}%
\pgfpathcurveto{\pgfqpoint{5.993832in}{0.612132in}}{\pgfqpoint{5.989442in}{0.601533in}}{\pgfqpoint{5.989442in}{0.590482in}}%
\pgfpathcurveto{\pgfqpoint{5.989442in}{0.579432in}}{\pgfqpoint{5.993832in}{0.568833in}}{\pgfqpoint{6.001646in}{0.561020in}}%
\pgfpathcurveto{\pgfqpoint{6.009460in}{0.553206in}}{\pgfqpoint{6.020059in}{0.548816in}}{\pgfqpoint{6.031109in}{0.548816in}}%
\pgfusepath{stroke}%
\end{pgfscope}%
\begin{pgfscope}%
\pgfpathrectangle{\pgfqpoint{0.847223in}{0.554012in}}{\pgfqpoint{6.200000in}{4.530000in}}%
\pgfusepath{clip}%
\pgfsetbuttcap%
\pgfsetroundjoin%
\pgfsetlinewidth{1.003750pt}%
\definecolor{currentstroke}{rgb}{1.000000,0.000000,0.000000}%
\pgfsetstrokecolor{currentstroke}%
\pgfsetdash{}{0pt}%
\pgfpathmoveto{\pgfqpoint{6.036442in}{0.548392in}}%
\pgfpathcurveto{\pgfqpoint{6.047492in}{0.548392in}}{\pgfqpoint{6.058091in}{0.552783in}}{\pgfqpoint{6.065905in}{0.560596in}}%
\pgfpathcurveto{\pgfqpoint{6.073718in}{0.568410in}}{\pgfqpoint{6.078109in}{0.579009in}}{\pgfqpoint{6.078109in}{0.590059in}}%
\pgfpathcurveto{\pgfqpoint{6.078109in}{0.601109in}}{\pgfqpoint{6.073718in}{0.611708in}}{\pgfqpoint{6.065905in}{0.619522in}}%
\pgfpathcurveto{\pgfqpoint{6.058091in}{0.627335in}}{\pgfqpoint{6.047492in}{0.631726in}}{\pgfqpoint{6.036442in}{0.631726in}}%
\pgfpathcurveto{\pgfqpoint{6.025392in}{0.631726in}}{\pgfqpoint{6.014793in}{0.627335in}}{\pgfqpoint{6.006979in}{0.619522in}}%
\pgfpathcurveto{\pgfqpoint{5.999165in}{0.611708in}}{\pgfqpoint{5.994775in}{0.601109in}}{\pgfqpoint{5.994775in}{0.590059in}}%
\pgfpathcurveto{\pgfqpoint{5.994775in}{0.579009in}}{\pgfqpoint{5.999165in}{0.568410in}}{\pgfqpoint{6.006979in}{0.560596in}}%
\pgfpathcurveto{\pgfqpoint{6.014793in}{0.552783in}}{\pgfqpoint{6.025392in}{0.548392in}}{\pgfqpoint{6.036442in}{0.548392in}}%
\pgfusepath{stroke}%
\end{pgfscope}%
\begin{pgfscope}%
\pgfpathrectangle{\pgfqpoint{0.847223in}{0.554012in}}{\pgfqpoint{6.200000in}{4.530000in}}%
\pgfusepath{clip}%
\pgfsetbuttcap%
\pgfsetroundjoin%
\pgfsetlinewidth{1.003750pt}%
\definecolor{currentstroke}{rgb}{1.000000,0.000000,0.000000}%
\pgfsetstrokecolor{currentstroke}%
\pgfsetdash{}{0pt}%
\pgfpathmoveto{\pgfqpoint{6.041775in}{0.547970in}}%
\pgfpathcurveto{\pgfqpoint{6.052825in}{0.547970in}}{\pgfqpoint{6.063424in}{0.552360in}}{\pgfqpoint{6.071238in}{0.560173in}}%
\pgfpathcurveto{\pgfqpoint{6.079052in}{0.567987in}}{\pgfqpoint{6.083442in}{0.578586in}}{\pgfqpoint{6.083442in}{0.589636in}}%
\pgfpathcurveto{\pgfqpoint{6.083442in}{0.600686in}}{\pgfqpoint{6.079052in}{0.611285in}}{\pgfqpoint{6.071238in}{0.619099in}}%
\pgfpathcurveto{\pgfqpoint{6.063424in}{0.626913in}}{\pgfqpoint{6.052825in}{0.631303in}}{\pgfqpoint{6.041775in}{0.631303in}}%
\pgfpathcurveto{\pgfqpoint{6.030725in}{0.631303in}}{\pgfqpoint{6.020126in}{0.626913in}}{\pgfqpoint{6.012312in}{0.619099in}}%
\pgfpathcurveto{\pgfqpoint{6.004499in}{0.611285in}}{\pgfqpoint{6.000108in}{0.600686in}}{\pgfqpoint{6.000108in}{0.589636in}}%
\pgfpathcurveto{\pgfqpoint{6.000108in}{0.578586in}}{\pgfqpoint{6.004499in}{0.567987in}}{\pgfqpoint{6.012312in}{0.560173in}}%
\pgfpathcurveto{\pgfqpoint{6.020126in}{0.552360in}}{\pgfqpoint{6.030725in}{0.547970in}}{\pgfqpoint{6.041775in}{0.547970in}}%
\pgfusepath{stroke}%
\end{pgfscope}%
\begin{pgfscope}%
\pgfpathrectangle{\pgfqpoint{0.847223in}{0.554012in}}{\pgfqpoint{6.200000in}{4.530000in}}%
\pgfusepath{clip}%
\pgfsetbuttcap%
\pgfsetroundjoin%
\pgfsetlinewidth{1.003750pt}%
\definecolor{currentstroke}{rgb}{1.000000,0.000000,0.000000}%
\pgfsetstrokecolor{currentstroke}%
\pgfsetdash{}{0pt}%
\pgfpathmoveto{\pgfqpoint{6.047108in}{0.547548in}}%
\pgfpathcurveto{\pgfqpoint{6.058158in}{0.547548in}}{\pgfqpoint{6.068757in}{0.551938in}}{\pgfqpoint{6.076571in}{0.559751in}}%
\pgfpathcurveto{\pgfqpoint{6.084385in}{0.567565in}}{\pgfqpoint{6.088775in}{0.578164in}}{\pgfqpoint{6.088775in}{0.589214in}}%
\pgfpathcurveto{\pgfqpoint{6.088775in}{0.600264in}}{\pgfqpoint{6.084385in}{0.610863in}}{\pgfqpoint{6.076571in}{0.618677in}}%
\pgfpathcurveto{\pgfqpoint{6.068757in}{0.626491in}}{\pgfqpoint{6.058158in}{0.630881in}}{\pgfqpoint{6.047108in}{0.630881in}}%
\pgfpathcurveto{\pgfqpoint{6.036058in}{0.630881in}}{\pgfqpoint{6.025459in}{0.626491in}}{\pgfqpoint{6.017646in}{0.618677in}}%
\pgfpathcurveto{\pgfqpoint{6.009832in}{0.610863in}}{\pgfqpoint{6.005442in}{0.600264in}}{\pgfqpoint{6.005442in}{0.589214in}}%
\pgfpathcurveto{\pgfqpoint{6.005442in}{0.578164in}}{\pgfqpoint{6.009832in}{0.567565in}}{\pgfqpoint{6.017646in}{0.559751in}}%
\pgfpathcurveto{\pgfqpoint{6.025459in}{0.551938in}}{\pgfqpoint{6.036058in}{0.547548in}}{\pgfqpoint{6.047108in}{0.547548in}}%
\pgfusepath{stroke}%
\end{pgfscope}%
\begin{pgfscope}%
\pgfpathrectangle{\pgfqpoint{0.847223in}{0.554012in}}{\pgfqpoint{6.200000in}{4.530000in}}%
\pgfusepath{clip}%
\pgfsetbuttcap%
\pgfsetroundjoin%
\pgfsetlinewidth{1.003750pt}%
\definecolor{currentstroke}{rgb}{1.000000,0.000000,0.000000}%
\pgfsetstrokecolor{currentstroke}%
\pgfsetdash{}{0pt}%
\pgfpathmoveto{\pgfqpoint{6.052442in}{0.547126in}}%
\pgfpathcurveto{\pgfqpoint{6.063492in}{0.547126in}}{\pgfqpoint{6.074091in}{0.551517in}}{\pgfqpoint{6.081904in}{0.559330in}}%
\pgfpathcurveto{\pgfqpoint{6.089718in}{0.567144in}}{\pgfqpoint{6.094108in}{0.577743in}}{\pgfqpoint{6.094108in}{0.588793in}}%
\pgfpathcurveto{\pgfqpoint{6.094108in}{0.599843in}}{\pgfqpoint{6.089718in}{0.610442in}}{\pgfqpoint{6.081904in}{0.618256in}}%
\pgfpathcurveto{\pgfqpoint{6.074091in}{0.626069in}}{\pgfqpoint{6.063492in}{0.630460in}}{\pgfqpoint{6.052442in}{0.630460in}}%
\pgfpathcurveto{\pgfqpoint{6.041391in}{0.630460in}}{\pgfqpoint{6.030792in}{0.626069in}}{\pgfqpoint{6.022979in}{0.618256in}}%
\pgfpathcurveto{\pgfqpoint{6.015165in}{0.610442in}}{\pgfqpoint{6.010775in}{0.599843in}}{\pgfqpoint{6.010775in}{0.588793in}}%
\pgfpathcurveto{\pgfqpoint{6.010775in}{0.577743in}}{\pgfqpoint{6.015165in}{0.567144in}}{\pgfqpoint{6.022979in}{0.559330in}}%
\pgfpathcurveto{\pgfqpoint{6.030792in}{0.551517in}}{\pgfqpoint{6.041391in}{0.547126in}}{\pgfqpoint{6.052442in}{0.547126in}}%
\pgfusepath{stroke}%
\end{pgfscope}%
\begin{pgfscope}%
\pgfpathrectangle{\pgfqpoint{0.847223in}{0.554012in}}{\pgfqpoint{6.200000in}{4.530000in}}%
\pgfusepath{clip}%
\pgfsetbuttcap%
\pgfsetroundjoin%
\pgfsetlinewidth{1.003750pt}%
\definecolor{currentstroke}{rgb}{1.000000,0.000000,0.000000}%
\pgfsetstrokecolor{currentstroke}%
\pgfsetdash{}{0pt}%
\pgfpathmoveto{\pgfqpoint{6.057775in}{0.546706in}}%
\pgfpathcurveto{\pgfqpoint{6.068825in}{0.546706in}}{\pgfqpoint{6.079424in}{0.551096in}}{\pgfqpoint{6.087238in}{0.558910in}}%
\pgfpathcurveto{\pgfqpoint{6.095051in}{0.566724in}}{\pgfqpoint{6.099441in}{0.577323in}}{\pgfqpoint{6.099441in}{0.588373in}}%
\pgfpathcurveto{\pgfqpoint{6.099441in}{0.599423in}}{\pgfqpoint{6.095051in}{0.610022in}}{\pgfqpoint{6.087238in}{0.617835in}}%
\pgfpathcurveto{\pgfqpoint{6.079424in}{0.625649in}}{\pgfqpoint{6.068825in}{0.630039in}}{\pgfqpoint{6.057775in}{0.630039in}}%
\pgfpathcurveto{\pgfqpoint{6.046725in}{0.630039in}}{\pgfqpoint{6.036126in}{0.625649in}}{\pgfqpoint{6.028312in}{0.617835in}}%
\pgfpathcurveto{\pgfqpoint{6.020498in}{0.610022in}}{\pgfqpoint{6.016108in}{0.599423in}}{\pgfqpoint{6.016108in}{0.588373in}}%
\pgfpathcurveto{\pgfqpoint{6.016108in}{0.577323in}}{\pgfqpoint{6.020498in}{0.566724in}}{\pgfqpoint{6.028312in}{0.558910in}}%
\pgfpathcurveto{\pgfqpoint{6.036126in}{0.551096in}}{\pgfqpoint{6.046725in}{0.546706in}}{\pgfqpoint{6.057775in}{0.546706in}}%
\pgfusepath{stroke}%
\end{pgfscope}%
\begin{pgfscope}%
\pgfpathrectangle{\pgfqpoint{0.847223in}{0.554012in}}{\pgfqpoint{6.200000in}{4.530000in}}%
\pgfusepath{clip}%
\pgfsetbuttcap%
\pgfsetroundjoin%
\pgfsetlinewidth{1.003750pt}%
\definecolor{currentstroke}{rgb}{1.000000,0.000000,0.000000}%
\pgfsetstrokecolor{currentstroke}%
\pgfsetdash{}{0pt}%
\pgfpathmoveto{\pgfqpoint{6.063108in}{0.546286in}}%
\pgfpathcurveto{\pgfqpoint{6.074158in}{0.546286in}}{\pgfqpoint{6.084757in}{0.550677in}}{\pgfqpoint{6.092571in}{0.558490in}}%
\pgfpathcurveto{\pgfqpoint{6.100384in}{0.566304in}}{\pgfqpoint{6.104775in}{0.576903in}}{\pgfqpoint{6.104775in}{0.587953in}}%
\pgfpathcurveto{\pgfqpoint{6.104775in}{0.599003in}}{\pgfqpoint{6.100384in}{0.609602in}}{\pgfqpoint{6.092571in}{0.617416in}}%
\pgfpathcurveto{\pgfqpoint{6.084757in}{0.625230in}}{\pgfqpoint{6.074158in}{0.629620in}}{\pgfqpoint{6.063108in}{0.629620in}}%
\pgfpathcurveto{\pgfqpoint{6.052058in}{0.629620in}}{\pgfqpoint{6.041459in}{0.625230in}}{\pgfqpoint{6.033645in}{0.617416in}}%
\pgfpathcurveto{\pgfqpoint{6.025832in}{0.609602in}}{\pgfqpoint{6.021441in}{0.599003in}}{\pgfqpoint{6.021441in}{0.587953in}}%
\pgfpathcurveto{\pgfqpoint{6.021441in}{0.576903in}}{\pgfqpoint{6.025832in}{0.566304in}}{\pgfqpoint{6.033645in}{0.558490in}}%
\pgfpathcurveto{\pgfqpoint{6.041459in}{0.550677in}}{\pgfqpoint{6.052058in}{0.546286in}}{\pgfqpoint{6.063108in}{0.546286in}}%
\pgfusepath{stroke}%
\end{pgfscope}%
\begin{pgfscope}%
\pgfpathrectangle{\pgfqpoint{0.847223in}{0.554012in}}{\pgfqpoint{6.200000in}{4.530000in}}%
\pgfusepath{clip}%
\pgfsetbuttcap%
\pgfsetroundjoin%
\pgfsetlinewidth{1.003750pt}%
\definecolor{currentstroke}{rgb}{1.000000,0.000000,0.000000}%
\pgfsetstrokecolor{currentstroke}%
\pgfsetdash{}{0pt}%
\pgfpathmoveto{\pgfqpoint{6.068441in}{0.545868in}}%
\pgfpathcurveto{\pgfqpoint{6.079491in}{0.545868in}}{\pgfqpoint{6.090090in}{0.550258in}}{\pgfqpoint{6.097904in}{0.558072in}}%
\pgfpathcurveto{\pgfqpoint{6.105718in}{0.565885in}}{\pgfqpoint{6.110108in}{0.576484in}}{\pgfqpoint{6.110108in}{0.587534in}}%
\pgfpathcurveto{\pgfqpoint{6.110108in}{0.598584in}}{\pgfqpoint{6.105718in}{0.609183in}}{\pgfqpoint{6.097904in}{0.616997in}}%
\pgfpathcurveto{\pgfqpoint{6.090090in}{0.624811in}}{\pgfqpoint{6.079491in}{0.629201in}}{\pgfqpoint{6.068441in}{0.629201in}}%
\pgfpathcurveto{\pgfqpoint{6.057391in}{0.629201in}}{\pgfqpoint{6.046792in}{0.624811in}}{\pgfqpoint{6.038978in}{0.616997in}}%
\pgfpathcurveto{\pgfqpoint{6.031165in}{0.609183in}}{\pgfqpoint{6.026775in}{0.598584in}}{\pgfqpoint{6.026775in}{0.587534in}}%
\pgfpathcurveto{\pgfqpoint{6.026775in}{0.576484in}}{\pgfqpoint{6.031165in}{0.565885in}}{\pgfqpoint{6.038978in}{0.558072in}}%
\pgfpathcurveto{\pgfqpoint{6.046792in}{0.550258in}}{\pgfqpoint{6.057391in}{0.545868in}}{\pgfqpoint{6.068441in}{0.545868in}}%
\pgfusepath{stroke}%
\end{pgfscope}%
\begin{pgfscope}%
\pgfpathrectangle{\pgfqpoint{0.847223in}{0.554012in}}{\pgfqpoint{6.200000in}{4.530000in}}%
\pgfusepath{clip}%
\pgfsetbuttcap%
\pgfsetroundjoin%
\pgfsetlinewidth{1.003750pt}%
\definecolor{currentstroke}{rgb}{1.000000,0.000000,0.000000}%
\pgfsetstrokecolor{currentstroke}%
\pgfsetdash{}{0pt}%
\pgfpathmoveto{\pgfqpoint{6.073774in}{0.545450in}}%
\pgfpathcurveto{\pgfqpoint{6.084825in}{0.545450in}}{\pgfqpoint{6.095424in}{0.549840in}}{\pgfqpoint{6.103237in}{0.557653in}}%
\pgfpathcurveto{\pgfqpoint{6.111051in}{0.565467in}}{\pgfqpoint{6.115441in}{0.576066in}}{\pgfqpoint{6.115441in}{0.587116in}}%
\pgfpathcurveto{\pgfqpoint{6.115441in}{0.598166in}}{\pgfqpoint{6.111051in}{0.608765in}}{\pgfqpoint{6.103237in}{0.616579in}}%
\pgfpathcurveto{\pgfqpoint{6.095424in}{0.624393in}}{\pgfqpoint{6.084825in}{0.628783in}}{\pgfqpoint{6.073774in}{0.628783in}}%
\pgfpathcurveto{\pgfqpoint{6.062724in}{0.628783in}}{\pgfqpoint{6.052125in}{0.624393in}}{\pgfqpoint{6.044312in}{0.616579in}}%
\pgfpathcurveto{\pgfqpoint{6.036498in}{0.608765in}}{\pgfqpoint{6.032108in}{0.598166in}}{\pgfqpoint{6.032108in}{0.587116in}}%
\pgfpathcurveto{\pgfqpoint{6.032108in}{0.576066in}}{\pgfqpoint{6.036498in}{0.565467in}}{\pgfqpoint{6.044312in}{0.557653in}}%
\pgfpathcurveto{\pgfqpoint{6.052125in}{0.549840in}}{\pgfqpoint{6.062724in}{0.545450in}}{\pgfqpoint{6.073774in}{0.545450in}}%
\pgfusepath{stroke}%
\end{pgfscope}%
\begin{pgfscope}%
\pgfpathrectangle{\pgfqpoint{0.847223in}{0.554012in}}{\pgfqpoint{6.200000in}{4.530000in}}%
\pgfusepath{clip}%
\pgfsetbuttcap%
\pgfsetroundjoin%
\pgfsetlinewidth{1.003750pt}%
\definecolor{currentstroke}{rgb}{1.000000,0.000000,0.000000}%
\pgfsetstrokecolor{currentstroke}%
\pgfsetdash{}{0pt}%
\pgfpathmoveto{\pgfqpoint{6.079108in}{0.545032in}}%
\pgfpathcurveto{\pgfqpoint{6.090158in}{0.545032in}}{\pgfqpoint{6.100757in}{0.549423in}}{\pgfqpoint{6.108570in}{0.557236in}}%
\pgfpathcurveto{\pgfqpoint{6.116384in}{0.565050in}}{\pgfqpoint{6.120774in}{0.575649in}}{\pgfqpoint{6.120774in}{0.586699in}}%
\pgfpathcurveto{\pgfqpoint{6.120774in}{0.597749in}}{\pgfqpoint{6.116384in}{0.608348in}}{\pgfqpoint{6.108570in}{0.616162in}}%
\pgfpathcurveto{\pgfqpoint{6.100757in}{0.623975in}}{\pgfqpoint{6.090158in}{0.628366in}}{\pgfqpoint{6.079108in}{0.628366in}}%
\pgfpathcurveto{\pgfqpoint{6.068057in}{0.628366in}}{\pgfqpoint{6.057458in}{0.623975in}}{\pgfqpoint{6.049645in}{0.616162in}}%
\pgfpathcurveto{\pgfqpoint{6.041831in}{0.608348in}}{\pgfqpoint{6.037441in}{0.597749in}}{\pgfqpoint{6.037441in}{0.586699in}}%
\pgfpathcurveto{\pgfqpoint{6.037441in}{0.575649in}}{\pgfqpoint{6.041831in}{0.565050in}}{\pgfqpoint{6.049645in}{0.557236in}}%
\pgfpathcurveto{\pgfqpoint{6.057458in}{0.549423in}}{\pgfqpoint{6.068057in}{0.545032in}}{\pgfqpoint{6.079108in}{0.545032in}}%
\pgfusepath{stroke}%
\end{pgfscope}%
\begin{pgfscope}%
\pgfpathrectangle{\pgfqpoint{0.847223in}{0.554012in}}{\pgfqpoint{6.200000in}{4.530000in}}%
\pgfusepath{clip}%
\pgfsetbuttcap%
\pgfsetroundjoin%
\pgfsetlinewidth{1.003750pt}%
\definecolor{currentstroke}{rgb}{1.000000,0.000000,0.000000}%
\pgfsetstrokecolor{currentstroke}%
\pgfsetdash{}{0pt}%
\pgfpathmoveto{\pgfqpoint{6.084441in}{0.544616in}}%
\pgfpathcurveto{\pgfqpoint{6.095491in}{0.544616in}}{\pgfqpoint{6.106090in}{0.549006in}}{\pgfqpoint{6.113904in}{0.556820in}}%
\pgfpathcurveto{\pgfqpoint{6.121717in}{0.564633in}}{\pgfqpoint{6.126108in}{0.575232in}}{\pgfqpoint{6.126108in}{0.586282in}}%
\pgfpathcurveto{\pgfqpoint{6.126108in}{0.597333in}}{\pgfqpoint{6.121717in}{0.607932in}}{\pgfqpoint{6.113904in}{0.615745in}}%
\pgfpathcurveto{\pgfqpoint{6.106090in}{0.623559in}}{\pgfqpoint{6.095491in}{0.627949in}}{\pgfqpoint{6.084441in}{0.627949in}}%
\pgfpathcurveto{\pgfqpoint{6.073391in}{0.627949in}}{\pgfqpoint{6.062792in}{0.623559in}}{\pgfqpoint{6.054978in}{0.615745in}}%
\pgfpathcurveto{\pgfqpoint{6.047164in}{0.607932in}}{\pgfqpoint{6.042774in}{0.597333in}}{\pgfqpoint{6.042774in}{0.586282in}}%
\pgfpathcurveto{\pgfqpoint{6.042774in}{0.575232in}}{\pgfqpoint{6.047164in}{0.564633in}}{\pgfqpoint{6.054978in}{0.556820in}}%
\pgfpathcurveto{\pgfqpoint{6.062792in}{0.549006in}}{\pgfqpoint{6.073391in}{0.544616in}}{\pgfqpoint{6.084441in}{0.544616in}}%
\pgfusepath{stroke}%
\end{pgfscope}%
\begin{pgfscope}%
\pgfpathrectangle{\pgfqpoint{0.847223in}{0.554012in}}{\pgfqpoint{6.200000in}{4.530000in}}%
\pgfusepath{clip}%
\pgfsetbuttcap%
\pgfsetroundjoin%
\pgfsetlinewidth{1.003750pt}%
\definecolor{currentstroke}{rgb}{1.000000,0.000000,0.000000}%
\pgfsetstrokecolor{currentstroke}%
\pgfsetdash{}{0pt}%
\pgfpathmoveto{\pgfqpoint{6.089774in}{0.544200in}}%
\pgfpathcurveto{\pgfqpoint{6.100824in}{0.544200in}}{\pgfqpoint{6.111423in}{0.548590in}}{\pgfqpoint{6.119237in}{0.556404in}}%
\pgfpathcurveto{\pgfqpoint{6.127050in}{0.564218in}}{\pgfqpoint{6.131441in}{0.574817in}}{\pgfqpoint{6.131441in}{0.585867in}}%
\pgfpathcurveto{\pgfqpoint{6.131441in}{0.596917in}}{\pgfqpoint{6.127050in}{0.607516in}}{\pgfqpoint{6.119237in}{0.615330in}}%
\pgfpathcurveto{\pgfqpoint{6.111423in}{0.623143in}}{\pgfqpoint{6.100824in}{0.627533in}}{\pgfqpoint{6.089774in}{0.627533in}}%
\pgfpathcurveto{\pgfqpoint{6.078724in}{0.627533in}}{\pgfqpoint{6.068125in}{0.623143in}}{\pgfqpoint{6.060311in}{0.615330in}}%
\pgfpathcurveto{\pgfqpoint{6.052498in}{0.607516in}}{\pgfqpoint{6.048107in}{0.596917in}}{\pgfqpoint{6.048107in}{0.585867in}}%
\pgfpathcurveto{\pgfqpoint{6.048107in}{0.574817in}}{\pgfqpoint{6.052498in}{0.564218in}}{\pgfqpoint{6.060311in}{0.556404in}}%
\pgfpathcurveto{\pgfqpoint{6.068125in}{0.548590in}}{\pgfqpoint{6.078724in}{0.544200in}}{\pgfqpoint{6.089774in}{0.544200in}}%
\pgfusepath{stroke}%
\end{pgfscope}%
\begin{pgfscope}%
\pgfpathrectangle{\pgfqpoint{0.847223in}{0.554012in}}{\pgfqpoint{6.200000in}{4.530000in}}%
\pgfusepath{clip}%
\pgfsetbuttcap%
\pgfsetroundjoin%
\pgfsetlinewidth{1.003750pt}%
\definecolor{currentstroke}{rgb}{1.000000,0.000000,0.000000}%
\pgfsetstrokecolor{currentstroke}%
\pgfsetdash{}{0pt}%
\pgfpathmoveto{\pgfqpoint{6.095107in}{0.543785in}}%
\pgfpathcurveto{\pgfqpoint{6.106157in}{0.543785in}}{\pgfqpoint{6.116756in}{0.548175in}}{\pgfqpoint{6.124570in}{0.555989in}}%
\pgfpathcurveto{\pgfqpoint{6.132384in}{0.563803in}}{\pgfqpoint{6.136774in}{0.574402in}}{\pgfqpoint{6.136774in}{0.585452in}}%
\pgfpathcurveto{\pgfqpoint{6.136774in}{0.596502in}}{\pgfqpoint{6.132384in}{0.607101in}}{\pgfqpoint{6.124570in}{0.614915in}}%
\pgfpathcurveto{\pgfqpoint{6.116756in}{0.622728in}}{\pgfqpoint{6.106157in}{0.627118in}}{\pgfqpoint{6.095107in}{0.627118in}}%
\pgfpathcurveto{\pgfqpoint{6.084057in}{0.627118in}}{\pgfqpoint{6.073458in}{0.622728in}}{\pgfqpoint{6.065644in}{0.614915in}}%
\pgfpathcurveto{\pgfqpoint{6.057831in}{0.607101in}}{\pgfqpoint{6.053441in}{0.596502in}}{\pgfqpoint{6.053441in}{0.585452in}}%
\pgfpathcurveto{\pgfqpoint{6.053441in}{0.574402in}}{\pgfqpoint{6.057831in}{0.563803in}}{\pgfqpoint{6.065644in}{0.555989in}}%
\pgfpathcurveto{\pgfqpoint{6.073458in}{0.548175in}}{\pgfqpoint{6.084057in}{0.543785in}}{\pgfqpoint{6.095107in}{0.543785in}}%
\pgfusepath{stroke}%
\end{pgfscope}%
\begin{pgfscope}%
\pgfpathrectangle{\pgfqpoint{0.847223in}{0.554012in}}{\pgfqpoint{6.200000in}{4.530000in}}%
\pgfusepath{clip}%
\pgfsetbuttcap%
\pgfsetroundjoin%
\pgfsetlinewidth{1.003750pt}%
\definecolor{currentstroke}{rgb}{1.000000,0.000000,0.000000}%
\pgfsetstrokecolor{currentstroke}%
\pgfsetdash{}{0pt}%
\pgfpathmoveto{\pgfqpoint{6.100440in}{0.543371in}}%
\pgfpathcurveto{\pgfqpoint{6.111491in}{0.543371in}}{\pgfqpoint{6.122090in}{0.547761in}}{\pgfqpoint{6.129903in}{0.555575in}}%
\pgfpathcurveto{\pgfqpoint{6.137717in}{0.563388in}}{\pgfqpoint{6.142107in}{0.573987in}}{\pgfqpoint{6.142107in}{0.585038in}}%
\pgfpathcurveto{\pgfqpoint{6.142107in}{0.596088in}}{\pgfqpoint{6.137717in}{0.606687in}}{\pgfqpoint{6.129903in}{0.614500in}}%
\pgfpathcurveto{\pgfqpoint{6.122090in}{0.622314in}}{\pgfqpoint{6.111491in}{0.626704in}}{\pgfqpoint{6.100440in}{0.626704in}}%
\pgfpathcurveto{\pgfqpoint{6.089390in}{0.626704in}}{\pgfqpoint{6.078791in}{0.622314in}}{\pgfqpoint{6.070978in}{0.614500in}}%
\pgfpathcurveto{\pgfqpoint{6.063164in}{0.606687in}}{\pgfqpoint{6.058774in}{0.596088in}}{\pgfqpoint{6.058774in}{0.585038in}}%
\pgfpathcurveto{\pgfqpoint{6.058774in}{0.573987in}}{\pgfqpoint{6.063164in}{0.563388in}}{\pgfqpoint{6.070978in}{0.555575in}}%
\pgfpathcurveto{\pgfqpoint{6.078791in}{0.547761in}}{\pgfqpoint{6.089390in}{0.543371in}}{\pgfqpoint{6.100440in}{0.543371in}}%
\pgfusepath{stroke}%
\end{pgfscope}%
\begin{pgfscope}%
\pgfpathrectangle{\pgfqpoint{0.847223in}{0.554012in}}{\pgfqpoint{6.200000in}{4.530000in}}%
\pgfusepath{clip}%
\pgfsetbuttcap%
\pgfsetroundjoin%
\pgfsetlinewidth{1.003750pt}%
\definecolor{currentstroke}{rgb}{1.000000,0.000000,0.000000}%
\pgfsetstrokecolor{currentstroke}%
\pgfsetdash{}{0pt}%
\pgfpathmoveto{\pgfqpoint{6.105774in}{0.542958in}}%
\pgfpathcurveto{\pgfqpoint{6.116824in}{0.542958in}}{\pgfqpoint{6.127423in}{0.547348in}}{\pgfqpoint{6.135236in}{0.555161in}}%
\pgfpathcurveto{\pgfqpoint{6.143050in}{0.562975in}}{\pgfqpoint{6.147440in}{0.573574in}}{\pgfqpoint{6.147440in}{0.584624in}}%
\pgfpathcurveto{\pgfqpoint{6.147440in}{0.595674in}}{\pgfqpoint{6.143050in}{0.606273in}}{\pgfqpoint{6.135236in}{0.614087in}}%
\pgfpathcurveto{\pgfqpoint{6.127423in}{0.621901in}}{\pgfqpoint{6.116824in}{0.626291in}}{\pgfqpoint{6.105774in}{0.626291in}}%
\pgfpathcurveto{\pgfqpoint{6.094724in}{0.626291in}}{\pgfqpoint{6.084125in}{0.621901in}}{\pgfqpoint{6.076311in}{0.614087in}}%
\pgfpathcurveto{\pgfqpoint{6.068497in}{0.606273in}}{\pgfqpoint{6.064107in}{0.595674in}}{\pgfqpoint{6.064107in}{0.584624in}}%
\pgfpathcurveto{\pgfqpoint{6.064107in}{0.573574in}}{\pgfqpoint{6.068497in}{0.562975in}}{\pgfqpoint{6.076311in}{0.555161in}}%
\pgfpathcurveto{\pgfqpoint{6.084125in}{0.547348in}}{\pgfqpoint{6.094724in}{0.542958in}}{\pgfqpoint{6.105774in}{0.542958in}}%
\pgfusepath{stroke}%
\end{pgfscope}%
\begin{pgfscope}%
\pgfpathrectangle{\pgfqpoint{0.847223in}{0.554012in}}{\pgfqpoint{6.200000in}{4.530000in}}%
\pgfusepath{clip}%
\pgfsetbuttcap%
\pgfsetroundjoin%
\pgfsetlinewidth{1.003750pt}%
\definecolor{currentstroke}{rgb}{1.000000,0.000000,0.000000}%
\pgfsetstrokecolor{currentstroke}%
\pgfsetdash{}{0pt}%
\pgfpathmoveto{\pgfqpoint{6.111107in}{0.542545in}}%
\pgfpathcurveto{\pgfqpoint{6.122157in}{0.542545in}}{\pgfqpoint{6.132756in}{0.546935in}}{\pgfqpoint{6.140570in}{0.554749in}}%
\pgfpathcurveto{\pgfqpoint{6.148383in}{0.562562in}}{\pgfqpoint{6.152774in}{0.573161in}}{\pgfqpoint{6.152774in}{0.584212in}}%
\pgfpathcurveto{\pgfqpoint{6.152774in}{0.595262in}}{\pgfqpoint{6.148383in}{0.605861in}}{\pgfqpoint{6.140570in}{0.613674in}}%
\pgfpathcurveto{\pgfqpoint{6.132756in}{0.621488in}}{\pgfqpoint{6.122157in}{0.625878in}}{\pgfqpoint{6.111107in}{0.625878in}}%
\pgfpathcurveto{\pgfqpoint{6.100057in}{0.625878in}}{\pgfqpoint{6.089458in}{0.621488in}}{\pgfqpoint{6.081644in}{0.613674in}}%
\pgfpathcurveto{\pgfqpoint{6.073831in}{0.605861in}}{\pgfqpoint{6.069440in}{0.595262in}}{\pgfqpoint{6.069440in}{0.584212in}}%
\pgfpathcurveto{\pgfqpoint{6.069440in}{0.573161in}}{\pgfqpoint{6.073831in}{0.562562in}}{\pgfqpoint{6.081644in}{0.554749in}}%
\pgfpathcurveto{\pgfqpoint{6.089458in}{0.546935in}}{\pgfqpoint{6.100057in}{0.542545in}}{\pgfqpoint{6.111107in}{0.542545in}}%
\pgfusepath{stroke}%
\end{pgfscope}%
\begin{pgfscope}%
\pgfpathrectangle{\pgfqpoint{0.847223in}{0.554012in}}{\pgfqpoint{6.200000in}{4.530000in}}%
\pgfusepath{clip}%
\pgfsetbuttcap%
\pgfsetroundjoin%
\pgfsetlinewidth{1.003750pt}%
\definecolor{currentstroke}{rgb}{1.000000,0.000000,0.000000}%
\pgfsetstrokecolor{currentstroke}%
\pgfsetdash{}{0pt}%
\pgfpathmoveto{\pgfqpoint{6.116440in}{0.542133in}}%
\pgfpathcurveto{\pgfqpoint{6.127490in}{0.542133in}}{\pgfqpoint{6.138089in}{0.546523in}}{\pgfqpoint{6.145903in}{0.554337in}}%
\pgfpathcurveto{\pgfqpoint{6.153717in}{0.562150in}}{\pgfqpoint{6.158107in}{0.572749in}}{\pgfqpoint{6.158107in}{0.583800in}}%
\pgfpathcurveto{\pgfqpoint{6.158107in}{0.594850in}}{\pgfqpoint{6.153717in}{0.605449in}}{\pgfqpoint{6.145903in}{0.613262in}}%
\pgfpathcurveto{\pgfqpoint{6.138089in}{0.621076in}}{\pgfqpoint{6.127490in}{0.625466in}}{\pgfqpoint{6.116440in}{0.625466in}}%
\pgfpathcurveto{\pgfqpoint{6.105390in}{0.625466in}}{\pgfqpoint{6.094791in}{0.621076in}}{\pgfqpoint{6.086977in}{0.613262in}}%
\pgfpathcurveto{\pgfqpoint{6.079164in}{0.605449in}}{\pgfqpoint{6.074773in}{0.594850in}}{\pgfqpoint{6.074773in}{0.583800in}}%
\pgfpathcurveto{\pgfqpoint{6.074773in}{0.572749in}}{\pgfqpoint{6.079164in}{0.562150in}}{\pgfqpoint{6.086977in}{0.554337in}}%
\pgfpathcurveto{\pgfqpoint{6.094791in}{0.546523in}}{\pgfqpoint{6.105390in}{0.542133in}}{\pgfqpoint{6.116440in}{0.542133in}}%
\pgfusepath{stroke}%
\end{pgfscope}%
\begin{pgfscope}%
\pgfpathrectangle{\pgfqpoint{0.847223in}{0.554012in}}{\pgfqpoint{6.200000in}{4.530000in}}%
\pgfusepath{clip}%
\pgfsetbuttcap%
\pgfsetroundjoin%
\pgfsetlinewidth{1.003750pt}%
\definecolor{currentstroke}{rgb}{1.000000,0.000000,0.000000}%
\pgfsetstrokecolor{currentstroke}%
\pgfsetdash{}{0pt}%
\pgfpathmoveto{\pgfqpoint{6.121773in}{0.541722in}}%
\pgfpathcurveto{\pgfqpoint{6.132823in}{0.541722in}}{\pgfqpoint{6.143423in}{0.546112in}}{\pgfqpoint{6.151236in}{0.553926in}}%
\pgfpathcurveto{\pgfqpoint{6.159050in}{0.561739in}}{\pgfqpoint{6.163440in}{0.572338in}}{\pgfqpoint{6.163440in}{0.583388in}}%
\pgfpathcurveto{\pgfqpoint{6.163440in}{0.594439in}}{\pgfqpoint{6.159050in}{0.605038in}}{\pgfqpoint{6.151236in}{0.612851in}}%
\pgfpathcurveto{\pgfqpoint{6.143423in}{0.620665in}}{\pgfqpoint{6.132823in}{0.625055in}}{\pgfqpoint{6.121773in}{0.625055in}}%
\pgfpathcurveto{\pgfqpoint{6.110723in}{0.625055in}}{\pgfqpoint{6.100124in}{0.620665in}}{\pgfqpoint{6.092311in}{0.612851in}}%
\pgfpathcurveto{\pgfqpoint{6.084497in}{0.605038in}}{\pgfqpoint{6.080107in}{0.594439in}}{\pgfqpoint{6.080107in}{0.583388in}}%
\pgfpathcurveto{\pgfqpoint{6.080107in}{0.572338in}}{\pgfqpoint{6.084497in}{0.561739in}}{\pgfqpoint{6.092311in}{0.553926in}}%
\pgfpathcurveto{\pgfqpoint{6.100124in}{0.546112in}}{\pgfqpoint{6.110723in}{0.541722in}}{\pgfqpoint{6.121773in}{0.541722in}}%
\pgfusepath{stroke}%
\end{pgfscope}%
\begin{pgfscope}%
\pgfpathrectangle{\pgfqpoint{0.847223in}{0.554012in}}{\pgfqpoint{6.200000in}{4.530000in}}%
\pgfusepath{clip}%
\pgfsetbuttcap%
\pgfsetroundjoin%
\pgfsetlinewidth{1.003750pt}%
\definecolor{currentstroke}{rgb}{1.000000,0.000000,0.000000}%
\pgfsetstrokecolor{currentstroke}%
\pgfsetdash{}{0pt}%
\pgfpathmoveto{\pgfqpoint{6.127107in}{0.541311in}}%
\pgfpathcurveto{\pgfqpoint{6.138157in}{0.541311in}}{\pgfqpoint{6.148756in}{0.545702in}}{\pgfqpoint{6.156569in}{0.553515in}}%
\pgfpathcurveto{\pgfqpoint{6.164383in}{0.561329in}}{\pgfqpoint{6.168773in}{0.571928in}}{\pgfqpoint{6.168773in}{0.582978in}}%
\pgfpathcurveto{\pgfqpoint{6.168773in}{0.594028in}}{\pgfqpoint{6.164383in}{0.604627in}}{\pgfqpoint{6.156569in}{0.612441in}}%
\pgfpathcurveto{\pgfqpoint{6.148756in}{0.620254in}}{\pgfqpoint{6.138157in}{0.624645in}}{\pgfqpoint{6.127107in}{0.624645in}}%
\pgfpathcurveto{\pgfqpoint{6.116056in}{0.624645in}}{\pgfqpoint{6.105457in}{0.620254in}}{\pgfqpoint{6.097644in}{0.612441in}}%
\pgfpathcurveto{\pgfqpoint{6.089830in}{0.604627in}}{\pgfqpoint{6.085440in}{0.594028in}}{\pgfqpoint{6.085440in}{0.582978in}}%
\pgfpathcurveto{\pgfqpoint{6.085440in}{0.571928in}}{\pgfqpoint{6.089830in}{0.561329in}}{\pgfqpoint{6.097644in}{0.553515in}}%
\pgfpathcurveto{\pgfqpoint{6.105457in}{0.545702in}}{\pgfqpoint{6.116056in}{0.541311in}}{\pgfqpoint{6.127107in}{0.541311in}}%
\pgfusepath{stroke}%
\end{pgfscope}%
\begin{pgfscope}%
\pgfpathrectangle{\pgfqpoint{0.847223in}{0.554012in}}{\pgfqpoint{6.200000in}{4.530000in}}%
\pgfusepath{clip}%
\pgfsetbuttcap%
\pgfsetroundjoin%
\pgfsetlinewidth{1.003750pt}%
\definecolor{currentstroke}{rgb}{1.000000,0.000000,0.000000}%
\pgfsetstrokecolor{currentstroke}%
\pgfsetdash{}{0pt}%
\pgfpathmoveto{\pgfqpoint{6.132440in}{0.540902in}}%
\pgfpathcurveto{\pgfqpoint{6.143490in}{0.540902in}}{\pgfqpoint{6.154089in}{0.545292in}}{\pgfqpoint{6.161903in}{0.553106in}}%
\pgfpathcurveto{\pgfqpoint{6.169716in}{0.560919in}}{\pgfqpoint{6.174106in}{0.571518in}}{\pgfqpoint{6.174106in}{0.582568in}}%
\pgfpathcurveto{\pgfqpoint{6.174106in}{0.593619in}}{\pgfqpoint{6.169716in}{0.604218in}}{\pgfqpoint{6.161903in}{0.612031in}}%
\pgfpathcurveto{\pgfqpoint{6.154089in}{0.619845in}}{\pgfqpoint{6.143490in}{0.624235in}}{\pgfqpoint{6.132440in}{0.624235in}}%
\pgfpathcurveto{\pgfqpoint{6.121390in}{0.624235in}}{\pgfqpoint{6.110791in}{0.619845in}}{\pgfqpoint{6.102977in}{0.612031in}}%
\pgfpathcurveto{\pgfqpoint{6.095163in}{0.604218in}}{\pgfqpoint{6.090773in}{0.593619in}}{\pgfqpoint{6.090773in}{0.582568in}}%
\pgfpathcurveto{\pgfqpoint{6.090773in}{0.571518in}}{\pgfqpoint{6.095163in}{0.560919in}}{\pgfqpoint{6.102977in}{0.553106in}}%
\pgfpathcurveto{\pgfqpoint{6.110791in}{0.545292in}}{\pgfqpoint{6.121390in}{0.540902in}}{\pgfqpoint{6.132440in}{0.540902in}}%
\pgfusepath{stroke}%
\end{pgfscope}%
\begin{pgfscope}%
\pgfpathrectangle{\pgfqpoint{0.847223in}{0.554012in}}{\pgfqpoint{6.200000in}{4.530000in}}%
\pgfusepath{clip}%
\pgfsetbuttcap%
\pgfsetroundjoin%
\pgfsetlinewidth{1.003750pt}%
\definecolor{currentstroke}{rgb}{1.000000,0.000000,0.000000}%
\pgfsetstrokecolor{currentstroke}%
\pgfsetdash{}{0pt}%
\pgfpathmoveto{\pgfqpoint{6.137773in}{0.540493in}}%
\pgfpathcurveto{\pgfqpoint{6.148823in}{0.540493in}}{\pgfqpoint{6.159422in}{0.544883in}}{\pgfqpoint{6.167236in}{0.552697in}}%
\pgfpathcurveto{\pgfqpoint{6.175049in}{0.560510in}}{\pgfqpoint{6.179440in}{0.571109in}}{\pgfqpoint{6.179440in}{0.582160in}}%
\pgfpathcurveto{\pgfqpoint{6.179440in}{0.593210in}}{\pgfqpoint{6.175049in}{0.603809in}}{\pgfqpoint{6.167236in}{0.611622in}}%
\pgfpathcurveto{\pgfqpoint{6.159422in}{0.619436in}}{\pgfqpoint{6.148823in}{0.623826in}}{\pgfqpoint{6.137773in}{0.623826in}}%
\pgfpathcurveto{\pgfqpoint{6.126723in}{0.623826in}}{\pgfqpoint{6.116124in}{0.619436in}}{\pgfqpoint{6.108310in}{0.611622in}}%
\pgfpathcurveto{\pgfqpoint{6.100497in}{0.603809in}}{\pgfqpoint{6.096106in}{0.593210in}}{\pgfqpoint{6.096106in}{0.582160in}}%
\pgfpathcurveto{\pgfqpoint{6.096106in}{0.571109in}}{\pgfqpoint{6.100497in}{0.560510in}}{\pgfqpoint{6.108310in}{0.552697in}}%
\pgfpathcurveto{\pgfqpoint{6.116124in}{0.544883in}}{\pgfqpoint{6.126723in}{0.540493in}}{\pgfqpoint{6.137773in}{0.540493in}}%
\pgfusepath{stroke}%
\end{pgfscope}%
\begin{pgfscope}%
\pgfpathrectangle{\pgfqpoint{0.847223in}{0.554012in}}{\pgfqpoint{6.200000in}{4.530000in}}%
\pgfusepath{clip}%
\pgfsetbuttcap%
\pgfsetroundjoin%
\pgfsetlinewidth{1.003750pt}%
\definecolor{currentstroke}{rgb}{1.000000,0.000000,0.000000}%
\pgfsetstrokecolor{currentstroke}%
\pgfsetdash{}{0pt}%
\pgfpathmoveto{\pgfqpoint{6.143106in}{0.540085in}}%
\pgfpathcurveto{\pgfqpoint{6.154156in}{0.540085in}}{\pgfqpoint{6.164755in}{0.544475in}}{\pgfqpoint{6.172569in}{0.552289in}}%
\pgfpathcurveto{\pgfqpoint{6.180383in}{0.560102in}}{\pgfqpoint{6.184773in}{0.570701in}}{\pgfqpoint{6.184773in}{0.581751in}}%
\pgfpathcurveto{\pgfqpoint{6.184773in}{0.592802in}}{\pgfqpoint{6.180383in}{0.603401in}}{\pgfqpoint{6.172569in}{0.611214in}}%
\pgfpathcurveto{\pgfqpoint{6.164755in}{0.619028in}}{\pgfqpoint{6.154156in}{0.623418in}}{\pgfqpoint{6.143106in}{0.623418in}}%
\pgfpathcurveto{\pgfqpoint{6.132056in}{0.623418in}}{\pgfqpoint{6.121457in}{0.619028in}}{\pgfqpoint{6.113643in}{0.611214in}}%
\pgfpathcurveto{\pgfqpoint{6.105830in}{0.603401in}}{\pgfqpoint{6.101440in}{0.592802in}}{\pgfqpoint{6.101440in}{0.581751in}}%
\pgfpathcurveto{\pgfqpoint{6.101440in}{0.570701in}}{\pgfqpoint{6.105830in}{0.560102in}}{\pgfqpoint{6.113643in}{0.552289in}}%
\pgfpathcurveto{\pgfqpoint{6.121457in}{0.544475in}}{\pgfqpoint{6.132056in}{0.540085in}}{\pgfqpoint{6.143106in}{0.540085in}}%
\pgfusepath{stroke}%
\end{pgfscope}%
\begin{pgfscope}%
\pgfpathrectangle{\pgfqpoint{0.847223in}{0.554012in}}{\pgfqpoint{6.200000in}{4.530000in}}%
\pgfusepath{clip}%
\pgfsetbuttcap%
\pgfsetroundjoin%
\pgfsetlinewidth{1.003750pt}%
\definecolor{currentstroke}{rgb}{1.000000,0.000000,0.000000}%
\pgfsetstrokecolor{currentstroke}%
\pgfsetdash{}{0pt}%
\pgfpathmoveto{\pgfqpoint{6.148439in}{0.539677in}}%
\pgfpathcurveto{\pgfqpoint{6.159490in}{0.539677in}}{\pgfqpoint{6.170089in}{0.544068in}}{\pgfqpoint{6.177902in}{0.551881in}}%
\pgfpathcurveto{\pgfqpoint{6.185716in}{0.559695in}}{\pgfqpoint{6.190106in}{0.570294in}}{\pgfqpoint{6.190106in}{0.581344in}}%
\pgfpathcurveto{\pgfqpoint{6.190106in}{0.592394in}}{\pgfqpoint{6.185716in}{0.602993in}}{\pgfqpoint{6.177902in}{0.610807in}}%
\pgfpathcurveto{\pgfqpoint{6.170089in}{0.618620in}}{\pgfqpoint{6.159490in}{0.623011in}}{\pgfqpoint{6.148439in}{0.623011in}}%
\pgfpathcurveto{\pgfqpoint{6.137389in}{0.623011in}}{\pgfqpoint{6.126790in}{0.618620in}}{\pgfqpoint{6.118977in}{0.610807in}}%
\pgfpathcurveto{\pgfqpoint{6.111163in}{0.602993in}}{\pgfqpoint{6.106773in}{0.592394in}}{\pgfqpoint{6.106773in}{0.581344in}}%
\pgfpathcurveto{\pgfqpoint{6.106773in}{0.570294in}}{\pgfqpoint{6.111163in}{0.559695in}}{\pgfqpoint{6.118977in}{0.551881in}}%
\pgfpathcurveto{\pgfqpoint{6.126790in}{0.544068in}}{\pgfqpoint{6.137389in}{0.539677in}}{\pgfqpoint{6.148439in}{0.539677in}}%
\pgfusepath{stroke}%
\end{pgfscope}%
\begin{pgfscope}%
\pgfpathrectangle{\pgfqpoint{0.847223in}{0.554012in}}{\pgfqpoint{6.200000in}{4.530000in}}%
\pgfusepath{clip}%
\pgfsetbuttcap%
\pgfsetroundjoin%
\pgfsetlinewidth{1.003750pt}%
\definecolor{currentstroke}{rgb}{1.000000,0.000000,0.000000}%
\pgfsetstrokecolor{currentstroke}%
\pgfsetdash{}{0pt}%
\pgfpathmoveto{\pgfqpoint{6.153773in}{0.539271in}}%
\pgfpathcurveto{\pgfqpoint{6.164823in}{0.539271in}}{\pgfqpoint{6.175422in}{0.543661in}}{\pgfqpoint{6.183235in}{0.551475in}}%
\pgfpathcurveto{\pgfqpoint{6.191049in}{0.559288in}}{\pgfqpoint{6.195439in}{0.569887in}}{\pgfqpoint{6.195439in}{0.580937in}}%
\pgfpathcurveto{\pgfqpoint{6.195439in}{0.591987in}}{\pgfqpoint{6.191049in}{0.602587in}}{\pgfqpoint{6.183235in}{0.610400in}}%
\pgfpathcurveto{\pgfqpoint{6.175422in}{0.618214in}}{\pgfqpoint{6.164823in}{0.622604in}}{\pgfqpoint{6.153773in}{0.622604in}}%
\pgfpathcurveto{\pgfqpoint{6.142723in}{0.622604in}}{\pgfqpoint{6.132123in}{0.618214in}}{\pgfqpoint{6.124310in}{0.610400in}}%
\pgfpathcurveto{\pgfqpoint{6.116496in}{0.602587in}}{\pgfqpoint{6.112106in}{0.591987in}}{\pgfqpoint{6.112106in}{0.580937in}}%
\pgfpathcurveto{\pgfqpoint{6.112106in}{0.569887in}}{\pgfqpoint{6.116496in}{0.559288in}}{\pgfqpoint{6.124310in}{0.551475in}}%
\pgfpathcurveto{\pgfqpoint{6.132123in}{0.543661in}}{\pgfqpoint{6.142723in}{0.539271in}}{\pgfqpoint{6.153773in}{0.539271in}}%
\pgfusepath{stroke}%
\end{pgfscope}%
\begin{pgfscope}%
\pgfpathrectangle{\pgfqpoint{0.847223in}{0.554012in}}{\pgfqpoint{6.200000in}{4.530000in}}%
\pgfusepath{clip}%
\pgfsetbuttcap%
\pgfsetroundjoin%
\pgfsetlinewidth{1.003750pt}%
\definecolor{currentstroke}{rgb}{1.000000,0.000000,0.000000}%
\pgfsetstrokecolor{currentstroke}%
\pgfsetdash{}{0pt}%
\pgfpathmoveto{\pgfqpoint{6.159106in}{0.538865in}}%
\pgfpathcurveto{\pgfqpoint{6.170156in}{0.538865in}}{\pgfqpoint{6.180755in}{0.543255in}}{\pgfqpoint{6.188569in}{0.551069in}}%
\pgfpathcurveto{\pgfqpoint{6.196382in}{0.558882in}}{\pgfqpoint{6.200773in}{0.569481in}}{\pgfqpoint{6.200773in}{0.580531in}}%
\pgfpathcurveto{\pgfqpoint{6.200773in}{0.591582in}}{\pgfqpoint{6.196382in}{0.602181in}}{\pgfqpoint{6.188569in}{0.609994in}}%
\pgfpathcurveto{\pgfqpoint{6.180755in}{0.617808in}}{\pgfqpoint{6.170156in}{0.622198in}}{\pgfqpoint{6.159106in}{0.622198in}}%
\pgfpathcurveto{\pgfqpoint{6.148056in}{0.622198in}}{\pgfqpoint{6.137457in}{0.617808in}}{\pgfqpoint{6.129643in}{0.609994in}}%
\pgfpathcurveto{\pgfqpoint{6.121829in}{0.602181in}}{\pgfqpoint{6.117439in}{0.591582in}}{\pgfqpoint{6.117439in}{0.580531in}}%
\pgfpathcurveto{\pgfqpoint{6.117439in}{0.569481in}}{\pgfqpoint{6.121829in}{0.558882in}}{\pgfqpoint{6.129643in}{0.551069in}}%
\pgfpathcurveto{\pgfqpoint{6.137457in}{0.543255in}}{\pgfqpoint{6.148056in}{0.538865in}}{\pgfqpoint{6.159106in}{0.538865in}}%
\pgfusepath{stroke}%
\end{pgfscope}%
\begin{pgfscope}%
\pgfpathrectangle{\pgfqpoint{0.847223in}{0.554012in}}{\pgfqpoint{6.200000in}{4.530000in}}%
\pgfusepath{clip}%
\pgfsetbuttcap%
\pgfsetroundjoin%
\pgfsetlinewidth{1.003750pt}%
\definecolor{currentstroke}{rgb}{1.000000,0.000000,0.000000}%
\pgfsetstrokecolor{currentstroke}%
\pgfsetdash{}{0pt}%
\pgfpathmoveto{\pgfqpoint{6.164439in}{0.538460in}}%
\pgfpathcurveto{\pgfqpoint{6.175489in}{0.538460in}}{\pgfqpoint{6.186088in}{0.542850in}}{\pgfqpoint{6.193902in}{0.550664in}}%
\pgfpathcurveto{\pgfqpoint{6.201715in}{0.558477in}}{\pgfqpoint{6.206106in}{0.569076in}}{\pgfqpoint{6.206106in}{0.580126in}}%
\pgfpathcurveto{\pgfqpoint{6.206106in}{0.591176in}}{\pgfqpoint{6.201715in}{0.601775in}}{\pgfqpoint{6.193902in}{0.609589in}}%
\pgfpathcurveto{\pgfqpoint{6.186088in}{0.617403in}}{\pgfqpoint{6.175489in}{0.621793in}}{\pgfqpoint{6.164439in}{0.621793in}}%
\pgfpathcurveto{\pgfqpoint{6.153389in}{0.621793in}}{\pgfqpoint{6.142790in}{0.617403in}}{\pgfqpoint{6.134976in}{0.609589in}}%
\pgfpathcurveto{\pgfqpoint{6.127163in}{0.601775in}}{\pgfqpoint{6.122772in}{0.591176in}}{\pgfqpoint{6.122772in}{0.580126in}}%
\pgfpathcurveto{\pgfqpoint{6.122772in}{0.569076in}}{\pgfqpoint{6.127163in}{0.558477in}}{\pgfqpoint{6.134976in}{0.550664in}}%
\pgfpathcurveto{\pgfqpoint{6.142790in}{0.542850in}}{\pgfqpoint{6.153389in}{0.538460in}}{\pgfqpoint{6.164439in}{0.538460in}}%
\pgfusepath{stroke}%
\end{pgfscope}%
\begin{pgfscope}%
\pgfpathrectangle{\pgfqpoint{0.847223in}{0.554012in}}{\pgfqpoint{6.200000in}{4.530000in}}%
\pgfusepath{clip}%
\pgfsetbuttcap%
\pgfsetroundjoin%
\pgfsetlinewidth{1.003750pt}%
\definecolor{currentstroke}{rgb}{1.000000,0.000000,0.000000}%
\pgfsetstrokecolor{currentstroke}%
\pgfsetdash{}{0pt}%
\pgfpathmoveto{\pgfqpoint{6.169772in}{0.538055in}}%
\pgfpathcurveto{\pgfqpoint{6.180822in}{0.538055in}}{\pgfqpoint{6.191421in}{0.542445in}}{\pgfqpoint{6.199235in}{0.550259in}}%
\pgfpathcurveto{\pgfqpoint{6.207049in}{0.558073in}}{\pgfqpoint{6.211439in}{0.568672in}}{\pgfqpoint{6.211439in}{0.579722in}}%
\pgfpathcurveto{\pgfqpoint{6.211439in}{0.590772in}}{\pgfqpoint{6.207049in}{0.601371in}}{\pgfqpoint{6.199235in}{0.609185in}}%
\pgfpathcurveto{\pgfqpoint{6.191421in}{0.616998in}}{\pgfqpoint{6.180822in}{0.621389in}}{\pgfqpoint{6.169772in}{0.621389in}}%
\pgfpathcurveto{\pgfqpoint{6.158722in}{0.621389in}}{\pgfqpoint{6.148123in}{0.616998in}}{\pgfqpoint{6.140310in}{0.609185in}}%
\pgfpathcurveto{\pgfqpoint{6.132496in}{0.601371in}}{\pgfqpoint{6.128106in}{0.590772in}}{\pgfqpoint{6.128106in}{0.579722in}}%
\pgfpathcurveto{\pgfqpoint{6.128106in}{0.568672in}}{\pgfqpoint{6.132496in}{0.558073in}}{\pgfqpoint{6.140310in}{0.550259in}}%
\pgfpathcurveto{\pgfqpoint{6.148123in}{0.542445in}}{\pgfqpoint{6.158722in}{0.538055in}}{\pgfqpoint{6.169772in}{0.538055in}}%
\pgfusepath{stroke}%
\end{pgfscope}%
\begin{pgfscope}%
\pgfpathrectangle{\pgfqpoint{0.847223in}{0.554012in}}{\pgfqpoint{6.200000in}{4.530000in}}%
\pgfusepath{clip}%
\pgfsetbuttcap%
\pgfsetroundjoin%
\pgfsetlinewidth{1.003750pt}%
\definecolor{currentstroke}{rgb}{1.000000,0.000000,0.000000}%
\pgfsetstrokecolor{currentstroke}%
\pgfsetdash{}{0pt}%
\pgfpathmoveto{\pgfqpoint{6.175105in}{0.537652in}}%
\pgfpathcurveto{\pgfqpoint{6.186156in}{0.537652in}}{\pgfqpoint{6.196755in}{0.542042in}}{\pgfqpoint{6.204568in}{0.549855in}}%
\pgfpathcurveto{\pgfqpoint{6.212382in}{0.557669in}}{\pgfqpoint{6.216772in}{0.568268in}}{\pgfqpoint{6.216772in}{0.579318in}}%
\pgfpathcurveto{\pgfqpoint{6.216772in}{0.590368in}}{\pgfqpoint{6.212382in}{0.600967in}}{\pgfqpoint{6.204568in}{0.608781in}}%
\pgfpathcurveto{\pgfqpoint{6.196755in}{0.616595in}}{\pgfqpoint{6.186156in}{0.620985in}}{\pgfqpoint{6.175105in}{0.620985in}}%
\pgfpathcurveto{\pgfqpoint{6.164055in}{0.620985in}}{\pgfqpoint{6.153456in}{0.616595in}}{\pgfqpoint{6.145643in}{0.608781in}}%
\pgfpathcurveto{\pgfqpoint{6.137829in}{0.600967in}}{\pgfqpoint{6.133439in}{0.590368in}}{\pgfqpoint{6.133439in}{0.579318in}}%
\pgfpathcurveto{\pgfqpoint{6.133439in}{0.568268in}}{\pgfqpoint{6.137829in}{0.557669in}}{\pgfqpoint{6.145643in}{0.549855in}}%
\pgfpathcurveto{\pgfqpoint{6.153456in}{0.542042in}}{\pgfqpoint{6.164055in}{0.537652in}}{\pgfqpoint{6.175105in}{0.537652in}}%
\pgfusepath{stroke}%
\end{pgfscope}%
\begin{pgfscope}%
\pgfpathrectangle{\pgfqpoint{0.847223in}{0.554012in}}{\pgfqpoint{6.200000in}{4.530000in}}%
\pgfusepath{clip}%
\pgfsetbuttcap%
\pgfsetroundjoin%
\pgfsetlinewidth{1.003750pt}%
\definecolor{currentstroke}{rgb}{1.000000,0.000000,0.000000}%
\pgfsetstrokecolor{currentstroke}%
\pgfsetdash{}{0pt}%
\pgfpathmoveto{\pgfqpoint{6.175105in}{0.537652in}}%
\pgfpathcurveto{\pgfqpoint{6.186156in}{0.537652in}}{\pgfqpoint{6.196755in}{0.542042in}}{\pgfqpoint{6.204568in}{0.549855in}}%
\pgfpathcurveto{\pgfqpoint{6.212382in}{0.557669in}}{\pgfqpoint{6.216772in}{0.568268in}}{\pgfqpoint{6.216772in}{0.579318in}}%
\pgfpathcurveto{\pgfqpoint{6.216772in}{0.590368in}}{\pgfqpoint{6.212382in}{0.600967in}}{\pgfqpoint{6.204568in}{0.608781in}}%
\pgfpathcurveto{\pgfqpoint{6.196755in}{0.616595in}}{\pgfqpoint{6.186156in}{0.620985in}}{\pgfqpoint{6.175105in}{0.620985in}}%
\pgfpathcurveto{\pgfqpoint{6.164055in}{0.620985in}}{\pgfqpoint{6.153456in}{0.616595in}}{\pgfqpoint{6.145643in}{0.608781in}}%
\pgfpathcurveto{\pgfqpoint{6.137829in}{0.600967in}}{\pgfqpoint{6.133439in}{0.590368in}}{\pgfqpoint{6.133439in}{0.579318in}}%
\pgfpathcurveto{\pgfqpoint{6.133439in}{0.568268in}}{\pgfqpoint{6.137829in}{0.557669in}}{\pgfqpoint{6.145643in}{0.549855in}}%
\pgfpathcurveto{\pgfqpoint{6.153456in}{0.542042in}}{\pgfqpoint{6.164055in}{0.537652in}}{\pgfqpoint{6.175105in}{0.537652in}}%
\pgfusepath{stroke}%
\end{pgfscope}%
\begin{pgfscope}%
\pgfpathrectangle{\pgfqpoint{0.847223in}{0.554012in}}{\pgfqpoint{6.200000in}{4.530000in}}%
\pgfusepath{clip}%
\pgfsetbuttcap%
\pgfsetroundjoin%
\pgfsetlinewidth{1.003750pt}%
\definecolor{currentstroke}{rgb}{1.000000,0.000000,0.000000}%
\pgfsetstrokecolor{currentstroke}%
\pgfsetdash{}{0pt}%
\pgfpathmoveto{\pgfqpoint{6.183915in}{0.536999in}}%
\pgfpathcurveto{\pgfqpoint{6.194965in}{0.536999in}}{\pgfqpoint{6.205564in}{0.541389in}}{\pgfqpoint{6.213378in}{0.549202in}}%
\pgfpathcurveto{\pgfqpoint{6.221191in}{0.557016in}}{\pgfqpoint{6.225581in}{0.567615in}}{\pgfqpoint{6.225581in}{0.578665in}}%
\pgfpathcurveto{\pgfqpoint{6.225581in}{0.589715in}}{\pgfqpoint{6.221191in}{0.600314in}}{\pgfqpoint{6.213378in}{0.608128in}}%
\pgfpathcurveto{\pgfqpoint{6.205564in}{0.615942in}}{\pgfqpoint{6.194965in}{0.620332in}}{\pgfqpoint{6.183915in}{0.620332in}}%
\pgfpathcurveto{\pgfqpoint{6.172865in}{0.620332in}}{\pgfqpoint{6.162266in}{0.615942in}}{\pgfqpoint{6.154452in}{0.608128in}}%
\pgfpathcurveto{\pgfqpoint{6.146638in}{0.600314in}}{\pgfqpoint{6.142248in}{0.589715in}}{\pgfqpoint{6.142248in}{0.578665in}}%
\pgfpathcurveto{\pgfqpoint{6.142248in}{0.567615in}}{\pgfqpoint{6.146638in}{0.557016in}}{\pgfqpoint{6.154452in}{0.549202in}}%
\pgfpathcurveto{\pgfqpoint{6.162266in}{0.541389in}}{\pgfqpoint{6.172865in}{0.536999in}}{\pgfqpoint{6.183915in}{0.536999in}}%
\pgfusepath{stroke}%
\end{pgfscope}%
\begin{pgfscope}%
\pgfpathrectangle{\pgfqpoint{0.847223in}{0.554012in}}{\pgfqpoint{6.200000in}{4.530000in}}%
\pgfusepath{clip}%
\pgfsetbuttcap%
\pgfsetroundjoin%
\pgfsetlinewidth{1.003750pt}%
\definecolor{currentstroke}{rgb}{1.000000,0.000000,0.000000}%
\pgfsetstrokecolor{currentstroke}%
\pgfsetdash{}{0pt}%
\pgfpathmoveto{\pgfqpoint{6.192724in}{0.536371in}}%
\pgfpathcurveto{\pgfqpoint{6.203774in}{0.536371in}}{\pgfqpoint{6.214373in}{0.540761in}}{\pgfqpoint{6.222187in}{0.548574in}}%
\pgfpathcurveto{\pgfqpoint{6.230000in}{0.556388in}}{\pgfqpoint{6.234391in}{0.566987in}}{\pgfqpoint{6.234391in}{0.578037in}}%
\pgfpathcurveto{\pgfqpoint{6.234391in}{0.589087in}}{\pgfqpoint{6.230000in}{0.599686in}}{\pgfqpoint{6.222187in}{0.607500in}}%
\pgfpathcurveto{\pgfqpoint{6.214373in}{0.615314in}}{\pgfqpoint{6.203774in}{0.619704in}}{\pgfqpoint{6.192724in}{0.619704in}}%
\pgfpathcurveto{\pgfqpoint{6.181674in}{0.619704in}}{\pgfqpoint{6.171075in}{0.615314in}}{\pgfqpoint{6.163261in}{0.607500in}}%
\pgfpathcurveto{\pgfqpoint{6.155448in}{0.599686in}}{\pgfqpoint{6.151057in}{0.589087in}}{\pgfqpoint{6.151057in}{0.578037in}}%
\pgfpathcurveto{\pgfqpoint{6.151057in}{0.566987in}}{\pgfqpoint{6.155448in}{0.556388in}}{\pgfqpoint{6.163261in}{0.548574in}}%
\pgfpathcurveto{\pgfqpoint{6.171075in}{0.540761in}}{\pgfqpoint{6.181674in}{0.536371in}}{\pgfqpoint{6.192724in}{0.536371in}}%
\pgfusepath{stroke}%
\end{pgfscope}%
\begin{pgfscope}%
\pgfpathrectangle{\pgfqpoint{0.847223in}{0.554012in}}{\pgfqpoint{6.200000in}{4.530000in}}%
\pgfusepath{clip}%
\pgfsetbuttcap%
\pgfsetroundjoin%
\pgfsetlinewidth{1.003750pt}%
\definecolor{currentstroke}{rgb}{1.000000,0.000000,0.000000}%
\pgfsetstrokecolor{currentstroke}%
\pgfsetdash{}{0pt}%
\pgfpathmoveto{\pgfqpoint{6.201533in}{0.535766in}}%
\pgfpathcurveto{\pgfqpoint{6.212583in}{0.535766in}}{\pgfqpoint{6.223182in}{0.540156in}}{\pgfqpoint{6.230996in}{0.547970in}}%
\pgfpathcurveto{\pgfqpoint{6.238810in}{0.555783in}}{\pgfqpoint{6.243200in}{0.566382in}}{\pgfqpoint{6.243200in}{0.577433in}}%
\pgfpathcurveto{\pgfqpoint{6.243200in}{0.588483in}}{\pgfqpoint{6.238810in}{0.599082in}}{\pgfqpoint{6.230996in}{0.606895in}}%
\pgfpathcurveto{\pgfqpoint{6.223182in}{0.614709in}}{\pgfqpoint{6.212583in}{0.619099in}}{\pgfqpoint{6.201533in}{0.619099in}}%
\pgfpathcurveto{\pgfqpoint{6.190483in}{0.619099in}}{\pgfqpoint{6.179884in}{0.614709in}}{\pgfqpoint{6.172071in}{0.606895in}}%
\pgfpathcurveto{\pgfqpoint{6.164257in}{0.599082in}}{\pgfqpoint{6.159867in}{0.588483in}}{\pgfqpoint{6.159867in}{0.577433in}}%
\pgfpathcurveto{\pgfqpoint{6.159867in}{0.566382in}}{\pgfqpoint{6.164257in}{0.555783in}}{\pgfqpoint{6.172071in}{0.547970in}}%
\pgfpathcurveto{\pgfqpoint{6.179884in}{0.540156in}}{\pgfqpoint{6.190483in}{0.535766in}}{\pgfqpoint{6.201533in}{0.535766in}}%
\pgfusepath{stroke}%
\end{pgfscope}%
\begin{pgfscope}%
\pgfpathrectangle{\pgfqpoint{0.847223in}{0.554012in}}{\pgfqpoint{6.200000in}{4.530000in}}%
\pgfusepath{clip}%
\pgfsetbuttcap%
\pgfsetroundjoin%
\pgfsetlinewidth{1.003750pt}%
\definecolor{currentstroke}{rgb}{1.000000,0.000000,0.000000}%
\pgfsetstrokecolor{currentstroke}%
\pgfsetdash{}{0pt}%
\pgfpathmoveto{\pgfqpoint{6.210343in}{0.535183in}}%
\pgfpathcurveto{\pgfqpoint{6.221393in}{0.535183in}}{\pgfqpoint{6.231992in}{0.539573in}}{\pgfqpoint{6.239805in}{0.547387in}}%
\pgfpathcurveto{\pgfqpoint{6.247619in}{0.555200in}}{\pgfqpoint{6.252009in}{0.565799in}}{\pgfqpoint{6.252009in}{0.576850in}}%
\pgfpathcurveto{\pgfqpoint{6.252009in}{0.587900in}}{\pgfqpoint{6.247619in}{0.598499in}}{\pgfqpoint{6.239805in}{0.606312in}}%
\pgfpathcurveto{\pgfqpoint{6.231992in}{0.614126in}}{\pgfqpoint{6.221393in}{0.618516in}}{\pgfqpoint{6.210343in}{0.618516in}}%
\pgfpathcurveto{\pgfqpoint{6.199292in}{0.618516in}}{\pgfqpoint{6.188693in}{0.614126in}}{\pgfqpoint{6.180880in}{0.606312in}}%
\pgfpathcurveto{\pgfqpoint{6.173066in}{0.598499in}}{\pgfqpoint{6.168676in}{0.587900in}}{\pgfqpoint{6.168676in}{0.576850in}}%
\pgfpathcurveto{\pgfqpoint{6.168676in}{0.565799in}}{\pgfqpoint{6.173066in}{0.555200in}}{\pgfqpoint{6.180880in}{0.547387in}}%
\pgfpathcurveto{\pgfqpoint{6.188693in}{0.539573in}}{\pgfqpoint{6.199292in}{0.535183in}}{\pgfqpoint{6.210343in}{0.535183in}}%
\pgfusepath{stroke}%
\end{pgfscope}%
\begin{pgfscope}%
\pgfpathrectangle{\pgfqpoint{0.847223in}{0.554012in}}{\pgfqpoint{6.200000in}{4.530000in}}%
\pgfusepath{clip}%
\pgfsetbuttcap%
\pgfsetroundjoin%
\pgfsetlinewidth{1.003750pt}%
\definecolor{currentstroke}{rgb}{1.000000,0.000000,0.000000}%
\pgfsetstrokecolor{currentstroke}%
\pgfsetdash{}{0pt}%
\pgfpathmoveto{\pgfqpoint{6.219152in}{0.534620in}}%
\pgfpathcurveto{\pgfqpoint{6.230202in}{0.534620in}}{\pgfqpoint{6.240801in}{0.539010in}}{\pgfqpoint{6.248615in}{0.546824in}}%
\pgfpathcurveto{\pgfqpoint{6.256428in}{0.554638in}}{\pgfqpoint{6.260819in}{0.565237in}}{\pgfqpoint{6.260819in}{0.576287in}}%
\pgfpathcurveto{\pgfqpoint{6.260819in}{0.587337in}}{\pgfqpoint{6.256428in}{0.597936in}}{\pgfqpoint{6.248615in}{0.605750in}}%
\pgfpathcurveto{\pgfqpoint{6.240801in}{0.613563in}}{\pgfqpoint{6.230202in}{0.617953in}}{\pgfqpoint{6.219152in}{0.617953in}}%
\pgfpathcurveto{\pgfqpoint{6.208102in}{0.617953in}}{\pgfqpoint{6.197503in}{0.613563in}}{\pgfqpoint{6.189689in}{0.605750in}}%
\pgfpathcurveto{\pgfqpoint{6.181875in}{0.597936in}}{\pgfqpoint{6.177485in}{0.587337in}}{\pgfqpoint{6.177485in}{0.576287in}}%
\pgfpathcurveto{\pgfqpoint{6.177485in}{0.565237in}}{\pgfqpoint{6.181875in}{0.554638in}}{\pgfqpoint{6.189689in}{0.546824in}}%
\pgfpathcurveto{\pgfqpoint{6.197503in}{0.539010in}}{\pgfqpoint{6.208102in}{0.534620in}}{\pgfqpoint{6.219152in}{0.534620in}}%
\pgfusepath{stroke}%
\end{pgfscope}%
\begin{pgfscope}%
\pgfpathrectangle{\pgfqpoint{0.847223in}{0.554012in}}{\pgfqpoint{6.200000in}{4.530000in}}%
\pgfusepath{clip}%
\pgfsetbuttcap%
\pgfsetroundjoin%
\pgfsetlinewidth{1.003750pt}%
\definecolor{currentstroke}{rgb}{1.000000,0.000000,0.000000}%
\pgfsetstrokecolor{currentstroke}%
\pgfsetdash{}{0pt}%
\pgfpathmoveto{\pgfqpoint{6.227961in}{0.534076in}}%
\pgfpathcurveto{\pgfqpoint{6.239011in}{0.534076in}}{\pgfqpoint{6.249610in}{0.538466in}}{\pgfqpoint{6.257424in}{0.546280in}}%
\pgfpathcurveto{\pgfqpoint{6.265238in}{0.554094in}}{\pgfqpoint{6.269628in}{0.564693in}}{\pgfqpoint{6.269628in}{0.575743in}}%
\pgfpathcurveto{\pgfqpoint{6.269628in}{0.586793in}}{\pgfqpoint{6.265238in}{0.597392in}}{\pgfqpoint{6.257424in}{0.605206in}}%
\pgfpathcurveto{\pgfqpoint{6.249610in}{0.613019in}}{\pgfqpoint{6.239011in}{0.617409in}}{\pgfqpoint{6.227961in}{0.617409in}}%
\pgfpathcurveto{\pgfqpoint{6.216911in}{0.617409in}}{\pgfqpoint{6.206312in}{0.613019in}}{\pgfqpoint{6.198498in}{0.605206in}}%
\pgfpathcurveto{\pgfqpoint{6.190685in}{0.597392in}}{\pgfqpoint{6.186294in}{0.586793in}}{\pgfqpoint{6.186294in}{0.575743in}}%
\pgfpathcurveto{\pgfqpoint{6.186294in}{0.564693in}}{\pgfqpoint{6.190685in}{0.554094in}}{\pgfqpoint{6.198498in}{0.546280in}}%
\pgfpathcurveto{\pgfqpoint{6.206312in}{0.538466in}}{\pgfqpoint{6.216911in}{0.534076in}}{\pgfqpoint{6.227961in}{0.534076in}}%
\pgfusepath{stroke}%
\end{pgfscope}%
\begin{pgfscope}%
\pgfpathrectangle{\pgfqpoint{0.847223in}{0.554012in}}{\pgfqpoint{6.200000in}{4.530000in}}%
\pgfusepath{clip}%
\pgfsetbuttcap%
\pgfsetroundjoin%
\pgfsetlinewidth{1.003750pt}%
\definecolor{currentstroke}{rgb}{1.000000,0.000000,0.000000}%
\pgfsetstrokecolor{currentstroke}%
\pgfsetdash{}{0pt}%
\pgfpathmoveto{\pgfqpoint{6.236770in}{0.533550in}}%
\pgfpathcurveto{\pgfqpoint{6.247821in}{0.533550in}}{\pgfqpoint{6.258420in}{0.537940in}}{\pgfqpoint{6.266233in}{0.545754in}}%
\pgfpathcurveto{\pgfqpoint{6.274047in}{0.553567in}}{\pgfqpoint{6.278437in}{0.564166in}}{\pgfqpoint{6.278437in}{0.575217in}}%
\pgfpathcurveto{\pgfqpoint{6.278437in}{0.586267in}}{\pgfqpoint{6.274047in}{0.596866in}}{\pgfqpoint{6.266233in}{0.604679in}}%
\pgfpathcurveto{\pgfqpoint{6.258420in}{0.612493in}}{\pgfqpoint{6.247821in}{0.616883in}}{\pgfqpoint{6.236770in}{0.616883in}}%
\pgfpathcurveto{\pgfqpoint{6.225720in}{0.616883in}}{\pgfqpoint{6.215121in}{0.612493in}}{\pgfqpoint{6.207308in}{0.604679in}}%
\pgfpathcurveto{\pgfqpoint{6.199494in}{0.596866in}}{\pgfqpoint{6.195104in}{0.586267in}}{\pgfqpoint{6.195104in}{0.575217in}}%
\pgfpathcurveto{\pgfqpoint{6.195104in}{0.564166in}}{\pgfqpoint{6.199494in}{0.553567in}}{\pgfqpoint{6.207308in}{0.545754in}}%
\pgfpathcurveto{\pgfqpoint{6.215121in}{0.537940in}}{\pgfqpoint{6.225720in}{0.533550in}}{\pgfqpoint{6.236770in}{0.533550in}}%
\pgfusepath{stroke}%
\end{pgfscope}%
\begin{pgfscope}%
\pgfpathrectangle{\pgfqpoint{0.847223in}{0.554012in}}{\pgfqpoint{6.200000in}{4.530000in}}%
\pgfusepath{clip}%
\pgfsetbuttcap%
\pgfsetroundjoin%
\pgfsetlinewidth{1.003750pt}%
\definecolor{currentstroke}{rgb}{1.000000,0.000000,0.000000}%
\pgfsetstrokecolor{currentstroke}%
\pgfsetdash{}{0pt}%
\pgfpathmoveto{\pgfqpoint{6.245580in}{0.533040in}}%
\pgfpathcurveto{\pgfqpoint{6.256630in}{0.533040in}}{\pgfqpoint{6.267229in}{0.537431in}}{\pgfqpoint{6.275042in}{0.545244in}}%
\pgfpathcurveto{\pgfqpoint{6.282856in}{0.553058in}}{\pgfqpoint{6.287246in}{0.563657in}}{\pgfqpoint{6.287246in}{0.574707in}}%
\pgfpathcurveto{\pgfqpoint{6.287246in}{0.585757in}}{\pgfqpoint{6.282856in}{0.596356in}}{\pgfqpoint{6.275042in}{0.604170in}}%
\pgfpathcurveto{\pgfqpoint{6.267229in}{0.611983in}}{\pgfqpoint{6.256630in}{0.616374in}}{\pgfqpoint{6.245580in}{0.616374in}}%
\pgfpathcurveto{\pgfqpoint{6.234530in}{0.616374in}}{\pgfqpoint{6.223930in}{0.611983in}}{\pgfqpoint{6.216117in}{0.604170in}}%
\pgfpathcurveto{\pgfqpoint{6.208303in}{0.596356in}}{\pgfqpoint{6.203913in}{0.585757in}}{\pgfqpoint{6.203913in}{0.574707in}}%
\pgfpathcurveto{\pgfqpoint{6.203913in}{0.563657in}}{\pgfqpoint{6.208303in}{0.553058in}}{\pgfqpoint{6.216117in}{0.545244in}}%
\pgfpathcurveto{\pgfqpoint{6.223930in}{0.537431in}}{\pgfqpoint{6.234530in}{0.533040in}}{\pgfqpoint{6.245580in}{0.533040in}}%
\pgfusepath{stroke}%
\end{pgfscope}%
\begin{pgfscope}%
\pgfpathrectangle{\pgfqpoint{0.847223in}{0.554012in}}{\pgfqpoint{6.200000in}{4.530000in}}%
\pgfusepath{clip}%
\pgfsetbuttcap%
\pgfsetroundjoin%
\pgfsetlinewidth{1.003750pt}%
\definecolor{currentstroke}{rgb}{1.000000,0.000000,0.000000}%
\pgfsetstrokecolor{currentstroke}%
\pgfsetdash{}{0pt}%
\pgfpathmoveto{\pgfqpoint{6.254389in}{0.532546in}}%
\pgfpathcurveto{\pgfqpoint{6.265439in}{0.532546in}}{\pgfqpoint{6.276038in}{0.536937in}}{\pgfqpoint{6.283852in}{0.544750in}}%
\pgfpathcurveto{\pgfqpoint{6.291665in}{0.552564in}}{\pgfqpoint{6.296056in}{0.563163in}}{\pgfqpoint{6.296056in}{0.574213in}}%
\pgfpathcurveto{\pgfqpoint{6.296056in}{0.585263in}}{\pgfqpoint{6.291665in}{0.595862in}}{\pgfqpoint{6.283852in}{0.603676in}}%
\pgfpathcurveto{\pgfqpoint{6.276038in}{0.611489in}}{\pgfqpoint{6.265439in}{0.615880in}}{\pgfqpoint{6.254389in}{0.615880in}}%
\pgfpathcurveto{\pgfqpoint{6.243339in}{0.615880in}}{\pgfqpoint{6.232740in}{0.611489in}}{\pgfqpoint{6.224926in}{0.603676in}}%
\pgfpathcurveto{\pgfqpoint{6.217113in}{0.595862in}}{\pgfqpoint{6.212722in}{0.585263in}}{\pgfqpoint{6.212722in}{0.574213in}}%
\pgfpathcurveto{\pgfqpoint{6.212722in}{0.563163in}}{\pgfqpoint{6.217113in}{0.552564in}}{\pgfqpoint{6.224926in}{0.544750in}}%
\pgfpathcurveto{\pgfqpoint{6.232740in}{0.536937in}}{\pgfqpoint{6.243339in}{0.532546in}}{\pgfqpoint{6.254389in}{0.532546in}}%
\pgfusepath{stroke}%
\end{pgfscope}%
\begin{pgfscope}%
\pgfpathrectangle{\pgfqpoint{0.847223in}{0.554012in}}{\pgfqpoint{6.200000in}{4.530000in}}%
\pgfusepath{clip}%
\pgfsetbuttcap%
\pgfsetroundjoin%
\pgfsetlinewidth{1.003750pt}%
\definecolor{currentstroke}{rgb}{1.000000,0.000000,0.000000}%
\pgfsetstrokecolor{currentstroke}%
\pgfsetdash{}{0pt}%
\pgfpathmoveto{\pgfqpoint{6.263198in}{0.532067in}}%
\pgfpathcurveto{\pgfqpoint{6.274248in}{0.532067in}}{\pgfqpoint{6.284847in}{0.536457in}}{\pgfqpoint{6.292661in}{0.544271in}}%
\pgfpathcurveto{\pgfqpoint{6.300475in}{0.552085in}}{\pgfqpoint{6.304865in}{0.562684in}}{\pgfqpoint{6.304865in}{0.573734in}}%
\pgfpathcurveto{\pgfqpoint{6.304865in}{0.584784in}}{\pgfqpoint{6.300475in}{0.595383in}}{\pgfqpoint{6.292661in}{0.603197in}}%
\pgfpathcurveto{\pgfqpoint{6.284847in}{0.611010in}}{\pgfqpoint{6.274248in}{0.615401in}}{\pgfqpoint{6.263198in}{0.615401in}}%
\pgfpathcurveto{\pgfqpoint{6.252148in}{0.615401in}}{\pgfqpoint{6.241549in}{0.611010in}}{\pgfqpoint{6.233735in}{0.603197in}}%
\pgfpathcurveto{\pgfqpoint{6.225922in}{0.595383in}}{\pgfqpoint{6.221532in}{0.584784in}}{\pgfqpoint{6.221532in}{0.573734in}}%
\pgfpathcurveto{\pgfqpoint{6.221532in}{0.562684in}}{\pgfqpoint{6.225922in}{0.552085in}}{\pgfqpoint{6.233735in}{0.544271in}}%
\pgfpathcurveto{\pgfqpoint{6.241549in}{0.536457in}}{\pgfqpoint{6.252148in}{0.532067in}}{\pgfqpoint{6.263198in}{0.532067in}}%
\pgfusepath{stroke}%
\end{pgfscope}%
\begin{pgfscope}%
\pgfpathrectangle{\pgfqpoint{0.847223in}{0.554012in}}{\pgfqpoint{6.200000in}{4.530000in}}%
\pgfusepath{clip}%
\pgfsetbuttcap%
\pgfsetroundjoin%
\pgfsetlinewidth{1.003750pt}%
\definecolor{currentstroke}{rgb}{1.000000,0.000000,0.000000}%
\pgfsetstrokecolor{currentstroke}%
\pgfsetdash{}{0pt}%
\pgfpathmoveto{\pgfqpoint{6.272007in}{0.531602in}}%
\pgfpathcurveto{\pgfqpoint{6.283058in}{0.531602in}}{\pgfqpoint{6.293657in}{0.535992in}}{\pgfqpoint{6.301470in}{0.543806in}}%
\pgfpathcurveto{\pgfqpoint{6.309284in}{0.551620in}}{\pgfqpoint{6.313674in}{0.562219in}}{\pgfqpoint{6.313674in}{0.573269in}}%
\pgfpathcurveto{\pgfqpoint{6.313674in}{0.584319in}}{\pgfqpoint{6.309284in}{0.594918in}}{\pgfqpoint{6.301470in}{0.602732in}}%
\pgfpathcurveto{\pgfqpoint{6.293657in}{0.610545in}}{\pgfqpoint{6.283058in}{0.614935in}}{\pgfqpoint{6.272007in}{0.614935in}}%
\pgfpathcurveto{\pgfqpoint{6.260957in}{0.614935in}}{\pgfqpoint{6.250358in}{0.610545in}}{\pgfqpoint{6.242545in}{0.602732in}}%
\pgfpathcurveto{\pgfqpoint{6.234731in}{0.594918in}}{\pgfqpoint{6.230341in}{0.584319in}}{\pgfqpoint{6.230341in}{0.573269in}}%
\pgfpathcurveto{\pgfqpoint{6.230341in}{0.562219in}}{\pgfqpoint{6.234731in}{0.551620in}}{\pgfqpoint{6.242545in}{0.543806in}}%
\pgfpathcurveto{\pgfqpoint{6.250358in}{0.535992in}}{\pgfqpoint{6.260957in}{0.531602in}}{\pgfqpoint{6.272007in}{0.531602in}}%
\pgfusepath{stroke}%
\end{pgfscope}%
\begin{pgfscope}%
\pgfpathrectangle{\pgfqpoint{0.847223in}{0.554012in}}{\pgfqpoint{6.200000in}{4.530000in}}%
\pgfusepath{clip}%
\pgfsetbuttcap%
\pgfsetroundjoin%
\pgfsetlinewidth{1.003750pt}%
\definecolor{currentstroke}{rgb}{1.000000,0.000000,0.000000}%
\pgfsetstrokecolor{currentstroke}%
\pgfsetdash{}{0pt}%
\pgfpathmoveto{\pgfqpoint{6.280817in}{0.531150in}}%
\pgfpathcurveto{\pgfqpoint{6.291867in}{0.531150in}}{\pgfqpoint{6.302466in}{0.535540in}}{\pgfqpoint{6.310280in}{0.543354in}}%
\pgfpathcurveto{\pgfqpoint{6.318093in}{0.551168in}}{\pgfqpoint{6.322483in}{0.561767in}}{\pgfqpoint{6.322483in}{0.572817in}}%
\pgfpathcurveto{\pgfqpoint{6.322483in}{0.583867in}}{\pgfqpoint{6.318093in}{0.594466in}}{\pgfqpoint{6.310280in}{0.602280in}}%
\pgfpathcurveto{\pgfqpoint{6.302466in}{0.610093in}}{\pgfqpoint{6.291867in}{0.614483in}}{\pgfqpoint{6.280817in}{0.614483in}}%
\pgfpathcurveto{\pgfqpoint{6.269767in}{0.614483in}}{\pgfqpoint{6.259168in}{0.610093in}}{\pgfqpoint{6.251354in}{0.602280in}}%
\pgfpathcurveto{\pgfqpoint{6.243540in}{0.594466in}}{\pgfqpoint{6.239150in}{0.583867in}}{\pgfqpoint{6.239150in}{0.572817in}}%
\pgfpathcurveto{\pgfqpoint{6.239150in}{0.561767in}}{\pgfqpoint{6.243540in}{0.551168in}}{\pgfqpoint{6.251354in}{0.543354in}}%
\pgfpathcurveto{\pgfqpoint{6.259168in}{0.535540in}}{\pgfqpoint{6.269767in}{0.531150in}}{\pgfqpoint{6.280817in}{0.531150in}}%
\pgfusepath{stroke}%
\end{pgfscope}%
\begin{pgfscope}%
\pgfpathrectangle{\pgfqpoint{0.847223in}{0.554012in}}{\pgfqpoint{6.200000in}{4.530000in}}%
\pgfusepath{clip}%
\pgfsetbuttcap%
\pgfsetroundjoin%
\pgfsetlinewidth{1.003750pt}%
\definecolor{currentstroke}{rgb}{1.000000,0.000000,0.000000}%
\pgfsetstrokecolor{currentstroke}%
\pgfsetdash{}{0pt}%
\pgfpathmoveto{\pgfqpoint{6.289626in}{0.530711in}}%
\pgfpathcurveto{\pgfqpoint{6.300676in}{0.530711in}}{\pgfqpoint{6.311275in}{0.535101in}}{\pgfqpoint{6.319089in}{0.542915in}}%
\pgfpathcurveto{\pgfqpoint{6.326902in}{0.550728in}}{\pgfqpoint{6.331293in}{0.561327in}}{\pgfqpoint{6.331293in}{0.572377in}}%
\pgfpathcurveto{\pgfqpoint{6.331293in}{0.583427in}}{\pgfqpoint{6.326902in}{0.594027in}}{\pgfqpoint{6.319089in}{0.601840in}}%
\pgfpathcurveto{\pgfqpoint{6.311275in}{0.609654in}}{\pgfqpoint{6.300676in}{0.614044in}}{\pgfqpoint{6.289626in}{0.614044in}}%
\pgfpathcurveto{\pgfqpoint{6.278576in}{0.614044in}}{\pgfqpoint{6.267977in}{0.609654in}}{\pgfqpoint{6.260163in}{0.601840in}}%
\pgfpathcurveto{\pgfqpoint{6.252350in}{0.594027in}}{\pgfqpoint{6.247959in}{0.583427in}}{\pgfqpoint{6.247959in}{0.572377in}}%
\pgfpathcurveto{\pgfqpoint{6.247959in}{0.561327in}}{\pgfqpoint{6.252350in}{0.550728in}}{\pgfqpoint{6.260163in}{0.542915in}}%
\pgfpathcurveto{\pgfqpoint{6.267977in}{0.535101in}}{\pgfqpoint{6.278576in}{0.530711in}}{\pgfqpoint{6.289626in}{0.530711in}}%
\pgfusepath{stroke}%
\end{pgfscope}%
\begin{pgfscope}%
\pgfpathrectangle{\pgfqpoint{0.847223in}{0.554012in}}{\pgfqpoint{6.200000in}{4.530000in}}%
\pgfusepath{clip}%
\pgfsetbuttcap%
\pgfsetroundjoin%
\pgfsetlinewidth{1.003750pt}%
\definecolor{currentstroke}{rgb}{1.000000,0.000000,0.000000}%
\pgfsetstrokecolor{currentstroke}%
\pgfsetdash{}{0pt}%
\pgfpathmoveto{\pgfqpoint{6.298435in}{0.530283in}}%
\pgfpathcurveto{\pgfqpoint{6.309485in}{0.530283in}}{\pgfqpoint{6.320084in}{0.534673in}}{\pgfqpoint{6.327898in}{0.542487in}}%
\pgfpathcurveto{\pgfqpoint{6.335712in}{0.550301in}}{\pgfqpoint{6.340102in}{0.560900in}}{\pgfqpoint{6.340102in}{0.571950in}}%
\pgfpathcurveto{\pgfqpoint{6.340102in}{0.583000in}}{\pgfqpoint{6.335712in}{0.593599in}}{\pgfqpoint{6.327898in}{0.601413in}}%
\pgfpathcurveto{\pgfqpoint{6.320084in}{0.609226in}}{\pgfqpoint{6.309485in}{0.613617in}}{\pgfqpoint{6.298435in}{0.613617in}}%
\pgfpathcurveto{\pgfqpoint{6.287385in}{0.613617in}}{\pgfqpoint{6.276786in}{0.609226in}}{\pgfqpoint{6.268972in}{0.601413in}}%
\pgfpathcurveto{\pgfqpoint{6.261159in}{0.593599in}}{\pgfqpoint{6.256769in}{0.583000in}}{\pgfqpoint{6.256769in}{0.571950in}}%
\pgfpathcurveto{\pgfqpoint{6.256769in}{0.560900in}}{\pgfqpoint{6.261159in}{0.550301in}}{\pgfqpoint{6.268972in}{0.542487in}}%
\pgfpathcurveto{\pgfqpoint{6.276786in}{0.534673in}}{\pgfqpoint{6.287385in}{0.530283in}}{\pgfqpoint{6.298435in}{0.530283in}}%
\pgfusepath{stroke}%
\end{pgfscope}%
\begin{pgfscope}%
\pgfpathrectangle{\pgfqpoint{0.847223in}{0.554012in}}{\pgfqpoint{6.200000in}{4.530000in}}%
\pgfusepath{clip}%
\pgfsetbuttcap%
\pgfsetroundjoin%
\pgfsetlinewidth{1.003750pt}%
\definecolor{currentstroke}{rgb}{1.000000,0.000000,0.000000}%
\pgfsetstrokecolor{currentstroke}%
\pgfsetdash{}{0pt}%
\pgfpathmoveto{\pgfqpoint{6.307245in}{0.529867in}}%
\pgfpathcurveto{\pgfqpoint{6.318295in}{0.529867in}}{\pgfqpoint{6.328894in}{0.534257in}}{\pgfqpoint{6.336707in}{0.542071in}}%
\pgfpathcurveto{\pgfqpoint{6.344521in}{0.549885in}}{\pgfqpoint{6.348911in}{0.560484in}}{\pgfqpoint{6.348911in}{0.571534in}}%
\pgfpathcurveto{\pgfqpoint{6.348911in}{0.582584in}}{\pgfqpoint{6.344521in}{0.593183in}}{\pgfqpoint{6.336707in}{0.600996in}}%
\pgfpathcurveto{\pgfqpoint{6.328894in}{0.608810in}}{\pgfqpoint{6.318295in}{0.613200in}}{\pgfqpoint{6.307245in}{0.613200in}}%
\pgfpathcurveto{\pgfqpoint{6.296194in}{0.613200in}}{\pgfqpoint{6.285595in}{0.608810in}}{\pgfqpoint{6.277782in}{0.600996in}}%
\pgfpathcurveto{\pgfqpoint{6.269968in}{0.593183in}}{\pgfqpoint{6.265578in}{0.582584in}}{\pgfqpoint{6.265578in}{0.571534in}}%
\pgfpathcurveto{\pgfqpoint{6.265578in}{0.560484in}}{\pgfqpoint{6.269968in}{0.549885in}}{\pgfqpoint{6.277782in}{0.542071in}}%
\pgfpathcurveto{\pgfqpoint{6.285595in}{0.534257in}}{\pgfqpoint{6.296194in}{0.529867in}}{\pgfqpoint{6.307245in}{0.529867in}}%
\pgfusepath{stroke}%
\end{pgfscope}%
\begin{pgfscope}%
\pgfpathrectangle{\pgfqpoint{0.847223in}{0.554012in}}{\pgfqpoint{6.200000in}{4.530000in}}%
\pgfusepath{clip}%
\pgfsetbuttcap%
\pgfsetroundjoin%
\pgfsetlinewidth{1.003750pt}%
\definecolor{currentstroke}{rgb}{1.000000,0.000000,0.000000}%
\pgfsetstrokecolor{currentstroke}%
\pgfsetdash{}{0pt}%
\pgfpathmoveto{\pgfqpoint{6.316054in}{0.529462in}}%
\pgfpathcurveto{\pgfqpoint{6.327104in}{0.529462in}}{\pgfqpoint{6.337703in}{0.533852in}}{\pgfqpoint{6.345517in}{0.541665in}}%
\pgfpathcurveto{\pgfqpoint{6.353330in}{0.549479in}}{\pgfqpoint{6.357720in}{0.560078in}}{\pgfqpoint{6.357720in}{0.571128in}}%
\pgfpathcurveto{\pgfqpoint{6.357720in}{0.582178in}}{\pgfqpoint{6.353330in}{0.592777in}}{\pgfqpoint{6.345517in}{0.600591in}}%
\pgfpathcurveto{\pgfqpoint{6.337703in}{0.608405in}}{\pgfqpoint{6.327104in}{0.612795in}}{\pgfqpoint{6.316054in}{0.612795in}}%
\pgfpathcurveto{\pgfqpoint{6.305004in}{0.612795in}}{\pgfqpoint{6.294405in}{0.608405in}}{\pgfqpoint{6.286591in}{0.600591in}}%
\pgfpathcurveto{\pgfqpoint{6.278777in}{0.592777in}}{\pgfqpoint{6.274387in}{0.582178in}}{\pgfqpoint{6.274387in}{0.571128in}}%
\pgfpathcurveto{\pgfqpoint{6.274387in}{0.560078in}}{\pgfqpoint{6.278777in}{0.549479in}}{\pgfqpoint{6.286591in}{0.541665in}}%
\pgfpathcurveto{\pgfqpoint{6.294405in}{0.533852in}}{\pgfqpoint{6.305004in}{0.529462in}}{\pgfqpoint{6.316054in}{0.529462in}}%
\pgfusepath{stroke}%
\end{pgfscope}%
\begin{pgfscope}%
\pgfpathrectangle{\pgfqpoint{0.847223in}{0.554012in}}{\pgfqpoint{6.200000in}{4.530000in}}%
\pgfusepath{clip}%
\pgfsetbuttcap%
\pgfsetroundjoin%
\pgfsetlinewidth{1.003750pt}%
\definecolor{currentstroke}{rgb}{1.000000,0.000000,0.000000}%
\pgfsetstrokecolor{currentstroke}%
\pgfsetdash{}{0pt}%
\pgfpathmoveto{\pgfqpoint{6.324863in}{0.529066in}}%
\pgfpathcurveto{\pgfqpoint{6.335913in}{0.529066in}}{\pgfqpoint{6.346512in}{0.533457in}}{\pgfqpoint{6.354326in}{0.541270in}}%
\pgfpathcurveto{\pgfqpoint{6.362139in}{0.549084in}}{\pgfqpoint{6.366530in}{0.559683in}}{\pgfqpoint{6.366530in}{0.570733in}}%
\pgfpathcurveto{\pgfqpoint{6.366530in}{0.581783in}}{\pgfqpoint{6.362139in}{0.592382in}}{\pgfqpoint{6.354326in}{0.600196in}}%
\pgfpathcurveto{\pgfqpoint{6.346512in}{0.608010in}}{\pgfqpoint{6.335913in}{0.612400in}}{\pgfqpoint{6.324863in}{0.612400in}}%
\pgfpathcurveto{\pgfqpoint{6.313813in}{0.612400in}}{\pgfqpoint{6.303214in}{0.608010in}}{\pgfqpoint{6.295400in}{0.600196in}}%
\pgfpathcurveto{\pgfqpoint{6.287587in}{0.592382in}}{\pgfqpoint{6.283196in}{0.581783in}}{\pgfqpoint{6.283196in}{0.570733in}}%
\pgfpathcurveto{\pgfqpoint{6.283196in}{0.559683in}}{\pgfqpoint{6.287587in}{0.549084in}}{\pgfqpoint{6.295400in}{0.541270in}}%
\pgfpathcurveto{\pgfqpoint{6.303214in}{0.533457in}}{\pgfqpoint{6.313813in}{0.529066in}}{\pgfqpoint{6.324863in}{0.529066in}}%
\pgfusepath{stroke}%
\end{pgfscope}%
\begin{pgfscope}%
\pgfpathrectangle{\pgfqpoint{0.847223in}{0.554012in}}{\pgfqpoint{6.200000in}{4.530000in}}%
\pgfusepath{clip}%
\pgfsetbuttcap%
\pgfsetroundjoin%
\pgfsetlinewidth{1.003750pt}%
\definecolor{currentstroke}{rgb}{1.000000,0.000000,0.000000}%
\pgfsetstrokecolor{currentstroke}%
\pgfsetdash{}{0pt}%
\pgfpathmoveto{\pgfqpoint{6.333672in}{0.528681in}}%
\pgfpathcurveto{\pgfqpoint{6.344722in}{0.528681in}}{\pgfqpoint{6.355321in}{0.533071in}}{\pgfqpoint{6.363135in}{0.540885in}}%
\pgfpathcurveto{\pgfqpoint{6.370949in}{0.548699in}}{\pgfqpoint{6.375339in}{0.559298in}}{\pgfqpoint{6.375339in}{0.570348in}}%
\pgfpathcurveto{\pgfqpoint{6.375339in}{0.581398in}}{\pgfqpoint{6.370949in}{0.591997in}}{\pgfqpoint{6.363135in}{0.599811in}}%
\pgfpathcurveto{\pgfqpoint{6.355321in}{0.607624in}}{\pgfqpoint{6.344722in}{0.612014in}}{\pgfqpoint{6.333672in}{0.612014in}}%
\pgfpathcurveto{\pgfqpoint{6.322622in}{0.612014in}}{\pgfqpoint{6.312023in}{0.607624in}}{\pgfqpoint{6.304210in}{0.599811in}}%
\pgfpathcurveto{\pgfqpoint{6.296396in}{0.591997in}}{\pgfqpoint{6.292006in}{0.581398in}}{\pgfqpoint{6.292006in}{0.570348in}}%
\pgfpathcurveto{\pgfqpoint{6.292006in}{0.559298in}}{\pgfqpoint{6.296396in}{0.548699in}}{\pgfqpoint{6.304210in}{0.540885in}}%
\pgfpathcurveto{\pgfqpoint{6.312023in}{0.533071in}}{\pgfqpoint{6.322622in}{0.528681in}}{\pgfqpoint{6.333672in}{0.528681in}}%
\pgfusepath{stroke}%
\end{pgfscope}%
\begin{pgfscope}%
\pgfpathrectangle{\pgfqpoint{0.847223in}{0.554012in}}{\pgfqpoint{6.200000in}{4.530000in}}%
\pgfusepath{clip}%
\pgfsetbuttcap%
\pgfsetroundjoin%
\pgfsetlinewidth{1.003750pt}%
\definecolor{currentstroke}{rgb}{1.000000,0.000000,0.000000}%
\pgfsetstrokecolor{currentstroke}%
\pgfsetdash{}{0pt}%
\pgfpathmoveto{\pgfqpoint{6.342482in}{0.528305in}}%
\pgfpathcurveto{\pgfqpoint{6.353532in}{0.528305in}}{\pgfqpoint{6.364131in}{0.532695in}}{\pgfqpoint{6.371944in}{0.540509in}}%
\pgfpathcurveto{\pgfqpoint{6.379758in}{0.548323in}}{\pgfqpoint{6.384148in}{0.558922in}}{\pgfqpoint{6.384148in}{0.569972in}}%
\pgfpathcurveto{\pgfqpoint{6.384148in}{0.581022in}}{\pgfqpoint{6.379758in}{0.591621in}}{\pgfqpoint{6.371944in}{0.599435in}}%
\pgfpathcurveto{\pgfqpoint{6.364131in}{0.607248in}}{\pgfqpoint{6.353532in}{0.611638in}}{\pgfqpoint{6.342482in}{0.611638in}}%
\pgfpathcurveto{\pgfqpoint{6.331431in}{0.611638in}}{\pgfqpoint{6.320832in}{0.607248in}}{\pgfqpoint{6.313019in}{0.599435in}}%
\pgfpathcurveto{\pgfqpoint{6.305205in}{0.591621in}}{\pgfqpoint{6.300815in}{0.581022in}}{\pgfqpoint{6.300815in}{0.569972in}}%
\pgfpathcurveto{\pgfqpoint{6.300815in}{0.558922in}}{\pgfqpoint{6.305205in}{0.548323in}}{\pgfqpoint{6.313019in}{0.540509in}}%
\pgfpathcurveto{\pgfqpoint{6.320832in}{0.532695in}}{\pgfqpoint{6.331431in}{0.528305in}}{\pgfqpoint{6.342482in}{0.528305in}}%
\pgfusepath{stroke}%
\end{pgfscope}%
\begin{pgfscope}%
\pgfpathrectangle{\pgfqpoint{0.847223in}{0.554012in}}{\pgfqpoint{6.200000in}{4.530000in}}%
\pgfusepath{clip}%
\pgfsetbuttcap%
\pgfsetroundjoin%
\pgfsetlinewidth{1.003750pt}%
\definecolor{currentstroke}{rgb}{1.000000,0.000000,0.000000}%
\pgfsetstrokecolor{currentstroke}%
\pgfsetdash{}{0pt}%
\pgfpathmoveto{\pgfqpoint{6.351291in}{0.527938in}}%
\pgfpathcurveto{\pgfqpoint{6.362341in}{0.527938in}}{\pgfqpoint{6.372940in}{0.532328in}}{\pgfqpoint{6.380754in}{0.540142in}}%
\pgfpathcurveto{\pgfqpoint{6.388567in}{0.547956in}}{\pgfqpoint{6.392958in}{0.558555in}}{\pgfqpoint{6.392958in}{0.569605in}}%
\pgfpathcurveto{\pgfqpoint{6.392958in}{0.580655in}}{\pgfqpoint{6.388567in}{0.591254in}}{\pgfqpoint{6.380754in}{0.599068in}}%
\pgfpathcurveto{\pgfqpoint{6.372940in}{0.606881in}}{\pgfqpoint{6.362341in}{0.611271in}}{\pgfqpoint{6.351291in}{0.611271in}}%
\pgfpathcurveto{\pgfqpoint{6.340241in}{0.611271in}}{\pgfqpoint{6.329642in}{0.606881in}}{\pgfqpoint{6.321828in}{0.599068in}}%
\pgfpathcurveto{\pgfqpoint{6.314014in}{0.591254in}}{\pgfqpoint{6.309624in}{0.580655in}}{\pgfqpoint{6.309624in}{0.569605in}}%
\pgfpathcurveto{\pgfqpoint{6.309624in}{0.558555in}}{\pgfqpoint{6.314014in}{0.547956in}}{\pgfqpoint{6.321828in}{0.540142in}}%
\pgfpathcurveto{\pgfqpoint{6.329642in}{0.532328in}}{\pgfqpoint{6.340241in}{0.527938in}}{\pgfqpoint{6.351291in}{0.527938in}}%
\pgfusepath{stroke}%
\end{pgfscope}%
\begin{pgfscope}%
\pgfpathrectangle{\pgfqpoint{0.847223in}{0.554012in}}{\pgfqpoint{6.200000in}{4.530000in}}%
\pgfusepath{clip}%
\pgfsetbuttcap%
\pgfsetroundjoin%
\pgfsetlinewidth{1.003750pt}%
\definecolor{currentstroke}{rgb}{1.000000,0.000000,0.000000}%
\pgfsetstrokecolor{currentstroke}%
\pgfsetdash{}{0pt}%
\pgfpathmoveto{\pgfqpoint{6.360100in}{0.527580in}}%
\pgfpathcurveto{\pgfqpoint{6.371150in}{0.527580in}}{\pgfqpoint{6.381749in}{0.531970in}}{\pgfqpoint{6.389563in}{0.539783in}}%
\pgfpathcurveto{\pgfqpoint{6.397377in}{0.547597in}}{\pgfqpoint{6.401767in}{0.558196in}}{\pgfqpoint{6.401767in}{0.569246in}}%
\pgfpathcurveto{\pgfqpoint{6.401767in}{0.580296in}}{\pgfqpoint{6.397377in}{0.590895in}}{\pgfqpoint{6.389563in}{0.598709in}}%
\pgfpathcurveto{\pgfqpoint{6.381749in}{0.606523in}}{\pgfqpoint{6.371150in}{0.610913in}}{\pgfqpoint{6.360100in}{0.610913in}}%
\pgfpathcurveto{\pgfqpoint{6.349050in}{0.610913in}}{\pgfqpoint{6.338451in}{0.606523in}}{\pgfqpoint{6.330637in}{0.598709in}}%
\pgfpathcurveto{\pgfqpoint{6.322824in}{0.590895in}}{\pgfqpoint{6.318433in}{0.580296in}}{\pgfqpoint{6.318433in}{0.569246in}}%
\pgfpathcurveto{\pgfqpoint{6.318433in}{0.558196in}}{\pgfqpoint{6.322824in}{0.547597in}}{\pgfqpoint{6.330637in}{0.539783in}}%
\pgfpathcurveto{\pgfqpoint{6.338451in}{0.531970in}}{\pgfqpoint{6.349050in}{0.527580in}}{\pgfqpoint{6.360100in}{0.527580in}}%
\pgfusepath{stroke}%
\end{pgfscope}%
\begin{pgfscope}%
\pgfpathrectangle{\pgfqpoint{0.847223in}{0.554012in}}{\pgfqpoint{6.200000in}{4.530000in}}%
\pgfusepath{clip}%
\pgfsetbuttcap%
\pgfsetroundjoin%
\pgfsetlinewidth{1.003750pt}%
\definecolor{currentstroke}{rgb}{1.000000,0.000000,0.000000}%
\pgfsetstrokecolor{currentstroke}%
\pgfsetdash{}{0pt}%
\pgfpathmoveto{\pgfqpoint{6.368909in}{0.527229in}}%
\pgfpathcurveto{\pgfqpoint{6.379960in}{0.527229in}}{\pgfqpoint{6.390559in}{0.531620in}}{\pgfqpoint{6.398372in}{0.539433in}}%
\pgfpathcurveto{\pgfqpoint{6.406186in}{0.547247in}}{\pgfqpoint{6.410576in}{0.557846in}}{\pgfqpoint{6.410576in}{0.568896in}}%
\pgfpathcurveto{\pgfqpoint{6.410576in}{0.579946in}}{\pgfqpoint{6.406186in}{0.590545in}}{\pgfqpoint{6.398372in}{0.598359in}}%
\pgfpathcurveto{\pgfqpoint{6.390559in}{0.606172in}}{\pgfqpoint{6.379960in}{0.610563in}}{\pgfqpoint{6.368909in}{0.610563in}}%
\pgfpathcurveto{\pgfqpoint{6.357859in}{0.610563in}}{\pgfqpoint{6.347260in}{0.606172in}}{\pgfqpoint{6.339447in}{0.598359in}}%
\pgfpathcurveto{\pgfqpoint{6.331633in}{0.590545in}}{\pgfqpoint{6.327243in}{0.579946in}}{\pgfqpoint{6.327243in}{0.568896in}}%
\pgfpathcurveto{\pgfqpoint{6.327243in}{0.557846in}}{\pgfqpoint{6.331633in}{0.547247in}}{\pgfqpoint{6.339447in}{0.539433in}}%
\pgfpathcurveto{\pgfqpoint{6.347260in}{0.531620in}}{\pgfqpoint{6.357859in}{0.527229in}}{\pgfqpoint{6.368909in}{0.527229in}}%
\pgfusepath{stroke}%
\end{pgfscope}%
\begin{pgfscope}%
\pgfpathrectangle{\pgfqpoint{0.847223in}{0.554012in}}{\pgfqpoint{6.200000in}{4.530000in}}%
\pgfusepath{clip}%
\pgfsetbuttcap%
\pgfsetroundjoin%
\pgfsetlinewidth{1.003750pt}%
\definecolor{currentstroke}{rgb}{1.000000,0.000000,0.000000}%
\pgfsetstrokecolor{currentstroke}%
\pgfsetdash{}{0pt}%
\pgfpathmoveto{\pgfqpoint{6.377719in}{0.526887in}}%
\pgfpathcurveto{\pgfqpoint{6.388769in}{0.526887in}}{\pgfqpoint{6.399368in}{0.531277in}}{\pgfqpoint{6.407181in}{0.539091in}}%
\pgfpathcurveto{\pgfqpoint{6.414995in}{0.546904in}}{\pgfqpoint{6.419385in}{0.557503in}}{\pgfqpoint{6.419385in}{0.568554in}}%
\pgfpathcurveto{\pgfqpoint{6.419385in}{0.579604in}}{\pgfqpoint{6.414995in}{0.590203in}}{\pgfqpoint{6.407181in}{0.598016in}}%
\pgfpathcurveto{\pgfqpoint{6.399368in}{0.605830in}}{\pgfqpoint{6.388769in}{0.610220in}}{\pgfqpoint{6.377719in}{0.610220in}}%
\pgfpathcurveto{\pgfqpoint{6.366669in}{0.610220in}}{\pgfqpoint{6.356070in}{0.605830in}}{\pgfqpoint{6.348256in}{0.598016in}}%
\pgfpathcurveto{\pgfqpoint{6.340442in}{0.590203in}}{\pgfqpoint{6.336052in}{0.579604in}}{\pgfqpoint{6.336052in}{0.568554in}}%
\pgfpathcurveto{\pgfqpoint{6.336052in}{0.557503in}}{\pgfqpoint{6.340442in}{0.546904in}}{\pgfqpoint{6.348256in}{0.539091in}}%
\pgfpathcurveto{\pgfqpoint{6.356070in}{0.531277in}}{\pgfqpoint{6.366669in}{0.526887in}}{\pgfqpoint{6.377719in}{0.526887in}}%
\pgfusepath{stroke}%
\end{pgfscope}%
\begin{pgfscope}%
\pgfpathrectangle{\pgfqpoint{0.847223in}{0.554012in}}{\pgfqpoint{6.200000in}{4.530000in}}%
\pgfusepath{clip}%
\pgfsetbuttcap%
\pgfsetroundjoin%
\pgfsetlinewidth{1.003750pt}%
\definecolor{currentstroke}{rgb}{1.000000,0.000000,0.000000}%
\pgfsetstrokecolor{currentstroke}%
\pgfsetdash{}{0pt}%
\pgfpathmoveto{\pgfqpoint{6.386528in}{0.526552in}}%
\pgfpathcurveto{\pgfqpoint{6.397578in}{0.526552in}}{\pgfqpoint{6.408177in}{0.530942in}}{\pgfqpoint{6.415991in}{0.538756in}}%
\pgfpathcurveto{\pgfqpoint{6.423804in}{0.546570in}}{\pgfqpoint{6.428195in}{0.557169in}}{\pgfqpoint{6.428195in}{0.568219in}}%
\pgfpathcurveto{\pgfqpoint{6.428195in}{0.579269in}}{\pgfqpoint{6.423804in}{0.589868in}}{\pgfqpoint{6.415991in}{0.597682in}}%
\pgfpathcurveto{\pgfqpoint{6.408177in}{0.605495in}}{\pgfqpoint{6.397578in}{0.609885in}}{\pgfqpoint{6.386528in}{0.609885in}}%
\pgfpathcurveto{\pgfqpoint{6.375478in}{0.609885in}}{\pgfqpoint{6.364879in}{0.605495in}}{\pgfqpoint{6.357065in}{0.597682in}}%
\pgfpathcurveto{\pgfqpoint{6.349252in}{0.589868in}}{\pgfqpoint{6.344861in}{0.579269in}}{\pgfqpoint{6.344861in}{0.568219in}}%
\pgfpathcurveto{\pgfqpoint{6.344861in}{0.557169in}}{\pgfqpoint{6.349252in}{0.546570in}}{\pgfqpoint{6.357065in}{0.538756in}}%
\pgfpathcurveto{\pgfqpoint{6.364879in}{0.530942in}}{\pgfqpoint{6.375478in}{0.526552in}}{\pgfqpoint{6.386528in}{0.526552in}}%
\pgfusepath{stroke}%
\end{pgfscope}%
\begin{pgfscope}%
\pgfpathrectangle{\pgfqpoint{0.847223in}{0.554012in}}{\pgfqpoint{6.200000in}{4.530000in}}%
\pgfusepath{clip}%
\pgfsetbuttcap%
\pgfsetroundjoin%
\pgfsetlinewidth{1.003750pt}%
\definecolor{currentstroke}{rgb}{1.000000,0.000000,0.000000}%
\pgfsetstrokecolor{currentstroke}%
\pgfsetdash{}{0pt}%
\pgfpathmoveto{\pgfqpoint{6.395337in}{0.526225in}}%
\pgfpathcurveto{\pgfqpoint{6.406387in}{0.526225in}}{\pgfqpoint{6.416986in}{0.530615in}}{\pgfqpoint{6.424800in}{0.538428in}}%
\pgfpathcurveto{\pgfqpoint{6.432614in}{0.546242in}}{\pgfqpoint{6.437004in}{0.556841in}}{\pgfqpoint{6.437004in}{0.567891in}}%
\pgfpathcurveto{\pgfqpoint{6.437004in}{0.578941in}}{\pgfqpoint{6.432614in}{0.589540in}}{\pgfqpoint{6.424800in}{0.597354in}}%
\pgfpathcurveto{\pgfqpoint{6.416986in}{0.605168in}}{\pgfqpoint{6.406387in}{0.609558in}}{\pgfqpoint{6.395337in}{0.609558in}}%
\pgfpathcurveto{\pgfqpoint{6.384287in}{0.609558in}}{\pgfqpoint{6.373688in}{0.605168in}}{\pgfqpoint{6.365874in}{0.597354in}}%
\pgfpathcurveto{\pgfqpoint{6.358061in}{0.589540in}}{\pgfqpoint{6.353671in}{0.578941in}}{\pgfqpoint{6.353671in}{0.567891in}}%
\pgfpathcurveto{\pgfqpoint{6.353671in}{0.556841in}}{\pgfqpoint{6.358061in}{0.546242in}}{\pgfqpoint{6.365874in}{0.538428in}}%
\pgfpathcurveto{\pgfqpoint{6.373688in}{0.530615in}}{\pgfqpoint{6.384287in}{0.526225in}}{\pgfqpoint{6.395337in}{0.526225in}}%
\pgfusepath{stroke}%
\end{pgfscope}%
\begin{pgfscope}%
\pgfpathrectangle{\pgfqpoint{0.847223in}{0.554012in}}{\pgfqpoint{6.200000in}{4.530000in}}%
\pgfusepath{clip}%
\pgfsetbuttcap%
\pgfsetroundjoin%
\pgfsetlinewidth{1.003750pt}%
\definecolor{currentstroke}{rgb}{1.000000,0.000000,0.000000}%
\pgfsetstrokecolor{currentstroke}%
\pgfsetdash{}{0pt}%
\pgfpathmoveto{\pgfqpoint{6.404146in}{0.525904in}}%
\pgfpathcurveto{\pgfqpoint{6.415197in}{0.525904in}}{\pgfqpoint{6.425796in}{0.530294in}}{\pgfqpoint{6.433609in}{0.538108in}}%
\pgfpathcurveto{\pgfqpoint{6.441423in}{0.545922in}}{\pgfqpoint{6.445813in}{0.556521in}}{\pgfqpoint{6.445813in}{0.567571in}}%
\pgfpathcurveto{\pgfqpoint{6.445813in}{0.578621in}}{\pgfqpoint{6.441423in}{0.589220in}}{\pgfqpoint{6.433609in}{0.597033in}}%
\pgfpathcurveto{\pgfqpoint{6.425796in}{0.604847in}}{\pgfqpoint{6.415197in}{0.609237in}}{\pgfqpoint{6.404146in}{0.609237in}}%
\pgfpathcurveto{\pgfqpoint{6.393096in}{0.609237in}}{\pgfqpoint{6.382497in}{0.604847in}}{\pgfqpoint{6.374684in}{0.597033in}}%
\pgfpathcurveto{\pgfqpoint{6.366870in}{0.589220in}}{\pgfqpoint{6.362480in}{0.578621in}}{\pgfqpoint{6.362480in}{0.567571in}}%
\pgfpathcurveto{\pgfqpoint{6.362480in}{0.556521in}}{\pgfqpoint{6.366870in}{0.545922in}}{\pgfqpoint{6.374684in}{0.538108in}}%
\pgfpathcurveto{\pgfqpoint{6.382497in}{0.530294in}}{\pgfqpoint{6.393096in}{0.525904in}}{\pgfqpoint{6.404146in}{0.525904in}}%
\pgfusepath{stroke}%
\end{pgfscope}%
\begin{pgfscope}%
\pgfpathrectangle{\pgfqpoint{0.847223in}{0.554012in}}{\pgfqpoint{6.200000in}{4.530000in}}%
\pgfusepath{clip}%
\pgfsetbuttcap%
\pgfsetroundjoin%
\pgfsetlinewidth{1.003750pt}%
\definecolor{currentstroke}{rgb}{1.000000,0.000000,0.000000}%
\pgfsetstrokecolor{currentstroke}%
\pgfsetdash{}{0pt}%
\pgfpathmoveto{\pgfqpoint{6.412956in}{0.525590in}}%
\pgfpathcurveto{\pgfqpoint{6.424006in}{0.525590in}}{\pgfqpoint{6.434605in}{0.529980in}}{\pgfqpoint{6.442419in}{0.537794in}}%
\pgfpathcurveto{\pgfqpoint{6.450232in}{0.545608in}}{\pgfqpoint{6.454622in}{0.556207in}}{\pgfqpoint{6.454622in}{0.567257in}}%
\pgfpathcurveto{\pgfqpoint{6.454622in}{0.578307in}}{\pgfqpoint{6.450232in}{0.588906in}}{\pgfqpoint{6.442419in}{0.596720in}}%
\pgfpathcurveto{\pgfqpoint{6.434605in}{0.604533in}}{\pgfqpoint{6.424006in}{0.608923in}}{\pgfqpoint{6.412956in}{0.608923in}}%
\pgfpathcurveto{\pgfqpoint{6.401906in}{0.608923in}}{\pgfqpoint{6.391307in}{0.604533in}}{\pgfqpoint{6.383493in}{0.596720in}}%
\pgfpathcurveto{\pgfqpoint{6.375679in}{0.588906in}}{\pgfqpoint{6.371289in}{0.578307in}}{\pgfqpoint{6.371289in}{0.567257in}}%
\pgfpathcurveto{\pgfqpoint{6.371289in}{0.556207in}}{\pgfqpoint{6.375679in}{0.545608in}}{\pgfqpoint{6.383493in}{0.537794in}}%
\pgfpathcurveto{\pgfqpoint{6.391307in}{0.529980in}}{\pgfqpoint{6.401906in}{0.525590in}}{\pgfqpoint{6.412956in}{0.525590in}}%
\pgfusepath{stroke}%
\end{pgfscope}%
\begin{pgfscope}%
\pgfpathrectangle{\pgfqpoint{0.847223in}{0.554012in}}{\pgfqpoint{6.200000in}{4.530000in}}%
\pgfusepath{clip}%
\pgfsetbuttcap%
\pgfsetroundjoin%
\pgfsetlinewidth{1.003750pt}%
\definecolor{currentstroke}{rgb}{1.000000,0.000000,0.000000}%
\pgfsetstrokecolor{currentstroke}%
\pgfsetdash{}{0pt}%
\pgfpathmoveto{\pgfqpoint{6.421765in}{0.525283in}}%
\pgfpathcurveto{\pgfqpoint{6.432815in}{0.525283in}}{\pgfqpoint{6.443414in}{0.529673in}}{\pgfqpoint{6.451228in}{0.537487in}}%
\pgfpathcurveto{\pgfqpoint{6.459041in}{0.545300in}}{\pgfqpoint{6.463432in}{0.555899in}}{\pgfqpoint{6.463432in}{0.566949in}}%
\pgfpathcurveto{\pgfqpoint{6.463432in}{0.578000in}}{\pgfqpoint{6.459041in}{0.588599in}}{\pgfqpoint{6.451228in}{0.596412in}}%
\pgfpathcurveto{\pgfqpoint{6.443414in}{0.604226in}}{\pgfqpoint{6.432815in}{0.608616in}}{\pgfqpoint{6.421765in}{0.608616in}}%
\pgfpathcurveto{\pgfqpoint{6.410715in}{0.608616in}}{\pgfqpoint{6.400116in}{0.604226in}}{\pgfqpoint{6.392302in}{0.596412in}}%
\pgfpathcurveto{\pgfqpoint{6.384489in}{0.588599in}}{\pgfqpoint{6.380098in}{0.578000in}}{\pgfqpoint{6.380098in}{0.566949in}}%
\pgfpathcurveto{\pgfqpoint{6.380098in}{0.555899in}}{\pgfqpoint{6.384489in}{0.545300in}}{\pgfqpoint{6.392302in}{0.537487in}}%
\pgfpathcurveto{\pgfqpoint{6.400116in}{0.529673in}}{\pgfqpoint{6.410715in}{0.525283in}}{\pgfqpoint{6.421765in}{0.525283in}}%
\pgfusepath{stroke}%
\end{pgfscope}%
\begin{pgfscope}%
\pgfpathrectangle{\pgfqpoint{0.847223in}{0.554012in}}{\pgfqpoint{6.200000in}{4.530000in}}%
\pgfusepath{clip}%
\pgfsetbuttcap%
\pgfsetroundjoin%
\pgfsetlinewidth{1.003750pt}%
\definecolor{currentstroke}{rgb}{1.000000,0.000000,0.000000}%
\pgfsetstrokecolor{currentstroke}%
\pgfsetdash{}{0pt}%
\pgfpathmoveto{\pgfqpoint{6.430574in}{0.524982in}}%
\pgfpathcurveto{\pgfqpoint{6.441624in}{0.524982in}}{\pgfqpoint{6.452223in}{0.529372in}}{\pgfqpoint{6.460037in}{0.537185in}}%
\pgfpathcurveto{\pgfqpoint{6.467851in}{0.544999in}}{\pgfqpoint{6.472241in}{0.555598in}}{\pgfqpoint{6.472241in}{0.566648in}}%
\pgfpathcurveto{\pgfqpoint{6.472241in}{0.577698in}}{\pgfqpoint{6.467851in}{0.588297in}}{\pgfqpoint{6.460037in}{0.596111in}}%
\pgfpathcurveto{\pgfqpoint{6.452223in}{0.603925in}}{\pgfqpoint{6.441624in}{0.608315in}}{\pgfqpoint{6.430574in}{0.608315in}}%
\pgfpathcurveto{\pgfqpoint{6.419524in}{0.608315in}}{\pgfqpoint{6.408925in}{0.603925in}}{\pgfqpoint{6.401111in}{0.596111in}}%
\pgfpathcurveto{\pgfqpoint{6.393298in}{0.588297in}}{\pgfqpoint{6.388908in}{0.577698in}}{\pgfqpoint{6.388908in}{0.566648in}}%
\pgfpathcurveto{\pgfqpoint{6.388908in}{0.555598in}}{\pgfqpoint{6.393298in}{0.544999in}}{\pgfqpoint{6.401111in}{0.537185in}}%
\pgfpathcurveto{\pgfqpoint{6.408925in}{0.529372in}}{\pgfqpoint{6.419524in}{0.524982in}}{\pgfqpoint{6.430574in}{0.524982in}}%
\pgfusepath{stroke}%
\end{pgfscope}%
\begin{pgfscope}%
\pgfpathrectangle{\pgfqpoint{0.847223in}{0.554012in}}{\pgfqpoint{6.200000in}{4.530000in}}%
\pgfusepath{clip}%
\pgfsetbuttcap%
\pgfsetroundjoin%
\pgfsetlinewidth{1.003750pt}%
\definecolor{currentstroke}{rgb}{1.000000,0.000000,0.000000}%
\pgfsetstrokecolor{currentstroke}%
\pgfsetdash{}{0pt}%
\pgfpathmoveto{\pgfqpoint{6.439384in}{0.524686in}}%
\pgfpathcurveto{\pgfqpoint{6.450434in}{0.524686in}}{\pgfqpoint{6.461033in}{0.529077in}}{\pgfqpoint{6.468846in}{0.536890in}}%
\pgfpathcurveto{\pgfqpoint{6.476660in}{0.544704in}}{\pgfqpoint{6.481050in}{0.555303in}}{\pgfqpoint{6.481050in}{0.566353in}}%
\pgfpathcurveto{\pgfqpoint{6.481050in}{0.577403in}}{\pgfqpoint{6.476660in}{0.588002in}}{\pgfqpoint{6.468846in}{0.595816in}}%
\pgfpathcurveto{\pgfqpoint{6.461033in}{0.603630in}}{\pgfqpoint{6.450434in}{0.608020in}}{\pgfqpoint{6.439384in}{0.608020in}}%
\pgfpathcurveto{\pgfqpoint{6.428333in}{0.608020in}}{\pgfqpoint{6.417734in}{0.603630in}}{\pgfqpoint{6.409921in}{0.595816in}}%
\pgfpathcurveto{\pgfqpoint{6.402107in}{0.588002in}}{\pgfqpoint{6.397717in}{0.577403in}}{\pgfqpoint{6.397717in}{0.566353in}}%
\pgfpathcurveto{\pgfqpoint{6.397717in}{0.555303in}}{\pgfqpoint{6.402107in}{0.544704in}}{\pgfqpoint{6.409921in}{0.536890in}}%
\pgfpathcurveto{\pgfqpoint{6.417734in}{0.529077in}}{\pgfqpoint{6.428333in}{0.524686in}}{\pgfqpoint{6.439384in}{0.524686in}}%
\pgfusepath{stroke}%
\end{pgfscope}%
\begin{pgfscope}%
\pgfpathrectangle{\pgfqpoint{0.847223in}{0.554012in}}{\pgfqpoint{6.200000in}{4.530000in}}%
\pgfusepath{clip}%
\pgfsetbuttcap%
\pgfsetroundjoin%
\pgfsetlinewidth{1.003750pt}%
\definecolor{currentstroke}{rgb}{1.000000,0.000000,0.000000}%
\pgfsetstrokecolor{currentstroke}%
\pgfsetdash{}{0pt}%
\pgfpathmoveto{\pgfqpoint{6.448193in}{0.524397in}}%
\pgfpathcurveto{\pgfqpoint{6.459243in}{0.524397in}}{\pgfqpoint{6.469842in}{0.528787in}}{\pgfqpoint{6.477656in}{0.536601in}}%
\pgfpathcurveto{\pgfqpoint{6.485469in}{0.544415in}}{\pgfqpoint{6.489859in}{0.555014in}}{\pgfqpoint{6.489859in}{0.566064in}}%
\pgfpathcurveto{\pgfqpoint{6.489859in}{0.577114in}}{\pgfqpoint{6.485469in}{0.587713in}}{\pgfqpoint{6.477656in}{0.595527in}}%
\pgfpathcurveto{\pgfqpoint{6.469842in}{0.603340in}}{\pgfqpoint{6.459243in}{0.607730in}}{\pgfqpoint{6.448193in}{0.607730in}}%
\pgfpathcurveto{\pgfqpoint{6.437143in}{0.607730in}}{\pgfqpoint{6.426544in}{0.603340in}}{\pgfqpoint{6.418730in}{0.595527in}}%
\pgfpathcurveto{\pgfqpoint{6.410916in}{0.587713in}}{\pgfqpoint{6.406526in}{0.577114in}}{\pgfqpoint{6.406526in}{0.566064in}}%
\pgfpathcurveto{\pgfqpoint{6.406526in}{0.555014in}}{\pgfqpoint{6.410916in}{0.544415in}}{\pgfqpoint{6.418730in}{0.536601in}}%
\pgfpathcurveto{\pgfqpoint{6.426544in}{0.528787in}}{\pgfqpoint{6.437143in}{0.524397in}}{\pgfqpoint{6.448193in}{0.524397in}}%
\pgfusepath{stroke}%
\end{pgfscope}%
\begin{pgfscope}%
\pgfpathrectangle{\pgfqpoint{0.847223in}{0.554012in}}{\pgfqpoint{6.200000in}{4.530000in}}%
\pgfusepath{clip}%
\pgfsetbuttcap%
\pgfsetroundjoin%
\pgfsetlinewidth{1.003750pt}%
\definecolor{currentstroke}{rgb}{1.000000,0.000000,0.000000}%
\pgfsetstrokecolor{currentstroke}%
\pgfsetdash{}{0pt}%
\pgfpathmoveto{\pgfqpoint{6.457002in}{0.524113in}}%
\pgfpathcurveto{\pgfqpoint{6.468052in}{0.524113in}}{\pgfqpoint{6.478651in}{0.528504in}}{\pgfqpoint{6.486465in}{0.536317in}}%
\pgfpathcurveto{\pgfqpoint{6.494278in}{0.544131in}}{\pgfqpoint{6.498669in}{0.554730in}}{\pgfqpoint{6.498669in}{0.565780in}}%
\pgfpathcurveto{\pgfqpoint{6.498669in}{0.576830in}}{\pgfqpoint{6.494278in}{0.587429in}}{\pgfqpoint{6.486465in}{0.595243in}}%
\pgfpathcurveto{\pgfqpoint{6.478651in}{0.603056in}}{\pgfqpoint{6.468052in}{0.607447in}}{\pgfqpoint{6.457002in}{0.607447in}}%
\pgfpathcurveto{\pgfqpoint{6.445952in}{0.607447in}}{\pgfqpoint{6.435353in}{0.603056in}}{\pgfqpoint{6.427539in}{0.595243in}}%
\pgfpathcurveto{\pgfqpoint{6.419726in}{0.587429in}}{\pgfqpoint{6.415335in}{0.576830in}}{\pgfqpoint{6.415335in}{0.565780in}}%
\pgfpathcurveto{\pgfqpoint{6.415335in}{0.554730in}}{\pgfqpoint{6.419726in}{0.544131in}}{\pgfqpoint{6.427539in}{0.536317in}}%
\pgfpathcurveto{\pgfqpoint{6.435353in}{0.528504in}}{\pgfqpoint{6.445952in}{0.524113in}}{\pgfqpoint{6.457002in}{0.524113in}}%
\pgfusepath{stroke}%
\end{pgfscope}%
\begin{pgfscope}%
\pgfpathrectangle{\pgfqpoint{0.847223in}{0.554012in}}{\pgfqpoint{6.200000in}{4.530000in}}%
\pgfusepath{clip}%
\pgfsetbuttcap%
\pgfsetroundjoin%
\pgfsetlinewidth{1.003750pt}%
\definecolor{currentstroke}{rgb}{1.000000,0.000000,0.000000}%
\pgfsetstrokecolor{currentstroke}%
\pgfsetdash{}{0pt}%
\pgfpathmoveto{\pgfqpoint{6.465811in}{0.523835in}}%
\pgfpathcurveto{\pgfqpoint{6.476861in}{0.523835in}}{\pgfqpoint{6.487461in}{0.528225in}}{\pgfqpoint{6.495274in}{0.536039in}}%
\pgfpathcurveto{\pgfqpoint{6.503088in}{0.543853in}}{\pgfqpoint{6.507478in}{0.554452in}}{\pgfqpoint{6.507478in}{0.565502in}}%
\pgfpathcurveto{\pgfqpoint{6.507478in}{0.576552in}}{\pgfqpoint{6.503088in}{0.587151in}}{\pgfqpoint{6.495274in}{0.594965in}}%
\pgfpathcurveto{\pgfqpoint{6.487461in}{0.602778in}}{\pgfqpoint{6.476861in}{0.607168in}}{\pgfqpoint{6.465811in}{0.607168in}}%
\pgfpathcurveto{\pgfqpoint{6.454761in}{0.607168in}}{\pgfqpoint{6.444162in}{0.602778in}}{\pgfqpoint{6.436349in}{0.594965in}}%
\pgfpathcurveto{\pgfqpoint{6.428535in}{0.587151in}}{\pgfqpoint{6.424145in}{0.576552in}}{\pgfqpoint{6.424145in}{0.565502in}}%
\pgfpathcurveto{\pgfqpoint{6.424145in}{0.554452in}}{\pgfqpoint{6.428535in}{0.543853in}}{\pgfqpoint{6.436349in}{0.536039in}}%
\pgfpathcurveto{\pgfqpoint{6.444162in}{0.528225in}}{\pgfqpoint{6.454761in}{0.523835in}}{\pgfqpoint{6.465811in}{0.523835in}}%
\pgfusepath{stroke}%
\end{pgfscope}%
\begin{pgfscope}%
\pgfpathrectangle{\pgfqpoint{0.847223in}{0.554012in}}{\pgfqpoint{6.200000in}{4.530000in}}%
\pgfusepath{clip}%
\pgfsetbuttcap%
\pgfsetroundjoin%
\pgfsetlinewidth{1.003750pt}%
\definecolor{currentstroke}{rgb}{1.000000,0.000000,0.000000}%
\pgfsetstrokecolor{currentstroke}%
\pgfsetdash{}{0pt}%
\pgfpathmoveto{\pgfqpoint{6.474621in}{0.523562in}}%
\pgfpathcurveto{\pgfqpoint{6.485671in}{0.523562in}}{\pgfqpoint{6.496270in}{0.527952in}}{\pgfqpoint{6.504083in}{0.535766in}}%
\pgfpathcurveto{\pgfqpoint{6.511897in}{0.543580in}}{\pgfqpoint{6.516287in}{0.554179in}}{\pgfqpoint{6.516287in}{0.565229in}}%
\pgfpathcurveto{\pgfqpoint{6.516287in}{0.576279in}}{\pgfqpoint{6.511897in}{0.586878in}}{\pgfqpoint{6.504083in}{0.594691in}}%
\pgfpathcurveto{\pgfqpoint{6.496270in}{0.602505in}}{\pgfqpoint{6.485671in}{0.606895in}}{\pgfqpoint{6.474621in}{0.606895in}}%
\pgfpathcurveto{\pgfqpoint{6.463570in}{0.606895in}}{\pgfqpoint{6.452971in}{0.602505in}}{\pgfqpoint{6.445158in}{0.594691in}}%
\pgfpathcurveto{\pgfqpoint{6.437344in}{0.586878in}}{\pgfqpoint{6.432954in}{0.576279in}}{\pgfqpoint{6.432954in}{0.565229in}}%
\pgfpathcurveto{\pgfqpoint{6.432954in}{0.554179in}}{\pgfqpoint{6.437344in}{0.543580in}}{\pgfqpoint{6.445158in}{0.535766in}}%
\pgfpathcurveto{\pgfqpoint{6.452971in}{0.527952in}}{\pgfqpoint{6.463570in}{0.523562in}}{\pgfqpoint{6.474621in}{0.523562in}}%
\pgfusepath{stroke}%
\end{pgfscope}%
\begin{pgfscope}%
\pgfpathrectangle{\pgfqpoint{0.847223in}{0.554012in}}{\pgfqpoint{6.200000in}{4.530000in}}%
\pgfusepath{clip}%
\pgfsetbuttcap%
\pgfsetroundjoin%
\pgfsetlinewidth{1.003750pt}%
\definecolor{currentstroke}{rgb}{1.000000,0.000000,0.000000}%
\pgfsetstrokecolor{currentstroke}%
\pgfsetdash{}{0pt}%
\pgfpathmoveto{\pgfqpoint{6.483430in}{0.523294in}}%
\pgfpathcurveto{\pgfqpoint{6.494480in}{0.523294in}}{\pgfqpoint{6.505079in}{0.527684in}}{\pgfqpoint{6.512893in}{0.535498in}}%
\pgfpathcurveto{\pgfqpoint{6.520706in}{0.543311in}}{\pgfqpoint{6.525097in}{0.553911in}}{\pgfqpoint{6.525097in}{0.564961in}}%
\pgfpathcurveto{\pgfqpoint{6.525097in}{0.576011in}}{\pgfqpoint{6.520706in}{0.586610in}}{\pgfqpoint{6.512893in}{0.594423in}}%
\pgfpathcurveto{\pgfqpoint{6.505079in}{0.602237in}}{\pgfqpoint{6.494480in}{0.606627in}}{\pgfqpoint{6.483430in}{0.606627in}}%
\pgfpathcurveto{\pgfqpoint{6.472380in}{0.606627in}}{\pgfqpoint{6.461781in}{0.602237in}}{\pgfqpoint{6.453967in}{0.594423in}}%
\pgfpathcurveto{\pgfqpoint{6.446153in}{0.586610in}}{\pgfqpoint{6.441763in}{0.576011in}}{\pgfqpoint{6.441763in}{0.564961in}}%
\pgfpathcurveto{\pgfqpoint{6.441763in}{0.553911in}}{\pgfqpoint{6.446153in}{0.543311in}}{\pgfqpoint{6.453967in}{0.535498in}}%
\pgfpathcurveto{\pgfqpoint{6.461781in}{0.527684in}}{\pgfqpoint{6.472380in}{0.523294in}}{\pgfqpoint{6.483430in}{0.523294in}}%
\pgfusepath{stroke}%
\end{pgfscope}%
\begin{pgfscope}%
\pgfpathrectangle{\pgfqpoint{0.847223in}{0.554012in}}{\pgfqpoint{6.200000in}{4.530000in}}%
\pgfusepath{clip}%
\pgfsetbuttcap%
\pgfsetroundjoin%
\pgfsetlinewidth{1.003750pt}%
\definecolor{currentstroke}{rgb}{1.000000,0.000000,0.000000}%
\pgfsetstrokecolor{currentstroke}%
\pgfsetdash{}{0pt}%
\pgfpathmoveto{\pgfqpoint{6.492239in}{0.523031in}}%
\pgfpathcurveto{\pgfqpoint{6.503289in}{0.523031in}}{\pgfqpoint{6.513888in}{0.527421in}}{\pgfqpoint{6.521702in}{0.535235in}}%
\pgfpathcurveto{\pgfqpoint{6.529516in}{0.543048in}}{\pgfqpoint{6.533906in}{0.553647in}}{\pgfqpoint{6.533906in}{0.564698in}}%
\pgfpathcurveto{\pgfqpoint{6.533906in}{0.575748in}}{\pgfqpoint{6.529516in}{0.586347in}}{\pgfqpoint{6.521702in}{0.594160in}}%
\pgfpathcurveto{\pgfqpoint{6.513888in}{0.601974in}}{\pgfqpoint{6.503289in}{0.606364in}}{\pgfqpoint{6.492239in}{0.606364in}}%
\pgfpathcurveto{\pgfqpoint{6.481189in}{0.606364in}}{\pgfqpoint{6.470590in}{0.601974in}}{\pgfqpoint{6.462776in}{0.594160in}}%
\pgfpathcurveto{\pgfqpoint{6.454963in}{0.586347in}}{\pgfqpoint{6.450572in}{0.575748in}}{\pgfqpoint{6.450572in}{0.564698in}}%
\pgfpathcurveto{\pgfqpoint{6.450572in}{0.553647in}}{\pgfqpoint{6.454963in}{0.543048in}}{\pgfqpoint{6.462776in}{0.535235in}}%
\pgfpathcurveto{\pgfqpoint{6.470590in}{0.527421in}}{\pgfqpoint{6.481189in}{0.523031in}}{\pgfqpoint{6.492239in}{0.523031in}}%
\pgfusepath{stroke}%
\end{pgfscope}%
\begin{pgfscope}%
\pgfpathrectangle{\pgfqpoint{0.847223in}{0.554012in}}{\pgfqpoint{6.200000in}{4.530000in}}%
\pgfusepath{clip}%
\pgfsetbuttcap%
\pgfsetroundjoin%
\pgfsetlinewidth{1.003750pt}%
\definecolor{currentstroke}{rgb}{1.000000,0.000000,0.000000}%
\pgfsetstrokecolor{currentstroke}%
\pgfsetdash{}{0pt}%
\pgfpathmoveto{\pgfqpoint{6.501048in}{0.522772in}}%
\pgfpathcurveto{\pgfqpoint{6.512099in}{0.522772in}}{\pgfqpoint{6.522698in}{0.527163in}}{\pgfqpoint{6.530511in}{0.534976in}}%
\pgfpathcurveto{\pgfqpoint{6.538325in}{0.542790in}}{\pgfqpoint{6.542715in}{0.553389in}}{\pgfqpoint{6.542715in}{0.564439in}}%
\pgfpathcurveto{\pgfqpoint{6.542715in}{0.575489in}}{\pgfqpoint{6.538325in}{0.586088in}}{\pgfqpoint{6.530511in}{0.593902in}}%
\pgfpathcurveto{\pgfqpoint{6.522698in}{0.601716in}}{\pgfqpoint{6.512099in}{0.606106in}}{\pgfqpoint{6.501048in}{0.606106in}}%
\pgfpathcurveto{\pgfqpoint{6.489998in}{0.606106in}}{\pgfqpoint{6.479399in}{0.601716in}}{\pgfqpoint{6.471586in}{0.593902in}}%
\pgfpathcurveto{\pgfqpoint{6.463772in}{0.586088in}}{\pgfqpoint{6.459382in}{0.575489in}}{\pgfqpoint{6.459382in}{0.564439in}}%
\pgfpathcurveto{\pgfqpoint{6.459382in}{0.553389in}}{\pgfqpoint{6.463772in}{0.542790in}}{\pgfqpoint{6.471586in}{0.534976in}}%
\pgfpathcurveto{\pgfqpoint{6.479399in}{0.527163in}}{\pgfqpoint{6.489998in}{0.522772in}}{\pgfqpoint{6.501048in}{0.522772in}}%
\pgfusepath{stroke}%
\end{pgfscope}%
\begin{pgfscope}%
\pgfpathrectangle{\pgfqpoint{0.847223in}{0.554012in}}{\pgfqpoint{6.200000in}{4.530000in}}%
\pgfusepath{clip}%
\pgfsetbuttcap%
\pgfsetroundjoin%
\pgfsetlinewidth{1.003750pt}%
\definecolor{currentstroke}{rgb}{1.000000,0.000000,0.000000}%
\pgfsetstrokecolor{currentstroke}%
\pgfsetdash{}{0pt}%
\pgfpathmoveto{\pgfqpoint{6.509858in}{0.522519in}}%
\pgfpathcurveto{\pgfqpoint{6.520908in}{0.522519in}}{\pgfqpoint{6.531507in}{0.526909in}}{\pgfqpoint{6.539320in}{0.534723in}}%
\pgfpathcurveto{\pgfqpoint{6.547134in}{0.542536in}}{\pgfqpoint{6.551524in}{0.553135in}}{\pgfqpoint{6.551524in}{0.564185in}}%
\pgfpathcurveto{\pgfqpoint{6.551524in}{0.575235in}}{\pgfqpoint{6.547134in}{0.585834in}}{\pgfqpoint{6.539320in}{0.593648in}}%
\pgfpathcurveto{\pgfqpoint{6.531507in}{0.601462in}}{\pgfqpoint{6.520908in}{0.605852in}}{\pgfqpoint{6.509858in}{0.605852in}}%
\pgfpathcurveto{\pgfqpoint{6.498808in}{0.605852in}}{\pgfqpoint{6.488209in}{0.601462in}}{\pgfqpoint{6.480395in}{0.593648in}}%
\pgfpathcurveto{\pgfqpoint{6.472581in}{0.585834in}}{\pgfqpoint{6.468191in}{0.575235in}}{\pgfqpoint{6.468191in}{0.564185in}}%
\pgfpathcurveto{\pgfqpoint{6.468191in}{0.553135in}}{\pgfqpoint{6.472581in}{0.542536in}}{\pgfqpoint{6.480395in}{0.534723in}}%
\pgfpathcurveto{\pgfqpoint{6.488209in}{0.526909in}}{\pgfqpoint{6.498808in}{0.522519in}}{\pgfqpoint{6.509858in}{0.522519in}}%
\pgfusepath{stroke}%
\end{pgfscope}%
\begin{pgfscope}%
\pgfpathrectangle{\pgfqpoint{0.847223in}{0.554012in}}{\pgfqpoint{6.200000in}{4.530000in}}%
\pgfusepath{clip}%
\pgfsetbuttcap%
\pgfsetroundjoin%
\pgfsetlinewidth{1.003750pt}%
\definecolor{currentstroke}{rgb}{1.000000,0.000000,0.000000}%
\pgfsetstrokecolor{currentstroke}%
\pgfsetdash{}{0pt}%
\pgfpathmoveto{\pgfqpoint{6.518667in}{0.522269in}}%
\pgfpathcurveto{\pgfqpoint{6.529717in}{0.522269in}}{\pgfqpoint{6.540316in}{0.526660in}}{\pgfqpoint{6.548130in}{0.534473in}}%
\pgfpathcurveto{\pgfqpoint{6.555943in}{0.542287in}}{\pgfqpoint{6.560334in}{0.552886in}}{\pgfqpoint{6.560334in}{0.563936in}}%
\pgfpathcurveto{\pgfqpoint{6.560334in}{0.574986in}}{\pgfqpoint{6.555943in}{0.585585in}}{\pgfqpoint{6.548130in}{0.593399in}}%
\pgfpathcurveto{\pgfqpoint{6.540316in}{0.601212in}}{\pgfqpoint{6.529717in}{0.605603in}}{\pgfqpoint{6.518667in}{0.605603in}}%
\pgfpathcurveto{\pgfqpoint{6.507617in}{0.605603in}}{\pgfqpoint{6.497018in}{0.601212in}}{\pgfqpoint{6.489204in}{0.593399in}}%
\pgfpathcurveto{\pgfqpoint{6.481391in}{0.585585in}}{\pgfqpoint{6.477000in}{0.574986in}}{\pgfqpoint{6.477000in}{0.563936in}}%
\pgfpathcurveto{\pgfqpoint{6.477000in}{0.552886in}}{\pgfqpoint{6.481391in}{0.542287in}}{\pgfqpoint{6.489204in}{0.534473in}}%
\pgfpathcurveto{\pgfqpoint{6.497018in}{0.526660in}}{\pgfqpoint{6.507617in}{0.522269in}}{\pgfqpoint{6.518667in}{0.522269in}}%
\pgfusepath{stroke}%
\end{pgfscope}%
\begin{pgfscope}%
\pgfpathrectangle{\pgfqpoint{0.847223in}{0.554012in}}{\pgfqpoint{6.200000in}{4.530000in}}%
\pgfusepath{clip}%
\pgfsetbuttcap%
\pgfsetroundjoin%
\pgfsetlinewidth{1.003750pt}%
\definecolor{currentstroke}{rgb}{1.000000,0.000000,0.000000}%
\pgfsetstrokecolor{currentstroke}%
\pgfsetdash{}{0pt}%
\pgfpathmoveto{\pgfqpoint{6.527476in}{0.522024in}}%
\pgfpathcurveto{\pgfqpoint{6.538526in}{0.522024in}}{\pgfqpoint{6.549125in}{0.526415in}}{\pgfqpoint{6.556939in}{0.534228in}}%
\pgfpathcurveto{\pgfqpoint{6.564753in}{0.542042in}}{\pgfqpoint{6.569143in}{0.552641in}}{\pgfqpoint{6.569143in}{0.563691in}}%
\pgfpathcurveto{\pgfqpoint{6.569143in}{0.574741in}}{\pgfqpoint{6.564753in}{0.585340in}}{\pgfqpoint{6.556939in}{0.593154in}}%
\pgfpathcurveto{\pgfqpoint{6.549125in}{0.600967in}}{\pgfqpoint{6.538526in}{0.605358in}}{\pgfqpoint{6.527476in}{0.605358in}}%
\pgfpathcurveto{\pgfqpoint{6.516426in}{0.605358in}}{\pgfqpoint{6.505827in}{0.600967in}}{\pgfqpoint{6.498013in}{0.593154in}}%
\pgfpathcurveto{\pgfqpoint{6.490200in}{0.585340in}}{\pgfqpoint{6.485810in}{0.574741in}}{\pgfqpoint{6.485810in}{0.563691in}}%
\pgfpathcurveto{\pgfqpoint{6.485810in}{0.552641in}}{\pgfqpoint{6.490200in}{0.542042in}}{\pgfqpoint{6.498013in}{0.534228in}}%
\pgfpathcurveto{\pgfqpoint{6.505827in}{0.526415in}}{\pgfqpoint{6.516426in}{0.522024in}}{\pgfqpoint{6.527476in}{0.522024in}}%
\pgfusepath{stroke}%
\end{pgfscope}%
\begin{pgfscope}%
\pgfpathrectangle{\pgfqpoint{0.847223in}{0.554012in}}{\pgfqpoint{6.200000in}{4.530000in}}%
\pgfusepath{clip}%
\pgfsetbuttcap%
\pgfsetroundjoin%
\pgfsetlinewidth{1.003750pt}%
\definecolor{currentstroke}{rgb}{1.000000,0.000000,0.000000}%
\pgfsetstrokecolor{currentstroke}%
\pgfsetdash{}{0pt}%
\pgfpathmoveto{\pgfqpoint{6.536285in}{0.521783in}}%
\pgfpathcurveto{\pgfqpoint{6.547336in}{0.521783in}}{\pgfqpoint{6.557935in}{0.526174in}}{\pgfqpoint{6.565748in}{0.533987in}}%
\pgfpathcurveto{\pgfqpoint{6.573562in}{0.541801in}}{\pgfqpoint{6.577952in}{0.552400in}}{\pgfqpoint{6.577952in}{0.563450in}}%
\pgfpathcurveto{\pgfqpoint{6.577952in}{0.574500in}}{\pgfqpoint{6.573562in}{0.585099in}}{\pgfqpoint{6.565748in}{0.592913in}}%
\pgfpathcurveto{\pgfqpoint{6.557935in}{0.600726in}}{\pgfqpoint{6.547336in}{0.605117in}}{\pgfqpoint{6.536285in}{0.605117in}}%
\pgfpathcurveto{\pgfqpoint{6.525235in}{0.605117in}}{\pgfqpoint{6.514636in}{0.600726in}}{\pgfqpoint{6.506823in}{0.592913in}}%
\pgfpathcurveto{\pgfqpoint{6.499009in}{0.585099in}}{\pgfqpoint{6.494619in}{0.574500in}}{\pgfqpoint{6.494619in}{0.563450in}}%
\pgfpathcurveto{\pgfqpoint{6.494619in}{0.552400in}}{\pgfqpoint{6.499009in}{0.541801in}}{\pgfqpoint{6.506823in}{0.533987in}}%
\pgfpathcurveto{\pgfqpoint{6.514636in}{0.526174in}}{\pgfqpoint{6.525235in}{0.521783in}}{\pgfqpoint{6.536285in}{0.521783in}}%
\pgfusepath{stroke}%
\end{pgfscope}%
\begin{pgfscope}%
\pgfpathrectangle{\pgfqpoint{0.847223in}{0.554012in}}{\pgfqpoint{6.200000in}{4.530000in}}%
\pgfusepath{clip}%
\pgfsetbuttcap%
\pgfsetroundjoin%
\pgfsetlinewidth{1.003750pt}%
\definecolor{currentstroke}{rgb}{1.000000,0.000000,0.000000}%
\pgfsetstrokecolor{currentstroke}%
\pgfsetdash{}{0pt}%
\pgfpathmoveto{\pgfqpoint{6.545095in}{0.521547in}}%
\pgfpathcurveto{\pgfqpoint{6.556145in}{0.521547in}}{\pgfqpoint{6.566744in}{0.525937in}}{\pgfqpoint{6.574558in}{0.533751in}}%
\pgfpathcurveto{\pgfqpoint{6.582371in}{0.541564in}}{\pgfqpoint{6.586761in}{0.552163in}}{\pgfqpoint{6.586761in}{0.563213in}}%
\pgfpathcurveto{\pgfqpoint{6.586761in}{0.574263in}}{\pgfqpoint{6.582371in}{0.584862in}}{\pgfqpoint{6.574558in}{0.592676in}}%
\pgfpathcurveto{\pgfqpoint{6.566744in}{0.600490in}}{\pgfqpoint{6.556145in}{0.604880in}}{\pgfqpoint{6.545095in}{0.604880in}}%
\pgfpathcurveto{\pgfqpoint{6.534045in}{0.604880in}}{\pgfqpoint{6.523446in}{0.600490in}}{\pgfqpoint{6.515632in}{0.592676in}}%
\pgfpathcurveto{\pgfqpoint{6.507818in}{0.584862in}}{\pgfqpoint{6.503428in}{0.574263in}}{\pgfqpoint{6.503428in}{0.563213in}}%
\pgfpathcurveto{\pgfqpoint{6.503428in}{0.552163in}}{\pgfqpoint{6.507818in}{0.541564in}}{\pgfqpoint{6.515632in}{0.533751in}}%
\pgfpathcurveto{\pgfqpoint{6.523446in}{0.525937in}}{\pgfqpoint{6.534045in}{0.521547in}}{\pgfqpoint{6.545095in}{0.521547in}}%
\pgfusepath{stroke}%
\end{pgfscope}%
\begin{pgfscope}%
\pgfpathrectangle{\pgfqpoint{0.847223in}{0.554012in}}{\pgfqpoint{6.200000in}{4.530000in}}%
\pgfusepath{clip}%
\pgfsetbuttcap%
\pgfsetroundjoin%
\pgfsetlinewidth{1.003750pt}%
\definecolor{currentstroke}{rgb}{1.000000,0.000000,0.000000}%
\pgfsetstrokecolor{currentstroke}%
\pgfsetdash{}{0pt}%
\pgfpathmoveto{\pgfqpoint{6.553904in}{0.521314in}}%
\pgfpathcurveto{\pgfqpoint{6.564954in}{0.521314in}}{\pgfqpoint{6.575553in}{0.525704in}}{\pgfqpoint{6.583367in}{0.533518in}}%
\pgfpathcurveto{\pgfqpoint{6.591180in}{0.541331in}}{\pgfqpoint{6.595571in}{0.551930in}}{\pgfqpoint{6.595571in}{0.562980in}}%
\pgfpathcurveto{\pgfqpoint{6.595571in}{0.574031in}}{\pgfqpoint{6.591180in}{0.584630in}}{\pgfqpoint{6.583367in}{0.592443in}}%
\pgfpathcurveto{\pgfqpoint{6.575553in}{0.600257in}}{\pgfqpoint{6.564954in}{0.604647in}}{\pgfqpoint{6.553904in}{0.604647in}}%
\pgfpathcurveto{\pgfqpoint{6.542854in}{0.604647in}}{\pgfqpoint{6.532255in}{0.600257in}}{\pgfqpoint{6.524441in}{0.592443in}}%
\pgfpathcurveto{\pgfqpoint{6.516628in}{0.584630in}}{\pgfqpoint{6.512237in}{0.574031in}}{\pgfqpoint{6.512237in}{0.562980in}}%
\pgfpathcurveto{\pgfqpoint{6.512237in}{0.551930in}}{\pgfqpoint{6.516628in}{0.541331in}}{\pgfqpoint{6.524441in}{0.533518in}}%
\pgfpathcurveto{\pgfqpoint{6.532255in}{0.525704in}}{\pgfqpoint{6.542854in}{0.521314in}}{\pgfqpoint{6.553904in}{0.521314in}}%
\pgfusepath{stroke}%
\end{pgfscope}%
\begin{pgfscope}%
\pgfpathrectangle{\pgfqpoint{0.847223in}{0.554012in}}{\pgfqpoint{6.200000in}{4.530000in}}%
\pgfusepath{clip}%
\pgfsetbuttcap%
\pgfsetroundjoin%
\pgfsetlinewidth{1.003750pt}%
\definecolor{currentstroke}{rgb}{1.000000,0.000000,0.000000}%
\pgfsetstrokecolor{currentstroke}%
\pgfsetdash{}{0pt}%
\pgfpathmoveto{\pgfqpoint{6.562713in}{0.521085in}}%
\pgfpathcurveto{\pgfqpoint{6.573763in}{0.521085in}}{\pgfqpoint{6.584362in}{0.525475in}}{\pgfqpoint{6.592176in}{0.533289in}}%
\pgfpathcurveto{\pgfqpoint{6.599990in}{0.541102in}}{\pgfqpoint{6.604380in}{0.551701in}}{\pgfqpoint{6.604380in}{0.562751in}}%
\pgfpathcurveto{\pgfqpoint{6.604380in}{0.573802in}}{\pgfqpoint{6.599990in}{0.584401in}}{\pgfqpoint{6.592176in}{0.592214in}}%
\pgfpathcurveto{\pgfqpoint{6.584362in}{0.600028in}}{\pgfqpoint{6.573763in}{0.604418in}}{\pgfqpoint{6.562713in}{0.604418in}}%
\pgfpathcurveto{\pgfqpoint{6.551663in}{0.604418in}}{\pgfqpoint{6.541064in}{0.600028in}}{\pgfqpoint{6.533251in}{0.592214in}}%
\pgfpathcurveto{\pgfqpoint{6.525437in}{0.584401in}}{\pgfqpoint{6.521047in}{0.573802in}}{\pgfqpoint{6.521047in}{0.562751in}}%
\pgfpathcurveto{\pgfqpoint{6.521047in}{0.551701in}}{\pgfqpoint{6.525437in}{0.541102in}}{\pgfqpoint{6.533251in}{0.533289in}}%
\pgfpathcurveto{\pgfqpoint{6.541064in}{0.525475in}}{\pgfqpoint{6.551663in}{0.521085in}}{\pgfqpoint{6.562713in}{0.521085in}}%
\pgfusepath{stroke}%
\end{pgfscope}%
\begin{pgfscope}%
\pgfpathrectangle{\pgfqpoint{0.847223in}{0.554012in}}{\pgfqpoint{6.200000in}{4.530000in}}%
\pgfusepath{clip}%
\pgfsetbuttcap%
\pgfsetroundjoin%
\pgfsetlinewidth{1.003750pt}%
\definecolor{currentstroke}{rgb}{1.000000,0.000000,0.000000}%
\pgfsetstrokecolor{currentstroke}%
\pgfsetdash{}{0pt}%
\pgfpathmoveto{\pgfqpoint{6.571523in}{0.520860in}}%
\pgfpathcurveto{\pgfqpoint{6.582573in}{0.520860in}}{\pgfqpoint{6.593172in}{0.525250in}}{\pgfqpoint{6.600985in}{0.533063in}}%
\pgfpathcurveto{\pgfqpoint{6.608799in}{0.540877in}}{\pgfqpoint{6.613189in}{0.551476in}}{\pgfqpoint{6.613189in}{0.562526in}}%
\pgfpathcurveto{\pgfqpoint{6.613189in}{0.573576in}}{\pgfqpoint{6.608799in}{0.584175in}}{\pgfqpoint{6.600985in}{0.591989in}}%
\pgfpathcurveto{\pgfqpoint{6.593172in}{0.599803in}}{\pgfqpoint{6.582573in}{0.604193in}}{\pgfqpoint{6.571523in}{0.604193in}}%
\pgfpathcurveto{\pgfqpoint{6.560472in}{0.604193in}}{\pgfqpoint{6.549873in}{0.599803in}}{\pgfqpoint{6.542060in}{0.591989in}}%
\pgfpathcurveto{\pgfqpoint{6.534246in}{0.584175in}}{\pgfqpoint{6.529856in}{0.573576in}}{\pgfqpoint{6.529856in}{0.562526in}}%
\pgfpathcurveto{\pgfqpoint{6.529856in}{0.551476in}}{\pgfqpoint{6.534246in}{0.540877in}}{\pgfqpoint{6.542060in}{0.533063in}}%
\pgfpathcurveto{\pgfqpoint{6.549873in}{0.525250in}}{\pgfqpoint{6.560472in}{0.520860in}}{\pgfqpoint{6.571523in}{0.520860in}}%
\pgfusepath{stroke}%
\end{pgfscope}%
\begin{pgfscope}%
\pgfpathrectangle{\pgfqpoint{0.847223in}{0.554012in}}{\pgfqpoint{6.200000in}{4.530000in}}%
\pgfusepath{clip}%
\pgfsetbuttcap%
\pgfsetroundjoin%
\pgfsetlinewidth{1.003750pt}%
\definecolor{currentstroke}{rgb}{1.000000,0.000000,0.000000}%
\pgfsetstrokecolor{currentstroke}%
\pgfsetdash{}{0pt}%
\pgfpathmoveto{\pgfqpoint{6.580332in}{0.520638in}}%
\pgfpathcurveto{\pgfqpoint{6.591382in}{0.520638in}}{\pgfqpoint{6.601981in}{0.525028in}}{\pgfqpoint{6.609795in}{0.532842in}}%
\pgfpathcurveto{\pgfqpoint{6.617608in}{0.540655in}}{\pgfqpoint{6.621999in}{0.551254in}}{\pgfqpoint{6.621999in}{0.562305in}}%
\pgfpathcurveto{\pgfqpoint{6.621999in}{0.573355in}}{\pgfqpoint{6.617608in}{0.583954in}}{\pgfqpoint{6.609795in}{0.591767in}}%
\pgfpathcurveto{\pgfqpoint{6.601981in}{0.599581in}}{\pgfqpoint{6.591382in}{0.603971in}}{\pgfqpoint{6.580332in}{0.603971in}}%
\pgfpathcurveto{\pgfqpoint{6.569282in}{0.603971in}}{\pgfqpoint{6.558683in}{0.599581in}}{\pgfqpoint{6.550869in}{0.591767in}}%
\pgfpathcurveto{\pgfqpoint{6.543055in}{0.583954in}}{\pgfqpoint{6.538665in}{0.573355in}}{\pgfqpoint{6.538665in}{0.562305in}}%
\pgfpathcurveto{\pgfqpoint{6.538665in}{0.551254in}}{\pgfqpoint{6.543055in}{0.540655in}}{\pgfqpoint{6.550869in}{0.532842in}}%
\pgfpathcurveto{\pgfqpoint{6.558683in}{0.525028in}}{\pgfqpoint{6.569282in}{0.520638in}}{\pgfqpoint{6.580332in}{0.520638in}}%
\pgfusepath{stroke}%
\end{pgfscope}%
\begin{pgfscope}%
\pgfpathrectangle{\pgfqpoint{0.847223in}{0.554012in}}{\pgfqpoint{6.200000in}{4.530000in}}%
\pgfusepath{clip}%
\pgfsetbuttcap%
\pgfsetroundjoin%
\pgfsetlinewidth{1.003750pt}%
\definecolor{currentstroke}{rgb}{1.000000,0.000000,0.000000}%
\pgfsetstrokecolor{currentstroke}%
\pgfsetdash{}{0pt}%
\pgfpathmoveto{\pgfqpoint{6.589141in}{0.520420in}}%
\pgfpathcurveto{\pgfqpoint{6.600191in}{0.520420in}}{\pgfqpoint{6.610790in}{0.524810in}}{\pgfqpoint{6.618604in}{0.532624in}}%
\pgfpathcurveto{\pgfqpoint{6.626418in}{0.540437in}}{\pgfqpoint{6.630808in}{0.551036in}}{\pgfqpoint{6.630808in}{0.562086in}}%
\pgfpathcurveto{\pgfqpoint{6.630808in}{0.573137in}}{\pgfqpoint{6.626418in}{0.583736in}}{\pgfqpoint{6.618604in}{0.591549in}}%
\pgfpathcurveto{\pgfqpoint{6.610790in}{0.599363in}}{\pgfqpoint{6.600191in}{0.603753in}}{\pgfqpoint{6.589141in}{0.603753in}}%
\pgfpathcurveto{\pgfqpoint{6.578091in}{0.603753in}}{\pgfqpoint{6.567492in}{0.599363in}}{\pgfqpoint{6.559678in}{0.591549in}}%
\pgfpathcurveto{\pgfqpoint{6.551865in}{0.583736in}}{\pgfqpoint{6.547474in}{0.573137in}}{\pgfqpoint{6.547474in}{0.562086in}}%
\pgfpathcurveto{\pgfqpoint{6.547474in}{0.551036in}}{\pgfqpoint{6.551865in}{0.540437in}}{\pgfqpoint{6.559678in}{0.532624in}}%
\pgfpathcurveto{\pgfqpoint{6.567492in}{0.524810in}}{\pgfqpoint{6.578091in}{0.520420in}}{\pgfqpoint{6.589141in}{0.520420in}}%
\pgfusepath{stroke}%
\end{pgfscope}%
\begin{pgfscope}%
\pgfpathrectangle{\pgfqpoint{0.847223in}{0.554012in}}{\pgfqpoint{6.200000in}{4.530000in}}%
\pgfusepath{clip}%
\pgfsetbuttcap%
\pgfsetroundjoin%
\pgfsetlinewidth{1.003750pt}%
\definecolor{currentstroke}{rgb}{1.000000,0.000000,0.000000}%
\pgfsetstrokecolor{currentstroke}%
\pgfsetdash{}{0pt}%
\pgfpathmoveto{\pgfqpoint{6.597950in}{0.520205in}}%
\pgfpathcurveto{\pgfqpoint{6.609001in}{0.520205in}}{\pgfqpoint{6.619600in}{0.524595in}}{\pgfqpoint{6.627413in}{0.532409in}}%
\pgfpathcurveto{\pgfqpoint{6.635227in}{0.540223in}}{\pgfqpoint{6.639617in}{0.550822in}}{\pgfqpoint{6.639617in}{0.561872in}}%
\pgfpathcurveto{\pgfqpoint{6.639617in}{0.572922in}}{\pgfqpoint{6.635227in}{0.583521in}}{\pgfqpoint{6.627413in}{0.591334in}}%
\pgfpathcurveto{\pgfqpoint{6.619600in}{0.599148in}}{\pgfqpoint{6.609001in}{0.603538in}}{\pgfqpoint{6.597950in}{0.603538in}}%
\pgfpathcurveto{\pgfqpoint{6.586900in}{0.603538in}}{\pgfqpoint{6.576301in}{0.599148in}}{\pgfqpoint{6.568488in}{0.591334in}}%
\pgfpathcurveto{\pgfqpoint{6.560674in}{0.583521in}}{\pgfqpoint{6.556284in}{0.572922in}}{\pgfqpoint{6.556284in}{0.561872in}}%
\pgfpathcurveto{\pgfqpoint{6.556284in}{0.550822in}}{\pgfqpoint{6.560674in}{0.540223in}}{\pgfqpoint{6.568488in}{0.532409in}}%
\pgfpathcurveto{\pgfqpoint{6.576301in}{0.524595in}}{\pgfqpoint{6.586900in}{0.520205in}}{\pgfqpoint{6.597950in}{0.520205in}}%
\pgfusepath{stroke}%
\end{pgfscope}%
\begin{pgfscope}%
\pgfpathrectangle{\pgfqpoint{0.847223in}{0.554012in}}{\pgfqpoint{6.200000in}{4.530000in}}%
\pgfusepath{clip}%
\pgfsetbuttcap%
\pgfsetroundjoin%
\pgfsetlinewidth{1.003750pt}%
\definecolor{currentstroke}{rgb}{1.000000,0.000000,0.000000}%
\pgfsetstrokecolor{currentstroke}%
\pgfsetdash{}{0pt}%
\pgfpathmoveto{\pgfqpoint{6.606760in}{0.519994in}}%
\pgfpathcurveto{\pgfqpoint{6.617810in}{0.519994in}}{\pgfqpoint{6.628409in}{0.524384in}}{\pgfqpoint{6.636222in}{0.532198in}}%
\pgfpathcurveto{\pgfqpoint{6.644036in}{0.540011in}}{\pgfqpoint{6.648426in}{0.550610in}}{\pgfqpoint{6.648426in}{0.561660in}}%
\pgfpathcurveto{\pgfqpoint{6.648426in}{0.572710in}}{\pgfqpoint{6.644036in}{0.583309in}}{\pgfqpoint{6.636222in}{0.591123in}}%
\pgfpathcurveto{\pgfqpoint{6.628409in}{0.598937in}}{\pgfqpoint{6.617810in}{0.603327in}}{\pgfqpoint{6.606760in}{0.603327in}}%
\pgfpathcurveto{\pgfqpoint{6.595710in}{0.603327in}}{\pgfqpoint{6.585110in}{0.598937in}}{\pgfqpoint{6.577297in}{0.591123in}}%
\pgfpathcurveto{\pgfqpoint{6.569483in}{0.583309in}}{\pgfqpoint{6.565093in}{0.572710in}}{\pgfqpoint{6.565093in}{0.561660in}}%
\pgfpathcurveto{\pgfqpoint{6.565093in}{0.550610in}}{\pgfqpoint{6.569483in}{0.540011in}}{\pgfqpoint{6.577297in}{0.532198in}}%
\pgfpathcurveto{\pgfqpoint{6.585110in}{0.524384in}}{\pgfqpoint{6.595710in}{0.519994in}}{\pgfqpoint{6.606760in}{0.519994in}}%
\pgfusepath{stroke}%
\end{pgfscope}%
\begin{pgfscope}%
\pgfpathrectangle{\pgfqpoint{0.847223in}{0.554012in}}{\pgfqpoint{6.200000in}{4.530000in}}%
\pgfusepath{clip}%
\pgfsetbuttcap%
\pgfsetroundjoin%
\pgfsetlinewidth{1.003750pt}%
\definecolor{currentstroke}{rgb}{1.000000,0.000000,0.000000}%
\pgfsetstrokecolor{currentstroke}%
\pgfsetdash{}{0pt}%
\pgfpathmoveto{\pgfqpoint{6.615569in}{0.519785in}}%
\pgfpathcurveto{\pgfqpoint{6.626619in}{0.519785in}}{\pgfqpoint{6.637218in}{0.524176in}}{\pgfqpoint{6.645032in}{0.531989in}}%
\pgfpathcurveto{\pgfqpoint{6.652845in}{0.539803in}}{\pgfqpoint{6.657236in}{0.550402in}}{\pgfqpoint{6.657236in}{0.561452in}}%
\pgfpathcurveto{\pgfqpoint{6.657236in}{0.572502in}}{\pgfqpoint{6.652845in}{0.583101in}}{\pgfqpoint{6.645032in}{0.590915in}}%
\pgfpathcurveto{\pgfqpoint{6.637218in}{0.598729in}}{\pgfqpoint{6.626619in}{0.603119in}}{\pgfqpoint{6.615569in}{0.603119in}}%
\pgfpathcurveto{\pgfqpoint{6.604519in}{0.603119in}}{\pgfqpoint{6.593920in}{0.598729in}}{\pgfqpoint{6.586106in}{0.590915in}}%
\pgfpathcurveto{\pgfqpoint{6.578293in}{0.583101in}}{\pgfqpoint{6.573902in}{0.572502in}}{\pgfqpoint{6.573902in}{0.561452in}}%
\pgfpathcurveto{\pgfqpoint{6.573902in}{0.550402in}}{\pgfqpoint{6.578293in}{0.539803in}}{\pgfqpoint{6.586106in}{0.531989in}}%
\pgfpathcurveto{\pgfqpoint{6.593920in}{0.524176in}}{\pgfqpoint{6.604519in}{0.519785in}}{\pgfqpoint{6.615569in}{0.519785in}}%
\pgfusepath{stroke}%
\end{pgfscope}%
\begin{pgfscope}%
\pgfpathrectangle{\pgfqpoint{0.847223in}{0.554012in}}{\pgfqpoint{6.200000in}{4.530000in}}%
\pgfusepath{clip}%
\pgfsetbuttcap%
\pgfsetroundjoin%
\pgfsetlinewidth{1.003750pt}%
\definecolor{currentstroke}{rgb}{1.000000,0.000000,0.000000}%
\pgfsetstrokecolor{currentstroke}%
\pgfsetdash{}{0pt}%
\pgfpathmoveto{\pgfqpoint{6.624378in}{0.519581in}}%
\pgfpathcurveto{\pgfqpoint{6.635428in}{0.519581in}}{\pgfqpoint{6.646027in}{0.523971in}}{\pgfqpoint{6.653841in}{0.531784in}}%
\pgfpathcurveto{\pgfqpoint{6.661655in}{0.539598in}}{\pgfqpoint{6.666045in}{0.550197in}}{\pgfqpoint{6.666045in}{0.561247in}}%
\pgfpathcurveto{\pgfqpoint{6.666045in}{0.572297in}}{\pgfqpoint{6.661655in}{0.582896in}}{\pgfqpoint{6.653841in}{0.590710in}}%
\pgfpathcurveto{\pgfqpoint{6.646027in}{0.598524in}}{\pgfqpoint{6.635428in}{0.602914in}}{\pgfqpoint{6.624378in}{0.602914in}}%
\pgfpathcurveto{\pgfqpoint{6.613328in}{0.602914in}}{\pgfqpoint{6.602729in}{0.598524in}}{\pgfqpoint{6.594915in}{0.590710in}}%
\pgfpathcurveto{\pgfqpoint{6.587102in}{0.582896in}}{\pgfqpoint{6.582712in}{0.572297in}}{\pgfqpoint{6.582712in}{0.561247in}}%
\pgfpathcurveto{\pgfqpoint{6.582712in}{0.550197in}}{\pgfqpoint{6.587102in}{0.539598in}}{\pgfqpoint{6.594915in}{0.531784in}}%
\pgfpathcurveto{\pgfqpoint{6.602729in}{0.523971in}}{\pgfqpoint{6.613328in}{0.519581in}}{\pgfqpoint{6.624378in}{0.519581in}}%
\pgfusepath{stroke}%
\end{pgfscope}%
\begin{pgfscope}%
\pgfpathrectangle{\pgfqpoint{0.847223in}{0.554012in}}{\pgfqpoint{6.200000in}{4.530000in}}%
\pgfusepath{clip}%
\pgfsetbuttcap%
\pgfsetroundjoin%
\pgfsetlinewidth{1.003750pt}%
\definecolor{currentstroke}{rgb}{1.000000,0.000000,0.000000}%
\pgfsetstrokecolor{currentstroke}%
\pgfsetdash{}{0pt}%
\pgfpathmoveto{\pgfqpoint{6.633187in}{0.519379in}}%
\pgfpathcurveto{\pgfqpoint{6.644238in}{0.519379in}}{\pgfqpoint{6.654837in}{0.523769in}}{\pgfqpoint{6.662650in}{0.531583in}}%
\pgfpathcurveto{\pgfqpoint{6.670464in}{0.539396in}}{\pgfqpoint{6.674854in}{0.549995in}}{\pgfqpoint{6.674854in}{0.561045in}}%
\pgfpathcurveto{\pgfqpoint{6.674854in}{0.572095in}}{\pgfqpoint{6.670464in}{0.582694in}}{\pgfqpoint{6.662650in}{0.590508in}}%
\pgfpathcurveto{\pgfqpoint{6.654837in}{0.598322in}}{\pgfqpoint{6.644238in}{0.602712in}}{\pgfqpoint{6.633187in}{0.602712in}}%
\pgfpathcurveto{\pgfqpoint{6.622137in}{0.602712in}}{\pgfqpoint{6.611538in}{0.598322in}}{\pgfqpoint{6.603725in}{0.590508in}}%
\pgfpathcurveto{\pgfqpoint{6.595911in}{0.582694in}}{\pgfqpoint{6.591521in}{0.572095in}}{\pgfqpoint{6.591521in}{0.561045in}}%
\pgfpathcurveto{\pgfqpoint{6.591521in}{0.549995in}}{\pgfqpoint{6.595911in}{0.539396in}}{\pgfqpoint{6.603725in}{0.531583in}}%
\pgfpathcurveto{\pgfqpoint{6.611538in}{0.523769in}}{\pgfqpoint{6.622137in}{0.519379in}}{\pgfqpoint{6.633187in}{0.519379in}}%
\pgfusepath{stroke}%
\end{pgfscope}%
\begin{pgfscope}%
\pgfpathrectangle{\pgfqpoint{0.847223in}{0.554012in}}{\pgfqpoint{6.200000in}{4.530000in}}%
\pgfusepath{clip}%
\pgfsetbuttcap%
\pgfsetroundjoin%
\pgfsetlinewidth{1.003750pt}%
\definecolor{currentstroke}{rgb}{1.000000,0.000000,0.000000}%
\pgfsetstrokecolor{currentstroke}%
\pgfsetdash{}{0pt}%
\pgfpathmoveto{\pgfqpoint{6.641997in}{0.519180in}}%
\pgfpathcurveto{\pgfqpoint{6.653047in}{0.519180in}}{\pgfqpoint{6.663646in}{0.523570in}}{\pgfqpoint{6.671459in}{0.531384in}}%
\pgfpathcurveto{\pgfqpoint{6.679273in}{0.539197in}}{\pgfqpoint{6.683663in}{0.549796in}}{\pgfqpoint{6.683663in}{0.560846in}}%
\pgfpathcurveto{\pgfqpoint{6.683663in}{0.571897in}}{\pgfqpoint{6.679273in}{0.582496in}}{\pgfqpoint{6.671459in}{0.590309in}}%
\pgfpathcurveto{\pgfqpoint{6.663646in}{0.598123in}}{\pgfqpoint{6.653047in}{0.602513in}}{\pgfqpoint{6.641997in}{0.602513in}}%
\pgfpathcurveto{\pgfqpoint{6.630947in}{0.602513in}}{\pgfqpoint{6.620348in}{0.598123in}}{\pgfqpoint{6.612534in}{0.590309in}}%
\pgfpathcurveto{\pgfqpoint{6.604720in}{0.582496in}}{\pgfqpoint{6.600330in}{0.571897in}}{\pgfqpoint{6.600330in}{0.560846in}}%
\pgfpathcurveto{\pgfqpoint{6.600330in}{0.549796in}}{\pgfqpoint{6.604720in}{0.539197in}}{\pgfqpoint{6.612534in}{0.531384in}}%
\pgfpathcurveto{\pgfqpoint{6.620348in}{0.523570in}}{\pgfqpoint{6.630947in}{0.519180in}}{\pgfqpoint{6.641997in}{0.519180in}}%
\pgfusepath{stroke}%
\end{pgfscope}%
\begin{pgfscope}%
\pgfpathrectangle{\pgfqpoint{0.847223in}{0.554012in}}{\pgfqpoint{6.200000in}{4.530000in}}%
\pgfusepath{clip}%
\pgfsetbuttcap%
\pgfsetroundjoin%
\pgfsetlinewidth{1.003750pt}%
\definecolor{currentstroke}{rgb}{1.000000,0.000000,0.000000}%
\pgfsetstrokecolor{currentstroke}%
\pgfsetdash{}{0pt}%
\pgfpathmoveto{\pgfqpoint{6.650806in}{0.518984in}}%
\pgfpathcurveto{\pgfqpoint{6.661856in}{0.518984in}}{\pgfqpoint{6.672455in}{0.523374in}}{\pgfqpoint{6.680269in}{0.531188in}}%
\pgfpathcurveto{\pgfqpoint{6.688082in}{0.539001in}}{\pgfqpoint{6.692473in}{0.549600in}}{\pgfqpoint{6.692473in}{0.560650in}}%
\pgfpathcurveto{\pgfqpoint{6.692473in}{0.571701in}}{\pgfqpoint{6.688082in}{0.582300in}}{\pgfqpoint{6.680269in}{0.590113in}}%
\pgfpathcurveto{\pgfqpoint{6.672455in}{0.597927in}}{\pgfqpoint{6.661856in}{0.602317in}}{\pgfqpoint{6.650806in}{0.602317in}}%
\pgfpathcurveto{\pgfqpoint{6.639756in}{0.602317in}}{\pgfqpoint{6.629157in}{0.597927in}}{\pgfqpoint{6.621343in}{0.590113in}}%
\pgfpathcurveto{\pgfqpoint{6.613530in}{0.582300in}}{\pgfqpoint{6.609139in}{0.571701in}}{\pgfqpoint{6.609139in}{0.560650in}}%
\pgfpathcurveto{\pgfqpoint{6.609139in}{0.549600in}}{\pgfqpoint{6.613530in}{0.539001in}}{\pgfqpoint{6.621343in}{0.531188in}}%
\pgfpathcurveto{\pgfqpoint{6.629157in}{0.523374in}}{\pgfqpoint{6.639756in}{0.518984in}}{\pgfqpoint{6.650806in}{0.518984in}}%
\pgfusepath{stroke}%
\end{pgfscope}%
\begin{pgfscope}%
\pgfpathrectangle{\pgfqpoint{0.847223in}{0.554012in}}{\pgfqpoint{6.200000in}{4.530000in}}%
\pgfusepath{clip}%
\pgfsetbuttcap%
\pgfsetroundjoin%
\pgfsetlinewidth{1.003750pt}%
\definecolor{currentstroke}{rgb}{1.000000,0.000000,0.000000}%
\pgfsetstrokecolor{currentstroke}%
\pgfsetdash{}{0pt}%
\pgfpathmoveto{\pgfqpoint{6.659615in}{0.518791in}}%
\pgfpathcurveto{\pgfqpoint{6.670665in}{0.518791in}}{\pgfqpoint{6.681264in}{0.523181in}}{\pgfqpoint{6.689078in}{0.530995in}}%
\pgfpathcurveto{\pgfqpoint{6.696892in}{0.538808in}}{\pgfqpoint{6.701282in}{0.549407in}}{\pgfqpoint{6.701282in}{0.560457in}}%
\pgfpathcurveto{\pgfqpoint{6.701282in}{0.571507in}}{\pgfqpoint{6.696892in}{0.582106in}}{\pgfqpoint{6.689078in}{0.589920in}}%
\pgfpathcurveto{\pgfqpoint{6.681264in}{0.597734in}}{\pgfqpoint{6.670665in}{0.602124in}}{\pgfqpoint{6.659615in}{0.602124in}}%
\pgfpathcurveto{\pgfqpoint{6.648565in}{0.602124in}}{\pgfqpoint{6.637966in}{0.597734in}}{\pgfqpoint{6.630152in}{0.589920in}}%
\pgfpathcurveto{\pgfqpoint{6.622339in}{0.582106in}}{\pgfqpoint{6.617949in}{0.571507in}}{\pgfqpoint{6.617949in}{0.560457in}}%
\pgfpathcurveto{\pgfqpoint{6.617949in}{0.549407in}}{\pgfqpoint{6.622339in}{0.538808in}}{\pgfqpoint{6.630152in}{0.530995in}}%
\pgfpathcurveto{\pgfqpoint{6.637966in}{0.523181in}}{\pgfqpoint{6.648565in}{0.518791in}}{\pgfqpoint{6.659615in}{0.518791in}}%
\pgfusepath{stroke}%
\end{pgfscope}%
\begin{pgfscope}%
\pgfpathrectangle{\pgfqpoint{0.847223in}{0.554012in}}{\pgfqpoint{6.200000in}{4.530000in}}%
\pgfusepath{clip}%
\pgfsetbuttcap%
\pgfsetroundjoin%
\pgfsetlinewidth{1.003750pt}%
\definecolor{currentstroke}{rgb}{1.000000,0.000000,0.000000}%
\pgfsetstrokecolor{currentstroke}%
\pgfsetdash{}{0pt}%
\pgfpathmoveto{\pgfqpoint{6.668425in}{0.518600in}}%
\pgfpathcurveto{\pgfqpoint{6.679475in}{0.518600in}}{\pgfqpoint{6.690074in}{0.522991in}}{\pgfqpoint{6.697887in}{0.530804in}}%
\pgfpathcurveto{\pgfqpoint{6.705701in}{0.538618in}}{\pgfqpoint{6.710091in}{0.549217in}}{\pgfqpoint{6.710091in}{0.560267in}}%
\pgfpathcurveto{\pgfqpoint{6.710091in}{0.571317in}}{\pgfqpoint{6.705701in}{0.581916in}}{\pgfqpoint{6.697887in}{0.589730in}}%
\pgfpathcurveto{\pgfqpoint{6.690074in}{0.597543in}}{\pgfqpoint{6.679475in}{0.601934in}}{\pgfqpoint{6.668425in}{0.601934in}}%
\pgfpathcurveto{\pgfqpoint{6.657374in}{0.601934in}}{\pgfqpoint{6.646775in}{0.597543in}}{\pgfqpoint{6.638962in}{0.589730in}}%
\pgfpathcurveto{\pgfqpoint{6.631148in}{0.581916in}}{\pgfqpoint{6.626758in}{0.571317in}}{\pgfqpoint{6.626758in}{0.560267in}}%
\pgfpathcurveto{\pgfqpoint{6.626758in}{0.549217in}}{\pgfqpoint{6.631148in}{0.538618in}}{\pgfqpoint{6.638962in}{0.530804in}}%
\pgfpathcurveto{\pgfqpoint{6.646775in}{0.522991in}}{\pgfqpoint{6.657374in}{0.518600in}}{\pgfqpoint{6.668425in}{0.518600in}}%
\pgfusepath{stroke}%
\end{pgfscope}%
\begin{pgfscope}%
\pgfpathrectangle{\pgfqpoint{0.847223in}{0.554012in}}{\pgfqpoint{6.200000in}{4.530000in}}%
\pgfusepath{clip}%
\pgfsetbuttcap%
\pgfsetroundjoin%
\pgfsetlinewidth{1.003750pt}%
\definecolor{currentstroke}{rgb}{1.000000,0.000000,0.000000}%
\pgfsetstrokecolor{currentstroke}%
\pgfsetdash{}{0pt}%
\pgfpathmoveto{\pgfqpoint{6.677234in}{0.518413in}}%
\pgfpathcurveto{\pgfqpoint{6.688284in}{0.518413in}}{\pgfqpoint{6.698883in}{0.522803in}}{\pgfqpoint{6.706697in}{0.530617in}}%
\pgfpathcurveto{\pgfqpoint{6.714510in}{0.538430in}}{\pgfqpoint{6.718900in}{0.549029in}}{\pgfqpoint{6.718900in}{0.560079in}}%
\pgfpathcurveto{\pgfqpoint{6.718900in}{0.571129in}}{\pgfqpoint{6.714510in}{0.581728in}}{\pgfqpoint{6.706697in}{0.589542in}}%
\pgfpathcurveto{\pgfqpoint{6.698883in}{0.597356in}}{\pgfqpoint{6.688284in}{0.601746in}}{\pgfqpoint{6.677234in}{0.601746in}}%
\pgfpathcurveto{\pgfqpoint{6.666184in}{0.601746in}}{\pgfqpoint{6.655585in}{0.597356in}}{\pgfqpoint{6.647771in}{0.589542in}}%
\pgfpathcurveto{\pgfqpoint{6.639957in}{0.581728in}}{\pgfqpoint{6.635567in}{0.571129in}}{\pgfqpoint{6.635567in}{0.560079in}}%
\pgfpathcurveto{\pgfqpoint{6.635567in}{0.549029in}}{\pgfqpoint{6.639957in}{0.538430in}}{\pgfqpoint{6.647771in}{0.530617in}}%
\pgfpathcurveto{\pgfqpoint{6.655585in}{0.522803in}}{\pgfqpoint{6.666184in}{0.518413in}}{\pgfqpoint{6.677234in}{0.518413in}}%
\pgfusepath{stroke}%
\end{pgfscope}%
\begin{pgfscope}%
\pgfpathrectangle{\pgfqpoint{0.847223in}{0.554012in}}{\pgfqpoint{6.200000in}{4.530000in}}%
\pgfusepath{clip}%
\pgfsetbuttcap%
\pgfsetroundjoin%
\pgfsetlinewidth{1.003750pt}%
\definecolor{currentstroke}{rgb}{1.000000,0.000000,0.000000}%
\pgfsetstrokecolor{currentstroke}%
\pgfsetdash{}{0pt}%
\pgfpathmoveto{\pgfqpoint{6.686043in}{0.518228in}}%
\pgfpathcurveto{\pgfqpoint{6.697093in}{0.518228in}}{\pgfqpoint{6.707692in}{0.522618in}}{\pgfqpoint{6.715506in}{0.530432in}}%
\pgfpathcurveto{\pgfqpoint{6.723319in}{0.538245in}}{\pgfqpoint{6.727710in}{0.548844in}}{\pgfqpoint{6.727710in}{0.559894in}}%
\pgfpathcurveto{\pgfqpoint{6.727710in}{0.570944in}}{\pgfqpoint{6.723319in}{0.581544in}}{\pgfqpoint{6.715506in}{0.589357in}}%
\pgfpathcurveto{\pgfqpoint{6.707692in}{0.597171in}}{\pgfqpoint{6.697093in}{0.601561in}}{\pgfqpoint{6.686043in}{0.601561in}}%
\pgfpathcurveto{\pgfqpoint{6.674993in}{0.601561in}}{\pgfqpoint{6.664394in}{0.597171in}}{\pgfqpoint{6.656580in}{0.589357in}}%
\pgfpathcurveto{\pgfqpoint{6.648767in}{0.581544in}}{\pgfqpoint{6.644376in}{0.570944in}}{\pgfqpoint{6.644376in}{0.559894in}}%
\pgfpathcurveto{\pgfqpoint{6.644376in}{0.548844in}}{\pgfqpoint{6.648767in}{0.538245in}}{\pgfqpoint{6.656580in}{0.530432in}}%
\pgfpathcurveto{\pgfqpoint{6.664394in}{0.522618in}}{\pgfqpoint{6.674993in}{0.518228in}}{\pgfqpoint{6.686043in}{0.518228in}}%
\pgfusepath{stroke}%
\end{pgfscope}%
\begin{pgfscope}%
\pgfpathrectangle{\pgfqpoint{0.847223in}{0.554012in}}{\pgfqpoint{6.200000in}{4.530000in}}%
\pgfusepath{clip}%
\pgfsetbuttcap%
\pgfsetroundjoin%
\pgfsetlinewidth{1.003750pt}%
\definecolor{currentstroke}{rgb}{1.000000,0.000000,0.000000}%
\pgfsetstrokecolor{currentstroke}%
\pgfsetdash{}{0pt}%
\pgfpathmoveto{\pgfqpoint{6.694852in}{0.518045in}}%
\pgfpathcurveto{\pgfqpoint{6.705902in}{0.518045in}}{\pgfqpoint{6.716501in}{0.522436in}}{\pgfqpoint{6.724315in}{0.530249in}}%
\pgfpathcurveto{\pgfqpoint{6.732129in}{0.538063in}}{\pgfqpoint{6.736519in}{0.548662in}}{\pgfqpoint{6.736519in}{0.559712in}}%
\pgfpathcurveto{\pgfqpoint{6.736519in}{0.570762in}}{\pgfqpoint{6.732129in}{0.581361in}}{\pgfqpoint{6.724315in}{0.589175in}}%
\pgfpathcurveto{\pgfqpoint{6.716501in}{0.596988in}}{\pgfqpoint{6.705902in}{0.601379in}}{\pgfqpoint{6.694852in}{0.601379in}}%
\pgfpathcurveto{\pgfqpoint{6.683802in}{0.601379in}}{\pgfqpoint{6.673203in}{0.596988in}}{\pgfqpoint{6.665390in}{0.589175in}}%
\pgfpathcurveto{\pgfqpoint{6.657576in}{0.581361in}}{\pgfqpoint{6.653186in}{0.570762in}}{\pgfqpoint{6.653186in}{0.559712in}}%
\pgfpathcurveto{\pgfqpoint{6.653186in}{0.548662in}}{\pgfqpoint{6.657576in}{0.538063in}}{\pgfqpoint{6.665390in}{0.530249in}}%
\pgfpathcurveto{\pgfqpoint{6.673203in}{0.522436in}}{\pgfqpoint{6.683802in}{0.518045in}}{\pgfqpoint{6.694852in}{0.518045in}}%
\pgfusepath{stroke}%
\end{pgfscope}%
\begin{pgfscope}%
\pgfpathrectangle{\pgfqpoint{0.847223in}{0.554012in}}{\pgfqpoint{6.200000in}{4.530000in}}%
\pgfusepath{clip}%
\pgfsetbuttcap%
\pgfsetroundjoin%
\pgfsetlinewidth{1.003750pt}%
\definecolor{currentstroke}{rgb}{1.000000,0.000000,0.000000}%
\pgfsetstrokecolor{currentstroke}%
\pgfsetdash{}{0pt}%
\pgfpathmoveto{\pgfqpoint{6.703662in}{0.517865in}}%
\pgfpathcurveto{\pgfqpoint{6.714712in}{0.517865in}}{\pgfqpoint{6.725311in}{0.522256in}}{\pgfqpoint{6.733124in}{0.530069in}}%
\pgfpathcurveto{\pgfqpoint{6.740938in}{0.537883in}}{\pgfqpoint{6.745328in}{0.548482in}}{\pgfqpoint{6.745328in}{0.559532in}}%
\pgfpathcurveto{\pgfqpoint{6.745328in}{0.570582in}}{\pgfqpoint{6.740938in}{0.581181in}}{\pgfqpoint{6.733124in}{0.588995in}}%
\pgfpathcurveto{\pgfqpoint{6.725311in}{0.596809in}}{\pgfqpoint{6.714712in}{0.601199in}}{\pgfqpoint{6.703662in}{0.601199in}}%
\pgfpathcurveto{\pgfqpoint{6.692611in}{0.601199in}}{\pgfqpoint{6.682012in}{0.596809in}}{\pgfqpoint{6.674199in}{0.588995in}}%
\pgfpathcurveto{\pgfqpoint{6.666385in}{0.581181in}}{\pgfqpoint{6.661995in}{0.570582in}}{\pgfqpoint{6.661995in}{0.559532in}}%
\pgfpathcurveto{\pgfqpoint{6.661995in}{0.548482in}}{\pgfqpoint{6.666385in}{0.537883in}}{\pgfqpoint{6.674199in}{0.530069in}}%
\pgfpathcurveto{\pgfqpoint{6.682012in}{0.522256in}}{\pgfqpoint{6.692611in}{0.517865in}}{\pgfqpoint{6.703662in}{0.517865in}}%
\pgfusepath{stroke}%
\end{pgfscope}%
\begin{pgfscope}%
\pgfpathrectangle{\pgfqpoint{0.847223in}{0.554012in}}{\pgfqpoint{6.200000in}{4.530000in}}%
\pgfusepath{clip}%
\pgfsetbuttcap%
\pgfsetroundjoin%
\pgfsetlinewidth{1.003750pt}%
\definecolor{currentstroke}{rgb}{1.000000,0.000000,0.000000}%
\pgfsetstrokecolor{currentstroke}%
\pgfsetdash{}{0pt}%
\pgfpathmoveto{\pgfqpoint{6.712471in}{0.517688in}}%
\pgfpathcurveto{\pgfqpoint{6.723521in}{0.517688in}}{\pgfqpoint{6.734120in}{0.522078in}}{\pgfqpoint{6.741934in}{0.529892in}}%
\pgfpathcurveto{\pgfqpoint{6.749747in}{0.537706in}}{\pgfqpoint{6.754138in}{0.548305in}}{\pgfqpoint{6.754138in}{0.559355in}}%
\pgfpathcurveto{\pgfqpoint{6.754138in}{0.570405in}}{\pgfqpoint{6.749747in}{0.581004in}}{\pgfqpoint{6.741934in}{0.588817in}}%
\pgfpathcurveto{\pgfqpoint{6.734120in}{0.596631in}}{\pgfqpoint{6.723521in}{0.601021in}}{\pgfqpoint{6.712471in}{0.601021in}}%
\pgfpathcurveto{\pgfqpoint{6.701421in}{0.601021in}}{\pgfqpoint{6.690822in}{0.596631in}}{\pgfqpoint{6.683008in}{0.588817in}}%
\pgfpathcurveto{\pgfqpoint{6.675194in}{0.581004in}}{\pgfqpoint{6.670804in}{0.570405in}}{\pgfqpoint{6.670804in}{0.559355in}}%
\pgfpathcurveto{\pgfqpoint{6.670804in}{0.548305in}}{\pgfqpoint{6.675194in}{0.537706in}}{\pgfqpoint{6.683008in}{0.529892in}}%
\pgfpathcurveto{\pgfqpoint{6.690822in}{0.522078in}}{\pgfqpoint{6.701421in}{0.517688in}}{\pgfqpoint{6.712471in}{0.517688in}}%
\pgfusepath{stroke}%
\end{pgfscope}%
\begin{pgfscope}%
\pgfpathrectangle{\pgfqpoint{0.847223in}{0.554012in}}{\pgfqpoint{6.200000in}{4.530000in}}%
\pgfusepath{clip}%
\pgfsetbuttcap%
\pgfsetroundjoin%
\pgfsetlinewidth{1.003750pt}%
\definecolor{currentstroke}{rgb}{1.000000,0.000000,0.000000}%
\pgfsetstrokecolor{currentstroke}%
\pgfsetdash{}{0pt}%
\pgfpathmoveto{\pgfqpoint{6.721280in}{0.517513in}}%
\pgfpathcurveto{\pgfqpoint{6.732330in}{0.517513in}}{\pgfqpoint{6.742929in}{0.521903in}}{\pgfqpoint{6.750743in}{0.529717in}}%
\pgfpathcurveto{\pgfqpoint{6.758557in}{0.537531in}}{\pgfqpoint{6.762947in}{0.548130in}}{\pgfqpoint{6.762947in}{0.559180in}}%
\pgfpathcurveto{\pgfqpoint{6.762947in}{0.570230in}}{\pgfqpoint{6.758557in}{0.580829in}}{\pgfqpoint{6.750743in}{0.588642in}}%
\pgfpathcurveto{\pgfqpoint{6.742929in}{0.596456in}}{\pgfqpoint{6.732330in}{0.600846in}}{\pgfqpoint{6.721280in}{0.600846in}}%
\pgfpathcurveto{\pgfqpoint{6.710230in}{0.600846in}}{\pgfqpoint{6.699631in}{0.596456in}}{\pgfqpoint{6.691817in}{0.588642in}}%
\pgfpathcurveto{\pgfqpoint{6.684004in}{0.580829in}}{\pgfqpoint{6.679613in}{0.570230in}}{\pgfqpoint{6.679613in}{0.559180in}}%
\pgfpathcurveto{\pgfqpoint{6.679613in}{0.548130in}}{\pgfqpoint{6.684004in}{0.537531in}}{\pgfqpoint{6.691817in}{0.529717in}}%
\pgfpathcurveto{\pgfqpoint{6.699631in}{0.521903in}}{\pgfqpoint{6.710230in}{0.517513in}}{\pgfqpoint{6.721280in}{0.517513in}}%
\pgfusepath{stroke}%
\end{pgfscope}%
\begin{pgfscope}%
\pgfpathrectangle{\pgfqpoint{0.847223in}{0.554012in}}{\pgfqpoint{6.200000in}{4.530000in}}%
\pgfusepath{clip}%
\pgfsetbuttcap%
\pgfsetroundjoin%
\pgfsetlinewidth{1.003750pt}%
\definecolor{currentstroke}{rgb}{1.000000,0.000000,0.000000}%
\pgfsetstrokecolor{currentstroke}%
\pgfsetdash{}{0pt}%
\pgfpathmoveto{\pgfqpoint{6.730089in}{0.517340in}}%
\pgfpathcurveto{\pgfqpoint{6.741140in}{0.517340in}}{\pgfqpoint{6.751739in}{0.521731in}}{\pgfqpoint{6.759552in}{0.529544in}}%
\pgfpathcurveto{\pgfqpoint{6.767366in}{0.537358in}}{\pgfqpoint{6.771756in}{0.547957in}}{\pgfqpoint{6.771756in}{0.559007in}}%
\pgfpathcurveto{\pgfqpoint{6.771756in}{0.570057in}}{\pgfqpoint{6.767366in}{0.580656in}}{\pgfqpoint{6.759552in}{0.588470in}}%
\pgfpathcurveto{\pgfqpoint{6.751739in}{0.596283in}}{\pgfqpoint{6.741140in}{0.600674in}}{\pgfqpoint{6.730089in}{0.600674in}}%
\pgfpathcurveto{\pgfqpoint{6.719039in}{0.600674in}}{\pgfqpoint{6.708440in}{0.596283in}}{\pgfqpoint{6.700627in}{0.588470in}}%
\pgfpathcurveto{\pgfqpoint{6.692813in}{0.580656in}}{\pgfqpoint{6.688423in}{0.570057in}}{\pgfqpoint{6.688423in}{0.559007in}}%
\pgfpathcurveto{\pgfqpoint{6.688423in}{0.547957in}}{\pgfqpoint{6.692813in}{0.537358in}}{\pgfqpoint{6.700627in}{0.529544in}}%
\pgfpathcurveto{\pgfqpoint{6.708440in}{0.521731in}}{\pgfqpoint{6.719039in}{0.517340in}}{\pgfqpoint{6.730089in}{0.517340in}}%
\pgfusepath{stroke}%
\end{pgfscope}%
\begin{pgfscope}%
\pgfpathrectangle{\pgfqpoint{0.847223in}{0.554012in}}{\pgfqpoint{6.200000in}{4.530000in}}%
\pgfusepath{clip}%
\pgfsetbuttcap%
\pgfsetroundjoin%
\pgfsetlinewidth{1.003750pt}%
\definecolor{currentstroke}{rgb}{1.000000,0.000000,0.000000}%
\pgfsetstrokecolor{currentstroke}%
\pgfsetdash{}{0pt}%
\pgfpathmoveto{\pgfqpoint{6.738899in}{0.517170in}}%
\pgfpathcurveto{\pgfqpoint{6.749949in}{0.517170in}}{\pgfqpoint{6.760548in}{0.521560in}}{\pgfqpoint{6.768361in}{0.529374in}}%
\pgfpathcurveto{\pgfqpoint{6.776175in}{0.537188in}}{\pgfqpoint{6.780565in}{0.547787in}}{\pgfqpoint{6.780565in}{0.558837in}}%
\pgfpathcurveto{\pgfqpoint{6.780565in}{0.569887in}}{\pgfqpoint{6.776175in}{0.580486in}}{\pgfqpoint{6.768361in}{0.588300in}}%
\pgfpathcurveto{\pgfqpoint{6.760548in}{0.596113in}}{\pgfqpoint{6.749949in}{0.600503in}}{\pgfqpoint{6.738899in}{0.600503in}}%
\pgfpathcurveto{\pgfqpoint{6.727849in}{0.600503in}}{\pgfqpoint{6.717250in}{0.596113in}}{\pgfqpoint{6.709436in}{0.588300in}}%
\pgfpathcurveto{\pgfqpoint{6.701622in}{0.580486in}}{\pgfqpoint{6.697232in}{0.569887in}}{\pgfqpoint{6.697232in}{0.558837in}}%
\pgfpathcurveto{\pgfqpoint{6.697232in}{0.547787in}}{\pgfqpoint{6.701622in}{0.537188in}}{\pgfqpoint{6.709436in}{0.529374in}}%
\pgfpathcurveto{\pgfqpoint{6.717250in}{0.521560in}}{\pgfqpoint{6.727849in}{0.517170in}}{\pgfqpoint{6.738899in}{0.517170in}}%
\pgfusepath{stroke}%
\end{pgfscope}%
\begin{pgfscope}%
\pgfpathrectangle{\pgfqpoint{0.847223in}{0.554012in}}{\pgfqpoint{6.200000in}{4.530000in}}%
\pgfusepath{clip}%
\pgfsetbuttcap%
\pgfsetroundjoin%
\pgfsetlinewidth{1.003750pt}%
\definecolor{currentstroke}{rgb}{1.000000,0.000000,0.000000}%
\pgfsetstrokecolor{currentstroke}%
\pgfsetdash{}{0pt}%
\pgfpathmoveto{\pgfqpoint{6.747708in}{0.517002in}}%
\pgfpathcurveto{\pgfqpoint{6.758758in}{0.517002in}}{\pgfqpoint{6.769357in}{0.521392in}}{\pgfqpoint{6.777171in}{0.529206in}}%
\pgfpathcurveto{\pgfqpoint{6.784984in}{0.537019in}}{\pgfqpoint{6.789375in}{0.547619in}}{\pgfqpoint{6.789375in}{0.558669in}}%
\pgfpathcurveto{\pgfqpoint{6.789375in}{0.569719in}}{\pgfqpoint{6.784984in}{0.580318in}}{\pgfqpoint{6.777171in}{0.588131in}}%
\pgfpathcurveto{\pgfqpoint{6.769357in}{0.595945in}}{\pgfqpoint{6.758758in}{0.600335in}}{\pgfqpoint{6.747708in}{0.600335in}}%
\pgfpathcurveto{\pgfqpoint{6.736658in}{0.600335in}}{\pgfqpoint{6.726059in}{0.595945in}}{\pgfqpoint{6.718245in}{0.588131in}}%
\pgfpathcurveto{\pgfqpoint{6.710432in}{0.580318in}}{\pgfqpoint{6.706041in}{0.569719in}}{\pgfqpoint{6.706041in}{0.558669in}}%
\pgfpathcurveto{\pgfqpoint{6.706041in}{0.547619in}}{\pgfqpoint{6.710432in}{0.537019in}}{\pgfqpoint{6.718245in}{0.529206in}}%
\pgfpathcurveto{\pgfqpoint{6.726059in}{0.521392in}}{\pgfqpoint{6.736658in}{0.517002in}}{\pgfqpoint{6.747708in}{0.517002in}}%
\pgfusepath{stroke}%
\end{pgfscope}%
\begin{pgfscope}%
\pgfpathrectangle{\pgfqpoint{0.847223in}{0.554012in}}{\pgfqpoint{6.200000in}{4.530000in}}%
\pgfusepath{clip}%
\pgfsetbuttcap%
\pgfsetroundjoin%
\pgfsetlinewidth{1.003750pt}%
\definecolor{currentstroke}{rgb}{1.000000,0.000000,0.000000}%
\pgfsetstrokecolor{currentstroke}%
\pgfsetdash{}{0pt}%
\pgfpathmoveto{\pgfqpoint{6.756517in}{0.516836in}}%
\pgfpathcurveto{\pgfqpoint{6.767567in}{0.516836in}}{\pgfqpoint{6.778166in}{0.521226in}}{\pgfqpoint{6.785980in}{0.529040in}}%
\pgfpathcurveto{\pgfqpoint{6.793794in}{0.536854in}}{\pgfqpoint{6.798184in}{0.547453in}}{\pgfqpoint{6.798184in}{0.558503in}}%
\pgfpathcurveto{\pgfqpoint{6.798184in}{0.569553in}}{\pgfqpoint{6.793794in}{0.580152in}}{\pgfqpoint{6.785980in}{0.587966in}}%
\pgfpathcurveto{\pgfqpoint{6.778166in}{0.595779in}}{\pgfqpoint{6.767567in}{0.600169in}}{\pgfqpoint{6.756517in}{0.600169in}}%
\pgfpathcurveto{\pgfqpoint{6.745467in}{0.600169in}}{\pgfqpoint{6.734868in}{0.595779in}}{\pgfqpoint{6.727054in}{0.587966in}}%
\pgfpathcurveto{\pgfqpoint{6.719241in}{0.580152in}}{\pgfqpoint{6.714851in}{0.569553in}}{\pgfqpoint{6.714851in}{0.558503in}}%
\pgfpathcurveto{\pgfqpoint{6.714851in}{0.547453in}}{\pgfqpoint{6.719241in}{0.536854in}}{\pgfqpoint{6.727054in}{0.529040in}}%
\pgfpathcurveto{\pgfqpoint{6.734868in}{0.521226in}}{\pgfqpoint{6.745467in}{0.516836in}}{\pgfqpoint{6.756517in}{0.516836in}}%
\pgfusepath{stroke}%
\end{pgfscope}%
\begin{pgfscope}%
\pgfpathrectangle{\pgfqpoint{0.847223in}{0.554012in}}{\pgfqpoint{6.200000in}{4.530000in}}%
\pgfusepath{clip}%
\pgfsetbuttcap%
\pgfsetroundjoin%
\pgfsetlinewidth{1.003750pt}%
\definecolor{currentstroke}{rgb}{1.000000,0.000000,0.000000}%
\pgfsetstrokecolor{currentstroke}%
\pgfsetdash{}{0pt}%
\pgfpathmoveto{\pgfqpoint{6.765326in}{0.516672in}}%
\pgfpathcurveto{\pgfqpoint{6.776377in}{0.516672in}}{\pgfqpoint{6.786976in}{0.521063in}}{\pgfqpoint{6.794789in}{0.528876in}}%
\pgfpathcurveto{\pgfqpoint{6.802603in}{0.536690in}}{\pgfqpoint{6.806993in}{0.547289in}}{\pgfqpoint{6.806993in}{0.558339in}}%
\pgfpathcurveto{\pgfqpoint{6.806993in}{0.569389in}}{\pgfqpoint{6.802603in}{0.579988in}}{\pgfqpoint{6.794789in}{0.587802in}}%
\pgfpathcurveto{\pgfqpoint{6.786976in}{0.595615in}}{\pgfqpoint{6.776377in}{0.600006in}}{\pgfqpoint{6.765326in}{0.600006in}}%
\pgfpathcurveto{\pgfqpoint{6.754276in}{0.600006in}}{\pgfqpoint{6.743677in}{0.595615in}}{\pgfqpoint{6.735864in}{0.587802in}}%
\pgfpathcurveto{\pgfqpoint{6.728050in}{0.579988in}}{\pgfqpoint{6.723660in}{0.569389in}}{\pgfqpoint{6.723660in}{0.558339in}}%
\pgfpathcurveto{\pgfqpoint{6.723660in}{0.547289in}}{\pgfqpoint{6.728050in}{0.536690in}}{\pgfqpoint{6.735864in}{0.528876in}}%
\pgfpathcurveto{\pgfqpoint{6.743677in}{0.521063in}}{\pgfqpoint{6.754276in}{0.516672in}}{\pgfqpoint{6.765326in}{0.516672in}}%
\pgfusepath{stroke}%
\end{pgfscope}%
\begin{pgfscope}%
\pgfpathrectangle{\pgfqpoint{0.847223in}{0.554012in}}{\pgfqpoint{6.200000in}{4.530000in}}%
\pgfusepath{clip}%
\pgfsetbuttcap%
\pgfsetroundjoin%
\pgfsetlinewidth{1.003750pt}%
\definecolor{currentstroke}{rgb}{1.000000,0.000000,0.000000}%
\pgfsetstrokecolor{currentstroke}%
\pgfsetdash{}{0pt}%
\pgfpathmoveto{\pgfqpoint{6.774136in}{0.516511in}}%
\pgfpathcurveto{\pgfqpoint{6.785186in}{0.516511in}}{\pgfqpoint{6.795785in}{0.520901in}}{\pgfqpoint{6.803599in}{0.528715in}}%
\pgfpathcurveto{\pgfqpoint{6.811412in}{0.536528in}}{\pgfqpoint{6.815802in}{0.547127in}}{\pgfqpoint{6.815802in}{0.558177in}}%
\pgfpathcurveto{\pgfqpoint{6.815802in}{0.569228in}}{\pgfqpoint{6.811412in}{0.579827in}}{\pgfqpoint{6.803599in}{0.587640in}}%
\pgfpathcurveto{\pgfqpoint{6.795785in}{0.595454in}}{\pgfqpoint{6.785186in}{0.599844in}}{\pgfqpoint{6.774136in}{0.599844in}}%
\pgfpathcurveto{\pgfqpoint{6.763086in}{0.599844in}}{\pgfqpoint{6.752487in}{0.595454in}}{\pgfqpoint{6.744673in}{0.587640in}}%
\pgfpathcurveto{\pgfqpoint{6.736859in}{0.579827in}}{\pgfqpoint{6.732469in}{0.569228in}}{\pgfqpoint{6.732469in}{0.558177in}}%
\pgfpathcurveto{\pgfqpoint{6.732469in}{0.547127in}}{\pgfqpoint{6.736859in}{0.536528in}}{\pgfqpoint{6.744673in}{0.528715in}}%
\pgfpathcurveto{\pgfqpoint{6.752487in}{0.520901in}}{\pgfqpoint{6.763086in}{0.516511in}}{\pgfqpoint{6.774136in}{0.516511in}}%
\pgfusepath{stroke}%
\end{pgfscope}%
\begin{pgfscope}%
\pgfpathrectangle{\pgfqpoint{0.847223in}{0.554012in}}{\pgfqpoint{6.200000in}{4.530000in}}%
\pgfusepath{clip}%
\pgfsetbuttcap%
\pgfsetroundjoin%
\pgfsetlinewidth{1.003750pt}%
\definecolor{currentstroke}{rgb}{1.000000,0.000000,0.000000}%
\pgfsetstrokecolor{currentstroke}%
\pgfsetdash{}{0pt}%
\pgfpathmoveto{\pgfqpoint{6.782945in}{0.516351in}}%
\pgfpathcurveto{\pgfqpoint{6.793995in}{0.516351in}}{\pgfqpoint{6.804594in}{0.520741in}}{\pgfqpoint{6.812408in}{0.528555in}}%
\pgfpathcurveto{\pgfqpoint{6.820221in}{0.536369in}}{\pgfqpoint{6.824612in}{0.546968in}}{\pgfqpoint{6.824612in}{0.558018in}}%
\pgfpathcurveto{\pgfqpoint{6.824612in}{0.569068in}}{\pgfqpoint{6.820221in}{0.579667in}}{\pgfqpoint{6.812408in}{0.587481in}}%
\pgfpathcurveto{\pgfqpoint{6.804594in}{0.595294in}}{\pgfqpoint{6.793995in}{0.599685in}}{\pgfqpoint{6.782945in}{0.599685in}}%
\pgfpathcurveto{\pgfqpoint{6.771895in}{0.599685in}}{\pgfqpoint{6.761296in}{0.595294in}}{\pgfqpoint{6.753482in}{0.587481in}}%
\pgfpathcurveto{\pgfqpoint{6.745669in}{0.579667in}}{\pgfqpoint{6.741278in}{0.569068in}}{\pgfqpoint{6.741278in}{0.558018in}}%
\pgfpathcurveto{\pgfqpoint{6.741278in}{0.546968in}}{\pgfqpoint{6.745669in}{0.536369in}}{\pgfqpoint{6.753482in}{0.528555in}}%
\pgfpathcurveto{\pgfqpoint{6.761296in}{0.520741in}}{\pgfqpoint{6.771895in}{0.516351in}}{\pgfqpoint{6.782945in}{0.516351in}}%
\pgfusepath{stroke}%
\end{pgfscope}%
\begin{pgfscope}%
\pgfpathrectangle{\pgfqpoint{0.847223in}{0.554012in}}{\pgfqpoint{6.200000in}{4.530000in}}%
\pgfusepath{clip}%
\pgfsetbuttcap%
\pgfsetroundjoin%
\pgfsetlinewidth{1.003750pt}%
\definecolor{currentstroke}{rgb}{1.000000,0.000000,0.000000}%
\pgfsetstrokecolor{currentstroke}%
\pgfsetdash{}{0pt}%
\pgfpathmoveto{\pgfqpoint{6.791754in}{0.516194in}}%
\pgfpathcurveto{\pgfqpoint{6.802804in}{0.516194in}}{\pgfqpoint{6.813403in}{0.520584in}}{\pgfqpoint{6.821217in}{0.528398in}}%
\pgfpathcurveto{\pgfqpoint{6.829031in}{0.536211in}}{\pgfqpoint{6.833421in}{0.546810in}}{\pgfqpoint{6.833421in}{0.557860in}}%
\pgfpathcurveto{\pgfqpoint{6.833421in}{0.568910in}}{\pgfqpoint{6.829031in}{0.579509in}}{\pgfqpoint{6.821217in}{0.587323in}}%
\pgfpathcurveto{\pgfqpoint{6.813403in}{0.595137in}}{\pgfqpoint{6.802804in}{0.599527in}}{\pgfqpoint{6.791754in}{0.599527in}}%
\pgfpathcurveto{\pgfqpoint{6.780704in}{0.599527in}}{\pgfqpoint{6.770105in}{0.595137in}}{\pgfqpoint{6.762291in}{0.587323in}}%
\pgfpathcurveto{\pgfqpoint{6.754478in}{0.579509in}}{\pgfqpoint{6.750088in}{0.568910in}}{\pgfqpoint{6.750088in}{0.557860in}}%
\pgfpathcurveto{\pgfqpoint{6.750088in}{0.546810in}}{\pgfqpoint{6.754478in}{0.536211in}}{\pgfqpoint{6.762291in}{0.528398in}}%
\pgfpathcurveto{\pgfqpoint{6.770105in}{0.520584in}}{\pgfqpoint{6.780704in}{0.516194in}}{\pgfqpoint{6.791754in}{0.516194in}}%
\pgfusepath{stroke}%
\end{pgfscope}%
\begin{pgfscope}%
\pgfpathrectangle{\pgfqpoint{0.847223in}{0.554012in}}{\pgfqpoint{6.200000in}{4.530000in}}%
\pgfusepath{clip}%
\pgfsetbuttcap%
\pgfsetroundjoin%
\pgfsetlinewidth{1.003750pt}%
\definecolor{currentstroke}{rgb}{1.000000,0.000000,0.000000}%
\pgfsetstrokecolor{currentstroke}%
\pgfsetdash{}{0pt}%
\pgfpathmoveto{\pgfqpoint{6.800564in}{0.516038in}}%
\pgfpathcurveto{\pgfqpoint{6.811614in}{0.516038in}}{\pgfqpoint{6.822213in}{0.520428in}}{\pgfqpoint{6.830026in}{0.528242in}}%
\pgfpathcurveto{\pgfqpoint{6.837840in}{0.536056in}}{\pgfqpoint{6.842230in}{0.546655in}}{\pgfqpoint{6.842230in}{0.557705in}}%
\pgfpathcurveto{\pgfqpoint{6.842230in}{0.568755in}}{\pgfqpoint{6.837840in}{0.579354in}}{\pgfqpoint{6.830026in}{0.587168in}}%
\pgfpathcurveto{\pgfqpoint{6.822213in}{0.594981in}}{\pgfqpoint{6.811614in}{0.599371in}}{\pgfqpoint{6.800564in}{0.599371in}}%
\pgfpathcurveto{\pgfqpoint{6.789513in}{0.599371in}}{\pgfqpoint{6.778914in}{0.594981in}}{\pgfqpoint{6.771101in}{0.587168in}}%
\pgfpathcurveto{\pgfqpoint{6.763287in}{0.579354in}}{\pgfqpoint{6.758897in}{0.568755in}}{\pgfqpoint{6.758897in}{0.557705in}}%
\pgfpathcurveto{\pgfqpoint{6.758897in}{0.546655in}}{\pgfqpoint{6.763287in}{0.536056in}}{\pgfqpoint{6.771101in}{0.528242in}}%
\pgfpathcurveto{\pgfqpoint{6.778914in}{0.520428in}}{\pgfqpoint{6.789513in}{0.516038in}}{\pgfqpoint{6.800564in}{0.516038in}}%
\pgfusepath{stroke}%
\end{pgfscope}%
\begin{pgfscope}%
\pgfpathrectangle{\pgfqpoint{0.847223in}{0.554012in}}{\pgfqpoint{6.200000in}{4.530000in}}%
\pgfusepath{clip}%
\pgfsetbuttcap%
\pgfsetroundjoin%
\pgfsetlinewidth{1.003750pt}%
\definecolor{currentstroke}{rgb}{1.000000,0.000000,0.000000}%
\pgfsetstrokecolor{currentstroke}%
\pgfsetdash{}{0pt}%
\pgfpathmoveto{\pgfqpoint{6.809373in}{0.515884in}}%
\pgfpathcurveto{\pgfqpoint{6.820423in}{0.515884in}}{\pgfqpoint{6.831022in}{0.520275in}}{\pgfqpoint{6.838836in}{0.528088in}}%
\pgfpathcurveto{\pgfqpoint{6.846649in}{0.535902in}}{\pgfqpoint{6.851039in}{0.546501in}}{\pgfqpoint{6.851039in}{0.557551in}}%
\pgfpathcurveto{\pgfqpoint{6.851039in}{0.568601in}}{\pgfqpoint{6.846649in}{0.579200in}}{\pgfqpoint{6.838836in}{0.587014in}}%
\pgfpathcurveto{\pgfqpoint{6.831022in}{0.594828in}}{\pgfqpoint{6.820423in}{0.599218in}}{\pgfqpoint{6.809373in}{0.599218in}}%
\pgfpathcurveto{\pgfqpoint{6.798323in}{0.599218in}}{\pgfqpoint{6.787724in}{0.594828in}}{\pgfqpoint{6.779910in}{0.587014in}}%
\pgfpathcurveto{\pgfqpoint{6.772096in}{0.579200in}}{\pgfqpoint{6.767706in}{0.568601in}}{\pgfqpoint{6.767706in}{0.557551in}}%
\pgfpathcurveto{\pgfqpoint{6.767706in}{0.546501in}}{\pgfqpoint{6.772096in}{0.535902in}}{\pgfqpoint{6.779910in}{0.528088in}}%
\pgfpathcurveto{\pgfqpoint{6.787724in}{0.520275in}}{\pgfqpoint{6.798323in}{0.515884in}}{\pgfqpoint{6.809373in}{0.515884in}}%
\pgfusepath{stroke}%
\end{pgfscope}%
\begin{pgfscope}%
\pgfpathrectangle{\pgfqpoint{0.847223in}{0.554012in}}{\pgfqpoint{6.200000in}{4.530000in}}%
\pgfusepath{clip}%
\pgfsetbuttcap%
\pgfsetroundjoin%
\pgfsetlinewidth{1.003750pt}%
\definecolor{currentstroke}{rgb}{1.000000,0.000000,0.000000}%
\pgfsetstrokecolor{currentstroke}%
\pgfsetdash{}{0pt}%
\pgfpathmoveto{\pgfqpoint{6.818182in}{0.515733in}}%
\pgfpathcurveto{\pgfqpoint{6.829232in}{0.515733in}}{\pgfqpoint{6.839831in}{0.520123in}}{\pgfqpoint{6.847645in}{0.527937in}}%
\pgfpathcurveto{\pgfqpoint{6.855458in}{0.535750in}}{\pgfqpoint{6.859849in}{0.546349in}}{\pgfqpoint{6.859849in}{0.557399in}}%
\pgfpathcurveto{\pgfqpoint{6.859849in}{0.568449in}}{\pgfqpoint{6.855458in}{0.579049in}}{\pgfqpoint{6.847645in}{0.586862in}}%
\pgfpathcurveto{\pgfqpoint{6.839831in}{0.594676in}}{\pgfqpoint{6.829232in}{0.599066in}}{\pgfqpoint{6.818182in}{0.599066in}}%
\pgfpathcurveto{\pgfqpoint{6.807132in}{0.599066in}}{\pgfqpoint{6.796533in}{0.594676in}}{\pgfqpoint{6.788719in}{0.586862in}}%
\pgfpathcurveto{\pgfqpoint{6.780906in}{0.579049in}}{\pgfqpoint{6.776515in}{0.568449in}}{\pgfqpoint{6.776515in}{0.557399in}}%
\pgfpathcurveto{\pgfqpoint{6.776515in}{0.546349in}}{\pgfqpoint{6.780906in}{0.535750in}}{\pgfqpoint{6.788719in}{0.527937in}}%
\pgfpathcurveto{\pgfqpoint{6.796533in}{0.520123in}}{\pgfqpoint{6.807132in}{0.515733in}}{\pgfqpoint{6.818182in}{0.515733in}}%
\pgfusepath{stroke}%
\end{pgfscope}%
\begin{pgfscope}%
\pgfpathrectangle{\pgfqpoint{0.847223in}{0.554012in}}{\pgfqpoint{6.200000in}{4.530000in}}%
\pgfusepath{clip}%
\pgfsetbuttcap%
\pgfsetroundjoin%
\pgfsetlinewidth{1.003750pt}%
\definecolor{currentstroke}{rgb}{1.000000,0.000000,0.000000}%
\pgfsetstrokecolor{currentstroke}%
\pgfsetdash{}{0pt}%
\pgfpathmoveto{\pgfqpoint{6.826991in}{0.515583in}}%
\pgfpathcurveto{\pgfqpoint{6.838041in}{0.515583in}}{\pgfqpoint{6.848641in}{0.519973in}}{\pgfqpoint{6.856454in}{0.527787in}}%
\pgfpathcurveto{\pgfqpoint{6.864268in}{0.535600in}}{\pgfqpoint{6.868658in}{0.546199in}}{\pgfqpoint{6.868658in}{0.557249in}}%
\pgfpathcurveto{\pgfqpoint{6.868658in}{0.568300in}}{\pgfqpoint{6.864268in}{0.578899in}}{\pgfqpoint{6.856454in}{0.586712in}}%
\pgfpathcurveto{\pgfqpoint{6.848641in}{0.594526in}}{\pgfqpoint{6.838041in}{0.598916in}}{\pgfqpoint{6.826991in}{0.598916in}}%
\pgfpathcurveto{\pgfqpoint{6.815941in}{0.598916in}}{\pgfqpoint{6.805342in}{0.594526in}}{\pgfqpoint{6.797529in}{0.586712in}}%
\pgfpathcurveto{\pgfqpoint{6.789715in}{0.578899in}}{\pgfqpoint{6.785325in}{0.568300in}}{\pgfqpoint{6.785325in}{0.557249in}}%
\pgfpathcurveto{\pgfqpoint{6.785325in}{0.546199in}}{\pgfqpoint{6.789715in}{0.535600in}}{\pgfqpoint{6.797529in}{0.527787in}}%
\pgfpathcurveto{\pgfqpoint{6.805342in}{0.519973in}}{\pgfqpoint{6.815941in}{0.515583in}}{\pgfqpoint{6.826991in}{0.515583in}}%
\pgfusepath{stroke}%
\end{pgfscope}%
\begin{pgfscope}%
\pgfpathrectangle{\pgfqpoint{0.847223in}{0.554012in}}{\pgfqpoint{6.200000in}{4.530000in}}%
\pgfusepath{clip}%
\pgfsetbuttcap%
\pgfsetroundjoin%
\pgfsetlinewidth{1.003750pt}%
\definecolor{currentstroke}{rgb}{1.000000,0.000000,0.000000}%
\pgfsetstrokecolor{currentstroke}%
\pgfsetdash{}{0pt}%
\pgfpathmoveto{\pgfqpoint{6.835801in}{0.515435in}}%
\pgfpathcurveto{\pgfqpoint{6.846851in}{0.515435in}}{\pgfqpoint{6.857450in}{0.519825in}}{\pgfqpoint{6.865263in}{0.527639in}}%
\pgfpathcurveto{\pgfqpoint{6.873077in}{0.535452in}}{\pgfqpoint{6.877467in}{0.546051in}}{\pgfqpoint{6.877467in}{0.557101in}}%
\pgfpathcurveto{\pgfqpoint{6.877467in}{0.568152in}}{\pgfqpoint{6.873077in}{0.578751in}}{\pgfqpoint{6.865263in}{0.586564in}}%
\pgfpathcurveto{\pgfqpoint{6.857450in}{0.594378in}}{\pgfqpoint{6.846851in}{0.598768in}}{\pgfqpoint{6.835801in}{0.598768in}}%
\pgfpathcurveto{\pgfqpoint{6.824750in}{0.598768in}}{\pgfqpoint{6.814151in}{0.594378in}}{\pgfqpoint{6.806338in}{0.586564in}}%
\pgfpathcurveto{\pgfqpoint{6.798524in}{0.578751in}}{\pgfqpoint{6.794134in}{0.568152in}}{\pgfqpoint{6.794134in}{0.557101in}}%
\pgfpathcurveto{\pgfqpoint{6.794134in}{0.546051in}}{\pgfqpoint{6.798524in}{0.535452in}}{\pgfqpoint{6.806338in}{0.527639in}}%
\pgfpathcurveto{\pgfqpoint{6.814151in}{0.519825in}}{\pgfqpoint{6.824750in}{0.515435in}}{\pgfqpoint{6.835801in}{0.515435in}}%
\pgfusepath{stroke}%
\end{pgfscope}%
\begin{pgfscope}%
\pgfpathrectangle{\pgfqpoint{0.847223in}{0.554012in}}{\pgfqpoint{6.200000in}{4.530000in}}%
\pgfusepath{clip}%
\pgfsetbuttcap%
\pgfsetroundjoin%
\pgfsetlinewidth{1.003750pt}%
\definecolor{currentstroke}{rgb}{1.000000,0.000000,0.000000}%
\pgfsetstrokecolor{currentstroke}%
\pgfsetdash{}{0pt}%
\pgfpathmoveto{\pgfqpoint{6.844610in}{0.515288in}}%
\pgfpathcurveto{\pgfqpoint{6.855660in}{0.515288in}}{\pgfqpoint{6.866259in}{0.519679in}}{\pgfqpoint{6.874073in}{0.527492in}}%
\pgfpathcurveto{\pgfqpoint{6.881886in}{0.535306in}}{\pgfqpoint{6.886277in}{0.545905in}}{\pgfqpoint{6.886277in}{0.556955in}}%
\pgfpathcurveto{\pgfqpoint{6.886277in}{0.568005in}}{\pgfqpoint{6.881886in}{0.578604in}}{\pgfqpoint{6.874073in}{0.586418in}}%
\pgfpathcurveto{\pgfqpoint{6.866259in}{0.594232in}}{\pgfqpoint{6.855660in}{0.598622in}}{\pgfqpoint{6.844610in}{0.598622in}}%
\pgfpathcurveto{\pgfqpoint{6.833560in}{0.598622in}}{\pgfqpoint{6.822961in}{0.594232in}}{\pgfqpoint{6.815147in}{0.586418in}}%
\pgfpathcurveto{\pgfqpoint{6.807333in}{0.578604in}}{\pgfqpoint{6.802943in}{0.568005in}}{\pgfqpoint{6.802943in}{0.556955in}}%
\pgfpathcurveto{\pgfqpoint{6.802943in}{0.545905in}}{\pgfqpoint{6.807333in}{0.535306in}}{\pgfqpoint{6.815147in}{0.527492in}}%
\pgfpathcurveto{\pgfqpoint{6.822961in}{0.519679in}}{\pgfqpoint{6.833560in}{0.515288in}}{\pgfqpoint{6.844610in}{0.515288in}}%
\pgfusepath{stroke}%
\end{pgfscope}%
\begin{pgfscope}%
\pgfpathrectangle{\pgfqpoint{0.847223in}{0.554012in}}{\pgfqpoint{6.200000in}{4.530000in}}%
\pgfusepath{clip}%
\pgfsetbuttcap%
\pgfsetroundjoin%
\pgfsetlinewidth{1.003750pt}%
\definecolor{currentstroke}{rgb}{1.000000,0.000000,0.000000}%
\pgfsetstrokecolor{currentstroke}%
\pgfsetdash{}{0pt}%
\pgfpathmoveto{\pgfqpoint{6.853419in}{0.515144in}}%
\pgfpathcurveto{\pgfqpoint{6.864469in}{0.515144in}}{\pgfqpoint{6.875068in}{0.519534in}}{\pgfqpoint{6.882882in}{0.527348in}}%
\pgfpathcurveto{\pgfqpoint{6.890696in}{0.535161in}}{\pgfqpoint{6.895086in}{0.545760in}}{\pgfqpoint{6.895086in}{0.556811in}}%
\pgfpathcurveto{\pgfqpoint{6.895086in}{0.567861in}}{\pgfqpoint{6.890696in}{0.578460in}}{\pgfqpoint{6.882882in}{0.586273in}}%
\pgfpathcurveto{\pgfqpoint{6.875068in}{0.594087in}}{\pgfqpoint{6.864469in}{0.598477in}}{\pgfqpoint{6.853419in}{0.598477in}}%
\pgfpathcurveto{\pgfqpoint{6.842369in}{0.598477in}}{\pgfqpoint{6.831770in}{0.594087in}}{\pgfqpoint{6.823956in}{0.586273in}}%
\pgfpathcurveto{\pgfqpoint{6.816143in}{0.578460in}}{\pgfqpoint{6.811752in}{0.567861in}}{\pgfqpoint{6.811752in}{0.556811in}}%
\pgfpathcurveto{\pgfqpoint{6.811752in}{0.545760in}}{\pgfqpoint{6.816143in}{0.535161in}}{\pgfqpoint{6.823956in}{0.527348in}}%
\pgfpathcurveto{\pgfqpoint{6.831770in}{0.519534in}}{\pgfqpoint{6.842369in}{0.515144in}}{\pgfqpoint{6.853419in}{0.515144in}}%
\pgfusepath{stroke}%
\end{pgfscope}%
\begin{pgfscope}%
\pgfpathrectangle{\pgfqpoint{0.847223in}{0.554012in}}{\pgfqpoint{6.200000in}{4.530000in}}%
\pgfusepath{clip}%
\pgfsetbuttcap%
\pgfsetroundjoin%
\pgfsetlinewidth{1.003750pt}%
\definecolor{currentstroke}{rgb}{1.000000,0.000000,0.000000}%
\pgfsetstrokecolor{currentstroke}%
\pgfsetdash{}{0pt}%
\pgfpathmoveto{\pgfqpoint{6.862228in}{0.515001in}}%
\pgfpathcurveto{\pgfqpoint{6.873279in}{0.515001in}}{\pgfqpoint{6.883878in}{0.519391in}}{\pgfqpoint{6.891691in}{0.527205in}}%
\pgfpathcurveto{\pgfqpoint{6.899505in}{0.535019in}}{\pgfqpoint{6.903895in}{0.545618in}}{\pgfqpoint{6.903895in}{0.556668in}}%
\pgfpathcurveto{\pgfqpoint{6.903895in}{0.567718in}}{\pgfqpoint{6.899505in}{0.578317in}}{\pgfqpoint{6.891691in}{0.586131in}}%
\pgfpathcurveto{\pgfqpoint{6.883878in}{0.593944in}}{\pgfqpoint{6.873279in}{0.598334in}}{\pgfqpoint{6.862228in}{0.598334in}}%
\pgfpathcurveto{\pgfqpoint{6.851178in}{0.598334in}}{\pgfqpoint{6.840579in}{0.593944in}}{\pgfqpoint{6.832766in}{0.586131in}}%
\pgfpathcurveto{\pgfqpoint{6.824952in}{0.578317in}}{\pgfqpoint{6.820562in}{0.567718in}}{\pgfqpoint{6.820562in}{0.556668in}}%
\pgfpathcurveto{\pgfqpoint{6.820562in}{0.545618in}}{\pgfqpoint{6.824952in}{0.535019in}}{\pgfqpoint{6.832766in}{0.527205in}}%
\pgfpathcurveto{\pgfqpoint{6.840579in}{0.519391in}}{\pgfqpoint{6.851178in}{0.515001in}}{\pgfqpoint{6.862228in}{0.515001in}}%
\pgfusepath{stroke}%
\end{pgfscope}%
\begin{pgfscope}%
\pgfpathrectangle{\pgfqpoint{0.847223in}{0.554012in}}{\pgfqpoint{6.200000in}{4.530000in}}%
\pgfusepath{clip}%
\pgfsetbuttcap%
\pgfsetroundjoin%
\pgfsetlinewidth{1.003750pt}%
\definecolor{currentstroke}{rgb}{1.000000,0.000000,0.000000}%
\pgfsetstrokecolor{currentstroke}%
\pgfsetdash{}{0pt}%
\pgfpathmoveto{\pgfqpoint{6.871038in}{0.514860in}}%
\pgfpathcurveto{\pgfqpoint{6.882088in}{0.514860in}}{\pgfqpoint{6.892687in}{0.519250in}}{\pgfqpoint{6.900500in}{0.527064in}}%
\pgfpathcurveto{\pgfqpoint{6.908314in}{0.534877in}}{\pgfqpoint{6.912704in}{0.545476in}}{\pgfqpoint{6.912704in}{0.556527in}}%
\pgfpathcurveto{\pgfqpoint{6.912704in}{0.567577in}}{\pgfqpoint{6.908314in}{0.578176in}}{\pgfqpoint{6.900500in}{0.585989in}}%
\pgfpathcurveto{\pgfqpoint{6.892687in}{0.593803in}}{\pgfqpoint{6.882088in}{0.598193in}}{\pgfqpoint{6.871038in}{0.598193in}}%
\pgfpathcurveto{\pgfqpoint{6.859988in}{0.598193in}}{\pgfqpoint{6.849389in}{0.593803in}}{\pgfqpoint{6.841575in}{0.585989in}}%
\pgfpathcurveto{\pgfqpoint{6.833761in}{0.578176in}}{\pgfqpoint{6.829371in}{0.567577in}}{\pgfqpoint{6.829371in}{0.556527in}}%
\pgfpathcurveto{\pgfqpoint{6.829371in}{0.545476in}}{\pgfqpoint{6.833761in}{0.534877in}}{\pgfqpoint{6.841575in}{0.527064in}}%
\pgfpathcurveto{\pgfqpoint{6.849389in}{0.519250in}}{\pgfqpoint{6.859988in}{0.514860in}}{\pgfqpoint{6.871038in}{0.514860in}}%
\pgfusepath{stroke}%
\end{pgfscope}%
\begin{pgfscope}%
\pgfpathrectangle{\pgfqpoint{0.847223in}{0.554012in}}{\pgfqpoint{6.200000in}{4.530000in}}%
\pgfusepath{clip}%
\pgfsetbuttcap%
\pgfsetroundjoin%
\pgfsetlinewidth{1.003750pt}%
\definecolor{currentstroke}{rgb}{1.000000,0.000000,0.000000}%
\pgfsetstrokecolor{currentstroke}%
\pgfsetdash{}{0pt}%
\pgfpathmoveto{\pgfqpoint{6.879847in}{0.514720in}}%
\pgfpathcurveto{\pgfqpoint{6.890897in}{0.514720in}}{\pgfqpoint{6.901496in}{0.519111in}}{\pgfqpoint{6.909310in}{0.526924in}}%
\pgfpathcurveto{\pgfqpoint{6.917123in}{0.534738in}}{\pgfqpoint{6.921514in}{0.545337in}}{\pgfqpoint{6.921514in}{0.556387in}}%
\pgfpathcurveto{\pgfqpoint{6.921514in}{0.567437in}}{\pgfqpoint{6.917123in}{0.578036in}}{\pgfqpoint{6.909310in}{0.585850in}}%
\pgfpathcurveto{\pgfqpoint{6.901496in}{0.593663in}}{\pgfqpoint{6.890897in}{0.598054in}}{\pgfqpoint{6.879847in}{0.598054in}}%
\pgfpathcurveto{\pgfqpoint{6.868797in}{0.598054in}}{\pgfqpoint{6.858198in}{0.593663in}}{\pgfqpoint{6.850384in}{0.585850in}}%
\pgfpathcurveto{\pgfqpoint{6.842571in}{0.578036in}}{\pgfqpoint{6.838180in}{0.567437in}}{\pgfqpoint{6.838180in}{0.556387in}}%
\pgfpathcurveto{\pgfqpoint{6.838180in}{0.545337in}}{\pgfqpoint{6.842571in}{0.534738in}}{\pgfqpoint{6.850384in}{0.526924in}}%
\pgfpathcurveto{\pgfqpoint{6.858198in}{0.519111in}}{\pgfqpoint{6.868797in}{0.514720in}}{\pgfqpoint{6.879847in}{0.514720in}}%
\pgfusepath{stroke}%
\end{pgfscope}%
\begin{pgfscope}%
\pgfpathrectangle{\pgfqpoint{0.847223in}{0.554012in}}{\pgfqpoint{6.200000in}{4.530000in}}%
\pgfusepath{clip}%
\pgfsetbuttcap%
\pgfsetroundjoin%
\pgfsetlinewidth{1.003750pt}%
\definecolor{currentstroke}{rgb}{1.000000,0.000000,0.000000}%
\pgfsetstrokecolor{currentstroke}%
\pgfsetdash{}{0pt}%
\pgfpathmoveto{\pgfqpoint{6.888656in}{0.514583in}}%
\pgfpathcurveto{\pgfqpoint{6.899706in}{0.514583in}}{\pgfqpoint{6.910305in}{0.518973in}}{\pgfqpoint{6.918119in}{0.526786in}}%
\pgfpathcurveto{\pgfqpoint{6.925933in}{0.534600in}}{\pgfqpoint{6.930323in}{0.545199in}}{\pgfqpoint{6.930323in}{0.556249in}}%
\pgfpathcurveto{\pgfqpoint{6.930323in}{0.567299in}}{\pgfqpoint{6.925933in}{0.577898in}}{\pgfqpoint{6.918119in}{0.585712in}}%
\pgfpathcurveto{\pgfqpoint{6.910305in}{0.593526in}}{\pgfqpoint{6.899706in}{0.597916in}}{\pgfqpoint{6.888656in}{0.597916in}}%
\pgfpathcurveto{\pgfqpoint{6.877606in}{0.597916in}}{\pgfqpoint{6.867007in}{0.593526in}}{\pgfqpoint{6.859193in}{0.585712in}}%
\pgfpathcurveto{\pgfqpoint{6.851380in}{0.577898in}}{\pgfqpoint{6.846990in}{0.567299in}}{\pgfqpoint{6.846990in}{0.556249in}}%
\pgfpathcurveto{\pgfqpoint{6.846990in}{0.545199in}}{\pgfqpoint{6.851380in}{0.534600in}}{\pgfqpoint{6.859193in}{0.526786in}}%
\pgfpathcurveto{\pgfqpoint{6.867007in}{0.518973in}}{\pgfqpoint{6.877606in}{0.514583in}}{\pgfqpoint{6.888656in}{0.514583in}}%
\pgfusepath{stroke}%
\end{pgfscope}%
\begin{pgfscope}%
\pgfpathrectangle{\pgfqpoint{0.847223in}{0.554012in}}{\pgfqpoint{6.200000in}{4.530000in}}%
\pgfusepath{clip}%
\pgfsetbuttcap%
\pgfsetroundjoin%
\pgfsetlinewidth{1.003750pt}%
\definecolor{currentstroke}{rgb}{1.000000,0.000000,0.000000}%
\pgfsetstrokecolor{currentstroke}%
\pgfsetdash{}{0pt}%
\pgfpathmoveto{\pgfqpoint{6.897465in}{0.514446in}}%
\pgfpathcurveto{\pgfqpoint{6.908516in}{0.514446in}}{\pgfqpoint{6.919115in}{0.518836in}}{\pgfqpoint{6.926928in}{0.526650in}}%
\pgfpathcurveto{\pgfqpoint{6.934742in}{0.534464in}}{\pgfqpoint{6.939132in}{0.545063in}}{\pgfqpoint{6.939132in}{0.556113in}}%
\pgfpathcurveto{\pgfqpoint{6.939132in}{0.567163in}}{\pgfqpoint{6.934742in}{0.577762in}}{\pgfqpoint{6.926928in}{0.585576in}}%
\pgfpathcurveto{\pgfqpoint{6.919115in}{0.593389in}}{\pgfqpoint{6.908516in}{0.597780in}}{\pgfqpoint{6.897465in}{0.597780in}}%
\pgfpathcurveto{\pgfqpoint{6.886415in}{0.597780in}}{\pgfqpoint{6.875816in}{0.593389in}}{\pgfqpoint{6.868003in}{0.585576in}}%
\pgfpathcurveto{\pgfqpoint{6.860189in}{0.577762in}}{\pgfqpoint{6.855799in}{0.567163in}}{\pgfqpoint{6.855799in}{0.556113in}}%
\pgfpathcurveto{\pgfqpoint{6.855799in}{0.545063in}}{\pgfqpoint{6.860189in}{0.534464in}}{\pgfqpoint{6.868003in}{0.526650in}}%
\pgfpathcurveto{\pgfqpoint{6.875816in}{0.518836in}}{\pgfqpoint{6.886415in}{0.514446in}}{\pgfqpoint{6.897465in}{0.514446in}}%
\pgfusepath{stroke}%
\end{pgfscope}%
\begin{pgfscope}%
\pgfpathrectangle{\pgfqpoint{0.847223in}{0.554012in}}{\pgfqpoint{6.200000in}{4.530000in}}%
\pgfusepath{clip}%
\pgfsetbuttcap%
\pgfsetroundjoin%
\pgfsetlinewidth{1.003750pt}%
\definecolor{currentstroke}{rgb}{1.000000,0.000000,0.000000}%
\pgfsetstrokecolor{currentstroke}%
\pgfsetdash{}{0pt}%
\pgfpathmoveto{\pgfqpoint{6.906275in}{0.514311in}}%
\pgfpathcurveto{\pgfqpoint{6.917325in}{0.514311in}}{\pgfqpoint{6.927924in}{0.518702in}}{\pgfqpoint{6.935738in}{0.526515in}}%
\pgfpathcurveto{\pgfqpoint{6.943551in}{0.534329in}}{\pgfqpoint{6.947941in}{0.544928in}}{\pgfqpoint{6.947941in}{0.555978in}}%
\pgfpathcurveto{\pgfqpoint{6.947941in}{0.567028in}}{\pgfqpoint{6.943551in}{0.577627in}}{\pgfqpoint{6.935738in}{0.585441in}}%
\pgfpathcurveto{\pgfqpoint{6.927924in}{0.593255in}}{\pgfqpoint{6.917325in}{0.597645in}}{\pgfqpoint{6.906275in}{0.597645in}}%
\pgfpathcurveto{\pgfqpoint{6.895225in}{0.597645in}}{\pgfqpoint{6.884626in}{0.593255in}}{\pgfqpoint{6.876812in}{0.585441in}}%
\pgfpathcurveto{\pgfqpoint{6.868998in}{0.577627in}}{\pgfqpoint{6.864608in}{0.567028in}}{\pgfqpoint{6.864608in}{0.555978in}}%
\pgfpathcurveto{\pgfqpoint{6.864608in}{0.544928in}}{\pgfqpoint{6.868998in}{0.534329in}}{\pgfqpoint{6.876812in}{0.526515in}}%
\pgfpathcurveto{\pgfqpoint{6.884626in}{0.518702in}}{\pgfqpoint{6.895225in}{0.514311in}}{\pgfqpoint{6.906275in}{0.514311in}}%
\pgfusepath{stroke}%
\end{pgfscope}%
\begin{pgfscope}%
\pgfpathrectangle{\pgfqpoint{0.847223in}{0.554012in}}{\pgfqpoint{6.200000in}{4.530000in}}%
\pgfusepath{clip}%
\pgfsetbuttcap%
\pgfsetroundjoin%
\pgfsetlinewidth{1.003750pt}%
\definecolor{currentstroke}{rgb}{1.000000,0.000000,0.000000}%
\pgfsetstrokecolor{currentstroke}%
\pgfsetdash{}{0pt}%
\pgfpathmoveto{\pgfqpoint{6.915084in}{0.514178in}}%
\pgfpathcurveto{\pgfqpoint{6.926134in}{0.514178in}}{\pgfqpoint{6.936733in}{0.518568in}}{\pgfqpoint{6.944547in}{0.526382in}}%
\pgfpathcurveto{\pgfqpoint{6.952360in}{0.534196in}}{\pgfqpoint{6.956751in}{0.544795in}}{\pgfqpoint{6.956751in}{0.555845in}}%
\pgfpathcurveto{\pgfqpoint{6.956751in}{0.566895in}}{\pgfqpoint{6.952360in}{0.577494in}}{\pgfqpoint{6.944547in}{0.585308in}}%
\pgfpathcurveto{\pgfqpoint{6.936733in}{0.593121in}}{\pgfqpoint{6.926134in}{0.597512in}}{\pgfqpoint{6.915084in}{0.597512in}}%
\pgfpathcurveto{\pgfqpoint{6.904034in}{0.597512in}}{\pgfqpoint{6.893435in}{0.593121in}}{\pgfqpoint{6.885621in}{0.585308in}}%
\pgfpathcurveto{\pgfqpoint{6.877808in}{0.577494in}}{\pgfqpoint{6.873417in}{0.566895in}}{\pgfqpoint{6.873417in}{0.555845in}}%
\pgfpathcurveto{\pgfqpoint{6.873417in}{0.544795in}}{\pgfqpoint{6.877808in}{0.534196in}}{\pgfqpoint{6.885621in}{0.526382in}}%
\pgfpathcurveto{\pgfqpoint{6.893435in}{0.518568in}}{\pgfqpoint{6.904034in}{0.514178in}}{\pgfqpoint{6.915084in}{0.514178in}}%
\pgfusepath{stroke}%
\end{pgfscope}%
\begin{pgfscope}%
\pgfpathrectangle{\pgfqpoint{0.847223in}{0.554012in}}{\pgfqpoint{6.200000in}{4.530000in}}%
\pgfusepath{clip}%
\pgfsetbuttcap%
\pgfsetroundjoin%
\pgfsetlinewidth{1.003750pt}%
\definecolor{currentstroke}{rgb}{1.000000,0.000000,0.000000}%
\pgfsetstrokecolor{currentstroke}%
\pgfsetdash{}{0pt}%
\pgfpathmoveto{\pgfqpoint{6.923893in}{0.514046in}}%
\pgfpathcurveto{\pgfqpoint{6.934943in}{0.514046in}}{\pgfqpoint{6.945542in}{0.518437in}}{\pgfqpoint{6.953356in}{0.526250in}}%
\pgfpathcurveto{\pgfqpoint{6.961170in}{0.534064in}}{\pgfqpoint{6.965560in}{0.544663in}}{\pgfqpoint{6.965560in}{0.555713in}}%
\pgfpathcurveto{\pgfqpoint{6.965560in}{0.566763in}}{\pgfqpoint{6.961170in}{0.577362in}}{\pgfqpoint{6.953356in}{0.585176in}}%
\pgfpathcurveto{\pgfqpoint{6.945542in}{0.592990in}}{\pgfqpoint{6.934943in}{0.597380in}}{\pgfqpoint{6.923893in}{0.597380in}}%
\pgfpathcurveto{\pgfqpoint{6.912843in}{0.597380in}}{\pgfqpoint{6.902244in}{0.592990in}}{\pgfqpoint{6.894431in}{0.585176in}}%
\pgfpathcurveto{\pgfqpoint{6.886617in}{0.577362in}}{\pgfqpoint{6.882227in}{0.566763in}}{\pgfqpoint{6.882227in}{0.555713in}}%
\pgfpathcurveto{\pgfqpoint{6.882227in}{0.544663in}}{\pgfqpoint{6.886617in}{0.534064in}}{\pgfqpoint{6.894431in}{0.526250in}}%
\pgfpathcurveto{\pgfqpoint{6.902244in}{0.518437in}}{\pgfqpoint{6.912843in}{0.514046in}}{\pgfqpoint{6.923893in}{0.514046in}}%
\pgfusepath{stroke}%
\end{pgfscope}%
\begin{pgfscope}%
\pgfpathrectangle{\pgfqpoint{0.847223in}{0.554012in}}{\pgfqpoint{6.200000in}{4.530000in}}%
\pgfusepath{clip}%
\pgfsetbuttcap%
\pgfsetroundjoin%
\pgfsetlinewidth{1.003750pt}%
\definecolor{currentstroke}{rgb}{1.000000,0.000000,0.000000}%
\pgfsetstrokecolor{currentstroke}%
\pgfsetdash{}{0pt}%
\pgfpathmoveto{\pgfqpoint{6.932703in}{0.513916in}}%
\pgfpathcurveto{\pgfqpoint{6.943753in}{0.513916in}}{\pgfqpoint{6.954352in}{0.518306in}}{\pgfqpoint{6.962165in}{0.526120in}}%
\pgfpathcurveto{\pgfqpoint{6.969979in}{0.533934in}}{\pgfqpoint{6.974369in}{0.544533in}}{\pgfqpoint{6.974369in}{0.555583in}}%
\pgfpathcurveto{\pgfqpoint{6.974369in}{0.566633in}}{\pgfqpoint{6.969979in}{0.577232in}}{\pgfqpoint{6.962165in}{0.585046in}}%
\pgfpathcurveto{\pgfqpoint{6.954352in}{0.592859in}}{\pgfqpoint{6.943753in}{0.597250in}}{\pgfqpoint{6.932703in}{0.597250in}}%
\pgfpathcurveto{\pgfqpoint{6.921652in}{0.597250in}}{\pgfqpoint{6.911053in}{0.592859in}}{\pgfqpoint{6.903240in}{0.585046in}}%
\pgfpathcurveto{\pgfqpoint{6.895426in}{0.577232in}}{\pgfqpoint{6.891036in}{0.566633in}}{\pgfqpoint{6.891036in}{0.555583in}}%
\pgfpathcurveto{\pgfqpoint{6.891036in}{0.544533in}}{\pgfqpoint{6.895426in}{0.533934in}}{\pgfqpoint{6.903240in}{0.526120in}}%
\pgfpathcurveto{\pgfqpoint{6.911053in}{0.518306in}}{\pgfqpoint{6.921652in}{0.513916in}}{\pgfqpoint{6.932703in}{0.513916in}}%
\pgfusepath{stroke}%
\end{pgfscope}%
\begin{pgfscope}%
\pgfpathrectangle{\pgfqpoint{0.847223in}{0.554012in}}{\pgfqpoint{6.200000in}{4.530000in}}%
\pgfusepath{clip}%
\pgfsetbuttcap%
\pgfsetroundjoin%
\pgfsetlinewidth{1.003750pt}%
\definecolor{currentstroke}{rgb}{1.000000,0.000000,0.000000}%
\pgfsetstrokecolor{currentstroke}%
\pgfsetdash{}{0pt}%
\pgfpathmoveto{\pgfqpoint{6.941512in}{0.513787in}}%
\pgfpathcurveto{\pgfqpoint{6.952562in}{0.513787in}}{\pgfqpoint{6.963161in}{0.518178in}}{\pgfqpoint{6.970975in}{0.525991in}}%
\pgfpathcurveto{\pgfqpoint{6.978788in}{0.533805in}}{\pgfqpoint{6.983179in}{0.544404in}}{\pgfqpoint{6.983179in}{0.555454in}}%
\pgfpathcurveto{\pgfqpoint{6.983179in}{0.566504in}}{\pgfqpoint{6.978788in}{0.577103in}}{\pgfqpoint{6.970975in}{0.584917in}}%
\pgfpathcurveto{\pgfqpoint{6.963161in}{0.592730in}}{\pgfqpoint{6.952562in}{0.597121in}}{\pgfqpoint{6.941512in}{0.597121in}}%
\pgfpathcurveto{\pgfqpoint{6.930462in}{0.597121in}}{\pgfqpoint{6.919863in}{0.592730in}}{\pgfqpoint{6.912049in}{0.584917in}}%
\pgfpathcurveto{\pgfqpoint{6.904235in}{0.577103in}}{\pgfqpoint{6.899845in}{0.566504in}}{\pgfqpoint{6.899845in}{0.555454in}}%
\pgfpathcurveto{\pgfqpoint{6.899845in}{0.544404in}}{\pgfqpoint{6.904235in}{0.533805in}}{\pgfqpoint{6.912049in}{0.525991in}}%
\pgfpathcurveto{\pgfqpoint{6.919863in}{0.518178in}}{\pgfqpoint{6.930462in}{0.513787in}}{\pgfqpoint{6.941512in}{0.513787in}}%
\pgfusepath{stroke}%
\end{pgfscope}%
\begin{pgfscope}%
\pgfpathrectangle{\pgfqpoint{0.847223in}{0.554012in}}{\pgfqpoint{6.200000in}{4.530000in}}%
\pgfusepath{clip}%
\pgfsetbuttcap%
\pgfsetroundjoin%
\pgfsetlinewidth{1.003750pt}%
\definecolor{currentstroke}{rgb}{1.000000,0.000000,0.000000}%
\pgfsetstrokecolor{currentstroke}%
\pgfsetdash{}{0pt}%
\pgfpathmoveto{\pgfqpoint{6.950321in}{0.513660in}}%
\pgfpathcurveto{\pgfqpoint{6.961371in}{0.513660in}}{\pgfqpoint{6.971970in}{0.518050in}}{\pgfqpoint{6.979784in}{0.525864in}}%
\pgfpathcurveto{\pgfqpoint{6.987598in}{0.533677in}}{\pgfqpoint{6.991988in}{0.544277in}}{\pgfqpoint{6.991988in}{0.555327in}}%
\pgfpathcurveto{\pgfqpoint{6.991988in}{0.566377in}}{\pgfqpoint{6.987598in}{0.576976in}}{\pgfqpoint{6.979784in}{0.584789in}}%
\pgfpathcurveto{\pgfqpoint{6.971970in}{0.592603in}}{\pgfqpoint{6.961371in}{0.596993in}}{\pgfqpoint{6.950321in}{0.596993in}}%
\pgfpathcurveto{\pgfqpoint{6.939271in}{0.596993in}}{\pgfqpoint{6.928672in}{0.592603in}}{\pgfqpoint{6.920858in}{0.584789in}}%
\pgfpathcurveto{\pgfqpoint{6.913045in}{0.576976in}}{\pgfqpoint{6.908654in}{0.566377in}}{\pgfqpoint{6.908654in}{0.555327in}}%
\pgfpathcurveto{\pgfqpoint{6.908654in}{0.544277in}}{\pgfqpoint{6.913045in}{0.533677in}}{\pgfqpoint{6.920858in}{0.525864in}}%
\pgfpathcurveto{\pgfqpoint{6.928672in}{0.518050in}}{\pgfqpoint{6.939271in}{0.513660in}}{\pgfqpoint{6.950321in}{0.513660in}}%
\pgfusepath{stroke}%
\end{pgfscope}%
\begin{pgfscope}%
\pgfpathrectangle{\pgfqpoint{0.847223in}{0.554012in}}{\pgfqpoint{6.200000in}{4.530000in}}%
\pgfusepath{clip}%
\pgfsetbuttcap%
\pgfsetroundjoin%
\pgfsetlinewidth{1.003750pt}%
\definecolor{currentstroke}{rgb}{1.000000,0.000000,0.000000}%
\pgfsetstrokecolor{currentstroke}%
\pgfsetdash{}{0pt}%
\pgfpathmoveto{\pgfqpoint{6.959130in}{0.513534in}}%
\pgfpathcurveto{\pgfqpoint{6.970181in}{0.513534in}}{\pgfqpoint{6.980780in}{0.517924in}}{\pgfqpoint{6.988593in}{0.525738in}}%
\pgfpathcurveto{\pgfqpoint{6.996407in}{0.533551in}}{\pgfqpoint{7.000797in}{0.544150in}}{\pgfqpoint{7.000797in}{0.555201in}}%
\pgfpathcurveto{\pgfqpoint{7.000797in}{0.566251in}}{\pgfqpoint{6.996407in}{0.576850in}}{\pgfqpoint{6.988593in}{0.584663in}}%
\pgfpathcurveto{\pgfqpoint{6.980780in}{0.592477in}}{\pgfqpoint{6.970181in}{0.596867in}}{\pgfqpoint{6.959130in}{0.596867in}}%
\pgfpathcurveto{\pgfqpoint{6.948080in}{0.596867in}}{\pgfqpoint{6.937481in}{0.592477in}}{\pgfqpoint{6.929668in}{0.584663in}}%
\pgfpathcurveto{\pgfqpoint{6.921854in}{0.576850in}}{\pgfqpoint{6.917464in}{0.566251in}}{\pgfqpoint{6.917464in}{0.555201in}}%
\pgfpathcurveto{\pgfqpoint{6.917464in}{0.544150in}}{\pgfqpoint{6.921854in}{0.533551in}}{\pgfqpoint{6.929668in}{0.525738in}}%
\pgfpathcurveto{\pgfqpoint{6.937481in}{0.517924in}}{\pgfqpoint{6.948080in}{0.513534in}}{\pgfqpoint{6.959130in}{0.513534in}}%
\pgfusepath{stroke}%
\end{pgfscope}%
\begin{pgfscope}%
\pgfpathrectangle{\pgfqpoint{0.847223in}{0.554012in}}{\pgfqpoint{6.200000in}{4.530000in}}%
\pgfusepath{clip}%
\pgfsetbuttcap%
\pgfsetroundjoin%
\pgfsetlinewidth{1.003750pt}%
\definecolor{currentstroke}{rgb}{1.000000,0.000000,0.000000}%
\pgfsetstrokecolor{currentstroke}%
\pgfsetdash{}{0pt}%
\pgfpathmoveto{\pgfqpoint{6.967940in}{0.513409in}}%
\pgfpathcurveto{\pgfqpoint{6.978990in}{0.513409in}}{\pgfqpoint{6.989589in}{0.517800in}}{\pgfqpoint{6.997402in}{0.525613in}}%
\pgfpathcurveto{\pgfqpoint{7.005216in}{0.533427in}}{\pgfqpoint{7.009606in}{0.544026in}}{\pgfqpoint{7.009606in}{0.555076in}}%
\pgfpathcurveto{\pgfqpoint{7.009606in}{0.566126in}}{\pgfqpoint{7.005216in}{0.576725in}}{\pgfqpoint{6.997402in}{0.584539in}}%
\pgfpathcurveto{\pgfqpoint{6.989589in}{0.592352in}}{\pgfqpoint{6.978990in}{0.596743in}}{\pgfqpoint{6.967940in}{0.596743in}}%
\pgfpathcurveto{\pgfqpoint{6.956890in}{0.596743in}}{\pgfqpoint{6.946290in}{0.592352in}}{\pgfqpoint{6.938477in}{0.584539in}}%
\pgfpathcurveto{\pgfqpoint{6.930663in}{0.576725in}}{\pgfqpoint{6.926273in}{0.566126in}}{\pgfqpoint{6.926273in}{0.555076in}}%
\pgfpathcurveto{\pgfqpoint{6.926273in}{0.544026in}}{\pgfqpoint{6.930663in}{0.533427in}}{\pgfqpoint{6.938477in}{0.525613in}}%
\pgfpathcurveto{\pgfqpoint{6.946290in}{0.517800in}}{\pgfqpoint{6.956890in}{0.513409in}}{\pgfqpoint{6.967940in}{0.513409in}}%
\pgfusepath{stroke}%
\end{pgfscope}%
\begin{pgfscope}%
\pgfpathrectangle{\pgfqpoint{0.847223in}{0.554012in}}{\pgfqpoint{6.200000in}{4.530000in}}%
\pgfusepath{clip}%
\pgfsetbuttcap%
\pgfsetroundjoin%
\pgfsetlinewidth{1.003750pt}%
\definecolor{currentstroke}{rgb}{1.000000,0.000000,0.000000}%
\pgfsetstrokecolor{currentstroke}%
\pgfsetdash{}{0pt}%
\pgfpathmoveto{\pgfqpoint{6.976749in}{0.513286in}}%
\pgfpathcurveto{\pgfqpoint{6.987799in}{0.513286in}}{\pgfqpoint{6.998398in}{0.517676in}}{\pgfqpoint{7.006212in}{0.525490in}}%
\pgfpathcurveto{\pgfqpoint{7.014025in}{0.533304in}}{\pgfqpoint{7.018416in}{0.543903in}}{\pgfqpoint{7.018416in}{0.554953in}}%
\pgfpathcurveto{\pgfqpoint{7.018416in}{0.566003in}}{\pgfqpoint{7.014025in}{0.576602in}}{\pgfqpoint{7.006212in}{0.584415in}}%
\pgfpathcurveto{\pgfqpoint{6.998398in}{0.592229in}}{\pgfqpoint{6.987799in}{0.596619in}}{\pgfqpoint{6.976749in}{0.596619in}}%
\pgfpathcurveto{\pgfqpoint{6.965699in}{0.596619in}}{\pgfqpoint{6.955100in}{0.592229in}}{\pgfqpoint{6.947286in}{0.584415in}}%
\pgfpathcurveto{\pgfqpoint{6.939473in}{0.576602in}}{\pgfqpoint{6.935082in}{0.566003in}}{\pgfqpoint{6.935082in}{0.554953in}}%
\pgfpathcurveto{\pgfqpoint{6.935082in}{0.543903in}}{\pgfqpoint{6.939473in}{0.533304in}}{\pgfqpoint{6.947286in}{0.525490in}}%
\pgfpathcurveto{\pgfqpoint{6.955100in}{0.517676in}}{\pgfqpoint{6.965699in}{0.513286in}}{\pgfqpoint{6.976749in}{0.513286in}}%
\pgfusepath{stroke}%
\end{pgfscope}%
\begin{pgfscope}%
\pgfpathrectangle{\pgfqpoint{0.847223in}{0.554012in}}{\pgfqpoint{6.200000in}{4.530000in}}%
\pgfusepath{clip}%
\pgfsetbuttcap%
\pgfsetroundjoin%
\pgfsetlinewidth{1.003750pt}%
\definecolor{currentstroke}{rgb}{1.000000,0.000000,0.000000}%
\pgfsetstrokecolor{currentstroke}%
\pgfsetdash{}{0pt}%
\pgfpathmoveto{\pgfqpoint{6.985558in}{0.513164in}}%
\pgfpathcurveto{\pgfqpoint{6.996608in}{0.513164in}}{\pgfqpoint{7.007207in}{0.517554in}}{\pgfqpoint{7.015021in}{0.525368in}}%
\pgfpathcurveto{\pgfqpoint{7.022835in}{0.533182in}}{\pgfqpoint{7.027225in}{0.543781in}}{\pgfqpoint{7.027225in}{0.554831in}}%
\pgfpathcurveto{\pgfqpoint{7.027225in}{0.565881in}}{\pgfqpoint{7.022835in}{0.576480in}}{\pgfqpoint{7.015021in}{0.584293in}}%
\pgfpathcurveto{\pgfqpoint{7.007207in}{0.592107in}}{\pgfqpoint{6.996608in}{0.596497in}}{\pgfqpoint{6.985558in}{0.596497in}}%
\pgfpathcurveto{\pgfqpoint{6.974508in}{0.596497in}}{\pgfqpoint{6.963909in}{0.592107in}}{\pgfqpoint{6.956095in}{0.584293in}}%
\pgfpathcurveto{\pgfqpoint{6.948282in}{0.576480in}}{\pgfqpoint{6.943892in}{0.565881in}}{\pgfqpoint{6.943892in}{0.554831in}}%
\pgfpathcurveto{\pgfqpoint{6.943892in}{0.543781in}}{\pgfqpoint{6.948282in}{0.533182in}}{\pgfqpoint{6.956095in}{0.525368in}}%
\pgfpathcurveto{\pgfqpoint{6.963909in}{0.517554in}}{\pgfqpoint{6.974508in}{0.513164in}}{\pgfqpoint{6.985558in}{0.513164in}}%
\pgfusepath{stroke}%
\end{pgfscope}%
\begin{pgfscope}%
\pgfpathrectangle{\pgfqpoint{0.847223in}{0.554012in}}{\pgfqpoint{6.200000in}{4.530000in}}%
\pgfusepath{clip}%
\pgfsetbuttcap%
\pgfsetroundjoin%
\pgfsetlinewidth{1.003750pt}%
\definecolor{currentstroke}{rgb}{1.000000,0.000000,0.000000}%
\pgfsetstrokecolor{currentstroke}%
\pgfsetdash{}{0pt}%
\pgfpathmoveto{\pgfqpoint{6.994367in}{0.513043in}}%
\pgfpathcurveto{\pgfqpoint{7.005418in}{0.513043in}}{\pgfqpoint{7.016017in}{0.517434in}}{\pgfqpoint{7.023830in}{0.525247in}}%
\pgfpathcurveto{\pgfqpoint{7.031644in}{0.533061in}}{\pgfqpoint{7.036034in}{0.543660in}}{\pgfqpoint{7.036034in}{0.554710in}}%
\pgfpathcurveto{\pgfqpoint{7.036034in}{0.565760in}}{\pgfqpoint{7.031644in}{0.576359in}}{\pgfqpoint{7.023830in}{0.584173in}}%
\pgfpathcurveto{\pgfqpoint{7.016017in}{0.591986in}}{\pgfqpoint{7.005418in}{0.596377in}}{\pgfqpoint{6.994367in}{0.596377in}}%
\pgfpathcurveto{\pgfqpoint{6.983317in}{0.596377in}}{\pgfqpoint{6.972718in}{0.591986in}}{\pgfqpoint{6.964905in}{0.584173in}}%
\pgfpathcurveto{\pgfqpoint{6.957091in}{0.576359in}}{\pgfqpoint{6.952701in}{0.565760in}}{\pgfqpoint{6.952701in}{0.554710in}}%
\pgfpathcurveto{\pgfqpoint{6.952701in}{0.543660in}}{\pgfqpoint{6.957091in}{0.533061in}}{\pgfqpoint{6.964905in}{0.525247in}}%
\pgfpathcurveto{\pgfqpoint{6.972718in}{0.517434in}}{\pgfqpoint{6.983317in}{0.513043in}}{\pgfqpoint{6.994367in}{0.513043in}}%
\pgfusepath{stroke}%
\end{pgfscope}%
\begin{pgfscope}%
\pgfpathrectangle{\pgfqpoint{0.847223in}{0.554012in}}{\pgfqpoint{6.200000in}{4.530000in}}%
\pgfusepath{clip}%
\pgfsetbuttcap%
\pgfsetroundjoin%
\pgfsetlinewidth{1.003750pt}%
\definecolor{currentstroke}{rgb}{1.000000,0.000000,0.000000}%
\pgfsetstrokecolor{currentstroke}%
\pgfsetdash{}{0pt}%
\pgfpathmoveto{\pgfqpoint{7.003177in}{0.512924in}}%
\pgfpathcurveto{\pgfqpoint{7.014227in}{0.512924in}}{\pgfqpoint{7.024826in}{0.517314in}}{\pgfqpoint{7.032639in}{0.525128in}}%
\pgfpathcurveto{\pgfqpoint{7.040453in}{0.532941in}}{\pgfqpoint{7.044843in}{0.543540in}}{\pgfqpoint{7.044843in}{0.554591in}}%
\pgfpathcurveto{\pgfqpoint{7.044843in}{0.565641in}}{\pgfqpoint{7.040453in}{0.576240in}}{\pgfqpoint{7.032639in}{0.584053in}}%
\pgfpathcurveto{\pgfqpoint{7.024826in}{0.591867in}}{\pgfqpoint{7.014227in}{0.596257in}}{\pgfqpoint{7.003177in}{0.596257in}}%
\pgfpathcurveto{\pgfqpoint{6.992127in}{0.596257in}}{\pgfqpoint{6.981528in}{0.591867in}}{\pgfqpoint{6.973714in}{0.584053in}}%
\pgfpathcurveto{\pgfqpoint{6.965900in}{0.576240in}}{\pgfqpoint{6.961510in}{0.565641in}}{\pgfqpoint{6.961510in}{0.554591in}}%
\pgfpathcurveto{\pgfqpoint{6.961510in}{0.543540in}}{\pgfqpoint{6.965900in}{0.532941in}}{\pgfqpoint{6.973714in}{0.525128in}}%
\pgfpathcurveto{\pgfqpoint{6.981528in}{0.517314in}}{\pgfqpoint{6.992127in}{0.512924in}}{\pgfqpoint{7.003177in}{0.512924in}}%
\pgfusepath{stroke}%
\end{pgfscope}%
\begin{pgfscope}%
\pgfpathrectangle{\pgfqpoint{0.847223in}{0.554012in}}{\pgfqpoint{6.200000in}{4.530000in}}%
\pgfusepath{clip}%
\pgfsetbuttcap%
\pgfsetroundjoin%
\pgfsetlinewidth{1.003750pt}%
\definecolor{currentstroke}{rgb}{1.000000,0.000000,0.000000}%
\pgfsetstrokecolor{currentstroke}%
\pgfsetdash{}{0pt}%
\pgfpathmoveto{\pgfqpoint{7.011986in}{0.512806in}}%
\pgfpathcurveto{\pgfqpoint{7.023036in}{0.512806in}}{\pgfqpoint{7.033635in}{0.517196in}}{\pgfqpoint{7.041449in}{0.525010in}}%
\pgfpathcurveto{\pgfqpoint{7.049262in}{0.532823in}}{\pgfqpoint{7.053653in}{0.543422in}}{\pgfqpoint{7.053653in}{0.554472in}}%
\pgfpathcurveto{\pgfqpoint{7.053653in}{0.565523in}}{\pgfqpoint{7.049262in}{0.576122in}}{\pgfqpoint{7.041449in}{0.583935in}}%
\pgfpathcurveto{\pgfqpoint{7.033635in}{0.591749in}}{\pgfqpoint{7.023036in}{0.596139in}}{\pgfqpoint{7.011986in}{0.596139in}}%
\pgfpathcurveto{\pgfqpoint{7.000936in}{0.596139in}}{\pgfqpoint{6.990337in}{0.591749in}}{\pgfqpoint{6.982523in}{0.583935in}}%
\pgfpathcurveto{\pgfqpoint{6.974710in}{0.576122in}}{\pgfqpoint{6.970319in}{0.565523in}}{\pgfqpoint{6.970319in}{0.554472in}}%
\pgfpathcurveto{\pgfqpoint{6.970319in}{0.543422in}}{\pgfqpoint{6.974710in}{0.532823in}}{\pgfqpoint{6.982523in}{0.525010in}}%
\pgfpathcurveto{\pgfqpoint{6.990337in}{0.517196in}}{\pgfqpoint{7.000936in}{0.512806in}}{\pgfqpoint{7.011986in}{0.512806in}}%
\pgfusepath{stroke}%
\end{pgfscope}%
\begin{pgfscope}%
\pgfpathrectangle{\pgfqpoint{0.847223in}{0.554012in}}{\pgfqpoint{6.200000in}{4.530000in}}%
\pgfusepath{clip}%
\pgfsetbuttcap%
\pgfsetroundjoin%
\pgfsetlinewidth{1.003750pt}%
\definecolor{currentstroke}{rgb}{1.000000,0.000000,0.000000}%
\pgfsetstrokecolor{currentstroke}%
\pgfsetdash{}{0pt}%
\pgfpathmoveto{\pgfqpoint{7.020795in}{0.512689in}}%
\pgfpathcurveto{\pgfqpoint{7.031845in}{0.512689in}}{\pgfqpoint{7.042444in}{0.517079in}}{\pgfqpoint{7.050258in}{0.524893in}}%
\pgfpathcurveto{\pgfqpoint{7.058072in}{0.532706in}}{\pgfqpoint{7.062462in}{0.543305in}}{\pgfqpoint{7.062462in}{0.554355in}}%
\pgfpathcurveto{\pgfqpoint{7.062462in}{0.565406in}}{\pgfqpoint{7.058072in}{0.576005in}}{\pgfqpoint{7.050258in}{0.583818in}}%
\pgfpathcurveto{\pgfqpoint{7.042444in}{0.591632in}}{\pgfqpoint{7.031845in}{0.596022in}}{\pgfqpoint{7.020795in}{0.596022in}}%
\pgfpathcurveto{\pgfqpoint{7.009745in}{0.596022in}}{\pgfqpoint{6.999146in}{0.591632in}}{\pgfqpoint{6.991332in}{0.583818in}}%
\pgfpathcurveto{\pgfqpoint{6.983519in}{0.576005in}}{\pgfqpoint{6.979129in}{0.565406in}}{\pgfqpoint{6.979129in}{0.554355in}}%
\pgfpathcurveto{\pgfqpoint{6.979129in}{0.543305in}}{\pgfqpoint{6.983519in}{0.532706in}}{\pgfqpoint{6.991332in}{0.524893in}}%
\pgfpathcurveto{\pgfqpoint{6.999146in}{0.517079in}}{\pgfqpoint{7.009745in}{0.512689in}}{\pgfqpoint{7.020795in}{0.512689in}}%
\pgfusepath{stroke}%
\end{pgfscope}%
\begin{pgfscope}%
\pgfpathrectangle{\pgfqpoint{0.847223in}{0.554012in}}{\pgfqpoint{6.200000in}{4.530000in}}%
\pgfusepath{clip}%
\pgfsetbuttcap%
\pgfsetroundjoin%
\pgfsetlinewidth{1.003750pt}%
\definecolor{currentstroke}{rgb}{1.000000,0.000000,0.000000}%
\pgfsetstrokecolor{currentstroke}%
\pgfsetdash{}{0pt}%
\pgfpathmoveto{\pgfqpoint{7.029605in}{0.512573in}}%
\pgfpathcurveto{\pgfqpoint{7.040655in}{0.512573in}}{\pgfqpoint{7.051254in}{0.516963in}}{\pgfqpoint{7.059067in}{0.524777in}}%
\pgfpathcurveto{\pgfqpoint{7.066881in}{0.532591in}}{\pgfqpoint{7.071271in}{0.543190in}}{\pgfqpoint{7.071271in}{0.554240in}}%
\pgfpathcurveto{\pgfqpoint{7.071271in}{0.565290in}}{\pgfqpoint{7.066881in}{0.575889in}}{\pgfqpoint{7.059067in}{0.583703in}}%
\pgfpathcurveto{\pgfqpoint{7.051254in}{0.591516in}}{\pgfqpoint{7.040655in}{0.595906in}}{\pgfqpoint{7.029605in}{0.595906in}}%
\pgfpathcurveto{\pgfqpoint{7.018554in}{0.595906in}}{\pgfqpoint{7.007955in}{0.591516in}}{\pgfqpoint{7.000142in}{0.583703in}}%
\pgfpathcurveto{\pgfqpoint{6.992328in}{0.575889in}}{\pgfqpoint{6.987938in}{0.565290in}}{\pgfqpoint{6.987938in}{0.554240in}}%
\pgfpathcurveto{\pgfqpoint{6.987938in}{0.543190in}}{\pgfqpoint{6.992328in}{0.532591in}}{\pgfqpoint{7.000142in}{0.524777in}}%
\pgfpathcurveto{\pgfqpoint{7.007955in}{0.516963in}}{\pgfqpoint{7.018554in}{0.512573in}}{\pgfqpoint{7.029605in}{0.512573in}}%
\pgfusepath{stroke}%
\end{pgfscope}%
\begin{pgfscope}%
\pgfpathrectangle{\pgfqpoint{0.847223in}{0.554012in}}{\pgfqpoint{6.200000in}{4.530000in}}%
\pgfusepath{clip}%
\pgfsetbuttcap%
\pgfsetroundjoin%
\pgfsetlinewidth{1.003750pt}%
\definecolor{currentstroke}{rgb}{1.000000,0.000000,0.000000}%
\pgfsetstrokecolor{currentstroke}%
\pgfsetdash{}{0pt}%
\pgfpathmoveto{\pgfqpoint{7.038414in}{0.512459in}}%
\pgfpathcurveto{\pgfqpoint{7.049464in}{0.512459in}}{\pgfqpoint{7.060063in}{0.516849in}}{\pgfqpoint{7.067877in}{0.524662in}}%
\pgfpathcurveto{\pgfqpoint{7.075690in}{0.532476in}}{\pgfqpoint{7.080080in}{0.543075in}}{\pgfqpoint{7.080080in}{0.554125in}}%
\pgfpathcurveto{\pgfqpoint{7.080080in}{0.565175in}}{\pgfqpoint{7.075690in}{0.575774in}}{\pgfqpoint{7.067877in}{0.583588in}}%
\pgfpathcurveto{\pgfqpoint{7.060063in}{0.591402in}}{\pgfqpoint{7.049464in}{0.595792in}}{\pgfqpoint{7.038414in}{0.595792in}}%
\pgfpathcurveto{\pgfqpoint{7.027364in}{0.595792in}}{\pgfqpoint{7.016765in}{0.591402in}}{\pgfqpoint{7.008951in}{0.583588in}}%
\pgfpathcurveto{\pgfqpoint{7.001137in}{0.575774in}}{\pgfqpoint{6.996747in}{0.565175in}}{\pgfqpoint{6.996747in}{0.554125in}}%
\pgfpathcurveto{\pgfqpoint{6.996747in}{0.543075in}}{\pgfqpoint{7.001137in}{0.532476in}}{\pgfqpoint{7.008951in}{0.524662in}}%
\pgfpathcurveto{\pgfqpoint{7.016765in}{0.516849in}}{\pgfqpoint{7.027364in}{0.512459in}}{\pgfqpoint{7.038414in}{0.512459in}}%
\pgfusepath{stroke}%
\end{pgfscope}%
\begin{pgfscope}%
\pgfpathrectangle{\pgfqpoint{0.847223in}{0.554012in}}{\pgfqpoint{6.200000in}{4.530000in}}%
\pgfusepath{clip}%
\pgfsetbuttcap%
\pgfsetroundjoin%
\pgfsetlinewidth{1.003750pt}%
\definecolor{currentstroke}{rgb}{1.000000,0.000000,0.000000}%
\pgfsetstrokecolor{currentstroke}%
\pgfsetdash{}{0pt}%
\pgfpathmoveto{\pgfqpoint{7.047223in}{0.512345in}}%
\pgfpathcurveto{\pgfqpoint{7.058273in}{0.512345in}}{\pgfqpoint{7.068872in}{0.516735in}}{\pgfqpoint{7.076686in}{0.524549in}}%
\pgfpathcurveto{\pgfqpoint{7.084499in}{0.532363in}}{\pgfqpoint{7.088890in}{0.542962in}}{\pgfqpoint{7.088890in}{0.554012in}}%
\pgfpathcurveto{\pgfqpoint{7.088890in}{0.565062in}}{\pgfqpoint{7.084499in}{0.575661in}}{\pgfqpoint{7.076686in}{0.583475in}}%
\pgfpathcurveto{\pgfqpoint{7.068872in}{0.591288in}}{\pgfqpoint{7.058273in}{0.595678in}}{\pgfqpoint{7.047223in}{0.595678in}}%
\pgfpathcurveto{\pgfqpoint{7.036173in}{0.595678in}}{\pgfqpoint{7.025574in}{0.591288in}}{\pgfqpoint{7.017760in}{0.583475in}}%
\pgfpathcurveto{\pgfqpoint{7.009947in}{0.575661in}}{\pgfqpoint{7.005556in}{0.565062in}}{\pgfqpoint{7.005556in}{0.554012in}}%
\pgfpathcurveto{\pgfqpoint{7.005556in}{0.542962in}}{\pgfqpoint{7.009947in}{0.532363in}}{\pgfqpoint{7.017760in}{0.524549in}}%
\pgfpathcurveto{\pgfqpoint{7.025574in}{0.516735in}}{\pgfqpoint{7.036173in}{0.512345in}}{\pgfqpoint{7.047223in}{0.512345in}}%
\pgfusepath{stroke}%
\end{pgfscope}%
\begin{pgfscope}%
\pgfpathrectangle{\pgfqpoint{0.847223in}{0.554012in}}{\pgfqpoint{6.200000in}{4.530000in}}%
\pgfusepath{clip}%
\pgfsetbuttcap%
\pgfsetroundjoin%
\definecolor{currentfill}{rgb}{0.501961,0.501961,0.501961}%
\pgfsetfillcolor{currentfill}%
\pgfsetfillopacity{0.200000}%
\pgfsetlinewidth{1.003750pt}%
\definecolor{currentstroke}{rgb}{0.501961,0.501961,0.501961}%
\pgfsetstrokecolor{currentstroke}%
\pgfsetstrokeopacity{0.200000}%
\pgfsetdash{}{0pt}%
\pgfsys@defobject{currentmarker}{\pgfqpoint{0.847223in}{0.136829in}}{\pgfqpoint{7.047223in}{5.084012in}}{%
\pgfpathmoveto{\pgfqpoint{0.847223in}{0.136829in}}%
\pgfpathlineto{\pgfqpoint{0.847223in}{5.084012in}}%
\pgfpathlineto{\pgfqpoint{0.852556in}{5.034106in}}%
\pgfpathlineto{\pgfqpoint{0.857889in}{4.985197in}}%
\pgfpathlineto{\pgfqpoint{0.863223in}{4.937255in}}%
\pgfpathlineto{\pgfqpoint{0.868556in}{4.890252in}}%
\pgfpathlineto{\pgfqpoint{0.873889in}{4.844161in}}%
\pgfpathlineto{\pgfqpoint{0.879222in}{4.798955in}}%
\pgfpathlineto{\pgfqpoint{0.884556in}{4.754608in}}%
\pgfpathlineto{\pgfqpoint{0.889889in}{4.711098in}}%
\pgfpathlineto{\pgfqpoint{0.895222in}{4.668400in}}%
\pgfpathlineto{\pgfqpoint{0.900555in}{4.626491in}}%
\pgfpathlineto{\pgfqpoint{0.905888in}{4.585351in}}%
\pgfpathlineto{\pgfqpoint{0.911222in}{4.544958in}}%
\pgfpathlineto{\pgfqpoint{0.916555in}{4.505291in}}%
\pgfpathlineto{\pgfqpoint{0.921888in}{4.466333in}}%
\pgfpathlineto{\pgfqpoint{0.927221in}{4.428063in}}%
\pgfpathlineto{\pgfqpoint{0.932555in}{4.390463in}}%
\pgfpathlineto{\pgfqpoint{0.937888in}{4.353517in}}%
\pgfpathlineto{\pgfqpoint{0.943221in}{4.317207in}}%
\pgfpathlineto{\pgfqpoint{0.948554in}{4.281517in}}%
\pgfpathlineto{\pgfqpoint{0.953887in}{4.246431in}}%
\pgfpathlineto{\pgfqpoint{0.959221in}{4.211935in}}%
\pgfpathlineto{\pgfqpoint{0.964554in}{4.178012in}}%
\pgfpathlineto{\pgfqpoint{0.969887in}{4.144650in}}%
\pgfpathlineto{\pgfqpoint{0.975220in}{4.111834in}}%
\pgfpathlineto{\pgfqpoint{0.980553in}{4.079551in}}%
\pgfpathlineto{\pgfqpoint{0.985887in}{4.047788in}}%
\pgfpathlineto{\pgfqpoint{0.991220in}{4.016533in}}%
\pgfpathlineto{\pgfqpoint{0.996553in}{3.985774in}}%
\pgfpathlineto{\pgfqpoint{1.001886in}{3.955498in}}%
\pgfpathlineto{\pgfqpoint{1.007220in}{3.925695in}}%
\pgfpathlineto{\pgfqpoint{1.012553in}{3.896354in}}%
\pgfpathlineto{\pgfqpoint{1.017886in}{3.867463in}}%
\pgfpathlineto{\pgfqpoint{1.023219in}{3.839013in}}%
\pgfpathlineto{\pgfqpoint{1.028552in}{3.810994in}}%
\pgfpathlineto{\pgfqpoint{1.033886in}{3.783396in}}%
\pgfpathlineto{\pgfqpoint{1.039219in}{3.756209in}}%
\pgfpathlineto{\pgfqpoint{1.044552in}{3.729425in}}%
\pgfpathlineto{\pgfqpoint{1.049885in}{3.703034in}}%
\pgfpathlineto{\pgfqpoint{1.055218in}{3.677028in}}%
\pgfpathlineto{\pgfqpoint{1.060552in}{3.651399in}}%
\pgfpathlineto{\pgfqpoint{1.065885in}{3.626138in}}%
\pgfpathlineto{\pgfqpoint{1.071218in}{3.601237in}}%
\pgfpathlineto{\pgfqpoint{1.076551in}{3.576690in}}%
\pgfpathlineto{\pgfqpoint{1.081885in}{3.552487in}}%
\pgfpathlineto{\pgfqpoint{1.087218in}{3.528623in}}%
\pgfpathlineto{\pgfqpoint{1.092551in}{3.505091in}}%
\pgfpathlineto{\pgfqpoint{1.097884in}{3.481882in}}%
\pgfpathlineto{\pgfqpoint{1.103217in}{3.458991in}}%
\pgfpathlineto{\pgfqpoint{1.108551in}{3.436411in}}%
\pgfpathlineto{\pgfqpoint{1.113884in}{3.414137in}}%
\pgfpathlineto{\pgfqpoint{1.119217in}{3.392160in}}%
\pgfpathlineto{\pgfqpoint{1.124550in}{3.370477in}}%
\pgfpathlineto{\pgfqpoint{1.129883in}{3.349081in}}%
\pgfpathlineto{\pgfqpoint{1.135217in}{3.327965in}}%
\pgfpathlineto{\pgfqpoint{1.140550in}{3.307126in}}%
\pgfpathlineto{\pgfqpoint{1.145883in}{3.286557in}}%
\pgfpathlineto{\pgfqpoint{1.151216in}{3.266253in}}%
\pgfpathlineto{\pgfqpoint{1.156550in}{3.246210in}}%
\pgfpathlineto{\pgfqpoint{1.161883in}{3.226421in}}%
\pgfpathlineto{\pgfqpoint{1.167216in}{3.206883in}}%
\pgfpathlineto{\pgfqpoint{1.172549in}{3.187590in}}%
\pgfpathlineto{\pgfqpoint{1.177882in}{3.168538in}}%
\pgfpathlineto{\pgfqpoint{1.183216in}{3.149723in}}%
\pgfpathlineto{\pgfqpoint{1.188549in}{3.131140in}}%
\pgfpathlineto{\pgfqpoint{1.193882in}{3.112785in}}%
\pgfpathlineto{\pgfqpoint{1.199215in}{3.094653in}}%
\pgfpathlineto{\pgfqpoint{1.204549in}{3.076741in}}%
\pgfpathlineto{\pgfqpoint{1.209882in}{3.059045in}}%
\pgfpathlineto{\pgfqpoint{1.215215in}{3.041560in}}%
\pgfpathlineto{\pgfqpoint{1.220548in}{3.024283in}}%
\pgfpathlineto{\pgfqpoint{1.225881in}{3.007211in}}%
\pgfpathlineto{\pgfqpoint{1.231215in}{2.990339in}}%
\pgfpathlineto{\pgfqpoint{1.236548in}{2.973665in}}%
\pgfpathlineto{\pgfqpoint{1.241881in}{2.957184in}}%
\pgfpathlineto{\pgfqpoint{1.247214in}{2.940894in}}%
\pgfpathlineto{\pgfqpoint{1.252547in}{2.924790in}}%
\pgfpathlineto{\pgfqpoint{1.257881in}{2.908871in}}%
\pgfpathlineto{\pgfqpoint{1.263214in}{2.893132in}}%
\pgfpathlineto{\pgfqpoint{1.268547in}{2.877572in}}%
\pgfpathlineto{\pgfqpoint{1.273880in}{2.862185in}}%
\pgfpathlineto{\pgfqpoint{1.279214in}{2.846971in}}%
\pgfpathlineto{\pgfqpoint{1.284547in}{2.831926in}}%
\pgfpathlineto{\pgfqpoint{1.289880in}{2.817046in}}%
\pgfpathlineto{\pgfqpoint{1.295213in}{2.802330in}}%
\pgfpathlineto{\pgfqpoint{1.300546in}{2.787775in}}%
\pgfpathlineto{\pgfqpoint{1.305880in}{2.773378in}}%
\pgfpathlineto{\pgfqpoint{1.311213in}{2.759136in}}%
\pgfpathlineto{\pgfqpoint{1.316546in}{2.745048in}}%
\pgfpathlineto{\pgfqpoint{1.321879in}{2.731110in}}%
\pgfpathlineto{\pgfqpoint{1.327212in}{2.717320in}}%
\pgfpathlineto{\pgfqpoint{1.332546in}{2.703676in}}%
\pgfpathlineto{\pgfqpoint{1.337879in}{2.690176in}}%
\pgfpathlineto{\pgfqpoint{1.343212in}{2.676816in}}%
\pgfpathlineto{\pgfqpoint{1.348545in}{2.663596in}}%
\pgfpathlineto{\pgfqpoint{1.353879in}{2.650513in}}%
\pgfpathlineto{\pgfqpoint{1.359212in}{2.637565in}}%
\pgfpathlineto{\pgfqpoint{1.364545in}{2.624749in}}%
\pgfpathlineto{\pgfqpoint{1.369878in}{2.612064in}}%
\pgfpathlineto{\pgfqpoint{1.375211in}{2.599507in}}%
\pgfpathlineto{\pgfqpoint{1.380545in}{2.587078in}}%
\pgfpathlineto{\pgfqpoint{1.385878in}{2.574773in}}%
\pgfpathlineto{\pgfqpoint{1.391211in}{2.562591in}}%
\pgfpathlineto{\pgfqpoint{1.396544in}{2.550531in}}%
\pgfpathlineto{\pgfqpoint{1.401877in}{2.538589in}}%
\pgfpathlineto{\pgfqpoint{1.407211in}{2.526766in}}%
\pgfpathlineto{\pgfqpoint{1.412544in}{2.515058in}}%
\pgfpathlineto{\pgfqpoint{1.417877in}{2.503464in}}%
\pgfpathlineto{\pgfqpoint{1.423210in}{2.491983in}}%
\pgfpathlineto{\pgfqpoint{1.428544in}{2.480612in}}%
\pgfpathlineto{\pgfqpoint{1.433877in}{2.469351in}}%
\pgfpathlineto{\pgfqpoint{1.439210in}{2.458198in}}%
\pgfpathlineto{\pgfqpoint{1.444543in}{2.447150in}}%
\pgfpathlineto{\pgfqpoint{1.449876in}{2.436208in}}%
\pgfpathlineto{\pgfqpoint{1.455210in}{2.425368in}}%
\pgfpathlineto{\pgfqpoint{1.460543in}{2.414631in}}%
\pgfpathlineto{\pgfqpoint{1.465876in}{2.403993in}}%
\pgfpathlineto{\pgfqpoint{1.471209in}{2.393455in}}%
\pgfpathlineto{\pgfqpoint{1.476543in}{2.383014in}}%
\pgfpathlineto{\pgfqpoint{1.481876in}{2.372669in}}%
\pgfpathlineto{\pgfqpoint{1.487209in}{2.362419in}}%
\pgfpathlineto{\pgfqpoint{1.492542in}{2.352262in}}%
\pgfpathlineto{\pgfqpoint{1.497875in}{2.342198in}}%
\pgfpathlineto{\pgfqpoint{1.503209in}{2.332225in}}%
\pgfpathlineto{\pgfqpoint{1.508542in}{2.322342in}}%
\pgfpathlineto{\pgfqpoint{1.513875in}{2.312547in}}%
\pgfpathlineto{\pgfqpoint{1.519208in}{2.302839in}}%
\pgfpathlineto{\pgfqpoint{1.524541in}{2.293218in}}%
\pgfpathlineto{\pgfqpoint{1.529875in}{2.283682in}}%
\pgfpathlineto{\pgfqpoint{1.535208in}{2.274230in}}%
\pgfpathlineto{\pgfqpoint{1.540541in}{2.264861in}}%
\pgfpathlineto{\pgfqpoint{1.545874in}{2.255573in}}%
\pgfpathlineto{\pgfqpoint{1.551208in}{2.246367in}}%
\pgfpathlineto{\pgfqpoint{1.556541in}{2.237240in}}%
\pgfpathlineto{\pgfqpoint{1.561874in}{2.228191in}}%
\pgfpathlineto{\pgfqpoint{1.567207in}{2.219220in}}%
\pgfpathlineto{\pgfqpoint{1.572540in}{2.210326in}}%
\pgfpathlineto{\pgfqpoint{1.577874in}{2.201508in}}%
\pgfpathlineto{\pgfqpoint{1.583207in}{2.192764in}}%
\pgfpathlineto{\pgfqpoint{1.588540in}{2.184094in}}%
\pgfpathlineto{\pgfqpoint{1.593873in}{2.175497in}}%
\pgfpathlineto{\pgfqpoint{1.599206in}{2.166971in}}%
\pgfpathlineto{\pgfqpoint{1.604540in}{2.158517in}}%
\pgfpathlineto{\pgfqpoint{1.609873in}{2.150133in}}%
\pgfpathlineto{\pgfqpoint{1.615206in}{2.141818in}}%
\pgfpathlineto{\pgfqpoint{1.620539in}{2.133571in}}%
\pgfpathlineto{\pgfqpoint{1.625873in}{2.125392in}}%
\pgfpathlineto{\pgfqpoint{1.631206in}{2.117280in}}%
\pgfpathlineto{\pgfqpoint{1.636539in}{2.109234in}}%
\pgfpathlineto{\pgfqpoint{1.641872in}{2.101252in}}%
\pgfpathlineto{\pgfqpoint{1.647205in}{2.093335in}}%
\pgfpathlineto{\pgfqpoint{1.652539in}{2.085482in}}%
\pgfpathlineto{\pgfqpoint{1.657872in}{2.077692in}}%
\pgfpathlineto{\pgfqpoint{1.663205in}{2.069963in}}%
\pgfpathlineto{\pgfqpoint{1.668538in}{2.062296in}}%
\pgfpathlineto{\pgfqpoint{1.673871in}{2.054689in}}%
\pgfpathlineto{\pgfqpoint{1.679205in}{2.047142in}}%
\pgfpathlineto{\pgfqpoint{1.684538in}{2.039655in}}%
\pgfpathlineto{\pgfqpoint{1.689871in}{2.032226in}}%
\pgfpathlineto{\pgfqpoint{1.695204in}{2.024854in}}%
\pgfpathlineto{\pgfqpoint{1.700538in}{2.017540in}}%
\pgfpathlineto{\pgfqpoint{1.705871in}{2.010282in}}%
\pgfpathlineto{\pgfqpoint{1.711204in}{2.003080in}}%
\pgfpathlineto{\pgfqpoint{1.716537in}{1.995934in}}%
\pgfpathlineto{\pgfqpoint{1.721870in}{1.988841in}}%
\pgfpathlineto{\pgfqpoint{1.727204in}{1.981803in}}%
\pgfpathlineto{\pgfqpoint{1.732537in}{1.974818in}}%
\pgfpathlineto{\pgfqpoint{1.737870in}{1.967885in}}%
\pgfpathlineto{\pgfqpoint{1.743203in}{1.961005in}}%
\pgfpathlineto{\pgfqpoint{1.748537in}{1.954176in}}%
\pgfpathlineto{\pgfqpoint{1.753870in}{1.947398in}}%
\pgfpathlineto{\pgfqpoint{1.759203in}{1.940671in}}%
\pgfpathlineto{\pgfqpoint{1.764536in}{1.933993in}}%
\pgfpathlineto{\pgfqpoint{1.769869in}{1.927365in}}%
\pgfpathlineto{\pgfqpoint{1.775203in}{1.920785in}}%
\pgfpathlineto{\pgfqpoint{1.780536in}{1.914254in}}%
\pgfpathlineto{\pgfqpoint{1.785869in}{1.907770in}}%
\pgfpathlineto{\pgfqpoint{1.791202in}{1.901333in}}%
\pgfpathlineto{\pgfqpoint{1.796535in}{1.894943in}}%
\pgfpathlineto{\pgfqpoint{1.801869in}{1.888599in}}%
\pgfpathlineto{\pgfqpoint{1.807202in}{1.882301in}}%
\pgfpathlineto{\pgfqpoint{1.812535in}{1.876047in}}%
\pgfpathlineto{\pgfqpoint{1.817868in}{1.869839in}}%
\pgfpathlineto{\pgfqpoint{1.823202in}{1.863674in}}%
\pgfpathlineto{\pgfqpoint{1.828535in}{1.857554in}}%
\pgfpathlineto{\pgfqpoint{1.833868in}{1.851476in}}%
\pgfpathlineto{\pgfqpoint{1.839201in}{1.845441in}}%
\pgfpathlineto{\pgfqpoint{1.844534in}{1.839449in}}%
\pgfpathlineto{\pgfqpoint{1.849868in}{1.833498in}}%
\pgfpathlineto{\pgfqpoint{1.855201in}{1.827589in}}%
\pgfpathlineto{\pgfqpoint{1.860534in}{1.821721in}}%
\pgfpathlineto{\pgfqpoint{1.865867in}{1.815894in}}%
\pgfpathlineto{\pgfqpoint{1.871200in}{1.810107in}}%
\pgfpathlineto{\pgfqpoint{1.876534in}{1.804359in}}%
\pgfpathlineto{\pgfqpoint{1.881867in}{1.798651in}}%
\pgfpathlineto{\pgfqpoint{1.887200in}{1.792982in}}%
\pgfpathlineto{\pgfqpoint{1.892533in}{1.787351in}}%
\pgfpathlineto{\pgfqpoint{1.897867in}{1.781758in}}%
\pgfpathlineto{\pgfqpoint{1.903200in}{1.776204in}}%
\pgfpathlineto{\pgfqpoint{1.908533in}{1.770686in}}%
\pgfpathlineto{\pgfqpoint{1.913866in}{1.765206in}}%
\pgfpathlineto{\pgfqpoint{1.919199in}{1.759762in}}%
\pgfpathlineto{\pgfqpoint{1.924533in}{1.754355in}}%
\pgfpathlineto{\pgfqpoint{1.929866in}{1.748983in}}%
\pgfpathlineto{\pgfqpoint{1.935199in}{1.743647in}}%
\pgfpathlineto{\pgfqpoint{1.940532in}{1.738347in}}%
\pgfpathlineto{\pgfqpoint{1.945865in}{1.733081in}}%
\pgfpathlineto{\pgfqpoint{1.951199in}{1.727849in}}%
\pgfpathlineto{\pgfqpoint{1.956532in}{1.722652in}}%
\pgfpathlineto{\pgfqpoint{1.961865in}{1.717489in}}%
\pgfpathlineto{\pgfqpoint{1.967198in}{1.712359in}}%
\pgfpathlineto{\pgfqpoint{1.972532in}{1.707262in}}%
\pgfpathlineto{\pgfqpoint{1.977865in}{1.702199in}}%
\pgfpathlineto{\pgfqpoint{1.983198in}{1.697167in}}%
\pgfpathlineto{\pgfqpoint{1.988531in}{1.692168in}}%
\pgfpathlineto{\pgfqpoint{1.993864in}{1.687201in}}%
\pgfpathlineto{\pgfqpoint{1.999198in}{1.682266in}}%
\pgfpathlineto{\pgfqpoint{2.004531in}{1.677362in}}%
\pgfpathlineto{\pgfqpoint{2.009864in}{1.672489in}}%
\pgfpathlineto{\pgfqpoint{2.015197in}{1.667646in}}%
\pgfpathlineto{\pgfqpoint{2.020531in}{1.662834in}}%
\pgfpathlineto{\pgfqpoint{2.025864in}{1.658053in}}%
\pgfpathlineto{\pgfqpoint{2.031197in}{1.653301in}}%
\pgfpathlineto{\pgfqpoint{2.036530in}{1.648578in}}%
\pgfpathlineto{\pgfqpoint{2.041863in}{1.643885in}}%
\pgfpathlineto{\pgfqpoint{2.047197in}{1.639221in}}%
\pgfpathlineto{\pgfqpoint{2.052530in}{1.634586in}}%
\pgfpathlineto{\pgfqpoint{2.057863in}{1.629980in}}%
\pgfpathlineto{\pgfqpoint{2.063196in}{1.625401in}}%
\pgfpathlineto{\pgfqpoint{2.068529in}{1.620851in}}%
\pgfpathlineto{\pgfqpoint{2.073863in}{1.616328in}}%
\pgfpathlineto{\pgfqpoint{2.079196in}{1.611833in}}%
\pgfpathlineto{\pgfqpoint{2.084529in}{1.607365in}}%
\pgfpathlineto{\pgfqpoint{2.089862in}{1.602924in}}%
\pgfpathlineto{\pgfqpoint{2.095196in}{1.598510in}}%
\pgfpathlineto{\pgfqpoint{2.100529in}{1.594122in}}%
\pgfpathlineto{\pgfqpoint{2.105862in}{1.589761in}}%
\pgfpathlineto{\pgfqpoint{2.111195in}{1.585425in}}%
\pgfpathlineto{\pgfqpoint{2.116528in}{1.581116in}}%
\pgfpathlineto{\pgfqpoint{2.121862in}{1.576832in}}%
\pgfpathlineto{\pgfqpoint{2.127195in}{1.572573in}}%
\pgfpathlineto{\pgfqpoint{2.132528in}{1.568339in}}%
\pgfpathlineto{\pgfqpoint{2.137861in}{1.564131in}}%
\pgfpathlineto{\pgfqpoint{2.143194in}{1.559947in}}%
\pgfpathlineto{\pgfqpoint{2.148528in}{1.555787in}}%
\pgfpathlineto{\pgfqpoint{2.153861in}{1.551652in}}%
\pgfpathlineto{\pgfqpoint{2.159194in}{1.547540in}}%
\pgfpathlineto{\pgfqpoint{2.164527in}{1.543453in}}%
\pgfpathlineto{\pgfqpoint{2.169861in}{1.539389in}}%
\pgfpathlineto{\pgfqpoint{2.175194in}{1.535349in}}%
\pgfpathlineto{\pgfqpoint{2.180527in}{1.531331in}}%
\pgfpathlineto{\pgfqpoint{2.185860in}{1.527337in}}%
\pgfpathlineto{\pgfqpoint{2.191193in}{1.523366in}}%
\pgfpathlineto{\pgfqpoint{2.196527in}{1.519417in}}%
\pgfpathlineto{\pgfqpoint{2.201860in}{1.515491in}}%
\pgfpathlineto{\pgfqpoint{2.207193in}{1.511586in}}%
\pgfpathlineto{\pgfqpoint{2.212526in}{1.507704in}}%
\pgfpathlineto{\pgfqpoint{2.217859in}{1.503844in}}%
\pgfpathlineto{\pgfqpoint{2.223193in}{1.500006in}}%
\pgfpathlineto{\pgfqpoint{2.228526in}{1.496189in}}%
\pgfpathlineto{\pgfqpoint{2.233859in}{1.492393in}}%
\pgfpathlineto{\pgfqpoint{2.239192in}{1.488618in}}%
\pgfpathlineto{\pgfqpoint{2.244526in}{1.484865in}}%
\pgfpathlineto{\pgfqpoint{2.249859in}{1.481132in}}%
\pgfpathlineto{\pgfqpoint{2.255192in}{1.477420in}}%
\pgfpathlineto{\pgfqpoint{2.260525in}{1.473728in}}%
\pgfpathlineto{\pgfqpoint{2.265858in}{1.470056in}}%
\pgfpathlineto{\pgfqpoint{2.271192in}{1.466405in}}%
\pgfpathlineto{\pgfqpoint{2.276525in}{1.462774in}}%
\pgfpathlineto{\pgfqpoint{2.281858in}{1.459162in}}%
\pgfpathlineto{\pgfqpoint{2.287191in}{1.455570in}}%
\pgfpathlineto{\pgfqpoint{2.292524in}{1.451997in}}%
\pgfpathlineto{\pgfqpoint{2.297858in}{1.448444in}}%
\pgfpathlineto{\pgfqpoint{2.303191in}{1.444910in}}%
\pgfpathlineto{\pgfqpoint{2.308524in}{1.441395in}}%
\pgfpathlineto{\pgfqpoint{2.313857in}{1.437899in}}%
\pgfpathlineto{\pgfqpoint{2.319191in}{1.434421in}}%
\pgfpathlineto{\pgfqpoint{2.324524in}{1.430962in}}%
\pgfpathlineto{\pgfqpoint{2.329857in}{1.427521in}}%
\pgfpathlineto{\pgfqpoint{2.335190in}{1.424099in}}%
\pgfpathlineto{\pgfqpoint{2.340523in}{1.420695in}}%
\pgfpathlineto{\pgfqpoint{2.345857in}{1.417308in}}%
\pgfpathlineto{\pgfqpoint{2.351190in}{1.413940in}}%
\pgfpathlineto{\pgfqpoint{2.356523in}{1.410589in}}%
\pgfpathlineto{\pgfqpoint{2.361856in}{1.407256in}}%
\pgfpathlineto{\pgfqpoint{2.367190in}{1.403940in}}%
\pgfpathlineto{\pgfqpoint{2.372523in}{1.400641in}}%
\pgfpathlineto{\pgfqpoint{2.377856in}{1.397360in}}%
\pgfpathlineto{\pgfqpoint{2.383189in}{1.394095in}}%
\pgfpathlineto{\pgfqpoint{2.388522in}{1.390847in}}%
\pgfpathlineto{\pgfqpoint{2.393856in}{1.387616in}}%
\pgfpathlineto{\pgfqpoint{2.399189in}{1.384402in}}%
\pgfpathlineto{\pgfqpoint{2.404522in}{1.381204in}}%
\pgfpathlineto{\pgfqpoint{2.409855in}{1.378023in}}%
\pgfpathlineto{\pgfqpoint{2.415188in}{1.374858in}}%
\pgfpathlineto{\pgfqpoint{2.420522in}{1.371708in}}%
\pgfpathlineto{\pgfqpoint{2.425855in}{1.368575in}}%
\pgfpathlineto{\pgfqpoint{2.431188in}{1.365458in}}%
\pgfpathlineto{\pgfqpoint{2.436521in}{1.362356in}}%
\pgfpathlineto{\pgfqpoint{2.441855in}{1.359270in}}%
\pgfpathlineto{\pgfqpoint{2.447188in}{1.356200in}}%
\pgfpathlineto{\pgfqpoint{2.452521in}{1.353145in}}%
\pgfpathlineto{\pgfqpoint{2.457854in}{1.350105in}}%
\pgfpathlineto{\pgfqpoint{2.463187in}{1.347080in}}%
\pgfpathlineto{\pgfqpoint{2.468521in}{1.344071in}}%
\pgfpathlineto{\pgfqpoint{2.473854in}{1.341076in}}%
\pgfpathlineto{\pgfqpoint{2.479187in}{1.338096in}}%
\pgfpathlineto{\pgfqpoint{2.484520in}{1.335131in}}%
\pgfpathlineto{\pgfqpoint{2.489853in}{1.332181in}}%
\pgfpathlineto{\pgfqpoint{2.495187in}{1.329245in}}%
\pgfpathlineto{\pgfqpoint{2.500520in}{1.326323in}}%
\pgfpathlineto{\pgfqpoint{2.505853in}{1.323416in}}%
\pgfpathlineto{\pgfqpoint{2.511186in}{1.320522in}}%
\pgfpathlineto{\pgfqpoint{2.516520in}{1.317643in}}%
\pgfpathlineto{\pgfqpoint{2.521853in}{1.314778in}}%
\pgfpathlineto{\pgfqpoint{2.527186in}{1.311927in}}%
\pgfpathlineto{\pgfqpoint{2.532519in}{1.309089in}}%
\pgfpathlineto{\pgfqpoint{2.537852in}{1.306265in}}%
\pgfpathlineto{\pgfqpoint{2.543186in}{1.303455in}}%
\pgfpathlineto{\pgfqpoint{2.548519in}{1.300658in}}%
\pgfpathlineto{\pgfqpoint{2.553852in}{1.297875in}}%
\pgfpathlineto{\pgfqpoint{2.559185in}{1.295105in}}%
\pgfpathlineto{\pgfqpoint{2.564518in}{1.292348in}}%
\pgfpathlineto{\pgfqpoint{2.569852in}{1.289604in}}%
\pgfpathlineto{\pgfqpoint{2.575185in}{1.286873in}}%
\pgfpathlineto{\pgfqpoint{2.580518in}{1.284155in}}%
\pgfpathlineto{\pgfqpoint{2.585851in}{1.281450in}}%
\pgfpathlineto{\pgfqpoint{2.591185in}{1.278758in}}%
\pgfpathlineto{\pgfqpoint{2.596518in}{1.276078in}}%
\pgfpathlineto{\pgfqpoint{2.601851in}{1.273411in}}%
\pgfpathlineto{\pgfqpoint{2.607184in}{1.270756in}}%
\pgfpathlineto{\pgfqpoint{2.612517in}{1.268114in}}%
\pgfpathlineto{\pgfqpoint{2.617851in}{1.265484in}}%
\pgfpathlineto{\pgfqpoint{2.623184in}{1.262866in}}%
\pgfpathlineto{\pgfqpoint{2.628517in}{1.260260in}}%
\pgfpathlineto{\pgfqpoint{2.633850in}{1.257666in}}%
\pgfpathlineto{\pgfqpoint{2.639184in}{1.255084in}}%
\pgfpathlineto{\pgfqpoint{2.644517in}{1.252514in}}%
\pgfpathlineto{\pgfqpoint{2.649850in}{1.249956in}}%
\pgfpathlineto{\pgfqpoint{2.655183in}{1.247410in}}%
\pgfpathlineto{\pgfqpoint{2.660516in}{1.244875in}}%
\pgfpathlineto{\pgfqpoint{2.665850in}{1.242352in}}%
\pgfpathlineto{\pgfqpoint{2.671183in}{1.239840in}}%
\pgfpathlineto{\pgfqpoint{2.676516in}{1.237339in}}%
\pgfpathlineto{\pgfqpoint{2.681849in}{1.234850in}}%
\pgfpathlineto{\pgfqpoint{2.687182in}{1.232372in}}%
\pgfpathlineto{\pgfqpoint{2.692516in}{1.229906in}}%
\pgfpathlineto{\pgfqpoint{2.697849in}{1.227450in}}%
\pgfpathlineto{\pgfqpoint{2.703182in}{1.225005in}}%
\pgfpathlineto{\pgfqpoint{2.708515in}{1.222572in}}%
\pgfpathlineto{\pgfqpoint{2.713849in}{1.220149in}}%
\pgfpathlineto{\pgfqpoint{2.719182in}{1.217737in}}%
\pgfpathlineto{\pgfqpoint{2.724515in}{1.215336in}}%
\pgfpathlineto{\pgfqpoint{2.729848in}{1.212945in}}%
\pgfpathlineto{\pgfqpoint{2.735181in}{1.210565in}}%
\pgfpathlineto{\pgfqpoint{2.740515in}{1.208195in}}%
\pgfpathlineto{\pgfqpoint{2.745848in}{1.205836in}}%
\pgfpathlineto{\pgfqpoint{2.751181in}{1.203487in}}%
\pgfpathlineto{\pgfqpoint{2.756514in}{1.201149in}}%
\pgfpathlineto{\pgfqpoint{2.761847in}{1.198821in}}%
\pgfpathlineto{\pgfqpoint{2.767181in}{1.196502in}}%
\pgfpathlineto{\pgfqpoint{2.772514in}{1.194194in}}%
\pgfpathlineto{\pgfqpoint{2.777847in}{1.191896in}}%
\pgfpathlineto{\pgfqpoint{2.783180in}{1.189608in}}%
\pgfpathlineto{\pgfqpoint{2.788514in}{1.187330in}}%
\pgfpathlineto{\pgfqpoint{2.793847in}{1.185062in}}%
\pgfpathlineto{\pgfqpoint{2.799180in}{1.182804in}}%
\pgfpathlineto{\pgfqpoint{2.804513in}{1.180555in}}%
\pgfpathlineto{\pgfqpoint{2.809846in}{1.178316in}}%
\pgfpathlineto{\pgfqpoint{2.815180in}{1.176086in}}%
\pgfpathlineto{\pgfqpoint{2.820513in}{1.173866in}}%
\pgfpathlineto{\pgfqpoint{2.825846in}{1.171656in}}%
\pgfpathlineto{\pgfqpoint{2.831179in}{1.169454in}}%
\pgfpathlineto{\pgfqpoint{2.836512in}{1.167263in}}%
\pgfpathlineto{\pgfqpoint{2.841846in}{1.165080in}}%
\pgfpathlineto{\pgfqpoint{2.847179in}{1.162907in}}%
\pgfpathlineto{\pgfqpoint{2.852512in}{1.160743in}}%
\pgfpathlineto{\pgfqpoint{2.857845in}{1.158588in}}%
\pgfpathlineto{\pgfqpoint{2.863179in}{1.156442in}}%
\pgfpathlineto{\pgfqpoint{2.868512in}{1.154305in}}%
\pgfpathlineto{\pgfqpoint{2.873845in}{1.152177in}}%
\pgfpathlineto{\pgfqpoint{2.879178in}{1.150058in}}%
\pgfpathlineto{\pgfqpoint{2.884511in}{1.147947in}}%
\pgfpathlineto{\pgfqpoint{2.889845in}{1.145846in}}%
\pgfpathlineto{\pgfqpoint{2.895178in}{1.143753in}}%
\pgfpathlineto{\pgfqpoint{2.900511in}{1.141669in}}%
\pgfpathlineto{\pgfqpoint{2.905844in}{1.139593in}}%
\pgfpathlineto{\pgfqpoint{2.911178in}{1.137526in}}%
\pgfpathlineto{\pgfqpoint{2.916511in}{1.135468in}}%
\pgfpathlineto{\pgfqpoint{2.921844in}{1.133418in}}%
\pgfpathlineto{\pgfqpoint{2.927177in}{1.131376in}}%
\pgfpathlineto{\pgfqpoint{2.932510in}{1.129343in}}%
\pgfpathlineto{\pgfqpoint{2.937844in}{1.127318in}}%
\pgfpathlineto{\pgfqpoint{2.943177in}{1.125301in}}%
\pgfpathlineto{\pgfqpoint{2.948510in}{1.123292in}}%
\pgfpathlineto{\pgfqpoint{2.953843in}{1.121292in}}%
\pgfpathlineto{\pgfqpoint{2.959176in}{1.119300in}}%
\pgfpathlineto{\pgfqpoint{2.964510in}{1.117315in}}%
\pgfpathlineto{\pgfqpoint{2.969843in}{1.115339in}}%
\pgfpathlineto{\pgfqpoint{2.975176in}{1.113371in}}%
\pgfpathlineto{\pgfqpoint{2.980509in}{1.111410in}}%
\pgfpathlineto{\pgfqpoint{2.985843in}{1.109458in}}%
\pgfpathlineto{\pgfqpoint{2.991176in}{1.107513in}}%
\pgfpathlineto{\pgfqpoint{2.996509in}{1.105576in}}%
\pgfpathlineto{\pgfqpoint{3.001842in}{1.103647in}}%
\pgfpathlineto{\pgfqpoint{3.007175in}{1.101725in}}%
\pgfpathlineto{\pgfqpoint{3.012509in}{1.099811in}}%
\pgfpathlineto{\pgfqpoint{3.017842in}{1.097905in}}%
\pgfpathlineto{\pgfqpoint{3.023175in}{1.096006in}}%
\pgfpathlineto{\pgfqpoint{3.028508in}{1.094115in}}%
\pgfpathlineto{\pgfqpoint{3.033841in}{1.092231in}}%
\pgfpathlineto{\pgfqpoint{3.039175in}{1.090354in}}%
\pgfpathlineto{\pgfqpoint{3.044508in}{1.088485in}}%
\pgfpathlineto{\pgfqpoint{3.049841in}{1.086623in}}%
\pgfpathlineto{\pgfqpoint{3.055174in}{1.084768in}}%
\pgfpathlineto{\pgfqpoint{3.060508in}{1.082921in}}%
\pgfpathlineto{\pgfqpoint{3.065841in}{1.081081in}}%
\pgfpathlineto{\pgfqpoint{3.071174in}{1.079248in}}%
\pgfpathlineto{\pgfqpoint{3.076507in}{1.077422in}}%
\pgfpathlineto{\pgfqpoint{3.081840in}{1.075603in}}%
\pgfpathlineto{\pgfqpoint{3.087174in}{1.073791in}}%
\pgfpathlineto{\pgfqpoint{3.092507in}{1.071986in}}%
\pgfpathlineto{\pgfqpoint{3.097840in}{1.070188in}}%
\pgfpathlineto{\pgfqpoint{3.103173in}{1.068397in}}%
\pgfpathlineto{\pgfqpoint{3.108506in}{1.066613in}}%
\pgfpathlineto{\pgfqpoint{3.113840in}{1.064836in}}%
\pgfpathlineto{\pgfqpoint{3.119173in}{1.063065in}}%
\pgfpathlineto{\pgfqpoint{3.124506in}{1.061301in}}%
\pgfpathlineto{\pgfqpoint{3.129839in}{1.059544in}}%
\pgfpathlineto{\pgfqpoint{3.135173in}{1.057794in}}%
\pgfpathlineto{\pgfqpoint{3.140506in}{1.056050in}}%
\pgfpathlineto{\pgfqpoint{3.145839in}{1.054313in}}%
\pgfpathlineto{\pgfqpoint{3.151172in}{1.052582in}}%
\pgfpathlineto{\pgfqpoint{3.156505in}{1.050858in}}%
\pgfpathlineto{\pgfqpoint{3.161839in}{1.049140in}}%
\pgfpathlineto{\pgfqpoint{3.167172in}{1.047429in}}%
\pgfpathlineto{\pgfqpoint{3.172505in}{1.045724in}}%
\pgfpathlineto{\pgfqpoint{3.177838in}{1.044026in}}%
\pgfpathlineto{\pgfqpoint{3.183172in}{1.042334in}}%
\pgfpathlineto{\pgfqpoint{3.188505in}{1.040648in}}%
\pgfpathlineto{\pgfqpoint{3.193838in}{1.038968in}}%
\pgfpathlineto{\pgfqpoint{3.199171in}{1.037295in}}%
\pgfpathlineto{\pgfqpoint{3.204504in}{1.035628in}}%
\pgfpathlineto{\pgfqpoint{3.209838in}{1.033967in}}%
\pgfpathlineto{\pgfqpoint{3.215171in}{1.032312in}}%
\pgfpathlineto{\pgfqpoint{3.220504in}{1.030663in}}%
\pgfpathlineto{\pgfqpoint{3.225837in}{1.029021in}}%
\pgfpathlineto{\pgfqpoint{3.231170in}{1.027384in}}%
\pgfpathlineto{\pgfqpoint{3.236504in}{1.025753in}}%
\pgfpathlineto{\pgfqpoint{3.241837in}{1.024129in}}%
\pgfpathlineto{\pgfqpoint{3.247170in}{1.022510in}}%
\pgfpathlineto{\pgfqpoint{3.252503in}{1.020897in}}%
\pgfpathlineto{\pgfqpoint{3.257837in}{1.019290in}}%
\pgfpathlineto{\pgfqpoint{3.263170in}{1.017689in}}%
\pgfpathlineto{\pgfqpoint{3.268503in}{1.016093in}}%
\pgfpathlineto{\pgfqpoint{3.273836in}{1.014504in}}%
\pgfpathlineto{\pgfqpoint{3.279169in}{1.012920in}}%
\pgfpathlineto{\pgfqpoint{3.284503in}{1.011342in}}%
\pgfpathlineto{\pgfqpoint{3.289836in}{1.009769in}}%
\pgfpathlineto{\pgfqpoint{3.295169in}{1.008202in}}%
\pgfpathlineto{\pgfqpoint{3.300502in}{1.006641in}}%
\pgfpathlineto{\pgfqpoint{3.305835in}{1.005085in}}%
\pgfpathlineto{\pgfqpoint{3.311169in}{1.003535in}}%
\pgfpathlineto{\pgfqpoint{3.316502in}{1.001991in}}%
\pgfpathlineto{\pgfqpoint{3.321835in}{1.000452in}}%
\pgfpathlineto{\pgfqpoint{3.327168in}{0.998918in}}%
\pgfpathlineto{\pgfqpoint{3.332502in}{0.997390in}}%
\pgfpathlineto{\pgfqpoint{3.337835in}{0.995867in}}%
\pgfpathlineto{\pgfqpoint{3.343168in}{0.994350in}}%
\pgfpathlineto{\pgfqpoint{3.348501in}{0.992838in}}%
\pgfpathlineto{\pgfqpoint{3.353834in}{0.991331in}}%
\pgfpathlineto{\pgfqpoint{3.359168in}{0.989830in}}%
\pgfpathlineto{\pgfqpoint{3.364501in}{0.988333in}}%
\pgfpathlineto{\pgfqpoint{3.369834in}{0.986843in}}%
\pgfpathlineto{\pgfqpoint{3.375167in}{0.985357in}}%
\pgfpathlineto{\pgfqpoint{3.380500in}{0.983876in}}%
\pgfpathlineto{\pgfqpoint{3.385834in}{0.982401in}}%
\pgfpathlineto{\pgfqpoint{3.391167in}{0.980931in}}%
\pgfpathlineto{\pgfqpoint{3.396500in}{0.979466in}}%
\pgfpathlineto{\pgfqpoint{3.401833in}{0.978006in}}%
\pgfpathlineto{\pgfqpoint{3.407167in}{0.976551in}}%
\pgfpathlineto{\pgfqpoint{3.412500in}{0.975101in}}%
\pgfpathlineto{\pgfqpoint{3.417833in}{0.973656in}}%
\pgfpathlineto{\pgfqpoint{3.423166in}{0.972216in}}%
\pgfpathlineto{\pgfqpoint{3.428499in}{0.970781in}}%
\pgfpathlineto{\pgfqpoint{3.433833in}{0.969350in}}%
\pgfpathlineto{\pgfqpoint{3.439166in}{0.967925in}}%
\pgfpathlineto{\pgfqpoint{3.444499in}{0.966505in}}%
\pgfpathlineto{\pgfqpoint{3.449832in}{0.965089in}}%
\pgfpathlineto{\pgfqpoint{3.455166in}{0.963679in}}%
\pgfpathlineto{\pgfqpoint{3.460499in}{0.962273in}}%
\pgfpathlineto{\pgfqpoint{3.465832in}{0.960872in}}%
\pgfpathlineto{\pgfqpoint{3.471165in}{0.959475in}}%
\pgfpathlineto{\pgfqpoint{3.476498in}{0.958084in}}%
\pgfpathlineto{\pgfqpoint{3.481832in}{0.956697in}}%
\pgfpathlineto{\pgfqpoint{3.487165in}{0.955314in}}%
\pgfpathlineto{\pgfqpoint{3.492498in}{0.953937in}}%
\pgfpathlineto{\pgfqpoint{3.497831in}{0.952564in}}%
\pgfpathlineto{\pgfqpoint{3.503164in}{0.951195in}}%
\pgfpathlineto{\pgfqpoint{3.508498in}{0.949832in}}%
\pgfpathlineto{\pgfqpoint{3.513831in}{0.948472in}}%
\pgfpathlineto{\pgfqpoint{3.519164in}{0.947118in}}%
\pgfpathlineto{\pgfqpoint{3.524497in}{0.945767in}}%
\pgfpathlineto{\pgfqpoint{3.529831in}{0.944422in}}%
\pgfpathlineto{\pgfqpoint{3.535164in}{0.943081in}}%
\pgfpathlineto{\pgfqpoint{3.540497in}{0.941744in}}%
\pgfpathlineto{\pgfqpoint{3.545830in}{0.940411in}}%
\pgfpathlineto{\pgfqpoint{3.551163in}{0.939083in}}%
\pgfpathlineto{\pgfqpoint{3.556497in}{0.937760in}}%
\pgfpathlineto{\pgfqpoint{3.561830in}{0.936441in}}%
\pgfpathlineto{\pgfqpoint{3.567163in}{0.935126in}}%
\pgfpathlineto{\pgfqpoint{3.572496in}{0.933815in}}%
\pgfpathlineto{\pgfqpoint{3.577829in}{0.932509in}}%
\pgfpathlineto{\pgfqpoint{3.583163in}{0.931207in}}%
\pgfpathlineto{\pgfqpoint{3.588496in}{0.929909in}}%
\pgfpathlineto{\pgfqpoint{3.593829in}{0.928616in}}%
\pgfpathlineto{\pgfqpoint{3.599162in}{0.927326in}}%
\pgfpathlineto{\pgfqpoint{3.604496in}{0.926041in}}%
\pgfpathlineto{\pgfqpoint{3.609829in}{0.924760in}}%
\pgfpathlineto{\pgfqpoint{3.615162in}{0.923484in}}%
\pgfpathlineto{\pgfqpoint{3.620495in}{0.922211in}}%
\pgfpathlineto{\pgfqpoint{3.625828in}{0.920943in}}%
\pgfpathlineto{\pgfqpoint{3.631162in}{0.919678in}}%
\pgfpathlineto{\pgfqpoint{3.636495in}{0.918418in}}%
\pgfpathlineto{\pgfqpoint{3.641828in}{0.917161in}}%
\pgfpathlineto{\pgfqpoint{3.647161in}{0.915909in}}%
\pgfpathlineto{\pgfqpoint{3.652494in}{0.914661in}}%
\pgfpathlineto{\pgfqpoint{3.657828in}{0.913417in}}%
\pgfpathlineto{\pgfqpoint{3.663161in}{0.912176in}}%
\pgfpathlineto{\pgfqpoint{3.668494in}{0.910940in}}%
\pgfpathlineto{\pgfqpoint{3.673827in}{0.909708in}}%
\pgfpathlineto{\pgfqpoint{3.679161in}{0.908479in}}%
\pgfpathlineto{\pgfqpoint{3.684494in}{0.907254in}}%
\pgfpathlineto{\pgfqpoint{3.689827in}{0.906034in}}%
\pgfpathlineto{\pgfqpoint{3.695160in}{0.904817in}}%
\pgfpathlineto{\pgfqpoint{3.700493in}{0.903604in}}%
\pgfpathlineto{\pgfqpoint{3.705827in}{0.902395in}}%
\pgfpathlineto{\pgfqpoint{3.711160in}{0.901189in}}%
\pgfpathlineto{\pgfqpoint{3.716493in}{0.899988in}}%
\pgfpathlineto{\pgfqpoint{3.721826in}{0.898790in}}%
\pgfpathlineto{\pgfqpoint{3.727160in}{0.897596in}}%
\pgfpathlineto{\pgfqpoint{3.732493in}{0.896406in}}%
\pgfpathlineto{\pgfqpoint{3.737826in}{0.895219in}}%
\pgfpathlineto{\pgfqpoint{3.743159in}{0.894036in}}%
\pgfpathlineto{\pgfqpoint{3.748492in}{0.892857in}}%
\pgfpathlineto{\pgfqpoint{3.753826in}{0.891681in}}%
\pgfpathlineto{\pgfqpoint{3.759159in}{0.890509in}}%
\pgfpathlineto{\pgfqpoint{3.764492in}{0.889341in}}%
\pgfpathlineto{\pgfqpoint{3.769825in}{0.888177in}}%
\pgfpathlineto{\pgfqpoint{3.775158in}{0.887016in}}%
\pgfpathlineto{\pgfqpoint{3.780492in}{0.885858in}}%
\pgfpathlineto{\pgfqpoint{3.785825in}{0.884704in}}%
\pgfpathlineto{\pgfqpoint{3.791158in}{0.883554in}}%
\pgfpathlineto{\pgfqpoint{3.796491in}{0.882407in}}%
\pgfpathlineto{\pgfqpoint{3.801825in}{0.881264in}}%
\pgfpathlineto{\pgfqpoint{3.807158in}{0.880124in}}%
\pgfpathlineto{\pgfqpoint{3.812491in}{0.878988in}}%
\pgfpathlineto{\pgfqpoint{3.817824in}{0.877855in}}%
\pgfpathlineto{\pgfqpoint{3.823157in}{0.876725in}}%
\pgfpathlineto{\pgfqpoint{3.828491in}{0.875599in}}%
\pgfpathlineto{\pgfqpoint{3.833824in}{0.874477in}}%
\pgfpathlineto{\pgfqpoint{3.839157in}{0.873358in}}%
\pgfpathlineto{\pgfqpoint{3.844490in}{0.872242in}}%
\pgfpathlineto{\pgfqpoint{3.849823in}{0.871130in}}%
\pgfpathlineto{\pgfqpoint{3.855157in}{0.870021in}}%
\pgfpathlineto{\pgfqpoint{3.860490in}{0.868915in}}%
\pgfpathlineto{\pgfqpoint{3.865823in}{0.867813in}}%
\pgfpathlineto{\pgfqpoint{3.871156in}{0.866714in}}%
\pgfpathlineto{\pgfqpoint{3.876490in}{0.865618in}}%
\pgfpathlineto{\pgfqpoint{3.881823in}{0.864526in}}%
\pgfpathlineto{\pgfqpoint{3.887156in}{0.863436in}}%
\pgfpathlineto{\pgfqpoint{3.892489in}{0.862351in}}%
\pgfpathlineto{\pgfqpoint{3.897822in}{0.861268in}}%
\pgfpathlineto{\pgfqpoint{3.903156in}{0.860188in}}%
\pgfpathlineto{\pgfqpoint{3.908489in}{0.859112in}}%
\pgfpathlineto{\pgfqpoint{3.913822in}{0.858039in}}%
\pgfpathlineto{\pgfqpoint{3.919155in}{0.856969in}}%
\pgfpathlineto{\pgfqpoint{3.924488in}{0.855903in}}%
\pgfpathlineto{\pgfqpoint{3.929822in}{0.854839in}}%
\pgfpathlineto{\pgfqpoint{3.935155in}{0.853779in}}%
\pgfpathlineto{\pgfqpoint{3.940488in}{0.852722in}}%
\pgfpathlineto{\pgfqpoint{3.945821in}{0.851667in}}%
\pgfpathlineto{\pgfqpoint{3.951155in}{0.850616in}}%
\pgfpathlineto{\pgfqpoint{3.956488in}{0.849568in}}%
\pgfpathlineto{\pgfqpoint{3.961821in}{0.848524in}}%
\pgfpathlineto{\pgfqpoint{3.967154in}{0.847482in}}%
\pgfpathlineto{\pgfqpoint{3.972487in}{0.846443in}}%
\pgfpathlineto{\pgfqpoint{3.977821in}{0.845407in}}%
\pgfpathlineto{\pgfqpoint{3.983154in}{0.844375in}}%
\pgfpathlineto{\pgfqpoint{3.988487in}{0.843345in}}%
\pgfpathlineto{\pgfqpoint{3.993820in}{0.842318in}}%
\pgfpathlineto{\pgfqpoint{3.999154in}{0.841294in}}%
\pgfpathlineto{\pgfqpoint{4.004487in}{0.840274in}}%
\pgfpathlineto{\pgfqpoint{4.009820in}{0.839256in}}%
\pgfpathlineto{\pgfqpoint{4.015153in}{0.838241in}}%
\pgfpathlineto{\pgfqpoint{4.020486in}{0.837229in}}%
\pgfpathlineto{\pgfqpoint{4.025820in}{0.836220in}}%
\pgfpathlineto{\pgfqpoint{4.031153in}{0.835214in}}%
\pgfpathlineto{\pgfqpoint{4.036486in}{0.834211in}}%
\pgfpathlineto{\pgfqpoint{4.041819in}{0.833210in}}%
\pgfpathlineto{\pgfqpoint{4.047152in}{0.832213in}}%
\pgfpathlineto{\pgfqpoint{4.052486in}{0.831218in}}%
\pgfpathlineto{\pgfqpoint{4.057819in}{0.830226in}}%
\pgfpathlineto{\pgfqpoint{4.063152in}{0.829237in}}%
\pgfpathlineto{\pgfqpoint{4.068485in}{0.828251in}}%
\pgfpathlineto{\pgfqpoint{4.073819in}{0.827268in}}%
\pgfpathlineto{\pgfqpoint{4.079152in}{0.826287in}}%
\pgfpathlineto{\pgfqpoint{4.084485in}{0.825309in}}%
\pgfpathlineto{\pgfqpoint{4.089818in}{0.824334in}}%
\pgfpathlineto{\pgfqpoint{4.095151in}{0.823362in}}%
\pgfpathlineto{\pgfqpoint{4.100485in}{0.822393in}}%
\pgfpathlineto{\pgfqpoint{4.105818in}{0.821426in}}%
\pgfpathlineto{\pgfqpoint{4.111151in}{0.820462in}}%
\pgfpathlineto{\pgfqpoint{4.116484in}{0.819501in}}%
\pgfpathlineto{\pgfqpoint{4.121817in}{0.818542in}}%
\pgfpathlineto{\pgfqpoint{4.127151in}{0.817586in}}%
\pgfpathlineto{\pgfqpoint{4.132484in}{0.816633in}}%
\pgfpathlineto{\pgfqpoint{4.137817in}{0.815682in}}%
\pgfpathlineto{\pgfqpoint{4.143150in}{0.814734in}}%
\pgfpathlineto{\pgfqpoint{4.148484in}{0.813789in}}%
\pgfpathlineto{\pgfqpoint{4.153817in}{0.812846in}}%
\pgfpathlineto{\pgfqpoint{4.159150in}{0.811906in}}%
\pgfpathlineto{\pgfqpoint{4.164483in}{0.810969in}}%
\pgfpathlineto{\pgfqpoint{4.169816in}{0.810034in}}%
\pgfpathlineto{\pgfqpoint{4.175150in}{0.809102in}}%
\pgfpathlineto{\pgfqpoint{4.180483in}{0.808172in}}%
\pgfpathlineto{\pgfqpoint{4.185816in}{0.807245in}}%
\pgfpathlineto{\pgfqpoint{4.191149in}{0.806320in}}%
\pgfpathlineto{\pgfqpoint{4.196482in}{0.805398in}}%
\pgfpathlineto{\pgfqpoint{4.201816in}{0.804479in}}%
\pgfpathlineto{\pgfqpoint{4.207149in}{0.803562in}}%
\pgfpathlineto{\pgfqpoint{4.212482in}{0.802647in}}%
\pgfpathlineto{\pgfqpoint{4.217815in}{0.801736in}}%
\pgfpathlineto{\pgfqpoint{4.223149in}{0.800826in}}%
\pgfpathlineto{\pgfqpoint{4.228482in}{0.799919in}}%
\pgfpathlineto{\pgfqpoint{4.233815in}{0.799015in}}%
\pgfpathlineto{\pgfqpoint{4.239148in}{0.798113in}}%
\pgfpathlineto{\pgfqpoint{4.244481in}{0.797213in}}%
\pgfpathlineto{\pgfqpoint{4.249815in}{0.796316in}}%
\pgfpathlineto{\pgfqpoint{4.255148in}{0.795421in}}%
\pgfpathlineto{\pgfqpoint{4.260481in}{0.794529in}}%
\pgfpathlineto{\pgfqpoint{4.265814in}{0.793639in}}%
\pgfpathlineto{\pgfqpoint{4.271148in}{0.792752in}}%
\pgfpathlineto{\pgfqpoint{4.276481in}{0.791867in}}%
\pgfpathlineto{\pgfqpoint{4.281814in}{0.790984in}}%
\pgfpathlineto{\pgfqpoint{4.287147in}{0.790104in}}%
\pgfpathlineto{\pgfqpoint{4.292480in}{0.789226in}}%
\pgfpathlineto{\pgfqpoint{4.297814in}{0.788351in}}%
\pgfpathlineto{\pgfqpoint{4.303147in}{0.787477in}}%
\pgfpathlineto{\pgfqpoint{4.308480in}{0.786606in}}%
\pgfpathlineto{\pgfqpoint{4.313813in}{0.785738in}}%
\pgfpathlineto{\pgfqpoint{4.319146in}{0.784872in}}%
\pgfpathlineto{\pgfqpoint{4.324480in}{0.784008in}}%
\pgfpathlineto{\pgfqpoint{4.329813in}{0.783146in}}%
\pgfpathlineto{\pgfqpoint{4.335146in}{0.782287in}}%
\pgfpathlineto{\pgfqpoint{4.340479in}{0.781430in}}%
\pgfpathlineto{\pgfqpoint{4.345813in}{0.780575in}}%
\pgfpathlineto{\pgfqpoint{4.351146in}{0.779723in}}%
\pgfpathlineto{\pgfqpoint{4.356479in}{0.778872in}}%
\pgfpathlineto{\pgfqpoint{4.361812in}{0.778024in}}%
\pgfpathlineto{\pgfqpoint{4.367145in}{0.777179in}}%
\pgfpathlineto{\pgfqpoint{4.372479in}{0.776335in}}%
\pgfpathlineto{\pgfqpoint{4.377812in}{0.775494in}}%
\pgfpathlineto{\pgfqpoint{4.383145in}{0.774655in}}%
\pgfpathlineto{\pgfqpoint{4.388478in}{0.773818in}}%
\pgfpathlineto{\pgfqpoint{4.393811in}{0.772983in}}%
\pgfpathlineto{\pgfqpoint{4.399145in}{0.772151in}}%
\pgfpathlineto{\pgfqpoint{4.404478in}{0.771320in}}%
\pgfpathlineto{\pgfqpoint{4.409811in}{0.770492in}}%
\pgfpathlineto{\pgfqpoint{4.415144in}{0.769666in}}%
\pgfpathlineto{\pgfqpoint{4.420478in}{0.768842in}}%
\pgfpathlineto{\pgfqpoint{4.425811in}{0.768020in}}%
\pgfpathlineto{\pgfqpoint{4.431144in}{0.767201in}}%
\pgfpathlineto{\pgfqpoint{4.436477in}{0.766383in}}%
\pgfpathlineto{\pgfqpoint{4.441810in}{0.765568in}}%
\pgfpathlineto{\pgfqpoint{4.447144in}{0.764755in}}%
\pgfpathlineto{\pgfqpoint{4.452477in}{0.763944in}}%
\pgfpathlineto{\pgfqpoint{4.457810in}{0.763135in}}%
\pgfpathlineto{\pgfqpoint{4.463143in}{0.762328in}}%
\pgfpathlineto{\pgfqpoint{4.468476in}{0.761523in}}%
\pgfpathlineto{\pgfqpoint{4.473810in}{0.760720in}}%
\pgfpathlineto{\pgfqpoint{4.479143in}{0.759919in}}%
\pgfpathlineto{\pgfqpoint{4.484476in}{0.759120in}}%
\pgfpathlineto{\pgfqpoint{4.489809in}{0.758324in}}%
\pgfpathlineto{\pgfqpoint{4.495143in}{0.757529in}}%
\pgfpathlineto{\pgfqpoint{4.500476in}{0.756737in}}%
\pgfpathlineto{\pgfqpoint{4.505809in}{0.755946in}}%
\pgfpathlineto{\pgfqpoint{4.511142in}{0.755157in}}%
\pgfpathlineto{\pgfqpoint{4.516475in}{0.754371in}}%
\pgfpathlineto{\pgfqpoint{4.521809in}{0.753586in}}%
\pgfpathlineto{\pgfqpoint{4.527142in}{0.752804in}}%
\pgfpathlineto{\pgfqpoint{4.532475in}{0.752023in}}%
\pgfpathlineto{\pgfqpoint{4.537808in}{0.751245in}}%
\pgfpathlineto{\pgfqpoint{4.543142in}{0.750468in}}%
\pgfpathlineto{\pgfqpoint{4.548475in}{0.749693in}}%
\pgfpathlineto{\pgfqpoint{4.553808in}{0.748921in}}%
\pgfpathlineto{\pgfqpoint{4.559141in}{0.748150in}}%
\pgfpathlineto{\pgfqpoint{4.564474in}{0.747381in}}%
\pgfpathlineto{\pgfqpoint{4.569808in}{0.746614in}}%
\pgfpathlineto{\pgfqpoint{4.575141in}{0.745849in}}%
\pgfpathlineto{\pgfqpoint{4.580474in}{0.745086in}}%
\pgfpathlineto{\pgfqpoint{4.585807in}{0.744325in}}%
\pgfpathlineto{\pgfqpoint{4.591140in}{0.743566in}}%
\pgfpathlineto{\pgfqpoint{4.596474in}{0.742808in}}%
\pgfpathlineto{\pgfqpoint{4.601807in}{0.742053in}}%
\pgfpathlineto{\pgfqpoint{4.607140in}{0.741299in}}%
\pgfpathlineto{\pgfqpoint{4.612473in}{0.740548in}}%
\pgfpathlineto{\pgfqpoint{4.617807in}{0.739798in}}%
\pgfpathlineto{\pgfqpoint{4.623140in}{0.739050in}}%
\pgfpathlineto{\pgfqpoint{4.628473in}{0.738304in}}%
\pgfpathlineto{\pgfqpoint{4.633806in}{0.737559in}}%
\pgfpathlineto{\pgfqpoint{4.639139in}{0.736817in}}%
\pgfpathlineto{\pgfqpoint{4.644473in}{0.736076in}}%
\pgfpathlineto{\pgfqpoint{4.649806in}{0.735338in}}%
\pgfpathlineto{\pgfqpoint{4.655139in}{0.734601in}}%
\pgfpathlineto{\pgfqpoint{4.660472in}{0.733865in}}%
\pgfpathlineto{\pgfqpoint{4.665805in}{0.733132in}}%
\pgfpathlineto{\pgfqpoint{4.671139in}{0.732401in}}%
\pgfpathlineto{\pgfqpoint{4.676472in}{0.731671in}}%
\pgfpathlineto{\pgfqpoint{4.681805in}{0.730943in}}%
\pgfpathlineto{\pgfqpoint{4.687138in}{0.730217in}}%
\pgfpathlineto{\pgfqpoint{4.692472in}{0.729492in}}%
\pgfpathlineto{\pgfqpoint{4.697805in}{0.728770in}}%
\pgfpathlineto{\pgfqpoint{4.703138in}{0.728049in}}%
\pgfpathlineto{\pgfqpoint{4.708471in}{0.727330in}}%
\pgfpathlineto{\pgfqpoint{4.713804in}{0.726612in}}%
\pgfpathlineto{\pgfqpoint{4.719138in}{0.725897in}}%
\pgfpathlineto{\pgfqpoint{4.724471in}{0.725183in}}%
\pgfpathlineto{\pgfqpoint{4.729804in}{0.724470in}}%
\pgfpathlineto{\pgfqpoint{4.735137in}{0.723760in}}%
\pgfpathlineto{\pgfqpoint{4.740470in}{0.723051in}}%
\pgfpathlineto{\pgfqpoint{4.745804in}{0.722344in}}%
\pgfpathlineto{\pgfqpoint{4.751137in}{0.721639in}}%
\pgfpathlineto{\pgfqpoint{4.756470in}{0.720935in}}%
\pgfpathlineto{\pgfqpoint{4.761803in}{0.720233in}}%
\pgfpathlineto{\pgfqpoint{4.767137in}{0.719533in}}%
\pgfpathlineto{\pgfqpoint{4.772470in}{0.718834in}}%
\pgfpathlineto{\pgfqpoint{4.777803in}{0.718138in}}%
\pgfpathlineto{\pgfqpoint{4.783136in}{0.717442in}}%
\pgfpathlineto{\pgfqpoint{4.788469in}{0.716749in}}%
\pgfpathlineto{\pgfqpoint{4.793803in}{0.716057in}}%
\pgfpathlineto{\pgfqpoint{4.799136in}{0.715367in}}%
\pgfpathlineto{\pgfqpoint{4.804469in}{0.714678in}}%
\pgfpathlineto{\pgfqpoint{4.809802in}{0.713991in}}%
\pgfpathlineto{\pgfqpoint{4.815136in}{0.713306in}}%
\pgfpathlineto{\pgfqpoint{4.820469in}{0.712622in}}%
\pgfpathlineto{\pgfqpoint{4.825802in}{0.711940in}}%
\pgfpathlineto{\pgfqpoint{4.831135in}{0.711259in}}%
\pgfpathlineto{\pgfqpoint{4.836468in}{0.710580in}}%
\pgfpathlineto{\pgfqpoint{4.841802in}{0.709903in}}%
\pgfpathlineto{\pgfqpoint{4.847135in}{0.709227in}}%
\pgfpathlineto{\pgfqpoint{4.852468in}{0.708553in}}%
\pgfpathlineto{\pgfqpoint{4.857801in}{0.707881in}}%
\pgfpathlineto{\pgfqpoint{4.863134in}{0.707210in}}%
\pgfpathlineto{\pgfqpoint{4.868468in}{0.706540in}}%
\pgfpathlineto{\pgfqpoint{4.873801in}{0.705873in}}%
\pgfpathlineto{\pgfqpoint{4.879134in}{0.705206in}}%
\pgfpathlineto{\pgfqpoint{4.884467in}{0.704542in}}%
\pgfpathlineto{\pgfqpoint{4.889801in}{0.703879in}}%
\pgfpathlineto{\pgfqpoint{4.895134in}{0.703217in}}%
\pgfpathlineto{\pgfqpoint{4.900467in}{0.702557in}}%
\pgfpathlineto{\pgfqpoint{4.905800in}{0.701899in}}%
\pgfpathlineto{\pgfqpoint{4.911133in}{0.701242in}}%
\pgfpathlineto{\pgfqpoint{4.916467in}{0.700586in}}%
\pgfpathlineto{\pgfqpoint{4.921800in}{0.699932in}}%
\pgfpathlineto{\pgfqpoint{4.927133in}{0.699280in}}%
\pgfpathlineto{\pgfqpoint{4.932466in}{0.698629in}}%
\pgfpathlineto{\pgfqpoint{4.937799in}{0.697980in}}%
\pgfpathlineto{\pgfqpoint{4.943133in}{0.697332in}}%
\pgfpathlineto{\pgfqpoint{4.948466in}{0.696685in}}%
\pgfpathlineto{\pgfqpoint{4.953799in}{0.696040in}}%
\pgfpathlineto{\pgfqpoint{4.959132in}{0.695397in}}%
\pgfpathlineto{\pgfqpoint{4.964466in}{0.694755in}}%
\pgfpathlineto{\pgfqpoint{4.969799in}{0.694115in}}%
\pgfpathlineto{\pgfqpoint{4.975132in}{0.693476in}}%
\pgfpathlineto{\pgfqpoint{4.980465in}{0.692838in}}%
\pgfpathlineto{\pgfqpoint{4.985798in}{0.692202in}}%
\pgfpathlineto{\pgfqpoint{4.991132in}{0.691567in}}%
\pgfpathlineto{\pgfqpoint{4.996465in}{0.690934in}}%
\pgfpathlineto{\pgfqpoint{5.001798in}{0.690303in}}%
\pgfpathlineto{\pgfqpoint{5.007131in}{0.689672in}}%
\pgfpathlineto{\pgfqpoint{5.012464in}{0.689043in}}%
\pgfpathlineto{\pgfqpoint{5.017798in}{0.688416in}}%
\pgfpathlineto{\pgfqpoint{5.023131in}{0.687790in}}%
\pgfpathlineto{\pgfqpoint{5.028464in}{0.687165in}}%
\pgfpathlineto{\pgfqpoint{5.033797in}{0.686542in}}%
\pgfpathlineto{\pgfqpoint{5.039131in}{0.685920in}}%
\pgfpathlineto{\pgfqpoint{5.044464in}{0.685300in}}%
\pgfpathlineto{\pgfqpoint{5.049797in}{0.684681in}}%
\pgfpathlineto{\pgfqpoint{5.055130in}{0.684064in}}%
\pgfpathlineto{\pgfqpoint{5.060463in}{0.683447in}}%
\pgfpathlineto{\pgfqpoint{5.065797in}{0.682833in}}%
\pgfpathlineto{\pgfqpoint{5.071130in}{0.682219in}}%
\pgfpathlineto{\pgfqpoint{5.076463in}{0.681607in}}%
\pgfpathlineto{\pgfqpoint{5.081796in}{0.680997in}}%
\pgfpathlineto{\pgfqpoint{5.087130in}{0.680387in}}%
\pgfpathlineto{\pgfqpoint{5.092463in}{0.679779in}}%
\pgfpathlineto{\pgfqpoint{5.097796in}{0.679173in}}%
\pgfpathlineto{\pgfqpoint{5.103129in}{0.678568in}}%
\pgfpathlineto{\pgfqpoint{5.108462in}{0.677964in}}%
\pgfpathlineto{\pgfqpoint{5.113796in}{0.677361in}}%
\pgfpathlineto{\pgfqpoint{5.119129in}{0.676760in}}%
\pgfpathlineto{\pgfqpoint{5.124462in}{0.676160in}}%
\pgfpathlineto{\pgfqpoint{5.129795in}{0.675562in}}%
\pgfpathlineto{\pgfqpoint{5.135128in}{0.674965in}}%
\pgfpathlineto{\pgfqpoint{5.140462in}{0.674369in}}%
\pgfpathlineto{\pgfqpoint{5.145795in}{0.673774in}}%
\pgfpathlineto{\pgfqpoint{5.151128in}{0.673181in}}%
\pgfpathlineto{\pgfqpoint{5.156461in}{0.672589in}}%
\pgfpathlineto{\pgfqpoint{5.161795in}{0.671998in}}%
\pgfpathlineto{\pgfqpoint{5.167128in}{0.671409in}}%
\pgfpathlineto{\pgfqpoint{5.172461in}{0.670821in}}%
\pgfpathlineto{\pgfqpoint{5.177794in}{0.670234in}}%
\pgfpathlineto{\pgfqpoint{5.183127in}{0.669649in}}%
\pgfpathlineto{\pgfqpoint{5.188461in}{0.669065in}}%
\pgfpathlineto{\pgfqpoint{5.193794in}{0.668482in}}%
\pgfpathlineto{\pgfqpoint{5.199127in}{0.667900in}}%
\pgfpathlineto{\pgfqpoint{5.204460in}{0.667320in}}%
\pgfpathlineto{\pgfqpoint{5.209793in}{0.666741in}}%
\pgfpathlineto{\pgfqpoint{5.215127in}{0.666163in}}%
\pgfpathlineto{\pgfqpoint{5.220460in}{0.665587in}}%
\pgfpathlineto{\pgfqpoint{5.225793in}{0.665011in}}%
\pgfpathlineto{\pgfqpoint{5.231126in}{0.664437in}}%
\pgfpathlineto{\pgfqpoint{5.236460in}{0.663864in}}%
\pgfpathlineto{\pgfqpoint{5.241793in}{0.663293in}}%
\pgfpathlineto{\pgfqpoint{5.247126in}{0.662723in}}%
\pgfpathlineto{\pgfqpoint{5.252459in}{0.662154in}}%
\pgfpathlineto{\pgfqpoint{5.257792in}{0.661586in}}%
\pgfpathlineto{\pgfqpoint{5.263126in}{0.661019in}}%
\pgfpathlineto{\pgfqpoint{5.268459in}{0.660454in}}%
\pgfpathlineto{\pgfqpoint{5.273792in}{0.659890in}}%
\pgfpathlineto{\pgfqpoint{5.279125in}{0.659327in}}%
\pgfpathlineto{\pgfqpoint{5.284458in}{0.658765in}}%
\pgfpathlineto{\pgfqpoint{5.289792in}{0.658204in}}%
\pgfpathlineto{\pgfqpoint{5.295125in}{0.657645in}}%
\pgfpathlineto{\pgfqpoint{5.300458in}{0.657087in}}%
\pgfpathlineto{\pgfqpoint{5.305791in}{0.656530in}}%
\pgfpathlineto{\pgfqpoint{5.311125in}{0.655974in}}%
\pgfpathlineto{\pgfqpoint{5.316458in}{0.655420in}}%
\pgfpathlineto{\pgfqpoint{5.321791in}{0.654866in}}%
\pgfpathlineto{\pgfqpoint{5.327124in}{0.654314in}}%
\pgfpathlineto{\pgfqpoint{5.332457in}{0.653763in}}%
\pgfpathlineto{\pgfqpoint{5.337791in}{0.653213in}}%
\pgfpathlineto{\pgfqpoint{5.343124in}{0.652664in}}%
\pgfpathlineto{\pgfqpoint{5.348457in}{0.652117in}}%
\pgfpathlineto{\pgfqpoint{5.353790in}{0.651571in}}%
\pgfpathlineto{\pgfqpoint{5.359124in}{0.651025in}}%
\pgfpathlineto{\pgfqpoint{5.364457in}{0.650481in}}%
\pgfpathlineto{\pgfqpoint{5.369790in}{0.649938in}}%
\pgfpathlineto{\pgfqpoint{5.375123in}{0.649397in}}%
\pgfpathlineto{\pgfqpoint{5.380456in}{0.648856in}}%
\pgfpathlineto{\pgfqpoint{5.385790in}{0.648317in}}%
\pgfpathlineto{\pgfqpoint{5.391123in}{0.647778in}}%
\pgfpathlineto{\pgfqpoint{5.396456in}{0.647241in}}%
\pgfpathlineto{\pgfqpoint{5.401789in}{0.646705in}}%
\pgfpathlineto{\pgfqpoint{5.407122in}{0.646170in}}%
\pgfpathlineto{\pgfqpoint{5.412456in}{0.645636in}}%
\pgfpathlineto{\pgfqpoint{5.417789in}{0.645103in}}%
\pgfpathlineto{\pgfqpoint{5.423122in}{0.644572in}}%
\pgfpathlineto{\pgfqpoint{5.428455in}{0.644041in}}%
\pgfpathlineto{\pgfqpoint{5.433789in}{0.643512in}}%
\pgfpathlineto{\pgfqpoint{5.439122in}{0.642984in}}%
\pgfpathlineto{\pgfqpoint{5.444455in}{0.642457in}}%
\pgfpathlineto{\pgfqpoint{5.449788in}{0.641930in}}%
\pgfpathlineto{\pgfqpoint{5.455121in}{0.641406in}}%
\pgfpathlineto{\pgfqpoint{5.460455in}{0.640882in}}%
\pgfpathlineto{\pgfqpoint{5.465788in}{0.640359in}}%
\pgfpathlineto{\pgfqpoint{5.471121in}{0.639837in}}%
\pgfpathlineto{\pgfqpoint{5.476454in}{0.639316in}}%
\pgfpathlineto{\pgfqpoint{5.481787in}{0.638797in}}%
\pgfpathlineto{\pgfqpoint{5.487121in}{0.638278in}}%
\pgfpathlineto{\pgfqpoint{5.492454in}{0.637761in}}%
\pgfpathlineto{\pgfqpoint{5.497787in}{0.637245in}}%
\pgfpathlineto{\pgfqpoint{5.503120in}{0.636729in}}%
\pgfpathlineto{\pgfqpoint{5.508454in}{0.636215in}}%
\pgfpathlineto{\pgfqpoint{5.513787in}{0.635702in}}%
\pgfpathlineto{\pgfqpoint{5.519120in}{0.635190in}}%
\pgfpathlineto{\pgfqpoint{5.524453in}{0.634679in}}%
\pgfpathlineto{\pgfqpoint{5.529786in}{0.634169in}}%
\pgfpathlineto{\pgfqpoint{5.535120in}{0.633660in}}%
\pgfpathlineto{\pgfqpoint{5.540453in}{0.633152in}}%
\pgfpathlineto{\pgfqpoint{5.545786in}{0.632645in}}%
\pgfpathlineto{\pgfqpoint{5.551119in}{0.632139in}}%
\pgfpathlineto{\pgfqpoint{5.556452in}{0.631634in}}%
\pgfpathlineto{\pgfqpoint{5.561786in}{0.631130in}}%
\pgfpathlineto{\pgfqpoint{5.567119in}{0.630628in}}%
\pgfpathlineto{\pgfqpoint{5.572452in}{0.630126in}}%
\pgfpathlineto{\pgfqpoint{5.577785in}{0.629625in}}%
\pgfpathlineto{\pgfqpoint{5.583119in}{0.629125in}}%
\pgfpathlineto{\pgfqpoint{5.588452in}{0.628627in}}%
\pgfpathlineto{\pgfqpoint{5.593785in}{0.628129in}}%
\pgfpathlineto{\pgfqpoint{5.599118in}{0.627632in}}%
\pgfpathlineto{\pgfqpoint{5.604451in}{0.627137in}}%
\pgfpathlineto{\pgfqpoint{5.609785in}{0.626642in}}%
\pgfpathlineto{\pgfqpoint{5.615118in}{0.626148in}}%
\pgfpathlineto{\pgfqpoint{5.620451in}{0.625655in}}%
\pgfpathlineto{\pgfqpoint{5.625784in}{0.625164in}}%
\pgfpathlineto{\pgfqpoint{5.631118in}{0.624673in}}%
\pgfpathlineto{\pgfqpoint{5.636451in}{0.624183in}}%
\pgfpathlineto{\pgfqpoint{5.641784in}{0.623695in}}%
\pgfpathlineto{\pgfqpoint{5.647117in}{0.623207in}}%
\pgfpathlineto{\pgfqpoint{5.652450in}{0.622720in}}%
\pgfpathlineto{\pgfqpoint{5.657784in}{0.622234in}}%
\pgfpathlineto{\pgfqpoint{5.663117in}{0.621749in}}%
\pgfpathlineto{\pgfqpoint{5.668450in}{0.621265in}}%
\pgfpathlineto{\pgfqpoint{5.673783in}{0.620782in}}%
\pgfpathlineto{\pgfqpoint{5.679116in}{0.620300in}}%
\pgfpathlineto{\pgfqpoint{5.684450in}{0.619819in}}%
\pgfpathlineto{\pgfqpoint{5.689783in}{0.619339in}}%
\pgfpathlineto{\pgfqpoint{5.695116in}{0.618860in}}%
\pgfpathlineto{\pgfqpoint{5.700449in}{0.618382in}}%
\pgfpathlineto{\pgfqpoint{5.705783in}{0.617905in}}%
\pgfpathlineto{\pgfqpoint{5.711116in}{0.617429in}}%
\pgfpathlineto{\pgfqpoint{5.716449in}{0.616953in}}%
\pgfpathlineto{\pgfqpoint{5.721782in}{0.616479in}}%
\pgfpathlineto{\pgfqpoint{5.727115in}{0.616006in}}%
\pgfpathlineto{\pgfqpoint{5.732449in}{0.615533in}}%
\pgfpathlineto{\pgfqpoint{5.737782in}{0.615062in}}%
\pgfpathlineto{\pgfqpoint{5.743115in}{0.614591in}}%
\pgfpathlineto{\pgfqpoint{5.748448in}{0.614121in}}%
\pgfpathlineto{\pgfqpoint{5.753781in}{0.613652in}}%
\pgfpathlineto{\pgfqpoint{5.759115in}{0.613185in}}%
\pgfpathlineto{\pgfqpoint{5.764448in}{0.612718in}}%
\pgfpathlineto{\pgfqpoint{5.769781in}{0.612252in}}%
\pgfpathlineto{\pgfqpoint{5.775114in}{0.611786in}}%
\pgfpathlineto{\pgfqpoint{5.780448in}{0.611322in}}%
\pgfpathlineto{\pgfqpoint{5.785781in}{0.610859in}}%
\pgfpathlineto{\pgfqpoint{5.791114in}{0.610396in}}%
\pgfpathlineto{\pgfqpoint{5.796447in}{0.609935in}}%
\pgfpathlineto{\pgfqpoint{5.801780in}{0.609474in}}%
\pgfpathlineto{\pgfqpoint{5.807114in}{0.609015in}}%
\pgfpathlineto{\pgfqpoint{5.812447in}{0.608556in}}%
\pgfpathlineto{\pgfqpoint{5.817780in}{0.608098in}}%
\pgfpathlineto{\pgfqpoint{5.823113in}{0.607641in}}%
\pgfpathlineto{\pgfqpoint{5.828446in}{0.607185in}}%
\pgfpathlineto{\pgfqpoint{5.833780in}{0.606729in}}%
\pgfpathlineto{\pgfqpoint{5.839113in}{0.606275in}}%
\pgfpathlineto{\pgfqpoint{5.844446in}{0.605821in}}%
\pgfpathlineto{\pgfqpoint{5.849779in}{0.605369in}}%
\pgfpathlineto{\pgfqpoint{5.855113in}{0.604917in}}%
\pgfpathlineto{\pgfqpoint{5.860446in}{0.604466in}}%
\pgfpathlineto{\pgfqpoint{5.865779in}{0.604016in}}%
\pgfpathlineto{\pgfqpoint{5.871112in}{0.603567in}}%
\pgfpathlineto{\pgfqpoint{5.876445in}{0.603119in}}%
\pgfpathlineto{\pgfqpoint{5.881779in}{0.602671in}}%
\pgfpathlineto{\pgfqpoint{5.887112in}{0.602225in}}%
\pgfpathlineto{\pgfqpoint{5.892445in}{0.601779in}}%
\pgfpathlineto{\pgfqpoint{5.897778in}{0.601334in}}%
\pgfpathlineto{\pgfqpoint{5.903112in}{0.600890in}}%
\pgfpathlineto{\pgfqpoint{5.908445in}{0.600447in}}%
\pgfpathlineto{\pgfqpoint{5.913778in}{0.600005in}}%
\pgfpathlineto{\pgfqpoint{5.919111in}{0.599563in}}%
\pgfpathlineto{\pgfqpoint{5.924444in}{0.599122in}}%
\pgfpathlineto{\pgfqpoint{5.929778in}{0.598683in}}%
\pgfpathlineto{\pgfqpoint{5.935111in}{0.598244in}}%
\pgfpathlineto{\pgfqpoint{5.940444in}{0.597806in}}%
\pgfpathlineto{\pgfqpoint{5.945777in}{0.597368in}}%
\pgfpathlineto{\pgfqpoint{5.951110in}{0.596932in}}%
\pgfpathlineto{\pgfqpoint{5.956444in}{0.596496in}}%
\pgfpathlineto{\pgfqpoint{5.961777in}{0.596061in}}%
\pgfpathlineto{\pgfqpoint{5.967110in}{0.595627in}}%
\pgfpathlineto{\pgfqpoint{5.972443in}{0.595194in}}%
\pgfpathlineto{\pgfqpoint{5.977777in}{0.594762in}}%
\pgfpathlineto{\pgfqpoint{5.983110in}{0.594330in}}%
\pgfpathlineto{\pgfqpoint{5.988443in}{0.593899in}}%
\pgfpathlineto{\pgfqpoint{5.993776in}{0.593469in}}%
\pgfpathlineto{\pgfqpoint{5.999109in}{0.593040in}}%
\pgfpathlineto{\pgfqpoint{6.004443in}{0.592612in}}%
\pgfpathlineto{\pgfqpoint{6.009776in}{0.592185in}}%
\pgfpathlineto{\pgfqpoint{6.015109in}{0.591758in}}%
\pgfpathlineto{\pgfqpoint{6.020442in}{0.591332in}}%
\pgfpathlineto{\pgfqpoint{6.025775in}{0.590907in}}%
\pgfpathlineto{\pgfqpoint{6.031109in}{0.590482in}}%
\pgfpathlineto{\pgfqpoint{6.036442in}{0.590059in}}%
\pgfpathlineto{\pgfqpoint{6.041775in}{0.589636in}}%
\pgfpathlineto{\pgfqpoint{6.047108in}{0.589214in}}%
\pgfpathlineto{\pgfqpoint{6.052442in}{0.588793in}}%
\pgfpathlineto{\pgfqpoint{6.057775in}{0.588373in}}%
\pgfpathlineto{\pgfqpoint{6.063108in}{0.587953in}}%
\pgfpathlineto{\pgfqpoint{6.068441in}{0.587534in}}%
\pgfpathlineto{\pgfqpoint{6.073774in}{0.587116in}}%
\pgfpathlineto{\pgfqpoint{6.079108in}{0.586699in}}%
\pgfpathlineto{\pgfqpoint{6.084441in}{0.586282in}}%
\pgfpathlineto{\pgfqpoint{6.089774in}{0.585867in}}%
\pgfpathlineto{\pgfqpoint{6.095107in}{0.585452in}}%
\pgfpathlineto{\pgfqpoint{6.100440in}{0.585038in}}%
\pgfpathlineto{\pgfqpoint{6.105774in}{0.584624in}}%
\pgfpathlineto{\pgfqpoint{6.111107in}{0.584212in}}%
\pgfpathlineto{\pgfqpoint{6.116440in}{0.583800in}}%
\pgfpathlineto{\pgfqpoint{6.121773in}{0.583388in}}%
\pgfpathlineto{\pgfqpoint{6.127107in}{0.582978in}}%
\pgfpathlineto{\pgfqpoint{6.132440in}{0.582568in}}%
\pgfpathlineto{\pgfqpoint{6.137773in}{0.582160in}}%
\pgfpathlineto{\pgfqpoint{6.143106in}{0.581751in}}%
\pgfpathlineto{\pgfqpoint{6.148439in}{0.581344in}}%
\pgfpathlineto{\pgfqpoint{6.153773in}{0.580937in}}%
\pgfpathlineto{\pgfqpoint{6.159106in}{0.580531in}}%
\pgfpathlineto{\pgfqpoint{6.164439in}{0.580126in}}%
\pgfpathlineto{\pgfqpoint{6.169772in}{0.579722in}}%
\pgfpathlineto{\pgfqpoint{6.175105in}{0.579318in}}%
\pgfpathlineto{\pgfqpoint{6.175105in}{0.579318in}}%
\pgfpathlineto{\pgfqpoint{6.183915in}{0.578665in}}%
\pgfpathlineto{\pgfqpoint{6.192724in}{0.578037in}}%
\pgfpathlineto{\pgfqpoint{6.201533in}{0.577433in}}%
\pgfpathlineto{\pgfqpoint{6.210343in}{0.576850in}}%
\pgfpathlineto{\pgfqpoint{6.219152in}{0.576287in}}%
\pgfpathlineto{\pgfqpoint{6.227961in}{0.575743in}}%
\pgfpathlineto{\pgfqpoint{6.236770in}{0.575217in}}%
\pgfpathlineto{\pgfqpoint{6.245580in}{0.574707in}}%
\pgfpathlineto{\pgfqpoint{6.254389in}{0.574213in}}%
\pgfpathlineto{\pgfqpoint{6.263198in}{0.573734in}}%
\pgfpathlineto{\pgfqpoint{6.272007in}{0.573269in}}%
\pgfpathlineto{\pgfqpoint{6.280817in}{0.572817in}}%
\pgfpathlineto{\pgfqpoint{6.289626in}{0.572377in}}%
\pgfpathlineto{\pgfqpoint{6.298435in}{0.571950in}}%
\pgfpathlineto{\pgfqpoint{6.307245in}{0.571534in}}%
\pgfpathlineto{\pgfqpoint{6.316054in}{0.571128in}}%
\pgfpathlineto{\pgfqpoint{6.324863in}{0.570733in}}%
\pgfpathlineto{\pgfqpoint{6.333672in}{0.570348in}}%
\pgfpathlineto{\pgfqpoint{6.342482in}{0.569972in}}%
\pgfpathlineto{\pgfqpoint{6.351291in}{0.569605in}}%
\pgfpathlineto{\pgfqpoint{6.360100in}{0.569246in}}%
\pgfpathlineto{\pgfqpoint{6.368909in}{0.568896in}}%
\pgfpathlineto{\pgfqpoint{6.377719in}{0.568554in}}%
\pgfpathlineto{\pgfqpoint{6.386528in}{0.568219in}}%
\pgfpathlineto{\pgfqpoint{6.395337in}{0.567891in}}%
\pgfpathlineto{\pgfqpoint{6.404146in}{0.567571in}}%
\pgfpathlineto{\pgfqpoint{6.412956in}{0.567257in}}%
\pgfpathlineto{\pgfqpoint{6.421765in}{0.566949in}}%
\pgfpathlineto{\pgfqpoint{6.430574in}{0.566648in}}%
\pgfpathlineto{\pgfqpoint{6.439384in}{0.566353in}}%
\pgfpathlineto{\pgfqpoint{6.448193in}{0.566064in}}%
\pgfpathlineto{\pgfqpoint{6.457002in}{0.565780in}}%
\pgfpathlineto{\pgfqpoint{6.465811in}{0.565502in}}%
\pgfpathlineto{\pgfqpoint{6.474621in}{0.565229in}}%
\pgfpathlineto{\pgfqpoint{6.483430in}{0.564961in}}%
\pgfpathlineto{\pgfqpoint{6.492239in}{0.564698in}}%
\pgfpathlineto{\pgfqpoint{6.501048in}{0.564439in}}%
\pgfpathlineto{\pgfqpoint{6.509858in}{0.564185in}}%
\pgfpathlineto{\pgfqpoint{6.518667in}{0.563936in}}%
\pgfpathlineto{\pgfqpoint{6.527476in}{0.563691in}}%
\pgfpathlineto{\pgfqpoint{6.536285in}{0.563450in}}%
\pgfpathlineto{\pgfqpoint{6.545095in}{0.563213in}}%
\pgfpathlineto{\pgfqpoint{6.553904in}{0.562980in}}%
\pgfpathlineto{\pgfqpoint{6.562713in}{0.562751in}}%
\pgfpathlineto{\pgfqpoint{6.571523in}{0.562526in}}%
\pgfpathlineto{\pgfqpoint{6.580332in}{0.562305in}}%
\pgfpathlineto{\pgfqpoint{6.589141in}{0.562086in}}%
\pgfpathlineto{\pgfqpoint{6.597950in}{0.561872in}}%
\pgfpathlineto{\pgfqpoint{6.606760in}{0.561660in}}%
\pgfpathlineto{\pgfqpoint{6.615569in}{0.561452in}}%
\pgfpathlineto{\pgfqpoint{6.624378in}{0.561247in}}%
\pgfpathlineto{\pgfqpoint{6.633187in}{0.561045in}}%
\pgfpathlineto{\pgfqpoint{6.641997in}{0.560846in}}%
\pgfpathlineto{\pgfqpoint{6.650806in}{0.560650in}}%
\pgfpathlineto{\pgfqpoint{6.659615in}{0.560457in}}%
\pgfpathlineto{\pgfqpoint{6.668425in}{0.560267in}}%
\pgfpathlineto{\pgfqpoint{6.677234in}{0.560079in}}%
\pgfpathlineto{\pgfqpoint{6.686043in}{0.559894in}}%
\pgfpathlineto{\pgfqpoint{6.694852in}{0.559712in}}%
\pgfpathlineto{\pgfqpoint{6.703662in}{0.559532in}}%
\pgfpathlineto{\pgfqpoint{6.712471in}{0.559355in}}%
\pgfpathlineto{\pgfqpoint{6.721280in}{0.559180in}}%
\pgfpathlineto{\pgfqpoint{6.730089in}{0.559007in}}%
\pgfpathlineto{\pgfqpoint{6.738899in}{0.558837in}}%
\pgfpathlineto{\pgfqpoint{6.747708in}{0.558669in}}%
\pgfpathlineto{\pgfqpoint{6.756517in}{0.558503in}}%
\pgfpathlineto{\pgfqpoint{6.765326in}{0.558339in}}%
\pgfpathlineto{\pgfqpoint{6.774136in}{0.558177in}}%
\pgfpathlineto{\pgfqpoint{6.782945in}{0.558018in}}%
\pgfpathlineto{\pgfqpoint{6.791754in}{0.557860in}}%
\pgfpathlineto{\pgfqpoint{6.800564in}{0.557705in}}%
\pgfpathlineto{\pgfqpoint{6.809373in}{0.557551in}}%
\pgfpathlineto{\pgfqpoint{6.818182in}{0.557399in}}%
\pgfpathlineto{\pgfqpoint{6.826991in}{0.557249in}}%
\pgfpathlineto{\pgfqpoint{6.835801in}{0.557101in}}%
\pgfpathlineto{\pgfqpoint{6.844610in}{0.556955in}}%
\pgfpathlineto{\pgfqpoint{6.853419in}{0.556811in}}%
\pgfpathlineto{\pgfqpoint{6.862228in}{0.556668in}}%
\pgfpathlineto{\pgfqpoint{6.871038in}{0.556527in}}%
\pgfpathlineto{\pgfqpoint{6.879847in}{0.556387in}}%
\pgfpathlineto{\pgfqpoint{6.888656in}{0.556249in}}%
\pgfpathlineto{\pgfqpoint{6.897465in}{0.556113in}}%
\pgfpathlineto{\pgfqpoint{6.906275in}{0.555978in}}%
\pgfpathlineto{\pgfqpoint{6.915084in}{0.555845in}}%
\pgfpathlineto{\pgfqpoint{6.923893in}{0.555713in}}%
\pgfpathlineto{\pgfqpoint{6.932703in}{0.555583in}}%
\pgfpathlineto{\pgfqpoint{6.941512in}{0.555454in}}%
\pgfpathlineto{\pgfqpoint{6.950321in}{0.555327in}}%
\pgfpathlineto{\pgfqpoint{6.959130in}{0.555201in}}%
\pgfpathlineto{\pgfqpoint{6.967940in}{0.555076in}}%
\pgfpathlineto{\pgfqpoint{6.976749in}{0.554953in}}%
\pgfpathlineto{\pgfqpoint{6.985558in}{0.554831in}}%
\pgfpathlineto{\pgfqpoint{6.994367in}{0.554710in}}%
\pgfpathlineto{\pgfqpoint{7.003177in}{0.554591in}}%
\pgfpathlineto{\pgfqpoint{7.011986in}{0.554472in}}%
\pgfpathlineto{\pgfqpoint{7.020795in}{0.554355in}}%
\pgfpathlineto{\pgfqpoint{7.029605in}{0.554240in}}%
\pgfpathlineto{\pgfqpoint{7.038414in}{0.554125in}}%
\pgfpathlineto{\pgfqpoint{7.047223in}{0.554012in}}%
\pgfpathlineto{\pgfqpoint{7.047223in}{0.136829in}}%
\pgfpathlineto{\pgfqpoint{7.047223in}{0.136829in}}%
\pgfpathlineto{\pgfqpoint{7.038414in}{0.136829in}}%
\pgfpathlineto{\pgfqpoint{7.029605in}{0.136829in}}%
\pgfpathlineto{\pgfqpoint{7.020795in}{0.136829in}}%
\pgfpathlineto{\pgfqpoint{7.011986in}{0.136829in}}%
\pgfpathlineto{\pgfqpoint{7.003177in}{0.136829in}}%
\pgfpathlineto{\pgfqpoint{6.994367in}{0.136829in}}%
\pgfpathlineto{\pgfqpoint{6.985558in}{0.136829in}}%
\pgfpathlineto{\pgfqpoint{6.976749in}{0.136829in}}%
\pgfpathlineto{\pgfqpoint{6.967940in}{0.136829in}}%
\pgfpathlineto{\pgfqpoint{6.959130in}{0.136829in}}%
\pgfpathlineto{\pgfqpoint{6.950321in}{0.136829in}}%
\pgfpathlineto{\pgfqpoint{6.941512in}{0.136829in}}%
\pgfpathlineto{\pgfqpoint{6.932703in}{0.136829in}}%
\pgfpathlineto{\pgfqpoint{6.923893in}{0.136829in}}%
\pgfpathlineto{\pgfqpoint{6.915084in}{0.136829in}}%
\pgfpathlineto{\pgfqpoint{6.906275in}{0.136829in}}%
\pgfpathlineto{\pgfqpoint{6.897465in}{0.136829in}}%
\pgfpathlineto{\pgfqpoint{6.888656in}{0.136829in}}%
\pgfpathlineto{\pgfqpoint{6.879847in}{0.136829in}}%
\pgfpathlineto{\pgfqpoint{6.871038in}{0.136829in}}%
\pgfpathlineto{\pgfqpoint{6.862228in}{0.136829in}}%
\pgfpathlineto{\pgfqpoint{6.853419in}{0.136829in}}%
\pgfpathlineto{\pgfqpoint{6.844610in}{0.136829in}}%
\pgfpathlineto{\pgfqpoint{6.835801in}{0.136829in}}%
\pgfpathlineto{\pgfqpoint{6.826991in}{0.136829in}}%
\pgfpathlineto{\pgfqpoint{6.818182in}{0.136829in}}%
\pgfpathlineto{\pgfqpoint{6.809373in}{0.136829in}}%
\pgfpathlineto{\pgfqpoint{6.800564in}{0.136829in}}%
\pgfpathlineto{\pgfqpoint{6.791754in}{0.136829in}}%
\pgfpathlineto{\pgfqpoint{6.782945in}{0.136829in}}%
\pgfpathlineto{\pgfqpoint{6.774136in}{0.136829in}}%
\pgfpathlineto{\pgfqpoint{6.765326in}{0.136829in}}%
\pgfpathlineto{\pgfqpoint{6.756517in}{0.136829in}}%
\pgfpathlineto{\pgfqpoint{6.747708in}{0.136829in}}%
\pgfpathlineto{\pgfqpoint{6.738899in}{0.136829in}}%
\pgfpathlineto{\pgfqpoint{6.730089in}{0.136829in}}%
\pgfpathlineto{\pgfqpoint{6.721280in}{0.136829in}}%
\pgfpathlineto{\pgfqpoint{6.712471in}{0.136829in}}%
\pgfpathlineto{\pgfqpoint{6.703662in}{0.136829in}}%
\pgfpathlineto{\pgfqpoint{6.694852in}{0.136829in}}%
\pgfpathlineto{\pgfqpoint{6.686043in}{0.136829in}}%
\pgfpathlineto{\pgfqpoint{6.677234in}{0.136829in}}%
\pgfpathlineto{\pgfqpoint{6.668425in}{0.136829in}}%
\pgfpathlineto{\pgfqpoint{6.659615in}{0.136829in}}%
\pgfpathlineto{\pgfqpoint{6.650806in}{0.136829in}}%
\pgfpathlineto{\pgfqpoint{6.641997in}{0.136829in}}%
\pgfpathlineto{\pgfqpoint{6.633187in}{0.136829in}}%
\pgfpathlineto{\pgfqpoint{6.624378in}{0.136829in}}%
\pgfpathlineto{\pgfqpoint{6.615569in}{0.136829in}}%
\pgfpathlineto{\pgfqpoint{6.606760in}{0.136829in}}%
\pgfpathlineto{\pgfqpoint{6.597950in}{0.136829in}}%
\pgfpathlineto{\pgfqpoint{6.589141in}{0.136829in}}%
\pgfpathlineto{\pgfqpoint{6.580332in}{0.136829in}}%
\pgfpathlineto{\pgfqpoint{6.571523in}{0.136829in}}%
\pgfpathlineto{\pgfqpoint{6.562713in}{0.136829in}}%
\pgfpathlineto{\pgfqpoint{6.553904in}{0.136829in}}%
\pgfpathlineto{\pgfqpoint{6.545095in}{0.136829in}}%
\pgfpathlineto{\pgfqpoint{6.536285in}{0.136829in}}%
\pgfpathlineto{\pgfqpoint{6.527476in}{0.136829in}}%
\pgfpathlineto{\pgfqpoint{6.518667in}{0.136829in}}%
\pgfpathlineto{\pgfqpoint{6.509858in}{0.136829in}}%
\pgfpathlineto{\pgfqpoint{6.501048in}{0.136829in}}%
\pgfpathlineto{\pgfqpoint{6.492239in}{0.136829in}}%
\pgfpathlineto{\pgfqpoint{6.483430in}{0.136829in}}%
\pgfpathlineto{\pgfqpoint{6.474621in}{0.136829in}}%
\pgfpathlineto{\pgfqpoint{6.465811in}{0.136829in}}%
\pgfpathlineto{\pgfqpoint{6.457002in}{0.136829in}}%
\pgfpathlineto{\pgfqpoint{6.448193in}{0.136829in}}%
\pgfpathlineto{\pgfqpoint{6.439384in}{0.136829in}}%
\pgfpathlineto{\pgfqpoint{6.430574in}{0.136829in}}%
\pgfpathlineto{\pgfqpoint{6.421765in}{0.136829in}}%
\pgfpathlineto{\pgfqpoint{6.412956in}{0.136829in}}%
\pgfpathlineto{\pgfqpoint{6.404146in}{0.136829in}}%
\pgfpathlineto{\pgfqpoint{6.395337in}{0.136829in}}%
\pgfpathlineto{\pgfqpoint{6.386528in}{0.136829in}}%
\pgfpathlineto{\pgfqpoint{6.377719in}{0.136829in}}%
\pgfpathlineto{\pgfqpoint{6.368909in}{0.136829in}}%
\pgfpathlineto{\pgfqpoint{6.360100in}{0.136829in}}%
\pgfpathlineto{\pgfqpoint{6.351291in}{0.136829in}}%
\pgfpathlineto{\pgfqpoint{6.342482in}{0.136829in}}%
\pgfpathlineto{\pgfqpoint{6.333672in}{0.136829in}}%
\pgfpathlineto{\pgfqpoint{6.324863in}{0.136829in}}%
\pgfpathlineto{\pgfqpoint{6.316054in}{0.136829in}}%
\pgfpathlineto{\pgfqpoint{6.307245in}{0.136829in}}%
\pgfpathlineto{\pgfqpoint{6.298435in}{0.136829in}}%
\pgfpathlineto{\pgfqpoint{6.289626in}{0.136829in}}%
\pgfpathlineto{\pgfqpoint{6.280817in}{0.136829in}}%
\pgfpathlineto{\pgfqpoint{6.272007in}{0.136829in}}%
\pgfpathlineto{\pgfqpoint{6.263198in}{0.136829in}}%
\pgfpathlineto{\pgfqpoint{6.254389in}{0.136829in}}%
\pgfpathlineto{\pgfqpoint{6.245580in}{0.136829in}}%
\pgfpathlineto{\pgfqpoint{6.236770in}{0.136829in}}%
\pgfpathlineto{\pgfqpoint{6.227961in}{0.136829in}}%
\pgfpathlineto{\pgfqpoint{6.219152in}{0.136829in}}%
\pgfpathlineto{\pgfqpoint{6.210343in}{0.136829in}}%
\pgfpathlineto{\pgfqpoint{6.201533in}{0.136829in}}%
\pgfpathlineto{\pgfqpoint{6.192724in}{0.136829in}}%
\pgfpathlineto{\pgfqpoint{6.183915in}{0.136829in}}%
\pgfpathlineto{\pgfqpoint{6.175105in}{0.136829in}}%
\pgfpathlineto{\pgfqpoint{6.175105in}{0.136829in}}%
\pgfpathlineto{\pgfqpoint{6.169772in}{0.136829in}}%
\pgfpathlineto{\pgfqpoint{6.164439in}{0.136829in}}%
\pgfpathlineto{\pgfqpoint{6.159106in}{0.136829in}}%
\pgfpathlineto{\pgfqpoint{6.153773in}{0.136829in}}%
\pgfpathlineto{\pgfqpoint{6.148439in}{0.136829in}}%
\pgfpathlineto{\pgfqpoint{6.143106in}{0.136829in}}%
\pgfpathlineto{\pgfqpoint{6.137773in}{0.136829in}}%
\pgfpathlineto{\pgfqpoint{6.132440in}{0.136829in}}%
\pgfpathlineto{\pgfqpoint{6.127107in}{0.136829in}}%
\pgfpathlineto{\pgfqpoint{6.121773in}{0.136829in}}%
\pgfpathlineto{\pgfqpoint{6.116440in}{0.136829in}}%
\pgfpathlineto{\pgfqpoint{6.111107in}{0.136829in}}%
\pgfpathlineto{\pgfqpoint{6.105774in}{0.136829in}}%
\pgfpathlineto{\pgfqpoint{6.100440in}{0.136829in}}%
\pgfpathlineto{\pgfqpoint{6.095107in}{0.136829in}}%
\pgfpathlineto{\pgfqpoint{6.089774in}{0.136829in}}%
\pgfpathlineto{\pgfqpoint{6.084441in}{0.136829in}}%
\pgfpathlineto{\pgfqpoint{6.079108in}{0.136829in}}%
\pgfpathlineto{\pgfqpoint{6.073774in}{0.136829in}}%
\pgfpathlineto{\pgfqpoint{6.068441in}{0.136829in}}%
\pgfpathlineto{\pgfqpoint{6.063108in}{0.136829in}}%
\pgfpathlineto{\pgfqpoint{6.057775in}{0.136829in}}%
\pgfpathlineto{\pgfqpoint{6.052442in}{0.136829in}}%
\pgfpathlineto{\pgfqpoint{6.047108in}{0.136829in}}%
\pgfpathlineto{\pgfqpoint{6.041775in}{0.136829in}}%
\pgfpathlineto{\pgfqpoint{6.036442in}{0.136829in}}%
\pgfpathlineto{\pgfqpoint{6.031109in}{0.136829in}}%
\pgfpathlineto{\pgfqpoint{6.025775in}{0.136829in}}%
\pgfpathlineto{\pgfqpoint{6.020442in}{0.136829in}}%
\pgfpathlineto{\pgfqpoint{6.015109in}{0.136829in}}%
\pgfpathlineto{\pgfqpoint{6.009776in}{0.136829in}}%
\pgfpathlineto{\pgfqpoint{6.004443in}{0.136829in}}%
\pgfpathlineto{\pgfqpoint{5.999109in}{0.136829in}}%
\pgfpathlineto{\pgfqpoint{5.993776in}{0.136829in}}%
\pgfpathlineto{\pgfqpoint{5.988443in}{0.136829in}}%
\pgfpathlineto{\pgfqpoint{5.983110in}{0.136829in}}%
\pgfpathlineto{\pgfqpoint{5.977777in}{0.136829in}}%
\pgfpathlineto{\pgfqpoint{5.972443in}{0.136829in}}%
\pgfpathlineto{\pgfqpoint{5.967110in}{0.136829in}}%
\pgfpathlineto{\pgfqpoint{5.961777in}{0.136829in}}%
\pgfpathlineto{\pgfqpoint{5.956444in}{0.136829in}}%
\pgfpathlineto{\pgfqpoint{5.951110in}{0.136829in}}%
\pgfpathlineto{\pgfqpoint{5.945777in}{0.136829in}}%
\pgfpathlineto{\pgfqpoint{5.940444in}{0.136829in}}%
\pgfpathlineto{\pgfqpoint{5.935111in}{0.136829in}}%
\pgfpathlineto{\pgfqpoint{5.929778in}{0.136829in}}%
\pgfpathlineto{\pgfqpoint{5.924444in}{0.136829in}}%
\pgfpathlineto{\pgfqpoint{5.919111in}{0.136829in}}%
\pgfpathlineto{\pgfqpoint{5.913778in}{0.136829in}}%
\pgfpathlineto{\pgfqpoint{5.908445in}{0.136829in}}%
\pgfpathlineto{\pgfqpoint{5.903112in}{0.136829in}}%
\pgfpathlineto{\pgfqpoint{5.897778in}{0.136829in}}%
\pgfpathlineto{\pgfqpoint{5.892445in}{0.136829in}}%
\pgfpathlineto{\pgfqpoint{5.887112in}{0.136829in}}%
\pgfpathlineto{\pgfqpoint{5.881779in}{0.136829in}}%
\pgfpathlineto{\pgfqpoint{5.876445in}{0.136829in}}%
\pgfpathlineto{\pgfqpoint{5.871112in}{0.136829in}}%
\pgfpathlineto{\pgfqpoint{5.865779in}{0.136829in}}%
\pgfpathlineto{\pgfqpoint{5.860446in}{0.136829in}}%
\pgfpathlineto{\pgfqpoint{5.855113in}{0.136829in}}%
\pgfpathlineto{\pgfqpoint{5.849779in}{0.136829in}}%
\pgfpathlineto{\pgfqpoint{5.844446in}{0.136829in}}%
\pgfpathlineto{\pgfqpoint{5.839113in}{0.136829in}}%
\pgfpathlineto{\pgfqpoint{5.833780in}{0.136829in}}%
\pgfpathlineto{\pgfqpoint{5.828446in}{0.136829in}}%
\pgfpathlineto{\pgfqpoint{5.823113in}{0.136829in}}%
\pgfpathlineto{\pgfqpoint{5.817780in}{0.136829in}}%
\pgfpathlineto{\pgfqpoint{5.812447in}{0.136829in}}%
\pgfpathlineto{\pgfqpoint{5.807114in}{0.136829in}}%
\pgfpathlineto{\pgfqpoint{5.801780in}{0.136829in}}%
\pgfpathlineto{\pgfqpoint{5.796447in}{0.136829in}}%
\pgfpathlineto{\pgfqpoint{5.791114in}{0.136829in}}%
\pgfpathlineto{\pgfqpoint{5.785781in}{0.136829in}}%
\pgfpathlineto{\pgfqpoint{5.780448in}{0.136829in}}%
\pgfpathlineto{\pgfqpoint{5.775114in}{0.136829in}}%
\pgfpathlineto{\pgfqpoint{5.769781in}{0.136829in}}%
\pgfpathlineto{\pgfqpoint{5.764448in}{0.136829in}}%
\pgfpathlineto{\pgfqpoint{5.759115in}{0.136829in}}%
\pgfpathlineto{\pgfqpoint{5.753781in}{0.136829in}}%
\pgfpathlineto{\pgfqpoint{5.748448in}{0.136829in}}%
\pgfpathlineto{\pgfqpoint{5.743115in}{0.136829in}}%
\pgfpathlineto{\pgfqpoint{5.737782in}{0.136829in}}%
\pgfpathlineto{\pgfqpoint{5.732449in}{0.136829in}}%
\pgfpathlineto{\pgfqpoint{5.727115in}{0.136829in}}%
\pgfpathlineto{\pgfqpoint{5.721782in}{0.136829in}}%
\pgfpathlineto{\pgfqpoint{5.716449in}{0.136829in}}%
\pgfpathlineto{\pgfqpoint{5.711116in}{0.136829in}}%
\pgfpathlineto{\pgfqpoint{5.705783in}{0.136829in}}%
\pgfpathlineto{\pgfqpoint{5.700449in}{0.136829in}}%
\pgfpathlineto{\pgfqpoint{5.695116in}{0.136829in}}%
\pgfpathlineto{\pgfqpoint{5.689783in}{0.136829in}}%
\pgfpathlineto{\pgfqpoint{5.684450in}{0.136829in}}%
\pgfpathlineto{\pgfqpoint{5.679116in}{0.136829in}}%
\pgfpathlineto{\pgfqpoint{5.673783in}{0.136829in}}%
\pgfpathlineto{\pgfqpoint{5.668450in}{0.136829in}}%
\pgfpathlineto{\pgfqpoint{5.663117in}{0.136829in}}%
\pgfpathlineto{\pgfqpoint{5.657784in}{0.136829in}}%
\pgfpathlineto{\pgfqpoint{5.652450in}{0.136829in}}%
\pgfpathlineto{\pgfqpoint{5.647117in}{0.136829in}}%
\pgfpathlineto{\pgfqpoint{5.641784in}{0.136829in}}%
\pgfpathlineto{\pgfqpoint{5.636451in}{0.136829in}}%
\pgfpathlineto{\pgfqpoint{5.631118in}{0.136829in}}%
\pgfpathlineto{\pgfqpoint{5.625784in}{0.136829in}}%
\pgfpathlineto{\pgfqpoint{5.620451in}{0.136829in}}%
\pgfpathlineto{\pgfqpoint{5.615118in}{0.136829in}}%
\pgfpathlineto{\pgfqpoint{5.609785in}{0.136829in}}%
\pgfpathlineto{\pgfqpoint{5.604451in}{0.136829in}}%
\pgfpathlineto{\pgfqpoint{5.599118in}{0.136829in}}%
\pgfpathlineto{\pgfqpoint{5.593785in}{0.136829in}}%
\pgfpathlineto{\pgfqpoint{5.588452in}{0.136829in}}%
\pgfpathlineto{\pgfqpoint{5.583119in}{0.136829in}}%
\pgfpathlineto{\pgfqpoint{5.577785in}{0.136829in}}%
\pgfpathlineto{\pgfqpoint{5.572452in}{0.136829in}}%
\pgfpathlineto{\pgfqpoint{5.567119in}{0.136829in}}%
\pgfpathlineto{\pgfqpoint{5.561786in}{0.136829in}}%
\pgfpathlineto{\pgfqpoint{5.556452in}{0.136829in}}%
\pgfpathlineto{\pgfqpoint{5.551119in}{0.136829in}}%
\pgfpathlineto{\pgfqpoint{5.545786in}{0.136829in}}%
\pgfpathlineto{\pgfqpoint{5.540453in}{0.136829in}}%
\pgfpathlineto{\pgfqpoint{5.535120in}{0.136829in}}%
\pgfpathlineto{\pgfqpoint{5.529786in}{0.136829in}}%
\pgfpathlineto{\pgfqpoint{5.524453in}{0.136829in}}%
\pgfpathlineto{\pgfqpoint{5.519120in}{0.136829in}}%
\pgfpathlineto{\pgfqpoint{5.513787in}{0.136829in}}%
\pgfpathlineto{\pgfqpoint{5.508454in}{0.136829in}}%
\pgfpathlineto{\pgfqpoint{5.503120in}{0.136829in}}%
\pgfpathlineto{\pgfqpoint{5.497787in}{0.136829in}}%
\pgfpathlineto{\pgfqpoint{5.492454in}{0.136829in}}%
\pgfpathlineto{\pgfqpoint{5.487121in}{0.136829in}}%
\pgfpathlineto{\pgfqpoint{5.481787in}{0.136829in}}%
\pgfpathlineto{\pgfqpoint{5.476454in}{0.136829in}}%
\pgfpathlineto{\pgfqpoint{5.471121in}{0.136829in}}%
\pgfpathlineto{\pgfqpoint{5.465788in}{0.136829in}}%
\pgfpathlineto{\pgfqpoint{5.460455in}{0.136829in}}%
\pgfpathlineto{\pgfqpoint{5.455121in}{0.136829in}}%
\pgfpathlineto{\pgfqpoint{5.449788in}{0.136829in}}%
\pgfpathlineto{\pgfqpoint{5.444455in}{0.136829in}}%
\pgfpathlineto{\pgfqpoint{5.439122in}{0.136829in}}%
\pgfpathlineto{\pgfqpoint{5.433789in}{0.136829in}}%
\pgfpathlineto{\pgfqpoint{5.428455in}{0.136829in}}%
\pgfpathlineto{\pgfqpoint{5.423122in}{0.136829in}}%
\pgfpathlineto{\pgfqpoint{5.417789in}{0.136829in}}%
\pgfpathlineto{\pgfqpoint{5.412456in}{0.136829in}}%
\pgfpathlineto{\pgfqpoint{5.407122in}{0.136829in}}%
\pgfpathlineto{\pgfqpoint{5.401789in}{0.136829in}}%
\pgfpathlineto{\pgfqpoint{5.396456in}{0.136829in}}%
\pgfpathlineto{\pgfqpoint{5.391123in}{0.136829in}}%
\pgfpathlineto{\pgfqpoint{5.385790in}{0.136829in}}%
\pgfpathlineto{\pgfqpoint{5.380456in}{0.136829in}}%
\pgfpathlineto{\pgfqpoint{5.375123in}{0.136829in}}%
\pgfpathlineto{\pgfqpoint{5.369790in}{0.136829in}}%
\pgfpathlineto{\pgfqpoint{5.364457in}{0.136829in}}%
\pgfpathlineto{\pgfqpoint{5.359124in}{0.136829in}}%
\pgfpathlineto{\pgfqpoint{5.353790in}{0.136829in}}%
\pgfpathlineto{\pgfqpoint{5.348457in}{0.136829in}}%
\pgfpathlineto{\pgfqpoint{5.343124in}{0.136829in}}%
\pgfpathlineto{\pgfqpoint{5.337791in}{0.136829in}}%
\pgfpathlineto{\pgfqpoint{5.332457in}{0.136829in}}%
\pgfpathlineto{\pgfqpoint{5.327124in}{0.136829in}}%
\pgfpathlineto{\pgfqpoint{5.321791in}{0.136829in}}%
\pgfpathlineto{\pgfqpoint{5.316458in}{0.136829in}}%
\pgfpathlineto{\pgfqpoint{5.311125in}{0.136829in}}%
\pgfpathlineto{\pgfqpoint{5.305791in}{0.136829in}}%
\pgfpathlineto{\pgfqpoint{5.300458in}{0.136829in}}%
\pgfpathlineto{\pgfqpoint{5.295125in}{0.136829in}}%
\pgfpathlineto{\pgfqpoint{5.289792in}{0.136829in}}%
\pgfpathlineto{\pgfqpoint{5.284458in}{0.136829in}}%
\pgfpathlineto{\pgfqpoint{5.279125in}{0.136829in}}%
\pgfpathlineto{\pgfqpoint{5.273792in}{0.136829in}}%
\pgfpathlineto{\pgfqpoint{5.268459in}{0.136829in}}%
\pgfpathlineto{\pgfqpoint{5.263126in}{0.136829in}}%
\pgfpathlineto{\pgfqpoint{5.257792in}{0.136829in}}%
\pgfpathlineto{\pgfqpoint{5.252459in}{0.136829in}}%
\pgfpathlineto{\pgfqpoint{5.247126in}{0.136829in}}%
\pgfpathlineto{\pgfqpoint{5.241793in}{0.136829in}}%
\pgfpathlineto{\pgfqpoint{5.236460in}{0.136829in}}%
\pgfpathlineto{\pgfqpoint{5.231126in}{0.136829in}}%
\pgfpathlineto{\pgfqpoint{5.225793in}{0.136829in}}%
\pgfpathlineto{\pgfqpoint{5.220460in}{0.136829in}}%
\pgfpathlineto{\pgfqpoint{5.215127in}{0.136829in}}%
\pgfpathlineto{\pgfqpoint{5.209793in}{0.136829in}}%
\pgfpathlineto{\pgfqpoint{5.204460in}{0.136829in}}%
\pgfpathlineto{\pgfqpoint{5.199127in}{0.136829in}}%
\pgfpathlineto{\pgfqpoint{5.193794in}{0.136829in}}%
\pgfpathlineto{\pgfqpoint{5.188461in}{0.136829in}}%
\pgfpathlineto{\pgfqpoint{5.183127in}{0.136829in}}%
\pgfpathlineto{\pgfqpoint{5.177794in}{0.136829in}}%
\pgfpathlineto{\pgfqpoint{5.172461in}{0.136829in}}%
\pgfpathlineto{\pgfqpoint{5.167128in}{0.136829in}}%
\pgfpathlineto{\pgfqpoint{5.161795in}{0.136829in}}%
\pgfpathlineto{\pgfqpoint{5.156461in}{0.136829in}}%
\pgfpathlineto{\pgfqpoint{5.151128in}{0.136829in}}%
\pgfpathlineto{\pgfqpoint{5.145795in}{0.136829in}}%
\pgfpathlineto{\pgfqpoint{5.140462in}{0.136829in}}%
\pgfpathlineto{\pgfqpoint{5.135128in}{0.136829in}}%
\pgfpathlineto{\pgfqpoint{5.129795in}{0.136829in}}%
\pgfpathlineto{\pgfqpoint{5.124462in}{0.136829in}}%
\pgfpathlineto{\pgfqpoint{5.119129in}{0.136829in}}%
\pgfpathlineto{\pgfqpoint{5.113796in}{0.136829in}}%
\pgfpathlineto{\pgfqpoint{5.108462in}{0.136829in}}%
\pgfpathlineto{\pgfqpoint{5.103129in}{0.136829in}}%
\pgfpathlineto{\pgfqpoint{5.097796in}{0.136829in}}%
\pgfpathlineto{\pgfqpoint{5.092463in}{0.136829in}}%
\pgfpathlineto{\pgfqpoint{5.087130in}{0.136829in}}%
\pgfpathlineto{\pgfqpoint{5.081796in}{0.136829in}}%
\pgfpathlineto{\pgfqpoint{5.076463in}{0.136829in}}%
\pgfpathlineto{\pgfqpoint{5.071130in}{0.136829in}}%
\pgfpathlineto{\pgfqpoint{5.065797in}{0.136829in}}%
\pgfpathlineto{\pgfqpoint{5.060463in}{0.136829in}}%
\pgfpathlineto{\pgfqpoint{5.055130in}{0.136829in}}%
\pgfpathlineto{\pgfqpoint{5.049797in}{0.136829in}}%
\pgfpathlineto{\pgfqpoint{5.044464in}{0.136829in}}%
\pgfpathlineto{\pgfqpoint{5.039131in}{0.136829in}}%
\pgfpathlineto{\pgfqpoint{5.033797in}{0.136829in}}%
\pgfpathlineto{\pgfqpoint{5.028464in}{0.136829in}}%
\pgfpathlineto{\pgfqpoint{5.023131in}{0.136829in}}%
\pgfpathlineto{\pgfqpoint{5.017798in}{0.136829in}}%
\pgfpathlineto{\pgfqpoint{5.012464in}{0.136829in}}%
\pgfpathlineto{\pgfqpoint{5.007131in}{0.136829in}}%
\pgfpathlineto{\pgfqpoint{5.001798in}{0.136829in}}%
\pgfpathlineto{\pgfqpoint{4.996465in}{0.136829in}}%
\pgfpathlineto{\pgfqpoint{4.991132in}{0.136829in}}%
\pgfpathlineto{\pgfqpoint{4.985798in}{0.136829in}}%
\pgfpathlineto{\pgfqpoint{4.980465in}{0.136829in}}%
\pgfpathlineto{\pgfqpoint{4.975132in}{0.136829in}}%
\pgfpathlineto{\pgfqpoint{4.969799in}{0.136829in}}%
\pgfpathlineto{\pgfqpoint{4.964466in}{0.136829in}}%
\pgfpathlineto{\pgfqpoint{4.959132in}{0.136829in}}%
\pgfpathlineto{\pgfqpoint{4.953799in}{0.136829in}}%
\pgfpathlineto{\pgfqpoint{4.948466in}{0.136829in}}%
\pgfpathlineto{\pgfqpoint{4.943133in}{0.136829in}}%
\pgfpathlineto{\pgfqpoint{4.937799in}{0.136829in}}%
\pgfpathlineto{\pgfqpoint{4.932466in}{0.136829in}}%
\pgfpathlineto{\pgfqpoint{4.927133in}{0.136829in}}%
\pgfpathlineto{\pgfqpoint{4.921800in}{0.136829in}}%
\pgfpathlineto{\pgfqpoint{4.916467in}{0.136829in}}%
\pgfpathlineto{\pgfqpoint{4.911133in}{0.136829in}}%
\pgfpathlineto{\pgfqpoint{4.905800in}{0.136829in}}%
\pgfpathlineto{\pgfqpoint{4.900467in}{0.136829in}}%
\pgfpathlineto{\pgfqpoint{4.895134in}{0.136829in}}%
\pgfpathlineto{\pgfqpoint{4.889801in}{0.136829in}}%
\pgfpathlineto{\pgfqpoint{4.884467in}{0.136829in}}%
\pgfpathlineto{\pgfqpoint{4.879134in}{0.136829in}}%
\pgfpathlineto{\pgfqpoint{4.873801in}{0.136829in}}%
\pgfpathlineto{\pgfqpoint{4.868468in}{0.136829in}}%
\pgfpathlineto{\pgfqpoint{4.863134in}{0.136829in}}%
\pgfpathlineto{\pgfqpoint{4.857801in}{0.136829in}}%
\pgfpathlineto{\pgfqpoint{4.852468in}{0.136829in}}%
\pgfpathlineto{\pgfqpoint{4.847135in}{0.136829in}}%
\pgfpathlineto{\pgfqpoint{4.841802in}{0.136829in}}%
\pgfpathlineto{\pgfqpoint{4.836468in}{0.136829in}}%
\pgfpathlineto{\pgfqpoint{4.831135in}{0.136829in}}%
\pgfpathlineto{\pgfqpoint{4.825802in}{0.136829in}}%
\pgfpathlineto{\pgfqpoint{4.820469in}{0.136829in}}%
\pgfpathlineto{\pgfqpoint{4.815136in}{0.136829in}}%
\pgfpathlineto{\pgfqpoint{4.809802in}{0.136829in}}%
\pgfpathlineto{\pgfqpoint{4.804469in}{0.136829in}}%
\pgfpathlineto{\pgfqpoint{4.799136in}{0.136829in}}%
\pgfpathlineto{\pgfqpoint{4.793803in}{0.136829in}}%
\pgfpathlineto{\pgfqpoint{4.788469in}{0.136829in}}%
\pgfpathlineto{\pgfqpoint{4.783136in}{0.136829in}}%
\pgfpathlineto{\pgfqpoint{4.777803in}{0.136829in}}%
\pgfpathlineto{\pgfqpoint{4.772470in}{0.136829in}}%
\pgfpathlineto{\pgfqpoint{4.767137in}{0.136829in}}%
\pgfpathlineto{\pgfqpoint{4.761803in}{0.136829in}}%
\pgfpathlineto{\pgfqpoint{4.756470in}{0.136829in}}%
\pgfpathlineto{\pgfqpoint{4.751137in}{0.136829in}}%
\pgfpathlineto{\pgfqpoint{4.745804in}{0.136829in}}%
\pgfpathlineto{\pgfqpoint{4.740470in}{0.136829in}}%
\pgfpathlineto{\pgfqpoint{4.735137in}{0.136829in}}%
\pgfpathlineto{\pgfqpoint{4.729804in}{0.136829in}}%
\pgfpathlineto{\pgfqpoint{4.724471in}{0.136829in}}%
\pgfpathlineto{\pgfqpoint{4.719138in}{0.136829in}}%
\pgfpathlineto{\pgfqpoint{4.713804in}{0.136829in}}%
\pgfpathlineto{\pgfqpoint{4.708471in}{0.136829in}}%
\pgfpathlineto{\pgfqpoint{4.703138in}{0.136829in}}%
\pgfpathlineto{\pgfqpoint{4.697805in}{0.136829in}}%
\pgfpathlineto{\pgfqpoint{4.692472in}{0.136829in}}%
\pgfpathlineto{\pgfqpoint{4.687138in}{0.136829in}}%
\pgfpathlineto{\pgfqpoint{4.681805in}{0.136829in}}%
\pgfpathlineto{\pgfqpoint{4.676472in}{0.136829in}}%
\pgfpathlineto{\pgfqpoint{4.671139in}{0.136829in}}%
\pgfpathlineto{\pgfqpoint{4.665805in}{0.136829in}}%
\pgfpathlineto{\pgfqpoint{4.660472in}{0.136829in}}%
\pgfpathlineto{\pgfqpoint{4.655139in}{0.136829in}}%
\pgfpathlineto{\pgfqpoint{4.649806in}{0.136829in}}%
\pgfpathlineto{\pgfqpoint{4.644473in}{0.136829in}}%
\pgfpathlineto{\pgfqpoint{4.639139in}{0.136829in}}%
\pgfpathlineto{\pgfqpoint{4.633806in}{0.136829in}}%
\pgfpathlineto{\pgfqpoint{4.628473in}{0.136829in}}%
\pgfpathlineto{\pgfqpoint{4.623140in}{0.136829in}}%
\pgfpathlineto{\pgfqpoint{4.617807in}{0.136829in}}%
\pgfpathlineto{\pgfqpoint{4.612473in}{0.136829in}}%
\pgfpathlineto{\pgfqpoint{4.607140in}{0.136829in}}%
\pgfpathlineto{\pgfqpoint{4.601807in}{0.136829in}}%
\pgfpathlineto{\pgfqpoint{4.596474in}{0.136829in}}%
\pgfpathlineto{\pgfqpoint{4.591140in}{0.136829in}}%
\pgfpathlineto{\pgfqpoint{4.585807in}{0.136829in}}%
\pgfpathlineto{\pgfqpoint{4.580474in}{0.136829in}}%
\pgfpathlineto{\pgfqpoint{4.575141in}{0.136829in}}%
\pgfpathlineto{\pgfqpoint{4.569808in}{0.136829in}}%
\pgfpathlineto{\pgfqpoint{4.564474in}{0.136829in}}%
\pgfpathlineto{\pgfqpoint{4.559141in}{0.136829in}}%
\pgfpathlineto{\pgfqpoint{4.553808in}{0.136829in}}%
\pgfpathlineto{\pgfqpoint{4.548475in}{0.136829in}}%
\pgfpathlineto{\pgfqpoint{4.543142in}{0.136829in}}%
\pgfpathlineto{\pgfqpoint{4.537808in}{0.136829in}}%
\pgfpathlineto{\pgfqpoint{4.532475in}{0.136829in}}%
\pgfpathlineto{\pgfqpoint{4.527142in}{0.136829in}}%
\pgfpathlineto{\pgfqpoint{4.521809in}{0.136829in}}%
\pgfpathlineto{\pgfqpoint{4.516475in}{0.136829in}}%
\pgfpathlineto{\pgfqpoint{4.511142in}{0.136829in}}%
\pgfpathlineto{\pgfqpoint{4.505809in}{0.136829in}}%
\pgfpathlineto{\pgfqpoint{4.500476in}{0.136829in}}%
\pgfpathlineto{\pgfqpoint{4.495143in}{0.136829in}}%
\pgfpathlineto{\pgfqpoint{4.489809in}{0.136829in}}%
\pgfpathlineto{\pgfqpoint{4.484476in}{0.136829in}}%
\pgfpathlineto{\pgfqpoint{4.479143in}{0.136829in}}%
\pgfpathlineto{\pgfqpoint{4.473810in}{0.136829in}}%
\pgfpathlineto{\pgfqpoint{4.468476in}{0.136829in}}%
\pgfpathlineto{\pgfqpoint{4.463143in}{0.136829in}}%
\pgfpathlineto{\pgfqpoint{4.457810in}{0.136829in}}%
\pgfpathlineto{\pgfqpoint{4.452477in}{0.136829in}}%
\pgfpathlineto{\pgfqpoint{4.447144in}{0.136829in}}%
\pgfpathlineto{\pgfqpoint{4.441810in}{0.136829in}}%
\pgfpathlineto{\pgfqpoint{4.436477in}{0.136829in}}%
\pgfpathlineto{\pgfqpoint{4.431144in}{0.136829in}}%
\pgfpathlineto{\pgfqpoint{4.425811in}{0.136829in}}%
\pgfpathlineto{\pgfqpoint{4.420478in}{0.136829in}}%
\pgfpathlineto{\pgfqpoint{4.415144in}{0.136829in}}%
\pgfpathlineto{\pgfqpoint{4.409811in}{0.136829in}}%
\pgfpathlineto{\pgfqpoint{4.404478in}{0.136829in}}%
\pgfpathlineto{\pgfqpoint{4.399145in}{0.136829in}}%
\pgfpathlineto{\pgfqpoint{4.393811in}{0.136829in}}%
\pgfpathlineto{\pgfqpoint{4.388478in}{0.136829in}}%
\pgfpathlineto{\pgfqpoint{4.383145in}{0.136829in}}%
\pgfpathlineto{\pgfqpoint{4.377812in}{0.136829in}}%
\pgfpathlineto{\pgfqpoint{4.372479in}{0.136829in}}%
\pgfpathlineto{\pgfqpoint{4.367145in}{0.136829in}}%
\pgfpathlineto{\pgfqpoint{4.361812in}{0.136829in}}%
\pgfpathlineto{\pgfqpoint{4.356479in}{0.136829in}}%
\pgfpathlineto{\pgfqpoint{4.351146in}{0.136829in}}%
\pgfpathlineto{\pgfqpoint{4.345813in}{0.136829in}}%
\pgfpathlineto{\pgfqpoint{4.340479in}{0.136829in}}%
\pgfpathlineto{\pgfqpoint{4.335146in}{0.136829in}}%
\pgfpathlineto{\pgfqpoint{4.329813in}{0.136829in}}%
\pgfpathlineto{\pgfqpoint{4.324480in}{0.136829in}}%
\pgfpathlineto{\pgfqpoint{4.319146in}{0.136829in}}%
\pgfpathlineto{\pgfqpoint{4.313813in}{0.136829in}}%
\pgfpathlineto{\pgfqpoint{4.308480in}{0.136829in}}%
\pgfpathlineto{\pgfqpoint{4.303147in}{0.136829in}}%
\pgfpathlineto{\pgfqpoint{4.297814in}{0.136829in}}%
\pgfpathlineto{\pgfqpoint{4.292480in}{0.136829in}}%
\pgfpathlineto{\pgfqpoint{4.287147in}{0.136829in}}%
\pgfpathlineto{\pgfqpoint{4.281814in}{0.136829in}}%
\pgfpathlineto{\pgfqpoint{4.276481in}{0.136829in}}%
\pgfpathlineto{\pgfqpoint{4.271148in}{0.136829in}}%
\pgfpathlineto{\pgfqpoint{4.265814in}{0.136829in}}%
\pgfpathlineto{\pgfqpoint{4.260481in}{0.136829in}}%
\pgfpathlineto{\pgfqpoint{4.255148in}{0.136829in}}%
\pgfpathlineto{\pgfqpoint{4.249815in}{0.136829in}}%
\pgfpathlineto{\pgfqpoint{4.244481in}{0.136829in}}%
\pgfpathlineto{\pgfqpoint{4.239148in}{0.136829in}}%
\pgfpathlineto{\pgfqpoint{4.233815in}{0.136829in}}%
\pgfpathlineto{\pgfqpoint{4.228482in}{0.136829in}}%
\pgfpathlineto{\pgfqpoint{4.223149in}{0.136829in}}%
\pgfpathlineto{\pgfqpoint{4.217815in}{0.136829in}}%
\pgfpathlineto{\pgfqpoint{4.212482in}{0.136829in}}%
\pgfpathlineto{\pgfqpoint{4.207149in}{0.136829in}}%
\pgfpathlineto{\pgfqpoint{4.201816in}{0.136829in}}%
\pgfpathlineto{\pgfqpoint{4.196482in}{0.136829in}}%
\pgfpathlineto{\pgfqpoint{4.191149in}{0.136829in}}%
\pgfpathlineto{\pgfqpoint{4.185816in}{0.136829in}}%
\pgfpathlineto{\pgfqpoint{4.180483in}{0.136829in}}%
\pgfpathlineto{\pgfqpoint{4.175150in}{0.136829in}}%
\pgfpathlineto{\pgfqpoint{4.169816in}{0.136829in}}%
\pgfpathlineto{\pgfqpoint{4.164483in}{0.136829in}}%
\pgfpathlineto{\pgfqpoint{4.159150in}{0.136829in}}%
\pgfpathlineto{\pgfqpoint{4.153817in}{0.136829in}}%
\pgfpathlineto{\pgfqpoint{4.148484in}{0.136829in}}%
\pgfpathlineto{\pgfqpoint{4.143150in}{0.136829in}}%
\pgfpathlineto{\pgfqpoint{4.137817in}{0.136829in}}%
\pgfpathlineto{\pgfqpoint{4.132484in}{0.136829in}}%
\pgfpathlineto{\pgfqpoint{4.127151in}{0.136829in}}%
\pgfpathlineto{\pgfqpoint{4.121817in}{0.136829in}}%
\pgfpathlineto{\pgfqpoint{4.116484in}{0.136829in}}%
\pgfpathlineto{\pgfqpoint{4.111151in}{0.136829in}}%
\pgfpathlineto{\pgfqpoint{4.105818in}{0.136829in}}%
\pgfpathlineto{\pgfqpoint{4.100485in}{0.136829in}}%
\pgfpathlineto{\pgfqpoint{4.095151in}{0.136829in}}%
\pgfpathlineto{\pgfqpoint{4.089818in}{0.136829in}}%
\pgfpathlineto{\pgfqpoint{4.084485in}{0.136829in}}%
\pgfpathlineto{\pgfqpoint{4.079152in}{0.136829in}}%
\pgfpathlineto{\pgfqpoint{4.073819in}{0.136829in}}%
\pgfpathlineto{\pgfqpoint{4.068485in}{0.136829in}}%
\pgfpathlineto{\pgfqpoint{4.063152in}{0.136829in}}%
\pgfpathlineto{\pgfqpoint{4.057819in}{0.136829in}}%
\pgfpathlineto{\pgfqpoint{4.052486in}{0.136829in}}%
\pgfpathlineto{\pgfqpoint{4.047152in}{0.136829in}}%
\pgfpathlineto{\pgfqpoint{4.041819in}{0.136829in}}%
\pgfpathlineto{\pgfqpoint{4.036486in}{0.136829in}}%
\pgfpathlineto{\pgfqpoint{4.031153in}{0.136829in}}%
\pgfpathlineto{\pgfqpoint{4.025820in}{0.136829in}}%
\pgfpathlineto{\pgfqpoint{4.020486in}{0.136829in}}%
\pgfpathlineto{\pgfqpoint{4.015153in}{0.136829in}}%
\pgfpathlineto{\pgfqpoint{4.009820in}{0.136829in}}%
\pgfpathlineto{\pgfqpoint{4.004487in}{0.136829in}}%
\pgfpathlineto{\pgfqpoint{3.999154in}{0.136829in}}%
\pgfpathlineto{\pgfqpoint{3.993820in}{0.136829in}}%
\pgfpathlineto{\pgfqpoint{3.988487in}{0.136829in}}%
\pgfpathlineto{\pgfqpoint{3.983154in}{0.136829in}}%
\pgfpathlineto{\pgfqpoint{3.977821in}{0.136829in}}%
\pgfpathlineto{\pgfqpoint{3.972487in}{0.136829in}}%
\pgfpathlineto{\pgfqpoint{3.967154in}{0.136829in}}%
\pgfpathlineto{\pgfqpoint{3.961821in}{0.136829in}}%
\pgfpathlineto{\pgfqpoint{3.956488in}{0.136829in}}%
\pgfpathlineto{\pgfqpoint{3.951155in}{0.136829in}}%
\pgfpathlineto{\pgfqpoint{3.945821in}{0.136829in}}%
\pgfpathlineto{\pgfqpoint{3.940488in}{0.136829in}}%
\pgfpathlineto{\pgfqpoint{3.935155in}{0.136829in}}%
\pgfpathlineto{\pgfqpoint{3.929822in}{0.136829in}}%
\pgfpathlineto{\pgfqpoint{3.924488in}{0.136829in}}%
\pgfpathlineto{\pgfqpoint{3.919155in}{0.136829in}}%
\pgfpathlineto{\pgfqpoint{3.913822in}{0.136829in}}%
\pgfpathlineto{\pgfqpoint{3.908489in}{0.136829in}}%
\pgfpathlineto{\pgfqpoint{3.903156in}{0.136829in}}%
\pgfpathlineto{\pgfqpoint{3.897822in}{0.136829in}}%
\pgfpathlineto{\pgfqpoint{3.892489in}{0.136829in}}%
\pgfpathlineto{\pgfqpoint{3.887156in}{0.136829in}}%
\pgfpathlineto{\pgfqpoint{3.881823in}{0.136829in}}%
\pgfpathlineto{\pgfqpoint{3.876490in}{0.136829in}}%
\pgfpathlineto{\pgfqpoint{3.871156in}{0.136829in}}%
\pgfpathlineto{\pgfqpoint{3.865823in}{0.136829in}}%
\pgfpathlineto{\pgfqpoint{3.860490in}{0.136829in}}%
\pgfpathlineto{\pgfqpoint{3.855157in}{0.136829in}}%
\pgfpathlineto{\pgfqpoint{3.849823in}{0.136829in}}%
\pgfpathlineto{\pgfqpoint{3.844490in}{0.136829in}}%
\pgfpathlineto{\pgfqpoint{3.839157in}{0.136829in}}%
\pgfpathlineto{\pgfqpoint{3.833824in}{0.136829in}}%
\pgfpathlineto{\pgfqpoint{3.828491in}{0.136829in}}%
\pgfpathlineto{\pgfqpoint{3.823157in}{0.136829in}}%
\pgfpathlineto{\pgfqpoint{3.817824in}{0.136829in}}%
\pgfpathlineto{\pgfqpoint{3.812491in}{0.136829in}}%
\pgfpathlineto{\pgfqpoint{3.807158in}{0.136829in}}%
\pgfpathlineto{\pgfqpoint{3.801825in}{0.136829in}}%
\pgfpathlineto{\pgfqpoint{3.796491in}{0.136829in}}%
\pgfpathlineto{\pgfqpoint{3.791158in}{0.136829in}}%
\pgfpathlineto{\pgfqpoint{3.785825in}{0.136829in}}%
\pgfpathlineto{\pgfqpoint{3.780492in}{0.136829in}}%
\pgfpathlineto{\pgfqpoint{3.775158in}{0.136829in}}%
\pgfpathlineto{\pgfqpoint{3.769825in}{0.136829in}}%
\pgfpathlineto{\pgfqpoint{3.764492in}{0.136829in}}%
\pgfpathlineto{\pgfqpoint{3.759159in}{0.136829in}}%
\pgfpathlineto{\pgfqpoint{3.753826in}{0.136829in}}%
\pgfpathlineto{\pgfqpoint{3.748492in}{0.136829in}}%
\pgfpathlineto{\pgfqpoint{3.743159in}{0.136829in}}%
\pgfpathlineto{\pgfqpoint{3.737826in}{0.136829in}}%
\pgfpathlineto{\pgfqpoint{3.732493in}{0.136829in}}%
\pgfpathlineto{\pgfqpoint{3.727160in}{0.136829in}}%
\pgfpathlineto{\pgfqpoint{3.721826in}{0.136829in}}%
\pgfpathlineto{\pgfqpoint{3.716493in}{0.136829in}}%
\pgfpathlineto{\pgfqpoint{3.711160in}{0.136829in}}%
\pgfpathlineto{\pgfqpoint{3.705827in}{0.136829in}}%
\pgfpathlineto{\pgfqpoint{3.700493in}{0.136829in}}%
\pgfpathlineto{\pgfqpoint{3.695160in}{0.136829in}}%
\pgfpathlineto{\pgfqpoint{3.689827in}{0.136829in}}%
\pgfpathlineto{\pgfqpoint{3.684494in}{0.136829in}}%
\pgfpathlineto{\pgfqpoint{3.679161in}{0.136829in}}%
\pgfpathlineto{\pgfqpoint{3.673827in}{0.136829in}}%
\pgfpathlineto{\pgfqpoint{3.668494in}{0.136829in}}%
\pgfpathlineto{\pgfqpoint{3.663161in}{0.136829in}}%
\pgfpathlineto{\pgfqpoint{3.657828in}{0.136829in}}%
\pgfpathlineto{\pgfqpoint{3.652494in}{0.136829in}}%
\pgfpathlineto{\pgfqpoint{3.647161in}{0.136829in}}%
\pgfpathlineto{\pgfqpoint{3.641828in}{0.136829in}}%
\pgfpathlineto{\pgfqpoint{3.636495in}{0.136829in}}%
\pgfpathlineto{\pgfqpoint{3.631162in}{0.136829in}}%
\pgfpathlineto{\pgfqpoint{3.625828in}{0.136829in}}%
\pgfpathlineto{\pgfqpoint{3.620495in}{0.136829in}}%
\pgfpathlineto{\pgfqpoint{3.615162in}{0.136829in}}%
\pgfpathlineto{\pgfqpoint{3.609829in}{0.136829in}}%
\pgfpathlineto{\pgfqpoint{3.604496in}{0.136829in}}%
\pgfpathlineto{\pgfqpoint{3.599162in}{0.136829in}}%
\pgfpathlineto{\pgfqpoint{3.593829in}{0.136829in}}%
\pgfpathlineto{\pgfqpoint{3.588496in}{0.136829in}}%
\pgfpathlineto{\pgfqpoint{3.583163in}{0.136829in}}%
\pgfpathlineto{\pgfqpoint{3.577829in}{0.136829in}}%
\pgfpathlineto{\pgfqpoint{3.572496in}{0.136829in}}%
\pgfpathlineto{\pgfqpoint{3.567163in}{0.136829in}}%
\pgfpathlineto{\pgfqpoint{3.561830in}{0.136829in}}%
\pgfpathlineto{\pgfqpoint{3.556497in}{0.136829in}}%
\pgfpathlineto{\pgfqpoint{3.551163in}{0.136829in}}%
\pgfpathlineto{\pgfqpoint{3.545830in}{0.136829in}}%
\pgfpathlineto{\pgfqpoint{3.540497in}{0.136829in}}%
\pgfpathlineto{\pgfqpoint{3.535164in}{0.136829in}}%
\pgfpathlineto{\pgfqpoint{3.529831in}{0.136829in}}%
\pgfpathlineto{\pgfqpoint{3.524497in}{0.136829in}}%
\pgfpathlineto{\pgfqpoint{3.519164in}{0.136829in}}%
\pgfpathlineto{\pgfqpoint{3.513831in}{0.136829in}}%
\pgfpathlineto{\pgfqpoint{3.508498in}{0.136829in}}%
\pgfpathlineto{\pgfqpoint{3.503164in}{0.136829in}}%
\pgfpathlineto{\pgfqpoint{3.497831in}{0.136829in}}%
\pgfpathlineto{\pgfqpoint{3.492498in}{0.136829in}}%
\pgfpathlineto{\pgfqpoint{3.487165in}{0.136829in}}%
\pgfpathlineto{\pgfqpoint{3.481832in}{0.136829in}}%
\pgfpathlineto{\pgfqpoint{3.476498in}{0.136829in}}%
\pgfpathlineto{\pgfqpoint{3.471165in}{0.136829in}}%
\pgfpathlineto{\pgfqpoint{3.465832in}{0.136829in}}%
\pgfpathlineto{\pgfqpoint{3.460499in}{0.136829in}}%
\pgfpathlineto{\pgfqpoint{3.455166in}{0.136829in}}%
\pgfpathlineto{\pgfqpoint{3.449832in}{0.136829in}}%
\pgfpathlineto{\pgfqpoint{3.444499in}{0.136829in}}%
\pgfpathlineto{\pgfqpoint{3.439166in}{0.136829in}}%
\pgfpathlineto{\pgfqpoint{3.433833in}{0.136829in}}%
\pgfpathlineto{\pgfqpoint{3.428499in}{0.136829in}}%
\pgfpathlineto{\pgfqpoint{3.423166in}{0.136829in}}%
\pgfpathlineto{\pgfqpoint{3.417833in}{0.136829in}}%
\pgfpathlineto{\pgfqpoint{3.412500in}{0.136829in}}%
\pgfpathlineto{\pgfqpoint{3.407167in}{0.136829in}}%
\pgfpathlineto{\pgfqpoint{3.401833in}{0.136829in}}%
\pgfpathlineto{\pgfqpoint{3.396500in}{0.136829in}}%
\pgfpathlineto{\pgfqpoint{3.391167in}{0.136829in}}%
\pgfpathlineto{\pgfqpoint{3.385834in}{0.136829in}}%
\pgfpathlineto{\pgfqpoint{3.380500in}{0.136829in}}%
\pgfpathlineto{\pgfqpoint{3.375167in}{0.136829in}}%
\pgfpathlineto{\pgfqpoint{3.369834in}{0.136829in}}%
\pgfpathlineto{\pgfqpoint{3.364501in}{0.136829in}}%
\pgfpathlineto{\pgfqpoint{3.359168in}{0.136829in}}%
\pgfpathlineto{\pgfqpoint{3.353834in}{0.136829in}}%
\pgfpathlineto{\pgfqpoint{3.348501in}{0.136829in}}%
\pgfpathlineto{\pgfqpoint{3.343168in}{0.136829in}}%
\pgfpathlineto{\pgfqpoint{3.337835in}{0.136829in}}%
\pgfpathlineto{\pgfqpoint{3.332502in}{0.136829in}}%
\pgfpathlineto{\pgfqpoint{3.327168in}{0.136829in}}%
\pgfpathlineto{\pgfqpoint{3.321835in}{0.136829in}}%
\pgfpathlineto{\pgfqpoint{3.316502in}{0.136829in}}%
\pgfpathlineto{\pgfqpoint{3.311169in}{0.136829in}}%
\pgfpathlineto{\pgfqpoint{3.305835in}{0.136829in}}%
\pgfpathlineto{\pgfqpoint{3.300502in}{0.136829in}}%
\pgfpathlineto{\pgfqpoint{3.295169in}{0.136829in}}%
\pgfpathlineto{\pgfqpoint{3.289836in}{0.136829in}}%
\pgfpathlineto{\pgfqpoint{3.284503in}{0.136829in}}%
\pgfpathlineto{\pgfqpoint{3.279169in}{0.136829in}}%
\pgfpathlineto{\pgfqpoint{3.273836in}{0.136829in}}%
\pgfpathlineto{\pgfqpoint{3.268503in}{0.136829in}}%
\pgfpathlineto{\pgfqpoint{3.263170in}{0.136829in}}%
\pgfpathlineto{\pgfqpoint{3.257837in}{0.136829in}}%
\pgfpathlineto{\pgfqpoint{3.252503in}{0.136829in}}%
\pgfpathlineto{\pgfqpoint{3.247170in}{0.136829in}}%
\pgfpathlineto{\pgfqpoint{3.241837in}{0.136829in}}%
\pgfpathlineto{\pgfqpoint{3.236504in}{0.136829in}}%
\pgfpathlineto{\pgfqpoint{3.231170in}{0.136829in}}%
\pgfpathlineto{\pgfqpoint{3.225837in}{0.136829in}}%
\pgfpathlineto{\pgfqpoint{3.220504in}{0.136829in}}%
\pgfpathlineto{\pgfqpoint{3.215171in}{0.136829in}}%
\pgfpathlineto{\pgfqpoint{3.209838in}{0.136829in}}%
\pgfpathlineto{\pgfqpoint{3.204504in}{0.136829in}}%
\pgfpathlineto{\pgfqpoint{3.199171in}{0.136829in}}%
\pgfpathlineto{\pgfqpoint{3.193838in}{0.136829in}}%
\pgfpathlineto{\pgfqpoint{3.188505in}{0.136829in}}%
\pgfpathlineto{\pgfqpoint{3.183172in}{0.136829in}}%
\pgfpathlineto{\pgfqpoint{3.177838in}{0.136829in}}%
\pgfpathlineto{\pgfqpoint{3.172505in}{0.136829in}}%
\pgfpathlineto{\pgfqpoint{3.167172in}{0.136829in}}%
\pgfpathlineto{\pgfqpoint{3.161839in}{0.136829in}}%
\pgfpathlineto{\pgfqpoint{3.156505in}{0.136829in}}%
\pgfpathlineto{\pgfqpoint{3.151172in}{0.136829in}}%
\pgfpathlineto{\pgfqpoint{3.145839in}{0.136829in}}%
\pgfpathlineto{\pgfqpoint{3.140506in}{0.136829in}}%
\pgfpathlineto{\pgfqpoint{3.135173in}{0.136829in}}%
\pgfpathlineto{\pgfqpoint{3.129839in}{0.136829in}}%
\pgfpathlineto{\pgfqpoint{3.124506in}{0.136829in}}%
\pgfpathlineto{\pgfqpoint{3.119173in}{0.136829in}}%
\pgfpathlineto{\pgfqpoint{3.113840in}{0.136829in}}%
\pgfpathlineto{\pgfqpoint{3.108506in}{0.136829in}}%
\pgfpathlineto{\pgfqpoint{3.103173in}{0.136829in}}%
\pgfpathlineto{\pgfqpoint{3.097840in}{0.136829in}}%
\pgfpathlineto{\pgfqpoint{3.092507in}{0.136829in}}%
\pgfpathlineto{\pgfqpoint{3.087174in}{0.136829in}}%
\pgfpathlineto{\pgfqpoint{3.081840in}{0.136829in}}%
\pgfpathlineto{\pgfqpoint{3.076507in}{0.136829in}}%
\pgfpathlineto{\pgfqpoint{3.071174in}{0.136829in}}%
\pgfpathlineto{\pgfqpoint{3.065841in}{0.136829in}}%
\pgfpathlineto{\pgfqpoint{3.060508in}{0.136829in}}%
\pgfpathlineto{\pgfqpoint{3.055174in}{0.136829in}}%
\pgfpathlineto{\pgfqpoint{3.049841in}{0.136829in}}%
\pgfpathlineto{\pgfqpoint{3.044508in}{0.136829in}}%
\pgfpathlineto{\pgfqpoint{3.039175in}{0.136829in}}%
\pgfpathlineto{\pgfqpoint{3.033841in}{0.136829in}}%
\pgfpathlineto{\pgfqpoint{3.028508in}{0.136829in}}%
\pgfpathlineto{\pgfqpoint{3.023175in}{0.136829in}}%
\pgfpathlineto{\pgfqpoint{3.017842in}{0.136829in}}%
\pgfpathlineto{\pgfqpoint{3.012509in}{0.136829in}}%
\pgfpathlineto{\pgfqpoint{3.007175in}{0.136829in}}%
\pgfpathlineto{\pgfqpoint{3.001842in}{0.136829in}}%
\pgfpathlineto{\pgfqpoint{2.996509in}{0.136829in}}%
\pgfpathlineto{\pgfqpoint{2.991176in}{0.136829in}}%
\pgfpathlineto{\pgfqpoint{2.985843in}{0.136829in}}%
\pgfpathlineto{\pgfqpoint{2.980509in}{0.136829in}}%
\pgfpathlineto{\pgfqpoint{2.975176in}{0.136829in}}%
\pgfpathlineto{\pgfqpoint{2.969843in}{0.136829in}}%
\pgfpathlineto{\pgfqpoint{2.964510in}{0.136829in}}%
\pgfpathlineto{\pgfqpoint{2.959176in}{0.136829in}}%
\pgfpathlineto{\pgfqpoint{2.953843in}{0.136829in}}%
\pgfpathlineto{\pgfqpoint{2.948510in}{0.136829in}}%
\pgfpathlineto{\pgfqpoint{2.943177in}{0.136829in}}%
\pgfpathlineto{\pgfqpoint{2.937844in}{0.136829in}}%
\pgfpathlineto{\pgfqpoint{2.932510in}{0.136829in}}%
\pgfpathlineto{\pgfqpoint{2.927177in}{0.136829in}}%
\pgfpathlineto{\pgfqpoint{2.921844in}{0.136829in}}%
\pgfpathlineto{\pgfqpoint{2.916511in}{0.136829in}}%
\pgfpathlineto{\pgfqpoint{2.911178in}{0.136829in}}%
\pgfpathlineto{\pgfqpoint{2.905844in}{0.136829in}}%
\pgfpathlineto{\pgfqpoint{2.900511in}{0.136829in}}%
\pgfpathlineto{\pgfqpoint{2.895178in}{0.136829in}}%
\pgfpathlineto{\pgfqpoint{2.889845in}{0.136829in}}%
\pgfpathlineto{\pgfqpoint{2.884511in}{0.136829in}}%
\pgfpathlineto{\pgfqpoint{2.879178in}{0.136829in}}%
\pgfpathlineto{\pgfqpoint{2.873845in}{0.136829in}}%
\pgfpathlineto{\pgfqpoint{2.868512in}{0.136829in}}%
\pgfpathlineto{\pgfqpoint{2.863179in}{0.136829in}}%
\pgfpathlineto{\pgfqpoint{2.857845in}{0.136829in}}%
\pgfpathlineto{\pgfqpoint{2.852512in}{0.136829in}}%
\pgfpathlineto{\pgfqpoint{2.847179in}{0.136829in}}%
\pgfpathlineto{\pgfqpoint{2.841846in}{0.136829in}}%
\pgfpathlineto{\pgfqpoint{2.836512in}{0.136829in}}%
\pgfpathlineto{\pgfqpoint{2.831179in}{0.136829in}}%
\pgfpathlineto{\pgfqpoint{2.825846in}{0.136829in}}%
\pgfpathlineto{\pgfqpoint{2.820513in}{0.136829in}}%
\pgfpathlineto{\pgfqpoint{2.815180in}{0.136829in}}%
\pgfpathlineto{\pgfqpoint{2.809846in}{0.136829in}}%
\pgfpathlineto{\pgfqpoint{2.804513in}{0.136829in}}%
\pgfpathlineto{\pgfqpoint{2.799180in}{0.136829in}}%
\pgfpathlineto{\pgfqpoint{2.793847in}{0.136829in}}%
\pgfpathlineto{\pgfqpoint{2.788514in}{0.136829in}}%
\pgfpathlineto{\pgfqpoint{2.783180in}{0.136829in}}%
\pgfpathlineto{\pgfqpoint{2.777847in}{0.136829in}}%
\pgfpathlineto{\pgfqpoint{2.772514in}{0.136829in}}%
\pgfpathlineto{\pgfqpoint{2.767181in}{0.136829in}}%
\pgfpathlineto{\pgfqpoint{2.761847in}{0.136829in}}%
\pgfpathlineto{\pgfqpoint{2.756514in}{0.136829in}}%
\pgfpathlineto{\pgfqpoint{2.751181in}{0.136829in}}%
\pgfpathlineto{\pgfqpoint{2.745848in}{0.136829in}}%
\pgfpathlineto{\pgfqpoint{2.740515in}{0.136829in}}%
\pgfpathlineto{\pgfqpoint{2.735181in}{0.136829in}}%
\pgfpathlineto{\pgfqpoint{2.729848in}{0.136829in}}%
\pgfpathlineto{\pgfqpoint{2.724515in}{0.136829in}}%
\pgfpathlineto{\pgfqpoint{2.719182in}{0.136829in}}%
\pgfpathlineto{\pgfqpoint{2.713849in}{0.136829in}}%
\pgfpathlineto{\pgfqpoint{2.708515in}{0.136829in}}%
\pgfpathlineto{\pgfqpoint{2.703182in}{0.136829in}}%
\pgfpathlineto{\pgfqpoint{2.697849in}{0.136829in}}%
\pgfpathlineto{\pgfqpoint{2.692516in}{0.136829in}}%
\pgfpathlineto{\pgfqpoint{2.687182in}{0.136829in}}%
\pgfpathlineto{\pgfqpoint{2.681849in}{0.136829in}}%
\pgfpathlineto{\pgfqpoint{2.676516in}{0.136829in}}%
\pgfpathlineto{\pgfqpoint{2.671183in}{0.136829in}}%
\pgfpathlineto{\pgfqpoint{2.665850in}{0.136829in}}%
\pgfpathlineto{\pgfqpoint{2.660516in}{0.136829in}}%
\pgfpathlineto{\pgfqpoint{2.655183in}{0.136829in}}%
\pgfpathlineto{\pgfqpoint{2.649850in}{0.136829in}}%
\pgfpathlineto{\pgfqpoint{2.644517in}{0.136829in}}%
\pgfpathlineto{\pgfqpoint{2.639184in}{0.136829in}}%
\pgfpathlineto{\pgfqpoint{2.633850in}{0.136829in}}%
\pgfpathlineto{\pgfqpoint{2.628517in}{0.136829in}}%
\pgfpathlineto{\pgfqpoint{2.623184in}{0.136829in}}%
\pgfpathlineto{\pgfqpoint{2.617851in}{0.136829in}}%
\pgfpathlineto{\pgfqpoint{2.612517in}{0.136829in}}%
\pgfpathlineto{\pgfqpoint{2.607184in}{0.136829in}}%
\pgfpathlineto{\pgfqpoint{2.601851in}{0.136829in}}%
\pgfpathlineto{\pgfqpoint{2.596518in}{0.136829in}}%
\pgfpathlineto{\pgfqpoint{2.591185in}{0.136829in}}%
\pgfpathlineto{\pgfqpoint{2.585851in}{0.136829in}}%
\pgfpathlineto{\pgfqpoint{2.580518in}{0.136829in}}%
\pgfpathlineto{\pgfqpoint{2.575185in}{0.136829in}}%
\pgfpathlineto{\pgfqpoint{2.569852in}{0.136829in}}%
\pgfpathlineto{\pgfqpoint{2.564518in}{0.136829in}}%
\pgfpathlineto{\pgfqpoint{2.559185in}{0.136829in}}%
\pgfpathlineto{\pgfqpoint{2.553852in}{0.136829in}}%
\pgfpathlineto{\pgfqpoint{2.548519in}{0.136829in}}%
\pgfpathlineto{\pgfqpoint{2.543186in}{0.136829in}}%
\pgfpathlineto{\pgfqpoint{2.537852in}{0.136829in}}%
\pgfpathlineto{\pgfqpoint{2.532519in}{0.136829in}}%
\pgfpathlineto{\pgfqpoint{2.527186in}{0.136829in}}%
\pgfpathlineto{\pgfqpoint{2.521853in}{0.136829in}}%
\pgfpathlineto{\pgfqpoint{2.516520in}{0.136829in}}%
\pgfpathlineto{\pgfqpoint{2.511186in}{0.136829in}}%
\pgfpathlineto{\pgfqpoint{2.505853in}{0.136829in}}%
\pgfpathlineto{\pgfqpoint{2.500520in}{0.136829in}}%
\pgfpathlineto{\pgfqpoint{2.495187in}{0.136829in}}%
\pgfpathlineto{\pgfqpoint{2.489853in}{0.136829in}}%
\pgfpathlineto{\pgfqpoint{2.484520in}{0.136829in}}%
\pgfpathlineto{\pgfqpoint{2.479187in}{0.136829in}}%
\pgfpathlineto{\pgfqpoint{2.473854in}{0.136829in}}%
\pgfpathlineto{\pgfqpoint{2.468521in}{0.136829in}}%
\pgfpathlineto{\pgfqpoint{2.463187in}{0.136829in}}%
\pgfpathlineto{\pgfqpoint{2.457854in}{0.136829in}}%
\pgfpathlineto{\pgfqpoint{2.452521in}{0.136829in}}%
\pgfpathlineto{\pgfqpoint{2.447188in}{0.136829in}}%
\pgfpathlineto{\pgfqpoint{2.441855in}{0.136829in}}%
\pgfpathlineto{\pgfqpoint{2.436521in}{0.136829in}}%
\pgfpathlineto{\pgfqpoint{2.431188in}{0.136829in}}%
\pgfpathlineto{\pgfqpoint{2.425855in}{0.136829in}}%
\pgfpathlineto{\pgfqpoint{2.420522in}{0.136829in}}%
\pgfpathlineto{\pgfqpoint{2.415188in}{0.136829in}}%
\pgfpathlineto{\pgfqpoint{2.409855in}{0.136829in}}%
\pgfpathlineto{\pgfqpoint{2.404522in}{0.136829in}}%
\pgfpathlineto{\pgfqpoint{2.399189in}{0.136829in}}%
\pgfpathlineto{\pgfqpoint{2.393856in}{0.136829in}}%
\pgfpathlineto{\pgfqpoint{2.388522in}{0.136829in}}%
\pgfpathlineto{\pgfqpoint{2.383189in}{0.136829in}}%
\pgfpathlineto{\pgfqpoint{2.377856in}{0.136829in}}%
\pgfpathlineto{\pgfqpoint{2.372523in}{0.136829in}}%
\pgfpathlineto{\pgfqpoint{2.367190in}{0.136829in}}%
\pgfpathlineto{\pgfqpoint{2.361856in}{0.136829in}}%
\pgfpathlineto{\pgfqpoint{2.356523in}{0.136829in}}%
\pgfpathlineto{\pgfqpoint{2.351190in}{0.136829in}}%
\pgfpathlineto{\pgfqpoint{2.345857in}{0.136829in}}%
\pgfpathlineto{\pgfqpoint{2.340523in}{0.136829in}}%
\pgfpathlineto{\pgfqpoint{2.335190in}{0.136829in}}%
\pgfpathlineto{\pgfqpoint{2.329857in}{0.136829in}}%
\pgfpathlineto{\pgfqpoint{2.324524in}{0.136829in}}%
\pgfpathlineto{\pgfqpoint{2.319191in}{0.136829in}}%
\pgfpathlineto{\pgfqpoint{2.313857in}{0.136829in}}%
\pgfpathlineto{\pgfqpoint{2.308524in}{0.136829in}}%
\pgfpathlineto{\pgfqpoint{2.303191in}{0.136829in}}%
\pgfpathlineto{\pgfqpoint{2.297858in}{0.136829in}}%
\pgfpathlineto{\pgfqpoint{2.292524in}{0.136829in}}%
\pgfpathlineto{\pgfqpoint{2.287191in}{0.136829in}}%
\pgfpathlineto{\pgfqpoint{2.281858in}{0.136829in}}%
\pgfpathlineto{\pgfqpoint{2.276525in}{0.136829in}}%
\pgfpathlineto{\pgfqpoint{2.271192in}{0.136829in}}%
\pgfpathlineto{\pgfqpoint{2.265858in}{0.136829in}}%
\pgfpathlineto{\pgfqpoint{2.260525in}{0.136829in}}%
\pgfpathlineto{\pgfqpoint{2.255192in}{0.136829in}}%
\pgfpathlineto{\pgfqpoint{2.249859in}{0.136829in}}%
\pgfpathlineto{\pgfqpoint{2.244526in}{0.136829in}}%
\pgfpathlineto{\pgfqpoint{2.239192in}{0.136829in}}%
\pgfpathlineto{\pgfqpoint{2.233859in}{0.136829in}}%
\pgfpathlineto{\pgfqpoint{2.228526in}{0.136829in}}%
\pgfpathlineto{\pgfqpoint{2.223193in}{0.136829in}}%
\pgfpathlineto{\pgfqpoint{2.217859in}{0.136829in}}%
\pgfpathlineto{\pgfqpoint{2.212526in}{0.136829in}}%
\pgfpathlineto{\pgfqpoint{2.207193in}{0.136829in}}%
\pgfpathlineto{\pgfqpoint{2.201860in}{0.136829in}}%
\pgfpathlineto{\pgfqpoint{2.196527in}{0.136829in}}%
\pgfpathlineto{\pgfqpoint{2.191193in}{0.136829in}}%
\pgfpathlineto{\pgfqpoint{2.185860in}{0.136829in}}%
\pgfpathlineto{\pgfqpoint{2.180527in}{0.136829in}}%
\pgfpathlineto{\pgfqpoint{2.175194in}{0.136829in}}%
\pgfpathlineto{\pgfqpoint{2.169861in}{0.136829in}}%
\pgfpathlineto{\pgfqpoint{2.164527in}{0.136829in}}%
\pgfpathlineto{\pgfqpoint{2.159194in}{0.136829in}}%
\pgfpathlineto{\pgfqpoint{2.153861in}{0.136829in}}%
\pgfpathlineto{\pgfqpoint{2.148528in}{0.136829in}}%
\pgfpathlineto{\pgfqpoint{2.143194in}{0.136829in}}%
\pgfpathlineto{\pgfqpoint{2.137861in}{0.136829in}}%
\pgfpathlineto{\pgfqpoint{2.132528in}{0.136829in}}%
\pgfpathlineto{\pgfqpoint{2.127195in}{0.136829in}}%
\pgfpathlineto{\pgfqpoint{2.121862in}{0.136829in}}%
\pgfpathlineto{\pgfqpoint{2.116528in}{0.136829in}}%
\pgfpathlineto{\pgfqpoint{2.111195in}{0.136829in}}%
\pgfpathlineto{\pgfqpoint{2.105862in}{0.136829in}}%
\pgfpathlineto{\pgfqpoint{2.100529in}{0.136829in}}%
\pgfpathlineto{\pgfqpoint{2.095196in}{0.136829in}}%
\pgfpathlineto{\pgfqpoint{2.089862in}{0.136829in}}%
\pgfpathlineto{\pgfqpoint{2.084529in}{0.136829in}}%
\pgfpathlineto{\pgfqpoint{2.079196in}{0.136829in}}%
\pgfpathlineto{\pgfqpoint{2.073863in}{0.136829in}}%
\pgfpathlineto{\pgfqpoint{2.068529in}{0.136829in}}%
\pgfpathlineto{\pgfqpoint{2.063196in}{0.136829in}}%
\pgfpathlineto{\pgfqpoint{2.057863in}{0.136829in}}%
\pgfpathlineto{\pgfqpoint{2.052530in}{0.136829in}}%
\pgfpathlineto{\pgfqpoint{2.047197in}{0.136829in}}%
\pgfpathlineto{\pgfqpoint{2.041863in}{0.136829in}}%
\pgfpathlineto{\pgfqpoint{2.036530in}{0.136829in}}%
\pgfpathlineto{\pgfqpoint{2.031197in}{0.136829in}}%
\pgfpathlineto{\pgfqpoint{2.025864in}{0.136829in}}%
\pgfpathlineto{\pgfqpoint{2.020531in}{0.136829in}}%
\pgfpathlineto{\pgfqpoint{2.015197in}{0.136829in}}%
\pgfpathlineto{\pgfqpoint{2.009864in}{0.136829in}}%
\pgfpathlineto{\pgfqpoint{2.004531in}{0.136829in}}%
\pgfpathlineto{\pgfqpoint{1.999198in}{0.136829in}}%
\pgfpathlineto{\pgfqpoint{1.993864in}{0.136829in}}%
\pgfpathlineto{\pgfqpoint{1.988531in}{0.136829in}}%
\pgfpathlineto{\pgfqpoint{1.983198in}{0.136829in}}%
\pgfpathlineto{\pgfqpoint{1.977865in}{0.136829in}}%
\pgfpathlineto{\pgfqpoint{1.972532in}{0.136829in}}%
\pgfpathlineto{\pgfqpoint{1.967198in}{0.136829in}}%
\pgfpathlineto{\pgfqpoint{1.961865in}{0.136829in}}%
\pgfpathlineto{\pgfqpoint{1.956532in}{0.136829in}}%
\pgfpathlineto{\pgfqpoint{1.951199in}{0.136829in}}%
\pgfpathlineto{\pgfqpoint{1.945865in}{0.136829in}}%
\pgfpathlineto{\pgfqpoint{1.940532in}{0.136829in}}%
\pgfpathlineto{\pgfqpoint{1.935199in}{0.136829in}}%
\pgfpathlineto{\pgfqpoint{1.929866in}{0.136829in}}%
\pgfpathlineto{\pgfqpoint{1.924533in}{0.136829in}}%
\pgfpathlineto{\pgfqpoint{1.919199in}{0.136829in}}%
\pgfpathlineto{\pgfqpoint{1.913866in}{0.136829in}}%
\pgfpathlineto{\pgfqpoint{1.908533in}{0.136829in}}%
\pgfpathlineto{\pgfqpoint{1.903200in}{0.136829in}}%
\pgfpathlineto{\pgfqpoint{1.897867in}{0.136829in}}%
\pgfpathlineto{\pgfqpoint{1.892533in}{0.136829in}}%
\pgfpathlineto{\pgfqpoint{1.887200in}{0.136829in}}%
\pgfpathlineto{\pgfqpoint{1.881867in}{0.136829in}}%
\pgfpathlineto{\pgfqpoint{1.876534in}{0.136829in}}%
\pgfpathlineto{\pgfqpoint{1.871200in}{0.136829in}}%
\pgfpathlineto{\pgfqpoint{1.865867in}{0.136829in}}%
\pgfpathlineto{\pgfqpoint{1.860534in}{0.136829in}}%
\pgfpathlineto{\pgfqpoint{1.855201in}{0.136829in}}%
\pgfpathlineto{\pgfqpoint{1.849868in}{0.136829in}}%
\pgfpathlineto{\pgfqpoint{1.844534in}{0.136829in}}%
\pgfpathlineto{\pgfqpoint{1.839201in}{0.136829in}}%
\pgfpathlineto{\pgfqpoint{1.833868in}{0.136829in}}%
\pgfpathlineto{\pgfqpoint{1.828535in}{0.136829in}}%
\pgfpathlineto{\pgfqpoint{1.823202in}{0.136829in}}%
\pgfpathlineto{\pgfqpoint{1.817868in}{0.136829in}}%
\pgfpathlineto{\pgfqpoint{1.812535in}{0.136829in}}%
\pgfpathlineto{\pgfqpoint{1.807202in}{0.136829in}}%
\pgfpathlineto{\pgfqpoint{1.801869in}{0.136829in}}%
\pgfpathlineto{\pgfqpoint{1.796535in}{0.136829in}}%
\pgfpathlineto{\pgfqpoint{1.791202in}{0.136829in}}%
\pgfpathlineto{\pgfqpoint{1.785869in}{0.136829in}}%
\pgfpathlineto{\pgfqpoint{1.780536in}{0.136829in}}%
\pgfpathlineto{\pgfqpoint{1.775203in}{0.136829in}}%
\pgfpathlineto{\pgfqpoint{1.769869in}{0.136829in}}%
\pgfpathlineto{\pgfqpoint{1.764536in}{0.136829in}}%
\pgfpathlineto{\pgfqpoint{1.759203in}{0.136829in}}%
\pgfpathlineto{\pgfqpoint{1.753870in}{0.136829in}}%
\pgfpathlineto{\pgfqpoint{1.748537in}{0.136829in}}%
\pgfpathlineto{\pgfqpoint{1.743203in}{0.136829in}}%
\pgfpathlineto{\pgfqpoint{1.737870in}{0.136829in}}%
\pgfpathlineto{\pgfqpoint{1.732537in}{0.136829in}}%
\pgfpathlineto{\pgfqpoint{1.727204in}{0.136829in}}%
\pgfpathlineto{\pgfqpoint{1.721870in}{0.136829in}}%
\pgfpathlineto{\pgfqpoint{1.716537in}{0.136829in}}%
\pgfpathlineto{\pgfqpoint{1.711204in}{0.136829in}}%
\pgfpathlineto{\pgfqpoint{1.705871in}{0.136829in}}%
\pgfpathlineto{\pgfqpoint{1.700538in}{0.136829in}}%
\pgfpathlineto{\pgfqpoint{1.695204in}{0.136829in}}%
\pgfpathlineto{\pgfqpoint{1.689871in}{0.136829in}}%
\pgfpathlineto{\pgfqpoint{1.684538in}{0.136829in}}%
\pgfpathlineto{\pgfqpoint{1.679205in}{0.136829in}}%
\pgfpathlineto{\pgfqpoint{1.673871in}{0.136829in}}%
\pgfpathlineto{\pgfqpoint{1.668538in}{0.136829in}}%
\pgfpathlineto{\pgfqpoint{1.663205in}{0.136829in}}%
\pgfpathlineto{\pgfqpoint{1.657872in}{0.136829in}}%
\pgfpathlineto{\pgfqpoint{1.652539in}{0.136829in}}%
\pgfpathlineto{\pgfqpoint{1.647205in}{0.136829in}}%
\pgfpathlineto{\pgfqpoint{1.641872in}{0.136829in}}%
\pgfpathlineto{\pgfqpoint{1.636539in}{0.136829in}}%
\pgfpathlineto{\pgfqpoint{1.631206in}{0.136829in}}%
\pgfpathlineto{\pgfqpoint{1.625873in}{0.136829in}}%
\pgfpathlineto{\pgfqpoint{1.620539in}{0.136829in}}%
\pgfpathlineto{\pgfqpoint{1.615206in}{0.136829in}}%
\pgfpathlineto{\pgfqpoint{1.609873in}{0.136829in}}%
\pgfpathlineto{\pgfqpoint{1.604540in}{0.136829in}}%
\pgfpathlineto{\pgfqpoint{1.599206in}{0.136829in}}%
\pgfpathlineto{\pgfqpoint{1.593873in}{0.136829in}}%
\pgfpathlineto{\pgfqpoint{1.588540in}{0.136829in}}%
\pgfpathlineto{\pgfqpoint{1.583207in}{0.136829in}}%
\pgfpathlineto{\pgfqpoint{1.577874in}{0.136829in}}%
\pgfpathlineto{\pgfqpoint{1.572540in}{0.136829in}}%
\pgfpathlineto{\pgfqpoint{1.567207in}{0.136829in}}%
\pgfpathlineto{\pgfqpoint{1.561874in}{0.136829in}}%
\pgfpathlineto{\pgfqpoint{1.556541in}{0.136829in}}%
\pgfpathlineto{\pgfqpoint{1.551208in}{0.136829in}}%
\pgfpathlineto{\pgfqpoint{1.545874in}{0.136829in}}%
\pgfpathlineto{\pgfqpoint{1.540541in}{0.136829in}}%
\pgfpathlineto{\pgfqpoint{1.535208in}{0.136829in}}%
\pgfpathlineto{\pgfqpoint{1.529875in}{0.136829in}}%
\pgfpathlineto{\pgfqpoint{1.524541in}{0.136829in}}%
\pgfpathlineto{\pgfqpoint{1.519208in}{0.136829in}}%
\pgfpathlineto{\pgfqpoint{1.513875in}{0.136829in}}%
\pgfpathlineto{\pgfqpoint{1.508542in}{0.136829in}}%
\pgfpathlineto{\pgfqpoint{1.503209in}{0.136829in}}%
\pgfpathlineto{\pgfqpoint{1.497875in}{0.136829in}}%
\pgfpathlineto{\pgfqpoint{1.492542in}{0.136829in}}%
\pgfpathlineto{\pgfqpoint{1.487209in}{0.136829in}}%
\pgfpathlineto{\pgfqpoint{1.481876in}{0.136829in}}%
\pgfpathlineto{\pgfqpoint{1.476543in}{0.136829in}}%
\pgfpathlineto{\pgfqpoint{1.471209in}{0.136829in}}%
\pgfpathlineto{\pgfqpoint{1.465876in}{0.136829in}}%
\pgfpathlineto{\pgfqpoint{1.460543in}{0.136829in}}%
\pgfpathlineto{\pgfqpoint{1.455210in}{0.136829in}}%
\pgfpathlineto{\pgfqpoint{1.449876in}{0.136829in}}%
\pgfpathlineto{\pgfqpoint{1.444543in}{0.136829in}}%
\pgfpathlineto{\pgfqpoint{1.439210in}{0.136829in}}%
\pgfpathlineto{\pgfqpoint{1.433877in}{0.136829in}}%
\pgfpathlineto{\pgfqpoint{1.428544in}{0.136829in}}%
\pgfpathlineto{\pgfqpoint{1.423210in}{0.136829in}}%
\pgfpathlineto{\pgfqpoint{1.417877in}{0.136829in}}%
\pgfpathlineto{\pgfqpoint{1.412544in}{0.136829in}}%
\pgfpathlineto{\pgfqpoint{1.407211in}{0.136829in}}%
\pgfpathlineto{\pgfqpoint{1.401877in}{0.136829in}}%
\pgfpathlineto{\pgfqpoint{1.396544in}{0.136829in}}%
\pgfpathlineto{\pgfqpoint{1.391211in}{0.136829in}}%
\pgfpathlineto{\pgfqpoint{1.385878in}{0.136829in}}%
\pgfpathlineto{\pgfqpoint{1.380545in}{0.136829in}}%
\pgfpathlineto{\pgfqpoint{1.375211in}{0.136829in}}%
\pgfpathlineto{\pgfqpoint{1.369878in}{0.136829in}}%
\pgfpathlineto{\pgfqpoint{1.364545in}{0.136829in}}%
\pgfpathlineto{\pgfqpoint{1.359212in}{0.136829in}}%
\pgfpathlineto{\pgfqpoint{1.353879in}{0.136829in}}%
\pgfpathlineto{\pgfqpoint{1.348545in}{0.136829in}}%
\pgfpathlineto{\pgfqpoint{1.343212in}{0.136829in}}%
\pgfpathlineto{\pgfqpoint{1.337879in}{0.136829in}}%
\pgfpathlineto{\pgfqpoint{1.332546in}{0.136829in}}%
\pgfpathlineto{\pgfqpoint{1.327212in}{0.136829in}}%
\pgfpathlineto{\pgfqpoint{1.321879in}{0.136829in}}%
\pgfpathlineto{\pgfqpoint{1.316546in}{0.136829in}}%
\pgfpathlineto{\pgfqpoint{1.311213in}{0.136829in}}%
\pgfpathlineto{\pgfqpoint{1.305880in}{0.136829in}}%
\pgfpathlineto{\pgfqpoint{1.300546in}{0.136829in}}%
\pgfpathlineto{\pgfqpoint{1.295213in}{0.136829in}}%
\pgfpathlineto{\pgfqpoint{1.289880in}{0.136829in}}%
\pgfpathlineto{\pgfqpoint{1.284547in}{0.136829in}}%
\pgfpathlineto{\pgfqpoint{1.279214in}{0.136829in}}%
\pgfpathlineto{\pgfqpoint{1.273880in}{0.136829in}}%
\pgfpathlineto{\pgfqpoint{1.268547in}{0.136829in}}%
\pgfpathlineto{\pgfqpoint{1.263214in}{0.136829in}}%
\pgfpathlineto{\pgfqpoint{1.257881in}{0.136829in}}%
\pgfpathlineto{\pgfqpoint{1.252547in}{0.136829in}}%
\pgfpathlineto{\pgfqpoint{1.247214in}{0.136829in}}%
\pgfpathlineto{\pgfqpoint{1.241881in}{0.136829in}}%
\pgfpathlineto{\pgfqpoint{1.236548in}{0.136829in}}%
\pgfpathlineto{\pgfqpoint{1.231215in}{0.136829in}}%
\pgfpathlineto{\pgfqpoint{1.225881in}{0.136829in}}%
\pgfpathlineto{\pgfqpoint{1.220548in}{0.136829in}}%
\pgfpathlineto{\pgfqpoint{1.215215in}{0.136829in}}%
\pgfpathlineto{\pgfqpoint{1.209882in}{0.136829in}}%
\pgfpathlineto{\pgfqpoint{1.204549in}{0.136829in}}%
\pgfpathlineto{\pgfqpoint{1.199215in}{0.136829in}}%
\pgfpathlineto{\pgfqpoint{1.193882in}{0.136829in}}%
\pgfpathlineto{\pgfqpoint{1.188549in}{0.136829in}}%
\pgfpathlineto{\pgfqpoint{1.183216in}{0.136829in}}%
\pgfpathlineto{\pgfqpoint{1.177882in}{0.136829in}}%
\pgfpathlineto{\pgfqpoint{1.172549in}{0.136829in}}%
\pgfpathlineto{\pgfqpoint{1.167216in}{0.136829in}}%
\pgfpathlineto{\pgfqpoint{1.161883in}{0.136829in}}%
\pgfpathlineto{\pgfqpoint{1.156550in}{0.136829in}}%
\pgfpathlineto{\pgfqpoint{1.151216in}{0.136829in}}%
\pgfpathlineto{\pgfqpoint{1.145883in}{0.136829in}}%
\pgfpathlineto{\pgfqpoint{1.140550in}{0.136829in}}%
\pgfpathlineto{\pgfqpoint{1.135217in}{0.136829in}}%
\pgfpathlineto{\pgfqpoint{1.129883in}{0.136829in}}%
\pgfpathlineto{\pgfqpoint{1.124550in}{0.136829in}}%
\pgfpathlineto{\pgfqpoint{1.119217in}{0.136829in}}%
\pgfpathlineto{\pgfqpoint{1.113884in}{0.136829in}}%
\pgfpathlineto{\pgfqpoint{1.108551in}{0.136829in}}%
\pgfpathlineto{\pgfqpoint{1.103217in}{0.136829in}}%
\pgfpathlineto{\pgfqpoint{1.097884in}{0.136829in}}%
\pgfpathlineto{\pgfqpoint{1.092551in}{0.136829in}}%
\pgfpathlineto{\pgfqpoint{1.087218in}{0.136829in}}%
\pgfpathlineto{\pgfqpoint{1.081885in}{0.136829in}}%
\pgfpathlineto{\pgfqpoint{1.076551in}{0.136829in}}%
\pgfpathlineto{\pgfqpoint{1.071218in}{0.136829in}}%
\pgfpathlineto{\pgfqpoint{1.065885in}{0.136829in}}%
\pgfpathlineto{\pgfqpoint{1.060552in}{0.136829in}}%
\pgfpathlineto{\pgfqpoint{1.055218in}{0.136829in}}%
\pgfpathlineto{\pgfqpoint{1.049885in}{0.136829in}}%
\pgfpathlineto{\pgfqpoint{1.044552in}{0.136829in}}%
\pgfpathlineto{\pgfqpoint{1.039219in}{0.136829in}}%
\pgfpathlineto{\pgfqpoint{1.033886in}{0.136829in}}%
\pgfpathlineto{\pgfqpoint{1.028552in}{0.136829in}}%
\pgfpathlineto{\pgfqpoint{1.023219in}{0.136829in}}%
\pgfpathlineto{\pgfqpoint{1.017886in}{0.136829in}}%
\pgfpathlineto{\pgfqpoint{1.012553in}{0.136829in}}%
\pgfpathlineto{\pgfqpoint{1.007220in}{0.136829in}}%
\pgfpathlineto{\pgfqpoint{1.001886in}{0.136829in}}%
\pgfpathlineto{\pgfqpoint{0.996553in}{0.136829in}}%
\pgfpathlineto{\pgfqpoint{0.991220in}{0.136829in}}%
\pgfpathlineto{\pgfqpoint{0.985887in}{0.136829in}}%
\pgfpathlineto{\pgfqpoint{0.980553in}{0.136829in}}%
\pgfpathlineto{\pgfqpoint{0.975220in}{0.136829in}}%
\pgfpathlineto{\pgfqpoint{0.969887in}{0.136829in}}%
\pgfpathlineto{\pgfqpoint{0.964554in}{0.136829in}}%
\pgfpathlineto{\pgfqpoint{0.959221in}{0.136829in}}%
\pgfpathlineto{\pgfqpoint{0.953887in}{0.136829in}}%
\pgfpathlineto{\pgfqpoint{0.948554in}{0.136829in}}%
\pgfpathlineto{\pgfqpoint{0.943221in}{0.136829in}}%
\pgfpathlineto{\pgfqpoint{0.937888in}{0.136829in}}%
\pgfpathlineto{\pgfqpoint{0.932555in}{0.136829in}}%
\pgfpathlineto{\pgfqpoint{0.927221in}{0.136829in}}%
\pgfpathlineto{\pgfqpoint{0.921888in}{0.136829in}}%
\pgfpathlineto{\pgfqpoint{0.916555in}{0.136829in}}%
\pgfpathlineto{\pgfqpoint{0.911222in}{0.136829in}}%
\pgfpathlineto{\pgfqpoint{0.905888in}{0.136829in}}%
\pgfpathlineto{\pgfqpoint{0.900555in}{0.136829in}}%
\pgfpathlineto{\pgfqpoint{0.895222in}{0.136829in}}%
\pgfpathlineto{\pgfqpoint{0.889889in}{0.136829in}}%
\pgfpathlineto{\pgfqpoint{0.884556in}{0.136829in}}%
\pgfpathlineto{\pgfqpoint{0.879222in}{0.136829in}}%
\pgfpathlineto{\pgfqpoint{0.873889in}{0.136829in}}%
\pgfpathlineto{\pgfqpoint{0.868556in}{0.136829in}}%
\pgfpathlineto{\pgfqpoint{0.863223in}{0.136829in}}%
\pgfpathlineto{\pgfqpoint{0.857889in}{0.136829in}}%
\pgfpathlineto{\pgfqpoint{0.852556in}{0.136829in}}%
\pgfpathlineto{\pgfqpoint{0.847223in}{0.136829in}}%
\pgfpathlineto{\pgfqpoint{0.847223in}{0.136829in}}%
\pgfpathclose%
\pgfusepath{stroke,fill}%
}%
\begin{pgfscope}%
\pgfsys@transformshift{0.000000in}{0.000000in}%
\pgfsys@useobject{currentmarker}{}%
\end{pgfscope}%
\end{pgfscope}%
\begin{pgfscope}%
\pgfpathrectangle{\pgfqpoint{0.847223in}{0.554012in}}{\pgfqpoint{6.200000in}{4.530000in}}%
\pgfusepath{clip}%
\pgfsetbuttcap%
\pgfsetroundjoin%
\definecolor{currentfill}{rgb}{0.121569,0.466667,0.705882}%
\pgfsetfillcolor{currentfill}%
\pgfsetfillopacity{0.200000}%
\pgfsetlinewidth{1.003750pt}%
\definecolor{currentstroke}{rgb}{0.121569,0.466667,0.705882}%
\pgfsetstrokecolor{currentstroke}%
\pgfsetstrokeopacity{0.200000}%
\pgfsetdash{}{0pt}%
\pgfsys@defobject{currentmarker}{\pgfqpoint{0.847223in}{0.554012in}}{\pgfqpoint{7.047223in}{5.084012in}}{%
\pgfpathmoveto{\pgfqpoint{0.847223in}{5.084012in}}%
\pgfpathlineto{\pgfqpoint{0.847223in}{5.084012in}}%
\pgfpathlineto{\pgfqpoint{0.852556in}{5.034106in}}%
\pgfpathlineto{\pgfqpoint{0.857889in}{4.985197in}}%
\pgfpathlineto{\pgfqpoint{0.863223in}{4.937255in}}%
\pgfpathlineto{\pgfqpoint{0.868556in}{4.890252in}}%
\pgfpathlineto{\pgfqpoint{0.873889in}{4.844161in}}%
\pgfpathlineto{\pgfqpoint{0.879222in}{4.798955in}}%
\pgfpathlineto{\pgfqpoint{0.884556in}{4.754608in}}%
\pgfpathlineto{\pgfqpoint{0.889889in}{4.711098in}}%
\pgfpathlineto{\pgfqpoint{0.895222in}{4.668400in}}%
\pgfpathlineto{\pgfqpoint{0.900555in}{4.626491in}}%
\pgfpathlineto{\pgfqpoint{0.905888in}{4.585351in}}%
\pgfpathlineto{\pgfqpoint{0.911222in}{4.544958in}}%
\pgfpathlineto{\pgfqpoint{0.916555in}{4.505291in}}%
\pgfpathlineto{\pgfqpoint{0.921888in}{4.466333in}}%
\pgfpathlineto{\pgfqpoint{0.927221in}{4.428063in}}%
\pgfpathlineto{\pgfqpoint{0.932555in}{4.390463in}}%
\pgfpathlineto{\pgfqpoint{0.937888in}{4.353517in}}%
\pgfpathlineto{\pgfqpoint{0.943221in}{4.317207in}}%
\pgfpathlineto{\pgfqpoint{0.948554in}{4.281517in}}%
\pgfpathlineto{\pgfqpoint{0.953887in}{4.246431in}}%
\pgfpathlineto{\pgfqpoint{0.959221in}{4.211935in}}%
\pgfpathlineto{\pgfqpoint{0.964554in}{4.178012in}}%
\pgfpathlineto{\pgfqpoint{0.969887in}{4.144650in}}%
\pgfpathlineto{\pgfqpoint{0.975220in}{4.111834in}}%
\pgfpathlineto{\pgfqpoint{0.980553in}{4.079551in}}%
\pgfpathlineto{\pgfqpoint{0.985887in}{4.047788in}}%
\pgfpathlineto{\pgfqpoint{0.991220in}{4.016533in}}%
\pgfpathlineto{\pgfqpoint{0.996553in}{3.985774in}}%
\pgfpathlineto{\pgfqpoint{1.001886in}{3.955498in}}%
\pgfpathlineto{\pgfqpoint{1.007220in}{3.925695in}}%
\pgfpathlineto{\pgfqpoint{1.012553in}{3.896354in}}%
\pgfpathlineto{\pgfqpoint{1.017886in}{3.867463in}}%
\pgfpathlineto{\pgfqpoint{1.023219in}{3.839013in}}%
\pgfpathlineto{\pgfqpoint{1.028552in}{3.810994in}}%
\pgfpathlineto{\pgfqpoint{1.033886in}{3.783396in}}%
\pgfpathlineto{\pgfqpoint{1.039219in}{3.756209in}}%
\pgfpathlineto{\pgfqpoint{1.044552in}{3.729425in}}%
\pgfpathlineto{\pgfqpoint{1.049885in}{3.703034in}}%
\pgfpathlineto{\pgfqpoint{1.055218in}{3.677028in}}%
\pgfpathlineto{\pgfqpoint{1.060552in}{3.651399in}}%
\pgfpathlineto{\pgfqpoint{1.065885in}{3.626138in}}%
\pgfpathlineto{\pgfqpoint{1.071218in}{3.601237in}}%
\pgfpathlineto{\pgfqpoint{1.076551in}{3.576690in}}%
\pgfpathlineto{\pgfqpoint{1.081885in}{3.552487in}}%
\pgfpathlineto{\pgfqpoint{1.087218in}{3.528623in}}%
\pgfpathlineto{\pgfqpoint{1.092551in}{3.505091in}}%
\pgfpathlineto{\pgfqpoint{1.097884in}{3.481882in}}%
\pgfpathlineto{\pgfqpoint{1.103217in}{3.458991in}}%
\pgfpathlineto{\pgfqpoint{1.108551in}{3.436411in}}%
\pgfpathlineto{\pgfqpoint{1.113884in}{3.414137in}}%
\pgfpathlineto{\pgfqpoint{1.119217in}{3.392160in}}%
\pgfpathlineto{\pgfqpoint{1.124550in}{3.370477in}}%
\pgfpathlineto{\pgfqpoint{1.129883in}{3.349081in}}%
\pgfpathlineto{\pgfqpoint{1.135217in}{3.327965in}}%
\pgfpathlineto{\pgfqpoint{1.140550in}{3.307126in}}%
\pgfpathlineto{\pgfqpoint{1.145883in}{3.286557in}}%
\pgfpathlineto{\pgfqpoint{1.151216in}{3.266253in}}%
\pgfpathlineto{\pgfqpoint{1.156550in}{3.246210in}}%
\pgfpathlineto{\pgfqpoint{1.161883in}{3.226421in}}%
\pgfpathlineto{\pgfqpoint{1.167216in}{3.206883in}}%
\pgfpathlineto{\pgfqpoint{1.172549in}{3.187590in}}%
\pgfpathlineto{\pgfqpoint{1.177882in}{3.168538in}}%
\pgfpathlineto{\pgfqpoint{1.183216in}{3.149723in}}%
\pgfpathlineto{\pgfqpoint{1.188549in}{3.131140in}}%
\pgfpathlineto{\pgfqpoint{1.193882in}{3.112785in}}%
\pgfpathlineto{\pgfqpoint{1.199215in}{3.094653in}}%
\pgfpathlineto{\pgfqpoint{1.204549in}{3.076741in}}%
\pgfpathlineto{\pgfqpoint{1.209882in}{3.059045in}}%
\pgfpathlineto{\pgfqpoint{1.215215in}{3.041560in}}%
\pgfpathlineto{\pgfqpoint{1.220548in}{3.024283in}}%
\pgfpathlineto{\pgfqpoint{1.225881in}{3.007211in}}%
\pgfpathlineto{\pgfqpoint{1.231215in}{2.990339in}}%
\pgfpathlineto{\pgfqpoint{1.236548in}{2.973665in}}%
\pgfpathlineto{\pgfqpoint{1.241881in}{2.957184in}}%
\pgfpathlineto{\pgfqpoint{1.247214in}{2.940894in}}%
\pgfpathlineto{\pgfqpoint{1.252547in}{2.924790in}}%
\pgfpathlineto{\pgfqpoint{1.257881in}{2.908871in}}%
\pgfpathlineto{\pgfqpoint{1.263214in}{2.893132in}}%
\pgfpathlineto{\pgfqpoint{1.268547in}{2.877572in}}%
\pgfpathlineto{\pgfqpoint{1.273880in}{2.862185in}}%
\pgfpathlineto{\pgfqpoint{1.279214in}{2.846971in}}%
\pgfpathlineto{\pgfqpoint{1.284547in}{2.831926in}}%
\pgfpathlineto{\pgfqpoint{1.289880in}{2.817046in}}%
\pgfpathlineto{\pgfqpoint{1.295213in}{2.802330in}}%
\pgfpathlineto{\pgfqpoint{1.300546in}{2.787775in}}%
\pgfpathlineto{\pgfqpoint{1.305880in}{2.773378in}}%
\pgfpathlineto{\pgfqpoint{1.311213in}{2.759136in}}%
\pgfpathlineto{\pgfqpoint{1.316546in}{2.745048in}}%
\pgfpathlineto{\pgfqpoint{1.321879in}{2.731110in}}%
\pgfpathlineto{\pgfqpoint{1.327212in}{2.717320in}}%
\pgfpathlineto{\pgfqpoint{1.332546in}{2.703676in}}%
\pgfpathlineto{\pgfqpoint{1.337879in}{2.690176in}}%
\pgfpathlineto{\pgfqpoint{1.343212in}{2.676816in}}%
\pgfpathlineto{\pgfqpoint{1.348545in}{2.663596in}}%
\pgfpathlineto{\pgfqpoint{1.353879in}{2.650513in}}%
\pgfpathlineto{\pgfqpoint{1.359212in}{2.637565in}}%
\pgfpathlineto{\pgfqpoint{1.364545in}{2.624749in}}%
\pgfpathlineto{\pgfqpoint{1.369878in}{2.612064in}}%
\pgfpathlineto{\pgfqpoint{1.375211in}{2.599507in}}%
\pgfpathlineto{\pgfqpoint{1.380545in}{2.587078in}}%
\pgfpathlineto{\pgfqpoint{1.385878in}{2.574773in}}%
\pgfpathlineto{\pgfqpoint{1.391211in}{2.562591in}}%
\pgfpathlineto{\pgfqpoint{1.396544in}{2.550531in}}%
\pgfpathlineto{\pgfqpoint{1.401877in}{2.538589in}}%
\pgfpathlineto{\pgfqpoint{1.407211in}{2.526766in}}%
\pgfpathlineto{\pgfqpoint{1.412544in}{2.515058in}}%
\pgfpathlineto{\pgfqpoint{1.417877in}{2.503464in}}%
\pgfpathlineto{\pgfqpoint{1.423210in}{2.491983in}}%
\pgfpathlineto{\pgfqpoint{1.428544in}{2.480612in}}%
\pgfpathlineto{\pgfqpoint{1.433877in}{2.469351in}}%
\pgfpathlineto{\pgfqpoint{1.439210in}{2.458198in}}%
\pgfpathlineto{\pgfqpoint{1.444543in}{2.447150in}}%
\pgfpathlineto{\pgfqpoint{1.449876in}{2.436208in}}%
\pgfpathlineto{\pgfqpoint{1.455210in}{2.425368in}}%
\pgfpathlineto{\pgfqpoint{1.460543in}{2.414631in}}%
\pgfpathlineto{\pgfqpoint{1.465876in}{2.403993in}}%
\pgfpathlineto{\pgfqpoint{1.471209in}{2.393455in}}%
\pgfpathlineto{\pgfqpoint{1.476543in}{2.383014in}}%
\pgfpathlineto{\pgfqpoint{1.481876in}{2.372669in}}%
\pgfpathlineto{\pgfqpoint{1.487209in}{2.362419in}}%
\pgfpathlineto{\pgfqpoint{1.492542in}{2.352262in}}%
\pgfpathlineto{\pgfqpoint{1.497875in}{2.342198in}}%
\pgfpathlineto{\pgfqpoint{1.503209in}{2.332225in}}%
\pgfpathlineto{\pgfqpoint{1.508542in}{2.322342in}}%
\pgfpathlineto{\pgfqpoint{1.513875in}{2.312547in}}%
\pgfpathlineto{\pgfqpoint{1.519208in}{2.302839in}}%
\pgfpathlineto{\pgfqpoint{1.524541in}{2.293218in}}%
\pgfpathlineto{\pgfqpoint{1.529875in}{2.283682in}}%
\pgfpathlineto{\pgfqpoint{1.535208in}{2.274230in}}%
\pgfpathlineto{\pgfqpoint{1.540541in}{2.264861in}}%
\pgfpathlineto{\pgfqpoint{1.545874in}{2.255573in}}%
\pgfpathlineto{\pgfqpoint{1.551208in}{2.246367in}}%
\pgfpathlineto{\pgfqpoint{1.556541in}{2.237240in}}%
\pgfpathlineto{\pgfqpoint{1.561874in}{2.228191in}}%
\pgfpathlineto{\pgfqpoint{1.567207in}{2.219220in}}%
\pgfpathlineto{\pgfqpoint{1.572540in}{2.210326in}}%
\pgfpathlineto{\pgfqpoint{1.577874in}{2.201508in}}%
\pgfpathlineto{\pgfqpoint{1.583207in}{2.192764in}}%
\pgfpathlineto{\pgfqpoint{1.588540in}{2.184094in}}%
\pgfpathlineto{\pgfqpoint{1.593873in}{2.175497in}}%
\pgfpathlineto{\pgfqpoint{1.599206in}{2.166971in}}%
\pgfpathlineto{\pgfqpoint{1.604540in}{2.158517in}}%
\pgfpathlineto{\pgfqpoint{1.609873in}{2.150133in}}%
\pgfpathlineto{\pgfqpoint{1.615206in}{2.141818in}}%
\pgfpathlineto{\pgfqpoint{1.620539in}{2.133571in}}%
\pgfpathlineto{\pgfqpoint{1.625873in}{2.125392in}}%
\pgfpathlineto{\pgfqpoint{1.631206in}{2.117280in}}%
\pgfpathlineto{\pgfqpoint{1.636539in}{2.109234in}}%
\pgfpathlineto{\pgfqpoint{1.641872in}{2.101252in}}%
\pgfpathlineto{\pgfqpoint{1.647205in}{2.093335in}}%
\pgfpathlineto{\pgfqpoint{1.652539in}{2.085482in}}%
\pgfpathlineto{\pgfqpoint{1.657872in}{2.077692in}}%
\pgfpathlineto{\pgfqpoint{1.663205in}{2.069963in}}%
\pgfpathlineto{\pgfqpoint{1.668538in}{2.062296in}}%
\pgfpathlineto{\pgfqpoint{1.673871in}{2.054689in}}%
\pgfpathlineto{\pgfqpoint{1.679205in}{2.047142in}}%
\pgfpathlineto{\pgfqpoint{1.684538in}{2.039655in}}%
\pgfpathlineto{\pgfqpoint{1.689871in}{2.032226in}}%
\pgfpathlineto{\pgfqpoint{1.695204in}{2.024854in}}%
\pgfpathlineto{\pgfqpoint{1.700538in}{2.017540in}}%
\pgfpathlineto{\pgfqpoint{1.705871in}{2.010282in}}%
\pgfpathlineto{\pgfqpoint{1.711204in}{2.003080in}}%
\pgfpathlineto{\pgfqpoint{1.716537in}{1.995934in}}%
\pgfpathlineto{\pgfqpoint{1.721870in}{1.988841in}}%
\pgfpathlineto{\pgfqpoint{1.727204in}{1.981803in}}%
\pgfpathlineto{\pgfqpoint{1.732537in}{1.974818in}}%
\pgfpathlineto{\pgfqpoint{1.737870in}{1.967885in}}%
\pgfpathlineto{\pgfqpoint{1.743203in}{1.961005in}}%
\pgfpathlineto{\pgfqpoint{1.748537in}{1.954176in}}%
\pgfpathlineto{\pgfqpoint{1.753870in}{1.947398in}}%
\pgfpathlineto{\pgfqpoint{1.759203in}{1.940671in}}%
\pgfpathlineto{\pgfqpoint{1.764536in}{1.933993in}}%
\pgfpathlineto{\pgfqpoint{1.769869in}{1.927365in}}%
\pgfpathlineto{\pgfqpoint{1.775203in}{1.920785in}}%
\pgfpathlineto{\pgfqpoint{1.780536in}{1.914254in}}%
\pgfpathlineto{\pgfqpoint{1.785869in}{1.907770in}}%
\pgfpathlineto{\pgfqpoint{1.791202in}{1.901333in}}%
\pgfpathlineto{\pgfqpoint{1.796535in}{1.894943in}}%
\pgfpathlineto{\pgfqpoint{1.801869in}{1.888599in}}%
\pgfpathlineto{\pgfqpoint{1.807202in}{1.882301in}}%
\pgfpathlineto{\pgfqpoint{1.812535in}{1.876047in}}%
\pgfpathlineto{\pgfqpoint{1.817868in}{1.869839in}}%
\pgfpathlineto{\pgfqpoint{1.823202in}{1.863674in}}%
\pgfpathlineto{\pgfqpoint{1.828535in}{1.857554in}}%
\pgfpathlineto{\pgfqpoint{1.833868in}{1.851476in}}%
\pgfpathlineto{\pgfqpoint{1.839201in}{1.845441in}}%
\pgfpathlineto{\pgfqpoint{1.844534in}{1.839449in}}%
\pgfpathlineto{\pgfqpoint{1.849868in}{1.833498in}}%
\pgfpathlineto{\pgfqpoint{1.855201in}{1.827589in}}%
\pgfpathlineto{\pgfqpoint{1.860534in}{1.821721in}}%
\pgfpathlineto{\pgfqpoint{1.865867in}{1.815894in}}%
\pgfpathlineto{\pgfqpoint{1.871200in}{1.810107in}}%
\pgfpathlineto{\pgfqpoint{1.876534in}{1.804359in}}%
\pgfpathlineto{\pgfqpoint{1.881867in}{1.798651in}}%
\pgfpathlineto{\pgfqpoint{1.887200in}{1.792982in}}%
\pgfpathlineto{\pgfqpoint{1.892533in}{1.787351in}}%
\pgfpathlineto{\pgfqpoint{1.897867in}{1.781758in}}%
\pgfpathlineto{\pgfqpoint{1.903200in}{1.776204in}}%
\pgfpathlineto{\pgfqpoint{1.908533in}{1.770686in}}%
\pgfpathlineto{\pgfqpoint{1.913866in}{1.765206in}}%
\pgfpathlineto{\pgfqpoint{1.919199in}{1.759762in}}%
\pgfpathlineto{\pgfqpoint{1.924533in}{1.754355in}}%
\pgfpathlineto{\pgfqpoint{1.929866in}{1.748983in}}%
\pgfpathlineto{\pgfqpoint{1.935199in}{1.743647in}}%
\pgfpathlineto{\pgfqpoint{1.940532in}{1.738347in}}%
\pgfpathlineto{\pgfqpoint{1.945865in}{1.733081in}}%
\pgfpathlineto{\pgfqpoint{1.951199in}{1.727849in}}%
\pgfpathlineto{\pgfqpoint{1.956532in}{1.722652in}}%
\pgfpathlineto{\pgfqpoint{1.961865in}{1.717489in}}%
\pgfpathlineto{\pgfqpoint{1.967198in}{1.712359in}}%
\pgfpathlineto{\pgfqpoint{1.972532in}{1.707262in}}%
\pgfpathlineto{\pgfqpoint{1.977865in}{1.702199in}}%
\pgfpathlineto{\pgfqpoint{1.983198in}{1.697167in}}%
\pgfpathlineto{\pgfqpoint{1.988531in}{1.692168in}}%
\pgfpathlineto{\pgfqpoint{1.993864in}{1.687201in}}%
\pgfpathlineto{\pgfqpoint{1.999198in}{1.682266in}}%
\pgfpathlineto{\pgfqpoint{2.004531in}{1.677362in}}%
\pgfpathlineto{\pgfqpoint{2.009864in}{1.672489in}}%
\pgfpathlineto{\pgfqpoint{2.015197in}{1.667646in}}%
\pgfpathlineto{\pgfqpoint{2.020531in}{1.662834in}}%
\pgfpathlineto{\pgfqpoint{2.025864in}{1.658053in}}%
\pgfpathlineto{\pgfqpoint{2.031197in}{1.653301in}}%
\pgfpathlineto{\pgfqpoint{2.036530in}{1.648578in}}%
\pgfpathlineto{\pgfqpoint{2.041863in}{1.643885in}}%
\pgfpathlineto{\pgfqpoint{2.047197in}{1.639221in}}%
\pgfpathlineto{\pgfqpoint{2.052530in}{1.634586in}}%
\pgfpathlineto{\pgfqpoint{2.057863in}{1.629980in}}%
\pgfpathlineto{\pgfqpoint{2.063196in}{1.625401in}}%
\pgfpathlineto{\pgfqpoint{2.068529in}{1.620851in}}%
\pgfpathlineto{\pgfqpoint{2.073863in}{1.616328in}}%
\pgfpathlineto{\pgfqpoint{2.079196in}{1.611833in}}%
\pgfpathlineto{\pgfqpoint{2.084529in}{1.607365in}}%
\pgfpathlineto{\pgfqpoint{2.089862in}{1.602924in}}%
\pgfpathlineto{\pgfqpoint{2.095196in}{1.598510in}}%
\pgfpathlineto{\pgfqpoint{2.100529in}{1.594122in}}%
\pgfpathlineto{\pgfqpoint{2.105862in}{1.589761in}}%
\pgfpathlineto{\pgfqpoint{2.111195in}{1.585425in}}%
\pgfpathlineto{\pgfqpoint{2.116528in}{1.581116in}}%
\pgfpathlineto{\pgfqpoint{2.121862in}{1.576832in}}%
\pgfpathlineto{\pgfqpoint{2.127195in}{1.572573in}}%
\pgfpathlineto{\pgfqpoint{2.132528in}{1.568339in}}%
\pgfpathlineto{\pgfqpoint{2.137861in}{1.564131in}}%
\pgfpathlineto{\pgfqpoint{2.143194in}{1.559947in}}%
\pgfpathlineto{\pgfqpoint{2.148528in}{1.555787in}}%
\pgfpathlineto{\pgfqpoint{2.153861in}{1.551652in}}%
\pgfpathlineto{\pgfqpoint{2.159194in}{1.547540in}}%
\pgfpathlineto{\pgfqpoint{2.164527in}{1.543453in}}%
\pgfpathlineto{\pgfqpoint{2.169861in}{1.539389in}}%
\pgfpathlineto{\pgfqpoint{2.175194in}{1.535349in}}%
\pgfpathlineto{\pgfqpoint{2.180527in}{1.531331in}}%
\pgfpathlineto{\pgfqpoint{2.185860in}{1.527337in}}%
\pgfpathlineto{\pgfqpoint{2.191193in}{1.523366in}}%
\pgfpathlineto{\pgfqpoint{2.196527in}{1.519417in}}%
\pgfpathlineto{\pgfqpoint{2.201860in}{1.515491in}}%
\pgfpathlineto{\pgfqpoint{2.207193in}{1.511586in}}%
\pgfpathlineto{\pgfqpoint{2.212526in}{1.507704in}}%
\pgfpathlineto{\pgfqpoint{2.217859in}{1.503844in}}%
\pgfpathlineto{\pgfqpoint{2.223193in}{1.500006in}}%
\pgfpathlineto{\pgfqpoint{2.228526in}{1.496189in}}%
\pgfpathlineto{\pgfqpoint{2.233859in}{1.492393in}}%
\pgfpathlineto{\pgfqpoint{2.239192in}{1.488618in}}%
\pgfpathlineto{\pgfqpoint{2.244526in}{1.484865in}}%
\pgfpathlineto{\pgfqpoint{2.249859in}{1.481132in}}%
\pgfpathlineto{\pgfqpoint{2.255192in}{1.477420in}}%
\pgfpathlineto{\pgfqpoint{2.260525in}{1.473728in}}%
\pgfpathlineto{\pgfqpoint{2.265858in}{1.470056in}}%
\pgfpathlineto{\pgfqpoint{2.271192in}{1.466405in}}%
\pgfpathlineto{\pgfqpoint{2.276525in}{1.462774in}}%
\pgfpathlineto{\pgfqpoint{2.281858in}{1.459162in}}%
\pgfpathlineto{\pgfqpoint{2.287191in}{1.455570in}}%
\pgfpathlineto{\pgfqpoint{2.292524in}{1.451997in}}%
\pgfpathlineto{\pgfqpoint{2.297858in}{1.448444in}}%
\pgfpathlineto{\pgfqpoint{2.303191in}{1.444910in}}%
\pgfpathlineto{\pgfqpoint{2.308524in}{1.441395in}}%
\pgfpathlineto{\pgfqpoint{2.313857in}{1.437899in}}%
\pgfpathlineto{\pgfqpoint{2.319191in}{1.434421in}}%
\pgfpathlineto{\pgfqpoint{2.324524in}{1.430962in}}%
\pgfpathlineto{\pgfqpoint{2.329857in}{1.427521in}}%
\pgfpathlineto{\pgfqpoint{2.335190in}{1.424099in}}%
\pgfpathlineto{\pgfqpoint{2.340523in}{1.420695in}}%
\pgfpathlineto{\pgfqpoint{2.345857in}{1.417308in}}%
\pgfpathlineto{\pgfqpoint{2.351190in}{1.413940in}}%
\pgfpathlineto{\pgfqpoint{2.356523in}{1.410589in}}%
\pgfpathlineto{\pgfqpoint{2.361856in}{1.407256in}}%
\pgfpathlineto{\pgfqpoint{2.367190in}{1.403940in}}%
\pgfpathlineto{\pgfqpoint{2.372523in}{1.400641in}}%
\pgfpathlineto{\pgfqpoint{2.377856in}{1.397360in}}%
\pgfpathlineto{\pgfqpoint{2.383189in}{1.394095in}}%
\pgfpathlineto{\pgfqpoint{2.388522in}{1.390847in}}%
\pgfpathlineto{\pgfqpoint{2.393856in}{1.387616in}}%
\pgfpathlineto{\pgfqpoint{2.399189in}{1.384402in}}%
\pgfpathlineto{\pgfqpoint{2.404522in}{1.381204in}}%
\pgfpathlineto{\pgfqpoint{2.409855in}{1.378023in}}%
\pgfpathlineto{\pgfqpoint{2.415188in}{1.374858in}}%
\pgfpathlineto{\pgfqpoint{2.420522in}{1.371708in}}%
\pgfpathlineto{\pgfqpoint{2.425855in}{1.368575in}}%
\pgfpathlineto{\pgfqpoint{2.431188in}{1.365458in}}%
\pgfpathlineto{\pgfqpoint{2.436521in}{1.362356in}}%
\pgfpathlineto{\pgfqpoint{2.441855in}{1.359270in}}%
\pgfpathlineto{\pgfqpoint{2.447188in}{1.356200in}}%
\pgfpathlineto{\pgfqpoint{2.452521in}{1.353145in}}%
\pgfpathlineto{\pgfqpoint{2.457854in}{1.350105in}}%
\pgfpathlineto{\pgfqpoint{2.463187in}{1.347080in}}%
\pgfpathlineto{\pgfqpoint{2.468521in}{1.344071in}}%
\pgfpathlineto{\pgfqpoint{2.473854in}{1.341076in}}%
\pgfpathlineto{\pgfqpoint{2.479187in}{1.338096in}}%
\pgfpathlineto{\pgfqpoint{2.484520in}{1.335131in}}%
\pgfpathlineto{\pgfqpoint{2.489853in}{1.332181in}}%
\pgfpathlineto{\pgfqpoint{2.495187in}{1.329245in}}%
\pgfpathlineto{\pgfqpoint{2.500520in}{1.326323in}}%
\pgfpathlineto{\pgfqpoint{2.505853in}{1.323416in}}%
\pgfpathlineto{\pgfqpoint{2.511186in}{1.320522in}}%
\pgfpathlineto{\pgfqpoint{2.516520in}{1.317643in}}%
\pgfpathlineto{\pgfqpoint{2.521853in}{1.314778in}}%
\pgfpathlineto{\pgfqpoint{2.527186in}{1.311927in}}%
\pgfpathlineto{\pgfqpoint{2.532519in}{1.309089in}}%
\pgfpathlineto{\pgfqpoint{2.537852in}{1.306265in}}%
\pgfpathlineto{\pgfqpoint{2.543186in}{1.303455in}}%
\pgfpathlineto{\pgfqpoint{2.548519in}{1.300658in}}%
\pgfpathlineto{\pgfqpoint{2.553852in}{1.297875in}}%
\pgfpathlineto{\pgfqpoint{2.559185in}{1.295105in}}%
\pgfpathlineto{\pgfqpoint{2.564518in}{1.292348in}}%
\pgfpathlineto{\pgfqpoint{2.569852in}{1.289604in}}%
\pgfpathlineto{\pgfqpoint{2.575185in}{1.286873in}}%
\pgfpathlineto{\pgfqpoint{2.580518in}{1.284155in}}%
\pgfpathlineto{\pgfqpoint{2.585851in}{1.281450in}}%
\pgfpathlineto{\pgfqpoint{2.591185in}{1.278758in}}%
\pgfpathlineto{\pgfqpoint{2.596518in}{1.276078in}}%
\pgfpathlineto{\pgfqpoint{2.601851in}{1.273411in}}%
\pgfpathlineto{\pgfqpoint{2.607184in}{1.270756in}}%
\pgfpathlineto{\pgfqpoint{2.612517in}{1.268114in}}%
\pgfpathlineto{\pgfqpoint{2.617851in}{1.265484in}}%
\pgfpathlineto{\pgfqpoint{2.623184in}{1.262866in}}%
\pgfpathlineto{\pgfqpoint{2.628517in}{1.260260in}}%
\pgfpathlineto{\pgfqpoint{2.633850in}{1.257666in}}%
\pgfpathlineto{\pgfqpoint{2.639184in}{1.255084in}}%
\pgfpathlineto{\pgfqpoint{2.644517in}{1.252514in}}%
\pgfpathlineto{\pgfqpoint{2.649850in}{1.249956in}}%
\pgfpathlineto{\pgfqpoint{2.655183in}{1.247410in}}%
\pgfpathlineto{\pgfqpoint{2.660516in}{1.244875in}}%
\pgfpathlineto{\pgfqpoint{2.665850in}{1.242352in}}%
\pgfpathlineto{\pgfqpoint{2.671183in}{1.239840in}}%
\pgfpathlineto{\pgfqpoint{2.676516in}{1.237339in}}%
\pgfpathlineto{\pgfqpoint{2.681849in}{1.234850in}}%
\pgfpathlineto{\pgfqpoint{2.687182in}{1.232372in}}%
\pgfpathlineto{\pgfqpoint{2.692516in}{1.229906in}}%
\pgfpathlineto{\pgfqpoint{2.697849in}{1.227450in}}%
\pgfpathlineto{\pgfqpoint{2.703182in}{1.225005in}}%
\pgfpathlineto{\pgfqpoint{2.708515in}{1.222572in}}%
\pgfpathlineto{\pgfqpoint{2.713849in}{1.220149in}}%
\pgfpathlineto{\pgfqpoint{2.719182in}{1.217737in}}%
\pgfpathlineto{\pgfqpoint{2.724515in}{1.215336in}}%
\pgfpathlineto{\pgfqpoint{2.729848in}{1.212945in}}%
\pgfpathlineto{\pgfqpoint{2.735181in}{1.210565in}}%
\pgfpathlineto{\pgfqpoint{2.740515in}{1.208195in}}%
\pgfpathlineto{\pgfqpoint{2.745848in}{1.205836in}}%
\pgfpathlineto{\pgfqpoint{2.751181in}{1.203487in}}%
\pgfpathlineto{\pgfqpoint{2.756514in}{1.201149in}}%
\pgfpathlineto{\pgfqpoint{2.761847in}{1.198821in}}%
\pgfpathlineto{\pgfqpoint{2.767181in}{1.196502in}}%
\pgfpathlineto{\pgfqpoint{2.772514in}{1.194194in}}%
\pgfpathlineto{\pgfqpoint{2.777847in}{1.191896in}}%
\pgfpathlineto{\pgfqpoint{2.783180in}{1.189608in}}%
\pgfpathlineto{\pgfqpoint{2.788514in}{1.187330in}}%
\pgfpathlineto{\pgfqpoint{2.793847in}{1.185062in}}%
\pgfpathlineto{\pgfqpoint{2.799180in}{1.182804in}}%
\pgfpathlineto{\pgfqpoint{2.804513in}{1.180555in}}%
\pgfpathlineto{\pgfqpoint{2.809846in}{1.178316in}}%
\pgfpathlineto{\pgfqpoint{2.815180in}{1.176086in}}%
\pgfpathlineto{\pgfqpoint{2.820513in}{1.173866in}}%
\pgfpathlineto{\pgfqpoint{2.825846in}{1.171656in}}%
\pgfpathlineto{\pgfqpoint{2.831179in}{1.169454in}}%
\pgfpathlineto{\pgfqpoint{2.836512in}{1.167263in}}%
\pgfpathlineto{\pgfqpoint{2.841846in}{1.165080in}}%
\pgfpathlineto{\pgfqpoint{2.847179in}{1.162907in}}%
\pgfpathlineto{\pgfqpoint{2.852512in}{1.160743in}}%
\pgfpathlineto{\pgfqpoint{2.857845in}{1.158588in}}%
\pgfpathlineto{\pgfqpoint{2.863179in}{1.156442in}}%
\pgfpathlineto{\pgfqpoint{2.868512in}{1.154305in}}%
\pgfpathlineto{\pgfqpoint{2.873845in}{1.152177in}}%
\pgfpathlineto{\pgfqpoint{2.879178in}{1.150058in}}%
\pgfpathlineto{\pgfqpoint{2.884511in}{1.147947in}}%
\pgfpathlineto{\pgfqpoint{2.889845in}{1.145846in}}%
\pgfpathlineto{\pgfqpoint{2.895178in}{1.143753in}}%
\pgfpathlineto{\pgfqpoint{2.900511in}{1.141669in}}%
\pgfpathlineto{\pgfqpoint{2.905844in}{1.139593in}}%
\pgfpathlineto{\pgfqpoint{2.911178in}{1.137526in}}%
\pgfpathlineto{\pgfqpoint{2.916511in}{1.135468in}}%
\pgfpathlineto{\pgfqpoint{2.921844in}{1.133418in}}%
\pgfpathlineto{\pgfqpoint{2.927177in}{1.131376in}}%
\pgfpathlineto{\pgfqpoint{2.932510in}{1.129343in}}%
\pgfpathlineto{\pgfqpoint{2.937844in}{1.127318in}}%
\pgfpathlineto{\pgfqpoint{2.943177in}{1.125301in}}%
\pgfpathlineto{\pgfqpoint{2.948510in}{1.123292in}}%
\pgfpathlineto{\pgfqpoint{2.953843in}{1.121292in}}%
\pgfpathlineto{\pgfqpoint{2.959176in}{1.119300in}}%
\pgfpathlineto{\pgfqpoint{2.964510in}{1.117315in}}%
\pgfpathlineto{\pgfqpoint{2.969843in}{1.115339in}}%
\pgfpathlineto{\pgfqpoint{2.975176in}{1.113371in}}%
\pgfpathlineto{\pgfqpoint{2.980509in}{1.111410in}}%
\pgfpathlineto{\pgfqpoint{2.985843in}{1.109458in}}%
\pgfpathlineto{\pgfqpoint{2.991176in}{1.107513in}}%
\pgfpathlineto{\pgfqpoint{2.996509in}{1.105576in}}%
\pgfpathlineto{\pgfqpoint{3.001842in}{1.103647in}}%
\pgfpathlineto{\pgfqpoint{3.007175in}{1.101725in}}%
\pgfpathlineto{\pgfqpoint{3.012509in}{1.099811in}}%
\pgfpathlineto{\pgfqpoint{3.017842in}{1.097905in}}%
\pgfpathlineto{\pgfqpoint{3.023175in}{1.096006in}}%
\pgfpathlineto{\pgfqpoint{3.028508in}{1.094115in}}%
\pgfpathlineto{\pgfqpoint{3.033841in}{1.092231in}}%
\pgfpathlineto{\pgfqpoint{3.039175in}{1.090354in}}%
\pgfpathlineto{\pgfqpoint{3.044508in}{1.088485in}}%
\pgfpathlineto{\pgfqpoint{3.049841in}{1.086623in}}%
\pgfpathlineto{\pgfqpoint{3.055174in}{1.084768in}}%
\pgfpathlineto{\pgfqpoint{3.060508in}{1.082921in}}%
\pgfpathlineto{\pgfqpoint{3.065841in}{1.081081in}}%
\pgfpathlineto{\pgfqpoint{3.071174in}{1.079248in}}%
\pgfpathlineto{\pgfqpoint{3.076507in}{1.077422in}}%
\pgfpathlineto{\pgfqpoint{3.081840in}{1.075603in}}%
\pgfpathlineto{\pgfqpoint{3.087174in}{1.073791in}}%
\pgfpathlineto{\pgfqpoint{3.092507in}{1.071986in}}%
\pgfpathlineto{\pgfqpoint{3.097840in}{1.070188in}}%
\pgfpathlineto{\pgfqpoint{3.103173in}{1.068397in}}%
\pgfpathlineto{\pgfqpoint{3.108506in}{1.066613in}}%
\pgfpathlineto{\pgfqpoint{3.113840in}{1.064836in}}%
\pgfpathlineto{\pgfqpoint{3.119173in}{1.063065in}}%
\pgfpathlineto{\pgfqpoint{3.124506in}{1.061301in}}%
\pgfpathlineto{\pgfqpoint{3.129839in}{1.059544in}}%
\pgfpathlineto{\pgfqpoint{3.135173in}{1.057794in}}%
\pgfpathlineto{\pgfqpoint{3.140506in}{1.056050in}}%
\pgfpathlineto{\pgfqpoint{3.145839in}{1.054313in}}%
\pgfpathlineto{\pgfqpoint{3.151172in}{1.052582in}}%
\pgfpathlineto{\pgfqpoint{3.156505in}{1.050858in}}%
\pgfpathlineto{\pgfqpoint{3.161839in}{1.049140in}}%
\pgfpathlineto{\pgfqpoint{3.167172in}{1.047429in}}%
\pgfpathlineto{\pgfqpoint{3.172505in}{1.045724in}}%
\pgfpathlineto{\pgfqpoint{3.177838in}{1.044026in}}%
\pgfpathlineto{\pgfqpoint{3.183172in}{1.042334in}}%
\pgfpathlineto{\pgfqpoint{3.188505in}{1.040648in}}%
\pgfpathlineto{\pgfqpoint{3.193838in}{1.038968in}}%
\pgfpathlineto{\pgfqpoint{3.199171in}{1.037295in}}%
\pgfpathlineto{\pgfqpoint{3.204504in}{1.035628in}}%
\pgfpathlineto{\pgfqpoint{3.209838in}{1.033967in}}%
\pgfpathlineto{\pgfqpoint{3.215171in}{1.032312in}}%
\pgfpathlineto{\pgfqpoint{3.220504in}{1.030663in}}%
\pgfpathlineto{\pgfqpoint{3.225837in}{1.029021in}}%
\pgfpathlineto{\pgfqpoint{3.231170in}{1.027384in}}%
\pgfpathlineto{\pgfqpoint{3.236504in}{1.025753in}}%
\pgfpathlineto{\pgfqpoint{3.241837in}{1.024129in}}%
\pgfpathlineto{\pgfqpoint{3.247170in}{1.022510in}}%
\pgfpathlineto{\pgfqpoint{3.252503in}{1.020897in}}%
\pgfpathlineto{\pgfqpoint{3.257837in}{1.019290in}}%
\pgfpathlineto{\pgfqpoint{3.263170in}{1.017689in}}%
\pgfpathlineto{\pgfqpoint{3.268503in}{1.016093in}}%
\pgfpathlineto{\pgfqpoint{3.273836in}{1.014504in}}%
\pgfpathlineto{\pgfqpoint{3.279169in}{1.012920in}}%
\pgfpathlineto{\pgfqpoint{3.284503in}{1.011342in}}%
\pgfpathlineto{\pgfqpoint{3.289836in}{1.009769in}}%
\pgfpathlineto{\pgfqpoint{3.295169in}{1.008202in}}%
\pgfpathlineto{\pgfqpoint{3.300502in}{1.006641in}}%
\pgfpathlineto{\pgfqpoint{3.305835in}{1.005085in}}%
\pgfpathlineto{\pgfqpoint{3.311169in}{1.003535in}}%
\pgfpathlineto{\pgfqpoint{3.316502in}{1.001991in}}%
\pgfpathlineto{\pgfqpoint{3.321835in}{1.000452in}}%
\pgfpathlineto{\pgfqpoint{3.327168in}{0.998918in}}%
\pgfpathlineto{\pgfqpoint{3.332502in}{0.997390in}}%
\pgfpathlineto{\pgfqpoint{3.337835in}{0.995867in}}%
\pgfpathlineto{\pgfqpoint{3.343168in}{0.994350in}}%
\pgfpathlineto{\pgfqpoint{3.348501in}{0.992838in}}%
\pgfpathlineto{\pgfqpoint{3.353834in}{0.991331in}}%
\pgfpathlineto{\pgfqpoint{3.359168in}{0.989830in}}%
\pgfpathlineto{\pgfqpoint{3.364501in}{0.988333in}}%
\pgfpathlineto{\pgfqpoint{3.369834in}{0.986843in}}%
\pgfpathlineto{\pgfqpoint{3.375167in}{0.985357in}}%
\pgfpathlineto{\pgfqpoint{3.380500in}{0.983876in}}%
\pgfpathlineto{\pgfqpoint{3.385834in}{0.982401in}}%
\pgfpathlineto{\pgfqpoint{3.391167in}{0.980931in}}%
\pgfpathlineto{\pgfqpoint{3.396500in}{0.979466in}}%
\pgfpathlineto{\pgfqpoint{3.401833in}{0.978006in}}%
\pgfpathlineto{\pgfqpoint{3.407167in}{0.976551in}}%
\pgfpathlineto{\pgfqpoint{3.412500in}{0.975101in}}%
\pgfpathlineto{\pgfqpoint{3.417833in}{0.973656in}}%
\pgfpathlineto{\pgfqpoint{3.423166in}{0.972216in}}%
\pgfpathlineto{\pgfqpoint{3.428499in}{0.970781in}}%
\pgfpathlineto{\pgfqpoint{3.433833in}{0.969350in}}%
\pgfpathlineto{\pgfqpoint{3.439166in}{0.967925in}}%
\pgfpathlineto{\pgfqpoint{3.444499in}{0.966505in}}%
\pgfpathlineto{\pgfqpoint{3.449832in}{0.965089in}}%
\pgfpathlineto{\pgfqpoint{3.455166in}{0.963679in}}%
\pgfpathlineto{\pgfqpoint{3.460499in}{0.962273in}}%
\pgfpathlineto{\pgfqpoint{3.465832in}{0.960872in}}%
\pgfpathlineto{\pgfqpoint{3.471165in}{0.959475in}}%
\pgfpathlineto{\pgfqpoint{3.476498in}{0.958084in}}%
\pgfpathlineto{\pgfqpoint{3.481832in}{0.956697in}}%
\pgfpathlineto{\pgfqpoint{3.487165in}{0.955314in}}%
\pgfpathlineto{\pgfqpoint{3.492498in}{0.953937in}}%
\pgfpathlineto{\pgfqpoint{3.497831in}{0.952564in}}%
\pgfpathlineto{\pgfqpoint{3.503164in}{0.951195in}}%
\pgfpathlineto{\pgfqpoint{3.508498in}{0.949832in}}%
\pgfpathlineto{\pgfqpoint{3.513831in}{0.948472in}}%
\pgfpathlineto{\pgfqpoint{3.519164in}{0.947118in}}%
\pgfpathlineto{\pgfqpoint{3.524497in}{0.945767in}}%
\pgfpathlineto{\pgfqpoint{3.529831in}{0.944422in}}%
\pgfpathlineto{\pgfqpoint{3.535164in}{0.943081in}}%
\pgfpathlineto{\pgfqpoint{3.540497in}{0.941744in}}%
\pgfpathlineto{\pgfqpoint{3.545830in}{0.940411in}}%
\pgfpathlineto{\pgfqpoint{3.551163in}{0.939083in}}%
\pgfpathlineto{\pgfqpoint{3.556497in}{0.937760in}}%
\pgfpathlineto{\pgfqpoint{3.561830in}{0.936441in}}%
\pgfpathlineto{\pgfqpoint{3.567163in}{0.935126in}}%
\pgfpathlineto{\pgfqpoint{3.572496in}{0.933815in}}%
\pgfpathlineto{\pgfqpoint{3.577829in}{0.932509in}}%
\pgfpathlineto{\pgfqpoint{3.583163in}{0.931207in}}%
\pgfpathlineto{\pgfqpoint{3.588496in}{0.929909in}}%
\pgfpathlineto{\pgfqpoint{3.593829in}{0.928616in}}%
\pgfpathlineto{\pgfqpoint{3.599162in}{0.927326in}}%
\pgfpathlineto{\pgfqpoint{3.604496in}{0.926041in}}%
\pgfpathlineto{\pgfqpoint{3.609829in}{0.924760in}}%
\pgfpathlineto{\pgfqpoint{3.615162in}{0.923484in}}%
\pgfpathlineto{\pgfqpoint{3.620495in}{0.922211in}}%
\pgfpathlineto{\pgfqpoint{3.625828in}{0.920943in}}%
\pgfpathlineto{\pgfqpoint{3.631162in}{0.919678in}}%
\pgfpathlineto{\pgfqpoint{3.636495in}{0.918418in}}%
\pgfpathlineto{\pgfqpoint{3.641828in}{0.917161in}}%
\pgfpathlineto{\pgfqpoint{3.647161in}{0.915909in}}%
\pgfpathlineto{\pgfqpoint{3.652494in}{0.914661in}}%
\pgfpathlineto{\pgfqpoint{3.657828in}{0.913417in}}%
\pgfpathlineto{\pgfqpoint{3.663161in}{0.912176in}}%
\pgfpathlineto{\pgfqpoint{3.668494in}{0.910940in}}%
\pgfpathlineto{\pgfqpoint{3.673827in}{0.909708in}}%
\pgfpathlineto{\pgfqpoint{3.679161in}{0.908479in}}%
\pgfpathlineto{\pgfqpoint{3.684494in}{0.907254in}}%
\pgfpathlineto{\pgfqpoint{3.689827in}{0.906034in}}%
\pgfpathlineto{\pgfqpoint{3.695160in}{0.904817in}}%
\pgfpathlineto{\pgfqpoint{3.700493in}{0.903604in}}%
\pgfpathlineto{\pgfqpoint{3.705827in}{0.902395in}}%
\pgfpathlineto{\pgfqpoint{3.711160in}{0.901189in}}%
\pgfpathlineto{\pgfqpoint{3.716493in}{0.899988in}}%
\pgfpathlineto{\pgfqpoint{3.721826in}{0.898790in}}%
\pgfpathlineto{\pgfqpoint{3.727160in}{0.897596in}}%
\pgfpathlineto{\pgfqpoint{3.732493in}{0.896406in}}%
\pgfpathlineto{\pgfqpoint{3.737826in}{0.895219in}}%
\pgfpathlineto{\pgfqpoint{3.743159in}{0.894036in}}%
\pgfpathlineto{\pgfqpoint{3.748492in}{0.892857in}}%
\pgfpathlineto{\pgfqpoint{3.753826in}{0.891681in}}%
\pgfpathlineto{\pgfqpoint{3.759159in}{0.890509in}}%
\pgfpathlineto{\pgfqpoint{3.764492in}{0.889341in}}%
\pgfpathlineto{\pgfqpoint{3.769825in}{0.888177in}}%
\pgfpathlineto{\pgfqpoint{3.775158in}{0.887016in}}%
\pgfpathlineto{\pgfqpoint{3.780492in}{0.885858in}}%
\pgfpathlineto{\pgfqpoint{3.785825in}{0.884704in}}%
\pgfpathlineto{\pgfqpoint{3.791158in}{0.883554in}}%
\pgfpathlineto{\pgfqpoint{3.796491in}{0.882407in}}%
\pgfpathlineto{\pgfqpoint{3.801825in}{0.881264in}}%
\pgfpathlineto{\pgfqpoint{3.807158in}{0.880124in}}%
\pgfpathlineto{\pgfqpoint{3.812491in}{0.878988in}}%
\pgfpathlineto{\pgfqpoint{3.817824in}{0.877855in}}%
\pgfpathlineto{\pgfqpoint{3.823157in}{0.876725in}}%
\pgfpathlineto{\pgfqpoint{3.828491in}{0.875599in}}%
\pgfpathlineto{\pgfqpoint{3.833824in}{0.874477in}}%
\pgfpathlineto{\pgfqpoint{3.839157in}{0.873358in}}%
\pgfpathlineto{\pgfqpoint{3.844490in}{0.872242in}}%
\pgfpathlineto{\pgfqpoint{3.849823in}{0.871130in}}%
\pgfpathlineto{\pgfqpoint{3.855157in}{0.870021in}}%
\pgfpathlineto{\pgfqpoint{3.860490in}{0.868915in}}%
\pgfpathlineto{\pgfqpoint{3.865823in}{0.867813in}}%
\pgfpathlineto{\pgfqpoint{3.871156in}{0.866714in}}%
\pgfpathlineto{\pgfqpoint{3.876490in}{0.865618in}}%
\pgfpathlineto{\pgfqpoint{3.881823in}{0.864526in}}%
\pgfpathlineto{\pgfqpoint{3.887156in}{0.863436in}}%
\pgfpathlineto{\pgfqpoint{3.892489in}{0.862351in}}%
\pgfpathlineto{\pgfqpoint{3.897822in}{0.861268in}}%
\pgfpathlineto{\pgfqpoint{3.903156in}{0.860188in}}%
\pgfpathlineto{\pgfqpoint{3.908489in}{0.859112in}}%
\pgfpathlineto{\pgfqpoint{3.913822in}{0.858039in}}%
\pgfpathlineto{\pgfqpoint{3.919155in}{0.856969in}}%
\pgfpathlineto{\pgfqpoint{3.924488in}{0.855903in}}%
\pgfpathlineto{\pgfqpoint{3.929822in}{0.854839in}}%
\pgfpathlineto{\pgfqpoint{3.935155in}{0.853779in}}%
\pgfpathlineto{\pgfqpoint{3.940488in}{0.852722in}}%
\pgfpathlineto{\pgfqpoint{3.945821in}{0.851667in}}%
\pgfpathlineto{\pgfqpoint{3.951155in}{0.850616in}}%
\pgfpathlineto{\pgfqpoint{3.956488in}{0.849568in}}%
\pgfpathlineto{\pgfqpoint{3.961821in}{0.848524in}}%
\pgfpathlineto{\pgfqpoint{3.967154in}{0.847482in}}%
\pgfpathlineto{\pgfqpoint{3.972487in}{0.846443in}}%
\pgfpathlineto{\pgfqpoint{3.977821in}{0.845407in}}%
\pgfpathlineto{\pgfqpoint{3.983154in}{0.844375in}}%
\pgfpathlineto{\pgfqpoint{3.988487in}{0.843345in}}%
\pgfpathlineto{\pgfqpoint{3.993820in}{0.842318in}}%
\pgfpathlineto{\pgfqpoint{3.999154in}{0.841294in}}%
\pgfpathlineto{\pgfqpoint{4.004487in}{0.840274in}}%
\pgfpathlineto{\pgfqpoint{4.009820in}{0.839256in}}%
\pgfpathlineto{\pgfqpoint{4.015153in}{0.838241in}}%
\pgfpathlineto{\pgfqpoint{4.020486in}{0.837229in}}%
\pgfpathlineto{\pgfqpoint{4.025820in}{0.836220in}}%
\pgfpathlineto{\pgfqpoint{4.031153in}{0.835214in}}%
\pgfpathlineto{\pgfqpoint{4.036486in}{0.834211in}}%
\pgfpathlineto{\pgfqpoint{4.041819in}{0.833210in}}%
\pgfpathlineto{\pgfqpoint{4.047152in}{0.832213in}}%
\pgfpathlineto{\pgfqpoint{4.052486in}{0.831218in}}%
\pgfpathlineto{\pgfqpoint{4.057819in}{0.830226in}}%
\pgfpathlineto{\pgfqpoint{4.063152in}{0.829237in}}%
\pgfpathlineto{\pgfqpoint{4.068485in}{0.828251in}}%
\pgfpathlineto{\pgfqpoint{4.073819in}{0.827268in}}%
\pgfpathlineto{\pgfqpoint{4.079152in}{0.826287in}}%
\pgfpathlineto{\pgfqpoint{4.084485in}{0.825309in}}%
\pgfpathlineto{\pgfqpoint{4.089818in}{0.824334in}}%
\pgfpathlineto{\pgfqpoint{4.095151in}{0.823362in}}%
\pgfpathlineto{\pgfqpoint{4.100485in}{0.822393in}}%
\pgfpathlineto{\pgfqpoint{4.105818in}{0.821426in}}%
\pgfpathlineto{\pgfqpoint{4.111151in}{0.820462in}}%
\pgfpathlineto{\pgfqpoint{4.116484in}{0.819501in}}%
\pgfpathlineto{\pgfqpoint{4.121817in}{0.818542in}}%
\pgfpathlineto{\pgfqpoint{4.127151in}{0.817586in}}%
\pgfpathlineto{\pgfqpoint{4.132484in}{0.816633in}}%
\pgfpathlineto{\pgfqpoint{4.137817in}{0.815682in}}%
\pgfpathlineto{\pgfqpoint{4.143150in}{0.814734in}}%
\pgfpathlineto{\pgfqpoint{4.148484in}{0.813789in}}%
\pgfpathlineto{\pgfqpoint{4.153817in}{0.812846in}}%
\pgfpathlineto{\pgfqpoint{4.159150in}{0.811906in}}%
\pgfpathlineto{\pgfqpoint{4.164483in}{0.810969in}}%
\pgfpathlineto{\pgfqpoint{4.169816in}{0.810034in}}%
\pgfpathlineto{\pgfqpoint{4.175150in}{0.809102in}}%
\pgfpathlineto{\pgfqpoint{4.180483in}{0.808172in}}%
\pgfpathlineto{\pgfqpoint{4.185816in}{0.807245in}}%
\pgfpathlineto{\pgfqpoint{4.191149in}{0.806320in}}%
\pgfpathlineto{\pgfqpoint{4.196482in}{0.805398in}}%
\pgfpathlineto{\pgfqpoint{4.201816in}{0.804479in}}%
\pgfpathlineto{\pgfqpoint{4.207149in}{0.803562in}}%
\pgfpathlineto{\pgfqpoint{4.212482in}{0.802647in}}%
\pgfpathlineto{\pgfqpoint{4.217815in}{0.801736in}}%
\pgfpathlineto{\pgfqpoint{4.223149in}{0.800826in}}%
\pgfpathlineto{\pgfqpoint{4.228482in}{0.799919in}}%
\pgfpathlineto{\pgfqpoint{4.233815in}{0.799015in}}%
\pgfpathlineto{\pgfqpoint{4.239148in}{0.798113in}}%
\pgfpathlineto{\pgfqpoint{4.244481in}{0.797213in}}%
\pgfpathlineto{\pgfqpoint{4.249815in}{0.796316in}}%
\pgfpathlineto{\pgfqpoint{4.255148in}{0.795421in}}%
\pgfpathlineto{\pgfqpoint{4.260481in}{0.794529in}}%
\pgfpathlineto{\pgfqpoint{4.265814in}{0.793639in}}%
\pgfpathlineto{\pgfqpoint{4.271148in}{0.792752in}}%
\pgfpathlineto{\pgfqpoint{4.276481in}{0.791867in}}%
\pgfpathlineto{\pgfqpoint{4.281814in}{0.790984in}}%
\pgfpathlineto{\pgfqpoint{4.287147in}{0.790104in}}%
\pgfpathlineto{\pgfqpoint{4.292480in}{0.789226in}}%
\pgfpathlineto{\pgfqpoint{4.297814in}{0.788351in}}%
\pgfpathlineto{\pgfqpoint{4.303147in}{0.787477in}}%
\pgfpathlineto{\pgfqpoint{4.308480in}{0.786606in}}%
\pgfpathlineto{\pgfqpoint{4.313813in}{0.785738in}}%
\pgfpathlineto{\pgfqpoint{4.319146in}{0.784872in}}%
\pgfpathlineto{\pgfqpoint{4.324480in}{0.784008in}}%
\pgfpathlineto{\pgfqpoint{4.329813in}{0.783146in}}%
\pgfpathlineto{\pgfqpoint{4.335146in}{0.782287in}}%
\pgfpathlineto{\pgfqpoint{4.340479in}{0.781430in}}%
\pgfpathlineto{\pgfqpoint{4.345813in}{0.780575in}}%
\pgfpathlineto{\pgfqpoint{4.351146in}{0.779723in}}%
\pgfpathlineto{\pgfqpoint{4.356479in}{0.778872in}}%
\pgfpathlineto{\pgfqpoint{4.361812in}{0.778024in}}%
\pgfpathlineto{\pgfqpoint{4.367145in}{0.777179in}}%
\pgfpathlineto{\pgfqpoint{4.372479in}{0.776335in}}%
\pgfpathlineto{\pgfqpoint{4.377812in}{0.775494in}}%
\pgfpathlineto{\pgfqpoint{4.383145in}{0.774655in}}%
\pgfpathlineto{\pgfqpoint{4.388478in}{0.773818in}}%
\pgfpathlineto{\pgfqpoint{4.393811in}{0.772983in}}%
\pgfpathlineto{\pgfqpoint{4.399145in}{0.772151in}}%
\pgfpathlineto{\pgfqpoint{4.404478in}{0.771320in}}%
\pgfpathlineto{\pgfqpoint{4.409811in}{0.770492in}}%
\pgfpathlineto{\pgfqpoint{4.415144in}{0.769666in}}%
\pgfpathlineto{\pgfqpoint{4.420478in}{0.768842in}}%
\pgfpathlineto{\pgfqpoint{4.425811in}{0.768020in}}%
\pgfpathlineto{\pgfqpoint{4.431144in}{0.767201in}}%
\pgfpathlineto{\pgfqpoint{4.436477in}{0.766383in}}%
\pgfpathlineto{\pgfqpoint{4.441810in}{0.765568in}}%
\pgfpathlineto{\pgfqpoint{4.447144in}{0.764755in}}%
\pgfpathlineto{\pgfqpoint{4.452477in}{0.763944in}}%
\pgfpathlineto{\pgfqpoint{4.457810in}{0.763135in}}%
\pgfpathlineto{\pgfqpoint{4.463143in}{0.762328in}}%
\pgfpathlineto{\pgfqpoint{4.468476in}{0.761523in}}%
\pgfpathlineto{\pgfqpoint{4.473810in}{0.760720in}}%
\pgfpathlineto{\pgfqpoint{4.479143in}{0.759919in}}%
\pgfpathlineto{\pgfqpoint{4.484476in}{0.759120in}}%
\pgfpathlineto{\pgfqpoint{4.489809in}{0.758324in}}%
\pgfpathlineto{\pgfqpoint{4.495143in}{0.757529in}}%
\pgfpathlineto{\pgfqpoint{4.500476in}{0.756737in}}%
\pgfpathlineto{\pgfqpoint{4.505809in}{0.755946in}}%
\pgfpathlineto{\pgfqpoint{4.511142in}{0.755157in}}%
\pgfpathlineto{\pgfqpoint{4.516475in}{0.754371in}}%
\pgfpathlineto{\pgfqpoint{4.521809in}{0.753586in}}%
\pgfpathlineto{\pgfqpoint{4.527142in}{0.752804in}}%
\pgfpathlineto{\pgfqpoint{4.532475in}{0.752023in}}%
\pgfpathlineto{\pgfqpoint{4.537808in}{0.751245in}}%
\pgfpathlineto{\pgfqpoint{4.543142in}{0.750468in}}%
\pgfpathlineto{\pgfqpoint{4.548475in}{0.749693in}}%
\pgfpathlineto{\pgfqpoint{4.553808in}{0.748921in}}%
\pgfpathlineto{\pgfqpoint{4.559141in}{0.748150in}}%
\pgfpathlineto{\pgfqpoint{4.564474in}{0.747381in}}%
\pgfpathlineto{\pgfqpoint{4.569808in}{0.746614in}}%
\pgfpathlineto{\pgfqpoint{4.575141in}{0.745849in}}%
\pgfpathlineto{\pgfqpoint{4.580474in}{0.745086in}}%
\pgfpathlineto{\pgfqpoint{4.585807in}{0.744325in}}%
\pgfpathlineto{\pgfqpoint{4.591140in}{0.743566in}}%
\pgfpathlineto{\pgfqpoint{4.596474in}{0.742808in}}%
\pgfpathlineto{\pgfqpoint{4.601807in}{0.742053in}}%
\pgfpathlineto{\pgfqpoint{4.607140in}{0.741299in}}%
\pgfpathlineto{\pgfqpoint{4.612473in}{0.740548in}}%
\pgfpathlineto{\pgfqpoint{4.617807in}{0.739798in}}%
\pgfpathlineto{\pgfqpoint{4.623140in}{0.739050in}}%
\pgfpathlineto{\pgfqpoint{4.628473in}{0.738304in}}%
\pgfpathlineto{\pgfqpoint{4.633806in}{0.737559in}}%
\pgfpathlineto{\pgfqpoint{4.639139in}{0.736817in}}%
\pgfpathlineto{\pgfqpoint{4.644473in}{0.736076in}}%
\pgfpathlineto{\pgfqpoint{4.649806in}{0.735338in}}%
\pgfpathlineto{\pgfqpoint{4.655139in}{0.734601in}}%
\pgfpathlineto{\pgfqpoint{4.660472in}{0.733865in}}%
\pgfpathlineto{\pgfqpoint{4.665805in}{0.733132in}}%
\pgfpathlineto{\pgfqpoint{4.671139in}{0.732401in}}%
\pgfpathlineto{\pgfqpoint{4.676472in}{0.731671in}}%
\pgfpathlineto{\pgfqpoint{4.681805in}{0.730943in}}%
\pgfpathlineto{\pgfqpoint{4.687138in}{0.730217in}}%
\pgfpathlineto{\pgfqpoint{4.692472in}{0.729492in}}%
\pgfpathlineto{\pgfqpoint{4.697805in}{0.728770in}}%
\pgfpathlineto{\pgfqpoint{4.703138in}{0.728049in}}%
\pgfpathlineto{\pgfqpoint{4.708471in}{0.727330in}}%
\pgfpathlineto{\pgfqpoint{4.713804in}{0.726612in}}%
\pgfpathlineto{\pgfqpoint{4.719138in}{0.725897in}}%
\pgfpathlineto{\pgfqpoint{4.724471in}{0.725183in}}%
\pgfpathlineto{\pgfqpoint{4.729804in}{0.724470in}}%
\pgfpathlineto{\pgfqpoint{4.735137in}{0.723760in}}%
\pgfpathlineto{\pgfqpoint{4.740470in}{0.723051in}}%
\pgfpathlineto{\pgfqpoint{4.745804in}{0.722344in}}%
\pgfpathlineto{\pgfqpoint{4.751137in}{0.721639in}}%
\pgfpathlineto{\pgfqpoint{4.756470in}{0.720935in}}%
\pgfpathlineto{\pgfqpoint{4.761803in}{0.720233in}}%
\pgfpathlineto{\pgfqpoint{4.767137in}{0.719533in}}%
\pgfpathlineto{\pgfqpoint{4.772470in}{0.718834in}}%
\pgfpathlineto{\pgfqpoint{4.777803in}{0.718138in}}%
\pgfpathlineto{\pgfqpoint{4.783136in}{0.717442in}}%
\pgfpathlineto{\pgfqpoint{4.788469in}{0.716749in}}%
\pgfpathlineto{\pgfqpoint{4.793803in}{0.716057in}}%
\pgfpathlineto{\pgfqpoint{4.799136in}{0.715367in}}%
\pgfpathlineto{\pgfqpoint{4.804469in}{0.714678in}}%
\pgfpathlineto{\pgfqpoint{4.809802in}{0.713991in}}%
\pgfpathlineto{\pgfqpoint{4.815136in}{0.713306in}}%
\pgfpathlineto{\pgfqpoint{4.820469in}{0.712622in}}%
\pgfpathlineto{\pgfqpoint{4.825802in}{0.711940in}}%
\pgfpathlineto{\pgfqpoint{4.831135in}{0.711259in}}%
\pgfpathlineto{\pgfqpoint{4.836468in}{0.710580in}}%
\pgfpathlineto{\pgfqpoint{4.841802in}{0.709903in}}%
\pgfpathlineto{\pgfqpoint{4.847135in}{0.709227in}}%
\pgfpathlineto{\pgfqpoint{4.852468in}{0.708553in}}%
\pgfpathlineto{\pgfqpoint{4.857801in}{0.707881in}}%
\pgfpathlineto{\pgfqpoint{4.863134in}{0.707210in}}%
\pgfpathlineto{\pgfqpoint{4.868468in}{0.706540in}}%
\pgfpathlineto{\pgfqpoint{4.873801in}{0.705873in}}%
\pgfpathlineto{\pgfqpoint{4.879134in}{0.705206in}}%
\pgfpathlineto{\pgfqpoint{4.884467in}{0.704542in}}%
\pgfpathlineto{\pgfqpoint{4.889801in}{0.703879in}}%
\pgfpathlineto{\pgfqpoint{4.895134in}{0.703217in}}%
\pgfpathlineto{\pgfqpoint{4.900467in}{0.702557in}}%
\pgfpathlineto{\pgfqpoint{4.905800in}{0.701899in}}%
\pgfpathlineto{\pgfqpoint{4.911133in}{0.701242in}}%
\pgfpathlineto{\pgfqpoint{4.916467in}{0.700586in}}%
\pgfpathlineto{\pgfqpoint{4.921800in}{0.699932in}}%
\pgfpathlineto{\pgfqpoint{4.927133in}{0.699280in}}%
\pgfpathlineto{\pgfqpoint{4.932466in}{0.698629in}}%
\pgfpathlineto{\pgfqpoint{4.937799in}{0.697980in}}%
\pgfpathlineto{\pgfqpoint{4.943133in}{0.697332in}}%
\pgfpathlineto{\pgfqpoint{4.948466in}{0.696685in}}%
\pgfpathlineto{\pgfqpoint{4.953799in}{0.696040in}}%
\pgfpathlineto{\pgfqpoint{4.959132in}{0.695397in}}%
\pgfpathlineto{\pgfqpoint{4.964466in}{0.694755in}}%
\pgfpathlineto{\pgfqpoint{4.969799in}{0.694115in}}%
\pgfpathlineto{\pgfqpoint{4.975132in}{0.693476in}}%
\pgfpathlineto{\pgfqpoint{4.980465in}{0.692838in}}%
\pgfpathlineto{\pgfqpoint{4.985798in}{0.692202in}}%
\pgfpathlineto{\pgfqpoint{4.991132in}{0.691567in}}%
\pgfpathlineto{\pgfqpoint{4.996465in}{0.690934in}}%
\pgfpathlineto{\pgfqpoint{5.001798in}{0.690303in}}%
\pgfpathlineto{\pgfqpoint{5.007131in}{0.689672in}}%
\pgfpathlineto{\pgfqpoint{5.012464in}{0.689043in}}%
\pgfpathlineto{\pgfqpoint{5.017798in}{0.688416in}}%
\pgfpathlineto{\pgfqpoint{5.023131in}{0.687790in}}%
\pgfpathlineto{\pgfqpoint{5.028464in}{0.687165in}}%
\pgfpathlineto{\pgfqpoint{5.033797in}{0.686542in}}%
\pgfpathlineto{\pgfqpoint{5.039131in}{0.685920in}}%
\pgfpathlineto{\pgfqpoint{5.044464in}{0.685300in}}%
\pgfpathlineto{\pgfqpoint{5.049797in}{0.684681in}}%
\pgfpathlineto{\pgfqpoint{5.055130in}{0.684064in}}%
\pgfpathlineto{\pgfqpoint{5.060463in}{0.683447in}}%
\pgfpathlineto{\pgfqpoint{5.065797in}{0.682833in}}%
\pgfpathlineto{\pgfqpoint{5.071130in}{0.682219in}}%
\pgfpathlineto{\pgfqpoint{5.076463in}{0.681607in}}%
\pgfpathlineto{\pgfqpoint{5.081796in}{0.680997in}}%
\pgfpathlineto{\pgfqpoint{5.087130in}{0.680387in}}%
\pgfpathlineto{\pgfqpoint{5.092463in}{0.679779in}}%
\pgfpathlineto{\pgfqpoint{5.097796in}{0.679173in}}%
\pgfpathlineto{\pgfqpoint{5.103129in}{0.678568in}}%
\pgfpathlineto{\pgfqpoint{5.108462in}{0.677964in}}%
\pgfpathlineto{\pgfqpoint{5.113796in}{0.677361in}}%
\pgfpathlineto{\pgfqpoint{5.119129in}{0.676760in}}%
\pgfpathlineto{\pgfqpoint{5.124462in}{0.676160in}}%
\pgfpathlineto{\pgfqpoint{5.129795in}{0.675562in}}%
\pgfpathlineto{\pgfqpoint{5.135128in}{0.674965in}}%
\pgfpathlineto{\pgfqpoint{5.140462in}{0.674369in}}%
\pgfpathlineto{\pgfqpoint{5.145795in}{0.673774in}}%
\pgfpathlineto{\pgfqpoint{5.151128in}{0.673181in}}%
\pgfpathlineto{\pgfqpoint{5.156461in}{0.672589in}}%
\pgfpathlineto{\pgfqpoint{5.161795in}{0.671998in}}%
\pgfpathlineto{\pgfqpoint{5.167128in}{0.671409in}}%
\pgfpathlineto{\pgfqpoint{5.172461in}{0.670821in}}%
\pgfpathlineto{\pgfqpoint{5.177794in}{0.670234in}}%
\pgfpathlineto{\pgfqpoint{5.183127in}{0.669649in}}%
\pgfpathlineto{\pgfqpoint{5.188461in}{0.669065in}}%
\pgfpathlineto{\pgfqpoint{5.193794in}{0.668482in}}%
\pgfpathlineto{\pgfqpoint{5.199127in}{0.667900in}}%
\pgfpathlineto{\pgfqpoint{5.204460in}{0.667320in}}%
\pgfpathlineto{\pgfqpoint{5.209793in}{0.666741in}}%
\pgfpathlineto{\pgfqpoint{5.215127in}{0.666163in}}%
\pgfpathlineto{\pgfqpoint{5.220460in}{0.665587in}}%
\pgfpathlineto{\pgfqpoint{5.225793in}{0.665011in}}%
\pgfpathlineto{\pgfqpoint{5.231126in}{0.664437in}}%
\pgfpathlineto{\pgfqpoint{5.236460in}{0.663864in}}%
\pgfpathlineto{\pgfqpoint{5.241793in}{0.663293in}}%
\pgfpathlineto{\pgfqpoint{5.247126in}{0.662723in}}%
\pgfpathlineto{\pgfqpoint{5.252459in}{0.662154in}}%
\pgfpathlineto{\pgfqpoint{5.257792in}{0.661586in}}%
\pgfpathlineto{\pgfqpoint{5.263126in}{0.661019in}}%
\pgfpathlineto{\pgfqpoint{5.268459in}{0.660454in}}%
\pgfpathlineto{\pgfqpoint{5.273792in}{0.659890in}}%
\pgfpathlineto{\pgfqpoint{5.279125in}{0.659327in}}%
\pgfpathlineto{\pgfqpoint{5.284458in}{0.658765in}}%
\pgfpathlineto{\pgfqpoint{5.289792in}{0.658204in}}%
\pgfpathlineto{\pgfqpoint{5.295125in}{0.657645in}}%
\pgfpathlineto{\pgfqpoint{5.300458in}{0.657087in}}%
\pgfpathlineto{\pgfqpoint{5.305791in}{0.656530in}}%
\pgfpathlineto{\pgfqpoint{5.311125in}{0.655974in}}%
\pgfpathlineto{\pgfqpoint{5.316458in}{0.655420in}}%
\pgfpathlineto{\pgfqpoint{5.321791in}{0.654866in}}%
\pgfpathlineto{\pgfqpoint{5.327124in}{0.654314in}}%
\pgfpathlineto{\pgfqpoint{5.332457in}{0.653763in}}%
\pgfpathlineto{\pgfqpoint{5.337791in}{0.653213in}}%
\pgfpathlineto{\pgfqpoint{5.343124in}{0.652664in}}%
\pgfpathlineto{\pgfqpoint{5.348457in}{0.652117in}}%
\pgfpathlineto{\pgfqpoint{5.353790in}{0.651571in}}%
\pgfpathlineto{\pgfqpoint{5.359124in}{0.651025in}}%
\pgfpathlineto{\pgfqpoint{5.364457in}{0.650481in}}%
\pgfpathlineto{\pgfqpoint{5.369790in}{0.649938in}}%
\pgfpathlineto{\pgfqpoint{5.375123in}{0.649397in}}%
\pgfpathlineto{\pgfqpoint{5.380456in}{0.648856in}}%
\pgfpathlineto{\pgfqpoint{5.385790in}{0.648317in}}%
\pgfpathlineto{\pgfqpoint{5.391123in}{0.647778in}}%
\pgfpathlineto{\pgfqpoint{5.396456in}{0.647241in}}%
\pgfpathlineto{\pgfqpoint{5.401789in}{0.646705in}}%
\pgfpathlineto{\pgfqpoint{5.407122in}{0.646170in}}%
\pgfpathlineto{\pgfqpoint{5.412456in}{0.645636in}}%
\pgfpathlineto{\pgfqpoint{5.417789in}{0.645103in}}%
\pgfpathlineto{\pgfqpoint{5.423122in}{0.644572in}}%
\pgfpathlineto{\pgfqpoint{5.428455in}{0.644041in}}%
\pgfpathlineto{\pgfqpoint{5.433789in}{0.643512in}}%
\pgfpathlineto{\pgfqpoint{5.439122in}{0.642984in}}%
\pgfpathlineto{\pgfqpoint{5.444455in}{0.642457in}}%
\pgfpathlineto{\pgfqpoint{5.449788in}{0.641930in}}%
\pgfpathlineto{\pgfqpoint{5.455121in}{0.641406in}}%
\pgfpathlineto{\pgfqpoint{5.460455in}{0.640882in}}%
\pgfpathlineto{\pgfqpoint{5.465788in}{0.640359in}}%
\pgfpathlineto{\pgfqpoint{5.471121in}{0.639837in}}%
\pgfpathlineto{\pgfqpoint{5.476454in}{0.639316in}}%
\pgfpathlineto{\pgfqpoint{5.481787in}{0.638797in}}%
\pgfpathlineto{\pgfqpoint{5.487121in}{0.638278in}}%
\pgfpathlineto{\pgfqpoint{5.492454in}{0.637761in}}%
\pgfpathlineto{\pgfqpoint{5.497787in}{0.637245in}}%
\pgfpathlineto{\pgfqpoint{5.503120in}{0.636729in}}%
\pgfpathlineto{\pgfqpoint{5.508454in}{0.636215in}}%
\pgfpathlineto{\pgfqpoint{5.513787in}{0.635702in}}%
\pgfpathlineto{\pgfqpoint{5.519120in}{0.635190in}}%
\pgfpathlineto{\pgfqpoint{5.524453in}{0.634679in}}%
\pgfpathlineto{\pgfqpoint{5.529786in}{0.634169in}}%
\pgfpathlineto{\pgfqpoint{5.535120in}{0.633660in}}%
\pgfpathlineto{\pgfqpoint{5.540453in}{0.633152in}}%
\pgfpathlineto{\pgfqpoint{5.545786in}{0.632645in}}%
\pgfpathlineto{\pgfqpoint{5.551119in}{0.632139in}}%
\pgfpathlineto{\pgfqpoint{5.556452in}{0.631634in}}%
\pgfpathlineto{\pgfqpoint{5.561786in}{0.631130in}}%
\pgfpathlineto{\pgfqpoint{5.567119in}{0.630628in}}%
\pgfpathlineto{\pgfqpoint{5.572452in}{0.630126in}}%
\pgfpathlineto{\pgfqpoint{5.577785in}{0.629625in}}%
\pgfpathlineto{\pgfqpoint{5.583119in}{0.629125in}}%
\pgfpathlineto{\pgfqpoint{5.588452in}{0.628627in}}%
\pgfpathlineto{\pgfqpoint{5.593785in}{0.628129in}}%
\pgfpathlineto{\pgfqpoint{5.599118in}{0.627632in}}%
\pgfpathlineto{\pgfqpoint{5.604451in}{0.627137in}}%
\pgfpathlineto{\pgfqpoint{5.609785in}{0.626642in}}%
\pgfpathlineto{\pgfqpoint{5.615118in}{0.626148in}}%
\pgfpathlineto{\pgfqpoint{5.620451in}{0.625655in}}%
\pgfpathlineto{\pgfqpoint{5.625784in}{0.625164in}}%
\pgfpathlineto{\pgfqpoint{5.631118in}{0.624673in}}%
\pgfpathlineto{\pgfqpoint{5.636451in}{0.624183in}}%
\pgfpathlineto{\pgfqpoint{5.641784in}{0.623695in}}%
\pgfpathlineto{\pgfqpoint{5.647117in}{0.623207in}}%
\pgfpathlineto{\pgfqpoint{5.652450in}{0.622720in}}%
\pgfpathlineto{\pgfqpoint{5.657784in}{0.622234in}}%
\pgfpathlineto{\pgfqpoint{5.663117in}{0.621749in}}%
\pgfpathlineto{\pgfqpoint{5.668450in}{0.621265in}}%
\pgfpathlineto{\pgfqpoint{5.673783in}{0.620782in}}%
\pgfpathlineto{\pgfqpoint{5.679116in}{0.620300in}}%
\pgfpathlineto{\pgfqpoint{5.684450in}{0.619819in}}%
\pgfpathlineto{\pgfqpoint{5.689783in}{0.619339in}}%
\pgfpathlineto{\pgfqpoint{5.695116in}{0.618860in}}%
\pgfpathlineto{\pgfqpoint{5.700449in}{0.618382in}}%
\pgfpathlineto{\pgfqpoint{5.705783in}{0.617905in}}%
\pgfpathlineto{\pgfqpoint{5.711116in}{0.617429in}}%
\pgfpathlineto{\pgfqpoint{5.716449in}{0.616953in}}%
\pgfpathlineto{\pgfqpoint{5.721782in}{0.616479in}}%
\pgfpathlineto{\pgfqpoint{5.727115in}{0.616006in}}%
\pgfpathlineto{\pgfqpoint{5.732449in}{0.615533in}}%
\pgfpathlineto{\pgfqpoint{5.737782in}{0.615062in}}%
\pgfpathlineto{\pgfqpoint{5.743115in}{0.614591in}}%
\pgfpathlineto{\pgfqpoint{5.748448in}{0.614121in}}%
\pgfpathlineto{\pgfqpoint{5.753781in}{0.613652in}}%
\pgfpathlineto{\pgfqpoint{5.759115in}{0.613185in}}%
\pgfpathlineto{\pgfqpoint{5.764448in}{0.612718in}}%
\pgfpathlineto{\pgfqpoint{5.769781in}{0.612252in}}%
\pgfpathlineto{\pgfqpoint{5.775114in}{0.611786in}}%
\pgfpathlineto{\pgfqpoint{5.780448in}{0.611322in}}%
\pgfpathlineto{\pgfqpoint{5.785781in}{0.610859in}}%
\pgfpathlineto{\pgfqpoint{5.791114in}{0.610396in}}%
\pgfpathlineto{\pgfqpoint{5.796447in}{0.609935in}}%
\pgfpathlineto{\pgfqpoint{5.801780in}{0.609474in}}%
\pgfpathlineto{\pgfqpoint{5.807114in}{0.609015in}}%
\pgfpathlineto{\pgfqpoint{5.812447in}{0.608556in}}%
\pgfpathlineto{\pgfqpoint{5.817780in}{0.608098in}}%
\pgfpathlineto{\pgfqpoint{5.823113in}{0.607641in}}%
\pgfpathlineto{\pgfqpoint{5.828446in}{0.607185in}}%
\pgfpathlineto{\pgfqpoint{5.833780in}{0.606729in}}%
\pgfpathlineto{\pgfqpoint{5.839113in}{0.606275in}}%
\pgfpathlineto{\pgfqpoint{5.844446in}{0.605821in}}%
\pgfpathlineto{\pgfqpoint{5.849779in}{0.605369in}}%
\pgfpathlineto{\pgfqpoint{5.855113in}{0.604917in}}%
\pgfpathlineto{\pgfqpoint{5.860446in}{0.604466in}}%
\pgfpathlineto{\pgfqpoint{5.865779in}{0.604016in}}%
\pgfpathlineto{\pgfqpoint{5.871112in}{0.603567in}}%
\pgfpathlineto{\pgfqpoint{5.876445in}{0.603119in}}%
\pgfpathlineto{\pgfqpoint{5.881779in}{0.602671in}}%
\pgfpathlineto{\pgfqpoint{5.887112in}{0.602225in}}%
\pgfpathlineto{\pgfqpoint{5.892445in}{0.601779in}}%
\pgfpathlineto{\pgfqpoint{5.897778in}{0.601334in}}%
\pgfpathlineto{\pgfqpoint{5.903112in}{0.600890in}}%
\pgfpathlineto{\pgfqpoint{5.908445in}{0.600447in}}%
\pgfpathlineto{\pgfqpoint{5.913778in}{0.600005in}}%
\pgfpathlineto{\pgfqpoint{5.919111in}{0.599563in}}%
\pgfpathlineto{\pgfqpoint{5.924444in}{0.599122in}}%
\pgfpathlineto{\pgfqpoint{5.929778in}{0.598683in}}%
\pgfpathlineto{\pgfqpoint{5.935111in}{0.598244in}}%
\pgfpathlineto{\pgfqpoint{5.940444in}{0.597806in}}%
\pgfpathlineto{\pgfqpoint{5.945777in}{0.597368in}}%
\pgfpathlineto{\pgfqpoint{5.951110in}{0.596932in}}%
\pgfpathlineto{\pgfqpoint{5.956444in}{0.596496in}}%
\pgfpathlineto{\pgfqpoint{5.961777in}{0.596061in}}%
\pgfpathlineto{\pgfqpoint{5.967110in}{0.595627in}}%
\pgfpathlineto{\pgfqpoint{5.972443in}{0.595194in}}%
\pgfpathlineto{\pgfqpoint{5.977777in}{0.594762in}}%
\pgfpathlineto{\pgfqpoint{5.983110in}{0.594330in}}%
\pgfpathlineto{\pgfqpoint{5.988443in}{0.593899in}}%
\pgfpathlineto{\pgfqpoint{5.993776in}{0.593469in}}%
\pgfpathlineto{\pgfqpoint{5.999109in}{0.593040in}}%
\pgfpathlineto{\pgfqpoint{6.004443in}{0.592612in}}%
\pgfpathlineto{\pgfqpoint{6.009776in}{0.592185in}}%
\pgfpathlineto{\pgfqpoint{6.015109in}{0.591758in}}%
\pgfpathlineto{\pgfqpoint{6.020442in}{0.591332in}}%
\pgfpathlineto{\pgfqpoint{6.025775in}{0.590907in}}%
\pgfpathlineto{\pgfqpoint{6.031109in}{0.590482in}}%
\pgfpathlineto{\pgfqpoint{6.036442in}{0.590059in}}%
\pgfpathlineto{\pgfqpoint{6.041775in}{0.589636in}}%
\pgfpathlineto{\pgfqpoint{6.047108in}{0.589214in}}%
\pgfpathlineto{\pgfqpoint{6.052442in}{0.588793in}}%
\pgfpathlineto{\pgfqpoint{6.057775in}{0.588373in}}%
\pgfpathlineto{\pgfqpoint{6.063108in}{0.587953in}}%
\pgfpathlineto{\pgfqpoint{6.068441in}{0.587534in}}%
\pgfpathlineto{\pgfqpoint{6.073774in}{0.587116in}}%
\pgfpathlineto{\pgfqpoint{6.079108in}{0.586699in}}%
\pgfpathlineto{\pgfqpoint{6.084441in}{0.586282in}}%
\pgfpathlineto{\pgfqpoint{6.089774in}{0.585867in}}%
\pgfpathlineto{\pgfqpoint{6.095107in}{0.585452in}}%
\pgfpathlineto{\pgfqpoint{6.100440in}{0.585038in}}%
\pgfpathlineto{\pgfqpoint{6.105774in}{0.584624in}}%
\pgfpathlineto{\pgfqpoint{6.111107in}{0.584212in}}%
\pgfpathlineto{\pgfqpoint{6.116440in}{0.583800in}}%
\pgfpathlineto{\pgfqpoint{6.121773in}{0.583388in}}%
\pgfpathlineto{\pgfqpoint{6.127107in}{0.582978in}}%
\pgfpathlineto{\pgfqpoint{6.132440in}{0.582568in}}%
\pgfpathlineto{\pgfqpoint{6.137773in}{0.582160in}}%
\pgfpathlineto{\pgfqpoint{6.143106in}{0.581751in}}%
\pgfpathlineto{\pgfqpoint{6.148439in}{0.581344in}}%
\pgfpathlineto{\pgfqpoint{6.153773in}{0.580937in}}%
\pgfpathlineto{\pgfqpoint{6.159106in}{0.580531in}}%
\pgfpathlineto{\pgfqpoint{6.164439in}{0.580126in}}%
\pgfpathlineto{\pgfqpoint{6.169772in}{0.579722in}}%
\pgfpathlineto{\pgfqpoint{6.175105in}{0.579318in}}%
\pgfpathlineto{\pgfqpoint{6.175105in}{0.579318in}}%
\pgfpathlineto{\pgfqpoint{6.183915in}{0.578665in}}%
\pgfpathlineto{\pgfqpoint{6.192724in}{0.578037in}}%
\pgfpathlineto{\pgfqpoint{6.201533in}{0.577433in}}%
\pgfpathlineto{\pgfqpoint{6.210343in}{0.576850in}}%
\pgfpathlineto{\pgfqpoint{6.219152in}{0.576287in}}%
\pgfpathlineto{\pgfqpoint{6.227961in}{0.575743in}}%
\pgfpathlineto{\pgfqpoint{6.236770in}{0.575217in}}%
\pgfpathlineto{\pgfqpoint{6.245580in}{0.574707in}}%
\pgfpathlineto{\pgfqpoint{6.254389in}{0.574213in}}%
\pgfpathlineto{\pgfqpoint{6.263198in}{0.573734in}}%
\pgfpathlineto{\pgfqpoint{6.272007in}{0.573269in}}%
\pgfpathlineto{\pgfqpoint{6.280817in}{0.572817in}}%
\pgfpathlineto{\pgfqpoint{6.289626in}{0.572377in}}%
\pgfpathlineto{\pgfqpoint{6.298435in}{0.571950in}}%
\pgfpathlineto{\pgfqpoint{6.307245in}{0.571534in}}%
\pgfpathlineto{\pgfqpoint{6.316054in}{0.571128in}}%
\pgfpathlineto{\pgfqpoint{6.324863in}{0.570733in}}%
\pgfpathlineto{\pgfqpoint{6.333672in}{0.570348in}}%
\pgfpathlineto{\pgfqpoint{6.342482in}{0.569972in}}%
\pgfpathlineto{\pgfqpoint{6.351291in}{0.569605in}}%
\pgfpathlineto{\pgfqpoint{6.360100in}{0.569246in}}%
\pgfpathlineto{\pgfqpoint{6.368909in}{0.568896in}}%
\pgfpathlineto{\pgfqpoint{6.377719in}{0.568554in}}%
\pgfpathlineto{\pgfqpoint{6.386528in}{0.568219in}}%
\pgfpathlineto{\pgfqpoint{6.395337in}{0.567891in}}%
\pgfpathlineto{\pgfqpoint{6.404146in}{0.567571in}}%
\pgfpathlineto{\pgfqpoint{6.412956in}{0.567257in}}%
\pgfpathlineto{\pgfqpoint{6.421765in}{0.566949in}}%
\pgfpathlineto{\pgfqpoint{6.430574in}{0.566648in}}%
\pgfpathlineto{\pgfqpoint{6.439384in}{0.566353in}}%
\pgfpathlineto{\pgfqpoint{6.448193in}{0.566064in}}%
\pgfpathlineto{\pgfqpoint{6.457002in}{0.565780in}}%
\pgfpathlineto{\pgfqpoint{6.465811in}{0.565502in}}%
\pgfpathlineto{\pgfqpoint{6.474621in}{0.565229in}}%
\pgfpathlineto{\pgfqpoint{6.483430in}{0.564961in}}%
\pgfpathlineto{\pgfqpoint{6.492239in}{0.564698in}}%
\pgfpathlineto{\pgfqpoint{6.501048in}{0.564439in}}%
\pgfpathlineto{\pgfqpoint{6.509858in}{0.564185in}}%
\pgfpathlineto{\pgfqpoint{6.518667in}{0.563936in}}%
\pgfpathlineto{\pgfqpoint{6.527476in}{0.563691in}}%
\pgfpathlineto{\pgfqpoint{6.536285in}{0.563450in}}%
\pgfpathlineto{\pgfqpoint{6.545095in}{0.563213in}}%
\pgfpathlineto{\pgfqpoint{6.553904in}{0.562980in}}%
\pgfpathlineto{\pgfqpoint{6.562713in}{0.562751in}}%
\pgfpathlineto{\pgfqpoint{6.571523in}{0.562526in}}%
\pgfpathlineto{\pgfqpoint{6.580332in}{0.562305in}}%
\pgfpathlineto{\pgfqpoint{6.589141in}{0.562086in}}%
\pgfpathlineto{\pgfqpoint{6.597950in}{0.561872in}}%
\pgfpathlineto{\pgfqpoint{6.606760in}{0.561660in}}%
\pgfpathlineto{\pgfqpoint{6.615569in}{0.561452in}}%
\pgfpathlineto{\pgfqpoint{6.624378in}{0.561247in}}%
\pgfpathlineto{\pgfqpoint{6.633187in}{0.561045in}}%
\pgfpathlineto{\pgfqpoint{6.641997in}{0.560846in}}%
\pgfpathlineto{\pgfqpoint{6.650806in}{0.560650in}}%
\pgfpathlineto{\pgfqpoint{6.659615in}{0.560457in}}%
\pgfpathlineto{\pgfqpoint{6.668425in}{0.560267in}}%
\pgfpathlineto{\pgfqpoint{6.677234in}{0.560079in}}%
\pgfpathlineto{\pgfqpoint{6.686043in}{0.559894in}}%
\pgfpathlineto{\pgfqpoint{6.694852in}{0.559712in}}%
\pgfpathlineto{\pgfqpoint{6.703662in}{0.559532in}}%
\pgfpathlineto{\pgfqpoint{6.712471in}{0.559355in}}%
\pgfpathlineto{\pgfqpoint{6.721280in}{0.559180in}}%
\pgfpathlineto{\pgfqpoint{6.730089in}{0.559007in}}%
\pgfpathlineto{\pgfqpoint{6.738899in}{0.558837in}}%
\pgfpathlineto{\pgfqpoint{6.747708in}{0.558669in}}%
\pgfpathlineto{\pgfqpoint{6.756517in}{0.558503in}}%
\pgfpathlineto{\pgfqpoint{6.765326in}{0.558339in}}%
\pgfpathlineto{\pgfqpoint{6.774136in}{0.558177in}}%
\pgfpathlineto{\pgfqpoint{6.782945in}{0.558018in}}%
\pgfpathlineto{\pgfqpoint{6.791754in}{0.557860in}}%
\pgfpathlineto{\pgfqpoint{6.800564in}{0.557705in}}%
\pgfpathlineto{\pgfqpoint{6.809373in}{0.557551in}}%
\pgfpathlineto{\pgfqpoint{6.818182in}{0.557399in}}%
\pgfpathlineto{\pgfqpoint{6.826991in}{0.557249in}}%
\pgfpathlineto{\pgfqpoint{6.835801in}{0.557101in}}%
\pgfpathlineto{\pgfqpoint{6.844610in}{0.556955in}}%
\pgfpathlineto{\pgfqpoint{6.853419in}{0.556811in}}%
\pgfpathlineto{\pgfqpoint{6.862228in}{0.556668in}}%
\pgfpathlineto{\pgfqpoint{6.871038in}{0.556527in}}%
\pgfpathlineto{\pgfqpoint{6.879847in}{0.556387in}}%
\pgfpathlineto{\pgfqpoint{6.888656in}{0.556249in}}%
\pgfpathlineto{\pgfqpoint{6.897465in}{0.556113in}}%
\pgfpathlineto{\pgfqpoint{6.906275in}{0.555978in}}%
\pgfpathlineto{\pgfqpoint{6.915084in}{0.555845in}}%
\pgfpathlineto{\pgfqpoint{6.923893in}{0.555713in}}%
\pgfpathlineto{\pgfqpoint{6.932703in}{0.555583in}}%
\pgfpathlineto{\pgfqpoint{6.941512in}{0.555454in}}%
\pgfpathlineto{\pgfqpoint{6.950321in}{0.555327in}}%
\pgfpathlineto{\pgfqpoint{6.959130in}{0.555201in}}%
\pgfpathlineto{\pgfqpoint{6.967940in}{0.555076in}}%
\pgfpathlineto{\pgfqpoint{6.976749in}{0.554953in}}%
\pgfpathlineto{\pgfqpoint{6.985558in}{0.554831in}}%
\pgfpathlineto{\pgfqpoint{6.994367in}{0.554710in}}%
\pgfpathlineto{\pgfqpoint{7.003177in}{0.554591in}}%
\pgfpathlineto{\pgfqpoint{7.011986in}{0.554472in}}%
\pgfpathlineto{\pgfqpoint{7.020795in}{0.554355in}}%
\pgfpathlineto{\pgfqpoint{7.029605in}{0.554240in}}%
\pgfpathlineto{\pgfqpoint{7.038414in}{0.554125in}}%
\pgfpathlineto{\pgfqpoint{7.047223in}{0.554012in}}%
\pgfpathlineto{\pgfqpoint{7.047223in}{5.084012in}}%
\pgfpathlineto{\pgfqpoint{7.047223in}{5.084012in}}%
\pgfpathlineto{\pgfqpoint{7.038414in}{5.084012in}}%
\pgfpathlineto{\pgfqpoint{7.029605in}{5.084012in}}%
\pgfpathlineto{\pgfqpoint{7.020795in}{5.084012in}}%
\pgfpathlineto{\pgfqpoint{7.011986in}{5.084012in}}%
\pgfpathlineto{\pgfqpoint{7.003177in}{5.084012in}}%
\pgfpathlineto{\pgfqpoint{6.994367in}{5.084012in}}%
\pgfpathlineto{\pgfqpoint{6.985558in}{5.084012in}}%
\pgfpathlineto{\pgfqpoint{6.976749in}{5.084012in}}%
\pgfpathlineto{\pgfqpoint{6.967940in}{5.084012in}}%
\pgfpathlineto{\pgfqpoint{6.959130in}{5.084012in}}%
\pgfpathlineto{\pgfqpoint{6.950321in}{5.084012in}}%
\pgfpathlineto{\pgfqpoint{6.941512in}{5.084012in}}%
\pgfpathlineto{\pgfqpoint{6.932703in}{5.084012in}}%
\pgfpathlineto{\pgfqpoint{6.923893in}{5.084012in}}%
\pgfpathlineto{\pgfqpoint{6.915084in}{5.084012in}}%
\pgfpathlineto{\pgfqpoint{6.906275in}{5.084012in}}%
\pgfpathlineto{\pgfqpoint{6.897465in}{5.084012in}}%
\pgfpathlineto{\pgfqpoint{6.888656in}{5.084012in}}%
\pgfpathlineto{\pgfqpoint{6.879847in}{5.084012in}}%
\pgfpathlineto{\pgfqpoint{6.871038in}{5.084012in}}%
\pgfpathlineto{\pgfqpoint{6.862228in}{5.084012in}}%
\pgfpathlineto{\pgfqpoint{6.853419in}{5.084012in}}%
\pgfpathlineto{\pgfqpoint{6.844610in}{5.084012in}}%
\pgfpathlineto{\pgfqpoint{6.835801in}{5.084012in}}%
\pgfpathlineto{\pgfqpoint{6.826991in}{5.084012in}}%
\pgfpathlineto{\pgfqpoint{6.818182in}{5.084012in}}%
\pgfpathlineto{\pgfqpoint{6.809373in}{5.084012in}}%
\pgfpathlineto{\pgfqpoint{6.800564in}{5.084012in}}%
\pgfpathlineto{\pgfqpoint{6.791754in}{5.084012in}}%
\pgfpathlineto{\pgfqpoint{6.782945in}{5.084012in}}%
\pgfpathlineto{\pgfqpoint{6.774136in}{5.084012in}}%
\pgfpathlineto{\pgfqpoint{6.765326in}{5.084012in}}%
\pgfpathlineto{\pgfqpoint{6.756517in}{5.084012in}}%
\pgfpathlineto{\pgfqpoint{6.747708in}{5.084012in}}%
\pgfpathlineto{\pgfqpoint{6.738899in}{5.084012in}}%
\pgfpathlineto{\pgfqpoint{6.730089in}{5.084012in}}%
\pgfpathlineto{\pgfqpoint{6.721280in}{5.084012in}}%
\pgfpathlineto{\pgfqpoint{6.712471in}{5.084012in}}%
\pgfpathlineto{\pgfqpoint{6.703662in}{5.084012in}}%
\pgfpathlineto{\pgfqpoint{6.694852in}{5.084012in}}%
\pgfpathlineto{\pgfqpoint{6.686043in}{5.084012in}}%
\pgfpathlineto{\pgfqpoint{6.677234in}{5.084012in}}%
\pgfpathlineto{\pgfqpoint{6.668425in}{5.084012in}}%
\pgfpathlineto{\pgfqpoint{6.659615in}{5.084012in}}%
\pgfpathlineto{\pgfqpoint{6.650806in}{5.084012in}}%
\pgfpathlineto{\pgfqpoint{6.641997in}{5.084012in}}%
\pgfpathlineto{\pgfqpoint{6.633187in}{5.084012in}}%
\pgfpathlineto{\pgfqpoint{6.624378in}{5.084012in}}%
\pgfpathlineto{\pgfqpoint{6.615569in}{5.084012in}}%
\pgfpathlineto{\pgfqpoint{6.606760in}{5.084012in}}%
\pgfpathlineto{\pgfqpoint{6.597950in}{5.084012in}}%
\pgfpathlineto{\pgfqpoint{6.589141in}{5.084012in}}%
\pgfpathlineto{\pgfqpoint{6.580332in}{5.084012in}}%
\pgfpathlineto{\pgfqpoint{6.571523in}{5.084012in}}%
\pgfpathlineto{\pgfqpoint{6.562713in}{5.084012in}}%
\pgfpathlineto{\pgfqpoint{6.553904in}{5.084012in}}%
\pgfpathlineto{\pgfqpoint{6.545095in}{5.084012in}}%
\pgfpathlineto{\pgfqpoint{6.536285in}{5.084012in}}%
\pgfpathlineto{\pgfqpoint{6.527476in}{5.084012in}}%
\pgfpathlineto{\pgfqpoint{6.518667in}{5.084012in}}%
\pgfpathlineto{\pgfqpoint{6.509858in}{5.084012in}}%
\pgfpathlineto{\pgfqpoint{6.501048in}{5.084012in}}%
\pgfpathlineto{\pgfqpoint{6.492239in}{5.084012in}}%
\pgfpathlineto{\pgfqpoint{6.483430in}{5.084012in}}%
\pgfpathlineto{\pgfqpoint{6.474621in}{5.084012in}}%
\pgfpathlineto{\pgfqpoint{6.465811in}{5.084012in}}%
\pgfpathlineto{\pgfqpoint{6.457002in}{5.084012in}}%
\pgfpathlineto{\pgfqpoint{6.448193in}{5.084012in}}%
\pgfpathlineto{\pgfqpoint{6.439384in}{5.084012in}}%
\pgfpathlineto{\pgfqpoint{6.430574in}{5.084012in}}%
\pgfpathlineto{\pgfqpoint{6.421765in}{5.084012in}}%
\pgfpathlineto{\pgfqpoint{6.412956in}{5.084012in}}%
\pgfpathlineto{\pgfqpoint{6.404146in}{5.084012in}}%
\pgfpathlineto{\pgfqpoint{6.395337in}{5.084012in}}%
\pgfpathlineto{\pgfqpoint{6.386528in}{5.084012in}}%
\pgfpathlineto{\pgfqpoint{6.377719in}{5.084012in}}%
\pgfpathlineto{\pgfqpoint{6.368909in}{5.084012in}}%
\pgfpathlineto{\pgfqpoint{6.360100in}{5.084012in}}%
\pgfpathlineto{\pgfqpoint{6.351291in}{5.084012in}}%
\pgfpathlineto{\pgfqpoint{6.342482in}{5.084012in}}%
\pgfpathlineto{\pgfqpoint{6.333672in}{5.084012in}}%
\pgfpathlineto{\pgfqpoint{6.324863in}{5.084012in}}%
\pgfpathlineto{\pgfqpoint{6.316054in}{5.084012in}}%
\pgfpathlineto{\pgfqpoint{6.307245in}{5.084012in}}%
\pgfpathlineto{\pgfqpoint{6.298435in}{5.084012in}}%
\pgfpathlineto{\pgfqpoint{6.289626in}{5.084012in}}%
\pgfpathlineto{\pgfqpoint{6.280817in}{5.084012in}}%
\pgfpathlineto{\pgfqpoint{6.272007in}{5.084012in}}%
\pgfpathlineto{\pgfqpoint{6.263198in}{5.084012in}}%
\pgfpathlineto{\pgfqpoint{6.254389in}{5.084012in}}%
\pgfpathlineto{\pgfqpoint{6.245580in}{5.084012in}}%
\pgfpathlineto{\pgfqpoint{6.236770in}{5.084012in}}%
\pgfpathlineto{\pgfqpoint{6.227961in}{5.084012in}}%
\pgfpathlineto{\pgfqpoint{6.219152in}{5.084012in}}%
\pgfpathlineto{\pgfqpoint{6.210343in}{5.084012in}}%
\pgfpathlineto{\pgfqpoint{6.201533in}{5.084012in}}%
\pgfpathlineto{\pgfqpoint{6.192724in}{5.084012in}}%
\pgfpathlineto{\pgfqpoint{6.183915in}{5.084012in}}%
\pgfpathlineto{\pgfqpoint{6.175105in}{5.084012in}}%
\pgfpathlineto{\pgfqpoint{6.175105in}{5.084012in}}%
\pgfpathlineto{\pgfqpoint{6.169772in}{5.084012in}}%
\pgfpathlineto{\pgfqpoint{6.164439in}{5.084012in}}%
\pgfpathlineto{\pgfqpoint{6.159106in}{5.084012in}}%
\pgfpathlineto{\pgfqpoint{6.153773in}{5.084012in}}%
\pgfpathlineto{\pgfqpoint{6.148439in}{5.084012in}}%
\pgfpathlineto{\pgfqpoint{6.143106in}{5.084012in}}%
\pgfpathlineto{\pgfqpoint{6.137773in}{5.084012in}}%
\pgfpathlineto{\pgfqpoint{6.132440in}{5.084012in}}%
\pgfpathlineto{\pgfqpoint{6.127107in}{5.084012in}}%
\pgfpathlineto{\pgfqpoint{6.121773in}{5.084012in}}%
\pgfpathlineto{\pgfqpoint{6.116440in}{5.084012in}}%
\pgfpathlineto{\pgfqpoint{6.111107in}{5.084012in}}%
\pgfpathlineto{\pgfqpoint{6.105774in}{5.084012in}}%
\pgfpathlineto{\pgfqpoint{6.100440in}{5.084012in}}%
\pgfpathlineto{\pgfqpoint{6.095107in}{5.084012in}}%
\pgfpathlineto{\pgfqpoint{6.089774in}{5.084012in}}%
\pgfpathlineto{\pgfqpoint{6.084441in}{5.084012in}}%
\pgfpathlineto{\pgfqpoint{6.079108in}{5.084012in}}%
\pgfpathlineto{\pgfqpoint{6.073774in}{5.084012in}}%
\pgfpathlineto{\pgfqpoint{6.068441in}{5.084012in}}%
\pgfpathlineto{\pgfqpoint{6.063108in}{5.084012in}}%
\pgfpathlineto{\pgfqpoint{6.057775in}{5.084012in}}%
\pgfpathlineto{\pgfqpoint{6.052442in}{5.084012in}}%
\pgfpathlineto{\pgfqpoint{6.047108in}{5.084012in}}%
\pgfpathlineto{\pgfqpoint{6.041775in}{5.084012in}}%
\pgfpathlineto{\pgfqpoint{6.036442in}{5.084012in}}%
\pgfpathlineto{\pgfqpoint{6.031109in}{5.084012in}}%
\pgfpathlineto{\pgfqpoint{6.025775in}{5.084012in}}%
\pgfpathlineto{\pgfqpoint{6.020442in}{5.084012in}}%
\pgfpathlineto{\pgfqpoint{6.015109in}{5.084012in}}%
\pgfpathlineto{\pgfqpoint{6.009776in}{5.084012in}}%
\pgfpathlineto{\pgfqpoint{6.004443in}{5.084012in}}%
\pgfpathlineto{\pgfqpoint{5.999109in}{5.084012in}}%
\pgfpathlineto{\pgfqpoint{5.993776in}{5.084012in}}%
\pgfpathlineto{\pgfqpoint{5.988443in}{5.084012in}}%
\pgfpathlineto{\pgfqpoint{5.983110in}{5.084012in}}%
\pgfpathlineto{\pgfqpoint{5.977777in}{5.084012in}}%
\pgfpathlineto{\pgfqpoint{5.972443in}{5.084012in}}%
\pgfpathlineto{\pgfqpoint{5.967110in}{5.084012in}}%
\pgfpathlineto{\pgfqpoint{5.961777in}{5.084012in}}%
\pgfpathlineto{\pgfqpoint{5.956444in}{5.084012in}}%
\pgfpathlineto{\pgfqpoint{5.951110in}{5.084012in}}%
\pgfpathlineto{\pgfqpoint{5.945777in}{5.084012in}}%
\pgfpathlineto{\pgfqpoint{5.940444in}{5.084012in}}%
\pgfpathlineto{\pgfqpoint{5.935111in}{5.084012in}}%
\pgfpathlineto{\pgfqpoint{5.929778in}{5.084012in}}%
\pgfpathlineto{\pgfqpoint{5.924444in}{5.084012in}}%
\pgfpathlineto{\pgfqpoint{5.919111in}{5.084012in}}%
\pgfpathlineto{\pgfqpoint{5.913778in}{5.084012in}}%
\pgfpathlineto{\pgfqpoint{5.908445in}{5.084012in}}%
\pgfpathlineto{\pgfqpoint{5.903112in}{5.084012in}}%
\pgfpathlineto{\pgfqpoint{5.897778in}{5.084012in}}%
\pgfpathlineto{\pgfqpoint{5.892445in}{5.084012in}}%
\pgfpathlineto{\pgfqpoint{5.887112in}{5.084012in}}%
\pgfpathlineto{\pgfqpoint{5.881779in}{5.084012in}}%
\pgfpathlineto{\pgfqpoint{5.876445in}{5.084012in}}%
\pgfpathlineto{\pgfqpoint{5.871112in}{5.084012in}}%
\pgfpathlineto{\pgfqpoint{5.865779in}{5.084012in}}%
\pgfpathlineto{\pgfqpoint{5.860446in}{5.084012in}}%
\pgfpathlineto{\pgfqpoint{5.855113in}{5.084012in}}%
\pgfpathlineto{\pgfqpoint{5.849779in}{5.084012in}}%
\pgfpathlineto{\pgfqpoint{5.844446in}{5.084012in}}%
\pgfpathlineto{\pgfqpoint{5.839113in}{5.084012in}}%
\pgfpathlineto{\pgfqpoint{5.833780in}{5.084012in}}%
\pgfpathlineto{\pgfqpoint{5.828446in}{5.084012in}}%
\pgfpathlineto{\pgfqpoint{5.823113in}{5.084012in}}%
\pgfpathlineto{\pgfqpoint{5.817780in}{5.084012in}}%
\pgfpathlineto{\pgfqpoint{5.812447in}{5.084012in}}%
\pgfpathlineto{\pgfqpoint{5.807114in}{5.084012in}}%
\pgfpathlineto{\pgfqpoint{5.801780in}{5.084012in}}%
\pgfpathlineto{\pgfqpoint{5.796447in}{5.084012in}}%
\pgfpathlineto{\pgfqpoint{5.791114in}{5.084012in}}%
\pgfpathlineto{\pgfqpoint{5.785781in}{5.084012in}}%
\pgfpathlineto{\pgfqpoint{5.780448in}{5.084012in}}%
\pgfpathlineto{\pgfqpoint{5.775114in}{5.084012in}}%
\pgfpathlineto{\pgfqpoint{5.769781in}{5.084012in}}%
\pgfpathlineto{\pgfqpoint{5.764448in}{5.084012in}}%
\pgfpathlineto{\pgfqpoint{5.759115in}{5.084012in}}%
\pgfpathlineto{\pgfqpoint{5.753781in}{5.084012in}}%
\pgfpathlineto{\pgfqpoint{5.748448in}{5.084012in}}%
\pgfpathlineto{\pgfqpoint{5.743115in}{5.084012in}}%
\pgfpathlineto{\pgfqpoint{5.737782in}{5.084012in}}%
\pgfpathlineto{\pgfqpoint{5.732449in}{5.084012in}}%
\pgfpathlineto{\pgfqpoint{5.727115in}{5.084012in}}%
\pgfpathlineto{\pgfqpoint{5.721782in}{5.084012in}}%
\pgfpathlineto{\pgfqpoint{5.716449in}{5.084012in}}%
\pgfpathlineto{\pgfqpoint{5.711116in}{5.084012in}}%
\pgfpathlineto{\pgfqpoint{5.705783in}{5.084012in}}%
\pgfpathlineto{\pgfqpoint{5.700449in}{5.084012in}}%
\pgfpathlineto{\pgfqpoint{5.695116in}{5.084012in}}%
\pgfpathlineto{\pgfqpoint{5.689783in}{5.084012in}}%
\pgfpathlineto{\pgfqpoint{5.684450in}{5.084012in}}%
\pgfpathlineto{\pgfqpoint{5.679116in}{5.084012in}}%
\pgfpathlineto{\pgfqpoint{5.673783in}{5.084012in}}%
\pgfpathlineto{\pgfqpoint{5.668450in}{5.084012in}}%
\pgfpathlineto{\pgfqpoint{5.663117in}{5.084012in}}%
\pgfpathlineto{\pgfqpoint{5.657784in}{5.084012in}}%
\pgfpathlineto{\pgfqpoint{5.652450in}{5.084012in}}%
\pgfpathlineto{\pgfqpoint{5.647117in}{5.084012in}}%
\pgfpathlineto{\pgfqpoint{5.641784in}{5.084012in}}%
\pgfpathlineto{\pgfqpoint{5.636451in}{5.084012in}}%
\pgfpathlineto{\pgfqpoint{5.631118in}{5.084012in}}%
\pgfpathlineto{\pgfqpoint{5.625784in}{5.084012in}}%
\pgfpathlineto{\pgfqpoint{5.620451in}{5.084012in}}%
\pgfpathlineto{\pgfqpoint{5.615118in}{5.084012in}}%
\pgfpathlineto{\pgfqpoint{5.609785in}{5.084012in}}%
\pgfpathlineto{\pgfqpoint{5.604451in}{5.084012in}}%
\pgfpathlineto{\pgfqpoint{5.599118in}{5.084012in}}%
\pgfpathlineto{\pgfqpoint{5.593785in}{5.084012in}}%
\pgfpathlineto{\pgfqpoint{5.588452in}{5.084012in}}%
\pgfpathlineto{\pgfqpoint{5.583119in}{5.084012in}}%
\pgfpathlineto{\pgfqpoint{5.577785in}{5.084012in}}%
\pgfpathlineto{\pgfqpoint{5.572452in}{5.084012in}}%
\pgfpathlineto{\pgfqpoint{5.567119in}{5.084012in}}%
\pgfpathlineto{\pgfqpoint{5.561786in}{5.084012in}}%
\pgfpathlineto{\pgfqpoint{5.556452in}{5.084012in}}%
\pgfpathlineto{\pgfqpoint{5.551119in}{5.084012in}}%
\pgfpathlineto{\pgfqpoint{5.545786in}{5.084012in}}%
\pgfpathlineto{\pgfqpoint{5.540453in}{5.084012in}}%
\pgfpathlineto{\pgfqpoint{5.535120in}{5.084012in}}%
\pgfpathlineto{\pgfqpoint{5.529786in}{5.084012in}}%
\pgfpathlineto{\pgfqpoint{5.524453in}{5.084012in}}%
\pgfpathlineto{\pgfqpoint{5.519120in}{5.084012in}}%
\pgfpathlineto{\pgfqpoint{5.513787in}{5.084012in}}%
\pgfpathlineto{\pgfqpoint{5.508454in}{5.084012in}}%
\pgfpathlineto{\pgfqpoint{5.503120in}{5.084012in}}%
\pgfpathlineto{\pgfqpoint{5.497787in}{5.084012in}}%
\pgfpathlineto{\pgfqpoint{5.492454in}{5.084012in}}%
\pgfpathlineto{\pgfqpoint{5.487121in}{5.084012in}}%
\pgfpathlineto{\pgfqpoint{5.481787in}{5.084012in}}%
\pgfpathlineto{\pgfqpoint{5.476454in}{5.084012in}}%
\pgfpathlineto{\pgfqpoint{5.471121in}{5.084012in}}%
\pgfpathlineto{\pgfqpoint{5.465788in}{5.084012in}}%
\pgfpathlineto{\pgfqpoint{5.460455in}{5.084012in}}%
\pgfpathlineto{\pgfqpoint{5.455121in}{5.084012in}}%
\pgfpathlineto{\pgfqpoint{5.449788in}{5.084012in}}%
\pgfpathlineto{\pgfqpoint{5.444455in}{5.084012in}}%
\pgfpathlineto{\pgfqpoint{5.439122in}{5.084012in}}%
\pgfpathlineto{\pgfqpoint{5.433789in}{5.084012in}}%
\pgfpathlineto{\pgfqpoint{5.428455in}{5.084012in}}%
\pgfpathlineto{\pgfqpoint{5.423122in}{5.084012in}}%
\pgfpathlineto{\pgfqpoint{5.417789in}{5.084012in}}%
\pgfpathlineto{\pgfqpoint{5.412456in}{5.084012in}}%
\pgfpathlineto{\pgfqpoint{5.407122in}{5.084012in}}%
\pgfpathlineto{\pgfqpoint{5.401789in}{5.084012in}}%
\pgfpathlineto{\pgfqpoint{5.396456in}{5.084012in}}%
\pgfpathlineto{\pgfqpoint{5.391123in}{5.084012in}}%
\pgfpathlineto{\pgfqpoint{5.385790in}{5.084012in}}%
\pgfpathlineto{\pgfqpoint{5.380456in}{5.084012in}}%
\pgfpathlineto{\pgfqpoint{5.375123in}{5.084012in}}%
\pgfpathlineto{\pgfqpoint{5.369790in}{5.084012in}}%
\pgfpathlineto{\pgfqpoint{5.364457in}{5.084012in}}%
\pgfpathlineto{\pgfqpoint{5.359124in}{5.084012in}}%
\pgfpathlineto{\pgfqpoint{5.353790in}{5.084012in}}%
\pgfpathlineto{\pgfqpoint{5.348457in}{5.084012in}}%
\pgfpathlineto{\pgfqpoint{5.343124in}{5.084012in}}%
\pgfpathlineto{\pgfqpoint{5.337791in}{5.084012in}}%
\pgfpathlineto{\pgfqpoint{5.332457in}{5.084012in}}%
\pgfpathlineto{\pgfqpoint{5.327124in}{5.084012in}}%
\pgfpathlineto{\pgfqpoint{5.321791in}{5.084012in}}%
\pgfpathlineto{\pgfqpoint{5.316458in}{5.084012in}}%
\pgfpathlineto{\pgfqpoint{5.311125in}{5.084012in}}%
\pgfpathlineto{\pgfqpoint{5.305791in}{5.084012in}}%
\pgfpathlineto{\pgfqpoint{5.300458in}{5.084012in}}%
\pgfpathlineto{\pgfqpoint{5.295125in}{5.084012in}}%
\pgfpathlineto{\pgfqpoint{5.289792in}{5.084012in}}%
\pgfpathlineto{\pgfqpoint{5.284458in}{5.084012in}}%
\pgfpathlineto{\pgfqpoint{5.279125in}{5.084012in}}%
\pgfpathlineto{\pgfqpoint{5.273792in}{5.084012in}}%
\pgfpathlineto{\pgfqpoint{5.268459in}{5.084012in}}%
\pgfpathlineto{\pgfqpoint{5.263126in}{5.084012in}}%
\pgfpathlineto{\pgfqpoint{5.257792in}{5.084012in}}%
\pgfpathlineto{\pgfqpoint{5.252459in}{5.084012in}}%
\pgfpathlineto{\pgfqpoint{5.247126in}{5.084012in}}%
\pgfpathlineto{\pgfqpoint{5.241793in}{5.084012in}}%
\pgfpathlineto{\pgfqpoint{5.236460in}{5.084012in}}%
\pgfpathlineto{\pgfqpoint{5.231126in}{5.084012in}}%
\pgfpathlineto{\pgfqpoint{5.225793in}{5.084012in}}%
\pgfpathlineto{\pgfqpoint{5.220460in}{5.084012in}}%
\pgfpathlineto{\pgfqpoint{5.215127in}{5.084012in}}%
\pgfpathlineto{\pgfqpoint{5.209793in}{5.084012in}}%
\pgfpathlineto{\pgfqpoint{5.204460in}{5.084012in}}%
\pgfpathlineto{\pgfqpoint{5.199127in}{5.084012in}}%
\pgfpathlineto{\pgfqpoint{5.193794in}{5.084012in}}%
\pgfpathlineto{\pgfqpoint{5.188461in}{5.084012in}}%
\pgfpathlineto{\pgfqpoint{5.183127in}{5.084012in}}%
\pgfpathlineto{\pgfqpoint{5.177794in}{5.084012in}}%
\pgfpathlineto{\pgfqpoint{5.172461in}{5.084012in}}%
\pgfpathlineto{\pgfqpoint{5.167128in}{5.084012in}}%
\pgfpathlineto{\pgfqpoint{5.161795in}{5.084012in}}%
\pgfpathlineto{\pgfqpoint{5.156461in}{5.084012in}}%
\pgfpathlineto{\pgfqpoint{5.151128in}{5.084012in}}%
\pgfpathlineto{\pgfqpoint{5.145795in}{5.084012in}}%
\pgfpathlineto{\pgfqpoint{5.140462in}{5.084012in}}%
\pgfpathlineto{\pgfqpoint{5.135128in}{5.084012in}}%
\pgfpathlineto{\pgfqpoint{5.129795in}{5.084012in}}%
\pgfpathlineto{\pgfqpoint{5.124462in}{5.084012in}}%
\pgfpathlineto{\pgfqpoint{5.119129in}{5.084012in}}%
\pgfpathlineto{\pgfqpoint{5.113796in}{5.084012in}}%
\pgfpathlineto{\pgfqpoint{5.108462in}{5.084012in}}%
\pgfpathlineto{\pgfqpoint{5.103129in}{5.084012in}}%
\pgfpathlineto{\pgfqpoint{5.097796in}{5.084012in}}%
\pgfpathlineto{\pgfqpoint{5.092463in}{5.084012in}}%
\pgfpathlineto{\pgfqpoint{5.087130in}{5.084012in}}%
\pgfpathlineto{\pgfqpoint{5.081796in}{5.084012in}}%
\pgfpathlineto{\pgfqpoint{5.076463in}{5.084012in}}%
\pgfpathlineto{\pgfqpoint{5.071130in}{5.084012in}}%
\pgfpathlineto{\pgfqpoint{5.065797in}{5.084012in}}%
\pgfpathlineto{\pgfqpoint{5.060463in}{5.084012in}}%
\pgfpathlineto{\pgfqpoint{5.055130in}{5.084012in}}%
\pgfpathlineto{\pgfqpoint{5.049797in}{5.084012in}}%
\pgfpathlineto{\pgfqpoint{5.044464in}{5.084012in}}%
\pgfpathlineto{\pgfqpoint{5.039131in}{5.084012in}}%
\pgfpathlineto{\pgfqpoint{5.033797in}{5.084012in}}%
\pgfpathlineto{\pgfqpoint{5.028464in}{5.084012in}}%
\pgfpathlineto{\pgfqpoint{5.023131in}{5.084012in}}%
\pgfpathlineto{\pgfqpoint{5.017798in}{5.084012in}}%
\pgfpathlineto{\pgfqpoint{5.012464in}{5.084012in}}%
\pgfpathlineto{\pgfqpoint{5.007131in}{5.084012in}}%
\pgfpathlineto{\pgfqpoint{5.001798in}{5.084012in}}%
\pgfpathlineto{\pgfqpoint{4.996465in}{5.084012in}}%
\pgfpathlineto{\pgfqpoint{4.991132in}{5.084012in}}%
\pgfpathlineto{\pgfqpoint{4.985798in}{5.084012in}}%
\pgfpathlineto{\pgfqpoint{4.980465in}{5.084012in}}%
\pgfpathlineto{\pgfqpoint{4.975132in}{5.084012in}}%
\pgfpathlineto{\pgfqpoint{4.969799in}{5.084012in}}%
\pgfpathlineto{\pgfqpoint{4.964466in}{5.084012in}}%
\pgfpathlineto{\pgfqpoint{4.959132in}{5.084012in}}%
\pgfpathlineto{\pgfqpoint{4.953799in}{5.084012in}}%
\pgfpathlineto{\pgfqpoint{4.948466in}{5.084012in}}%
\pgfpathlineto{\pgfqpoint{4.943133in}{5.084012in}}%
\pgfpathlineto{\pgfqpoint{4.937799in}{5.084012in}}%
\pgfpathlineto{\pgfqpoint{4.932466in}{5.084012in}}%
\pgfpathlineto{\pgfqpoint{4.927133in}{5.084012in}}%
\pgfpathlineto{\pgfqpoint{4.921800in}{5.084012in}}%
\pgfpathlineto{\pgfqpoint{4.916467in}{5.084012in}}%
\pgfpathlineto{\pgfqpoint{4.911133in}{5.084012in}}%
\pgfpathlineto{\pgfqpoint{4.905800in}{5.084012in}}%
\pgfpathlineto{\pgfqpoint{4.900467in}{5.084012in}}%
\pgfpathlineto{\pgfqpoint{4.895134in}{5.084012in}}%
\pgfpathlineto{\pgfqpoint{4.889801in}{5.084012in}}%
\pgfpathlineto{\pgfqpoint{4.884467in}{5.084012in}}%
\pgfpathlineto{\pgfqpoint{4.879134in}{5.084012in}}%
\pgfpathlineto{\pgfqpoint{4.873801in}{5.084012in}}%
\pgfpathlineto{\pgfqpoint{4.868468in}{5.084012in}}%
\pgfpathlineto{\pgfqpoint{4.863134in}{5.084012in}}%
\pgfpathlineto{\pgfqpoint{4.857801in}{5.084012in}}%
\pgfpathlineto{\pgfqpoint{4.852468in}{5.084012in}}%
\pgfpathlineto{\pgfqpoint{4.847135in}{5.084012in}}%
\pgfpathlineto{\pgfqpoint{4.841802in}{5.084012in}}%
\pgfpathlineto{\pgfqpoint{4.836468in}{5.084012in}}%
\pgfpathlineto{\pgfqpoint{4.831135in}{5.084012in}}%
\pgfpathlineto{\pgfqpoint{4.825802in}{5.084012in}}%
\pgfpathlineto{\pgfqpoint{4.820469in}{5.084012in}}%
\pgfpathlineto{\pgfqpoint{4.815136in}{5.084012in}}%
\pgfpathlineto{\pgfqpoint{4.809802in}{5.084012in}}%
\pgfpathlineto{\pgfqpoint{4.804469in}{5.084012in}}%
\pgfpathlineto{\pgfqpoint{4.799136in}{5.084012in}}%
\pgfpathlineto{\pgfqpoint{4.793803in}{5.084012in}}%
\pgfpathlineto{\pgfqpoint{4.788469in}{5.084012in}}%
\pgfpathlineto{\pgfqpoint{4.783136in}{5.084012in}}%
\pgfpathlineto{\pgfqpoint{4.777803in}{5.084012in}}%
\pgfpathlineto{\pgfqpoint{4.772470in}{5.084012in}}%
\pgfpathlineto{\pgfqpoint{4.767137in}{5.084012in}}%
\pgfpathlineto{\pgfqpoint{4.761803in}{5.084012in}}%
\pgfpathlineto{\pgfqpoint{4.756470in}{5.084012in}}%
\pgfpathlineto{\pgfqpoint{4.751137in}{5.084012in}}%
\pgfpathlineto{\pgfqpoint{4.745804in}{5.084012in}}%
\pgfpathlineto{\pgfqpoint{4.740470in}{5.084012in}}%
\pgfpathlineto{\pgfqpoint{4.735137in}{5.084012in}}%
\pgfpathlineto{\pgfqpoint{4.729804in}{5.084012in}}%
\pgfpathlineto{\pgfqpoint{4.724471in}{5.084012in}}%
\pgfpathlineto{\pgfqpoint{4.719138in}{5.084012in}}%
\pgfpathlineto{\pgfqpoint{4.713804in}{5.084012in}}%
\pgfpathlineto{\pgfqpoint{4.708471in}{5.084012in}}%
\pgfpathlineto{\pgfqpoint{4.703138in}{5.084012in}}%
\pgfpathlineto{\pgfqpoint{4.697805in}{5.084012in}}%
\pgfpathlineto{\pgfqpoint{4.692472in}{5.084012in}}%
\pgfpathlineto{\pgfqpoint{4.687138in}{5.084012in}}%
\pgfpathlineto{\pgfqpoint{4.681805in}{5.084012in}}%
\pgfpathlineto{\pgfqpoint{4.676472in}{5.084012in}}%
\pgfpathlineto{\pgfqpoint{4.671139in}{5.084012in}}%
\pgfpathlineto{\pgfqpoint{4.665805in}{5.084012in}}%
\pgfpathlineto{\pgfqpoint{4.660472in}{5.084012in}}%
\pgfpathlineto{\pgfqpoint{4.655139in}{5.084012in}}%
\pgfpathlineto{\pgfqpoint{4.649806in}{5.084012in}}%
\pgfpathlineto{\pgfqpoint{4.644473in}{5.084012in}}%
\pgfpathlineto{\pgfqpoint{4.639139in}{5.084012in}}%
\pgfpathlineto{\pgfqpoint{4.633806in}{5.084012in}}%
\pgfpathlineto{\pgfqpoint{4.628473in}{5.084012in}}%
\pgfpathlineto{\pgfqpoint{4.623140in}{5.084012in}}%
\pgfpathlineto{\pgfqpoint{4.617807in}{5.084012in}}%
\pgfpathlineto{\pgfqpoint{4.612473in}{5.084012in}}%
\pgfpathlineto{\pgfqpoint{4.607140in}{5.084012in}}%
\pgfpathlineto{\pgfqpoint{4.601807in}{5.084012in}}%
\pgfpathlineto{\pgfqpoint{4.596474in}{5.084012in}}%
\pgfpathlineto{\pgfqpoint{4.591140in}{5.084012in}}%
\pgfpathlineto{\pgfqpoint{4.585807in}{5.084012in}}%
\pgfpathlineto{\pgfqpoint{4.580474in}{5.084012in}}%
\pgfpathlineto{\pgfqpoint{4.575141in}{5.084012in}}%
\pgfpathlineto{\pgfqpoint{4.569808in}{5.084012in}}%
\pgfpathlineto{\pgfqpoint{4.564474in}{5.084012in}}%
\pgfpathlineto{\pgfqpoint{4.559141in}{5.084012in}}%
\pgfpathlineto{\pgfqpoint{4.553808in}{5.084012in}}%
\pgfpathlineto{\pgfqpoint{4.548475in}{5.084012in}}%
\pgfpathlineto{\pgfqpoint{4.543142in}{5.084012in}}%
\pgfpathlineto{\pgfqpoint{4.537808in}{5.084012in}}%
\pgfpathlineto{\pgfqpoint{4.532475in}{5.084012in}}%
\pgfpathlineto{\pgfqpoint{4.527142in}{5.084012in}}%
\pgfpathlineto{\pgfqpoint{4.521809in}{5.084012in}}%
\pgfpathlineto{\pgfqpoint{4.516475in}{5.084012in}}%
\pgfpathlineto{\pgfqpoint{4.511142in}{5.084012in}}%
\pgfpathlineto{\pgfqpoint{4.505809in}{5.084012in}}%
\pgfpathlineto{\pgfqpoint{4.500476in}{5.084012in}}%
\pgfpathlineto{\pgfqpoint{4.495143in}{5.084012in}}%
\pgfpathlineto{\pgfqpoint{4.489809in}{5.084012in}}%
\pgfpathlineto{\pgfqpoint{4.484476in}{5.084012in}}%
\pgfpathlineto{\pgfqpoint{4.479143in}{5.084012in}}%
\pgfpathlineto{\pgfqpoint{4.473810in}{5.084012in}}%
\pgfpathlineto{\pgfqpoint{4.468476in}{5.084012in}}%
\pgfpathlineto{\pgfqpoint{4.463143in}{5.084012in}}%
\pgfpathlineto{\pgfqpoint{4.457810in}{5.084012in}}%
\pgfpathlineto{\pgfqpoint{4.452477in}{5.084012in}}%
\pgfpathlineto{\pgfqpoint{4.447144in}{5.084012in}}%
\pgfpathlineto{\pgfqpoint{4.441810in}{5.084012in}}%
\pgfpathlineto{\pgfqpoint{4.436477in}{5.084012in}}%
\pgfpathlineto{\pgfqpoint{4.431144in}{5.084012in}}%
\pgfpathlineto{\pgfqpoint{4.425811in}{5.084012in}}%
\pgfpathlineto{\pgfqpoint{4.420478in}{5.084012in}}%
\pgfpathlineto{\pgfqpoint{4.415144in}{5.084012in}}%
\pgfpathlineto{\pgfqpoint{4.409811in}{5.084012in}}%
\pgfpathlineto{\pgfqpoint{4.404478in}{5.084012in}}%
\pgfpathlineto{\pgfqpoint{4.399145in}{5.084012in}}%
\pgfpathlineto{\pgfqpoint{4.393811in}{5.084012in}}%
\pgfpathlineto{\pgfqpoint{4.388478in}{5.084012in}}%
\pgfpathlineto{\pgfqpoint{4.383145in}{5.084012in}}%
\pgfpathlineto{\pgfqpoint{4.377812in}{5.084012in}}%
\pgfpathlineto{\pgfqpoint{4.372479in}{5.084012in}}%
\pgfpathlineto{\pgfqpoint{4.367145in}{5.084012in}}%
\pgfpathlineto{\pgfqpoint{4.361812in}{5.084012in}}%
\pgfpathlineto{\pgfqpoint{4.356479in}{5.084012in}}%
\pgfpathlineto{\pgfqpoint{4.351146in}{5.084012in}}%
\pgfpathlineto{\pgfqpoint{4.345813in}{5.084012in}}%
\pgfpathlineto{\pgfqpoint{4.340479in}{5.084012in}}%
\pgfpathlineto{\pgfqpoint{4.335146in}{5.084012in}}%
\pgfpathlineto{\pgfqpoint{4.329813in}{5.084012in}}%
\pgfpathlineto{\pgfqpoint{4.324480in}{5.084012in}}%
\pgfpathlineto{\pgfqpoint{4.319146in}{5.084012in}}%
\pgfpathlineto{\pgfqpoint{4.313813in}{5.084012in}}%
\pgfpathlineto{\pgfqpoint{4.308480in}{5.084012in}}%
\pgfpathlineto{\pgfqpoint{4.303147in}{5.084012in}}%
\pgfpathlineto{\pgfqpoint{4.297814in}{5.084012in}}%
\pgfpathlineto{\pgfqpoint{4.292480in}{5.084012in}}%
\pgfpathlineto{\pgfqpoint{4.287147in}{5.084012in}}%
\pgfpathlineto{\pgfqpoint{4.281814in}{5.084012in}}%
\pgfpathlineto{\pgfqpoint{4.276481in}{5.084012in}}%
\pgfpathlineto{\pgfqpoint{4.271148in}{5.084012in}}%
\pgfpathlineto{\pgfqpoint{4.265814in}{5.084012in}}%
\pgfpathlineto{\pgfqpoint{4.260481in}{5.084012in}}%
\pgfpathlineto{\pgfqpoint{4.255148in}{5.084012in}}%
\pgfpathlineto{\pgfqpoint{4.249815in}{5.084012in}}%
\pgfpathlineto{\pgfqpoint{4.244481in}{5.084012in}}%
\pgfpathlineto{\pgfqpoint{4.239148in}{5.084012in}}%
\pgfpathlineto{\pgfqpoint{4.233815in}{5.084012in}}%
\pgfpathlineto{\pgfqpoint{4.228482in}{5.084012in}}%
\pgfpathlineto{\pgfqpoint{4.223149in}{5.084012in}}%
\pgfpathlineto{\pgfqpoint{4.217815in}{5.084012in}}%
\pgfpathlineto{\pgfqpoint{4.212482in}{5.084012in}}%
\pgfpathlineto{\pgfqpoint{4.207149in}{5.084012in}}%
\pgfpathlineto{\pgfqpoint{4.201816in}{5.084012in}}%
\pgfpathlineto{\pgfqpoint{4.196482in}{5.084012in}}%
\pgfpathlineto{\pgfqpoint{4.191149in}{5.084012in}}%
\pgfpathlineto{\pgfqpoint{4.185816in}{5.084012in}}%
\pgfpathlineto{\pgfqpoint{4.180483in}{5.084012in}}%
\pgfpathlineto{\pgfqpoint{4.175150in}{5.084012in}}%
\pgfpathlineto{\pgfqpoint{4.169816in}{5.084012in}}%
\pgfpathlineto{\pgfqpoint{4.164483in}{5.084012in}}%
\pgfpathlineto{\pgfqpoint{4.159150in}{5.084012in}}%
\pgfpathlineto{\pgfqpoint{4.153817in}{5.084012in}}%
\pgfpathlineto{\pgfqpoint{4.148484in}{5.084012in}}%
\pgfpathlineto{\pgfqpoint{4.143150in}{5.084012in}}%
\pgfpathlineto{\pgfqpoint{4.137817in}{5.084012in}}%
\pgfpathlineto{\pgfqpoint{4.132484in}{5.084012in}}%
\pgfpathlineto{\pgfqpoint{4.127151in}{5.084012in}}%
\pgfpathlineto{\pgfqpoint{4.121817in}{5.084012in}}%
\pgfpathlineto{\pgfqpoint{4.116484in}{5.084012in}}%
\pgfpathlineto{\pgfqpoint{4.111151in}{5.084012in}}%
\pgfpathlineto{\pgfqpoint{4.105818in}{5.084012in}}%
\pgfpathlineto{\pgfqpoint{4.100485in}{5.084012in}}%
\pgfpathlineto{\pgfqpoint{4.095151in}{5.084012in}}%
\pgfpathlineto{\pgfqpoint{4.089818in}{5.084012in}}%
\pgfpathlineto{\pgfqpoint{4.084485in}{5.084012in}}%
\pgfpathlineto{\pgfqpoint{4.079152in}{5.084012in}}%
\pgfpathlineto{\pgfqpoint{4.073819in}{5.084012in}}%
\pgfpathlineto{\pgfqpoint{4.068485in}{5.084012in}}%
\pgfpathlineto{\pgfqpoint{4.063152in}{5.084012in}}%
\pgfpathlineto{\pgfqpoint{4.057819in}{5.084012in}}%
\pgfpathlineto{\pgfqpoint{4.052486in}{5.084012in}}%
\pgfpathlineto{\pgfqpoint{4.047152in}{5.084012in}}%
\pgfpathlineto{\pgfqpoint{4.041819in}{5.084012in}}%
\pgfpathlineto{\pgfqpoint{4.036486in}{5.084012in}}%
\pgfpathlineto{\pgfqpoint{4.031153in}{5.084012in}}%
\pgfpathlineto{\pgfqpoint{4.025820in}{5.084012in}}%
\pgfpathlineto{\pgfqpoint{4.020486in}{5.084012in}}%
\pgfpathlineto{\pgfqpoint{4.015153in}{5.084012in}}%
\pgfpathlineto{\pgfqpoint{4.009820in}{5.084012in}}%
\pgfpathlineto{\pgfqpoint{4.004487in}{5.084012in}}%
\pgfpathlineto{\pgfqpoint{3.999154in}{5.084012in}}%
\pgfpathlineto{\pgfqpoint{3.993820in}{5.084012in}}%
\pgfpathlineto{\pgfqpoint{3.988487in}{5.084012in}}%
\pgfpathlineto{\pgfqpoint{3.983154in}{5.084012in}}%
\pgfpathlineto{\pgfqpoint{3.977821in}{5.084012in}}%
\pgfpathlineto{\pgfqpoint{3.972487in}{5.084012in}}%
\pgfpathlineto{\pgfqpoint{3.967154in}{5.084012in}}%
\pgfpathlineto{\pgfqpoint{3.961821in}{5.084012in}}%
\pgfpathlineto{\pgfqpoint{3.956488in}{5.084012in}}%
\pgfpathlineto{\pgfqpoint{3.951155in}{5.084012in}}%
\pgfpathlineto{\pgfqpoint{3.945821in}{5.084012in}}%
\pgfpathlineto{\pgfqpoint{3.940488in}{5.084012in}}%
\pgfpathlineto{\pgfqpoint{3.935155in}{5.084012in}}%
\pgfpathlineto{\pgfqpoint{3.929822in}{5.084012in}}%
\pgfpathlineto{\pgfqpoint{3.924488in}{5.084012in}}%
\pgfpathlineto{\pgfqpoint{3.919155in}{5.084012in}}%
\pgfpathlineto{\pgfqpoint{3.913822in}{5.084012in}}%
\pgfpathlineto{\pgfqpoint{3.908489in}{5.084012in}}%
\pgfpathlineto{\pgfqpoint{3.903156in}{5.084012in}}%
\pgfpathlineto{\pgfqpoint{3.897822in}{5.084012in}}%
\pgfpathlineto{\pgfqpoint{3.892489in}{5.084012in}}%
\pgfpathlineto{\pgfqpoint{3.887156in}{5.084012in}}%
\pgfpathlineto{\pgfqpoint{3.881823in}{5.084012in}}%
\pgfpathlineto{\pgfqpoint{3.876490in}{5.084012in}}%
\pgfpathlineto{\pgfqpoint{3.871156in}{5.084012in}}%
\pgfpathlineto{\pgfqpoint{3.865823in}{5.084012in}}%
\pgfpathlineto{\pgfqpoint{3.860490in}{5.084012in}}%
\pgfpathlineto{\pgfqpoint{3.855157in}{5.084012in}}%
\pgfpathlineto{\pgfqpoint{3.849823in}{5.084012in}}%
\pgfpathlineto{\pgfqpoint{3.844490in}{5.084012in}}%
\pgfpathlineto{\pgfqpoint{3.839157in}{5.084012in}}%
\pgfpathlineto{\pgfqpoint{3.833824in}{5.084012in}}%
\pgfpathlineto{\pgfqpoint{3.828491in}{5.084012in}}%
\pgfpathlineto{\pgfqpoint{3.823157in}{5.084012in}}%
\pgfpathlineto{\pgfqpoint{3.817824in}{5.084012in}}%
\pgfpathlineto{\pgfqpoint{3.812491in}{5.084012in}}%
\pgfpathlineto{\pgfqpoint{3.807158in}{5.084012in}}%
\pgfpathlineto{\pgfqpoint{3.801825in}{5.084012in}}%
\pgfpathlineto{\pgfqpoint{3.796491in}{5.084012in}}%
\pgfpathlineto{\pgfqpoint{3.791158in}{5.084012in}}%
\pgfpathlineto{\pgfqpoint{3.785825in}{5.084012in}}%
\pgfpathlineto{\pgfqpoint{3.780492in}{5.084012in}}%
\pgfpathlineto{\pgfqpoint{3.775158in}{5.084012in}}%
\pgfpathlineto{\pgfqpoint{3.769825in}{5.084012in}}%
\pgfpathlineto{\pgfqpoint{3.764492in}{5.084012in}}%
\pgfpathlineto{\pgfqpoint{3.759159in}{5.084012in}}%
\pgfpathlineto{\pgfqpoint{3.753826in}{5.084012in}}%
\pgfpathlineto{\pgfqpoint{3.748492in}{5.084012in}}%
\pgfpathlineto{\pgfqpoint{3.743159in}{5.084012in}}%
\pgfpathlineto{\pgfqpoint{3.737826in}{5.084012in}}%
\pgfpathlineto{\pgfqpoint{3.732493in}{5.084012in}}%
\pgfpathlineto{\pgfqpoint{3.727160in}{5.084012in}}%
\pgfpathlineto{\pgfqpoint{3.721826in}{5.084012in}}%
\pgfpathlineto{\pgfqpoint{3.716493in}{5.084012in}}%
\pgfpathlineto{\pgfqpoint{3.711160in}{5.084012in}}%
\pgfpathlineto{\pgfqpoint{3.705827in}{5.084012in}}%
\pgfpathlineto{\pgfqpoint{3.700493in}{5.084012in}}%
\pgfpathlineto{\pgfqpoint{3.695160in}{5.084012in}}%
\pgfpathlineto{\pgfqpoint{3.689827in}{5.084012in}}%
\pgfpathlineto{\pgfqpoint{3.684494in}{5.084012in}}%
\pgfpathlineto{\pgfqpoint{3.679161in}{5.084012in}}%
\pgfpathlineto{\pgfqpoint{3.673827in}{5.084012in}}%
\pgfpathlineto{\pgfqpoint{3.668494in}{5.084012in}}%
\pgfpathlineto{\pgfqpoint{3.663161in}{5.084012in}}%
\pgfpathlineto{\pgfqpoint{3.657828in}{5.084012in}}%
\pgfpathlineto{\pgfqpoint{3.652494in}{5.084012in}}%
\pgfpathlineto{\pgfqpoint{3.647161in}{5.084012in}}%
\pgfpathlineto{\pgfqpoint{3.641828in}{5.084012in}}%
\pgfpathlineto{\pgfqpoint{3.636495in}{5.084012in}}%
\pgfpathlineto{\pgfqpoint{3.631162in}{5.084012in}}%
\pgfpathlineto{\pgfqpoint{3.625828in}{5.084012in}}%
\pgfpathlineto{\pgfqpoint{3.620495in}{5.084012in}}%
\pgfpathlineto{\pgfqpoint{3.615162in}{5.084012in}}%
\pgfpathlineto{\pgfqpoint{3.609829in}{5.084012in}}%
\pgfpathlineto{\pgfqpoint{3.604496in}{5.084012in}}%
\pgfpathlineto{\pgfqpoint{3.599162in}{5.084012in}}%
\pgfpathlineto{\pgfqpoint{3.593829in}{5.084012in}}%
\pgfpathlineto{\pgfqpoint{3.588496in}{5.084012in}}%
\pgfpathlineto{\pgfqpoint{3.583163in}{5.084012in}}%
\pgfpathlineto{\pgfqpoint{3.577829in}{5.084012in}}%
\pgfpathlineto{\pgfqpoint{3.572496in}{5.084012in}}%
\pgfpathlineto{\pgfqpoint{3.567163in}{5.084012in}}%
\pgfpathlineto{\pgfqpoint{3.561830in}{5.084012in}}%
\pgfpathlineto{\pgfqpoint{3.556497in}{5.084012in}}%
\pgfpathlineto{\pgfqpoint{3.551163in}{5.084012in}}%
\pgfpathlineto{\pgfqpoint{3.545830in}{5.084012in}}%
\pgfpathlineto{\pgfqpoint{3.540497in}{5.084012in}}%
\pgfpathlineto{\pgfqpoint{3.535164in}{5.084012in}}%
\pgfpathlineto{\pgfqpoint{3.529831in}{5.084012in}}%
\pgfpathlineto{\pgfqpoint{3.524497in}{5.084012in}}%
\pgfpathlineto{\pgfqpoint{3.519164in}{5.084012in}}%
\pgfpathlineto{\pgfqpoint{3.513831in}{5.084012in}}%
\pgfpathlineto{\pgfqpoint{3.508498in}{5.084012in}}%
\pgfpathlineto{\pgfqpoint{3.503164in}{5.084012in}}%
\pgfpathlineto{\pgfqpoint{3.497831in}{5.084012in}}%
\pgfpathlineto{\pgfqpoint{3.492498in}{5.084012in}}%
\pgfpathlineto{\pgfqpoint{3.487165in}{5.084012in}}%
\pgfpathlineto{\pgfqpoint{3.481832in}{5.084012in}}%
\pgfpathlineto{\pgfqpoint{3.476498in}{5.084012in}}%
\pgfpathlineto{\pgfqpoint{3.471165in}{5.084012in}}%
\pgfpathlineto{\pgfqpoint{3.465832in}{5.084012in}}%
\pgfpathlineto{\pgfqpoint{3.460499in}{5.084012in}}%
\pgfpathlineto{\pgfqpoint{3.455166in}{5.084012in}}%
\pgfpathlineto{\pgfqpoint{3.449832in}{5.084012in}}%
\pgfpathlineto{\pgfqpoint{3.444499in}{5.084012in}}%
\pgfpathlineto{\pgfqpoint{3.439166in}{5.084012in}}%
\pgfpathlineto{\pgfqpoint{3.433833in}{5.084012in}}%
\pgfpathlineto{\pgfqpoint{3.428499in}{5.084012in}}%
\pgfpathlineto{\pgfqpoint{3.423166in}{5.084012in}}%
\pgfpathlineto{\pgfqpoint{3.417833in}{5.084012in}}%
\pgfpathlineto{\pgfqpoint{3.412500in}{5.084012in}}%
\pgfpathlineto{\pgfqpoint{3.407167in}{5.084012in}}%
\pgfpathlineto{\pgfqpoint{3.401833in}{5.084012in}}%
\pgfpathlineto{\pgfqpoint{3.396500in}{5.084012in}}%
\pgfpathlineto{\pgfqpoint{3.391167in}{5.084012in}}%
\pgfpathlineto{\pgfqpoint{3.385834in}{5.084012in}}%
\pgfpathlineto{\pgfqpoint{3.380500in}{5.084012in}}%
\pgfpathlineto{\pgfqpoint{3.375167in}{5.084012in}}%
\pgfpathlineto{\pgfqpoint{3.369834in}{5.084012in}}%
\pgfpathlineto{\pgfqpoint{3.364501in}{5.084012in}}%
\pgfpathlineto{\pgfqpoint{3.359168in}{5.084012in}}%
\pgfpathlineto{\pgfqpoint{3.353834in}{5.084012in}}%
\pgfpathlineto{\pgfqpoint{3.348501in}{5.084012in}}%
\pgfpathlineto{\pgfqpoint{3.343168in}{5.084012in}}%
\pgfpathlineto{\pgfqpoint{3.337835in}{5.084012in}}%
\pgfpathlineto{\pgfqpoint{3.332502in}{5.084012in}}%
\pgfpathlineto{\pgfqpoint{3.327168in}{5.084012in}}%
\pgfpathlineto{\pgfqpoint{3.321835in}{5.084012in}}%
\pgfpathlineto{\pgfqpoint{3.316502in}{5.084012in}}%
\pgfpathlineto{\pgfqpoint{3.311169in}{5.084012in}}%
\pgfpathlineto{\pgfqpoint{3.305835in}{5.084012in}}%
\pgfpathlineto{\pgfqpoint{3.300502in}{5.084012in}}%
\pgfpathlineto{\pgfqpoint{3.295169in}{5.084012in}}%
\pgfpathlineto{\pgfqpoint{3.289836in}{5.084012in}}%
\pgfpathlineto{\pgfqpoint{3.284503in}{5.084012in}}%
\pgfpathlineto{\pgfqpoint{3.279169in}{5.084012in}}%
\pgfpathlineto{\pgfqpoint{3.273836in}{5.084012in}}%
\pgfpathlineto{\pgfqpoint{3.268503in}{5.084012in}}%
\pgfpathlineto{\pgfqpoint{3.263170in}{5.084012in}}%
\pgfpathlineto{\pgfqpoint{3.257837in}{5.084012in}}%
\pgfpathlineto{\pgfqpoint{3.252503in}{5.084012in}}%
\pgfpathlineto{\pgfqpoint{3.247170in}{5.084012in}}%
\pgfpathlineto{\pgfqpoint{3.241837in}{5.084012in}}%
\pgfpathlineto{\pgfqpoint{3.236504in}{5.084012in}}%
\pgfpathlineto{\pgfqpoint{3.231170in}{5.084012in}}%
\pgfpathlineto{\pgfqpoint{3.225837in}{5.084012in}}%
\pgfpathlineto{\pgfqpoint{3.220504in}{5.084012in}}%
\pgfpathlineto{\pgfqpoint{3.215171in}{5.084012in}}%
\pgfpathlineto{\pgfqpoint{3.209838in}{5.084012in}}%
\pgfpathlineto{\pgfqpoint{3.204504in}{5.084012in}}%
\pgfpathlineto{\pgfqpoint{3.199171in}{5.084012in}}%
\pgfpathlineto{\pgfqpoint{3.193838in}{5.084012in}}%
\pgfpathlineto{\pgfqpoint{3.188505in}{5.084012in}}%
\pgfpathlineto{\pgfqpoint{3.183172in}{5.084012in}}%
\pgfpathlineto{\pgfqpoint{3.177838in}{5.084012in}}%
\pgfpathlineto{\pgfqpoint{3.172505in}{5.084012in}}%
\pgfpathlineto{\pgfqpoint{3.167172in}{5.084012in}}%
\pgfpathlineto{\pgfqpoint{3.161839in}{5.084012in}}%
\pgfpathlineto{\pgfqpoint{3.156505in}{5.084012in}}%
\pgfpathlineto{\pgfqpoint{3.151172in}{5.084012in}}%
\pgfpathlineto{\pgfqpoint{3.145839in}{5.084012in}}%
\pgfpathlineto{\pgfqpoint{3.140506in}{5.084012in}}%
\pgfpathlineto{\pgfqpoint{3.135173in}{5.084012in}}%
\pgfpathlineto{\pgfqpoint{3.129839in}{5.084012in}}%
\pgfpathlineto{\pgfqpoint{3.124506in}{5.084012in}}%
\pgfpathlineto{\pgfqpoint{3.119173in}{5.084012in}}%
\pgfpathlineto{\pgfqpoint{3.113840in}{5.084012in}}%
\pgfpathlineto{\pgfqpoint{3.108506in}{5.084012in}}%
\pgfpathlineto{\pgfqpoint{3.103173in}{5.084012in}}%
\pgfpathlineto{\pgfqpoint{3.097840in}{5.084012in}}%
\pgfpathlineto{\pgfqpoint{3.092507in}{5.084012in}}%
\pgfpathlineto{\pgfqpoint{3.087174in}{5.084012in}}%
\pgfpathlineto{\pgfqpoint{3.081840in}{5.084012in}}%
\pgfpathlineto{\pgfqpoint{3.076507in}{5.084012in}}%
\pgfpathlineto{\pgfqpoint{3.071174in}{5.084012in}}%
\pgfpathlineto{\pgfqpoint{3.065841in}{5.084012in}}%
\pgfpathlineto{\pgfqpoint{3.060508in}{5.084012in}}%
\pgfpathlineto{\pgfqpoint{3.055174in}{5.084012in}}%
\pgfpathlineto{\pgfqpoint{3.049841in}{5.084012in}}%
\pgfpathlineto{\pgfqpoint{3.044508in}{5.084012in}}%
\pgfpathlineto{\pgfqpoint{3.039175in}{5.084012in}}%
\pgfpathlineto{\pgfqpoint{3.033841in}{5.084012in}}%
\pgfpathlineto{\pgfqpoint{3.028508in}{5.084012in}}%
\pgfpathlineto{\pgfqpoint{3.023175in}{5.084012in}}%
\pgfpathlineto{\pgfqpoint{3.017842in}{5.084012in}}%
\pgfpathlineto{\pgfqpoint{3.012509in}{5.084012in}}%
\pgfpathlineto{\pgfqpoint{3.007175in}{5.084012in}}%
\pgfpathlineto{\pgfqpoint{3.001842in}{5.084012in}}%
\pgfpathlineto{\pgfqpoint{2.996509in}{5.084012in}}%
\pgfpathlineto{\pgfqpoint{2.991176in}{5.084012in}}%
\pgfpathlineto{\pgfqpoint{2.985843in}{5.084012in}}%
\pgfpathlineto{\pgfqpoint{2.980509in}{5.084012in}}%
\pgfpathlineto{\pgfqpoint{2.975176in}{5.084012in}}%
\pgfpathlineto{\pgfqpoint{2.969843in}{5.084012in}}%
\pgfpathlineto{\pgfqpoint{2.964510in}{5.084012in}}%
\pgfpathlineto{\pgfqpoint{2.959176in}{5.084012in}}%
\pgfpathlineto{\pgfqpoint{2.953843in}{5.084012in}}%
\pgfpathlineto{\pgfqpoint{2.948510in}{5.084012in}}%
\pgfpathlineto{\pgfqpoint{2.943177in}{5.084012in}}%
\pgfpathlineto{\pgfqpoint{2.937844in}{5.084012in}}%
\pgfpathlineto{\pgfqpoint{2.932510in}{5.084012in}}%
\pgfpathlineto{\pgfqpoint{2.927177in}{5.084012in}}%
\pgfpathlineto{\pgfqpoint{2.921844in}{5.084012in}}%
\pgfpathlineto{\pgfqpoint{2.916511in}{5.084012in}}%
\pgfpathlineto{\pgfqpoint{2.911178in}{5.084012in}}%
\pgfpathlineto{\pgfqpoint{2.905844in}{5.084012in}}%
\pgfpathlineto{\pgfqpoint{2.900511in}{5.084012in}}%
\pgfpathlineto{\pgfqpoint{2.895178in}{5.084012in}}%
\pgfpathlineto{\pgfqpoint{2.889845in}{5.084012in}}%
\pgfpathlineto{\pgfqpoint{2.884511in}{5.084012in}}%
\pgfpathlineto{\pgfqpoint{2.879178in}{5.084012in}}%
\pgfpathlineto{\pgfqpoint{2.873845in}{5.084012in}}%
\pgfpathlineto{\pgfqpoint{2.868512in}{5.084012in}}%
\pgfpathlineto{\pgfqpoint{2.863179in}{5.084012in}}%
\pgfpathlineto{\pgfqpoint{2.857845in}{5.084012in}}%
\pgfpathlineto{\pgfqpoint{2.852512in}{5.084012in}}%
\pgfpathlineto{\pgfqpoint{2.847179in}{5.084012in}}%
\pgfpathlineto{\pgfqpoint{2.841846in}{5.084012in}}%
\pgfpathlineto{\pgfqpoint{2.836512in}{5.084012in}}%
\pgfpathlineto{\pgfqpoint{2.831179in}{5.084012in}}%
\pgfpathlineto{\pgfqpoint{2.825846in}{5.084012in}}%
\pgfpathlineto{\pgfqpoint{2.820513in}{5.084012in}}%
\pgfpathlineto{\pgfqpoint{2.815180in}{5.084012in}}%
\pgfpathlineto{\pgfqpoint{2.809846in}{5.084012in}}%
\pgfpathlineto{\pgfqpoint{2.804513in}{5.084012in}}%
\pgfpathlineto{\pgfqpoint{2.799180in}{5.084012in}}%
\pgfpathlineto{\pgfqpoint{2.793847in}{5.084012in}}%
\pgfpathlineto{\pgfqpoint{2.788514in}{5.084012in}}%
\pgfpathlineto{\pgfqpoint{2.783180in}{5.084012in}}%
\pgfpathlineto{\pgfqpoint{2.777847in}{5.084012in}}%
\pgfpathlineto{\pgfqpoint{2.772514in}{5.084012in}}%
\pgfpathlineto{\pgfqpoint{2.767181in}{5.084012in}}%
\pgfpathlineto{\pgfqpoint{2.761847in}{5.084012in}}%
\pgfpathlineto{\pgfqpoint{2.756514in}{5.084012in}}%
\pgfpathlineto{\pgfqpoint{2.751181in}{5.084012in}}%
\pgfpathlineto{\pgfqpoint{2.745848in}{5.084012in}}%
\pgfpathlineto{\pgfqpoint{2.740515in}{5.084012in}}%
\pgfpathlineto{\pgfqpoint{2.735181in}{5.084012in}}%
\pgfpathlineto{\pgfqpoint{2.729848in}{5.084012in}}%
\pgfpathlineto{\pgfqpoint{2.724515in}{5.084012in}}%
\pgfpathlineto{\pgfqpoint{2.719182in}{5.084012in}}%
\pgfpathlineto{\pgfqpoint{2.713849in}{5.084012in}}%
\pgfpathlineto{\pgfqpoint{2.708515in}{5.084012in}}%
\pgfpathlineto{\pgfqpoint{2.703182in}{5.084012in}}%
\pgfpathlineto{\pgfqpoint{2.697849in}{5.084012in}}%
\pgfpathlineto{\pgfqpoint{2.692516in}{5.084012in}}%
\pgfpathlineto{\pgfqpoint{2.687182in}{5.084012in}}%
\pgfpathlineto{\pgfqpoint{2.681849in}{5.084012in}}%
\pgfpathlineto{\pgfqpoint{2.676516in}{5.084012in}}%
\pgfpathlineto{\pgfqpoint{2.671183in}{5.084012in}}%
\pgfpathlineto{\pgfqpoint{2.665850in}{5.084012in}}%
\pgfpathlineto{\pgfqpoint{2.660516in}{5.084012in}}%
\pgfpathlineto{\pgfqpoint{2.655183in}{5.084012in}}%
\pgfpathlineto{\pgfqpoint{2.649850in}{5.084012in}}%
\pgfpathlineto{\pgfqpoint{2.644517in}{5.084012in}}%
\pgfpathlineto{\pgfqpoint{2.639184in}{5.084012in}}%
\pgfpathlineto{\pgfqpoint{2.633850in}{5.084012in}}%
\pgfpathlineto{\pgfqpoint{2.628517in}{5.084012in}}%
\pgfpathlineto{\pgfqpoint{2.623184in}{5.084012in}}%
\pgfpathlineto{\pgfqpoint{2.617851in}{5.084012in}}%
\pgfpathlineto{\pgfqpoint{2.612517in}{5.084012in}}%
\pgfpathlineto{\pgfqpoint{2.607184in}{5.084012in}}%
\pgfpathlineto{\pgfqpoint{2.601851in}{5.084012in}}%
\pgfpathlineto{\pgfqpoint{2.596518in}{5.084012in}}%
\pgfpathlineto{\pgfqpoint{2.591185in}{5.084012in}}%
\pgfpathlineto{\pgfqpoint{2.585851in}{5.084012in}}%
\pgfpathlineto{\pgfqpoint{2.580518in}{5.084012in}}%
\pgfpathlineto{\pgfqpoint{2.575185in}{5.084012in}}%
\pgfpathlineto{\pgfqpoint{2.569852in}{5.084012in}}%
\pgfpathlineto{\pgfqpoint{2.564518in}{5.084012in}}%
\pgfpathlineto{\pgfqpoint{2.559185in}{5.084012in}}%
\pgfpathlineto{\pgfqpoint{2.553852in}{5.084012in}}%
\pgfpathlineto{\pgfqpoint{2.548519in}{5.084012in}}%
\pgfpathlineto{\pgfqpoint{2.543186in}{5.084012in}}%
\pgfpathlineto{\pgfqpoint{2.537852in}{5.084012in}}%
\pgfpathlineto{\pgfqpoint{2.532519in}{5.084012in}}%
\pgfpathlineto{\pgfqpoint{2.527186in}{5.084012in}}%
\pgfpathlineto{\pgfqpoint{2.521853in}{5.084012in}}%
\pgfpathlineto{\pgfqpoint{2.516520in}{5.084012in}}%
\pgfpathlineto{\pgfqpoint{2.511186in}{5.084012in}}%
\pgfpathlineto{\pgfqpoint{2.505853in}{5.084012in}}%
\pgfpathlineto{\pgfqpoint{2.500520in}{5.084012in}}%
\pgfpathlineto{\pgfqpoint{2.495187in}{5.084012in}}%
\pgfpathlineto{\pgfqpoint{2.489853in}{5.084012in}}%
\pgfpathlineto{\pgfqpoint{2.484520in}{5.084012in}}%
\pgfpathlineto{\pgfqpoint{2.479187in}{5.084012in}}%
\pgfpathlineto{\pgfqpoint{2.473854in}{5.084012in}}%
\pgfpathlineto{\pgfqpoint{2.468521in}{5.084012in}}%
\pgfpathlineto{\pgfqpoint{2.463187in}{5.084012in}}%
\pgfpathlineto{\pgfqpoint{2.457854in}{5.084012in}}%
\pgfpathlineto{\pgfqpoint{2.452521in}{5.084012in}}%
\pgfpathlineto{\pgfqpoint{2.447188in}{5.084012in}}%
\pgfpathlineto{\pgfqpoint{2.441855in}{5.084012in}}%
\pgfpathlineto{\pgfqpoint{2.436521in}{5.084012in}}%
\pgfpathlineto{\pgfqpoint{2.431188in}{5.084012in}}%
\pgfpathlineto{\pgfqpoint{2.425855in}{5.084012in}}%
\pgfpathlineto{\pgfqpoint{2.420522in}{5.084012in}}%
\pgfpathlineto{\pgfqpoint{2.415188in}{5.084012in}}%
\pgfpathlineto{\pgfqpoint{2.409855in}{5.084012in}}%
\pgfpathlineto{\pgfqpoint{2.404522in}{5.084012in}}%
\pgfpathlineto{\pgfqpoint{2.399189in}{5.084012in}}%
\pgfpathlineto{\pgfqpoint{2.393856in}{5.084012in}}%
\pgfpathlineto{\pgfqpoint{2.388522in}{5.084012in}}%
\pgfpathlineto{\pgfqpoint{2.383189in}{5.084012in}}%
\pgfpathlineto{\pgfqpoint{2.377856in}{5.084012in}}%
\pgfpathlineto{\pgfqpoint{2.372523in}{5.084012in}}%
\pgfpathlineto{\pgfqpoint{2.367190in}{5.084012in}}%
\pgfpathlineto{\pgfqpoint{2.361856in}{5.084012in}}%
\pgfpathlineto{\pgfqpoint{2.356523in}{5.084012in}}%
\pgfpathlineto{\pgfqpoint{2.351190in}{5.084012in}}%
\pgfpathlineto{\pgfqpoint{2.345857in}{5.084012in}}%
\pgfpathlineto{\pgfqpoint{2.340523in}{5.084012in}}%
\pgfpathlineto{\pgfqpoint{2.335190in}{5.084012in}}%
\pgfpathlineto{\pgfqpoint{2.329857in}{5.084012in}}%
\pgfpathlineto{\pgfqpoint{2.324524in}{5.084012in}}%
\pgfpathlineto{\pgfqpoint{2.319191in}{5.084012in}}%
\pgfpathlineto{\pgfqpoint{2.313857in}{5.084012in}}%
\pgfpathlineto{\pgfqpoint{2.308524in}{5.084012in}}%
\pgfpathlineto{\pgfqpoint{2.303191in}{5.084012in}}%
\pgfpathlineto{\pgfqpoint{2.297858in}{5.084012in}}%
\pgfpathlineto{\pgfqpoint{2.292524in}{5.084012in}}%
\pgfpathlineto{\pgfqpoint{2.287191in}{5.084012in}}%
\pgfpathlineto{\pgfqpoint{2.281858in}{5.084012in}}%
\pgfpathlineto{\pgfqpoint{2.276525in}{5.084012in}}%
\pgfpathlineto{\pgfqpoint{2.271192in}{5.084012in}}%
\pgfpathlineto{\pgfqpoint{2.265858in}{5.084012in}}%
\pgfpathlineto{\pgfqpoint{2.260525in}{5.084012in}}%
\pgfpathlineto{\pgfqpoint{2.255192in}{5.084012in}}%
\pgfpathlineto{\pgfqpoint{2.249859in}{5.084012in}}%
\pgfpathlineto{\pgfqpoint{2.244526in}{5.084012in}}%
\pgfpathlineto{\pgfqpoint{2.239192in}{5.084012in}}%
\pgfpathlineto{\pgfqpoint{2.233859in}{5.084012in}}%
\pgfpathlineto{\pgfqpoint{2.228526in}{5.084012in}}%
\pgfpathlineto{\pgfqpoint{2.223193in}{5.084012in}}%
\pgfpathlineto{\pgfqpoint{2.217859in}{5.084012in}}%
\pgfpathlineto{\pgfqpoint{2.212526in}{5.084012in}}%
\pgfpathlineto{\pgfqpoint{2.207193in}{5.084012in}}%
\pgfpathlineto{\pgfqpoint{2.201860in}{5.084012in}}%
\pgfpathlineto{\pgfqpoint{2.196527in}{5.084012in}}%
\pgfpathlineto{\pgfqpoint{2.191193in}{5.084012in}}%
\pgfpathlineto{\pgfqpoint{2.185860in}{5.084012in}}%
\pgfpathlineto{\pgfqpoint{2.180527in}{5.084012in}}%
\pgfpathlineto{\pgfqpoint{2.175194in}{5.084012in}}%
\pgfpathlineto{\pgfqpoint{2.169861in}{5.084012in}}%
\pgfpathlineto{\pgfqpoint{2.164527in}{5.084012in}}%
\pgfpathlineto{\pgfqpoint{2.159194in}{5.084012in}}%
\pgfpathlineto{\pgfqpoint{2.153861in}{5.084012in}}%
\pgfpathlineto{\pgfqpoint{2.148528in}{5.084012in}}%
\pgfpathlineto{\pgfqpoint{2.143194in}{5.084012in}}%
\pgfpathlineto{\pgfqpoint{2.137861in}{5.084012in}}%
\pgfpathlineto{\pgfqpoint{2.132528in}{5.084012in}}%
\pgfpathlineto{\pgfqpoint{2.127195in}{5.084012in}}%
\pgfpathlineto{\pgfqpoint{2.121862in}{5.084012in}}%
\pgfpathlineto{\pgfqpoint{2.116528in}{5.084012in}}%
\pgfpathlineto{\pgfqpoint{2.111195in}{5.084012in}}%
\pgfpathlineto{\pgfqpoint{2.105862in}{5.084012in}}%
\pgfpathlineto{\pgfqpoint{2.100529in}{5.084012in}}%
\pgfpathlineto{\pgfqpoint{2.095196in}{5.084012in}}%
\pgfpathlineto{\pgfqpoint{2.089862in}{5.084012in}}%
\pgfpathlineto{\pgfqpoint{2.084529in}{5.084012in}}%
\pgfpathlineto{\pgfqpoint{2.079196in}{5.084012in}}%
\pgfpathlineto{\pgfqpoint{2.073863in}{5.084012in}}%
\pgfpathlineto{\pgfqpoint{2.068529in}{5.084012in}}%
\pgfpathlineto{\pgfqpoint{2.063196in}{5.084012in}}%
\pgfpathlineto{\pgfqpoint{2.057863in}{5.084012in}}%
\pgfpathlineto{\pgfqpoint{2.052530in}{5.084012in}}%
\pgfpathlineto{\pgfqpoint{2.047197in}{5.084012in}}%
\pgfpathlineto{\pgfqpoint{2.041863in}{5.084012in}}%
\pgfpathlineto{\pgfqpoint{2.036530in}{5.084012in}}%
\pgfpathlineto{\pgfqpoint{2.031197in}{5.084012in}}%
\pgfpathlineto{\pgfqpoint{2.025864in}{5.084012in}}%
\pgfpathlineto{\pgfqpoint{2.020531in}{5.084012in}}%
\pgfpathlineto{\pgfqpoint{2.015197in}{5.084012in}}%
\pgfpathlineto{\pgfqpoint{2.009864in}{5.084012in}}%
\pgfpathlineto{\pgfqpoint{2.004531in}{5.084012in}}%
\pgfpathlineto{\pgfqpoint{1.999198in}{5.084012in}}%
\pgfpathlineto{\pgfqpoint{1.993864in}{5.084012in}}%
\pgfpathlineto{\pgfqpoint{1.988531in}{5.084012in}}%
\pgfpathlineto{\pgfqpoint{1.983198in}{5.084012in}}%
\pgfpathlineto{\pgfqpoint{1.977865in}{5.084012in}}%
\pgfpathlineto{\pgfqpoint{1.972532in}{5.084012in}}%
\pgfpathlineto{\pgfqpoint{1.967198in}{5.084012in}}%
\pgfpathlineto{\pgfqpoint{1.961865in}{5.084012in}}%
\pgfpathlineto{\pgfqpoint{1.956532in}{5.084012in}}%
\pgfpathlineto{\pgfqpoint{1.951199in}{5.084012in}}%
\pgfpathlineto{\pgfqpoint{1.945865in}{5.084012in}}%
\pgfpathlineto{\pgfqpoint{1.940532in}{5.084012in}}%
\pgfpathlineto{\pgfqpoint{1.935199in}{5.084012in}}%
\pgfpathlineto{\pgfqpoint{1.929866in}{5.084012in}}%
\pgfpathlineto{\pgfqpoint{1.924533in}{5.084012in}}%
\pgfpathlineto{\pgfqpoint{1.919199in}{5.084012in}}%
\pgfpathlineto{\pgfqpoint{1.913866in}{5.084012in}}%
\pgfpathlineto{\pgfqpoint{1.908533in}{5.084012in}}%
\pgfpathlineto{\pgfqpoint{1.903200in}{5.084012in}}%
\pgfpathlineto{\pgfqpoint{1.897867in}{5.084012in}}%
\pgfpathlineto{\pgfqpoint{1.892533in}{5.084012in}}%
\pgfpathlineto{\pgfqpoint{1.887200in}{5.084012in}}%
\pgfpathlineto{\pgfqpoint{1.881867in}{5.084012in}}%
\pgfpathlineto{\pgfqpoint{1.876534in}{5.084012in}}%
\pgfpathlineto{\pgfqpoint{1.871200in}{5.084012in}}%
\pgfpathlineto{\pgfqpoint{1.865867in}{5.084012in}}%
\pgfpathlineto{\pgfqpoint{1.860534in}{5.084012in}}%
\pgfpathlineto{\pgfqpoint{1.855201in}{5.084012in}}%
\pgfpathlineto{\pgfqpoint{1.849868in}{5.084012in}}%
\pgfpathlineto{\pgfqpoint{1.844534in}{5.084012in}}%
\pgfpathlineto{\pgfqpoint{1.839201in}{5.084012in}}%
\pgfpathlineto{\pgfqpoint{1.833868in}{5.084012in}}%
\pgfpathlineto{\pgfqpoint{1.828535in}{5.084012in}}%
\pgfpathlineto{\pgfqpoint{1.823202in}{5.084012in}}%
\pgfpathlineto{\pgfqpoint{1.817868in}{5.084012in}}%
\pgfpathlineto{\pgfqpoint{1.812535in}{5.084012in}}%
\pgfpathlineto{\pgfqpoint{1.807202in}{5.084012in}}%
\pgfpathlineto{\pgfqpoint{1.801869in}{5.084012in}}%
\pgfpathlineto{\pgfqpoint{1.796535in}{5.084012in}}%
\pgfpathlineto{\pgfqpoint{1.791202in}{5.084012in}}%
\pgfpathlineto{\pgfqpoint{1.785869in}{5.084012in}}%
\pgfpathlineto{\pgfqpoint{1.780536in}{5.084012in}}%
\pgfpathlineto{\pgfqpoint{1.775203in}{5.084012in}}%
\pgfpathlineto{\pgfqpoint{1.769869in}{5.084012in}}%
\pgfpathlineto{\pgfqpoint{1.764536in}{5.084012in}}%
\pgfpathlineto{\pgfqpoint{1.759203in}{5.084012in}}%
\pgfpathlineto{\pgfqpoint{1.753870in}{5.084012in}}%
\pgfpathlineto{\pgfqpoint{1.748537in}{5.084012in}}%
\pgfpathlineto{\pgfqpoint{1.743203in}{5.084012in}}%
\pgfpathlineto{\pgfqpoint{1.737870in}{5.084012in}}%
\pgfpathlineto{\pgfqpoint{1.732537in}{5.084012in}}%
\pgfpathlineto{\pgfqpoint{1.727204in}{5.084012in}}%
\pgfpathlineto{\pgfqpoint{1.721870in}{5.084012in}}%
\pgfpathlineto{\pgfqpoint{1.716537in}{5.084012in}}%
\pgfpathlineto{\pgfqpoint{1.711204in}{5.084012in}}%
\pgfpathlineto{\pgfqpoint{1.705871in}{5.084012in}}%
\pgfpathlineto{\pgfqpoint{1.700538in}{5.084012in}}%
\pgfpathlineto{\pgfqpoint{1.695204in}{5.084012in}}%
\pgfpathlineto{\pgfqpoint{1.689871in}{5.084012in}}%
\pgfpathlineto{\pgfqpoint{1.684538in}{5.084012in}}%
\pgfpathlineto{\pgfqpoint{1.679205in}{5.084012in}}%
\pgfpathlineto{\pgfqpoint{1.673871in}{5.084012in}}%
\pgfpathlineto{\pgfqpoint{1.668538in}{5.084012in}}%
\pgfpathlineto{\pgfqpoint{1.663205in}{5.084012in}}%
\pgfpathlineto{\pgfqpoint{1.657872in}{5.084012in}}%
\pgfpathlineto{\pgfqpoint{1.652539in}{5.084012in}}%
\pgfpathlineto{\pgfqpoint{1.647205in}{5.084012in}}%
\pgfpathlineto{\pgfqpoint{1.641872in}{5.084012in}}%
\pgfpathlineto{\pgfqpoint{1.636539in}{5.084012in}}%
\pgfpathlineto{\pgfqpoint{1.631206in}{5.084012in}}%
\pgfpathlineto{\pgfqpoint{1.625873in}{5.084012in}}%
\pgfpathlineto{\pgfqpoint{1.620539in}{5.084012in}}%
\pgfpathlineto{\pgfqpoint{1.615206in}{5.084012in}}%
\pgfpathlineto{\pgfqpoint{1.609873in}{5.084012in}}%
\pgfpathlineto{\pgfqpoint{1.604540in}{5.084012in}}%
\pgfpathlineto{\pgfqpoint{1.599206in}{5.084012in}}%
\pgfpathlineto{\pgfqpoint{1.593873in}{5.084012in}}%
\pgfpathlineto{\pgfqpoint{1.588540in}{5.084012in}}%
\pgfpathlineto{\pgfqpoint{1.583207in}{5.084012in}}%
\pgfpathlineto{\pgfqpoint{1.577874in}{5.084012in}}%
\pgfpathlineto{\pgfqpoint{1.572540in}{5.084012in}}%
\pgfpathlineto{\pgfqpoint{1.567207in}{5.084012in}}%
\pgfpathlineto{\pgfqpoint{1.561874in}{5.084012in}}%
\pgfpathlineto{\pgfqpoint{1.556541in}{5.084012in}}%
\pgfpathlineto{\pgfqpoint{1.551208in}{5.084012in}}%
\pgfpathlineto{\pgfqpoint{1.545874in}{5.084012in}}%
\pgfpathlineto{\pgfqpoint{1.540541in}{5.084012in}}%
\pgfpathlineto{\pgfqpoint{1.535208in}{5.084012in}}%
\pgfpathlineto{\pgfqpoint{1.529875in}{5.084012in}}%
\pgfpathlineto{\pgfqpoint{1.524541in}{5.084012in}}%
\pgfpathlineto{\pgfqpoint{1.519208in}{5.084012in}}%
\pgfpathlineto{\pgfqpoint{1.513875in}{5.084012in}}%
\pgfpathlineto{\pgfqpoint{1.508542in}{5.084012in}}%
\pgfpathlineto{\pgfqpoint{1.503209in}{5.084012in}}%
\pgfpathlineto{\pgfqpoint{1.497875in}{5.084012in}}%
\pgfpathlineto{\pgfqpoint{1.492542in}{5.084012in}}%
\pgfpathlineto{\pgfqpoint{1.487209in}{5.084012in}}%
\pgfpathlineto{\pgfqpoint{1.481876in}{5.084012in}}%
\pgfpathlineto{\pgfqpoint{1.476543in}{5.084012in}}%
\pgfpathlineto{\pgfqpoint{1.471209in}{5.084012in}}%
\pgfpathlineto{\pgfqpoint{1.465876in}{5.084012in}}%
\pgfpathlineto{\pgfqpoint{1.460543in}{5.084012in}}%
\pgfpathlineto{\pgfqpoint{1.455210in}{5.084012in}}%
\pgfpathlineto{\pgfqpoint{1.449876in}{5.084012in}}%
\pgfpathlineto{\pgfqpoint{1.444543in}{5.084012in}}%
\pgfpathlineto{\pgfqpoint{1.439210in}{5.084012in}}%
\pgfpathlineto{\pgfqpoint{1.433877in}{5.084012in}}%
\pgfpathlineto{\pgfqpoint{1.428544in}{5.084012in}}%
\pgfpathlineto{\pgfqpoint{1.423210in}{5.084012in}}%
\pgfpathlineto{\pgfqpoint{1.417877in}{5.084012in}}%
\pgfpathlineto{\pgfqpoint{1.412544in}{5.084012in}}%
\pgfpathlineto{\pgfqpoint{1.407211in}{5.084012in}}%
\pgfpathlineto{\pgfqpoint{1.401877in}{5.084012in}}%
\pgfpathlineto{\pgfqpoint{1.396544in}{5.084012in}}%
\pgfpathlineto{\pgfqpoint{1.391211in}{5.084012in}}%
\pgfpathlineto{\pgfqpoint{1.385878in}{5.084012in}}%
\pgfpathlineto{\pgfqpoint{1.380545in}{5.084012in}}%
\pgfpathlineto{\pgfqpoint{1.375211in}{5.084012in}}%
\pgfpathlineto{\pgfqpoint{1.369878in}{5.084012in}}%
\pgfpathlineto{\pgfqpoint{1.364545in}{5.084012in}}%
\pgfpathlineto{\pgfqpoint{1.359212in}{5.084012in}}%
\pgfpathlineto{\pgfqpoint{1.353879in}{5.084012in}}%
\pgfpathlineto{\pgfqpoint{1.348545in}{5.084012in}}%
\pgfpathlineto{\pgfqpoint{1.343212in}{5.084012in}}%
\pgfpathlineto{\pgfqpoint{1.337879in}{5.084012in}}%
\pgfpathlineto{\pgfqpoint{1.332546in}{5.084012in}}%
\pgfpathlineto{\pgfqpoint{1.327212in}{5.084012in}}%
\pgfpathlineto{\pgfqpoint{1.321879in}{5.084012in}}%
\pgfpathlineto{\pgfqpoint{1.316546in}{5.084012in}}%
\pgfpathlineto{\pgfqpoint{1.311213in}{5.084012in}}%
\pgfpathlineto{\pgfqpoint{1.305880in}{5.084012in}}%
\pgfpathlineto{\pgfqpoint{1.300546in}{5.084012in}}%
\pgfpathlineto{\pgfqpoint{1.295213in}{5.084012in}}%
\pgfpathlineto{\pgfqpoint{1.289880in}{5.084012in}}%
\pgfpathlineto{\pgfqpoint{1.284547in}{5.084012in}}%
\pgfpathlineto{\pgfqpoint{1.279214in}{5.084012in}}%
\pgfpathlineto{\pgfqpoint{1.273880in}{5.084012in}}%
\pgfpathlineto{\pgfqpoint{1.268547in}{5.084012in}}%
\pgfpathlineto{\pgfqpoint{1.263214in}{5.084012in}}%
\pgfpathlineto{\pgfqpoint{1.257881in}{5.084012in}}%
\pgfpathlineto{\pgfqpoint{1.252547in}{5.084012in}}%
\pgfpathlineto{\pgfqpoint{1.247214in}{5.084012in}}%
\pgfpathlineto{\pgfqpoint{1.241881in}{5.084012in}}%
\pgfpathlineto{\pgfqpoint{1.236548in}{5.084012in}}%
\pgfpathlineto{\pgfqpoint{1.231215in}{5.084012in}}%
\pgfpathlineto{\pgfqpoint{1.225881in}{5.084012in}}%
\pgfpathlineto{\pgfqpoint{1.220548in}{5.084012in}}%
\pgfpathlineto{\pgfqpoint{1.215215in}{5.084012in}}%
\pgfpathlineto{\pgfqpoint{1.209882in}{5.084012in}}%
\pgfpathlineto{\pgfqpoint{1.204549in}{5.084012in}}%
\pgfpathlineto{\pgfqpoint{1.199215in}{5.084012in}}%
\pgfpathlineto{\pgfqpoint{1.193882in}{5.084012in}}%
\pgfpathlineto{\pgfqpoint{1.188549in}{5.084012in}}%
\pgfpathlineto{\pgfqpoint{1.183216in}{5.084012in}}%
\pgfpathlineto{\pgfqpoint{1.177882in}{5.084012in}}%
\pgfpathlineto{\pgfqpoint{1.172549in}{5.084012in}}%
\pgfpathlineto{\pgfqpoint{1.167216in}{5.084012in}}%
\pgfpathlineto{\pgfqpoint{1.161883in}{5.084012in}}%
\pgfpathlineto{\pgfqpoint{1.156550in}{5.084012in}}%
\pgfpathlineto{\pgfqpoint{1.151216in}{5.084012in}}%
\pgfpathlineto{\pgfqpoint{1.145883in}{5.084012in}}%
\pgfpathlineto{\pgfqpoint{1.140550in}{5.084012in}}%
\pgfpathlineto{\pgfqpoint{1.135217in}{5.084012in}}%
\pgfpathlineto{\pgfqpoint{1.129883in}{5.084012in}}%
\pgfpathlineto{\pgfqpoint{1.124550in}{5.084012in}}%
\pgfpathlineto{\pgfqpoint{1.119217in}{5.084012in}}%
\pgfpathlineto{\pgfqpoint{1.113884in}{5.084012in}}%
\pgfpathlineto{\pgfqpoint{1.108551in}{5.084012in}}%
\pgfpathlineto{\pgfqpoint{1.103217in}{5.084012in}}%
\pgfpathlineto{\pgfqpoint{1.097884in}{5.084012in}}%
\pgfpathlineto{\pgfqpoint{1.092551in}{5.084012in}}%
\pgfpathlineto{\pgfqpoint{1.087218in}{5.084012in}}%
\pgfpathlineto{\pgfqpoint{1.081885in}{5.084012in}}%
\pgfpathlineto{\pgfqpoint{1.076551in}{5.084012in}}%
\pgfpathlineto{\pgfqpoint{1.071218in}{5.084012in}}%
\pgfpathlineto{\pgfqpoint{1.065885in}{5.084012in}}%
\pgfpathlineto{\pgfqpoint{1.060552in}{5.084012in}}%
\pgfpathlineto{\pgfqpoint{1.055218in}{5.084012in}}%
\pgfpathlineto{\pgfqpoint{1.049885in}{5.084012in}}%
\pgfpathlineto{\pgfqpoint{1.044552in}{5.084012in}}%
\pgfpathlineto{\pgfqpoint{1.039219in}{5.084012in}}%
\pgfpathlineto{\pgfqpoint{1.033886in}{5.084012in}}%
\pgfpathlineto{\pgfqpoint{1.028552in}{5.084012in}}%
\pgfpathlineto{\pgfqpoint{1.023219in}{5.084012in}}%
\pgfpathlineto{\pgfqpoint{1.017886in}{5.084012in}}%
\pgfpathlineto{\pgfqpoint{1.012553in}{5.084012in}}%
\pgfpathlineto{\pgfqpoint{1.007220in}{5.084012in}}%
\pgfpathlineto{\pgfqpoint{1.001886in}{5.084012in}}%
\pgfpathlineto{\pgfqpoint{0.996553in}{5.084012in}}%
\pgfpathlineto{\pgfqpoint{0.991220in}{5.084012in}}%
\pgfpathlineto{\pgfqpoint{0.985887in}{5.084012in}}%
\pgfpathlineto{\pgfqpoint{0.980553in}{5.084012in}}%
\pgfpathlineto{\pgfqpoint{0.975220in}{5.084012in}}%
\pgfpathlineto{\pgfqpoint{0.969887in}{5.084012in}}%
\pgfpathlineto{\pgfqpoint{0.964554in}{5.084012in}}%
\pgfpathlineto{\pgfqpoint{0.959221in}{5.084012in}}%
\pgfpathlineto{\pgfqpoint{0.953887in}{5.084012in}}%
\pgfpathlineto{\pgfqpoint{0.948554in}{5.084012in}}%
\pgfpathlineto{\pgfqpoint{0.943221in}{5.084012in}}%
\pgfpathlineto{\pgfqpoint{0.937888in}{5.084012in}}%
\pgfpathlineto{\pgfqpoint{0.932555in}{5.084012in}}%
\pgfpathlineto{\pgfqpoint{0.927221in}{5.084012in}}%
\pgfpathlineto{\pgfqpoint{0.921888in}{5.084012in}}%
\pgfpathlineto{\pgfqpoint{0.916555in}{5.084012in}}%
\pgfpathlineto{\pgfqpoint{0.911222in}{5.084012in}}%
\pgfpathlineto{\pgfqpoint{0.905888in}{5.084012in}}%
\pgfpathlineto{\pgfqpoint{0.900555in}{5.084012in}}%
\pgfpathlineto{\pgfqpoint{0.895222in}{5.084012in}}%
\pgfpathlineto{\pgfqpoint{0.889889in}{5.084012in}}%
\pgfpathlineto{\pgfqpoint{0.884556in}{5.084012in}}%
\pgfpathlineto{\pgfqpoint{0.879222in}{5.084012in}}%
\pgfpathlineto{\pgfqpoint{0.873889in}{5.084012in}}%
\pgfpathlineto{\pgfqpoint{0.868556in}{5.084012in}}%
\pgfpathlineto{\pgfqpoint{0.863223in}{5.084012in}}%
\pgfpathlineto{\pgfqpoint{0.857889in}{5.084012in}}%
\pgfpathlineto{\pgfqpoint{0.852556in}{5.084012in}}%
\pgfpathlineto{\pgfqpoint{0.847223in}{5.084012in}}%
\pgfpathlineto{\pgfqpoint{0.847223in}{5.084012in}}%
\pgfpathclose%
\pgfusepath{stroke,fill}%
}%
\begin{pgfscope}%
\pgfsys@transformshift{0.000000in}{0.000000in}%
\pgfsys@useobject{currentmarker}{}%
\end{pgfscope}%
\end{pgfscope}%
\begin{pgfscope}%
\pgfpathrectangle{\pgfqpoint{0.847223in}{0.554012in}}{\pgfqpoint{6.200000in}{4.530000in}}%
\pgfusepath{clip}%
\pgfsetbuttcap%
\pgfsetroundjoin%
\definecolor{currentfill}{rgb}{0.121569,0.466667,0.705882}%
\pgfsetfillcolor{currentfill}%
\pgfsetlinewidth{1.003750pt}%
\definecolor{currentstroke}{rgb}{0.121569,0.466667,0.705882}%
\pgfsetstrokecolor{currentstroke}%
\pgfsetdash{}{0pt}%
\pgfsys@defobject{currentmarker}{\pgfqpoint{-0.114394in}{-0.097310in}}{\pgfqpoint{0.114394in}{0.120281in}}{%
\pgfpathmoveto{\pgfqpoint{0.000000in}{0.120281in}}%
\pgfpathlineto{\pgfqpoint{-0.027005in}{0.037169in}}%
\pgfpathlineto{\pgfqpoint{-0.114394in}{0.037169in}}%
\pgfpathlineto{\pgfqpoint{-0.043695in}{-0.014197in}}%
\pgfpathlineto{\pgfqpoint{-0.070700in}{-0.097310in}}%
\pgfpathlineto{\pgfqpoint{-0.000000in}{-0.045943in}}%
\pgfpathlineto{\pgfqpoint{0.070700in}{-0.097310in}}%
\pgfpathlineto{\pgfqpoint{0.043695in}{-0.014197in}}%
\pgfpathlineto{\pgfqpoint{0.114394in}{0.037169in}}%
\pgfpathlineto{\pgfqpoint{0.027005in}{0.037169in}}%
\pgfpathlineto{\pgfqpoint{0.000000in}{0.120281in}}%
\pgfpathclose%
\pgfusepath{stroke,fill}%
}%
\begin{pgfscope}%
\pgfsys@transformshift{0.978061in}{0.658308in}%
\pgfsys@useobject{currentmarker}{}%
\end{pgfscope}%
\end{pgfscope}%
\begin{pgfscope}%
\pgfpathrectangle{\pgfqpoint{0.847223in}{0.554012in}}{\pgfqpoint{6.200000in}{4.530000in}}%
\pgfusepath{clip}%
\pgfsetbuttcap%
\pgfsetroundjoin%
\definecolor{currentfill}{rgb}{1.000000,0.498039,0.054902}%
\pgfsetfillcolor{currentfill}%
\pgfsetlinewidth{1.003750pt}%
\definecolor{currentstroke}{rgb}{1.000000,0.498039,0.054902}%
\pgfsetstrokecolor{currentstroke}%
\pgfsetdash{}{0pt}%
\pgfsys@defobject{currentmarker}{\pgfqpoint{-0.098209in}{-0.098209in}}{\pgfqpoint{0.098209in}{0.098209in}}{%
\pgfpathmoveto{\pgfqpoint{-0.098209in}{-0.098209in}}%
\pgfpathlineto{\pgfqpoint{0.098209in}{-0.098209in}}%
\pgfpathlineto{\pgfqpoint{0.098209in}{0.098209in}}%
\pgfpathlineto{\pgfqpoint{-0.098209in}{0.098209in}}%
\pgfpathlineto{\pgfqpoint{-0.098209in}{-0.098209in}}%
\pgfpathclose%
\pgfusepath{stroke,fill}%
}%
\begin{pgfscope}%
\pgfsys@transformshift{6.711056in}{4.836653in}%
\pgfsys@useobject{currentmarker}{}%
\end{pgfscope}%
\end{pgfscope}%
\begin{pgfscope}%
\pgfsetbuttcap%
\pgfsetroundjoin%
\definecolor{currentfill}{rgb}{0.000000,0.000000,0.000000}%
\pgfsetfillcolor{currentfill}%
\pgfsetlinewidth{0.803000pt}%
\definecolor{currentstroke}{rgb}{0.000000,0.000000,0.000000}%
\pgfsetstrokecolor{currentstroke}%
\pgfsetdash{}{0pt}%
\pgfsys@defobject{currentmarker}{\pgfqpoint{0.000000in}{-0.048611in}}{\pgfqpoint{0.000000in}{0.000000in}}{%
\pgfpathmoveto{\pgfqpoint{0.000000in}{0.000000in}}%
\pgfpathlineto{\pgfqpoint{0.000000in}{-0.048611in}}%
\pgfusepath{stroke,fill}%
}%
\begin{pgfscope}%
\pgfsys@transformshift{1.632248in}{0.554012in}%
\pgfsys@useobject{currentmarker}{}%
\end{pgfscope}%
\end{pgfscope}%
\begin{pgfscope}%
\definecolor{textcolor}{rgb}{0.000000,0.000000,0.000000}%
\pgfsetstrokecolor{textcolor}%
\pgfsetfillcolor{textcolor}%
\pgftext[x=1.632248in,y=0.456790in,,top]{\color{textcolor}\rmfamily\fontsize{10.000000}{12.000000}\selectfont \(\displaystyle {0.01}\)}%
\end{pgfscope}%
\begin{pgfscope}%
\pgfsetbuttcap%
\pgfsetroundjoin%
\definecolor{currentfill}{rgb}{0.000000,0.000000,0.000000}%
\pgfsetfillcolor{currentfill}%
\pgfsetlinewidth{0.803000pt}%
\definecolor{currentstroke}{rgb}{0.000000,0.000000,0.000000}%
\pgfsetstrokecolor{currentstroke}%
\pgfsetdash{}{0pt}%
\pgfsys@defobject{currentmarker}{\pgfqpoint{0.000000in}{-0.048611in}}{\pgfqpoint{0.000000in}{0.000000in}}{%
\pgfpathmoveto{\pgfqpoint{0.000000in}{0.000000in}}%
\pgfpathlineto{\pgfqpoint{0.000000in}{-0.048611in}}%
\pgfusepath{stroke,fill}%
}%
\begin{pgfscope}%
\pgfsys@transformshift{2.940624in}{0.554012in}%
\pgfsys@useobject{currentmarker}{}%
\end{pgfscope}%
\end{pgfscope}%
\begin{pgfscope}%
\definecolor{textcolor}{rgb}{0.000000,0.000000,0.000000}%
\pgfsetstrokecolor{textcolor}%
\pgfsetfillcolor{textcolor}%
\pgftext[x=2.940624in,y=0.456790in,,top]{\color{textcolor}\rmfamily\fontsize{10.000000}{12.000000}\selectfont \(\displaystyle {0.02}\)}%
\end{pgfscope}%
\begin{pgfscope}%
\pgfsetbuttcap%
\pgfsetroundjoin%
\definecolor{currentfill}{rgb}{0.000000,0.000000,0.000000}%
\pgfsetfillcolor{currentfill}%
\pgfsetlinewidth{0.803000pt}%
\definecolor{currentstroke}{rgb}{0.000000,0.000000,0.000000}%
\pgfsetstrokecolor{currentstroke}%
\pgfsetdash{}{0pt}%
\pgfsys@defobject{currentmarker}{\pgfqpoint{0.000000in}{-0.048611in}}{\pgfqpoint{0.000000in}{0.000000in}}{%
\pgfpathmoveto{\pgfqpoint{0.000000in}{0.000000in}}%
\pgfpathlineto{\pgfqpoint{0.000000in}{-0.048611in}}%
\pgfusepath{stroke,fill}%
}%
\begin{pgfscope}%
\pgfsys@transformshift{4.248999in}{0.554012in}%
\pgfsys@useobject{currentmarker}{}%
\end{pgfscope}%
\end{pgfscope}%
\begin{pgfscope}%
\definecolor{textcolor}{rgb}{0.000000,0.000000,0.000000}%
\pgfsetstrokecolor{textcolor}%
\pgfsetfillcolor{textcolor}%
\pgftext[x=4.248999in,y=0.456790in,,top]{\color{textcolor}\rmfamily\fontsize{10.000000}{12.000000}\selectfont \(\displaystyle {0.03}\)}%
\end{pgfscope}%
\begin{pgfscope}%
\pgfsetbuttcap%
\pgfsetroundjoin%
\definecolor{currentfill}{rgb}{0.000000,0.000000,0.000000}%
\pgfsetfillcolor{currentfill}%
\pgfsetlinewidth{0.803000pt}%
\definecolor{currentstroke}{rgb}{0.000000,0.000000,0.000000}%
\pgfsetstrokecolor{currentstroke}%
\pgfsetdash{}{0pt}%
\pgfsys@defobject{currentmarker}{\pgfqpoint{0.000000in}{-0.048611in}}{\pgfqpoint{0.000000in}{0.000000in}}{%
\pgfpathmoveto{\pgfqpoint{0.000000in}{0.000000in}}%
\pgfpathlineto{\pgfqpoint{0.000000in}{-0.048611in}}%
\pgfusepath{stroke,fill}%
}%
\begin{pgfscope}%
\pgfsys@transformshift{5.557374in}{0.554012in}%
\pgfsys@useobject{currentmarker}{}%
\end{pgfscope}%
\end{pgfscope}%
\begin{pgfscope}%
\definecolor{textcolor}{rgb}{0.000000,0.000000,0.000000}%
\pgfsetstrokecolor{textcolor}%
\pgfsetfillcolor{textcolor}%
\pgftext[x=5.557374in,y=0.456790in,,top]{\color{textcolor}\rmfamily\fontsize{10.000000}{12.000000}\selectfont \(\displaystyle {0.04}\)}%
\end{pgfscope}%
\begin{pgfscope}%
\pgfsetbuttcap%
\pgfsetroundjoin%
\definecolor{currentfill}{rgb}{0.000000,0.000000,0.000000}%
\pgfsetfillcolor{currentfill}%
\pgfsetlinewidth{0.803000pt}%
\definecolor{currentstroke}{rgb}{0.000000,0.000000,0.000000}%
\pgfsetstrokecolor{currentstroke}%
\pgfsetdash{}{0pt}%
\pgfsys@defobject{currentmarker}{\pgfqpoint{0.000000in}{-0.048611in}}{\pgfqpoint{0.000000in}{0.000000in}}{%
\pgfpathmoveto{\pgfqpoint{0.000000in}{0.000000in}}%
\pgfpathlineto{\pgfqpoint{0.000000in}{-0.048611in}}%
\pgfusepath{stroke,fill}%
}%
\begin{pgfscope}%
\pgfsys@transformshift{6.865750in}{0.554012in}%
\pgfsys@useobject{currentmarker}{}%
\end{pgfscope}%
\end{pgfscope}%
\begin{pgfscope}%
\definecolor{textcolor}{rgb}{0.000000,0.000000,0.000000}%
\pgfsetstrokecolor{textcolor}%
\pgfsetfillcolor{textcolor}%
\pgftext[x=6.865750in,y=0.456790in,,top]{\color{textcolor}\rmfamily\fontsize{10.000000}{12.000000}\selectfont \(\displaystyle {0.05}\)}%
\end{pgfscope}%
\begin{pgfscope}%
\definecolor{textcolor}{rgb}{0.000000,0.000000,0.000000}%
\pgfsetstrokecolor{textcolor}%
\pgfsetfillcolor{textcolor}%
\pgftext[x=3.947223in,y=0.277777in,,top]{\color{textcolor}\rmfamily\fontsize{14.000000}{16.800000}\selectfont f1}%
\end{pgfscope}%
\begin{pgfscope}%
\pgfsetbuttcap%
\pgfsetroundjoin%
\definecolor{currentfill}{rgb}{0.000000,0.000000,0.000000}%
\pgfsetfillcolor{currentfill}%
\pgfsetlinewidth{0.803000pt}%
\definecolor{currentstroke}{rgb}{0.000000,0.000000,0.000000}%
\pgfsetstrokecolor{currentstroke}%
\pgfsetdash{}{0pt}%
\pgfsys@defobject{currentmarker}{\pgfqpoint{-0.048611in}{0.000000in}}{\pgfqpoint{-0.000000in}{0.000000in}}{%
\pgfpathmoveto{\pgfqpoint{-0.000000in}{0.000000in}}%
\pgfpathlineto{\pgfqpoint{-0.048611in}{0.000000in}}%
\pgfusepath{stroke,fill}%
}%
\begin{pgfscope}%
\pgfsys@transformshift{0.847223in}{1.126265in}%
\pgfsys@useobject{currentmarker}{}%
\end{pgfscope}%
\end{pgfscope}%
\begin{pgfscope}%
\definecolor{textcolor}{rgb}{0.000000,0.000000,0.000000}%
\pgfsetstrokecolor{textcolor}%
\pgfsetfillcolor{textcolor}%
\pgftext[x=0.402777in, y=1.078040in, left, base]{\color{textcolor}\rmfamily\fontsize{10.000000}{12.000000}\selectfont \(\displaystyle {20000}\)}%
\end{pgfscope}%
\begin{pgfscope}%
\pgfsetbuttcap%
\pgfsetroundjoin%
\definecolor{currentfill}{rgb}{0.000000,0.000000,0.000000}%
\pgfsetfillcolor{currentfill}%
\pgfsetlinewidth{0.803000pt}%
\definecolor{currentstroke}{rgb}{0.000000,0.000000,0.000000}%
\pgfsetstrokecolor{currentstroke}%
\pgfsetdash{}{0pt}%
\pgfsys@defobject{currentmarker}{\pgfqpoint{-0.048611in}{0.000000in}}{\pgfqpoint{-0.000000in}{0.000000in}}{%
\pgfpathmoveto{\pgfqpoint{-0.000000in}{0.000000in}}%
\pgfpathlineto{\pgfqpoint{-0.048611in}{0.000000in}}%
\pgfusepath{stroke,fill}%
}%
\begin{pgfscope}%
\pgfsys@transformshift{0.847223in}{2.115702in}%
\pgfsys@useobject{currentmarker}{}%
\end{pgfscope}%
\end{pgfscope}%
\begin{pgfscope}%
\definecolor{textcolor}{rgb}{0.000000,0.000000,0.000000}%
\pgfsetstrokecolor{textcolor}%
\pgfsetfillcolor{textcolor}%
\pgftext[x=0.402777in, y=2.067477in, left, base]{\color{textcolor}\rmfamily\fontsize{10.000000}{12.000000}\selectfont \(\displaystyle {40000}\)}%
\end{pgfscope}%
\begin{pgfscope}%
\pgfsetbuttcap%
\pgfsetroundjoin%
\definecolor{currentfill}{rgb}{0.000000,0.000000,0.000000}%
\pgfsetfillcolor{currentfill}%
\pgfsetlinewidth{0.803000pt}%
\definecolor{currentstroke}{rgb}{0.000000,0.000000,0.000000}%
\pgfsetstrokecolor{currentstroke}%
\pgfsetdash{}{0pt}%
\pgfsys@defobject{currentmarker}{\pgfqpoint{-0.048611in}{0.000000in}}{\pgfqpoint{-0.000000in}{0.000000in}}{%
\pgfpathmoveto{\pgfqpoint{-0.000000in}{0.000000in}}%
\pgfpathlineto{\pgfqpoint{-0.048611in}{0.000000in}}%
\pgfusepath{stroke,fill}%
}%
\begin{pgfscope}%
\pgfsys@transformshift{0.847223in}{3.105139in}%
\pgfsys@useobject{currentmarker}{}%
\end{pgfscope}%
\end{pgfscope}%
\begin{pgfscope}%
\definecolor{textcolor}{rgb}{0.000000,0.000000,0.000000}%
\pgfsetstrokecolor{textcolor}%
\pgfsetfillcolor{textcolor}%
\pgftext[x=0.402777in, y=3.056913in, left, base]{\color{textcolor}\rmfamily\fontsize{10.000000}{12.000000}\selectfont \(\displaystyle {60000}\)}%
\end{pgfscope}%
\begin{pgfscope}%
\pgfsetbuttcap%
\pgfsetroundjoin%
\definecolor{currentfill}{rgb}{0.000000,0.000000,0.000000}%
\pgfsetfillcolor{currentfill}%
\pgfsetlinewidth{0.803000pt}%
\definecolor{currentstroke}{rgb}{0.000000,0.000000,0.000000}%
\pgfsetstrokecolor{currentstroke}%
\pgfsetdash{}{0pt}%
\pgfsys@defobject{currentmarker}{\pgfqpoint{-0.048611in}{0.000000in}}{\pgfqpoint{-0.000000in}{0.000000in}}{%
\pgfpathmoveto{\pgfqpoint{-0.000000in}{0.000000in}}%
\pgfpathlineto{\pgfqpoint{-0.048611in}{0.000000in}}%
\pgfusepath{stroke,fill}%
}%
\begin{pgfscope}%
\pgfsys@transformshift{0.847223in}{4.094575in}%
\pgfsys@useobject{currentmarker}{}%
\end{pgfscope}%
\end{pgfscope}%
\begin{pgfscope}%
\definecolor{textcolor}{rgb}{0.000000,0.000000,0.000000}%
\pgfsetstrokecolor{textcolor}%
\pgfsetfillcolor{textcolor}%
\pgftext[x=0.402777in, y=4.046350in, left, base]{\color{textcolor}\rmfamily\fontsize{10.000000}{12.000000}\selectfont \(\displaystyle {80000}\)}%
\end{pgfscope}%
\begin{pgfscope}%
\pgfsetbuttcap%
\pgfsetroundjoin%
\definecolor{currentfill}{rgb}{0.000000,0.000000,0.000000}%
\pgfsetfillcolor{currentfill}%
\pgfsetlinewidth{0.803000pt}%
\definecolor{currentstroke}{rgb}{0.000000,0.000000,0.000000}%
\pgfsetstrokecolor{currentstroke}%
\pgfsetdash{}{0pt}%
\pgfsys@defobject{currentmarker}{\pgfqpoint{-0.048611in}{0.000000in}}{\pgfqpoint{-0.000000in}{0.000000in}}{%
\pgfpathmoveto{\pgfqpoint{-0.000000in}{0.000000in}}%
\pgfpathlineto{\pgfqpoint{-0.048611in}{0.000000in}}%
\pgfusepath{stroke,fill}%
}%
\begin{pgfscope}%
\pgfsys@transformshift{0.847223in}{5.084012in}%
\pgfsys@useobject{currentmarker}{}%
\end{pgfscope}%
\end{pgfscope}%
\begin{pgfscope}%
\definecolor{textcolor}{rgb}{0.000000,0.000000,0.000000}%
\pgfsetstrokecolor{textcolor}%
\pgfsetfillcolor{textcolor}%
\pgftext[x=0.333333in, y=5.035787in, left, base]{\color{textcolor}\rmfamily\fontsize{10.000000}{12.000000}\selectfont \(\displaystyle {100000}\)}%
\end{pgfscope}%
\begin{pgfscope}%
\definecolor{textcolor}{rgb}{0.000000,0.000000,0.000000}%
\pgfsetstrokecolor{textcolor}%
\pgfsetfillcolor{textcolor}%
\pgftext[x=0.277777in,y=2.819012in,,bottom,rotate=90.000000]{\color{textcolor}\rmfamily\fontsize{14.000000}{16.800000}\selectfont f2}%
\end{pgfscope}%
\begin{pgfscope}%
\pgfpathrectangle{\pgfqpoint{0.847223in}{0.554012in}}{\pgfqpoint{6.200000in}{4.530000in}}%
\pgfusepath{clip}%
\pgfsetrectcap%
\pgfsetroundjoin%
\pgfsetlinewidth{2.509375pt}%
\definecolor{currentstroke}{rgb}{0.000000,0.000000,0.000000}%
\pgfsetstrokecolor{currentstroke}%
\pgfsetdash{}{0pt}%
\pgfpathmoveto{\pgfqpoint{0.847223in}{5.084012in}}%
\pgfpathlineto{\pgfqpoint{0.857889in}{4.985197in}}%
\pgfpathlineto{\pgfqpoint{0.868556in}{4.890252in}}%
\pgfpathlineto{\pgfqpoint{0.879222in}{4.798955in}}%
\pgfpathlineto{\pgfqpoint{0.889889in}{4.711098in}}%
\pgfpathlineto{\pgfqpoint{0.900555in}{4.626491in}}%
\pgfpathlineto{\pgfqpoint{0.911222in}{4.544958in}}%
\pgfpathlineto{\pgfqpoint{0.921888in}{4.466333in}}%
\pgfpathlineto{\pgfqpoint{0.932555in}{4.390463in}}%
\pgfpathlineto{\pgfqpoint{0.943221in}{4.317207in}}%
\pgfpathlineto{\pgfqpoint{0.953887in}{4.246431in}}%
\pgfpathlineto{\pgfqpoint{0.964554in}{4.178012in}}%
\pgfpathlineto{\pgfqpoint{0.975220in}{4.111834in}}%
\pgfpathlineto{\pgfqpoint{0.985887in}{4.047788in}}%
\pgfpathlineto{\pgfqpoint{0.996553in}{3.985774in}}%
\pgfpathlineto{\pgfqpoint{1.007220in}{3.925695in}}%
\pgfpathlineto{\pgfqpoint{1.017886in}{3.867463in}}%
\pgfpathlineto{\pgfqpoint{1.028552in}{3.810994in}}%
\pgfpathlineto{\pgfqpoint{1.039219in}{3.756209in}}%
\pgfpathlineto{\pgfqpoint{1.049885in}{3.703034in}}%
\pgfpathlineto{\pgfqpoint{1.060552in}{3.651399in}}%
\pgfpathlineto{\pgfqpoint{1.071218in}{3.601237in}}%
\pgfpathlineto{\pgfqpoint{1.081885in}{3.552487in}}%
\pgfpathlineto{\pgfqpoint{1.092551in}{3.505091in}}%
\pgfpathlineto{\pgfqpoint{1.103217in}{3.458991in}}%
\pgfpathlineto{\pgfqpoint{1.113884in}{3.414137in}}%
\pgfpathlineto{\pgfqpoint{1.124550in}{3.370477in}}%
\pgfpathlineto{\pgfqpoint{1.135217in}{3.327965in}}%
\pgfpathlineto{\pgfqpoint{1.145883in}{3.286557in}}%
\pgfpathlineto{\pgfqpoint{1.156550in}{3.246210in}}%
\pgfpathlineto{\pgfqpoint{1.167216in}{3.206883in}}%
\pgfpathlineto{\pgfqpoint{1.177882in}{3.168538in}}%
\pgfpathlineto{\pgfqpoint{1.188549in}{3.131140in}}%
\pgfpathlineto{\pgfqpoint{1.199215in}{3.094653in}}%
\pgfpathlineto{\pgfqpoint{1.215215in}{3.041560in}}%
\pgfpathlineto{\pgfqpoint{1.231215in}{2.990339in}}%
\pgfpathlineto{\pgfqpoint{1.247214in}{2.940894in}}%
\pgfpathlineto{\pgfqpoint{1.263214in}{2.893132in}}%
\pgfpathlineto{\pgfqpoint{1.279214in}{2.846971in}}%
\pgfpathlineto{\pgfqpoint{1.295213in}{2.802330in}}%
\pgfpathlineto{\pgfqpoint{1.311213in}{2.759136in}}%
\pgfpathlineto{\pgfqpoint{1.327212in}{2.717320in}}%
\pgfpathlineto{\pgfqpoint{1.343212in}{2.676816in}}%
\pgfpathlineto{\pgfqpoint{1.359212in}{2.637565in}}%
\pgfpathlineto{\pgfqpoint{1.375211in}{2.599507in}}%
\pgfpathlineto{\pgfqpoint{1.391211in}{2.562591in}}%
\pgfpathlineto{\pgfqpoint{1.407211in}{2.526766in}}%
\pgfpathlineto{\pgfqpoint{1.423210in}{2.491983in}}%
\pgfpathlineto{\pgfqpoint{1.439210in}{2.458198in}}%
\pgfpathlineto{\pgfqpoint{1.455210in}{2.425368in}}%
\pgfpathlineto{\pgfqpoint{1.471209in}{2.393455in}}%
\pgfpathlineto{\pgfqpoint{1.487209in}{2.362419in}}%
\pgfpathlineto{\pgfqpoint{1.503209in}{2.332225in}}%
\pgfpathlineto{\pgfqpoint{1.519208in}{2.302839in}}%
\pgfpathlineto{\pgfqpoint{1.535208in}{2.274230in}}%
\pgfpathlineto{\pgfqpoint{1.551208in}{2.246367in}}%
\pgfpathlineto{\pgfqpoint{1.567207in}{2.219220in}}%
\pgfpathlineto{\pgfqpoint{1.583207in}{2.192764in}}%
\pgfpathlineto{\pgfqpoint{1.599206in}{2.166971in}}%
\pgfpathlineto{\pgfqpoint{1.615206in}{2.141818in}}%
\pgfpathlineto{\pgfqpoint{1.631206in}{2.117280in}}%
\pgfpathlineto{\pgfqpoint{1.647205in}{2.093335in}}%
\pgfpathlineto{\pgfqpoint{1.663205in}{2.069963in}}%
\pgfpathlineto{\pgfqpoint{1.679205in}{2.047142in}}%
\pgfpathlineto{\pgfqpoint{1.695204in}{2.024854in}}%
\pgfpathlineto{\pgfqpoint{1.711204in}{2.003080in}}%
\pgfpathlineto{\pgfqpoint{1.727204in}{1.981803in}}%
\pgfpathlineto{\pgfqpoint{1.743203in}{1.961005in}}%
\pgfpathlineto{\pgfqpoint{1.759203in}{1.940671in}}%
\pgfpathlineto{\pgfqpoint{1.780536in}{1.914254in}}%
\pgfpathlineto{\pgfqpoint{1.801869in}{1.888599in}}%
\pgfpathlineto{\pgfqpoint{1.823202in}{1.863674in}}%
\pgfpathlineto{\pgfqpoint{1.844534in}{1.839449in}}%
\pgfpathlineto{\pgfqpoint{1.865867in}{1.815894in}}%
\pgfpathlineto{\pgfqpoint{1.887200in}{1.792982in}}%
\pgfpathlineto{\pgfqpoint{1.908533in}{1.770686in}}%
\pgfpathlineto{\pgfqpoint{1.929866in}{1.748983in}}%
\pgfpathlineto{\pgfqpoint{1.951199in}{1.727849in}}%
\pgfpathlineto{\pgfqpoint{1.972532in}{1.707262in}}%
\pgfpathlineto{\pgfqpoint{1.993864in}{1.687201in}}%
\pgfpathlineto{\pgfqpoint{2.015197in}{1.667646in}}%
\pgfpathlineto{\pgfqpoint{2.036530in}{1.648578in}}%
\pgfpathlineto{\pgfqpoint{2.057863in}{1.629980in}}%
\pgfpathlineto{\pgfqpoint{2.079196in}{1.611833in}}%
\pgfpathlineto{\pgfqpoint{2.100529in}{1.594122in}}%
\pgfpathlineto{\pgfqpoint{2.121862in}{1.576832in}}%
\pgfpathlineto{\pgfqpoint{2.143194in}{1.559947in}}%
\pgfpathlineto{\pgfqpoint{2.164527in}{1.543453in}}%
\pgfpathlineto{\pgfqpoint{2.191193in}{1.523366in}}%
\pgfpathlineto{\pgfqpoint{2.217859in}{1.503844in}}%
\pgfpathlineto{\pgfqpoint{2.244526in}{1.484865in}}%
\pgfpathlineto{\pgfqpoint{2.271192in}{1.466405in}}%
\pgfpathlineto{\pgfqpoint{2.297858in}{1.448444in}}%
\pgfpathlineto{\pgfqpoint{2.324524in}{1.430962in}}%
\pgfpathlineto{\pgfqpoint{2.351190in}{1.413940in}}%
\pgfpathlineto{\pgfqpoint{2.377856in}{1.397360in}}%
\pgfpathlineto{\pgfqpoint{2.404522in}{1.381204in}}%
\pgfpathlineto{\pgfqpoint{2.431188in}{1.365458in}}%
\pgfpathlineto{\pgfqpoint{2.457854in}{1.350105in}}%
\pgfpathlineto{\pgfqpoint{2.484520in}{1.335131in}}%
\pgfpathlineto{\pgfqpoint{2.511186in}{1.320522in}}%
\pgfpathlineto{\pgfqpoint{2.543186in}{1.303455in}}%
\pgfpathlineto{\pgfqpoint{2.575185in}{1.286873in}}%
\pgfpathlineto{\pgfqpoint{2.607184in}{1.270756in}}%
\pgfpathlineto{\pgfqpoint{2.639184in}{1.255084in}}%
\pgfpathlineto{\pgfqpoint{2.671183in}{1.239840in}}%
\pgfpathlineto{\pgfqpoint{2.703182in}{1.225005in}}%
\pgfpathlineto{\pgfqpoint{2.735181in}{1.210565in}}%
\pgfpathlineto{\pgfqpoint{2.767181in}{1.196502in}}%
\pgfpathlineto{\pgfqpoint{2.799180in}{1.182804in}}%
\pgfpathlineto{\pgfqpoint{2.836512in}{1.167263in}}%
\pgfpathlineto{\pgfqpoint{2.873845in}{1.152177in}}%
\pgfpathlineto{\pgfqpoint{2.911178in}{1.137526in}}%
\pgfpathlineto{\pgfqpoint{2.948510in}{1.123292in}}%
\pgfpathlineto{\pgfqpoint{2.985843in}{1.109458in}}%
\pgfpathlineto{\pgfqpoint{3.023175in}{1.096006in}}%
\pgfpathlineto{\pgfqpoint{3.060508in}{1.082921in}}%
\pgfpathlineto{\pgfqpoint{3.097840in}{1.070188in}}%
\pgfpathlineto{\pgfqpoint{3.140506in}{1.056050in}}%
\pgfpathlineto{\pgfqpoint{3.183172in}{1.042334in}}%
\pgfpathlineto{\pgfqpoint{3.225837in}{1.029021in}}%
\pgfpathlineto{\pgfqpoint{3.268503in}{1.016093in}}%
\pgfpathlineto{\pgfqpoint{3.311169in}{1.003535in}}%
\pgfpathlineto{\pgfqpoint{3.353834in}{0.991331in}}%
\pgfpathlineto{\pgfqpoint{3.401833in}{0.978006in}}%
\pgfpathlineto{\pgfqpoint{3.449832in}{0.965089in}}%
\pgfpathlineto{\pgfqpoint{3.497831in}{0.952564in}}%
\pgfpathlineto{\pgfqpoint{3.545830in}{0.940411in}}%
\pgfpathlineto{\pgfqpoint{3.593829in}{0.928616in}}%
\pgfpathlineto{\pgfqpoint{3.647161in}{0.915909in}}%
\pgfpathlineto{\pgfqpoint{3.700493in}{0.903604in}}%
\pgfpathlineto{\pgfqpoint{3.753826in}{0.891681in}}%
\pgfpathlineto{\pgfqpoint{3.807158in}{0.880124in}}%
\pgfpathlineto{\pgfqpoint{3.865823in}{0.867813in}}%
\pgfpathlineto{\pgfqpoint{3.924488in}{0.855903in}}%
\pgfpathlineto{\pgfqpoint{3.983154in}{0.844375in}}%
\pgfpathlineto{\pgfqpoint{4.041819in}{0.833210in}}%
\pgfpathlineto{\pgfqpoint{4.105818in}{0.821426in}}%
\pgfpathlineto{\pgfqpoint{4.169816in}{0.810034in}}%
\pgfpathlineto{\pgfqpoint{4.233815in}{0.799015in}}%
\pgfpathlineto{\pgfqpoint{4.297814in}{0.788351in}}%
\pgfpathlineto{\pgfqpoint{4.367145in}{0.777179in}}%
\pgfpathlineto{\pgfqpoint{4.436477in}{0.766383in}}%
\pgfpathlineto{\pgfqpoint{4.505809in}{0.755946in}}%
\pgfpathlineto{\pgfqpoint{4.580474in}{0.745086in}}%
\pgfpathlineto{\pgfqpoint{4.655139in}{0.734601in}}%
\pgfpathlineto{\pgfqpoint{4.735137in}{0.723760in}}%
\pgfpathlineto{\pgfqpoint{4.815136in}{0.713306in}}%
\pgfpathlineto{\pgfqpoint{4.895134in}{0.703217in}}%
\pgfpathlineto{\pgfqpoint{4.980465in}{0.692838in}}%
\pgfpathlineto{\pgfqpoint{5.065797in}{0.682833in}}%
\pgfpathlineto{\pgfqpoint{5.156461in}{0.672589in}}%
\pgfpathlineto{\pgfqpoint{5.247126in}{0.662723in}}%
\pgfpathlineto{\pgfqpoint{5.343124in}{0.652664in}}%
\pgfpathlineto{\pgfqpoint{5.439122in}{0.642984in}}%
\pgfpathlineto{\pgfqpoint{5.540453in}{0.633152in}}%
\pgfpathlineto{\pgfqpoint{5.641784in}{0.623695in}}%
\pgfpathlineto{\pgfqpoint{5.748448in}{0.614121in}}%
\pgfpathlineto{\pgfqpoint{5.855113in}{0.604917in}}%
\pgfpathlineto{\pgfqpoint{5.967110in}{0.595627in}}%
\pgfpathlineto{\pgfqpoint{6.084441in}{0.586282in}}%
\pgfpathlineto{\pgfqpoint{6.183915in}{0.578665in}}%
\pgfpathlineto{\pgfqpoint{6.219152in}{0.576287in}}%
\pgfpathlineto{\pgfqpoint{6.263198in}{0.573734in}}%
\pgfpathlineto{\pgfqpoint{6.316054in}{0.571128in}}%
\pgfpathlineto{\pgfqpoint{6.368909in}{0.568896in}}%
\pgfpathlineto{\pgfqpoint{6.430574in}{0.566648in}}%
\pgfpathlineto{\pgfqpoint{6.501048in}{0.564439in}}%
\pgfpathlineto{\pgfqpoint{6.580332in}{0.562305in}}%
\pgfpathlineto{\pgfqpoint{6.677234in}{0.560079in}}%
\pgfpathlineto{\pgfqpoint{6.782945in}{0.558018in}}%
\pgfpathlineto{\pgfqpoint{6.906275in}{0.555978in}}%
\pgfpathlineto{\pgfqpoint{7.047223in}{0.554012in}}%
\pgfpathlineto{\pgfqpoint{7.047223in}{0.554012in}}%
\pgfusepath{stroke}%
\end{pgfscope}%
\begin{pgfscope}%
\pgfsetrectcap%
\pgfsetmiterjoin%
\pgfsetlinewidth{0.803000pt}%
\definecolor{currentstroke}{rgb}{0.000000,0.000000,0.000000}%
\pgfsetstrokecolor{currentstroke}%
\pgfsetdash{}{0pt}%
\pgfpathmoveto{\pgfqpoint{0.847223in}{0.554012in}}%
\pgfpathlineto{\pgfqpoint{0.847223in}{5.084012in}}%
\pgfusepath{stroke}%
\end{pgfscope}%
\begin{pgfscope}%
\pgfsetrectcap%
\pgfsetmiterjoin%
\pgfsetlinewidth{0.803000pt}%
\definecolor{currentstroke}{rgb}{0.000000,0.000000,0.000000}%
\pgfsetstrokecolor{currentstroke}%
\pgfsetdash{}{0pt}%
\pgfpathmoveto{\pgfqpoint{7.047223in}{0.554012in}}%
\pgfpathlineto{\pgfqpoint{7.047223in}{5.084012in}}%
\pgfusepath{stroke}%
\end{pgfscope}%
\begin{pgfscope}%
\pgfsetrectcap%
\pgfsetmiterjoin%
\pgfsetlinewidth{0.803000pt}%
\definecolor{currentstroke}{rgb}{0.000000,0.000000,0.000000}%
\pgfsetstrokecolor{currentstroke}%
\pgfsetdash{}{0pt}%
\pgfpathmoveto{\pgfqpoint{0.847223in}{0.554012in}}%
\pgfpathlineto{\pgfqpoint{7.047223in}{0.554012in}}%
\pgfusepath{stroke}%
\end{pgfscope}%
\begin{pgfscope}%
\pgfsetrectcap%
\pgfsetmiterjoin%
\pgfsetlinewidth{0.803000pt}%
\definecolor{currentstroke}{rgb}{0.000000,0.000000,0.000000}%
\pgfsetstrokecolor{currentstroke}%
\pgfsetdash{}{0pt}%
\pgfpathmoveto{\pgfqpoint{0.847223in}{5.084012in}}%
\pgfpathlineto{\pgfqpoint{7.047223in}{5.084012in}}%
\pgfusepath{stroke}%
\end{pgfscope}%
\begin{pgfscope}%
\pgfsetbuttcap%
\pgfsetmiterjoin%
\definecolor{currentfill}{rgb}{1.000000,1.000000,1.000000}%
\pgfsetfillcolor{currentfill}%
\pgfsetfillopacity{0.800000}%
\pgfsetlinewidth{1.003750pt}%
\definecolor{currentstroke}{rgb}{0.800000,0.800000,0.800000}%
\pgfsetstrokecolor{currentstroke}%
\pgfsetstrokeopacity{0.800000}%
\pgfsetdash{}{0pt}%
\pgfpathmoveto{\pgfqpoint{4.927935in}{1.964846in}}%
\pgfpathlineto{\pgfqpoint{6.911112in}{1.964846in}}%
\pgfpathquadraticcurveto{\pgfqpoint{6.950001in}{1.964846in}}{\pgfqpoint{6.950001in}{2.003735in}}%
\pgfpathlineto{\pgfqpoint{6.950001in}{3.634288in}}%
\pgfpathquadraticcurveto{\pgfqpoint{6.950001in}{3.673177in}}{\pgfqpoint{6.911112in}{3.673177in}}%
\pgfpathlineto{\pgfqpoint{4.927935in}{3.673177in}}%
\pgfpathquadraticcurveto{\pgfqpoint{4.889046in}{3.673177in}}{\pgfqpoint{4.889046in}{3.634288in}}%
\pgfpathlineto{\pgfqpoint{4.889046in}{2.003735in}}%
\pgfpathquadraticcurveto{\pgfqpoint{4.889046in}{1.964846in}}{\pgfqpoint{4.927935in}{1.964846in}}%
\pgfpathlineto{\pgfqpoint{4.927935in}{1.964846in}}%
\pgfpathclose%
\pgfusepath{stroke,fill}%
\end{pgfscope}%
\begin{pgfscope}%
\pgfsetbuttcap%
\pgfsetroundjoin%
\pgfsetlinewidth{1.003750pt}%
\definecolor{currentstroke}{rgb}{1.000000,0.000000,0.000000}%
\pgfsetstrokecolor{currentstroke}%
\pgfsetdash{}{0pt}%
\pgfpathmoveto{\pgfqpoint{5.161268in}{3.465886in}}%
\pgfpathcurveto{\pgfqpoint{5.172318in}{3.465886in}}{\pgfqpoint{5.182917in}{3.470276in}}{\pgfqpoint{5.190731in}{3.478090in}}%
\pgfpathcurveto{\pgfqpoint{5.198544in}{3.485903in}}{\pgfqpoint{5.202935in}{3.496502in}}{\pgfqpoint{5.202935in}{3.507553in}}%
\pgfpathcurveto{\pgfqpoint{5.202935in}{3.518603in}}{\pgfqpoint{5.198544in}{3.529202in}}{\pgfqpoint{5.190731in}{3.537015in}}%
\pgfpathcurveto{\pgfqpoint{5.182917in}{3.544829in}}{\pgfqpoint{5.172318in}{3.549219in}}{\pgfqpoint{5.161268in}{3.549219in}}%
\pgfpathcurveto{\pgfqpoint{5.150218in}{3.549219in}}{\pgfqpoint{5.139619in}{3.544829in}}{\pgfqpoint{5.131805in}{3.537015in}}%
\pgfpathcurveto{\pgfqpoint{5.123992in}{3.529202in}}{\pgfqpoint{5.119601in}{3.518603in}}{\pgfqpoint{5.119601in}{3.507553in}}%
\pgfpathcurveto{\pgfqpoint{5.119601in}{3.496502in}}{\pgfqpoint{5.123992in}{3.485903in}}{\pgfqpoint{5.131805in}{3.478090in}}%
\pgfpathcurveto{\pgfqpoint{5.139619in}{3.470276in}}{\pgfqpoint{5.150218in}{3.465886in}}{\pgfqpoint{5.161268in}{3.465886in}}%
\pgfpathlineto{\pgfqpoint{5.161268in}{3.465886in}}%
\pgfpathclose%
\pgfusepath{stroke}%
\end{pgfscope}%
\begin{pgfscope}%
\definecolor{textcolor}{rgb}{0.000000,0.000000,0.000000}%
\pgfsetstrokecolor{textcolor}%
\pgfsetfillcolor{textcolor}%
\pgftext[x=5.511268in,y=3.456511in,left,base]{\color{textcolor}\rmfamily\fontsize{14.000000}{16.800000}\selectfont Solutions}%
\end{pgfscope}%
\begin{pgfscope}%
\pgfsetrectcap%
\pgfsetroundjoin%
\pgfsetlinewidth{2.509375pt}%
\definecolor{currentstroke}{rgb}{0.000000,0.000000,0.000000}%
\pgfsetstrokecolor{currentstroke}%
\pgfsetdash{}{0pt}%
\pgfpathmoveto{\pgfqpoint{4.966823in}{3.249567in}}%
\pgfpathlineto{\pgfqpoint{5.161268in}{3.249567in}}%
\pgfpathlineto{\pgfqpoint{5.355712in}{3.249567in}}%
\pgfusepath{stroke}%
\end{pgfscope}%
\begin{pgfscope}%
\definecolor{textcolor}{rgb}{0.000000,0.000000,0.000000}%
\pgfsetstrokecolor{textcolor}%
\pgfsetfillcolor{textcolor}%
\pgftext[x=5.511268in,y=3.181511in,left,base]{\color{textcolor}\rmfamily\fontsize{14.000000}{16.800000}\selectfont Pareto-front}%
\end{pgfscope}%
\begin{pgfscope}%
\pgfsetbuttcap%
\pgfsetmiterjoin%
\definecolor{currentfill}{rgb}{0.501961,0.501961,0.501961}%
\pgfsetfillcolor{currentfill}%
\pgfsetfillopacity{0.200000}%
\pgfsetlinewidth{1.003750pt}%
\definecolor{currentstroke}{rgb}{0.501961,0.501961,0.501961}%
\pgfsetstrokecolor{currentstroke}%
\pgfsetstrokeopacity{0.200000}%
\pgfsetdash{}{0pt}%
\pgfpathmoveto{\pgfqpoint{4.966823in}{2.906512in}}%
\pgfpathlineto{\pgfqpoint{5.355712in}{2.906512in}}%
\pgfpathlineto{\pgfqpoint{5.355712in}{3.042623in}}%
\pgfpathlineto{\pgfqpoint{4.966823in}{3.042623in}}%
\pgfpathlineto{\pgfqpoint{4.966823in}{2.906512in}}%
\pgfpathclose%
\pgfusepath{stroke,fill}%
\end{pgfscope}%
\begin{pgfscope}%
\definecolor{textcolor}{rgb}{0.000000,0.000000,0.000000}%
\pgfsetstrokecolor{textcolor}%
\pgfsetfillcolor{textcolor}%
\pgftext[x=5.511268in,y=2.906512in,left,base]{\color{textcolor}\rmfamily\fontsize{14.000000}{16.800000}\selectfont Infeasible Space}%
\end{pgfscope}%
\begin{pgfscope}%
\pgfsetbuttcap%
\pgfsetmiterjoin%
\definecolor{currentfill}{rgb}{0.121569,0.466667,0.705882}%
\pgfsetfillcolor{currentfill}%
\pgfsetfillopacity{0.200000}%
\pgfsetlinewidth{1.003750pt}%
\definecolor{currentstroke}{rgb}{0.121569,0.466667,0.705882}%
\pgfsetstrokecolor{currentstroke}%
\pgfsetstrokeopacity{0.200000}%
\pgfsetdash{}{0pt}%
\pgfpathmoveto{\pgfqpoint{4.966823in}{2.631512in}}%
\pgfpathlineto{\pgfqpoint{5.355712in}{2.631512in}}%
\pgfpathlineto{\pgfqpoint{5.355712in}{2.767623in}}%
\pgfpathlineto{\pgfqpoint{4.966823in}{2.767623in}}%
\pgfpathlineto{\pgfqpoint{4.966823in}{2.631512in}}%
\pgfpathclose%
\pgfusepath{stroke,fill}%
\end{pgfscope}%
\begin{pgfscope}%
\definecolor{textcolor}{rgb}{0.000000,0.000000,0.000000}%
\pgfsetstrokecolor{textcolor}%
\pgfsetfillcolor{textcolor}%
\pgftext[x=5.511268in,y=2.631512in,left,base]{\color{textcolor}\rmfamily\fontsize{14.000000}{16.800000}\selectfont Feasible Space}%
\end{pgfscope}%
\begin{pgfscope}%
\pgfsetbuttcap%
\pgfsetroundjoin%
\definecolor{currentfill}{rgb}{0.121569,0.466667,0.705882}%
\pgfsetfillcolor{currentfill}%
\pgfsetlinewidth{1.003750pt}%
\definecolor{currentstroke}{rgb}{0.121569,0.466667,0.705882}%
\pgfsetstrokecolor{currentstroke}%
\pgfsetdash{}{0pt}%
\pgfsys@defobject{currentmarker}{\pgfqpoint{-0.114394in}{-0.097310in}}{\pgfqpoint{0.114394in}{0.120281in}}{%
\pgfpathmoveto{\pgfqpoint{0.000000in}{0.120281in}}%
\pgfpathlineto{\pgfqpoint{-0.027005in}{0.037169in}}%
\pgfpathlineto{\pgfqpoint{-0.114394in}{0.037169in}}%
\pgfpathlineto{\pgfqpoint{-0.043695in}{-0.014197in}}%
\pgfpathlineto{\pgfqpoint{-0.070700in}{-0.097310in}}%
\pgfpathlineto{\pgfqpoint{-0.000000in}{-0.045943in}}%
\pgfpathlineto{\pgfqpoint{0.070700in}{-0.097310in}}%
\pgfpathlineto{\pgfqpoint{0.043695in}{-0.014197in}}%
\pgfpathlineto{\pgfqpoint{0.114394in}{0.037169in}}%
\pgfpathlineto{\pgfqpoint{0.027005in}{0.037169in}}%
\pgfpathlineto{\pgfqpoint{0.000000in}{0.120281in}}%
\pgfpathclose%
\pgfusepath{stroke,fill}%
}%
\begin{pgfscope}%
\pgfsys@transformshift{5.161268in}{2.407554in}%
\pgfsys@useobject{currentmarker}{}%
\end{pgfscope}%
\end{pgfscope}%
\begin{pgfscope}%
\definecolor{textcolor}{rgb}{0.000000,0.000000,0.000000}%
\pgfsetstrokecolor{textcolor}%
\pgfsetfillcolor{textcolor}%
\pgftext[x=5.511268in,y=2.356513in,left,base]{\color{textcolor}\rmfamily\fontsize{14.000000}{16.800000}\selectfont Ideal}%
\end{pgfscope}%
\begin{pgfscope}%
\pgfsetbuttcap%
\pgfsetroundjoin%
\definecolor{currentfill}{rgb}{1.000000,0.498039,0.054902}%
\pgfsetfillcolor{currentfill}%
\pgfsetlinewidth{1.003750pt}%
\definecolor{currentstroke}{rgb}{1.000000,0.498039,0.054902}%
\pgfsetstrokecolor{currentstroke}%
\pgfsetdash{}{0pt}%
\pgfsys@defobject{currentmarker}{\pgfqpoint{-0.098209in}{-0.098209in}}{\pgfqpoint{0.098209in}{0.098209in}}{%
\pgfpathmoveto{\pgfqpoint{-0.098209in}{-0.098209in}}%
\pgfpathlineto{\pgfqpoint{0.098209in}{-0.098209in}}%
\pgfpathlineto{\pgfqpoint{0.098209in}{0.098209in}}%
\pgfpathlineto{\pgfqpoint{-0.098209in}{0.098209in}}%
\pgfpathlineto{\pgfqpoint{-0.098209in}{-0.098209in}}%
\pgfpathclose%
\pgfusepath{stroke,fill}%
}%
\begin{pgfscope}%
\pgfsys@transformshift{5.161268in}{2.132555in}%
\pgfsys@useobject{currentmarker}{}%
\end{pgfscope}%
\end{pgfscope}%
\begin{pgfscope}%
\definecolor{textcolor}{rgb}{0.000000,0.000000,0.000000}%
\pgfsetstrokecolor{textcolor}%
\pgfsetfillcolor{textcolor}%
\pgftext[x=5.511268in,y=2.081513in,left,base]{\color{textcolor}\rmfamily\fontsize{14.000000}{16.800000}\selectfont Nadir}%
\end{pgfscope}%
\end{pgfpicture}%
\makeatother%
\endgroup%
} \caption{An
  example \textit{convex} Pareto-front from \acs{pymoo} \cite{blank_pymoo_2020,
  deb_omni-optimizer_2008}.}
  \label{fig:truss-pareto}
\end{figure}

 There are broadly two classes of \ac{moo} algorithms for solving Equation
\ref{eqn:generic-moop}, \textit{scalarization} and \textit{population-based}
\cite{gunantara_review_2018, emmerich_tutorial_2018}. Scalarization approaches
map the multi-objective problem onto a set of single-objective problems using
variation of parameters. In the \ac{ws} algorithm, the objectives are assigned
weights, $w_i$, and the aggregated objective becomes
\begin{align}
    \label{eqn:weighted-sum}
    \text{min}\quad J(x) &= \sum_i w_i F_i(x)
    \intertext{subject to:}
&g(x, p) \leq 0\nonumber\\
&x \in \vec{X}\nonumber
\intertext{where}
&F_i \text{ is an arbitrary objective function,}\nonumber\\
&w_i \text{ is the weight for objective function $F_i$}\\
&J \text{ is the aggregated objective,}\nonumber\\
&g \text{ is a constraint,}\nonumber\\
&p \text{ is an arbitrary parameter of $g$,}\nonumber\\
&\vec{X} \text{ is the set of decision variables.}
\end{align}
\noindent
These weights are varied in order to sample points along the Pareto-front. 

Alternatively, the \ac{ec}
algorithm for scalarization chooses one objective from $\{F_n\}$ to solve and converts the others
into constraints, whose bounds are denoted by $\epsilon$. These bounds are
varied until the desired number of points on the Pareto-front is reached
\cite{gunantara_review_2018, emmerich_tutorial_2018}. This problem can be
written as
\begin{align}
\label{eqn:epsilon-constraint}
    &\text{min}\quad F_j(x),
    \intertext{subject to:}
    &F_2(x) - \epsilon_j \leq 0\nonumber\\
    &\vdots\nonumber\\
    &F_i(x) - \epsilon_j \leq 0\nonumber\\
    &g(x, p) \leq 0,\nonumber\\
    &x \in \vec{X}.\nonumber
\end{align}
\noindent
The sub-problem, Equation \ref{eqn:epsilon-constraint}, must be repeated for
each $F_j(x)$ and corresponding $\epsilon_j$ in $\{F_n\}$.

Scalarization is attractive due to its simplicity. However, this approach is
sensitive to problem convexity. \ac{ws} will never be able to sample points in a
concave region of the Pareto-front, and \ac{ec} will have poorly spaced samples
along a concave region. Further, these algorithms can only sample points on the
frontier, not the sub-optimal feasible space. Thus supporting the critique of
using \ac{moo} for handling structural uncertainty \cite{decarolis_using_2011}.

Fortunately, population-based algorithms, also called \textit{\ac{ga}} or
\textit{evolutionary algorithms}, resolve some of these issues by solving
Equation \ref{eqn:generic-moop} directly. \Acp{ga} are based on the principle of
natural selection. In a \ac{ga}, such as \ac{nsga2}, an initial population is
randomly generated using the problem's decision variables, the `fitness' of this
population (i.e., performance on each objective) is calculated, then a new
population is selected from the `fittest' (most optimal) individuals. This
process continues until a convergence criterion is reached. The advantages of
this method are
\begin{enumerate}
    \item a guaranteed solution, regardless of convexity,
    \item no prior knowledge is required to initialize the problem, as with
    \ac{ec},
    \item greater diversity of solutions (i.e., spacing of points along the
    Pareto-front),
    \item the sub-optimal space is sampled through the iterative process (though
    not uniformly).
\end{enumerate}
Specifically, point four address one of the primary criticisms of using \ac{moo}
to reduce structural uncertainty by obtaining points in the inferior region
\cite{loughlin_genetic_2001,zechman_evolutionary_2004,
zechman_evolutionary_2013}. An additional advantage of \acp{ga} is the ability
to incorporate more physics and simulations into the optimization procedure than
\ac{lp}, \ac{milp}, or scalarization allow \cite{loughlin_genetic_2001} because 
\acp{moo} can incorporate data from external models. 

Previous work handled structural uncertainty using \ac{mga} which samples unique
solutions from the sub-optimal space in a neighborhood around the global minimum
for a single objective \cite{decarolis_using_2011}. Researchers argue that this
approach is valid because there will always be structural uncertainty and
sampling the inferior region may offer insight for decision-makers.
\textcolor{red}{While structural uncertainty may persist it is not
\textit{irreducible}.} By increasing the number of modeled objectives \ac{moo}
reduces structural uncertainty. Further, ideas from \ac{mga} can be applied to
\ac{moo} by efficiently sampling the near-optimal space
\cite{loughlin_genetic_2001,
zechman_evolutionary_2004,zechman_evolutionary_2013,pajares_comparison_2021}.
The goal of \ac{mga} is to find a \textit{reduced} set of maximally different
alternatives to provide insight, where analyzing the full set of alternatives
would be overwhelming \cite{decarolis_using_2011, pajares_comparison_2021}.
Figure \ref{fig:near-opt-pareto} shows the near-optimal space around the
Pareto-front from Figure \ref{fig:truss-pareto}.

\begin{figure}[H]
  \centering
  \resizebox{0.6\columnwidth}{!}{%% Creator: Matplotlib, PGF backend
%%
%% To include the figure in your LaTeX document, write
%%   \input{<filename>.pgf}
%%
%% Make sure the required packages are loaded in your preamble
%%   \usepackage{pgf}
%%
%% Also ensure that all the required font packages are loaded; for instance,
%% the lmodern package is sometimes necessary when using math font.
%%   \usepackage{lmodern}
%%
%% Figures using additional raster images can only be included by \input if
%% they are in the same directory as the main LaTeX file. For loading figures
%% from other directories you can use the `import` package
%%   \usepackage{import}
%%
%% and then include the figures with
%%   \import{<path to file>}{<filename>.pgf}
%%
%% Matplotlib used the following preamble
%%
\begingroup%
\makeatletter%
\begin{pgfpicture}%
\pgfpathrectangle{\pgfpointorigin}{\pgfqpoint{7.147223in}{5.232237in}}%
\pgfusepath{use as bounding box, clip}%
\begin{pgfscope}%
\pgfsetbuttcap%
\pgfsetmiterjoin%
\definecolor{currentfill}{rgb}{1.000000,1.000000,1.000000}%
\pgfsetfillcolor{currentfill}%
\pgfsetlinewidth{0.000000pt}%
\definecolor{currentstroke}{rgb}{0.000000,0.000000,0.000000}%
\pgfsetstrokecolor{currentstroke}%
\pgfsetdash{}{0pt}%
\pgfpathmoveto{\pgfqpoint{0.000000in}{0.000000in}}%
\pgfpathlineto{\pgfqpoint{7.147223in}{0.000000in}}%
\pgfpathlineto{\pgfqpoint{7.147223in}{5.232237in}}%
\pgfpathlineto{\pgfqpoint{0.000000in}{5.232237in}}%
\pgfpathlineto{\pgfqpoint{0.000000in}{0.000000in}}%
\pgfpathclose%
\pgfusepath{fill}%
\end{pgfscope}%
\begin{pgfscope}%
\pgfsetbuttcap%
\pgfsetmiterjoin%
\definecolor{currentfill}{rgb}{1.000000,1.000000,1.000000}%
\pgfsetfillcolor{currentfill}%
\pgfsetlinewidth{0.000000pt}%
\definecolor{currentstroke}{rgb}{0.000000,0.000000,0.000000}%
\pgfsetstrokecolor{currentstroke}%
\pgfsetstrokeopacity{0.000000}%
\pgfsetdash{}{0pt}%
\pgfpathmoveto{\pgfqpoint{0.847223in}{0.554012in}}%
\pgfpathlineto{\pgfqpoint{7.047223in}{0.554012in}}%
\pgfpathlineto{\pgfqpoint{7.047223in}{5.084012in}}%
\pgfpathlineto{\pgfqpoint{0.847223in}{5.084012in}}%
\pgfpathlineto{\pgfqpoint{0.847223in}{0.554012in}}%
\pgfpathclose%
\pgfusepath{fill}%
\end{pgfscope}%
\begin{pgfscope}%
\pgfpathrectangle{\pgfqpoint{0.847223in}{0.554012in}}{\pgfqpoint{6.200000in}{4.530000in}}%
\pgfusepath{clip}%
\pgfsetbuttcap%
\pgfsetmiterjoin%
\definecolor{currentfill}{rgb}{0.827451,0.827451,0.827451}%
\pgfsetfillcolor{currentfill}%
\pgfsetfillopacity{0.500000}%
\pgfsetlinewidth{0.000000pt}%
\definecolor{currentstroke}{rgb}{0.000000,0.000000,0.000000}%
\pgfsetstrokecolor{currentstroke}%
\pgfsetstrokeopacity{0.500000}%
\pgfsetdash{}{0pt}%
\pgfpathmoveto{\pgfqpoint{0.759998in}{4.259481in}}%
\pgfpathlineto{\pgfqpoint{0.764442in}{4.217893in}}%
\pgfpathlineto{\pgfqpoint{0.768887in}{4.177136in}}%
\pgfpathlineto{\pgfqpoint{0.773331in}{4.137184in}}%
\pgfpathlineto{\pgfqpoint{0.777775in}{4.098015in}}%
\pgfpathlineto{\pgfqpoint{0.782220in}{4.059605in}}%
\pgfpathlineto{\pgfqpoint{0.786664in}{4.021934in}}%
\pgfpathlineto{\pgfqpoint{0.791108in}{3.984978in}}%
\pgfpathlineto{\pgfqpoint{0.795553in}{3.948720in}}%
\pgfpathlineto{\pgfqpoint{0.799997in}{3.913138in}}%
\pgfpathlineto{\pgfqpoint{0.804441in}{3.878214in}}%
\pgfpathlineto{\pgfqpoint{0.808886in}{3.843931in}}%
\pgfpathlineto{\pgfqpoint{0.813330in}{3.810270in}}%
\pgfpathlineto{\pgfqpoint{0.817775in}{3.777214in}}%
\pgfpathlineto{\pgfqpoint{0.822219in}{3.744749in}}%
\pgfpathlineto{\pgfqpoint{0.826663in}{3.712857in}}%
\pgfpathlineto{\pgfqpoint{0.831108in}{3.681524in}}%
\pgfpathlineto{\pgfqpoint{0.835552in}{3.650736in}}%
\pgfpathlineto{\pgfqpoint{0.839996in}{3.620477in}}%
\pgfpathlineto{\pgfqpoint{0.844441in}{3.590736in}}%
\pgfpathlineto{\pgfqpoint{0.848885in}{3.561497in}}%
\pgfpathlineto{\pgfqpoint{0.853329in}{3.532750in}}%
\pgfpathlineto{\pgfqpoint{0.857774in}{3.504482in}}%
\pgfpathlineto{\pgfqpoint{0.862218in}{3.476680in}}%
\pgfpathlineto{\pgfqpoint{0.866662in}{3.449333in}}%
\pgfpathlineto{\pgfqpoint{0.871107in}{3.422431in}}%
\pgfpathlineto{\pgfqpoint{0.875551in}{3.395962in}}%
\pgfpathlineto{\pgfqpoint{0.879995in}{3.369916in}}%
\pgfpathlineto{\pgfqpoint{0.884440in}{3.344283in}}%
\pgfpathlineto{\pgfqpoint{0.888884in}{3.319053in}}%
\pgfpathlineto{\pgfqpoint{0.893328in}{3.294217in}}%
\pgfpathlineto{\pgfqpoint{0.897773in}{3.269766in}}%
\pgfpathlineto{\pgfqpoint{0.902217in}{3.245691in}}%
\pgfpathlineto{\pgfqpoint{0.906661in}{3.221983in}}%
\pgfpathlineto{\pgfqpoint{0.911106in}{3.198633in}}%
\pgfpathlineto{\pgfqpoint{0.915550in}{3.175635in}}%
\pgfpathlineto{\pgfqpoint{0.919994in}{3.152979in}}%
\pgfpathlineto{\pgfqpoint{0.924439in}{3.130659in}}%
\pgfpathlineto{\pgfqpoint{0.928883in}{3.108666in}}%
\pgfpathlineto{\pgfqpoint{0.933328in}{3.086995in}}%
\pgfpathlineto{\pgfqpoint{0.937772in}{3.065637in}}%
\pgfpathlineto{\pgfqpoint{0.942216in}{3.044586in}}%
\pgfpathlineto{\pgfqpoint{0.946661in}{3.023836in}}%
\pgfpathlineto{\pgfqpoint{0.951105in}{3.003379in}}%
\pgfpathlineto{\pgfqpoint{0.955549in}{2.983211in}}%
\pgfpathlineto{\pgfqpoint{0.959994in}{2.963324in}}%
\pgfpathlineto{\pgfqpoint{0.964438in}{2.943714in}}%
\pgfpathlineto{\pgfqpoint{0.968882in}{2.924373in}}%
\pgfpathlineto{\pgfqpoint{0.973327in}{2.905297in}}%
\pgfpathlineto{\pgfqpoint{0.977771in}{2.886481in}}%
\pgfpathlineto{\pgfqpoint{0.982215in}{2.867919in}}%
\pgfpathlineto{\pgfqpoint{0.986660in}{2.849605in}}%
\pgfpathlineto{\pgfqpoint{0.991104in}{2.831536in}}%
\pgfpathlineto{\pgfqpoint{0.995548in}{2.813705in}}%
\pgfpathlineto{\pgfqpoint{0.999993in}{2.796109in}}%
\pgfpathlineto{\pgfqpoint{1.004437in}{2.778743in}}%
\pgfpathlineto{\pgfqpoint{1.008881in}{2.761602in}}%
\pgfpathlineto{\pgfqpoint{1.013326in}{2.744683in}}%
\pgfpathlineto{\pgfqpoint{1.017770in}{2.727980in}}%
\pgfpathlineto{\pgfqpoint{1.022214in}{2.711489in}}%
\pgfpathlineto{\pgfqpoint{1.026659in}{2.695207in}}%
\pgfpathlineto{\pgfqpoint{1.031103in}{2.679130in}}%
\pgfpathlineto{\pgfqpoint{1.035548in}{2.663254in}}%
\pgfpathlineto{\pgfqpoint{1.039992in}{2.647574in}}%
\pgfpathlineto{\pgfqpoint{1.044436in}{2.632088in}}%
\pgfpathlineto{\pgfqpoint{1.048881in}{2.616792in}}%
\pgfpathlineto{\pgfqpoint{1.053325in}{2.601682in}}%
\pgfpathlineto{\pgfqpoint{1.057769in}{2.586756in}}%
\pgfpathlineto{\pgfqpoint{1.062214in}{2.572009in}}%
\pgfpathlineto{\pgfqpoint{1.066658in}{2.557438in}}%
\pgfpathlineto{\pgfqpoint{1.071102in}{2.543041in}}%
\pgfpathlineto{\pgfqpoint{1.075547in}{2.528814in}}%
\pgfpathlineto{\pgfqpoint{1.079991in}{2.514754in}}%
\pgfpathlineto{\pgfqpoint{1.084435in}{2.500859in}}%
\pgfpathlineto{\pgfqpoint{1.088880in}{2.487125in}}%
\pgfpathlineto{\pgfqpoint{1.093324in}{2.473549in}}%
\pgfpathlineto{\pgfqpoint{1.097768in}{2.460130in}}%
\pgfpathlineto{\pgfqpoint{1.102213in}{2.446864in}}%
\pgfpathlineto{\pgfqpoint{1.106657in}{2.433748in}}%
\pgfpathlineto{\pgfqpoint{1.111101in}{2.420781in}}%
\pgfpathlineto{\pgfqpoint{1.115546in}{2.407959in}}%
\pgfpathlineto{\pgfqpoint{1.119990in}{2.395281in}}%
\pgfpathlineto{\pgfqpoint{1.124434in}{2.382743in}}%
\pgfpathlineto{\pgfqpoint{1.128879in}{2.370343in}}%
\pgfpathlineto{\pgfqpoint{1.133323in}{2.358080in}}%
\pgfpathlineto{\pgfqpoint{1.137767in}{2.345951in}}%
\pgfpathlineto{\pgfqpoint{1.142212in}{2.333953in}}%
\pgfpathlineto{\pgfqpoint{1.146656in}{2.322085in}}%
\pgfpathlineto{\pgfqpoint{1.151101in}{2.310345in}}%
\pgfpathlineto{\pgfqpoint{1.155545in}{2.298730in}}%
\pgfpathlineto{\pgfqpoint{1.159989in}{2.287238in}}%
\pgfpathlineto{\pgfqpoint{1.164434in}{2.275868in}}%
\pgfpathlineto{\pgfqpoint{1.168878in}{2.264618in}}%
\pgfpathlineto{\pgfqpoint{1.173322in}{2.253485in}}%
\pgfpathlineto{\pgfqpoint{1.177767in}{2.242468in}}%
\pgfpathlineto{\pgfqpoint{1.182211in}{2.231566in}}%
\pgfpathlineto{\pgfqpoint{1.186655in}{2.220775in}}%
\pgfpathlineto{\pgfqpoint{1.191100in}{2.210096in}}%
\pgfpathlineto{\pgfqpoint{1.195544in}{2.199525in}}%
\pgfpathlineto{\pgfqpoint{1.199988in}{2.189061in}}%
\pgfpathlineto{\pgfqpoint{1.204433in}{2.178703in}}%
\pgfpathlineto{\pgfqpoint{1.208877in}{2.168449in}}%
\pgfpathlineto{\pgfqpoint{1.213321in}{2.158298in}}%
\pgfpathlineto{\pgfqpoint{1.217766in}{2.148247in}}%
\pgfpathlineto{\pgfqpoint{1.222210in}{2.138296in}}%
\pgfpathlineto{\pgfqpoint{1.226654in}{2.128443in}}%
\pgfpathlineto{\pgfqpoint{1.231099in}{2.118686in}}%
\pgfpathlineto{\pgfqpoint{1.235543in}{2.109025in}}%
\pgfpathlineto{\pgfqpoint{1.239987in}{2.099457in}}%
\pgfpathlineto{\pgfqpoint{1.244432in}{2.089982in}}%
\pgfpathlineto{\pgfqpoint{1.248876in}{2.080597in}}%
\pgfpathlineto{\pgfqpoint{1.253320in}{2.071303in}}%
\pgfpathlineto{\pgfqpoint{1.257765in}{2.062097in}}%
\pgfpathlineto{\pgfqpoint{1.262209in}{2.052978in}}%
\pgfpathlineto{\pgfqpoint{1.266654in}{2.043945in}}%
\pgfpathlineto{\pgfqpoint{1.271098in}{2.034997in}}%
\pgfpathlineto{\pgfqpoint{1.275542in}{2.026132in}}%
\pgfpathlineto{\pgfqpoint{1.279987in}{2.017350in}}%
\pgfpathlineto{\pgfqpoint{1.284431in}{2.008649in}}%
\pgfpathlineto{\pgfqpoint{1.288875in}{2.000029in}}%
\pgfpathlineto{\pgfqpoint{1.293320in}{1.991487in}}%
\pgfpathlineto{\pgfqpoint{1.297764in}{1.983023in}}%
\pgfpathlineto{\pgfqpoint{1.302208in}{1.974636in}}%
\pgfpathlineto{\pgfqpoint{1.306653in}{1.966326in}}%
\pgfpathlineto{\pgfqpoint{1.311097in}{1.958089in}}%
\pgfpathlineto{\pgfqpoint{1.315541in}{1.949927in}}%
\pgfpathlineto{\pgfqpoint{1.319986in}{1.941838in}}%
\pgfpathlineto{\pgfqpoint{1.324430in}{1.933820in}}%
\pgfpathlineto{\pgfqpoint{1.328874in}{1.925873in}}%
\pgfpathlineto{\pgfqpoint{1.333319in}{1.917997in}}%
\pgfpathlineto{\pgfqpoint{1.337763in}{1.910189in}}%
\pgfpathlineto{\pgfqpoint{1.342207in}{1.902449in}}%
\pgfpathlineto{\pgfqpoint{1.346652in}{1.894777in}}%
\pgfpathlineto{\pgfqpoint{1.351096in}{1.887171in}}%
\pgfpathlineto{\pgfqpoint{1.355540in}{1.879631in}}%
\pgfpathlineto{\pgfqpoint{1.359985in}{1.872155in}}%
\pgfpathlineto{\pgfqpoint{1.364429in}{1.864743in}}%
\pgfpathlineto{\pgfqpoint{1.368873in}{1.857395in}}%
\pgfpathlineto{\pgfqpoint{1.373318in}{1.850108in}}%
\pgfpathlineto{\pgfqpoint{1.377762in}{1.842883in}}%
\pgfpathlineto{\pgfqpoint{1.382207in}{1.835719in}}%
\pgfpathlineto{\pgfqpoint{1.386651in}{1.828614in}}%
\pgfpathlineto{\pgfqpoint{1.391095in}{1.821569in}}%
\pgfpathlineto{\pgfqpoint{1.395540in}{1.814582in}}%
\pgfpathlineto{\pgfqpoint{1.399984in}{1.807653in}}%
\pgfpathlineto{\pgfqpoint{1.404428in}{1.800781in}}%
\pgfpathlineto{\pgfqpoint{1.408873in}{1.793965in}}%
\pgfpathlineto{\pgfqpoint{1.413317in}{1.787205in}}%
\pgfpathlineto{\pgfqpoint{1.417761in}{1.780499in}}%
\pgfpathlineto{\pgfqpoint{1.422206in}{1.773848in}}%
\pgfpathlineto{\pgfqpoint{1.426650in}{1.767251in}}%
\pgfpathlineto{\pgfqpoint{1.431094in}{1.760707in}}%
\pgfpathlineto{\pgfqpoint{1.435539in}{1.754214in}}%
\pgfpathlineto{\pgfqpoint{1.439983in}{1.747774in}}%
\pgfpathlineto{\pgfqpoint{1.444427in}{1.741385in}}%
\pgfpathlineto{\pgfqpoint{1.448872in}{1.735046in}}%
\pgfpathlineto{\pgfqpoint{1.453316in}{1.728757in}}%
\pgfpathlineto{\pgfqpoint{1.457760in}{1.722517in}}%
\pgfpathlineto{\pgfqpoint{1.462205in}{1.716326in}}%
\pgfpathlineto{\pgfqpoint{1.466649in}{1.710183in}}%
\pgfpathlineto{\pgfqpoint{1.471093in}{1.704088in}}%
\pgfpathlineto{\pgfqpoint{1.475538in}{1.698040in}}%
\pgfpathlineto{\pgfqpoint{1.479982in}{1.692038in}}%
\pgfpathlineto{\pgfqpoint{1.484426in}{1.686083in}}%
\pgfpathlineto{\pgfqpoint{1.488871in}{1.680173in}}%
\pgfpathlineto{\pgfqpoint{1.493315in}{1.674307in}}%
\pgfpathlineto{\pgfqpoint{1.497760in}{1.668486in}}%
\pgfpathlineto{\pgfqpoint{1.502204in}{1.662709in}}%
\pgfpathlineto{\pgfqpoint{1.506648in}{1.656976in}}%
\pgfpathlineto{\pgfqpoint{1.511093in}{1.651285in}}%
\pgfpathlineto{\pgfqpoint{1.515537in}{1.645637in}}%
\pgfpathlineto{\pgfqpoint{1.519981in}{1.640031in}}%
\pgfpathlineto{\pgfqpoint{1.524426in}{1.634466in}}%
\pgfpathlineto{\pgfqpoint{1.528870in}{1.628942in}}%
\pgfpathlineto{\pgfqpoint{1.533314in}{1.623459in}}%
\pgfpathlineto{\pgfqpoint{1.537759in}{1.618016in}}%
\pgfpathlineto{\pgfqpoint{1.542203in}{1.612613in}}%
\pgfpathlineto{\pgfqpoint{1.546647in}{1.607249in}}%
\pgfpathlineto{\pgfqpoint{1.551092in}{1.601924in}}%
\pgfpathlineto{\pgfqpoint{1.555536in}{1.596637in}}%
\pgfpathlineto{\pgfqpoint{1.559980in}{1.591389in}}%
\pgfpathlineto{\pgfqpoint{1.564425in}{1.586178in}}%
\pgfpathlineto{\pgfqpoint{1.568869in}{1.581004in}}%
\pgfpathlineto{\pgfqpoint{1.573313in}{1.575867in}}%
\pgfpathlineto{\pgfqpoint{1.577758in}{1.570766in}}%
\pgfpathlineto{\pgfqpoint{1.582202in}{1.565702in}}%
\pgfpathlineto{\pgfqpoint{1.586646in}{1.560673in}}%
\pgfpathlineto{\pgfqpoint{1.591091in}{1.555679in}}%
\pgfpathlineto{\pgfqpoint{1.595535in}{1.550720in}}%
\pgfpathlineto{\pgfqpoint{1.599979in}{1.545796in}}%
\pgfpathlineto{\pgfqpoint{1.604424in}{1.540906in}}%
\pgfpathlineto{\pgfqpoint{1.608868in}{1.536050in}}%
\pgfpathlineto{\pgfqpoint{1.613313in}{1.531227in}}%
\pgfpathlineto{\pgfqpoint{1.617757in}{1.526437in}}%
\pgfpathlineto{\pgfqpoint{1.622201in}{1.521681in}}%
\pgfpathlineto{\pgfqpoint{1.626646in}{1.516956in}}%
\pgfpathlineto{\pgfqpoint{1.631090in}{1.512264in}}%
\pgfpathlineto{\pgfqpoint{1.635534in}{1.507604in}}%
\pgfpathlineto{\pgfqpoint{1.639979in}{1.502975in}}%
\pgfpathlineto{\pgfqpoint{1.644423in}{1.498377in}}%
\pgfpathlineto{\pgfqpoint{1.648867in}{1.493810in}}%
\pgfpathlineto{\pgfqpoint{1.653312in}{1.489273in}}%
\pgfpathlineto{\pgfqpoint{1.657756in}{1.484767in}}%
\pgfpathlineto{\pgfqpoint{1.662200in}{1.480291in}}%
\pgfpathlineto{\pgfqpoint{1.666645in}{1.475844in}}%
\pgfpathlineto{\pgfqpoint{1.671089in}{1.471427in}}%
\pgfpathlineto{\pgfqpoint{1.675533in}{1.467039in}}%
\pgfpathlineto{\pgfqpoint{1.679978in}{1.462679in}}%
\pgfpathlineto{\pgfqpoint{1.684422in}{1.458348in}}%
\pgfpathlineto{\pgfqpoint{1.688866in}{1.454045in}}%
\pgfpathlineto{\pgfqpoint{1.693311in}{1.449771in}}%
\pgfpathlineto{\pgfqpoint{1.697755in}{1.445523in}}%
\pgfpathlineto{\pgfqpoint{1.702199in}{1.441304in}}%
\pgfpathlineto{\pgfqpoint{1.706644in}{1.437111in}}%
\pgfpathlineto{\pgfqpoint{1.711088in}{1.432945in}}%
\pgfpathlineto{\pgfqpoint{1.715533in}{1.428806in}}%
\pgfpathlineto{\pgfqpoint{1.719977in}{1.424693in}}%
\pgfpathlineto{\pgfqpoint{1.724421in}{1.420606in}}%
\pgfpathlineto{\pgfqpoint{1.728866in}{1.416545in}}%
\pgfpathlineto{\pgfqpoint{1.733310in}{1.412510in}}%
\pgfpathlineto{\pgfqpoint{1.737754in}{1.408500in}}%
\pgfpathlineto{\pgfqpoint{1.742199in}{1.404515in}}%
\pgfpathlineto{\pgfqpoint{1.746643in}{1.400555in}}%
\pgfpathlineto{\pgfqpoint{1.751087in}{1.396620in}}%
\pgfpathlineto{\pgfqpoint{1.755532in}{1.392709in}}%
\pgfpathlineto{\pgfqpoint{1.759976in}{1.388823in}}%
\pgfpathlineto{\pgfqpoint{1.764420in}{1.384960in}}%
\pgfpathlineto{\pgfqpoint{1.768865in}{1.381121in}}%
\pgfpathlineto{\pgfqpoint{1.773309in}{1.377306in}}%
\pgfpathlineto{\pgfqpoint{1.777753in}{1.373514in}}%
\pgfpathlineto{\pgfqpoint{1.782198in}{1.369745in}}%
\pgfpathlineto{\pgfqpoint{1.786642in}{1.365999in}}%
\pgfpathlineto{\pgfqpoint{1.791086in}{1.362276in}}%
\pgfpathlineto{\pgfqpoint{1.795531in}{1.358575in}}%
\pgfpathlineto{\pgfqpoint{1.799975in}{1.354896in}}%
\pgfpathlineto{\pgfqpoint{1.804419in}{1.351240in}}%
\pgfpathlineto{\pgfqpoint{1.808864in}{1.347605in}}%
\pgfpathlineto{\pgfqpoint{1.813308in}{1.343993in}}%
\pgfpathlineto{\pgfqpoint{1.817752in}{1.340401in}}%
\pgfpathlineto{\pgfqpoint{1.822197in}{1.336831in}}%
\pgfpathlineto{\pgfqpoint{1.826641in}{1.333282in}}%
\pgfpathlineto{\pgfqpoint{1.831086in}{1.329754in}}%
\pgfpathlineto{\pgfqpoint{1.835530in}{1.326247in}}%
\pgfpathlineto{\pgfqpoint{1.839974in}{1.322760in}}%
\pgfpathlineto{\pgfqpoint{1.844419in}{1.319294in}}%
\pgfpathlineto{\pgfqpoint{1.848863in}{1.315848in}}%
\pgfpathlineto{\pgfqpoint{1.853307in}{1.312422in}}%
\pgfpathlineto{\pgfqpoint{1.857752in}{1.309016in}}%
\pgfpathlineto{\pgfqpoint{1.862196in}{1.305629in}}%
\pgfpathlineto{\pgfqpoint{1.866640in}{1.302262in}}%
\pgfpathlineto{\pgfqpoint{1.871085in}{1.298914in}}%
\pgfpathlineto{\pgfqpoint{1.875529in}{1.295586in}}%
\pgfpathlineto{\pgfqpoint{1.879973in}{1.292276in}}%
\pgfpathlineto{\pgfqpoint{1.884418in}{1.288986in}}%
\pgfpathlineto{\pgfqpoint{1.888862in}{1.285714in}}%
\pgfpathlineto{\pgfqpoint{1.893306in}{1.282460in}}%
\pgfpathlineto{\pgfqpoint{1.897751in}{1.279225in}}%
\pgfpathlineto{\pgfqpoint{1.902195in}{1.276008in}}%
\pgfpathlineto{\pgfqpoint{1.906639in}{1.272810in}}%
\pgfpathlineto{\pgfqpoint{1.911084in}{1.269629in}}%
\pgfpathlineto{\pgfqpoint{1.915528in}{1.266466in}}%
\pgfpathlineto{\pgfqpoint{1.919972in}{1.263320in}}%
\pgfpathlineto{\pgfqpoint{1.924417in}{1.260192in}}%
\pgfpathlineto{\pgfqpoint{1.928861in}{1.257081in}}%
\pgfpathlineto{\pgfqpoint{1.933305in}{1.253988in}}%
\pgfpathlineto{\pgfqpoint{1.937750in}{1.250911in}}%
\pgfpathlineto{\pgfqpoint{1.942194in}{1.247852in}}%
\pgfpathlineto{\pgfqpoint{1.946639in}{1.244809in}}%
\pgfpathlineto{\pgfqpoint{1.951083in}{1.241783in}}%
\pgfpathlineto{\pgfqpoint{1.955527in}{1.238773in}}%
\pgfpathlineto{\pgfqpoint{1.959972in}{1.235780in}}%
\pgfpathlineto{\pgfqpoint{1.964416in}{1.232803in}}%
\pgfpathlineto{\pgfqpoint{1.968860in}{1.229842in}}%
\pgfpathlineto{\pgfqpoint{1.973305in}{1.226897in}}%
\pgfpathlineto{\pgfqpoint{1.977749in}{1.223967in}}%
\pgfpathlineto{\pgfqpoint{1.982193in}{1.221054in}}%
\pgfpathlineto{\pgfqpoint{1.986638in}{1.218156in}}%
\pgfpathlineto{\pgfqpoint{1.991082in}{1.215273in}}%
\pgfpathlineto{\pgfqpoint{1.995526in}{1.212406in}}%
\pgfpathlineto{\pgfqpoint{1.999971in}{1.209554in}}%
\pgfpathlineto{\pgfqpoint{2.004415in}{1.206717in}}%
\pgfpathlineto{\pgfqpoint{2.008859in}{1.203895in}}%
\pgfpathlineto{\pgfqpoint{2.013304in}{1.201088in}}%
\pgfpathlineto{\pgfqpoint{2.017748in}{1.198296in}}%
\pgfpathlineto{\pgfqpoint{2.022192in}{1.195518in}}%
\pgfpathlineto{\pgfqpoint{2.026637in}{1.192755in}}%
\pgfpathlineto{\pgfqpoint{2.031081in}{1.190006in}}%
\pgfpathlineto{\pgfqpoint{2.035525in}{1.187271in}}%
\pgfpathlineto{\pgfqpoint{2.039970in}{1.184551in}}%
\pgfpathlineto{\pgfqpoint{2.044414in}{1.181844in}}%
\pgfpathlineto{\pgfqpoint{2.048858in}{1.179152in}}%
\pgfpathlineto{\pgfqpoint{2.053303in}{1.176473in}}%
\pgfpathlineto{\pgfqpoint{2.057747in}{1.173808in}}%
\pgfpathlineto{\pgfqpoint{2.062192in}{1.171157in}}%
\pgfpathlineto{\pgfqpoint{2.066636in}{1.168519in}}%
\pgfpathlineto{\pgfqpoint{2.071080in}{1.165895in}}%
\pgfpathlineto{\pgfqpoint{2.075525in}{1.163284in}}%
\pgfpathlineto{\pgfqpoint{2.079969in}{1.160686in}}%
\pgfpathlineto{\pgfqpoint{2.084413in}{1.158102in}}%
\pgfpathlineto{\pgfqpoint{2.088858in}{1.155530in}}%
\pgfpathlineto{\pgfqpoint{2.093302in}{1.152971in}}%
\pgfpathlineto{\pgfqpoint{2.097746in}{1.150425in}}%
\pgfpathlineto{\pgfqpoint{2.102191in}{1.147892in}}%
\pgfpathlineto{\pgfqpoint{2.106635in}{1.145372in}}%
\pgfpathlineto{\pgfqpoint{2.111079in}{1.142864in}}%
\pgfpathlineto{\pgfqpoint{2.115524in}{1.140368in}}%
\pgfpathlineto{\pgfqpoint{2.119968in}{1.137885in}}%
\pgfpathlineto{\pgfqpoint{2.124412in}{1.135414in}}%
\pgfpathlineto{\pgfqpoint{2.128857in}{1.132955in}}%
\pgfpathlineto{\pgfqpoint{2.133301in}{1.130509in}}%
\pgfpathlineto{\pgfqpoint{2.137745in}{1.128074in}}%
\pgfpathlineto{\pgfqpoint{2.142190in}{1.125651in}}%
\pgfpathlineto{\pgfqpoint{2.146634in}{1.123240in}}%
\pgfpathlineto{\pgfqpoint{2.151078in}{1.120841in}}%
\pgfpathlineto{\pgfqpoint{2.155523in}{1.118453in}}%
\pgfpathlineto{\pgfqpoint{2.159967in}{1.116077in}}%
\pgfpathlineto{\pgfqpoint{2.164411in}{1.113713in}}%
\pgfpathlineto{\pgfqpoint{2.168856in}{1.111359in}}%
\pgfpathlineto{\pgfqpoint{2.173300in}{1.109017in}}%
\pgfpathlineto{\pgfqpoint{2.177745in}{1.106687in}}%
\pgfpathlineto{\pgfqpoint{2.182189in}{1.104367in}}%
\pgfpathlineto{\pgfqpoint{2.186633in}{1.102059in}}%
\pgfpathlineto{\pgfqpoint{2.191078in}{1.099761in}}%
\pgfpathlineto{\pgfqpoint{2.195522in}{1.097475in}}%
\pgfpathlineto{\pgfqpoint{2.199966in}{1.095199in}}%
\pgfpathlineto{\pgfqpoint{2.204411in}{1.092934in}}%
\pgfpathlineto{\pgfqpoint{2.208855in}{1.090680in}}%
\pgfpathlineto{\pgfqpoint{2.213299in}{1.088436in}}%
\pgfpathlineto{\pgfqpoint{2.217744in}{1.086203in}}%
\pgfpathlineto{\pgfqpoint{2.222188in}{1.083980in}}%
\pgfpathlineto{\pgfqpoint{2.226632in}{1.081768in}}%
\pgfpathlineto{\pgfqpoint{2.231077in}{1.079566in}}%
\pgfpathlineto{\pgfqpoint{2.235521in}{1.077374in}}%
\pgfpathlineto{\pgfqpoint{2.239965in}{1.075193in}}%
\pgfpathlineto{\pgfqpoint{2.244410in}{1.073021in}}%
\pgfpathlineto{\pgfqpoint{2.248854in}{1.070860in}}%
\pgfpathlineto{\pgfqpoint{2.253298in}{1.068708in}}%
\pgfpathlineto{\pgfqpoint{2.257743in}{1.066567in}}%
\pgfpathlineto{\pgfqpoint{2.262187in}{1.064435in}}%
\pgfpathlineto{\pgfqpoint{2.266631in}{1.062313in}}%
\pgfpathlineto{\pgfqpoint{2.271076in}{1.060201in}}%
\pgfpathlineto{\pgfqpoint{2.275520in}{1.058098in}}%
\pgfpathlineto{\pgfqpoint{2.279964in}{1.056005in}}%
\pgfpathlineto{\pgfqpoint{2.284409in}{1.053921in}}%
\pgfpathlineto{\pgfqpoint{2.288853in}{1.051847in}}%
\pgfpathlineto{\pgfqpoint{2.293298in}{1.049782in}}%
\pgfpathlineto{\pgfqpoint{2.297742in}{1.047726in}}%
\pgfpathlineto{\pgfqpoint{2.302186in}{1.045680in}}%
\pgfpathlineto{\pgfqpoint{2.306631in}{1.043643in}}%
\pgfpathlineto{\pgfqpoint{2.311075in}{1.041615in}}%
\pgfpathlineto{\pgfqpoint{2.315519in}{1.039596in}}%
\pgfpathlineto{\pgfqpoint{2.319964in}{1.037586in}}%
\pgfpathlineto{\pgfqpoint{2.324408in}{1.035584in}}%
\pgfpathlineto{\pgfqpoint{2.328852in}{1.033592in}}%
\pgfpathlineto{\pgfqpoint{2.333297in}{1.031609in}}%
\pgfpathlineto{\pgfqpoint{2.337741in}{1.029634in}}%
\pgfpathlineto{\pgfqpoint{2.342185in}{1.027668in}}%
\pgfpathlineto{\pgfqpoint{2.346630in}{1.025711in}}%
\pgfpathlineto{\pgfqpoint{2.351074in}{1.023762in}}%
\pgfpathlineto{\pgfqpoint{2.355518in}{1.021822in}}%
\pgfpathlineto{\pgfqpoint{2.359963in}{1.019890in}}%
\pgfpathlineto{\pgfqpoint{2.364407in}{1.017967in}}%
\pgfpathlineto{\pgfqpoint{2.368851in}{1.016052in}}%
\pgfpathlineto{\pgfqpoint{2.373296in}{1.014145in}}%
\pgfpathlineto{\pgfqpoint{2.377740in}{1.012247in}}%
\pgfpathlineto{\pgfqpoint{2.382184in}{1.010357in}}%
\pgfpathlineto{\pgfqpoint{2.386629in}{1.008474in}}%
\pgfpathlineto{\pgfqpoint{2.391073in}{1.006600in}}%
\pgfpathlineto{\pgfqpoint{2.395517in}{1.004735in}}%
\pgfpathlineto{\pgfqpoint{2.399962in}{1.002877in}}%
\pgfpathlineto{\pgfqpoint{2.404406in}{1.001027in}}%
\pgfpathlineto{\pgfqpoint{2.408851in}{0.999184in}}%
\pgfpathlineto{\pgfqpoint{2.413295in}{0.997350in}}%
\pgfpathlineto{\pgfqpoint{2.417739in}{0.995524in}}%
\pgfpathlineto{\pgfqpoint{2.422184in}{0.993705in}}%
\pgfpathlineto{\pgfqpoint{2.426628in}{0.991894in}}%
\pgfpathlineto{\pgfqpoint{2.431072in}{0.990090in}}%
\pgfpathlineto{\pgfqpoint{2.435517in}{0.988294in}}%
\pgfpathlineto{\pgfqpoint{2.439961in}{0.986506in}}%
\pgfpathlineto{\pgfqpoint{2.444405in}{0.984725in}}%
\pgfpathlineto{\pgfqpoint{2.448850in}{0.982952in}}%
\pgfpathlineto{\pgfqpoint{2.453294in}{0.981186in}}%
\pgfpathlineto{\pgfqpoint{2.457738in}{0.979427in}}%
\pgfpathlineto{\pgfqpoint{2.462183in}{0.977676in}}%
\pgfpathlineto{\pgfqpoint{2.466627in}{0.975932in}}%
\pgfpathlineto{\pgfqpoint{2.471071in}{0.974195in}}%
\pgfpathlineto{\pgfqpoint{2.475516in}{0.972466in}}%
\pgfpathlineto{\pgfqpoint{2.479960in}{0.970743in}}%
\pgfpathlineto{\pgfqpoint{2.484404in}{0.969028in}}%
\pgfpathlineto{\pgfqpoint{2.488849in}{0.967319in}}%
\pgfpathlineto{\pgfqpoint{2.493293in}{0.965618in}}%
\pgfpathlineto{\pgfqpoint{2.497737in}{0.963924in}}%
\pgfpathlineto{\pgfqpoint{2.502182in}{0.962236in}}%
\pgfpathlineto{\pgfqpoint{2.506626in}{0.960556in}}%
\pgfpathlineto{\pgfqpoint{2.511071in}{0.958882in}}%
\pgfpathlineto{\pgfqpoint{2.515515in}{0.957215in}}%
\pgfpathlineto{\pgfqpoint{2.519959in}{0.955554in}}%
\pgfpathlineto{\pgfqpoint{2.524404in}{0.953901in}}%
\pgfpathlineto{\pgfqpoint{2.528848in}{0.952254in}}%
\pgfpathlineto{\pgfqpoint{2.533292in}{0.950614in}}%
\pgfpathlineto{\pgfqpoint{2.537737in}{0.948980in}}%
\pgfpathlineto{\pgfqpoint{2.542181in}{0.947353in}}%
\pgfpathlineto{\pgfqpoint{2.546625in}{0.945732in}}%
\pgfpathlineto{\pgfqpoint{2.551070in}{0.944118in}}%
\pgfpathlineto{\pgfqpoint{2.555514in}{0.942510in}}%
\pgfpathlineto{\pgfqpoint{2.559958in}{0.940909in}}%
\pgfpathlineto{\pgfqpoint{2.564403in}{0.939314in}}%
\pgfpathlineto{\pgfqpoint{2.568847in}{0.937725in}}%
\pgfpathlineto{\pgfqpoint{2.573291in}{0.936143in}}%
\pgfpathlineto{\pgfqpoint{2.577736in}{0.934567in}}%
\pgfpathlineto{\pgfqpoint{2.582180in}{0.932997in}}%
\pgfpathlineto{\pgfqpoint{2.586624in}{0.931433in}}%
\pgfpathlineto{\pgfqpoint{2.591069in}{0.929875in}}%
\pgfpathlineto{\pgfqpoint{2.595513in}{0.928324in}}%
\pgfpathlineto{\pgfqpoint{2.599957in}{0.926778in}}%
\pgfpathlineto{\pgfqpoint{2.604402in}{0.925239in}}%
\pgfpathlineto{\pgfqpoint{2.608846in}{0.923705in}}%
\pgfpathlineto{\pgfqpoint{2.613290in}{0.922178in}}%
\pgfpathlineto{\pgfqpoint{2.617735in}{0.920656in}}%
\pgfpathlineto{\pgfqpoint{2.622179in}{0.919141in}}%
\pgfpathlineto{\pgfqpoint{2.626624in}{0.917631in}}%
\pgfpathlineto{\pgfqpoint{2.631068in}{0.916127in}}%
\pgfpathlineto{\pgfqpoint{2.635512in}{0.914628in}}%
\pgfpathlineto{\pgfqpoint{2.639957in}{0.913136in}}%
\pgfpathlineto{\pgfqpoint{2.644401in}{0.911649in}}%
\pgfpathlineto{\pgfqpoint{2.648845in}{0.910168in}}%
\pgfpathlineto{\pgfqpoint{2.653290in}{0.908692in}}%
\pgfpathlineto{\pgfqpoint{2.657734in}{0.907223in}}%
\pgfpathlineto{\pgfqpoint{2.662178in}{0.905758in}}%
\pgfpathlineto{\pgfqpoint{2.666623in}{0.904300in}}%
\pgfpathlineto{\pgfqpoint{2.671067in}{0.902846in}}%
\pgfpathlineto{\pgfqpoint{2.675511in}{0.901399in}}%
\pgfpathlineto{\pgfqpoint{2.679956in}{0.899957in}}%
\pgfpathlineto{\pgfqpoint{2.684400in}{0.898520in}}%
\pgfpathlineto{\pgfqpoint{2.688844in}{0.897088in}}%
\pgfpathlineto{\pgfqpoint{2.693289in}{0.895662in}}%
\pgfpathlineto{\pgfqpoint{2.697733in}{0.894242in}}%
\pgfpathlineto{\pgfqpoint{2.702177in}{0.892826in}}%
\pgfpathlineto{\pgfqpoint{2.706622in}{0.891416in}}%
\pgfpathlineto{\pgfqpoint{2.711066in}{0.890011in}}%
\pgfpathlineto{\pgfqpoint{2.715510in}{0.888612in}}%
\pgfpathlineto{\pgfqpoint{2.719955in}{0.887217in}}%
\pgfpathlineto{\pgfqpoint{2.724399in}{0.885828in}}%
\pgfpathlineto{\pgfqpoint{2.728843in}{0.884444in}}%
\pgfpathlineto{\pgfqpoint{2.733288in}{0.883065in}}%
\pgfpathlineto{\pgfqpoint{2.737732in}{0.881691in}}%
\pgfpathlineto{\pgfqpoint{2.742177in}{0.880322in}}%
\pgfpathlineto{\pgfqpoint{2.746621in}{0.878958in}}%
\pgfpathlineto{\pgfqpoint{2.751065in}{0.877599in}}%
\pgfpathlineto{\pgfqpoint{2.755510in}{0.876245in}}%
\pgfpathlineto{\pgfqpoint{2.759954in}{0.874896in}}%
\pgfpathlineto{\pgfqpoint{2.764398in}{0.873552in}}%
\pgfpathlineto{\pgfqpoint{2.768843in}{0.872213in}}%
\pgfpathlineto{\pgfqpoint{2.773287in}{0.870879in}}%
\pgfpathlineto{\pgfqpoint{2.777731in}{0.869549in}}%
\pgfpathlineto{\pgfqpoint{2.782176in}{0.868225in}}%
\pgfpathlineto{\pgfqpoint{2.786620in}{0.866905in}}%
\pgfpathlineto{\pgfqpoint{2.791064in}{0.865589in}}%
\pgfpathlineto{\pgfqpoint{2.795509in}{0.864279in}}%
\pgfpathlineto{\pgfqpoint{2.799953in}{0.862973in}}%
\pgfpathlineto{\pgfqpoint{2.804397in}{0.861672in}}%
\pgfpathlineto{\pgfqpoint{2.808842in}{0.860376in}}%
\pgfpathlineto{\pgfqpoint{2.813286in}{0.859084in}}%
\pgfpathlineto{\pgfqpoint{2.817730in}{0.857797in}}%
\pgfpathlineto{\pgfqpoint{2.822175in}{0.856515in}}%
\pgfpathlineto{\pgfqpoint{2.826619in}{0.855236in}}%
\pgfpathlineto{\pgfqpoint{2.831063in}{0.853963in}}%
\pgfpathlineto{\pgfqpoint{2.835508in}{0.852694in}}%
\pgfpathlineto{\pgfqpoint{2.839952in}{0.851430in}}%
\pgfpathlineto{\pgfqpoint{2.844396in}{0.850170in}}%
\pgfpathlineto{\pgfqpoint{2.848841in}{0.848914in}}%
\pgfpathlineto{\pgfqpoint{2.853285in}{0.847663in}}%
\pgfpathlineto{\pgfqpoint{2.857730in}{0.846416in}}%
\pgfpathlineto{\pgfqpoint{2.862174in}{0.845174in}}%
\pgfpathlineto{\pgfqpoint{2.866618in}{0.843935in}}%
\pgfpathlineto{\pgfqpoint{2.871063in}{0.842702in}}%
\pgfpathlineto{\pgfqpoint{2.875507in}{0.841472in}}%
\pgfpathlineto{\pgfqpoint{2.879951in}{0.840247in}}%
\pgfpathlineto{\pgfqpoint{2.884396in}{0.839026in}}%
\pgfpathlineto{\pgfqpoint{2.888840in}{0.837809in}}%
\pgfpathlineto{\pgfqpoint{2.893284in}{0.836597in}}%
\pgfpathlineto{\pgfqpoint{2.897729in}{0.835389in}}%
\pgfpathlineto{\pgfqpoint{2.902173in}{0.834184in}}%
\pgfpathlineto{\pgfqpoint{2.906617in}{0.832984in}}%
\pgfpathlineto{\pgfqpoint{2.911062in}{0.831789in}}%
\pgfpathlineto{\pgfqpoint{2.915506in}{0.830597in}}%
\pgfpathlineto{\pgfqpoint{2.919950in}{0.829409in}}%
\pgfpathlineto{\pgfqpoint{2.924395in}{0.828225in}}%
\pgfpathlineto{\pgfqpoint{2.928839in}{0.827046in}}%
\pgfpathlineto{\pgfqpoint{2.933283in}{0.825870in}}%
\pgfpathlineto{\pgfqpoint{2.937728in}{0.824699in}}%
\pgfpathlineto{\pgfqpoint{2.942172in}{0.823531in}}%
\pgfpathlineto{\pgfqpoint{2.946616in}{0.822367in}}%
\pgfpathlineto{\pgfqpoint{2.951061in}{0.821208in}}%
\pgfpathlineto{\pgfqpoint{2.955505in}{0.820052in}}%
\pgfpathlineto{\pgfqpoint{2.959949in}{0.818900in}}%
\pgfpathlineto{\pgfqpoint{2.964394in}{0.817752in}}%
\pgfpathlineto{\pgfqpoint{2.968838in}{0.816608in}}%
\pgfpathlineto{\pgfqpoint{2.973283in}{0.815468in}}%
\pgfpathlineto{\pgfqpoint{2.977727in}{0.814331in}}%
\pgfpathlineto{\pgfqpoint{2.982171in}{0.813198in}}%
\pgfpathlineto{\pgfqpoint{2.986616in}{0.812069in}}%
\pgfpathlineto{\pgfqpoint{2.991060in}{0.810944in}}%
\pgfpathlineto{\pgfqpoint{2.995504in}{0.809823in}}%
\pgfpathlineto{\pgfqpoint{2.999949in}{0.808705in}}%
\pgfpathlineto{\pgfqpoint{3.004393in}{0.807591in}}%
\pgfpathlineto{\pgfqpoint{3.008837in}{0.806481in}}%
\pgfpathlineto{\pgfqpoint{3.013282in}{0.805374in}}%
\pgfpathlineto{\pgfqpoint{3.017726in}{0.804271in}}%
\pgfpathlineto{\pgfqpoint{3.022170in}{0.803172in}}%
\pgfpathlineto{\pgfqpoint{3.026615in}{0.802076in}}%
\pgfpathlineto{\pgfqpoint{3.031059in}{0.800984in}}%
\pgfpathlineto{\pgfqpoint{3.035503in}{0.799896in}}%
\pgfpathlineto{\pgfqpoint{3.039948in}{0.798811in}}%
\pgfpathlineto{\pgfqpoint{3.044392in}{0.797729in}}%
\pgfpathlineto{\pgfqpoint{3.048836in}{0.796651in}}%
\pgfpathlineto{\pgfqpoint{3.053281in}{0.795577in}}%
\pgfpathlineto{\pgfqpoint{3.057725in}{0.794506in}}%
\pgfpathlineto{\pgfqpoint{3.062169in}{0.793439in}}%
\pgfpathlineto{\pgfqpoint{3.066614in}{0.792375in}}%
\pgfpathlineto{\pgfqpoint{3.071058in}{0.791314in}}%
\pgfpathlineto{\pgfqpoint{3.075502in}{0.790257in}}%
\pgfpathlineto{\pgfqpoint{3.079947in}{0.789203in}}%
\pgfpathlineto{\pgfqpoint{3.084391in}{0.788153in}}%
\pgfpathlineto{\pgfqpoint{3.088836in}{0.787106in}}%
\pgfpathlineto{\pgfqpoint{3.093280in}{0.786062in}}%
\pgfpathlineto{\pgfqpoint{3.097724in}{0.785022in}}%
\pgfpathlineto{\pgfqpoint{3.102169in}{0.783985in}}%
\pgfpathlineto{\pgfqpoint{3.106613in}{0.782952in}}%
\pgfpathlineto{\pgfqpoint{3.111057in}{0.781921in}}%
\pgfpathlineto{\pgfqpoint{3.115502in}{0.780894in}}%
\pgfpathlineto{\pgfqpoint{3.119946in}{0.779871in}}%
\pgfpathlineto{\pgfqpoint{3.124390in}{0.778850in}}%
\pgfpathlineto{\pgfqpoint{3.128835in}{0.777833in}}%
\pgfpathlineto{\pgfqpoint{3.133279in}{0.776819in}}%
\pgfpathlineto{\pgfqpoint{3.137723in}{0.775808in}}%
\pgfpathlineto{\pgfqpoint{3.142168in}{0.774800in}}%
\pgfpathlineto{\pgfqpoint{3.146612in}{0.773796in}}%
\pgfpathlineto{\pgfqpoint{3.151056in}{0.772795in}}%
\pgfpathlineto{\pgfqpoint{3.155501in}{0.771796in}}%
\pgfpathlineto{\pgfqpoint{3.159945in}{0.770801in}}%
\pgfpathlineto{\pgfqpoint{3.164389in}{0.769809in}}%
\pgfpathlineto{\pgfqpoint{3.168834in}{0.768821in}}%
\pgfpathlineto{\pgfqpoint{3.173278in}{0.767835in}}%
\pgfpathlineto{\pgfqpoint{3.177722in}{0.766852in}}%
\pgfpathlineto{\pgfqpoint{3.182167in}{0.765873in}}%
\pgfpathlineto{\pgfqpoint{3.186611in}{0.764896in}}%
\pgfpathlineto{\pgfqpoint{3.191056in}{0.763922in}}%
\pgfpathlineto{\pgfqpoint{3.195500in}{0.762952in}}%
\pgfpathlineto{\pgfqpoint{3.199944in}{0.761984in}}%
\pgfpathlineto{\pgfqpoint{3.204389in}{0.761020in}}%
\pgfpathlineto{\pgfqpoint{3.208833in}{0.760058in}}%
\pgfpathlineto{\pgfqpoint{3.213277in}{0.759100in}}%
\pgfpathlineto{\pgfqpoint{3.217722in}{0.758144in}}%
\pgfpathlineto{\pgfqpoint{3.222166in}{0.757191in}}%
\pgfpathlineto{\pgfqpoint{3.226610in}{0.756241in}}%
\pgfpathlineto{\pgfqpoint{3.231055in}{0.755294in}}%
\pgfpathlineto{\pgfqpoint{3.235499in}{0.754350in}}%
\pgfpathlineto{\pgfqpoint{3.239943in}{0.753409in}}%
\pgfpathlineto{\pgfqpoint{3.244388in}{0.752471in}}%
\pgfpathlineto{\pgfqpoint{3.248832in}{0.751536in}}%
\pgfpathlineto{\pgfqpoint{3.253276in}{0.750603in}}%
\pgfpathlineto{\pgfqpoint{3.257721in}{0.749673in}}%
\pgfpathlineto{\pgfqpoint{3.262165in}{0.748746in}}%
\pgfpathlineto{\pgfqpoint{3.266609in}{0.747822in}}%
\pgfpathlineto{\pgfqpoint{3.271054in}{0.746901in}}%
\pgfpathlineto{\pgfqpoint{3.275498in}{0.745982in}}%
\pgfpathlineto{\pgfqpoint{3.279942in}{0.745066in}}%
\pgfpathlineto{\pgfqpoint{3.284387in}{0.744153in}}%
\pgfpathlineto{\pgfqpoint{3.288831in}{0.743243in}}%
\pgfpathlineto{\pgfqpoint{3.293275in}{0.742335in}}%
\pgfpathlineto{\pgfqpoint{3.297720in}{0.741430in}}%
\pgfpathlineto{\pgfqpoint{3.302164in}{0.740528in}}%
\pgfpathlineto{\pgfqpoint{3.306609in}{0.739628in}}%
\pgfpathlineto{\pgfqpoint{3.311053in}{0.738732in}}%
\pgfpathlineto{\pgfqpoint{3.315497in}{0.737837in}}%
\pgfpathlineto{\pgfqpoint{3.319942in}{0.736946in}}%
\pgfpathlineto{\pgfqpoint{3.324386in}{0.736057in}}%
\pgfpathlineto{\pgfqpoint{3.328830in}{0.735171in}}%
\pgfpathlineto{\pgfqpoint{3.333275in}{0.734287in}}%
\pgfpathlineto{\pgfqpoint{3.337719in}{0.733406in}}%
\pgfpathlineto{\pgfqpoint{3.342163in}{0.732528in}}%
\pgfpathlineto{\pgfqpoint{3.346608in}{0.731652in}}%
\pgfpathlineto{\pgfqpoint{3.351052in}{0.730778in}}%
\pgfpathlineto{\pgfqpoint{3.355496in}{0.729908in}}%
\pgfpathlineto{\pgfqpoint{3.359941in}{0.729040in}}%
\pgfpathlineto{\pgfqpoint{3.364385in}{0.728174in}}%
\pgfpathlineto{\pgfqpoint{3.368829in}{0.727311in}}%
\pgfpathlineto{\pgfqpoint{3.373274in}{0.726450in}}%
\pgfpathlineto{\pgfqpoint{3.377718in}{0.725592in}}%
\pgfpathlineto{\pgfqpoint{3.382162in}{0.724737in}}%
\pgfpathlineto{\pgfqpoint{3.386607in}{0.723883in}}%
\pgfpathlineto{\pgfqpoint{3.391051in}{0.723033in}}%
\pgfpathlineto{\pgfqpoint{3.395495in}{0.722185in}}%
\pgfpathlineto{\pgfqpoint{3.399940in}{0.721339in}}%
\pgfpathlineto{\pgfqpoint{3.404384in}{0.720496in}}%
\pgfpathlineto{\pgfqpoint{3.408828in}{0.719655in}}%
\pgfpathlineto{\pgfqpoint{3.413273in}{0.718816in}}%
\pgfpathlineto{\pgfqpoint{3.417717in}{0.717980in}}%
\pgfpathlineto{\pgfqpoint{3.422162in}{0.717147in}}%
\pgfpathlineto{\pgfqpoint{3.426606in}{0.716315in}}%
\pgfpathlineto{\pgfqpoint{3.431050in}{0.715486in}}%
\pgfpathlineto{\pgfqpoint{3.435495in}{0.714660in}}%
\pgfpathlineto{\pgfqpoint{3.439939in}{0.713836in}}%
\pgfpathlineto{\pgfqpoint{3.444383in}{0.713014in}}%
\pgfpathlineto{\pgfqpoint{3.448828in}{0.712195in}}%
\pgfpathlineto{\pgfqpoint{3.453272in}{0.711377in}}%
\pgfpathlineto{\pgfqpoint{3.457716in}{0.710563in}}%
\pgfpathlineto{\pgfqpoint{3.462161in}{0.709750in}}%
\pgfpathlineto{\pgfqpoint{3.466605in}{0.708940in}}%
\pgfpathlineto{\pgfqpoint{3.471049in}{0.708132in}}%
\pgfpathlineto{\pgfqpoint{3.475494in}{0.707326in}}%
\pgfpathlineto{\pgfqpoint{3.479938in}{0.706523in}}%
\pgfpathlineto{\pgfqpoint{3.484382in}{0.705722in}}%
\pgfpathlineto{\pgfqpoint{3.488827in}{0.704923in}}%
\pgfpathlineto{\pgfqpoint{3.493271in}{0.704126in}}%
\pgfpathlineto{\pgfqpoint{3.497715in}{0.703332in}}%
\pgfpathlineto{\pgfqpoint{3.502160in}{0.702540in}}%
\pgfpathlineto{\pgfqpoint{3.506604in}{0.701750in}}%
\pgfpathlineto{\pgfqpoint{3.511048in}{0.700962in}}%
\pgfpathlineto{\pgfqpoint{3.515493in}{0.700177in}}%
\pgfpathlineto{\pgfqpoint{3.519937in}{0.699393in}}%
\pgfpathlineto{\pgfqpoint{3.524381in}{0.698612in}}%
\pgfpathlineto{\pgfqpoint{3.528826in}{0.697833in}}%
\pgfpathlineto{\pgfqpoint{3.533270in}{0.697056in}}%
\pgfpathlineto{\pgfqpoint{3.537715in}{0.696281in}}%
\pgfpathlineto{\pgfqpoint{3.542159in}{0.695509in}}%
\pgfpathlineto{\pgfqpoint{3.546603in}{0.694738in}}%
\pgfpathlineto{\pgfqpoint{3.551048in}{0.693970in}}%
\pgfpathlineto{\pgfqpoint{3.555492in}{0.693204in}}%
\pgfpathlineto{\pgfqpoint{3.559936in}{0.692440in}}%
\pgfpathlineto{\pgfqpoint{3.564381in}{0.691678in}}%
\pgfpathlineto{\pgfqpoint{3.568825in}{0.690918in}}%
\pgfpathlineto{\pgfqpoint{3.573269in}{0.690160in}}%
\pgfpathlineto{\pgfqpoint{3.577714in}{0.689404in}}%
\pgfpathlineto{\pgfqpoint{3.582158in}{0.688650in}}%
\pgfpathlineto{\pgfqpoint{3.586602in}{0.687899in}}%
\pgfpathlineto{\pgfqpoint{3.591047in}{0.687149in}}%
\pgfpathlineto{\pgfqpoint{3.595491in}{0.686402in}}%
\pgfpathlineto{\pgfqpoint{3.599935in}{0.685656in}}%
\pgfpathlineto{\pgfqpoint{3.604380in}{0.684912in}}%
\pgfpathlineto{\pgfqpoint{3.608824in}{0.684171in}}%
\pgfpathlineto{\pgfqpoint{3.613268in}{0.683431in}}%
\pgfpathlineto{\pgfqpoint{3.617713in}{0.682694in}}%
\pgfpathlineto{\pgfqpoint{3.622157in}{0.681958in}}%
\pgfpathlineto{\pgfqpoint{3.626601in}{0.681225in}}%
\pgfpathlineto{\pgfqpoint{3.631046in}{0.680493in}}%
\pgfpathlineto{\pgfqpoint{3.635490in}{0.679764in}}%
\pgfpathlineto{\pgfqpoint{3.639934in}{0.679036in}}%
\pgfpathlineto{\pgfqpoint{3.644379in}{0.678310in}}%
\pgfpathlineto{\pgfqpoint{3.648823in}{0.677586in}}%
\pgfpathlineto{\pgfqpoint{3.653268in}{0.676865in}}%
\pgfpathlineto{\pgfqpoint{3.657712in}{0.676145in}}%
\pgfpathlineto{\pgfqpoint{3.662156in}{0.675427in}}%
\pgfpathlineto{\pgfqpoint{3.666601in}{0.674711in}}%
\pgfpathlineto{\pgfqpoint{3.671045in}{0.673996in}}%
\pgfpathlineto{\pgfqpoint{3.675489in}{0.673284in}}%
\pgfpathlineto{\pgfqpoint{3.679934in}{0.672574in}}%
\pgfpathlineto{\pgfqpoint{3.684378in}{0.671865in}}%
\pgfpathlineto{\pgfqpoint{3.688822in}{0.671158in}}%
\pgfpathlineto{\pgfqpoint{3.693267in}{0.670454in}}%
\pgfpathlineto{\pgfqpoint{3.697711in}{0.669751in}}%
\pgfpathlineto{\pgfqpoint{3.702155in}{0.669050in}}%
\pgfpathlineto{\pgfqpoint{3.706600in}{0.668350in}}%
\pgfpathlineto{\pgfqpoint{3.711044in}{0.667653in}}%
\pgfpathlineto{\pgfqpoint{3.715488in}{0.666957in}}%
\pgfpathlineto{\pgfqpoint{3.719933in}{0.666264in}}%
\pgfpathlineto{\pgfqpoint{3.724377in}{0.665572in}}%
\pgfpathlineto{\pgfqpoint{3.728821in}{0.664881in}}%
\pgfpathlineto{\pgfqpoint{3.733266in}{0.664193in}}%
\pgfpathlineto{\pgfqpoint{3.737710in}{0.663507in}}%
\pgfpathlineto{\pgfqpoint{3.742154in}{0.662822in}}%
\pgfpathlineto{\pgfqpoint{3.746599in}{0.662139in}}%
\pgfpathlineto{\pgfqpoint{3.751043in}{0.661458in}}%
\pgfpathlineto{\pgfqpoint{3.755487in}{0.660778in}}%
\pgfpathlineto{\pgfqpoint{3.759932in}{0.660100in}}%
\pgfpathlineto{\pgfqpoint{3.764376in}{0.659424in}}%
\pgfpathlineto{\pgfqpoint{3.768821in}{0.658750in}}%
\pgfpathlineto{\pgfqpoint{3.773265in}{0.658078in}}%
\pgfpathlineto{\pgfqpoint{3.777709in}{0.657407in}}%
\pgfpathlineto{\pgfqpoint{3.782154in}{0.656738in}}%
\pgfpathlineto{\pgfqpoint{3.786598in}{0.656071in}}%
\pgfpathlineto{\pgfqpoint{3.791042in}{0.655405in}}%
\pgfpathlineto{\pgfqpoint{3.795487in}{0.654741in}}%
\pgfpathlineto{\pgfqpoint{3.799931in}{0.654079in}}%
\pgfpathlineto{\pgfqpoint{3.804375in}{0.653419in}}%
\pgfpathlineto{\pgfqpoint{3.808820in}{0.652760in}}%
\pgfpathlineto{\pgfqpoint{3.813264in}{0.652103in}}%
\pgfpathlineto{\pgfqpoint{3.817708in}{0.651447in}}%
\pgfpathlineto{\pgfqpoint{3.822153in}{0.650793in}}%
\pgfpathlineto{\pgfqpoint{3.826597in}{0.650141in}}%
\pgfpathlineto{\pgfqpoint{3.831041in}{0.649491in}}%
\pgfpathlineto{\pgfqpoint{3.835486in}{0.648842in}}%
\pgfpathlineto{\pgfqpoint{3.839930in}{0.648195in}}%
\pgfpathlineto{\pgfqpoint{3.844374in}{0.647549in}}%
\pgfpathlineto{\pgfqpoint{3.848819in}{0.646905in}}%
\pgfpathlineto{\pgfqpoint{3.853263in}{0.646263in}}%
\pgfpathlineto{\pgfqpoint{3.857707in}{0.645622in}}%
\pgfpathlineto{\pgfqpoint{3.862152in}{0.644983in}}%
\pgfpathlineto{\pgfqpoint{3.866596in}{0.644346in}}%
\pgfpathlineto{\pgfqpoint{3.871040in}{0.643710in}}%
\pgfpathlineto{\pgfqpoint{3.875485in}{0.643076in}}%
\pgfpathlineto{\pgfqpoint{3.879929in}{0.642443in}}%
\pgfpathlineto{\pgfqpoint{3.884374in}{0.641812in}}%
\pgfpathlineto{\pgfqpoint{3.888818in}{0.641182in}}%
\pgfpathlineto{\pgfqpoint{3.893262in}{0.640554in}}%
\pgfpathlineto{\pgfqpoint{3.897707in}{0.639928in}}%
\pgfpathlineto{\pgfqpoint{3.902151in}{0.639303in}}%
\pgfpathlineto{\pgfqpoint{3.906595in}{0.638680in}}%
\pgfpathlineto{\pgfqpoint{3.911040in}{0.638058in}}%
\pgfpathlineto{\pgfqpoint{3.915484in}{0.637438in}}%
\pgfpathlineto{\pgfqpoint{3.919928in}{0.636819in}}%
\pgfpathlineto{\pgfqpoint{3.924373in}{0.636202in}}%
\pgfpathlineto{\pgfqpoint{3.928817in}{0.635586in}}%
\pgfpathlineto{\pgfqpoint{3.933261in}{0.634972in}}%
\pgfpathlineto{\pgfqpoint{3.937706in}{0.634359in}}%
\pgfpathlineto{\pgfqpoint{3.942150in}{0.633748in}}%
\pgfpathlineto{\pgfqpoint{3.946594in}{0.633139in}}%
\pgfpathlineto{\pgfqpoint{3.951039in}{0.632530in}}%
\pgfpathlineto{\pgfqpoint{3.955483in}{0.631924in}}%
\pgfpathlineto{\pgfqpoint{3.959927in}{0.631319in}}%
\pgfpathlineto{\pgfqpoint{3.964372in}{0.630715in}}%
\pgfpathlineto{\pgfqpoint{3.968816in}{0.630113in}}%
\pgfpathlineto{\pgfqpoint{3.973260in}{0.629512in}}%
\pgfpathlineto{\pgfqpoint{3.977705in}{0.628913in}}%
\pgfpathlineto{\pgfqpoint{3.982149in}{0.628315in}}%
\pgfpathlineto{\pgfqpoint{3.986594in}{0.627719in}}%
\pgfpathlineto{\pgfqpoint{3.991038in}{0.627124in}}%
\pgfpathlineto{\pgfqpoint{3.995482in}{0.626530in}}%
\pgfpathlineto{\pgfqpoint{3.999927in}{0.625938in}}%
\pgfpathlineto{\pgfqpoint{4.004371in}{0.625347in}}%
\pgfpathlineto{\pgfqpoint{4.008815in}{0.624758in}}%
\pgfpathlineto{\pgfqpoint{4.013260in}{0.624171in}}%
\pgfpathlineto{\pgfqpoint{4.017704in}{0.623584in}}%
\pgfpathlineto{\pgfqpoint{4.022148in}{0.622999in}}%
\pgfpathlineto{\pgfqpoint{4.026593in}{0.622416in}}%
\pgfpathlineto{\pgfqpoint{4.031037in}{0.621834in}}%
\pgfpathlineto{\pgfqpoint{4.035481in}{0.621253in}}%
\pgfpathlineto{\pgfqpoint{4.039926in}{0.620673in}}%
\pgfpathlineto{\pgfqpoint{4.044370in}{0.620095in}}%
\pgfpathlineto{\pgfqpoint{4.048814in}{0.619519in}}%
\pgfpathlineto{\pgfqpoint{4.053259in}{0.618944in}}%
\pgfpathlineto{\pgfqpoint{4.057703in}{0.618370in}}%
\pgfpathlineto{\pgfqpoint{4.062147in}{0.617797in}}%
\pgfpathlineto{\pgfqpoint{4.066592in}{0.617226in}}%
\pgfpathlineto{\pgfqpoint{4.071036in}{0.616656in}}%
\pgfpathlineto{\pgfqpoint{4.075480in}{0.616088in}}%
\pgfpathlineto{\pgfqpoint{4.079925in}{0.615521in}}%
\pgfpathlineto{\pgfqpoint{4.084369in}{0.614955in}}%
\pgfpathlineto{\pgfqpoint{4.088813in}{0.614391in}}%
\pgfpathlineto{\pgfqpoint{4.093258in}{0.613828in}}%
\pgfpathlineto{\pgfqpoint{4.097702in}{0.613266in}}%
\pgfpathlineto{\pgfqpoint{4.102147in}{0.612705in}}%
\pgfpathlineto{\pgfqpoint{4.106591in}{0.612146in}}%
\pgfpathlineto{\pgfqpoint{4.111035in}{0.611588in}}%
\pgfpathlineto{\pgfqpoint{4.115480in}{0.611032in}}%
\pgfpathlineto{\pgfqpoint{4.119924in}{0.610477in}}%
\pgfpathlineto{\pgfqpoint{4.124368in}{0.609923in}}%
\pgfpathlineto{\pgfqpoint{4.128813in}{0.609370in}}%
\pgfpathlineto{\pgfqpoint{4.133257in}{0.608819in}}%
\pgfpathlineto{\pgfqpoint{4.137701in}{0.608269in}}%
\pgfpathlineto{\pgfqpoint{4.142146in}{0.607720in}}%
\pgfpathlineto{\pgfqpoint{4.146590in}{0.607173in}}%
\pgfpathlineto{\pgfqpoint{4.151034in}{0.606627in}}%
\pgfpathlineto{\pgfqpoint{4.155479in}{0.606082in}}%
\pgfpathlineto{\pgfqpoint{4.159923in}{0.605538in}}%
\pgfpathlineto{\pgfqpoint{4.164367in}{0.604996in}}%
\pgfpathlineto{\pgfqpoint{4.168812in}{0.604454in}}%
\pgfpathlineto{\pgfqpoint{4.173256in}{0.603915in}}%
\pgfpathlineto{\pgfqpoint{4.177700in}{0.603376in}}%
\pgfpathlineto{\pgfqpoint{4.182145in}{0.602838in}}%
\pgfpathlineto{\pgfqpoint{4.186589in}{0.602302in}}%
\pgfpathlineto{\pgfqpoint{4.191033in}{0.601767in}}%
\pgfpathlineto{\pgfqpoint{4.195478in}{0.601234in}}%
\pgfpathlineto{\pgfqpoint{4.199922in}{0.600701in}}%
\pgfpathlineto{\pgfqpoint{4.204366in}{0.600170in}}%
\pgfpathlineto{\pgfqpoint{4.208811in}{0.599640in}}%
\pgfpathlineto{\pgfqpoint{4.213255in}{0.599111in}}%
\pgfpathlineto{\pgfqpoint{4.217700in}{0.598583in}}%
\pgfpathlineto{\pgfqpoint{4.222144in}{0.598057in}}%
\pgfpathlineto{\pgfqpoint{4.226588in}{0.597532in}}%
\pgfpathlineto{\pgfqpoint{4.231033in}{0.597008in}}%
\pgfpathlineto{\pgfqpoint{4.235477in}{0.596485in}}%
\pgfpathlineto{\pgfqpoint{4.239921in}{0.595963in}}%
\pgfpathlineto{\pgfqpoint{4.244366in}{0.595443in}}%
\pgfpathlineto{\pgfqpoint{4.248810in}{0.594923in}}%
\pgfpathlineto{\pgfqpoint{4.253254in}{0.594405in}}%
\pgfpathlineto{\pgfqpoint{4.257699in}{0.593888in}}%
\pgfpathlineto{\pgfqpoint{4.262143in}{0.593372in}}%
\pgfpathlineto{\pgfqpoint{4.266587in}{0.592858in}}%
\pgfpathlineto{\pgfqpoint{4.271032in}{0.592344in}}%
\pgfpathlineto{\pgfqpoint{4.275476in}{0.591832in}}%
\pgfpathlineto{\pgfqpoint{4.279920in}{0.591321in}}%
\pgfpathlineto{\pgfqpoint{4.284365in}{0.590811in}}%
\pgfpathlineto{\pgfqpoint{4.288809in}{0.590302in}}%
\pgfpathlineto{\pgfqpoint{4.293253in}{0.589794in}}%
\pgfpathlineto{\pgfqpoint{4.297698in}{0.589288in}}%
\pgfpathlineto{\pgfqpoint{4.302142in}{0.588782in}}%
\pgfpathlineto{\pgfqpoint{4.306586in}{0.588278in}}%
\pgfpathlineto{\pgfqpoint{4.311031in}{0.587775in}}%
\pgfpathlineto{\pgfqpoint{4.315475in}{0.587272in}}%
\pgfpathlineto{\pgfqpoint{4.319919in}{0.586771in}}%
\pgfpathlineto{\pgfqpoint{4.324364in}{0.586272in}}%
\pgfpathlineto{\pgfqpoint{4.328808in}{0.585773in}}%
\pgfpathlineto{\pgfqpoint{4.333253in}{0.585275in}}%
\pgfpathlineto{\pgfqpoint{4.337697in}{0.584779in}}%
\pgfpathlineto{\pgfqpoint{4.342141in}{0.584283in}}%
\pgfpathlineto{\pgfqpoint{4.346586in}{0.583789in}}%
\pgfpathlineto{\pgfqpoint{4.351030in}{0.583296in}}%
\pgfpathlineto{\pgfqpoint{4.355474in}{0.582803in}}%
\pgfpathlineto{\pgfqpoint{4.359919in}{0.582312in}}%
\pgfpathlineto{\pgfqpoint{4.364363in}{0.581822in}}%
\pgfpathlineto{\pgfqpoint{4.368807in}{0.581333in}}%
\pgfpathlineto{\pgfqpoint{4.373252in}{0.580846in}}%
\pgfpathlineto{\pgfqpoint{4.377696in}{0.580359in}}%
\pgfpathlineto{\pgfqpoint{4.382140in}{0.579873in}}%
\pgfpathlineto{\pgfqpoint{4.386585in}{0.579388in}}%
\pgfpathlineto{\pgfqpoint{4.391029in}{0.578905in}}%
\pgfpathlineto{\pgfqpoint{4.395473in}{0.578422in}}%
\pgfpathlineto{\pgfqpoint{4.399918in}{0.577941in}}%
\pgfpathlineto{\pgfqpoint{4.404362in}{0.577460in}}%
\pgfpathlineto{\pgfqpoint{4.408806in}{0.576981in}}%
\pgfpathlineto{\pgfqpoint{4.413251in}{0.576503in}}%
\pgfpathlineto{\pgfqpoint{4.417695in}{0.576025in}}%
\pgfpathlineto{\pgfqpoint{4.422139in}{0.575549in}}%
\pgfpathlineto{\pgfqpoint{4.426584in}{0.575074in}}%
\pgfpathlineto{\pgfqpoint{4.431028in}{0.574599in}}%
\pgfpathlineto{\pgfqpoint{4.435472in}{0.574126in}}%
\pgfpathlineto{\pgfqpoint{4.439917in}{0.573654in}}%
\pgfpathlineto{\pgfqpoint{4.444361in}{0.573183in}}%
\pgfpathlineto{\pgfqpoint{4.448806in}{0.572713in}}%
\pgfpathlineto{\pgfqpoint{4.453250in}{0.572244in}}%
\pgfpathlineto{\pgfqpoint{4.457694in}{0.571776in}}%
\pgfpathlineto{\pgfqpoint{4.462139in}{0.571308in}}%
\pgfpathlineto{\pgfqpoint{4.466583in}{0.570842in}}%
\pgfpathlineto{\pgfqpoint{4.471027in}{0.570377in}}%
\pgfpathlineto{\pgfqpoint{4.475472in}{0.569913in}}%
\pgfpathlineto{\pgfqpoint{4.479916in}{0.569450in}}%
\pgfpathlineto{\pgfqpoint{4.484360in}{0.568988in}}%
\pgfpathlineto{\pgfqpoint{4.488805in}{0.568527in}}%
\pgfpathlineto{\pgfqpoint{4.493249in}{0.568066in}}%
\pgfpathlineto{\pgfqpoint{4.497693in}{0.567607in}}%
\pgfpathlineto{\pgfqpoint{4.502138in}{0.567149in}}%
\pgfpathlineto{\pgfqpoint{4.506582in}{0.566692in}}%
\pgfpathlineto{\pgfqpoint{4.511026in}{0.566236in}}%
\pgfpathlineto{\pgfqpoint{4.515471in}{0.565780in}}%
\pgfpathlineto{\pgfqpoint{4.519915in}{0.565326in}}%
\pgfpathlineto{\pgfqpoint{4.524359in}{0.564873in}}%
\pgfpathlineto{\pgfqpoint{4.528804in}{0.564420in}}%
\pgfpathlineto{\pgfqpoint{4.533248in}{0.563969in}}%
\pgfpathlineto{\pgfqpoint{4.537692in}{0.563518in}}%
\pgfpathlineto{\pgfqpoint{4.542137in}{0.563069in}}%
\pgfpathlineto{\pgfqpoint{4.546581in}{0.562620in}}%
\pgfpathlineto{\pgfqpoint{4.551025in}{0.562172in}}%
\pgfpathlineto{\pgfqpoint{4.555470in}{0.561726in}}%
\pgfpathlineto{\pgfqpoint{4.559914in}{0.561280in}}%
\pgfpathlineto{\pgfqpoint{4.564359in}{0.560835in}}%
\pgfpathlineto{\pgfqpoint{4.568803in}{0.560391in}}%
\pgfpathlineto{\pgfqpoint{4.573247in}{0.559948in}}%
\pgfpathlineto{\pgfqpoint{4.577692in}{0.559506in}}%
\pgfpathlineto{\pgfqpoint{4.582136in}{0.559065in}}%
\pgfpathlineto{\pgfqpoint{4.586580in}{0.558625in}}%
\pgfpathlineto{\pgfqpoint{4.591025in}{0.558185in}}%
\pgfpathlineto{\pgfqpoint{4.595469in}{0.557747in}}%
\pgfpathlineto{\pgfqpoint{4.599913in}{0.557309in}}%
\pgfpathlineto{\pgfqpoint{4.604358in}{0.556873in}}%
\pgfpathlineto{\pgfqpoint{4.608802in}{0.556437in}}%
\pgfpathlineto{\pgfqpoint{4.613246in}{0.556002in}}%
\pgfpathlineto{\pgfqpoint{4.617691in}{0.555568in}}%
\pgfpathlineto{\pgfqpoint{4.622135in}{0.555136in}}%
\pgfpathlineto{\pgfqpoint{4.626579in}{0.554703in}}%
\pgfpathlineto{\pgfqpoint{4.631024in}{0.554272in}}%
\pgfpathlineto{\pgfqpoint{4.635468in}{0.553842in}}%
\pgfpathlineto{\pgfqpoint{4.639912in}{0.553413in}}%
\pgfpathlineto{\pgfqpoint{4.644357in}{0.552984in}}%
\pgfpathlineto{\pgfqpoint{4.648801in}{0.552556in}}%
\pgfpathlineto{\pgfqpoint{4.653245in}{0.552130in}}%
\pgfpathlineto{\pgfqpoint{4.657690in}{0.551704in}}%
\pgfpathlineto{\pgfqpoint{4.662134in}{0.551279in}}%
\pgfpathlineto{\pgfqpoint{4.666579in}{0.550855in}}%
\pgfpathlineto{\pgfqpoint{4.671023in}{0.550431in}}%
\pgfpathlineto{\pgfqpoint{4.675467in}{0.550009in}}%
\pgfpathlineto{\pgfqpoint{4.679912in}{0.549587in}}%
\pgfpathlineto{\pgfqpoint{4.684356in}{0.549167in}}%
\pgfpathlineto{\pgfqpoint{4.688800in}{0.548747in}}%
\pgfpathlineto{\pgfqpoint{4.693245in}{0.548328in}}%
\pgfpathlineto{\pgfqpoint{4.697689in}{0.547910in}}%
\pgfpathlineto{\pgfqpoint{4.702133in}{0.547492in}}%
\pgfpathlineto{\pgfqpoint{4.706578in}{0.547076in}}%
\pgfpathlineto{\pgfqpoint{4.711022in}{0.546660in}}%
\pgfpathlineto{\pgfqpoint{4.715466in}{0.546246in}}%
\pgfpathlineto{\pgfqpoint{4.719911in}{0.545832in}}%
\pgfpathlineto{\pgfqpoint{4.724355in}{0.545419in}}%
\pgfpathlineto{\pgfqpoint{4.728799in}{0.545006in}}%
\pgfpathlineto{\pgfqpoint{4.733244in}{0.544595in}}%
\pgfpathlineto{\pgfqpoint{4.737688in}{0.544184in}}%
\pgfpathlineto{\pgfqpoint{4.742132in}{0.543775in}}%
\pgfpathlineto{\pgfqpoint{4.746577in}{0.543366in}}%
\pgfpathlineto{\pgfqpoint{4.751021in}{0.542957in}}%
\pgfpathlineto{\pgfqpoint{4.755465in}{0.542550in}}%
\pgfpathlineto{\pgfqpoint{4.759910in}{0.542144in}}%
\pgfpathlineto{\pgfqpoint{4.764354in}{0.541738in}}%
\pgfpathlineto{\pgfqpoint{4.768798in}{0.541333in}}%
\pgfpathlineto{\pgfqpoint{4.773243in}{0.540929in}}%
\pgfpathlineto{\pgfqpoint{4.777687in}{0.540526in}}%
\pgfpathlineto{\pgfqpoint{4.782132in}{0.540123in}}%
\pgfpathlineto{\pgfqpoint{4.786576in}{0.539722in}}%
\pgfpathlineto{\pgfqpoint{4.791020in}{0.539321in}}%
\pgfpathlineto{\pgfqpoint{4.795465in}{0.538921in}}%
\pgfpathlineto{\pgfqpoint{4.799909in}{0.538522in}}%
\pgfpathlineto{\pgfqpoint{4.804353in}{0.538123in}}%
\pgfpathlineto{\pgfqpoint{4.808798in}{0.537726in}}%
\pgfpathlineto{\pgfqpoint{4.813242in}{0.537329in}}%
\pgfpathlineto{\pgfqpoint{4.817686in}{0.536933in}}%
\pgfpathlineto{\pgfqpoint{4.822131in}{0.536537in}}%
\pgfpathlineto{\pgfqpoint{4.826575in}{0.536143in}}%
\pgfpathlineto{\pgfqpoint{4.831019in}{0.535749in}}%
\pgfpathlineto{\pgfqpoint{4.835464in}{0.535356in}}%
\pgfpathlineto{\pgfqpoint{4.839908in}{0.534964in}}%
\pgfpathlineto{\pgfqpoint{4.844352in}{0.534572in}}%
\pgfpathlineto{\pgfqpoint{4.848797in}{0.534182in}}%
\pgfpathlineto{\pgfqpoint{4.853241in}{0.533792in}}%
\pgfpathlineto{\pgfqpoint{4.857685in}{0.533403in}}%
\pgfpathlineto{\pgfqpoint{4.862130in}{0.533014in}}%
\pgfpathlineto{\pgfqpoint{4.866574in}{0.532627in}}%
\pgfpathlineto{\pgfqpoint{4.871018in}{0.532240in}}%
\pgfpathlineto{\pgfqpoint{4.875463in}{0.531854in}}%
\pgfpathlineto{\pgfqpoint{4.879907in}{0.531468in}}%
\pgfpathlineto{\pgfqpoint{4.884351in}{0.531084in}}%
\pgfpathlineto{\pgfqpoint{4.888796in}{0.530700in}}%
\pgfpathlineto{\pgfqpoint{4.893240in}{0.530317in}}%
\pgfpathlineto{\pgfqpoint{4.897685in}{0.529935in}}%
\pgfpathlineto{\pgfqpoint{4.902129in}{0.529553in}}%
\pgfpathlineto{\pgfqpoint{4.906573in}{0.529172in}}%
\pgfpathlineto{\pgfqpoint{4.911018in}{0.528792in}}%
\pgfpathlineto{\pgfqpoint{4.915462in}{0.528413in}}%
\pgfpathlineto{\pgfqpoint{4.919906in}{0.528034in}}%
\pgfpathlineto{\pgfqpoint{4.924351in}{0.527656in}}%
\pgfpathlineto{\pgfqpoint{4.928795in}{0.527279in}}%
\pgfpathlineto{\pgfqpoint{4.933239in}{0.526902in}}%
\pgfpathlineto{\pgfqpoint{4.937684in}{0.526527in}}%
\pgfpathlineto{\pgfqpoint{4.942128in}{0.526152in}}%
\pgfpathlineto{\pgfqpoint{4.946572in}{0.525777in}}%
\pgfpathlineto{\pgfqpoint{4.951017in}{0.525404in}}%
\pgfpathlineto{\pgfqpoint{4.955461in}{0.525031in}}%
\pgfpathlineto{\pgfqpoint{4.959905in}{0.524659in}}%
\pgfpathlineto{\pgfqpoint{4.964350in}{0.524287in}}%
\pgfpathlineto{\pgfqpoint{4.968794in}{0.523917in}}%
\pgfpathlineto{\pgfqpoint{4.973238in}{0.523546in}}%
\pgfpathlineto{\pgfqpoint{4.977683in}{0.523177in}}%
\pgfpathlineto{\pgfqpoint{4.982127in}{0.522809in}}%
\pgfpathlineto{\pgfqpoint{4.986571in}{0.522441in}}%
\pgfpathlineto{\pgfqpoint{4.991016in}{0.522073in}}%
\pgfpathlineto{\pgfqpoint{4.995460in}{0.521707in}}%
\pgfpathlineto{\pgfqpoint{4.999904in}{0.521341in}}%
\pgfpathlineto{\pgfqpoint{5.004349in}{0.520976in}}%
\pgfpathlineto{\pgfqpoint{5.008793in}{0.520612in}}%
\pgfpathlineto{\pgfqpoint{5.013238in}{0.520248in}}%
\pgfpathlineto{\pgfqpoint{5.017682in}{0.519885in}}%
\pgfpathlineto{\pgfqpoint{5.022126in}{0.519522in}}%
\pgfpathlineto{\pgfqpoint{5.026571in}{0.519161in}}%
\pgfpathlineto{\pgfqpoint{5.031015in}{0.518800in}}%
\pgfpathlineto{\pgfqpoint{5.035459in}{0.518440in}}%
\pgfpathlineto{\pgfqpoint{5.039904in}{0.518080in}}%
\pgfpathlineto{\pgfqpoint{5.044348in}{0.517721in}}%
\pgfpathlineto{\pgfqpoint{5.048792in}{0.517363in}}%
\pgfpathlineto{\pgfqpoint{5.053237in}{0.517005in}}%
\pgfpathlineto{\pgfqpoint{5.057681in}{0.516648in}}%
\pgfpathlineto{\pgfqpoint{5.062125in}{0.516292in}}%
\pgfpathlineto{\pgfqpoint{5.066570in}{0.515936in}}%
\pgfpathlineto{\pgfqpoint{5.071014in}{0.515581in}}%
\pgfpathlineto{\pgfqpoint{5.075458in}{0.515227in}}%
\pgfpathlineto{\pgfqpoint{5.079903in}{0.514874in}}%
\pgfpathlineto{\pgfqpoint{5.084347in}{0.514521in}}%
\pgfpathlineto{\pgfqpoint{5.088791in}{0.514168in}}%
\pgfpathlineto{\pgfqpoint{5.093236in}{0.513817in}}%
\pgfpathlineto{\pgfqpoint{5.097680in}{0.513466in}}%
\pgfpathlineto{\pgfqpoint{5.102124in}{0.513115in}}%
\pgfpathlineto{\pgfqpoint{5.106569in}{0.512766in}}%
\pgfpathlineto{\pgfqpoint{5.111013in}{0.512417in}}%
\pgfpathlineto{\pgfqpoint{5.115457in}{0.512068in}}%
\pgfpathlineto{\pgfqpoint{5.119902in}{0.511721in}}%
\pgfpathlineto{\pgfqpoint{5.124346in}{0.511374in}}%
\pgfpathlineto{\pgfqpoint{5.128791in}{0.511027in}}%
\pgfpathlineto{\pgfqpoint{5.133235in}{0.510681in}}%
\pgfpathlineto{\pgfqpoint{5.137679in}{0.510336in}}%
\pgfpathlineto{\pgfqpoint{5.142124in}{0.509992in}}%
\pgfpathlineto{\pgfqpoint{5.146568in}{0.509648in}}%
\pgfpathlineto{\pgfqpoint{5.151012in}{0.509304in}}%
\pgfpathlineto{\pgfqpoint{5.155457in}{0.508962in}}%
\pgfpathlineto{\pgfqpoint{5.159901in}{0.508620in}}%
\pgfpathlineto{\pgfqpoint{5.164345in}{0.508278in}}%
\pgfpathlineto{\pgfqpoint{5.168790in}{0.507938in}}%
\pgfpathlineto{\pgfqpoint{5.173234in}{0.507598in}}%
\pgfpathlineto{\pgfqpoint{5.177678in}{0.507258in}}%
\pgfpathlineto{\pgfqpoint{5.182123in}{0.506919in}}%
\pgfpathlineto{\pgfqpoint{5.186567in}{0.506581in}}%
\pgfpathlineto{\pgfqpoint{5.191011in}{0.506243in}}%
\pgfpathlineto{\pgfqpoint{5.195456in}{0.505906in}}%
\pgfpathlineto{\pgfqpoint{5.199900in}{0.505570in}}%
\pgfpathlineto{\pgfqpoint{5.199900in}{0.505570in}}%
\pgfpathlineto{\pgfqpoint{5.207241in}{0.505026in}}%
\pgfpathlineto{\pgfqpoint{5.214582in}{0.504503in}}%
\pgfpathlineto{\pgfqpoint{5.221923in}{0.503999in}}%
\pgfpathlineto{\pgfqpoint{5.229264in}{0.503513in}}%
\pgfpathlineto{\pgfqpoint{5.236605in}{0.503044in}}%
\pgfpathlineto{\pgfqpoint{5.243946in}{0.502590in}}%
\pgfpathlineto{\pgfqpoint{5.251287in}{0.502152in}}%
\pgfpathlineto{\pgfqpoint{5.258629in}{0.501727in}}%
\pgfpathlineto{\pgfqpoint{5.265970in}{0.501316in}}%
\pgfpathlineto{\pgfqpoint{5.273311in}{0.500916in}}%
\pgfpathlineto{\pgfqpoint{5.280652in}{0.500529in}}%
\pgfpathlineto{\pgfqpoint{5.287993in}{0.500152in}}%
\pgfpathlineto{\pgfqpoint{5.295334in}{0.499786in}}%
\pgfpathlineto{\pgfqpoint{5.302675in}{0.499430in}}%
\pgfpathlineto{\pgfqpoint{5.310016in}{0.499083in}}%
\pgfpathlineto{\pgfqpoint{5.317357in}{0.498745in}}%
\pgfpathlineto{\pgfqpoint{5.324698in}{0.498416in}}%
\pgfpathlineto{\pgfqpoint{5.332039in}{0.498095in}}%
\pgfpathlineto{\pgfqpoint{5.339380in}{0.497781in}}%
\pgfpathlineto{\pgfqpoint{5.346721in}{0.497475in}}%
\pgfpathlineto{\pgfqpoint{5.354062in}{0.497177in}}%
\pgfpathlineto{\pgfqpoint{5.361403in}{0.496885in}}%
\pgfpathlineto{\pgfqpoint{5.368744in}{0.496599in}}%
\pgfpathlineto{\pgfqpoint{5.376085in}{0.496320in}}%
\pgfpathlineto{\pgfqpoint{5.383426in}{0.496047in}}%
\pgfpathlineto{\pgfqpoint{5.390768in}{0.495780in}}%
\pgfpathlineto{\pgfqpoint{5.398109in}{0.495519in}}%
\pgfpathlineto{\pgfqpoint{5.405450in}{0.495263in}}%
\pgfpathlineto{\pgfqpoint{5.412791in}{0.495012in}}%
\pgfpathlineto{\pgfqpoint{5.420132in}{0.494766in}}%
\pgfpathlineto{\pgfqpoint{5.427473in}{0.494525in}}%
\pgfpathlineto{\pgfqpoint{5.434814in}{0.494288in}}%
\pgfpathlineto{\pgfqpoint{5.442155in}{0.494056in}}%
\pgfpathlineto{\pgfqpoint{5.449496in}{0.493829in}}%
\pgfpathlineto{\pgfqpoint{5.456837in}{0.493605in}}%
\pgfpathlineto{\pgfqpoint{5.464178in}{0.493386in}}%
\pgfpathlineto{\pgfqpoint{5.471519in}{0.493171in}}%
\pgfpathlineto{\pgfqpoint{5.478860in}{0.492959in}}%
\pgfpathlineto{\pgfqpoint{5.486201in}{0.492751in}}%
\pgfpathlineto{\pgfqpoint{5.493542in}{0.492547in}}%
\pgfpathlineto{\pgfqpoint{5.500883in}{0.492347in}}%
\pgfpathlineto{\pgfqpoint{5.508224in}{0.492149in}}%
\pgfpathlineto{\pgfqpoint{5.515566in}{0.491955in}}%
\pgfpathlineto{\pgfqpoint{5.522907in}{0.491764in}}%
\pgfpathlineto{\pgfqpoint{5.530248in}{0.491577in}}%
\pgfpathlineto{\pgfqpoint{5.537589in}{0.491392in}}%
\pgfpathlineto{\pgfqpoint{5.544930in}{0.491210in}}%
\pgfpathlineto{\pgfqpoint{5.552271in}{0.491031in}}%
\pgfpathlineto{\pgfqpoint{5.559612in}{0.490855in}}%
\pgfpathlineto{\pgfqpoint{5.566953in}{0.490682in}}%
\pgfpathlineto{\pgfqpoint{5.574294in}{0.490511in}}%
\pgfpathlineto{\pgfqpoint{5.581635in}{0.490343in}}%
\pgfpathlineto{\pgfqpoint{5.588976in}{0.490177in}}%
\pgfpathlineto{\pgfqpoint{5.596317in}{0.490013in}}%
\pgfpathlineto{\pgfqpoint{5.603658in}{0.489853in}}%
\pgfpathlineto{\pgfqpoint{5.610999in}{0.489694in}}%
\pgfpathlineto{\pgfqpoint{5.618340in}{0.489538in}}%
\pgfpathlineto{\pgfqpoint{5.625681in}{0.489383in}}%
\pgfpathlineto{\pgfqpoint{5.633022in}{0.489231in}}%
\pgfpathlineto{\pgfqpoint{5.640363in}{0.489082in}}%
\pgfpathlineto{\pgfqpoint{5.647705in}{0.488934in}}%
\pgfpathlineto{\pgfqpoint{5.655046in}{0.488788in}}%
\pgfpathlineto{\pgfqpoint{5.662387in}{0.488644in}}%
\pgfpathlineto{\pgfqpoint{5.669728in}{0.488502in}}%
\pgfpathlineto{\pgfqpoint{5.677069in}{0.488362in}}%
\pgfpathlineto{\pgfqpoint{5.684410in}{0.488224in}}%
\pgfpathlineto{\pgfqpoint{5.691751in}{0.488087in}}%
\pgfpathlineto{\pgfqpoint{5.699092in}{0.487953in}}%
\pgfpathlineto{\pgfqpoint{5.706433in}{0.487820in}}%
\pgfpathlineto{\pgfqpoint{5.713774in}{0.487688in}}%
\pgfpathlineto{\pgfqpoint{5.721115in}{0.487559in}}%
\pgfpathlineto{\pgfqpoint{5.728456in}{0.487431in}}%
\pgfpathlineto{\pgfqpoint{5.735797in}{0.487304in}}%
\pgfpathlineto{\pgfqpoint{5.743138in}{0.487179in}}%
\pgfpathlineto{\pgfqpoint{5.750479in}{0.487056in}}%
\pgfpathlineto{\pgfqpoint{5.757820in}{0.486934in}}%
\pgfpathlineto{\pgfqpoint{5.765161in}{0.486814in}}%
\pgfpathlineto{\pgfqpoint{5.772503in}{0.486695in}}%
\pgfpathlineto{\pgfqpoint{5.779844in}{0.486577in}}%
\pgfpathlineto{\pgfqpoint{5.787185in}{0.486461in}}%
\pgfpathlineto{\pgfqpoint{5.794526in}{0.486346in}}%
\pgfpathlineto{\pgfqpoint{5.801867in}{0.486232in}}%
\pgfpathlineto{\pgfqpoint{5.809208in}{0.486120in}}%
\pgfpathlineto{\pgfqpoint{5.816549in}{0.486009in}}%
\pgfpathlineto{\pgfqpoint{5.823890in}{0.485899in}}%
\pgfpathlineto{\pgfqpoint{5.831231in}{0.485791in}}%
\pgfpathlineto{\pgfqpoint{5.838572in}{0.485683in}}%
\pgfpathlineto{\pgfqpoint{5.845913in}{0.485577in}}%
\pgfpathlineto{\pgfqpoint{5.853254in}{0.485472in}}%
\pgfpathlineto{\pgfqpoint{5.860595in}{0.485368in}}%
\pgfpathlineto{\pgfqpoint{5.867936in}{0.485265in}}%
\pgfpathlineto{\pgfqpoint{5.875277in}{0.485164in}}%
\pgfpathlineto{\pgfqpoint{5.882618in}{0.485063in}}%
\pgfpathlineto{\pgfqpoint{5.889959in}{0.484964in}}%
\pgfpathlineto{\pgfqpoint{5.897300in}{0.484865in}}%
\pgfpathlineto{\pgfqpoint{5.904642in}{0.484768in}}%
\pgfpathlineto{\pgfqpoint{5.911983in}{0.484671in}}%
\pgfpathlineto{\pgfqpoint{5.919324in}{0.484576in}}%
\pgfpathlineto{\pgfqpoint{5.926665in}{0.484481in}}%
\pgfpathlineto{\pgfqpoint{7.047223in}{0.554012in}}%
\pgfpathlineto{\pgfqpoint{7.038414in}{0.554125in}}%
\pgfpathlineto{\pgfqpoint{7.029605in}{0.554240in}}%
\pgfpathlineto{\pgfqpoint{7.020795in}{0.554355in}}%
\pgfpathlineto{\pgfqpoint{7.011986in}{0.554472in}}%
\pgfpathlineto{\pgfqpoint{7.003177in}{0.554591in}}%
\pgfpathlineto{\pgfqpoint{6.994367in}{0.554710in}}%
\pgfpathlineto{\pgfqpoint{6.985558in}{0.554831in}}%
\pgfpathlineto{\pgfqpoint{6.976749in}{0.554953in}}%
\pgfpathlineto{\pgfqpoint{6.967940in}{0.555076in}}%
\pgfpathlineto{\pgfqpoint{6.959130in}{0.555201in}}%
\pgfpathlineto{\pgfqpoint{6.950321in}{0.555327in}}%
\pgfpathlineto{\pgfqpoint{6.941512in}{0.555454in}}%
\pgfpathlineto{\pgfqpoint{6.932703in}{0.555583in}}%
\pgfpathlineto{\pgfqpoint{6.923893in}{0.555713in}}%
\pgfpathlineto{\pgfqpoint{6.915084in}{0.555845in}}%
\pgfpathlineto{\pgfqpoint{6.906275in}{0.555978in}}%
\pgfpathlineto{\pgfqpoint{6.897465in}{0.556113in}}%
\pgfpathlineto{\pgfqpoint{6.888656in}{0.556249in}}%
\pgfpathlineto{\pgfqpoint{6.879847in}{0.556387in}}%
\pgfpathlineto{\pgfqpoint{6.871038in}{0.556527in}}%
\pgfpathlineto{\pgfqpoint{6.862228in}{0.556668in}}%
\pgfpathlineto{\pgfqpoint{6.853419in}{0.556811in}}%
\pgfpathlineto{\pgfqpoint{6.844610in}{0.556955in}}%
\pgfpathlineto{\pgfqpoint{6.835801in}{0.557101in}}%
\pgfpathlineto{\pgfqpoint{6.826991in}{0.557249in}}%
\pgfpathlineto{\pgfqpoint{6.818182in}{0.557399in}}%
\pgfpathlineto{\pgfqpoint{6.809373in}{0.557551in}}%
\pgfpathlineto{\pgfqpoint{6.800564in}{0.557705in}}%
\pgfpathlineto{\pgfqpoint{6.791754in}{0.557860in}}%
\pgfpathlineto{\pgfqpoint{6.782945in}{0.558018in}}%
\pgfpathlineto{\pgfqpoint{6.774136in}{0.558177in}}%
\pgfpathlineto{\pgfqpoint{6.765326in}{0.558339in}}%
\pgfpathlineto{\pgfqpoint{6.756517in}{0.558503in}}%
\pgfpathlineto{\pgfqpoint{6.747708in}{0.558669in}}%
\pgfpathlineto{\pgfqpoint{6.738899in}{0.558837in}}%
\pgfpathlineto{\pgfqpoint{6.730089in}{0.559007in}}%
\pgfpathlineto{\pgfqpoint{6.721280in}{0.559180in}}%
\pgfpathlineto{\pgfqpoint{6.712471in}{0.559355in}}%
\pgfpathlineto{\pgfqpoint{6.703662in}{0.559532in}}%
\pgfpathlineto{\pgfqpoint{6.694852in}{0.559712in}}%
\pgfpathlineto{\pgfqpoint{6.686043in}{0.559894in}}%
\pgfpathlineto{\pgfqpoint{6.677234in}{0.560079in}}%
\pgfpathlineto{\pgfqpoint{6.668425in}{0.560267in}}%
\pgfpathlineto{\pgfqpoint{6.659615in}{0.560457in}}%
\pgfpathlineto{\pgfqpoint{6.650806in}{0.560650in}}%
\pgfpathlineto{\pgfqpoint{6.641997in}{0.560846in}}%
\pgfpathlineto{\pgfqpoint{6.633187in}{0.561045in}}%
\pgfpathlineto{\pgfqpoint{6.624378in}{0.561247in}}%
\pgfpathlineto{\pgfqpoint{6.615569in}{0.561452in}}%
\pgfpathlineto{\pgfqpoint{6.606760in}{0.561660in}}%
\pgfpathlineto{\pgfqpoint{6.597950in}{0.561872in}}%
\pgfpathlineto{\pgfqpoint{6.589141in}{0.562086in}}%
\pgfpathlineto{\pgfqpoint{6.580332in}{0.562305in}}%
\pgfpathlineto{\pgfqpoint{6.571523in}{0.562526in}}%
\pgfpathlineto{\pgfqpoint{6.562713in}{0.562751in}}%
\pgfpathlineto{\pgfqpoint{6.553904in}{0.562980in}}%
\pgfpathlineto{\pgfqpoint{6.545095in}{0.563213in}}%
\pgfpathlineto{\pgfqpoint{6.536285in}{0.563450in}}%
\pgfpathlineto{\pgfqpoint{6.527476in}{0.563691in}}%
\pgfpathlineto{\pgfqpoint{6.518667in}{0.563936in}}%
\pgfpathlineto{\pgfqpoint{6.509858in}{0.564185in}}%
\pgfpathlineto{\pgfqpoint{6.501048in}{0.564439in}}%
\pgfpathlineto{\pgfqpoint{6.492239in}{0.564698in}}%
\pgfpathlineto{\pgfqpoint{6.483430in}{0.564961in}}%
\pgfpathlineto{\pgfqpoint{6.474621in}{0.565229in}}%
\pgfpathlineto{\pgfqpoint{6.465811in}{0.565502in}}%
\pgfpathlineto{\pgfqpoint{6.457002in}{0.565780in}}%
\pgfpathlineto{\pgfqpoint{6.448193in}{0.566064in}}%
\pgfpathlineto{\pgfqpoint{6.439384in}{0.566353in}}%
\pgfpathlineto{\pgfqpoint{6.430574in}{0.566648in}}%
\pgfpathlineto{\pgfqpoint{6.421765in}{0.566949in}}%
\pgfpathlineto{\pgfqpoint{6.412956in}{0.567257in}}%
\pgfpathlineto{\pgfqpoint{6.404146in}{0.567571in}}%
\pgfpathlineto{\pgfqpoint{6.395337in}{0.567891in}}%
\pgfpathlineto{\pgfqpoint{6.386528in}{0.568219in}}%
\pgfpathlineto{\pgfqpoint{6.377719in}{0.568554in}}%
\pgfpathlineto{\pgfqpoint{6.368909in}{0.568896in}}%
\pgfpathlineto{\pgfqpoint{6.360100in}{0.569246in}}%
\pgfpathlineto{\pgfqpoint{6.351291in}{0.569605in}}%
\pgfpathlineto{\pgfqpoint{6.342482in}{0.569972in}}%
\pgfpathlineto{\pgfqpoint{6.333672in}{0.570348in}}%
\pgfpathlineto{\pgfqpoint{6.324863in}{0.570733in}}%
\pgfpathlineto{\pgfqpoint{6.316054in}{0.571128in}}%
\pgfpathlineto{\pgfqpoint{6.307245in}{0.571534in}}%
\pgfpathlineto{\pgfqpoint{6.298435in}{0.571950in}}%
\pgfpathlineto{\pgfqpoint{6.289626in}{0.572377in}}%
\pgfpathlineto{\pgfqpoint{6.280817in}{0.572817in}}%
\pgfpathlineto{\pgfqpoint{6.272007in}{0.573269in}}%
\pgfpathlineto{\pgfqpoint{6.263198in}{0.573734in}}%
\pgfpathlineto{\pgfqpoint{6.254389in}{0.574213in}}%
\pgfpathlineto{\pgfqpoint{6.245580in}{0.574707in}}%
\pgfpathlineto{\pgfqpoint{6.236770in}{0.575217in}}%
\pgfpathlineto{\pgfqpoint{6.227961in}{0.575743in}}%
\pgfpathlineto{\pgfqpoint{6.219152in}{0.576287in}}%
\pgfpathlineto{\pgfqpoint{6.210343in}{0.576850in}}%
\pgfpathlineto{\pgfqpoint{6.201533in}{0.577433in}}%
\pgfpathlineto{\pgfqpoint{6.192724in}{0.578037in}}%
\pgfpathlineto{\pgfqpoint{6.183915in}{0.578665in}}%
\pgfpathlineto{\pgfqpoint{6.175105in}{0.579318in}}%
\pgfpathlineto{\pgfqpoint{6.175105in}{0.579318in}}%
\pgfpathlineto{\pgfqpoint{6.169772in}{0.579722in}}%
\pgfpathlineto{\pgfqpoint{6.164439in}{0.580126in}}%
\pgfpathlineto{\pgfqpoint{6.159106in}{0.580531in}}%
\pgfpathlineto{\pgfqpoint{6.153773in}{0.580937in}}%
\pgfpathlineto{\pgfqpoint{6.148439in}{0.581344in}}%
\pgfpathlineto{\pgfqpoint{6.143106in}{0.581751in}}%
\pgfpathlineto{\pgfqpoint{6.137773in}{0.582160in}}%
\pgfpathlineto{\pgfqpoint{6.132440in}{0.582568in}}%
\pgfpathlineto{\pgfqpoint{6.127107in}{0.582978in}}%
\pgfpathlineto{\pgfqpoint{6.121773in}{0.583388in}}%
\pgfpathlineto{\pgfqpoint{6.116440in}{0.583800in}}%
\pgfpathlineto{\pgfqpoint{6.111107in}{0.584212in}}%
\pgfpathlineto{\pgfqpoint{6.105774in}{0.584624in}}%
\pgfpathlineto{\pgfqpoint{6.100440in}{0.585038in}}%
\pgfpathlineto{\pgfqpoint{6.095107in}{0.585452in}}%
\pgfpathlineto{\pgfqpoint{6.089774in}{0.585867in}}%
\pgfpathlineto{\pgfqpoint{6.084441in}{0.586282in}}%
\pgfpathlineto{\pgfqpoint{6.079108in}{0.586699in}}%
\pgfpathlineto{\pgfqpoint{6.073774in}{0.587116in}}%
\pgfpathlineto{\pgfqpoint{6.068441in}{0.587534in}}%
\pgfpathlineto{\pgfqpoint{6.063108in}{0.587953in}}%
\pgfpathlineto{\pgfqpoint{6.057775in}{0.588373in}}%
\pgfpathlineto{\pgfqpoint{6.052442in}{0.588793in}}%
\pgfpathlineto{\pgfqpoint{6.047108in}{0.589214in}}%
\pgfpathlineto{\pgfqpoint{6.041775in}{0.589636in}}%
\pgfpathlineto{\pgfqpoint{6.036442in}{0.590059in}}%
\pgfpathlineto{\pgfqpoint{6.031109in}{0.590482in}}%
\pgfpathlineto{\pgfqpoint{6.025775in}{0.590907in}}%
\pgfpathlineto{\pgfqpoint{6.020442in}{0.591332in}}%
\pgfpathlineto{\pgfqpoint{6.015109in}{0.591758in}}%
\pgfpathlineto{\pgfqpoint{6.009776in}{0.592185in}}%
\pgfpathlineto{\pgfqpoint{6.004443in}{0.592612in}}%
\pgfpathlineto{\pgfqpoint{5.999109in}{0.593040in}}%
\pgfpathlineto{\pgfqpoint{5.993776in}{0.593469in}}%
\pgfpathlineto{\pgfqpoint{5.988443in}{0.593899in}}%
\pgfpathlineto{\pgfqpoint{5.983110in}{0.594330in}}%
\pgfpathlineto{\pgfqpoint{5.977777in}{0.594762in}}%
\pgfpathlineto{\pgfqpoint{5.972443in}{0.595194in}}%
\pgfpathlineto{\pgfqpoint{5.967110in}{0.595627in}}%
\pgfpathlineto{\pgfqpoint{5.961777in}{0.596061in}}%
\pgfpathlineto{\pgfqpoint{5.956444in}{0.596496in}}%
\pgfpathlineto{\pgfqpoint{5.951110in}{0.596932in}}%
\pgfpathlineto{\pgfqpoint{5.945777in}{0.597368in}}%
\pgfpathlineto{\pgfqpoint{5.940444in}{0.597806in}}%
\pgfpathlineto{\pgfqpoint{5.935111in}{0.598244in}}%
\pgfpathlineto{\pgfqpoint{5.929778in}{0.598683in}}%
\pgfpathlineto{\pgfqpoint{5.924444in}{0.599122in}}%
\pgfpathlineto{\pgfqpoint{5.919111in}{0.599563in}}%
\pgfpathlineto{\pgfqpoint{5.913778in}{0.600005in}}%
\pgfpathlineto{\pgfqpoint{5.908445in}{0.600447in}}%
\pgfpathlineto{\pgfqpoint{5.903112in}{0.600890in}}%
\pgfpathlineto{\pgfqpoint{5.897778in}{0.601334in}}%
\pgfpathlineto{\pgfqpoint{5.892445in}{0.601779in}}%
\pgfpathlineto{\pgfqpoint{5.887112in}{0.602225in}}%
\pgfpathlineto{\pgfqpoint{5.881779in}{0.602671in}}%
\pgfpathlineto{\pgfqpoint{5.876445in}{0.603119in}}%
\pgfpathlineto{\pgfqpoint{5.871112in}{0.603567in}}%
\pgfpathlineto{\pgfqpoint{5.865779in}{0.604016in}}%
\pgfpathlineto{\pgfqpoint{5.860446in}{0.604466in}}%
\pgfpathlineto{\pgfqpoint{5.855113in}{0.604917in}}%
\pgfpathlineto{\pgfqpoint{5.849779in}{0.605369in}}%
\pgfpathlineto{\pgfqpoint{5.844446in}{0.605821in}}%
\pgfpathlineto{\pgfqpoint{5.839113in}{0.606275in}}%
\pgfpathlineto{\pgfqpoint{5.833780in}{0.606729in}}%
\pgfpathlineto{\pgfqpoint{5.828446in}{0.607185in}}%
\pgfpathlineto{\pgfqpoint{5.823113in}{0.607641in}}%
\pgfpathlineto{\pgfqpoint{5.817780in}{0.608098in}}%
\pgfpathlineto{\pgfqpoint{5.812447in}{0.608556in}}%
\pgfpathlineto{\pgfqpoint{5.807114in}{0.609015in}}%
\pgfpathlineto{\pgfqpoint{5.801780in}{0.609474in}}%
\pgfpathlineto{\pgfqpoint{5.796447in}{0.609935in}}%
\pgfpathlineto{\pgfqpoint{5.791114in}{0.610396in}}%
\pgfpathlineto{\pgfqpoint{5.785781in}{0.610859in}}%
\pgfpathlineto{\pgfqpoint{5.780448in}{0.611322in}}%
\pgfpathlineto{\pgfqpoint{5.775114in}{0.611786in}}%
\pgfpathlineto{\pgfqpoint{5.769781in}{0.612252in}}%
\pgfpathlineto{\pgfqpoint{5.764448in}{0.612718in}}%
\pgfpathlineto{\pgfqpoint{5.759115in}{0.613185in}}%
\pgfpathlineto{\pgfqpoint{5.753781in}{0.613652in}}%
\pgfpathlineto{\pgfqpoint{5.748448in}{0.614121in}}%
\pgfpathlineto{\pgfqpoint{5.743115in}{0.614591in}}%
\pgfpathlineto{\pgfqpoint{5.737782in}{0.615062in}}%
\pgfpathlineto{\pgfqpoint{5.732449in}{0.615533in}}%
\pgfpathlineto{\pgfqpoint{5.727115in}{0.616006in}}%
\pgfpathlineto{\pgfqpoint{5.721782in}{0.616479in}}%
\pgfpathlineto{\pgfqpoint{5.716449in}{0.616953in}}%
\pgfpathlineto{\pgfqpoint{5.711116in}{0.617429in}}%
\pgfpathlineto{\pgfqpoint{5.705783in}{0.617905in}}%
\pgfpathlineto{\pgfqpoint{5.700449in}{0.618382in}}%
\pgfpathlineto{\pgfqpoint{5.695116in}{0.618860in}}%
\pgfpathlineto{\pgfqpoint{5.689783in}{0.619339in}}%
\pgfpathlineto{\pgfqpoint{5.684450in}{0.619819in}}%
\pgfpathlineto{\pgfqpoint{5.679116in}{0.620300in}}%
\pgfpathlineto{\pgfqpoint{5.673783in}{0.620782in}}%
\pgfpathlineto{\pgfqpoint{5.668450in}{0.621265in}}%
\pgfpathlineto{\pgfqpoint{5.663117in}{0.621749in}}%
\pgfpathlineto{\pgfqpoint{5.657784in}{0.622234in}}%
\pgfpathlineto{\pgfqpoint{5.652450in}{0.622720in}}%
\pgfpathlineto{\pgfqpoint{5.647117in}{0.623207in}}%
\pgfpathlineto{\pgfqpoint{5.641784in}{0.623695in}}%
\pgfpathlineto{\pgfqpoint{5.636451in}{0.624183in}}%
\pgfpathlineto{\pgfqpoint{5.631118in}{0.624673in}}%
\pgfpathlineto{\pgfqpoint{5.625784in}{0.625164in}}%
\pgfpathlineto{\pgfqpoint{5.620451in}{0.625655in}}%
\pgfpathlineto{\pgfqpoint{5.615118in}{0.626148in}}%
\pgfpathlineto{\pgfqpoint{5.609785in}{0.626642in}}%
\pgfpathlineto{\pgfqpoint{5.604451in}{0.627137in}}%
\pgfpathlineto{\pgfqpoint{5.599118in}{0.627632in}}%
\pgfpathlineto{\pgfqpoint{5.593785in}{0.628129in}}%
\pgfpathlineto{\pgfqpoint{5.588452in}{0.628627in}}%
\pgfpathlineto{\pgfqpoint{5.583119in}{0.629125in}}%
\pgfpathlineto{\pgfqpoint{5.577785in}{0.629625in}}%
\pgfpathlineto{\pgfqpoint{5.572452in}{0.630126in}}%
\pgfpathlineto{\pgfqpoint{5.567119in}{0.630628in}}%
\pgfpathlineto{\pgfqpoint{5.561786in}{0.631130in}}%
\pgfpathlineto{\pgfqpoint{5.556452in}{0.631634in}}%
\pgfpathlineto{\pgfqpoint{5.551119in}{0.632139in}}%
\pgfpathlineto{\pgfqpoint{5.545786in}{0.632645in}}%
\pgfpathlineto{\pgfqpoint{5.540453in}{0.633152in}}%
\pgfpathlineto{\pgfqpoint{5.535120in}{0.633660in}}%
\pgfpathlineto{\pgfqpoint{5.529786in}{0.634169in}}%
\pgfpathlineto{\pgfqpoint{5.524453in}{0.634679in}}%
\pgfpathlineto{\pgfqpoint{5.519120in}{0.635190in}}%
\pgfpathlineto{\pgfqpoint{5.513787in}{0.635702in}}%
\pgfpathlineto{\pgfqpoint{5.508454in}{0.636215in}}%
\pgfpathlineto{\pgfqpoint{5.503120in}{0.636729in}}%
\pgfpathlineto{\pgfqpoint{5.497787in}{0.637245in}}%
\pgfpathlineto{\pgfqpoint{5.492454in}{0.637761in}}%
\pgfpathlineto{\pgfqpoint{5.487121in}{0.638278in}}%
\pgfpathlineto{\pgfqpoint{5.481787in}{0.638797in}}%
\pgfpathlineto{\pgfqpoint{5.476454in}{0.639316in}}%
\pgfpathlineto{\pgfqpoint{5.471121in}{0.639837in}}%
\pgfpathlineto{\pgfqpoint{5.465788in}{0.640359in}}%
\pgfpathlineto{\pgfqpoint{5.460455in}{0.640882in}}%
\pgfpathlineto{\pgfqpoint{5.455121in}{0.641406in}}%
\pgfpathlineto{\pgfqpoint{5.449788in}{0.641930in}}%
\pgfpathlineto{\pgfqpoint{5.444455in}{0.642457in}}%
\pgfpathlineto{\pgfqpoint{5.439122in}{0.642984in}}%
\pgfpathlineto{\pgfqpoint{5.433789in}{0.643512in}}%
\pgfpathlineto{\pgfqpoint{5.428455in}{0.644041in}}%
\pgfpathlineto{\pgfqpoint{5.423122in}{0.644572in}}%
\pgfpathlineto{\pgfqpoint{5.417789in}{0.645103in}}%
\pgfpathlineto{\pgfqpoint{5.412456in}{0.645636in}}%
\pgfpathlineto{\pgfqpoint{5.407122in}{0.646170in}}%
\pgfpathlineto{\pgfqpoint{5.401789in}{0.646705in}}%
\pgfpathlineto{\pgfqpoint{5.396456in}{0.647241in}}%
\pgfpathlineto{\pgfqpoint{5.391123in}{0.647778in}}%
\pgfpathlineto{\pgfqpoint{5.385790in}{0.648317in}}%
\pgfpathlineto{\pgfqpoint{5.380456in}{0.648856in}}%
\pgfpathlineto{\pgfqpoint{5.375123in}{0.649397in}}%
\pgfpathlineto{\pgfqpoint{5.369790in}{0.649938in}}%
\pgfpathlineto{\pgfqpoint{5.364457in}{0.650481in}}%
\pgfpathlineto{\pgfqpoint{5.359124in}{0.651025in}}%
\pgfpathlineto{\pgfqpoint{5.353790in}{0.651571in}}%
\pgfpathlineto{\pgfqpoint{5.348457in}{0.652117in}}%
\pgfpathlineto{\pgfqpoint{5.343124in}{0.652664in}}%
\pgfpathlineto{\pgfqpoint{5.337791in}{0.653213in}}%
\pgfpathlineto{\pgfqpoint{5.332457in}{0.653763in}}%
\pgfpathlineto{\pgfqpoint{5.327124in}{0.654314in}}%
\pgfpathlineto{\pgfqpoint{5.321791in}{0.654866in}}%
\pgfpathlineto{\pgfqpoint{5.316458in}{0.655420in}}%
\pgfpathlineto{\pgfqpoint{5.311125in}{0.655974in}}%
\pgfpathlineto{\pgfqpoint{5.305791in}{0.656530in}}%
\pgfpathlineto{\pgfqpoint{5.300458in}{0.657087in}}%
\pgfpathlineto{\pgfqpoint{5.295125in}{0.657645in}}%
\pgfpathlineto{\pgfqpoint{5.289792in}{0.658204in}}%
\pgfpathlineto{\pgfqpoint{5.284458in}{0.658765in}}%
\pgfpathlineto{\pgfqpoint{5.279125in}{0.659327in}}%
\pgfpathlineto{\pgfqpoint{5.273792in}{0.659890in}}%
\pgfpathlineto{\pgfqpoint{5.268459in}{0.660454in}}%
\pgfpathlineto{\pgfqpoint{5.263126in}{0.661019in}}%
\pgfpathlineto{\pgfqpoint{5.257792in}{0.661586in}}%
\pgfpathlineto{\pgfqpoint{5.252459in}{0.662154in}}%
\pgfpathlineto{\pgfqpoint{5.247126in}{0.662723in}}%
\pgfpathlineto{\pgfqpoint{5.241793in}{0.663293in}}%
\pgfpathlineto{\pgfqpoint{5.236460in}{0.663864in}}%
\pgfpathlineto{\pgfqpoint{5.231126in}{0.664437in}}%
\pgfpathlineto{\pgfqpoint{5.225793in}{0.665011in}}%
\pgfpathlineto{\pgfqpoint{5.220460in}{0.665587in}}%
\pgfpathlineto{\pgfqpoint{5.215127in}{0.666163in}}%
\pgfpathlineto{\pgfqpoint{5.209793in}{0.666741in}}%
\pgfpathlineto{\pgfqpoint{5.204460in}{0.667320in}}%
\pgfpathlineto{\pgfqpoint{5.199127in}{0.667900in}}%
\pgfpathlineto{\pgfqpoint{5.193794in}{0.668482in}}%
\pgfpathlineto{\pgfqpoint{5.188461in}{0.669065in}}%
\pgfpathlineto{\pgfqpoint{5.183127in}{0.669649in}}%
\pgfpathlineto{\pgfqpoint{5.177794in}{0.670234in}}%
\pgfpathlineto{\pgfqpoint{5.172461in}{0.670821in}}%
\pgfpathlineto{\pgfqpoint{5.167128in}{0.671409in}}%
\pgfpathlineto{\pgfqpoint{5.161795in}{0.671998in}}%
\pgfpathlineto{\pgfqpoint{5.156461in}{0.672589in}}%
\pgfpathlineto{\pgfqpoint{5.151128in}{0.673181in}}%
\pgfpathlineto{\pgfqpoint{5.145795in}{0.673774in}}%
\pgfpathlineto{\pgfqpoint{5.140462in}{0.674369in}}%
\pgfpathlineto{\pgfqpoint{5.135128in}{0.674965in}}%
\pgfpathlineto{\pgfqpoint{5.129795in}{0.675562in}}%
\pgfpathlineto{\pgfqpoint{5.124462in}{0.676160in}}%
\pgfpathlineto{\pgfqpoint{5.119129in}{0.676760in}}%
\pgfpathlineto{\pgfqpoint{5.113796in}{0.677361in}}%
\pgfpathlineto{\pgfqpoint{5.108462in}{0.677964in}}%
\pgfpathlineto{\pgfqpoint{5.103129in}{0.678568in}}%
\pgfpathlineto{\pgfqpoint{5.097796in}{0.679173in}}%
\pgfpathlineto{\pgfqpoint{5.092463in}{0.679779in}}%
\pgfpathlineto{\pgfqpoint{5.087130in}{0.680387in}}%
\pgfpathlineto{\pgfqpoint{5.081796in}{0.680997in}}%
\pgfpathlineto{\pgfqpoint{5.076463in}{0.681607in}}%
\pgfpathlineto{\pgfqpoint{5.071130in}{0.682219in}}%
\pgfpathlineto{\pgfqpoint{5.065797in}{0.682833in}}%
\pgfpathlineto{\pgfqpoint{5.060463in}{0.683447in}}%
\pgfpathlineto{\pgfqpoint{5.055130in}{0.684064in}}%
\pgfpathlineto{\pgfqpoint{5.049797in}{0.684681in}}%
\pgfpathlineto{\pgfqpoint{5.044464in}{0.685300in}}%
\pgfpathlineto{\pgfqpoint{5.039131in}{0.685920in}}%
\pgfpathlineto{\pgfqpoint{5.033797in}{0.686542in}}%
\pgfpathlineto{\pgfqpoint{5.028464in}{0.687165in}}%
\pgfpathlineto{\pgfqpoint{5.023131in}{0.687790in}}%
\pgfpathlineto{\pgfqpoint{5.017798in}{0.688416in}}%
\pgfpathlineto{\pgfqpoint{5.012464in}{0.689043in}}%
\pgfpathlineto{\pgfqpoint{5.007131in}{0.689672in}}%
\pgfpathlineto{\pgfqpoint{5.001798in}{0.690303in}}%
\pgfpathlineto{\pgfqpoint{4.996465in}{0.690934in}}%
\pgfpathlineto{\pgfqpoint{4.991132in}{0.691567in}}%
\pgfpathlineto{\pgfqpoint{4.985798in}{0.692202in}}%
\pgfpathlineto{\pgfqpoint{4.980465in}{0.692838in}}%
\pgfpathlineto{\pgfqpoint{4.975132in}{0.693476in}}%
\pgfpathlineto{\pgfqpoint{4.969799in}{0.694115in}}%
\pgfpathlineto{\pgfqpoint{4.964466in}{0.694755in}}%
\pgfpathlineto{\pgfqpoint{4.959132in}{0.695397in}}%
\pgfpathlineto{\pgfqpoint{4.953799in}{0.696040in}}%
\pgfpathlineto{\pgfqpoint{4.948466in}{0.696685in}}%
\pgfpathlineto{\pgfqpoint{4.943133in}{0.697332in}}%
\pgfpathlineto{\pgfqpoint{4.937799in}{0.697980in}}%
\pgfpathlineto{\pgfqpoint{4.932466in}{0.698629in}}%
\pgfpathlineto{\pgfqpoint{4.927133in}{0.699280in}}%
\pgfpathlineto{\pgfqpoint{4.921800in}{0.699932in}}%
\pgfpathlineto{\pgfqpoint{4.916467in}{0.700586in}}%
\pgfpathlineto{\pgfqpoint{4.911133in}{0.701242in}}%
\pgfpathlineto{\pgfqpoint{4.905800in}{0.701899in}}%
\pgfpathlineto{\pgfqpoint{4.900467in}{0.702557in}}%
\pgfpathlineto{\pgfqpoint{4.895134in}{0.703217in}}%
\pgfpathlineto{\pgfqpoint{4.889801in}{0.703879in}}%
\pgfpathlineto{\pgfqpoint{4.884467in}{0.704542in}}%
\pgfpathlineto{\pgfqpoint{4.879134in}{0.705206in}}%
\pgfpathlineto{\pgfqpoint{4.873801in}{0.705873in}}%
\pgfpathlineto{\pgfqpoint{4.868468in}{0.706540in}}%
\pgfpathlineto{\pgfqpoint{4.863134in}{0.707210in}}%
\pgfpathlineto{\pgfqpoint{4.857801in}{0.707881in}}%
\pgfpathlineto{\pgfqpoint{4.852468in}{0.708553in}}%
\pgfpathlineto{\pgfqpoint{4.847135in}{0.709227in}}%
\pgfpathlineto{\pgfqpoint{4.841802in}{0.709903in}}%
\pgfpathlineto{\pgfqpoint{4.836468in}{0.710580in}}%
\pgfpathlineto{\pgfqpoint{4.831135in}{0.711259in}}%
\pgfpathlineto{\pgfqpoint{4.825802in}{0.711940in}}%
\pgfpathlineto{\pgfqpoint{4.820469in}{0.712622in}}%
\pgfpathlineto{\pgfqpoint{4.815136in}{0.713306in}}%
\pgfpathlineto{\pgfqpoint{4.809802in}{0.713991in}}%
\pgfpathlineto{\pgfqpoint{4.804469in}{0.714678in}}%
\pgfpathlineto{\pgfqpoint{4.799136in}{0.715367in}}%
\pgfpathlineto{\pgfqpoint{4.793803in}{0.716057in}}%
\pgfpathlineto{\pgfqpoint{4.788469in}{0.716749in}}%
\pgfpathlineto{\pgfqpoint{4.783136in}{0.717442in}}%
\pgfpathlineto{\pgfqpoint{4.777803in}{0.718138in}}%
\pgfpathlineto{\pgfqpoint{4.772470in}{0.718834in}}%
\pgfpathlineto{\pgfqpoint{4.767137in}{0.719533in}}%
\pgfpathlineto{\pgfqpoint{4.761803in}{0.720233in}}%
\pgfpathlineto{\pgfqpoint{4.756470in}{0.720935in}}%
\pgfpathlineto{\pgfqpoint{4.751137in}{0.721639in}}%
\pgfpathlineto{\pgfqpoint{4.745804in}{0.722344in}}%
\pgfpathlineto{\pgfqpoint{4.740470in}{0.723051in}}%
\pgfpathlineto{\pgfqpoint{4.735137in}{0.723760in}}%
\pgfpathlineto{\pgfqpoint{4.729804in}{0.724470in}}%
\pgfpathlineto{\pgfqpoint{4.724471in}{0.725183in}}%
\pgfpathlineto{\pgfqpoint{4.719138in}{0.725897in}}%
\pgfpathlineto{\pgfqpoint{4.713804in}{0.726612in}}%
\pgfpathlineto{\pgfqpoint{4.708471in}{0.727330in}}%
\pgfpathlineto{\pgfqpoint{4.703138in}{0.728049in}}%
\pgfpathlineto{\pgfqpoint{4.697805in}{0.728770in}}%
\pgfpathlineto{\pgfqpoint{4.692472in}{0.729492in}}%
\pgfpathlineto{\pgfqpoint{4.687138in}{0.730217in}}%
\pgfpathlineto{\pgfqpoint{4.681805in}{0.730943in}}%
\pgfpathlineto{\pgfqpoint{4.676472in}{0.731671in}}%
\pgfpathlineto{\pgfqpoint{4.671139in}{0.732401in}}%
\pgfpathlineto{\pgfqpoint{4.665805in}{0.733132in}}%
\pgfpathlineto{\pgfqpoint{4.660472in}{0.733865in}}%
\pgfpathlineto{\pgfqpoint{4.655139in}{0.734601in}}%
\pgfpathlineto{\pgfqpoint{4.649806in}{0.735338in}}%
\pgfpathlineto{\pgfqpoint{4.644473in}{0.736076in}}%
\pgfpathlineto{\pgfqpoint{4.639139in}{0.736817in}}%
\pgfpathlineto{\pgfqpoint{4.633806in}{0.737559in}}%
\pgfpathlineto{\pgfqpoint{4.628473in}{0.738304in}}%
\pgfpathlineto{\pgfqpoint{4.623140in}{0.739050in}}%
\pgfpathlineto{\pgfqpoint{4.617807in}{0.739798in}}%
\pgfpathlineto{\pgfqpoint{4.612473in}{0.740548in}}%
\pgfpathlineto{\pgfqpoint{4.607140in}{0.741299in}}%
\pgfpathlineto{\pgfqpoint{4.601807in}{0.742053in}}%
\pgfpathlineto{\pgfqpoint{4.596474in}{0.742808in}}%
\pgfpathlineto{\pgfqpoint{4.591140in}{0.743566in}}%
\pgfpathlineto{\pgfqpoint{4.585807in}{0.744325in}}%
\pgfpathlineto{\pgfqpoint{4.580474in}{0.745086in}}%
\pgfpathlineto{\pgfqpoint{4.575141in}{0.745849in}}%
\pgfpathlineto{\pgfqpoint{4.569808in}{0.746614in}}%
\pgfpathlineto{\pgfqpoint{4.564474in}{0.747381in}}%
\pgfpathlineto{\pgfqpoint{4.559141in}{0.748150in}}%
\pgfpathlineto{\pgfqpoint{4.553808in}{0.748921in}}%
\pgfpathlineto{\pgfqpoint{4.548475in}{0.749693in}}%
\pgfpathlineto{\pgfqpoint{4.543142in}{0.750468in}}%
\pgfpathlineto{\pgfqpoint{4.537808in}{0.751245in}}%
\pgfpathlineto{\pgfqpoint{4.532475in}{0.752023in}}%
\pgfpathlineto{\pgfqpoint{4.527142in}{0.752804in}}%
\pgfpathlineto{\pgfqpoint{4.521809in}{0.753586in}}%
\pgfpathlineto{\pgfqpoint{4.516475in}{0.754371in}}%
\pgfpathlineto{\pgfqpoint{4.511142in}{0.755157in}}%
\pgfpathlineto{\pgfqpoint{4.505809in}{0.755946in}}%
\pgfpathlineto{\pgfqpoint{4.500476in}{0.756737in}}%
\pgfpathlineto{\pgfqpoint{4.495143in}{0.757529in}}%
\pgfpathlineto{\pgfqpoint{4.489809in}{0.758324in}}%
\pgfpathlineto{\pgfqpoint{4.484476in}{0.759120in}}%
\pgfpathlineto{\pgfqpoint{4.479143in}{0.759919in}}%
\pgfpathlineto{\pgfqpoint{4.473810in}{0.760720in}}%
\pgfpathlineto{\pgfqpoint{4.468476in}{0.761523in}}%
\pgfpathlineto{\pgfqpoint{4.463143in}{0.762328in}}%
\pgfpathlineto{\pgfqpoint{4.457810in}{0.763135in}}%
\pgfpathlineto{\pgfqpoint{4.452477in}{0.763944in}}%
\pgfpathlineto{\pgfqpoint{4.447144in}{0.764755in}}%
\pgfpathlineto{\pgfqpoint{4.441810in}{0.765568in}}%
\pgfpathlineto{\pgfqpoint{4.436477in}{0.766383in}}%
\pgfpathlineto{\pgfqpoint{4.431144in}{0.767201in}}%
\pgfpathlineto{\pgfqpoint{4.425811in}{0.768020in}}%
\pgfpathlineto{\pgfqpoint{4.420478in}{0.768842in}}%
\pgfpathlineto{\pgfqpoint{4.415144in}{0.769666in}}%
\pgfpathlineto{\pgfqpoint{4.409811in}{0.770492in}}%
\pgfpathlineto{\pgfqpoint{4.404478in}{0.771320in}}%
\pgfpathlineto{\pgfqpoint{4.399145in}{0.772151in}}%
\pgfpathlineto{\pgfqpoint{4.393811in}{0.772983in}}%
\pgfpathlineto{\pgfqpoint{4.388478in}{0.773818in}}%
\pgfpathlineto{\pgfqpoint{4.383145in}{0.774655in}}%
\pgfpathlineto{\pgfqpoint{4.377812in}{0.775494in}}%
\pgfpathlineto{\pgfqpoint{4.372479in}{0.776335in}}%
\pgfpathlineto{\pgfqpoint{4.367145in}{0.777179in}}%
\pgfpathlineto{\pgfqpoint{4.361812in}{0.778024in}}%
\pgfpathlineto{\pgfqpoint{4.356479in}{0.778872in}}%
\pgfpathlineto{\pgfqpoint{4.351146in}{0.779723in}}%
\pgfpathlineto{\pgfqpoint{4.345813in}{0.780575in}}%
\pgfpathlineto{\pgfqpoint{4.340479in}{0.781430in}}%
\pgfpathlineto{\pgfqpoint{4.335146in}{0.782287in}}%
\pgfpathlineto{\pgfqpoint{4.329813in}{0.783146in}}%
\pgfpathlineto{\pgfqpoint{4.324480in}{0.784008in}}%
\pgfpathlineto{\pgfqpoint{4.319146in}{0.784872in}}%
\pgfpathlineto{\pgfqpoint{4.313813in}{0.785738in}}%
\pgfpathlineto{\pgfqpoint{4.308480in}{0.786606in}}%
\pgfpathlineto{\pgfqpoint{4.303147in}{0.787477in}}%
\pgfpathlineto{\pgfqpoint{4.297814in}{0.788351in}}%
\pgfpathlineto{\pgfqpoint{4.292480in}{0.789226in}}%
\pgfpathlineto{\pgfqpoint{4.287147in}{0.790104in}}%
\pgfpathlineto{\pgfqpoint{4.281814in}{0.790984in}}%
\pgfpathlineto{\pgfqpoint{4.276481in}{0.791867in}}%
\pgfpathlineto{\pgfqpoint{4.271148in}{0.792752in}}%
\pgfpathlineto{\pgfqpoint{4.265814in}{0.793639in}}%
\pgfpathlineto{\pgfqpoint{4.260481in}{0.794529in}}%
\pgfpathlineto{\pgfqpoint{4.255148in}{0.795421in}}%
\pgfpathlineto{\pgfqpoint{4.249815in}{0.796316in}}%
\pgfpathlineto{\pgfqpoint{4.244481in}{0.797213in}}%
\pgfpathlineto{\pgfqpoint{4.239148in}{0.798113in}}%
\pgfpathlineto{\pgfqpoint{4.233815in}{0.799015in}}%
\pgfpathlineto{\pgfqpoint{4.228482in}{0.799919in}}%
\pgfpathlineto{\pgfqpoint{4.223149in}{0.800826in}}%
\pgfpathlineto{\pgfqpoint{4.217815in}{0.801736in}}%
\pgfpathlineto{\pgfqpoint{4.212482in}{0.802647in}}%
\pgfpathlineto{\pgfqpoint{4.207149in}{0.803562in}}%
\pgfpathlineto{\pgfqpoint{4.201816in}{0.804479in}}%
\pgfpathlineto{\pgfqpoint{4.196482in}{0.805398in}}%
\pgfpathlineto{\pgfqpoint{4.191149in}{0.806320in}}%
\pgfpathlineto{\pgfqpoint{4.185816in}{0.807245in}}%
\pgfpathlineto{\pgfqpoint{4.180483in}{0.808172in}}%
\pgfpathlineto{\pgfqpoint{4.175150in}{0.809102in}}%
\pgfpathlineto{\pgfqpoint{4.169816in}{0.810034in}}%
\pgfpathlineto{\pgfqpoint{4.164483in}{0.810969in}}%
\pgfpathlineto{\pgfqpoint{4.159150in}{0.811906in}}%
\pgfpathlineto{\pgfqpoint{4.153817in}{0.812846in}}%
\pgfpathlineto{\pgfqpoint{4.148484in}{0.813789in}}%
\pgfpathlineto{\pgfqpoint{4.143150in}{0.814734in}}%
\pgfpathlineto{\pgfqpoint{4.137817in}{0.815682in}}%
\pgfpathlineto{\pgfqpoint{4.132484in}{0.816633in}}%
\pgfpathlineto{\pgfqpoint{4.127151in}{0.817586in}}%
\pgfpathlineto{\pgfqpoint{4.121817in}{0.818542in}}%
\pgfpathlineto{\pgfqpoint{4.116484in}{0.819501in}}%
\pgfpathlineto{\pgfqpoint{4.111151in}{0.820462in}}%
\pgfpathlineto{\pgfqpoint{4.105818in}{0.821426in}}%
\pgfpathlineto{\pgfqpoint{4.100485in}{0.822393in}}%
\pgfpathlineto{\pgfqpoint{4.095151in}{0.823362in}}%
\pgfpathlineto{\pgfqpoint{4.089818in}{0.824334in}}%
\pgfpathlineto{\pgfqpoint{4.084485in}{0.825309in}}%
\pgfpathlineto{\pgfqpoint{4.079152in}{0.826287in}}%
\pgfpathlineto{\pgfqpoint{4.073819in}{0.827268in}}%
\pgfpathlineto{\pgfqpoint{4.068485in}{0.828251in}}%
\pgfpathlineto{\pgfqpoint{4.063152in}{0.829237in}}%
\pgfpathlineto{\pgfqpoint{4.057819in}{0.830226in}}%
\pgfpathlineto{\pgfqpoint{4.052486in}{0.831218in}}%
\pgfpathlineto{\pgfqpoint{4.047152in}{0.832213in}}%
\pgfpathlineto{\pgfqpoint{4.041819in}{0.833210in}}%
\pgfpathlineto{\pgfqpoint{4.036486in}{0.834211in}}%
\pgfpathlineto{\pgfqpoint{4.031153in}{0.835214in}}%
\pgfpathlineto{\pgfqpoint{4.025820in}{0.836220in}}%
\pgfpathlineto{\pgfqpoint{4.020486in}{0.837229in}}%
\pgfpathlineto{\pgfqpoint{4.015153in}{0.838241in}}%
\pgfpathlineto{\pgfqpoint{4.009820in}{0.839256in}}%
\pgfpathlineto{\pgfqpoint{4.004487in}{0.840274in}}%
\pgfpathlineto{\pgfqpoint{3.999154in}{0.841294in}}%
\pgfpathlineto{\pgfqpoint{3.993820in}{0.842318in}}%
\pgfpathlineto{\pgfqpoint{3.988487in}{0.843345in}}%
\pgfpathlineto{\pgfqpoint{3.983154in}{0.844375in}}%
\pgfpathlineto{\pgfqpoint{3.977821in}{0.845407in}}%
\pgfpathlineto{\pgfqpoint{3.972487in}{0.846443in}}%
\pgfpathlineto{\pgfqpoint{3.967154in}{0.847482in}}%
\pgfpathlineto{\pgfqpoint{3.961821in}{0.848524in}}%
\pgfpathlineto{\pgfqpoint{3.956488in}{0.849568in}}%
\pgfpathlineto{\pgfqpoint{3.951155in}{0.850616in}}%
\pgfpathlineto{\pgfqpoint{3.945821in}{0.851667in}}%
\pgfpathlineto{\pgfqpoint{3.940488in}{0.852722in}}%
\pgfpathlineto{\pgfqpoint{3.935155in}{0.853779in}}%
\pgfpathlineto{\pgfqpoint{3.929822in}{0.854839in}}%
\pgfpathlineto{\pgfqpoint{3.924488in}{0.855903in}}%
\pgfpathlineto{\pgfqpoint{3.919155in}{0.856969in}}%
\pgfpathlineto{\pgfqpoint{3.913822in}{0.858039in}}%
\pgfpathlineto{\pgfqpoint{3.908489in}{0.859112in}}%
\pgfpathlineto{\pgfqpoint{3.903156in}{0.860188in}}%
\pgfpathlineto{\pgfqpoint{3.897822in}{0.861268in}}%
\pgfpathlineto{\pgfqpoint{3.892489in}{0.862351in}}%
\pgfpathlineto{\pgfqpoint{3.887156in}{0.863436in}}%
\pgfpathlineto{\pgfqpoint{3.881823in}{0.864526in}}%
\pgfpathlineto{\pgfqpoint{3.876490in}{0.865618in}}%
\pgfpathlineto{\pgfqpoint{3.871156in}{0.866714in}}%
\pgfpathlineto{\pgfqpoint{3.865823in}{0.867813in}}%
\pgfpathlineto{\pgfqpoint{3.860490in}{0.868915in}}%
\pgfpathlineto{\pgfqpoint{3.855157in}{0.870021in}}%
\pgfpathlineto{\pgfqpoint{3.849823in}{0.871130in}}%
\pgfpathlineto{\pgfqpoint{3.844490in}{0.872242in}}%
\pgfpathlineto{\pgfqpoint{3.839157in}{0.873358in}}%
\pgfpathlineto{\pgfqpoint{3.833824in}{0.874477in}}%
\pgfpathlineto{\pgfqpoint{3.828491in}{0.875599in}}%
\pgfpathlineto{\pgfqpoint{3.823157in}{0.876725in}}%
\pgfpathlineto{\pgfqpoint{3.817824in}{0.877855in}}%
\pgfpathlineto{\pgfqpoint{3.812491in}{0.878988in}}%
\pgfpathlineto{\pgfqpoint{3.807158in}{0.880124in}}%
\pgfpathlineto{\pgfqpoint{3.801825in}{0.881264in}}%
\pgfpathlineto{\pgfqpoint{3.796491in}{0.882407in}}%
\pgfpathlineto{\pgfqpoint{3.791158in}{0.883554in}}%
\pgfpathlineto{\pgfqpoint{3.785825in}{0.884704in}}%
\pgfpathlineto{\pgfqpoint{3.780492in}{0.885858in}}%
\pgfpathlineto{\pgfqpoint{3.775158in}{0.887016in}}%
\pgfpathlineto{\pgfqpoint{3.769825in}{0.888177in}}%
\pgfpathlineto{\pgfqpoint{3.764492in}{0.889341in}}%
\pgfpathlineto{\pgfqpoint{3.759159in}{0.890509in}}%
\pgfpathlineto{\pgfqpoint{3.753826in}{0.891681in}}%
\pgfpathlineto{\pgfqpoint{3.748492in}{0.892857in}}%
\pgfpathlineto{\pgfqpoint{3.743159in}{0.894036in}}%
\pgfpathlineto{\pgfqpoint{3.737826in}{0.895219in}}%
\pgfpathlineto{\pgfqpoint{3.732493in}{0.896406in}}%
\pgfpathlineto{\pgfqpoint{3.727160in}{0.897596in}}%
\pgfpathlineto{\pgfqpoint{3.721826in}{0.898790in}}%
\pgfpathlineto{\pgfqpoint{3.716493in}{0.899988in}}%
\pgfpathlineto{\pgfqpoint{3.711160in}{0.901189in}}%
\pgfpathlineto{\pgfqpoint{3.705827in}{0.902395in}}%
\pgfpathlineto{\pgfqpoint{3.700493in}{0.903604in}}%
\pgfpathlineto{\pgfqpoint{3.695160in}{0.904817in}}%
\pgfpathlineto{\pgfqpoint{3.689827in}{0.906034in}}%
\pgfpathlineto{\pgfqpoint{3.684494in}{0.907254in}}%
\pgfpathlineto{\pgfqpoint{3.679161in}{0.908479in}}%
\pgfpathlineto{\pgfqpoint{3.673827in}{0.909708in}}%
\pgfpathlineto{\pgfqpoint{3.668494in}{0.910940in}}%
\pgfpathlineto{\pgfqpoint{3.663161in}{0.912176in}}%
\pgfpathlineto{\pgfqpoint{3.657828in}{0.913417in}}%
\pgfpathlineto{\pgfqpoint{3.652494in}{0.914661in}}%
\pgfpathlineto{\pgfqpoint{3.647161in}{0.915909in}}%
\pgfpathlineto{\pgfqpoint{3.641828in}{0.917161in}}%
\pgfpathlineto{\pgfqpoint{3.636495in}{0.918418in}}%
\pgfpathlineto{\pgfqpoint{3.631162in}{0.919678in}}%
\pgfpathlineto{\pgfqpoint{3.625828in}{0.920943in}}%
\pgfpathlineto{\pgfqpoint{3.620495in}{0.922211in}}%
\pgfpathlineto{\pgfqpoint{3.615162in}{0.923484in}}%
\pgfpathlineto{\pgfqpoint{3.609829in}{0.924760in}}%
\pgfpathlineto{\pgfqpoint{3.604496in}{0.926041in}}%
\pgfpathlineto{\pgfqpoint{3.599162in}{0.927326in}}%
\pgfpathlineto{\pgfqpoint{3.593829in}{0.928616in}}%
\pgfpathlineto{\pgfqpoint{3.588496in}{0.929909in}}%
\pgfpathlineto{\pgfqpoint{3.583163in}{0.931207in}}%
\pgfpathlineto{\pgfqpoint{3.577829in}{0.932509in}}%
\pgfpathlineto{\pgfqpoint{3.572496in}{0.933815in}}%
\pgfpathlineto{\pgfqpoint{3.567163in}{0.935126in}}%
\pgfpathlineto{\pgfqpoint{3.561830in}{0.936441in}}%
\pgfpathlineto{\pgfqpoint{3.556497in}{0.937760in}}%
\pgfpathlineto{\pgfqpoint{3.551163in}{0.939083in}}%
\pgfpathlineto{\pgfqpoint{3.545830in}{0.940411in}}%
\pgfpathlineto{\pgfqpoint{3.540497in}{0.941744in}}%
\pgfpathlineto{\pgfqpoint{3.535164in}{0.943081in}}%
\pgfpathlineto{\pgfqpoint{3.529831in}{0.944422in}}%
\pgfpathlineto{\pgfqpoint{3.524497in}{0.945767in}}%
\pgfpathlineto{\pgfqpoint{3.519164in}{0.947118in}}%
\pgfpathlineto{\pgfqpoint{3.513831in}{0.948472in}}%
\pgfpathlineto{\pgfqpoint{3.508498in}{0.949832in}}%
\pgfpathlineto{\pgfqpoint{3.503164in}{0.951195in}}%
\pgfpathlineto{\pgfqpoint{3.497831in}{0.952564in}}%
\pgfpathlineto{\pgfqpoint{3.492498in}{0.953937in}}%
\pgfpathlineto{\pgfqpoint{3.487165in}{0.955314in}}%
\pgfpathlineto{\pgfqpoint{3.481832in}{0.956697in}}%
\pgfpathlineto{\pgfqpoint{3.476498in}{0.958084in}}%
\pgfpathlineto{\pgfqpoint{3.471165in}{0.959475in}}%
\pgfpathlineto{\pgfqpoint{3.465832in}{0.960872in}}%
\pgfpathlineto{\pgfqpoint{3.460499in}{0.962273in}}%
\pgfpathlineto{\pgfqpoint{3.455166in}{0.963679in}}%
\pgfpathlineto{\pgfqpoint{3.449832in}{0.965089in}}%
\pgfpathlineto{\pgfqpoint{3.444499in}{0.966505in}}%
\pgfpathlineto{\pgfqpoint{3.439166in}{0.967925in}}%
\pgfpathlineto{\pgfqpoint{3.433833in}{0.969350in}}%
\pgfpathlineto{\pgfqpoint{3.428499in}{0.970781in}}%
\pgfpathlineto{\pgfqpoint{3.423166in}{0.972216in}}%
\pgfpathlineto{\pgfqpoint{3.417833in}{0.973656in}}%
\pgfpathlineto{\pgfqpoint{3.412500in}{0.975101in}}%
\pgfpathlineto{\pgfqpoint{3.407167in}{0.976551in}}%
\pgfpathlineto{\pgfqpoint{3.401833in}{0.978006in}}%
\pgfpathlineto{\pgfqpoint{3.396500in}{0.979466in}}%
\pgfpathlineto{\pgfqpoint{3.391167in}{0.980931in}}%
\pgfpathlineto{\pgfqpoint{3.385834in}{0.982401in}}%
\pgfpathlineto{\pgfqpoint{3.380500in}{0.983876in}}%
\pgfpathlineto{\pgfqpoint{3.375167in}{0.985357in}}%
\pgfpathlineto{\pgfqpoint{3.369834in}{0.986843in}}%
\pgfpathlineto{\pgfqpoint{3.364501in}{0.988333in}}%
\pgfpathlineto{\pgfqpoint{3.359168in}{0.989830in}}%
\pgfpathlineto{\pgfqpoint{3.353834in}{0.991331in}}%
\pgfpathlineto{\pgfqpoint{3.348501in}{0.992838in}}%
\pgfpathlineto{\pgfqpoint{3.343168in}{0.994350in}}%
\pgfpathlineto{\pgfqpoint{3.337835in}{0.995867in}}%
\pgfpathlineto{\pgfqpoint{3.332502in}{0.997390in}}%
\pgfpathlineto{\pgfqpoint{3.327168in}{0.998918in}}%
\pgfpathlineto{\pgfqpoint{3.321835in}{1.000452in}}%
\pgfpathlineto{\pgfqpoint{3.316502in}{1.001991in}}%
\pgfpathlineto{\pgfqpoint{3.311169in}{1.003535in}}%
\pgfpathlineto{\pgfqpoint{3.305835in}{1.005085in}}%
\pgfpathlineto{\pgfqpoint{3.300502in}{1.006641in}}%
\pgfpathlineto{\pgfqpoint{3.295169in}{1.008202in}}%
\pgfpathlineto{\pgfqpoint{3.289836in}{1.009769in}}%
\pgfpathlineto{\pgfqpoint{3.284503in}{1.011342in}}%
\pgfpathlineto{\pgfqpoint{3.279169in}{1.012920in}}%
\pgfpathlineto{\pgfqpoint{3.273836in}{1.014504in}}%
\pgfpathlineto{\pgfqpoint{3.268503in}{1.016093in}}%
\pgfpathlineto{\pgfqpoint{3.263170in}{1.017689in}}%
\pgfpathlineto{\pgfqpoint{3.257837in}{1.019290in}}%
\pgfpathlineto{\pgfqpoint{3.252503in}{1.020897in}}%
\pgfpathlineto{\pgfqpoint{3.247170in}{1.022510in}}%
\pgfpathlineto{\pgfqpoint{3.241837in}{1.024129in}}%
\pgfpathlineto{\pgfqpoint{3.236504in}{1.025753in}}%
\pgfpathlineto{\pgfqpoint{3.231170in}{1.027384in}}%
\pgfpathlineto{\pgfqpoint{3.225837in}{1.029021in}}%
\pgfpathlineto{\pgfqpoint{3.220504in}{1.030663in}}%
\pgfpathlineto{\pgfqpoint{3.215171in}{1.032312in}}%
\pgfpathlineto{\pgfqpoint{3.209838in}{1.033967in}}%
\pgfpathlineto{\pgfqpoint{3.204504in}{1.035628in}}%
\pgfpathlineto{\pgfqpoint{3.199171in}{1.037295in}}%
\pgfpathlineto{\pgfqpoint{3.193838in}{1.038968in}}%
\pgfpathlineto{\pgfqpoint{3.188505in}{1.040648in}}%
\pgfpathlineto{\pgfqpoint{3.183172in}{1.042334in}}%
\pgfpathlineto{\pgfqpoint{3.177838in}{1.044026in}}%
\pgfpathlineto{\pgfqpoint{3.172505in}{1.045724in}}%
\pgfpathlineto{\pgfqpoint{3.167172in}{1.047429in}}%
\pgfpathlineto{\pgfqpoint{3.161839in}{1.049140in}}%
\pgfpathlineto{\pgfqpoint{3.156505in}{1.050858in}}%
\pgfpathlineto{\pgfqpoint{3.151172in}{1.052582in}}%
\pgfpathlineto{\pgfqpoint{3.145839in}{1.054313in}}%
\pgfpathlineto{\pgfqpoint{3.140506in}{1.056050in}}%
\pgfpathlineto{\pgfqpoint{3.135173in}{1.057794in}}%
\pgfpathlineto{\pgfqpoint{3.129839in}{1.059544in}}%
\pgfpathlineto{\pgfqpoint{3.124506in}{1.061301in}}%
\pgfpathlineto{\pgfqpoint{3.119173in}{1.063065in}}%
\pgfpathlineto{\pgfqpoint{3.113840in}{1.064836in}}%
\pgfpathlineto{\pgfqpoint{3.108506in}{1.066613in}}%
\pgfpathlineto{\pgfqpoint{3.103173in}{1.068397in}}%
\pgfpathlineto{\pgfqpoint{3.097840in}{1.070188in}}%
\pgfpathlineto{\pgfqpoint{3.092507in}{1.071986in}}%
\pgfpathlineto{\pgfqpoint{3.087174in}{1.073791in}}%
\pgfpathlineto{\pgfqpoint{3.081840in}{1.075603in}}%
\pgfpathlineto{\pgfqpoint{3.076507in}{1.077422in}}%
\pgfpathlineto{\pgfqpoint{3.071174in}{1.079248in}}%
\pgfpathlineto{\pgfqpoint{3.065841in}{1.081081in}}%
\pgfpathlineto{\pgfqpoint{3.060508in}{1.082921in}}%
\pgfpathlineto{\pgfqpoint{3.055174in}{1.084768in}}%
\pgfpathlineto{\pgfqpoint{3.049841in}{1.086623in}}%
\pgfpathlineto{\pgfqpoint{3.044508in}{1.088485in}}%
\pgfpathlineto{\pgfqpoint{3.039175in}{1.090354in}}%
\pgfpathlineto{\pgfqpoint{3.033841in}{1.092231in}}%
\pgfpathlineto{\pgfqpoint{3.028508in}{1.094115in}}%
\pgfpathlineto{\pgfqpoint{3.023175in}{1.096006in}}%
\pgfpathlineto{\pgfqpoint{3.017842in}{1.097905in}}%
\pgfpathlineto{\pgfqpoint{3.012509in}{1.099811in}}%
\pgfpathlineto{\pgfqpoint{3.007175in}{1.101725in}}%
\pgfpathlineto{\pgfqpoint{3.001842in}{1.103647in}}%
\pgfpathlineto{\pgfqpoint{2.996509in}{1.105576in}}%
\pgfpathlineto{\pgfqpoint{2.991176in}{1.107513in}}%
\pgfpathlineto{\pgfqpoint{2.985843in}{1.109458in}}%
\pgfpathlineto{\pgfqpoint{2.980509in}{1.111410in}}%
\pgfpathlineto{\pgfqpoint{2.975176in}{1.113371in}}%
\pgfpathlineto{\pgfqpoint{2.969843in}{1.115339in}}%
\pgfpathlineto{\pgfqpoint{2.964510in}{1.117315in}}%
\pgfpathlineto{\pgfqpoint{2.959176in}{1.119300in}}%
\pgfpathlineto{\pgfqpoint{2.953843in}{1.121292in}}%
\pgfpathlineto{\pgfqpoint{2.948510in}{1.123292in}}%
\pgfpathlineto{\pgfqpoint{2.943177in}{1.125301in}}%
\pgfpathlineto{\pgfqpoint{2.937844in}{1.127318in}}%
\pgfpathlineto{\pgfqpoint{2.932510in}{1.129343in}}%
\pgfpathlineto{\pgfqpoint{2.927177in}{1.131376in}}%
\pgfpathlineto{\pgfqpoint{2.921844in}{1.133418in}}%
\pgfpathlineto{\pgfqpoint{2.916511in}{1.135468in}}%
\pgfpathlineto{\pgfqpoint{2.911178in}{1.137526in}}%
\pgfpathlineto{\pgfqpoint{2.905844in}{1.139593in}}%
\pgfpathlineto{\pgfqpoint{2.900511in}{1.141669in}}%
\pgfpathlineto{\pgfqpoint{2.895178in}{1.143753in}}%
\pgfpathlineto{\pgfqpoint{2.889845in}{1.145846in}}%
\pgfpathlineto{\pgfqpoint{2.884511in}{1.147947in}}%
\pgfpathlineto{\pgfqpoint{2.879178in}{1.150058in}}%
\pgfpathlineto{\pgfqpoint{2.873845in}{1.152177in}}%
\pgfpathlineto{\pgfqpoint{2.868512in}{1.154305in}}%
\pgfpathlineto{\pgfqpoint{2.863179in}{1.156442in}}%
\pgfpathlineto{\pgfqpoint{2.857845in}{1.158588in}}%
\pgfpathlineto{\pgfqpoint{2.852512in}{1.160743in}}%
\pgfpathlineto{\pgfqpoint{2.847179in}{1.162907in}}%
\pgfpathlineto{\pgfqpoint{2.841846in}{1.165080in}}%
\pgfpathlineto{\pgfqpoint{2.836512in}{1.167263in}}%
\pgfpathlineto{\pgfqpoint{2.831179in}{1.169454in}}%
\pgfpathlineto{\pgfqpoint{2.825846in}{1.171656in}}%
\pgfpathlineto{\pgfqpoint{2.820513in}{1.173866in}}%
\pgfpathlineto{\pgfqpoint{2.815180in}{1.176086in}}%
\pgfpathlineto{\pgfqpoint{2.809846in}{1.178316in}}%
\pgfpathlineto{\pgfqpoint{2.804513in}{1.180555in}}%
\pgfpathlineto{\pgfqpoint{2.799180in}{1.182804in}}%
\pgfpathlineto{\pgfqpoint{2.793847in}{1.185062in}}%
\pgfpathlineto{\pgfqpoint{2.788514in}{1.187330in}}%
\pgfpathlineto{\pgfqpoint{2.783180in}{1.189608in}}%
\pgfpathlineto{\pgfqpoint{2.777847in}{1.191896in}}%
\pgfpathlineto{\pgfqpoint{2.772514in}{1.194194in}}%
\pgfpathlineto{\pgfqpoint{2.767181in}{1.196502in}}%
\pgfpathlineto{\pgfqpoint{2.761847in}{1.198821in}}%
\pgfpathlineto{\pgfqpoint{2.756514in}{1.201149in}}%
\pgfpathlineto{\pgfqpoint{2.751181in}{1.203487in}}%
\pgfpathlineto{\pgfqpoint{2.745848in}{1.205836in}}%
\pgfpathlineto{\pgfqpoint{2.740515in}{1.208195in}}%
\pgfpathlineto{\pgfqpoint{2.735181in}{1.210565in}}%
\pgfpathlineto{\pgfqpoint{2.729848in}{1.212945in}}%
\pgfpathlineto{\pgfqpoint{2.724515in}{1.215336in}}%
\pgfpathlineto{\pgfqpoint{2.719182in}{1.217737in}}%
\pgfpathlineto{\pgfqpoint{2.713849in}{1.220149in}}%
\pgfpathlineto{\pgfqpoint{2.708515in}{1.222572in}}%
\pgfpathlineto{\pgfqpoint{2.703182in}{1.225005in}}%
\pgfpathlineto{\pgfqpoint{2.697849in}{1.227450in}}%
\pgfpathlineto{\pgfqpoint{2.692516in}{1.229906in}}%
\pgfpathlineto{\pgfqpoint{2.687182in}{1.232372in}}%
\pgfpathlineto{\pgfqpoint{2.681849in}{1.234850in}}%
\pgfpathlineto{\pgfqpoint{2.676516in}{1.237339in}}%
\pgfpathlineto{\pgfqpoint{2.671183in}{1.239840in}}%
\pgfpathlineto{\pgfqpoint{2.665850in}{1.242352in}}%
\pgfpathlineto{\pgfqpoint{2.660516in}{1.244875in}}%
\pgfpathlineto{\pgfqpoint{2.655183in}{1.247410in}}%
\pgfpathlineto{\pgfqpoint{2.649850in}{1.249956in}}%
\pgfpathlineto{\pgfqpoint{2.644517in}{1.252514in}}%
\pgfpathlineto{\pgfqpoint{2.639184in}{1.255084in}}%
\pgfpathlineto{\pgfqpoint{2.633850in}{1.257666in}}%
\pgfpathlineto{\pgfqpoint{2.628517in}{1.260260in}}%
\pgfpathlineto{\pgfqpoint{2.623184in}{1.262866in}}%
\pgfpathlineto{\pgfqpoint{2.617851in}{1.265484in}}%
\pgfpathlineto{\pgfqpoint{2.612517in}{1.268114in}}%
\pgfpathlineto{\pgfqpoint{2.607184in}{1.270756in}}%
\pgfpathlineto{\pgfqpoint{2.601851in}{1.273411in}}%
\pgfpathlineto{\pgfqpoint{2.596518in}{1.276078in}}%
\pgfpathlineto{\pgfqpoint{2.591185in}{1.278758in}}%
\pgfpathlineto{\pgfqpoint{2.585851in}{1.281450in}}%
\pgfpathlineto{\pgfqpoint{2.580518in}{1.284155in}}%
\pgfpathlineto{\pgfqpoint{2.575185in}{1.286873in}}%
\pgfpathlineto{\pgfqpoint{2.569852in}{1.289604in}}%
\pgfpathlineto{\pgfqpoint{2.564518in}{1.292348in}}%
\pgfpathlineto{\pgfqpoint{2.559185in}{1.295105in}}%
\pgfpathlineto{\pgfqpoint{2.553852in}{1.297875in}}%
\pgfpathlineto{\pgfqpoint{2.548519in}{1.300658in}}%
\pgfpathlineto{\pgfqpoint{2.543186in}{1.303455in}}%
\pgfpathlineto{\pgfqpoint{2.537852in}{1.306265in}}%
\pgfpathlineto{\pgfqpoint{2.532519in}{1.309089in}}%
\pgfpathlineto{\pgfqpoint{2.527186in}{1.311927in}}%
\pgfpathlineto{\pgfqpoint{2.521853in}{1.314778in}}%
\pgfpathlineto{\pgfqpoint{2.516520in}{1.317643in}}%
\pgfpathlineto{\pgfqpoint{2.511186in}{1.320522in}}%
\pgfpathlineto{\pgfqpoint{2.505853in}{1.323416in}}%
\pgfpathlineto{\pgfqpoint{2.500520in}{1.326323in}}%
\pgfpathlineto{\pgfqpoint{2.495187in}{1.329245in}}%
\pgfpathlineto{\pgfqpoint{2.489853in}{1.332181in}}%
\pgfpathlineto{\pgfqpoint{2.484520in}{1.335131in}}%
\pgfpathlineto{\pgfqpoint{2.479187in}{1.338096in}}%
\pgfpathlineto{\pgfqpoint{2.473854in}{1.341076in}}%
\pgfpathlineto{\pgfqpoint{2.468521in}{1.344071in}}%
\pgfpathlineto{\pgfqpoint{2.463187in}{1.347080in}}%
\pgfpathlineto{\pgfqpoint{2.457854in}{1.350105in}}%
\pgfpathlineto{\pgfqpoint{2.452521in}{1.353145in}}%
\pgfpathlineto{\pgfqpoint{2.447188in}{1.356200in}}%
\pgfpathlineto{\pgfqpoint{2.441855in}{1.359270in}}%
\pgfpathlineto{\pgfqpoint{2.436521in}{1.362356in}}%
\pgfpathlineto{\pgfqpoint{2.431188in}{1.365458in}}%
\pgfpathlineto{\pgfqpoint{2.425855in}{1.368575in}}%
\pgfpathlineto{\pgfqpoint{2.420522in}{1.371708in}}%
\pgfpathlineto{\pgfqpoint{2.415188in}{1.374858in}}%
\pgfpathlineto{\pgfqpoint{2.409855in}{1.378023in}}%
\pgfpathlineto{\pgfqpoint{2.404522in}{1.381204in}}%
\pgfpathlineto{\pgfqpoint{2.399189in}{1.384402in}}%
\pgfpathlineto{\pgfqpoint{2.393856in}{1.387616in}}%
\pgfpathlineto{\pgfqpoint{2.388522in}{1.390847in}}%
\pgfpathlineto{\pgfqpoint{2.383189in}{1.394095in}}%
\pgfpathlineto{\pgfqpoint{2.377856in}{1.397360in}}%
\pgfpathlineto{\pgfqpoint{2.372523in}{1.400641in}}%
\pgfpathlineto{\pgfqpoint{2.367190in}{1.403940in}}%
\pgfpathlineto{\pgfqpoint{2.361856in}{1.407256in}}%
\pgfpathlineto{\pgfqpoint{2.356523in}{1.410589in}}%
\pgfpathlineto{\pgfqpoint{2.351190in}{1.413940in}}%
\pgfpathlineto{\pgfqpoint{2.345857in}{1.417308in}}%
\pgfpathlineto{\pgfqpoint{2.340523in}{1.420695in}}%
\pgfpathlineto{\pgfqpoint{2.335190in}{1.424099in}}%
\pgfpathlineto{\pgfqpoint{2.329857in}{1.427521in}}%
\pgfpathlineto{\pgfqpoint{2.324524in}{1.430962in}}%
\pgfpathlineto{\pgfqpoint{2.319191in}{1.434421in}}%
\pgfpathlineto{\pgfqpoint{2.313857in}{1.437899in}}%
\pgfpathlineto{\pgfqpoint{2.308524in}{1.441395in}}%
\pgfpathlineto{\pgfqpoint{2.303191in}{1.444910in}}%
\pgfpathlineto{\pgfqpoint{2.297858in}{1.448444in}}%
\pgfpathlineto{\pgfqpoint{2.292524in}{1.451997in}}%
\pgfpathlineto{\pgfqpoint{2.287191in}{1.455570in}}%
\pgfpathlineto{\pgfqpoint{2.281858in}{1.459162in}}%
\pgfpathlineto{\pgfqpoint{2.276525in}{1.462774in}}%
\pgfpathlineto{\pgfqpoint{2.271192in}{1.466405in}}%
\pgfpathlineto{\pgfqpoint{2.265858in}{1.470056in}}%
\pgfpathlineto{\pgfqpoint{2.260525in}{1.473728in}}%
\pgfpathlineto{\pgfqpoint{2.255192in}{1.477420in}}%
\pgfpathlineto{\pgfqpoint{2.249859in}{1.481132in}}%
\pgfpathlineto{\pgfqpoint{2.244526in}{1.484865in}}%
\pgfpathlineto{\pgfqpoint{2.239192in}{1.488618in}}%
\pgfpathlineto{\pgfqpoint{2.233859in}{1.492393in}}%
\pgfpathlineto{\pgfqpoint{2.228526in}{1.496189in}}%
\pgfpathlineto{\pgfqpoint{2.223193in}{1.500006in}}%
\pgfpathlineto{\pgfqpoint{2.217859in}{1.503844in}}%
\pgfpathlineto{\pgfqpoint{2.212526in}{1.507704in}}%
\pgfpathlineto{\pgfqpoint{2.207193in}{1.511586in}}%
\pgfpathlineto{\pgfqpoint{2.201860in}{1.515491in}}%
\pgfpathlineto{\pgfqpoint{2.196527in}{1.519417in}}%
\pgfpathlineto{\pgfqpoint{2.191193in}{1.523366in}}%
\pgfpathlineto{\pgfqpoint{2.185860in}{1.527337in}}%
\pgfpathlineto{\pgfqpoint{2.180527in}{1.531331in}}%
\pgfpathlineto{\pgfqpoint{2.175194in}{1.535349in}}%
\pgfpathlineto{\pgfqpoint{2.169861in}{1.539389in}}%
\pgfpathlineto{\pgfqpoint{2.164527in}{1.543453in}}%
\pgfpathlineto{\pgfqpoint{2.159194in}{1.547540in}}%
\pgfpathlineto{\pgfqpoint{2.153861in}{1.551652in}}%
\pgfpathlineto{\pgfqpoint{2.148528in}{1.555787in}}%
\pgfpathlineto{\pgfqpoint{2.143194in}{1.559947in}}%
\pgfpathlineto{\pgfqpoint{2.137861in}{1.564131in}}%
\pgfpathlineto{\pgfqpoint{2.132528in}{1.568339in}}%
\pgfpathlineto{\pgfqpoint{2.127195in}{1.572573in}}%
\pgfpathlineto{\pgfqpoint{2.121862in}{1.576832in}}%
\pgfpathlineto{\pgfqpoint{2.116528in}{1.581116in}}%
\pgfpathlineto{\pgfqpoint{2.111195in}{1.585425in}}%
\pgfpathlineto{\pgfqpoint{2.105862in}{1.589761in}}%
\pgfpathlineto{\pgfqpoint{2.100529in}{1.594122in}}%
\pgfpathlineto{\pgfqpoint{2.095196in}{1.598510in}}%
\pgfpathlineto{\pgfqpoint{2.089862in}{1.602924in}}%
\pgfpathlineto{\pgfqpoint{2.084529in}{1.607365in}}%
\pgfpathlineto{\pgfqpoint{2.079196in}{1.611833in}}%
\pgfpathlineto{\pgfqpoint{2.073863in}{1.616328in}}%
\pgfpathlineto{\pgfqpoint{2.068529in}{1.620851in}}%
\pgfpathlineto{\pgfqpoint{2.063196in}{1.625401in}}%
\pgfpathlineto{\pgfqpoint{2.057863in}{1.629980in}}%
\pgfpathlineto{\pgfqpoint{2.052530in}{1.634586in}}%
\pgfpathlineto{\pgfqpoint{2.047197in}{1.639221in}}%
\pgfpathlineto{\pgfqpoint{2.041863in}{1.643885in}}%
\pgfpathlineto{\pgfqpoint{2.036530in}{1.648578in}}%
\pgfpathlineto{\pgfqpoint{2.031197in}{1.653301in}}%
\pgfpathlineto{\pgfqpoint{2.025864in}{1.658053in}}%
\pgfpathlineto{\pgfqpoint{2.020531in}{1.662834in}}%
\pgfpathlineto{\pgfqpoint{2.015197in}{1.667646in}}%
\pgfpathlineto{\pgfqpoint{2.009864in}{1.672489in}}%
\pgfpathlineto{\pgfqpoint{2.004531in}{1.677362in}}%
\pgfpathlineto{\pgfqpoint{1.999198in}{1.682266in}}%
\pgfpathlineto{\pgfqpoint{1.993864in}{1.687201in}}%
\pgfpathlineto{\pgfqpoint{1.988531in}{1.692168in}}%
\pgfpathlineto{\pgfqpoint{1.983198in}{1.697167in}}%
\pgfpathlineto{\pgfqpoint{1.977865in}{1.702199in}}%
\pgfpathlineto{\pgfqpoint{1.972532in}{1.707262in}}%
\pgfpathlineto{\pgfqpoint{1.967198in}{1.712359in}}%
\pgfpathlineto{\pgfqpoint{1.961865in}{1.717489in}}%
\pgfpathlineto{\pgfqpoint{1.956532in}{1.722652in}}%
\pgfpathlineto{\pgfqpoint{1.951199in}{1.727849in}}%
\pgfpathlineto{\pgfqpoint{1.945865in}{1.733081in}}%
\pgfpathlineto{\pgfqpoint{1.940532in}{1.738347in}}%
\pgfpathlineto{\pgfqpoint{1.935199in}{1.743647in}}%
\pgfpathlineto{\pgfqpoint{1.929866in}{1.748983in}}%
\pgfpathlineto{\pgfqpoint{1.924533in}{1.754355in}}%
\pgfpathlineto{\pgfqpoint{1.919199in}{1.759762in}}%
\pgfpathlineto{\pgfqpoint{1.913866in}{1.765206in}}%
\pgfpathlineto{\pgfqpoint{1.908533in}{1.770686in}}%
\pgfpathlineto{\pgfqpoint{1.903200in}{1.776204in}}%
\pgfpathlineto{\pgfqpoint{1.897867in}{1.781758in}}%
\pgfpathlineto{\pgfqpoint{1.892533in}{1.787351in}}%
\pgfpathlineto{\pgfqpoint{1.887200in}{1.792982in}}%
\pgfpathlineto{\pgfqpoint{1.881867in}{1.798651in}}%
\pgfpathlineto{\pgfqpoint{1.876534in}{1.804359in}}%
\pgfpathlineto{\pgfqpoint{1.871200in}{1.810107in}}%
\pgfpathlineto{\pgfqpoint{1.865867in}{1.815894in}}%
\pgfpathlineto{\pgfqpoint{1.860534in}{1.821721in}}%
\pgfpathlineto{\pgfqpoint{1.855201in}{1.827589in}}%
\pgfpathlineto{\pgfqpoint{1.849868in}{1.833498in}}%
\pgfpathlineto{\pgfqpoint{1.844534in}{1.839449in}}%
\pgfpathlineto{\pgfqpoint{1.839201in}{1.845441in}}%
\pgfpathlineto{\pgfqpoint{1.833868in}{1.851476in}}%
\pgfpathlineto{\pgfqpoint{1.828535in}{1.857554in}}%
\pgfpathlineto{\pgfqpoint{1.823202in}{1.863674in}}%
\pgfpathlineto{\pgfqpoint{1.817868in}{1.869839in}}%
\pgfpathlineto{\pgfqpoint{1.812535in}{1.876047in}}%
\pgfpathlineto{\pgfqpoint{1.807202in}{1.882301in}}%
\pgfpathlineto{\pgfqpoint{1.801869in}{1.888599in}}%
\pgfpathlineto{\pgfqpoint{1.796535in}{1.894943in}}%
\pgfpathlineto{\pgfqpoint{1.791202in}{1.901333in}}%
\pgfpathlineto{\pgfqpoint{1.785869in}{1.907770in}}%
\pgfpathlineto{\pgfqpoint{1.780536in}{1.914254in}}%
\pgfpathlineto{\pgfqpoint{1.775203in}{1.920785in}}%
\pgfpathlineto{\pgfqpoint{1.769869in}{1.927365in}}%
\pgfpathlineto{\pgfqpoint{1.764536in}{1.933993in}}%
\pgfpathlineto{\pgfqpoint{1.759203in}{1.940671in}}%
\pgfpathlineto{\pgfqpoint{1.753870in}{1.947398in}}%
\pgfpathlineto{\pgfqpoint{1.748537in}{1.954176in}}%
\pgfpathlineto{\pgfqpoint{1.743203in}{1.961005in}}%
\pgfpathlineto{\pgfqpoint{1.737870in}{1.967885in}}%
\pgfpathlineto{\pgfqpoint{1.732537in}{1.974818in}}%
\pgfpathlineto{\pgfqpoint{1.727204in}{1.981803in}}%
\pgfpathlineto{\pgfqpoint{1.721870in}{1.988841in}}%
\pgfpathlineto{\pgfqpoint{1.716537in}{1.995934in}}%
\pgfpathlineto{\pgfqpoint{1.711204in}{2.003080in}}%
\pgfpathlineto{\pgfqpoint{1.705871in}{2.010282in}}%
\pgfpathlineto{\pgfqpoint{1.700538in}{2.017540in}}%
\pgfpathlineto{\pgfqpoint{1.695204in}{2.024854in}}%
\pgfpathlineto{\pgfqpoint{1.689871in}{2.032226in}}%
\pgfpathlineto{\pgfqpoint{1.684538in}{2.039655in}}%
\pgfpathlineto{\pgfqpoint{1.679205in}{2.047142in}}%
\pgfpathlineto{\pgfqpoint{1.673871in}{2.054689in}}%
\pgfpathlineto{\pgfqpoint{1.668538in}{2.062296in}}%
\pgfpathlineto{\pgfqpoint{1.663205in}{2.069963in}}%
\pgfpathlineto{\pgfqpoint{1.657872in}{2.077692in}}%
\pgfpathlineto{\pgfqpoint{1.652539in}{2.085482in}}%
\pgfpathlineto{\pgfqpoint{1.647205in}{2.093335in}}%
\pgfpathlineto{\pgfqpoint{1.641872in}{2.101252in}}%
\pgfpathlineto{\pgfqpoint{1.636539in}{2.109234in}}%
\pgfpathlineto{\pgfqpoint{1.631206in}{2.117280in}}%
\pgfpathlineto{\pgfqpoint{1.625873in}{2.125392in}}%
\pgfpathlineto{\pgfqpoint{1.620539in}{2.133571in}}%
\pgfpathlineto{\pgfqpoint{1.615206in}{2.141818in}}%
\pgfpathlineto{\pgfqpoint{1.609873in}{2.150133in}}%
\pgfpathlineto{\pgfqpoint{1.604540in}{2.158517in}}%
\pgfpathlineto{\pgfqpoint{1.599206in}{2.166971in}}%
\pgfpathlineto{\pgfqpoint{1.593873in}{2.175497in}}%
\pgfpathlineto{\pgfqpoint{1.588540in}{2.184094in}}%
\pgfpathlineto{\pgfqpoint{1.583207in}{2.192764in}}%
\pgfpathlineto{\pgfqpoint{1.577874in}{2.201508in}}%
\pgfpathlineto{\pgfqpoint{1.572540in}{2.210326in}}%
\pgfpathlineto{\pgfqpoint{1.567207in}{2.219220in}}%
\pgfpathlineto{\pgfqpoint{1.561874in}{2.228191in}}%
\pgfpathlineto{\pgfqpoint{1.556541in}{2.237240in}}%
\pgfpathlineto{\pgfqpoint{1.551208in}{2.246367in}}%
\pgfpathlineto{\pgfqpoint{1.545874in}{2.255573in}}%
\pgfpathlineto{\pgfqpoint{1.540541in}{2.264861in}}%
\pgfpathlineto{\pgfqpoint{1.535208in}{2.274230in}}%
\pgfpathlineto{\pgfqpoint{1.529875in}{2.283682in}}%
\pgfpathlineto{\pgfqpoint{1.524541in}{2.293218in}}%
\pgfpathlineto{\pgfqpoint{1.519208in}{2.302839in}}%
\pgfpathlineto{\pgfqpoint{1.513875in}{2.312547in}}%
\pgfpathlineto{\pgfqpoint{1.508542in}{2.322342in}}%
\pgfpathlineto{\pgfqpoint{1.503209in}{2.332225in}}%
\pgfpathlineto{\pgfqpoint{1.497875in}{2.342198in}}%
\pgfpathlineto{\pgfqpoint{1.492542in}{2.352262in}}%
\pgfpathlineto{\pgfqpoint{1.487209in}{2.362419in}}%
\pgfpathlineto{\pgfqpoint{1.481876in}{2.372669in}}%
\pgfpathlineto{\pgfqpoint{1.476543in}{2.383014in}}%
\pgfpathlineto{\pgfqpoint{1.471209in}{2.393455in}}%
\pgfpathlineto{\pgfqpoint{1.465876in}{2.403993in}}%
\pgfpathlineto{\pgfqpoint{1.460543in}{2.414631in}}%
\pgfpathlineto{\pgfqpoint{1.455210in}{2.425368in}}%
\pgfpathlineto{\pgfqpoint{1.449876in}{2.436208in}}%
\pgfpathlineto{\pgfqpoint{1.444543in}{2.447150in}}%
\pgfpathlineto{\pgfqpoint{1.439210in}{2.458198in}}%
\pgfpathlineto{\pgfqpoint{1.433877in}{2.469351in}}%
\pgfpathlineto{\pgfqpoint{1.428544in}{2.480612in}}%
\pgfpathlineto{\pgfqpoint{1.423210in}{2.491983in}}%
\pgfpathlineto{\pgfqpoint{1.417877in}{2.503464in}}%
\pgfpathlineto{\pgfqpoint{1.412544in}{2.515058in}}%
\pgfpathlineto{\pgfqpoint{1.407211in}{2.526766in}}%
\pgfpathlineto{\pgfqpoint{1.401877in}{2.538589in}}%
\pgfpathlineto{\pgfqpoint{1.396544in}{2.550531in}}%
\pgfpathlineto{\pgfqpoint{1.391211in}{2.562591in}}%
\pgfpathlineto{\pgfqpoint{1.385878in}{2.574773in}}%
\pgfpathlineto{\pgfqpoint{1.380545in}{2.587078in}}%
\pgfpathlineto{\pgfqpoint{1.375211in}{2.599507in}}%
\pgfpathlineto{\pgfqpoint{1.369878in}{2.612064in}}%
\pgfpathlineto{\pgfqpoint{1.364545in}{2.624749in}}%
\pgfpathlineto{\pgfqpoint{1.359212in}{2.637565in}}%
\pgfpathlineto{\pgfqpoint{1.353879in}{2.650513in}}%
\pgfpathlineto{\pgfqpoint{1.348545in}{2.663596in}}%
\pgfpathlineto{\pgfqpoint{1.343212in}{2.676816in}}%
\pgfpathlineto{\pgfqpoint{1.337879in}{2.690176in}}%
\pgfpathlineto{\pgfqpoint{1.332546in}{2.703676in}}%
\pgfpathlineto{\pgfqpoint{1.327212in}{2.717320in}}%
\pgfpathlineto{\pgfqpoint{1.321879in}{2.731110in}}%
\pgfpathlineto{\pgfqpoint{1.316546in}{2.745048in}}%
\pgfpathlineto{\pgfqpoint{1.311213in}{2.759136in}}%
\pgfpathlineto{\pgfqpoint{1.305880in}{2.773378in}}%
\pgfpathlineto{\pgfqpoint{1.300546in}{2.787775in}}%
\pgfpathlineto{\pgfqpoint{1.295213in}{2.802330in}}%
\pgfpathlineto{\pgfqpoint{1.289880in}{2.817046in}}%
\pgfpathlineto{\pgfqpoint{1.284547in}{2.831926in}}%
\pgfpathlineto{\pgfqpoint{1.279214in}{2.846971in}}%
\pgfpathlineto{\pgfqpoint{1.273880in}{2.862185in}}%
\pgfpathlineto{\pgfqpoint{1.268547in}{2.877572in}}%
\pgfpathlineto{\pgfqpoint{1.263214in}{2.893132in}}%
\pgfpathlineto{\pgfqpoint{1.257881in}{2.908871in}}%
\pgfpathlineto{\pgfqpoint{1.252547in}{2.924790in}}%
\pgfpathlineto{\pgfqpoint{1.247214in}{2.940894in}}%
\pgfpathlineto{\pgfqpoint{1.241881in}{2.957184in}}%
\pgfpathlineto{\pgfqpoint{1.236548in}{2.973665in}}%
\pgfpathlineto{\pgfqpoint{1.231215in}{2.990339in}}%
\pgfpathlineto{\pgfqpoint{1.225881in}{3.007211in}}%
\pgfpathlineto{\pgfqpoint{1.220548in}{3.024283in}}%
\pgfpathlineto{\pgfqpoint{1.215215in}{3.041560in}}%
\pgfpathlineto{\pgfqpoint{1.209882in}{3.059045in}}%
\pgfpathlineto{\pgfqpoint{1.204549in}{3.076741in}}%
\pgfpathlineto{\pgfqpoint{1.199215in}{3.094653in}}%
\pgfpathlineto{\pgfqpoint{1.193882in}{3.112785in}}%
\pgfpathlineto{\pgfqpoint{1.188549in}{3.131140in}}%
\pgfpathlineto{\pgfqpoint{1.183216in}{3.149723in}}%
\pgfpathlineto{\pgfqpoint{1.177882in}{3.168538in}}%
\pgfpathlineto{\pgfqpoint{1.172549in}{3.187590in}}%
\pgfpathlineto{\pgfqpoint{1.167216in}{3.206883in}}%
\pgfpathlineto{\pgfqpoint{1.161883in}{3.226421in}}%
\pgfpathlineto{\pgfqpoint{1.156550in}{3.246210in}}%
\pgfpathlineto{\pgfqpoint{1.151216in}{3.266253in}}%
\pgfpathlineto{\pgfqpoint{1.145883in}{3.286557in}}%
\pgfpathlineto{\pgfqpoint{1.140550in}{3.307126in}}%
\pgfpathlineto{\pgfqpoint{1.135217in}{3.327965in}}%
\pgfpathlineto{\pgfqpoint{1.129883in}{3.349081in}}%
\pgfpathlineto{\pgfqpoint{1.124550in}{3.370477in}}%
\pgfpathlineto{\pgfqpoint{1.119217in}{3.392160in}}%
\pgfpathlineto{\pgfqpoint{1.113884in}{3.414137in}}%
\pgfpathlineto{\pgfqpoint{1.108551in}{3.436411in}}%
\pgfpathlineto{\pgfqpoint{1.103217in}{3.458991in}}%
\pgfpathlineto{\pgfqpoint{1.097884in}{3.481882in}}%
\pgfpathlineto{\pgfqpoint{1.092551in}{3.505091in}}%
\pgfpathlineto{\pgfqpoint{1.087218in}{3.528623in}}%
\pgfpathlineto{\pgfqpoint{1.081885in}{3.552487in}}%
\pgfpathlineto{\pgfqpoint{1.076551in}{3.576690in}}%
\pgfpathlineto{\pgfqpoint{1.071218in}{3.601237in}}%
\pgfpathlineto{\pgfqpoint{1.065885in}{3.626138in}}%
\pgfpathlineto{\pgfqpoint{1.060552in}{3.651399in}}%
\pgfpathlineto{\pgfqpoint{1.055218in}{3.677028in}}%
\pgfpathlineto{\pgfqpoint{1.049885in}{3.703034in}}%
\pgfpathlineto{\pgfqpoint{1.044552in}{3.729425in}}%
\pgfpathlineto{\pgfqpoint{1.039219in}{3.756209in}}%
\pgfpathlineto{\pgfqpoint{1.033886in}{3.783396in}}%
\pgfpathlineto{\pgfqpoint{1.028552in}{3.810994in}}%
\pgfpathlineto{\pgfqpoint{1.023219in}{3.839013in}}%
\pgfpathlineto{\pgfqpoint{1.017886in}{3.867463in}}%
\pgfpathlineto{\pgfqpoint{1.012553in}{3.896354in}}%
\pgfpathlineto{\pgfqpoint{1.007220in}{3.925695in}}%
\pgfpathlineto{\pgfqpoint{1.001886in}{3.955498in}}%
\pgfpathlineto{\pgfqpoint{0.996553in}{3.985774in}}%
\pgfpathlineto{\pgfqpoint{0.991220in}{4.016533in}}%
\pgfpathlineto{\pgfqpoint{0.985887in}{4.047788in}}%
\pgfpathlineto{\pgfqpoint{0.980553in}{4.079551in}}%
\pgfpathlineto{\pgfqpoint{0.975220in}{4.111834in}}%
\pgfpathlineto{\pgfqpoint{0.969887in}{4.144650in}}%
\pgfpathlineto{\pgfqpoint{0.964554in}{4.178012in}}%
\pgfpathlineto{\pgfqpoint{0.959221in}{4.211935in}}%
\pgfpathlineto{\pgfqpoint{0.953887in}{4.246431in}}%
\pgfpathlineto{\pgfqpoint{0.948554in}{4.281517in}}%
\pgfpathlineto{\pgfqpoint{0.943221in}{4.317207in}}%
\pgfpathlineto{\pgfqpoint{0.937888in}{4.353517in}}%
\pgfpathlineto{\pgfqpoint{0.932555in}{4.390463in}}%
\pgfpathlineto{\pgfqpoint{0.927221in}{4.428063in}}%
\pgfpathlineto{\pgfqpoint{0.921888in}{4.466333in}}%
\pgfpathlineto{\pgfqpoint{0.916555in}{4.505291in}}%
\pgfpathlineto{\pgfqpoint{0.911222in}{4.544958in}}%
\pgfpathlineto{\pgfqpoint{0.905888in}{4.585351in}}%
\pgfpathlineto{\pgfqpoint{0.900555in}{4.626491in}}%
\pgfpathlineto{\pgfqpoint{0.895222in}{4.668400in}}%
\pgfpathlineto{\pgfqpoint{0.889889in}{4.711098in}}%
\pgfpathlineto{\pgfqpoint{0.884556in}{4.754608in}}%
\pgfpathlineto{\pgfqpoint{0.879222in}{4.798955in}}%
\pgfpathlineto{\pgfqpoint{0.873889in}{4.844161in}}%
\pgfpathlineto{\pgfqpoint{0.868556in}{4.890252in}}%
\pgfpathlineto{\pgfqpoint{0.863223in}{4.937255in}}%
\pgfpathlineto{\pgfqpoint{0.857889in}{4.985197in}}%
\pgfpathlineto{\pgfqpoint{0.852556in}{5.034106in}}%
\pgfpathlineto{\pgfqpoint{0.847223in}{5.084012in}}%
\pgfpathlineto{\pgfqpoint{0.759998in}{4.259481in}}%
\pgfpathclose%
\pgfusepath{fill}%
\end{pgfscope}%
\begin{pgfscope}%
\pgfsetbuttcap%
\pgfsetroundjoin%
\definecolor{currentfill}{rgb}{0.000000,0.000000,0.000000}%
\pgfsetfillcolor{currentfill}%
\pgfsetlinewidth{0.803000pt}%
\definecolor{currentstroke}{rgb}{0.000000,0.000000,0.000000}%
\pgfsetstrokecolor{currentstroke}%
\pgfsetdash{}{0pt}%
\pgfsys@defobject{currentmarker}{\pgfqpoint{0.000000in}{-0.048611in}}{\pgfqpoint{0.000000in}{0.000000in}}{%
\pgfpathmoveto{\pgfqpoint{0.000000in}{0.000000in}}%
\pgfpathlineto{\pgfqpoint{0.000000in}{-0.048611in}}%
\pgfusepath{stroke,fill}%
}%
\begin{pgfscope}%
\pgfsys@transformshift{1.414186in}{0.554012in}%
\pgfsys@useobject{currentmarker}{}%
\end{pgfscope}%
\end{pgfscope}%
\begin{pgfscope}%
\definecolor{textcolor}{rgb}{0.000000,0.000000,0.000000}%
\pgfsetstrokecolor{textcolor}%
\pgfsetfillcolor{textcolor}%
\pgftext[x=1.414186in,y=0.456790in,,top]{\color{textcolor}\rmfamily\fontsize{10.000000}{12.000000}\selectfont \(\displaystyle {0.01}\)}%
\end{pgfscope}%
\begin{pgfscope}%
\pgfsetbuttcap%
\pgfsetroundjoin%
\definecolor{currentfill}{rgb}{0.000000,0.000000,0.000000}%
\pgfsetfillcolor{currentfill}%
\pgfsetlinewidth{0.803000pt}%
\definecolor{currentstroke}{rgb}{0.000000,0.000000,0.000000}%
\pgfsetstrokecolor{currentstroke}%
\pgfsetdash{}{0pt}%
\pgfsys@defobject{currentmarker}{\pgfqpoint{0.000000in}{-0.048611in}}{\pgfqpoint{0.000000in}{0.000000in}}{%
\pgfpathmoveto{\pgfqpoint{0.000000in}{0.000000in}}%
\pgfpathlineto{\pgfqpoint{0.000000in}{-0.048611in}}%
\pgfusepath{stroke,fill}%
}%
\begin{pgfscope}%
\pgfsys@transformshift{2.504499in}{0.554012in}%
\pgfsys@useobject{currentmarker}{}%
\end{pgfscope}%
\end{pgfscope}%
\begin{pgfscope}%
\definecolor{textcolor}{rgb}{0.000000,0.000000,0.000000}%
\pgfsetstrokecolor{textcolor}%
\pgfsetfillcolor{textcolor}%
\pgftext[x=2.504499in,y=0.456790in,,top]{\color{textcolor}\rmfamily\fontsize{10.000000}{12.000000}\selectfont \(\displaystyle {0.02}\)}%
\end{pgfscope}%
\begin{pgfscope}%
\pgfsetbuttcap%
\pgfsetroundjoin%
\definecolor{currentfill}{rgb}{0.000000,0.000000,0.000000}%
\pgfsetfillcolor{currentfill}%
\pgfsetlinewidth{0.803000pt}%
\definecolor{currentstroke}{rgb}{0.000000,0.000000,0.000000}%
\pgfsetstrokecolor{currentstroke}%
\pgfsetdash{}{0pt}%
\pgfsys@defobject{currentmarker}{\pgfqpoint{0.000000in}{-0.048611in}}{\pgfqpoint{0.000000in}{0.000000in}}{%
\pgfpathmoveto{\pgfqpoint{0.000000in}{0.000000in}}%
\pgfpathlineto{\pgfqpoint{0.000000in}{-0.048611in}}%
\pgfusepath{stroke,fill}%
}%
\begin{pgfscope}%
\pgfsys@transformshift{3.594811in}{0.554012in}%
\pgfsys@useobject{currentmarker}{}%
\end{pgfscope}%
\end{pgfscope}%
\begin{pgfscope}%
\definecolor{textcolor}{rgb}{0.000000,0.000000,0.000000}%
\pgfsetstrokecolor{textcolor}%
\pgfsetfillcolor{textcolor}%
\pgftext[x=3.594811in,y=0.456790in,,top]{\color{textcolor}\rmfamily\fontsize{10.000000}{12.000000}\selectfont \(\displaystyle {0.03}\)}%
\end{pgfscope}%
\begin{pgfscope}%
\pgfsetbuttcap%
\pgfsetroundjoin%
\definecolor{currentfill}{rgb}{0.000000,0.000000,0.000000}%
\pgfsetfillcolor{currentfill}%
\pgfsetlinewidth{0.803000pt}%
\definecolor{currentstroke}{rgb}{0.000000,0.000000,0.000000}%
\pgfsetstrokecolor{currentstroke}%
\pgfsetdash{}{0pt}%
\pgfsys@defobject{currentmarker}{\pgfqpoint{0.000000in}{-0.048611in}}{\pgfqpoint{0.000000in}{0.000000in}}{%
\pgfpathmoveto{\pgfqpoint{0.000000in}{0.000000in}}%
\pgfpathlineto{\pgfqpoint{0.000000in}{-0.048611in}}%
\pgfusepath{stroke,fill}%
}%
\begin{pgfscope}%
\pgfsys@transformshift{4.685124in}{0.554012in}%
\pgfsys@useobject{currentmarker}{}%
\end{pgfscope}%
\end{pgfscope}%
\begin{pgfscope}%
\definecolor{textcolor}{rgb}{0.000000,0.000000,0.000000}%
\pgfsetstrokecolor{textcolor}%
\pgfsetfillcolor{textcolor}%
\pgftext[x=4.685124in,y=0.456790in,,top]{\color{textcolor}\rmfamily\fontsize{10.000000}{12.000000}\selectfont \(\displaystyle {0.04}\)}%
\end{pgfscope}%
\begin{pgfscope}%
\pgfsetbuttcap%
\pgfsetroundjoin%
\definecolor{currentfill}{rgb}{0.000000,0.000000,0.000000}%
\pgfsetfillcolor{currentfill}%
\pgfsetlinewidth{0.803000pt}%
\definecolor{currentstroke}{rgb}{0.000000,0.000000,0.000000}%
\pgfsetstrokecolor{currentstroke}%
\pgfsetdash{}{0pt}%
\pgfsys@defobject{currentmarker}{\pgfqpoint{0.000000in}{-0.048611in}}{\pgfqpoint{0.000000in}{0.000000in}}{%
\pgfpathmoveto{\pgfqpoint{0.000000in}{0.000000in}}%
\pgfpathlineto{\pgfqpoint{0.000000in}{-0.048611in}}%
\pgfusepath{stroke,fill}%
}%
\begin{pgfscope}%
\pgfsys@transformshift{5.775437in}{0.554012in}%
\pgfsys@useobject{currentmarker}{}%
\end{pgfscope}%
\end{pgfscope}%
\begin{pgfscope}%
\definecolor{textcolor}{rgb}{0.000000,0.000000,0.000000}%
\pgfsetstrokecolor{textcolor}%
\pgfsetfillcolor{textcolor}%
\pgftext[x=5.775437in,y=0.456790in,,top]{\color{textcolor}\rmfamily\fontsize{10.000000}{12.000000}\selectfont \(\displaystyle {0.05}\)}%
\end{pgfscope}%
\begin{pgfscope}%
\pgfsetbuttcap%
\pgfsetroundjoin%
\definecolor{currentfill}{rgb}{0.000000,0.000000,0.000000}%
\pgfsetfillcolor{currentfill}%
\pgfsetlinewidth{0.803000pt}%
\definecolor{currentstroke}{rgb}{0.000000,0.000000,0.000000}%
\pgfsetstrokecolor{currentstroke}%
\pgfsetdash{}{0pt}%
\pgfsys@defobject{currentmarker}{\pgfqpoint{0.000000in}{-0.048611in}}{\pgfqpoint{0.000000in}{0.000000in}}{%
\pgfpathmoveto{\pgfqpoint{0.000000in}{0.000000in}}%
\pgfpathlineto{\pgfqpoint{0.000000in}{-0.048611in}}%
\pgfusepath{stroke,fill}%
}%
\begin{pgfscope}%
\pgfsys@transformshift{6.865750in}{0.554012in}%
\pgfsys@useobject{currentmarker}{}%
\end{pgfscope}%
\end{pgfscope}%
\begin{pgfscope}%
\definecolor{textcolor}{rgb}{0.000000,0.000000,0.000000}%
\pgfsetstrokecolor{textcolor}%
\pgfsetfillcolor{textcolor}%
\pgftext[x=6.865750in,y=0.456790in,,top]{\color{textcolor}\rmfamily\fontsize{10.000000}{12.000000}\selectfont \(\displaystyle {0.06}\)}%
\end{pgfscope}%
\begin{pgfscope}%
\definecolor{textcolor}{rgb}{0.000000,0.000000,0.000000}%
\pgfsetstrokecolor{textcolor}%
\pgfsetfillcolor{textcolor}%
\pgftext[x=3.947223in,y=0.277777in,,top]{\color{textcolor}\rmfamily\fontsize{14.000000}{16.800000}\selectfont f1}%
\end{pgfscope}%
\begin{pgfscope}%
\pgfsetbuttcap%
\pgfsetroundjoin%
\definecolor{currentfill}{rgb}{0.000000,0.000000,0.000000}%
\pgfsetfillcolor{currentfill}%
\pgfsetlinewidth{0.803000pt}%
\definecolor{currentstroke}{rgb}{0.000000,0.000000,0.000000}%
\pgfsetstrokecolor{currentstroke}%
\pgfsetdash{}{0pt}%
\pgfsys@defobject{currentmarker}{\pgfqpoint{-0.048611in}{0.000000in}}{\pgfqpoint{-0.000000in}{0.000000in}}{%
\pgfpathmoveto{\pgfqpoint{-0.000000in}{0.000000in}}%
\pgfpathlineto{\pgfqpoint{-0.048611in}{0.000000in}}%
\pgfusepath{stroke,fill}%
}%
\begin{pgfscope}%
\pgfsys@transformshift{0.847223in}{0.961359in}%
\pgfsys@useobject{currentmarker}{}%
\end{pgfscope}%
\end{pgfscope}%
\begin{pgfscope}%
\definecolor{textcolor}{rgb}{0.000000,0.000000,0.000000}%
\pgfsetstrokecolor{textcolor}%
\pgfsetfillcolor{textcolor}%
\pgftext[x=0.402777in, y=0.913134in, left, base]{\color{textcolor}\rmfamily\fontsize{10.000000}{12.000000}\selectfont \(\displaystyle {20000}\)}%
\end{pgfscope}%
\begin{pgfscope}%
\pgfsetbuttcap%
\pgfsetroundjoin%
\definecolor{currentfill}{rgb}{0.000000,0.000000,0.000000}%
\pgfsetfillcolor{currentfill}%
\pgfsetlinewidth{0.803000pt}%
\definecolor{currentstroke}{rgb}{0.000000,0.000000,0.000000}%
\pgfsetstrokecolor{currentstroke}%
\pgfsetdash{}{0pt}%
\pgfsys@defobject{currentmarker}{\pgfqpoint{-0.048611in}{0.000000in}}{\pgfqpoint{-0.000000in}{0.000000in}}{%
\pgfpathmoveto{\pgfqpoint{-0.000000in}{0.000000in}}%
\pgfpathlineto{\pgfqpoint{-0.048611in}{0.000000in}}%
\pgfusepath{stroke,fill}%
}%
\begin{pgfscope}%
\pgfsys@transformshift{0.847223in}{1.785890in}%
\pgfsys@useobject{currentmarker}{}%
\end{pgfscope}%
\end{pgfscope}%
\begin{pgfscope}%
\definecolor{textcolor}{rgb}{0.000000,0.000000,0.000000}%
\pgfsetstrokecolor{textcolor}%
\pgfsetfillcolor{textcolor}%
\pgftext[x=0.402777in, y=1.737664in, left, base]{\color{textcolor}\rmfamily\fontsize{10.000000}{12.000000}\selectfont \(\displaystyle {40000}\)}%
\end{pgfscope}%
\begin{pgfscope}%
\pgfsetbuttcap%
\pgfsetroundjoin%
\definecolor{currentfill}{rgb}{0.000000,0.000000,0.000000}%
\pgfsetfillcolor{currentfill}%
\pgfsetlinewidth{0.803000pt}%
\definecolor{currentstroke}{rgb}{0.000000,0.000000,0.000000}%
\pgfsetstrokecolor{currentstroke}%
\pgfsetdash{}{0pt}%
\pgfsys@defobject{currentmarker}{\pgfqpoint{-0.048611in}{0.000000in}}{\pgfqpoint{-0.000000in}{0.000000in}}{%
\pgfpathmoveto{\pgfqpoint{-0.000000in}{0.000000in}}%
\pgfpathlineto{\pgfqpoint{-0.048611in}{0.000000in}}%
\pgfusepath{stroke,fill}%
}%
\begin{pgfscope}%
\pgfsys@transformshift{0.847223in}{2.610420in}%
\pgfsys@useobject{currentmarker}{}%
\end{pgfscope}%
\end{pgfscope}%
\begin{pgfscope}%
\definecolor{textcolor}{rgb}{0.000000,0.000000,0.000000}%
\pgfsetstrokecolor{textcolor}%
\pgfsetfillcolor{textcolor}%
\pgftext[x=0.402777in, y=2.562195in, left, base]{\color{textcolor}\rmfamily\fontsize{10.000000}{12.000000}\selectfont \(\displaystyle {60000}\)}%
\end{pgfscope}%
\begin{pgfscope}%
\pgfsetbuttcap%
\pgfsetroundjoin%
\definecolor{currentfill}{rgb}{0.000000,0.000000,0.000000}%
\pgfsetfillcolor{currentfill}%
\pgfsetlinewidth{0.803000pt}%
\definecolor{currentstroke}{rgb}{0.000000,0.000000,0.000000}%
\pgfsetstrokecolor{currentstroke}%
\pgfsetdash{}{0pt}%
\pgfsys@defobject{currentmarker}{\pgfqpoint{-0.048611in}{0.000000in}}{\pgfqpoint{-0.000000in}{0.000000in}}{%
\pgfpathmoveto{\pgfqpoint{-0.000000in}{0.000000in}}%
\pgfpathlineto{\pgfqpoint{-0.048611in}{0.000000in}}%
\pgfusepath{stroke,fill}%
}%
\begin{pgfscope}%
\pgfsys@transformshift{0.847223in}{3.434951in}%
\pgfsys@useobject{currentmarker}{}%
\end{pgfscope}%
\end{pgfscope}%
\begin{pgfscope}%
\definecolor{textcolor}{rgb}{0.000000,0.000000,0.000000}%
\pgfsetstrokecolor{textcolor}%
\pgfsetfillcolor{textcolor}%
\pgftext[x=0.402777in, y=3.386725in, left, base]{\color{textcolor}\rmfamily\fontsize{10.000000}{12.000000}\selectfont \(\displaystyle {80000}\)}%
\end{pgfscope}%
\begin{pgfscope}%
\pgfsetbuttcap%
\pgfsetroundjoin%
\definecolor{currentfill}{rgb}{0.000000,0.000000,0.000000}%
\pgfsetfillcolor{currentfill}%
\pgfsetlinewidth{0.803000pt}%
\definecolor{currentstroke}{rgb}{0.000000,0.000000,0.000000}%
\pgfsetstrokecolor{currentstroke}%
\pgfsetdash{}{0pt}%
\pgfsys@defobject{currentmarker}{\pgfqpoint{-0.048611in}{0.000000in}}{\pgfqpoint{-0.000000in}{0.000000in}}{%
\pgfpathmoveto{\pgfqpoint{-0.000000in}{0.000000in}}%
\pgfpathlineto{\pgfqpoint{-0.048611in}{0.000000in}}%
\pgfusepath{stroke,fill}%
}%
\begin{pgfscope}%
\pgfsys@transformshift{0.847223in}{4.259481in}%
\pgfsys@useobject{currentmarker}{}%
\end{pgfscope}%
\end{pgfscope}%
\begin{pgfscope}%
\definecolor{textcolor}{rgb}{0.000000,0.000000,0.000000}%
\pgfsetstrokecolor{textcolor}%
\pgfsetfillcolor{textcolor}%
\pgftext[x=0.333333in, y=4.211256in, left, base]{\color{textcolor}\rmfamily\fontsize{10.000000}{12.000000}\selectfont \(\displaystyle {100000}\)}%
\end{pgfscope}%
\begin{pgfscope}%
\pgfsetbuttcap%
\pgfsetroundjoin%
\definecolor{currentfill}{rgb}{0.000000,0.000000,0.000000}%
\pgfsetfillcolor{currentfill}%
\pgfsetlinewidth{0.803000pt}%
\definecolor{currentstroke}{rgb}{0.000000,0.000000,0.000000}%
\pgfsetstrokecolor{currentstroke}%
\pgfsetdash{}{0pt}%
\pgfsys@defobject{currentmarker}{\pgfqpoint{-0.048611in}{0.000000in}}{\pgfqpoint{-0.000000in}{0.000000in}}{%
\pgfpathmoveto{\pgfqpoint{-0.000000in}{0.000000in}}%
\pgfpathlineto{\pgfqpoint{-0.048611in}{0.000000in}}%
\pgfusepath{stroke,fill}%
}%
\begin{pgfscope}%
\pgfsys@transformshift{0.847223in}{5.084012in}%
\pgfsys@useobject{currentmarker}{}%
\end{pgfscope}%
\end{pgfscope}%
\begin{pgfscope}%
\definecolor{textcolor}{rgb}{0.000000,0.000000,0.000000}%
\pgfsetstrokecolor{textcolor}%
\pgfsetfillcolor{textcolor}%
\pgftext[x=0.333333in, y=5.035787in, left, base]{\color{textcolor}\rmfamily\fontsize{10.000000}{12.000000}\selectfont \(\displaystyle {120000}\)}%
\end{pgfscope}%
\begin{pgfscope}%
\definecolor{textcolor}{rgb}{0.000000,0.000000,0.000000}%
\pgfsetstrokecolor{textcolor}%
\pgfsetfillcolor{textcolor}%
\pgftext[x=0.277777in,y=2.819012in,,bottom,rotate=90.000000]{\color{textcolor}\rmfamily\fontsize{14.000000}{16.800000}\selectfont f2}%
\end{pgfscope}%
\begin{pgfscope}%
\pgfpathrectangle{\pgfqpoint{0.847223in}{0.554012in}}{\pgfqpoint{6.200000in}{4.530000in}}%
\pgfusepath{clip}%
\pgfsetrectcap%
\pgfsetroundjoin%
\pgfsetlinewidth{1.505625pt}%
\definecolor{currentstroke}{rgb}{0.827451,0.827451,0.827451}%
\pgfsetstrokecolor{currentstroke}%
\pgfsetstrokeopacity{0.500000}%
\pgfsetdash{}{0pt}%
\pgfpathmoveto{\pgfqpoint{0.847223in}{5.084012in}}%
\pgfpathlineto{\pgfqpoint{0.857889in}{4.985197in}}%
\pgfpathlineto{\pgfqpoint{0.868556in}{4.890252in}}%
\pgfpathlineto{\pgfqpoint{0.879222in}{4.798955in}}%
\pgfpathlineto{\pgfqpoint{0.889889in}{4.711098in}}%
\pgfpathlineto{\pgfqpoint{0.900555in}{4.626491in}}%
\pgfpathlineto{\pgfqpoint{0.911222in}{4.544958in}}%
\pgfpathlineto{\pgfqpoint{0.921888in}{4.466333in}}%
\pgfpathlineto{\pgfqpoint{0.932555in}{4.390463in}}%
\pgfpathlineto{\pgfqpoint{0.943221in}{4.317207in}}%
\pgfpathlineto{\pgfqpoint{0.953887in}{4.246431in}}%
\pgfpathlineto{\pgfqpoint{0.964554in}{4.178012in}}%
\pgfpathlineto{\pgfqpoint{0.975220in}{4.111834in}}%
\pgfpathlineto{\pgfqpoint{0.985887in}{4.047788in}}%
\pgfpathlineto{\pgfqpoint{0.996553in}{3.985774in}}%
\pgfpathlineto{\pgfqpoint{1.007220in}{3.925695in}}%
\pgfpathlineto{\pgfqpoint{1.017886in}{3.867463in}}%
\pgfpathlineto{\pgfqpoint{1.028552in}{3.810994in}}%
\pgfpathlineto{\pgfqpoint{1.039219in}{3.756209in}}%
\pgfpathlineto{\pgfqpoint{1.049885in}{3.703034in}}%
\pgfpathlineto{\pgfqpoint{1.060552in}{3.651399in}}%
\pgfpathlineto{\pgfqpoint{1.071218in}{3.601237in}}%
\pgfpathlineto{\pgfqpoint{1.081885in}{3.552487in}}%
\pgfpathlineto{\pgfqpoint{1.092551in}{3.505091in}}%
\pgfpathlineto{\pgfqpoint{1.103217in}{3.458991in}}%
\pgfpathlineto{\pgfqpoint{1.113884in}{3.414137in}}%
\pgfpathlineto{\pgfqpoint{1.124550in}{3.370477in}}%
\pgfpathlineto{\pgfqpoint{1.135217in}{3.327965in}}%
\pgfpathlineto{\pgfqpoint{1.145883in}{3.286557in}}%
\pgfpathlineto{\pgfqpoint{1.156550in}{3.246210in}}%
\pgfpathlineto{\pgfqpoint{1.167216in}{3.206883in}}%
\pgfpathlineto{\pgfqpoint{1.177882in}{3.168538in}}%
\pgfpathlineto{\pgfqpoint{1.188549in}{3.131140in}}%
\pgfpathlineto{\pgfqpoint{1.199215in}{3.094653in}}%
\pgfpathlineto{\pgfqpoint{1.215215in}{3.041560in}}%
\pgfpathlineto{\pgfqpoint{1.231215in}{2.990339in}}%
\pgfpathlineto{\pgfqpoint{1.247214in}{2.940894in}}%
\pgfpathlineto{\pgfqpoint{1.263214in}{2.893132in}}%
\pgfpathlineto{\pgfqpoint{1.279214in}{2.846971in}}%
\pgfpathlineto{\pgfqpoint{1.295213in}{2.802330in}}%
\pgfpathlineto{\pgfqpoint{1.311213in}{2.759136in}}%
\pgfpathlineto{\pgfqpoint{1.327212in}{2.717320in}}%
\pgfpathlineto{\pgfqpoint{1.343212in}{2.676816in}}%
\pgfpathlineto{\pgfqpoint{1.359212in}{2.637565in}}%
\pgfpathlineto{\pgfqpoint{1.375211in}{2.599507in}}%
\pgfpathlineto{\pgfqpoint{1.391211in}{2.562591in}}%
\pgfpathlineto{\pgfqpoint{1.407211in}{2.526766in}}%
\pgfpathlineto{\pgfqpoint{1.423210in}{2.491983in}}%
\pgfpathlineto{\pgfqpoint{1.439210in}{2.458198in}}%
\pgfpathlineto{\pgfqpoint{1.455210in}{2.425368in}}%
\pgfpathlineto{\pgfqpoint{1.471209in}{2.393455in}}%
\pgfpathlineto{\pgfqpoint{1.487209in}{2.362419in}}%
\pgfpathlineto{\pgfqpoint{1.503209in}{2.332225in}}%
\pgfpathlineto{\pgfqpoint{1.519208in}{2.302839in}}%
\pgfpathlineto{\pgfqpoint{1.535208in}{2.274230in}}%
\pgfpathlineto{\pgfqpoint{1.551208in}{2.246367in}}%
\pgfpathlineto{\pgfqpoint{1.567207in}{2.219220in}}%
\pgfpathlineto{\pgfqpoint{1.583207in}{2.192764in}}%
\pgfpathlineto{\pgfqpoint{1.599206in}{2.166971in}}%
\pgfpathlineto{\pgfqpoint{1.615206in}{2.141818in}}%
\pgfpathlineto{\pgfqpoint{1.631206in}{2.117280in}}%
\pgfpathlineto{\pgfqpoint{1.647205in}{2.093335in}}%
\pgfpathlineto{\pgfqpoint{1.663205in}{2.069963in}}%
\pgfpathlineto{\pgfqpoint{1.679205in}{2.047142in}}%
\pgfpathlineto{\pgfqpoint{1.695204in}{2.024854in}}%
\pgfpathlineto{\pgfqpoint{1.711204in}{2.003080in}}%
\pgfpathlineto{\pgfqpoint{1.727204in}{1.981803in}}%
\pgfpathlineto{\pgfqpoint{1.743203in}{1.961005in}}%
\pgfpathlineto{\pgfqpoint{1.759203in}{1.940671in}}%
\pgfpathlineto{\pgfqpoint{1.780536in}{1.914254in}}%
\pgfpathlineto{\pgfqpoint{1.801869in}{1.888599in}}%
\pgfpathlineto{\pgfqpoint{1.823202in}{1.863674in}}%
\pgfpathlineto{\pgfqpoint{1.844534in}{1.839449in}}%
\pgfpathlineto{\pgfqpoint{1.865867in}{1.815894in}}%
\pgfpathlineto{\pgfqpoint{1.887200in}{1.792982in}}%
\pgfpathlineto{\pgfqpoint{1.908533in}{1.770686in}}%
\pgfpathlineto{\pgfqpoint{1.929866in}{1.748983in}}%
\pgfpathlineto{\pgfqpoint{1.951199in}{1.727849in}}%
\pgfpathlineto{\pgfqpoint{1.972532in}{1.707262in}}%
\pgfpathlineto{\pgfqpoint{1.993864in}{1.687201in}}%
\pgfpathlineto{\pgfqpoint{2.015197in}{1.667646in}}%
\pgfpathlineto{\pgfqpoint{2.036530in}{1.648578in}}%
\pgfpathlineto{\pgfqpoint{2.057863in}{1.629980in}}%
\pgfpathlineto{\pgfqpoint{2.079196in}{1.611833in}}%
\pgfpathlineto{\pgfqpoint{2.100529in}{1.594122in}}%
\pgfpathlineto{\pgfqpoint{2.121862in}{1.576832in}}%
\pgfpathlineto{\pgfqpoint{2.143194in}{1.559947in}}%
\pgfpathlineto{\pgfqpoint{2.164527in}{1.543453in}}%
\pgfpathlineto{\pgfqpoint{2.191193in}{1.523366in}}%
\pgfpathlineto{\pgfqpoint{2.217859in}{1.503844in}}%
\pgfpathlineto{\pgfqpoint{2.244526in}{1.484865in}}%
\pgfpathlineto{\pgfqpoint{2.271192in}{1.466405in}}%
\pgfpathlineto{\pgfqpoint{2.297858in}{1.448444in}}%
\pgfpathlineto{\pgfqpoint{2.324524in}{1.430962in}}%
\pgfpathlineto{\pgfqpoint{2.351190in}{1.413940in}}%
\pgfpathlineto{\pgfqpoint{2.377856in}{1.397360in}}%
\pgfpathlineto{\pgfqpoint{2.404522in}{1.381204in}}%
\pgfpathlineto{\pgfqpoint{2.431188in}{1.365458in}}%
\pgfpathlineto{\pgfqpoint{2.457854in}{1.350105in}}%
\pgfpathlineto{\pgfqpoint{2.484520in}{1.335131in}}%
\pgfpathlineto{\pgfqpoint{2.511186in}{1.320522in}}%
\pgfpathlineto{\pgfqpoint{2.543186in}{1.303455in}}%
\pgfpathlineto{\pgfqpoint{2.575185in}{1.286873in}}%
\pgfpathlineto{\pgfqpoint{2.607184in}{1.270756in}}%
\pgfpathlineto{\pgfqpoint{2.639184in}{1.255084in}}%
\pgfpathlineto{\pgfqpoint{2.671183in}{1.239840in}}%
\pgfpathlineto{\pgfqpoint{2.703182in}{1.225005in}}%
\pgfpathlineto{\pgfqpoint{2.735181in}{1.210565in}}%
\pgfpathlineto{\pgfqpoint{2.767181in}{1.196502in}}%
\pgfpathlineto{\pgfqpoint{2.799180in}{1.182804in}}%
\pgfpathlineto{\pgfqpoint{2.836512in}{1.167263in}}%
\pgfpathlineto{\pgfqpoint{2.873845in}{1.152177in}}%
\pgfpathlineto{\pgfqpoint{2.911178in}{1.137526in}}%
\pgfpathlineto{\pgfqpoint{2.948510in}{1.123292in}}%
\pgfpathlineto{\pgfqpoint{2.985843in}{1.109458in}}%
\pgfpathlineto{\pgfqpoint{3.023175in}{1.096006in}}%
\pgfpathlineto{\pgfqpoint{3.060508in}{1.082921in}}%
\pgfpathlineto{\pgfqpoint{3.097840in}{1.070188in}}%
\pgfpathlineto{\pgfqpoint{3.140506in}{1.056050in}}%
\pgfpathlineto{\pgfqpoint{3.183172in}{1.042334in}}%
\pgfpathlineto{\pgfqpoint{3.225837in}{1.029021in}}%
\pgfpathlineto{\pgfqpoint{3.268503in}{1.016093in}}%
\pgfpathlineto{\pgfqpoint{3.311169in}{1.003535in}}%
\pgfpathlineto{\pgfqpoint{3.353834in}{0.991331in}}%
\pgfpathlineto{\pgfqpoint{3.401833in}{0.978006in}}%
\pgfpathlineto{\pgfqpoint{3.449832in}{0.965089in}}%
\pgfpathlineto{\pgfqpoint{3.497831in}{0.952564in}}%
\pgfpathlineto{\pgfqpoint{3.545830in}{0.940411in}}%
\pgfpathlineto{\pgfqpoint{3.593829in}{0.928616in}}%
\pgfpathlineto{\pgfqpoint{3.647161in}{0.915909in}}%
\pgfpathlineto{\pgfqpoint{3.700493in}{0.903604in}}%
\pgfpathlineto{\pgfqpoint{3.753826in}{0.891681in}}%
\pgfpathlineto{\pgfqpoint{3.807158in}{0.880124in}}%
\pgfpathlineto{\pgfqpoint{3.865823in}{0.867813in}}%
\pgfpathlineto{\pgfqpoint{3.924488in}{0.855903in}}%
\pgfpathlineto{\pgfqpoint{3.983154in}{0.844375in}}%
\pgfpathlineto{\pgfqpoint{4.041819in}{0.833210in}}%
\pgfpathlineto{\pgfqpoint{4.105818in}{0.821426in}}%
\pgfpathlineto{\pgfqpoint{4.169816in}{0.810034in}}%
\pgfpathlineto{\pgfqpoint{4.233815in}{0.799015in}}%
\pgfpathlineto{\pgfqpoint{4.297814in}{0.788351in}}%
\pgfpathlineto{\pgfqpoint{4.367145in}{0.777179in}}%
\pgfpathlineto{\pgfqpoint{4.436477in}{0.766383in}}%
\pgfpathlineto{\pgfqpoint{4.505809in}{0.755946in}}%
\pgfpathlineto{\pgfqpoint{4.580474in}{0.745086in}}%
\pgfpathlineto{\pgfqpoint{4.655139in}{0.734601in}}%
\pgfpathlineto{\pgfqpoint{4.735137in}{0.723760in}}%
\pgfpathlineto{\pgfqpoint{4.815136in}{0.713306in}}%
\pgfpathlineto{\pgfqpoint{4.895134in}{0.703217in}}%
\pgfpathlineto{\pgfqpoint{4.980465in}{0.692838in}}%
\pgfpathlineto{\pgfqpoint{5.065797in}{0.682833in}}%
\pgfpathlineto{\pgfqpoint{5.156461in}{0.672589in}}%
\pgfpathlineto{\pgfqpoint{5.247126in}{0.662723in}}%
\pgfpathlineto{\pgfqpoint{5.343124in}{0.652664in}}%
\pgfpathlineto{\pgfqpoint{5.439122in}{0.642984in}}%
\pgfpathlineto{\pgfqpoint{5.540453in}{0.633152in}}%
\pgfpathlineto{\pgfqpoint{5.641784in}{0.623695in}}%
\pgfpathlineto{\pgfqpoint{5.748448in}{0.614121in}}%
\pgfpathlineto{\pgfqpoint{5.855113in}{0.604917in}}%
\pgfpathlineto{\pgfqpoint{5.967110in}{0.595627in}}%
\pgfpathlineto{\pgfqpoint{6.084441in}{0.586282in}}%
\pgfpathlineto{\pgfqpoint{6.183915in}{0.578665in}}%
\pgfpathlineto{\pgfqpoint{6.219152in}{0.576287in}}%
\pgfpathlineto{\pgfqpoint{6.263198in}{0.573734in}}%
\pgfpathlineto{\pgfqpoint{6.316054in}{0.571128in}}%
\pgfpathlineto{\pgfqpoint{6.368909in}{0.568896in}}%
\pgfpathlineto{\pgfqpoint{6.430574in}{0.566648in}}%
\pgfpathlineto{\pgfqpoint{6.501048in}{0.564439in}}%
\pgfpathlineto{\pgfqpoint{6.580332in}{0.562305in}}%
\pgfpathlineto{\pgfqpoint{6.677234in}{0.560079in}}%
\pgfpathlineto{\pgfqpoint{6.782945in}{0.558018in}}%
\pgfpathlineto{\pgfqpoint{6.906275in}{0.555978in}}%
\pgfpathlineto{\pgfqpoint{7.047223in}{0.554012in}}%
\pgfpathlineto{\pgfqpoint{7.047223in}{0.554012in}}%
\pgfusepath{stroke}%
\end{pgfscope}%
\begin{pgfscope}%
\pgfpathrectangle{\pgfqpoint{0.847223in}{0.554012in}}{\pgfqpoint{6.200000in}{4.530000in}}%
\pgfusepath{clip}%
\pgfsetrectcap%
\pgfsetroundjoin%
\pgfsetlinewidth{2.509375pt}%
\definecolor{currentstroke}{rgb}{0.000000,0.000000,0.000000}%
\pgfsetstrokecolor{currentstroke}%
\pgfsetdash{}{0pt}%
\pgfpathmoveto{\pgfqpoint{0.845556in}{3.583395in}}%
\pgfpathlineto{\pgfqpoint{0.853329in}{3.532750in}}%
\pgfpathlineto{\pgfqpoint{0.862218in}{3.476680in}}%
\pgfpathlineto{\pgfqpoint{0.871107in}{3.422431in}}%
\pgfpathlineto{\pgfqpoint{0.879995in}{3.369916in}}%
\pgfpathlineto{\pgfqpoint{0.888884in}{3.319053in}}%
\pgfpathlineto{\pgfqpoint{0.897773in}{3.269766in}}%
\pgfpathlineto{\pgfqpoint{0.906661in}{3.221983in}}%
\pgfpathlineto{\pgfqpoint{0.915550in}{3.175635in}}%
\pgfpathlineto{\pgfqpoint{0.924439in}{3.130659in}}%
\pgfpathlineto{\pgfqpoint{0.933328in}{3.086995in}}%
\pgfpathlineto{\pgfqpoint{0.942216in}{3.044586in}}%
\pgfpathlineto{\pgfqpoint{0.955549in}{2.983211in}}%
\pgfpathlineto{\pgfqpoint{0.968882in}{2.924373in}}%
\pgfpathlineto{\pgfqpoint{0.982215in}{2.867919in}}%
\pgfpathlineto{\pgfqpoint{0.995548in}{2.813705in}}%
\pgfpathlineto{\pgfqpoint{1.008881in}{2.761602in}}%
\pgfpathlineto{\pgfqpoint{1.022214in}{2.711489in}}%
\pgfpathlineto{\pgfqpoint{1.035548in}{2.663254in}}%
\pgfpathlineto{\pgfqpoint{1.048881in}{2.616792in}}%
\pgfpathlineto{\pgfqpoint{1.062214in}{2.572009in}}%
\pgfpathlineto{\pgfqpoint{1.075547in}{2.528814in}}%
\pgfpathlineto{\pgfqpoint{1.088880in}{2.487125in}}%
\pgfpathlineto{\pgfqpoint{1.102213in}{2.446864in}}%
\pgfpathlineto{\pgfqpoint{1.115546in}{2.407959in}}%
\pgfpathlineto{\pgfqpoint{1.128879in}{2.370343in}}%
\pgfpathlineto{\pgfqpoint{1.142212in}{2.333953in}}%
\pgfpathlineto{\pgfqpoint{1.155545in}{2.298730in}}%
\pgfpathlineto{\pgfqpoint{1.168878in}{2.264618in}}%
\pgfpathlineto{\pgfqpoint{1.182211in}{2.231566in}}%
\pgfpathlineto{\pgfqpoint{1.195544in}{2.199525in}}%
\pgfpathlineto{\pgfqpoint{1.208877in}{2.168449in}}%
\pgfpathlineto{\pgfqpoint{1.222210in}{2.138296in}}%
\pgfpathlineto{\pgfqpoint{1.235543in}{2.109025in}}%
\pgfpathlineto{\pgfqpoint{1.248876in}{2.080597in}}%
\pgfpathlineto{\pgfqpoint{1.262209in}{2.052978in}}%
\pgfpathlineto{\pgfqpoint{1.275542in}{2.026132in}}%
\pgfpathlineto{\pgfqpoint{1.288875in}{2.000029in}}%
\pgfpathlineto{\pgfqpoint{1.302208in}{1.974636in}}%
\pgfpathlineto{\pgfqpoint{1.315541in}{1.949927in}}%
\pgfpathlineto{\pgfqpoint{1.328874in}{1.925873in}}%
\pgfpathlineto{\pgfqpoint{1.342207in}{1.902449in}}%
\pgfpathlineto{\pgfqpoint{1.355540in}{1.879631in}}%
\pgfpathlineto{\pgfqpoint{1.368873in}{1.857395in}}%
\pgfpathlineto{\pgfqpoint{1.382207in}{1.835719in}}%
\pgfpathlineto{\pgfqpoint{1.395540in}{1.814582in}}%
\pgfpathlineto{\pgfqpoint{1.413317in}{1.787205in}}%
\pgfpathlineto{\pgfqpoint{1.431094in}{1.760707in}}%
\pgfpathlineto{\pgfqpoint{1.448872in}{1.735046in}}%
\pgfpathlineto{\pgfqpoint{1.466649in}{1.710183in}}%
\pgfpathlineto{\pgfqpoint{1.484426in}{1.686083in}}%
\pgfpathlineto{\pgfqpoint{1.502204in}{1.662709in}}%
\pgfpathlineto{\pgfqpoint{1.519981in}{1.640031in}}%
\pgfpathlineto{\pgfqpoint{1.537759in}{1.618016in}}%
\pgfpathlineto{\pgfqpoint{1.555536in}{1.596637in}}%
\pgfpathlineto{\pgfqpoint{1.573313in}{1.575867in}}%
\pgfpathlineto{\pgfqpoint{1.591091in}{1.555679in}}%
\pgfpathlineto{\pgfqpoint{1.608868in}{1.536050in}}%
\pgfpathlineto{\pgfqpoint{1.626646in}{1.516956in}}%
\pgfpathlineto{\pgfqpoint{1.644423in}{1.498377in}}%
\pgfpathlineto{\pgfqpoint{1.662200in}{1.480291in}}%
\pgfpathlineto{\pgfqpoint{1.679978in}{1.462679in}}%
\pgfpathlineto{\pgfqpoint{1.697755in}{1.445523in}}%
\pgfpathlineto{\pgfqpoint{1.715533in}{1.428806in}}%
\pgfpathlineto{\pgfqpoint{1.733310in}{1.412510in}}%
\pgfpathlineto{\pgfqpoint{1.755532in}{1.392709in}}%
\pgfpathlineto{\pgfqpoint{1.777753in}{1.373514in}}%
\pgfpathlineto{\pgfqpoint{1.799975in}{1.354896in}}%
\pgfpathlineto{\pgfqpoint{1.822197in}{1.336831in}}%
\pgfpathlineto{\pgfqpoint{1.844419in}{1.319294in}}%
\pgfpathlineto{\pgfqpoint{1.866640in}{1.302262in}}%
\pgfpathlineto{\pgfqpoint{1.888862in}{1.285714in}}%
\pgfpathlineto{\pgfqpoint{1.911084in}{1.269629in}}%
\pgfpathlineto{\pgfqpoint{1.933305in}{1.253988in}}%
\pgfpathlineto{\pgfqpoint{1.955527in}{1.238773in}}%
\pgfpathlineto{\pgfqpoint{1.977749in}{1.223967in}}%
\pgfpathlineto{\pgfqpoint{1.999971in}{1.209554in}}%
\pgfpathlineto{\pgfqpoint{2.022192in}{1.195518in}}%
\pgfpathlineto{\pgfqpoint{2.048858in}{1.179152in}}%
\pgfpathlineto{\pgfqpoint{2.075525in}{1.163284in}}%
\pgfpathlineto{\pgfqpoint{2.102191in}{1.147892in}}%
\pgfpathlineto{\pgfqpoint{2.128857in}{1.132955in}}%
\pgfpathlineto{\pgfqpoint{2.155523in}{1.118453in}}%
\pgfpathlineto{\pgfqpoint{2.182189in}{1.104367in}}%
\pgfpathlineto{\pgfqpoint{2.208855in}{1.090680in}}%
\pgfpathlineto{\pgfqpoint{2.235521in}{1.077374in}}%
\pgfpathlineto{\pgfqpoint{2.262187in}{1.064435in}}%
\pgfpathlineto{\pgfqpoint{2.293298in}{1.049782in}}%
\pgfpathlineto{\pgfqpoint{2.324408in}{1.035584in}}%
\pgfpathlineto{\pgfqpoint{2.355518in}{1.021822in}}%
\pgfpathlineto{\pgfqpoint{2.386629in}{1.008474in}}%
\pgfpathlineto{\pgfqpoint{2.417739in}{0.995524in}}%
\pgfpathlineto{\pgfqpoint{2.448850in}{0.982952in}}%
\pgfpathlineto{\pgfqpoint{2.479960in}{0.970743in}}%
\pgfpathlineto{\pgfqpoint{2.515515in}{0.957215in}}%
\pgfpathlineto{\pgfqpoint{2.551070in}{0.944118in}}%
\pgfpathlineto{\pgfqpoint{2.586624in}{0.931433in}}%
\pgfpathlineto{\pgfqpoint{2.622179in}{0.919141in}}%
\pgfpathlineto{\pgfqpoint{2.657734in}{0.907223in}}%
\pgfpathlineto{\pgfqpoint{2.693289in}{0.895662in}}%
\pgfpathlineto{\pgfqpoint{2.733288in}{0.883065in}}%
\pgfpathlineto{\pgfqpoint{2.773287in}{0.870879in}}%
\pgfpathlineto{\pgfqpoint{2.813286in}{0.859084in}}%
\pgfpathlineto{\pgfqpoint{2.853285in}{0.847663in}}%
\pgfpathlineto{\pgfqpoint{2.893284in}{0.836597in}}%
\pgfpathlineto{\pgfqpoint{2.937728in}{0.824699in}}%
\pgfpathlineto{\pgfqpoint{2.982171in}{0.813198in}}%
\pgfpathlineto{\pgfqpoint{3.026615in}{0.802076in}}%
\pgfpathlineto{\pgfqpoint{3.071058in}{0.791314in}}%
\pgfpathlineto{\pgfqpoint{3.119946in}{0.779871in}}%
\pgfpathlineto{\pgfqpoint{3.168834in}{0.768821in}}%
\pgfpathlineto{\pgfqpoint{3.217722in}{0.758144in}}%
\pgfpathlineto{\pgfqpoint{3.266609in}{0.747822in}}%
\pgfpathlineto{\pgfqpoint{3.319942in}{0.736946in}}%
\pgfpathlineto{\pgfqpoint{3.373274in}{0.726450in}}%
\pgfpathlineto{\pgfqpoint{3.426606in}{0.716315in}}%
\pgfpathlineto{\pgfqpoint{3.484382in}{0.705722in}}%
\pgfpathlineto{\pgfqpoint{3.542159in}{0.695509in}}%
\pgfpathlineto{\pgfqpoint{3.599935in}{0.685656in}}%
\pgfpathlineto{\pgfqpoint{3.662156in}{0.675427in}}%
\pgfpathlineto{\pgfqpoint{3.724377in}{0.665572in}}%
\pgfpathlineto{\pgfqpoint{3.791042in}{0.655405in}}%
\pgfpathlineto{\pgfqpoint{3.857707in}{0.645622in}}%
\pgfpathlineto{\pgfqpoint{3.924373in}{0.636202in}}%
\pgfpathlineto{\pgfqpoint{3.995482in}{0.626530in}}%
\pgfpathlineto{\pgfqpoint{4.066592in}{0.617226in}}%
\pgfpathlineto{\pgfqpoint{4.142146in}{0.607720in}}%
\pgfpathlineto{\pgfqpoint{4.217700in}{0.598583in}}%
\pgfpathlineto{\pgfqpoint{4.297698in}{0.589288in}}%
\pgfpathlineto{\pgfqpoint{4.377696in}{0.580359in}}%
\pgfpathlineto{\pgfqpoint{4.462139in}{0.571308in}}%
\pgfpathlineto{\pgfqpoint{4.546581in}{0.562620in}}%
\pgfpathlineto{\pgfqpoint{4.635468in}{0.553842in}}%
\pgfpathlineto{\pgfqpoint{4.651001in}{0.552345in}}%
\pgfpathlineto{\pgfqpoint{4.651001in}{0.552345in}}%
\pgfusepath{stroke}%
\end{pgfscope}%
\begin{pgfscope}%
\pgfsetrectcap%
\pgfsetmiterjoin%
\pgfsetlinewidth{0.803000pt}%
\definecolor{currentstroke}{rgb}{0.000000,0.000000,0.000000}%
\pgfsetstrokecolor{currentstroke}%
\pgfsetdash{}{0pt}%
\pgfpathmoveto{\pgfqpoint{0.847223in}{0.554012in}}%
\pgfpathlineto{\pgfqpoint{0.847223in}{5.084012in}}%
\pgfusepath{stroke}%
\end{pgfscope}%
\begin{pgfscope}%
\pgfsetrectcap%
\pgfsetmiterjoin%
\pgfsetlinewidth{0.803000pt}%
\definecolor{currentstroke}{rgb}{0.000000,0.000000,0.000000}%
\pgfsetstrokecolor{currentstroke}%
\pgfsetdash{}{0pt}%
\pgfpathmoveto{\pgfqpoint{7.047223in}{0.554012in}}%
\pgfpathlineto{\pgfqpoint{7.047223in}{5.084012in}}%
\pgfusepath{stroke}%
\end{pgfscope}%
\begin{pgfscope}%
\pgfsetrectcap%
\pgfsetmiterjoin%
\pgfsetlinewidth{0.803000pt}%
\definecolor{currentstroke}{rgb}{0.000000,0.000000,0.000000}%
\pgfsetstrokecolor{currentstroke}%
\pgfsetdash{}{0pt}%
\pgfpathmoveto{\pgfqpoint{0.847223in}{0.554012in}}%
\pgfpathlineto{\pgfqpoint{7.047223in}{0.554012in}}%
\pgfusepath{stroke}%
\end{pgfscope}%
\begin{pgfscope}%
\pgfsetrectcap%
\pgfsetmiterjoin%
\pgfsetlinewidth{0.803000pt}%
\definecolor{currentstroke}{rgb}{0.000000,0.000000,0.000000}%
\pgfsetstrokecolor{currentstroke}%
\pgfsetdash{}{0pt}%
\pgfpathmoveto{\pgfqpoint{0.847223in}{5.084012in}}%
\pgfpathlineto{\pgfqpoint{7.047223in}{5.084012in}}%
\pgfusepath{stroke}%
\end{pgfscope}%
\begin{pgfscope}%
\pgfsetbuttcap%
\pgfsetmiterjoin%
\definecolor{currentfill}{rgb}{1.000000,1.000000,1.000000}%
\pgfsetfillcolor{currentfill}%
\pgfsetfillopacity{0.800000}%
\pgfsetlinewidth{1.003750pt}%
\definecolor{currentstroke}{rgb}{0.800000,0.800000,0.800000}%
\pgfsetstrokecolor{currentstroke}%
\pgfsetstrokeopacity{0.800000}%
\pgfsetdash{}{0pt}%
\pgfpathmoveto{\pgfqpoint{4.639628in}{4.378457in}}%
\pgfpathlineto{\pgfqpoint{6.911112in}{4.378457in}}%
\pgfpathquadraticcurveto{\pgfqpoint{6.950001in}{4.378457in}}{\pgfqpoint{6.950001in}{4.417346in}}%
\pgfpathlineto{\pgfqpoint{6.950001in}{4.947901in}}%
\pgfpathquadraticcurveto{\pgfqpoint{6.950001in}{4.986790in}}{\pgfqpoint{6.911112in}{4.986790in}}%
\pgfpathlineto{\pgfqpoint{4.639628in}{4.986790in}}%
\pgfpathquadraticcurveto{\pgfqpoint{4.600739in}{4.986790in}}{\pgfqpoint{4.600739in}{4.947901in}}%
\pgfpathlineto{\pgfqpoint{4.600739in}{4.417346in}}%
\pgfpathquadraticcurveto{\pgfqpoint{4.600739in}{4.378457in}}{\pgfqpoint{4.639628in}{4.378457in}}%
\pgfpathlineto{\pgfqpoint{4.639628in}{4.378457in}}%
\pgfpathclose%
\pgfusepath{stroke,fill}%
\end{pgfscope}%
\begin{pgfscope}%
\pgfsetrectcap%
\pgfsetroundjoin%
\pgfsetlinewidth{2.509375pt}%
\definecolor{currentstroke}{rgb}{0.000000,0.000000,0.000000}%
\pgfsetstrokecolor{currentstroke}%
\pgfsetdash{}{0pt}%
\pgfpathmoveto{\pgfqpoint{4.678517in}{4.838179in}}%
\pgfpathlineto{\pgfqpoint{4.872961in}{4.838179in}}%
\pgfpathlineto{\pgfqpoint{5.067406in}{4.838179in}}%
\pgfusepath{stroke}%
\end{pgfscope}%
\begin{pgfscope}%
\definecolor{textcolor}{rgb}{0.000000,0.000000,0.000000}%
\pgfsetstrokecolor{textcolor}%
\pgfsetfillcolor{textcolor}%
\pgftext[x=5.222961in,y=4.770123in,left,base]{\color{textcolor}\rmfamily\fontsize{14.000000}{16.800000}\selectfont Pareto-front}%
\end{pgfscope}%
\begin{pgfscope}%
\pgfsetbuttcap%
\pgfsetmiterjoin%
\definecolor{currentfill}{rgb}{0.827451,0.827451,0.827451}%
\pgfsetfillcolor{currentfill}%
\pgfsetfillopacity{0.500000}%
\pgfsetlinewidth{0.000000pt}%
\definecolor{currentstroke}{rgb}{0.000000,0.000000,0.000000}%
\pgfsetstrokecolor{currentstroke}%
\pgfsetstrokeopacity{0.500000}%
\pgfsetdash{}{0pt}%
\pgfpathmoveto{\pgfqpoint{4.678517in}{4.495124in}}%
\pgfpathlineto{\pgfqpoint{5.067406in}{4.495124in}}%
\pgfpathlineto{\pgfqpoint{5.067406in}{4.631235in}}%
\pgfpathlineto{\pgfqpoint{4.678517in}{4.631235in}}%
\pgfpathlineto{\pgfqpoint{4.678517in}{4.495124in}}%
\pgfpathclose%
\pgfusepath{fill}%
\end{pgfscope}%
\begin{pgfscope}%
\definecolor{textcolor}{rgb}{0.000000,0.000000,0.000000}%
\pgfsetstrokecolor{textcolor}%
\pgfsetfillcolor{textcolor}%
\pgftext[x=5.222961in,y=4.495124in,left,base]{\color{textcolor}\rmfamily\fontsize{14.000000}{16.800000}\selectfont Near-optimal space}%
\end{pgfscope}%
\end{pgfpicture}%
\makeatother%
\endgroup%
}
  \caption{The near-optimal space around the Pareto-front.}
  \label{fig:near-opt-pareto}
\end{figure}

For these reasons, this thesis explores energy systems optimization and the
handling of structural uncertainty through \ac{moo} and \acp{ga}. Section
\ref{section:genetic-algorithms} reviews the details of the \ac{ga} used in this
thesis.


% \subsection{Working title: \Ac{moo} in other fields}





\subsection{Energy System Applications}
It is well understood that engineering and policy problems, which include energy
systems optimization, often require satisfying multiple antagonistic objectives
\cite{loughlin_genetic_2001,zechman_evolutionary_2004,
zechman_evolutionary_2013, chattopadhyay_need_2021}. However, the application of
\ac{moo} to energy systems in the literature is limited. Table
\ref{tab:moop-literature} summarizes the current body of work. As before, the
``public code'' column only indicates if the source code is accessible.
Additionally, the ``sector'' columns only indicate the presence of a feature,
not the relative maturity or sophistication of the modeling. There are six
``objective columns,'' indicating which objectives are considered the in the
model or study. A ``technology'' objective might optimize a specific technology
or set of technologies. For example, maximizing the percentage of renewable
energy in a system. The ``reliability'' metric varies among studies, but
generally refers to the potential for load loss. For all of the studies in Table
\ref{tab:moop-literature}, the ``environmental'' objective refers to \ac{ghg} or
``global warming potential'' \cite{de-leon_almaraz_deployment_2015}. Although it
could refer to other environmental impacts such as land use, water use, or
thermal pollution. 

\begin{table}[ht!]
    \centering
    \caption{\ac{moo} used with energy systems.}
    \label{tab:moop-literature}
    \resizebox*{\textwidth}{!}{\begin{tabular}{lll*{6}{c}|*{4}{c}}
\toprule
Citation & Model & Algorithm & \multicolumn{6}{c}{Objectives} & \multicolumn{3}{c}{Sector} &  \\
  &    &    & Economic & Social & Environment & Reliability & Technology & User-defined & Heat & Electricity & Transport &  Public Code  \\
\midrule
\cite{riou_multi-objective_2021}  &    &  \acs{nsga2}  & \checkmark &  &  & \checkmark & \checkmark &  &  & \checkmark &  &   \\
\cite{laha_low_2021}  &    &  \acs{nsga2}  & \checkmark &  &  &  & \checkmark &  &  & \checkmark &  &   \\
\cite{mayer_environmental_2020}  &    &  \acs{nsga2}  & \checkmark &  & \checkmark &  &  &  & \checkmark & \checkmark &  &   \\
\cite{prina_multi-objective_2020}  &  Oemof-moea  &  \acs{nsga2}  & \checkmark &  & \checkmark &  &  &  & \checkmark & \checkmark & \checkmark &  \checkmark  \\
\cite{donado_hyres_2020}  &  HYRES  &  GAToolbox  & \checkmark &  &  & \checkmark &  &  &  & \checkmark &  &   \\
\cite{prina_multi-objective_2018}&    &  \acs{nsga2}  & \checkmark &  & \checkmark &  &  &  & \checkmark & \checkmark &  &   \\
\cite{samsatli_multi-objective_2018}& &  \acs{ws}  & \checkmark &  & \checkmark &  &  &  & \checkmark & \checkmark & \checkmark &   \\
\cite{falke_multi-objective_2016}&    &  \acs{nsga2}  & \checkmark &  & \checkmark &  &  &  & \checkmark & \checkmark &  &   \\
\cite{mahbub_combining_2016}&    &  \acs{nsga2}  & \checkmark &  & \checkmark &  &  &  & \checkmark & \checkmark &  &   \\
\cite{sustainable_energy_now_renewable_2016}& SIREN  &  \acs{ws} & \checkmark &  & \checkmark & \checkmark &  &  & & \checkmark &  & \checkmark \\
\cite{de-leon_almaraz_deployment_2015}&    &  \acs{ec}  & \checkmark &  & \checkmark & \checkmark &  &  & \checkmark & \checkmark &  &   \\
\cite{kamjoo_multi-objective_2016}  &    &  \acs{nsga2}  & \checkmark &  &  & \checkmark &  &  &  & \checkmark &  &   \\
\cite{bilil_multiobjective_2014}  &    &  \acs{nsga2}  & \checkmark &  &  & \checkmark &  &  &  & \checkmark &  &   \\
\cite{fazlollahi_multi-objectives_2014}  &  &  \acs{nsga2}  & \checkmark &  & \checkmark &  &  &  & \checkmark & \checkmark &  &   \\
\cite{de-leon_almaraz_assessment_2013}  &    &  \acs{ec}  & \checkmark &  & \checkmark & \checkmark &  &  & \checkmark &  & \checkmark &   \\
\cite{katsigiannis_multiobjective_2010}  &    &  \acs{nsga2}  & \checkmark &  & \checkmark &  &  &  &  & \checkmark &  &   \\
\bottomrule
\end{tabular}
}
\end{table}
Most of the studies in Table \ref{tab:moop-literature} used \ac{nsga2} to
identify the Pareto-front with a few using scalarization. Consistent with the
trend shown in Table \ref{tab:esoms}, every study in Table
\ref{tab:moop-literature} uses some economic or ``cost'' metric as one of the
objectives. Also consistent, is that none of these studies identified a metric
to optimize over social concerns. Laha et al. \cite{laha_low_2021} used
fatalities per GWh and employment per GWh as criteria for social sustainability,
but these were not objectives in their model, rather they were calculated
\textit{ex post facto} with scenario analysis. Riou et al.
\cite{riou_multi-objective_2021} investigated the tradeoffs among renewable
share, reliability, and total cost. Their findings were consistent with single
objective scenario analysis \cite{de_sisternes_value_2016}, that greater
renewable penetration leads to greater costs and less reliable energy with a
100\% renewable energy system being the least reliable or incurring the greatest
costs \cite{riou_multi-objective_2021}. 

Although previous work demonstrated the applicability of \ac{moo} to energy
systems optimization, there are significant limitations. 
\begin{itemize}
    \item{There are at most three modeled objectives
    \cite{riou_multi-objective_2021,de-leon_almaraz_deployment_2015,
    de-leon_almaraz_assessment_2013}.}
    \item{Where traditional \acp{esom} have many mature frameworks (as shown in
    Table \ref{tab:esoms}, there are no frameworks that use \ac{moo}.
    Simultaneously, none of the studies in Table \ref{tab:moop-literature}
    developed a framework. Prina et al. developed a bespoke and unlicensed model
    called ``Oemof-moea,'' however this does not constitute a framework.}
    \item{None of the studies in Table \ref{tab:moop-literature} allow
    arbitrary user-defined objectives.}
    \item{None of the studies incorporate social metrics into the modeled objectives.}
\end{itemize}

This thesis develops, \ac{osier}, a novel energy systems framework using
\ac{moo} that fills these gaps by using \acp{ga} that allows for efficient
modeling of many objectives, enabling user-defined objectives, providing the
option to make metrics of interest either objectives or constraints, and
incorporating ideas from \ac{mga} to provide insight from the sub-optimal
objective space.


% \textcolor{red}{If carbon emissions should not be considered an objective, but
% rather a constraint, because there are hard emissions budgets why can't the
% same argument be made with respect to cost? Some places might have limited
% funds to allocate for energy infrastructure. }

The next section outlines attempts to incorporate social justice concerns with
energy system models.


% \subsection{\textcolor{red}{More context for decision making with \ac{moo} in energy systems.}}

% Answer the following questions for each paper listed below (this will help demonstrate the 
% novelty of my work):

% \begin{enumerate}
%     \item Does the study develop a framework?
%     \item Is the framework open source (if applicable)? Is the input data transparent?
%     \item Does the study use genetic algorithms? If so, which one?
%     \item What methods does the study use?
%     \item What objectives does the study optimize?
%     \item Are there ``users'' that can arbitrarily add new objectives?
%     \item How many objectives can be optimized at once?
%     \item Does the study discuss energy justice? How do they define justice? What aspects are discussed?
%     \item Does the study describe how their analysis can inform decision making processes and improve
%     justice outcomes?
% \end{enumerate}


% The papers of interest

% \begin{enumerate}
%     \item \cite{kamenopoulos_assessment_2019}
%     \item \cite{kasprzyk_many_2013}
%     \item \cite{jafino_enabling_2021}
%     \item \cite{granacher_overcoming_2022}
% \end{enumerate}