\section{Modeling Energy Systems}
\label{section:esoms}

\Acp{esom} have broad utility, including forecasting future quantities,
generating insight for policy development, or energy system planning for
scheduling and acquisition \cite{decarolis_using_2011, yue_review_2018}.
However, analyses using currently available \acp{esom} seldom consider the role
of energy systems in creating and maintaining inequitable distributions of
burdens. \acp{esom} vary significantly by the energy sectors they
choose to model, the degree of physical detail, uncertainty quanitification, 
and forecasting capabilities. Table \ref{tab:esoms} summarizes the capabilities 
for a comprehensive
list of energy system analysis tools. These tools are approximately sorted by
mathematical formulation, e.g. explicit optimization or simulation. The
``\ac{milp}'' column indicates whether the framework uses a linear-programming
approach to optimize an objective function. The ``objective'' column specifies
the nature of the objective function if one exists. ``Cost'' objectives minimize
total or annual energy costs, while ``welfare'' maximizes social welfare. Some
entries have more than one objective listed. This means users may choose which
objective to optimize. None of the tools in Table \ref{tab:esoms} are designed
to handle simultaneous optimization (i.e., \ac{moo}). For those modeling
frameworks that have an ``objective'' in Table \ref{tab:esoms}, virtually all of
them optimize system costs. EnergyScope is the only exception to this, which
allows users to optimize \ac{ghg} emissions \cite{limpens_energyscope_2019}.
\textcolor{black}{The ``uncertainty'' column indicates a feature to
algorithmically generate model runs for testing either parametric or structural
uncertainties. \textcolor{black}{For example, EnergyScope is \textit{suitable}
for uncertainty analysis (i.e., many runs are computationally tractable) but
does not have any built-in capabilities \cite{limpens_energyscope_2019}.} Some
tools, such as NEMS \cite{nalley_national_2019}, incorporate uncertainty into
their calculations via learning curves. However, these learning curves require
assumptions about learning factors and technological ``optimism'' --  which are
themselves uncertain \cite{nalley_national_2019}.} Table \ref{tab:esoms} also
indicates whether the tool is a ``public code.'' This simply means users can
download and inspect the source code. Other considerations for openness, such as
licensing and development, vary among the listed frameworks.
\textcolor{black}{The other columns simply indicate the existence of particular
features rather than the relative maturity or sophistication of each feature.} 

Frameworks, such as MEDEAS \cite{capellan-perez_medeas_2020}, and MultiMod
\cite{huppmann_market_2014}, are general equilibrium models which embed energy
systems within the macro-economy and facilitate the modeling of strategic
behavior. The latter formulates a non-linear problem with the Karush-Kuhn-Tucker
optimality condition \cite{huppmann_market_2014}, as opposed to more traditional
linear programming methods. Models of this type are helpful for analyzing the
economy-wide influence of policies but lack sufficient operational detail to be
prescriptive for energy system planning.

\textcolor{black}{Agent-based models are useful for modeling the market
behaviors of different actors, such as firms (which produce power), transmission
operators, and consumers. The latter category is typically aggregated for
tractability. Modeled behaviors include technology preferences
\cite{anwar_modeling_2022, zade_quantifying_2020}, risk aversion
\cite{anwar_modeling_2022}, financial characteristics \cite{anwar_modeling_2022,
nitsch_economic_2021}, and information asymmetry among agents
\cite{anwar_modeling_2022, nitsch_economic_2021}. Due to agent heterogeneity,
agent-based models are considered useful for capturing social phenomena
\cite{yue_review_2018,fattahi_systemic_2020}.}

A further set of tools focus on simulating power flow and demand fluctuations.
\textcolor{black}{CAPOW \cite{su_open_2020} generates synthetic data with
statistical methods to explore uncertainties in energy dispatch and extreme
demand events, but does not include any investment optimization based on these
uncertainties.} \textcolor{black}{CESAR-P, SAM, Demod, and DESSTinEE focus on
modeling demand profiles
\cite{leoniefierz_hues-platformcesar-p-core_2021,bosmann_shape_2015,barsanti_socio-technical_2021}.
CESAR-P models individual building demand for energy based on the physical
parameters of the building. However, it has no dispatch or investment
optimization capabilities.} Other tools such as Pandapower, GridCal, and SciGRID
power model the infrastructure aspects of electricity systems -- transmission
and distribution -- rather than the optimal dispatch of electricity producers
\cite{thurner_pandapower_2018, vera_gridcal_2022, matke_structure_2017}.

There is an overwhelming number of models with varying levels sophistication and
capabilities. However, the inability to optimize any objective besides cost presents
a significant gap in the existing space of energy modeling tools. Further, since none 
of these tools allow for multiple objectives, true trade off analysis is rendered impossible.



\begin{table}
    \centering
    \caption{Summary of \ac{esom} frameworks.}
    \label{tab:esoms}
    \resizebox*{\textwidth}{0.95\textheight}{\begin{tabular}{lllll*{8}{c}rc}
\toprule
Model &   Citation   &   math model type   &  MILP  &   Objective   & Transmission & \multicolumn{3}{c}{Sector}& Investment & Physical & Forecasting & Agent  &  Uncertainty& Public  \\ 
&  & & & & & Heat & Electric & Transport & Optimization & Models & & Based& Analysis& Code \\
% Model  &  &  &  &  &  &  &  &  &  &  &  &  &  &  \\
\midrule
% JMM    &    -    &    -    &   nan   &    -    &  nan  &  nan  &  nan  &  nan  &  nan  &  nan  &  nan  &  nan  & nan &    \xmark     \\
% PowerMatcher    &    -    &    -    &   nan   &    -    &  nan  &  nan  &  nan  &  nan  &  nan  &  nan  &  nan  &  nan  & nan &    \checkmark     \\
% USENSYS    &    Cost    &    Optimization    &   nan   &    -    &  nan  &  nan  &  nan  &  nan  &  nan  &  nan  &  nan  &  nan  & nan &    \checkmark     \\
% GAMAMOD    &    -    &    Optimization     &   nan   &    -    &  nan  &  nan  &  nan  &  nan  &  nan  &  nan  &  nan  &  nan  & nan &    \xmark     \\
% NEMO (SEI)    &    Minimize total discounted costs    &    Optimization     &   nan   &    In preparation    &  nan  &  nan  &  nan  &  nan  &  nan  &  nan  &  nan  &  nan  & nan &    \checkmark     \\
% OMEGAlpes    &    -    &    Optimization     &   nan   &    -    &  nan  &  nan  &  nan  &  nan  &  nan  &  nan  &  nan  &  nan  & nan &    \checkmark     \\
% PowerSimulations.jl    &    Least Cost    &    Optimization     &   nan   &    -    &  nan  &  nan  &  nan  &  nan  &  nan  &  nan  &  nan  &  nan  & nan &    \checkmark     \\
% Region4FLEX    &    -    &    Optimization     &   nan   &    -    &  nan  &  nan  &  nan  &  nan  &  nan  &  nan  &  nan  &  nan  & nan &    \xmark     \\
% TIMES Évora    &    Minimise total discounted cost of the energy system    &    Optimization     &   nan   &    -    &  nan  &  nan  &  nan  &  nan  &  nan  &  nan  &  nan  &  nan  & nan &    \checkmark     \\
% TIMES-PT    &    Minimise total discounted cost of the energy system    &    Optimization     &   nan   &    -    &  nan  &  nan  &  nan  &  nan  &  nan  &  nan  &  nan  &  nan  & nan &    \checkmark     \\
AnyMOD     &    \cite{goke_graph-based_2021}    &    Optimization     &   \checkmark   &    Cost    &    &  \checkmark  &  \checkmark  &   &  \checkmark  &   &    &    & &    \checkmark     \\
Backbone     &    \cite{helisto_backboneadaptable_2019}    &    Optimization     &   \checkmark   &    Cost    &  \checkmark  &  \checkmark  &  \checkmark  &  \checkmark  &  \checkmark  &   &  \checkmark  &   & \acs{sp} &    \checkmark     \\
Balmorel     &    \cite{goransson_cost-optimized_2013}    &    Optimization     &   \checkmark   &    Cost    &  \checkmark  &    &  \checkmark  &   &  \checkmark  &   &   &    & &    \checkmark     \\
Calliope     &    \cite{pfenninger_calliope_2018}    &    Optimization     &   \checkmark   &    Cost    &   &  \checkmark  &  \checkmark  &  \checkmark  &  \checkmark  &   &   &   & &    \checkmark     \\
CapacityExpansion     &    \cite{kuepper_capacityexpansion_2020}    &    Optimization     &   \checkmark   &    Cost    &  \checkmark  &    &  \checkmark  &   &  \checkmark  &    &    &    &  &    \checkmark     \\
DIETER     &    \cite{zerrahn_long-run_2017}    &    Optimization     &   \checkmark   &    Cost    &   & \checkmark  & \checkmark  &   &  \checkmark  &   &   &   & &    \checkmark     \\
Dispa-SET    &    \cite{quoilin_modelling_2017}    &    Optimization     &   \checkmark   &    Cost    &  \checkmark  &   &  \checkmark  &   &  \checkmark  &   &   &   & &    \checkmark     \\
ELMOD     &    \cite{leuthold_elmod_2008}    &    Optimization     &   \checkmark   &    Welfare    &  \checkmark  &   &  \checkmark  &   &  \checkmark &   &   &   & &    \checkmark     \\
ELTRAMOD     &    \cite{ladwig_demand_2018}    &    Optimization     &   \checkmark   &    Cost    &   &   &  \checkmark  &   &  \checkmark  &   &   &   & &   \\
EMMA     &    \cite{hirth_european_2021}    &    Optimization     &   \checkmark   &  Cost  &   &   &  \checkmark  &   &  \checkmark  &   &   &   & &    \checkmark     \\
% ENSYSI     &    \cite{limpens_energyscope_2019}   &    Optimization     &   \checkmark   &    Cost    &  nan  &  \checkmark  &  \checkmark  &  \checkmark  &  nan  &  nan  &  nan  &  nan  & nan &    \checkmark     \\
EOLES elec    &    \cite{shirizadeh_how_2022}    &    Optimization &   \checkmark   &    Cost    &   &   &  \checkmark  &   &  \checkmark  &    &    &    & &    \checkmark     \\
ESME     &    \cite{heaton_modelling_2014}    &    Optimization     &   \checkmark   &    Cost    &   &  \checkmark  &  \checkmark  &  \checkmark  &  \checkmark  &   &   &   & \acs{mc} &   \\
ESO-X    &    \cite{heuberger_power_2017}    &    Optimization     &   \checkmark   &    Cost    &    &   &   \checkmark  &    &  \checkmark  &   & &   & &    \checkmark     \\
EnergyRt     &    \cite{lugovoy_energyrt_2022}    &    Optimization     &   \checkmark   &    Cost    &   &   &  \checkmark  &   &  \checkmark  &   &   &   & &    \checkmark     \\
EnergyScope    &    \cite{limpens_energyscope_2019}    &    Optimization     &   \checkmark   &    Cost, \acs{ghg}    &   &  \checkmark  &  \checkmark  &  \checkmark  &  \checkmark  &   &   &   & &    \checkmark     \\
Ficus     &   \cite{atabay_open-source_2017}    &    Optimization     &   \checkmark   &    Cost    &   &   &  \checkmark  &   &  \checkmark  &   &   & & &    \checkmark     \\
FlexiGIS     &    \cite{alhamwi_gis-based_2017}    &    Optimization     & \checkmark &    Cost    &   & \checkmark  & \checkmark  & \checkmark  &   &   &   &   & &    \checkmark     \\
GAMAMOD-DE    &    \cite{hauser_modelling_2019}    &    Optimization    &   \checkmark   &    Cost    &    &   &  \checkmark  &   &  \checkmark  &   &   &   &  &   \\
GenX     &    \cite{jenkins_genx_2022}    &    Optimization     &   \checkmark   &    Cost    &  \checkmark   &    &  \checkmark  &   & \checkmark  &   &   & & \acs{mga}& \checkmark     \\
GRIMSEL-FLEX    &    \cite{rinaldi_what_2022}    &    Optimization     &   \checkmark   &    Cost    &    &  \checkmark  &  \checkmark  &    &  \checkmark  &  \checkmark  &    &    & &    \checkmark     \\
HighRES     &    \cite{zeyringer_designing_2018}    &    Optimization     &   \checkmark   &    Cost    &    &  \checkmark  &  \checkmark  &    &  \checkmark  &    &    &    & &   \\
MARKAL     &    \cite{loulou_documentation_2004}    &    Optimization     &   \checkmark   &    Cost    &    &  \checkmark  &  \checkmark  &  \checkmark  &  \checkmark  &    &    &    & \acs{mc}, \acs{sp} &    \checkmark     \\
METIS     &    \cite{sakellaris_metis_2018}    &    Optimization     &   \checkmark   &    Cost    &  \checkmark  &  \checkmark  &  \checkmark  &    &  \checkmark  &    &    &    & \acs{mc} &    \checkmark     \\
Medea     &    \cite{wehrle_cost_2021}    &    Optimization     &   \checkmark   &    Cost    &   &   &  \checkmark  &    &  \checkmark  &   &   &   & &    \checkmark     \\
Oemof     &    \cite{hilpert_open_2018}    &    Optimization  &   \checkmark   &    Cost    &    &  \checkmark  &  \checkmark  &  \checkmark  &  \checkmark  &     &   &  &  &  \checkmark     \\
OPERA     &    \cite{van_stralen_opera_2021}    &    Optimization     &   \checkmark   &    Cost    &  \checkmark  &  \checkmark  &  \checkmark  &   &  \checkmark  &   &   &   & &    \checkmark     \\
OSeMOSYS     &    \cite{howells_osemosys_2011}    &    Optimization     &   \checkmark   &    Cost    &    &  \checkmark  &  \checkmark  &  \checkmark  &  \checkmark  &    &    &    & &    \checkmark     \\
OnSSET     &    \cite{mentis_gis-based_2015}    &    Optimization     &   \checkmark   &    Cost    &  \checkmark  &    &  \checkmark  &    &  \checkmark  &    &    &    & &    \checkmark     \\
PLEXOS     &   \cite{deane_quantifying_2015}    &    Optimization     &   \checkmark   &    Cost    &   &   &  \checkmark  &   &  \checkmark  &   &   &   & \acs{mc} &   \\
POLES     &    \cite{vandyck_global_2016}     &    Optimization     &   \checkmark   &    Cost    &   &   &  \checkmark  &   &  \checkmark  &   &   &   &  &   \\
POMATO     &    \cite{weinhold_power_2021}     &    Optimization     &   \checkmark   &    Cost    &  \checkmark  &  \checkmark  &  \checkmark  &   &    &   &   &   &  &    \checkmark     \\
PRIMES     &    \cite{antoniou_decision_1999}    &    Optimization     &   \checkmark   &    Cost    &  \checkmark  &  \checkmark  &  \checkmark  &  \checkmark  &  \checkmark  &   &   &   & &  \\
PyPSA     &    \cite{brown_pypsa_2018}    &    Optimization  &   \checkmark   &    Cost    &  \checkmark  &  \checkmark  &  \checkmark  &  \checkmark  &  \checkmark  &  \checkmark  &   &   & \acs{mga} &    \checkmark     \\
REMix     &    \cite{gils_integrated_2017}    &    Optimization     &   \checkmark   &    Cost    &  \checkmark  &  \checkmark  &  \checkmark  &  \checkmark  &  \checkmark  &   &   &   &  &    \checkmark     \\
REopt     &    \cite{simpkins_reopt_nodate}    &    Optimization     &   \checkmark   &    Cost    &   &  \checkmark  &  \checkmark  &   &  \checkmark  &   &   &   &  &    \checkmark     \\
% SMS     &    \cite{the_sms_team_sms_2022}    &    Optimization     &   \checkmark   &    Cost    &  nan  &  nan  &  nan  &  nan  &  nan  &  nan  &  nan  &  nan  & nan &    \checkmark     \\
StELMOD     &    \cite{abrell_integrating_2015}    &    Optimization     &   \checkmark   &    Cost    &  \checkmark  &   &  \checkmark  &   &  \checkmark  &   &   &   &  &    \checkmark     \\
Switch     &    \cite{johnston_switch_2019}    &    Optimization     &   \checkmark   &    Cost    &  \checkmark  &  \checkmark  &  \checkmark  &  \checkmark  &  \checkmark  &   &   &   &  &    \checkmark     \\
TIMES     &    \cite{loulou_documentation_2016}    &    Optimization     &   \checkmark   &    Cost, Welfare    &   & \checkmark  & \checkmark  &    & \checkmark  &    &    &    & \acs{sp} &    \checkmark     \\
Temoa     &    \cite{hunter_modeling_2013}    &    Optimization     &   \checkmark   &    Cost    &   &  \checkmark  &  \checkmark  &  \checkmark  &  \checkmark  &   &   &   & \acs{mga}, \acs{mc}, \acs{sp} &    \checkmark     \\
TransiEnt     &    \cite{andresen_status_2015}    &    Simulation     &   \checkmark   & Cost & \checkmark  &  \checkmark  &  \checkmark  &   &   &   &   &  &  &    \checkmark     \\
URBS     &    \cite{dorfner_open_2015}    &    Optimization     &   \checkmark   &    Cost    &  \checkmark  &  \checkmark  &  \checkmark  &  \checkmark  &  \checkmark  &   &   &   & &    \checkmark     \\
\arrayrulecolor{lightgray}\hline
% DynPP    &    operation, cost, emissions, thermal stress    &    Optimization  and Simulation     &   nan   &    Modelling and simulation of a coal-fired power plant for start-up optimisation    &  nan  &  nan  &  nan  &  nan  &  nan  &  nan  &  nan  &  nan  & nan &    \xmark     \\
% EA-PSM Electric Arc Flash    &    -    &    Optimization  and Simulation     &   nan   &    -    &  nan  &  nan  &  nan  &  nan  &  nan  &  nan  &  nan  &  nan  & nan &    \xmark     \\
% EA-PSM Electric Short Circuit    &    -    &    Optimization  and Simulation     &   nan   &    -    &  nan  &  nan  &  nan  &  nan  &  nan  &  nan  &  nan  &  nan  & nan &    \xmark     \\
% NEMO    &    minimise average cost of electricity    &    Optimization  and Simulation     &   nan   &    -    &  nan  &  nan  &  nan  &  nan  &  nan  &  nan  &  nan  &  nan  & nan &    \checkmark     \\
Genesys     &    \cite{bussar_optimal_2014}    &    Optimization  and Simulation     &      &    Cost    &    &    &  \checkmark  &    &  \checkmark  &    &    &  &  &      \\
OpenTUMFlex     &    \cite{zade_quantifying_2020}    &    Optimization  and Simulation     &   \checkmark   &    Cost    &   &  \checkmark  &  \checkmark&  \checkmark  &   &   &  \checkmark &   & &    \checkmark     \\
PowNet     &    \cite{chowdhury_pownet_2020}    &    Optimization  and Simulation     &   \checkmark   &    Cost    &  \checkmark  &  &  \checkmark  &  &   &  \checkmark  &   &   & &    \checkmark     \\
Renpass     &    \cite{frauke_wiese_renpass_2015}    &    Optimization  and Simulation     &    &    Cost    &    &    &  \checkmark&    &   &   &   &   & &    \checkmark     \\
SimSEE     &    \cite{chaer_simulacion_2008}    &    Optimization  and Simulation     &      &    Cost    &    &    &  \checkmark  &    &    &  \checkmark  &    &    &  &    \checkmark     \\
% Maon    &    Economic welfare at electricity spot, electricity reserve, and emission markets in Europe    &    Optimization , Simulation  and Other     &   nan   &    Ketov, Mihail (2019). "Marktsimulationen unter Berücksichtigung der Strom-Wärme-Sektorenkopplung", Print Production, Aachener Beiträge zur Energieversorgung, volume 189, PhD thesis, RWTH Aachen University.    &  nan  &  nan  &  nan  &  nan  &  nan  &  nan  &  nan  &  nan  & nan &    \xmark     \\
MEDEAS     &    \cite{capellan-perez_medeas_2020}    &    Other     &    &    &   & \checkmark  & \checkmark  & \checkmark  &   & \checkmark  &   &   & \acs{mc} &  \checkmark    \\
MultiMod     &    \cite{huppmann_market_2014}    &    Other     &  & Welfare &  \checkmark  &  \checkmark  &  \checkmark  &  \checkmark  &  \checkmark  &  &   &   & &\\
NEMS     &    \cite{nalley_national_2019}    &    Other     &   \checkmark   & Cost &  \checkmark & \checkmark  &  \checkmark &  \checkmark  &  \checkmark  &   &  \checkmark  & & &  \\
\arrayrulecolor{lightgray}\hline
% Energy Policy Simulator    &    -    &    Simulation     &   nan   &    -    &  nan  &  nan  &  nan  &  nan  &  nan  &  nan  &  nan  &  nan  & nan &    \checkmark     \\
% EnergyNumbers-Balancing    &    -    &    Simulation     &   nan   &    -    &  nan  &  nan  &  nan  &  nan  &  nan  &  nan  &  nan  &  nan  & nan &    \xmark     \\
% PowerSimulationsDynamics.jl    &    -    &    Simulation     &   nan   &    -    &  nan  &  nan  &  nan  &  nan  &  nan  &  nan  &  nan  &  nan  & nan &    \checkmark     \\
% PowerSystems.jl    &    -    &    Simulation     &   nan   &    -    &  nan  &  nan  &  nan  &  nan  &  nan  &  nan  &  nan  &  nan  & nan &    \checkmark     \\
Breakthrough Energy Model    &    \cite{xu_us_2020}    & Simulation     &  & &  \checkmark  &   & \checkmark  &   &  \checkmark&   &   &   & &    \checkmark     \\
CAPOW     &    \cite{su_open_2020}    &    Simulation     & \checkmark &    Cost    &  \checkmark  &   &  \checkmark  &   &   & &  \checkmark  &   & \checkmark &    \checkmark     \\
CESAR-P    &    \cite{leoniefierz_hues-platformcesar-p-core_2021}    &    Simulation     & & & & & \checkmark & & & \checkmark & \checkmark & & & \checkmark \\
DESSTinEE    &    \cite{bosmann_shape_2015}    &    Simulation     & \checkmark & Cost & \checkmark  & \checkmark  & \checkmark  &   &   & \checkmark  &   &   & &    \checkmark     \\
Demod    &    \cite{barsanti_socio-technical_2021}    &    Simulation     & & &  &  \checkmark  &  \checkmark  &   &   &  \checkmark  &  \checkmark  &    &  \acs{mc}&    \checkmark     \\
EMLab-Generation    &    \cite{richstein_cross-border_2014}    & Simulation & & Cost & & & \checkmark &  & \checkmark & & & & \acs{mc} &    \checkmark     \\
EnergyPLAN     &    \cite{lund_energyplan_2021}   &    Simulation     &   &    Cost    &  \checkmark  &  \checkmark  &  \checkmark  &  \checkmark  &  \checkmark  &   &   &   & &     \\
Energy Transition Model    &    \cite{quintel_etm_2022}    &    Simulation     & & &   &   &  \checkmark  &  &   &   &   &    &  &    \checkmark     \\
GridCal     &    \cite{vera_gridcal_2022}    &    Simulation     &   &     &  \checkmark  &   &  &   &   &  \checkmark  &   &   & &    \checkmark     \\
LoadProfileGenerator     &    \cite{pflugradt_modelling_2016}    &    Simulation    &    & &     &  \checkmark  &  \checkmark  &   &   &  \checkmark  &  \checkmark  &  \checkmark  & &    \checkmark     \\
Pandapower     &    \cite{thurner_pandapower_2018}    &    Simulation     &    &  &  \checkmark &  & & & & \checkmark & & & &    \checkmark     \\
Pvlib     &    \cite{holmgren_pvlib_2018}    &    Simulation     &    & &   & \checkmark& &  & & \checkmark & \checkmark &  & &    \checkmark     \\
PyLESA     &    \cite{lyden_pylesa_2021}    &    Simulation     & &    Cost    &  \checkmark  &   &  \checkmark &  \checkmark  & &  \checkmark &   &   & &    \checkmark     \\
SAM    &    \cite{blair_system_2014}    &    Simulation     & & & & & & & &\checkmark&\checkmark& & \acs{pa} &    \checkmark     \\
SciGRID power    &    \cite{matke_structure_2017}    &    Simulation     & & &  \checkmark  &   & \checkmark  &    &    &    &   &   & &    \checkmark     \\
SimSES     &    \cite{naumann_simses_2017}    &    Simulation     &   & & & & \checkmark &  & & \checkmark& & & &    \checkmark     \\
\arrayrulecolor{lightgray}\hline
AMIRIS     &    \cite{nitsch_economic_2021}    &    Simulation  and Agent-based     &  &  &  &  & \checkmark  &  & \checkmark  &  &  &  \checkmark  & &    \checkmark     \\
ASAM     &    \cite{glismann_ancillary_2021}    &    Simulation  and Agent-based     & & & \checkmark &  & \checkmark & & & & &  \checkmark  & &    \checkmark     \\
EMIS-AS     &    \cite{anwar_modeling_2022}    &    Simulation  and Agent-based     & \checkmark & Welfare & \checkmark &  & \checkmark & & & & &  \checkmark  & \checkmark & \checkmark \\
Lemlab     &    \cite{zade_satisfying_2022}    &    Simulation  and Agent-based     &   \checkmark   & Welfare &  &  &  \checkmark  & & &   &   &  \checkmark  & &    \checkmark     \\
MOCES     &    \cite{exel_multi-domain_2015}    &    Simulation  and Agent-based     & & Cost &   &   &  \checkmark  &   &   &  \checkmark  & &  \checkmark  & &  \\
% Mosaik     &    \cite{ofenloch_mosaik_2022}    &    Simulation  and Agent-based     &   nan   &    -    &  nan  &  nan  &  nan  &  nan  &  nan  &  nan  &  nan  &  \checkmark  & nan &    \checkmark     \\
% SIREN     &    \cite{sustainable_energy_now_renewable_2016}    &    Simulation  and Other     &   nan   &    Cost    &  nan  &  nan  &  nan  &  nan  &  nan  &  nan  &  nan  &  nan  & nan &    \checkmark     \\
% SciGRID gas    &    -    &    Simulation  and Other     &   nan   &    -    &  nan  &  nan  &  nan  &  nan  &  nan  &  nan  &  nan  &  nan  & nan &    \checkmark     \\
\arrayrulecolor{black}\bottomrule
\end{tabular}
}
\end{table}


% \begin{itemize} \item METIS has the following motivation
%     \cite{sakellaris_metis_2018} \begin{enumerate} \item Close the gap between
%     modellers and policy-makers, enabling policy-makers to become modellers.
%     \item reconciles user-friendliness with powerful capabilities \item
%     modularity \item \textcolor{red}{METIS does not support multi-objective
%     optimization!}    
%     \end{enumerate} \item energyRt has similar motivations
%     \cite{lugovoy_energyrt_2022}. \begin{enumerate} \item enhance
%     reproducibility \item reduce the learning curve \item minimize model
%     development time \end{enumerate} \item GENESYS cannot \sout{can} do
%     multi-objective optimization \cite{bussar_optimal_2014}. It uses a logical
%     flow to model dispatch behavior for a genetic algorithm! \begin{enumerate}
%     \item One drawback of the "hierarchical system management" method (i.e. a
%     logical flow chart) is the difficulty of modeling ramping rates... (test
%     this). \end{enumerate} \end{itemize}

\subsection{Economic Dispatch and Social Welfare}
\Ac{lp} or \ac{milp} are the dominant optimization approaches among the
frameworks in Table \ref{tab:esoms}. Economic dispatch models optimize the power
output of \textit{dispatchable} generators in a model system
\cite{de_queiroz_repurposing_2019, neumann_near-optimal_2021}. They all share
the same fundamental formulation.
\begin{align}
    \intertext{Minimize}
    \label{eqn:generic_objective}
    &F(x) = \sum_i C_i x_i\\
    % &F(x) = \sum_t^T\left[\sum_i^{N_g}{C_i^g x_i} - \sum_j^{N_d}{C_j^d x_j}\right],\\
    \intertext{subject to,}
    % &\sum_i{x_i} - \sum_j{x_j} = 0\\
    &g(x, p) \leq 0.\nonumber\\
    &x \in \vec{X}\nonumber\\
    \intertext{where}
    &\vec{X} \text{ is the set of decision variables,}\nonumber\\
    & C_i \text{ is the \textit{i-th} cost,}\nonumber\\
    & g \text{ is some linear inequality constraint,}\nonumber\\
    & p \text{ is some arbitrary parameter.}\nonumber
\end{align}
The exact formulation of Equation \ref{eqn:generic_objective} may vary slightly
across models, but the objective for most economic dispatch models is to
minimize total cost. The near universality of a cost-based objective function
comes from the concept of \textit{social welfare maximization}. This concept is
illustrated in Figure \ref{fig:social-max}.

\begin{figure}[H]
  \centering
  \resizebox{\columnwidth}{!}{%% Creator: Matplotlib, PGF backend
%%
%% To include the figure in your LaTeX document, write
%%   \input{<filename>.pgf}
%%
%% Make sure the required packages are loaded in your preamble
%%   \usepackage{pgf}
%%
%% Also ensure that all the required font packages are loaded; for instance,
%% the lmodern package is sometimes necessary when using math font.
%%   \usepackage{lmodern}
%%
%% Figures using additional raster images can only be included by \input if
%% they are in the same directory as the main LaTeX file. For loading figures
%% from other directories you can use the `import` package
%%   \usepackage{import}
%%
%% and then include the figures with
%%   \import{<path to file>}{<filename>.pgf}
%%
%% Matplotlib used the following preamble
%%
\begingroup%
\makeatletter%
\begin{pgfpicture}%
\pgfpathrectangle{\pgfpointorigin}{\pgfqpoint{13.900000in}{5.900000in}}%
\pgfusepath{use as bounding box, clip}%
\begin{pgfscope}%
\pgfsetbuttcap%
\pgfsetmiterjoin%
\definecolor{currentfill}{rgb}{1.000000,1.000000,1.000000}%
\pgfsetfillcolor{currentfill}%
\pgfsetlinewidth{0.000000pt}%
\definecolor{currentstroke}{rgb}{0.000000,0.000000,0.000000}%
\pgfsetstrokecolor{currentstroke}%
\pgfsetdash{}{0pt}%
\pgfpathmoveto{\pgfqpoint{0.000000in}{0.000000in}}%
\pgfpathlineto{\pgfqpoint{13.900000in}{0.000000in}}%
\pgfpathlineto{\pgfqpoint{13.900000in}{5.900000in}}%
\pgfpathlineto{\pgfqpoint{0.000000in}{5.900000in}}%
\pgfpathlineto{\pgfqpoint{0.000000in}{0.000000in}}%
\pgfpathclose%
\pgfusepath{fill}%
\end{pgfscope}%
\begin{pgfscope}%
\pgfsetbuttcap%
\pgfsetmiterjoin%
\definecolor{currentfill}{rgb}{1.000000,1.000000,1.000000}%
\pgfsetfillcolor{currentfill}%
\pgfsetlinewidth{0.000000pt}%
\definecolor{currentstroke}{rgb}{0.000000,0.000000,0.000000}%
\pgfsetstrokecolor{currentstroke}%
\pgfsetstrokeopacity{0.000000}%
\pgfsetdash{}{0pt}%
\pgfpathmoveto{\pgfqpoint{0.608025in}{0.554012in}}%
\pgfpathlineto{\pgfqpoint{6.995910in}{0.554012in}}%
\pgfpathlineto{\pgfqpoint{6.995910in}{5.751775in}}%
\pgfpathlineto{\pgfqpoint{0.608025in}{5.751775in}}%
\pgfpathlineto{\pgfqpoint{0.608025in}{0.554012in}}%
\pgfpathclose%
\pgfusepath{fill}%
\end{pgfscope}%
\begin{pgfscope}%
\pgfpathrectangle{\pgfqpoint{0.608025in}{0.554012in}}{\pgfqpoint{6.387885in}{5.197763in}}%
\pgfusepath{clip}%
\pgfsetbuttcap%
\pgfsetroundjoin%
\definecolor{currentfill}{rgb}{1.000000,0.000000,0.000000}%
\pgfsetfillcolor{currentfill}%
\pgfsetlinewidth{1.003750pt}%
\definecolor{currentstroke}{rgb}{1.000000,0.000000,0.000000}%
\pgfsetstrokecolor{currentstroke}%
\pgfsetdash{}{0pt}%
\pgfsys@defobject{currentmarker}{\pgfqpoint{-0.069444in}{-0.069444in}}{\pgfqpoint{0.069444in}{0.069444in}}{%
\pgfpathmoveto{\pgfqpoint{0.000000in}{-0.069444in}}%
\pgfpathcurveto{\pgfqpoint{0.018417in}{-0.069444in}}{\pgfqpoint{0.036082in}{-0.062127in}}{\pgfqpoint{0.049105in}{-0.049105in}}%
\pgfpathcurveto{\pgfqpoint{0.062127in}{-0.036082in}}{\pgfqpoint{0.069444in}{-0.018417in}}{\pgfqpoint{0.069444in}{0.000000in}}%
\pgfpathcurveto{\pgfqpoint{0.069444in}{0.018417in}}{\pgfqpoint{0.062127in}{0.036082in}}{\pgfqpoint{0.049105in}{0.049105in}}%
\pgfpathcurveto{\pgfqpoint{0.036082in}{0.062127in}}{\pgfqpoint{0.018417in}{0.069444in}}{\pgfqpoint{0.000000in}{0.069444in}}%
\pgfpathcurveto{\pgfqpoint{-0.018417in}{0.069444in}}{\pgfqpoint{-0.036082in}{0.062127in}}{\pgfqpoint{-0.049105in}{0.049105in}}%
\pgfpathcurveto{\pgfqpoint{-0.062127in}{0.036082in}}{\pgfqpoint{-0.069444in}{0.018417in}}{\pgfqpoint{-0.069444in}{0.000000in}}%
\pgfpathcurveto{\pgfqpoint{-0.069444in}{-0.018417in}}{\pgfqpoint{-0.062127in}{-0.036082in}}{\pgfqpoint{-0.049105in}{-0.049105in}}%
\pgfpathcurveto{\pgfqpoint{-0.036082in}{-0.062127in}}{\pgfqpoint{-0.018417in}{-0.069444in}}{\pgfqpoint{0.000000in}{-0.069444in}}%
\pgfpathlineto{\pgfqpoint{0.000000in}{-0.069444in}}%
\pgfpathclose%
\pgfusepath{stroke,fill}%
}%
\begin{pgfscope}%
\pgfsys@transformshift{3.801968in}{3.152893in}%
\pgfsys@useobject{currentmarker}{}%
\end{pgfscope}%
\end{pgfscope}%
\begin{pgfscope}%
\pgfpathrectangle{\pgfqpoint{0.608025in}{0.554012in}}{\pgfqpoint{6.387885in}{5.197763in}}%
\pgfusepath{clip}%
\pgfsetbuttcap%
\pgfsetroundjoin%
\definecolor{currentfill}{rgb}{0.121569,0.466667,0.705882}%
\pgfsetfillcolor{currentfill}%
\pgfsetfillopacity{0.200000}%
\pgfsetlinewidth{0.000000pt}%
\definecolor{currentstroke}{rgb}{0.000000,0.000000,0.000000}%
\pgfsetstrokecolor{currentstroke}%
\pgfsetdash{}{0pt}%
\pgfpathmoveto{\pgfqpoint{0.608025in}{5.751775in}}%
\pgfpathlineto{\pgfqpoint{0.608025in}{3.152893in}}%
\pgfpathlineto{\pgfqpoint{0.614419in}{3.152893in}}%
\pgfpathlineto{\pgfqpoint{0.620813in}{3.152893in}}%
\pgfpathlineto{\pgfqpoint{0.627208in}{3.152893in}}%
\pgfpathlineto{\pgfqpoint{0.633602in}{3.152893in}}%
\pgfpathlineto{\pgfqpoint{0.639996in}{3.152893in}}%
\pgfpathlineto{\pgfqpoint{0.646391in}{3.152893in}}%
\pgfpathlineto{\pgfqpoint{0.652785in}{3.152893in}}%
\pgfpathlineto{\pgfqpoint{0.659179in}{3.152893in}}%
\pgfpathlineto{\pgfqpoint{0.665573in}{3.152893in}}%
\pgfpathlineto{\pgfqpoint{0.671968in}{3.152893in}}%
\pgfpathlineto{\pgfqpoint{0.678362in}{3.152893in}}%
\pgfpathlineto{\pgfqpoint{0.684756in}{3.152893in}}%
\pgfpathlineto{\pgfqpoint{0.691150in}{3.152893in}}%
\pgfpathlineto{\pgfqpoint{0.697545in}{3.152893in}}%
\pgfpathlineto{\pgfqpoint{0.703939in}{3.152893in}}%
\pgfpathlineto{\pgfqpoint{0.710333in}{3.152893in}}%
\pgfpathlineto{\pgfqpoint{0.716728in}{3.152893in}}%
\pgfpathlineto{\pgfqpoint{0.723122in}{3.152893in}}%
\pgfpathlineto{\pgfqpoint{0.729516in}{3.152893in}}%
\pgfpathlineto{\pgfqpoint{0.735910in}{3.152893in}}%
\pgfpathlineto{\pgfqpoint{0.742305in}{3.152893in}}%
\pgfpathlineto{\pgfqpoint{0.748699in}{3.152893in}}%
\pgfpathlineto{\pgfqpoint{0.755093in}{3.152893in}}%
\pgfpathlineto{\pgfqpoint{0.761488in}{3.152893in}}%
\pgfpathlineto{\pgfqpoint{0.767882in}{3.152893in}}%
\pgfpathlineto{\pgfqpoint{0.774276in}{3.152893in}}%
\pgfpathlineto{\pgfqpoint{0.780670in}{3.152893in}}%
\pgfpathlineto{\pgfqpoint{0.787065in}{3.152893in}}%
\pgfpathlineto{\pgfqpoint{0.793459in}{3.152893in}}%
\pgfpathlineto{\pgfqpoint{0.799853in}{3.152893in}}%
\pgfpathlineto{\pgfqpoint{0.806248in}{3.152893in}}%
\pgfpathlineto{\pgfqpoint{0.812642in}{3.152893in}}%
\pgfpathlineto{\pgfqpoint{0.819036in}{3.152893in}}%
\pgfpathlineto{\pgfqpoint{0.825430in}{3.152893in}}%
\pgfpathlineto{\pgfqpoint{0.831825in}{3.152893in}}%
\pgfpathlineto{\pgfqpoint{0.838219in}{3.152893in}}%
\pgfpathlineto{\pgfqpoint{0.844613in}{3.152893in}}%
\pgfpathlineto{\pgfqpoint{0.851007in}{3.152893in}}%
\pgfpathlineto{\pgfqpoint{0.857402in}{3.152893in}}%
\pgfpathlineto{\pgfqpoint{0.863796in}{3.152893in}}%
\pgfpathlineto{\pgfqpoint{0.870190in}{3.152893in}}%
\pgfpathlineto{\pgfqpoint{0.876585in}{3.152893in}}%
\pgfpathlineto{\pgfqpoint{0.882979in}{3.152893in}}%
\pgfpathlineto{\pgfqpoint{0.889373in}{3.152893in}}%
\pgfpathlineto{\pgfqpoint{0.895767in}{3.152893in}}%
\pgfpathlineto{\pgfqpoint{0.902162in}{3.152893in}}%
\pgfpathlineto{\pgfqpoint{0.908556in}{3.152893in}}%
\pgfpathlineto{\pgfqpoint{0.914950in}{3.152893in}}%
\pgfpathlineto{\pgfqpoint{0.921345in}{3.152893in}}%
\pgfpathlineto{\pgfqpoint{0.927739in}{3.152893in}}%
\pgfpathlineto{\pgfqpoint{0.934133in}{3.152893in}}%
\pgfpathlineto{\pgfqpoint{0.940527in}{3.152893in}}%
\pgfpathlineto{\pgfqpoint{0.946922in}{3.152893in}}%
\pgfpathlineto{\pgfqpoint{0.953316in}{3.152893in}}%
\pgfpathlineto{\pgfqpoint{0.959710in}{3.152893in}}%
\pgfpathlineto{\pgfqpoint{0.966105in}{3.152893in}}%
\pgfpathlineto{\pgfqpoint{0.972499in}{3.152893in}}%
\pgfpathlineto{\pgfqpoint{0.978893in}{3.152893in}}%
\pgfpathlineto{\pgfqpoint{0.985287in}{3.152893in}}%
\pgfpathlineto{\pgfqpoint{0.991682in}{3.152893in}}%
\pgfpathlineto{\pgfqpoint{0.998076in}{3.152893in}}%
\pgfpathlineto{\pgfqpoint{1.004470in}{3.152893in}}%
\pgfpathlineto{\pgfqpoint{1.010864in}{3.152893in}}%
\pgfpathlineto{\pgfqpoint{1.017259in}{3.152893in}}%
\pgfpathlineto{\pgfqpoint{1.023653in}{3.152893in}}%
\pgfpathlineto{\pgfqpoint{1.030047in}{3.152893in}}%
\pgfpathlineto{\pgfqpoint{1.036442in}{3.152893in}}%
\pgfpathlineto{\pgfqpoint{1.042836in}{3.152893in}}%
\pgfpathlineto{\pgfqpoint{1.049230in}{3.152893in}}%
\pgfpathlineto{\pgfqpoint{1.055624in}{3.152893in}}%
\pgfpathlineto{\pgfqpoint{1.062019in}{3.152893in}}%
\pgfpathlineto{\pgfqpoint{1.068413in}{3.152893in}}%
\pgfpathlineto{\pgfqpoint{1.074807in}{3.152893in}}%
\pgfpathlineto{\pgfqpoint{1.081202in}{3.152893in}}%
\pgfpathlineto{\pgfqpoint{1.087596in}{3.152893in}}%
\pgfpathlineto{\pgfqpoint{1.093990in}{3.152893in}}%
\pgfpathlineto{\pgfqpoint{1.100384in}{3.152893in}}%
\pgfpathlineto{\pgfqpoint{1.106779in}{3.152893in}}%
\pgfpathlineto{\pgfqpoint{1.113173in}{3.152893in}}%
\pgfpathlineto{\pgfqpoint{1.119567in}{3.152893in}}%
\pgfpathlineto{\pgfqpoint{1.125962in}{3.152893in}}%
\pgfpathlineto{\pgfqpoint{1.132356in}{3.152893in}}%
\pgfpathlineto{\pgfqpoint{1.138750in}{3.152893in}}%
\pgfpathlineto{\pgfqpoint{1.145144in}{3.152893in}}%
\pgfpathlineto{\pgfqpoint{1.151539in}{3.152893in}}%
\pgfpathlineto{\pgfqpoint{1.157933in}{3.152893in}}%
\pgfpathlineto{\pgfqpoint{1.164327in}{3.152893in}}%
\pgfpathlineto{\pgfqpoint{1.170721in}{3.152893in}}%
\pgfpathlineto{\pgfqpoint{1.177116in}{3.152893in}}%
\pgfpathlineto{\pgfqpoint{1.183510in}{3.152893in}}%
\pgfpathlineto{\pgfqpoint{1.189904in}{3.152893in}}%
\pgfpathlineto{\pgfqpoint{1.196299in}{3.152893in}}%
\pgfpathlineto{\pgfqpoint{1.202693in}{3.152893in}}%
\pgfpathlineto{\pgfqpoint{1.209087in}{3.152893in}}%
\pgfpathlineto{\pgfqpoint{1.215481in}{3.152893in}}%
\pgfpathlineto{\pgfqpoint{1.221876in}{3.152893in}}%
\pgfpathlineto{\pgfqpoint{1.228270in}{3.152893in}}%
\pgfpathlineto{\pgfqpoint{1.234664in}{3.152893in}}%
\pgfpathlineto{\pgfqpoint{1.241059in}{3.152893in}}%
\pgfpathlineto{\pgfqpoint{1.247453in}{3.152893in}}%
\pgfpathlineto{\pgfqpoint{1.253847in}{3.152893in}}%
\pgfpathlineto{\pgfqpoint{1.260241in}{3.152893in}}%
\pgfpathlineto{\pgfqpoint{1.266636in}{3.152893in}}%
\pgfpathlineto{\pgfqpoint{1.273030in}{3.152893in}}%
\pgfpathlineto{\pgfqpoint{1.279424in}{3.152893in}}%
\pgfpathlineto{\pgfqpoint{1.285818in}{3.152893in}}%
\pgfpathlineto{\pgfqpoint{1.292213in}{3.152893in}}%
\pgfpathlineto{\pgfqpoint{1.298607in}{3.152893in}}%
\pgfpathlineto{\pgfqpoint{1.305001in}{3.152893in}}%
\pgfpathlineto{\pgfqpoint{1.311396in}{3.152893in}}%
\pgfpathlineto{\pgfqpoint{1.317790in}{3.152893in}}%
\pgfpathlineto{\pgfqpoint{1.324184in}{3.152893in}}%
\pgfpathlineto{\pgfqpoint{1.330578in}{3.152893in}}%
\pgfpathlineto{\pgfqpoint{1.336973in}{3.152893in}}%
\pgfpathlineto{\pgfqpoint{1.343367in}{3.152893in}}%
\pgfpathlineto{\pgfqpoint{1.349761in}{3.152893in}}%
\pgfpathlineto{\pgfqpoint{1.356156in}{3.152893in}}%
\pgfpathlineto{\pgfqpoint{1.362550in}{3.152893in}}%
\pgfpathlineto{\pgfqpoint{1.368944in}{3.152893in}}%
\pgfpathlineto{\pgfqpoint{1.375338in}{3.152893in}}%
\pgfpathlineto{\pgfqpoint{1.381733in}{3.152893in}}%
\pgfpathlineto{\pgfqpoint{1.388127in}{3.152893in}}%
\pgfpathlineto{\pgfqpoint{1.394521in}{3.152893in}}%
\pgfpathlineto{\pgfqpoint{1.400916in}{3.152893in}}%
\pgfpathlineto{\pgfqpoint{1.407310in}{3.152893in}}%
\pgfpathlineto{\pgfqpoint{1.413704in}{3.152893in}}%
\pgfpathlineto{\pgfqpoint{1.420098in}{3.152893in}}%
\pgfpathlineto{\pgfqpoint{1.426493in}{3.152893in}}%
\pgfpathlineto{\pgfqpoint{1.432887in}{3.152893in}}%
\pgfpathlineto{\pgfqpoint{1.439281in}{3.152893in}}%
\pgfpathlineto{\pgfqpoint{1.445675in}{3.152893in}}%
\pgfpathlineto{\pgfqpoint{1.452070in}{3.152893in}}%
\pgfpathlineto{\pgfqpoint{1.458464in}{3.152893in}}%
\pgfpathlineto{\pgfqpoint{1.464858in}{3.152893in}}%
\pgfpathlineto{\pgfqpoint{1.471253in}{3.152893in}}%
\pgfpathlineto{\pgfqpoint{1.477647in}{3.152893in}}%
\pgfpathlineto{\pgfqpoint{1.484041in}{3.152893in}}%
\pgfpathlineto{\pgfqpoint{1.490435in}{3.152893in}}%
\pgfpathlineto{\pgfqpoint{1.496830in}{3.152893in}}%
\pgfpathlineto{\pgfqpoint{1.503224in}{3.152893in}}%
\pgfpathlineto{\pgfqpoint{1.509618in}{3.152893in}}%
\pgfpathlineto{\pgfqpoint{1.516013in}{3.152893in}}%
\pgfpathlineto{\pgfqpoint{1.522407in}{3.152893in}}%
\pgfpathlineto{\pgfqpoint{1.528801in}{3.152893in}}%
\pgfpathlineto{\pgfqpoint{1.535195in}{3.152893in}}%
\pgfpathlineto{\pgfqpoint{1.541590in}{3.152893in}}%
\pgfpathlineto{\pgfqpoint{1.547984in}{3.152893in}}%
\pgfpathlineto{\pgfqpoint{1.554378in}{3.152893in}}%
\pgfpathlineto{\pgfqpoint{1.560773in}{3.152893in}}%
\pgfpathlineto{\pgfqpoint{1.567167in}{3.152893in}}%
\pgfpathlineto{\pgfqpoint{1.573561in}{3.152893in}}%
\pgfpathlineto{\pgfqpoint{1.579955in}{3.152893in}}%
\pgfpathlineto{\pgfqpoint{1.586350in}{3.152893in}}%
\pgfpathlineto{\pgfqpoint{1.592744in}{3.152893in}}%
\pgfpathlineto{\pgfqpoint{1.599138in}{3.152893in}}%
\pgfpathlineto{\pgfqpoint{1.605532in}{3.152893in}}%
\pgfpathlineto{\pgfqpoint{1.611927in}{3.152893in}}%
\pgfpathlineto{\pgfqpoint{1.618321in}{3.152893in}}%
\pgfpathlineto{\pgfqpoint{1.624715in}{3.152893in}}%
\pgfpathlineto{\pgfqpoint{1.631110in}{3.152893in}}%
\pgfpathlineto{\pgfqpoint{1.637504in}{3.152893in}}%
\pgfpathlineto{\pgfqpoint{1.643898in}{3.152893in}}%
\pgfpathlineto{\pgfqpoint{1.650292in}{3.152893in}}%
\pgfpathlineto{\pgfqpoint{1.656687in}{3.152893in}}%
\pgfpathlineto{\pgfqpoint{1.663081in}{3.152893in}}%
\pgfpathlineto{\pgfqpoint{1.669475in}{3.152893in}}%
\pgfpathlineto{\pgfqpoint{1.675870in}{3.152893in}}%
\pgfpathlineto{\pgfqpoint{1.682264in}{3.152893in}}%
\pgfpathlineto{\pgfqpoint{1.688658in}{3.152893in}}%
\pgfpathlineto{\pgfqpoint{1.695052in}{3.152893in}}%
\pgfpathlineto{\pgfqpoint{1.701447in}{3.152893in}}%
\pgfpathlineto{\pgfqpoint{1.707841in}{3.152893in}}%
\pgfpathlineto{\pgfqpoint{1.714235in}{3.152893in}}%
\pgfpathlineto{\pgfqpoint{1.720630in}{3.152893in}}%
\pgfpathlineto{\pgfqpoint{1.727024in}{3.152893in}}%
\pgfpathlineto{\pgfqpoint{1.733418in}{3.152893in}}%
\pgfpathlineto{\pgfqpoint{1.739812in}{3.152893in}}%
\pgfpathlineto{\pgfqpoint{1.746207in}{3.152893in}}%
\pgfpathlineto{\pgfqpoint{1.752601in}{3.152893in}}%
\pgfpathlineto{\pgfqpoint{1.758995in}{3.152893in}}%
\pgfpathlineto{\pgfqpoint{1.765389in}{3.152893in}}%
\pgfpathlineto{\pgfqpoint{1.771784in}{3.152893in}}%
\pgfpathlineto{\pgfqpoint{1.778178in}{3.152893in}}%
\pgfpathlineto{\pgfqpoint{1.784572in}{3.152893in}}%
\pgfpathlineto{\pgfqpoint{1.790967in}{3.152893in}}%
\pgfpathlineto{\pgfqpoint{1.797361in}{3.152893in}}%
\pgfpathlineto{\pgfqpoint{1.803755in}{3.152893in}}%
\pgfpathlineto{\pgfqpoint{1.810149in}{3.152893in}}%
\pgfpathlineto{\pgfqpoint{1.816544in}{3.152893in}}%
\pgfpathlineto{\pgfqpoint{1.822938in}{3.152893in}}%
\pgfpathlineto{\pgfqpoint{1.829332in}{3.152893in}}%
\pgfpathlineto{\pgfqpoint{1.835727in}{3.152893in}}%
\pgfpathlineto{\pgfqpoint{1.842121in}{3.152893in}}%
\pgfpathlineto{\pgfqpoint{1.848515in}{3.152893in}}%
\pgfpathlineto{\pgfqpoint{1.854909in}{3.152893in}}%
\pgfpathlineto{\pgfqpoint{1.861304in}{3.152893in}}%
\pgfpathlineto{\pgfqpoint{1.867698in}{3.152893in}}%
\pgfpathlineto{\pgfqpoint{1.874092in}{3.152893in}}%
\pgfpathlineto{\pgfqpoint{1.880486in}{3.152893in}}%
\pgfpathlineto{\pgfqpoint{1.886881in}{3.152893in}}%
\pgfpathlineto{\pgfqpoint{1.893275in}{3.152893in}}%
\pgfpathlineto{\pgfqpoint{1.899669in}{3.152893in}}%
\pgfpathlineto{\pgfqpoint{1.906064in}{3.152893in}}%
\pgfpathlineto{\pgfqpoint{1.912458in}{3.152893in}}%
\pgfpathlineto{\pgfqpoint{1.918852in}{3.152893in}}%
\pgfpathlineto{\pgfqpoint{1.925246in}{3.152893in}}%
\pgfpathlineto{\pgfqpoint{1.931641in}{3.152893in}}%
\pgfpathlineto{\pgfqpoint{1.938035in}{3.152893in}}%
\pgfpathlineto{\pgfqpoint{1.944429in}{3.152893in}}%
\pgfpathlineto{\pgfqpoint{1.950824in}{3.152893in}}%
\pgfpathlineto{\pgfqpoint{1.957218in}{3.152893in}}%
\pgfpathlineto{\pgfqpoint{1.963612in}{3.152893in}}%
\pgfpathlineto{\pgfqpoint{1.970006in}{3.152893in}}%
\pgfpathlineto{\pgfqpoint{1.976401in}{3.152893in}}%
\pgfpathlineto{\pgfqpoint{1.982795in}{3.152893in}}%
\pgfpathlineto{\pgfqpoint{1.989189in}{3.152893in}}%
\pgfpathlineto{\pgfqpoint{1.995584in}{3.152893in}}%
\pgfpathlineto{\pgfqpoint{2.001978in}{3.152893in}}%
\pgfpathlineto{\pgfqpoint{2.008372in}{3.152893in}}%
\pgfpathlineto{\pgfqpoint{2.014766in}{3.152893in}}%
\pgfpathlineto{\pgfqpoint{2.021161in}{3.152893in}}%
\pgfpathlineto{\pgfqpoint{2.027555in}{3.152893in}}%
\pgfpathlineto{\pgfqpoint{2.033949in}{3.152893in}}%
\pgfpathlineto{\pgfqpoint{2.040343in}{3.152893in}}%
\pgfpathlineto{\pgfqpoint{2.046738in}{3.152893in}}%
\pgfpathlineto{\pgfqpoint{2.053132in}{3.152893in}}%
\pgfpathlineto{\pgfqpoint{2.059526in}{3.152893in}}%
\pgfpathlineto{\pgfqpoint{2.065921in}{3.152893in}}%
\pgfpathlineto{\pgfqpoint{2.072315in}{3.152893in}}%
\pgfpathlineto{\pgfqpoint{2.078709in}{3.152893in}}%
\pgfpathlineto{\pgfqpoint{2.085103in}{3.152893in}}%
\pgfpathlineto{\pgfqpoint{2.091498in}{3.152893in}}%
\pgfpathlineto{\pgfqpoint{2.097892in}{3.152893in}}%
\pgfpathlineto{\pgfqpoint{2.104286in}{3.152893in}}%
\pgfpathlineto{\pgfqpoint{2.110681in}{3.152893in}}%
\pgfpathlineto{\pgfqpoint{2.117075in}{3.152893in}}%
\pgfpathlineto{\pgfqpoint{2.123469in}{3.152893in}}%
\pgfpathlineto{\pgfqpoint{2.129863in}{3.152893in}}%
\pgfpathlineto{\pgfqpoint{2.136258in}{3.152893in}}%
\pgfpathlineto{\pgfqpoint{2.142652in}{3.152893in}}%
\pgfpathlineto{\pgfqpoint{2.149046in}{3.152893in}}%
\pgfpathlineto{\pgfqpoint{2.155441in}{3.152893in}}%
\pgfpathlineto{\pgfqpoint{2.161835in}{3.152893in}}%
\pgfpathlineto{\pgfqpoint{2.168229in}{3.152893in}}%
\pgfpathlineto{\pgfqpoint{2.174623in}{3.152893in}}%
\pgfpathlineto{\pgfqpoint{2.181018in}{3.152893in}}%
\pgfpathlineto{\pgfqpoint{2.187412in}{3.152893in}}%
\pgfpathlineto{\pgfqpoint{2.193806in}{3.152893in}}%
\pgfpathlineto{\pgfqpoint{2.200200in}{3.152893in}}%
\pgfpathlineto{\pgfqpoint{2.206595in}{3.152893in}}%
\pgfpathlineto{\pgfqpoint{2.212989in}{3.152893in}}%
\pgfpathlineto{\pgfqpoint{2.219383in}{3.152893in}}%
\pgfpathlineto{\pgfqpoint{2.225778in}{3.152893in}}%
\pgfpathlineto{\pgfqpoint{2.232172in}{3.152893in}}%
\pgfpathlineto{\pgfqpoint{2.238566in}{3.152893in}}%
\pgfpathlineto{\pgfqpoint{2.244960in}{3.152893in}}%
\pgfpathlineto{\pgfqpoint{2.251355in}{3.152893in}}%
\pgfpathlineto{\pgfqpoint{2.257749in}{3.152893in}}%
\pgfpathlineto{\pgfqpoint{2.264143in}{3.152893in}}%
\pgfpathlineto{\pgfqpoint{2.270538in}{3.152893in}}%
\pgfpathlineto{\pgfqpoint{2.276932in}{3.152893in}}%
\pgfpathlineto{\pgfqpoint{2.283326in}{3.152893in}}%
\pgfpathlineto{\pgfqpoint{2.289720in}{3.152893in}}%
\pgfpathlineto{\pgfqpoint{2.296115in}{3.152893in}}%
\pgfpathlineto{\pgfqpoint{2.302509in}{3.152893in}}%
\pgfpathlineto{\pgfqpoint{2.308903in}{3.152893in}}%
\pgfpathlineto{\pgfqpoint{2.315298in}{3.152893in}}%
\pgfpathlineto{\pgfqpoint{2.321692in}{3.152893in}}%
\pgfpathlineto{\pgfqpoint{2.328086in}{3.152893in}}%
\pgfpathlineto{\pgfqpoint{2.334480in}{3.152893in}}%
\pgfpathlineto{\pgfqpoint{2.340875in}{3.152893in}}%
\pgfpathlineto{\pgfqpoint{2.347269in}{3.152893in}}%
\pgfpathlineto{\pgfqpoint{2.353663in}{3.152893in}}%
\pgfpathlineto{\pgfqpoint{2.360057in}{3.152893in}}%
\pgfpathlineto{\pgfqpoint{2.366452in}{3.152893in}}%
\pgfpathlineto{\pgfqpoint{2.372846in}{3.152893in}}%
\pgfpathlineto{\pgfqpoint{2.379240in}{3.152893in}}%
\pgfpathlineto{\pgfqpoint{2.385635in}{3.152893in}}%
\pgfpathlineto{\pgfqpoint{2.392029in}{3.152893in}}%
\pgfpathlineto{\pgfqpoint{2.398423in}{3.152893in}}%
\pgfpathlineto{\pgfqpoint{2.404817in}{3.152893in}}%
\pgfpathlineto{\pgfqpoint{2.411212in}{3.152893in}}%
\pgfpathlineto{\pgfqpoint{2.417606in}{3.152893in}}%
\pgfpathlineto{\pgfqpoint{2.424000in}{3.152893in}}%
\pgfpathlineto{\pgfqpoint{2.430395in}{3.152893in}}%
\pgfpathlineto{\pgfqpoint{2.436789in}{3.152893in}}%
\pgfpathlineto{\pgfqpoint{2.443183in}{3.152893in}}%
\pgfpathlineto{\pgfqpoint{2.449577in}{3.152893in}}%
\pgfpathlineto{\pgfqpoint{2.455972in}{3.152893in}}%
\pgfpathlineto{\pgfqpoint{2.462366in}{3.152893in}}%
\pgfpathlineto{\pgfqpoint{2.468760in}{3.152893in}}%
\pgfpathlineto{\pgfqpoint{2.475154in}{3.152893in}}%
\pgfpathlineto{\pgfqpoint{2.481549in}{3.152893in}}%
\pgfpathlineto{\pgfqpoint{2.487943in}{3.152893in}}%
\pgfpathlineto{\pgfqpoint{2.494337in}{3.152893in}}%
\pgfpathlineto{\pgfqpoint{2.500732in}{3.152893in}}%
\pgfpathlineto{\pgfqpoint{2.507126in}{3.152893in}}%
\pgfpathlineto{\pgfqpoint{2.513520in}{3.152893in}}%
\pgfpathlineto{\pgfqpoint{2.519914in}{3.152893in}}%
\pgfpathlineto{\pgfqpoint{2.526309in}{3.152893in}}%
\pgfpathlineto{\pgfqpoint{2.532703in}{3.152893in}}%
\pgfpathlineto{\pgfqpoint{2.539097in}{3.152893in}}%
\pgfpathlineto{\pgfqpoint{2.545492in}{3.152893in}}%
\pgfpathlineto{\pgfqpoint{2.551886in}{3.152893in}}%
\pgfpathlineto{\pgfqpoint{2.558280in}{3.152893in}}%
\pgfpathlineto{\pgfqpoint{2.564674in}{3.152893in}}%
\pgfpathlineto{\pgfqpoint{2.571069in}{3.152893in}}%
\pgfpathlineto{\pgfqpoint{2.577463in}{3.152893in}}%
\pgfpathlineto{\pgfqpoint{2.583857in}{3.152893in}}%
\pgfpathlineto{\pgfqpoint{2.590252in}{3.152893in}}%
\pgfpathlineto{\pgfqpoint{2.596646in}{3.152893in}}%
\pgfpathlineto{\pgfqpoint{2.603040in}{3.152893in}}%
\pgfpathlineto{\pgfqpoint{2.609434in}{3.152893in}}%
\pgfpathlineto{\pgfqpoint{2.615829in}{3.152893in}}%
\pgfpathlineto{\pgfqpoint{2.622223in}{3.152893in}}%
\pgfpathlineto{\pgfqpoint{2.628617in}{3.152893in}}%
\pgfpathlineto{\pgfqpoint{2.635011in}{3.152893in}}%
\pgfpathlineto{\pgfqpoint{2.641406in}{3.152893in}}%
\pgfpathlineto{\pgfqpoint{2.647800in}{3.152893in}}%
\pgfpathlineto{\pgfqpoint{2.654194in}{3.152893in}}%
\pgfpathlineto{\pgfqpoint{2.660589in}{3.152893in}}%
\pgfpathlineto{\pgfqpoint{2.666983in}{3.152893in}}%
\pgfpathlineto{\pgfqpoint{2.673377in}{3.152893in}}%
\pgfpathlineto{\pgfqpoint{2.679771in}{3.152893in}}%
\pgfpathlineto{\pgfqpoint{2.686166in}{3.152893in}}%
\pgfpathlineto{\pgfqpoint{2.692560in}{3.152893in}}%
\pgfpathlineto{\pgfqpoint{2.698954in}{3.152893in}}%
\pgfpathlineto{\pgfqpoint{2.705349in}{3.152893in}}%
\pgfpathlineto{\pgfqpoint{2.711743in}{3.152893in}}%
\pgfpathlineto{\pgfqpoint{2.718137in}{3.152893in}}%
\pgfpathlineto{\pgfqpoint{2.724531in}{3.152893in}}%
\pgfpathlineto{\pgfqpoint{2.730926in}{3.152893in}}%
\pgfpathlineto{\pgfqpoint{2.737320in}{3.152893in}}%
\pgfpathlineto{\pgfqpoint{2.743714in}{3.152893in}}%
\pgfpathlineto{\pgfqpoint{2.750109in}{3.152893in}}%
\pgfpathlineto{\pgfqpoint{2.756503in}{3.152893in}}%
\pgfpathlineto{\pgfqpoint{2.762897in}{3.152893in}}%
\pgfpathlineto{\pgfqpoint{2.769291in}{3.152893in}}%
\pgfpathlineto{\pgfqpoint{2.775686in}{3.152893in}}%
\pgfpathlineto{\pgfqpoint{2.782080in}{3.152893in}}%
\pgfpathlineto{\pgfqpoint{2.788474in}{3.152893in}}%
\pgfpathlineto{\pgfqpoint{2.794868in}{3.152893in}}%
\pgfpathlineto{\pgfqpoint{2.801263in}{3.152893in}}%
\pgfpathlineto{\pgfqpoint{2.807657in}{3.152893in}}%
\pgfpathlineto{\pgfqpoint{2.814051in}{3.152893in}}%
\pgfpathlineto{\pgfqpoint{2.820446in}{3.152893in}}%
\pgfpathlineto{\pgfqpoint{2.826840in}{3.152893in}}%
\pgfpathlineto{\pgfqpoint{2.833234in}{3.152893in}}%
\pgfpathlineto{\pgfqpoint{2.839628in}{3.152893in}}%
\pgfpathlineto{\pgfqpoint{2.846023in}{3.152893in}}%
\pgfpathlineto{\pgfqpoint{2.852417in}{3.152893in}}%
\pgfpathlineto{\pgfqpoint{2.858811in}{3.152893in}}%
\pgfpathlineto{\pgfqpoint{2.865206in}{3.152893in}}%
\pgfpathlineto{\pgfqpoint{2.871600in}{3.152893in}}%
\pgfpathlineto{\pgfqpoint{2.877994in}{3.152893in}}%
\pgfpathlineto{\pgfqpoint{2.884388in}{3.152893in}}%
\pgfpathlineto{\pgfqpoint{2.890783in}{3.152893in}}%
\pgfpathlineto{\pgfqpoint{2.897177in}{3.152893in}}%
\pgfpathlineto{\pgfqpoint{2.903571in}{3.152893in}}%
\pgfpathlineto{\pgfqpoint{2.909966in}{3.152893in}}%
\pgfpathlineto{\pgfqpoint{2.916360in}{3.152893in}}%
\pgfpathlineto{\pgfqpoint{2.922754in}{3.152893in}}%
\pgfpathlineto{\pgfqpoint{2.929148in}{3.152893in}}%
\pgfpathlineto{\pgfqpoint{2.935543in}{3.152893in}}%
\pgfpathlineto{\pgfqpoint{2.941937in}{3.152893in}}%
\pgfpathlineto{\pgfqpoint{2.948331in}{3.152893in}}%
\pgfpathlineto{\pgfqpoint{2.954725in}{3.152893in}}%
\pgfpathlineto{\pgfqpoint{2.961120in}{3.152893in}}%
\pgfpathlineto{\pgfqpoint{2.967514in}{3.152893in}}%
\pgfpathlineto{\pgfqpoint{2.973908in}{3.152893in}}%
\pgfpathlineto{\pgfqpoint{2.980303in}{3.152893in}}%
\pgfpathlineto{\pgfqpoint{2.986697in}{3.152893in}}%
\pgfpathlineto{\pgfqpoint{2.993091in}{3.152893in}}%
\pgfpathlineto{\pgfqpoint{2.999485in}{3.152893in}}%
\pgfpathlineto{\pgfqpoint{3.005880in}{3.152893in}}%
\pgfpathlineto{\pgfqpoint{3.012274in}{3.152893in}}%
\pgfpathlineto{\pgfqpoint{3.018668in}{3.152893in}}%
\pgfpathlineto{\pgfqpoint{3.025063in}{3.152893in}}%
\pgfpathlineto{\pgfqpoint{3.031457in}{3.152893in}}%
\pgfpathlineto{\pgfqpoint{3.037851in}{3.152893in}}%
\pgfpathlineto{\pgfqpoint{3.044245in}{3.152893in}}%
\pgfpathlineto{\pgfqpoint{3.050640in}{3.152893in}}%
\pgfpathlineto{\pgfqpoint{3.057034in}{3.152893in}}%
\pgfpathlineto{\pgfqpoint{3.063428in}{3.152893in}}%
\pgfpathlineto{\pgfqpoint{3.069822in}{3.152893in}}%
\pgfpathlineto{\pgfqpoint{3.076217in}{3.152893in}}%
\pgfpathlineto{\pgfqpoint{3.082611in}{3.152893in}}%
\pgfpathlineto{\pgfqpoint{3.089005in}{3.152893in}}%
\pgfpathlineto{\pgfqpoint{3.095400in}{3.152893in}}%
\pgfpathlineto{\pgfqpoint{3.101794in}{3.152893in}}%
\pgfpathlineto{\pgfqpoint{3.108188in}{3.152893in}}%
\pgfpathlineto{\pgfqpoint{3.114582in}{3.152893in}}%
\pgfpathlineto{\pgfqpoint{3.120977in}{3.152893in}}%
\pgfpathlineto{\pgfqpoint{3.127371in}{3.152893in}}%
\pgfpathlineto{\pgfqpoint{3.133765in}{3.152893in}}%
\pgfpathlineto{\pgfqpoint{3.140160in}{3.152893in}}%
\pgfpathlineto{\pgfqpoint{3.146554in}{3.152893in}}%
\pgfpathlineto{\pgfqpoint{3.152948in}{3.152893in}}%
\pgfpathlineto{\pgfqpoint{3.159342in}{3.152893in}}%
\pgfpathlineto{\pgfqpoint{3.165737in}{3.152893in}}%
\pgfpathlineto{\pgfqpoint{3.172131in}{3.152893in}}%
\pgfpathlineto{\pgfqpoint{3.178525in}{3.152893in}}%
\pgfpathlineto{\pgfqpoint{3.184920in}{3.152893in}}%
\pgfpathlineto{\pgfqpoint{3.191314in}{3.152893in}}%
\pgfpathlineto{\pgfqpoint{3.197708in}{3.152893in}}%
\pgfpathlineto{\pgfqpoint{3.204102in}{3.152893in}}%
\pgfpathlineto{\pgfqpoint{3.210497in}{3.152893in}}%
\pgfpathlineto{\pgfqpoint{3.216891in}{3.152893in}}%
\pgfpathlineto{\pgfqpoint{3.223285in}{3.152893in}}%
\pgfpathlineto{\pgfqpoint{3.229679in}{3.152893in}}%
\pgfpathlineto{\pgfqpoint{3.236074in}{3.152893in}}%
\pgfpathlineto{\pgfqpoint{3.242468in}{3.152893in}}%
\pgfpathlineto{\pgfqpoint{3.248862in}{3.152893in}}%
\pgfpathlineto{\pgfqpoint{3.255257in}{3.152893in}}%
\pgfpathlineto{\pgfqpoint{3.261651in}{3.152893in}}%
\pgfpathlineto{\pgfqpoint{3.268045in}{3.152893in}}%
\pgfpathlineto{\pgfqpoint{3.274439in}{3.152893in}}%
\pgfpathlineto{\pgfqpoint{3.280834in}{3.152893in}}%
\pgfpathlineto{\pgfqpoint{3.287228in}{3.152893in}}%
\pgfpathlineto{\pgfqpoint{3.293622in}{3.152893in}}%
\pgfpathlineto{\pgfqpoint{3.300017in}{3.152893in}}%
\pgfpathlineto{\pgfqpoint{3.306411in}{3.152893in}}%
\pgfpathlineto{\pgfqpoint{3.312805in}{3.152893in}}%
\pgfpathlineto{\pgfqpoint{3.319199in}{3.152893in}}%
\pgfpathlineto{\pgfqpoint{3.325594in}{3.152893in}}%
\pgfpathlineto{\pgfqpoint{3.331988in}{3.152893in}}%
\pgfpathlineto{\pgfqpoint{3.338382in}{3.152893in}}%
\pgfpathlineto{\pgfqpoint{3.344777in}{3.152893in}}%
\pgfpathlineto{\pgfqpoint{3.351171in}{3.152893in}}%
\pgfpathlineto{\pgfqpoint{3.357565in}{3.152893in}}%
\pgfpathlineto{\pgfqpoint{3.363959in}{3.152893in}}%
\pgfpathlineto{\pgfqpoint{3.370354in}{3.152893in}}%
\pgfpathlineto{\pgfqpoint{3.376748in}{3.152893in}}%
\pgfpathlineto{\pgfqpoint{3.383142in}{3.152893in}}%
\pgfpathlineto{\pgfqpoint{3.389536in}{3.152893in}}%
\pgfpathlineto{\pgfqpoint{3.395931in}{3.152893in}}%
\pgfpathlineto{\pgfqpoint{3.402325in}{3.152893in}}%
\pgfpathlineto{\pgfqpoint{3.408719in}{3.152893in}}%
\pgfpathlineto{\pgfqpoint{3.415114in}{3.152893in}}%
\pgfpathlineto{\pgfqpoint{3.421508in}{3.152893in}}%
\pgfpathlineto{\pgfqpoint{3.427902in}{3.152893in}}%
\pgfpathlineto{\pgfqpoint{3.434296in}{3.152893in}}%
\pgfpathlineto{\pgfqpoint{3.440691in}{3.152893in}}%
\pgfpathlineto{\pgfqpoint{3.447085in}{3.152893in}}%
\pgfpathlineto{\pgfqpoint{3.453479in}{3.152893in}}%
\pgfpathlineto{\pgfqpoint{3.459874in}{3.152893in}}%
\pgfpathlineto{\pgfqpoint{3.466268in}{3.152893in}}%
\pgfpathlineto{\pgfqpoint{3.472662in}{3.152893in}}%
\pgfpathlineto{\pgfqpoint{3.479056in}{3.152893in}}%
\pgfpathlineto{\pgfqpoint{3.485451in}{3.152893in}}%
\pgfpathlineto{\pgfqpoint{3.491845in}{3.152893in}}%
\pgfpathlineto{\pgfqpoint{3.498239in}{3.152893in}}%
\pgfpathlineto{\pgfqpoint{3.504634in}{3.152893in}}%
\pgfpathlineto{\pgfqpoint{3.511028in}{3.152893in}}%
\pgfpathlineto{\pgfqpoint{3.517422in}{3.152893in}}%
\pgfpathlineto{\pgfqpoint{3.523816in}{3.152893in}}%
\pgfpathlineto{\pgfqpoint{3.530211in}{3.152893in}}%
\pgfpathlineto{\pgfqpoint{3.536605in}{3.152893in}}%
\pgfpathlineto{\pgfqpoint{3.542999in}{3.152893in}}%
\pgfpathlineto{\pgfqpoint{3.549393in}{3.152893in}}%
\pgfpathlineto{\pgfqpoint{3.555788in}{3.152893in}}%
\pgfpathlineto{\pgfqpoint{3.562182in}{3.152893in}}%
\pgfpathlineto{\pgfqpoint{3.568576in}{3.152893in}}%
\pgfpathlineto{\pgfqpoint{3.574971in}{3.152893in}}%
\pgfpathlineto{\pgfqpoint{3.581365in}{3.152893in}}%
\pgfpathlineto{\pgfqpoint{3.587759in}{3.152893in}}%
\pgfpathlineto{\pgfqpoint{3.594153in}{3.152893in}}%
\pgfpathlineto{\pgfqpoint{3.600548in}{3.152893in}}%
\pgfpathlineto{\pgfqpoint{3.606942in}{3.152893in}}%
\pgfpathlineto{\pgfqpoint{3.613336in}{3.152893in}}%
\pgfpathlineto{\pgfqpoint{3.619731in}{3.152893in}}%
\pgfpathlineto{\pgfqpoint{3.626125in}{3.152893in}}%
\pgfpathlineto{\pgfqpoint{3.632519in}{3.152893in}}%
\pgfpathlineto{\pgfqpoint{3.638913in}{3.152893in}}%
\pgfpathlineto{\pgfqpoint{3.645308in}{3.152893in}}%
\pgfpathlineto{\pgfqpoint{3.651702in}{3.152893in}}%
\pgfpathlineto{\pgfqpoint{3.658096in}{3.152893in}}%
\pgfpathlineto{\pgfqpoint{3.664490in}{3.152893in}}%
\pgfpathlineto{\pgfqpoint{3.670885in}{3.152893in}}%
\pgfpathlineto{\pgfqpoint{3.677279in}{3.152893in}}%
\pgfpathlineto{\pgfqpoint{3.683673in}{3.152893in}}%
\pgfpathlineto{\pgfqpoint{3.690068in}{3.152893in}}%
\pgfpathlineto{\pgfqpoint{3.696462in}{3.152893in}}%
\pgfpathlineto{\pgfqpoint{3.702856in}{3.152893in}}%
\pgfpathlineto{\pgfqpoint{3.709250in}{3.152893in}}%
\pgfpathlineto{\pgfqpoint{3.715645in}{3.152893in}}%
\pgfpathlineto{\pgfqpoint{3.722039in}{3.152893in}}%
\pgfpathlineto{\pgfqpoint{3.728433in}{3.152893in}}%
\pgfpathlineto{\pgfqpoint{3.734828in}{3.152893in}}%
\pgfpathlineto{\pgfqpoint{3.741222in}{3.152893in}}%
\pgfpathlineto{\pgfqpoint{3.747616in}{3.152893in}}%
\pgfpathlineto{\pgfqpoint{3.754010in}{3.152893in}}%
\pgfpathlineto{\pgfqpoint{3.760405in}{3.152893in}}%
\pgfpathlineto{\pgfqpoint{3.766799in}{3.152893in}}%
\pgfpathlineto{\pgfqpoint{3.773193in}{3.152893in}}%
\pgfpathlineto{\pgfqpoint{3.779588in}{3.152893in}}%
\pgfpathlineto{\pgfqpoint{3.785982in}{3.152893in}}%
\pgfpathlineto{\pgfqpoint{3.792376in}{3.152893in}}%
\pgfpathlineto{\pgfqpoint{3.798770in}{3.152893in}}%
\pgfpathlineto{\pgfqpoint{3.798770in}{3.155495in}}%
\pgfpathlineto{\pgfqpoint{3.798770in}{3.155495in}}%
\pgfpathlineto{\pgfqpoint{3.792376in}{3.160698in}}%
\pgfpathlineto{\pgfqpoint{3.785982in}{3.165901in}}%
\pgfpathlineto{\pgfqpoint{3.779588in}{3.171104in}}%
\pgfpathlineto{\pgfqpoint{3.773193in}{3.176307in}}%
\pgfpathlineto{\pgfqpoint{3.766799in}{3.181510in}}%
\pgfpathlineto{\pgfqpoint{3.760405in}{3.186713in}}%
\pgfpathlineto{\pgfqpoint{3.754010in}{3.191916in}}%
\pgfpathlineto{\pgfqpoint{3.747616in}{3.197118in}}%
\pgfpathlineto{\pgfqpoint{3.741222in}{3.202321in}}%
\pgfpathlineto{\pgfqpoint{3.734828in}{3.207524in}}%
\pgfpathlineto{\pgfqpoint{3.728433in}{3.212727in}}%
\pgfpathlineto{\pgfqpoint{3.722039in}{3.217930in}}%
\pgfpathlineto{\pgfqpoint{3.715645in}{3.223133in}}%
\pgfpathlineto{\pgfqpoint{3.709250in}{3.228336in}}%
\pgfpathlineto{\pgfqpoint{3.702856in}{3.233539in}}%
\pgfpathlineto{\pgfqpoint{3.696462in}{3.238742in}}%
\pgfpathlineto{\pgfqpoint{3.690068in}{3.243945in}}%
\pgfpathlineto{\pgfqpoint{3.683673in}{3.249148in}}%
\pgfpathlineto{\pgfqpoint{3.677279in}{3.254351in}}%
\pgfpathlineto{\pgfqpoint{3.670885in}{3.259554in}}%
\pgfpathlineto{\pgfqpoint{3.664490in}{3.264757in}}%
\pgfpathlineto{\pgfqpoint{3.658096in}{3.269960in}}%
\pgfpathlineto{\pgfqpoint{3.651702in}{3.275163in}}%
\pgfpathlineto{\pgfqpoint{3.645308in}{3.280366in}}%
\pgfpathlineto{\pgfqpoint{3.638913in}{3.285569in}}%
\pgfpathlineto{\pgfqpoint{3.632519in}{3.290772in}}%
\pgfpathlineto{\pgfqpoint{3.626125in}{3.295975in}}%
\pgfpathlineto{\pgfqpoint{3.619731in}{3.301178in}}%
\pgfpathlineto{\pgfqpoint{3.613336in}{3.306381in}}%
\pgfpathlineto{\pgfqpoint{3.606942in}{3.311584in}}%
\pgfpathlineto{\pgfqpoint{3.600548in}{3.316787in}}%
\pgfpathlineto{\pgfqpoint{3.594153in}{3.321990in}}%
\pgfpathlineto{\pgfqpoint{3.587759in}{3.327193in}}%
\pgfpathlineto{\pgfqpoint{3.581365in}{3.332396in}}%
\pgfpathlineto{\pgfqpoint{3.574971in}{3.337599in}}%
\pgfpathlineto{\pgfqpoint{3.568576in}{3.342802in}}%
\pgfpathlineto{\pgfqpoint{3.562182in}{3.348004in}}%
\pgfpathlineto{\pgfqpoint{3.555788in}{3.353207in}}%
\pgfpathlineto{\pgfqpoint{3.549393in}{3.358410in}}%
\pgfpathlineto{\pgfqpoint{3.542999in}{3.363613in}}%
\pgfpathlineto{\pgfqpoint{3.536605in}{3.368816in}}%
\pgfpathlineto{\pgfqpoint{3.530211in}{3.374019in}}%
\pgfpathlineto{\pgfqpoint{3.523816in}{3.379222in}}%
\pgfpathlineto{\pgfqpoint{3.517422in}{3.384425in}}%
\pgfpathlineto{\pgfqpoint{3.511028in}{3.389628in}}%
\pgfpathlineto{\pgfqpoint{3.504634in}{3.394831in}}%
\pgfpathlineto{\pgfqpoint{3.498239in}{3.400034in}}%
\pgfpathlineto{\pgfqpoint{3.491845in}{3.405237in}}%
\pgfpathlineto{\pgfqpoint{3.485451in}{3.410440in}}%
\pgfpathlineto{\pgfqpoint{3.479056in}{3.415643in}}%
\pgfpathlineto{\pgfqpoint{3.472662in}{3.420846in}}%
\pgfpathlineto{\pgfqpoint{3.466268in}{3.426049in}}%
\pgfpathlineto{\pgfqpoint{3.459874in}{3.431252in}}%
\pgfpathlineto{\pgfqpoint{3.453479in}{3.436455in}}%
\pgfpathlineto{\pgfqpoint{3.447085in}{3.441658in}}%
\pgfpathlineto{\pgfqpoint{3.440691in}{3.446861in}}%
\pgfpathlineto{\pgfqpoint{3.434296in}{3.452064in}}%
\pgfpathlineto{\pgfqpoint{3.427902in}{3.457267in}}%
\pgfpathlineto{\pgfqpoint{3.421508in}{3.462470in}}%
\pgfpathlineto{\pgfqpoint{3.415114in}{3.467673in}}%
\pgfpathlineto{\pgfqpoint{3.408719in}{3.472876in}}%
\pgfpathlineto{\pgfqpoint{3.402325in}{3.478079in}}%
\pgfpathlineto{\pgfqpoint{3.395931in}{3.483282in}}%
\pgfpathlineto{\pgfqpoint{3.389536in}{3.488485in}}%
\pgfpathlineto{\pgfqpoint{3.383142in}{3.493688in}}%
\pgfpathlineto{\pgfqpoint{3.376748in}{3.498890in}}%
\pgfpathlineto{\pgfqpoint{3.370354in}{3.504093in}}%
\pgfpathlineto{\pgfqpoint{3.363959in}{3.509296in}}%
\pgfpathlineto{\pgfqpoint{3.357565in}{3.514499in}}%
\pgfpathlineto{\pgfqpoint{3.351171in}{3.519702in}}%
\pgfpathlineto{\pgfqpoint{3.344777in}{3.524905in}}%
\pgfpathlineto{\pgfqpoint{3.338382in}{3.530108in}}%
\pgfpathlineto{\pgfqpoint{3.331988in}{3.535311in}}%
\pgfpathlineto{\pgfqpoint{3.325594in}{3.540514in}}%
\pgfpathlineto{\pgfqpoint{3.319199in}{3.545717in}}%
\pgfpathlineto{\pgfqpoint{3.312805in}{3.550920in}}%
\pgfpathlineto{\pgfqpoint{3.306411in}{3.556123in}}%
\pgfpathlineto{\pgfqpoint{3.300017in}{3.561326in}}%
\pgfpathlineto{\pgfqpoint{3.293622in}{3.566529in}}%
\pgfpathlineto{\pgfqpoint{3.287228in}{3.571732in}}%
\pgfpathlineto{\pgfqpoint{3.280834in}{3.576935in}}%
\pgfpathlineto{\pgfqpoint{3.274439in}{3.582138in}}%
\pgfpathlineto{\pgfqpoint{3.268045in}{3.587341in}}%
\pgfpathlineto{\pgfqpoint{3.261651in}{3.592544in}}%
\pgfpathlineto{\pgfqpoint{3.255257in}{3.597747in}}%
\pgfpathlineto{\pgfqpoint{3.248862in}{3.602950in}}%
\pgfpathlineto{\pgfqpoint{3.242468in}{3.608153in}}%
\pgfpathlineto{\pgfqpoint{3.236074in}{3.613356in}}%
\pgfpathlineto{\pgfqpoint{3.229679in}{3.618559in}}%
\pgfpathlineto{\pgfqpoint{3.223285in}{3.623762in}}%
\pgfpathlineto{\pgfqpoint{3.216891in}{3.628965in}}%
\pgfpathlineto{\pgfqpoint{3.210497in}{3.634168in}}%
\pgfpathlineto{\pgfqpoint{3.204102in}{3.639371in}}%
\pgfpathlineto{\pgfqpoint{3.197708in}{3.644574in}}%
\pgfpathlineto{\pgfqpoint{3.191314in}{3.649777in}}%
\pgfpathlineto{\pgfqpoint{3.184920in}{3.654979in}}%
\pgfpathlineto{\pgfqpoint{3.178525in}{3.660182in}}%
\pgfpathlineto{\pgfqpoint{3.172131in}{3.665385in}}%
\pgfpathlineto{\pgfqpoint{3.165737in}{3.670588in}}%
\pgfpathlineto{\pgfqpoint{3.159342in}{3.675791in}}%
\pgfpathlineto{\pgfqpoint{3.152948in}{3.680994in}}%
\pgfpathlineto{\pgfqpoint{3.146554in}{3.686197in}}%
\pgfpathlineto{\pgfqpoint{3.140160in}{3.691400in}}%
\pgfpathlineto{\pgfqpoint{3.133765in}{3.696603in}}%
\pgfpathlineto{\pgfqpoint{3.127371in}{3.701806in}}%
\pgfpathlineto{\pgfqpoint{3.120977in}{3.707009in}}%
\pgfpathlineto{\pgfqpoint{3.114582in}{3.712212in}}%
\pgfpathlineto{\pgfqpoint{3.108188in}{3.717415in}}%
\pgfpathlineto{\pgfqpoint{3.101794in}{3.722618in}}%
\pgfpathlineto{\pgfqpoint{3.095400in}{3.727821in}}%
\pgfpathlineto{\pgfqpoint{3.089005in}{3.733024in}}%
\pgfpathlineto{\pgfqpoint{3.082611in}{3.738227in}}%
\pgfpathlineto{\pgfqpoint{3.076217in}{3.743430in}}%
\pgfpathlineto{\pgfqpoint{3.069822in}{3.748633in}}%
\pgfpathlineto{\pgfqpoint{3.063428in}{3.753836in}}%
\pgfpathlineto{\pgfqpoint{3.057034in}{3.759039in}}%
\pgfpathlineto{\pgfqpoint{3.050640in}{3.764242in}}%
\pgfpathlineto{\pgfqpoint{3.044245in}{3.769445in}}%
\pgfpathlineto{\pgfqpoint{3.037851in}{3.774648in}}%
\pgfpathlineto{\pgfqpoint{3.031457in}{3.779851in}}%
\pgfpathlineto{\pgfqpoint{3.025063in}{3.785054in}}%
\pgfpathlineto{\pgfqpoint{3.018668in}{3.790257in}}%
\pgfpathlineto{\pgfqpoint{3.012274in}{3.795460in}}%
\pgfpathlineto{\pgfqpoint{3.005880in}{3.800663in}}%
\pgfpathlineto{\pgfqpoint{2.999485in}{3.805865in}}%
\pgfpathlineto{\pgfqpoint{2.993091in}{3.811068in}}%
\pgfpathlineto{\pgfqpoint{2.986697in}{3.816271in}}%
\pgfpathlineto{\pgfqpoint{2.980303in}{3.821474in}}%
\pgfpathlineto{\pgfqpoint{2.973908in}{3.826677in}}%
\pgfpathlineto{\pgfqpoint{2.967514in}{3.831880in}}%
\pgfpathlineto{\pgfqpoint{2.961120in}{3.837083in}}%
\pgfpathlineto{\pgfqpoint{2.954725in}{3.842286in}}%
\pgfpathlineto{\pgfqpoint{2.948331in}{3.847489in}}%
\pgfpathlineto{\pgfqpoint{2.941937in}{3.852692in}}%
\pgfpathlineto{\pgfqpoint{2.935543in}{3.857895in}}%
\pgfpathlineto{\pgfqpoint{2.929148in}{3.863098in}}%
\pgfpathlineto{\pgfqpoint{2.922754in}{3.868301in}}%
\pgfpathlineto{\pgfqpoint{2.916360in}{3.873504in}}%
\pgfpathlineto{\pgfqpoint{2.909966in}{3.878707in}}%
\pgfpathlineto{\pgfqpoint{2.903571in}{3.883910in}}%
\pgfpathlineto{\pgfqpoint{2.897177in}{3.889113in}}%
\pgfpathlineto{\pgfqpoint{2.890783in}{3.894316in}}%
\pgfpathlineto{\pgfqpoint{2.884388in}{3.899519in}}%
\pgfpathlineto{\pgfqpoint{2.877994in}{3.904722in}}%
\pgfpathlineto{\pgfqpoint{2.871600in}{3.909925in}}%
\pgfpathlineto{\pgfqpoint{2.865206in}{3.915128in}}%
\pgfpathlineto{\pgfqpoint{2.858811in}{3.920331in}}%
\pgfpathlineto{\pgfqpoint{2.852417in}{3.925534in}}%
\pgfpathlineto{\pgfqpoint{2.846023in}{3.930737in}}%
\pgfpathlineto{\pgfqpoint{2.839628in}{3.935940in}}%
\pgfpathlineto{\pgfqpoint{2.833234in}{3.941143in}}%
\pgfpathlineto{\pgfqpoint{2.826840in}{3.946346in}}%
\pgfpathlineto{\pgfqpoint{2.820446in}{3.951549in}}%
\pgfpathlineto{\pgfqpoint{2.814051in}{3.956751in}}%
\pgfpathlineto{\pgfqpoint{2.807657in}{3.961954in}}%
\pgfpathlineto{\pgfqpoint{2.801263in}{3.967157in}}%
\pgfpathlineto{\pgfqpoint{2.794868in}{3.972360in}}%
\pgfpathlineto{\pgfqpoint{2.788474in}{3.977563in}}%
\pgfpathlineto{\pgfqpoint{2.782080in}{3.982766in}}%
\pgfpathlineto{\pgfqpoint{2.775686in}{3.987969in}}%
\pgfpathlineto{\pgfqpoint{2.769291in}{3.993172in}}%
\pgfpathlineto{\pgfqpoint{2.762897in}{3.998375in}}%
\pgfpathlineto{\pgfqpoint{2.756503in}{4.003578in}}%
\pgfpathlineto{\pgfqpoint{2.750109in}{4.008781in}}%
\pgfpathlineto{\pgfqpoint{2.743714in}{4.013984in}}%
\pgfpathlineto{\pgfqpoint{2.737320in}{4.019187in}}%
\pgfpathlineto{\pgfqpoint{2.730926in}{4.024390in}}%
\pgfpathlineto{\pgfqpoint{2.724531in}{4.029593in}}%
\pgfpathlineto{\pgfqpoint{2.718137in}{4.034796in}}%
\pgfpathlineto{\pgfqpoint{2.711743in}{4.039999in}}%
\pgfpathlineto{\pgfqpoint{2.705349in}{4.045202in}}%
\pgfpathlineto{\pgfqpoint{2.698954in}{4.050405in}}%
\pgfpathlineto{\pgfqpoint{2.692560in}{4.055608in}}%
\pgfpathlineto{\pgfqpoint{2.686166in}{4.060811in}}%
\pgfpathlineto{\pgfqpoint{2.679771in}{4.066014in}}%
\pgfpathlineto{\pgfqpoint{2.673377in}{4.071217in}}%
\pgfpathlineto{\pgfqpoint{2.666983in}{4.076420in}}%
\pgfpathlineto{\pgfqpoint{2.660589in}{4.081623in}}%
\pgfpathlineto{\pgfqpoint{2.654194in}{4.086826in}}%
\pgfpathlineto{\pgfqpoint{2.647800in}{4.092029in}}%
\pgfpathlineto{\pgfqpoint{2.641406in}{4.097232in}}%
\pgfpathlineto{\pgfqpoint{2.635011in}{4.102435in}}%
\pgfpathlineto{\pgfqpoint{2.628617in}{4.107638in}}%
\pgfpathlineto{\pgfqpoint{2.622223in}{4.112840in}}%
\pgfpathlineto{\pgfqpoint{2.615829in}{4.118043in}}%
\pgfpathlineto{\pgfqpoint{2.609434in}{4.123246in}}%
\pgfpathlineto{\pgfqpoint{2.603040in}{4.128449in}}%
\pgfpathlineto{\pgfqpoint{2.596646in}{4.133652in}}%
\pgfpathlineto{\pgfqpoint{2.590252in}{4.138855in}}%
\pgfpathlineto{\pgfqpoint{2.583857in}{4.144058in}}%
\pgfpathlineto{\pgfqpoint{2.577463in}{4.149261in}}%
\pgfpathlineto{\pgfqpoint{2.571069in}{4.154464in}}%
\pgfpathlineto{\pgfqpoint{2.564674in}{4.159667in}}%
\pgfpathlineto{\pgfqpoint{2.558280in}{4.164870in}}%
\pgfpathlineto{\pgfqpoint{2.551886in}{4.170073in}}%
\pgfpathlineto{\pgfqpoint{2.545492in}{4.175276in}}%
\pgfpathlineto{\pgfqpoint{2.539097in}{4.180479in}}%
\pgfpathlineto{\pgfqpoint{2.532703in}{4.185682in}}%
\pgfpathlineto{\pgfqpoint{2.526309in}{4.190885in}}%
\pgfpathlineto{\pgfqpoint{2.519914in}{4.196088in}}%
\pgfpathlineto{\pgfqpoint{2.513520in}{4.201291in}}%
\pgfpathlineto{\pgfqpoint{2.507126in}{4.206494in}}%
\pgfpathlineto{\pgfqpoint{2.500732in}{4.211697in}}%
\pgfpathlineto{\pgfqpoint{2.494337in}{4.216900in}}%
\pgfpathlineto{\pgfqpoint{2.487943in}{4.222103in}}%
\pgfpathlineto{\pgfqpoint{2.481549in}{4.227306in}}%
\pgfpathlineto{\pgfqpoint{2.475154in}{4.232509in}}%
\pgfpathlineto{\pgfqpoint{2.468760in}{4.237712in}}%
\pgfpathlineto{\pgfqpoint{2.462366in}{4.242915in}}%
\pgfpathlineto{\pgfqpoint{2.455972in}{4.248118in}}%
\pgfpathlineto{\pgfqpoint{2.449577in}{4.253321in}}%
\pgfpathlineto{\pgfqpoint{2.443183in}{4.258524in}}%
\pgfpathlineto{\pgfqpoint{2.436789in}{4.263726in}}%
\pgfpathlineto{\pgfqpoint{2.430395in}{4.268929in}}%
\pgfpathlineto{\pgfqpoint{2.424000in}{4.274132in}}%
\pgfpathlineto{\pgfqpoint{2.417606in}{4.279335in}}%
\pgfpathlineto{\pgfqpoint{2.411212in}{4.284538in}}%
\pgfpathlineto{\pgfqpoint{2.404817in}{4.289741in}}%
\pgfpathlineto{\pgfqpoint{2.398423in}{4.294944in}}%
\pgfpathlineto{\pgfqpoint{2.392029in}{4.300147in}}%
\pgfpathlineto{\pgfqpoint{2.385635in}{4.305350in}}%
\pgfpathlineto{\pgfqpoint{2.379240in}{4.310553in}}%
\pgfpathlineto{\pgfqpoint{2.372846in}{4.315756in}}%
\pgfpathlineto{\pgfqpoint{2.366452in}{4.320959in}}%
\pgfpathlineto{\pgfqpoint{2.360057in}{4.326162in}}%
\pgfpathlineto{\pgfqpoint{2.353663in}{4.331365in}}%
\pgfpathlineto{\pgfqpoint{2.347269in}{4.336568in}}%
\pgfpathlineto{\pgfqpoint{2.340875in}{4.341771in}}%
\pgfpathlineto{\pgfqpoint{2.334480in}{4.346974in}}%
\pgfpathlineto{\pgfqpoint{2.328086in}{4.352177in}}%
\pgfpathlineto{\pgfqpoint{2.321692in}{4.357380in}}%
\pgfpathlineto{\pgfqpoint{2.315298in}{4.362583in}}%
\pgfpathlineto{\pgfqpoint{2.308903in}{4.367786in}}%
\pgfpathlineto{\pgfqpoint{2.302509in}{4.372989in}}%
\pgfpathlineto{\pgfqpoint{2.296115in}{4.378192in}}%
\pgfpathlineto{\pgfqpoint{2.289720in}{4.383395in}}%
\pgfpathlineto{\pgfqpoint{2.283326in}{4.388598in}}%
\pgfpathlineto{\pgfqpoint{2.276932in}{4.393801in}}%
\pgfpathlineto{\pgfqpoint{2.270538in}{4.399004in}}%
\pgfpathlineto{\pgfqpoint{2.264143in}{4.404207in}}%
\pgfpathlineto{\pgfqpoint{2.257749in}{4.409410in}}%
\pgfpathlineto{\pgfqpoint{2.251355in}{4.414612in}}%
\pgfpathlineto{\pgfqpoint{2.244960in}{4.419815in}}%
\pgfpathlineto{\pgfqpoint{2.238566in}{4.425018in}}%
\pgfpathlineto{\pgfqpoint{2.232172in}{4.430221in}}%
\pgfpathlineto{\pgfqpoint{2.225778in}{4.435424in}}%
\pgfpathlineto{\pgfqpoint{2.219383in}{4.440627in}}%
\pgfpathlineto{\pgfqpoint{2.212989in}{4.445830in}}%
\pgfpathlineto{\pgfqpoint{2.206595in}{4.451033in}}%
\pgfpathlineto{\pgfqpoint{2.200200in}{4.456236in}}%
\pgfpathlineto{\pgfqpoint{2.193806in}{4.461439in}}%
\pgfpathlineto{\pgfqpoint{2.187412in}{4.466642in}}%
\pgfpathlineto{\pgfqpoint{2.181018in}{4.471845in}}%
\pgfpathlineto{\pgfqpoint{2.174623in}{4.477048in}}%
\pgfpathlineto{\pgfqpoint{2.168229in}{4.482251in}}%
\pgfpathlineto{\pgfqpoint{2.161835in}{4.487454in}}%
\pgfpathlineto{\pgfqpoint{2.155441in}{4.492657in}}%
\pgfpathlineto{\pgfqpoint{2.149046in}{4.497860in}}%
\pgfpathlineto{\pgfqpoint{2.142652in}{4.503063in}}%
\pgfpathlineto{\pgfqpoint{2.136258in}{4.508266in}}%
\pgfpathlineto{\pgfqpoint{2.129863in}{4.513469in}}%
\pgfpathlineto{\pgfqpoint{2.123469in}{4.518672in}}%
\pgfpathlineto{\pgfqpoint{2.117075in}{4.523875in}}%
\pgfpathlineto{\pgfqpoint{2.110681in}{4.529078in}}%
\pgfpathlineto{\pgfqpoint{2.104286in}{4.534281in}}%
\pgfpathlineto{\pgfqpoint{2.097892in}{4.539484in}}%
\pgfpathlineto{\pgfqpoint{2.091498in}{4.544687in}}%
\pgfpathlineto{\pgfqpoint{2.085103in}{4.549890in}}%
\pgfpathlineto{\pgfqpoint{2.078709in}{4.555093in}}%
\pgfpathlineto{\pgfqpoint{2.072315in}{4.560296in}}%
\pgfpathlineto{\pgfqpoint{2.065921in}{4.565499in}}%
\pgfpathlineto{\pgfqpoint{2.059526in}{4.570701in}}%
\pgfpathlineto{\pgfqpoint{2.053132in}{4.575904in}}%
\pgfpathlineto{\pgfqpoint{2.046738in}{4.581107in}}%
\pgfpathlineto{\pgfqpoint{2.040343in}{4.586310in}}%
\pgfpathlineto{\pgfqpoint{2.033949in}{4.591513in}}%
\pgfpathlineto{\pgfqpoint{2.027555in}{4.596716in}}%
\pgfpathlineto{\pgfqpoint{2.021161in}{4.601919in}}%
\pgfpathlineto{\pgfqpoint{2.014766in}{4.607122in}}%
\pgfpathlineto{\pgfqpoint{2.008372in}{4.612325in}}%
\pgfpathlineto{\pgfqpoint{2.001978in}{4.617528in}}%
\pgfpathlineto{\pgfqpoint{1.995584in}{4.622731in}}%
\pgfpathlineto{\pgfqpoint{1.989189in}{4.627934in}}%
\pgfpathlineto{\pgfqpoint{1.982795in}{4.633137in}}%
\pgfpathlineto{\pgfqpoint{1.976401in}{4.638340in}}%
\pgfpathlineto{\pgfqpoint{1.970006in}{4.643543in}}%
\pgfpathlineto{\pgfqpoint{1.963612in}{4.648746in}}%
\pgfpathlineto{\pgfqpoint{1.957218in}{4.653949in}}%
\pgfpathlineto{\pgfqpoint{1.950824in}{4.659152in}}%
\pgfpathlineto{\pgfqpoint{1.944429in}{4.664355in}}%
\pgfpathlineto{\pgfqpoint{1.938035in}{4.669558in}}%
\pgfpathlineto{\pgfqpoint{1.931641in}{4.674761in}}%
\pgfpathlineto{\pgfqpoint{1.925246in}{4.679964in}}%
\pgfpathlineto{\pgfqpoint{1.918852in}{4.685167in}}%
\pgfpathlineto{\pgfqpoint{1.912458in}{4.690370in}}%
\pgfpathlineto{\pgfqpoint{1.906064in}{4.695573in}}%
\pgfpathlineto{\pgfqpoint{1.899669in}{4.700776in}}%
\pgfpathlineto{\pgfqpoint{1.893275in}{4.705979in}}%
\pgfpathlineto{\pgfqpoint{1.886881in}{4.711182in}}%
\pgfpathlineto{\pgfqpoint{1.880486in}{4.716385in}}%
\pgfpathlineto{\pgfqpoint{1.874092in}{4.721587in}}%
\pgfpathlineto{\pgfqpoint{1.867698in}{4.726790in}}%
\pgfpathlineto{\pgfqpoint{1.861304in}{4.731993in}}%
\pgfpathlineto{\pgfqpoint{1.854909in}{4.737196in}}%
\pgfpathlineto{\pgfqpoint{1.848515in}{4.742399in}}%
\pgfpathlineto{\pgfqpoint{1.842121in}{4.747602in}}%
\pgfpathlineto{\pgfqpoint{1.835727in}{4.752805in}}%
\pgfpathlineto{\pgfqpoint{1.829332in}{4.758008in}}%
\pgfpathlineto{\pgfqpoint{1.822938in}{4.763211in}}%
\pgfpathlineto{\pgfqpoint{1.816544in}{4.768414in}}%
\pgfpathlineto{\pgfqpoint{1.810149in}{4.773617in}}%
\pgfpathlineto{\pgfqpoint{1.803755in}{4.778820in}}%
\pgfpathlineto{\pgfqpoint{1.797361in}{4.784023in}}%
\pgfpathlineto{\pgfqpoint{1.790967in}{4.789226in}}%
\pgfpathlineto{\pgfqpoint{1.784572in}{4.794429in}}%
\pgfpathlineto{\pgfqpoint{1.778178in}{4.799632in}}%
\pgfpathlineto{\pgfqpoint{1.771784in}{4.804835in}}%
\pgfpathlineto{\pgfqpoint{1.765389in}{4.810038in}}%
\pgfpathlineto{\pgfqpoint{1.758995in}{4.815241in}}%
\pgfpathlineto{\pgfqpoint{1.752601in}{4.820444in}}%
\pgfpathlineto{\pgfqpoint{1.746207in}{4.825647in}}%
\pgfpathlineto{\pgfqpoint{1.739812in}{4.830850in}}%
\pgfpathlineto{\pgfqpoint{1.733418in}{4.836053in}}%
\pgfpathlineto{\pgfqpoint{1.727024in}{4.841256in}}%
\pgfpathlineto{\pgfqpoint{1.720630in}{4.846459in}}%
\pgfpathlineto{\pgfqpoint{1.714235in}{4.851662in}}%
\pgfpathlineto{\pgfqpoint{1.707841in}{4.856865in}}%
\pgfpathlineto{\pgfqpoint{1.701447in}{4.862068in}}%
\pgfpathlineto{\pgfqpoint{1.695052in}{4.867271in}}%
\pgfpathlineto{\pgfqpoint{1.688658in}{4.872473in}}%
\pgfpathlineto{\pgfqpoint{1.682264in}{4.877676in}}%
\pgfpathlineto{\pgfqpoint{1.675870in}{4.882879in}}%
\pgfpathlineto{\pgfqpoint{1.669475in}{4.888082in}}%
\pgfpathlineto{\pgfqpoint{1.663081in}{4.893285in}}%
\pgfpathlineto{\pgfqpoint{1.656687in}{4.898488in}}%
\pgfpathlineto{\pgfqpoint{1.650292in}{4.903691in}}%
\pgfpathlineto{\pgfqpoint{1.643898in}{4.908894in}}%
\pgfpathlineto{\pgfqpoint{1.637504in}{4.914097in}}%
\pgfpathlineto{\pgfqpoint{1.631110in}{4.919300in}}%
\pgfpathlineto{\pgfqpoint{1.624715in}{4.924503in}}%
\pgfpathlineto{\pgfqpoint{1.618321in}{4.929706in}}%
\pgfpathlineto{\pgfqpoint{1.611927in}{4.934909in}}%
\pgfpathlineto{\pgfqpoint{1.605532in}{4.940112in}}%
\pgfpathlineto{\pgfqpoint{1.599138in}{4.945315in}}%
\pgfpathlineto{\pgfqpoint{1.592744in}{4.950518in}}%
\pgfpathlineto{\pgfqpoint{1.586350in}{4.955721in}}%
\pgfpathlineto{\pgfqpoint{1.579955in}{4.960924in}}%
\pgfpathlineto{\pgfqpoint{1.573561in}{4.966127in}}%
\pgfpathlineto{\pgfqpoint{1.567167in}{4.971330in}}%
\pgfpathlineto{\pgfqpoint{1.560773in}{4.976533in}}%
\pgfpathlineto{\pgfqpoint{1.554378in}{4.981736in}}%
\pgfpathlineto{\pgfqpoint{1.547984in}{4.986939in}}%
\pgfpathlineto{\pgfqpoint{1.541590in}{4.992142in}}%
\pgfpathlineto{\pgfqpoint{1.535195in}{4.997345in}}%
\pgfpathlineto{\pgfqpoint{1.528801in}{5.002548in}}%
\pgfpathlineto{\pgfqpoint{1.522407in}{5.007751in}}%
\pgfpathlineto{\pgfqpoint{1.516013in}{5.012954in}}%
\pgfpathlineto{\pgfqpoint{1.509618in}{5.018157in}}%
\pgfpathlineto{\pgfqpoint{1.503224in}{5.023359in}}%
\pgfpathlineto{\pgfqpoint{1.496830in}{5.028562in}}%
\pgfpathlineto{\pgfqpoint{1.490435in}{5.033765in}}%
\pgfpathlineto{\pgfqpoint{1.484041in}{5.038968in}}%
\pgfpathlineto{\pgfqpoint{1.477647in}{5.044171in}}%
\pgfpathlineto{\pgfqpoint{1.471253in}{5.049374in}}%
\pgfpathlineto{\pgfqpoint{1.464858in}{5.054577in}}%
\pgfpathlineto{\pgfqpoint{1.458464in}{5.059780in}}%
\pgfpathlineto{\pgfqpoint{1.452070in}{5.064983in}}%
\pgfpathlineto{\pgfqpoint{1.445675in}{5.070186in}}%
\pgfpathlineto{\pgfqpoint{1.439281in}{5.075389in}}%
\pgfpathlineto{\pgfqpoint{1.432887in}{5.080592in}}%
\pgfpathlineto{\pgfqpoint{1.426493in}{5.085795in}}%
\pgfpathlineto{\pgfqpoint{1.420098in}{5.090998in}}%
\pgfpathlineto{\pgfqpoint{1.413704in}{5.096201in}}%
\pgfpathlineto{\pgfqpoint{1.407310in}{5.101404in}}%
\pgfpathlineto{\pgfqpoint{1.400916in}{5.106607in}}%
\pgfpathlineto{\pgfqpoint{1.394521in}{5.111810in}}%
\pgfpathlineto{\pgfqpoint{1.388127in}{5.117013in}}%
\pgfpathlineto{\pgfqpoint{1.381733in}{5.122216in}}%
\pgfpathlineto{\pgfqpoint{1.375338in}{5.127419in}}%
\pgfpathlineto{\pgfqpoint{1.368944in}{5.132622in}}%
\pgfpathlineto{\pgfqpoint{1.362550in}{5.137825in}}%
\pgfpathlineto{\pgfqpoint{1.356156in}{5.143028in}}%
\pgfpathlineto{\pgfqpoint{1.349761in}{5.148231in}}%
\pgfpathlineto{\pgfqpoint{1.343367in}{5.153434in}}%
\pgfpathlineto{\pgfqpoint{1.336973in}{5.158637in}}%
\pgfpathlineto{\pgfqpoint{1.330578in}{5.163840in}}%
\pgfpathlineto{\pgfqpoint{1.324184in}{5.169043in}}%
\pgfpathlineto{\pgfqpoint{1.317790in}{5.174246in}}%
\pgfpathlineto{\pgfqpoint{1.311396in}{5.179448in}}%
\pgfpathlineto{\pgfqpoint{1.305001in}{5.184651in}}%
\pgfpathlineto{\pgfqpoint{1.298607in}{5.189854in}}%
\pgfpathlineto{\pgfqpoint{1.292213in}{5.195057in}}%
\pgfpathlineto{\pgfqpoint{1.285818in}{5.200260in}}%
\pgfpathlineto{\pgfqpoint{1.279424in}{5.205463in}}%
\pgfpathlineto{\pgfqpoint{1.273030in}{5.210666in}}%
\pgfpathlineto{\pgfqpoint{1.266636in}{5.215869in}}%
\pgfpathlineto{\pgfqpoint{1.260241in}{5.221072in}}%
\pgfpathlineto{\pgfqpoint{1.253847in}{5.226275in}}%
\pgfpathlineto{\pgfqpoint{1.247453in}{5.231478in}}%
\pgfpathlineto{\pgfqpoint{1.241059in}{5.236681in}}%
\pgfpathlineto{\pgfqpoint{1.234664in}{5.241884in}}%
\pgfpathlineto{\pgfqpoint{1.228270in}{5.247087in}}%
\pgfpathlineto{\pgfqpoint{1.221876in}{5.252290in}}%
\pgfpathlineto{\pgfqpoint{1.215481in}{5.257493in}}%
\pgfpathlineto{\pgfqpoint{1.209087in}{5.262696in}}%
\pgfpathlineto{\pgfqpoint{1.202693in}{5.267899in}}%
\pgfpathlineto{\pgfqpoint{1.196299in}{5.273102in}}%
\pgfpathlineto{\pgfqpoint{1.189904in}{5.278305in}}%
\pgfpathlineto{\pgfqpoint{1.183510in}{5.283508in}}%
\pgfpathlineto{\pgfqpoint{1.177116in}{5.288711in}}%
\pgfpathlineto{\pgfqpoint{1.170721in}{5.293914in}}%
\pgfpathlineto{\pgfqpoint{1.164327in}{5.299117in}}%
\pgfpathlineto{\pgfqpoint{1.157933in}{5.304320in}}%
\pgfpathlineto{\pgfqpoint{1.151539in}{5.309523in}}%
\pgfpathlineto{\pgfqpoint{1.145144in}{5.314726in}}%
\pgfpathlineto{\pgfqpoint{1.138750in}{5.319929in}}%
\pgfpathlineto{\pgfqpoint{1.132356in}{5.325132in}}%
\pgfpathlineto{\pgfqpoint{1.125962in}{5.330334in}}%
\pgfpathlineto{\pgfqpoint{1.119567in}{5.335537in}}%
\pgfpathlineto{\pgfqpoint{1.113173in}{5.340740in}}%
\pgfpathlineto{\pgfqpoint{1.106779in}{5.345943in}}%
\pgfpathlineto{\pgfqpoint{1.100384in}{5.351146in}}%
\pgfpathlineto{\pgfqpoint{1.093990in}{5.356349in}}%
\pgfpathlineto{\pgfqpoint{1.087596in}{5.361552in}}%
\pgfpathlineto{\pgfqpoint{1.081202in}{5.366755in}}%
\pgfpathlineto{\pgfqpoint{1.074807in}{5.371958in}}%
\pgfpathlineto{\pgfqpoint{1.068413in}{5.377161in}}%
\pgfpathlineto{\pgfqpoint{1.062019in}{5.382364in}}%
\pgfpathlineto{\pgfqpoint{1.055624in}{5.387567in}}%
\pgfpathlineto{\pgfqpoint{1.049230in}{5.392770in}}%
\pgfpathlineto{\pgfqpoint{1.042836in}{5.397973in}}%
\pgfpathlineto{\pgfqpoint{1.036442in}{5.403176in}}%
\pgfpathlineto{\pgfqpoint{1.030047in}{5.408379in}}%
\pgfpathlineto{\pgfqpoint{1.023653in}{5.413582in}}%
\pgfpathlineto{\pgfqpoint{1.017259in}{5.418785in}}%
\pgfpathlineto{\pgfqpoint{1.010864in}{5.423988in}}%
\pgfpathlineto{\pgfqpoint{1.004470in}{5.429191in}}%
\pgfpathlineto{\pgfqpoint{0.998076in}{5.434394in}}%
\pgfpathlineto{\pgfqpoint{0.991682in}{5.439597in}}%
\pgfpathlineto{\pgfqpoint{0.985287in}{5.444800in}}%
\pgfpathlineto{\pgfqpoint{0.978893in}{5.450003in}}%
\pgfpathlineto{\pgfqpoint{0.972499in}{5.455206in}}%
\pgfpathlineto{\pgfqpoint{0.966105in}{5.460409in}}%
\pgfpathlineto{\pgfqpoint{0.959710in}{5.465612in}}%
\pgfpathlineto{\pgfqpoint{0.953316in}{5.470815in}}%
\pgfpathlineto{\pgfqpoint{0.946922in}{5.476018in}}%
\pgfpathlineto{\pgfqpoint{0.940527in}{5.481220in}}%
\pgfpathlineto{\pgfqpoint{0.934133in}{5.486423in}}%
\pgfpathlineto{\pgfqpoint{0.927739in}{5.491626in}}%
\pgfpathlineto{\pgfqpoint{0.921345in}{5.496829in}}%
\pgfpathlineto{\pgfqpoint{0.914950in}{5.502032in}}%
\pgfpathlineto{\pgfqpoint{0.908556in}{5.507235in}}%
\pgfpathlineto{\pgfqpoint{0.902162in}{5.512438in}}%
\pgfpathlineto{\pgfqpoint{0.895767in}{5.517641in}}%
\pgfpathlineto{\pgfqpoint{0.889373in}{5.522844in}}%
\pgfpathlineto{\pgfqpoint{0.882979in}{5.528047in}}%
\pgfpathlineto{\pgfqpoint{0.876585in}{5.533250in}}%
\pgfpathlineto{\pgfqpoint{0.870190in}{5.538453in}}%
\pgfpathlineto{\pgfqpoint{0.863796in}{5.543656in}}%
\pgfpathlineto{\pgfqpoint{0.857402in}{5.548859in}}%
\pgfpathlineto{\pgfqpoint{0.851007in}{5.554062in}}%
\pgfpathlineto{\pgfqpoint{0.844613in}{5.559265in}}%
\pgfpathlineto{\pgfqpoint{0.838219in}{5.564468in}}%
\pgfpathlineto{\pgfqpoint{0.831825in}{5.569671in}}%
\pgfpathlineto{\pgfqpoint{0.825430in}{5.574874in}}%
\pgfpathlineto{\pgfqpoint{0.819036in}{5.580077in}}%
\pgfpathlineto{\pgfqpoint{0.812642in}{5.585280in}}%
\pgfpathlineto{\pgfqpoint{0.806248in}{5.590483in}}%
\pgfpathlineto{\pgfqpoint{0.799853in}{5.595686in}}%
\pgfpathlineto{\pgfqpoint{0.793459in}{5.600889in}}%
\pgfpathlineto{\pgfqpoint{0.787065in}{5.606092in}}%
\pgfpathlineto{\pgfqpoint{0.780670in}{5.611295in}}%
\pgfpathlineto{\pgfqpoint{0.774276in}{5.616498in}}%
\pgfpathlineto{\pgfqpoint{0.767882in}{5.621701in}}%
\pgfpathlineto{\pgfqpoint{0.761488in}{5.626904in}}%
\pgfpathlineto{\pgfqpoint{0.755093in}{5.632107in}}%
\pgfpathlineto{\pgfqpoint{0.748699in}{5.637309in}}%
\pgfpathlineto{\pgfqpoint{0.742305in}{5.642512in}}%
\pgfpathlineto{\pgfqpoint{0.735910in}{5.647715in}}%
\pgfpathlineto{\pgfqpoint{0.729516in}{5.652918in}}%
\pgfpathlineto{\pgfqpoint{0.723122in}{5.658121in}}%
\pgfpathlineto{\pgfqpoint{0.716728in}{5.663324in}}%
\pgfpathlineto{\pgfqpoint{0.710333in}{5.668527in}}%
\pgfpathlineto{\pgfqpoint{0.703939in}{5.673730in}}%
\pgfpathlineto{\pgfqpoint{0.697545in}{5.678933in}}%
\pgfpathlineto{\pgfqpoint{0.691150in}{5.684136in}}%
\pgfpathlineto{\pgfqpoint{0.684756in}{5.689339in}}%
\pgfpathlineto{\pgfqpoint{0.678362in}{5.694542in}}%
\pgfpathlineto{\pgfqpoint{0.671968in}{5.699745in}}%
\pgfpathlineto{\pgfqpoint{0.665573in}{5.704948in}}%
\pgfpathlineto{\pgfqpoint{0.659179in}{5.710151in}}%
\pgfpathlineto{\pgfqpoint{0.652785in}{5.715354in}}%
\pgfpathlineto{\pgfqpoint{0.646391in}{5.720557in}}%
\pgfpathlineto{\pgfqpoint{0.639996in}{5.725760in}}%
\pgfpathlineto{\pgfqpoint{0.633602in}{5.730963in}}%
\pgfpathlineto{\pgfqpoint{0.627208in}{5.736166in}}%
\pgfpathlineto{\pgfqpoint{0.620813in}{5.741369in}}%
\pgfpathlineto{\pgfqpoint{0.614419in}{5.746572in}}%
\pgfpathlineto{\pgfqpoint{0.608025in}{5.751775in}}%
\pgfpathlineto{\pgfqpoint{0.608025in}{5.751775in}}%
\pgfpathclose%
\pgfusepath{fill}%
\end{pgfscope}%
\begin{pgfscope}%
\pgfpathrectangle{\pgfqpoint{0.608025in}{0.554012in}}{\pgfqpoint{6.387885in}{5.197763in}}%
\pgfusepath{clip}%
\pgfsetbuttcap%
\pgfsetroundjoin%
\definecolor{currentfill}{rgb}{1.000000,0.498039,0.054902}%
\pgfsetfillcolor{currentfill}%
\pgfsetfillopacity{0.200000}%
\pgfsetlinewidth{1.003750pt}%
\definecolor{currentstroke}{rgb}{1.000000,0.498039,0.054902}%
\pgfsetstrokecolor{currentstroke}%
\pgfsetstrokeopacity{0.200000}%
\pgfsetdash{}{0pt}%
\pgfsys@defobject{currentmarker}{\pgfqpoint{0.608025in}{0.554012in}}{\pgfqpoint{3.798770in}{3.152893in}}{%
\pgfpathmoveto{\pgfqpoint{0.608025in}{3.152893in}}%
\pgfpathlineto{\pgfqpoint{0.608025in}{0.554012in}}%
\pgfpathlineto{\pgfqpoint{0.614419in}{0.559215in}}%
\pgfpathlineto{\pgfqpoint{0.620813in}{0.564418in}}%
\pgfpathlineto{\pgfqpoint{0.627208in}{0.569621in}}%
\pgfpathlineto{\pgfqpoint{0.633602in}{0.574824in}}%
\pgfpathlineto{\pgfqpoint{0.639996in}{0.580027in}}%
\pgfpathlineto{\pgfqpoint{0.646391in}{0.585230in}}%
\pgfpathlineto{\pgfqpoint{0.652785in}{0.590433in}}%
\pgfpathlineto{\pgfqpoint{0.659179in}{0.595636in}}%
\pgfpathlineto{\pgfqpoint{0.665573in}{0.600838in}}%
\pgfpathlineto{\pgfqpoint{0.671968in}{0.606041in}}%
\pgfpathlineto{\pgfqpoint{0.678362in}{0.611244in}}%
\pgfpathlineto{\pgfqpoint{0.684756in}{0.616447in}}%
\pgfpathlineto{\pgfqpoint{0.691150in}{0.621650in}}%
\pgfpathlineto{\pgfqpoint{0.697545in}{0.626853in}}%
\pgfpathlineto{\pgfqpoint{0.703939in}{0.632056in}}%
\pgfpathlineto{\pgfqpoint{0.710333in}{0.637259in}}%
\pgfpathlineto{\pgfqpoint{0.716728in}{0.642462in}}%
\pgfpathlineto{\pgfqpoint{0.723122in}{0.647665in}}%
\pgfpathlineto{\pgfqpoint{0.729516in}{0.652868in}}%
\pgfpathlineto{\pgfqpoint{0.735910in}{0.658071in}}%
\pgfpathlineto{\pgfqpoint{0.742305in}{0.663274in}}%
\pgfpathlineto{\pgfqpoint{0.748699in}{0.668477in}}%
\pgfpathlineto{\pgfqpoint{0.755093in}{0.673680in}}%
\pgfpathlineto{\pgfqpoint{0.761488in}{0.678883in}}%
\pgfpathlineto{\pgfqpoint{0.767882in}{0.684086in}}%
\pgfpathlineto{\pgfqpoint{0.774276in}{0.689289in}}%
\pgfpathlineto{\pgfqpoint{0.780670in}{0.694492in}}%
\pgfpathlineto{\pgfqpoint{0.787065in}{0.699695in}}%
\pgfpathlineto{\pgfqpoint{0.793459in}{0.704898in}}%
\pgfpathlineto{\pgfqpoint{0.799853in}{0.710101in}}%
\pgfpathlineto{\pgfqpoint{0.806248in}{0.715304in}}%
\pgfpathlineto{\pgfqpoint{0.812642in}{0.720507in}}%
\pgfpathlineto{\pgfqpoint{0.819036in}{0.725710in}}%
\pgfpathlineto{\pgfqpoint{0.825430in}{0.730913in}}%
\pgfpathlineto{\pgfqpoint{0.831825in}{0.736116in}}%
\pgfpathlineto{\pgfqpoint{0.838219in}{0.741319in}}%
\pgfpathlineto{\pgfqpoint{0.844613in}{0.746522in}}%
\pgfpathlineto{\pgfqpoint{0.851007in}{0.751725in}}%
\pgfpathlineto{\pgfqpoint{0.857402in}{0.756927in}}%
\pgfpathlineto{\pgfqpoint{0.863796in}{0.762130in}}%
\pgfpathlineto{\pgfqpoint{0.870190in}{0.767333in}}%
\pgfpathlineto{\pgfqpoint{0.876585in}{0.772536in}}%
\pgfpathlineto{\pgfqpoint{0.882979in}{0.777739in}}%
\pgfpathlineto{\pgfqpoint{0.889373in}{0.782942in}}%
\pgfpathlineto{\pgfqpoint{0.895767in}{0.788145in}}%
\pgfpathlineto{\pgfqpoint{0.902162in}{0.793348in}}%
\pgfpathlineto{\pgfqpoint{0.908556in}{0.798551in}}%
\pgfpathlineto{\pgfqpoint{0.914950in}{0.803754in}}%
\pgfpathlineto{\pgfqpoint{0.921345in}{0.808957in}}%
\pgfpathlineto{\pgfqpoint{0.927739in}{0.814160in}}%
\pgfpathlineto{\pgfqpoint{0.934133in}{0.819363in}}%
\pgfpathlineto{\pgfqpoint{0.940527in}{0.824566in}}%
\pgfpathlineto{\pgfqpoint{0.946922in}{0.829769in}}%
\pgfpathlineto{\pgfqpoint{0.953316in}{0.834972in}}%
\pgfpathlineto{\pgfqpoint{0.959710in}{0.840175in}}%
\pgfpathlineto{\pgfqpoint{0.966105in}{0.845378in}}%
\pgfpathlineto{\pgfqpoint{0.972499in}{0.850581in}}%
\pgfpathlineto{\pgfqpoint{0.978893in}{0.855784in}}%
\pgfpathlineto{\pgfqpoint{0.985287in}{0.860987in}}%
\pgfpathlineto{\pgfqpoint{0.991682in}{0.866190in}}%
\pgfpathlineto{\pgfqpoint{0.998076in}{0.871393in}}%
\pgfpathlineto{\pgfqpoint{1.004470in}{0.876596in}}%
\pgfpathlineto{\pgfqpoint{1.010864in}{0.881799in}}%
\pgfpathlineto{\pgfqpoint{1.017259in}{0.887002in}}%
\pgfpathlineto{\pgfqpoint{1.023653in}{0.892205in}}%
\pgfpathlineto{\pgfqpoint{1.030047in}{0.897408in}}%
\pgfpathlineto{\pgfqpoint{1.036442in}{0.902611in}}%
\pgfpathlineto{\pgfqpoint{1.042836in}{0.907813in}}%
\pgfpathlineto{\pgfqpoint{1.049230in}{0.913016in}}%
\pgfpathlineto{\pgfqpoint{1.055624in}{0.918219in}}%
\pgfpathlineto{\pgfqpoint{1.062019in}{0.923422in}}%
\pgfpathlineto{\pgfqpoint{1.068413in}{0.928625in}}%
\pgfpathlineto{\pgfqpoint{1.074807in}{0.933828in}}%
\pgfpathlineto{\pgfqpoint{1.081202in}{0.939031in}}%
\pgfpathlineto{\pgfqpoint{1.087596in}{0.944234in}}%
\pgfpathlineto{\pgfqpoint{1.093990in}{0.949437in}}%
\pgfpathlineto{\pgfqpoint{1.100384in}{0.954640in}}%
\pgfpathlineto{\pgfqpoint{1.106779in}{0.959843in}}%
\pgfpathlineto{\pgfqpoint{1.113173in}{0.965046in}}%
\pgfpathlineto{\pgfqpoint{1.119567in}{0.970249in}}%
\pgfpathlineto{\pgfqpoint{1.125962in}{0.975452in}}%
\pgfpathlineto{\pgfqpoint{1.132356in}{0.980655in}}%
\pgfpathlineto{\pgfqpoint{1.138750in}{0.985858in}}%
\pgfpathlineto{\pgfqpoint{1.145144in}{0.991061in}}%
\pgfpathlineto{\pgfqpoint{1.151539in}{0.996264in}}%
\pgfpathlineto{\pgfqpoint{1.157933in}{1.001467in}}%
\pgfpathlineto{\pgfqpoint{1.164327in}{1.006670in}}%
\pgfpathlineto{\pgfqpoint{1.170721in}{1.011873in}}%
\pgfpathlineto{\pgfqpoint{1.177116in}{1.017076in}}%
\pgfpathlineto{\pgfqpoint{1.183510in}{1.022279in}}%
\pgfpathlineto{\pgfqpoint{1.189904in}{1.027482in}}%
\pgfpathlineto{\pgfqpoint{1.196299in}{1.032685in}}%
\pgfpathlineto{\pgfqpoint{1.202693in}{1.037888in}}%
\pgfpathlineto{\pgfqpoint{1.209087in}{1.043091in}}%
\pgfpathlineto{\pgfqpoint{1.215481in}{1.048294in}}%
\pgfpathlineto{\pgfqpoint{1.221876in}{1.053497in}}%
\pgfpathlineto{\pgfqpoint{1.228270in}{1.058699in}}%
\pgfpathlineto{\pgfqpoint{1.234664in}{1.063902in}}%
\pgfpathlineto{\pgfqpoint{1.241059in}{1.069105in}}%
\pgfpathlineto{\pgfqpoint{1.247453in}{1.074308in}}%
\pgfpathlineto{\pgfqpoint{1.253847in}{1.079511in}}%
\pgfpathlineto{\pgfqpoint{1.260241in}{1.084714in}}%
\pgfpathlineto{\pgfqpoint{1.266636in}{1.089917in}}%
\pgfpathlineto{\pgfqpoint{1.273030in}{1.095120in}}%
\pgfpathlineto{\pgfqpoint{1.279424in}{1.100323in}}%
\pgfpathlineto{\pgfqpoint{1.285818in}{1.105526in}}%
\pgfpathlineto{\pgfqpoint{1.292213in}{1.110729in}}%
\pgfpathlineto{\pgfqpoint{1.298607in}{1.115932in}}%
\pgfpathlineto{\pgfqpoint{1.305001in}{1.121135in}}%
\pgfpathlineto{\pgfqpoint{1.311396in}{1.126338in}}%
\pgfpathlineto{\pgfqpoint{1.317790in}{1.131541in}}%
\pgfpathlineto{\pgfqpoint{1.324184in}{1.136744in}}%
\pgfpathlineto{\pgfqpoint{1.330578in}{1.141947in}}%
\pgfpathlineto{\pgfqpoint{1.336973in}{1.147150in}}%
\pgfpathlineto{\pgfqpoint{1.343367in}{1.152353in}}%
\pgfpathlineto{\pgfqpoint{1.349761in}{1.157556in}}%
\pgfpathlineto{\pgfqpoint{1.356156in}{1.162759in}}%
\pgfpathlineto{\pgfqpoint{1.362550in}{1.167962in}}%
\pgfpathlineto{\pgfqpoint{1.368944in}{1.173165in}}%
\pgfpathlineto{\pgfqpoint{1.375338in}{1.178368in}}%
\pgfpathlineto{\pgfqpoint{1.381733in}{1.183571in}}%
\pgfpathlineto{\pgfqpoint{1.388127in}{1.188774in}}%
\pgfpathlineto{\pgfqpoint{1.394521in}{1.193977in}}%
\pgfpathlineto{\pgfqpoint{1.400916in}{1.199180in}}%
\pgfpathlineto{\pgfqpoint{1.407310in}{1.204383in}}%
\pgfpathlineto{\pgfqpoint{1.413704in}{1.209586in}}%
\pgfpathlineto{\pgfqpoint{1.420098in}{1.214788in}}%
\pgfpathlineto{\pgfqpoint{1.426493in}{1.219991in}}%
\pgfpathlineto{\pgfqpoint{1.432887in}{1.225194in}}%
\pgfpathlineto{\pgfqpoint{1.439281in}{1.230397in}}%
\pgfpathlineto{\pgfqpoint{1.445675in}{1.235600in}}%
\pgfpathlineto{\pgfqpoint{1.452070in}{1.240803in}}%
\pgfpathlineto{\pgfqpoint{1.458464in}{1.246006in}}%
\pgfpathlineto{\pgfqpoint{1.464858in}{1.251209in}}%
\pgfpathlineto{\pgfqpoint{1.471253in}{1.256412in}}%
\pgfpathlineto{\pgfqpoint{1.477647in}{1.261615in}}%
\pgfpathlineto{\pgfqpoint{1.484041in}{1.266818in}}%
\pgfpathlineto{\pgfqpoint{1.490435in}{1.272021in}}%
\pgfpathlineto{\pgfqpoint{1.496830in}{1.277224in}}%
\pgfpathlineto{\pgfqpoint{1.503224in}{1.282427in}}%
\pgfpathlineto{\pgfqpoint{1.509618in}{1.287630in}}%
\pgfpathlineto{\pgfqpoint{1.516013in}{1.292833in}}%
\pgfpathlineto{\pgfqpoint{1.522407in}{1.298036in}}%
\pgfpathlineto{\pgfqpoint{1.528801in}{1.303239in}}%
\pgfpathlineto{\pgfqpoint{1.535195in}{1.308442in}}%
\pgfpathlineto{\pgfqpoint{1.541590in}{1.313645in}}%
\pgfpathlineto{\pgfqpoint{1.547984in}{1.318848in}}%
\pgfpathlineto{\pgfqpoint{1.554378in}{1.324051in}}%
\pgfpathlineto{\pgfqpoint{1.560773in}{1.329254in}}%
\pgfpathlineto{\pgfqpoint{1.567167in}{1.334457in}}%
\pgfpathlineto{\pgfqpoint{1.573561in}{1.339660in}}%
\pgfpathlineto{\pgfqpoint{1.579955in}{1.344863in}}%
\pgfpathlineto{\pgfqpoint{1.586350in}{1.350066in}}%
\pgfpathlineto{\pgfqpoint{1.592744in}{1.355269in}}%
\pgfpathlineto{\pgfqpoint{1.599138in}{1.360472in}}%
\pgfpathlineto{\pgfqpoint{1.605532in}{1.365674in}}%
\pgfpathlineto{\pgfqpoint{1.611927in}{1.370877in}}%
\pgfpathlineto{\pgfqpoint{1.618321in}{1.376080in}}%
\pgfpathlineto{\pgfqpoint{1.624715in}{1.381283in}}%
\pgfpathlineto{\pgfqpoint{1.631110in}{1.386486in}}%
\pgfpathlineto{\pgfqpoint{1.637504in}{1.391689in}}%
\pgfpathlineto{\pgfqpoint{1.643898in}{1.396892in}}%
\pgfpathlineto{\pgfqpoint{1.650292in}{1.402095in}}%
\pgfpathlineto{\pgfqpoint{1.656687in}{1.407298in}}%
\pgfpathlineto{\pgfqpoint{1.663081in}{1.412501in}}%
\pgfpathlineto{\pgfqpoint{1.669475in}{1.417704in}}%
\pgfpathlineto{\pgfqpoint{1.675870in}{1.422907in}}%
\pgfpathlineto{\pgfqpoint{1.682264in}{1.428110in}}%
\pgfpathlineto{\pgfqpoint{1.688658in}{1.433313in}}%
\pgfpathlineto{\pgfqpoint{1.695052in}{1.438516in}}%
\pgfpathlineto{\pgfqpoint{1.701447in}{1.443719in}}%
\pgfpathlineto{\pgfqpoint{1.707841in}{1.448922in}}%
\pgfpathlineto{\pgfqpoint{1.714235in}{1.454125in}}%
\pgfpathlineto{\pgfqpoint{1.720630in}{1.459328in}}%
\pgfpathlineto{\pgfqpoint{1.727024in}{1.464531in}}%
\pgfpathlineto{\pgfqpoint{1.733418in}{1.469734in}}%
\pgfpathlineto{\pgfqpoint{1.739812in}{1.474937in}}%
\pgfpathlineto{\pgfqpoint{1.746207in}{1.480140in}}%
\pgfpathlineto{\pgfqpoint{1.752601in}{1.485343in}}%
\pgfpathlineto{\pgfqpoint{1.758995in}{1.490546in}}%
\pgfpathlineto{\pgfqpoint{1.765389in}{1.495749in}}%
\pgfpathlineto{\pgfqpoint{1.771784in}{1.500952in}}%
\pgfpathlineto{\pgfqpoint{1.778178in}{1.506155in}}%
\pgfpathlineto{\pgfqpoint{1.784572in}{1.511358in}}%
\pgfpathlineto{\pgfqpoint{1.790967in}{1.516560in}}%
\pgfpathlineto{\pgfqpoint{1.797361in}{1.521763in}}%
\pgfpathlineto{\pgfqpoint{1.803755in}{1.526966in}}%
\pgfpathlineto{\pgfqpoint{1.810149in}{1.532169in}}%
\pgfpathlineto{\pgfqpoint{1.816544in}{1.537372in}}%
\pgfpathlineto{\pgfqpoint{1.822938in}{1.542575in}}%
\pgfpathlineto{\pgfqpoint{1.829332in}{1.547778in}}%
\pgfpathlineto{\pgfqpoint{1.835727in}{1.552981in}}%
\pgfpathlineto{\pgfqpoint{1.842121in}{1.558184in}}%
\pgfpathlineto{\pgfqpoint{1.848515in}{1.563387in}}%
\pgfpathlineto{\pgfqpoint{1.854909in}{1.568590in}}%
\pgfpathlineto{\pgfqpoint{1.861304in}{1.573793in}}%
\pgfpathlineto{\pgfqpoint{1.867698in}{1.578996in}}%
\pgfpathlineto{\pgfqpoint{1.874092in}{1.584199in}}%
\pgfpathlineto{\pgfqpoint{1.880486in}{1.589402in}}%
\pgfpathlineto{\pgfqpoint{1.886881in}{1.594605in}}%
\pgfpathlineto{\pgfqpoint{1.893275in}{1.599808in}}%
\pgfpathlineto{\pgfqpoint{1.899669in}{1.605011in}}%
\pgfpathlineto{\pgfqpoint{1.906064in}{1.610214in}}%
\pgfpathlineto{\pgfqpoint{1.912458in}{1.615417in}}%
\pgfpathlineto{\pgfqpoint{1.918852in}{1.620620in}}%
\pgfpathlineto{\pgfqpoint{1.925246in}{1.625823in}}%
\pgfpathlineto{\pgfqpoint{1.931641in}{1.631026in}}%
\pgfpathlineto{\pgfqpoint{1.938035in}{1.636229in}}%
\pgfpathlineto{\pgfqpoint{1.944429in}{1.641432in}}%
\pgfpathlineto{\pgfqpoint{1.950824in}{1.646635in}}%
\pgfpathlineto{\pgfqpoint{1.957218in}{1.651838in}}%
\pgfpathlineto{\pgfqpoint{1.963612in}{1.657041in}}%
\pgfpathlineto{\pgfqpoint{1.970006in}{1.662244in}}%
\pgfpathlineto{\pgfqpoint{1.976401in}{1.667447in}}%
\pgfpathlineto{\pgfqpoint{1.982795in}{1.672649in}}%
\pgfpathlineto{\pgfqpoint{1.989189in}{1.677852in}}%
\pgfpathlineto{\pgfqpoint{1.995584in}{1.683055in}}%
\pgfpathlineto{\pgfqpoint{2.001978in}{1.688258in}}%
\pgfpathlineto{\pgfqpoint{2.008372in}{1.693461in}}%
\pgfpathlineto{\pgfqpoint{2.014766in}{1.698664in}}%
\pgfpathlineto{\pgfqpoint{2.021161in}{1.703867in}}%
\pgfpathlineto{\pgfqpoint{2.027555in}{1.709070in}}%
\pgfpathlineto{\pgfqpoint{2.033949in}{1.714273in}}%
\pgfpathlineto{\pgfqpoint{2.040343in}{1.719476in}}%
\pgfpathlineto{\pgfqpoint{2.046738in}{1.724679in}}%
\pgfpathlineto{\pgfqpoint{2.053132in}{1.729882in}}%
\pgfpathlineto{\pgfqpoint{2.059526in}{1.735085in}}%
\pgfpathlineto{\pgfqpoint{2.065921in}{1.740288in}}%
\pgfpathlineto{\pgfqpoint{2.072315in}{1.745491in}}%
\pgfpathlineto{\pgfqpoint{2.078709in}{1.750694in}}%
\pgfpathlineto{\pgfqpoint{2.085103in}{1.755897in}}%
\pgfpathlineto{\pgfqpoint{2.091498in}{1.761100in}}%
\pgfpathlineto{\pgfqpoint{2.097892in}{1.766303in}}%
\pgfpathlineto{\pgfqpoint{2.104286in}{1.771506in}}%
\pgfpathlineto{\pgfqpoint{2.110681in}{1.776709in}}%
\pgfpathlineto{\pgfqpoint{2.117075in}{1.781912in}}%
\pgfpathlineto{\pgfqpoint{2.123469in}{1.787115in}}%
\pgfpathlineto{\pgfqpoint{2.129863in}{1.792318in}}%
\pgfpathlineto{\pgfqpoint{2.136258in}{1.797521in}}%
\pgfpathlineto{\pgfqpoint{2.142652in}{1.802724in}}%
\pgfpathlineto{\pgfqpoint{2.149046in}{1.807927in}}%
\pgfpathlineto{\pgfqpoint{2.155441in}{1.813130in}}%
\pgfpathlineto{\pgfqpoint{2.161835in}{1.818333in}}%
\pgfpathlineto{\pgfqpoint{2.168229in}{1.823535in}}%
\pgfpathlineto{\pgfqpoint{2.174623in}{1.828738in}}%
\pgfpathlineto{\pgfqpoint{2.181018in}{1.833941in}}%
\pgfpathlineto{\pgfqpoint{2.187412in}{1.839144in}}%
\pgfpathlineto{\pgfqpoint{2.193806in}{1.844347in}}%
\pgfpathlineto{\pgfqpoint{2.200200in}{1.849550in}}%
\pgfpathlineto{\pgfqpoint{2.206595in}{1.854753in}}%
\pgfpathlineto{\pgfqpoint{2.212989in}{1.859956in}}%
\pgfpathlineto{\pgfqpoint{2.219383in}{1.865159in}}%
\pgfpathlineto{\pgfqpoint{2.225778in}{1.870362in}}%
\pgfpathlineto{\pgfqpoint{2.232172in}{1.875565in}}%
\pgfpathlineto{\pgfqpoint{2.238566in}{1.880768in}}%
\pgfpathlineto{\pgfqpoint{2.244960in}{1.885971in}}%
\pgfpathlineto{\pgfqpoint{2.251355in}{1.891174in}}%
\pgfpathlineto{\pgfqpoint{2.257749in}{1.896377in}}%
\pgfpathlineto{\pgfqpoint{2.264143in}{1.901580in}}%
\pgfpathlineto{\pgfqpoint{2.270538in}{1.906783in}}%
\pgfpathlineto{\pgfqpoint{2.276932in}{1.911986in}}%
\pgfpathlineto{\pgfqpoint{2.283326in}{1.917189in}}%
\pgfpathlineto{\pgfqpoint{2.289720in}{1.922392in}}%
\pgfpathlineto{\pgfqpoint{2.296115in}{1.927595in}}%
\pgfpathlineto{\pgfqpoint{2.302509in}{1.932798in}}%
\pgfpathlineto{\pgfqpoint{2.308903in}{1.938001in}}%
\pgfpathlineto{\pgfqpoint{2.315298in}{1.943204in}}%
\pgfpathlineto{\pgfqpoint{2.321692in}{1.948407in}}%
\pgfpathlineto{\pgfqpoint{2.328086in}{1.953610in}}%
\pgfpathlineto{\pgfqpoint{2.334480in}{1.958813in}}%
\pgfpathlineto{\pgfqpoint{2.340875in}{1.964016in}}%
\pgfpathlineto{\pgfqpoint{2.347269in}{1.969219in}}%
\pgfpathlineto{\pgfqpoint{2.353663in}{1.974421in}}%
\pgfpathlineto{\pgfqpoint{2.360057in}{1.979624in}}%
\pgfpathlineto{\pgfqpoint{2.366452in}{1.984827in}}%
\pgfpathlineto{\pgfqpoint{2.372846in}{1.990030in}}%
\pgfpathlineto{\pgfqpoint{2.379240in}{1.995233in}}%
\pgfpathlineto{\pgfqpoint{2.385635in}{2.000436in}}%
\pgfpathlineto{\pgfqpoint{2.392029in}{2.005639in}}%
\pgfpathlineto{\pgfqpoint{2.398423in}{2.010842in}}%
\pgfpathlineto{\pgfqpoint{2.404817in}{2.016045in}}%
\pgfpathlineto{\pgfqpoint{2.411212in}{2.021248in}}%
\pgfpathlineto{\pgfqpoint{2.417606in}{2.026451in}}%
\pgfpathlineto{\pgfqpoint{2.424000in}{2.031654in}}%
\pgfpathlineto{\pgfqpoint{2.430395in}{2.036857in}}%
\pgfpathlineto{\pgfqpoint{2.436789in}{2.042060in}}%
\pgfpathlineto{\pgfqpoint{2.443183in}{2.047263in}}%
\pgfpathlineto{\pgfqpoint{2.449577in}{2.052466in}}%
\pgfpathlineto{\pgfqpoint{2.455972in}{2.057669in}}%
\pgfpathlineto{\pgfqpoint{2.462366in}{2.062872in}}%
\pgfpathlineto{\pgfqpoint{2.468760in}{2.068075in}}%
\pgfpathlineto{\pgfqpoint{2.475154in}{2.073278in}}%
\pgfpathlineto{\pgfqpoint{2.481549in}{2.078481in}}%
\pgfpathlineto{\pgfqpoint{2.487943in}{2.083684in}}%
\pgfpathlineto{\pgfqpoint{2.494337in}{2.088887in}}%
\pgfpathlineto{\pgfqpoint{2.500732in}{2.094090in}}%
\pgfpathlineto{\pgfqpoint{2.507126in}{2.099293in}}%
\pgfpathlineto{\pgfqpoint{2.513520in}{2.104496in}}%
\pgfpathlineto{\pgfqpoint{2.519914in}{2.109699in}}%
\pgfpathlineto{\pgfqpoint{2.526309in}{2.114902in}}%
\pgfpathlineto{\pgfqpoint{2.532703in}{2.120105in}}%
\pgfpathlineto{\pgfqpoint{2.539097in}{2.125308in}}%
\pgfpathlineto{\pgfqpoint{2.545492in}{2.130510in}}%
\pgfpathlineto{\pgfqpoint{2.551886in}{2.135713in}}%
\pgfpathlineto{\pgfqpoint{2.558280in}{2.140916in}}%
\pgfpathlineto{\pgfqpoint{2.564674in}{2.146119in}}%
\pgfpathlineto{\pgfqpoint{2.571069in}{2.151322in}}%
\pgfpathlineto{\pgfqpoint{2.577463in}{2.156525in}}%
\pgfpathlineto{\pgfqpoint{2.583857in}{2.161728in}}%
\pgfpathlineto{\pgfqpoint{2.590252in}{2.166931in}}%
\pgfpathlineto{\pgfqpoint{2.596646in}{2.172134in}}%
\pgfpathlineto{\pgfqpoint{2.603040in}{2.177337in}}%
\pgfpathlineto{\pgfqpoint{2.609434in}{2.182540in}}%
\pgfpathlineto{\pgfqpoint{2.615829in}{2.187743in}}%
\pgfpathlineto{\pgfqpoint{2.622223in}{2.192946in}}%
\pgfpathlineto{\pgfqpoint{2.628617in}{2.198149in}}%
\pgfpathlineto{\pgfqpoint{2.635011in}{2.203352in}}%
\pgfpathlineto{\pgfqpoint{2.641406in}{2.208555in}}%
\pgfpathlineto{\pgfqpoint{2.647800in}{2.213758in}}%
\pgfpathlineto{\pgfqpoint{2.654194in}{2.218961in}}%
\pgfpathlineto{\pgfqpoint{2.660589in}{2.224164in}}%
\pgfpathlineto{\pgfqpoint{2.666983in}{2.229367in}}%
\pgfpathlineto{\pgfqpoint{2.673377in}{2.234570in}}%
\pgfpathlineto{\pgfqpoint{2.679771in}{2.239773in}}%
\pgfpathlineto{\pgfqpoint{2.686166in}{2.244976in}}%
\pgfpathlineto{\pgfqpoint{2.692560in}{2.250179in}}%
\pgfpathlineto{\pgfqpoint{2.698954in}{2.255382in}}%
\pgfpathlineto{\pgfqpoint{2.705349in}{2.260585in}}%
\pgfpathlineto{\pgfqpoint{2.711743in}{2.265788in}}%
\pgfpathlineto{\pgfqpoint{2.718137in}{2.270991in}}%
\pgfpathlineto{\pgfqpoint{2.724531in}{2.276194in}}%
\pgfpathlineto{\pgfqpoint{2.730926in}{2.281396in}}%
\pgfpathlineto{\pgfqpoint{2.737320in}{2.286599in}}%
\pgfpathlineto{\pgfqpoint{2.743714in}{2.291802in}}%
\pgfpathlineto{\pgfqpoint{2.750109in}{2.297005in}}%
\pgfpathlineto{\pgfqpoint{2.756503in}{2.302208in}}%
\pgfpathlineto{\pgfqpoint{2.762897in}{2.307411in}}%
\pgfpathlineto{\pgfqpoint{2.769291in}{2.312614in}}%
\pgfpathlineto{\pgfqpoint{2.775686in}{2.317817in}}%
\pgfpathlineto{\pgfqpoint{2.782080in}{2.323020in}}%
\pgfpathlineto{\pgfqpoint{2.788474in}{2.328223in}}%
\pgfpathlineto{\pgfqpoint{2.794868in}{2.333426in}}%
\pgfpathlineto{\pgfqpoint{2.801263in}{2.338629in}}%
\pgfpathlineto{\pgfqpoint{2.807657in}{2.343832in}}%
\pgfpathlineto{\pgfqpoint{2.814051in}{2.349035in}}%
\pgfpathlineto{\pgfqpoint{2.820446in}{2.354238in}}%
\pgfpathlineto{\pgfqpoint{2.826840in}{2.359441in}}%
\pgfpathlineto{\pgfqpoint{2.833234in}{2.364644in}}%
\pgfpathlineto{\pgfqpoint{2.839628in}{2.369847in}}%
\pgfpathlineto{\pgfqpoint{2.846023in}{2.375050in}}%
\pgfpathlineto{\pgfqpoint{2.852417in}{2.380253in}}%
\pgfpathlineto{\pgfqpoint{2.858811in}{2.385456in}}%
\pgfpathlineto{\pgfqpoint{2.865206in}{2.390659in}}%
\pgfpathlineto{\pgfqpoint{2.871600in}{2.395862in}}%
\pgfpathlineto{\pgfqpoint{2.877994in}{2.401065in}}%
\pgfpathlineto{\pgfqpoint{2.884388in}{2.406268in}}%
\pgfpathlineto{\pgfqpoint{2.890783in}{2.411471in}}%
\pgfpathlineto{\pgfqpoint{2.897177in}{2.416674in}}%
\pgfpathlineto{\pgfqpoint{2.903571in}{2.421877in}}%
\pgfpathlineto{\pgfqpoint{2.909966in}{2.427080in}}%
\pgfpathlineto{\pgfqpoint{2.916360in}{2.432282in}}%
\pgfpathlineto{\pgfqpoint{2.922754in}{2.437485in}}%
\pgfpathlineto{\pgfqpoint{2.929148in}{2.442688in}}%
\pgfpathlineto{\pgfqpoint{2.935543in}{2.447891in}}%
\pgfpathlineto{\pgfqpoint{2.941937in}{2.453094in}}%
\pgfpathlineto{\pgfqpoint{2.948331in}{2.458297in}}%
\pgfpathlineto{\pgfqpoint{2.954725in}{2.463500in}}%
\pgfpathlineto{\pgfqpoint{2.961120in}{2.468703in}}%
\pgfpathlineto{\pgfqpoint{2.967514in}{2.473906in}}%
\pgfpathlineto{\pgfqpoint{2.973908in}{2.479109in}}%
\pgfpathlineto{\pgfqpoint{2.980303in}{2.484312in}}%
\pgfpathlineto{\pgfqpoint{2.986697in}{2.489515in}}%
\pgfpathlineto{\pgfqpoint{2.993091in}{2.494718in}}%
\pgfpathlineto{\pgfqpoint{2.999485in}{2.499921in}}%
\pgfpathlineto{\pgfqpoint{3.005880in}{2.505124in}}%
\pgfpathlineto{\pgfqpoint{3.012274in}{2.510327in}}%
\pgfpathlineto{\pgfqpoint{3.018668in}{2.515530in}}%
\pgfpathlineto{\pgfqpoint{3.025063in}{2.520733in}}%
\pgfpathlineto{\pgfqpoint{3.031457in}{2.525936in}}%
\pgfpathlineto{\pgfqpoint{3.037851in}{2.531139in}}%
\pgfpathlineto{\pgfqpoint{3.044245in}{2.536342in}}%
\pgfpathlineto{\pgfqpoint{3.050640in}{2.541545in}}%
\pgfpathlineto{\pgfqpoint{3.057034in}{2.546748in}}%
\pgfpathlineto{\pgfqpoint{3.063428in}{2.551951in}}%
\pgfpathlineto{\pgfqpoint{3.069822in}{2.557154in}}%
\pgfpathlineto{\pgfqpoint{3.076217in}{2.562357in}}%
\pgfpathlineto{\pgfqpoint{3.082611in}{2.567560in}}%
\pgfpathlineto{\pgfqpoint{3.089005in}{2.572763in}}%
\pgfpathlineto{\pgfqpoint{3.095400in}{2.577966in}}%
\pgfpathlineto{\pgfqpoint{3.101794in}{2.583168in}}%
\pgfpathlineto{\pgfqpoint{3.108188in}{2.588371in}}%
\pgfpathlineto{\pgfqpoint{3.114582in}{2.593574in}}%
\pgfpathlineto{\pgfqpoint{3.120977in}{2.598777in}}%
\pgfpathlineto{\pgfqpoint{3.127371in}{2.603980in}}%
\pgfpathlineto{\pgfqpoint{3.133765in}{2.609183in}}%
\pgfpathlineto{\pgfqpoint{3.140160in}{2.614386in}}%
\pgfpathlineto{\pgfqpoint{3.146554in}{2.619589in}}%
\pgfpathlineto{\pgfqpoint{3.152948in}{2.624792in}}%
\pgfpathlineto{\pgfqpoint{3.159342in}{2.629995in}}%
\pgfpathlineto{\pgfqpoint{3.165737in}{2.635198in}}%
\pgfpathlineto{\pgfqpoint{3.172131in}{2.640401in}}%
\pgfpathlineto{\pgfqpoint{3.178525in}{2.645604in}}%
\pgfpathlineto{\pgfqpoint{3.184920in}{2.650807in}}%
\pgfpathlineto{\pgfqpoint{3.191314in}{2.656010in}}%
\pgfpathlineto{\pgfqpoint{3.197708in}{2.661213in}}%
\pgfpathlineto{\pgfqpoint{3.204102in}{2.666416in}}%
\pgfpathlineto{\pgfqpoint{3.210497in}{2.671619in}}%
\pgfpathlineto{\pgfqpoint{3.216891in}{2.676822in}}%
\pgfpathlineto{\pgfqpoint{3.223285in}{2.682025in}}%
\pgfpathlineto{\pgfqpoint{3.229679in}{2.687228in}}%
\pgfpathlineto{\pgfqpoint{3.236074in}{2.692431in}}%
\pgfpathlineto{\pgfqpoint{3.242468in}{2.697634in}}%
\pgfpathlineto{\pgfqpoint{3.248862in}{2.702837in}}%
\pgfpathlineto{\pgfqpoint{3.255257in}{2.708040in}}%
\pgfpathlineto{\pgfqpoint{3.261651in}{2.713243in}}%
\pgfpathlineto{\pgfqpoint{3.268045in}{2.718446in}}%
\pgfpathlineto{\pgfqpoint{3.274439in}{2.723649in}}%
\pgfpathlineto{\pgfqpoint{3.280834in}{2.728852in}}%
\pgfpathlineto{\pgfqpoint{3.287228in}{2.734055in}}%
\pgfpathlineto{\pgfqpoint{3.293622in}{2.739257in}}%
\pgfpathlineto{\pgfqpoint{3.300017in}{2.744460in}}%
\pgfpathlineto{\pgfqpoint{3.306411in}{2.749663in}}%
\pgfpathlineto{\pgfqpoint{3.312805in}{2.754866in}}%
\pgfpathlineto{\pgfqpoint{3.319199in}{2.760069in}}%
\pgfpathlineto{\pgfqpoint{3.325594in}{2.765272in}}%
\pgfpathlineto{\pgfqpoint{3.331988in}{2.770475in}}%
\pgfpathlineto{\pgfqpoint{3.338382in}{2.775678in}}%
\pgfpathlineto{\pgfqpoint{3.344777in}{2.780881in}}%
\pgfpathlineto{\pgfqpoint{3.351171in}{2.786084in}}%
\pgfpathlineto{\pgfqpoint{3.357565in}{2.791287in}}%
\pgfpathlineto{\pgfqpoint{3.363959in}{2.796490in}}%
\pgfpathlineto{\pgfqpoint{3.370354in}{2.801693in}}%
\pgfpathlineto{\pgfqpoint{3.376748in}{2.806896in}}%
\pgfpathlineto{\pgfqpoint{3.383142in}{2.812099in}}%
\pgfpathlineto{\pgfqpoint{3.389536in}{2.817302in}}%
\pgfpathlineto{\pgfqpoint{3.395931in}{2.822505in}}%
\pgfpathlineto{\pgfqpoint{3.402325in}{2.827708in}}%
\pgfpathlineto{\pgfqpoint{3.408719in}{2.832911in}}%
\pgfpathlineto{\pgfqpoint{3.415114in}{2.838114in}}%
\pgfpathlineto{\pgfqpoint{3.421508in}{2.843317in}}%
\pgfpathlineto{\pgfqpoint{3.427902in}{2.848520in}}%
\pgfpathlineto{\pgfqpoint{3.434296in}{2.853723in}}%
\pgfpathlineto{\pgfqpoint{3.440691in}{2.858926in}}%
\pgfpathlineto{\pgfqpoint{3.447085in}{2.864129in}}%
\pgfpathlineto{\pgfqpoint{3.453479in}{2.869332in}}%
\pgfpathlineto{\pgfqpoint{3.459874in}{2.874535in}}%
\pgfpathlineto{\pgfqpoint{3.466268in}{2.879738in}}%
\pgfpathlineto{\pgfqpoint{3.472662in}{2.884941in}}%
\pgfpathlineto{\pgfqpoint{3.479056in}{2.890143in}}%
\pgfpathlineto{\pgfqpoint{3.485451in}{2.895346in}}%
\pgfpathlineto{\pgfqpoint{3.491845in}{2.900549in}}%
\pgfpathlineto{\pgfqpoint{3.498239in}{2.905752in}}%
\pgfpathlineto{\pgfqpoint{3.504634in}{2.910955in}}%
\pgfpathlineto{\pgfqpoint{3.511028in}{2.916158in}}%
\pgfpathlineto{\pgfqpoint{3.517422in}{2.921361in}}%
\pgfpathlineto{\pgfqpoint{3.523816in}{2.926564in}}%
\pgfpathlineto{\pgfqpoint{3.530211in}{2.931767in}}%
\pgfpathlineto{\pgfqpoint{3.536605in}{2.936970in}}%
\pgfpathlineto{\pgfqpoint{3.542999in}{2.942173in}}%
\pgfpathlineto{\pgfqpoint{3.549393in}{2.947376in}}%
\pgfpathlineto{\pgfqpoint{3.555788in}{2.952579in}}%
\pgfpathlineto{\pgfqpoint{3.562182in}{2.957782in}}%
\pgfpathlineto{\pgfqpoint{3.568576in}{2.962985in}}%
\pgfpathlineto{\pgfqpoint{3.574971in}{2.968188in}}%
\pgfpathlineto{\pgfqpoint{3.581365in}{2.973391in}}%
\pgfpathlineto{\pgfqpoint{3.587759in}{2.978594in}}%
\pgfpathlineto{\pgfqpoint{3.594153in}{2.983797in}}%
\pgfpathlineto{\pgfqpoint{3.600548in}{2.989000in}}%
\pgfpathlineto{\pgfqpoint{3.606942in}{2.994203in}}%
\pgfpathlineto{\pgfqpoint{3.613336in}{2.999406in}}%
\pgfpathlineto{\pgfqpoint{3.619731in}{3.004609in}}%
\pgfpathlineto{\pgfqpoint{3.626125in}{3.009812in}}%
\pgfpathlineto{\pgfqpoint{3.632519in}{3.015015in}}%
\pgfpathlineto{\pgfqpoint{3.638913in}{3.020218in}}%
\pgfpathlineto{\pgfqpoint{3.645308in}{3.025421in}}%
\pgfpathlineto{\pgfqpoint{3.651702in}{3.030624in}}%
\pgfpathlineto{\pgfqpoint{3.658096in}{3.035827in}}%
\pgfpathlineto{\pgfqpoint{3.664490in}{3.041029in}}%
\pgfpathlineto{\pgfqpoint{3.670885in}{3.046232in}}%
\pgfpathlineto{\pgfqpoint{3.677279in}{3.051435in}}%
\pgfpathlineto{\pgfqpoint{3.683673in}{3.056638in}}%
\pgfpathlineto{\pgfqpoint{3.690068in}{3.061841in}}%
\pgfpathlineto{\pgfqpoint{3.696462in}{3.067044in}}%
\pgfpathlineto{\pgfqpoint{3.702856in}{3.072247in}}%
\pgfpathlineto{\pgfqpoint{3.709250in}{3.077450in}}%
\pgfpathlineto{\pgfqpoint{3.715645in}{3.082653in}}%
\pgfpathlineto{\pgfqpoint{3.722039in}{3.087856in}}%
\pgfpathlineto{\pgfqpoint{3.728433in}{3.093059in}}%
\pgfpathlineto{\pgfqpoint{3.734828in}{3.098262in}}%
\pgfpathlineto{\pgfqpoint{3.741222in}{3.103465in}}%
\pgfpathlineto{\pgfqpoint{3.747616in}{3.108668in}}%
\pgfpathlineto{\pgfqpoint{3.754010in}{3.113871in}}%
\pgfpathlineto{\pgfqpoint{3.760405in}{3.119074in}}%
\pgfpathlineto{\pgfqpoint{3.766799in}{3.124277in}}%
\pgfpathlineto{\pgfqpoint{3.773193in}{3.129480in}}%
\pgfpathlineto{\pgfqpoint{3.779588in}{3.134683in}}%
\pgfpathlineto{\pgfqpoint{3.785982in}{3.139886in}}%
\pgfpathlineto{\pgfqpoint{3.792376in}{3.145089in}}%
\pgfpathlineto{\pgfqpoint{3.798770in}{3.150292in}}%
\pgfpathlineto{\pgfqpoint{3.798770in}{3.152893in}}%
\pgfpathlineto{\pgfqpoint{3.798770in}{3.152893in}}%
\pgfpathlineto{\pgfqpoint{3.792376in}{3.152893in}}%
\pgfpathlineto{\pgfqpoint{3.785982in}{3.152893in}}%
\pgfpathlineto{\pgfqpoint{3.779588in}{3.152893in}}%
\pgfpathlineto{\pgfqpoint{3.773193in}{3.152893in}}%
\pgfpathlineto{\pgfqpoint{3.766799in}{3.152893in}}%
\pgfpathlineto{\pgfqpoint{3.760405in}{3.152893in}}%
\pgfpathlineto{\pgfqpoint{3.754010in}{3.152893in}}%
\pgfpathlineto{\pgfqpoint{3.747616in}{3.152893in}}%
\pgfpathlineto{\pgfqpoint{3.741222in}{3.152893in}}%
\pgfpathlineto{\pgfqpoint{3.734828in}{3.152893in}}%
\pgfpathlineto{\pgfqpoint{3.728433in}{3.152893in}}%
\pgfpathlineto{\pgfqpoint{3.722039in}{3.152893in}}%
\pgfpathlineto{\pgfqpoint{3.715645in}{3.152893in}}%
\pgfpathlineto{\pgfqpoint{3.709250in}{3.152893in}}%
\pgfpathlineto{\pgfqpoint{3.702856in}{3.152893in}}%
\pgfpathlineto{\pgfqpoint{3.696462in}{3.152893in}}%
\pgfpathlineto{\pgfqpoint{3.690068in}{3.152893in}}%
\pgfpathlineto{\pgfqpoint{3.683673in}{3.152893in}}%
\pgfpathlineto{\pgfqpoint{3.677279in}{3.152893in}}%
\pgfpathlineto{\pgfqpoint{3.670885in}{3.152893in}}%
\pgfpathlineto{\pgfqpoint{3.664490in}{3.152893in}}%
\pgfpathlineto{\pgfqpoint{3.658096in}{3.152893in}}%
\pgfpathlineto{\pgfqpoint{3.651702in}{3.152893in}}%
\pgfpathlineto{\pgfqpoint{3.645308in}{3.152893in}}%
\pgfpathlineto{\pgfqpoint{3.638913in}{3.152893in}}%
\pgfpathlineto{\pgfqpoint{3.632519in}{3.152893in}}%
\pgfpathlineto{\pgfqpoint{3.626125in}{3.152893in}}%
\pgfpathlineto{\pgfqpoint{3.619731in}{3.152893in}}%
\pgfpathlineto{\pgfqpoint{3.613336in}{3.152893in}}%
\pgfpathlineto{\pgfqpoint{3.606942in}{3.152893in}}%
\pgfpathlineto{\pgfqpoint{3.600548in}{3.152893in}}%
\pgfpathlineto{\pgfqpoint{3.594153in}{3.152893in}}%
\pgfpathlineto{\pgfqpoint{3.587759in}{3.152893in}}%
\pgfpathlineto{\pgfqpoint{3.581365in}{3.152893in}}%
\pgfpathlineto{\pgfqpoint{3.574971in}{3.152893in}}%
\pgfpathlineto{\pgfqpoint{3.568576in}{3.152893in}}%
\pgfpathlineto{\pgfqpoint{3.562182in}{3.152893in}}%
\pgfpathlineto{\pgfqpoint{3.555788in}{3.152893in}}%
\pgfpathlineto{\pgfqpoint{3.549393in}{3.152893in}}%
\pgfpathlineto{\pgfqpoint{3.542999in}{3.152893in}}%
\pgfpathlineto{\pgfqpoint{3.536605in}{3.152893in}}%
\pgfpathlineto{\pgfqpoint{3.530211in}{3.152893in}}%
\pgfpathlineto{\pgfqpoint{3.523816in}{3.152893in}}%
\pgfpathlineto{\pgfqpoint{3.517422in}{3.152893in}}%
\pgfpathlineto{\pgfqpoint{3.511028in}{3.152893in}}%
\pgfpathlineto{\pgfqpoint{3.504634in}{3.152893in}}%
\pgfpathlineto{\pgfqpoint{3.498239in}{3.152893in}}%
\pgfpathlineto{\pgfqpoint{3.491845in}{3.152893in}}%
\pgfpathlineto{\pgfqpoint{3.485451in}{3.152893in}}%
\pgfpathlineto{\pgfqpoint{3.479056in}{3.152893in}}%
\pgfpathlineto{\pgfqpoint{3.472662in}{3.152893in}}%
\pgfpathlineto{\pgfqpoint{3.466268in}{3.152893in}}%
\pgfpathlineto{\pgfqpoint{3.459874in}{3.152893in}}%
\pgfpathlineto{\pgfqpoint{3.453479in}{3.152893in}}%
\pgfpathlineto{\pgfqpoint{3.447085in}{3.152893in}}%
\pgfpathlineto{\pgfqpoint{3.440691in}{3.152893in}}%
\pgfpathlineto{\pgfqpoint{3.434296in}{3.152893in}}%
\pgfpathlineto{\pgfqpoint{3.427902in}{3.152893in}}%
\pgfpathlineto{\pgfqpoint{3.421508in}{3.152893in}}%
\pgfpathlineto{\pgfqpoint{3.415114in}{3.152893in}}%
\pgfpathlineto{\pgfqpoint{3.408719in}{3.152893in}}%
\pgfpathlineto{\pgfqpoint{3.402325in}{3.152893in}}%
\pgfpathlineto{\pgfqpoint{3.395931in}{3.152893in}}%
\pgfpathlineto{\pgfqpoint{3.389536in}{3.152893in}}%
\pgfpathlineto{\pgfqpoint{3.383142in}{3.152893in}}%
\pgfpathlineto{\pgfqpoint{3.376748in}{3.152893in}}%
\pgfpathlineto{\pgfqpoint{3.370354in}{3.152893in}}%
\pgfpathlineto{\pgfqpoint{3.363959in}{3.152893in}}%
\pgfpathlineto{\pgfqpoint{3.357565in}{3.152893in}}%
\pgfpathlineto{\pgfqpoint{3.351171in}{3.152893in}}%
\pgfpathlineto{\pgfqpoint{3.344777in}{3.152893in}}%
\pgfpathlineto{\pgfqpoint{3.338382in}{3.152893in}}%
\pgfpathlineto{\pgfqpoint{3.331988in}{3.152893in}}%
\pgfpathlineto{\pgfqpoint{3.325594in}{3.152893in}}%
\pgfpathlineto{\pgfqpoint{3.319199in}{3.152893in}}%
\pgfpathlineto{\pgfqpoint{3.312805in}{3.152893in}}%
\pgfpathlineto{\pgfqpoint{3.306411in}{3.152893in}}%
\pgfpathlineto{\pgfqpoint{3.300017in}{3.152893in}}%
\pgfpathlineto{\pgfqpoint{3.293622in}{3.152893in}}%
\pgfpathlineto{\pgfqpoint{3.287228in}{3.152893in}}%
\pgfpathlineto{\pgfqpoint{3.280834in}{3.152893in}}%
\pgfpathlineto{\pgfqpoint{3.274439in}{3.152893in}}%
\pgfpathlineto{\pgfqpoint{3.268045in}{3.152893in}}%
\pgfpathlineto{\pgfqpoint{3.261651in}{3.152893in}}%
\pgfpathlineto{\pgfqpoint{3.255257in}{3.152893in}}%
\pgfpathlineto{\pgfqpoint{3.248862in}{3.152893in}}%
\pgfpathlineto{\pgfqpoint{3.242468in}{3.152893in}}%
\pgfpathlineto{\pgfqpoint{3.236074in}{3.152893in}}%
\pgfpathlineto{\pgfqpoint{3.229679in}{3.152893in}}%
\pgfpathlineto{\pgfqpoint{3.223285in}{3.152893in}}%
\pgfpathlineto{\pgfqpoint{3.216891in}{3.152893in}}%
\pgfpathlineto{\pgfqpoint{3.210497in}{3.152893in}}%
\pgfpathlineto{\pgfqpoint{3.204102in}{3.152893in}}%
\pgfpathlineto{\pgfqpoint{3.197708in}{3.152893in}}%
\pgfpathlineto{\pgfqpoint{3.191314in}{3.152893in}}%
\pgfpathlineto{\pgfqpoint{3.184920in}{3.152893in}}%
\pgfpathlineto{\pgfqpoint{3.178525in}{3.152893in}}%
\pgfpathlineto{\pgfqpoint{3.172131in}{3.152893in}}%
\pgfpathlineto{\pgfqpoint{3.165737in}{3.152893in}}%
\pgfpathlineto{\pgfqpoint{3.159342in}{3.152893in}}%
\pgfpathlineto{\pgfqpoint{3.152948in}{3.152893in}}%
\pgfpathlineto{\pgfqpoint{3.146554in}{3.152893in}}%
\pgfpathlineto{\pgfqpoint{3.140160in}{3.152893in}}%
\pgfpathlineto{\pgfqpoint{3.133765in}{3.152893in}}%
\pgfpathlineto{\pgfqpoint{3.127371in}{3.152893in}}%
\pgfpathlineto{\pgfqpoint{3.120977in}{3.152893in}}%
\pgfpathlineto{\pgfqpoint{3.114582in}{3.152893in}}%
\pgfpathlineto{\pgfqpoint{3.108188in}{3.152893in}}%
\pgfpathlineto{\pgfqpoint{3.101794in}{3.152893in}}%
\pgfpathlineto{\pgfqpoint{3.095400in}{3.152893in}}%
\pgfpathlineto{\pgfqpoint{3.089005in}{3.152893in}}%
\pgfpathlineto{\pgfqpoint{3.082611in}{3.152893in}}%
\pgfpathlineto{\pgfqpoint{3.076217in}{3.152893in}}%
\pgfpathlineto{\pgfqpoint{3.069822in}{3.152893in}}%
\pgfpathlineto{\pgfqpoint{3.063428in}{3.152893in}}%
\pgfpathlineto{\pgfqpoint{3.057034in}{3.152893in}}%
\pgfpathlineto{\pgfqpoint{3.050640in}{3.152893in}}%
\pgfpathlineto{\pgfqpoint{3.044245in}{3.152893in}}%
\pgfpathlineto{\pgfqpoint{3.037851in}{3.152893in}}%
\pgfpathlineto{\pgfqpoint{3.031457in}{3.152893in}}%
\pgfpathlineto{\pgfqpoint{3.025063in}{3.152893in}}%
\pgfpathlineto{\pgfqpoint{3.018668in}{3.152893in}}%
\pgfpathlineto{\pgfqpoint{3.012274in}{3.152893in}}%
\pgfpathlineto{\pgfqpoint{3.005880in}{3.152893in}}%
\pgfpathlineto{\pgfqpoint{2.999485in}{3.152893in}}%
\pgfpathlineto{\pgfqpoint{2.993091in}{3.152893in}}%
\pgfpathlineto{\pgfqpoint{2.986697in}{3.152893in}}%
\pgfpathlineto{\pgfqpoint{2.980303in}{3.152893in}}%
\pgfpathlineto{\pgfqpoint{2.973908in}{3.152893in}}%
\pgfpathlineto{\pgfqpoint{2.967514in}{3.152893in}}%
\pgfpathlineto{\pgfqpoint{2.961120in}{3.152893in}}%
\pgfpathlineto{\pgfqpoint{2.954725in}{3.152893in}}%
\pgfpathlineto{\pgfqpoint{2.948331in}{3.152893in}}%
\pgfpathlineto{\pgfqpoint{2.941937in}{3.152893in}}%
\pgfpathlineto{\pgfqpoint{2.935543in}{3.152893in}}%
\pgfpathlineto{\pgfqpoint{2.929148in}{3.152893in}}%
\pgfpathlineto{\pgfqpoint{2.922754in}{3.152893in}}%
\pgfpathlineto{\pgfqpoint{2.916360in}{3.152893in}}%
\pgfpathlineto{\pgfqpoint{2.909966in}{3.152893in}}%
\pgfpathlineto{\pgfqpoint{2.903571in}{3.152893in}}%
\pgfpathlineto{\pgfqpoint{2.897177in}{3.152893in}}%
\pgfpathlineto{\pgfqpoint{2.890783in}{3.152893in}}%
\pgfpathlineto{\pgfqpoint{2.884388in}{3.152893in}}%
\pgfpathlineto{\pgfqpoint{2.877994in}{3.152893in}}%
\pgfpathlineto{\pgfqpoint{2.871600in}{3.152893in}}%
\pgfpathlineto{\pgfqpoint{2.865206in}{3.152893in}}%
\pgfpathlineto{\pgfqpoint{2.858811in}{3.152893in}}%
\pgfpathlineto{\pgfqpoint{2.852417in}{3.152893in}}%
\pgfpathlineto{\pgfqpoint{2.846023in}{3.152893in}}%
\pgfpathlineto{\pgfqpoint{2.839628in}{3.152893in}}%
\pgfpathlineto{\pgfqpoint{2.833234in}{3.152893in}}%
\pgfpathlineto{\pgfqpoint{2.826840in}{3.152893in}}%
\pgfpathlineto{\pgfqpoint{2.820446in}{3.152893in}}%
\pgfpathlineto{\pgfqpoint{2.814051in}{3.152893in}}%
\pgfpathlineto{\pgfqpoint{2.807657in}{3.152893in}}%
\pgfpathlineto{\pgfqpoint{2.801263in}{3.152893in}}%
\pgfpathlineto{\pgfqpoint{2.794868in}{3.152893in}}%
\pgfpathlineto{\pgfqpoint{2.788474in}{3.152893in}}%
\pgfpathlineto{\pgfqpoint{2.782080in}{3.152893in}}%
\pgfpathlineto{\pgfqpoint{2.775686in}{3.152893in}}%
\pgfpathlineto{\pgfqpoint{2.769291in}{3.152893in}}%
\pgfpathlineto{\pgfqpoint{2.762897in}{3.152893in}}%
\pgfpathlineto{\pgfqpoint{2.756503in}{3.152893in}}%
\pgfpathlineto{\pgfqpoint{2.750109in}{3.152893in}}%
\pgfpathlineto{\pgfqpoint{2.743714in}{3.152893in}}%
\pgfpathlineto{\pgfqpoint{2.737320in}{3.152893in}}%
\pgfpathlineto{\pgfqpoint{2.730926in}{3.152893in}}%
\pgfpathlineto{\pgfqpoint{2.724531in}{3.152893in}}%
\pgfpathlineto{\pgfqpoint{2.718137in}{3.152893in}}%
\pgfpathlineto{\pgfqpoint{2.711743in}{3.152893in}}%
\pgfpathlineto{\pgfqpoint{2.705349in}{3.152893in}}%
\pgfpathlineto{\pgfqpoint{2.698954in}{3.152893in}}%
\pgfpathlineto{\pgfqpoint{2.692560in}{3.152893in}}%
\pgfpathlineto{\pgfqpoint{2.686166in}{3.152893in}}%
\pgfpathlineto{\pgfqpoint{2.679771in}{3.152893in}}%
\pgfpathlineto{\pgfqpoint{2.673377in}{3.152893in}}%
\pgfpathlineto{\pgfqpoint{2.666983in}{3.152893in}}%
\pgfpathlineto{\pgfqpoint{2.660589in}{3.152893in}}%
\pgfpathlineto{\pgfqpoint{2.654194in}{3.152893in}}%
\pgfpathlineto{\pgfqpoint{2.647800in}{3.152893in}}%
\pgfpathlineto{\pgfqpoint{2.641406in}{3.152893in}}%
\pgfpathlineto{\pgfqpoint{2.635011in}{3.152893in}}%
\pgfpathlineto{\pgfqpoint{2.628617in}{3.152893in}}%
\pgfpathlineto{\pgfqpoint{2.622223in}{3.152893in}}%
\pgfpathlineto{\pgfqpoint{2.615829in}{3.152893in}}%
\pgfpathlineto{\pgfqpoint{2.609434in}{3.152893in}}%
\pgfpathlineto{\pgfqpoint{2.603040in}{3.152893in}}%
\pgfpathlineto{\pgfqpoint{2.596646in}{3.152893in}}%
\pgfpathlineto{\pgfqpoint{2.590252in}{3.152893in}}%
\pgfpathlineto{\pgfqpoint{2.583857in}{3.152893in}}%
\pgfpathlineto{\pgfqpoint{2.577463in}{3.152893in}}%
\pgfpathlineto{\pgfqpoint{2.571069in}{3.152893in}}%
\pgfpathlineto{\pgfqpoint{2.564674in}{3.152893in}}%
\pgfpathlineto{\pgfqpoint{2.558280in}{3.152893in}}%
\pgfpathlineto{\pgfqpoint{2.551886in}{3.152893in}}%
\pgfpathlineto{\pgfqpoint{2.545492in}{3.152893in}}%
\pgfpathlineto{\pgfqpoint{2.539097in}{3.152893in}}%
\pgfpathlineto{\pgfqpoint{2.532703in}{3.152893in}}%
\pgfpathlineto{\pgfqpoint{2.526309in}{3.152893in}}%
\pgfpathlineto{\pgfqpoint{2.519914in}{3.152893in}}%
\pgfpathlineto{\pgfqpoint{2.513520in}{3.152893in}}%
\pgfpathlineto{\pgfqpoint{2.507126in}{3.152893in}}%
\pgfpathlineto{\pgfqpoint{2.500732in}{3.152893in}}%
\pgfpathlineto{\pgfqpoint{2.494337in}{3.152893in}}%
\pgfpathlineto{\pgfqpoint{2.487943in}{3.152893in}}%
\pgfpathlineto{\pgfqpoint{2.481549in}{3.152893in}}%
\pgfpathlineto{\pgfqpoint{2.475154in}{3.152893in}}%
\pgfpathlineto{\pgfqpoint{2.468760in}{3.152893in}}%
\pgfpathlineto{\pgfqpoint{2.462366in}{3.152893in}}%
\pgfpathlineto{\pgfqpoint{2.455972in}{3.152893in}}%
\pgfpathlineto{\pgfqpoint{2.449577in}{3.152893in}}%
\pgfpathlineto{\pgfqpoint{2.443183in}{3.152893in}}%
\pgfpathlineto{\pgfqpoint{2.436789in}{3.152893in}}%
\pgfpathlineto{\pgfqpoint{2.430395in}{3.152893in}}%
\pgfpathlineto{\pgfqpoint{2.424000in}{3.152893in}}%
\pgfpathlineto{\pgfqpoint{2.417606in}{3.152893in}}%
\pgfpathlineto{\pgfqpoint{2.411212in}{3.152893in}}%
\pgfpathlineto{\pgfqpoint{2.404817in}{3.152893in}}%
\pgfpathlineto{\pgfqpoint{2.398423in}{3.152893in}}%
\pgfpathlineto{\pgfqpoint{2.392029in}{3.152893in}}%
\pgfpathlineto{\pgfqpoint{2.385635in}{3.152893in}}%
\pgfpathlineto{\pgfqpoint{2.379240in}{3.152893in}}%
\pgfpathlineto{\pgfqpoint{2.372846in}{3.152893in}}%
\pgfpathlineto{\pgfqpoint{2.366452in}{3.152893in}}%
\pgfpathlineto{\pgfqpoint{2.360057in}{3.152893in}}%
\pgfpathlineto{\pgfqpoint{2.353663in}{3.152893in}}%
\pgfpathlineto{\pgfqpoint{2.347269in}{3.152893in}}%
\pgfpathlineto{\pgfqpoint{2.340875in}{3.152893in}}%
\pgfpathlineto{\pgfqpoint{2.334480in}{3.152893in}}%
\pgfpathlineto{\pgfqpoint{2.328086in}{3.152893in}}%
\pgfpathlineto{\pgfqpoint{2.321692in}{3.152893in}}%
\pgfpathlineto{\pgfqpoint{2.315298in}{3.152893in}}%
\pgfpathlineto{\pgfqpoint{2.308903in}{3.152893in}}%
\pgfpathlineto{\pgfqpoint{2.302509in}{3.152893in}}%
\pgfpathlineto{\pgfqpoint{2.296115in}{3.152893in}}%
\pgfpathlineto{\pgfqpoint{2.289720in}{3.152893in}}%
\pgfpathlineto{\pgfqpoint{2.283326in}{3.152893in}}%
\pgfpathlineto{\pgfqpoint{2.276932in}{3.152893in}}%
\pgfpathlineto{\pgfqpoint{2.270538in}{3.152893in}}%
\pgfpathlineto{\pgfqpoint{2.264143in}{3.152893in}}%
\pgfpathlineto{\pgfqpoint{2.257749in}{3.152893in}}%
\pgfpathlineto{\pgfqpoint{2.251355in}{3.152893in}}%
\pgfpathlineto{\pgfqpoint{2.244960in}{3.152893in}}%
\pgfpathlineto{\pgfqpoint{2.238566in}{3.152893in}}%
\pgfpathlineto{\pgfqpoint{2.232172in}{3.152893in}}%
\pgfpathlineto{\pgfqpoint{2.225778in}{3.152893in}}%
\pgfpathlineto{\pgfqpoint{2.219383in}{3.152893in}}%
\pgfpathlineto{\pgfqpoint{2.212989in}{3.152893in}}%
\pgfpathlineto{\pgfqpoint{2.206595in}{3.152893in}}%
\pgfpathlineto{\pgfqpoint{2.200200in}{3.152893in}}%
\pgfpathlineto{\pgfqpoint{2.193806in}{3.152893in}}%
\pgfpathlineto{\pgfqpoint{2.187412in}{3.152893in}}%
\pgfpathlineto{\pgfqpoint{2.181018in}{3.152893in}}%
\pgfpathlineto{\pgfqpoint{2.174623in}{3.152893in}}%
\pgfpathlineto{\pgfqpoint{2.168229in}{3.152893in}}%
\pgfpathlineto{\pgfqpoint{2.161835in}{3.152893in}}%
\pgfpathlineto{\pgfqpoint{2.155441in}{3.152893in}}%
\pgfpathlineto{\pgfqpoint{2.149046in}{3.152893in}}%
\pgfpathlineto{\pgfqpoint{2.142652in}{3.152893in}}%
\pgfpathlineto{\pgfqpoint{2.136258in}{3.152893in}}%
\pgfpathlineto{\pgfqpoint{2.129863in}{3.152893in}}%
\pgfpathlineto{\pgfqpoint{2.123469in}{3.152893in}}%
\pgfpathlineto{\pgfqpoint{2.117075in}{3.152893in}}%
\pgfpathlineto{\pgfqpoint{2.110681in}{3.152893in}}%
\pgfpathlineto{\pgfqpoint{2.104286in}{3.152893in}}%
\pgfpathlineto{\pgfqpoint{2.097892in}{3.152893in}}%
\pgfpathlineto{\pgfqpoint{2.091498in}{3.152893in}}%
\pgfpathlineto{\pgfqpoint{2.085103in}{3.152893in}}%
\pgfpathlineto{\pgfqpoint{2.078709in}{3.152893in}}%
\pgfpathlineto{\pgfqpoint{2.072315in}{3.152893in}}%
\pgfpathlineto{\pgfqpoint{2.065921in}{3.152893in}}%
\pgfpathlineto{\pgfqpoint{2.059526in}{3.152893in}}%
\pgfpathlineto{\pgfqpoint{2.053132in}{3.152893in}}%
\pgfpathlineto{\pgfqpoint{2.046738in}{3.152893in}}%
\pgfpathlineto{\pgfqpoint{2.040343in}{3.152893in}}%
\pgfpathlineto{\pgfqpoint{2.033949in}{3.152893in}}%
\pgfpathlineto{\pgfqpoint{2.027555in}{3.152893in}}%
\pgfpathlineto{\pgfqpoint{2.021161in}{3.152893in}}%
\pgfpathlineto{\pgfqpoint{2.014766in}{3.152893in}}%
\pgfpathlineto{\pgfqpoint{2.008372in}{3.152893in}}%
\pgfpathlineto{\pgfqpoint{2.001978in}{3.152893in}}%
\pgfpathlineto{\pgfqpoint{1.995584in}{3.152893in}}%
\pgfpathlineto{\pgfqpoint{1.989189in}{3.152893in}}%
\pgfpathlineto{\pgfqpoint{1.982795in}{3.152893in}}%
\pgfpathlineto{\pgfqpoint{1.976401in}{3.152893in}}%
\pgfpathlineto{\pgfqpoint{1.970006in}{3.152893in}}%
\pgfpathlineto{\pgfqpoint{1.963612in}{3.152893in}}%
\pgfpathlineto{\pgfqpoint{1.957218in}{3.152893in}}%
\pgfpathlineto{\pgfqpoint{1.950824in}{3.152893in}}%
\pgfpathlineto{\pgfqpoint{1.944429in}{3.152893in}}%
\pgfpathlineto{\pgfqpoint{1.938035in}{3.152893in}}%
\pgfpathlineto{\pgfqpoint{1.931641in}{3.152893in}}%
\pgfpathlineto{\pgfqpoint{1.925246in}{3.152893in}}%
\pgfpathlineto{\pgfqpoint{1.918852in}{3.152893in}}%
\pgfpathlineto{\pgfqpoint{1.912458in}{3.152893in}}%
\pgfpathlineto{\pgfqpoint{1.906064in}{3.152893in}}%
\pgfpathlineto{\pgfqpoint{1.899669in}{3.152893in}}%
\pgfpathlineto{\pgfqpoint{1.893275in}{3.152893in}}%
\pgfpathlineto{\pgfqpoint{1.886881in}{3.152893in}}%
\pgfpathlineto{\pgfqpoint{1.880486in}{3.152893in}}%
\pgfpathlineto{\pgfqpoint{1.874092in}{3.152893in}}%
\pgfpathlineto{\pgfqpoint{1.867698in}{3.152893in}}%
\pgfpathlineto{\pgfqpoint{1.861304in}{3.152893in}}%
\pgfpathlineto{\pgfqpoint{1.854909in}{3.152893in}}%
\pgfpathlineto{\pgfqpoint{1.848515in}{3.152893in}}%
\pgfpathlineto{\pgfqpoint{1.842121in}{3.152893in}}%
\pgfpathlineto{\pgfqpoint{1.835727in}{3.152893in}}%
\pgfpathlineto{\pgfqpoint{1.829332in}{3.152893in}}%
\pgfpathlineto{\pgfqpoint{1.822938in}{3.152893in}}%
\pgfpathlineto{\pgfqpoint{1.816544in}{3.152893in}}%
\pgfpathlineto{\pgfqpoint{1.810149in}{3.152893in}}%
\pgfpathlineto{\pgfqpoint{1.803755in}{3.152893in}}%
\pgfpathlineto{\pgfqpoint{1.797361in}{3.152893in}}%
\pgfpathlineto{\pgfqpoint{1.790967in}{3.152893in}}%
\pgfpathlineto{\pgfqpoint{1.784572in}{3.152893in}}%
\pgfpathlineto{\pgfqpoint{1.778178in}{3.152893in}}%
\pgfpathlineto{\pgfqpoint{1.771784in}{3.152893in}}%
\pgfpathlineto{\pgfqpoint{1.765389in}{3.152893in}}%
\pgfpathlineto{\pgfqpoint{1.758995in}{3.152893in}}%
\pgfpathlineto{\pgfqpoint{1.752601in}{3.152893in}}%
\pgfpathlineto{\pgfqpoint{1.746207in}{3.152893in}}%
\pgfpathlineto{\pgfqpoint{1.739812in}{3.152893in}}%
\pgfpathlineto{\pgfqpoint{1.733418in}{3.152893in}}%
\pgfpathlineto{\pgfqpoint{1.727024in}{3.152893in}}%
\pgfpathlineto{\pgfqpoint{1.720630in}{3.152893in}}%
\pgfpathlineto{\pgfqpoint{1.714235in}{3.152893in}}%
\pgfpathlineto{\pgfqpoint{1.707841in}{3.152893in}}%
\pgfpathlineto{\pgfqpoint{1.701447in}{3.152893in}}%
\pgfpathlineto{\pgfqpoint{1.695052in}{3.152893in}}%
\pgfpathlineto{\pgfqpoint{1.688658in}{3.152893in}}%
\pgfpathlineto{\pgfqpoint{1.682264in}{3.152893in}}%
\pgfpathlineto{\pgfqpoint{1.675870in}{3.152893in}}%
\pgfpathlineto{\pgfqpoint{1.669475in}{3.152893in}}%
\pgfpathlineto{\pgfqpoint{1.663081in}{3.152893in}}%
\pgfpathlineto{\pgfqpoint{1.656687in}{3.152893in}}%
\pgfpathlineto{\pgfqpoint{1.650292in}{3.152893in}}%
\pgfpathlineto{\pgfqpoint{1.643898in}{3.152893in}}%
\pgfpathlineto{\pgfqpoint{1.637504in}{3.152893in}}%
\pgfpathlineto{\pgfqpoint{1.631110in}{3.152893in}}%
\pgfpathlineto{\pgfqpoint{1.624715in}{3.152893in}}%
\pgfpathlineto{\pgfqpoint{1.618321in}{3.152893in}}%
\pgfpathlineto{\pgfqpoint{1.611927in}{3.152893in}}%
\pgfpathlineto{\pgfqpoint{1.605532in}{3.152893in}}%
\pgfpathlineto{\pgfqpoint{1.599138in}{3.152893in}}%
\pgfpathlineto{\pgfqpoint{1.592744in}{3.152893in}}%
\pgfpathlineto{\pgfqpoint{1.586350in}{3.152893in}}%
\pgfpathlineto{\pgfqpoint{1.579955in}{3.152893in}}%
\pgfpathlineto{\pgfqpoint{1.573561in}{3.152893in}}%
\pgfpathlineto{\pgfqpoint{1.567167in}{3.152893in}}%
\pgfpathlineto{\pgfqpoint{1.560773in}{3.152893in}}%
\pgfpathlineto{\pgfqpoint{1.554378in}{3.152893in}}%
\pgfpathlineto{\pgfqpoint{1.547984in}{3.152893in}}%
\pgfpathlineto{\pgfqpoint{1.541590in}{3.152893in}}%
\pgfpathlineto{\pgfqpoint{1.535195in}{3.152893in}}%
\pgfpathlineto{\pgfqpoint{1.528801in}{3.152893in}}%
\pgfpathlineto{\pgfqpoint{1.522407in}{3.152893in}}%
\pgfpathlineto{\pgfqpoint{1.516013in}{3.152893in}}%
\pgfpathlineto{\pgfqpoint{1.509618in}{3.152893in}}%
\pgfpathlineto{\pgfqpoint{1.503224in}{3.152893in}}%
\pgfpathlineto{\pgfqpoint{1.496830in}{3.152893in}}%
\pgfpathlineto{\pgfqpoint{1.490435in}{3.152893in}}%
\pgfpathlineto{\pgfqpoint{1.484041in}{3.152893in}}%
\pgfpathlineto{\pgfqpoint{1.477647in}{3.152893in}}%
\pgfpathlineto{\pgfqpoint{1.471253in}{3.152893in}}%
\pgfpathlineto{\pgfqpoint{1.464858in}{3.152893in}}%
\pgfpathlineto{\pgfqpoint{1.458464in}{3.152893in}}%
\pgfpathlineto{\pgfqpoint{1.452070in}{3.152893in}}%
\pgfpathlineto{\pgfqpoint{1.445675in}{3.152893in}}%
\pgfpathlineto{\pgfqpoint{1.439281in}{3.152893in}}%
\pgfpathlineto{\pgfqpoint{1.432887in}{3.152893in}}%
\pgfpathlineto{\pgfqpoint{1.426493in}{3.152893in}}%
\pgfpathlineto{\pgfqpoint{1.420098in}{3.152893in}}%
\pgfpathlineto{\pgfqpoint{1.413704in}{3.152893in}}%
\pgfpathlineto{\pgfqpoint{1.407310in}{3.152893in}}%
\pgfpathlineto{\pgfqpoint{1.400916in}{3.152893in}}%
\pgfpathlineto{\pgfqpoint{1.394521in}{3.152893in}}%
\pgfpathlineto{\pgfqpoint{1.388127in}{3.152893in}}%
\pgfpathlineto{\pgfqpoint{1.381733in}{3.152893in}}%
\pgfpathlineto{\pgfqpoint{1.375338in}{3.152893in}}%
\pgfpathlineto{\pgfqpoint{1.368944in}{3.152893in}}%
\pgfpathlineto{\pgfqpoint{1.362550in}{3.152893in}}%
\pgfpathlineto{\pgfqpoint{1.356156in}{3.152893in}}%
\pgfpathlineto{\pgfqpoint{1.349761in}{3.152893in}}%
\pgfpathlineto{\pgfqpoint{1.343367in}{3.152893in}}%
\pgfpathlineto{\pgfqpoint{1.336973in}{3.152893in}}%
\pgfpathlineto{\pgfqpoint{1.330578in}{3.152893in}}%
\pgfpathlineto{\pgfqpoint{1.324184in}{3.152893in}}%
\pgfpathlineto{\pgfqpoint{1.317790in}{3.152893in}}%
\pgfpathlineto{\pgfqpoint{1.311396in}{3.152893in}}%
\pgfpathlineto{\pgfqpoint{1.305001in}{3.152893in}}%
\pgfpathlineto{\pgfqpoint{1.298607in}{3.152893in}}%
\pgfpathlineto{\pgfqpoint{1.292213in}{3.152893in}}%
\pgfpathlineto{\pgfqpoint{1.285818in}{3.152893in}}%
\pgfpathlineto{\pgfqpoint{1.279424in}{3.152893in}}%
\pgfpathlineto{\pgfqpoint{1.273030in}{3.152893in}}%
\pgfpathlineto{\pgfqpoint{1.266636in}{3.152893in}}%
\pgfpathlineto{\pgfqpoint{1.260241in}{3.152893in}}%
\pgfpathlineto{\pgfqpoint{1.253847in}{3.152893in}}%
\pgfpathlineto{\pgfqpoint{1.247453in}{3.152893in}}%
\pgfpathlineto{\pgfqpoint{1.241059in}{3.152893in}}%
\pgfpathlineto{\pgfqpoint{1.234664in}{3.152893in}}%
\pgfpathlineto{\pgfqpoint{1.228270in}{3.152893in}}%
\pgfpathlineto{\pgfqpoint{1.221876in}{3.152893in}}%
\pgfpathlineto{\pgfqpoint{1.215481in}{3.152893in}}%
\pgfpathlineto{\pgfqpoint{1.209087in}{3.152893in}}%
\pgfpathlineto{\pgfqpoint{1.202693in}{3.152893in}}%
\pgfpathlineto{\pgfqpoint{1.196299in}{3.152893in}}%
\pgfpathlineto{\pgfqpoint{1.189904in}{3.152893in}}%
\pgfpathlineto{\pgfqpoint{1.183510in}{3.152893in}}%
\pgfpathlineto{\pgfqpoint{1.177116in}{3.152893in}}%
\pgfpathlineto{\pgfqpoint{1.170721in}{3.152893in}}%
\pgfpathlineto{\pgfqpoint{1.164327in}{3.152893in}}%
\pgfpathlineto{\pgfqpoint{1.157933in}{3.152893in}}%
\pgfpathlineto{\pgfqpoint{1.151539in}{3.152893in}}%
\pgfpathlineto{\pgfqpoint{1.145144in}{3.152893in}}%
\pgfpathlineto{\pgfqpoint{1.138750in}{3.152893in}}%
\pgfpathlineto{\pgfqpoint{1.132356in}{3.152893in}}%
\pgfpathlineto{\pgfqpoint{1.125962in}{3.152893in}}%
\pgfpathlineto{\pgfqpoint{1.119567in}{3.152893in}}%
\pgfpathlineto{\pgfqpoint{1.113173in}{3.152893in}}%
\pgfpathlineto{\pgfqpoint{1.106779in}{3.152893in}}%
\pgfpathlineto{\pgfqpoint{1.100384in}{3.152893in}}%
\pgfpathlineto{\pgfqpoint{1.093990in}{3.152893in}}%
\pgfpathlineto{\pgfqpoint{1.087596in}{3.152893in}}%
\pgfpathlineto{\pgfqpoint{1.081202in}{3.152893in}}%
\pgfpathlineto{\pgfqpoint{1.074807in}{3.152893in}}%
\pgfpathlineto{\pgfqpoint{1.068413in}{3.152893in}}%
\pgfpathlineto{\pgfqpoint{1.062019in}{3.152893in}}%
\pgfpathlineto{\pgfqpoint{1.055624in}{3.152893in}}%
\pgfpathlineto{\pgfqpoint{1.049230in}{3.152893in}}%
\pgfpathlineto{\pgfqpoint{1.042836in}{3.152893in}}%
\pgfpathlineto{\pgfqpoint{1.036442in}{3.152893in}}%
\pgfpathlineto{\pgfqpoint{1.030047in}{3.152893in}}%
\pgfpathlineto{\pgfqpoint{1.023653in}{3.152893in}}%
\pgfpathlineto{\pgfqpoint{1.017259in}{3.152893in}}%
\pgfpathlineto{\pgfqpoint{1.010864in}{3.152893in}}%
\pgfpathlineto{\pgfqpoint{1.004470in}{3.152893in}}%
\pgfpathlineto{\pgfqpoint{0.998076in}{3.152893in}}%
\pgfpathlineto{\pgfqpoint{0.991682in}{3.152893in}}%
\pgfpathlineto{\pgfqpoint{0.985287in}{3.152893in}}%
\pgfpathlineto{\pgfqpoint{0.978893in}{3.152893in}}%
\pgfpathlineto{\pgfqpoint{0.972499in}{3.152893in}}%
\pgfpathlineto{\pgfqpoint{0.966105in}{3.152893in}}%
\pgfpathlineto{\pgfqpoint{0.959710in}{3.152893in}}%
\pgfpathlineto{\pgfqpoint{0.953316in}{3.152893in}}%
\pgfpathlineto{\pgfqpoint{0.946922in}{3.152893in}}%
\pgfpathlineto{\pgfqpoint{0.940527in}{3.152893in}}%
\pgfpathlineto{\pgfqpoint{0.934133in}{3.152893in}}%
\pgfpathlineto{\pgfqpoint{0.927739in}{3.152893in}}%
\pgfpathlineto{\pgfqpoint{0.921345in}{3.152893in}}%
\pgfpathlineto{\pgfqpoint{0.914950in}{3.152893in}}%
\pgfpathlineto{\pgfqpoint{0.908556in}{3.152893in}}%
\pgfpathlineto{\pgfqpoint{0.902162in}{3.152893in}}%
\pgfpathlineto{\pgfqpoint{0.895767in}{3.152893in}}%
\pgfpathlineto{\pgfqpoint{0.889373in}{3.152893in}}%
\pgfpathlineto{\pgfqpoint{0.882979in}{3.152893in}}%
\pgfpathlineto{\pgfqpoint{0.876585in}{3.152893in}}%
\pgfpathlineto{\pgfqpoint{0.870190in}{3.152893in}}%
\pgfpathlineto{\pgfqpoint{0.863796in}{3.152893in}}%
\pgfpathlineto{\pgfqpoint{0.857402in}{3.152893in}}%
\pgfpathlineto{\pgfqpoint{0.851007in}{3.152893in}}%
\pgfpathlineto{\pgfqpoint{0.844613in}{3.152893in}}%
\pgfpathlineto{\pgfqpoint{0.838219in}{3.152893in}}%
\pgfpathlineto{\pgfqpoint{0.831825in}{3.152893in}}%
\pgfpathlineto{\pgfqpoint{0.825430in}{3.152893in}}%
\pgfpathlineto{\pgfqpoint{0.819036in}{3.152893in}}%
\pgfpathlineto{\pgfqpoint{0.812642in}{3.152893in}}%
\pgfpathlineto{\pgfqpoint{0.806248in}{3.152893in}}%
\pgfpathlineto{\pgfqpoint{0.799853in}{3.152893in}}%
\pgfpathlineto{\pgfqpoint{0.793459in}{3.152893in}}%
\pgfpathlineto{\pgfqpoint{0.787065in}{3.152893in}}%
\pgfpathlineto{\pgfqpoint{0.780670in}{3.152893in}}%
\pgfpathlineto{\pgfqpoint{0.774276in}{3.152893in}}%
\pgfpathlineto{\pgfqpoint{0.767882in}{3.152893in}}%
\pgfpathlineto{\pgfqpoint{0.761488in}{3.152893in}}%
\pgfpathlineto{\pgfqpoint{0.755093in}{3.152893in}}%
\pgfpathlineto{\pgfqpoint{0.748699in}{3.152893in}}%
\pgfpathlineto{\pgfqpoint{0.742305in}{3.152893in}}%
\pgfpathlineto{\pgfqpoint{0.735910in}{3.152893in}}%
\pgfpathlineto{\pgfqpoint{0.729516in}{3.152893in}}%
\pgfpathlineto{\pgfqpoint{0.723122in}{3.152893in}}%
\pgfpathlineto{\pgfqpoint{0.716728in}{3.152893in}}%
\pgfpathlineto{\pgfqpoint{0.710333in}{3.152893in}}%
\pgfpathlineto{\pgfqpoint{0.703939in}{3.152893in}}%
\pgfpathlineto{\pgfqpoint{0.697545in}{3.152893in}}%
\pgfpathlineto{\pgfqpoint{0.691150in}{3.152893in}}%
\pgfpathlineto{\pgfqpoint{0.684756in}{3.152893in}}%
\pgfpathlineto{\pgfqpoint{0.678362in}{3.152893in}}%
\pgfpathlineto{\pgfqpoint{0.671968in}{3.152893in}}%
\pgfpathlineto{\pgfqpoint{0.665573in}{3.152893in}}%
\pgfpathlineto{\pgfqpoint{0.659179in}{3.152893in}}%
\pgfpathlineto{\pgfqpoint{0.652785in}{3.152893in}}%
\pgfpathlineto{\pgfqpoint{0.646391in}{3.152893in}}%
\pgfpathlineto{\pgfqpoint{0.639996in}{3.152893in}}%
\pgfpathlineto{\pgfqpoint{0.633602in}{3.152893in}}%
\pgfpathlineto{\pgfqpoint{0.627208in}{3.152893in}}%
\pgfpathlineto{\pgfqpoint{0.620813in}{3.152893in}}%
\pgfpathlineto{\pgfqpoint{0.614419in}{3.152893in}}%
\pgfpathlineto{\pgfqpoint{0.608025in}{3.152893in}}%
\pgfpathlineto{\pgfqpoint{0.608025in}{3.152893in}}%
\pgfpathclose%
\pgfusepath{stroke,fill}%
}%
\begin{pgfscope}%
\pgfsys@transformshift{0.000000in}{0.000000in}%
\pgfsys@useobject{currentmarker}{}%
\end{pgfscope}%
\end{pgfscope}%
\begin{pgfscope}%
\pgfsetbuttcap%
\pgfsetroundjoin%
\definecolor{currentfill}{rgb}{0.000000,0.000000,0.000000}%
\pgfsetfillcolor{currentfill}%
\pgfsetlinewidth{0.803000pt}%
\definecolor{currentstroke}{rgb}{0.000000,0.000000,0.000000}%
\pgfsetstrokecolor{currentstroke}%
\pgfsetdash{}{0pt}%
\pgfsys@defobject{currentmarker}{\pgfqpoint{0.000000in}{-0.048611in}}{\pgfqpoint{0.000000in}{0.000000in}}{%
\pgfpathmoveto{\pgfqpoint{0.000000in}{0.000000in}}%
\pgfpathlineto{\pgfqpoint{0.000000in}{-0.048611in}}%
\pgfusepath{stroke,fill}%
}%
\begin{pgfscope}%
\pgfsys@transformshift{0.608025in}{0.554012in}%
\pgfsys@useobject{currentmarker}{}%
\end{pgfscope}%
\end{pgfscope}%
\begin{pgfscope}%
\definecolor{textcolor}{rgb}{0.000000,0.000000,0.000000}%
\pgfsetstrokecolor{textcolor}%
\pgfsetfillcolor{textcolor}%
\pgftext[x=0.608025in,y=0.456790in,,top]{\color{textcolor}\rmfamily\fontsize{10.000000}{12.000000}\selectfont \(\displaystyle {0.0}\)}%
\end{pgfscope}%
\begin{pgfscope}%
\pgfsetbuttcap%
\pgfsetroundjoin%
\definecolor{currentfill}{rgb}{0.000000,0.000000,0.000000}%
\pgfsetfillcolor{currentfill}%
\pgfsetlinewidth{0.803000pt}%
\definecolor{currentstroke}{rgb}{0.000000,0.000000,0.000000}%
\pgfsetstrokecolor{currentstroke}%
\pgfsetdash{}{0pt}%
\pgfsys@defobject{currentmarker}{\pgfqpoint{0.000000in}{-0.048611in}}{\pgfqpoint{0.000000in}{0.000000in}}{%
\pgfpathmoveto{\pgfqpoint{0.000000in}{0.000000in}}%
\pgfpathlineto{\pgfqpoint{0.000000in}{-0.048611in}}%
\pgfusepath{stroke,fill}%
}%
\begin{pgfscope}%
\pgfsys@transformshift{1.885602in}{0.554012in}%
\pgfsys@useobject{currentmarker}{}%
\end{pgfscope}%
\end{pgfscope}%
\begin{pgfscope}%
\definecolor{textcolor}{rgb}{0.000000,0.000000,0.000000}%
\pgfsetstrokecolor{textcolor}%
\pgfsetfillcolor{textcolor}%
\pgftext[x=1.885602in,y=0.456790in,,top]{\color{textcolor}\rmfamily\fontsize{10.000000}{12.000000}\selectfont \(\displaystyle {0.2}\)}%
\end{pgfscope}%
\begin{pgfscope}%
\pgfsetbuttcap%
\pgfsetroundjoin%
\definecolor{currentfill}{rgb}{0.000000,0.000000,0.000000}%
\pgfsetfillcolor{currentfill}%
\pgfsetlinewidth{0.803000pt}%
\definecolor{currentstroke}{rgb}{0.000000,0.000000,0.000000}%
\pgfsetstrokecolor{currentstroke}%
\pgfsetdash{}{0pt}%
\pgfsys@defobject{currentmarker}{\pgfqpoint{0.000000in}{-0.048611in}}{\pgfqpoint{0.000000in}{0.000000in}}{%
\pgfpathmoveto{\pgfqpoint{0.000000in}{0.000000in}}%
\pgfpathlineto{\pgfqpoint{0.000000in}{-0.048611in}}%
\pgfusepath{stroke,fill}%
}%
\begin{pgfscope}%
\pgfsys@transformshift{3.163179in}{0.554012in}%
\pgfsys@useobject{currentmarker}{}%
\end{pgfscope}%
\end{pgfscope}%
\begin{pgfscope}%
\definecolor{textcolor}{rgb}{0.000000,0.000000,0.000000}%
\pgfsetstrokecolor{textcolor}%
\pgfsetfillcolor{textcolor}%
\pgftext[x=3.163179in,y=0.456790in,,top]{\color{textcolor}\rmfamily\fontsize{10.000000}{12.000000}\selectfont \(\displaystyle {0.4}\)}%
\end{pgfscope}%
\begin{pgfscope}%
\pgfsetbuttcap%
\pgfsetroundjoin%
\definecolor{currentfill}{rgb}{0.000000,0.000000,0.000000}%
\pgfsetfillcolor{currentfill}%
\pgfsetlinewidth{0.803000pt}%
\definecolor{currentstroke}{rgb}{0.000000,0.000000,0.000000}%
\pgfsetstrokecolor{currentstroke}%
\pgfsetdash{}{0pt}%
\pgfsys@defobject{currentmarker}{\pgfqpoint{0.000000in}{-0.048611in}}{\pgfqpoint{0.000000in}{0.000000in}}{%
\pgfpathmoveto{\pgfqpoint{0.000000in}{0.000000in}}%
\pgfpathlineto{\pgfqpoint{0.000000in}{-0.048611in}}%
\pgfusepath{stroke,fill}%
}%
\begin{pgfscope}%
\pgfsys@transformshift{4.440756in}{0.554012in}%
\pgfsys@useobject{currentmarker}{}%
\end{pgfscope}%
\end{pgfscope}%
\begin{pgfscope}%
\definecolor{textcolor}{rgb}{0.000000,0.000000,0.000000}%
\pgfsetstrokecolor{textcolor}%
\pgfsetfillcolor{textcolor}%
\pgftext[x=4.440756in,y=0.456790in,,top]{\color{textcolor}\rmfamily\fontsize{10.000000}{12.000000}\selectfont \(\displaystyle {0.6}\)}%
\end{pgfscope}%
\begin{pgfscope}%
\pgfsetbuttcap%
\pgfsetroundjoin%
\definecolor{currentfill}{rgb}{0.000000,0.000000,0.000000}%
\pgfsetfillcolor{currentfill}%
\pgfsetlinewidth{0.803000pt}%
\definecolor{currentstroke}{rgb}{0.000000,0.000000,0.000000}%
\pgfsetstrokecolor{currentstroke}%
\pgfsetdash{}{0pt}%
\pgfsys@defobject{currentmarker}{\pgfqpoint{0.000000in}{-0.048611in}}{\pgfqpoint{0.000000in}{0.000000in}}{%
\pgfpathmoveto{\pgfqpoint{0.000000in}{0.000000in}}%
\pgfpathlineto{\pgfqpoint{0.000000in}{-0.048611in}}%
\pgfusepath{stroke,fill}%
}%
\begin{pgfscope}%
\pgfsys@transformshift{5.718333in}{0.554012in}%
\pgfsys@useobject{currentmarker}{}%
\end{pgfscope}%
\end{pgfscope}%
\begin{pgfscope}%
\definecolor{textcolor}{rgb}{0.000000,0.000000,0.000000}%
\pgfsetstrokecolor{textcolor}%
\pgfsetfillcolor{textcolor}%
\pgftext[x=5.718333in,y=0.456790in,,top]{\color{textcolor}\rmfamily\fontsize{10.000000}{12.000000}\selectfont \(\displaystyle {0.8}\)}%
\end{pgfscope}%
\begin{pgfscope}%
\pgfsetbuttcap%
\pgfsetroundjoin%
\definecolor{currentfill}{rgb}{0.000000,0.000000,0.000000}%
\pgfsetfillcolor{currentfill}%
\pgfsetlinewidth{0.803000pt}%
\definecolor{currentstroke}{rgb}{0.000000,0.000000,0.000000}%
\pgfsetstrokecolor{currentstroke}%
\pgfsetdash{}{0pt}%
\pgfsys@defobject{currentmarker}{\pgfqpoint{0.000000in}{-0.048611in}}{\pgfqpoint{0.000000in}{0.000000in}}{%
\pgfpathmoveto{\pgfqpoint{0.000000in}{0.000000in}}%
\pgfpathlineto{\pgfqpoint{0.000000in}{-0.048611in}}%
\pgfusepath{stroke,fill}%
}%
\begin{pgfscope}%
\pgfsys@transformshift{6.995910in}{0.554012in}%
\pgfsys@useobject{currentmarker}{}%
\end{pgfscope}%
\end{pgfscope}%
\begin{pgfscope}%
\definecolor{textcolor}{rgb}{0.000000,0.000000,0.000000}%
\pgfsetstrokecolor{textcolor}%
\pgfsetfillcolor{textcolor}%
\pgftext[x=6.995910in,y=0.456790in,,top]{\color{textcolor}\rmfamily\fontsize{10.000000}{12.000000}\selectfont \(\displaystyle {1.0}\)}%
\end{pgfscope}%
\begin{pgfscope}%
\definecolor{textcolor}{rgb}{0.000000,0.000000,0.000000}%
\pgfsetstrokecolor{textcolor}%
\pgfsetfillcolor{textcolor}%
\pgftext[x=3.801968in,y=0.277777in,,top]{\color{textcolor}\rmfamily\fontsize{14.000000}{16.800000}\selectfont Normalized Quantity}%
\end{pgfscope}%
\begin{pgfscope}%
\pgfsetbuttcap%
\pgfsetroundjoin%
\definecolor{currentfill}{rgb}{0.000000,0.000000,0.000000}%
\pgfsetfillcolor{currentfill}%
\pgfsetlinewidth{0.803000pt}%
\definecolor{currentstroke}{rgb}{0.000000,0.000000,0.000000}%
\pgfsetstrokecolor{currentstroke}%
\pgfsetdash{}{0pt}%
\pgfsys@defobject{currentmarker}{\pgfqpoint{-0.048611in}{0.000000in}}{\pgfqpoint{-0.000000in}{0.000000in}}{%
\pgfpathmoveto{\pgfqpoint{-0.000000in}{0.000000in}}%
\pgfpathlineto{\pgfqpoint{-0.048611in}{0.000000in}}%
\pgfusepath{stroke,fill}%
}%
\begin{pgfscope}%
\pgfsys@transformshift{0.608025in}{0.554012in}%
\pgfsys@useobject{currentmarker}{}%
\end{pgfscope}%
\end{pgfscope}%
\begin{pgfscope}%
\definecolor{textcolor}{rgb}{0.000000,0.000000,0.000000}%
\pgfsetstrokecolor{textcolor}%
\pgfsetfillcolor{textcolor}%
\pgftext[x=0.333333in, y=0.505787in, left, base]{\color{textcolor}\rmfamily\fontsize{10.000000}{12.000000}\selectfont \(\displaystyle {0.0}\)}%
\end{pgfscope}%
\begin{pgfscope}%
\pgfsetbuttcap%
\pgfsetroundjoin%
\definecolor{currentfill}{rgb}{0.000000,0.000000,0.000000}%
\pgfsetfillcolor{currentfill}%
\pgfsetlinewidth{0.803000pt}%
\definecolor{currentstroke}{rgb}{0.000000,0.000000,0.000000}%
\pgfsetstrokecolor{currentstroke}%
\pgfsetdash{}{0pt}%
\pgfsys@defobject{currentmarker}{\pgfqpoint{-0.048611in}{0.000000in}}{\pgfqpoint{-0.000000in}{0.000000in}}{%
\pgfpathmoveto{\pgfqpoint{-0.000000in}{0.000000in}}%
\pgfpathlineto{\pgfqpoint{-0.048611in}{0.000000in}}%
\pgfusepath{stroke,fill}%
}%
\begin{pgfscope}%
\pgfsys@transformshift{0.608025in}{1.593564in}%
\pgfsys@useobject{currentmarker}{}%
\end{pgfscope}%
\end{pgfscope}%
\begin{pgfscope}%
\definecolor{textcolor}{rgb}{0.000000,0.000000,0.000000}%
\pgfsetstrokecolor{textcolor}%
\pgfsetfillcolor{textcolor}%
\pgftext[x=0.333333in, y=1.545339in, left, base]{\color{textcolor}\rmfamily\fontsize{10.000000}{12.000000}\selectfont \(\displaystyle {0.2}\)}%
\end{pgfscope}%
\begin{pgfscope}%
\pgfsetbuttcap%
\pgfsetroundjoin%
\definecolor{currentfill}{rgb}{0.000000,0.000000,0.000000}%
\pgfsetfillcolor{currentfill}%
\pgfsetlinewidth{0.803000pt}%
\definecolor{currentstroke}{rgb}{0.000000,0.000000,0.000000}%
\pgfsetstrokecolor{currentstroke}%
\pgfsetdash{}{0pt}%
\pgfsys@defobject{currentmarker}{\pgfqpoint{-0.048611in}{0.000000in}}{\pgfqpoint{-0.000000in}{0.000000in}}{%
\pgfpathmoveto{\pgfqpoint{-0.000000in}{0.000000in}}%
\pgfpathlineto{\pgfqpoint{-0.048611in}{0.000000in}}%
\pgfusepath{stroke,fill}%
}%
\begin{pgfscope}%
\pgfsys@transformshift{0.608025in}{2.633117in}%
\pgfsys@useobject{currentmarker}{}%
\end{pgfscope}%
\end{pgfscope}%
\begin{pgfscope}%
\definecolor{textcolor}{rgb}{0.000000,0.000000,0.000000}%
\pgfsetstrokecolor{textcolor}%
\pgfsetfillcolor{textcolor}%
\pgftext[x=0.333333in, y=2.584892in, left, base]{\color{textcolor}\rmfamily\fontsize{10.000000}{12.000000}\selectfont \(\displaystyle {0.4}\)}%
\end{pgfscope}%
\begin{pgfscope}%
\pgfsetbuttcap%
\pgfsetroundjoin%
\definecolor{currentfill}{rgb}{0.000000,0.000000,0.000000}%
\pgfsetfillcolor{currentfill}%
\pgfsetlinewidth{0.803000pt}%
\definecolor{currentstroke}{rgb}{0.000000,0.000000,0.000000}%
\pgfsetstrokecolor{currentstroke}%
\pgfsetdash{}{0pt}%
\pgfsys@defobject{currentmarker}{\pgfqpoint{-0.048611in}{0.000000in}}{\pgfqpoint{-0.000000in}{0.000000in}}{%
\pgfpathmoveto{\pgfqpoint{-0.000000in}{0.000000in}}%
\pgfpathlineto{\pgfqpoint{-0.048611in}{0.000000in}}%
\pgfusepath{stroke,fill}%
}%
\begin{pgfscope}%
\pgfsys@transformshift{0.608025in}{3.672670in}%
\pgfsys@useobject{currentmarker}{}%
\end{pgfscope}%
\end{pgfscope}%
\begin{pgfscope}%
\definecolor{textcolor}{rgb}{0.000000,0.000000,0.000000}%
\pgfsetstrokecolor{textcolor}%
\pgfsetfillcolor{textcolor}%
\pgftext[x=0.333333in, y=3.624444in, left, base]{\color{textcolor}\rmfamily\fontsize{10.000000}{12.000000}\selectfont \(\displaystyle {0.6}\)}%
\end{pgfscope}%
\begin{pgfscope}%
\pgfsetbuttcap%
\pgfsetroundjoin%
\definecolor{currentfill}{rgb}{0.000000,0.000000,0.000000}%
\pgfsetfillcolor{currentfill}%
\pgfsetlinewidth{0.803000pt}%
\definecolor{currentstroke}{rgb}{0.000000,0.000000,0.000000}%
\pgfsetstrokecolor{currentstroke}%
\pgfsetdash{}{0pt}%
\pgfsys@defobject{currentmarker}{\pgfqpoint{-0.048611in}{0.000000in}}{\pgfqpoint{-0.000000in}{0.000000in}}{%
\pgfpathmoveto{\pgfqpoint{-0.000000in}{0.000000in}}%
\pgfpathlineto{\pgfqpoint{-0.048611in}{0.000000in}}%
\pgfusepath{stroke,fill}%
}%
\begin{pgfscope}%
\pgfsys@transformshift{0.608025in}{4.712222in}%
\pgfsys@useobject{currentmarker}{}%
\end{pgfscope}%
\end{pgfscope}%
\begin{pgfscope}%
\definecolor{textcolor}{rgb}{0.000000,0.000000,0.000000}%
\pgfsetstrokecolor{textcolor}%
\pgfsetfillcolor{textcolor}%
\pgftext[x=0.333333in, y=4.663997in, left, base]{\color{textcolor}\rmfamily\fontsize{10.000000}{12.000000}\selectfont \(\displaystyle {0.8}\)}%
\end{pgfscope}%
\begin{pgfscope}%
\pgfsetbuttcap%
\pgfsetroundjoin%
\definecolor{currentfill}{rgb}{0.000000,0.000000,0.000000}%
\pgfsetfillcolor{currentfill}%
\pgfsetlinewidth{0.803000pt}%
\definecolor{currentstroke}{rgb}{0.000000,0.000000,0.000000}%
\pgfsetstrokecolor{currentstroke}%
\pgfsetdash{}{0pt}%
\pgfsys@defobject{currentmarker}{\pgfqpoint{-0.048611in}{0.000000in}}{\pgfqpoint{-0.000000in}{0.000000in}}{%
\pgfpathmoveto{\pgfqpoint{-0.000000in}{0.000000in}}%
\pgfpathlineto{\pgfqpoint{-0.048611in}{0.000000in}}%
\pgfusepath{stroke,fill}%
}%
\begin{pgfscope}%
\pgfsys@transformshift{0.608025in}{5.751775in}%
\pgfsys@useobject{currentmarker}{}%
\end{pgfscope}%
\end{pgfscope}%
\begin{pgfscope}%
\definecolor{textcolor}{rgb}{0.000000,0.000000,0.000000}%
\pgfsetstrokecolor{textcolor}%
\pgfsetfillcolor{textcolor}%
\pgftext[x=0.333333in, y=5.703549in, left, base]{\color{textcolor}\rmfamily\fontsize{10.000000}{12.000000}\selectfont \(\displaystyle {1.0}\)}%
\end{pgfscope}%
\begin{pgfscope}%
\definecolor{textcolor}{rgb}{0.000000,0.000000,0.000000}%
\pgfsetstrokecolor{textcolor}%
\pgfsetfillcolor{textcolor}%
\pgftext[x=0.277777in,y=3.152893in,,bottom,rotate=90.000000]{\color{textcolor}\rmfamily\fontsize{14.000000}{16.800000}\selectfont Normalized Price}%
\end{pgfscope}%
\begin{pgfscope}%
\pgfpathrectangle{\pgfqpoint{0.608025in}{0.554012in}}{\pgfqpoint{6.387885in}{5.197763in}}%
\pgfusepath{clip}%
\pgfsetrectcap%
\pgfsetroundjoin%
\pgfsetlinewidth{1.505625pt}%
\definecolor{currentstroke}{rgb}{0.121569,0.466667,0.705882}%
\pgfsetstrokecolor{currentstroke}%
\pgfsetdash{}{0pt}%
\pgfpathmoveto{\pgfqpoint{0.608025in}{5.751775in}}%
\pgfpathlineto{\pgfqpoint{6.995910in}{0.554012in}}%
\pgfpathlineto{\pgfqpoint{6.995910in}{0.554012in}}%
\pgfusepath{stroke}%
\end{pgfscope}%
\begin{pgfscope}%
\pgfpathrectangle{\pgfqpoint{0.608025in}{0.554012in}}{\pgfqpoint{6.387885in}{5.197763in}}%
\pgfusepath{clip}%
\pgfsetrectcap%
\pgfsetroundjoin%
\pgfsetlinewidth{1.505625pt}%
\definecolor{currentstroke}{rgb}{1.000000,0.498039,0.054902}%
\pgfsetstrokecolor{currentstroke}%
\pgfsetdash{}{0pt}%
\pgfpathmoveto{\pgfqpoint{0.608025in}{0.554012in}}%
\pgfpathlineto{\pgfqpoint{6.995910in}{5.751775in}}%
\pgfpathlineto{\pgfqpoint{6.995910in}{5.751775in}}%
\pgfusepath{stroke}%
\end{pgfscope}%
\begin{pgfscope}%
\pgfpathrectangle{\pgfqpoint{0.608025in}{0.554012in}}{\pgfqpoint{6.387885in}{5.197763in}}%
\pgfusepath{clip}%
\pgfsetbuttcap%
\pgfsetroundjoin%
\pgfsetlinewidth{1.505625pt}%
\definecolor{currentstroke}{rgb}{1.000000,0.000000,0.000000}%
\pgfsetstrokecolor{currentstroke}%
\pgfsetstrokeopacity{0.600000}%
\pgfsetdash{{5.550000pt}{2.400000pt}}{0.000000pt}%
\pgfpathmoveto{\pgfqpoint{0.608025in}{3.152893in}}%
\pgfpathlineto{\pgfqpoint{3.801968in}{3.152893in}}%
\pgfusepath{stroke}%
\end{pgfscope}%
\begin{pgfscope}%
\pgfsetrectcap%
\pgfsetmiterjoin%
\pgfsetlinewidth{0.803000pt}%
\definecolor{currentstroke}{rgb}{0.000000,0.000000,0.000000}%
\pgfsetstrokecolor{currentstroke}%
\pgfsetdash{}{0pt}%
\pgfpathmoveto{\pgfqpoint{0.608025in}{0.554012in}}%
\pgfpathlineto{\pgfqpoint{0.608025in}{5.751775in}}%
\pgfusepath{stroke}%
\end{pgfscope}%
\begin{pgfscope}%
\pgfsetrectcap%
\pgfsetmiterjoin%
\pgfsetlinewidth{0.803000pt}%
\definecolor{currentstroke}{rgb}{0.000000,0.000000,0.000000}%
\pgfsetstrokecolor{currentstroke}%
\pgfsetdash{}{0pt}%
\pgfpathmoveto{\pgfqpoint{6.995910in}{0.554012in}}%
\pgfpathlineto{\pgfqpoint{6.995910in}{5.751775in}}%
\pgfusepath{stroke}%
\end{pgfscope}%
\begin{pgfscope}%
\pgfsetrectcap%
\pgfsetmiterjoin%
\pgfsetlinewidth{0.803000pt}%
\definecolor{currentstroke}{rgb}{0.000000,0.000000,0.000000}%
\pgfsetstrokecolor{currentstroke}%
\pgfsetdash{}{0pt}%
\pgfpathmoveto{\pgfqpoint{0.608025in}{0.554012in}}%
\pgfpathlineto{\pgfqpoint{6.995910in}{0.554012in}}%
\pgfusepath{stroke}%
\end{pgfscope}%
\begin{pgfscope}%
\pgfsetrectcap%
\pgfsetmiterjoin%
\pgfsetlinewidth{0.803000pt}%
\definecolor{currentstroke}{rgb}{0.000000,0.000000,0.000000}%
\pgfsetstrokecolor{currentstroke}%
\pgfsetdash{}{0pt}%
\pgfpathmoveto{\pgfqpoint{0.608025in}{5.751775in}}%
\pgfpathlineto{\pgfqpoint{6.995910in}{5.751775in}}%
\pgfusepath{stroke}%
\end{pgfscope}%
\begin{pgfscope}%
\definecolor{textcolor}{rgb}{0.000000,0.000000,0.000000}%
\pgfsetstrokecolor{textcolor}%
\pgfsetfillcolor{textcolor}%
\pgftext[x=0.927419in,y=3.672670in,left,base]{\color{textcolor}\rmfamily\fontsize{12.000000}{14.400000}\selectfont Consumer Surplus}%
\end{pgfscope}%
\begin{pgfscope}%
\definecolor{textcolor}{rgb}{0.000000,0.000000,0.000000}%
\pgfsetstrokecolor{textcolor}%
\pgfsetfillcolor{textcolor}%
\pgftext[x=0.927419in,y=2.633117in,left,base]{\color{textcolor}\rmfamily\fontsize{12.000000}{14.400000}\selectfont Producer Surplus}%
\end{pgfscope}%
\begin{pgfscope}%
\definecolor{textcolor}{rgb}{0.000000,0.000000,0.000000}%
\pgfsetstrokecolor{textcolor}%
\pgfsetfillcolor{textcolor}%
\pgftext[x=5.573101in, y=4.280992in, left, base,rotate=45.000000]{\color{textcolor}\rmfamily\fontsize{12.000000}{14.400000}\selectfont Supply}%
\end{pgfscope}%
\begin{pgfscope}%
\definecolor{textcolor}{rgb}{0.000000,0.000000,0.000000}%
\pgfsetstrokecolor{textcolor}%
\pgfsetfillcolor{textcolor}%
\pgftext[x=5.463239in, y=1.504902in, left, base,rotate=320.000000]{\color{textcolor}\rmfamily\fontsize{12.000000}{14.400000}\selectfont Demand}%
\end{pgfscope}%
\begin{pgfscope}%
\pgfsetbuttcap%
\pgfsetmiterjoin%
\definecolor{currentfill}{rgb}{1.000000,1.000000,1.000000}%
\pgfsetfillcolor{currentfill}%
\pgfsetlinewidth{1.003750pt}%
\definecolor{currentstroke}{rgb}{0.000000,0.000000,0.000000}%
\pgfsetstrokecolor{currentstroke}%
\pgfsetdash{}{0pt}%
\pgfpathmoveto{\pgfqpoint{0.615515in}{5.385498in}}%
\pgfpathlineto{\pgfqpoint{0.902364in}{5.385498in}}%
\pgfpathlineto{\pgfqpoint{0.902364in}{5.698275in}}%
\pgfpathlineto{\pgfqpoint{0.615515in}{5.698275in}}%
\pgfpathlineto{\pgfqpoint{0.615515in}{5.385498in}}%
\pgfpathclose%
\pgfusepath{stroke,fill}%
\end{pgfscope}%
\begin{pgfscope}%
\definecolor{textcolor}{rgb}{0.000000,0.000000,0.000000}%
\pgfsetstrokecolor{textcolor}%
\pgfsetfillcolor{textcolor}%
\pgftext[x=0.671904in,y=5.491887in,left,base]{\color{textcolor}\rmfamily\fontsize{14.000000}{16.800000}\selectfont a)}%
\end{pgfscope}%
\begin{pgfscope}%
\pgfsetbuttcap%
\pgfsetmiterjoin%
\definecolor{currentfill}{rgb}{1.000000,1.000000,1.000000}%
\pgfsetfillcolor{currentfill}%
\pgfsetlinewidth{0.000000pt}%
\definecolor{currentstroke}{rgb}{0.000000,0.000000,0.000000}%
\pgfsetstrokecolor{currentstroke}%
\pgfsetstrokeopacity{0.000000}%
\pgfsetdash{}{0pt}%
\pgfpathmoveto{\pgfqpoint{7.323380in}{0.554012in}}%
\pgfpathlineto{\pgfqpoint{13.711265in}{0.554012in}}%
\pgfpathlineto{\pgfqpoint{13.711265in}{5.751775in}}%
\pgfpathlineto{\pgfqpoint{7.323380in}{5.751775in}}%
\pgfpathlineto{\pgfqpoint{7.323380in}{0.554012in}}%
\pgfpathclose%
\pgfusepath{fill}%
\end{pgfscope}%
\begin{pgfscope}%
\pgfpathrectangle{\pgfqpoint{7.323380in}{0.554012in}}{\pgfqpoint{6.387885in}{5.197763in}}%
\pgfusepath{clip}%
\pgfsetbuttcap%
\pgfsetroundjoin%
\definecolor{currentfill}{rgb}{1.000000,0.000000,0.000000}%
\pgfsetfillcolor{currentfill}%
\pgfsetlinewidth{1.003750pt}%
\definecolor{currentstroke}{rgb}{1.000000,0.000000,0.000000}%
\pgfsetstrokecolor{currentstroke}%
\pgfsetdash{}{0pt}%
\pgfsys@defobject{currentmarker}{\pgfqpoint{-0.069444in}{-0.069444in}}{\pgfqpoint{0.069444in}{0.069444in}}{%
\pgfpathmoveto{\pgfqpoint{0.000000in}{-0.069444in}}%
\pgfpathcurveto{\pgfqpoint{0.018417in}{-0.069444in}}{\pgfqpoint{0.036082in}{-0.062127in}}{\pgfqpoint{0.049105in}{-0.049105in}}%
\pgfpathcurveto{\pgfqpoint{0.062127in}{-0.036082in}}{\pgfqpoint{0.069444in}{-0.018417in}}{\pgfqpoint{0.069444in}{0.000000in}}%
\pgfpathcurveto{\pgfqpoint{0.069444in}{0.018417in}}{\pgfqpoint{0.062127in}{0.036082in}}{\pgfqpoint{0.049105in}{0.049105in}}%
\pgfpathcurveto{\pgfqpoint{0.036082in}{0.062127in}}{\pgfqpoint{0.018417in}{0.069444in}}{\pgfqpoint{0.000000in}{0.069444in}}%
\pgfpathcurveto{\pgfqpoint{-0.018417in}{0.069444in}}{\pgfqpoint{-0.036082in}{0.062127in}}{\pgfqpoint{-0.049105in}{0.049105in}}%
\pgfpathcurveto{\pgfqpoint{-0.062127in}{0.036082in}}{\pgfqpoint{-0.069444in}{0.018417in}}{\pgfqpoint{-0.069444in}{0.000000in}}%
\pgfpathcurveto{\pgfqpoint{-0.069444in}{-0.018417in}}{\pgfqpoint{-0.062127in}{-0.036082in}}{\pgfqpoint{-0.049105in}{-0.049105in}}%
\pgfpathcurveto{\pgfqpoint{-0.036082in}{-0.062127in}}{\pgfqpoint{-0.018417in}{-0.069444in}}{\pgfqpoint{0.000000in}{-0.069444in}}%
\pgfpathlineto{\pgfqpoint{0.000000in}{-0.069444in}}%
\pgfpathclose%
\pgfusepath{stroke,fill}%
}%
\begin{pgfscope}%
\pgfsys@transformshift{9.239745in}{2.113341in}%
\pgfsys@useobject{currentmarker}{}%
\end{pgfscope}%
\end{pgfscope}%
\begin{pgfscope}%
\pgfpathrectangle{\pgfqpoint{7.323380in}{0.554012in}}{\pgfqpoint{6.387885in}{5.197763in}}%
\pgfusepath{clip}%
\pgfsetbuttcap%
\pgfsetroundjoin%
\definecolor{currentfill}{rgb}{0.121569,0.466667,0.705882}%
\pgfsetfillcolor{currentfill}%
\pgfsetfillopacity{0.200000}%
\pgfsetlinewidth{0.000000pt}%
\definecolor{currentstroke}{rgb}{0.000000,0.000000,0.000000}%
\pgfsetstrokecolor{currentstroke}%
\pgfsetdash{}{0pt}%
\pgfpathmoveto{\pgfqpoint{7.323380in}{5.751775in}}%
\pgfpathlineto{\pgfqpoint{7.323380in}{2.113341in}}%
\pgfpathlineto{\pgfqpoint{7.329774in}{2.113341in}}%
\pgfpathlineto{\pgfqpoint{7.336168in}{2.113341in}}%
\pgfpathlineto{\pgfqpoint{7.342563in}{2.113341in}}%
\pgfpathlineto{\pgfqpoint{7.348957in}{2.113341in}}%
\pgfpathlineto{\pgfqpoint{7.355351in}{2.113341in}}%
\pgfpathlineto{\pgfqpoint{7.361746in}{2.113341in}}%
\pgfpathlineto{\pgfqpoint{7.368140in}{2.113341in}}%
\pgfpathlineto{\pgfqpoint{7.374534in}{2.113341in}}%
\pgfpathlineto{\pgfqpoint{7.380928in}{2.113341in}}%
\pgfpathlineto{\pgfqpoint{7.387323in}{2.113341in}}%
\pgfpathlineto{\pgfqpoint{7.393717in}{2.113341in}}%
\pgfpathlineto{\pgfqpoint{7.400111in}{2.113341in}}%
\pgfpathlineto{\pgfqpoint{7.406505in}{2.113341in}}%
\pgfpathlineto{\pgfqpoint{7.412900in}{2.113341in}}%
\pgfpathlineto{\pgfqpoint{7.419294in}{2.113341in}}%
\pgfpathlineto{\pgfqpoint{7.425688in}{2.113341in}}%
\pgfpathlineto{\pgfqpoint{7.432083in}{2.113341in}}%
\pgfpathlineto{\pgfqpoint{7.438477in}{2.113341in}}%
\pgfpathlineto{\pgfqpoint{7.444871in}{2.113341in}}%
\pgfpathlineto{\pgfqpoint{7.451265in}{2.113341in}}%
\pgfpathlineto{\pgfqpoint{7.457660in}{2.113341in}}%
\pgfpathlineto{\pgfqpoint{7.464054in}{2.113341in}}%
\pgfpathlineto{\pgfqpoint{7.470448in}{2.113341in}}%
\pgfpathlineto{\pgfqpoint{7.476843in}{2.113341in}}%
\pgfpathlineto{\pgfqpoint{7.483237in}{2.113341in}}%
\pgfpathlineto{\pgfqpoint{7.489631in}{2.113341in}}%
\pgfpathlineto{\pgfqpoint{7.496025in}{2.113341in}}%
\pgfpathlineto{\pgfqpoint{7.502420in}{2.113341in}}%
\pgfpathlineto{\pgfqpoint{7.508814in}{2.113341in}}%
\pgfpathlineto{\pgfqpoint{7.515208in}{2.113341in}}%
\pgfpathlineto{\pgfqpoint{7.521603in}{2.113341in}}%
\pgfpathlineto{\pgfqpoint{7.527997in}{2.113341in}}%
\pgfpathlineto{\pgfqpoint{7.534391in}{2.113341in}}%
\pgfpathlineto{\pgfqpoint{7.540785in}{2.113341in}}%
\pgfpathlineto{\pgfqpoint{7.547180in}{2.113341in}}%
\pgfpathlineto{\pgfqpoint{7.553574in}{2.113341in}}%
\pgfpathlineto{\pgfqpoint{7.559968in}{2.113341in}}%
\pgfpathlineto{\pgfqpoint{7.566362in}{2.113341in}}%
\pgfpathlineto{\pgfqpoint{7.572757in}{2.113341in}}%
\pgfpathlineto{\pgfqpoint{7.579151in}{2.113341in}}%
\pgfpathlineto{\pgfqpoint{7.585545in}{2.113341in}}%
\pgfpathlineto{\pgfqpoint{7.591940in}{2.113341in}}%
\pgfpathlineto{\pgfqpoint{7.598334in}{2.113341in}}%
\pgfpathlineto{\pgfqpoint{7.604728in}{2.113341in}}%
\pgfpathlineto{\pgfqpoint{7.611122in}{2.113341in}}%
\pgfpathlineto{\pgfqpoint{7.617517in}{2.113341in}}%
\pgfpathlineto{\pgfqpoint{7.623911in}{2.113341in}}%
\pgfpathlineto{\pgfqpoint{7.630305in}{2.113341in}}%
\pgfpathlineto{\pgfqpoint{7.636700in}{2.113341in}}%
\pgfpathlineto{\pgfqpoint{7.643094in}{2.113341in}}%
\pgfpathlineto{\pgfqpoint{7.649488in}{2.113341in}}%
\pgfpathlineto{\pgfqpoint{7.655882in}{2.113341in}}%
\pgfpathlineto{\pgfqpoint{7.662277in}{2.113341in}}%
\pgfpathlineto{\pgfqpoint{7.668671in}{2.113341in}}%
\pgfpathlineto{\pgfqpoint{7.675065in}{2.113341in}}%
\pgfpathlineto{\pgfqpoint{7.681460in}{2.113341in}}%
\pgfpathlineto{\pgfqpoint{7.687854in}{2.113341in}}%
\pgfpathlineto{\pgfqpoint{7.694248in}{2.113341in}}%
\pgfpathlineto{\pgfqpoint{7.700642in}{2.113341in}}%
\pgfpathlineto{\pgfqpoint{7.707037in}{2.113341in}}%
\pgfpathlineto{\pgfqpoint{7.713431in}{2.113341in}}%
\pgfpathlineto{\pgfqpoint{7.719825in}{2.113341in}}%
\pgfpathlineto{\pgfqpoint{7.726219in}{2.113341in}}%
\pgfpathlineto{\pgfqpoint{7.732614in}{2.113341in}}%
\pgfpathlineto{\pgfqpoint{7.739008in}{2.113341in}}%
\pgfpathlineto{\pgfqpoint{7.745402in}{2.113341in}}%
\pgfpathlineto{\pgfqpoint{7.751797in}{2.113341in}}%
\pgfpathlineto{\pgfqpoint{7.758191in}{2.113341in}}%
\pgfpathlineto{\pgfqpoint{7.764585in}{2.113341in}}%
\pgfpathlineto{\pgfqpoint{7.770979in}{2.113341in}}%
\pgfpathlineto{\pgfqpoint{7.777374in}{2.113341in}}%
\pgfpathlineto{\pgfqpoint{7.783768in}{2.113341in}}%
\pgfpathlineto{\pgfqpoint{7.790162in}{2.113341in}}%
\pgfpathlineto{\pgfqpoint{7.796557in}{2.113341in}}%
\pgfpathlineto{\pgfqpoint{7.802951in}{2.113341in}}%
\pgfpathlineto{\pgfqpoint{7.809345in}{2.113341in}}%
\pgfpathlineto{\pgfqpoint{7.815739in}{2.113341in}}%
\pgfpathlineto{\pgfqpoint{7.822134in}{2.113341in}}%
\pgfpathlineto{\pgfqpoint{7.828528in}{2.113341in}}%
\pgfpathlineto{\pgfqpoint{7.834922in}{2.113341in}}%
\pgfpathlineto{\pgfqpoint{7.841317in}{2.113341in}}%
\pgfpathlineto{\pgfqpoint{7.847711in}{2.113341in}}%
\pgfpathlineto{\pgfqpoint{7.854105in}{2.113341in}}%
\pgfpathlineto{\pgfqpoint{7.860499in}{2.113341in}}%
\pgfpathlineto{\pgfqpoint{7.866894in}{2.113341in}}%
\pgfpathlineto{\pgfqpoint{7.873288in}{2.113341in}}%
\pgfpathlineto{\pgfqpoint{7.879682in}{2.113341in}}%
\pgfpathlineto{\pgfqpoint{7.886076in}{2.113341in}}%
\pgfpathlineto{\pgfqpoint{7.892471in}{2.113341in}}%
\pgfpathlineto{\pgfqpoint{7.898865in}{2.113341in}}%
\pgfpathlineto{\pgfqpoint{7.905259in}{2.113341in}}%
\pgfpathlineto{\pgfqpoint{7.911654in}{2.113341in}}%
\pgfpathlineto{\pgfqpoint{7.918048in}{2.113341in}}%
\pgfpathlineto{\pgfqpoint{7.924442in}{2.113341in}}%
\pgfpathlineto{\pgfqpoint{7.930836in}{2.113341in}}%
\pgfpathlineto{\pgfqpoint{7.937231in}{2.113341in}}%
\pgfpathlineto{\pgfqpoint{7.943625in}{2.113341in}}%
\pgfpathlineto{\pgfqpoint{7.950019in}{2.113341in}}%
\pgfpathlineto{\pgfqpoint{7.956414in}{2.113341in}}%
\pgfpathlineto{\pgfqpoint{7.962808in}{2.113341in}}%
\pgfpathlineto{\pgfqpoint{7.969202in}{2.113341in}}%
\pgfpathlineto{\pgfqpoint{7.975596in}{2.113341in}}%
\pgfpathlineto{\pgfqpoint{7.981991in}{2.113341in}}%
\pgfpathlineto{\pgfqpoint{7.988385in}{2.113341in}}%
\pgfpathlineto{\pgfqpoint{7.994779in}{2.113341in}}%
\pgfpathlineto{\pgfqpoint{8.001173in}{2.113341in}}%
\pgfpathlineto{\pgfqpoint{8.007568in}{2.113341in}}%
\pgfpathlineto{\pgfqpoint{8.013962in}{2.113341in}}%
\pgfpathlineto{\pgfqpoint{8.020356in}{2.113341in}}%
\pgfpathlineto{\pgfqpoint{8.026751in}{2.113341in}}%
\pgfpathlineto{\pgfqpoint{8.033145in}{2.113341in}}%
\pgfpathlineto{\pgfqpoint{8.039539in}{2.113341in}}%
\pgfpathlineto{\pgfqpoint{8.045933in}{2.113341in}}%
\pgfpathlineto{\pgfqpoint{8.052328in}{2.113341in}}%
\pgfpathlineto{\pgfqpoint{8.058722in}{2.113341in}}%
\pgfpathlineto{\pgfqpoint{8.065116in}{2.113341in}}%
\pgfpathlineto{\pgfqpoint{8.071511in}{2.113341in}}%
\pgfpathlineto{\pgfqpoint{8.077905in}{2.113341in}}%
\pgfpathlineto{\pgfqpoint{8.084299in}{2.113341in}}%
\pgfpathlineto{\pgfqpoint{8.090693in}{2.113341in}}%
\pgfpathlineto{\pgfqpoint{8.097088in}{2.113341in}}%
\pgfpathlineto{\pgfqpoint{8.103482in}{2.113341in}}%
\pgfpathlineto{\pgfqpoint{8.109876in}{2.113341in}}%
\pgfpathlineto{\pgfqpoint{8.116271in}{2.113341in}}%
\pgfpathlineto{\pgfqpoint{8.122665in}{2.113341in}}%
\pgfpathlineto{\pgfqpoint{8.129059in}{2.113341in}}%
\pgfpathlineto{\pgfqpoint{8.135453in}{2.113341in}}%
\pgfpathlineto{\pgfqpoint{8.141848in}{2.113341in}}%
\pgfpathlineto{\pgfqpoint{8.148242in}{2.113341in}}%
\pgfpathlineto{\pgfqpoint{8.154636in}{2.113341in}}%
\pgfpathlineto{\pgfqpoint{8.161030in}{2.113341in}}%
\pgfpathlineto{\pgfqpoint{8.167425in}{2.113341in}}%
\pgfpathlineto{\pgfqpoint{8.173819in}{2.113341in}}%
\pgfpathlineto{\pgfqpoint{8.180213in}{2.113341in}}%
\pgfpathlineto{\pgfqpoint{8.186608in}{2.113341in}}%
\pgfpathlineto{\pgfqpoint{8.193002in}{2.113341in}}%
\pgfpathlineto{\pgfqpoint{8.199396in}{2.113341in}}%
\pgfpathlineto{\pgfqpoint{8.205790in}{2.113341in}}%
\pgfpathlineto{\pgfqpoint{8.212185in}{2.113341in}}%
\pgfpathlineto{\pgfqpoint{8.218579in}{2.113341in}}%
\pgfpathlineto{\pgfqpoint{8.224973in}{2.113341in}}%
\pgfpathlineto{\pgfqpoint{8.231368in}{2.113341in}}%
\pgfpathlineto{\pgfqpoint{8.237762in}{2.113341in}}%
\pgfpathlineto{\pgfqpoint{8.244156in}{2.113341in}}%
\pgfpathlineto{\pgfqpoint{8.250550in}{2.113341in}}%
\pgfpathlineto{\pgfqpoint{8.256945in}{2.113341in}}%
\pgfpathlineto{\pgfqpoint{8.263339in}{2.113341in}}%
\pgfpathlineto{\pgfqpoint{8.269733in}{2.113341in}}%
\pgfpathlineto{\pgfqpoint{8.276128in}{2.113341in}}%
\pgfpathlineto{\pgfqpoint{8.282522in}{2.113341in}}%
\pgfpathlineto{\pgfqpoint{8.288916in}{2.113341in}}%
\pgfpathlineto{\pgfqpoint{8.295310in}{2.113341in}}%
\pgfpathlineto{\pgfqpoint{8.301705in}{2.113341in}}%
\pgfpathlineto{\pgfqpoint{8.308099in}{2.113341in}}%
\pgfpathlineto{\pgfqpoint{8.314493in}{2.113341in}}%
\pgfpathlineto{\pgfqpoint{8.320887in}{2.113341in}}%
\pgfpathlineto{\pgfqpoint{8.327282in}{2.113341in}}%
\pgfpathlineto{\pgfqpoint{8.333676in}{2.113341in}}%
\pgfpathlineto{\pgfqpoint{8.340070in}{2.113341in}}%
\pgfpathlineto{\pgfqpoint{8.346465in}{2.113341in}}%
\pgfpathlineto{\pgfqpoint{8.352859in}{2.113341in}}%
\pgfpathlineto{\pgfqpoint{8.359253in}{2.113341in}}%
\pgfpathlineto{\pgfqpoint{8.365647in}{2.113341in}}%
\pgfpathlineto{\pgfqpoint{8.372042in}{2.113341in}}%
\pgfpathlineto{\pgfqpoint{8.378436in}{2.113341in}}%
\pgfpathlineto{\pgfqpoint{8.384830in}{2.113341in}}%
\pgfpathlineto{\pgfqpoint{8.391225in}{2.113341in}}%
\pgfpathlineto{\pgfqpoint{8.397619in}{2.113341in}}%
\pgfpathlineto{\pgfqpoint{8.404013in}{2.113341in}}%
\pgfpathlineto{\pgfqpoint{8.410407in}{2.113341in}}%
\pgfpathlineto{\pgfqpoint{8.416802in}{2.113341in}}%
\pgfpathlineto{\pgfqpoint{8.423196in}{2.113341in}}%
\pgfpathlineto{\pgfqpoint{8.429590in}{2.113341in}}%
\pgfpathlineto{\pgfqpoint{8.435985in}{2.113341in}}%
\pgfpathlineto{\pgfqpoint{8.442379in}{2.113341in}}%
\pgfpathlineto{\pgfqpoint{8.448773in}{2.113341in}}%
\pgfpathlineto{\pgfqpoint{8.455167in}{2.113341in}}%
\pgfpathlineto{\pgfqpoint{8.461562in}{2.113341in}}%
\pgfpathlineto{\pgfqpoint{8.467956in}{2.113341in}}%
\pgfpathlineto{\pgfqpoint{8.474350in}{2.113341in}}%
\pgfpathlineto{\pgfqpoint{8.480744in}{2.113341in}}%
\pgfpathlineto{\pgfqpoint{8.487139in}{2.113341in}}%
\pgfpathlineto{\pgfqpoint{8.493533in}{2.113341in}}%
\pgfpathlineto{\pgfqpoint{8.499927in}{2.113341in}}%
\pgfpathlineto{\pgfqpoint{8.506322in}{2.113341in}}%
\pgfpathlineto{\pgfqpoint{8.512716in}{2.113341in}}%
\pgfpathlineto{\pgfqpoint{8.519110in}{2.113341in}}%
\pgfpathlineto{\pgfqpoint{8.525504in}{2.113341in}}%
\pgfpathlineto{\pgfqpoint{8.531899in}{2.113341in}}%
\pgfpathlineto{\pgfqpoint{8.538293in}{2.113341in}}%
\pgfpathlineto{\pgfqpoint{8.544687in}{2.113341in}}%
\pgfpathlineto{\pgfqpoint{8.551082in}{2.113341in}}%
\pgfpathlineto{\pgfqpoint{8.557476in}{2.113341in}}%
\pgfpathlineto{\pgfqpoint{8.563870in}{2.113341in}}%
\pgfpathlineto{\pgfqpoint{8.570264in}{2.113341in}}%
\pgfpathlineto{\pgfqpoint{8.576659in}{2.113341in}}%
\pgfpathlineto{\pgfqpoint{8.583053in}{2.113341in}}%
\pgfpathlineto{\pgfqpoint{8.589447in}{2.113341in}}%
\pgfpathlineto{\pgfqpoint{8.595841in}{2.113341in}}%
\pgfpathlineto{\pgfqpoint{8.602236in}{2.113341in}}%
\pgfpathlineto{\pgfqpoint{8.608630in}{2.113341in}}%
\pgfpathlineto{\pgfqpoint{8.615024in}{2.113341in}}%
\pgfpathlineto{\pgfqpoint{8.621419in}{2.113341in}}%
\pgfpathlineto{\pgfqpoint{8.627813in}{2.113341in}}%
\pgfpathlineto{\pgfqpoint{8.634207in}{2.113341in}}%
\pgfpathlineto{\pgfqpoint{8.640601in}{2.113341in}}%
\pgfpathlineto{\pgfqpoint{8.646996in}{2.113341in}}%
\pgfpathlineto{\pgfqpoint{8.653390in}{2.113341in}}%
\pgfpathlineto{\pgfqpoint{8.659784in}{2.113341in}}%
\pgfpathlineto{\pgfqpoint{8.666179in}{2.113341in}}%
\pgfpathlineto{\pgfqpoint{8.672573in}{2.113341in}}%
\pgfpathlineto{\pgfqpoint{8.678967in}{2.113341in}}%
\pgfpathlineto{\pgfqpoint{8.685361in}{2.113341in}}%
\pgfpathlineto{\pgfqpoint{8.691756in}{2.113341in}}%
\pgfpathlineto{\pgfqpoint{8.698150in}{2.113341in}}%
\pgfpathlineto{\pgfqpoint{8.704544in}{2.113341in}}%
\pgfpathlineto{\pgfqpoint{8.710939in}{2.113341in}}%
\pgfpathlineto{\pgfqpoint{8.717333in}{2.113341in}}%
\pgfpathlineto{\pgfqpoint{8.723727in}{2.113341in}}%
\pgfpathlineto{\pgfqpoint{8.730121in}{2.113341in}}%
\pgfpathlineto{\pgfqpoint{8.736516in}{2.113341in}}%
\pgfpathlineto{\pgfqpoint{8.742910in}{2.113341in}}%
\pgfpathlineto{\pgfqpoint{8.749304in}{2.113341in}}%
\pgfpathlineto{\pgfqpoint{8.755698in}{2.113341in}}%
\pgfpathlineto{\pgfqpoint{8.762093in}{2.113341in}}%
\pgfpathlineto{\pgfqpoint{8.768487in}{2.113341in}}%
\pgfpathlineto{\pgfqpoint{8.774881in}{2.113341in}}%
\pgfpathlineto{\pgfqpoint{8.781276in}{2.113341in}}%
\pgfpathlineto{\pgfqpoint{8.787670in}{2.113341in}}%
\pgfpathlineto{\pgfqpoint{8.794064in}{2.113341in}}%
\pgfpathlineto{\pgfqpoint{8.800458in}{2.113341in}}%
\pgfpathlineto{\pgfqpoint{8.806853in}{2.113341in}}%
\pgfpathlineto{\pgfqpoint{8.813247in}{2.113341in}}%
\pgfpathlineto{\pgfqpoint{8.819641in}{2.113341in}}%
\pgfpathlineto{\pgfqpoint{8.826036in}{2.113341in}}%
\pgfpathlineto{\pgfqpoint{8.832430in}{2.113341in}}%
\pgfpathlineto{\pgfqpoint{8.838824in}{2.113341in}}%
\pgfpathlineto{\pgfqpoint{8.845218in}{2.113341in}}%
\pgfpathlineto{\pgfqpoint{8.851613in}{2.113341in}}%
\pgfpathlineto{\pgfqpoint{8.858007in}{2.113341in}}%
\pgfpathlineto{\pgfqpoint{8.864401in}{2.113341in}}%
\pgfpathlineto{\pgfqpoint{8.870796in}{2.113341in}}%
\pgfpathlineto{\pgfqpoint{8.877190in}{2.113341in}}%
\pgfpathlineto{\pgfqpoint{8.883584in}{2.113341in}}%
\pgfpathlineto{\pgfqpoint{8.889978in}{2.113341in}}%
\pgfpathlineto{\pgfqpoint{8.896373in}{2.113341in}}%
\pgfpathlineto{\pgfqpoint{8.902767in}{2.113341in}}%
\pgfpathlineto{\pgfqpoint{8.909161in}{2.113341in}}%
\pgfpathlineto{\pgfqpoint{8.915555in}{2.113341in}}%
\pgfpathlineto{\pgfqpoint{8.921950in}{2.113341in}}%
\pgfpathlineto{\pgfqpoint{8.928344in}{2.113341in}}%
\pgfpathlineto{\pgfqpoint{8.934738in}{2.113341in}}%
\pgfpathlineto{\pgfqpoint{8.941133in}{2.113341in}}%
\pgfpathlineto{\pgfqpoint{8.947527in}{2.113341in}}%
\pgfpathlineto{\pgfqpoint{8.953921in}{2.113341in}}%
\pgfpathlineto{\pgfqpoint{8.960315in}{2.113341in}}%
\pgfpathlineto{\pgfqpoint{8.966710in}{2.113341in}}%
\pgfpathlineto{\pgfqpoint{8.973104in}{2.113341in}}%
\pgfpathlineto{\pgfqpoint{8.979498in}{2.113341in}}%
\pgfpathlineto{\pgfqpoint{8.985893in}{2.113341in}}%
\pgfpathlineto{\pgfqpoint{8.992287in}{2.113341in}}%
\pgfpathlineto{\pgfqpoint{8.998681in}{2.113341in}}%
\pgfpathlineto{\pgfqpoint{9.005075in}{2.113341in}}%
\pgfpathlineto{\pgfqpoint{9.011470in}{2.113341in}}%
\pgfpathlineto{\pgfqpoint{9.017864in}{2.113341in}}%
\pgfpathlineto{\pgfqpoint{9.024258in}{2.113341in}}%
\pgfpathlineto{\pgfqpoint{9.030653in}{2.113341in}}%
\pgfpathlineto{\pgfqpoint{9.037047in}{2.113341in}}%
\pgfpathlineto{\pgfqpoint{9.043441in}{2.113341in}}%
\pgfpathlineto{\pgfqpoint{9.049835in}{2.113341in}}%
\pgfpathlineto{\pgfqpoint{9.056230in}{2.113341in}}%
\pgfpathlineto{\pgfqpoint{9.062624in}{2.113341in}}%
\pgfpathlineto{\pgfqpoint{9.069018in}{2.113341in}}%
\pgfpathlineto{\pgfqpoint{9.075412in}{2.113341in}}%
\pgfpathlineto{\pgfqpoint{9.081807in}{2.113341in}}%
\pgfpathlineto{\pgfqpoint{9.088201in}{2.113341in}}%
\pgfpathlineto{\pgfqpoint{9.094595in}{2.113341in}}%
\pgfpathlineto{\pgfqpoint{9.100990in}{2.113341in}}%
\pgfpathlineto{\pgfqpoint{9.107384in}{2.113341in}}%
\pgfpathlineto{\pgfqpoint{9.113778in}{2.113341in}}%
\pgfpathlineto{\pgfqpoint{9.120172in}{2.113341in}}%
\pgfpathlineto{\pgfqpoint{9.126567in}{2.113341in}}%
\pgfpathlineto{\pgfqpoint{9.132961in}{2.113341in}}%
\pgfpathlineto{\pgfqpoint{9.139355in}{2.113341in}}%
\pgfpathlineto{\pgfqpoint{9.145750in}{2.113341in}}%
\pgfpathlineto{\pgfqpoint{9.152144in}{2.113341in}}%
\pgfpathlineto{\pgfqpoint{9.158538in}{2.113341in}}%
\pgfpathlineto{\pgfqpoint{9.164932in}{2.113341in}}%
\pgfpathlineto{\pgfqpoint{9.171327in}{2.113341in}}%
\pgfpathlineto{\pgfqpoint{9.177721in}{2.113341in}}%
\pgfpathlineto{\pgfqpoint{9.184115in}{2.113341in}}%
\pgfpathlineto{\pgfqpoint{9.190509in}{2.113341in}}%
\pgfpathlineto{\pgfqpoint{9.196904in}{2.113341in}}%
\pgfpathlineto{\pgfqpoint{9.203298in}{2.113341in}}%
\pgfpathlineto{\pgfqpoint{9.209692in}{2.113341in}}%
\pgfpathlineto{\pgfqpoint{9.216087in}{2.113341in}}%
\pgfpathlineto{\pgfqpoint{9.222481in}{2.113341in}}%
\pgfpathlineto{\pgfqpoint{9.228875in}{2.113341in}}%
\pgfpathlineto{\pgfqpoint{9.235269in}{2.113341in}}%
\pgfpathlineto{\pgfqpoint{9.235269in}{4.196088in}}%
\pgfpathlineto{\pgfqpoint{9.235269in}{4.196088in}}%
\pgfpathlineto{\pgfqpoint{9.228875in}{4.201291in}}%
\pgfpathlineto{\pgfqpoint{9.222481in}{4.206494in}}%
\pgfpathlineto{\pgfqpoint{9.216087in}{4.211697in}}%
\pgfpathlineto{\pgfqpoint{9.209692in}{4.216900in}}%
\pgfpathlineto{\pgfqpoint{9.203298in}{4.222103in}}%
\pgfpathlineto{\pgfqpoint{9.196904in}{4.227306in}}%
\pgfpathlineto{\pgfqpoint{9.190509in}{4.232509in}}%
\pgfpathlineto{\pgfqpoint{9.184115in}{4.237712in}}%
\pgfpathlineto{\pgfqpoint{9.177721in}{4.242915in}}%
\pgfpathlineto{\pgfqpoint{9.171327in}{4.248118in}}%
\pgfpathlineto{\pgfqpoint{9.164932in}{4.253321in}}%
\pgfpathlineto{\pgfqpoint{9.158538in}{4.258524in}}%
\pgfpathlineto{\pgfqpoint{9.152144in}{4.263726in}}%
\pgfpathlineto{\pgfqpoint{9.145750in}{4.268929in}}%
\pgfpathlineto{\pgfqpoint{9.139355in}{4.274132in}}%
\pgfpathlineto{\pgfqpoint{9.132961in}{4.279335in}}%
\pgfpathlineto{\pgfqpoint{9.126567in}{4.284538in}}%
\pgfpathlineto{\pgfqpoint{9.120172in}{4.289741in}}%
\pgfpathlineto{\pgfqpoint{9.113778in}{4.294944in}}%
\pgfpathlineto{\pgfqpoint{9.107384in}{4.300147in}}%
\pgfpathlineto{\pgfqpoint{9.100990in}{4.305350in}}%
\pgfpathlineto{\pgfqpoint{9.094595in}{4.310553in}}%
\pgfpathlineto{\pgfqpoint{9.088201in}{4.315756in}}%
\pgfpathlineto{\pgfqpoint{9.081807in}{4.320959in}}%
\pgfpathlineto{\pgfqpoint{9.075412in}{4.326162in}}%
\pgfpathlineto{\pgfqpoint{9.069018in}{4.331365in}}%
\pgfpathlineto{\pgfqpoint{9.062624in}{4.336568in}}%
\pgfpathlineto{\pgfqpoint{9.056230in}{4.341771in}}%
\pgfpathlineto{\pgfqpoint{9.049835in}{4.346974in}}%
\pgfpathlineto{\pgfqpoint{9.043441in}{4.352177in}}%
\pgfpathlineto{\pgfqpoint{9.037047in}{4.357380in}}%
\pgfpathlineto{\pgfqpoint{9.030653in}{4.362583in}}%
\pgfpathlineto{\pgfqpoint{9.024258in}{4.367786in}}%
\pgfpathlineto{\pgfqpoint{9.017864in}{4.372989in}}%
\pgfpathlineto{\pgfqpoint{9.011470in}{4.378192in}}%
\pgfpathlineto{\pgfqpoint{9.005075in}{4.383395in}}%
\pgfpathlineto{\pgfqpoint{8.998681in}{4.388598in}}%
\pgfpathlineto{\pgfqpoint{8.992287in}{4.393801in}}%
\pgfpathlineto{\pgfqpoint{8.985893in}{4.399004in}}%
\pgfpathlineto{\pgfqpoint{8.979498in}{4.404207in}}%
\pgfpathlineto{\pgfqpoint{8.973104in}{4.409410in}}%
\pgfpathlineto{\pgfqpoint{8.966710in}{4.414612in}}%
\pgfpathlineto{\pgfqpoint{8.960315in}{4.419815in}}%
\pgfpathlineto{\pgfqpoint{8.953921in}{4.425018in}}%
\pgfpathlineto{\pgfqpoint{8.947527in}{4.430221in}}%
\pgfpathlineto{\pgfqpoint{8.941133in}{4.435424in}}%
\pgfpathlineto{\pgfqpoint{8.934738in}{4.440627in}}%
\pgfpathlineto{\pgfqpoint{8.928344in}{4.445830in}}%
\pgfpathlineto{\pgfqpoint{8.921950in}{4.451033in}}%
\pgfpathlineto{\pgfqpoint{8.915555in}{4.456236in}}%
\pgfpathlineto{\pgfqpoint{8.909161in}{4.461439in}}%
\pgfpathlineto{\pgfqpoint{8.902767in}{4.466642in}}%
\pgfpathlineto{\pgfqpoint{8.896373in}{4.471845in}}%
\pgfpathlineto{\pgfqpoint{8.889978in}{4.477048in}}%
\pgfpathlineto{\pgfqpoint{8.883584in}{4.482251in}}%
\pgfpathlineto{\pgfqpoint{8.877190in}{4.487454in}}%
\pgfpathlineto{\pgfqpoint{8.870796in}{4.492657in}}%
\pgfpathlineto{\pgfqpoint{8.864401in}{4.497860in}}%
\pgfpathlineto{\pgfqpoint{8.858007in}{4.503063in}}%
\pgfpathlineto{\pgfqpoint{8.851613in}{4.508266in}}%
\pgfpathlineto{\pgfqpoint{8.845218in}{4.513469in}}%
\pgfpathlineto{\pgfqpoint{8.838824in}{4.518672in}}%
\pgfpathlineto{\pgfqpoint{8.832430in}{4.523875in}}%
\pgfpathlineto{\pgfqpoint{8.826036in}{4.529078in}}%
\pgfpathlineto{\pgfqpoint{8.819641in}{4.534281in}}%
\pgfpathlineto{\pgfqpoint{8.813247in}{4.539484in}}%
\pgfpathlineto{\pgfqpoint{8.806853in}{4.544687in}}%
\pgfpathlineto{\pgfqpoint{8.800458in}{4.549890in}}%
\pgfpathlineto{\pgfqpoint{8.794064in}{4.555093in}}%
\pgfpathlineto{\pgfqpoint{8.787670in}{4.560296in}}%
\pgfpathlineto{\pgfqpoint{8.781276in}{4.565499in}}%
\pgfpathlineto{\pgfqpoint{8.774881in}{4.570701in}}%
\pgfpathlineto{\pgfqpoint{8.768487in}{4.575904in}}%
\pgfpathlineto{\pgfqpoint{8.762093in}{4.581107in}}%
\pgfpathlineto{\pgfqpoint{8.755698in}{4.586310in}}%
\pgfpathlineto{\pgfqpoint{8.749304in}{4.591513in}}%
\pgfpathlineto{\pgfqpoint{8.742910in}{4.596716in}}%
\pgfpathlineto{\pgfqpoint{8.736516in}{4.601919in}}%
\pgfpathlineto{\pgfqpoint{8.730121in}{4.607122in}}%
\pgfpathlineto{\pgfqpoint{8.723727in}{4.612325in}}%
\pgfpathlineto{\pgfqpoint{8.717333in}{4.617528in}}%
\pgfpathlineto{\pgfqpoint{8.710939in}{4.622731in}}%
\pgfpathlineto{\pgfqpoint{8.704544in}{4.627934in}}%
\pgfpathlineto{\pgfqpoint{8.698150in}{4.633137in}}%
\pgfpathlineto{\pgfqpoint{8.691756in}{4.638340in}}%
\pgfpathlineto{\pgfqpoint{8.685361in}{4.643543in}}%
\pgfpathlineto{\pgfqpoint{8.678967in}{4.648746in}}%
\pgfpathlineto{\pgfqpoint{8.672573in}{4.653949in}}%
\pgfpathlineto{\pgfqpoint{8.666179in}{4.659152in}}%
\pgfpathlineto{\pgfqpoint{8.659784in}{4.664355in}}%
\pgfpathlineto{\pgfqpoint{8.653390in}{4.669558in}}%
\pgfpathlineto{\pgfqpoint{8.646996in}{4.674761in}}%
\pgfpathlineto{\pgfqpoint{8.640601in}{4.679964in}}%
\pgfpathlineto{\pgfqpoint{8.634207in}{4.685167in}}%
\pgfpathlineto{\pgfqpoint{8.627813in}{4.690370in}}%
\pgfpathlineto{\pgfqpoint{8.621419in}{4.695573in}}%
\pgfpathlineto{\pgfqpoint{8.615024in}{4.700776in}}%
\pgfpathlineto{\pgfqpoint{8.608630in}{4.705979in}}%
\pgfpathlineto{\pgfqpoint{8.602236in}{4.711182in}}%
\pgfpathlineto{\pgfqpoint{8.595841in}{4.716385in}}%
\pgfpathlineto{\pgfqpoint{8.589447in}{4.721587in}}%
\pgfpathlineto{\pgfqpoint{8.583053in}{4.726790in}}%
\pgfpathlineto{\pgfqpoint{8.576659in}{4.731993in}}%
\pgfpathlineto{\pgfqpoint{8.570264in}{4.737196in}}%
\pgfpathlineto{\pgfqpoint{8.563870in}{4.742399in}}%
\pgfpathlineto{\pgfqpoint{8.557476in}{4.747602in}}%
\pgfpathlineto{\pgfqpoint{8.551082in}{4.752805in}}%
\pgfpathlineto{\pgfqpoint{8.544687in}{4.758008in}}%
\pgfpathlineto{\pgfqpoint{8.538293in}{4.763211in}}%
\pgfpathlineto{\pgfqpoint{8.531899in}{4.768414in}}%
\pgfpathlineto{\pgfqpoint{8.525504in}{4.773617in}}%
\pgfpathlineto{\pgfqpoint{8.519110in}{4.778820in}}%
\pgfpathlineto{\pgfqpoint{8.512716in}{4.784023in}}%
\pgfpathlineto{\pgfqpoint{8.506322in}{4.789226in}}%
\pgfpathlineto{\pgfqpoint{8.499927in}{4.794429in}}%
\pgfpathlineto{\pgfqpoint{8.493533in}{4.799632in}}%
\pgfpathlineto{\pgfqpoint{8.487139in}{4.804835in}}%
\pgfpathlineto{\pgfqpoint{8.480744in}{4.810038in}}%
\pgfpathlineto{\pgfqpoint{8.474350in}{4.815241in}}%
\pgfpathlineto{\pgfqpoint{8.467956in}{4.820444in}}%
\pgfpathlineto{\pgfqpoint{8.461562in}{4.825647in}}%
\pgfpathlineto{\pgfqpoint{8.455167in}{4.830850in}}%
\pgfpathlineto{\pgfqpoint{8.448773in}{4.836053in}}%
\pgfpathlineto{\pgfqpoint{8.442379in}{4.841256in}}%
\pgfpathlineto{\pgfqpoint{8.435985in}{4.846459in}}%
\pgfpathlineto{\pgfqpoint{8.429590in}{4.851662in}}%
\pgfpathlineto{\pgfqpoint{8.423196in}{4.856865in}}%
\pgfpathlineto{\pgfqpoint{8.416802in}{4.862068in}}%
\pgfpathlineto{\pgfqpoint{8.410407in}{4.867271in}}%
\pgfpathlineto{\pgfqpoint{8.404013in}{4.872473in}}%
\pgfpathlineto{\pgfqpoint{8.397619in}{4.877676in}}%
\pgfpathlineto{\pgfqpoint{8.391225in}{4.882879in}}%
\pgfpathlineto{\pgfqpoint{8.384830in}{4.888082in}}%
\pgfpathlineto{\pgfqpoint{8.378436in}{4.893285in}}%
\pgfpathlineto{\pgfqpoint{8.372042in}{4.898488in}}%
\pgfpathlineto{\pgfqpoint{8.365647in}{4.903691in}}%
\pgfpathlineto{\pgfqpoint{8.359253in}{4.908894in}}%
\pgfpathlineto{\pgfqpoint{8.352859in}{4.914097in}}%
\pgfpathlineto{\pgfqpoint{8.346465in}{4.919300in}}%
\pgfpathlineto{\pgfqpoint{8.340070in}{4.924503in}}%
\pgfpathlineto{\pgfqpoint{8.333676in}{4.929706in}}%
\pgfpathlineto{\pgfqpoint{8.327282in}{4.934909in}}%
\pgfpathlineto{\pgfqpoint{8.320887in}{4.940112in}}%
\pgfpathlineto{\pgfqpoint{8.314493in}{4.945315in}}%
\pgfpathlineto{\pgfqpoint{8.308099in}{4.950518in}}%
\pgfpathlineto{\pgfqpoint{8.301705in}{4.955721in}}%
\pgfpathlineto{\pgfqpoint{8.295310in}{4.960924in}}%
\pgfpathlineto{\pgfqpoint{8.288916in}{4.966127in}}%
\pgfpathlineto{\pgfqpoint{8.282522in}{4.971330in}}%
\pgfpathlineto{\pgfqpoint{8.276128in}{4.976533in}}%
\pgfpathlineto{\pgfqpoint{8.269733in}{4.981736in}}%
\pgfpathlineto{\pgfqpoint{8.263339in}{4.986939in}}%
\pgfpathlineto{\pgfqpoint{8.256945in}{4.992142in}}%
\pgfpathlineto{\pgfqpoint{8.250550in}{4.997345in}}%
\pgfpathlineto{\pgfqpoint{8.244156in}{5.002548in}}%
\pgfpathlineto{\pgfqpoint{8.237762in}{5.007751in}}%
\pgfpathlineto{\pgfqpoint{8.231368in}{5.012954in}}%
\pgfpathlineto{\pgfqpoint{8.224973in}{5.018157in}}%
\pgfpathlineto{\pgfqpoint{8.218579in}{5.023359in}}%
\pgfpathlineto{\pgfqpoint{8.212185in}{5.028562in}}%
\pgfpathlineto{\pgfqpoint{8.205790in}{5.033765in}}%
\pgfpathlineto{\pgfqpoint{8.199396in}{5.038968in}}%
\pgfpathlineto{\pgfqpoint{8.193002in}{5.044171in}}%
\pgfpathlineto{\pgfqpoint{8.186608in}{5.049374in}}%
\pgfpathlineto{\pgfqpoint{8.180213in}{5.054577in}}%
\pgfpathlineto{\pgfqpoint{8.173819in}{5.059780in}}%
\pgfpathlineto{\pgfqpoint{8.167425in}{5.064983in}}%
\pgfpathlineto{\pgfqpoint{8.161030in}{5.070186in}}%
\pgfpathlineto{\pgfqpoint{8.154636in}{5.075389in}}%
\pgfpathlineto{\pgfqpoint{8.148242in}{5.080592in}}%
\pgfpathlineto{\pgfqpoint{8.141848in}{5.085795in}}%
\pgfpathlineto{\pgfqpoint{8.135453in}{5.090998in}}%
\pgfpathlineto{\pgfqpoint{8.129059in}{5.096201in}}%
\pgfpathlineto{\pgfqpoint{8.122665in}{5.101404in}}%
\pgfpathlineto{\pgfqpoint{8.116271in}{5.106607in}}%
\pgfpathlineto{\pgfqpoint{8.109876in}{5.111810in}}%
\pgfpathlineto{\pgfqpoint{8.103482in}{5.117013in}}%
\pgfpathlineto{\pgfqpoint{8.097088in}{5.122216in}}%
\pgfpathlineto{\pgfqpoint{8.090693in}{5.127419in}}%
\pgfpathlineto{\pgfqpoint{8.084299in}{5.132622in}}%
\pgfpathlineto{\pgfqpoint{8.077905in}{5.137825in}}%
\pgfpathlineto{\pgfqpoint{8.071511in}{5.143028in}}%
\pgfpathlineto{\pgfqpoint{8.065116in}{5.148231in}}%
\pgfpathlineto{\pgfqpoint{8.058722in}{5.153434in}}%
\pgfpathlineto{\pgfqpoint{8.052328in}{5.158637in}}%
\pgfpathlineto{\pgfqpoint{8.045933in}{5.163840in}}%
\pgfpathlineto{\pgfqpoint{8.039539in}{5.169043in}}%
\pgfpathlineto{\pgfqpoint{8.033145in}{5.174246in}}%
\pgfpathlineto{\pgfqpoint{8.026751in}{5.179448in}}%
\pgfpathlineto{\pgfqpoint{8.020356in}{5.184651in}}%
\pgfpathlineto{\pgfqpoint{8.013962in}{5.189854in}}%
\pgfpathlineto{\pgfqpoint{8.007568in}{5.195057in}}%
\pgfpathlineto{\pgfqpoint{8.001173in}{5.200260in}}%
\pgfpathlineto{\pgfqpoint{7.994779in}{5.205463in}}%
\pgfpathlineto{\pgfqpoint{7.988385in}{5.210666in}}%
\pgfpathlineto{\pgfqpoint{7.981991in}{5.215869in}}%
\pgfpathlineto{\pgfqpoint{7.975596in}{5.221072in}}%
\pgfpathlineto{\pgfqpoint{7.969202in}{5.226275in}}%
\pgfpathlineto{\pgfqpoint{7.962808in}{5.231478in}}%
\pgfpathlineto{\pgfqpoint{7.956414in}{5.236681in}}%
\pgfpathlineto{\pgfqpoint{7.950019in}{5.241884in}}%
\pgfpathlineto{\pgfqpoint{7.943625in}{5.247087in}}%
\pgfpathlineto{\pgfqpoint{7.937231in}{5.252290in}}%
\pgfpathlineto{\pgfqpoint{7.930836in}{5.257493in}}%
\pgfpathlineto{\pgfqpoint{7.924442in}{5.262696in}}%
\pgfpathlineto{\pgfqpoint{7.918048in}{5.267899in}}%
\pgfpathlineto{\pgfqpoint{7.911654in}{5.273102in}}%
\pgfpathlineto{\pgfqpoint{7.905259in}{5.278305in}}%
\pgfpathlineto{\pgfqpoint{7.898865in}{5.283508in}}%
\pgfpathlineto{\pgfqpoint{7.892471in}{5.288711in}}%
\pgfpathlineto{\pgfqpoint{7.886076in}{5.293914in}}%
\pgfpathlineto{\pgfqpoint{7.879682in}{5.299117in}}%
\pgfpathlineto{\pgfqpoint{7.873288in}{5.304320in}}%
\pgfpathlineto{\pgfqpoint{7.866894in}{5.309523in}}%
\pgfpathlineto{\pgfqpoint{7.860499in}{5.314726in}}%
\pgfpathlineto{\pgfqpoint{7.854105in}{5.319929in}}%
\pgfpathlineto{\pgfqpoint{7.847711in}{5.325132in}}%
\pgfpathlineto{\pgfqpoint{7.841317in}{5.330334in}}%
\pgfpathlineto{\pgfqpoint{7.834922in}{5.335537in}}%
\pgfpathlineto{\pgfqpoint{7.828528in}{5.340740in}}%
\pgfpathlineto{\pgfqpoint{7.822134in}{5.345943in}}%
\pgfpathlineto{\pgfqpoint{7.815739in}{5.351146in}}%
\pgfpathlineto{\pgfqpoint{7.809345in}{5.356349in}}%
\pgfpathlineto{\pgfqpoint{7.802951in}{5.361552in}}%
\pgfpathlineto{\pgfqpoint{7.796557in}{5.366755in}}%
\pgfpathlineto{\pgfqpoint{7.790162in}{5.371958in}}%
\pgfpathlineto{\pgfqpoint{7.783768in}{5.377161in}}%
\pgfpathlineto{\pgfqpoint{7.777374in}{5.382364in}}%
\pgfpathlineto{\pgfqpoint{7.770979in}{5.387567in}}%
\pgfpathlineto{\pgfqpoint{7.764585in}{5.392770in}}%
\pgfpathlineto{\pgfqpoint{7.758191in}{5.397973in}}%
\pgfpathlineto{\pgfqpoint{7.751797in}{5.403176in}}%
\pgfpathlineto{\pgfqpoint{7.745402in}{5.408379in}}%
\pgfpathlineto{\pgfqpoint{7.739008in}{5.413582in}}%
\pgfpathlineto{\pgfqpoint{7.732614in}{5.418785in}}%
\pgfpathlineto{\pgfqpoint{7.726219in}{5.423988in}}%
\pgfpathlineto{\pgfqpoint{7.719825in}{5.429191in}}%
\pgfpathlineto{\pgfqpoint{7.713431in}{5.434394in}}%
\pgfpathlineto{\pgfqpoint{7.707037in}{5.439597in}}%
\pgfpathlineto{\pgfqpoint{7.700642in}{5.444800in}}%
\pgfpathlineto{\pgfqpoint{7.694248in}{5.450003in}}%
\pgfpathlineto{\pgfqpoint{7.687854in}{5.455206in}}%
\pgfpathlineto{\pgfqpoint{7.681460in}{5.460409in}}%
\pgfpathlineto{\pgfqpoint{7.675065in}{5.465612in}}%
\pgfpathlineto{\pgfqpoint{7.668671in}{5.470815in}}%
\pgfpathlineto{\pgfqpoint{7.662277in}{5.476018in}}%
\pgfpathlineto{\pgfqpoint{7.655882in}{5.481220in}}%
\pgfpathlineto{\pgfqpoint{7.649488in}{5.486423in}}%
\pgfpathlineto{\pgfqpoint{7.643094in}{5.491626in}}%
\pgfpathlineto{\pgfqpoint{7.636700in}{5.496829in}}%
\pgfpathlineto{\pgfqpoint{7.630305in}{5.502032in}}%
\pgfpathlineto{\pgfqpoint{7.623911in}{5.507235in}}%
\pgfpathlineto{\pgfqpoint{7.617517in}{5.512438in}}%
\pgfpathlineto{\pgfqpoint{7.611122in}{5.517641in}}%
\pgfpathlineto{\pgfqpoint{7.604728in}{5.522844in}}%
\pgfpathlineto{\pgfqpoint{7.598334in}{5.528047in}}%
\pgfpathlineto{\pgfqpoint{7.591940in}{5.533250in}}%
\pgfpathlineto{\pgfqpoint{7.585545in}{5.538453in}}%
\pgfpathlineto{\pgfqpoint{7.579151in}{5.543656in}}%
\pgfpathlineto{\pgfqpoint{7.572757in}{5.548859in}}%
\pgfpathlineto{\pgfqpoint{7.566362in}{5.554062in}}%
\pgfpathlineto{\pgfqpoint{7.559968in}{5.559265in}}%
\pgfpathlineto{\pgfqpoint{7.553574in}{5.564468in}}%
\pgfpathlineto{\pgfqpoint{7.547180in}{5.569671in}}%
\pgfpathlineto{\pgfqpoint{7.540785in}{5.574874in}}%
\pgfpathlineto{\pgfqpoint{7.534391in}{5.580077in}}%
\pgfpathlineto{\pgfqpoint{7.527997in}{5.585280in}}%
\pgfpathlineto{\pgfqpoint{7.521603in}{5.590483in}}%
\pgfpathlineto{\pgfqpoint{7.515208in}{5.595686in}}%
\pgfpathlineto{\pgfqpoint{7.508814in}{5.600889in}}%
\pgfpathlineto{\pgfqpoint{7.502420in}{5.606092in}}%
\pgfpathlineto{\pgfqpoint{7.496025in}{5.611295in}}%
\pgfpathlineto{\pgfqpoint{7.489631in}{5.616498in}}%
\pgfpathlineto{\pgfqpoint{7.483237in}{5.621701in}}%
\pgfpathlineto{\pgfqpoint{7.476843in}{5.626904in}}%
\pgfpathlineto{\pgfqpoint{7.470448in}{5.632107in}}%
\pgfpathlineto{\pgfqpoint{7.464054in}{5.637309in}}%
\pgfpathlineto{\pgfqpoint{7.457660in}{5.642512in}}%
\pgfpathlineto{\pgfqpoint{7.451265in}{5.647715in}}%
\pgfpathlineto{\pgfqpoint{7.444871in}{5.652918in}}%
\pgfpathlineto{\pgfqpoint{7.438477in}{5.658121in}}%
\pgfpathlineto{\pgfqpoint{7.432083in}{5.663324in}}%
\pgfpathlineto{\pgfqpoint{7.425688in}{5.668527in}}%
\pgfpathlineto{\pgfqpoint{7.419294in}{5.673730in}}%
\pgfpathlineto{\pgfqpoint{7.412900in}{5.678933in}}%
\pgfpathlineto{\pgfqpoint{7.406505in}{5.684136in}}%
\pgfpathlineto{\pgfqpoint{7.400111in}{5.689339in}}%
\pgfpathlineto{\pgfqpoint{7.393717in}{5.694542in}}%
\pgfpathlineto{\pgfqpoint{7.387323in}{5.699745in}}%
\pgfpathlineto{\pgfqpoint{7.380928in}{5.704948in}}%
\pgfpathlineto{\pgfqpoint{7.374534in}{5.710151in}}%
\pgfpathlineto{\pgfqpoint{7.368140in}{5.715354in}}%
\pgfpathlineto{\pgfqpoint{7.361746in}{5.720557in}}%
\pgfpathlineto{\pgfqpoint{7.355351in}{5.725760in}}%
\pgfpathlineto{\pgfqpoint{7.348957in}{5.730963in}}%
\pgfpathlineto{\pgfqpoint{7.342563in}{5.736166in}}%
\pgfpathlineto{\pgfqpoint{7.336168in}{5.741369in}}%
\pgfpathlineto{\pgfqpoint{7.329774in}{5.746572in}}%
\pgfpathlineto{\pgfqpoint{7.323380in}{5.751775in}}%
\pgfpathlineto{\pgfqpoint{7.323380in}{5.751775in}}%
\pgfpathclose%
\pgfusepath{fill}%
\end{pgfscope}%
\begin{pgfscope}%
\pgfpathrectangle{\pgfqpoint{7.323380in}{0.554012in}}{\pgfqpoint{6.387885in}{5.197763in}}%
\pgfusepath{clip}%
\pgfsetbuttcap%
\pgfsetroundjoin%
\definecolor{currentfill}{rgb}{1.000000,0.498039,0.054902}%
\pgfsetfillcolor{currentfill}%
\pgfsetfillopacity{0.200000}%
\pgfsetlinewidth{1.003750pt}%
\definecolor{currentstroke}{rgb}{1.000000,0.498039,0.054902}%
\pgfsetstrokecolor{currentstroke}%
\pgfsetstrokeopacity{0.200000}%
\pgfsetdash{}{0pt}%
\pgfsys@defobject{currentmarker}{\pgfqpoint{7.323380in}{0.554012in}}{\pgfqpoint{9.235269in}{2.113341in}}{%
\pgfpathmoveto{\pgfqpoint{7.323380in}{2.113341in}}%
\pgfpathlineto{\pgfqpoint{7.323380in}{0.554012in}}%
\pgfpathlineto{\pgfqpoint{7.329774in}{0.559215in}}%
\pgfpathlineto{\pgfqpoint{7.336168in}{0.564418in}}%
\pgfpathlineto{\pgfqpoint{7.342563in}{0.569621in}}%
\pgfpathlineto{\pgfqpoint{7.348957in}{0.574824in}}%
\pgfpathlineto{\pgfqpoint{7.355351in}{0.580027in}}%
\pgfpathlineto{\pgfqpoint{7.361746in}{0.585230in}}%
\pgfpathlineto{\pgfqpoint{7.368140in}{0.590433in}}%
\pgfpathlineto{\pgfqpoint{7.374534in}{0.595636in}}%
\pgfpathlineto{\pgfqpoint{7.380928in}{0.600838in}}%
\pgfpathlineto{\pgfqpoint{7.387323in}{0.606041in}}%
\pgfpathlineto{\pgfqpoint{7.393717in}{0.611244in}}%
\pgfpathlineto{\pgfqpoint{7.400111in}{0.616447in}}%
\pgfpathlineto{\pgfqpoint{7.406505in}{0.621650in}}%
\pgfpathlineto{\pgfqpoint{7.412900in}{0.626853in}}%
\pgfpathlineto{\pgfqpoint{7.419294in}{0.632056in}}%
\pgfpathlineto{\pgfqpoint{7.425688in}{0.637259in}}%
\pgfpathlineto{\pgfqpoint{7.432083in}{0.642462in}}%
\pgfpathlineto{\pgfqpoint{7.438477in}{0.647665in}}%
\pgfpathlineto{\pgfqpoint{7.444871in}{0.652868in}}%
\pgfpathlineto{\pgfqpoint{7.451265in}{0.658071in}}%
\pgfpathlineto{\pgfqpoint{7.457660in}{0.663274in}}%
\pgfpathlineto{\pgfqpoint{7.464054in}{0.668477in}}%
\pgfpathlineto{\pgfqpoint{7.470448in}{0.673680in}}%
\pgfpathlineto{\pgfqpoint{7.476843in}{0.678883in}}%
\pgfpathlineto{\pgfqpoint{7.483237in}{0.684086in}}%
\pgfpathlineto{\pgfqpoint{7.489631in}{0.689289in}}%
\pgfpathlineto{\pgfqpoint{7.496025in}{0.694492in}}%
\pgfpathlineto{\pgfqpoint{7.502420in}{0.699695in}}%
\pgfpathlineto{\pgfqpoint{7.508814in}{0.704898in}}%
\pgfpathlineto{\pgfqpoint{7.515208in}{0.710101in}}%
\pgfpathlineto{\pgfqpoint{7.521603in}{0.715304in}}%
\pgfpathlineto{\pgfqpoint{7.527997in}{0.720507in}}%
\pgfpathlineto{\pgfqpoint{7.534391in}{0.725710in}}%
\pgfpathlineto{\pgfqpoint{7.540785in}{0.730913in}}%
\pgfpathlineto{\pgfqpoint{7.547180in}{0.736116in}}%
\pgfpathlineto{\pgfqpoint{7.553574in}{0.741319in}}%
\pgfpathlineto{\pgfqpoint{7.559968in}{0.746522in}}%
\pgfpathlineto{\pgfqpoint{7.566362in}{0.751725in}}%
\pgfpathlineto{\pgfqpoint{7.572757in}{0.756927in}}%
\pgfpathlineto{\pgfqpoint{7.579151in}{0.762130in}}%
\pgfpathlineto{\pgfqpoint{7.585545in}{0.767333in}}%
\pgfpathlineto{\pgfqpoint{7.591940in}{0.772536in}}%
\pgfpathlineto{\pgfqpoint{7.598334in}{0.777739in}}%
\pgfpathlineto{\pgfqpoint{7.604728in}{0.782942in}}%
\pgfpathlineto{\pgfqpoint{7.611122in}{0.788145in}}%
\pgfpathlineto{\pgfqpoint{7.617517in}{0.793348in}}%
\pgfpathlineto{\pgfqpoint{7.623911in}{0.798551in}}%
\pgfpathlineto{\pgfqpoint{7.630305in}{0.803754in}}%
\pgfpathlineto{\pgfqpoint{7.636700in}{0.808957in}}%
\pgfpathlineto{\pgfqpoint{7.643094in}{0.814160in}}%
\pgfpathlineto{\pgfqpoint{7.649488in}{0.819363in}}%
\pgfpathlineto{\pgfqpoint{7.655882in}{0.824566in}}%
\pgfpathlineto{\pgfqpoint{7.662277in}{0.829769in}}%
\pgfpathlineto{\pgfqpoint{7.668671in}{0.834972in}}%
\pgfpathlineto{\pgfqpoint{7.675065in}{0.840175in}}%
\pgfpathlineto{\pgfqpoint{7.681460in}{0.845378in}}%
\pgfpathlineto{\pgfqpoint{7.687854in}{0.850581in}}%
\pgfpathlineto{\pgfqpoint{7.694248in}{0.855784in}}%
\pgfpathlineto{\pgfqpoint{7.700642in}{0.860987in}}%
\pgfpathlineto{\pgfqpoint{7.707037in}{0.866190in}}%
\pgfpathlineto{\pgfqpoint{7.713431in}{0.871393in}}%
\pgfpathlineto{\pgfqpoint{7.719825in}{0.876596in}}%
\pgfpathlineto{\pgfqpoint{7.726219in}{0.881799in}}%
\pgfpathlineto{\pgfqpoint{7.732614in}{0.887002in}}%
\pgfpathlineto{\pgfqpoint{7.739008in}{0.892205in}}%
\pgfpathlineto{\pgfqpoint{7.745402in}{0.897408in}}%
\pgfpathlineto{\pgfqpoint{7.751797in}{0.902611in}}%
\pgfpathlineto{\pgfqpoint{7.758191in}{0.907813in}}%
\pgfpathlineto{\pgfqpoint{7.764585in}{0.913016in}}%
\pgfpathlineto{\pgfqpoint{7.770979in}{0.918219in}}%
\pgfpathlineto{\pgfqpoint{7.777374in}{0.923422in}}%
\pgfpathlineto{\pgfqpoint{7.783768in}{0.928625in}}%
\pgfpathlineto{\pgfqpoint{7.790162in}{0.933828in}}%
\pgfpathlineto{\pgfqpoint{7.796557in}{0.939031in}}%
\pgfpathlineto{\pgfqpoint{7.802951in}{0.944234in}}%
\pgfpathlineto{\pgfqpoint{7.809345in}{0.949437in}}%
\pgfpathlineto{\pgfqpoint{7.815739in}{0.954640in}}%
\pgfpathlineto{\pgfqpoint{7.822134in}{0.959843in}}%
\pgfpathlineto{\pgfqpoint{7.828528in}{0.965046in}}%
\pgfpathlineto{\pgfqpoint{7.834922in}{0.970249in}}%
\pgfpathlineto{\pgfqpoint{7.841317in}{0.975452in}}%
\pgfpathlineto{\pgfqpoint{7.847711in}{0.980655in}}%
\pgfpathlineto{\pgfqpoint{7.854105in}{0.985858in}}%
\pgfpathlineto{\pgfqpoint{7.860499in}{0.991061in}}%
\pgfpathlineto{\pgfqpoint{7.866894in}{0.996264in}}%
\pgfpathlineto{\pgfqpoint{7.873288in}{1.001467in}}%
\pgfpathlineto{\pgfqpoint{7.879682in}{1.006670in}}%
\pgfpathlineto{\pgfqpoint{7.886076in}{1.011873in}}%
\pgfpathlineto{\pgfqpoint{7.892471in}{1.017076in}}%
\pgfpathlineto{\pgfqpoint{7.898865in}{1.022279in}}%
\pgfpathlineto{\pgfqpoint{7.905259in}{1.027482in}}%
\pgfpathlineto{\pgfqpoint{7.911654in}{1.032685in}}%
\pgfpathlineto{\pgfqpoint{7.918048in}{1.037888in}}%
\pgfpathlineto{\pgfqpoint{7.924442in}{1.043091in}}%
\pgfpathlineto{\pgfqpoint{7.930836in}{1.048294in}}%
\pgfpathlineto{\pgfqpoint{7.937231in}{1.053497in}}%
\pgfpathlineto{\pgfqpoint{7.943625in}{1.058699in}}%
\pgfpathlineto{\pgfqpoint{7.950019in}{1.063902in}}%
\pgfpathlineto{\pgfqpoint{7.956414in}{1.069105in}}%
\pgfpathlineto{\pgfqpoint{7.962808in}{1.074308in}}%
\pgfpathlineto{\pgfqpoint{7.969202in}{1.079511in}}%
\pgfpathlineto{\pgfqpoint{7.975596in}{1.084714in}}%
\pgfpathlineto{\pgfqpoint{7.981991in}{1.089917in}}%
\pgfpathlineto{\pgfqpoint{7.988385in}{1.095120in}}%
\pgfpathlineto{\pgfqpoint{7.994779in}{1.100323in}}%
\pgfpathlineto{\pgfqpoint{8.001173in}{1.105526in}}%
\pgfpathlineto{\pgfqpoint{8.007568in}{1.110729in}}%
\pgfpathlineto{\pgfqpoint{8.013962in}{1.115932in}}%
\pgfpathlineto{\pgfqpoint{8.020356in}{1.121135in}}%
\pgfpathlineto{\pgfqpoint{8.026751in}{1.126338in}}%
\pgfpathlineto{\pgfqpoint{8.033145in}{1.131541in}}%
\pgfpathlineto{\pgfqpoint{8.039539in}{1.136744in}}%
\pgfpathlineto{\pgfqpoint{8.045933in}{1.141947in}}%
\pgfpathlineto{\pgfqpoint{8.052328in}{1.147150in}}%
\pgfpathlineto{\pgfqpoint{8.058722in}{1.152353in}}%
\pgfpathlineto{\pgfqpoint{8.065116in}{1.157556in}}%
\pgfpathlineto{\pgfqpoint{8.071511in}{1.162759in}}%
\pgfpathlineto{\pgfqpoint{8.077905in}{1.167962in}}%
\pgfpathlineto{\pgfqpoint{8.084299in}{1.173165in}}%
\pgfpathlineto{\pgfqpoint{8.090693in}{1.178368in}}%
\pgfpathlineto{\pgfqpoint{8.097088in}{1.183571in}}%
\pgfpathlineto{\pgfqpoint{8.103482in}{1.188774in}}%
\pgfpathlineto{\pgfqpoint{8.109876in}{1.193977in}}%
\pgfpathlineto{\pgfqpoint{8.116271in}{1.199180in}}%
\pgfpathlineto{\pgfqpoint{8.122665in}{1.204383in}}%
\pgfpathlineto{\pgfqpoint{8.129059in}{1.209586in}}%
\pgfpathlineto{\pgfqpoint{8.135453in}{1.214788in}}%
\pgfpathlineto{\pgfqpoint{8.141848in}{1.219991in}}%
\pgfpathlineto{\pgfqpoint{8.148242in}{1.225194in}}%
\pgfpathlineto{\pgfqpoint{8.154636in}{1.230397in}}%
\pgfpathlineto{\pgfqpoint{8.161030in}{1.235600in}}%
\pgfpathlineto{\pgfqpoint{8.167425in}{1.240803in}}%
\pgfpathlineto{\pgfqpoint{8.173819in}{1.246006in}}%
\pgfpathlineto{\pgfqpoint{8.180213in}{1.251209in}}%
\pgfpathlineto{\pgfqpoint{8.186608in}{1.256412in}}%
\pgfpathlineto{\pgfqpoint{8.193002in}{1.261615in}}%
\pgfpathlineto{\pgfqpoint{8.199396in}{1.266818in}}%
\pgfpathlineto{\pgfqpoint{8.205790in}{1.272021in}}%
\pgfpathlineto{\pgfqpoint{8.212185in}{1.277224in}}%
\pgfpathlineto{\pgfqpoint{8.218579in}{1.282427in}}%
\pgfpathlineto{\pgfqpoint{8.224973in}{1.287630in}}%
\pgfpathlineto{\pgfqpoint{8.231368in}{1.292833in}}%
\pgfpathlineto{\pgfqpoint{8.237762in}{1.298036in}}%
\pgfpathlineto{\pgfqpoint{8.244156in}{1.303239in}}%
\pgfpathlineto{\pgfqpoint{8.250550in}{1.308442in}}%
\pgfpathlineto{\pgfqpoint{8.256945in}{1.313645in}}%
\pgfpathlineto{\pgfqpoint{8.263339in}{1.318848in}}%
\pgfpathlineto{\pgfqpoint{8.269733in}{1.324051in}}%
\pgfpathlineto{\pgfqpoint{8.276128in}{1.329254in}}%
\pgfpathlineto{\pgfqpoint{8.282522in}{1.334457in}}%
\pgfpathlineto{\pgfqpoint{8.288916in}{1.339660in}}%
\pgfpathlineto{\pgfqpoint{8.295310in}{1.344863in}}%
\pgfpathlineto{\pgfqpoint{8.301705in}{1.350066in}}%
\pgfpathlineto{\pgfqpoint{8.308099in}{1.355269in}}%
\pgfpathlineto{\pgfqpoint{8.314493in}{1.360472in}}%
\pgfpathlineto{\pgfqpoint{8.320887in}{1.365674in}}%
\pgfpathlineto{\pgfqpoint{8.327282in}{1.370877in}}%
\pgfpathlineto{\pgfqpoint{8.333676in}{1.376080in}}%
\pgfpathlineto{\pgfqpoint{8.340070in}{1.381283in}}%
\pgfpathlineto{\pgfqpoint{8.346465in}{1.386486in}}%
\pgfpathlineto{\pgfqpoint{8.352859in}{1.391689in}}%
\pgfpathlineto{\pgfqpoint{8.359253in}{1.396892in}}%
\pgfpathlineto{\pgfqpoint{8.365647in}{1.402095in}}%
\pgfpathlineto{\pgfqpoint{8.372042in}{1.407298in}}%
\pgfpathlineto{\pgfqpoint{8.378436in}{1.412501in}}%
\pgfpathlineto{\pgfqpoint{8.384830in}{1.417704in}}%
\pgfpathlineto{\pgfqpoint{8.391225in}{1.422907in}}%
\pgfpathlineto{\pgfqpoint{8.397619in}{1.428110in}}%
\pgfpathlineto{\pgfqpoint{8.404013in}{1.433313in}}%
\pgfpathlineto{\pgfqpoint{8.410407in}{1.438516in}}%
\pgfpathlineto{\pgfqpoint{8.416802in}{1.443719in}}%
\pgfpathlineto{\pgfqpoint{8.423196in}{1.448922in}}%
\pgfpathlineto{\pgfqpoint{8.429590in}{1.454125in}}%
\pgfpathlineto{\pgfqpoint{8.435985in}{1.459328in}}%
\pgfpathlineto{\pgfqpoint{8.442379in}{1.464531in}}%
\pgfpathlineto{\pgfqpoint{8.448773in}{1.469734in}}%
\pgfpathlineto{\pgfqpoint{8.455167in}{1.474937in}}%
\pgfpathlineto{\pgfqpoint{8.461562in}{1.480140in}}%
\pgfpathlineto{\pgfqpoint{8.467956in}{1.485343in}}%
\pgfpathlineto{\pgfqpoint{8.474350in}{1.490546in}}%
\pgfpathlineto{\pgfqpoint{8.480744in}{1.495749in}}%
\pgfpathlineto{\pgfqpoint{8.487139in}{1.500952in}}%
\pgfpathlineto{\pgfqpoint{8.493533in}{1.506155in}}%
\pgfpathlineto{\pgfqpoint{8.499927in}{1.511358in}}%
\pgfpathlineto{\pgfqpoint{8.506322in}{1.516560in}}%
\pgfpathlineto{\pgfqpoint{8.512716in}{1.521763in}}%
\pgfpathlineto{\pgfqpoint{8.519110in}{1.526966in}}%
\pgfpathlineto{\pgfqpoint{8.525504in}{1.532169in}}%
\pgfpathlineto{\pgfqpoint{8.531899in}{1.537372in}}%
\pgfpathlineto{\pgfqpoint{8.538293in}{1.542575in}}%
\pgfpathlineto{\pgfqpoint{8.544687in}{1.547778in}}%
\pgfpathlineto{\pgfqpoint{8.551082in}{1.552981in}}%
\pgfpathlineto{\pgfqpoint{8.557476in}{1.558184in}}%
\pgfpathlineto{\pgfqpoint{8.563870in}{1.563387in}}%
\pgfpathlineto{\pgfqpoint{8.570264in}{1.568590in}}%
\pgfpathlineto{\pgfqpoint{8.576659in}{1.573793in}}%
\pgfpathlineto{\pgfqpoint{8.583053in}{1.578996in}}%
\pgfpathlineto{\pgfqpoint{8.589447in}{1.584199in}}%
\pgfpathlineto{\pgfqpoint{8.595841in}{1.589402in}}%
\pgfpathlineto{\pgfqpoint{8.602236in}{1.594605in}}%
\pgfpathlineto{\pgfqpoint{8.608630in}{1.599808in}}%
\pgfpathlineto{\pgfqpoint{8.615024in}{1.605011in}}%
\pgfpathlineto{\pgfqpoint{8.621419in}{1.610214in}}%
\pgfpathlineto{\pgfqpoint{8.627813in}{1.615417in}}%
\pgfpathlineto{\pgfqpoint{8.634207in}{1.620620in}}%
\pgfpathlineto{\pgfqpoint{8.640601in}{1.625823in}}%
\pgfpathlineto{\pgfqpoint{8.646996in}{1.631026in}}%
\pgfpathlineto{\pgfqpoint{8.653390in}{1.636229in}}%
\pgfpathlineto{\pgfqpoint{8.659784in}{1.641432in}}%
\pgfpathlineto{\pgfqpoint{8.666179in}{1.646635in}}%
\pgfpathlineto{\pgfqpoint{8.672573in}{1.651838in}}%
\pgfpathlineto{\pgfqpoint{8.678967in}{1.657041in}}%
\pgfpathlineto{\pgfqpoint{8.685361in}{1.662244in}}%
\pgfpathlineto{\pgfqpoint{8.691756in}{1.667447in}}%
\pgfpathlineto{\pgfqpoint{8.698150in}{1.672649in}}%
\pgfpathlineto{\pgfqpoint{8.704544in}{1.677852in}}%
\pgfpathlineto{\pgfqpoint{8.710939in}{1.683055in}}%
\pgfpathlineto{\pgfqpoint{8.717333in}{1.688258in}}%
\pgfpathlineto{\pgfqpoint{8.723727in}{1.693461in}}%
\pgfpathlineto{\pgfqpoint{8.730121in}{1.698664in}}%
\pgfpathlineto{\pgfqpoint{8.736516in}{1.703867in}}%
\pgfpathlineto{\pgfqpoint{8.742910in}{1.709070in}}%
\pgfpathlineto{\pgfqpoint{8.749304in}{1.714273in}}%
\pgfpathlineto{\pgfqpoint{8.755698in}{1.719476in}}%
\pgfpathlineto{\pgfqpoint{8.762093in}{1.724679in}}%
\pgfpathlineto{\pgfqpoint{8.768487in}{1.729882in}}%
\pgfpathlineto{\pgfqpoint{8.774881in}{1.735085in}}%
\pgfpathlineto{\pgfqpoint{8.781276in}{1.740288in}}%
\pgfpathlineto{\pgfqpoint{8.787670in}{1.745491in}}%
\pgfpathlineto{\pgfqpoint{8.794064in}{1.750694in}}%
\pgfpathlineto{\pgfqpoint{8.800458in}{1.755897in}}%
\pgfpathlineto{\pgfqpoint{8.806853in}{1.761100in}}%
\pgfpathlineto{\pgfqpoint{8.813247in}{1.766303in}}%
\pgfpathlineto{\pgfqpoint{8.819641in}{1.771506in}}%
\pgfpathlineto{\pgfqpoint{8.826036in}{1.776709in}}%
\pgfpathlineto{\pgfqpoint{8.832430in}{1.781912in}}%
\pgfpathlineto{\pgfqpoint{8.838824in}{1.787115in}}%
\pgfpathlineto{\pgfqpoint{8.845218in}{1.792318in}}%
\pgfpathlineto{\pgfqpoint{8.851613in}{1.797521in}}%
\pgfpathlineto{\pgfqpoint{8.858007in}{1.802724in}}%
\pgfpathlineto{\pgfqpoint{8.864401in}{1.807927in}}%
\pgfpathlineto{\pgfqpoint{8.870796in}{1.813130in}}%
\pgfpathlineto{\pgfqpoint{8.877190in}{1.818333in}}%
\pgfpathlineto{\pgfqpoint{8.883584in}{1.823535in}}%
\pgfpathlineto{\pgfqpoint{8.889978in}{1.828738in}}%
\pgfpathlineto{\pgfqpoint{8.896373in}{1.833941in}}%
\pgfpathlineto{\pgfqpoint{8.902767in}{1.839144in}}%
\pgfpathlineto{\pgfqpoint{8.909161in}{1.844347in}}%
\pgfpathlineto{\pgfqpoint{8.915555in}{1.849550in}}%
\pgfpathlineto{\pgfqpoint{8.921950in}{1.854753in}}%
\pgfpathlineto{\pgfqpoint{8.928344in}{1.859956in}}%
\pgfpathlineto{\pgfqpoint{8.934738in}{1.865159in}}%
\pgfpathlineto{\pgfqpoint{8.941133in}{1.870362in}}%
\pgfpathlineto{\pgfqpoint{8.947527in}{1.875565in}}%
\pgfpathlineto{\pgfqpoint{8.953921in}{1.880768in}}%
\pgfpathlineto{\pgfqpoint{8.960315in}{1.885971in}}%
\pgfpathlineto{\pgfqpoint{8.966710in}{1.891174in}}%
\pgfpathlineto{\pgfqpoint{8.973104in}{1.896377in}}%
\pgfpathlineto{\pgfqpoint{8.979498in}{1.901580in}}%
\pgfpathlineto{\pgfqpoint{8.985893in}{1.906783in}}%
\pgfpathlineto{\pgfqpoint{8.992287in}{1.911986in}}%
\pgfpathlineto{\pgfqpoint{8.998681in}{1.917189in}}%
\pgfpathlineto{\pgfqpoint{9.005075in}{1.922392in}}%
\pgfpathlineto{\pgfqpoint{9.011470in}{1.927595in}}%
\pgfpathlineto{\pgfqpoint{9.017864in}{1.932798in}}%
\pgfpathlineto{\pgfqpoint{9.024258in}{1.938001in}}%
\pgfpathlineto{\pgfqpoint{9.030653in}{1.943204in}}%
\pgfpathlineto{\pgfqpoint{9.037047in}{1.948407in}}%
\pgfpathlineto{\pgfqpoint{9.043441in}{1.953610in}}%
\pgfpathlineto{\pgfqpoint{9.049835in}{1.958813in}}%
\pgfpathlineto{\pgfqpoint{9.056230in}{1.964016in}}%
\pgfpathlineto{\pgfqpoint{9.062624in}{1.969219in}}%
\pgfpathlineto{\pgfqpoint{9.069018in}{1.974421in}}%
\pgfpathlineto{\pgfqpoint{9.075412in}{1.979624in}}%
\pgfpathlineto{\pgfqpoint{9.081807in}{1.984827in}}%
\pgfpathlineto{\pgfqpoint{9.088201in}{1.990030in}}%
\pgfpathlineto{\pgfqpoint{9.094595in}{1.995233in}}%
\pgfpathlineto{\pgfqpoint{9.100990in}{2.000436in}}%
\pgfpathlineto{\pgfqpoint{9.107384in}{2.005639in}}%
\pgfpathlineto{\pgfqpoint{9.113778in}{2.010842in}}%
\pgfpathlineto{\pgfqpoint{9.120172in}{2.016045in}}%
\pgfpathlineto{\pgfqpoint{9.126567in}{2.021248in}}%
\pgfpathlineto{\pgfqpoint{9.132961in}{2.026451in}}%
\pgfpathlineto{\pgfqpoint{9.139355in}{2.031654in}}%
\pgfpathlineto{\pgfqpoint{9.145750in}{2.036857in}}%
\pgfpathlineto{\pgfqpoint{9.152144in}{2.042060in}}%
\pgfpathlineto{\pgfqpoint{9.158538in}{2.047263in}}%
\pgfpathlineto{\pgfqpoint{9.164932in}{2.052466in}}%
\pgfpathlineto{\pgfqpoint{9.171327in}{2.057669in}}%
\pgfpathlineto{\pgfqpoint{9.177721in}{2.062872in}}%
\pgfpathlineto{\pgfqpoint{9.184115in}{2.068075in}}%
\pgfpathlineto{\pgfqpoint{9.190509in}{2.073278in}}%
\pgfpathlineto{\pgfqpoint{9.196904in}{2.078481in}}%
\pgfpathlineto{\pgfqpoint{9.203298in}{2.083684in}}%
\pgfpathlineto{\pgfqpoint{9.209692in}{2.088887in}}%
\pgfpathlineto{\pgfqpoint{9.216087in}{2.094090in}}%
\pgfpathlineto{\pgfqpoint{9.222481in}{2.099293in}}%
\pgfpathlineto{\pgfqpoint{9.228875in}{2.104496in}}%
\pgfpathlineto{\pgfqpoint{9.235269in}{2.109699in}}%
\pgfpathlineto{\pgfqpoint{9.235269in}{2.113341in}}%
\pgfpathlineto{\pgfqpoint{9.235269in}{2.113341in}}%
\pgfpathlineto{\pgfqpoint{9.228875in}{2.113341in}}%
\pgfpathlineto{\pgfqpoint{9.222481in}{2.113341in}}%
\pgfpathlineto{\pgfqpoint{9.216087in}{2.113341in}}%
\pgfpathlineto{\pgfqpoint{9.209692in}{2.113341in}}%
\pgfpathlineto{\pgfqpoint{9.203298in}{2.113341in}}%
\pgfpathlineto{\pgfqpoint{9.196904in}{2.113341in}}%
\pgfpathlineto{\pgfqpoint{9.190509in}{2.113341in}}%
\pgfpathlineto{\pgfqpoint{9.184115in}{2.113341in}}%
\pgfpathlineto{\pgfqpoint{9.177721in}{2.113341in}}%
\pgfpathlineto{\pgfqpoint{9.171327in}{2.113341in}}%
\pgfpathlineto{\pgfqpoint{9.164932in}{2.113341in}}%
\pgfpathlineto{\pgfqpoint{9.158538in}{2.113341in}}%
\pgfpathlineto{\pgfqpoint{9.152144in}{2.113341in}}%
\pgfpathlineto{\pgfqpoint{9.145750in}{2.113341in}}%
\pgfpathlineto{\pgfqpoint{9.139355in}{2.113341in}}%
\pgfpathlineto{\pgfqpoint{9.132961in}{2.113341in}}%
\pgfpathlineto{\pgfqpoint{9.126567in}{2.113341in}}%
\pgfpathlineto{\pgfqpoint{9.120172in}{2.113341in}}%
\pgfpathlineto{\pgfqpoint{9.113778in}{2.113341in}}%
\pgfpathlineto{\pgfqpoint{9.107384in}{2.113341in}}%
\pgfpathlineto{\pgfqpoint{9.100990in}{2.113341in}}%
\pgfpathlineto{\pgfqpoint{9.094595in}{2.113341in}}%
\pgfpathlineto{\pgfqpoint{9.088201in}{2.113341in}}%
\pgfpathlineto{\pgfqpoint{9.081807in}{2.113341in}}%
\pgfpathlineto{\pgfqpoint{9.075412in}{2.113341in}}%
\pgfpathlineto{\pgfqpoint{9.069018in}{2.113341in}}%
\pgfpathlineto{\pgfqpoint{9.062624in}{2.113341in}}%
\pgfpathlineto{\pgfqpoint{9.056230in}{2.113341in}}%
\pgfpathlineto{\pgfqpoint{9.049835in}{2.113341in}}%
\pgfpathlineto{\pgfqpoint{9.043441in}{2.113341in}}%
\pgfpathlineto{\pgfqpoint{9.037047in}{2.113341in}}%
\pgfpathlineto{\pgfqpoint{9.030653in}{2.113341in}}%
\pgfpathlineto{\pgfqpoint{9.024258in}{2.113341in}}%
\pgfpathlineto{\pgfqpoint{9.017864in}{2.113341in}}%
\pgfpathlineto{\pgfqpoint{9.011470in}{2.113341in}}%
\pgfpathlineto{\pgfqpoint{9.005075in}{2.113341in}}%
\pgfpathlineto{\pgfqpoint{8.998681in}{2.113341in}}%
\pgfpathlineto{\pgfqpoint{8.992287in}{2.113341in}}%
\pgfpathlineto{\pgfqpoint{8.985893in}{2.113341in}}%
\pgfpathlineto{\pgfqpoint{8.979498in}{2.113341in}}%
\pgfpathlineto{\pgfqpoint{8.973104in}{2.113341in}}%
\pgfpathlineto{\pgfqpoint{8.966710in}{2.113341in}}%
\pgfpathlineto{\pgfqpoint{8.960315in}{2.113341in}}%
\pgfpathlineto{\pgfqpoint{8.953921in}{2.113341in}}%
\pgfpathlineto{\pgfqpoint{8.947527in}{2.113341in}}%
\pgfpathlineto{\pgfqpoint{8.941133in}{2.113341in}}%
\pgfpathlineto{\pgfqpoint{8.934738in}{2.113341in}}%
\pgfpathlineto{\pgfqpoint{8.928344in}{2.113341in}}%
\pgfpathlineto{\pgfqpoint{8.921950in}{2.113341in}}%
\pgfpathlineto{\pgfqpoint{8.915555in}{2.113341in}}%
\pgfpathlineto{\pgfqpoint{8.909161in}{2.113341in}}%
\pgfpathlineto{\pgfqpoint{8.902767in}{2.113341in}}%
\pgfpathlineto{\pgfqpoint{8.896373in}{2.113341in}}%
\pgfpathlineto{\pgfqpoint{8.889978in}{2.113341in}}%
\pgfpathlineto{\pgfqpoint{8.883584in}{2.113341in}}%
\pgfpathlineto{\pgfqpoint{8.877190in}{2.113341in}}%
\pgfpathlineto{\pgfqpoint{8.870796in}{2.113341in}}%
\pgfpathlineto{\pgfqpoint{8.864401in}{2.113341in}}%
\pgfpathlineto{\pgfqpoint{8.858007in}{2.113341in}}%
\pgfpathlineto{\pgfqpoint{8.851613in}{2.113341in}}%
\pgfpathlineto{\pgfqpoint{8.845218in}{2.113341in}}%
\pgfpathlineto{\pgfqpoint{8.838824in}{2.113341in}}%
\pgfpathlineto{\pgfqpoint{8.832430in}{2.113341in}}%
\pgfpathlineto{\pgfqpoint{8.826036in}{2.113341in}}%
\pgfpathlineto{\pgfqpoint{8.819641in}{2.113341in}}%
\pgfpathlineto{\pgfqpoint{8.813247in}{2.113341in}}%
\pgfpathlineto{\pgfqpoint{8.806853in}{2.113341in}}%
\pgfpathlineto{\pgfqpoint{8.800458in}{2.113341in}}%
\pgfpathlineto{\pgfqpoint{8.794064in}{2.113341in}}%
\pgfpathlineto{\pgfqpoint{8.787670in}{2.113341in}}%
\pgfpathlineto{\pgfqpoint{8.781276in}{2.113341in}}%
\pgfpathlineto{\pgfqpoint{8.774881in}{2.113341in}}%
\pgfpathlineto{\pgfqpoint{8.768487in}{2.113341in}}%
\pgfpathlineto{\pgfqpoint{8.762093in}{2.113341in}}%
\pgfpathlineto{\pgfqpoint{8.755698in}{2.113341in}}%
\pgfpathlineto{\pgfqpoint{8.749304in}{2.113341in}}%
\pgfpathlineto{\pgfqpoint{8.742910in}{2.113341in}}%
\pgfpathlineto{\pgfqpoint{8.736516in}{2.113341in}}%
\pgfpathlineto{\pgfqpoint{8.730121in}{2.113341in}}%
\pgfpathlineto{\pgfqpoint{8.723727in}{2.113341in}}%
\pgfpathlineto{\pgfqpoint{8.717333in}{2.113341in}}%
\pgfpathlineto{\pgfqpoint{8.710939in}{2.113341in}}%
\pgfpathlineto{\pgfqpoint{8.704544in}{2.113341in}}%
\pgfpathlineto{\pgfqpoint{8.698150in}{2.113341in}}%
\pgfpathlineto{\pgfqpoint{8.691756in}{2.113341in}}%
\pgfpathlineto{\pgfqpoint{8.685361in}{2.113341in}}%
\pgfpathlineto{\pgfqpoint{8.678967in}{2.113341in}}%
\pgfpathlineto{\pgfqpoint{8.672573in}{2.113341in}}%
\pgfpathlineto{\pgfqpoint{8.666179in}{2.113341in}}%
\pgfpathlineto{\pgfqpoint{8.659784in}{2.113341in}}%
\pgfpathlineto{\pgfqpoint{8.653390in}{2.113341in}}%
\pgfpathlineto{\pgfqpoint{8.646996in}{2.113341in}}%
\pgfpathlineto{\pgfqpoint{8.640601in}{2.113341in}}%
\pgfpathlineto{\pgfqpoint{8.634207in}{2.113341in}}%
\pgfpathlineto{\pgfqpoint{8.627813in}{2.113341in}}%
\pgfpathlineto{\pgfqpoint{8.621419in}{2.113341in}}%
\pgfpathlineto{\pgfqpoint{8.615024in}{2.113341in}}%
\pgfpathlineto{\pgfqpoint{8.608630in}{2.113341in}}%
\pgfpathlineto{\pgfqpoint{8.602236in}{2.113341in}}%
\pgfpathlineto{\pgfqpoint{8.595841in}{2.113341in}}%
\pgfpathlineto{\pgfqpoint{8.589447in}{2.113341in}}%
\pgfpathlineto{\pgfqpoint{8.583053in}{2.113341in}}%
\pgfpathlineto{\pgfqpoint{8.576659in}{2.113341in}}%
\pgfpathlineto{\pgfqpoint{8.570264in}{2.113341in}}%
\pgfpathlineto{\pgfqpoint{8.563870in}{2.113341in}}%
\pgfpathlineto{\pgfqpoint{8.557476in}{2.113341in}}%
\pgfpathlineto{\pgfqpoint{8.551082in}{2.113341in}}%
\pgfpathlineto{\pgfqpoint{8.544687in}{2.113341in}}%
\pgfpathlineto{\pgfqpoint{8.538293in}{2.113341in}}%
\pgfpathlineto{\pgfqpoint{8.531899in}{2.113341in}}%
\pgfpathlineto{\pgfqpoint{8.525504in}{2.113341in}}%
\pgfpathlineto{\pgfqpoint{8.519110in}{2.113341in}}%
\pgfpathlineto{\pgfqpoint{8.512716in}{2.113341in}}%
\pgfpathlineto{\pgfqpoint{8.506322in}{2.113341in}}%
\pgfpathlineto{\pgfqpoint{8.499927in}{2.113341in}}%
\pgfpathlineto{\pgfqpoint{8.493533in}{2.113341in}}%
\pgfpathlineto{\pgfqpoint{8.487139in}{2.113341in}}%
\pgfpathlineto{\pgfqpoint{8.480744in}{2.113341in}}%
\pgfpathlineto{\pgfqpoint{8.474350in}{2.113341in}}%
\pgfpathlineto{\pgfqpoint{8.467956in}{2.113341in}}%
\pgfpathlineto{\pgfqpoint{8.461562in}{2.113341in}}%
\pgfpathlineto{\pgfqpoint{8.455167in}{2.113341in}}%
\pgfpathlineto{\pgfqpoint{8.448773in}{2.113341in}}%
\pgfpathlineto{\pgfqpoint{8.442379in}{2.113341in}}%
\pgfpathlineto{\pgfqpoint{8.435985in}{2.113341in}}%
\pgfpathlineto{\pgfqpoint{8.429590in}{2.113341in}}%
\pgfpathlineto{\pgfqpoint{8.423196in}{2.113341in}}%
\pgfpathlineto{\pgfqpoint{8.416802in}{2.113341in}}%
\pgfpathlineto{\pgfqpoint{8.410407in}{2.113341in}}%
\pgfpathlineto{\pgfqpoint{8.404013in}{2.113341in}}%
\pgfpathlineto{\pgfqpoint{8.397619in}{2.113341in}}%
\pgfpathlineto{\pgfqpoint{8.391225in}{2.113341in}}%
\pgfpathlineto{\pgfqpoint{8.384830in}{2.113341in}}%
\pgfpathlineto{\pgfqpoint{8.378436in}{2.113341in}}%
\pgfpathlineto{\pgfqpoint{8.372042in}{2.113341in}}%
\pgfpathlineto{\pgfqpoint{8.365647in}{2.113341in}}%
\pgfpathlineto{\pgfqpoint{8.359253in}{2.113341in}}%
\pgfpathlineto{\pgfqpoint{8.352859in}{2.113341in}}%
\pgfpathlineto{\pgfqpoint{8.346465in}{2.113341in}}%
\pgfpathlineto{\pgfqpoint{8.340070in}{2.113341in}}%
\pgfpathlineto{\pgfqpoint{8.333676in}{2.113341in}}%
\pgfpathlineto{\pgfqpoint{8.327282in}{2.113341in}}%
\pgfpathlineto{\pgfqpoint{8.320887in}{2.113341in}}%
\pgfpathlineto{\pgfqpoint{8.314493in}{2.113341in}}%
\pgfpathlineto{\pgfqpoint{8.308099in}{2.113341in}}%
\pgfpathlineto{\pgfqpoint{8.301705in}{2.113341in}}%
\pgfpathlineto{\pgfqpoint{8.295310in}{2.113341in}}%
\pgfpathlineto{\pgfqpoint{8.288916in}{2.113341in}}%
\pgfpathlineto{\pgfqpoint{8.282522in}{2.113341in}}%
\pgfpathlineto{\pgfqpoint{8.276128in}{2.113341in}}%
\pgfpathlineto{\pgfqpoint{8.269733in}{2.113341in}}%
\pgfpathlineto{\pgfqpoint{8.263339in}{2.113341in}}%
\pgfpathlineto{\pgfqpoint{8.256945in}{2.113341in}}%
\pgfpathlineto{\pgfqpoint{8.250550in}{2.113341in}}%
\pgfpathlineto{\pgfqpoint{8.244156in}{2.113341in}}%
\pgfpathlineto{\pgfqpoint{8.237762in}{2.113341in}}%
\pgfpathlineto{\pgfqpoint{8.231368in}{2.113341in}}%
\pgfpathlineto{\pgfqpoint{8.224973in}{2.113341in}}%
\pgfpathlineto{\pgfqpoint{8.218579in}{2.113341in}}%
\pgfpathlineto{\pgfqpoint{8.212185in}{2.113341in}}%
\pgfpathlineto{\pgfqpoint{8.205790in}{2.113341in}}%
\pgfpathlineto{\pgfqpoint{8.199396in}{2.113341in}}%
\pgfpathlineto{\pgfqpoint{8.193002in}{2.113341in}}%
\pgfpathlineto{\pgfqpoint{8.186608in}{2.113341in}}%
\pgfpathlineto{\pgfqpoint{8.180213in}{2.113341in}}%
\pgfpathlineto{\pgfqpoint{8.173819in}{2.113341in}}%
\pgfpathlineto{\pgfqpoint{8.167425in}{2.113341in}}%
\pgfpathlineto{\pgfqpoint{8.161030in}{2.113341in}}%
\pgfpathlineto{\pgfqpoint{8.154636in}{2.113341in}}%
\pgfpathlineto{\pgfqpoint{8.148242in}{2.113341in}}%
\pgfpathlineto{\pgfqpoint{8.141848in}{2.113341in}}%
\pgfpathlineto{\pgfqpoint{8.135453in}{2.113341in}}%
\pgfpathlineto{\pgfqpoint{8.129059in}{2.113341in}}%
\pgfpathlineto{\pgfqpoint{8.122665in}{2.113341in}}%
\pgfpathlineto{\pgfqpoint{8.116271in}{2.113341in}}%
\pgfpathlineto{\pgfqpoint{8.109876in}{2.113341in}}%
\pgfpathlineto{\pgfqpoint{8.103482in}{2.113341in}}%
\pgfpathlineto{\pgfqpoint{8.097088in}{2.113341in}}%
\pgfpathlineto{\pgfqpoint{8.090693in}{2.113341in}}%
\pgfpathlineto{\pgfqpoint{8.084299in}{2.113341in}}%
\pgfpathlineto{\pgfqpoint{8.077905in}{2.113341in}}%
\pgfpathlineto{\pgfqpoint{8.071511in}{2.113341in}}%
\pgfpathlineto{\pgfqpoint{8.065116in}{2.113341in}}%
\pgfpathlineto{\pgfqpoint{8.058722in}{2.113341in}}%
\pgfpathlineto{\pgfqpoint{8.052328in}{2.113341in}}%
\pgfpathlineto{\pgfqpoint{8.045933in}{2.113341in}}%
\pgfpathlineto{\pgfqpoint{8.039539in}{2.113341in}}%
\pgfpathlineto{\pgfqpoint{8.033145in}{2.113341in}}%
\pgfpathlineto{\pgfqpoint{8.026751in}{2.113341in}}%
\pgfpathlineto{\pgfqpoint{8.020356in}{2.113341in}}%
\pgfpathlineto{\pgfqpoint{8.013962in}{2.113341in}}%
\pgfpathlineto{\pgfqpoint{8.007568in}{2.113341in}}%
\pgfpathlineto{\pgfqpoint{8.001173in}{2.113341in}}%
\pgfpathlineto{\pgfqpoint{7.994779in}{2.113341in}}%
\pgfpathlineto{\pgfqpoint{7.988385in}{2.113341in}}%
\pgfpathlineto{\pgfqpoint{7.981991in}{2.113341in}}%
\pgfpathlineto{\pgfqpoint{7.975596in}{2.113341in}}%
\pgfpathlineto{\pgfqpoint{7.969202in}{2.113341in}}%
\pgfpathlineto{\pgfqpoint{7.962808in}{2.113341in}}%
\pgfpathlineto{\pgfqpoint{7.956414in}{2.113341in}}%
\pgfpathlineto{\pgfqpoint{7.950019in}{2.113341in}}%
\pgfpathlineto{\pgfqpoint{7.943625in}{2.113341in}}%
\pgfpathlineto{\pgfqpoint{7.937231in}{2.113341in}}%
\pgfpathlineto{\pgfqpoint{7.930836in}{2.113341in}}%
\pgfpathlineto{\pgfqpoint{7.924442in}{2.113341in}}%
\pgfpathlineto{\pgfqpoint{7.918048in}{2.113341in}}%
\pgfpathlineto{\pgfqpoint{7.911654in}{2.113341in}}%
\pgfpathlineto{\pgfqpoint{7.905259in}{2.113341in}}%
\pgfpathlineto{\pgfqpoint{7.898865in}{2.113341in}}%
\pgfpathlineto{\pgfqpoint{7.892471in}{2.113341in}}%
\pgfpathlineto{\pgfqpoint{7.886076in}{2.113341in}}%
\pgfpathlineto{\pgfqpoint{7.879682in}{2.113341in}}%
\pgfpathlineto{\pgfqpoint{7.873288in}{2.113341in}}%
\pgfpathlineto{\pgfqpoint{7.866894in}{2.113341in}}%
\pgfpathlineto{\pgfqpoint{7.860499in}{2.113341in}}%
\pgfpathlineto{\pgfqpoint{7.854105in}{2.113341in}}%
\pgfpathlineto{\pgfqpoint{7.847711in}{2.113341in}}%
\pgfpathlineto{\pgfqpoint{7.841317in}{2.113341in}}%
\pgfpathlineto{\pgfqpoint{7.834922in}{2.113341in}}%
\pgfpathlineto{\pgfqpoint{7.828528in}{2.113341in}}%
\pgfpathlineto{\pgfqpoint{7.822134in}{2.113341in}}%
\pgfpathlineto{\pgfqpoint{7.815739in}{2.113341in}}%
\pgfpathlineto{\pgfqpoint{7.809345in}{2.113341in}}%
\pgfpathlineto{\pgfqpoint{7.802951in}{2.113341in}}%
\pgfpathlineto{\pgfqpoint{7.796557in}{2.113341in}}%
\pgfpathlineto{\pgfqpoint{7.790162in}{2.113341in}}%
\pgfpathlineto{\pgfqpoint{7.783768in}{2.113341in}}%
\pgfpathlineto{\pgfqpoint{7.777374in}{2.113341in}}%
\pgfpathlineto{\pgfqpoint{7.770979in}{2.113341in}}%
\pgfpathlineto{\pgfqpoint{7.764585in}{2.113341in}}%
\pgfpathlineto{\pgfqpoint{7.758191in}{2.113341in}}%
\pgfpathlineto{\pgfqpoint{7.751797in}{2.113341in}}%
\pgfpathlineto{\pgfqpoint{7.745402in}{2.113341in}}%
\pgfpathlineto{\pgfqpoint{7.739008in}{2.113341in}}%
\pgfpathlineto{\pgfqpoint{7.732614in}{2.113341in}}%
\pgfpathlineto{\pgfqpoint{7.726219in}{2.113341in}}%
\pgfpathlineto{\pgfqpoint{7.719825in}{2.113341in}}%
\pgfpathlineto{\pgfqpoint{7.713431in}{2.113341in}}%
\pgfpathlineto{\pgfqpoint{7.707037in}{2.113341in}}%
\pgfpathlineto{\pgfqpoint{7.700642in}{2.113341in}}%
\pgfpathlineto{\pgfqpoint{7.694248in}{2.113341in}}%
\pgfpathlineto{\pgfqpoint{7.687854in}{2.113341in}}%
\pgfpathlineto{\pgfqpoint{7.681460in}{2.113341in}}%
\pgfpathlineto{\pgfqpoint{7.675065in}{2.113341in}}%
\pgfpathlineto{\pgfqpoint{7.668671in}{2.113341in}}%
\pgfpathlineto{\pgfqpoint{7.662277in}{2.113341in}}%
\pgfpathlineto{\pgfqpoint{7.655882in}{2.113341in}}%
\pgfpathlineto{\pgfqpoint{7.649488in}{2.113341in}}%
\pgfpathlineto{\pgfqpoint{7.643094in}{2.113341in}}%
\pgfpathlineto{\pgfqpoint{7.636700in}{2.113341in}}%
\pgfpathlineto{\pgfqpoint{7.630305in}{2.113341in}}%
\pgfpathlineto{\pgfqpoint{7.623911in}{2.113341in}}%
\pgfpathlineto{\pgfqpoint{7.617517in}{2.113341in}}%
\pgfpathlineto{\pgfqpoint{7.611122in}{2.113341in}}%
\pgfpathlineto{\pgfqpoint{7.604728in}{2.113341in}}%
\pgfpathlineto{\pgfqpoint{7.598334in}{2.113341in}}%
\pgfpathlineto{\pgfqpoint{7.591940in}{2.113341in}}%
\pgfpathlineto{\pgfqpoint{7.585545in}{2.113341in}}%
\pgfpathlineto{\pgfqpoint{7.579151in}{2.113341in}}%
\pgfpathlineto{\pgfqpoint{7.572757in}{2.113341in}}%
\pgfpathlineto{\pgfqpoint{7.566362in}{2.113341in}}%
\pgfpathlineto{\pgfqpoint{7.559968in}{2.113341in}}%
\pgfpathlineto{\pgfqpoint{7.553574in}{2.113341in}}%
\pgfpathlineto{\pgfqpoint{7.547180in}{2.113341in}}%
\pgfpathlineto{\pgfqpoint{7.540785in}{2.113341in}}%
\pgfpathlineto{\pgfqpoint{7.534391in}{2.113341in}}%
\pgfpathlineto{\pgfqpoint{7.527997in}{2.113341in}}%
\pgfpathlineto{\pgfqpoint{7.521603in}{2.113341in}}%
\pgfpathlineto{\pgfqpoint{7.515208in}{2.113341in}}%
\pgfpathlineto{\pgfqpoint{7.508814in}{2.113341in}}%
\pgfpathlineto{\pgfqpoint{7.502420in}{2.113341in}}%
\pgfpathlineto{\pgfqpoint{7.496025in}{2.113341in}}%
\pgfpathlineto{\pgfqpoint{7.489631in}{2.113341in}}%
\pgfpathlineto{\pgfqpoint{7.483237in}{2.113341in}}%
\pgfpathlineto{\pgfqpoint{7.476843in}{2.113341in}}%
\pgfpathlineto{\pgfqpoint{7.470448in}{2.113341in}}%
\pgfpathlineto{\pgfqpoint{7.464054in}{2.113341in}}%
\pgfpathlineto{\pgfqpoint{7.457660in}{2.113341in}}%
\pgfpathlineto{\pgfqpoint{7.451265in}{2.113341in}}%
\pgfpathlineto{\pgfqpoint{7.444871in}{2.113341in}}%
\pgfpathlineto{\pgfqpoint{7.438477in}{2.113341in}}%
\pgfpathlineto{\pgfqpoint{7.432083in}{2.113341in}}%
\pgfpathlineto{\pgfqpoint{7.425688in}{2.113341in}}%
\pgfpathlineto{\pgfqpoint{7.419294in}{2.113341in}}%
\pgfpathlineto{\pgfqpoint{7.412900in}{2.113341in}}%
\pgfpathlineto{\pgfqpoint{7.406505in}{2.113341in}}%
\pgfpathlineto{\pgfqpoint{7.400111in}{2.113341in}}%
\pgfpathlineto{\pgfqpoint{7.393717in}{2.113341in}}%
\pgfpathlineto{\pgfqpoint{7.387323in}{2.113341in}}%
\pgfpathlineto{\pgfqpoint{7.380928in}{2.113341in}}%
\pgfpathlineto{\pgfqpoint{7.374534in}{2.113341in}}%
\pgfpathlineto{\pgfqpoint{7.368140in}{2.113341in}}%
\pgfpathlineto{\pgfqpoint{7.361746in}{2.113341in}}%
\pgfpathlineto{\pgfqpoint{7.355351in}{2.113341in}}%
\pgfpathlineto{\pgfqpoint{7.348957in}{2.113341in}}%
\pgfpathlineto{\pgfqpoint{7.342563in}{2.113341in}}%
\pgfpathlineto{\pgfqpoint{7.336168in}{2.113341in}}%
\pgfpathlineto{\pgfqpoint{7.329774in}{2.113341in}}%
\pgfpathlineto{\pgfqpoint{7.323380in}{2.113341in}}%
\pgfpathlineto{\pgfqpoint{7.323380in}{2.113341in}}%
\pgfpathclose%
\pgfusepath{stroke,fill}%
}%
\begin{pgfscope}%
\pgfsys@transformshift{0.000000in}{0.000000in}%
\pgfsys@useobject{currentmarker}{}%
\end{pgfscope}%
\end{pgfscope}%
\begin{pgfscope}%
\pgfpathrectangle{\pgfqpoint{7.323380in}{0.554012in}}{\pgfqpoint{6.387885in}{5.197763in}}%
\pgfusepath{clip}%
\pgfsetbuttcap%
\pgfsetroundjoin%
\definecolor{currentfill}{rgb}{0.501961,0.501961,0.501961}%
\pgfsetfillcolor{currentfill}%
\pgfsetfillopacity{0.200000}%
\pgfsetlinewidth{1.003750pt}%
\definecolor{currentstroke}{rgb}{0.501961,0.501961,0.501961}%
\pgfsetstrokecolor{currentstroke}%
\pgfsetstrokeopacity{0.200000}%
\pgfsetdash{}{0pt}%
\pgfsys@defobject{currentmarker}{\pgfqpoint{9.241664in}{2.114902in}}{\pgfqpoint{10.514125in}{4.190885in}}{%
\pgfpathmoveto{\pgfqpoint{9.241664in}{4.190885in}}%
\pgfpathlineto{\pgfqpoint{9.241664in}{2.114902in}}%
\pgfpathlineto{\pgfqpoint{9.248058in}{2.120105in}}%
\pgfpathlineto{\pgfqpoint{9.254452in}{2.125308in}}%
\pgfpathlineto{\pgfqpoint{9.260847in}{2.130510in}}%
\pgfpathlineto{\pgfqpoint{9.267241in}{2.135713in}}%
\pgfpathlineto{\pgfqpoint{9.273635in}{2.140916in}}%
\pgfpathlineto{\pgfqpoint{9.280029in}{2.146119in}}%
\pgfpathlineto{\pgfqpoint{9.286424in}{2.151322in}}%
\pgfpathlineto{\pgfqpoint{9.292818in}{2.156525in}}%
\pgfpathlineto{\pgfqpoint{9.299212in}{2.161728in}}%
\pgfpathlineto{\pgfqpoint{9.305607in}{2.166931in}}%
\pgfpathlineto{\pgfqpoint{9.312001in}{2.172134in}}%
\pgfpathlineto{\pgfqpoint{9.318395in}{2.177337in}}%
\pgfpathlineto{\pgfqpoint{9.324789in}{2.182540in}}%
\pgfpathlineto{\pgfqpoint{9.331184in}{2.187743in}}%
\pgfpathlineto{\pgfqpoint{9.337578in}{2.192946in}}%
\pgfpathlineto{\pgfqpoint{9.343972in}{2.198149in}}%
\pgfpathlineto{\pgfqpoint{9.350366in}{2.203352in}}%
\pgfpathlineto{\pgfqpoint{9.356761in}{2.208555in}}%
\pgfpathlineto{\pgfqpoint{9.363155in}{2.213758in}}%
\pgfpathlineto{\pgfqpoint{9.369549in}{2.218961in}}%
\pgfpathlineto{\pgfqpoint{9.375944in}{2.224164in}}%
\pgfpathlineto{\pgfqpoint{9.382338in}{2.229367in}}%
\pgfpathlineto{\pgfqpoint{9.388732in}{2.234570in}}%
\pgfpathlineto{\pgfqpoint{9.395126in}{2.239773in}}%
\pgfpathlineto{\pgfqpoint{9.401521in}{2.244976in}}%
\pgfpathlineto{\pgfqpoint{9.407915in}{2.250179in}}%
\pgfpathlineto{\pgfqpoint{9.414309in}{2.255382in}}%
\pgfpathlineto{\pgfqpoint{9.420704in}{2.260585in}}%
\pgfpathlineto{\pgfqpoint{9.427098in}{2.265788in}}%
\pgfpathlineto{\pgfqpoint{9.433492in}{2.270991in}}%
\pgfpathlineto{\pgfqpoint{9.439886in}{2.276194in}}%
\pgfpathlineto{\pgfqpoint{9.446281in}{2.281396in}}%
\pgfpathlineto{\pgfqpoint{9.452675in}{2.286599in}}%
\pgfpathlineto{\pgfqpoint{9.459069in}{2.291802in}}%
\pgfpathlineto{\pgfqpoint{9.465464in}{2.297005in}}%
\pgfpathlineto{\pgfqpoint{9.471858in}{2.302208in}}%
\pgfpathlineto{\pgfqpoint{9.478252in}{2.307411in}}%
\pgfpathlineto{\pgfqpoint{9.484646in}{2.312614in}}%
\pgfpathlineto{\pgfqpoint{9.491041in}{2.317817in}}%
\pgfpathlineto{\pgfqpoint{9.497435in}{2.323020in}}%
\pgfpathlineto{\pgfqpoint{9.503829in}{2.328223in}}%
\pgfpathlineto{\pgfqpoint{9.510223in}{2.333426in}}%
\pgfpathlineto{\pgfqpoint{9.516618in}{2.338629in}}%
\pgfpathlineto{\pgfqpoint{9.523012in}{2.343832in}}%
\pgfpathlineto{\pgfqpoint{9.529406in}{2.349035in}}%
\pgfpathlineto{\pgfqpoint{9.535801in}{2.354238in}}%
\pgfpathlineto{\pgfqpoint{9.542195in}{2.359441in}}%
\pgfpathlineto{\pgfqpoint{9.548589in}{2.364644in}}%
\pgfpathlineto{\pgfqpoint{9.554983in}{2.369847in}}%
\pgfpathlineto{\pgfqpoint{9.561378in}{2.375050in}}%
\pgfpathlineto{\pgfqpoint{9.567772in}{2.380253in}}%
\pgfpathlineto{\pgfqpoint{9.574166in}{2.385456in}}%
\pgfpathlineto{\pgfqpoint{9.580561in}{2.390659in}}%
\pgfpathlineto{\pgfqpoint{9.586955in}{2.395862in}}%
\pgfpathlineto{\pgfqpoint{9.593349in}{2.401065in}}%
\pgfpathlineto{\pgfqpoint{9.599743in}{2.406268in}}%
\pgfpathlineto{\pgfqpoint{9.606138in}{2.411471in}}%
\pgfpathlineto{\pgfqpoint{9.612532in}{2.416674in}}%
\pgfpathlineto{\pgfqpoint{9.618926in}{2.421877in}}%
\pgfpathlineto{\pgfqpoint{9.625321in}{2.427080in}}%
\pgfpathlineto{\pgfqpoint{9.631715in}{2.432282in}}%
\pgfpathlineto{\pgfqpoint{9.638109in}{2.437485in}}%
\pgfpathlineto{\pgfqpoint{9.644503in}{2.442688in}}%
\pgfpathlineto{\pgfqpoint{9.650898in}{2.447891in}}%
\pgfpathlineto{\pgfqpoint{9.657292in}{2.453094in}}%
\pgfpathlineto{\pgfqpoint{9.663686in}{2.458297in}}%
\pgfpathlineto{\pgfqpoint{9.670080in}{2.463500in}}%
\pgfpathlineto{\pgfqpoint{9.676475in}{2.468703in}}%
\pgfpathlineto{\pgfqpoint{9.682869in}{2.473906in}}%
\pgfpathlineto{\pgfqpoint{9.689263in}{2.479109in}}%
\pgfpathlineto{\pgfqpoint{9.695658in}{2.484312in}}%
\pgfpathlineto{\pgfqpoint{9.702052in}{2.489515in}}%
\pgfpathlineto{\pgfqpoint{9.708446in}{2.494718in}}%
\pgfpathlineto{\pgfqpoint{9.714840in}{2.499921in}}%
\pgfpathlineto{\pgfqpoint{9.721235in}{2.505124in}}%
\pgfpathlineto{\pgfqpoint{9.727629in}{2.510327in}}%
\pgfpathlineto{\pgfqpoint{9.734023in}{2.515530in}}%
\pgfpathlineto{\pgfqpoint{9.740418in}{2.520733in}}%
\pgfpathlineto{\pgfqpoint{9.746812in}{2.525936in}}%
\pgfpathlineto{\pgfqpoint{9.753206in}{2.531139in}}%
\pgfpathlineto{\pgfqpoint{9.759600in}{2.536342in}}%
\pgfpathlineto{\pgfqpoint{9.765995in}{2.541545in}}%
\pgfpathlineto{\pgfqpoint{9.772389in}{2.546748in}}%
\pgfpathlineto{\pgfqpoint{9.778783in}{2.551951in}}%
\pgfpathlineto{\pgfqpoint{9.785177in}{2.557154in}}%
\pgfpathlineto{\pgfqpoint{9.791572in}{2.562357in}}%
\pgfpathlineto{\pgfqpoint{9.797966in}{2.567560in}}%
\pgfpathlineto{\pgfqpoint{9.804360in}{2.572763in}}%
\pgfpathlineto{\pgfqpoint{9.810755in}{2.577966in}}%
\pgfpathlineto{\pgfqpoint{9.817149in}{2.583168in}}%
\pgfpathlineto{\pgfqpoint{9.823543in}{2.588371in}}%
\pgfpathlineto{\pgfqpoint{9.829937in}{2.593574in}}%
\pgfpathlineto{\pgfqpoint{9.836332in}{2.598777in}}%
\pgfpathlineto{\pgfqpoint{9.842726in}{2.603980in}}%
\pgfpathlineto{\pgfqpoint{9.849120in}{2.609183in}}%
\pgfpathlineto{\pgfqpoint{9.855515in}{2.614386in}}%
\pgfpathlineto{\pgfqpoint{9.861909in}{2.619589in}}%
\pgfpathlineto{\pgfqpoint{9.868303in}{2.624792in}}%
\pgfpathlineto{\pgfqpoint{9.874697in}{2.629995in}}%
\pgfpathlineto{\pgfqpoint{9.881092in}{2.635198in}}%
\pgfpathlineto{\pgfqpoint{9.887486in}{2.640401in}}%
\pgfpathlineto{\pgfqpoint{9.893880in}{2.645604in}}%
\pgfpathlineto{\pgfqpoint{9.900275in}{2.650807in}}%
\pgfpathlineto{\pgfqpoint{9.906669in}{2.656010in}}%
\pgfpathlineto{\pgfqpoint{9.913063in}{2.661213in}}%
\pgfpathlineto{\pgfqpoint{9.919457in}{2.666416in}}%
\pgfpathlineto{\pgfqpoint{9.925852in}{2.671619in}}%
\pgfpathlineto{\pgfqpoint{9.932246in}{2.676822in}}%
\pgfpathlineto{\pgfqpoint{9.938640in}{2.682025in}}%
\pgfpathlineto{\pgfqpoint{9.945034in}{2.687228in}}%
\pgfpathlineto{\pgfqpoint{9.951429in}{2.692431in}}%
\pgfpathlineto{\pgfqpoint{9.957823in}{2.697634in}}%
\pgfpathlineto{\pgfqpoint{9.964217in}{2.702837in}}%
\pgfpathlineto{\pgfqpoint{9.970612in}{2.708040in}}%
\pgfpathlineto{\pgfqpoint{9.977006in}{2.713243in}}%
\pgfpathlineto{\pgfqpoint{9.983400in}{2.718446in}}%
\pgfpathlineto{\pgfqpoint{9.989794in}{2.723649in}}%
\pgfpathlineto{\pgfqpoint{9.996189in}{2.728852in}}%
\pgfpathlineto{\pgfqpoint{10.002583in}{2.734055in}}%
\pgfpathlineto{\pgfqpoint{10.008977in}{2.739257in}}%
\pgfpathlineto{\pgfqpoint{10.015372in}{2.744460in}}%
\pgfpathlineto{\pgfqpoint{10.021766in}{2.749663in}}%
\pgfpathlineto{\pgfqpoint{10.028160in}{2.754866in}}%
\pgfpathlineto{\pgfqpoint{10.034554in}{2.760069in}}%
\pgfpathlineto{\pgfqpoint{10.040949in}{2.765272in}}%
\pgfpathlineto{\pgfqpoint{10.047343in}{2.770475in}}%
\pgfpathlineto{\pgfqpoint{10.053737in}{2.775678in}}%
\pgfpathlineto{\pgfqpoint{10.060132in}{2.780881in}}%
\pgfpathlineto{\pgfqpoint{10.066526in}{2.786084in}}%
\pgfpathlineto{\pgfqpoint{10.072920in}{2.791287in}}%
\pgfpathlineto{\pgfqpoint{10.079314in}{2.796490in}}%
\pgfpathlineto{\pgfqpoint{10.085709in}{2.801693in}}%
\pgfpathlineto{\pgfqpoint{10.092103in}{2.806896in}}%
\pgfpathlineto{\pgfqpoint{10.098497in}{2.812099in}}%
\pgfpathlineto{\pgfqpoint{10.104891in}{2.817302in}}%
\pgfpathlineto{\pgfqpoint{10.111286in}{2.822505in}}%
\pgfpathlineto{\pgfqpoint{10.117680in}{2.827708in}}%
\pgfpathlineto{\pgfqpoint{10.124074in}{2.832911in}}%
\pgfpathlineto{\pgfqpoint{10.130469in}{2.838114in}}%
\pgfpathlineto{\pgfqpoint{10.136863in}{2.843317in}}%
\pgfpathlineto{\pgfqpoint{10.143257in}{2.848520in}}%
\pgfpathlineto{\pgfqpoint{10.149651in}{2.853723in}}%
\pgfpathlineto{\pgfqpoint{10.156046in}{2.858926in}}%
\pgfpathlineto{\pgfqpoint{10.162440in}{2.864129in}}%
\pgfpathlineto{\pgfqpoint{10.168834in}{2.869332in}}%
\pgfpathlineto{\pgfqpoint{10.175229in}{2.874535in}}%
\pgfpathlineto{\pgfqpoint{10.181623in}{2.879738in}}%
\pgfpathlineto{\pgfqpoint{10.188017in}{2.884941in}}%
\pgfpathlineto{\pgfqpoint{10.194411in}{2.890143in}}%
\pgfpathlineto{\pgfqpoint{10.200806in}{2.895346in}}%
\pgfpathlineto{\pgfqpoint{10.207200in}{2.900549in}}%
\pgfpathlineto{\pgfqpoint{10.213594in}{2.905752in}}%
\pgfpathlineto{\pgfqpoint{10.219989in}{2.910955in}}%
\pgfpathlineto{\pgfqpoint{10.226383in}{2.916158in}}%
\pgfpathlineto{\pgfqpoint{10.232777in}{2.921361in}}%
\pgfpathlineto{\pgfqpoint{10.239171in}{2.926564in}}%
\pgfpathlineto{\pgfqpoint{10.245566in}{2.931767in}}%
\pgfpathlineto{\pgfqpoint{10.251960in}{2.936970in}}%
\pgfpathlineto{\pgfqpoint{10.258354in}{2.942173in}}%
\pgfpathlineto{\pgfqpoint{10.264748in}{2.947376in}}%
\pgfpathlineto{\pgfqpoint{10.271143in}{2.952579in}}%
\pgfpathlineto{\pgfqpoint{10.277537in}{2.957782in}}%
\pgfpathlineto{\pgfqpoint{10.283931in}{2.962985in}}%
\pgfpathlineto{\pgfqpoint{10.290326in}{2.968188in}}%
\pgfpathlineto{\pgfqpoint{10.296720in}{2.973391in}}%
\pgfpathlineto{\pgfqpoint{10.303114in}{2.978594in}}%
\pgfpathlineto{\pgfqpoint{10.309508in}{2.983797in}}%
\pgfpathlineto{\pgfqpoint{10.315903in}{2.989000in}}%
\pgfpathlineto{\pgfqpoint{10.322297in}{2.994203in}}%
\pgfpathlineto{\pgfqpoint{10.328691in}{2.999406in}}%
\pgfpathlineto{\pgfqpoint{10.335086in}{3.004609in}}%
\pgfpathlineto{\pgfqpoint{10.341480in}{3.009812in}}%
\pgfpathlineto{\pgfqpoint{10.347874in}{3.015015in}}%
\pgfpathlineto{\pgfqpoint{10.354268in}{3.020218in}}%
\pgfpathlineto{\pgfqpoint{10.360663in}{3.025421in}}%
\pgfpathlineto{\pgfqpoint{10.367057in}{3.030624in}}%
\pgfpathlineto{\pgfqpoint{10.373451in}{3.035827in}}%
\pgfpathlineto{\pgfqpoint{10.379845in}{3.041029in}}%
\pgfpathlineto{\pgfqpoint{10.386240in}{3.046232in}}%
\pgfpathlineto{\pgfqpoint{10.392634in}{3.051435in}}%
\pgfpathlineto{\pgfqpoint{10.399028in}{3.056638in}}%
\pgfpathlineto{\pgfqpoint{10.405423in}{3.061841in}}%
\pgfpathlineto{\pgfqpoint{10.411817in}{3.067044in}}%
\pgfpathlineto{\pgfqpoint{10.418211in}{3.072247in}}%
\pgfpathlineto{\pgfqpoint{10.424605in}{3.077450in}}%
\pgfpathlineto{\pgfqpoint{10.431000in}{3.082653in}}%
\pgfpathlineto{\pgfqpoint{10.437394in}{3.087856in}}%
\pgfpathlineto{\pgfqpoint{10.443788in}{3.093059in}}%
\pgfpathlineto{\pgfqpoint{10.450183in}{3.098262in}}%
\pgfpathlineto{\pgfqpoint{10.456577in}{3.103465in}}%
\pgfpathlineto{\pgfqpoint{10.462971in}{3.108668in}}%
\pgfpathlineto{\pgfqpoint{10.469365in}{3.113871in}}%
\pgfpathlineto{\pgfqpoint{10.475760in}{3.119074in}}%
\pgfpathlineto{\pgfqpoint{10.482154in}{3.124277in}}%
\pgfpathlineto{\pgfqpoint{10.488548in}{3.129480in}}%
\pgfpathlineto{\pgfqpoint{10.494943in}{3.134683in}}%
\pgfpathlineto{\pgfqpoint{10.501337in}{3.139886in}}%
\pgfpathlineto{\pgfqpoint{10.507731in}{3.145089in}}%
\pgfpathlineto{\pgfqpoint{10.514125in}{3.150292in}}%
\pgfpathlineto{\pgfqpoint{10.514125in}{3.155495in}}%
\pgfpathlineto{\pgfqpoint{10.514125in}{3.155495in}}%
\pgfpathlineto{\pgfqpoint{10.507731in}{3.160698in}}%
\pgfpathlineto{\pgfqpoint{10.501337in}{3.165901in}}%
\pgfpathlineto{\pgfqpoint{10.494943in}{3.171104in}}%
\pgfpathlineto{\pgfqpoint{10.488548in}{3.176307in}}%
\pgfpathlineto{\pgfqpoint{10.482154in}{3.181510in}}%
\pgfpathlineto{\pgfqpoint{10.475760in}{3.186713in}}%
\pgfpathlineto{\pgfqpoint{10.469365in}{3.191916in}}%
\pgfpathlineto{\pgfqpoint{10.462971in}{3.197118in}}%
\pgfpathlineto{\pgfqpoint{10.456577in}{3.202321in}}%
\pgfpathlineto{\pgfqpoint{10.450183in}{3.207524in}}%
\pgfpathlineto{\pgfqpoint{10.443788in}{3.212727in}}%
\pgfpathlineto{\pgfqpoint{10.437394in}{3.217930in}}%
\pgfpathlineto{\pgfqpoint{10.431000in}{3.223133in}}%
\pgfpathlineto{\pgfqpoint{10.424605in}{3.228336in}}%
\pgfpathlineto{\pgfqpoint{10.418211in}{3.233539in}}%
\pgfpathlineto{\pgfqpoint{10.411817in}{3.238742in}}%
\pgfpathlineto{\pgfqpoint{10.405423in}{3.243945in}}%
\pgfpathlineto{\pgfqpoint{10.399028in}{3.249148in}}%
\pgfpathlineto{\pgfqpoint{10.392634in}{3.254351in}}%
\pgfpathlineto{\pgfqpoint{10.386240in}{3.259554in}}%
\pgfpathlineto{\pgfqpoint{10.379845in}{3.264757in}}%
\pgfpathlineto{\pgfqpoint{10.373451in}{3.269960in}}%
\pgfpathlineto{\pgfqpoint{10.367057in}{3.275163in}}%
\pgfpathlineto{\pgfqpoint{10.360663in}{3.280366in}}%
\pgfpathlineto{\pgfqpoint{10.354268in}{3.285569in}}%
\pgfpathlineto{\pgfqpoint{10.347874in}{3.290772in}}%
\pgfpathlineto{\pgfqpoint{10.341480in}{3.295975in}}%
\pgfpathlineto{\pgfqpoint{10.335086in}{3.301178in}}%
\pgfpathlineto{\pgfqpoint{10.328691in}{3.306381in}}%
\pgfpathlineto{\pgfqpoint{10.322297in}{3.311584in}}%
\pgfpathlineto{\pgfqpoint{10.315903in}{3.316787in}}%
\pgfpathlineto{\pgfqpoint{10.309508in}{3.321990in}}%
\pgfpathlineto{\pgfqpoint{10.303114in}{3.327193in}}%
\pgfpathlineto{\pgfqpoint{10.296720in}{3.332396in}}%
\pgfpathlineto{\pgfqpoint{10.290326in}{3.337599in}}%
\pgfpathlineto{\pgfqpoint{10.283931in}{3.342802in}}%
\pgfpathlineto{\pgfqpoint{10.277537in}{3.348004in}}%
\pgfpathlineto{\pgfqpoint{10.271143in}{3.353207in}}%
\pgfpathlineto{\pgfqpoint{10.264748in}{3.358410in}}%
\pgfpathlineto{\pgfqpoint{10.258354in}{3.363613in}}%
\pgfpathlineto{\pgfqpoint{10.251960in}{3.368816in}}%
\pgfpathlineto{\pgfqpoint{10.245566in}{3.374019in}}%
\pgfpathlineto{\pgfqpoint{10.239171in}{3.379222in}}%
\pgfpathlineto{\pgfqpoint{10.232777in}{3.384425in}}%
\pgfpathlineto{\pgfqpoint{10.226383in}{3.389628in}}%
\pgfpathlineto{\pgfqpoint{10.219989in}{3.394831in}}%
\pgfpathlineto{\pgfqpoint{10.213594in}{3.400034in}}%
\pgfpathlineto{\pgfqpoint{10.207200in}{3.405237in}}%
\pgfpathlineto{\pgfqpoint{10.200806in}{3.410440in}}%
\pgfpathlineto{\pgfqpoint{10.194411in}{3.415643in}}%
\pgfpathlineto{\pgfqpoint{10.188017in}{3.420846in}}%
\pgfpathlineto{\pgfqpoint{10.181623in}{3.426049in}}%
\pgfpathlineto{\pgfqpoint{10.175229in}{3.431252in}}%
\pgfpathlineto{\pgfqpoint{10.168834in}{3.436455in}}%
\pgfpathlineto{\pgfqpoint{10.162440in}{3.441658in}}%
\pgfpathlineto{\pgfqpoint{10.156046in}{3.446861in}}%
\pgfpathlineto{\pgfqpoint{10.149651in}{3.452064in}}%
\pgfpathlineto{\pgfqpoint{10.143257in}{3.457267in}}%
\pgfpathlineto{\pgfqpoint{10.136863in}{3.462470in}}%
\pgfpathlineto{\pgfqpoint{10.130469in}{3.467673in}}%
\pgfpathlineto{\pgfqpoint{10.124074in}{3.472876in}}%
\pgfpathlineto{\pgfqpoint{10.117680in}{3.478079in}}%
\pgfpathlineto{\pgfqpoint{10.111286in}{3.483282in}}%
\pgfpathlineto{\pgfqpoint{10.104891in}{3.488485in}}%
\pgfpathlineto{\pgfqpoint{10.098497in}{3.493688in}}%
\pgfpathlineto{\pgfqpoint{10.092103in}{3.498890in}}%
\pgfpathlineto{\pgfqpoint{10.085709in}{3.504093in}}%
\pgfpathlineto{\pgfqpoint{10.079314in}{3.509296in}}%
\pgfpathlineto{\pgfqpoint{10.072920in}{3.514499in}}%
\pgfpathlineto{\pgfqpoint{10.066526in}{3.519702in}}%
\pgfpathlineto{\pgfqpoint{10.060132in}{3.524905in}}%
\pgfpathlineto{\pgfqpoint{10.053737in}{3.530108in}}%
\pgfpathlineto{\pgfqpoint{10.047343in}{3.535311in}}%
\pgfpathlineto{\pgfqpoint{10.040949in}{3.540514in}}%
\pgfpathlineto{\pgfqpoint{10.034554in}{3.545717in}}%
\pgfpathlineto{\pgfqpoint{10.028160in}{3.550920in}}%
\pgfpathlineto{\pgfqpoint{10.021766in}{3.556123in}}%
\pgfpathlineto{\pgfqpoint{10.015372in}{3.561326in}}%
\pgfpathlineto{\pgfqpoint{10.008977in}{3.566529in}}%
\pgfpathlineto{\pgfqpoint{10.002583in}{3.571732in}}%
\pgfpathlineto{\pgfqpoint{9.996189in}{3.576935in}}%
\pgfpathlineto{\pgfqpoint{9.989794in}{3.582138in}}%
\pgfpathlineto{\pgfqpoint{9.983400in}{3.587341in}}%
\pgfpathlineto{\pgfqpoint{9.977006in}{3.592544in}}%
\pgfpathlineto{\pgfqpoint{9.970612in}{3.597747in}}%
\pgfpathlineto{\pgfqpoint{9.964217in}{3.602950in}}%
\pgfpathlineto{\pgfqpoint{9.957823in}{3.608153in}}%
\pgfpathlineto{\pgfqpoint{9.951429in}{3.613356in}}%
\pgfpathlineto{\pgfqpoint{9.945034in}{3.618559in}}%
\pgfpathlineto{\pgfqpoint{9.938640in}{3.623762in}}%
\pgfpathlineto{\pgfqpoint{9.932246in}{3.628965in}}%
\pgfpathlineto{\pgfqpoint{9.925852in}{3.634168in}}%
\pgfpathlineto{\pgfqpoint{9.919457in}{3.639371in}}%
\pgfpathlineto{\pgfqpoint{9.913063in}{3.644574in}}%
\pgfpathlineto{\pgfqpoint{9.906669in}{3.649777in}}%
\pgfpathlineto{\pgfqpoint{9.900275in}{3.654979in}}%
\pgfpathlineto{\pgfqpoint{9.893880in}{3.660182in}}%
\pgfpathlineto{\pgfqpoint{9.887486in}{3.665385in}}%
\pgfpathlineto{\pgfqpoint{9.881092in}{3.670588in}}%
\pgfpathlineto{\pgfqpoint{9.874697in}{3.675791in}}%
\pgfpathlineto{\pgfqpoint{9.868303in}{3.680994in}}%
\pgfpathlineto{\pgfqpoint{9.861909in}{3.686197in}}%
\pgfpathlineto{\pgfqpoint{9.855515in}{3.691400in}}%
\pgfpathlineto{\pgfqpoint{9.849120in}{3.696603in}}%
\pgfpathlineto{\pgfqpoint{9.842726in}{3.701806in}}%
\pgfpathlineto{\pgfqpoint{9.836332in}{3.707009in}}%
\pgfpathlineto{\pgfqpoint{9.829937in}{3.712212in}}%
\pgfpathlineto{\pgfqpoint{9.823543in}{3.717415in}}%
\pgfpathlineto{\pgfqpoint{9.817149in}{3.722618in}}%
\pgfpathlineto{\pgfqpoint{9.810755in}{3.727821in}}%
\pgfpathlineto{\pgfqpoint{9.804360in}{3.733024in}}%
\pgfpathlineto{\pgfqpoint{9.797966in}{3.738227in}}%
\pgfpathlineto{\pgfqpoint{9.791572in}{3.743430in}}%
\pgfpathlineto{\pgfqpoint{9.785177in}{3.748633in}}%
\pgfpathlineto{\pgfqpoint{9.778783in}{3.753836in}}%
\pgfpathlineto{\pgfqpoint{9.772389in}{3.759039in}}%
\pgfpathlineto{\pgfqpoint{9.765995in}{3.764242in}}%
\pgfpathlineto{\pgfqpoint{9.759600in}{3.769445in}}%
\pgfpathlineto{\pgfqpoint{9.753206in}{3.774648in}}%
\pgfpathlineto{\pgfqpoint{9.746812in}{3.779851in}}%
\pgfpathlineto{\pgfqpoint{9.740418in}{3.785054in}}%
\pgfpathlineto{\pgfqpoint{9.734023in}{3.790257in}}%
\pgfpathlineto{\pgfqpoint{9.727629in}{3.795460in}}%
\pgfpathlineto{\pgfqpoint{9.721235in}{3.800663in}}%
\pgfpathlineto{\pgfqpoint{9.714840in}{3.805865in}}%
\pgfpathlineto{\pgfqpoint{9.708446in}{3.811068in}}%
\pgfpathlineto{\pgfqpoint{9.702052in}{3.816271in}}%
\pgfpathlineto{\pgfqpoint{9.695658in}{3.821474in}}%
\pgfpathlineto{\pgfqpoint{9.689263in}{3.826677in}}%
\pgfpathlineto{\pgfqpoint{9.682869in}{3.831880in}}%
\pgfpathlineto{\pgfqpoint{9.676475in}{3.837083in}}%
\pgfpathlineto{\pgfqpoint{9.670080in}{3.842286in}}%
\pgfpathlineto{\pgfqpoint{9.663686in}{3.847489in}}%
\pgfpathlineto{\pgfqpoint{9.657292in}{3.852692in}}%
\pgfpathlineto{\pgfqpoint{9.650898in}{3.857895in}}%
\pgfpathlineto{\pgfqpoint{9.644503in}{3.863098in}}%
\pgfpathlineto{\pgfqpoint{9.638109in}{3.868301in}}%
\pgfpathlineto{\pgfqpoint{9.631715in}{3.873504in}}%
\pgfpathlineto{\pgfqpoint{9.625321in}{3.878707in}}%
\pgfpathlineto{\pgfqpoint{9.618926in}{3.883910in}}%
\pgfpathlineto{\pgfqpoint{9.612532in}{3.889113in}}%
\pgfpathlineto{\pgfqpoint{9.606138in}{3.894316in}}%
\pgfpathlineto{\pgfqpoint{9.599743in}{3.899519in}}%
\pgfpathlineto{\pgfqpoint{9.593349in}{3.904722in}}%
\pgfpathlineto{\pgfqpoint{9.586955in}{3.909925in}}%
\pgfpathlineto{\pgfqpoint{9.580561in}{3.915128in}}%
\pgfpathlineto{\pgfqpoint{9.574166in}{3.920331in}}%
\pgfpathlineto{\pgfqpoint{9.567772in}{3.925534in}}%
\pgfpathlineto{\pgfqpoint{9.561378in}{3.930737in}}%
\pgfpathlineto{\pgfqpoint{9.554983in}{3.935940in}}%
\pgfpathlineto{\pgfqpoint{9.548589in}{3.941143in}}%
\pgfpathlineto{\pgfqpoint{9.542195in}{3.946346in}}%
\pgfpathlineto{\pgfqpoint{9.535801in}{3.951549in}}%
\pgfpathlineto{\pgfqpoint{9.529406in}{3.956751in}}%
\pgfpathlineto{\pgfqpoint{9.523012in}{3.961954in}}%
\pgfpathlineto{\pgfqpoint{9.516618in}{3.967157in}}%
\pgfpathlineto{\pgfqpoint{9.510223in}{3.972360in}}%
\pgfpathlineto{\pgfqpoint{9.503829in}{3.977563in}}%
\pgfpathlineto{\pgfqpoint{9.497435in}{3.982766in}}%
\pgfpathlineto{\pgfqpoint{9.491041in}{3.987969in}}%
\pgfpathlineto{\pgfqpoint{9.484646in}{3.993172in}}%
\pgfpathlineto{\pgfqpoint{9.478252in}{3.998375in}}%
\pgfpathlineto{\pgfqpoint{9.471858in}{4.003578in}}%
\pgfpathlineto{\pgfqpoint{9.465464in}{4.008781in}}%
\pgfpathlineto{\pgfqpoint{9.459069in}{4.013984in}}%
\pgfpathlineto{\pgfqpoint{9.452675in}{4.019187in}}%
\pgfpathlineto{\pgfqpoint{9.446281in}{4.024390in}}%
\pgfpathlineto{\pgfqpoint{9.439886in}{4.029593in}}%
\pgfpathlineto{\pgfqpoint{9.433492in}{4.034796in}}%
\pgfpathlineto{\pgfqpoint{9.427098in}{4.039999in}}%
\pgfpathlineto{\pgfqpoint{9.420704in}{4.045202in}}%
\pgfpathlineto{\pgfqpoint{9.414309in}{4.050405in}}%
\pgfpathlineto{\pgfqpoint{9.407915in}{4.055608in}}%
\pgfpathlineto{\pgfqpoint{9.401521in}{4.060811in}}%
\pgfpathlineto{\pgfqpoint{9.395126in}{4.066014in}}%
\pgfpathlineto{\pgfqpoint{9.388732in}{4.071217in}}%
\pgfpathlineto{\pgfqpoint{9.382338in}{4.076420in}}%
\pgfpathlineto{\pgfqpoint{9.375944in}{4.081623in}}%
\pgfpathlineto{\pgfqpoint{9.369549in}{4.086826in}}%
\pgfpathlineto{\pgfqpoint{9.363155in}{4.092029in}}%
\pgfpathlineto{\pgfqpoint{9.356761in}{4.097232in}}%
\pgfpathlineto{\pgfqpoint{9.350366in}{4.102435in}}%
\pgfpathlineto{\pgfqpoint{9.343972in}{4.107638in}}%
\pgfpathlineto{\pgfqpoint{9.337578in}{4.112840in}}%
\pgfpathlineto{\pgfqpoint{9.331184in}{4.118043in}}%
\pgfpathlineto{\pgfqpoint{9.324789in}{4.123246in}}%
\pgfpathlineto{\pgfqpoint{9.318395in}{4.128449in}}%
\pgfpathlineto{\pgfqpoint{9.312001in}{4.133652in}}%
\pgfpathlineto{\pgfqpoint{9.305607in}{4.138855in}}%
\pgfpathlineto{\pgfqpoint{9.299212in}{4.144058in}}%
\pgfpathlineto{\pgfqpoint{9.292818in}{4.149261in}}%
\pgfpathlineto{\pgfqpoint{9.286424in}{4.154464in}}%
\pgfpathlineto{\pgfqpoint{9.280029in}{4.159667in}}%
\pgfpathlineto{\pgfqpoint{9.273635in}{4.164870in}}%
\pgfpathlineto{\pgfqpoint{9.267241in}{4.170073in}}%
\pgfpathlineto{\pgfqpoint{9.260847in}{4.175276in}}%
\pgfpathlineto{\pgfqpoint{9.254452in}{4.180479in}}%
\pgfpathlineto{\pgfqpoint{9.248058in}{4.185682in}}%
\pgfpathlineto{\pgfqpoint{9.241664in}{4.190885in}}%
\pgfpathlineto{\pgfqpoint{9.241664in}{4.190885in}}%
\pgfpathclose%
\pgfusepath{stroke,fill}%
}%
\begin{pgfscope}%
\pgfsys@transformshift{0.000000in}{0.000000in}%
\pgfsys@useobject{currentmarker}{}%
\end{pgfscope}%
\end{pgfscope}%
\begin{pgfscope}%
\pgfsetbuttcap%
\pgfsetroundjoin%
\definecolor{currentfill}{rgb}{0.000000,0.000000,0.000000}%
\pgfsetfillcolor{currentfill}%
\pgfsetlinewidth{0.803000pt}%
\definecolor{currentstroke}{rgb}{0.000000,0.000000,0.000000}%
\pgfsetstrokecolor{currentstroke}%
\pgfsetdash{}{0pt}%
\pgfsys@defobject{currentmarker}{\pgfqpoint{0.000000in}{-0.048611in}}{\pgfqpoint{0.000000in}{0.000000in}}{%
\pgfpathmoveto{\pgfqpoint{0.000000in}{0.000000in}}%
\pgfpathlineto{\pgfqpoint{0.000000in}{-0.048611in}}%
\pgfusepath{stroke,fill}%
}%
\begin{pgfscope}%
\pgfsys@transformshift{7.323380in}{0.554012in}%
\pgfsys@useobject{currentmarker}{}%
\end{pgfscope}%
\end{pgfscope}%
\begin{pgfscope}%
\definecolor{textcolor}{rgb}{0.000000,0.000000,0.000000}%
\pgfsetstrokecolor{textcolor}%
\pgfsetfillcolor{textcolor}%
\pgftext[x=7.323380in,y=0.456790in,,top]{\color{textcolor}\rmfamily\fontsize{10.000000}{12.000000}\selectfont \(\displaystyle {0.0}\)}%
\end{pgfscope}%
\begin{pgfscope}%
\pgfsetbuttcap%
\pgfsetroundjoin%
\definecolor{currentfill}{rgb}{0.000000,0.000000,0.000000}%
\pgfsetfillcolor{currentfill}%
\pgfsetlinewidth{0.803000pt}%
\definecolor{currentstroke}{rgb}{0.000000,0.000000,0.000000}%
\pgfsetstrokecolor{currentstroke}%
\pgfsetdash{}{0pt}%
\pgfsys@defobject{currentmarker}{\pgfqpoint{0.000000in}{-0.048611in}}{\pgfqpoint{0.000000in}{0.000000in}}{%
\pgfpathmoveto{\pgfqpoint{0.000000in}{0.000000in}}%
\pgfpathlineto{\pgfqpoint{0.000000in}{-0.048611in}}%
\pgfusepath{stroke,fill}%
}%
\begin{pgfscope}%
\pgfsys@transformshift{8.600957in}{0.554012in}%
\pgfsys@useobject{currentmarker}{}%
\end{pgfscope}%
\end{pgfscope}%
\begin{pgfscope}%
\definecolor{textcolor}{rgb}{0.000000,0.000000,0.000000}%
\pgfsetstrokecolor{textcolor}%
\pgfsetfillcolor{textcolor}%
\pgftext[x=8.600957in,y=0.456790in,,top]{\color{textcolor}\rmfamily\fontsize{10.000000}{12.000000}\selectfont \(\displaystyle {0.2}\)}%
\end{pgfscope}%
\begin{pgfscope}%
\pgfsetbuttcap%
\pgfsetroundjoin%
\definecolor{currentfill}{rgb}{0.000000,0.000000,0.000000}%
\pgfsetfillcolor{currentfill}%
\pgfsetlinewidth{0.803000pt}%
\definecolor{currentstroke}{rgb}{0.000000,0.000000,0.000000}%
\pgfsetstrokecolor{currentstroke}%
\pgfsetdash{}{0pt}%
\pgfsys@defobject{currentmarker}{\pgfqpoint{0.000000in}{-0.048611in}}{\pgfqpoint{0.000000in}{0.000000in}}{%
\pgfpathmoveto{\pgfqpoint{0.000000in}{0.000000in}}%
\pgfpathlineto{\pgfqpoint{0.000000in}{-0.048611in}}%
\pgfusepath{stroke,fill}%
}%
\begin{pgfscope}%
\pgfsys@transformshift{9.878534in}{0.554012in}%
\pgfsys@useobject{currentmarker}{}%
\end{pgfscope}%
\end{pgfscope}%
\begin{pgfscope}%
\definecolor{textcolor}{rgb}{0.000000,0.000000,0.000000}%
\pgfsetstrokecolor{textcolor}%
\pgfsetfillcolor{textcolor}%
\pgftext[x=9.878534in,y=0.456790in,,top]{\color{textcolor}\rmfamily\fontsize{10.000000}{12.000000}\selectfont \(\displaystyle {0.4}\)}%
\end{pgfscope}%
\begin{pgfscope}%
\pgfsetbuttcap%
\pgfsetroundjoin%
\definecolor{currentfill}{rgb}{0.000000,0.000000,0.000000}%
\pgfsetfillcolor{currentfill}%
\pgfsetlinewidth{0.803000pt}%
\definecolor{currentstroke}{rgb}{0.000000,0.000000,0.000000}%
\pgfsetstrokecolor{currentstroke}%
\pgfsetdash{}{0pt}%
\pgfsys@defobject{currentmarker}{\pgfqpoint{0.000000in}{-0.048611in}}{\pgfqpoint{0.000000in}{0.000000in}}{%
\pgfpathmoveto{\pgfqpoint{0.000000in}{0.000000in}}%
\pgfpathlineto{\pgfqpoint{0.000000in}{-0.048611in}}%
\pgfusepath{stroke,fill}%
}%
\begin{pgfscope}%
\pgfsys@transformshift{11.156111in}{0.554012in}%
\pgfsys@useobject{currentmarker}{}%
\end{pgfscope}%
\end{pgfscope}%
\begin{pgfscope}%
\definecolor{textcolor}{rgb}{0.000000,0.000000,0.000000}%
\pgfsetstrokecolor{textcolor}%
\pgfsetfillcolor{textcolor}%
\pgftext[x=11.156111in,y=0.456790in,,top]{\color{textcolor}\rmfamily\fontsize{10.000000}{12.000000}\selectfont \(\displaystyle {0.6}\)}%
\end{pgfscope}%
\begin{pgfscope}%
\pgfsetbuttcap%
\pgfsetroundjoin%
\definecolor{currentfill}{rgb}{0.000000,0.000000,0.000000}%
\pgfsetfillcolor{currentfill}%
\pgfsetlinewidth{0.803000pt}%
\definecolor{currentstroke}{rgb}{0.000000,0.000000,0.000000}%
\pgfsetstrokecolor{currentstroke}%
\pgfsetdash{}{0pt}%
\pgfsys@defobject{currentmarker}{\pgfqpoint{0.000000in}{-0.048611in}}{\pgfqpoint{0.000000in}{0.000000in}}{%
\pgfpathmoveto{\pgfqpoint{0.000000in}{0.000000in}}%
\pgfpathlineto{\pgfqpoint{0.000000in}{-0.048611in}}%
\pgfusepath{stroke,fill}%
}%
\begin{pgfscope}%
\pgfsys@transformshift{12.433688in}{0.554012in}%
\pgfsys@useobject{currentmarker}{}%
\end{pgfscope}%
\end{pgfscope}%
\begin{pgfscope}%
\definecolor{textcolor}{rgb}{0.000000,0.000000,0.000000}%
\pgfsetstrokecolor{textcolor}%
\pgfsetfillcolor{textcolor}%
\pgftext[x=12.433688in,y=0.456790in,,top]{\color{textcolor}\rmfamily\fontsize{10.000000}{12.000000}\selectfont \(\displaystyle {0.8}\)}%
\end{pgfscope}%
\begin{pgfscope}%
\pgfsetbuttcap%
\pgfsetroundjoin%
\definecolor{currentfill}{rgb}{0.000000,0.000000,0.000000}%
\pgfsetfillcolor{currentfill}%
\pgfsetlinewidth{0.803000pt}%
\definecolor{currentstroke}{rgb}{0.000000,0.000000,0.000000}%
\pgfsetstrokecolor{currentstroke}%
\pgfsetdash{}{0pt}%
\pgfsys@defobject{currentmarker}{\pgfqpoint{0.000000in}{-0.048611in}}{\pgfqpoint{0.000000in}{0.000000in}}{%
\pgfpathmoveto{\pgfqpoint{0.000000in}{0.000000in}}%
\pgfpathlineto{\pgfqpoint{0.000000in}{-0.048611in}}%
\pgfusepath{stroke,fill}%
}%
\begin{pgfscope}%
\pgfsys@transformshift{13.711265in}{0.554012in}%
\pgfsys@useobject{currentmarker}{}%
\end{pgfscope}%
\end{pgfscope}%
\begin{pgfscope}%
\definecolor{textcolor}{rgb}{0.000000,0.000000,0.000000}%
\pgfsetstrokecolor{textcolor}%
\pgfsetfillcolor{textcolor}%
\pgftext[x=13.711265in,y=0.456790in,,top]{\color{textcolor}\rmfamily\fontsize{10.000000}{12.000000}\selectfont \(\displaystyle {1.0}\)}%
\end{pgfscope}%
\begin{pgfscope}%
\definecolor{textcolor}{rgb}{0.000000,0.000000,0.000000}%
\pgfsetstrokecolor{textcolor}%
\pgfsetfillcolor{textcolor}%
\pgftext[x=10.517323in,y=0.277777in,,top]{\color{textcolor}\rmfamily\fontsize{14.000000}{16.800000}\selectfont Normalized Quantity}%
\end{pgfscope}%
\begin{pgfscope}%
\pgfsetbuttcap%
\pgfsetroundjoin%
\definecolor{currentfill}{rgb}{0.000000,0.000000,0.000000}%
\pgfsetfillcolor{currentfill}%
\pgfsetlinewidth{0.803000pt}%
\definecolor{currentstroke}{rgb}{0.000000,0.000000,0.000000}%
\pgfsetstrokecolor{currentstroke}%
\pgfsetdash{}{0pt}%
\pgfsys@defobject{currentmarker}{\pgfqpoint{-0.048611in}{0.000000in}}{\pgfqpoint{-0.000000in}{0.000000in}}{%
\pgfpathmoveto{\pgfqpoint{-0.000000in}{0.000000in}}%
\pgfpathlineto{\pgfqpoint{-0.048611in}{0.000000in}}%
\pgfusepath{stroke,fill}%
}%
\begin{pgfscope}%
\pgfsys@transformshift{7.323380in}{0.554012in}%
\pgfsys@useobject{currentmarker}{}%
\end{pgfscope}%
\end{pgfscope}%
\begin{pgfscope}%
\pgfsetbuttcap%
\pgfsetroundjoin%
\definecolor{currentfill}{rgb}{0.000000,0.000000,0.000000}%
\pgfsetfillcolor{currentfill}%
\pgfsetlinewidth{0.803000pt}%
\definecolor{currentstroke}{rgb}{0.000000,0.000000,0.000000}%
\pgfsetstrokecolor{currentstroke}%
\pgfsetdash{}{0pt}%
\pgfsys@defobject{currentmarker}{\pgfqpoint{-0.048611in}{0.000000in}}{\pgfqpoint{-0.000000in}{0.000000in}}{%
\pgfpathmoveto{\pgfqpoint{-0.000000in}{0.000000in}}%
\pgfpathlineto{\pgfqpoint{-0.048611in}{0.000000in}}%
\pgfusepath{stroke,fill}%
}%
\begin{pgfscope}%
\pgfsys@transformshift{7.323380in}{1.593564in}%
\pgfsys@useobject{currentmarker}{}%
\end{pgfscope}%
\end{pgfscope}%
\begin{pgfscope}%
\pgfsetbuttcap%
\pgfsetroundjoin%
\definecolor{currentfill}{rgb}{0.000000,0.000000,0.000000}%
\pgfsetfillcolor{currentfill}%
\pgfsetlinewidth{0.803000pt}%
\definecolor{currentstroke}{rgb}{0.000000,0.000000,0.000000}%
\pgfsetstrokecolor{currentstroke}%
\pgfsetdash{}{0pt}%
\pgfsys@defobject{currentmarker}{\pgfqpoint{-0.048611in}{0.000000in}}{\pgfqpoint{-0.000000in}{0.000000in}}{%
\pgfpathmoveto{\pgfqpoint{-0.000000in}{0.000000in}}%
\pgfpathlineto{\pgfqpoint{-0.048611in}{0.000000in}}%
\pgfusepath{stroke,fill}%
}%
\begin{pgfscope}%
\pgfsys@transformshift{7.323380in}{2.633117in}%
\pgfsys@useobject{currentmarker}{}%
\end{pgfscope}%
\end{pgfscope}%
\begin{pgfscope}%
\pgfsetbuttcap%
\pgfsetroundjoin%
\definecolor{currentfill}{rgb}{0.000000,0.000000,0.000000}%
\pgfsetfillcolor{currentfill}%
\pgfsetlinewidth{0.803000pt}%
\definecolor{currentstroke}{rgb}{0.000000,0.000000,0.000000}%
\pgfsetstrokecolor{currentstroke}%
\pgfsetdash{}{0pt}%
\pgfsys@defobject{currentmarker}{\pgfqpoint{-0.048611in}{0.000000in}}{\pgfqpoint{-0.000000in}{0.000000in}}{%
\pgfpathmoveto{\pgfqpoint{-0.000000in}{0.000000in}}%
\pgfpathlineto{\pgfqpoint{-0.048611in}{0.000000in}}%
\pgfusepath{stroke,fill}%
}%
\begin{pgfscope}%
\pgfsys@transformshift{7.323380in}{3.672670in}%
\pgfsys@useobject{currentmarker}{}%
\end{pgfscope}%
\end{pgfscope}%
\begin{pgfscope}%
\pgfsetbuttcap%
\pgfsetroundjoin%
\definecolor{currentfill}{rgb}{0.000000,0.000000,0.000000}%
\pgfsetfillcolor{currentfill}%
\pgfsetlinewidth{0.803000pt}%
\definecolor{currentstroke}{rgb}{0.000000,0.000000,0.000000}%
\pgfsetstrokecolor{currentstroke}%
\pgfsetdash{}{0pt}%
\pgfsys@defobject{currentmarker}{\pgfqpoint{-0.048611in}{0.000000in}}{\pgfqpoint{-0.000000in}{0.000000in}}{%
\pgfpathmoveto{\pgfqpoint{-0.000000in}{0.000000in}}%
\pgfpathlineto{\pgfqpoint{-0.048611in}{0.000000in}}%
\pgfusepath{stroke,fill}%
}%
\begin{pgfscope}%
\pgfsys@transformshift{7.323380in}{4.712222in}%
\pgfsys@useobject{currentmarker}{}%
\end{pgfscope}%
\end{pgfscope}%
\begin{pgfscope}%
\pgfsetbuttcap%
\pgfsetroundjoin%
\definecolor{currentfill}{rgb}{0.000000,0.000000,0.000000}%
\pgfsetfillcolor{currentfill}%
\pgfsetlinewidth{0.803000pt}%
\definecolor{currentstroke}{rgb}{0.000000,0.000000,0.000000}%
\pgfsetstrokecolor{currentstroke}%
\pgfsetdash{}{0pt}%
\pgfsys@defobject{currentmarker}{\pgfqpoint{-0.048611in}{0.000000in}}{\pgfqpoint{-0.000000in}{0.000000in}}{%
\pgfpathmoveto{\pgfqpoint{-0.000000in}{0.000000in}}%
\pgfpathlineto{\pgfqpoint{-0.048611in}{0.000000in}}%
\pgfusepath{stroke,fill}%
}%
\begin{pgfscope}%
\pgfsys@transformshift{7.323380in}{5.751775in}%
\pgfsys@useobject{currentmarker}{}%
\end{pgfscope}%
\end{pgfscope}%
\begin{pgfscope}%
\pgfpathrectangle{\pgfqpoint{7.323380in}{0.554012in}}{\pgfqpoint{6.387885in}{5.197763in}}%
\pgfusepath{clip}%
\pgfsetrectcap%
\pgfsetroundjoin%
\pgfsetlinewidth{1.505625pt}%
\definecolor{currentstroke}{rgb}{0.121569,0.466667,0.705882}%
\pgfsetstrokecolor{currentstroke}%
\pgfsetdash{}{0pt}%
\pgfpathmoveto{\pgfqpoint{7.323380in}{5.751775in}}%
\pgfpathlineto{\pgfqpoint{13.711265in}{0.554012in}}%
\pgfpathlineto{\pgfqpoint{13.711265in}{0.554012in}}%
\pgfusepath{stroke}%
\end{pgfscope}%
\begin{pgfscope}%
\pgfpathrectangle{\pgfqpoint{7.323380in}{0.554012in}}{\pgfqpoint{6.387885in}{5.197763in}}%
\pgfusepath{clip}%
\pgfsetrectcap%
\pgfsetroundjoin%
\pgfsetlinewidth{1.505625pt}%
\definecolor{currentstroke}{rgb}{1.000000,0.498039,0.054902}%
\pgfsetstrokecolor{currentstroke}%
\pgfsetdash{}{0pt}%
\pgfpathmoveto{\pgfqpoint{7.323380in}{0.554012in}}%
\pgfpathlineto{\pgfqpoint{13.711265in}{5.751775in}}%
\pgfpathlineto{\pgfqpoint{13.711265in}{5.751775in}}%
\pgfusepath{stroke}%
\end{pgfscope}%
\begin{pgfscope}%
\pgfpathrectangle{\pgfqpoint{7.323380in}{0.554012in}}{\pgfqpoint{6.387885in}{5.197763in}}%
\pgfusepath{clip}%
\pgfsetbuttcap%
\pgfsetroundjoin%
\pgfsetlinewidth{1.505625pt}%
\definecolor{currentstroke}{rgb}{1.000000,0.000000,0.000000}%
\pgfsetstrokecolor{currentstroke}%
\pgfsetstrokeopacity{0.600000}%
\pgfsetdash{{5.550000pt}{2.400000pt}}{0.000000pt}%
\pgfpathmoveto{\pgfqpoint{7.323380in}{2.113341in}}%
\pgfpathlineto{\pgfqpoint{9.239745in}{2.113341in}}%
\pgfusepath{stroke}%
\end{pgfscope}%
\begin{pgfscope}%
\pgfsetrectcap%
\pgfsetmiterjoin%
\pgfsetlinewidth{0.803000pt}%
\definecolor{currentstroke}{rgb}{0.000000,0.000000,0.000000}%
\pgfsetstrokecolor{currentstroke}%
\pgfsetdash{}{0pt}%
\pgfpathmoveto{\pgfqpoint{7.323380in}{0.554012in}}%
\pgfpathlineto{\pgfqpoint{7.323380in}{5.751775in}}%
\pgfusepath{stroke}%
\end{pgfscope}%
\begin{pgfscope}%
\pgfsetrectcap%
\pgfsetmiterjoin%
\pgfsetlinewidth{0.803000pt}%
\definecolor{currentstroke}{rgb}{0.000000,0.000000,0.000000}%
\pgfsetstrokecolor{currentstroke}%
\pgfsetdash{}{0pt}%
\pgfpathmoveto{\pgfqpoint{13.711265in}{0.554012in}}%
\pgfpathlineto{\pgfqpoint{13.711265in}{5.751775in}}%
\pgfusepath{stroke}%
\end{pgfscope}%
\begin{pgfscope}%
\pgfsetrectcap%
\pgfsetmiterjoin%
\pgfsetlinewidth{0.803000pt}%
\definecolor{currentstroke}{rgb}{0.000000,0.000000,0.000000}%
\pgfsetstrokecolor{currentstroke}%
\pgfsetdash{}{0pt}%
\pgfpathmoveto{\pgfqpoint{7.323380in}{0.554012in}}%
\pgfpathlineto{\pgfqpoint{13.711265in}{0.554012in}}%
\pgfusepath{stroke}%
\end{pgfscope}%
\begin{pgfscope}%
\pgfsetrectcap%
\pgfsetmiterjoin%
\pgfsetlinewidth{0.803000pt}%
\definecolor{currentstroke}{rgb}{0.000000,0.000000,0.000000}%
\pgfsetstrokecolor{currentstroke}%
\pgfsetdash{}{0pt}%
\pgfpathmoveto{\pgfqpoint{7.323380in}{5.751775in}}%
\pgfpathlineto{\pgfqpoint{13.711265in}{5.751775in}}%
\pgfusepath{stroke}%
\end{pgfscope}%
\begin{pgfscope}%
\definecolor{textcolor}{rgb}{0.000000,0.000000,0.000000}%
\pgfsetstrokecolor{textcolor}%
\pgfsetfillcolor{textcolor}%
\pgftext[x=7.642774in, y=3.324189in, left, base]{\color{textcolor}\rmfamily\fontsize{12.000000}{14.400000}\selectfont Consumer }%
\end{pgfscope}%
\begin{pgfscope}%
\definecolor{textcolor}{rgb}{0.000000,0.000000,0.000000}%
\pgfsetstrokecolor{textcolor}%
\pgfsetfillcolor{textcolor}%
\pgftext[x=7.642774in, y=3.152893in, left, base]{\color{textcolor}\rmfamily\fontsize{12.000000}{14.400000}\selectfont Surplus}%
\end{pgfscope}%
\begin{pgfscope}%
\definecolor{textcolor}{rgb}{0.000000,0.000000,0.000000}%
\pgfsetstrokecolor{textcolor}%
\pgfsetfillcolor{textcolor}%
\pgftext[x=7.642774in, y=1.764860in, left, base]{\color{textcolor}\rmfamily\fontsize{12.000000}{14.400000}\selectfont Producer }%
\end{pgfscope}%
\begin{pgfscope}%
\definecolor{textcolor}{rgb}{0.000000,0.000000,0.000000}%
\pgfsetstrokecolor{textcolor}%
\pgfsetfillcolor{textcolor}%
\pgftext[x=7.642774in, y=1.593564in, left, base]{\color{textcolor}\rmfamily\fontsize{12.000000}{14.400000}\selectfont Surplus}%
\end{pgfscope}%
\begin{pgfscope}%
\definecolor{textcolor}{rgb}{0.000000,0.000000,0.000000}%
\pgfsetstrokecolor{textcolor}%
\pgfsetfillcolor{textcolor}%
\pgftext[x=9.367503in, y=3.064301in, left, base]{\color{textcolor}\rmfamily\fontsize{12.000000}{14.400000}\selectfont Lost }%
\end{pgfscope}%
\begin{pgfscope}%
\definecolor{textcolor}{rgb}{0.000000,0.000000,0.000000}%
\pgfsetstrokecolor{textcolor}%
\pgfsetfillcolor{textcolor}%
\pgftext[x=9.367503in, y=2.893005in, left, base]{\color{textcolor}\rmfamily\fontsize{12.000000}{14.400000}\selectfont Surplus}%
\end{pgfscope}%
\begin{pgfscope}%
\definecolor{textcolor}{rgb}{0.000000,0.000000,0.000000}%
\pgfsetstrokecolor{textcolor}%
\pgfsetfillcolor{textcolor}%
\pgftext[x=12.288456in, y=4.280992in, left, base,rotate=45.000000]{\color{textcolor}\rmfamily\fontsize{12.000000}{14.400000}\selectfont Supply}%
\end{pgfscope}%
\begin{pgfscope}%
\definecolor{textcolor}{rgb}{0.000000,0.000000,0.000000}%
\pgfsetstrokecolor{textcolor}%
\pgfsetfillcolor{textcolor}%
\pgftext[x=12.178594in, y=1.504902in, left, base,rotate=320.000000]{\color{textcolor}\rmfamily\fontsize{12.000000}{14.400000}\selectfont Demand}%
\end{pgfscope}%
\begin{pgfscope}%
\pgfsetbuttcap%
\pgfsetmiterjoin%
\definecolor{currentfill}{rgb}{1.000000,1.000000,1.000000}%
\pgfsetfillcolor{currentfill}%
\pgfsetlinewidth{1.003750pt}%
\definecolor{currentstroke}{rgb}{0.000000,0.000000,0.000000}%
\pgfsetstrokecolor{currentstroke}%
\pgfsetdash{}{0pt}%
\pgfpathmoveto{\pgfqpoint{7.330870in}{5.385498in}}%
\pgfpathlineto{\pgfqpoint{7.628599in}{5.385498in}}%
\pgfpathlineto{\pgfqpoint{7.628599in}{5.698275in}}%
\pgfpathlineto{\pgfqpoint{7.330870in}{5.698275in}}%
\pgfpathlineto{\pgfqpoint{7.330870in}{5.385498in}}%
\pgfpathclose%
\pgfusepath{stroke,fill}%
\end{pgfscope}%
\begin{pgfscope}%
\definecolor{textcolor}{rgb}{0.000000,0.000000,0.000000}%
\pgfsetstrokecolor{textcolor}%
\pgfsetfillcolor{textcolor}%
\pgftext[x=7.387259in,y=5.491887in,left,base]{\color{textcolor}\rmfamily\fontsize{14.000000}{16.800000}\selectfont b)}%
\end{pgfscope}%
\end{pgfpicture}%
\makeatother%
\endgroup%
}
  \caption{Demonstration of ``social welfare maximization.'' Plot a) shows the
  total surplus when the price is at equilibrium. Plot b) shows the total
  surplus when the price is artificially depressed.}
  \label{fig:social-max}
\end{figure}

In microeconomics, social welfare is identical to the sum of consumer and
producer surplus. Therefore social welfare is maximized when the sum of these
two quantities is maximized. Figure \ref{fig:social-max} shows this case on the
left panel. However, suppose an economic policy capped the price of some product
at a price lower than the equilibrium price. In that case, the consumer surplus
expands, and the producer surplus contracts, as shown in the right panel of
Figure \ref{fig:social-max}. Nobody receives the ``lost surplus'' because
suppliers do not produce more despite unmet demand for the product because the
price is capped. Typically, modeling tools consolidate the demand curve to a
single value. \textcolor{black}{In this case, social welfare maximization is
approximated by minimizing the total cost of energy
\cite{richstein_cross-border_2014}}. This simplification is valid because demand
for energy is highly inelastic \cite{heuberger_power_2017, eia_price_2021,
labandeira_meta-analysis_2017, csereklyei_price_2020}. Figure
\ref{fig:inelastic} shows the impact of highly inelastic demand.

\begin{figure}[H]
  \centering
  \resizebox{\columnwidth}{!}{%% Creator: Matplotlib, PGF backend
%%
%% To include the figure in your LaTeX document, write
%%   \input{<filename>.pgf}
%%
%% Make sure the required packages are loaded in your preamble
%%   \usepackage{pgf}
%%
%% Also ensure that all the required font packages are loaded; for instance,
%% the lmodern package is sometimes necessary when using math font.
%%   \usepackage{lmodern}
%%
%% Figures using additional raster images can only be included by \input if
%% they are in the same directory as the main LaTeX file. For loading figures
%% from other directories you can use the `import` package
%%   \usepackage{import}
%%
%% and then include the figures with
%%   \import{<path to file>}{<filename>.pgf}
%%
%% Matplotlib used the following preamble
%%
\begingroup%
\makeatletter%
\begin{pgfpicture}%
\pgfpathrectangle{\pgfpointorigin}{\pgfqpoint{13.900000in}{5.930000in}}%
\pgfusepath{use as bounding box, clip}%
\begin{pgfscope}%
\pgfsetbuttcap%
\pgfsetmiterjoin%
\definecolor{currentfill}{rgb}{1.000000,1.000000,1.000000}%
\pgfsetfillcolor{currentfill}%
\pgfsetlinewidth{0.000000pt}%
\definecolor{currentstroke}{rgb}{0.000000,0.000000,0.000000}%
\pgfsetstrokecolor{currentstroke}%
\pgfsetdash{}{0pt}%
\pgfpathmoveto{\pgfqpoint{0.000000in}{0.000000in}}%
\pgfpathlineto{\pgfqpoint{13.900000in}{0.000000in}}%
\pgfpathlineto{\pgfqpoint{13.900000in}{5.930000in}}%
\pgfpathlineto{\pgfqpoint{0.000000in}{5.930000in}}%
\pgfpathlineto{\pgfqpoint{0.000000in}{0.000000in}}%
\pgfpathclose%
\pgfusepath{fill}%
\end{pgfscope}%
\begin{pgfscope}%
\pgfsetbuttcap%
\pgfsetmiterjoin%
\definecolor{currentfill}{rgb}{1.000000,1.000000,1.000000}%
\pgfsetfillcolor{currentfill}%
\pgfsetlinewidth{0.000000pt}%
\definecolor{currentstroke}{rgb}{0.000000,0.000000,0.000000}%
\pgfsetstrokecolor{currentstroke}%
\pgfsetstrokeopacity{0.000000}%
\pgfsetdash{}{0pt}%
\pgfpathmoveto{\pgfqpoint{0.608025in}{0.554012in}}%
\pgfpathlineto{\pgfqpoint{6.995910in}{0.554012in}}%
\pgfpathlineto{\pgfqpoint{6.995910in}{5.388426in}}%
\pgfpathlineto{\pgfqpoint{0.608025in}{5.388426in}}%
\pgfpathlineto{\pgfqpoint{0.608025in}{0.554012in}}%
\pgfpathclose%
\pgfusepath{fill}%
\end{pgfscope}%
\begin{pgfscope}%
\pgfpathrectangle{\pgfqpoint{0.608025in}{0.554012in}}{\pgfqpoint{6.387885in}{4.834414in}}%
\pgfusepath{clip}%
\pgfsetbuttcap%
\pgfsetroundjoin%
\definecolor{currentfill}{rgb}{0.121569,0.466667,0.705882}%
\pgfsetfillcolor{currentfill}%
\pgfsetfillopacity{0.200000}%
\pgfsetlinewidth{0.000000pt}%
\definecolor{currentstroke}{rgb}{0.000000,0.000000,0.000000}%
\pgfsetstrokecolor{currentstroke}%
\pgfsetdash{}{0pt}%
\pgfpathmoveto{\pgfqpoint{0.608025in}{5.388426in}}%
\pgfpathlineto{\pgfqpoint{0.608025in}{2.971219in}}%
\pgfpathlineto{\pgfqpoint{0.614419in}{2.971219in}}%
\pgfpathlineto{\pgfqpoint{0.620813in}{2.971219in}}%
\pgfpathlineto{\pgfqpoint{0.627208in}{2.971219in}}%
\pgfpathlineto{\pgfqpoint{0.633602in}{2.971219in}}%
\pgfpathlineto{\pgfqpoint{0.639996in}{2.971219in}}%
\pgfpathlineto{\pgfqpoint{0.646391in}{2.971219in}}%
\pgfpathlineto{\pgfqpoint{0.652785in}{2.971219in}}%
\pgfpathlineto{\pgfqpoint{0.659179in}{2.971219in}}%
\pgfpathlineto{\pgfqpoint{0.665573in}{2.971219in}}%
\pgfpathlineto{\pgfqpoint{0.671968in}{2.971219in}}%
\pgfpathlineto{\pgfqpoint{0.678362in}{2.971219in}}%
\pgfpathlineto{\pgfqpoint{0.684756in}{2.971219in}}%
\pgfpathlineto{\pgfqpoint{0.691150in}{2.971219in}}%
\pgfpathlineto{\pgfqpoint{0.697545in}{2.971219in}}%
\pgfpathlineto{\pgfqpoint{0.703939in}{2.971219in}}%
\pgfpathlineto{\pgfqpoint{0.710333in}{2.971219in}}%
\pgfpathlineto{\pgfqpoint{0.716728in}{2.971219in}}%
\pgfpathlineto{\pgfqpoint{0.723122in}{2.971219in}}%
\pgfpathlineto{\pgfqpoint{0.729516in}{2.971219in}}%
\pgfpathlineto{\pgfqpoint{0.735910in}{2.971219in}}%
\pgfpathlineto{\pgfqpoint{0.742305in}{2.971219in}}%
\pgfpathlineto{\pgfqpoint{0.748699in}{2.971219in}}%
\pgfpathlineto{\pgfqpoint{0.755093in}{2.971219in}}%
\pgfpathlineto{\pgfqpoint{0.761488in}{2.971219in}}%
\pgfpathlineto{\pgfqpoint{0.767882in}{2.971219in}}%
\pgfpathlineto{\pgfqpoint{0.774276in}{2.971219in}}%
\pgfpathlineto{\pgfqpoint{0.780670in}{2.971219in}}%
\pgfpathlineto{\pgfqpoint{0.787065in}{2.971219in}}%
\pgfpathlineto{\pgfqpoint{0.793459in}{2.971219in}}%
\pgfpathlineto{\pgfqpoint{0.799853in}{2.971219in}}%
\pgfpathlineto{\pgfqpoint{0.806248in}{2.971219in}}%
\pgfpathlineto{\pgfqpoint{0.812642in}{2.971219in}}%
\pgfpathlineto{\pgfqpoint{0.819036in}{2.971219in}}%
\pgfpathlineto{\pgfqpoint{0.825430in}{2.971219in}}%
\pgfpathlineto{\pgfqpoint{0.831825in}{2.971219in}}%
\pgfpathlineto{\pgfqpoint{0.838219in}{2.971219in}}%
\pgfpathlineto{\pgfqpoint{0.844613in}{2.971219in}}%
\pgfpathlineto{\pgfqpoint{0.851007in}{2.971219in}}%
\pgfpathlineto{\pgfqpoint{0.857402in}{2.971219in}}%
\pgfpathlineto{\pgfqpoint{0.863796in}{2.971219in}}%
\pgfpathlineto{\pgfqpoint{0.870190in}{2.971219in}}%
\pgfpathlineto{\pgfqpoint{0.876585in}{2.971219in}}%
\pgfpathlineto{\pgfqpoint{0.882979in}{2.971219in}}%
\pgfpathlineto{\pgfqpoint{0.889373in}{2.971219in}}%
\pgfpathlineto{\pgfqpoint{0.895767in}{2.971219in}}%
\pgfpathlineto{\pgfqpoint{0.902162in}{2.971219in}}%
\pgfpathlineto{\pgfqpoint{0.908556in}{2.971219in}}%
\pgfpathlineto{\pgfqpoint{0.914950in}{2.971219in}}%
\pgfpathlineto{\pgfqpoint{0.921345in}{2.971219in}}%
\pgfpathlineto{\pgfqpoint{0.927739in}{2.971219in}}%
\pgfpathlineto{\pgfqpoint{0.934133in}{2.971219in}}%
\pgfpathlineto{\pgfqpoint{0.940527in}{2.971219in}}%
\pgfpathlineto{\pgfqpoint{0.946922in}{2.971219in}}%
\pgfpathlineto{\pgfqpoint{0.953316in}{2.971219in}}%
\pgfpathlineto{\pgfqpoint{0.959710in}{2.971219in}}%
\pgfpathlineto{\pgfqpoint{0.966105in}{2.971219in}}%
\pgfpathlineto{\pgfqpoint{0.972499in}{2.971219in}}%
\pgfpathlineto{\pgfqpoint{0.978893in}{2.971219in}}%
\pgfpathlineto{\pgfqpoint{0.985287in}{2.971219in}}%
\pgfpathlineto{\pgfqpoint{0.991682in}{2.971219in}}%
\pgfpathlineto{\pgfqpoint{0.998076in}{2.971219in}}%
\pgfpathlineto{\pgfqpoint{1.004470in}{2.971219in}}%
\pgfpathlineto{\pgfqpoint{1.010864in}{2.971219in}}%
\pgfpathlineto{\pgfqpoint{1.017259in}{2.971219in}}%
\pgfpathlineto{\pgfqpoint{1.023653in}{2.971219in}}%
\pgfpathlineto{\pgfqpoint{1.030047in}{2.971219in}}%
\pgfpathlineto{\pgfqpoint{1.036442in}{2.971219in}}%
\pgfpathlineto{\pgfqpoint{1.042836in}{2.971219in}}%
\pgfpathlineto{\pgfqpoint{1.049230in}{2.971219in}}%
\pgfpathlineto{\pgfqpoint{1.055624in}{2.971219in}}%
\pgfpathlineto{\pgfqpoint{1.062019in}{2.971219in}}%
\pgfpathlineto{\pgfqpoint{1.068413in}{2.971219in}}%
\pgfpathlineto{\pgfqpoint{1.074807in}{2.971219in}}%
\pgfpathlineto{\pgfqpoint{1.081202in}{2.971219in}}%
\pgfpathlineto{\pgfqpoint{1.087596in}{2.971219in}}%
\pgfpathlineto{\pgfqpoint{1.093990in}{2.971219in}}%
\pgfpathlineto{\pgfqpoint{1.100384in}{2.971219in}}%
\pgfpathlineto{\pgfqpoint{1.106779in}{2.971219in}}%
\pgfpathlineto{\pgfqpoint{1.113173in}{2.971219in}}%
\pgfpathlineto{\pgfqpoint{1.119567in}{2.971219in}}%
\pgfpathlineto{\pgfqpoint{1.125962in}{2.971219in}}%
\pgfpathlineto{\pgfqpoint{1.132356in}{2.971219in}}%
\pgfpathlineto{\pgfqpoint{1.138750in}{2.971219in}}%
\pgfpathlineto{\pgfqpoint{1.145144in}{2.971219in}}%
\pgfpathlineto{\pgfqpoint{1.151539in}{2.971219in}}%
\pgfpathlineto{\pgfqpoint{1.157933in}{2.971219in}}%
\pgfpathlineto{\pgfqpoint{1.164327in}{2.971219in}}%
\pgfpathlineto{\pgfqpoint{1.170721in}{2.971219in}}%
\pgfpathlineto{\pgfqpoint{1.177116in}{2.971219in}}%
\pgfpathlineto{\pgfqpoint{1.183510in}{2.971219in}}%
\pgfpathlineto{\pgfqpoint{1.189904in}{2.971219in}}%
\pgfpathlineto{\pgfqpoint{1.196299in}{2.971219in}}%
\pgfpathlineto{\pgfqpoint{1.202693in}{2.971219in}}%
\pgfpathlineto{\pgfqpoint{1.209087in}{2.971219in}}%
\pgfpathlineto{\pgfqpoint{1.215481in}{2.971219in}}%
\pgfpathlineto{\pgfqpoint{1.221876in}{2.971219in}}%
\pgfpathlineto{\pgfqpoint{1.228270in}{2.971219in}}%
\pgfpathlineto{\pgfqpoint{1.234664in}{2.971219in}}%
\pgfpathlineto{\pgfqpoint{1.241059in}{2.971219in}}%
\pgfpathlineto{\pgfqpoint{1.247453in}{2.971219in}}%
\pgfpathlineto{\pgfqpoint{1.253847in}{2.971219in}}%
\pgfpathlineto{\pgfqpoint{1.260241in}{2.971219in}}%
\pgfpathlineto{\pgfqpoint{1.266636in}{2.971219in}}%
\pgfpathlineto{\pgfqpoint{1.273030in}{2.971219in}}%
\pgfpathlineto{\pgfqpoint{1.279424in}{2.971219in}}%
\pgfpathlineto{\pgfqpoint{1.285818in}{2.971219in}}%
\pgfpathlineto{\pgfqpoint{1.292213in}{2.971219in}}%
\pgfpathlineto{\pgfqpoint{1.298607in}{2.971219in}}%
\pgfpathlineto{\pgfqpoint{1.305001in}{2.971219in}}%
\pgfpathlineto{\pgfqpoint{1.311396in}{2.971219in}}%
\pgfpathlineto{\pgfqpoint{1.317790in}{2.971219in}}%
\pgfpathlineto{\pgfqpoint{1.324184in}{2.971219in}}%
\pgfpathlineto{\pgfqpoint{1.330578in}{2.971219in}}%
\pgfpathlineto{\pgfqpoint{1.336973in}{2.971219in}}%
\pgfpathlineto{\pgfqpoint{1.343367in}{2.971219in}}%
\pgfpathlineto{\pgfqpoint{1.349761in}{2.971219in}}%
\pgfpathlineto{\pgfqpoint{1.356156in}{2.971219in}}%
\pgfpathlineto{\pgfqpoint{1.362550in}{2.971219in}}%
\pgfpathlineto{\pgfqpoint{1.368944in}{2.971219in}}%
\pgfpathlineto{\pgfqpoint{1.375338in}{2.971219in}}%
\pgfpathlineto{\pgfqpoint{1.381733in}{2.971219in}}%
\pgfpathlineto{\pgfqpoint{1.388127in}{2.971219in}}%
\pgfpathlineto{\pgfqpoint{1.394521in}{2.971219in}}%
\pgfpathlineto{\pgfqpoint{1.400916in}{2.971219in}}%
\pgfpathlineto{\pgfqpoint{1.407310in}{2.971219in}}%
\pgfpathlineto{\pgfqpoint{1.413704in}{2.971219in}}%
\pgfpathlineto{\pgfqpoint{1.420098in}{2.971219in}}%
\pgfpathlineto{\pgfqpoint{1.426493in}{2.971219in}}%
\pgfpathlineto{\pgfqpoint{1.432887in}{2.971219in}}%
\pgfpathlineto{\pgfqpoint{1.439281in}{2.971219in}}%
\pgfpathlineto{\pgfqpoint{1.445675in}{2.971219in}}%
\pgfpathlineto{\pgfqpoint{1.452070in}{2.971219in}}%
\pgfpathlineto{\pgfqpoint{1.458464in}{2.971219in}}%
\pgfpathlineto{\pgfqpoint{1.464858in}{2.971219in}}%
\pgfpathlineto{\pgfqpoint{1.471253in}{2.971219in}}%
\pgfpathlineto{\pgfqpoint{1.477647in}{2.971219in}}%
\pgfpathlineto{\pgfqpoint{1.484041in}{2.971219in}}%
\pgfpathlineto{\pgfqpoint{1.490435in}{2.971219in}}%
\pgfpathlineto{\pgfqpoint{1.496830in}{2.971219in}}%
\pgfpathlineto{\pgfqpoint{1.503224in}{2.971219in}}%
\pgfpathlineto{\pgfqpoint{1.509618in}{2.971219in}}%
\pgfpathlineto{\pgfqpoint{1.516013in}{2.971219in}}%
\pgfpathlineto{\pgfqpoint{1.522407in}{2.971219in}}%
\pgfpathlineto{\pgfqpoint{1.528801in}{2.971219in}}%
\pgfpathlineto{\pgfqpoint{1.535195in}{2.971219in}}%
\pgfpathlineto{\pgfqpoint{1.541590in}{2.971219in}}%
\pgfpathlineto{\pgfqpoint{1.547984in}{2.971219in}}%
\pgfpathlineto{\pgfqpoint{1.554378in}{2.971219in}}%
\pgfpathlineto{\pgfqpoint{1.560773in}{2.971219in}}%
\pgfpathlineto{\pgfqpoint{1.567167in}{2.971219in}}%
\pgfpathlineto{\pgfqpoint{1.573561in}{2.971219in}}%
\pgfpathlineto{\pgfqpoint{1.579955in}{2.971219in}}%
\pgfpathlineto{\pgfqpoint{1.586350in}{2.971219in}}%
\pgfpathlineto{\pgfqpoint{1.592744in}{2.971219in}}%
\pgfpathlineto{\pgfqpoint{1.599138in}{2.971219in}}%
\pgfpathlineto{\pgfqpoint{1.605532in}{2.971219in}}%
\pgfpathlineto{\pgfqpoint{1.611927in}{2.971219in}}%
\pgfpathlineto{\pgfqpoint{1.618321in}{2.971219in}}%
\pgfpathlineto{\pgfqpoint{1.624715in}{2.971219in}}%
\pgfpathlineto{\pgfqpoint{1.631110in}{2.971219in}}%
\pgfpathlineto{\pgfqpoint{1.637504in}{2.971219in}}%
\pgfpathlineto{\pgfqpoint{1.643898in}{2.971219in}}%
\pgfpathlineto{\pgfqpoint{1.650292in}{2.971219in}}%
\pgfpathlineto{\pgfqpoint{1.656687in}{2.971219in}}%
\pgfpathlineto{\pgfqpoint{1.663081in}{2.971219in}}%
\pgfpathlineto{\pgfqpoint{1.669475in}{2.971219in}}%
\pgfpathlineto{\pgfqpoint{1.675870in}{2.971219in}}%
\pgfpathlineto{\pgfqpoint{1.682264in}{2.971219in}}%
\pgfpathlineto{\pgfqpoint{1.688658in}{2.971219in}}%
\pgfpathlineto{\pgfqpoint{1.695052in}{2.971219in}}%
\pgfpathlineto{\pgfqpoint{1.701447in}{2.971219in}}%
\pgfpathlineto{\pgfqpoint{1.707841in}{2.971219in}}%
\pgfpathlineto{\pgfqpoint{1.714235in}{2.971219in}}%
\pgfpathlineto{\pgfqpoint{1.720630in}{2.971219in}}%
\pgfpathlineto{\pgfqpoint{1.727024in}{2.971219in}}%
\pgfpathlineto{\pgfqpoint{1.733418in}{2.971219in}}%
\pgfpathlineto{\pgfqpoint{1.739812in}{2.971219in}}%
\pgfpathlineto{\pgfqpoint{1.746207in}{2.971219in}}%
\pgfpathlineto{\pgfqpoint{1.752601in}{2.971219in}}%
\pgfpathlineto{\pgfqpoint{1.758995in}{2.971219in}}%
\pgfpathlineto{\pgfqpoint{1.765389in}{2.971219in}}%
\pgfpathlineto{\pgfqpoint{1.771784in}{2.971219in}}%
\pgfpathlineto{\pgfqpoint{1.778178in}{2.971219in}}%
\pgfpathlineto{\pgfqpoint{1.784572in}{2.971219in}}%
\pgfpathlineto{\pgfqpoint{1.790967in}{2.971219in}}%
\pgfpathlineto{\pgfqpoint{1.797361in}{2.971219in}}%
\pgfpathlineto{\pgfqpoint{1.803755in}{2.971219in}}%
\pgfpathlineto{\pgfqpoint{1.810149in}{2.971219in}}%
\pgfpathlineto{\pgfqpoint{1.816544in}{2.971219in}}%
\pgfpathlineto{\pgfqpoint{1.822938in}{2.971219in}}%
\pgfpathlineto{\pgfqpoint{1.829332in}{2.971219in}}%
\pgfpathlineto{\pgfqpoint{1.835727in}{2.971219in}}%
\pgfpathlineto{\pgfqpoint{1.842121in}{2.971219in}}%
\pgfpathlineto{\pgfqpoint{1.848515in}{2.971219in}}%
\pgfpathlineto{\pgfqpoint{1.854909in}{2.971219in}}%
\pgfpathlineto{\pgfqpoint{1.861304in}{2.971219in}}%
\pgfpathlineto{\pgfqpoint{1.867698in}{2.971219in}}%
\pgfpathlineto{\pgfqpoint{1.874092in}{2.971219in}}%
\pgfpathlineto{\pgfqpoint{1.880486in}{2.971219in}}%
\pgfpathlineto{\pgfqpoint{1.886881in}{2.971219in}}%
\pgfpathlineto{\pgfqpoint{1.893275in}{2.971219in}}%
\pgfpathlineto{\pgfqpoint{1.899669in}{2.971219in}}%
\pgfpathlineto{\pgfqpoint{1.906064in}{2.971219in}}%
\pgfpathlineto{\pgfqpoint{1.912458in}{2.971219in}}%
\pgfpathlineto{\pgfqpoint{1.918852in}{2.971219in}}%
\pgfpathlineto{\pgfqpoint{1.925246in}{2.971219in}}%
\pgfpathlineto{\pgfqpoint{1.931641in}{2.971219in}}%
\pgfpathlineto{\pgfqpoint{1.938035in}{2.971219in}}%
\pgfpathlineto{\pgfqpoint{1.944429in}{2.971219in}}%
\pgfpathlineto{\pgfqpoint{1.950824in}{2.971219in}}%
\pgfpathlineto{\pgfqpoint{1.957218in}{2.971219in}}%
\pgfpathlineto{\pgfqpoint{1.963612in}{2.971219in}}%
\pgfpathlineto{\pgfqpoint{1.970006in}{2.971219in}}%
\pgfpathlineto{\pgfqpoint{1.976401in}{2.971219in}}%
\pgfpathlineto{\pgfqpoint{1.982795in}{2.971219in}}%
\pgfpathlineto{\pgfqpoint{1.989189in}{2.971219in}}%
\pgfpathlineto{\pgfqpoint{1.995584in}{2.971219in}}%
\pgfpathlineto{\pgfqpoint{2.001978in}{2.971219in}}%
\pgfpathlineto{\pgfqpoint{2.008372in}{2.971219in}}%
\pgfpathlineto{\pgfqpoint{2.014766in}{2.971219in}}%
\pgfpathlineto{\pgfqpoint{2.021161in}{2.971219in}}%
\pgfpathlineto{\pgfqpoint{2.027555in}{2.971219in}}%
\pgfpathlineto{\pgfqpoint{2.033949in}{2.971219in}}%
\pgfpathlineto{\pgfqpoint{2.040343in}{2.971219in}}%
\pgfpathlineto{\pgfqpoint{2.046738in}{2.971219in}}%
\pgfpathlineto{\pgfqpoint{2.053132in}{2.971219in}}%
\pgfpathlineto{\pgfqpoint{2.059526in}{2.971219in}}%
\pgfpathlineto{\pgfqpoint{2.065921in}{2.971219in}}%
\pgfpathlineto{\pgfqpoint{2.072315in}{2.971219in}}%
\pgfpathlineto{\pgfqpoint{2.078709in}{2.971219in}}%
\pgfpathlineto{\pgfqpoint{2.085103in}{2.971219in}}%
\pgfpathlineto{\pgfqpoint{2.091498in}{2.971219in}}%
\pgfpathlineto{\pgfqpoint{2.097892in}{2.971219in}}%
\pgfpathlineto{\pgfqpoint{2.104286in}{2.971219in}}%
\pgfpathlineto{\pgfqpoint{2.110681in}{2.971219in}}%
\pgfpathlineto{\pgfqpoint{2.117075in}{2.971219in}}%
\pgfpathlineto{\pgfqpoint{2.123469in}{2.971219in}}%
\pgfpathlineto{\pgfqpoint{2.129863in}{2.971219in}}%
\pgfpathlineto{\pgfqpoint{2.136258in}{2.971219in}}%
\pgfpathlineto{\pgfqpoint{2.142652in}{2.971219in}}%
\pgfpathlineto{\pgfqpoint{2.149046in}{2.971219in}}%
\pgfpathlineto{\pgfqpoint{2.155441in}{2.971219in}}%
\pgfpathlineto{\pgfqpoint{2.161835in}{2.971219in}}%
\pgfpathlineto{\pgfqpoint{2.168229in}{2.971219in}}%
\pgfpathlineto{\pgfqpoint{2.174623in}{2.971219in}}%
\pgfpathlineto{\pgfqpoint{2.181018in}{2.971219in}}%
\pgfpathlineto{\pgfqpoint{2.187412in}{2.971219in}}%
\pgfpathlineto{\pgfqpoint{2.193806in}{2.971219in}}%
\pgfpathlineto{\pgfqpoint{2.200200in}{2.971219in}}%
\pgfpathlineto{\pgfqpoint{2.206595in}{2.971219in}}%
\pgfpathlineto{\pgfqpoint{2.212989in}{2.971219in}}%
\pgfpathlineto{\pgfqpoint{2.219383in}{2.971219in}}%
\pgfpathlineto{\pgfqpoint{2.225778in}{2.971219in}}%
\pgfpathlineto{\pgfqpoint{2.232172in}{2.971219in}}%
\pgfpathlineto{\pgfqpoint{2.238566in}{2.971219in}}%
\pgfpathlineto{\pgfqpoint{2.244960in}{2.971219in}}%
\pgfpathlineto{\pgfqpoint{2.251355in}{2.971219in}}%
\pgfpathlineto{\pgfqpoint{2.257749in}{2.971219in}}%
\pgfpathlineto{\pgfqpoint{2.264143in}{2.971219in}}%
\pgfpathlineto{\pgfqpoint{2.270538in}{2.971219in}}%
\pgfpathlineto{\pgfqpoint{2.276932in}{2.971219in}}%
\pgfpathlineto{\pgfqpoint{2.283326in}{2.971219in}}%
\pgfpathlineto{\pgfqpoint{2.289720in}{2.971219in}}%
\pgfpathlineto{\pgfqpoint{2.296115in}{2.971219in}}%
\pgfpathlineto{\pgfqpoint{2.302509in}{2.971219in}}%
\pgfpathlineto{\pgfqpoint{2.308903in}{2.971219in}}%
\pgfpathlineto{\pgfqpoint{2.315298in}{2.971219in}}%
\pgfpathlineto{\pgfqpoint{2.321692in}{2.971219in}}%
\pgfpathlineto{\pgfqpoint{2.328086in}{2.971219in}}%
\pgfpathlineto{\pgfqpoint{2.334480in}{2.971219in}}%
\pgfpathlineto{\pgfqpoint{2.340875in}{2.971219in}}%
\pgfpathlineto{\pgfqpoint{2.347269in}{2.971219in}}%
\pgfpathlineto{\pgfqpoint{2.353663in}{2.971219in}}%
\pgfpathlineto{\pgfqpoint{2.360057in}{2.971219in}}%
\pgfpathlineto{\pgfqpoint{2.366452in}{2.971219in}}%
\pgfpathlineto{\pgfqpoint{2.372846in}{2.971219in}}%
\pgfpathlineto{\pgfqpoint{2.379240in}{2.971219in}}%
\pgfpathlineto{\pgfqpoint{2.385635in}{2.971219in}}%
\pgfpathlineto{\pgfqpoint{2.392029in}{2.971219in}}%
\pgfpathlineto{\pgfqpoint{2.398423in}{2.971219in}}%
\pgfpathlineto{\pgfqpoint{2.404817in}{2.971219in}}%
\pgfpathlineto{\pgfqpoint{2.411212in}{2.971219in}}%
\pgfpathlineto{\pgfqpoint{2.417606in}{2.971219in}}%
\pgfpathlineto{\pgfqpoint{2.424000in}{2.971219in}}%
\pgfpathlineto{\pgfqpoint{2.430395in}{2.971219in}}%
\pgfpathlineto{\pgfqpoint{2.436789in}{2.971219in}}%
\pgfpathlineto{\pgfqpoint{2.443183in}{2.971219in}}%
\pgfpathlineto{\pgfqpoint{2.449577in}{2.971219in}}%
\pgfpathlineto{\pgfqpoint{2.455972in}{2.971219in}}%
\pgfpathlineto{\pgfqpoint{2.462366in}{2.971219in}}%
\pgfpathlineto{\pgfqpoint{2.468760in}{2.971219in}}%
\pgfpathlineto{\pgfqpoint{2.475154in}{2.971219in}}%
\pgfpathlineto{\pgfqpoint{2.481549in}{2.971219in}}%
\pgfpathlineto{\pgfqpoint{2.487943in}{2.971219in}}%
\pgfpathlineto{\pgfqpoint{2.494337in}{2.971219in}}%
\pgfpathlineto{\pgfqpoint{2.500732in}{2.971219in}}%
\pgfpathlineto{\pgfqpoint{2.507126in}{2.971219in}}%
\pgfpathlineto{\pgfqpoint{2.513520in}{2.971219in}}%
\pgfpathlineto{\pgfqpoint{2.519914in}{2.971219in}}%
\pgfpathlineto{\pgfqpoint{2.526309in}{2.971219in}}%
\pgfpathlineto{\pgfqpoint{2.532703in}{2.971219in}}%
\pgfpathlineto{\pgfqpoint{2.539097in}{2.971219in}}%
\pgfpathlineto{\pgfqpoint{2.545492in}{2.971219in}}%
\pgfpathlineto{\pgfqpoint{2.551886in}{2.971219in}}%
\pgfpathlineto{\pgfqpoint{2.558280in}{2.971219in}}%
\pgfpathlineto{\pgfqpoint{2.564674in}{2.971219in}}%
\pgfpathlineto{\pgfqpoint{2.571069in}{2.971219in}}%
\pgfpathlineto{\pgfqpoint{2.577463in}{2.971219in}}%
\pgfpathlineto{\pgfqpoint{2.583857in}{2.971219in}}%
\pgfpathlineto{\pgfqpoint{2.590252in}{2.971219in}}%
\pgfpathlineto{\pgfqpoint{2.596646in}{2.971219in}}%
\pgfpathlineto{\pgfqpoint{2.603040in}{2.971219in}}%
\pgfpathlineto{\pgfqpoint{2.609434in}{2.971219in}}%
\pgfpathlineto{\pgfqpoint{2.615829in}{2.971219in}}%
\pgfpathlineto{\pgfqpoint{2.622223in}{2.971219in}}%
\pgfpathlineto{\pgfqpoint{2.628617in}{2.971219in}}%
\pgfpathlineto{\pgfqpoint{2.635011in}{2.971219in}}%
\pgfpathlineto{\pgfqpoint{2.641406in}{2.971219in}}%
\pgfpathlineto{\pgfqpoint{2.647800in}{2.971219in}}%
\pgfpathlineto{\pgfqpoint{2.654194in}{2.971219in}}%
\pgfpathlineto{\pgfqpoint{2.660589in}{2.971219in}}%
\pgfpathlineto{\pgfqpoint{2.666983in}{2.971219in}}%
\pgfpathlineto{\pgfqpoint{2.673377in}{2.971219in}}%
\pgfpathlineto{\pgfqpoint{2.679771in}{2.971219in}}%
\pgfpathlineto{\pgfqpoint{2.686166in}{2.971219in}}%
\pgfpathlineto{\pgfqpoint{2.692560in}{2.971219in}}%
\pgfpathlineto{\pgfqpoint{2.698954in}{2.971219in}}%
\pgfpathlineto{\pgfqpoint{2.705349in}{2.971219in}}%
\pgfpathlineto{\pgfqpoint{2.711743in}{2.971219in}}%
\pgfpathlineto{\pgfqpoint{2.718137in}{2.971219in}}%
\pgfpathlineto{\pgfqpoint{2.724531in}{2.971219in}}%
\pgfpathlineto{\pgfqpoint{2.730926in}{2.971219in}}%
\pgfpathlineto{\pgfqpoint{2.737320in}{2.971219in}}%
\pgfpathlineto{\pgfqpoint{2.743714in}{2.971219in}}%
\pgfpathlineto{\pgfqpoint{2.750109in}{2.971219in}}%
\pgfpathlineto{\pgfqpoint{2.756503in}{2.971219in}}%
\pgfpathlineto{\pgfqpoint{2.762897in}{2.971219in}}%
\pgfpathlineto{\pgfqpoint{2.769291in}{2.971219in}}%
\pgfpathlineto{\pgfqpoint{2.775686in}{2.971219in}}%
\pgfpathlineto{\pgfqpoint{2.782080in}{2.971219in}}%
\pgfpathlineto{\pgfqpoint{2.788474in}{2.971219in}}%
\pgfpathlineto{\pgfqpoint{2.794868in}{2.971219in}}%
\pgfpathlineto{\pgfqpoint{2.801263in}{2.971219in}}%
\pgfpathlineto{\pgfqpoint{2.807657in}{2.971219in}}%
\pgfpathlineto{\pgfqpoint{2.814051in}{2.971219in}}%
\pgfpathlineto{\pgfqpoint{2.820446in}{2.971219in}}%
\pgfpathlineto{\pgfqpoint{2.826840in}{2.971219in}}%
\pgfpathlineto{\pgfqpoint{2.833234in}{2.971219in}}%
\pgfpathlineto{\pgfqpoint{2.839628in}{2.971219in}}%
\pgfpathlineto{\pgfqpoint{2.846023in}{2.971219in}}%
\pgfpathlineto{\pgfqpoint{2.852417in}{2.971219in}}%
\pgfpathlineto{\pgfqpoint{2.858811in}{2.971219in}}%
\pgfpathlineto{\pgfqpoint{2.865206in}{2.971219in}}%
\pgfpathlineto{\pgfqpoint{2.871600in}{2.971219in}}%
\pgfpathlineto{\pgfqpoint{2.877994in}{2.971219in}}%
\pgfpathlineto{\pgfqpoint{2.884388in}{2.971219in}}%
\pgfpathlineto{\pgfqpoint{2.890783in}{2.971219in}}%
\pgfpathlineto{\pgfqpoint{2.897177in}{2.971219in}}%
\pgfpathlineto{\pgfqpoint{2.903571in}{2.971219in}}%
\pgfpathlineto{\pgfqpoint{2.909966in}{2.971219in}}%
\pgfpathlineto{\pgfqpoint{2.916360in}{2.971219in}}%
\pgfpathlineto{\pgfqpoint{2.922754in}{2.971219in}}%
\pgfpathlineto{\pgfqpoint{2.929148in}{2.971219in}}%
\pgfpathlineto{\pgfqpoint{2.935543in}{2.971219in}}%
\pgfpathlineto{\pgfqpoint{2.941937in}{2.971219in}}%
\pgfpathlineto{\pgfqpoint{2.948331in}{2.971219in}}%
\pgfpathlineto{\pgfqpoint{2.954725in}{2.971219in}}%
\pgfpathlineto{\pgfqpoint{2.961120in}{2.971219in}}%
\pgfpathlineto{\pgfqpoint{2.967514in}{2.971219in}}%
\pgfpathlineto{\pgfqpoint{2.973908in}{2.971219in}}%
\pgfpathlineto{\pgfqpoint{2.980303in}{2.971219in}}%
\pgfpathlineto{\pgfqpoint{2.986697in}{2.971219in}}%
\pgfpathlineto{\pgfqpoint{2.993091in}{2.971219in}}%
\pgfpathlineto{\pgfqpoint{2.999485in}{2.971219in}}%
\pgfpathlineto{\pgfqpoint{3.005880in}{2.971219in}}%
\pgfpathlineto{\pgfqpoint{3.012274in}{2.971219in}}%
\pgfpathlineto{\pgfqpoint{3.018668in}{2.971219in}}%
\pgfpathlineto{\pgfqpoint{3.025063in}{2.971219in}}%
\pgfpathlineto{\pgfqpoint{3.031457in}{2.971219in}}%
\pgfpathlineto{\pgfqpoint{3.037851in}{2.971219in}}%
\pgfpathlineto{\pgfqpoint{3.044245in}{2.971219in}}%
\pgfpathlineto{\pgfqpoint{3.050640in}{2.971219in}}%
\pgfpathlineto{\pgfqpoint{3.057034in}{2.971219in}}%
\pgfpathlineto{\pgfqpoint{3.063428in}{2.971219in}}%
\pgfpathlineto{\pgfqpoint{3.069822in}{2.971219in}}%
\pgfpathlineto{\pgfqpoint{3.076217in}{2.971219in}}%
\pgfpathlineto{\pgfqpoint{3.082611in}{2.971219in}}%
\pgfpathlineto{\pgfqpoint{3.089005in}{2.971219in}}%
\pgfpathlineto{\pgfqpoint{3.095400in}{2.971219in}}%
\pgfpathlineto{\pgfqpoint{3.101794in}{2.971219in}}%
\pgfpathlineto{\pgfqpoint{3.108188in}{2.971219in}}%
\pgfpathlineto{\pgfqpoint{3.114582in}{2.971219in}}%
\pgfpathlineto{\pgfqpoint{3.120977in}{2.971219in}}%
\pgfpathlineto{\pgfqpoint{3.127371in}{2.971219in}}%
\pgfpathlineto{\pgfqpoint{3.133765in}{2.971219in}}%
\pgfpathlineto{\pgfqpoint{3.140160in}{2.971219in}}%
\pgfpathlineto{\pgfqpoint{3.146554in}{2.971219in}}%
\pgfpathlineto{\pgfqpoint{3.152948in}{2.971219in}}%
\pgfpathlineto{\pgfqpoint{3.159342in}{2.971219in}}%
\pgfpathlineto{\pgfqpoint{3.165737in}{2.971219in}}%
\pgfpathlineto{\pgfqpoint{3.172131in}{2.971219in}}%
\pgfpathlineto{\pgfqpoint{3.178525in}{2.971219in}}%
\pgfpathlineto{\pgfqpoint{3.184920in}{2.971219in}}%
\pgfpathlineto{\pgfqpoint{3.191314in}{2.971219in}}%
\pgfpathlineto{\pgfqpoint{3.197708in}{2.971219in}}%
\pgfpathlineto{\pgfqpoint{3.204102in}{2.971219in}}%
\pgfpathlineto{\pgfqpoint{3.210497in}{2.971219in}}%
\pgfpathlineto{\pgfqpoint{3.216891in}{2.971219in}}%
\pgfpathlineto{\pgfqpoint{3.223285in}{2.971219in}}%
\pgfpathlineto{\pgfqpoint{3.229679in}{2.971219in}}%
\pgfpathlineto{\pgfqpoint{3.236074in}{2.971219in}}%
\pgfpathlineto{\pgfqpoint{3.242468in}{2.971219in}}%
\pgfpathlineto{\pgfqpoint{3.248862in}{2.971219in}}%
\pgfpathlineto{\pgfqpoint{3.255257in}{2.971219in}}%
\pgfpathlineto{\pgfqpoint{3.261651in}{2.971219in}}%
\pgfpathlineto{\pgfqpoint{3.268045in}{2.971219in}}%
\pgfpathlineto{\pgfqpoint{3.274439in}{2.971219in}}%
\pgfpathlineto{\pgfqpoint{3.280834in}{2.971219in}}%
\pgfpathlineto{\pgfqpoint{3.287228in}{2.971219in}}%
\pgfpathlineto{\pgfqpoint{3.293622in}{2.971219in}}%
\pgfpathlineto{\pgfqpoint{3.300017in}{2.971219in}}%
\pgfpathlineto{\pgfqpoint{3.306411in}{2.971219in}}%
\pgfpathlineto{\pgfqpoint{3.312805in}{2.971219in}}%
\pgfpathlineto{\pgfqpoint{3.319199in}{2.971219in}}%
\pgfpathlineto{\pgfqpoint{3.325594in}{2.971219in}}%
\pgfpathlineto{\pgfqpoint{3.331988in}{2.971219in}}%
\pgfpathlineto{\pgfqpoint{3.338382in}{2.971219in}}%
\pgfpathlineto{\pgfqpoint{3.344777in}{2.971219in}}%
\pgfpathlineto{\pgfqpoint{3.351171in}{2.971219in}}%
\pgfpathlineto{\pgfqpoint{3.357565in}{2.971219in}}%
\pgfpathlineto{\pgfqpoint{3.363959in}{2.971219in}}%
\pgfpathlineto{\pgfqpoint{3.370354in}{2.971219in}}%
\pgfpathlineto{\pgfqpoint{3.376748in}{2.971219in}}%
\pgfpathlineto{\pgfqpoint{3.383142in}{2.971219in}}%
\pgfpathlineto{\pgfqpoint{3.389536in}{2.971219in}}%
\pgfpathlineto{\pgfqpoint{3.395931in}{2.971219in}}%
\pgfpathlineto{\pgfqpoint{3.402325in}{2.971219in}}%
\pgfpathlineto{\pgfqpoint{3.408719in}{2.971219in}}%
\pgfpathlineto{\pgfqpoint{3.415114in}{2.971219in}}%
\pgfpathlineto{\pgfqpoint{3.421508in}{2.971219in}}%
\pgfpathlineto{\pgfqpoint{3.427902in}{2.971219in}}%
\pgfpathlineto{\pgfqpoint{3.434296in}{2.971219in}}%
\pgfpathlineto{\pgfqpoint{3.440691in}{2.971219in}}%
\pgfpathlineto{\pgfqpoint{3.447085in}{2.971219in}}%
\pgfpathlineto{\pgfqpoint{3.453479in}{2.971219in}}%
\pgfpathlineto{\pgfqpoint{3.459874in}{2.971219in}}%
\pgfpathlineto{\pgfqpoint{3.466268in}{2.971219in}}%
\pgfpathlineto{\pgfqpoint{3.472662in}{2.971219in}}%
\pgfpathlineto{\pgfqpoint{3.479056in}{2.971219in}}%
\pgfpathlineto{\pgfqpoint{3.485451in}{2.971219in}}%
\pgfpathlineto{\pgfqpoint{3.491845in}{2.971219in}}%
\pgfpathlineto{\pgfqpoint{3.498239in}{2.971219in}}%
\pgfpathlineto{\pgfqpoint{3.504634in}{2.971219in}}%
\pgfpathlineto{\pgfqpoint{3.511028in}{2.971219in}}%
\pgfpathlineto{\pgfqpoint{3.517422in}{2.971219in}}%
\pgfpathlineto{\pgfqpoint{3.523816in}{2.971219in}}%
\pgfpathlineto{\pgfqpoint{3.530211in}{2.971219in}}%
\pgfpathlineto{\pgfqpoint{3.536605in}{2.971219in}}%
\pgfpathlineto{\pgfqpoint{3.542999in}{2.971219in}}%
\pgfpathlineto{\pgfqpoint{3.549393in}{2.971219in}}%
\pgfpathlineto{\pgfqpoint{3.555788in}{2.971219in}}%
\pgfpathlineto{\pgfqpoint{3.562182in}{2.971219in}}%
\pgfpathlineto{\pgfqpoint{3.568576in}{2.971219in}}%
\pgfpathlineto{\pgfqpoint{3.574971in}{2.971219in}}%
\pgfpathlineto{\pgfqpoint{3.581365in}{2.971219in}}%
\pgfpathlineto{\pgfqpoint{3.587759in}{2.971219in}}%
\pgfpathlineto{\pgfqpoint{3.594153in}{2.971219in}}%
\pgfpathlineto{\pgfqpoint{3.600548in}{2.971219in}}%
\pgfpathlineto{\pgfqpoint{3.606942in}{2.971219in}}%
\pgfpathlineto{\pgfqpoint{3.613336in}{2.971219in}}%
\pgfpathlineto{\pgfqpoint{3.619731in}{2.971219in}}%
\pgfpathlineto{\pgfqpoint{3.626125in}{2.971219in}}%
\pgfpathlineto{\pgfqpoint{3.632519in}{2.971219in}}%
\pgfpathlineto{\pgfqpoint{3.638913in}{2.971219in}}%
\pgfpathlineto{\pgfqpoint{3.645308in}{2.971219in}}%
\pgfpathlineto{\pgfqpoint{3.651702in}{2.971219in}}%
\pgfpathlineto{\pgfqpoint{3.658096in}{2.971219in}}%
\pgfpathlineto{\pgfqpoint{3.664490in}{2.971219in}}%
\pgfpathlineto{\pgfqpoint{3.670885in}{2.971219in}}%
\pgfpathlineto{\pgfqpoint{3.677279in}{2.971219in}}%
\pgfpathlineto{\pgfqpoint{3.683673in}{2.971219in}}%
\pgfpathlineto{\pgfqpoint{3.690068in}{2.971219in}}%
\pgfpathlineto{\pgfqpoint{3.696462in}{2.971219in}}%
\pgfpathlineto{\pgfqpoint{3.702856in}{2.971219in}}%
\pgfpathlineto{\pgfqpoint{3.709250in}{2.971219in}}%
\pgfpathlineto{\pgfqpoint{3.715645in}{2.971219in}}%
\pgfpathlineto{\pgfqpoint{3.722039in}{2.971219in}}%
\pgfpathlineto{\pgfqpoint{3.728433in}{2.971219in}}%
\pgfpathlineto{\pgfqpoint{3.734828in}{2.971219in}}%
\pgfpathlineto{\pgfqpoint{3.741222in}{2.971219in}}%
\pgfpathlineto{\pgfqpoint{3.747616in}{2.971219in}}%
\pgfpathlineto{\pgfqpoint{3.754010in}{2.971219in}}%
\pgfpathlineto{\pgfqpoint{3.760405in}{2.971219in}}%
\pgfpathlineto{\pgfqpoint{3.766799in}{2.971219in}}%
\pgfpathlineto{\pgfqpoint{3.773193in}{2.971219in}}%
\pgfpathlineto{\pgfqpoint{3.779588in}{2.971219in}}%
\pgfpathlineto{\pgfqpoint{3.785982in}{2.971219in}}%
\pgfpathlineto{\pgfqpoint{3.792376in}{2.971219in}}%
\pgfpathlineto{\pgfqpoint{3.798770in}{2.971219in}}%
\pgfpathlineto{\pgfqpoint{3.798770in}{2.973639in}}%
\pgfpathlineto{\pgfqpoint{3.798770in}{2.973639in}}%
\pgfpathlineto{\pgfqpoint{3.792376in}{2.978478in}}%
\pgfpathlineto{\pgfqpoint{3.785982in}{2.983317in}}%
\pgfpathlineto{\pgfqpoint{3.779588in}{2.988156in}}%
\pgfpathlineto{\pgfqpoint{3.773193in}{2.992996in}}%
\pgfpathlineto{\pgfqpoint{3.766799in}{2.997835in}}%
\pgfpathlineto{\pgfqpoint{3.760405in}{3.002674in}}%
\pgfpathlineto{\pgfqpoint{3.754010in}{3.007513in}}%
\pgfpathlineto{\pgfqpoint{3.747616in}{3.012353in}}%
\pgfpathlineto{\pgfqpoint{3.741222in}{3.017192in}}%
\pgfpathlineto{\pgfqpoint{3.734828in}{3.022031in}}%
\pgfpathlineto{\pgfqpoint{3.728433in}{3.026870in}}%
\pgfpathlineto{\pgfqpoint{3.722039in}{3.031710in}}%
\pgfpathlineto{\pgfqpoint{3.715645in}{3.036549in}}%
\pgfpathlineto{\pgfqpoint{3.709250in}{3.041388in}}%
\pgfpathlineto{\pgfqpoint{3.702856in}{3.046227in}}%
\pgfpathlineto{\pgfqpoint{3.696462in}{3.051067in}}%
\pgfpathlineto{\pgfqpoint{3.690068in}{3.055906in}}%
\pgfpathlineto{\pgfqpoint{3.683673in}{3.060745in}}%
\pgfpathlineto{\pgfqpoint{3.677279in}{3.065584in}}%
\pgfpathlineto{\pgfqpoint{3.670885in}{3.070424in}}%
\pgfpathlineto{\pgfqpoint{3.664490in}{3.075263in}}%
\pgfpathlineto{\pgfqpoint{3.658096in}{3.080102in}}%
\pgfpathlineto{\pgfqpoint{3.651702in}{3.084941in}}%
\pgfpathlineto{\pgfqpoint{3.645308in}{3.089781in}}%
\pgfpathlineto{\pgfqpoint{3.638913in}{3.094620in}}%
\pgfpathlineto{\pgfqpoint{3.632519in}{3.099459in}}%
\pgfpathlineto{\pgfqpoint{3.626125in}{3.104298in}}%
\pgfpathlineto{\pgfqpoint{3.619731in}{3.109138in}}%
\pgfpathlineto{\pgfqpoint{3.613336in}{3.113977in}}%
\pgfpathlineto{\pgfqpoint{3.606942in}{3.118816in}}%
\pgfpathlineto{\pgfqpoint{3.600548in}{3.123655in}}%
\pgfpathlineto{\pgfqpoint{3.594153in}{3.128495in}}%
\pgfpathlineto{\pgfqpoint{3.587759in}{3.133334in}}%
\pgfpathlineto{\pgfqpoint{3.581365in}{3.138173in}}%
\pgfpathlineto{\pgfqpoint{3.574971in}{3.143012in}}%
\pgfpathlineto{\pgfqpoint{3.568576in}{3.147852in}}%
\pgfpathlineto{\pgfqpoint{3.562182in}{3.152691in}}%
\pgfpathlineto{\pgfqpoint{3.555788in}{3.157530in}}%
\pgfpathlineto{\pgfqpoint{3.549393in}{3.162369in}}%
\pgfpathlineto{\pgfqpoint{3.542999in}{3.167209in}}%
\pgfpathlineto{\pgfqpoint{3.536605in}{3.172048in}}%
\pgfpathlineto{\pgfqpoint{3.530211in}{3.176887in}}%
\pgfpathlineto{\pgfqpoint{3.523816in}{3.181726in}}%
\pgfpathlineto{\pgfqpoint{3.517422in}{3.186566in}}%
\pgfpathlineto{\pgfqpoint{3.511028in}{3.191405in}}%
\pgfpathlineto{\pgfqpoint{3.504634in}{3.196244in}}%
\pgfpathlineto{\pgfqpoint{3.498239in}{3.201084in}}%
\pgfpathlineto{\pgfqpoint{3.491845in}{3.205923in}}%
\pgfpathlineto{\pgfqpoint{3.485451in}{3.210762in}}%
\pgfpathlineto{\pgfqpoint{3.479056in}{3.215601in}}%
\pgfpathlineto{\pgfqpoint{3.472662in}{3.220441in}}%
\pgfpathlineto{\pgfqpoint{3.466268in}{3.225280in}}%
\pgfpathlineto{\pgfqpoint{3.459874in}{3.230119in}}%
\pgfpathlineto{\pgfqpoint{3.453479in}{3.234958in}}%
\pgfpathlineto{\pgfqpoint{3.447085in}{3.239798in}}%
\pgfpathlineto{\pgfqpoint{3.440691in}{3.244637in}}%
\pgfpathlineto{\pgfqpoint{3.434296in}{3.249476in}}%
\pgfpathlineto{\pgfqpoint{3.427902in}{3.254315in}}%
\pgfpathlineto{\pgfqpoint{3.421508in}{3.259155in}}%
\pgfpathlineto{\pgfqpoint{3.415114in}{3.263994in}}%
\pgfpathlineto{\pgfqpoint{3.408719in}{3.268833in}}%
\pgfpathlineto{\pgfqpoint{3.402325in}{3.273672in}}%
\pgfpathlineto{\pgfqpoint{3.395931in}{3.278512in}}%
\pgfpathlineto{\pgfqpoint{3.389536in}{3.283351in}}%
\pgfpathlineto{\pgfqpoint{3.383142in}{3.288190in}}%
\pgfpathlineto{\pgfqpoint{3.376748in}{3.293029in}}%
\pgfpathlineto{\pgfqpoint{3.370354in}{3.297869in}}%
\pgfpathlineto{\pgfqpoint{3.363959in}{3.302708in}}%
\pgfpathlineto{\pgfqpoint{3.357565in}{3.307547in}}%
\pgfpathlineto{\pgfqpoint{3.351171in}{3.312386in}}%
\pgfpathlineto{\pgfqpoint{3.344777in}{3.317226in}}%
\pgfpathlineto{\pgfqpoint{3.338382in}{3.322065in}}%
\pgfpathlineto{\pgfqpoint{3.331988in}{3.326904in}}%
\pgfpathlineto{\pgfqpoint{3.325594in}{3.331743in}}%
\pgfpathlineto{\pgfqpoint{3.319199in}{3.336583in}}%
\pgfpathlineto{\pgfqpoint{3.312805in}{3.341422in}}%
\pgfpathlineto{\pgfqpoint{3.306411in}{3.346261in}}%
\pgfpathlineto{\pgfqpoint{3.300017in}{3.351100in}}%
\pgfpathlineto{\pgfqpoint{3.293622in}{3.355940in}}%
\pgfpathlineto{\pgfqpoint{3.287228in}{3.360779in}}%
\pgfpathlineto{\pgfqpoint{3.280834in}{3.365618in}}%
\pgfpathlineto{\pgfqpoint{3.274439in}{3.370457in}}%
\pgfpathlineto{\pgfqpoint{3.268045in}{3.375297in}}%
\pgfpathlineto{\pgfqpoint{3.261651in}{3.380136in}}%
\pgfpathlineto{\pgfqpoint{3.255257in}{3.384975in}}%
\pgfpathlineto{\pgfqpoint{3.248862in}{3.389814in}}%
\pgfpathlineto{\pgfqpoint{3.242468in}{3.394654in}}%
\pgfpathlineto{\pgfqpoint{3.236074in}{3.399493in}}%
\pgfpathlineto{\pgfqpoint{3.229679in}{3.404332in}}%
\pgfpathlineto{\pgfqpoint{3.223285in}{3.409171in}}%
\pgfpathlineto{\pgfqpoint{3.216891in}{3.414011in}}%
\pgfpathlineto{\pgfqpoint{3.210497in}{3.418850in}}%
\pgfpathlineto{\pgfqpoint{3.204102in}{3.423689in}}%
\pgfpathlineto{\pgfqpoint{3.197708in}{3.428528in}}%
\pgfpathlineto{\pgfqpoint{3.191314in}{3.433368in}}%
\pgfpathlineto{\pgfqpoint{3.184920in}{3.438207in}}%
\pgfpathlineto{\pgfqpoint{3.178525in}{3.443046in}}%
\pgfpathlineto{\pgfqpoint{3.172131in}{3.447885in}}%
\pgfpathlineto{\pgfqpoint{3.165737in}{3.452725in}}%
\pgfpathlineto{\pgfqpoint{3.159342in}{3.457564in}}%
\pgfpathlineto{\pgfqpoint{3.152948in}{3.462403in}}%
\pgfpathlineto{\pgfqpoint{3.146554in}{3.467242in}}%
\pgfpathlineto{\pgfqpoint{3.140160in}{3.472082in}}%
\pgfpathlineto{\pgfqpoint{3.133765in}{3.476921in}}%
\pgfpathlineto{\pgfqpoint{3.127371in}{3.481760in}}%
\pgfpathlineto{\pgfqpoint{3.120977in}{3.486599in}}%
\pgfpathlineto{\pgfqpoint{3.114582in}{3.491439in}}%
\pgfpathlineto{\pgfqpoint{3.108188in}{3.496278in}}%
\pgfpathlineto{\pgfqpoint{3.101794in}{3.501117in}}%
\pgfpathlineto{\pgfqpoint{3.095400in}{3.505956in}}%
\pgfpathlineto{\pgfqpoint{3.089005in}{3.510796in}}%
\pgfpathlineto{\pgfqpoint{3.082611in}{3.515635in}}%
\pgfpathlineto{\pgfqpoint{3.076217in}{3.520474in}}%
\pgfpathlineto{\pgfqpoint{3.069822in}{3.525313in}}%
\pgfpathlineto{\pgfqpoint{3.063428in}{3.530153in}}%
\pgfpathlineto{\pgfqpoint{3.057034in}{3.534992in}}%
\pgfpathlineto{\pgfqpoint{3.050640in}{3.539831in}}%
\pgfpathlineto{\pgfqpoint{3.044245in}{3.544671in}}%
\pgfpathlineto{\pgfqpoint{3.037851in}{3.549510in}}%
\pgfpathlineto{\pgfqpoint{3.031457in}{3.554349in}}%
\pgfpathlineto{\pgfqpoint{3.025063in}{3.559188in}}%
\pgfpathlineto{\pgfqpoint{3.018668in}{3.564028in}}%
\pgfpathlineto{\pgfqpoint{3.012274in}{3.568867in}}%
\pgfpathlineto{\pgfqpoint{3.005880in}{3.573706in}}%
\pgfpathlineto{\pgfqpoint{2.999485in}{3.578545in}}%
\pgfpathlineto{\pgfqpoint{2.993091in}{3.583385in}}%
\pgfpathlineto{\pgfqpoint{2.986697in}{3.588224in}}%
\pgfpathlineto{\pgfqpoint{2.980303in}{3.593063in}}%
\pgfpathlineto{\pgfqpoint{2.973908in}{3.597902in}}%
\pgfpathlineto{\pgfqpoint{2.967514in}{3.602742in}}%
\pgfpathlineto{\pgfqpoint{2.961120in}{3.607581in}}%
\pgfpathlineto{\pgfqpoint{2.954725in}{3.612420in}}%
\pgfpathlineto{\pgfqpoint{2.948331in}{3.617259in}}%
\pgfpathlineto{\pgfqpoint{2.941937in}{3.622099in}}%
\pgfpathlineto{\pgfqpoint{2.935543in}{3.626938in}}%
\pgfpathlineto{\pgfqpoint{2.929148in}{3.631777in}}%
\pgfpathlineto{\pgfqpoint{2.922754in}{3.636616in}}%
\pgfpathlineto{\pgfqpoint{2.916360in}{3.641456in}}%
\pgfpathlineto{\pgfqpoint{2.909966in}{3.646295in}}%
\pgfpathlineto{\pgfqpoint{2.903571in}{3.651134in}}%
\pgfpathlineto{\pgfqpoint{2.897177in}{3.655973in}}%
\pgfpathlineto{\pgfqpoint{2.890783in}{3.660813in}}%
\pgfpathlineto{\pgfqpoint{2.884388in}{3.665652in}}%
\pgfpathlineto{\pgfqpoint{2.877994in}{3.670491in}}%
\pgfpathlineto{\pgfqpoint{2.871600in}{3.675330in}}%
\pgfpathlineto{\pgfqpoint{2.865206in}{3.680170in}}%
\pgfpathlineto{\pgfqpoint{2.858811in}{3.685009in}}%
\pgfpathlineto{\pgfqpoint{2.852417in}{3.689848in}}%
\pgfpathlineto{\pgfqpoint{2.846023in}{3.694687in}}%
\pgfpathlineto{\pgfqpoint{2.839628in}{3.699527in}}%
\pgfpathlineto{\pgfqpoint{2.833234in}{3.704366in}}%
\pgfpathlineto{\pgfqpoint{2.826840in}{3.709205in}}%
\pgfpathlineto{\pgfqpoint{2.820446in}{3.714044in}}%
\pgfpathlineto{\pgfqpoint{2.814051in}{3.718884in}}%
\pgfpathlineto{\pgfqpoint{2.807657in}{3.723723in}}%
\pgfpathlineto{\pgfqpoint{2.801263in}{3.728562in}}%
\pgfpathlineto{\pgfqpoint{2.794868in}{3.733401in}}%
\pgfpathlineto{\pgfqpoint{2.788474in}{3.738241in}}%
\pgfpathlineto{\pgfqpoint{2.782080in}{3.743080in}}%
\pgfpathlineto{\pgfqpoint{2.775686in}{3.747919in}}%
\pgfpathlineto{\pgfqpoint{2.769291in}{3.752758in}}%
\pgfpathlineto{\pgfqpoint{2.762897in}{3.757598in}}%
\pgfpathlineto{\pgfqpoint{2.756503in}{3.762437in}}%
\pgfpathlineto{\pgfqpoint{2.750109in}{3.767276in}}%
\pgfpathlineto{\pgfqpoint{2.743714in}{3.772115in}}%
\pgfpathlineto{\pgfqpoint{2.737320in}{3.776955in}}%
\pgfpathlineto{\pgfqpoint{2.730926in}{3.781794in}}%
\pgfpathlineto{\pgfqpoint{2.724531in}{3.786633in}}%
\pgfpathlineto{\pgfqpoint{2.718137in}{3.791472in}}%
\pgfpathlineto{\pgfqpoint{2.711743in}{3.796312in}}%
\pgfpathlineto{\pgfqpoint{2.705349in}{3.801151in}}%
\pgfpathlineto{\pgfqpoint{2.698954in}{3.805990in}}%
\pgfpathlineto{\pgfqpoint{2.692560in}{3.810829in}}%
\pgfpathlineto{\pgfqpoint{2.686166in}{3.815669in}}%
\pgfpathlineto{\pgfqpoint{2.679771in}{3.820508in}}%
\pgfpathlineto{\pgfqpoint{2.673377in}{3.825347in}}%
\pgfpathlineto{\pgfqpoint{2.666983in}{3.830186in}}%
\pgfpathlineto{\pgfqpoint{2.660589in}{3.835026in}}%
\pgfpathlineto{\pgfqpoint{2.654194in}{3.839865in}}%
\pgfpathlineto{\pgfqpoint{2.647800in}{3.844704in}}%
\pgfpathlineto{\pgfqpoint{2.641406in}{3.849543in}}%
\pgfpathlineto{\pgfqpoint{2.635011in}{3.854383in}}%
\pgfpathlineto{\pgfqpoint{2.628617in}{3.859222in}}%
\pgfpathlineto{\pgfqpoint{2.622223in}{3.864061in}}%
\pgfpathlineto{\pgfqpoint{2.615829in}{3.868900in}}%
\pgfpathlineto{\pgfqpoint{2.609434in}{3.873740in}}%
\pgfpathlineto{\pgfqpoint{2.603040in}{3.878579in}}%
\pgfpathlineto{\pgfqpoint{2.596646in}{3.883418in}}%
\pgfpathlineto{\pgfqpoint{2.590252in}{3.888258in}}%
\pgfpathlineto{\pgfqpoint{2.583857in}{3.893097in}}%
\pgfpathlineto{\pgfqpoint{2.577463in}{3.897936in}}%
\pgfpathlineto{\pgfqpoint{2.571069in}{3.902775in}}%
\pgfpathlineto{\pgfqpoint{2.564674in}{3.907615in}}%
\pgfpathlineto{\pgfqpoint{2.558280in}{3.912454in}}%
\pgfpathlineto{\pgfqpoint{2.551886in}{3.917293in}}%
\pgfpathlineto{\pgfqpoint{2.545492in}{3.922132in}}%
\pgfpathlineto{\pgfqpoint{2.539097in}{3.926972in}}%
\pgfpathlineto{\pgfqpoint{2.532703in}{3.931811in}}%
\pgfpathlineto{\pgfqpoint{2.526309in}{3.936650in}}%
\pgfpathlineto{\pgfqpoint{2.519914in}{3.941489in}}%
\pgfpathlineto{\pgfqpoint{2.513520in}{3.946329in}}%
\pgfpathlineto{\pgfqpoint{2.507126in}{3.951168in}}%
\pgfpathlineto{\pgfqpoint{2.500732in}{3.956007in}}%
\pgfpathlineto{\pgfqpoint{2.494337in}{3.960846in}}%
\pgfpathlineto{\pgfqpoint{2.487943in}{3.965686in}}%
\pgfpathlineto{\pgfqpoint{2.481549in}{3.970525in}}%
\pgfpathlineto{\pgfqpoint{2.475154in}{3.975364in}}%
\pgfpathlineto{\pgfqpoint{2.468760in}{3.980203in}}%
\pgfpathlineto{\pgfqpoint{2.462366in}{3.985043in}}%
\pgfpathlineto{\pgfqpoint{2.455972in}{3.989882in}}%
\pgfpathlineto{\pgfqpoint{2.449577in}{3.994721in}}%
\pgfpathlineto{\pgfqpoint{2.443183in}{3.999560in}}%
\pgfpathlineto{\pgfqpoint{2.436789in}{4.004400in}}%
\pgfpathlineto{\pgfqpoint{2.430395in}{4.009239in}}%
\pgfpathlineto{\pgfqpoint{2.424000in}{4.014078in}}%
\pgfpathlineto{\pgfqpoint{2.417606in}{4.018917in}}%
\pgfpathlineto{\pgfqpoint{2.411212in}{4.023757in}}%
\pgfpathlineto{\pgfqpoint{2.404817in}{4.028596in}}%
\pgfpathlineto{\pgfqpoint{2.398423in}{4.033435in}}%
\pgfpathlineto{\pgfqpoint{2.392029in}{4.038274in}}%
\pgfpathlineto{\pgfqpoint{2.385635in}{4.043114in}}%
\pgfpathlineto{\pgfqpoint{2.379240in}{4.047953in}}%
\pgfpathlineto{\pgfqpoint{2.372846in}{4.052792in}}%
\pgfpathlineto{\pgfqpoint{2.366452in}{4.057631in}}%
\pgfpathlineto{\pgfqpoint{2.360057in}{4.062471in}}%
\pgfpathlineto{\pgfqpoint{2.353663in}{4.067310in}}%
\pgfpathlineto{\pgfqpoint{2.347269in}{4.072149in}}%
\pgfpathlineto{\pgfqpoint{2.340875in}{4.076988in}}%
\pgfpathlineto{\pgfqpoint{2.334480in}{4.081828in}}%
\pgfpathlineto{\pgfqpoint{2.328086in}{4.086667in}}%
\pgfpathlineto{\pgfqpoint{2.321692in}{4.091506in}}%
\pgfpathlineto{\pgfqpoint{2.315298in}{4.096345in}}%
\pgfpathlineto{\pgfqpoint{2.308903in}{4.101185in}}%
\pgfpathlineto{\pgfqpoint{2.302509in}{4.106024in}}%
\pgfpathlineto{\pgfqpoint{2.296115in}{4.110863in}}%
\pgfpathlineto{\pgfqpoint{2.289720in}{4.115702in}}%
\pgfpathlineto{\pgfqpoint{2.283326in}{4.120542in}}%
\pgfpathlineto{\pgfqpoint{2.276932in}{4.125381in}}%
\pgfpathlineto{\pgfqpoint{2.270538in}{4.130220in}}%
\pgfpathlineto{\pgfqpoint{2.264143in}{4.135059in}}%
\pgfpathlineto{\pgfqpoint{2.257749in}{4.139899in}}%
\pgfpathlineto{\pgfqpoint{2.251355in}{4.144738in}}%
\pgfpathlineto{\pgfqpoint{2.244960in}{4.149577in}}%
\pgfpathlineto{\pgfqpoint{2.238566in}{4.154416in}}%
\pgfpathlineto{\pgfqpoint{2.232172in}{4.159256in}}%
\pgfpathlineto{\pgfqpoint{2.225778in}{4.164095in}}%
\pgfpathlineto{\pgfqpoint{2.219383in}{4.168934in}}%
\pgfpathlineto{\pgfqpoint{2.212989in}{4.173773in}}%
\pgfpathlineto{\pgfqpoint{2.206595in}{4.178613in}}%
\pgfpathlineto{\pgfqpoint{2.200200in}{4.183452in}}%
\pgfpathlineto{\pgfqpoint{2.193806in}{4.188291in}}%
\pgfpathlineto{\pgfqpoint{2.187412in}{4.193130in}}%
\pgfpathlineto{\pgfqpoint{2.181018in}{4.197970in}}%
\pgfpathlineto{\pgfqpoint{2.174623in}{4.202809in}}%
\pgfpathlineto{\pgfqpoint{2.168229in}{4.207648in}}%
\pgfpathlineto{\pgfqpoint{2.161835in}{4.212487in}}%
\pgfpathlineto{\pgfqpoint{2.155441in}{4.217327in}}%
\pgfpathlineto{\pgfqpoint{2.149046in}{4.222166in}}%
\pgfpathlineto{\pgfqpoint{2.142652in}{4.227005in}}%
\pgfpathlineto{\pgfqpoint{2.136258in}{4.231845in}}%
\pgfpathlineto{\pgfqpoint{2.129863in}{4.236684in}}%
\pgfpathlineto{\pgfqpoint{2.123469in}{4.241523in}}%
\pgfpathlineto{\pgfqpoint{2.117075in}{4.246362in}}%
\pgfpathlineto{\pgfqpoint{2.110681in}{4.251202in}}%
\pgfpathlineto{\pgfqpoint{2.104286in}{4.256041in}}%
\pgfpathlineto{\pgfqpoint{2.097892in}{4.260880in}}%
\pgfpathlineto{\pgfqpoint{2.091498in}{4.265719in}}%
\pgfpathlineto{\pgfqpoint{2.085103in}{4.270559in}}%
\pgfpathlineto{\pgfqpoint{2.078709in}{4.275398in}}%
\pgfpathlineto{\pgfqpoint{2.072315in}{4.280237in}}%
\pgfpathlineto{\pgfqpoint{2.065921in}{4.285076in}}%
\pgfpathlineto{\pgfqpoint{2.059526in}{4.289916in}}%
\pgfpathlineto{\pgfqpoint{2.053132in}{4.294755in}}%
\pgfpathlineto{\pgfqpoint{2.046738in}{4.299594in}}%
\pgfpathlineto{\pgfqpoint{2.040343in}{4.304433in}}%
\pgfpathlineto{\pgfqpoint{2.033949in}{4.309273in}}%
\pgfpathlineto{\pgfqpoint{2.027555in}{4.314112in}}%
\pgfpathlineto{\pgfqpoint{2.021161in}{4.318951in}}%
\pgfpathlineto{\pgfqpoint{2.014766in}{4.323790in}}%
\pgfpathlineto{\pgfqpoint{2.008372in}{4.328630in}}%
\pgfpathlineto{\pgfqpoint{2.001978in}{4.333469in}}%
\pgfpathlineto{\pgfqpoint{1.995584in}{4.338308in}}%
\pgfpathlineto{\pgfqpoint{1.989189in}{4.343147in}}%
\pgfpathlineto{\pgfqpoint{1.982795in}{4.347987in}}%
\pgfpathlineto{\pgfqpoint{1.976401in}{4.352826in}}%
\pgfpathlineto{\pgfqpoint{1.970006in}{4.357665in}}%
\pgfpathlineto{\pgfqpoint{1.963612in}{4.362504in}}%
\pgfpathlineto{\pgfqpoint{1.957218in}{4.367344in}}%
\pgfpathlineto{\pgfqpoint{1.950824in}{4.372183in}}%
\pgfpathlineto{\pgfqpoint{1.944429in}{4.377022in}}%
\pgfpathlineto{\pgfqpoint{1.938035in}{4.381861in}}%
\pgfpathlineto{\pgfqpoint{1.931641in}{4.386701in}}%
\pgfpathlineto{\pgfqpoint{1.925246in}{4.391540in}}%
\pgfpathlineto{\pgfqpoint{1.918852in}{4.396379in}}%
\pgfpathlineto{\pgfqpoint{1.912458in}{4.401218in}}%
\pgfpathlineto{\pgfqpoint{1.906064in}{4.406058in}}%
\pgfpathlineto{\pgfqpoint{1.899669in}{4.410897in}}%
\pgfpathlineto{\pgfqpoint{1.893275in}{4.415736in}}%
\pgfpathlineto{\pgfqpoint{1.886881in}{4.420575in}}%
\pgfpathlineto{\pgfqpoint{1.880486in}{4.425415in}}%
\pgfpathlineto{\pgfqpoint{1.874092in}{4.430254in}}%
\pgfpathlineto{\pgfqpoint{1.867698in}{4.435093in}}%
\pgfpathlineto{\pgfqpoint{1.861304in}{4.439932in}}%
\pgfpathlineto{\pgfqpoint{1.854909in}{4.444772in}}%
\pgfpathlineto{\pgfqpoint{1.848515in}{4.449611in}}%
\pgfpathlineto{\pgfqpoint{1.842121in}{4.454450in}}%
\pgfpathlineto{\pgfqpoint{1.835727in}{4.459289in}}%
\pgfpathlineto{\pgfqpoint{1.829332in}{4.464129in}}%
\pgfpathlineto{\pgfqpoint{1.822938in}{4.468968in}}%
\pgfpathlineto{\pgfqpoint{1.816544in}{4.473807in}}%
\pgfpathlineto{\pgfqpoint{1.810149in}{4.478646in}}%
\pgfpathlineto{\pgfqpoint{1.803755in}{4.483486in}}%
\pgfpathlineto{\pgfqpoint{1.797361in}{4.488325in}}%
\pgfpathlineto{\pgfqpoint{1.790967in}{4.493164in}}%
\pgfpathlineto{\pgfqpoint{1.784572in}{4.498003in}}%
\pgfpathlineto{\pgfqpoint{1.778178in}{4.502843in}}%
\pgfpathlineto{\pgfqpoint{1.771784in}{4.507682in}}%
\pgfpathlineto{\pgfqpoint{1.765389in}{4.512521in}}%
\pgfpathlineto{\pgfqpoint{1.758995in}{4.517360in}}%
\pgfpathlineto{\pgfqpoint{1.752601in}{4.522200in}}%
\pgfpathlineto{\pgfqpoint{1.746207in}{4.527039in}}%
\pgfpathlineto{\pgfqpoint{1.739812in}{4.531878in}}%
\pgfpathlineto{\pgfqpoint{1.733418in}{4.536717in}}%
\pgfpathlineto{\pgfqpoint{1.727024in}{4.541557in}}%
\pgfpathlineto{\pgfqpoint{1.720630in}{4.546396in}}%
\pgfpathlineto{\pgfqpoint{1.714235in}{4.551235in}}%
\pgfpathlineto{\pgfqpoint{1.707841in}{4.556074in}}%
\pgfpathlineto{\pgfqpoint{1.701447in}{4.560914in}}%
\pgfpathlineto{\pgfqpoint{1.695052in}{4.565753in}}%
\pgfpathlineto{\pgfqpoint{1.688658in}{4.570592in}}%
\pgfpathlineto{\pgfqpoint{1.682264in}{4.575432in}}%
\pgfpathlineto{\pgfqpoint{1.675870in}{4.580271in}}%
\pgfpathlineto{\pgfqpoint{1.669475in}{4.585110in}}%
\pgfpathlineto{\pgfqpoint{1.663081in}{4.589949in}}%
\pgfpathlineto{\pgfqpoint{1.656687in}{4.594789in}}%
\pgfpathlineto{\pgfqpoint{1.650292in}{4.599628in}}%
\pgfpathlineto{\pgfqpoint{1.643898in}{4.604467in}}%
\pgfpathlineto{\pgfqpoint{1.637504in}{4.609306in}}%
\pgfpathlineto{\pgfqpoint{1.631110in}{4.614146in}}%
\pgfpathlineto{\pgfqpoint{1.624715in}{4.618985in}}%
\pgfpathlineto{\pgfqpoint{1.618321in}{4.623824in}}%
\pgfpathlineto{\pgfqpoint{1.611927in}{4.628663in}}%
\pgfpathlineto{\pgfqpoint{1.605532in}{4.633503in}}%
\pgfpathlineto{\pgfqpoint{1.599138in}{4.638342in}}%
\pgfpathlineto{\pgfqpoint{1.592744in}{4.643181in}}%
\pgfpathlineto{\pgfqpoint{1.586350in}{4.648020in}}%
\pgfpathlineto{\pgfqpoint{1.579955in}{4.652860in}}%
\pgfpathlineto{\pgfqpoint{1.573561in}{4.657699in}}%
\pgfpathlineto{\pgfqpoint{1.567167in}{4.662538in}}%
\pgfpathlineto{\pgfqpoint{1.560773in}{4.667377in}}%
\pgfpathlineto{\pgfqpoint{1.554378in}{4.672217in}}%
\pgfpathlineto{\pgfqpoint{1.547984in}{4.677056in}}%
\pgfpathlineto{\pgfqpoint{1.541590in}{4.681895in}}%
\pgfpathlineto{\pgfqpoint{1.535195in}{4.686734in}}%
\pgfpathlineto{\pgfqpoint{1.528801in}{4.691574in}}%
\pgfpathlineto{\pgfqpoint{1.522407in}{4.696413in}}%
\pgfpathlineto{\pgfqpoint{1.516013in}{4.701252in}}%
\pgfpathlineto{\pgfqpoint{1.509618in}{4.706091in}}%
\pgfpathlineto{\pgfqpoint{1.503224in}{4.710931in}}%
\pgfpathlineto{\pgfqpoint{1.496830in}{4.715770in}}%
\pgfpathlineto{\pgfqpoint{1.490435in}{4.720609in}}%
\pgfpathlineto{\pgfqpoint{1.484041in}{4.725448in}}%
\pgfpathlineto{\pgfqpoint{1.477647in}{4.730288in}}%
\pgfpathlineto{\pgfqpoint{1.471253in}{4.735127in}}%
\pgfpathlineto{\pgfqpoint{1.464858in}{4.739966in}}%
\pgfpathlineto{\pgfqpoint{1.458464in}{4.744805in}}%
\pgfpathlineto{\pgfqpoint{1.452070in}{4.749645in}}%
\pgfpathlineto{\pgfqpoint{1.445675in}{4.754484in}}%
\pgfpathlineto{\pgfqpoint{1.439281in}{4.759323in}}%
\pgfpathlineto{\pgfqpoint{1.432887in}{4.764162in}}%
\pgfpathlineto{\pgfqpoint{1.426493in}{4.769002in}}%
\pgfpathlineto{\pgfqpoint{1.420098in}{4.773841in}}%
\pgfpathlineto{\pgfqpoint{1.413704in}{4.778680in}}%
\pgfpathlineto{\pgfqpoint{1.407310in}{4.783519in}}%
\pgfpathlineto{\pgfqpoint{1.400916in}{4.788359in}}%
\pgfpathlineto{\pgfqpoint{1.394521in}{4.793198in}}%
\pgfpathlineto{\pgfqpoint{1.388127in}{4.798037in}}%
\pgfpathlineto{\pgfqpoint{1.381733in}{4.802876in}}%
\pgfpathlineto{\pgfqpoint{1.375338in}{4.807716in}}%
\pgfpathlineto{\pgfqpoint{1.368944in}{4.812555in}}%
\pgfpathlineto{\pgfqpoint{1.362550in}{4.817394in}}%
\pgfpathlineto{\pgfqpoint{1.356156in}{4.822233in}}%
\pgfpathlineto{\pgfqpoint{1.349761in}{4.827073in}}%
\pgfpathlineto{\pgfqpoint{1.343367in}{4.831912in}}%
\pgfpathlineto{\pgfqpoint{1.336973in}{4.836751in}}%
\pgfpathlineto{\pgfqpoint{1.330578in}{4.841590in}}%
\pgfpathlineto{\pgfqpoint{1.324184in}{4.846430in}}%
\pgfpathlineto{\pgfqpoint{1.317790in}{4.851269in}}%
\pgfpathlineto{\pgfqpoint{1.311396in}{4.856108in}}%
\pgfpathlineto{\pgfqpoint{1.305001in}{4.860947in}}%
\pgfpathlineto{\pgfqpoint{1.298607in}{4.865787in}}%
\pgfpathlineto{\pgfqpoint{1.292213in}{4.870626in}}%
\pgfpathlineto{\pgfqpoint{1.285818in}{4.875465in}}%
\pgfpathlineto{\pgfqpoint{1.279424in}{4.880304in}}%
\pgfpathlineto{\pgfqpoint{1.273030in}{4.885144in}}%
\pgfpathlineto{\pgfqpoint{1.266636in}{4.889983in}}%
\pgfpathlineto{\pgfqpoint{1.260241in}{4.894822in}}%
\pgfpathlineto{\pgfqpoint{1.253847in}{4.899662in}}%
\pgfpathlineto{\pgfqpoint{1.247453in}{4.904501in}}%
\pgfpathlineto{\pgfqpoint{1.241059in}{4.909340in}}%
\pgfpathlineto{\pgfqpoint{1.234664in}{4.914179in}}%
\pgfpathlineto{\pgfqpoint{1.228270in}{4.919019in}}%
\pgfpathlineto{\pgfqpoint{1.221876in}{4.923858in}}%
\pgfpathlineto{\pgfqpoint{1.215481in}{4.928697in}}%
\pgfpathlineto{\pgfqpoint{1.209087in}{4.933536in}}%
\pgfpathlineto{\pgfqpoint{1.202693in}{4.938376in}}%
\pgfpathlineto{\pgfqpoint{1.196299in}{4.943215in}}%
\pgfpathlineto{\pgfqpoint{1.189904in}{4.948054in}}%
\pgfpathlineto{\pgfqpoint{1.183510in}{4.952893in}}%
\pgfpathlineto{\pgfqpoint{1.177116in}{4.957733in}}%
\pgfpathlineto{\pgfqpoint{1.170721in}{4.962572in}}%
\pgfpathlineto{\pgfqpoint{1.164327in}{4.967411in}}%
\pgfpathlineto{\pgfqpoint{1.157933in}{4.972250in}}%
\pgfpathlineto{\pgfqpoint{1.151539in}{4.977090in}}%
\pgfpathlineto{\pgfqpoint{1.145144in}{4.981929in}}%
\pgfpathlineto{\pgfqpoint{1.138750in}{4.986768in}}%
\pgfpathlineto{\pgfqpoint{1.132356in}{4.991607in}}%
\pgfpathlineto{\pgfqpoint{1.125962in}{4.996447in}}%
\pgfpathlineto{\pgfqpoint{1.119567in}{5.001286in}}%
\pgfpathlineto{\pgfqpoint{1.113173in}{5.006125in}}%
\pgfpathlineto{\pgfqpoint{1.106779in}{5.010964in}}%
\pgfpathlineto{\pgfqpoint{1.100384in}{5.015804in}}%
\pgfpathlineto{\pgfqpoint{1.093990in}{5.020643in}}%
\pgfpathlineto{\pgfqpoint{1.087596in}{5.025482in}}%
\pgfpathlineto{\pgfqpoint{1.081202in}{5.030321in}}%
\pgfpathlineto{\pgfqpoint{1.074807in}{5.035161in}}%
\pgfpathlineto{\pgfqpoint{1.068413in}{5.040000in}}%
\pgfpathlineto{\pgfqpoint{1.062019in}{5.044839in}}%
\pgfpathlineto{\pgfqpoint{1.055624in}{5.049678in}}%
\pgfpathlineto{\pgfqpoint{1.049230in}{5.054518in}}%
\pgfpathlineto{\pgfqpoint{1.042836in}{5.059357in}}%
\pgfpathlineto{\pgfqpoint{1.036442in}{5.064196in}}%
\pgfpathlineto{\pgfqpoint{1.030047in}{5.069035in}}%
\pgfpathlineto{\pgfqpoint{1.023653in}{5.073875in}}%
\pgfpathlineto{\pgfqpoint{1.017259in}{5.078714in}}%
\pgfpathlineto{\pgfqpoint{1.010864in}{5.083553in}}%
\pgfpathlineto{\pgfqpoint{1.004470in}{5.088392in}}%
\pgfpathlineto{\pgfqpoint{0.998076in}{5.093232in}}%
\pgfpathlineto{\pgfqpoint{0.991682in}{5.098071in}}%
\pgfpathlineto{\pgfqpoint{0.985287in}{5.102910in}}%
\pgfpathlineto{\pgfqpoint{0.978893in}{5.107749in}}%
\pgfpathlineto{\pgfqpoint{0.972499in}{5.112589in}}%
\pgfpathlineto{\pgfqpoint{0.966105in}{5.117428in}}%
\pgfpathlineto{\pgfqpoint{0.959710in}{5.122267in}}%
\pgfpathlineto{\pgfqpoint{0.953316in}{5.127106in}}%
\pgfpathlineto{\pgfqpoint{0.946922in}{5.131946in}}%
\pgfpathlineto{\pgfqpoint{0.940527in}{5.136785in}}%
\pgfpathlineto{\pgfqpoint{0.934133in}{5.141624in}}%
\pgfpathlineto{\pgfqpoint{0.927739in}{5.146463in}}%
\pgfpathlineto{\pgfqpoint{0.921345in}{5.151303in}}%
\pgfpathlineto{\pgfqpoint{0.914950in}{5.156142in}}%
\pgfpathlineto{\pgfqpoint{0.908556in}{5.160981in}}%
\pgfpathlineto{\pgfqpoint{0.902162in}{5.165820in}}%
\pgfpathlineto{\pgfqpoint{0.895767in}{5.170660in}}%
\pgfpathlineto{\pgfqpoint{0.889373in}{5.175499in}}%
\pgfpathlineto{\pgfqpoint{0.882979in}{5.180338in}}%
\pgfpathlineto{\pgfqpoint{0.876585in}{5.185177in}}%
\pgfpathlineto{\pgfqpoint{0.870190in}{5.190017in}}%
\pgfpathlineto{\pgfqpoint{0.863796in}{5.194856in}}%
\pgfpathlineto{\pgfqpoint{0.857402in}{5.199695in}}%
\pgfpathlineto{\pgfqpoint{0.851007in}{5.204534in}}%
\pgfpathlineto{\pgfqpoint{0.844613in}{5.209374in}}%
\pgfpathlineto{\pgfqpoint{0.838219in}{5.214213in}}%
\pgfpathlineto{\pgfqpoint{0.831825in}{5.219052in}}%
\pgfpathlineto{\pgfqpoint{0.825430in}{5.223891in}}%
\pgfpathlineto{\pgfqpoint{0.819036in}{5.228731in}}%
\pgfpathlineto{\pgfqpoint{0.812642in}{5.233570in}}%
\pgfpathlineto{\pgfqpoint{0.806248in}{5.238409in}}%
\pgfpathlineto{\pgfqpoint{0.799853in}{5.243249in}}%
\pgfpathlineto{\pgfqpoint{0.793459in}{5.248088in}}%
\pgfpathlineto{\pgfqpoint{0.787065in}{5.252927in}}%
\pgfpathlineto{\pgfqpoint{0.780670in}{5.257766in}}%
\pgfpathlineto{\pgfqpoint{0.774276in}{5.262606in}}%
\pgfpathlineto{\pgfqpoint{0.767882in}{5.267445in}}%
\pgfpathlineto{\pgfqpoint{0.761488in}{5.272284in}}%
\pgfpathlineto{\pgfqpoint{0.755093in}{5.277123in}}%
\pgfpathlineto{\pgfqpoint{0.748699in}{5.281963in}}%
\pgfpathlineto{\pgfqpoint{0.742305in}{5.286802in}}%
\pgfpathlineto{\pgfqpoint{0.735910in}{5.291641in}}%
\pgfpathlineto{\pgfqpoint{0.729516in}{5.296480in}}%
\pgfpathlineto{\pgfqpoint{0.723122in}{5.301320in}}%
\pgfpathlineto{\pgfqpoint{0.716728in}{5.306159in}}%
\pgfpathlineto{\pgfqpoint{0.710333in}{5.310998in}}%
\pgfpathlineto{\pgfqpoint{0.703939in}{5.315837in}}%
\pgfpathlineto{\pgfqpoint{0.697545in}{5.320677in}}%
\pgfpathlineto{\pgfqpoint{0.691150in}{5.325516in}}%
\pgfpathlineto{\pgfqpoint{0.684756in}{5.330355in}}%
\pgfpathlineto{\pgfqpoint{0.678362in}{5.335194in}}%
\pgfpathlineto{\pgfqpoint{0.671968in}{5.340034in}}%
\pgfpathlineto{\pgfqpoint{0.665573in}{5.344873in}}%
\pgfpathlineto{\pgfqpoint{0.659179in}{5.349712in}}%
\pgfpathlineto{\pgfqpoint{0.652785in}{5.354551in}}%
\pgfpathlineto{\pgfqpoint{0.646391in}{5.359391in}}%
\pgfpathlineto{\pgfqpoint{0.639996in}{5.364230in}}%
\pgfpathlineto{\pgfqpoint{0.633602in}{5.369069in}}%
\pgfpathlineto{\pgfqpoint{0.627208in}{5.373908in}}%
\pgfpathlineto{\pgfqpoint{0.620813in}{5.378748in}}%
\pgfpathlineto{\pgfqpoint{0.614419in}{5.383587in}}%
\pgfpathlineto{\pgfqpoint{0.608025in}{5.388426in}}%
\pgfpathlineto{\pgfqpoint{0.608025in}{5.388426in}}%
\pgfpathclose%
\pgfusepath{fill}%
\end{pgfscope}%
\begin{pgfscope}%
\pgfpathrectangle{\pgfqpoint{0.608025in}{0.554012in}}{\pgfqpoint{6.387885in}{4.834414in}}%
\pgfusepath{clip}%
\pgfsetbuttcap%
\pgfsetroundjoin%
\definecolor{currentfill}{rgb}{1.000000,0.498039,0.054902}%
\pgfsetfillcolor{currentfill}%
\pgfsetfillopacity{0.200000}%
\pgfsetlinewidth{1.003750pt}%
\definecolor{currentstroke}{rgb}{1.000000,0.498039,0.054902}%
\pgfsetstrokecolor{currentstroke}%
\pgfsetstrokeopacity{0.200000}%
\pgfsetdash{}{0pt}%
\pgfsys@defobject{currentmarker}{\pgfqpoint{0.608025in}{0.554012in}}{\pgfqpoint{3.798770in}{2.971219in}}{%
\pgfpathmoveto{\pgfqpoint{0.608025in}{2.971219in}}%
\pgfpathlineto{\pgfqpoint{0.608025in}{0.554012in}}%
\pgfpathlineto{\pgfqpoint{0.614419in}{0.558851in}}%
\pgfpathlineto{\pgfqpoint{0.620813in}{0.563690in}}%
\pgfpathlineto{\pgfqpoint{0.627208in}{0.568530in}}%
\pgfpathlineto{\pgfqpoint{0.633602in}{0.573369in}}%
\pgfpathlineto{\pgfqpoint{0.639996in}{0.578208in}}%
\pgfpathlineto{\pgfqpoint{0.646391in}{0.583047in}}%
\pgfpathlineto{\pgfqpoint{0.652785in}{0.587887in}}%
\pgfpathlineto{\pgfqpoint{0.659179in}{0.592726in}}%
\pgfpathlineto{\pgfqpoint{0.665573in}{0.597565in}}%
\pgfpathlineto{\pgfqpoint{0.671968in}{0.602404in}}%
\pgfpathlineto{\pgfqpoint{0.678362in}{0.607244in}}%
\pgfpathlineto{\pgfqpoint{0.684756in}{0.612083in}}%
\pgfpathlineto{\pgfqpoint{0.691150in}{0.616922in}}%
\pgfpathlineto{\pgfqpoint{0.697545in}{0.621761in}}%
\pgfpathlineto{\pgfqpoint{0.703939in}{0.626601in}}%
\pgfpathlineto{\pgfqpoint{0.710333in}{0.631440in}}%
\pgfpathlineto{\pgfqpoint{0.716728in}{0.636279in}}%
\pgfpathlineto{\pgfqpoint{0.723122in}{0.641118in}}%
\pgfpathlineto{\pgfqpoint{0.729516in}{0.645958in}}%
\pgfpathlineto{\pgfqpoint{0.735910in}{0.650797in}}%
\pgfpathlineto{\pgfqpoint{0.742305in}{0.655636in}}%
\pgfpathlineto{\pgfqpoint{0.748699in}{0.660475in}}%
\pgfpathlineto{\pgfqpoint{0.755093in}{0.665315in}}%
\pgfpathlineto{\pgfqpoint{0.761488in}{0.670154in}}%
\pgfpathlineto{\pgfqpoint{0.767882in}{0.674993in}}%
\pgfpathlineto{\pgfqpoint{0.774276in}{0.679832in}}%
\pgfpathlineto{\pgfqpoint{0.780670in}{0.684672in}}%
\pgfpathlineto{\pgfqpoint{0.787065in}{0.689511in}}%
\pgfpathlineto{\pgfqpoint{0.793459in}{0.694350in}}%
\pgfpathlineto{\pgfqpoint{0.799853in}{0.699189in}}%
\pgfpathlineto{\pgfqpoint{0.806248in}{0.704029in}}%
\pgfpathlineto{\pgfqpoint{0.812642in}{0.708868in}}%
\pgfpathlineto{\pgfqpoint{0.819036in}{0.713707in}}%
\pgfpathlineto{\pgfqpoint{0.825430in}{0.718546in}}%
\pgfpathlineto{\pgfqpoint{0.831825in}{0.723386in}}%
\pgfpathlineto{\pgfqpoint{0.838219in}{0.728225in}}%
\pgfpathlineto{\pgfqpoint{0.844613in}{0.733064in}}%
\pgfpathlineto{\pgfqpoint{0.851007in}{0.737903in}}%
\pgfpathlineto{\pgfqpoint{0.857402in}{0.742743in}}%
\pgfpathlineto{\pgfqpoint{0.863796in}{0.747582in}}%
\pgfpathlineto{\pgfqpoint{0.870190in}{0.752421in}}%
\pgfpathlineto{\pgfqpoint{0.876585in}{0.757260in}}%
\pgfpathlineto{\pgfqpoint{0.882979in}{0.762100in}}%
\pgfpathlineto{\pgfqpoint{0.889373in}{0.766939in}}%
\pgfpathlineto{\pgfqpoint{0.895767in}{0.771778in}}%
\pgfpathlineto{\pgfqpoint{0.902162in}{0.776617in}}%
\pgfpathlineto{\pgfqpoint{0.908556in}{0.781457in}}%
\pgfpathlineto{\pgfqpoint{0.914950in}{0.786296in}}%
\pgfpathlineto{\pgfqpoint{0.921345in}{0.791135in}}%
\pgfpathlineto{\pgfqpoint{0.927739in}{0.795974in}}%
\pgfpathlineto{\pgfqpoint{0.934133in}{0.800814in}}%
\pgfpathlineto{\pgfqpoint{0.940527in}{0.805653in}}%
\pgfpathlineto{\pgfqpoint{0.946922in}{0.810492in}}%
\pgfpathlineto{\pgfqpoint{0.953316in}{0.815331in}}%
\pgfpathlineto{\pgfqpoint{0.959710in}{0.820171in}}%
\pgfpathlineto{\pgfqpoint{0.966105in}{0.825010in}}%
\pgfpathlineto{\pgfqpoint{0.972499in}{0.829849in}}%
\pgfpathlineto{\pgfqpoint{0.978893in}{0.834689in}}%
\pgfpathlineto{\pgfqpoint{0.985287in}{0.839528in}}%
\pgfpathlineto{\pgfqpoint{0.991682in}{0.844367in}}%
\pgfpathlineto{\pgfqpoint{0.998076in}{0.849206in}}%
\pgfpathlineto{\pgfqpoint{1.004470in}{0.854046in}}%
\pgfpathlineto{\pgfqpoint{1.010864in}{0.858885in}}%
\pgfpathlineto{\pgfqpoint{1.017259in}{0.863724in}}%
\pgfpathlineto{\pgfqpoint{1.023653in}{0.868563in}}%
\pgfpathlineto{\pgfqpoint{1.030047in}{0.873403in}}%
\pgfpathlineto{\pgfqpoint{1.036442in}{0.878242in}}%
\pgfpathlineto{\pgfqpoint{1.042836in}{0.883081in}}%
\pgfpathlineto{\pgfqpoint{1.049230in}{0.887920in}}%
\pgfpathlineto{\pgfqpoint{1.055624in}{0.892760in}}%
\pgfpathlineto{\pgfqpoint{1.062019in}{0.897599in}}%
\pgfpathlineto{\pgfqpoint{1.068413in}{0.902438in}}%
\pgfpathlineto{\pgfqpoint{1.074807in}{0.907277in}}%
\pgfpathlineto{\pgfqpoint{1.081202in}{0.912117in}}%
\pgfpathlineto{\pgfqpoint{1.087596in}{0.916956in}}%
\pgfpathlineto{\pgfqpoint{1.093990in}{0.921795in}}%
\pgfpathlineto{\pgfqpoint{1.100384in}{0.926634in}}%
\pgfpathlineto{\pgfqpoint{1.106779in}{0.931474in}}%
\pgfpathlineto{\pgfqpoint{1.113173in}{0.936313in}}%
\pgfpathlineto{\pgfqpoint{1.119567in}{0.941152in}}%
\pgfpathlineto{\pgfqpoint{1.125962in}{0.945991in}}%
\pgfpathlineto{\pgfqpoint{1.132356in}{0.950831in}}%
\pgfpathlineto{\pgfqpoint{1.138750in}{0.955670in}}%
\pgfpathlineto{\pgfqpoint{1.145144in}{0.960509in}}%
\pgfpathlineto{\pgfqpoint{1.151539in}{0.965348in}}%
\pgfpathlineto{\pgfqpoint{1.157933in}{0.970188in}}%
\pgfpathlineto{\pgfqpoint{1.164327in}{0.975027in}}%
\pgfpathlineto{\pgfqpoint{1.170721in}{0.979866in}}%
\pgfpathlineto{\pgfqpoint{1.177116in}{0.984705in}}%
\pgfpathlineto{\pgfqpoint{1.183510in}{0.989545in}}%
\pgfpathlineto{\pgfqpoint{1.189904in}{0.994384in}}%
\pgfpathlineto{\pgfqpoint{1.196299in}{0.999223in}}%
\pgfpathlineto{\pgfqpoint{1.202693in}{1.004062in}}%
\pgfpathlineto{\pgfqpoint{1.209087in}{1.008902in}}%
\pgfpathlineto{\pgfqpoint{1.215481in}{1.013741in}}%
\pgfpathlineto{\pgfqpoint{1.221876in}{1.018580in}}%
\pgfpathlineto{\pgfqpoint{1.228270in}{1.023419in}}%
\pgfpathlineto{\pgfqpoint{1.234664in}{1.028259in}}%
\pgfpathlineto{\pgfqpoint{1.241059in}{1.033098in}}%
\pgfpathlineto{\pgfqpoint{1.247453in}{1.037937in}}%
\pgfpathlineto{\pgfqpoint{1.253847in}{1.042776in}}%
\pgfpathlineto{\pgfqpoint{1.260241in}{1.047616in}}%
\pgfpathlineto{\pgfqpoint{1.266636in}{1.052455in}}%
\pgfpathlineto{\pgfqpoint{1.273030in}{1.057294in}}%
\pgfpathlineto{\pgfqpoint{1.279424in}{1.062133in}}%
\pgfpathlineto{\pgfqpoint{1.285818in}{1.066973in}}%
\pgfpathlineto{\pgfqpoint{1.292213in}{1.071812in}}%
\pgfpathlineto{\pgfqpoint{1.298607in}{1.076651in}}%
\pgfpathlineto{\pgfqpoint{1.305001in}{1.081490in}}%
\pgfpathlineto{\pgfqpoint{1.311396in}{1.086330in}}%
\pgfpathlineto{\pgfqpoint{1.317790in}{1.091169in}}%
\pgfpathlineto{\pgfqpoint{1.324184in}{1.096008in}}%
\pgfpathlineto{\pgfqpoint{1.330578in}{1.100847in}}%
\pgfpathlineto{\pgfqpoint{1.336973in}{1.105687in}}%
\pgfpathlineto{\pgfqpoint{1.343367in}{1.110526in}}%
\pgfpathlineto{\pgfqpoint{1.349761in}{1.115365in}}%
\pgfpathlineto{\pgfqpoint{1.356156in}{1.120204in}}%
\pgfpathlineto{\pgfqpoint{1.362550in}{1.125044in}}%
\pgfpathlineto{\pgfqpoint{1.368944in}{1.129883in}}%
\pgfpathlineto{\pgfqpoint{1.375338in}{1.134722in}}%
\pgfpathlineto{\pgfqpoint{1.381733in}{1.139561in}}%
\pgfpathlineto{\pgfqpoint{1.388127in}{1.144401in}}%
\pgfpathlineto{\pgfqpoint{1.394521in}{1.149240in}}%
\pgfpathlineto{\pgfqpoint{1.400916in}{1.154079in}}%
\pgfpathlineto{\pgfqpoint{1.407310in}{1.158919in}}%
\pgfpathlineto{\pgfqpoint{1.413704in}{1.163758in}}%
\pgfpathlineto{\pgfqpoint{1.420098in}{1.168597in}}%
\pgfpathlineto{\pgfqpoint{1.426493in}{1.173436in}}%
\pgfpathlineto{\pgfqpoint{1.432887in}{1.178276in}}%
\pgfpathlineto{\pgfqpoint{1.439281in}{1.183115in}}%
\pgfpathlineto{\pgfqpoint{1.445675in}{1.187954in}}%
\pgfpathlineto{\pgfqpoint{1.452070in}{1.192793in}}%
\pgfpathlineto{\pgfqpoint{1.458464in}{1.197633in}}%
\pgfpathlineto{\pgfqpoint{1.464858in}{1.202472in}}%
\pgfpathlineto{\pgfqpoint{1.471253in}{1.207311in}}%
\pgfpathlineto{\pgfqpoint{1.477647in}{1.212150in}}%
\pgfpathlineto{\pgfqpoint{1.484041in}{1.216990in}}%
\pgfpathlineto{\pgfqpoint{1.490435in}{1.221829in}}%
\pgfpathlineto{\pgfqpoint{1.496830in}{1.226668in}}%
\pgfpathlineto{\pgfqpoint{1.503224in}{1.231507in}}%
\pgfpathlineto{\pgfqpoint{1.509618in}{1.236347in}}%
\pgfpathlineto{\pgfqpoint{1.516013in}{1.241186in}}%
\pgfpathlineto{\pgfqpoint{1.522407in}{1.246025in}}%
\pgfpathlineto{\pgfqpoint{1.528801in}{1.250864in}}%
\pgfpathlineto{\pgfqpoint{1.535195in}{1.255704in}}%
\pgfpathlineto{\pgfqpoint{1.541590in}{1.260543in}}%
\pgfpathlineto{\pgfqpoint{1.547984in}{1.265382in}}%
\pgfpathlineto{\pgfqpoint{1.554378in}{1.270221in}}%
\pgfpathlineto{\pgfqpoint{1.560773in}{1.275061in}}%
\pgfpathlineto{\pgfqpoint{1.567167in}{1.279900in}}%
\pgfpathlineto{\pgfqpoint{1.573561in}{1.284739in}}%
\pgfpathlineto{\pgfqpoint{1.579955in}{1.289578in}}%
\pgfpathlineto{\pgfqpoint{1.586350in}{1.294418in}}%
\pgfpathlineto{\pgfqpoint{1.592744in}{1.299257in}}%
\pgfpathlineto{\pgfqpoint{1.599138in}{1.304096in}}%
\pgfpathlineto{\pgfqpoint{1.605532in}{1.308935in}}%
\pgfpathlineto{\pgfqpoint{1.611927in}{1.313775in}}%
\pgfpathlineto{\pgfqpoint{1.618321in}{1.318614in}}%
\pgfpathlineto{\pgfqpoint{1.624715in}{1.323453in}}%
\pgfpathlineto{\pgfqpoint{1.631110in}{1.328292in}}%
\pgfpathlineto{\pgfqpoint{1.637504in}{1.333132in}}%
\pgfpathlineto{\pgfqpoint{1.643898in}{1.337971in}}%
\pgfpathlineto{\pgfqpoint{1.650292in}{1.342810in}}%
\pgfpathlineto{\pgfqpoint{1.656687in}{1.347649in}}%
\pgfpathlineto{\pgfqpoint{1.663081in}{1.352489in}}%
\pgfpathlineto{\pgfqpoint{1.669475in}{1.357328in}}%
\pgfpathlineto{\pgfqpoint{1.675870in}{1.362167in}}%
\pgfpathlineto{\pgfqpoint{1.682264in}{1.367006in}}%
\pgfpathlineto{\pgfqpoint{1.688658in}{1.371846in}}%
\pgfpathlineto{\pgfqpoint{1.695052in}{1.376685in}}%
\pgfpathlineto{\pgfqpoint{1.701447in}{1.381524in}}%
\pgfpathlineto{\pgfqpoint{1.707841in}{1.386363in}}%
\pgfpathlineto{\pgfqpoint{1.714235in}{1.391203in}}%
\pgfpathlineto{\pgfqpoint{1.720630in}{1.396042in}}%
\pgfpathlineto{\pgfqpoint{1.727024in}{1.400881in}}%
\pgfpathlineto{\pgfqpoint{1.733418in}{1.405720in}}%
\pgfpathlineto{\pgfqpoint{1.739812in}{1.410560in}}%
\pgfpathlineto{\pgfqpoint{1.746207in}{1.415399in}}%
\pgfpathlineto{\pgfqpoint{1.752601in}{1.420238in}}%
\pgfpathlineto{\pgfqpoint{1.758995in}{1.425077in}}%
\pgfpathlineto{\pgfqpoint{1.765389in}{1.429917in}}%
\pgfpathlineto{\pgfqpoint{1.771784in}{1.434756in}}%
\pgfpathlineto{\pgfqpoint{1.778178in}{1.439595in}}%
\pgfpathlineto{\pgfqpoint{1.784572in}{1.444434in}}%
\pgfpathlineto{\pgfqpoint{1.790967in}{1.449274in}}%
\pgfpathlineto{\pgfqpoint{1.797361in}{1.454113in}}%
\pgfpathlineto{\pgfqpoint{1.803755in}{1.458952in}}%
\pgfpathlineto{\pgfqpoint{1.810149in}{1.463791in}}%
\pgfpathlineto{\pgfqpoint{1.816544in}{1.468631in}}%
\pgfpathlineto{\pgfqpoint{1.822938in}{1.473470in}}%
\pgfpathlineto{\pgfqpoint{1.829332in}{1.478309in}}%
\pgfpathlineto{\pgfqpoint{1.835727in}{1.483148in}}%
\pgfpathlineto{\pgfqpoint{1.842121in}{1.487988in}}%
\pgfpathlineto{\pgfqpoint{1.848515in}{1.492827in}}%
\pgfpathlineto{\pgfqpoint{1.854909in}{1.497666in}}%
\pgfpathlineto{\pgfqpoint{1.861304in}{1.502506in}}%
\pgfpathlineto{\pgfqpoint{1.867698in}{1.507345in}}%
\pgfpathlineto{\pgfqpoint{1.874092in}{1.512184in}}%
\pgfpathlineto{\pgfqpoint{1.880486in}{1.517023in}}%
\pgfpathlineto{\pgfqpoint{1.886881in}{1.521863in}}%
\pgfpathlineto{\pgfqpoint{1.893275in}{1.526702in}}%
\pgfpathlineto{\pgfqpoint{1.899669in}{1.531541in}}%
\pgfpathlineto{\pgfqpoint{1.906064in}{1.536380in}}%
\pgfpathlineto{\pgfqpoint{1.912458in}{1.541220in}}%
\pgfpathlineto{\pgfqpoint{1.918852in}{1.546059in}}%
\pgfpathlineto{\pgfqpoint{1.925246in}{1.550898in}}%
\pgfpathlineto{\pgfqpoint{1.931641in}{1.555737in}}%
\pgfpathlineto{\pgfqpoint{1.938035in}{1.560577in}}%
\pgfpathlineto{\pgfqpoint{1.944429in}{1.565416in}}%
\pgfpathlineto{\pgfqpoint{1.950824in}{1.570255in}}%
\pgfpathlineto{\pgfqpoint{1.957218in}{1.575094in}}%
\pgfpathlineto{\pgfqpoint{1.963612in}{1.579934in}}%
\pgfpathlineto{\pgfqpoint{1.970006in}{1.584773in}}%
\pgfpathlineto{\pgfqpoint{1.976401in}{1.589612in}}%
\pgfpathlineto{\pgfqpoint{1.982795in}{1.594451in}}%
\pgfpathlineto{\pgfqpoint{1.989189in}{1.599291in}}%
\pgfpathlineto{\pgfqpoint{1.995584in}{1.604130in}}%
\pgfpathlineto{\pgfqpoint{2.001978in}{1.608969in}}%
\pgfpathlineto{\pgfqpoint{2.008372in}{1.613808in}}%
\pgfpathlineto{\pgfqpoint{2.014766in}{1.618648in}}%
\pgfpathlineto{\pgfqpoint{2.021161in}{1.623487in}}%
\pgfpathlineto{\pgfqpoint{2.027555in}{1.628326in}}%
\pgfpathlineto{\pgfqpoint{2.033949in}{1.633165in}}%
\pgfpathlineto{\pgfqpoint{2.040343in}{1.638005in}}%
\pgfpathlineto{\pgfqpoint{2.046738in}{1.642844in}}%
\pgfpathlineto{\pgfqpoint{2.053132in}{1.647683in}}%
\pgfpathlineto{\pgfqpoint{2.059526in}{1.652522in}}%
\pgfpathlineto{\pgfqpoint{2.065921in}{1.657362in}}%
\pgfpathlineto{\pgfqpoint{2.072315in}{1.662201in}}%
\pgfpathlineto{\pgfqpoint{2.078709in}{1.667040in}}%
\pgfpathlineto{\pgfqpoint{2.085103in}{1.671879in}}%
\pgfpathlineto{\pgfqpoint{2.091498in}{1.676719in}}%
\pgfpathlineto{\pgfqpoint{2.097892in}{1.681558in}}%
\pgfpathlineto{\pgfqpoint{2.104286in}{1.686397in}}%
\pgfpathlineto{\pgfqpoint{2.110681in}{1.691236in}}%
\pgfpathlineto{\pgfqpoint{2.117075in}{1.696076in}}%
\pgfpathlineto{\pgfqpoint{2.123469in}{1.700915in}}%
\pgfpathlineto{\pgfqpoint{2.129863in}{1.705754in}}%
\pgfpathlineto{\pgfqpoint{2.136258in}{1.710593in}}%
\pgfpathlineto{\pgfqpoint{2.142652in}{1.715433in}}%
\pgfpathlineto{\pgfqpoint{2.149046in}{1.720272in}}%
\pgfpathlineto{\pgfqpoint{2.155441in}{1.725111in}}%
\pgfpathlineto{\pgfqpoint{2.161835in}{1.729950in}}%
\pgfpathlineto{\pgfqpoint{2.168229in}{1.734790in}}%
\pgfpathlineto{\pgfqpoint{2.174623in}{1.739629in}}%
\pgfpathlineto{\pgfqpoint{2.181018in}{1.744468in}}%
\pgfpathlineto{\pgfqpoint{2.187412in}{1.749307in}}%
\pgfpathlineto{\pgfqpoint{2.193806in}{1.754147in}}%
\pgfpathlineto{\pgfqpoint{2.200200in}{1.758986in}}%
\pgfpathlineto{\pgfqpoint{2.206595in}{1.763825in}}%
\pgfpathlineto{\pgfqpoint{2.212989in}{1.768664in}}%
\pgfpathlineto{\pgfqpoint{2.219383in}{1.773504in}}%
\pgfpathlineto{\pgfqpoint{2.225778in}{1.778343in}}%
\pgfpathlineto{\pgfqpoint{2.232172in}{1.783182in}}%
\pgfpathlineto{\pgfqpoint{2.238566in}{1.788021in}}%
\pgfpathlineto{\pgfqpoint{2.244960in}{1.792861in}}%
\pgfpathlineto{\pgfqpoint{2.251355in}{1.797700in}}%
\pgfpathlineto{\pgfqpoint{2.257749in}{1.802539in}}%
\pgfpathlineto{\pgfqpoint{2.264143in}{1.807378in}}%
\pgfpathlineto{\pgfqpoint{2.270538in}{1.812218in}}%
\pgfpathlineto{\pgfqpoint{2.276932in}{1.817057in}}%
\pgfpathlineto{\pgfqpoint{2.283326in}{1.821896in}}%
\pgfpathlineto{\pgfqpoint{2.289720in}{1.826735in}}%
\pgfpathlineto{\pgfqpoint{2.296115in}{1.831575in}}%
\pgfpathlineto{\pgfqpoint{2.302509in}{1.836414in}}%
\pgfpathlineto{\pgfqpoint{2.308903in}{1.841253in}}%
\pgfpathlineto{\pgfqpoint{2.315298in}{1.846093in}}%
\pgfpathlineto{\pgfqpoint{2.321692in}{1.850932in}}%
\pgfpathlineto{\pgfqpoint{2.328086in}{1.855771in}}%
\pgfpathlineto{\pgfqpoint{2.334480in}{1.860610in}}%
\pgfpathlineto{\pgfqpoint{2.340875in}{1.865450in}}%
\pgfpathlineto{\pgfqpoint{2.347269in}{1.870289in}}%
\pgfpathlineto{\pgfqpoint{2.353663in}{1.875128in}}%
\pgfpathlineto{\pgfqpoint{2.360057in}{1.879967in}}%
\pgfpathlineto{\pgfqpoint{2.366452in}{1.884807in}}%
\pgfpathlineto{\pgfqpoint{2.372846in}{1.889646in}}%
\pgfpathlineto{\pgfqpoint{2.379240in}{1.894485in}}%
\pgfpathlineto{\pgfqpoint{2.385635in}{1.899324in}}%
\pgfpathlineto{\pgfqpoint{2.392029in}{1.904164in}}%
\pgfpathlineto{\pgfqpoint{2.398423in}{1.909003in}}%
\pgfpathlineto{\pgfqpoint{2.404817in}{1.913842in}}%
\pgfpathlineto{\pgfqpoint{2.411212in}{1.918681in}}%
\pgfpathlineto{\pgfqpoint{2.417606in}{1.923521in}}%
\pgfpathlineto{\pgfqpoint{2.424000in}{1.928360in}}%
\pgfpathlineto{\pgfqpoint{2.430395in}{1.933199in}}%
\pgfpathlineto{\pgfqpoint{2.436789in}{1.938038in}}%
\pgfpathlineto{\pgfqpoint{2.443183in}{1.942878in}}%
\pgfpathlineto{\pgfqpoint{2.449577in}{1.947717in}}%
\pgfpathlineto{\pgfqpoint{2.455972in}{1.952556in}}%
\pgfpathlineto{\pgfqpoint{2.462366in}{1.957395in}}%
\pgfpathlineto{\pgfqpoint{2.468760in}{1.962235in}}%
\pgfpathlineto{\pgfqpoint{2.475154in}{1.967074in}}%
\pgfpathlineto{\pgfqpoint{2.481549in}{1.971913in}}%
\pgfpathlineto{\pgfqpoint{2.487943in}{1.976752in}}%
\pgfpathlineto{\pgfqpoint{2.494337in}{1.981592in}}%
\pgfpathlineto{\pgfqpoint{2.500732in}{1.986431in}}%
\pgfpathlineto{\pgfqpoint{2.507126in}{1.991270in}}%
\pgfpathlineto{\pgfqpoint{2.513520in}{1.996109in}}%
\pgfpathlineto{\pgfqpoint{2.519914in}{2.000949in}}%
\pgfpathlineto{\pgfqpoint{2.526309in}{2.005788in}}%
\pgfpathlineto{\pgfqpoint{2.532703in}{2.010627in}}%
\pgfpathlineto{\pgfqpoint{2.539097in}{2.015466in}}%
\pgfpathlineto{\pgfqpoint{2.545492in}{2.020306in}}%
\pgfpathlineto{\pgfqpoint{2.551886in}{2.025145in}}%
\pgfpathlineto{\pgfqpoint{2.558280in}{2.029984in}}%
\pgfpathlineto{\pgfqpoint{2.564674in}{2.034823in}}%
\pgfpathlineto{\pgfqpoint{2.571069in}{2.039663in}}%
\pgfpathlineto{\pgfqpoint{2.577463in}{2.044502in}}%
\pgfpathlineto{\pgfqpoint{2.583857in}{2.049341in}}%
\pgfpathlineto{\pgfqpoint{2.590252in}{2.054180in}}%
\pgfpathlineto{\pgfqpoint{2.596646in}{2.059020in}}%
\pgfpathlineto{\pgfqpoint{2.603040in}{2.063859in}}%
\pgfpathlineto{\pgfqpoint{2.609434in}{2.068698in}}%
\pgfpathlineto{\pgfqpoint{2.615829in}{2.073537in}}%
\pgfpathlineto{\pgfqpoint{2.622223in}{2.078377in}}%
\pgfpathlineto{\pgfqpoint{2.628617in}{2.083216in}}%
\pgfpathlineto{\pgfqpoint{2.635011in}{2.088055in}}%
\pgfpathlineto{\pgfqpoint{2.641406in}{2.092894in}}%
\pgfpathlineto{\pgfqpoint{2.647800in}{2.097734in}}%
\pgfpathlineto{\pgfqpoint{2.654194in}{2.102573in}}%
\pgfpathlineto{\pgfqpoint{2.660589in}{2.107412in}}%
\pgfpathlineto{\pgfqpoint{2.666983in}{2.112251in}}%
\pgfpathlineto{\pgfqpoint{2.673377in}{2.117091in}}%
\pgfpathlineto{\pgfqpoint{2.679771in}{2.121930in}}%
\pgfpathlineto{\pgfqpoint{2.686166in}{2.126769in}}%
\pgfpathlineto{\pgfqpoint{2.692560in}{2.131608in}}%
\pgfpathlineto{\pgfqpoint{2.698954in}{2.136448in}}%
\pgfpathlineto{\pgfqpoint{2.705349in}{2.141287in}}%
\pgfpathlineto{\pgfqpoint{2.711743in}{2.146126in}}%
\pgfpathlineto{\pgfqpoint{2.718137in}{2.150965in}}%
\pgfpathlineto{\pgfqpoint{2.724531in}{2.155805in}}%
\pgfpathlineto{\pgfqpoint{2.730926in}{2.160644in}}%
\pgfpathlineto{\pgfqpoint{2.737320in}{2.165483in}}%
\pgfpathlineto{\pgfqpoint{2.743714in}{2.170322in}}%
\pgfpathlineto{\pgfqpoint{2.750109in}{2.175162in}}%
\pgfpathlineto{\pgfqpoint{2.756503in}{2.180001in}}%
\pgfpathlineto{\pgfqpoint{2.762897in}{2.184840in}}%
\pgfpathlineto{\pgfqpoint{2.769291in}{2.189680in}}%
\pgfpathlineto{\pgfqpoint{2.775686in}{2.194519in}}%
\pgfpathlineto{\pgfqpoint{2.782080in}{2.199358in}}%
\pgfpathlineto{\pgfqpoint{2.788474in}{2.204197in}}%
\pgfpathlineto{\pgfqpoint{2.794868in}{2.209037in}}%
\pgfpathlineto{\pgfqpoint{2.801263in}{2.213876in}}%
\pgfpathlineto{\pgfqpoint{2.807657in}{2.218715in}}%
\pgfpathlineto{\pgfqpoint{2.814051in}{2.223554in}}%
\pgfpathlineto{\pgfqpoint{2.820446in}{2.228394in}}%
\pgfpathlineto{\pgfqpoint{2.826840in}{2.233233in}}%
\pgfpathlineto{\pgfqpoint{2.833234in}{2.238072in}}%
\pgfpathlineto{\pgfqpoint{2.839628in}{2.242911in}}%
\pgfpathlineto{\pgfqpoint{2.846023in}{2.247751in}}%
\pgfpathlineto{\pgfqpoint{2.852417in}{2.252590in}}%
\pgfpathlineto{\pgfqpoint{2.858811in}{2.257429in}}%
\pgfpathlineto{\pgfqpoint{2.865206in}{2.262268in}}%
\pgfpathlineto{\pgfqpoint{2.871600in}{2.267108in}}%
\pgfpathlineto{\pgfqpoint{2.877994in}{2.271947in}}%
\pgfpathlineto{\pgfqpoint{2.884388in}{2.276786in}}%
\pgfpathlineto{\pgfqpoint{2.890783in}{2.281625in}}%
\pgfpathlineto{\pgfqpoint{2.897177in}{2.286465in}}%
\pgfpathlineto{\pgfqpoint{2.903571in}{2.291304in}}%
\pgfpathlineto{\pgfqpoint{2.909966in}{2.296143in}}%
\pgfpathlineto{\pgfqpoint{2.916360in}{2.300982in}}%
\pgfpathlineto{\pgfqpoint{2.922754in}{2.305822in}}%
\pgfpathlineto{\pgfqpoint{2.929148in}{2.310661in}}%
\pgfpathlineto{\pgfqpoint{2.935543in}{2.315500in}}%
\pgfpathlineto{\pgfqpoint{2.941937in}{2.320339in}}%
\pgfpathlineto{\pgfqpoint{2.948331in}{2.325179in}}%
\pgfpathlineto{\pgfqpoint{2.954725in}{2.330018in}}%
\pgfpathlineto{\pgfqpoint{2.961120in}{2.334857in}}%
\pgfpathlineto{\pgfqpoint{2.967514in}{2.339696in}}%
\pgfpathlineto{\pgfqpoint{2.973908in}{2.344536in}}%
\pgfpathlineto{\pgfqpoint{2.980303in}{2.349375in}}%
\pgfpathlineto{\pgfqpoint{2.986697in}{2.354214in}}%
\pgfpathlineto{\pgfqpoint{2.993091in}{2.359053in}}%
\pgfpathlineto{\pgfqpoint{2.999485in}{2.363893in}}%
\pgfpathlineto{\pgfqpoint{3.005880in}{2.368732in}}%
\pgfpathlineto{\pgfqpoint{3.012274in}{2.373571in}}%
\pgfpathlineto{\pgfqpoint{3.018668in}{2.378410in}}%
\pgfpathlineto{\pgfqpoint{3.025063in}{2.383250in}}%
\pgfpathlineto{\pgfqpoint{3.031457in}{2.388089in}}%
\pgfpathlineto{\pgfqpoint{3.037851in}{2.392928in}}%
\pgfpathlineto{\pgfqpoint{3.044245in}{2.397767in}}%
\pgfpathlineto{\pgfqpoint{3.050640in}{2.402607in}}%
\pgfpathlineto{\pgfqpoint{3.057034in}{2.407446in}}%
\pgfpathlineto{\pgfqpoint{3.063428in}{2.412285in}}%
\pgfpathlineto{\pgfqpoint{3.069822in}{2.417124in}}%
\pgfpathlineto{\pgfqpoint{3.076217in}{2.421964in}}%
\pgfpathlineto{\pgfqpoint{3.082611in}{2.426803in}}%
\pgfpathlineto{\pgfqpoint{3.089005in}{2.431642in}}%
\pgfpathlineto{\pgfqpoint{3.095400in}{2.436481in}}%
\pgfpathlineto{\pgfqpoint{3.101794in}{2.441321in}}%
\pgfpathlineto{\pgfqpoint{3.108188in}{2.446160in}}%
\pgfpathlineto{\pgfqpoint{3.114582in}{2.450999in}}%
\pgfpathlineto{\pgfqpoint{3.120977in}{2.455838in}}%
\pgfpathlineto{\pgfqpoint{3.127371in}{2.460678in}}%
\pgfpathlineto{\pgfqpoint{3.133765in}{2.465517in}}%
\pgfpathlineto{\pgfqpoint{3.140160in}{2.470356in}}%
\pgfpathlineto{\pgfqpoint{3.146554in}{2.475195in}}%
\pgfpathlineto{\pgfqpoint{3.152948in}{2.480035in}}%
\pgfpathlineto{\pgfqpoint{3.159342in}{2.484874in}}%
\pgfpathlineto{\pgfqpoint{3.165737in}{2.489713in}}%
\pgfpathlineto{\pgfqpoint{3.172131in}{2.494552in}}%
\pgfpathlineto{\pgfqpoint{3.178525in}{2.499392in}}%
\pgfpathlineto{\pgfqpoint{3.184920in}{2.504231in}}%
\pgfpathlineto{\pgfqpoint{3.191314in}{2.509070in}}%
\pgfpathlineto{\pgfqpoint{3.197708in}{2.513909in}}%
\pgfpathlineto{\pgfqpoint{3.204102in}{2.518749in}}%
\pgfpathlineto{\pgfqpoint{3.210497in}{2.523588in}}%
\pgfpathlineto{\pgfqpoint{3.216891in}{2.528427in}}%
\pgfpathlineto{\pgfqpoint{3.223285in}{2.533267in}}%
\pgfpathlineto{\pgfqpoint{3.229679in}{2.538106in}}%
\pgfpathlineto{\pgfqpoint{3.236074in}{2.542945in}}%
\pgfpathlineto{\pgfqpoint{3.242468in}{2.547784in}}%
\pgfpathlineto{\pgfqpoint{3.248862in}{2.552624in}}%
\pgfpathlineto{\pgfqpoint{3.255257in}{2.557463in}}%
\pgfpathlineto{\pgfqpoint{3.261651in}{2.562302in}}%
\pgfpathlineto{\pgfqpoint{3.268045in}{2.567141in}}%
\pgfpathlineto{\pgfqpoint{3.274439in}{2.571981in}}%
\pgfpathlineto{\pgfqpoint{3.280834in}{2.576820in}}%
\pgfpathlineto{\pgfqpoint{3.287228in}{2.581659in}}%
\pgfpathlineto{\pgfqpoint{3.293622in}{2.586498in}}%
\pgfpathlineto{\pgfqpoint{3.300017in}{2.591338in}}%
\pgfpathlineto{\pgfqpoint{3.306411in}{2.596177in}}%
\pgfpathlineto{\pgfqpoint{3.312805in}{2.601016in}}%
\pgfpathlineto{\pgfqpoint{3.319199in}{2.605855in}}%
\pgfpathlineto{\pgfqpoint{3.325594in}{2.610695in}}%
\pgfpathlineto{\pgfqpoint{3.331988in}{2.615534in}}%
\pgfpathlineto{\pgfqpoint{3.338382in}{2.620373in}}%
\pgfpathlineto{\pgfqpoint{3.344777in}{2.625212in}}%
\pgfpathlineto{\pgfqpoint{3.351171in}{2.630052in}}%
\pgfpathlineto{\pgfqpoint{3.357565in}{2.634891in}}%
\pgfpathlineto{\pgfqpoint{3.363959in}{2.639730in}}%
\pgfpathlineto{\pgfqpoint{3.370354in}{2.644569in}}%
\pgfpathlineto{\pgfqpoint{3.376748in}{2.649409in}}%
\pgfpathlineto{\pgfqpoint{3.383142in}{2.654248in}}%
\pgfpathlineto{\pgfqpoint{3.389536in}{2.659087in}}%
\pgfpathlineto{\pgfqpoint{3.395931in}{2.663926in}}%
\pgfpathlineto{\pgfqpoint{3.402325in}{2.668766in}}%
\pgfpathlineto{\pgfqpoint{3.408719in}{2.673605in}}%
\pgfpathlineto{\pgfqpoint{3.415114in}{2.678444in}}%
\pgfpathlineto{\pgfqpoint{3.421508in}{2.683283in}}%
\pgfpathlineto{\pgfqpoint{3.427902in}{2.688123in}}%
\pgfpathlineto{\pgfqpoint{3.434296in}{2.692962in}}%
\pgfpathlineto{\pgfqpoint{3.440691in}{2.697801in}}%
\pgfpathlineto{\pgfqpoint{3.447085in}{2.702640in}}%
\pgfpathlineto{\pgfqpoint{3.453479in}{2.707480in}}%
\pgfpathlineto{\pgfqpoint{3.459874in}{2.712319in}}%
\pgfpathlineto{\pgfqpoint{3.466268in}{2.717158in}}%
\pgfpathlineto{\pgfqpoint{3.472662in}{2.721997in}}%
\pgfpathlineto{\pgfqpoint{3.479056in}{2.726837in}}%
\pgfpathlineto{\pgfqpoint{3.485451in}{2.731676in}}%
\pgfpathlineto{\pgfqpoint{3.491845in}{2.736515in}}%
\pgfpathlineto{\pgfqpoint{3.498239in}{2.741354in}}%
\pgfpathlineto{\pgfqpoint{3.504634in}{2.746194in}}%
\pgfpathlineto{\pgfqpoint{3.511028in}{2.751033in}}%
\pgfpathlineto{\pgfqpoint{3.517422in}{2.755872in}}%
\pgfpathlineto{\pgfqpoint{3.523816in}{2.760711in}}%
\pgfpathlineto{\pgfqpoint{3.530211in}{2.765551in}}%
\pgfpathlineto{\pgfqpoint{3.536605in}{2.770390in}}%
\pgfpathlineto{\pgfqpoint{3.542999in}{2.775229in}}%
\pgfpathlineto{\pgfqpoint{3.549393in}{2.780068in}}%
\pgfpathlineto{\pgfqpoint{3.555788in}{2.784908in}}%
\pgfpathlineto{\pgfqpoint{3.562182in}{2.789747in}}%
\pgfpathlineto{\pgfqpoint{3.568576in}{2.794586in}}%
\pgfpathlineto{\pgfqpoint{3.574971in}{2.799425in}}%
\pgfpathlineto{\pgfqpoint{3.581365in}{2.804265in}}%
\pgfpathlineto{\pgfqpoint{3.587759in}{2.809104in}}%
\pgfpathlineto{\pgfqpoint{3.594153in}{2.813943in}}%
\pgfpathlineto{\pgfqpoint{3.600548in}{2.818782in}}%
\pgfpathlineto{\pgfqpoint{3.606942in}{2.823622in}}%
\pgfpathlineto{\pgfqpoint{3.613336in}{2.828461in}}%
\pgfpathlineto{\pgfqpoint{3.619731in}{2.833300in}}%
\pgfpathlineto{\pgfqpoint{3.626125in}{2.838139in}}%
\pgfpathlineto{\pgfqpoint{3.632519in}{2.842979in}}%
\pgfpathlineto{\pgfqpoint{3.638913in}{2.847818in}}%
\pgfpathlineto{\pgfqpoint{3.645308in}{2.852657in}}%
\pgfpathlineto{\pgfqpoint{3.651702in}{2.857496in}}%
\pgfpathlineto{\pgfqpoint{3.658096in}{2.862336in}}%
\pgfpathlineto{\pgfqpoint{3.664490in}{2.867175in}}%
\pgfpathlineto{\pgfqpoint{3.670885in}{2.872014in}}%
\pgfpathlineto{\pgfqpoint{3.677279in}{2.876854in}}%
\pgfpathlineto{\pgfqpoint{3.683673in}{2.881693in}}%
\pgfpathlineto{\pgfqpoint{3.690068in}{2.886532in}}%
\pgfpathlineto{\pgfqpoint{3.696462in}{2.891371in}}%
\pgfpathlineto{\pgfqpoint{3.702856in}{2.896211in}}%
\pgfpathlineto{\pgfqpoint{3.709250in}{2.901050in}}%
\pgfpathlineto{\pgfqpoint{3.715645in}{2.905889in}}%
\pgfpathlineto{\pgfqpoint{3.722039in}{2.910728in}}%
\pgfpathlineto{\pgfqpoint{3.728433in}{2.915568in}}%
\pgfpathlineto{\pgfqpoint{3.734828in}{2.920407in}}%
\pgfpathlineto{\pgfqpoint{3.741222in}{2.925246in}}%
\pgfpathlineto{\pgfqpoint{3.747616in}{2.930085in}}%
\pgfpathlineto{\pgfqpoint{3.754010in}{2.934925in}}%
\pgfpathlineto{\pgfqpoint{3.760405in}{2.939764in}}%
\pgfpathlineto{\pgfqpoint{3.766799in}{2.944603in}}%
\pgfpathlineto{\pgfqpoint{3.773193in}{2.949442in}}%
\pgfpathlineto{\pgfqpoint{3.779588in}{2.954282in}}%
\pgfpathlineto{\pgfqpoint{3.785982in}{2.959121in}}%
\pgfpathlineto{\pgfqpoint{3.792376in}{2.963960in}}%
\pgfpathlineto{\pgfqpoint{3.798770in}{2.968799in}}%
\pgfpathlineto{\pgfqpoint{3.798770in}{2.971219in}}%
\pgfpathlineto{\pgfqpoint{3.798770in}{2.971219in}}%
\pgfpathlineto{\pgfqpoint{3.792376in}{2.971219in}}%
\pgfpathlineto{\pgfqpoint{3.785982in}{2.971219in}}%
\pgfpathlineto{\pgfqpoint{3.779588in}{2.971219in}}%
\pgfpathlineto{\pgfqpoint{3.773193in}{2.971219in}}%
\pgfpathlineto{\pgfqpoint{3.766799in}{2.971219in}}%
\pgfpathlineto{\pgfqpoint{3.760405in}{2.971219in}}%
\pgfpathlineto{\pgfqpoint{3.754010in}{2.971219in}}%
\pgfpathlineto{\pgfqpoint{3.747616in}{2.971219in}}%
\pgfpathlineto{\pgfqpoint{3.741222in}{2.971219in}}%
\pgfpathlineto{\pgfqpoint{3.734828in}{2.971219in}}%
\pgfpathlineto{\pgfqpoint{3.728433in}{2.971219in}}%
\pgfpathlineto{\pgfqpoint{3.722039in}{2.971219in}}%
\pgfpathlineto{\pgfqpoint{3.715645in}{2.971219in}}%
\pgfpathlineto{\pgfqpoint{3.709250in}{2.971219in}}%
\pgfpathlineto{\pgfqpoint{3.702856in}{2.971219in}}%
\pgfpathlineto{\pgfqpoint{3.696462in}{2.971219in}}%
\pgfpathlineto{\pgfqpoint{3.690068in}{2.971219in}}%
\pgfpathlineto{\pgfqpoint{3.683673in}{2.971219in}}%
\pgfpathlineto{\pgfqpoint{3.677279in}{2.971219in}}%
\pgfpathlineto{\pgfqpoint{3.670885in}{2.971219in}}%
\pgfpathlineto{\pgfqpoint{3.664490in}{2.971219in}}%
\pgfpathlineto{\pgfqpoint{3.658096in}{2.971219in}}%
\pgfpathlineto{\pgfqpoint{3.651702in}{2.971219in}}%
\pgfpathlineto{\pgfqpoint{3.645308in}{2.971219in}}%
\pgfpathlineto{\pgfqpoint{3.638913in}{2.971219in}}%
\pgfpathlineto{\pgfqpoint{3.632519in}{2.971219in}}%
\pgfpathlineto{\pgfqpoint{3.626125in}{2.971219in}}%
\pgfpathlineto{\pgfqpoint{3.619731in}{2.971219in}}%
\pgfpathlineto{\pgfqpoint{3.613336in}{2.971219in}}%
\pgfpathlineto{\pgfqpoint{3.606942in}{2.971219in}}%
\pgfpathlineto{\pgfqpoint{3.600548in}{2.971219in}}%
\pgfpathlineto{\pgfqpoint{3.594153in}{2.971219in}}%
\pgfpathlineto{\pgfqpoint{3.587759in}{2.971219in}}%
\pgfpathlineto{\pgfqpoint{3.581365in}{2.971219in}}%
\pgfpathlineto{\pgfqpoint{3.574971in}{2.971219in}}%
\pgfpathlineto{\pgfqpoint{3.568576in}{2.971219in}}%
\pgfpathlineto{\pgfqpoint{3.562182in}{2.971219in}}%
\pgfpathlineto{\pgfqpoint{3.555788in}{2.971219in}}%
\pgfpathlineto{\pgfqpoint{3.549393in}{2.971219in}}%
\pgfpathlineto{\pgfqpoint{3.542999in}{2.971219in}}%
\pgfpathlineto{\pgfqpoint{3.536605in}{2.971219in}}%
\pgfpathlineto{\pgfqpoint{3.530211in}{2.971219in}}%
\pgfpathlineto{\pgfqpoint{3.523816in}{2.971219in}}%
\pgfpathlineto{\pgfqpoint{3.517422in}{2.971219in}}%
\pgfpathlineto{\pgfqpoint{3.511028in}{2.971219in}}%
\pgfpathlineto{\pgfqpoint{3.504634in}{2.971219in}}%
\pgfpathlineto{\pgfqpoint{3.498239in}{2.971219in}}%
\pgfpathlineto{\pgfqpoint{3.491845in}{2.971219in}}%
\pgfpathlineto{\pgfqpoint{3.485451in}{2.971219in}}%
\pgfpathlineto{\pgfqpoint{3.479056in}{2.971219in}}%
\pgfpathlineto{\pgfqpoint{3.472662in}{2.971219in}}%
\pgfpathlineto{\pgfqpoint{3.466268in}{2.971219in}}%
\pgfpathlineto{\pgfqpoint{3.459874in}{2.971219in}}%
\pgfpathlineto{\pgfqpoint{3.453479in}{2.971219in}}%
\pgfpathlineto{\pgfqpoint{3.447085in}{2.971219in}}%
\pgfpathlineto{\pgfqpoint{3.440691in}{2.971219in}}%
\pgfpathlineto{\pgfqpoint{3.434296in}{2.971219in}}%
\pgfpathlineto{\pgfqpoint{3.427902in}{2.971219in}}%
\pgfpathlineto{\pgfqpoint{3.421508in}{2.971219in}}%
\pgfpathlineto{\pgfqpoint{3.415114in}{2.971219in}}%
\pgfpathlineto{\pgfqpoint{3.408719in}{2.971219in}}%
\pgfpathlineto{\pgfqpoint{3.402325in}{2.971219in}}%
\pgfpathlineto{\pgfqpoint{3.395931in}{2.971219in}}%
\pgfpathlineto{\pgfqpoint{3.389536in}{2.971219in}}%
\pgfpathlineto{\pgfqpoint{3.383142in}{2.971219in}}%
\pgfpathlineto{\pgfqpoint{3.376748in}{2.971219in}}%
\pgfpathlineto{\pgfqpoint{3.370354in}{2.971219in}}%
\pgfpathlineto{\pgfqpoint{3.363959in}{2.971219in}}%
\pgfpathlineto{\pgfqpoint{3.357565in}{2.971219in}}%
\pgfpathlineto{\pgfqpoint{3.351171in}{2.971219in}}%
\pgfpathlineto{\pgfqpoint{3.344777in}{2.971219in}}%
\pgfpathlineto{\pgfqpoint{3.338382in}{2.971219in}}%
\pgfpathlineto{\pgfqpoint{3.331988in}{2.971219in}}%
\pgfpathlineto{\pgfqpoint{3.325594in}{2.971219in}}%
\pgfpathlineto{\pgfqpoint{3.319199in}{2.971219in}}%
\pgfpathlineto{\pgfqpoint{3.312805in}{2.971219in}}%
\pgfpathlineto{\pgfqpoint{3.306411in}{2.971219in}}%
\pgfpathlineto{\pgfqpoint{3.300017in}{2.971219in}}%
\pgfpathlineto{\pgfqpoint{3.293622in}{2.971219in}}%
\pgfpathlineto{\pgfqpoint{3.287228in}{2.971219in}}%
\pgfpathlineto{\pgfqpoint{3.280834in}{2.971219in}}%
\pgfpathlineto{\pgfqpoint{3.274439in}{2.971219in}}%
\pgfpathlineto{\pgfqpoint{3.268045in}{2.971219in}}%
\pgfpathlineto{\pgfqpoint{3.261651in}{2.971219in}}%
\pgfpathlineto{\pgfqpoint{3.255257in}{2.971219in}}%
\pgfpathlineto{\pgfqpoint{3.248862in}{2.971219in}}%
\pgfpathlineto{\pgfqpoint{3.242468in}{2.971219in}}%
\pgfpathlineto{\pgfqpoint{3.236074in}{2.971219in}}%
\pgfpathlineto{\pgfqpoint{3.229679in}{2.971219in}}%
\pgfpathlineto{\pgfqpoint{3.223285in}{2.971219in}}%
\pgfpathlineto{\pgfqpoint{3.216891in}{2.971219in}}%
\pgfpathlineto{\pgfqpoint{3.210497in}{2.971219in}}%
\pgfpathlineto{\pgfqpoint{3.204102in}{2.971219in}}%
\pgfpathlineto{\pgfqpoint{3.197708in}{2.971219in}}%
\pgfpathlineto{\pgfqpoint{3.191314in}{2.971219in}}%
\pgfpathlineto{\pgfqpoint{3.184920in}{2.971219in}}%
\pgfpathlineto{\pgfqpoint{3.178525in}{2.971219in}}%
\pgfpathlineto{\pgfqpoint{3.172131in}{2.971219in}}%
\pgfpathlineto{\pgfqpoint{3.165737in}{2.971219in}}%
\pgfpathlineto{\pgfqpoint{3.159342in}{2.971219in}}%
\pgfpathlineto{\pgfqpoint{3.152948in}{2.971219in}}%
\pgfpathlineto{\pgfqpoint{3.146554in}{2.971219in}}%
\pgfpathlineto{\pgfqpoint{3.140160in}{2.971219in}}%
\pgfpathlineto{\pgfqpoint{3.133765in}{2.971219in}}%
\pgfpathlineto{\pgfqpoint{3.127371in}{2.971219in}}%
\pgfpathlineto{\pgfqpoint{3.120977in}{2.971219in}}%
\pgfpathlineto{\pgfqpoint{3.114582in}{2.971219in}}%
\pgfpathlineto{\pgfqpoint{3.108188in}{2.971219in}}%
\pgfpathlineto{\pgfqpoint{3.101794in}{2.971219in}}%
\pgfpathlineto{\pgfqpoint{3.095400in}{2.971219in}}%
\pgfpathlineto{\pgfqpoint{3.089005in}{2.971219in}}%
\pgfpathlineto{\pgfqpoint{3.082611in}{2.971219in}}%
\pgfpathlineto{\pgfqpoint{3.076217in}{2.971219in}}%
\pgfpathlineto{\pgfqpoint{3.069822in}{2.971219in}}%
\pgfpathlineto{\pgfqpoint{3.063428in}{2.971219in}}%
\pgfpathlineto{\pgfqpoint{3.057034in}{2.971219in}}%
\pgfpathlineto{\pgfqpoint{3.050640in}{2.971219in}}%
\pgfpathlineto{\pgfqpoint{3.044245in}{2.971219in}}%
\pgfpathlineto{\pgfqpoint{3.037851in}{2.971219in}}%
\pgfpathlineto{\pgfqpoint{3.031457in}{2.971219in}}%
\pgfpathlineto{\pgfqpoint{3.025063in}{2.971219in}}%
\pgfpathlineto{\pgfqpoint{3.018668in}{2.971219in}}%
\pgfpathlineto{\pgfqpoint{3.012274in}{2.971219in}}%
\pgfpathlineto{\pgfqpoint{3.005880in}{2.971219in}}%
\pgfpathlineto{\pgfqpoint{2.999485in}{2.971219in}}%
\pgfpathlineto{\pgfqpoint{2.993091in}{2.971219in}}%
\pgfpathlineto{\pgfqpoint{2.986697in}{2.971219in}}%
\pgfpathlineto{\pgfqpoint{2.980303in}{2.971219in}}%
\pgfpathlineto{\pgfqpoint{2.973908in}{2.971219in}}%
\pgfpathlineto{\pgfqpoint{2.967514in}{2.971219in}}%
\pgfpathlineto{\pgfqpoint{2.961120in}{2.971219in}}%
\pgfpathlineto{\pgfqpoint{2.954725in}{2.971219in}}%
\pgfpathlineto{\pgfqpoint{2.948331in}{2.971219in}}%
\pgfpathlineto{\pgfqpoint{2.941937in}{2.971219in}}%
\pgfpathlineto{\pgfqpoint{2.935543in}{2.971219in}}%
\pgfpathlineto{\pgfqpoint{2.929148in}{2.971219in}}%
\pgfpathlineto{\pgfqpoint{2.922754in}{2.971219in}}%
\pgfpathlineto{\pgfqpoint{2.916360in}{2.971219in}}%
\pgfpathlineto{\pgfqpoint{2.909966in}{2.971219in}}%
\pgfpathlineto{\pgfqpoint{2.903571in}{2.971219in}}%
\pgfpathlineto{\pgfqpoint{2.897177in}{2.971219in}}%
\pgfpathlineto{\pgfqpoint{2.890783in}{2.971219in}}%
\pgfpathlineto{\pgfqpoint{2.884388in}{2.971219in}}%
\pgfpathlineto{\pgfqpoint{2.877994in}{2.971219in}}%
\pgfpathlineto{\pgfqpoint{2.871600in}{2.971219in}}%
\pgfpathlineto{\pgfqpoint{2.865206in}{2.971219in}}%
\pgfpathlineto{\pgfqpoint{2.858811in}{2.971219in}}%
\pgfpathlineto{\pgfqpoint{2.852417in}{2.971219in}}%
\pgfpathlineto{\pgfqpoint{2.846023in}{2.971219in}}%
\pgfpathlineto{\pgfqpoint{2.839628in}{2.971219in}}%
\pgfpathlineto{\pgfqpoint{2.833234in}{2.971219in}}%
\pgfpathlineto{\pgfqpoint{2.826840in}{2.971219in}}%
\pgfpathlineto{\pgfqpoint{2.820446in}{2.971219in}}%
\pgfpathlineto{\pgfqpoint{2.814051in}{2.971219in}}%
\pgfpathlineto{\pgfqpoint{2.807657in}{2.971219in}}%
\pgfpathlineto{\pgfqpoint{2.801263in}{2.971219in}}%
\pgfpathlineto{\pgfqpoint{2.794868in}{2.971219in}}%
\pgfpathlineto{\pgfqpoint{2.788474in}{2.971219in}}%
\pgfpathlineto{\pgfqpoint{2.782080in}{2.971219in}}%
\pgfpathlineto{\pgfqpoint{2.775686in}{2.971219in}}%
\pgfpathlineto{\pgfqpoint{2.769291in}{2.971219in}}%
\pgfpathlineto{\pgfqpoint{2.762897in}{2.971219in}}%
\pgfpathlineto{\pgfqpoint{2.756503in}{2.971219in}}%
\pgfpathlineto{\pgfqpoint{2.750109in}{2.971219in}}%
\pgfpathlineto{\pgfqpoint{2.743714in}{2.971219in}}%
\pgfpathlineto{\pgfqpoint{2.737320in}{2.971219in}}%
\pgfpathlineto{\pgfqpoint{2.730926in}{2.971219in}}%
\pgfpathlineto{\pgfqpoint{2.724531in}{2.971219in}}%
\pgfpathlineto{\pgfqpoint{2.718137in}{2.971219in}}%
\pgfpathlineto{\pgfqpoint{2.711743in}{2.971219in}}%
\pgfpathlineto{\pgfqpoint{2.705349in}{2.971219in}}%
\pgfpathlineto{\pgfqpoint{2.698954in}{2.971219in}}%
\pgfpathlineto{\pgfqpoint{2.692560in}{2.971219in}}%
\pgfpathlineto{\pgfqpoint{2.686166in}{2.971219in}}%
\pgfpathlineto{\pgfqpoint{2.679771in}{2.971219in}}%
\pgfpathlineto{\pgfqpoint{2.673377in}{2.971219in}}%
\pgfpathlineto{\pgfqpoint{2.666983in}{2.971219in}}%
\pgfpathlineto{\pgfqpoint{2.660589in}{2.971219in}}%
\pgfpathlineto{\pgfqpoint{2.654194in}{2.971219in}}%
\pgfpathlineto{\pgfqpoint{2.647800in}{2.971219in}}%
\pgfpathlineto{\pgfqpoint{2.641406in}{2.971219in}}%
\pgfpathlineto{\pgfqpoint{2.635011in}{2.971219in}}%
\pgfpathlineto{\pgfqpoint{2.628617in}{2.971219in}}%
\pgfpathlineto{\pgfqpoint{2.622223in}{2.971219in}}%
\pgfpathlineto{\pgfqpoint{2.615829in}{2.971219in}}%
\pgfpathlineto{\pgfqpoint{2.609434in}{2.971219in}}%
\pgfpathlineto{\pgfqpoint{2.603040in}{2.971219in}}%
\pgfpathlineto{\pgfqpoint{2.596646in}{2.971219in}}%
\pgfpathlineto{\pgfqpoint{2.590252in}{2.971219in}}%
\pgfpathlineto{\pgfqpoint{2.583857in}{2.971219in}}%
\pgfpathlineto{\pgfqpoint{2.577463in}{2.971219in}}%
\pgfpathlineto{\pgfqpoint{2.571069in}{2.971219in}}%
\pgfpathlineto{\pgfqpoint{2.564674in}{2.971219in}}%
\pgfpathlineto{\pgfqpoint{2.558280in}{2.971219in}}%
\pgfpathlineto{\pgfqpoint{2.551886in}{2.971219in}}%
\pgfpathlineto{\pgfqpoint{2.545492in}{2.971219in}}%
\pgfpathlineto{\pgfqpoint{2.539097in}{2.971219in}}%
\pgfpathlineto{\pgfqpoint{2.532703in}{2.971219in}}%
\pgfpathlineto{\pgfqpoint{2.526309in}{2.971219in}}%
\pgfpathlineto{\pgfqpoint{2.519914in}{2.971219in}}%
\pgfpathlineto{\pgfqpoint{2.513520in}{2.971219in}}%
\pgfpathlineto{\pgfqpoint{2.507126in}{2.971219in}}%
\pgfpathlineto{\pgfqpoint{2.500732in}{2.971219in}}%
\pgfpathlineto{\pgfqpoint{2.494337in}{2.971219in}}%
\pgfpathlineto{\pgfqpoint{2.487943in}{2.971219in}}%
\pgfpathlineto{\pgfqpoint{2.481549in}{2.971219in}}%
\pgfpathlineto{\pgfqpoint{2.475154in}{2.971219in}}%
\pgfpathlineto{\pgfqpoint{2.468760in}{2.971219in}}%
\pgfpathlineto{\pgfqpoint{2.462366in}{2.971219in}}%
\pgfpathlineto{\pgfqpoint{2.455972in}{2.971219in}}%
\pgfpathlineto{\pgfqpoint{2.449577in}{2.971219in}}%
\pgfpathlineto{\pgfqpoint{2.443183in}{2.971219in}}%
\pgfpathlineto{\pgfqpoint{2.436789in}{2.971219in}}%
\pgfpathlineto{\pgfqpoint{2.430395in}{2.971219in}}%
\pgfpathlineto{\pgfqpoint{2.424000in}{2.971219in}}%
\pgfpathlineto{\pgfqpoint{2.417606in}{2.971219in}}%
\pgfpathlineto{\pgfqpoint{2.411212in}{2.971219in}}%
\pgfpathlineto{\pgfqpoint{2.404817in}{2.971219in}}%
\pgfpathlineto{\pgfqpoint{2.398423in}{2.971219in}}%
\pgfpathlineto{\pgfqpoint{2.392029in}{2.971219in}}%
\pgfpathlineto{\pgfqpoint{2.385635in}{2.971219in}}%
\pgfpathlineto{\pgfqpoint{2.379240in}{2.971219in}}%
\pgfpathlineto{\pgfqpoint{2.372846in}{2.971219in}}%
\pgfpathlineto{\pgfqpoint{2.366452in}{2.971219in}}%
\pgfpathlineto{\pgfqpoint{2.360057in}{2.971219in}}%
\pgfpathlineto{\pgfqpoint{2.353663in}{2.971219in}}%
\pgfpathlineto{\pgfqpoint{2.347269in}{2.971219in}}%
\pgfpathlineto{\pgfqpoint{2.340875in}{2.971219in}}%
\pgfpathlineto{\pgfqpoint{2.334480in}{2.971219in}}%
\pgfpathlineto{\pgfqpoint{2.328086in}{2.971219in}}%
\pgfpathlineto{\pgfqpoint{2.321692in}{2.971219in}}%
\pgfpathlineto{\pgfqpoint{2.315298in}{2.971219in}}%
\pgfpathlineto{\pgfqpoint{2.308903in}{2.971219in}}%
\pgfpathlineto{\pgfqpoint{2.302509in}{2.971219in}}%
\pgfpathlineto{\pgfqpoint{2.296115in}{2.971219in}}%
\pgfpathlineto{\pgfqpoint{2.289720in}{2.971219in}}%
\pgfpathlineto{\pgfqpoint{2.283326in}{2.971219in}}%
\pgfpathlineto{\pgfqpoint{2.276932in}{2.971219in}}%
\pgfpathlineto{\pgfqpoint{2.270538in}{2.971219in}}%
\pgfpathlineto{\pgfqpoint{2.264143in}{2.971219in}}%
\pgfpathlineto{\pgfqpoint{2.257749in}{2.971219in}}%
\pgfpathlineto{\pgfqpoint{2.251355in}{2.971219in}}%
\pgfpathlineto{\pgfqpoint{2.244960in}{2.971219in}}%
\pgfpathlineto{\pgfqpoint{2.238566in}{2.971219in}}%
\pgfpathlineto{\pgfqpoint{2.232172in}{2.971219in}}%
\pgfpathlineto{\pgfqpoint{2.225778in}{2.971219in}}%
\pgfpathlineto{\pgfqpoint{2.219383in}{2.971219in}}%
\pgfpathlineto{\pgfqpoint{2.212989in}{2.971219in}}%
\pgfpathlineto{\pgfqpoint{2.206595in}{2.971219in}}%
\pgfpathlineto{\pgfqpoint{2.200200in}{2.971219in}}%
\pgfpathlineto{\pgfqpoint{2.193806in}{2.971219in}}%
\pgfpathlineto{\pgfqpoint{2.187412in}{2.971219in}}%
\pgfpathlineto{\pgfqpoint{2.181018in}{2.971219in}}%
\pgfpathlineto{\pgfqpoint{2.174623in}{2.971219in}}%
\pgfpathlineto{\pgfqpoint{2.168229in}{2.971219in}}%
\pgfpathlineto{\pgfqpoint{2.161835in}{2.971219in}}%
\pgfpathlineto{\pgfqpoint{2.155441in}{2.971219in}}%
\pgfpathlineto{\pgfqpoint{2.149046in}{2.971219in}}%
\pgfpathlineto{\pgfqpoint{2.142652in}{2.971219in}}%
\pgfpathlineto{\pgfqpoint{2.136258in}{2.971219in}}%
\pgfpathlineto{\pgfqpoint{2.129863in}{2.971219in}}%
\pgfpathlineto{\pgfqpoint{2.123469in}{2.971219in}}%
\pgfpathlineto{\pgfqpoint{2.117075in}{2.971219in}}%
\pgfpathlineto{\pgfqpoint{2.110681in}{2.971219in}}%
\pgfpathlineto{\pgfqpoint{2.104286in}{2.971219in}}%
\pgfpathlineto{\pgfqpoint{2.097892in}{2.971219in}}%
\pgfpathlineto{\pgfqpoint{2.091498in}{2.971219in}}%
\pgfpathlineto{\pgfqpoint{2.085103in}{2.971219in}}%
\pgfpathlineto{\pgfqpoint{2.078709in}{2.971219in}}%
\pgfpathlineto{\pgfqpoint{2.072315in}{2.971219in}}%
\pgfpathlineto{\pgfqpoint{2.065921in}{2.971219in}}%
\pgfpathlineto{\pgfqpoint{2.059526in}{2.971219in}}%
\pgfpathlineto{\pgfqpoint{2.053132in}{2.971219in}}%
\pgfpathlineto{\pgfqpoint{2.046738in}{2.971219in}}%
\pgfpathlineto{\pgfqpoint{2.040343in}{2.971219in}}%
\pgfpathlineto{\pgfqpoint{2.033949in}{2.971219in}}%
\pgfpathlineto{\pgfqpoint{2.027555in}{2.971219in}}%
\pgfpathlineto{\pgfqpoint{2.021161in}{2.971219in}}%
\pgfpathlineto{\pgfqpoint{2.014766in}{2.971219in}}%
\pgfpathlineto{\pgfqpoint{2.008372in}{2.971219in}}%
\pgfpathlineto{\pgfqpoint{2.001978in}{2.971219in}}%
\pgfpathlineto{\pgfqpoint{1.995584in}{2.971219in}}%
\pgfpathlineto{\pgfqpoint{1.989189in}{2.971219in}}%
\pgfpathlineto{\pgfqpoint{1.982795in}{2.971219in}}%
\pgfpathlineto{\pgfqpoint{1.976401in}{2.971219in}}%
\pgfpathlineto{\pgfqpoint{1.970006in}{2.971219in}}%
\pgfpathlineto{\pgfqpoint{1.963612in}{2.971219in}}%
\pgfpathlineto{\pgfqpoint{1.957218in}{2.971219in}}%
\pgfpathlineto{\pgfqpoint{1.950824in}{2.971219in}}%
\pgfpathlineto{\pgfqpoint{1.944429in}{2.971219in}}%
\pgfpathlineto{\pgfqpoint{1.938035in}{2.971219in}}%
\pgfpathlineto{\pgfqpoint{1.931641in}{2.971219in}}%
\pgfpathlineto{\pgfqpoint{1.925246in}{2.971219in}}%
\pgfpathlineto{\pgfqpoint{1.918852in}{2.971219in}}%
\pgfpathlineto{\pgfqpoint{1.912458in}{2.971219in}}%
\pgfpathlineto{\pgfqpoint{1.906064in}{2.971219in}}%
\pgfpathlineto{\pgfqpoint{1.899669in}{2.971219in}}%
\pgfpathlineto{\pgfqpoint{1.893275in}{2.971219in}}%
\pgfpathlineto{\pgfqpoint{1.886881in}{2.971219in}}%
\pgfpathlineto{\pgfqpoint{1.880486in}{2.971219in}}%
\pgfpathlineto{\pgfqpoint{1.874092in}{2.971219in}}%
\pgfpathlineto{\pgfqpoint{1.867698in}{2.971219in}}%
\pgfpathlineto{\pgfqpoint{1.861304in}{2.971219in}}%
\pgfpathlineto{\pgfqpoint{1.854909in}{2.971219in}}%
\pgfpathlineto{\pgfqpoint{1.848515in}{2.971219in}}%
\pgfpathlineto{\pgfqpoint{1.842121in}{2.971219in}}%
\pgfpathlineto{\pgfqpoint{1.835727in}{2.971219in}}%
\pgfpathlineto{\pgfqpoint{1.829332in}{2.971219in}}%
\pgfpathlineto{\pgfqpoint{1.822938in}{2.971219in}}%
\pgfpathlineto{\pgfqpoint{1.816544in}{2.971219in}}%
\pgfpathlineto{\pgfqpoint{1.810149in}{2.971219in}}%
\pgfpathlineto{\pgfqpoint{1.803755in}{2.971219in}}%
\pgfpathlineto{\pgfqpoint{1.797361in}{2.971219in}}%
\pgfpathlineto{\pgfqpoint{1.790967in}{2.971219in}}%
\pgfpathlineto{\pgfqpoint{1.784572in}{2.971219in}}%
\pgfpathlineto{\pgfqpoint{1.778178in}{2.971219in}}%
\pgfpathlineto{\pgfqpoint{1.771784in}{2.971219in}}%
\pgfpathlineto{\pgfqpoint{1.765389in}{2.971219in}}%
\pgfpathlineto{\pgfqpoint{1.758995in}{2.971219in}}%
\pgfpathlineto{\pgfqpoint{1.752601in}{2.971219in}}%
\pgfpathlineto{\pgfqpoint{1.746207in}{2.971219in}}%
\pgfpathlineto{\pgfqpoint{1.739812in}{2.971219in}}%
\pgfpathlineto{\pgfqpoint{1.733418in}{2.971219in}}%
\pgfpathlineto{\pgfqpoint{1.727024in}{2.971219in}}%
\pgfpathlineto{\pgfqpoint{1.720630in}{2.971219in}}%
\pgfpathlineto{\pgfqpoint{1.714235in}{2.971219in}}%
\pgfpathlineto{\pgfqpoint{1.707841in}{2.971219in}}%
\pgfpathlineto{\pgfqpoint{1.701447in}{2.971219in}}%
\pgfpathlineto{\pgfqpoint{1.695052in}{2.971219in}}%
\pgfpathlineto{\pgfqpoint{1.688658in}{2.971219in}}%
\pgfpathlineto{\pgfqpoint{1.682264in}{2.971219in}}%
\pgfpathlineto{\pgfqpoint{1.675870in}{2.971219in}}%
\pgfpathlineto{\pgfqpoint{1.669475in}{2.971219in}}%
\pgfpathlineto{\pgfqpoint{1.663081in}{2.971219in}}%
\pgfpathlineto{\pgfqpoint{1.656687in}{2.971219in}}%
\pgfpathlineto{\pgfqpoint{1.650292in}{2.971219in}}%
\pgfpathlineto{\pgfqpoint{1.643898in}{2.971219in}}%
\pgfpathlineto{\pgfqpoint{1.637504in}{2.971219in}}%
\pgfpathlineto{\pgfqpoint{1.631110in}{2.971219in}}%
\pgfpathlineto{\pgfqpoint{1.624715in}{2.971219in}}%
\pgfpathlineto{\pgfqpoint{1.618321in}{2.971219in}}%
\pgfpathlineto{\pgfqpoint{1.611927in}{2.971219in}}%
\pgfpathlineto{\pgfqpoint{1.605532in}{2.971219in}}%
\pgfpathlineto{\pgfqpoint{1.599138in}{2.971219in}}%
\pgfpathlineto{\pgfqpoint{1.592744in}{2.971219in}}%
\pgfpathlineto{\pgfqpoint{1.586350in}{2.971219in}}%
\pgfpathlineto{\pgfqpoint{1.579955in}{2.971219in}}%
\pgfpathlineto{\pgfqpoint{1.573561in}{2.971219in}}%
\pgfpathlineto{\pgfqpoint{1.567167in}{2.971219in}}%
\pgfpathlineto{\pgfqpoint{1.560773in}{2.971219in}}%
\pgfpathlineto{\pgfqpoint{1.554378in}{2.971219in}}%
\pgfpathlineto{\pgfqpoint{1.547984in}{2.971219in}}%
\pgfpathlineto{\pgfqpoint{1.541590in}{2.971219in}}%
\pgfpathlineto{\pgfqpoint{1.535195in}{2.971219in}}%
\pgfpathlineto{\pgfqpoint{1.528801in}{2.971219in}}%
\pgfpathlineto{\pgfqpoint{1.522407in}{2.971219in}}%
\pgfpathlineto{\pgfqpoint{1.516013in}{2.971219in}}%
\pgfpathlineto{\pgfqpoint{1.509618in}{2.971219in}}%
\pgfpathlineto{\pgfqpoint{1.503224in}{2.971219in}}%
\pgfpathlineto{\pgfqpoint{1.496830in}{2.971219in}}%
\pgfpathlineto{\pgfqpoint{1.490435in}{2.971219in}}%
\pgfpathlineto{\pgfqpoint{1.484041in}{2.971219in}}%
\pgfpathlineto{\pgfqpoint{1.477647in}{2.971219in}}%
\pgfpathlineto{\pgfqpoint{1.471253in}{2.971219in}}%
\pgfpathlineto{\pgfqpoint{1.464858in}{2.971219in}}%
\pgfpathlineto{\pgfqpoint{1.458464in}{2.971219in}}%
\pgfpathlineto{\pgfqpoint{1.452070in}{2.971219in}}%
\pgfpathlineto{\pgfqpoint{1.445675in}{2.971219in}}%
\pgfpathlineto{\pgfqpoint{1.439281in}{2.971219in}}%
\pgfpathlineto{\pgfqpoint{1.432887in}{2.971219in}}%
\pgfpathlineto{\pgfqpoint{1.426493in}{2.971219in}}%
\pgfpathlineto{\pgfqpoint{1.420098in}{2.971219in}}%
\pgfpathlineto{\pgfqpoint{1.413704in}{2.971219in}}%
\pgfpathlineto{\pgfqpoint{1.407310in}{2.971219in}}%
\pgfpathlineto{\pgfqpoint{1.400916in}{2.971219in}}%
\pgfpathlineto{\pgfqpoint{1.394521in}{2.971219in}}%
\pgfpathlineto{\pgfqpoint{1.388127in}{2.971219in}}%
\pgfpathlineto{\pgfqpoint{1.381733in}{2.971219in}}%
\pgfpathlineto{\pgfqpoint{1.375338in}{2.971219in}}%
\pgfpathlineto{\pgfqpoint{1.368944in}{2.971219in}}%
\pgfpathlineto{\pgfqpoint{1.362550in}{2.971219in}}%
\pgfpathlineto{\pgfqpoint{1.356156in}{2.971219in}}%
\pgfpathlineto{\pgfqpoint{1.349761in}{2.971219in}}%
\pgfpathlineto{\pgfqpoint{1.343367in}{2.971219in}}%
\pgfpathlineto{\pgfqpoint{1.336973in}{2.971219in}}%
\pgfpathlineto{\pgfqpoint{1.330578in}{2.971219in}}%
\pgfpathlineto{\pgfqpoint{1.324184in}{2.971219in}}%
\pgfpathlineto{\pgfqpoint{1.317790in}{2.971219in}}%
\pgfpathlineto{\pgfqpoint{1.311396in}{2.971219in}}%
\pgfpathlineto{\pgfqpoint{1.305001in}{2.971219in}}%
\pgfpathlineto{\pgfqpoint{1.298607in}{2.971219in}}%
\pgfpathlineto{\pgfqpoint{1.292213in}{2.971219in}}%
\pgfpathlineto{\pgfqpoint{1.285818in}{2.971219in}}%
\pgfpathlineto{\pgfqpoint{1.279424in}{2.971219in}}%
\pgfpathlineto{\pgfqpoint{1.273030in}{2.971219in}}%
\pgfpathlineto{\pgfqpoint{1.266636in}{2.971219in}}%
\pgfpathlineto{\pgfqpoint{1.260241in}{2.971219in}}%
\pgfpathlineto{\pgfqpoint{1.253847in}{2.971219in}}%
\pgfpathlineto{\pgfqpoint{1.247453in}{2.971219in}}%
\pgfpathlineto{\pgfqpoint{1.241059in}{2.971219in}}%
\pgfpathlineto{\pgfqpoint{1.234664in}{2.971219in}}%
\pgfpathlineto{\pgfqpoint{1.228270in}{2.971219in}}%
\pgfpathlineto{\pgfqpoint{1.221876in}{2.971219in}}%
\pgfpathlineto{\pgfqpoint{1.215481in}{2.971219in}}%
\pgfpathlineto{\pgfqpoint{1.209087in}{2.971219in}}%
\pgfpathlineto{\pgfqpoint{1.202693in}{2.971219in}}%
\pgfpathlineto{\pgfqpoint{1.196299in}{2.971219in}}%
\pgfpathlineto{\pgfqpoint{1.189904in}{2.971219in}}%
\pgfpathlineto{\pgfqpoint{1.183510in}{2.971219in}}%
\pgfpathlineto{\pgfqpoint{1.177116in}{2.971219in}}%
\pgfpathlineto{\pgfqpoint{1.170721in}{2.971219in}}%
\pgfpathlineto{\pgfqpoint{1.164327in}{2.971219in}}%
\pgfpathlineto{\pgfqpoint{1.157933in}{2.971219in}}%
\pgfpathlineto{\pgfqpoint{1.151539in}{2.971219in}}%
\pgfpathlineto{\pgfqpoint{1.145144in}{2.971219in}}%
\pgfpathlineto{\pgfqpoint{1.138750in}{2.971219in}}%
\pgfpathlineto{\pgfqpoint{1.132356in}{2.971219in}}%
\pgfpathlineto{\pgfqpoint{1.125962in}{2.971219in}}%
\pgfpathlineto{\pgfqpoint{1.119567in}{2.971219in}}%
\pgfpathlineto{\pgfqpoint{1.113173in}{2.971219in}}%
\pgfpathlineto{\pgfqpoint{1.106779in}{2.971219in}}%
\pgfpathlineto{\pgfqpoint{1.100384in}{2.971219in}}%
\pgfpathlineto{\pgfqpoint{1.093990in}{2.971219in}}%
\pgfpathlineto{\pgfqpoint{1.087596in}{2.971219in}}%
\pgfpathlineto{\pgfqpoint{1.081202in}{2.971219in}}%
\pgfpathlineto{\pgfqpoint{1.074807in}{2.971219in}}%
\pgfpathlineto{\pgfqpoint{1.068413in}{2.971219in}}%
\pgfpathlineto{\pgfqpoint{1.062019in}{2.971219in}}%
\pgfpathlineto{\pgfqpoint{1.055624in}{2.971219in}}%
\pgfpathlineto{\pgfqpoint{1.049230in}{2.971219in}}%
\pgfpathlineto{\pgfqpoint{1.042836in}{2.971219in}}%
\pgfpathlineto{\pgfqpoint{1.036442in}{2.971219in}}%
\pgfpathlineto{\pgfqpoint{1.030047in}{2.971219in}}%
\pgfpathlineto{\pgfqpoint{1.023653in}{2.971219in}}%
\pgfpathlineto{\pgfqpoint{1.017259in}{2.971219in}}%
\pgfpathlineto{\pgfqpoint{1.010864in}{2.971219in}}%
\pgfpathlineto{\pgfqpoint{1.004470in}{2.971219in}}%
\pgfpathlineto{\pgfqpoint{0.998076in}{2.971219in}}%
\pgfpathlineto{\pgfqpoint{0.991682in}{2.971219in}}%
\pgfpathlineto{\pgfqpoint{0.985287in}{2.971219in}}%
\pgfpathlineto{\pgfqpoint{0.978893in}{2.971219in}}%
\pgfpathlineto{\pgfqpoint{0.972499in}{2.971219in}}%
\pgfpathlineto{\pgfqpoint{0.966105in}{2.971219in}}%
\pgfpathlineto{\pgfqpoint{0.959710in}{2.971219in}}%
\pgfpathlineto{\pgfqpoint{0.953316in}{2.971219in}}%
\pgfpathlineto{\pgfqpoint{0.946922in}{2.971219in}}%
\pgfpathlineto{\pgfqpoint{0.940527in}{2.971219in}}%
\pgfpathlineto{\pgfqpoint{0.934133in}{2.971219in}}%
\pgfpathlineto{\pgfqpoint{0.927739in}{2.971219in}}%
\pgfpathlineto{\pgfqpoint{0.921345in}{2.971219in}}%
\pgfpathlineto{\pgfqpoint{0.914950in}{2.971219in}}%
\pgfpathlineto{\pgfqpoint{0.908556in}{2.971219in}}%
\pgfpathlineto{\pgfqpoint{0.902162in}{2.971219in}}%
\pgfpathlineto{\pgfqpoint{0.895767in}{2.971219in}}%
\pgfpathlineto{\pgfqpoint{0.889373in}{2.971219in}}%
\pgfpathlineto{\pgfqpoint{0.882979in}{2.971219in}}%
\pgfpathlineto{\pgfqpoint{0.876585in}{2.971219in}}%
\pgfpathlineto{\pgfqpoint{0.870190in}{2.971219in}}%
\pgfpathlineto{\pgfqpoint{0.863796in}{2.971219in}}%
\pgfpathlineto{\pgfqpoint{0.857402in}{2.971219in}}%
\pgfpathlineto{\pgfqpoint{0.851007in}{2.971219in}}%
\pgfpathlineto{\pgfqpoint{0.844613in}{2.971219in}}%
\pgfpathlineto{\pgfqpoint{0.838219in}{2.971219in}}%
\pgfpathlineto{\pgfqpoint{0.831825in}{2.971219in}}%
\pgfpathlineto{\pgfqpoint{0.825430in}{2.971219in}}%
\pgfpathlineto{\pgfqpoint{0.819036in}{2.971219in}}%
\pgfpathlineto{\pgfqpoint{0.812642in}{2.971219in}}%
\pgfpathlineto{\pgfqpoint{0.806248in}{2.971219in}}%
\pgfpathlineto{\pgfqpoint{0.799853in}{2.971219in}}%
\pgfpathlineto{\pgfqpoint{0.793459in}{2.971219in}}%
\pgfpathlineto{\pgfqpoint{0.787065in}{2.971219in}}%
\pgfpathlineto{\pgfqpoint{0.780670in}{2.971219in}}%
\pgfpathlineto{\pgfqpoint{0.774276in}{2.971219in}}%
\pgfpathlineto{\pgfqpoint{0.767882in}{2.971219in}}%
\pgfpathlineto{\pgfqpoint{0.761488in}{2.971219in}}%
\pgfpathlineto{\pgfqpoint{0.755093in}{2.971219in}}%
\pgfpathlineto{\pgfqpoint{0.748699in}{2.971219in}}%
\pgfpathlineto{\pgfqpoint{0.742305in}{2.971219in}}%
\pgfpathlineto{\pgfqpoint{0.735910in}{2.971219in}}%
\pgfpathlineto{\pgfqpoint{0.729516in}{2.971219in}}%
\pgfpathlineto{\pgfqpoint{0.723122in}{2.971219in}}%
\pgfpathlineto{\pgfqpoint{0.716728in}{2.971219in}}%
\pgfpathlineto{\pgfqpoint{0.710333in}{2.971219in}}%
\pgfpathlineto{\pgfqpoint{0.703939in}{2.971219in}}%
\pgfpathlineto{\pgfqpoint{0.697545in}{2.971219in}}%
\pgfpathlineto{\pgfqpoint{0.691150in}{2.971219in}}%
\pgfpathlineto{\pgfqpoint{0.684756in}{2.971219in}}%
\pgfpathlineto{\pgfqpoint{0.678362in}{2.971219in}}%
\pgfpathlineto{\pgfqpoint{0.671968in}{2.971219in}}%
\pgfpathlineto{\pgfqpoint{0.665573in}{2.971219in}}%
\pgfpathlineto{\pgfqpoint{0.659179in}{2.971219in}}%
\pgfpathlineto{\pgfqpoint{0.652785in}{2.971219in}}%
\pgfpathlineto{\pgfqpoint{0.646391in}{2.971219in}}%
\pgfpathlineto{\pgfqpoint{0.639996in}{2.971219in}}%
\pgfpathlineto{\pgfqpoint{0.633602in}{2.971219in}}%
\pgfpathlineto{\pgfqpoint{0.627208in}{2.971219in}}%
\pgfpathlineto{\pgfqpoint{0.620813in}{2.971219in}}%
\pgfpathlineto{\pgfqpoint{0.614419in}{2.971219in}}%
\pgfpathlineto{\pgfqpoint{0.608025in}{2.971219in}}%
\pgfpathlineto{\pgfqpoint{0.608025in}{2.971219in}}%
\pgfpathclose%
\pgfusepath{stroke,fill}%
}%
\begin{pgfscope}%
\pgfsys@transformshift{0.000000in}{0.000000in}%
\pgfsys@useobject{currentmarker}{}%
\end{pgfscope}%
\end{pgfscope}%
\begin{pgfscope}%
\pgfsetbuttcap%
\pgfsetroundjoin%
\definecolor{currentfill}{rgb}{0.000000,0.000000,0.000000}%
\pgfsetfillcolor{currentfill}%
\pgfsetlinewidth{0.803000pt}%
\definecolor{currentstroke}{rgb}{0.000000,0.000000,0.000000}%
\pgfsetstrokecolor{currentstroke}%
\pgfsetdash{}{0pt}%
\pgfsys@defobject{currentmarker}{\pgfqpoint{0.000000in}{-0.048611in}}{\pgfqpoint{0.000000in}{0.000000in}}{%
\pgfpathmoveto{\pgfqpoint{0.000000in}{0.000000in}}%
\pgfpathlineto{\pgfqpoint{0.000000in}{-0.048611in}}%
\pgfusepath{stroke,fill}%
}%
\begin{pgfscope}%
\pgfsys@transformshift{0.608025in}{0.554012in}%
\pgfsys@useobject{currentmarker}{}%
\end{pgfscope}%
\end{pgfscope}%
\begin{pgfscope}%
\definecolor{textcolor}{rgb}{0.000000,0.000000,0.000000}%
\pgfsetstrokecolor{textcolor}%
\pgfsetfillcolor{textcolor}%
\pgftext[x=0.608025in,y=0.456790in,,top]{\color{textcolor}\rmfamily\fontsize{10.000000}{12.000000}\selectfont \(\displaystyle {0.0}\)}%
\end{pgfscope}%
\begin{pgfscope}%
\pgfsetbuttcap%
\pgfsetroundjoin%
\definecolor{currentfill}{rgb}{0.000000,0.000000,0.000000}%
\pgfsetfillcolor{currentfill}%
\pgfsetlinewidth{0.803000pt}%
\definecolor{currentstroke}{rgb}{0.000000,0.000000,0.000000}%
\pgfsetstrokecolor{currentstroke}%
\pgfsetdash{}{0pt}%
\pgfsys@defobject{currentmarker}{\pgfqpoint{0.000000in}{-0.048611in}}{\pgfqpoint{0.000000in}{0.000000in}}{%
\pgfpathmoveto{\pgfqpoint{0.000000in}{0.000000in}}%
\pgfpathlineto{\pgfqpoint{0.000000in}{-0.048611in}}%
\pgfusepath{stroke,fill}%
}%
\begin{pgfscope}%
\pgfsys@transformshift{1.885602in}{0.554012in}%
\pgfsys@useobject{currentmarker}{}%
\end{pgfscope}%
\end{pgfscope}%
\begin{pgfscope}%
\definecolor{textcolor}{rgb}{0.000000,0.000000,0.000000}%
\pgfsetstrokecolor{textcolor}%
\pgfsetfillcolor{textcolor}%
\pgftext[x=1.885602in,y=0.456790in,,top]{\color{textcolor}\rmfamily\fontsize{10.000000}{12.000000}\selectfont \(\displaystyle {0.2}\)}%
\end{pgfscope}%
\begin{pgfscope}%
\pgfsetbuttcap%
\pgfsetroundjoin%
\definecolor{currentfill}{rgb}{0.000000,0.000000,0.000000}%
\pgfsetfillcolor{currentfill}%
\pgfsetlinewidth{0.803000pt}%
\definecolor{currentstroke}{rgb}{0.000000,0.000000,0.000000}%
\pgfsetstrokecolor{currentstroke}%
\pgfsetdash{}{0pt}%
\pgfsys@defobject{currentmarker}{\pgfqpoint{0.000000in}{-0.048611in}}{\pgfqpoint{0.000000in}{0.000000in}}{%
\pgfpathmoveto{\pgfqpoint{0.000000in}{0.000000in}}%
\pgfpathlineto{\pgfqpoint{0.000000in}{-0.048611in}}%
\pgfusepath{stroke,fill}%
}%
\begin{pgfscope}%
\pgfsys@transformshift{3.163179in}{0.554012in}%
\pgfsys@useobject{currentmarker}{}%
\end{pgfscope}%
\end{pgfscope}%
\begin{pgfscope}%
\definecolor{textcolor}{rgb}{0.000000,0.000000,0.000000}%
\pgfsetstrokecolor{textcolor}%
\pgfsetfillcolor{textcolor}%
\pgftext[x=3.163179in,y=0.456790in,,top]{\color{textcolor}\rmfamily\fontsize{10.000000}{12.000000}\selectfont \(\displaystyle {0.4}\)}%
\end{pgfscope}%
\begin{pgfscope}%
\pgfsetbuttcap%
\pgfsetroundjoin%
\definecolor{currentfill}{rgb}{0.000000,0.000000,0.000000}%
\pgfsetfillcolor{currentfill}%
\pgfsetlinewidth{0.803000pt}%
\definecolor{currentstroke}{rgb}{0.000000,0.000000,0.000000}%
\pgfsetstrokecolor{currentstroke}%
\pgfsetdash{}{0pt}%
\pgfsys@defobject{currentmarker}{\pgfqpoint{0.000000in}{-0.048611in}}{\pgfqpoint{0.000000in}{0.000000in}}{%
\pgfpathmoveto{\pgfqpoint{0.000000in}{0.000000in}}%
\pgfpathlineto{\pgfqpoint{0.000000in}{-0.048611in}}%
\pgfusepath{stroke,fill}%
}%
\begin{pgfscope}%
\pgfsys@transformshift{4.440756in}{0.554012in}%
\pgfsys@useobject{currentmarker}{}%
\end{pgfscope}%
\end{pgfscope}%
\begin{pgfscope}%
\definecolor{textcolor}{rgb}{0.000000,0.000000,0.000000}%
\pgfsetstrokecolor{textcolor}%
\pgfsetfillcolor{textcolor}%
\pgftext[x=4.440756in,y=0.456790in,,top]{\color{textcolor}\rmfamily\fontsize{10.000000}{12.000000}\selectfont \(\displaystyle {0.6}\)}%
\end{pgfscope}%
\begin{pgfscope}%
\pgfsetbuttcap%
\pgfsetroundjoin%
\definecolor{currentfill}{rgb}{0.000000,0.000000,0.000000}%
\pgfsetfillcolor{currentfill}%
\pgfsetlinewidth{0.803000pt}%
\definecolor{currentstroke}{rgb}{0.000000,0.000000,0.000000}%
\pgfsetstrokecolor{currentstroke}%
\pgfsetdash{}{0pt}%
\pgfsys@defobject{currentmarker}{\pgfqpoint{0.000000in}{-0.048611in}}{\pgfqpoint{0.000000in}{0.000000in}}{%
\pgfpathmoveto{\pgfqpoint{0.000000in}{0.000000in}}%
\pgfpathlineto{\pgfqpoint{0.000000in}{-0.048611in}}%
\pgfusepath{stroke,fill}%
}%
\begin{pgfscope}%
\pgfsys@transformshift{5.718333in}{0.554012in}%
\pgfsys@useobject{currentmarker}{}%
\end{pgfscope}%
\end{pgfscope}%
\begin{pgfscope}%
\definecolor{textcolor}{rgb}{0.000000,0.000000,0.000000}%
\pgfsetstrokecolor{textcolor}%
\pgfsetfillcolor{textcolor}%
\pgftext[x=5.718333in,y=0.456790in,,top]{\color{textcolor}\rmfamily\fontsize{10.000000}{12.000000}\selectfont \(\displaystyle {0.8}\)}%
\end{pgfscope}%
\begin{pgfscope}%
\pgfsetbuttcap%
\pgfsetroundjoin%
\definecolor{currentfill}{rgb}{0.000000,0.000000,0.000000}%
\pgfsetfillcolor{currentfill}%
\pgfsetlinewidth{0.803000pt}%
\definecolor{currentstroke}{rgb}{0.000000,0.000000,0.000000}%
\pgfsetstrokecolor{currentstroke}%
\pgfsetdash{}{0pt}%
\pgfsys@defobject{currentmarker}{\pgfqpoint{0.000000in}{-0.048611in}}{\pgfqpoint{0.000000in}{0.000000in}}{%
\pgfpathmoveto{\pgfqpoint{0.000000in}{0.000000in}}%
\pgfpathlineto{\pgfqpoint{0.000000in}{-0.048611in}}%
\pgfusepath{stroke,fill}%
}%
\begin{pgfscope}%
\pgfsys@transformshift{6.995910in}{0.554012in}%
\pgfsys@useobject{currentmarker}{}%
\end{pgfscope}%
\end{pgfscope}%
\begin{pgfscope}%
\definecolor{textcolor}{rgb}{0.000000,0.000000,0.000000}%
\pgfsetstrokecolor{textcolor}%
\pgfsetfillcolor{textcolor}%
\pgftext[x=6.995910in,y=0.456790in,,top]{\color{textcolor}\rmfamily\fontsize{10.000000}{12.000000}\selectfont \(\displaystyle {1.0}\)}%
\end{pgfscope}%
\begin{pgfscope}%
\definecolor{textcolor}{rgb}{0.000000,0.000000,0.000000}%
\pgfsetstrokecolor{textcolor}%
\pgfsetfillcolor{textcolor}%
\pgftext[x=3.801968in,y=0.277777in,,top]{\color{textcolor}\rmfamily\fontsize{14.000000}{16.800000}\selectfont Normalized Quantity}%
\end{pgfscope}%
\begin{pgfscope}%
\pgfsetbuttcap%
\pgfsetroundjoin%
\definecolor{currentfill}{rgb}{0.000000,0.000000,0.000000}%
\pgfsetfillcolor{currentfill}%
\pgfsetlinewidth{0.803000pt}%
\definecolor{currentstroke}{rgb}{0.000000,0.000000,0.000000}%
\pgfsetstrokecolor{currentstroke}%
\pgfsetdash{}{0pt}%
\pgfsys@defobject{currentmarker}{\pgfqpoint{-0.048611in}{0.000000in}}{\pgfqpoint{-0.000000in}{0.000000in}}{%
\pgfpathmoveto{\pgfqpoint{-0.000000in}{0.000000in}}%
\pgfpathlineto{\pgfqpoint{-0.048611in}{0.000000in}}%
\pgfusepath{stroke,fill}%
}%
\begin{pgfscope}%
\pgfsys@transformshift{0.608025in}{0.554012in}%
\pgfsys@useobject{currentmarker}{}%
\end{pgfscope}%
\end{pgfscope}%
\begin{pgfscope}%
\definecolor{textcolor}{rgb}{0.000000,0.000000,0.000000}%
\pgfsetstrokecolor{textcolor}%
\pgfsetfillcolor{textcolor}%
\pgftext[x=0.333333in, y=0.505787in, left, base]{\color{textcolor}\rmfamily\fontsize{10.000000}{12.000000}\selectfont \(\displaystyle {0.0}\)}%
\end{pgfscope}%
\begin{pgfscope}%
\pgfsetbuttcap%
\pgfsetroundjoin%
\definecolor{currentfill}{rgb}{0.000000,0.000000,0.000000}%
\pgfsetfillcolor{currentfill}%
\pgfsetlinewidth{0.803000pt}%
\definecolor{currentstroke}{rgb}{0.000000,0.000000,0.000000}%
\pgfsetstrokecolor{currentstroke}%
\pgfsetdash{}{0pt}%
\pgfsys@defobject{currentmarker}{\pgfqpoint{-0.048611in}{0.000000in}}{\pgfqpoint{-0.000000in}{0.000000in}}{%
\pgfpathmoveto{\pgfqpoint{-0.000000in}{0.000000in}}%
\pgfpathlineto{\pgfqpoint{-0.048611in}{0.000000in}}%
\pgfusepath{stroke,fill}%
}%
\begin{pgfscope}%
\pgfsys@transformshift{0.608025in}{1.520895in}%
\pgfsys@useobject{currentmarker}{}%
\end{pgfscope}%
\end{pgfscope}%
\begin{pgfscope}%
\definecolor{textcolor}{rgb}{0.000000,0.000000,0.000000}%
\pgfsetstrokecolor{textcolor}%
\pgfsetfillcolor{textcolor}%
\pgftext[x=0.333333in, y=1.472669in, left, base]{\color{textcolor}\rmfamily\fontsize{10.000000}{12.000000}\selectfont \(\displaystyle {0.2}\)}%
\end{pgfscope}%
\begin{pgfscope}%
\pgfsetbuttcap%
\pgfsetroundjoin%
\definecolor{currentfill}{rgb}{0.000000,0.000000,0.000000}%
\pgfsetfillcolor{currentfill}%
\pgfsetlinewidth{0.803000pt}%
\definecolor{currentstroke}{rgb}{0.000000,0.000000,0.000000}%
\pgfsetstrokecolor{currentstroke}%
\pgfsetdash{}{0pt}%
\pgfsys@defobject{currentmarker}{\pgfqpoint{-0.048611in}{0.000000in}}{\pgfqpoint{-0.000000in}{0.000000in}}{%
\pgfpathmoveto{\pgfqpoint{-0.000000in}{0.000000in}}%
\pgfpathlineto{\pgfqpoint{-0.048611in}{0.000000in}}%
\pgfusepath{stroke,fill}%
}%
\begin{pgfscope}%
\pgfsys@transformshift{0.608025in}{2.487778in}%
\pgfsys@useobject{currentmarker}{}%
\end{pgfscope}%
\end{pgfscope}%
\begin{pgfscope}%
\definecolor{textcolor}{rgb}{0.000000,0.000000,0.000000}%
\pgfsetstrokecolor{textcolor}%
\pgfsetfillcolor{textcolor}%
\pgftext[x=0.333333in, y=2.439552in, left, base]{\color{textcolor}\rmfamily\fontsize{10.000000}{12.000000}\selectfont \(\displaystyle {0.4}\)}%
\end{pgfscope}%
\begin{pgfscope}%
\pgfsetbuttcap%
\pgfsetroundjoin%
\definecolor{currentfill}{rgb}{0.000000,0.000000,0.000000}%
\pgfsetfillcolor{currentfill}%
\pgfsetlinewidth{0.803000pt}%
\definecolor{currentstroke}{rgb}{0.000000,0.000000,0.000000}%
\pgfsetstrokecolor{currentstroke}%
\pgfsetdash{}{0pt}%
\pgfsys@defobject{currentmarker}{\pgfqpoint{-0.048611in}{0.000000in}}{\pgfqpoint{-0.000000in}{0.000000in}}{%
\pgfpathmoveto{\pgfqpoint{-0.000000in}{0.000000in}}%
\pgfpathlineto{\pgfqpoint{-0.048611in}{0.000000in}}%
\pgfusepath{stroke,fill}%
}%
\begin{pgfscope}%
\pgfsys@transformshift{0.608025in}{3.454660in}%
\pgfsys@useobject{currentmarker}{}%
\end{pgfscope}%
\end{pgfscope}%
\begin{pgfscope}%
\definecolor{textcolor}{rgb}{0.000000,0.000000,0.000000}%
\pgfsetstrokecolor{textcolor}%
\pgfsetfillcolor{textcolor}%
\pgftext[x=0.333333in, y=3.406435in, left, base]{\color{textcolor}\rmfamily\fontsize{10.000000}{12.000000}\selectfont \(\displaystyle {0.6}\)}%
\end{pgfscope}%
\begin{pgfscope}%
\pgfsetbuttcap%
\pgfsetroundjoin%
\definecolor{currentfill}{rgb}{0.000000,0.000000,0.000000}%
\pgfsetfillcolor{currentfill}%
\pgfsetlinewidth{0.803000pt}%
\definecolor{currentstroke}{rgb}{0.000000,0.000000,0.000000}%
\pgfsetstrokecolor{currentstroke}%
\pgfsetdash{}{0pt}%
\pgfsys@defobject{currentmarker}{\pgfqpoint{-0.048611in}{0.000000in}}{\pgfqpoint{-0.000000in}{0.000000in}}{%
\pgfpathmoveto{\pgfqpoint{-0.000000in}{0.000000in}}%
\pgfpathlineto{\pgfqpoint{-0.048611in}{0.000000in}}%
\pgfusepath{stroke,fill}%
}%
\begin{pgfscope}%
\pgfsys@transformshift{0.608025in}{4.421543in}%
\pgfsys@useobject{currentmarker}{}%
\end{pgfscope}%
\end{pgfscope}%
\begin{pgfscope}%
\definecolor{textcolor}{rgb}{0.000000,0.000000,0.000000}%
\pgfsetstrokecolor{textcolor}%
\pgfsetfillcolor{textcolor}%
\pgftext[x=0.333333in, y=4.373318in, left, base]{\color{textcolor}\rmfamily\fontsize{10.000000}{12.000000}\selectfont \(\displaystyle {0.8}\)}%
\end{pgfscope}%
\begin{pgfscope}%
\pgfsetbuttcap%
\pgfsetroundjoin%
\definecolor{currentfill}{rgb}{0.000000,0.000000,0.000000}%
\pgfsetfillcolor{currentfill}%
\pgfsetlinewidth{0.803000pt}%
\definecolor{currentstroke}{rgb}{0.000000,0.000000,0.000000}%
\pgfsetstrokecolor{currentstroke}%
\pgfsetdash{}{0pt}%
\pgfsys@defobject{currentmarker}{\pgfqpoint{-0.048611in}{0.000000in}}{\pgfqpoint{-0.000000in}{0.000000in}}{%
\pgfpathmoveto{\pgfqpoint{-0.000000in}{0.000000in}}%
\pgfpathlineto{\pgfqpoint{-0.048611in}{0.000000in}}%
\pgfusepath{stroke,fill}%
}%
\begin{pgfscope}%
\pgfsys@transformshift{0.608025in}{5.388426in}%
\pgfsys@useobject{currentmarker}{}%
\end{pgfscope}%
\end{pgfscope}%
\begin{pgfscope}%
\definecolor{textcolor}{rgb}{0.000000,0.000000,0.000000}%
\pgfsetstrokecolor{textcolor}%
\pgfsetfillcolor{textcolor}%
\pgftext[x=0.333333in, y=5.340201in, left, base]{\color{textcolor}\rmfamily\fontsize{10.000000}{12.000000}\selectfont \(\displaystyle {1.0}\)}%
\end{pgfscope}%
\begin{pgfscope}%
\definecolor{textcolor}{rgb}{0.000000,0.000000,0.000000}%
\pgfsetstrokecolor{textcolor}%
\pgfsetfillcolor{textcolor}%
\pgftext[x=0.277777in,y=2.971219in,,bottom,rotate=90.000000]{\color{textcolor}\rmfamily\fontsize{14.000000}{16.800000}\selectfont Normalized Price}%
\end{pgfscope}%
\begin{pgfscope}%
\pgfpathrectangle{\pgfqpoint{0.608025in}{0.554012in}}{\pgfqpoint{6.387885in}{4.834414in}}%
\pgfusepath{clip}%
\pgfsetrectcap%
\pgfsetroundjoin%
\pgfsetlinewidth{1.505625pt}%
\definecolor{currentstroke}{rgb}{0.121569,0.466667,0.705882}%
\pgfsetstrokecolor{currentstroke}%
\pgfsetdash{}{0pt}%
\pgfpathmoveto{\pgfqpoint{0.608025in}{5.388426in}}%
\pgfpathlineto{\pgfqpoint{6.995910in}{0.554012in}}%
\pgfpathlineto{\pgfqpoint{6.995910in}{0.554012in}}%
\pgfusepath{stroke}%
\end{pgfscope}%
\begin{pgfscope}%
\pgfpathrectangle{\pgfqpoint{0.608025in}{0.554012in}}{\pgfqpoint{6.387885in}{4.834414in}}%
\pgfusepath{clip}%
\pgfsetrectcap%
\pgfsetroundjoin%
\pgfsetlinewidth{1.505625pt}%
\definecolor{currentstroke}{rgb}{1.000000,0.498039,0.054902}%
\pgfsetstrokecolor{currentstroke}%
\pgfsetdash{}{0pt}%
\pgfpathmoveto{\pgfqpoint{0.608025in}{0.554012in}}%
\pgfpathlineto{\pgfqpoint{6.995910in}{5.388426in}}%
\pgfpathlineto{\pgfqpoint{6.995910in}{5.388426in}}%
\pgfusepath{stroke}%
\end{pgfscope}%
\begin{pgfscope}%
\pgfpathrectangle{\pgfqpoint{0.608025in}{0.554012in}}{\pgfqpoint{6.387885in}{4.834414in}}%
\pgfusepath{clip}%
\pgfsetbuttcap%
\pgfsetroundjoin%
\pgfsetlinewidth{1.505625pt}%
\definecolor{currentstroke}{rgb}{1.000000,0.000000,0.000000}%
\pgfsetstrokecolor{currentstroke}%
\pgfsetstrokeopacity{0.600000}%
\pgfsetdash{{5.550000pt}{2.400000pt}}{0.000000pt}%
\pgfpathmoveto{\pgfqpoint{0.608025in}{2.971219in}}%
\pgfpathlineto{\pgfqpoint{3.801968in}{2.971219in}}%
\pgfusepath{stroke}%
\end{pgfscope}%
\begin{pgfscope}%
\pgfpathrectangle{\pgfqpoint{0.608025in}{0.554012in}}{\pgfqpoint{6.387885in}{4.834414in}}%
\pgfusepath{clip}%
\pgfsetbuttcap%
\pgfsetroundjoin%
\definecolor{currentfill}{rgb}{1.000000,0.000000,0.000000}%
\pgfsetfillcolor{currentfill}%
\pgfsetlinewidth{1.003750pt}%
\definecolor{currentstroke}{rgb}{1.000000,0.000000,0.000000}%
\pgfsetstrokecolor{currentstroke}%
\pgfsetdash{}{0pt}%
\pgfsys@defobject{currentmarker}{\pgfqpoint{-0.069444in}{-0.069444in}}{\pgfqpoint{0.069444in}{0.069444in}}{%
\pgfpathmoveto{\pgfqpoint{0.000000in}{-0.069444in}}%
\pgfpathcurveto{\pgfqpoint{0.018417in}{-0.069444in}}{\pgfqpoint{0.036082in}{-0.062127in}}{\pgfqpoint{0.049105in}{-0.049105in}}%
\pgfpathcurveto{\pgfqpoint{0.062127in}{-0.036082in}}{\pgfqpoint{0.069444in}{-0.018417in}}{\pgfqpoint{0.069444in}{0.000000in}}%
\pgfpathcurveto{\pgfqpoint{0.069444in}{0.018417in}}{\pgfqpoint{0.062127in}{0.036082in}}{\pgfqpoint{0.049105in}{0.049105in}}%
\pgfpathcurveto{\pgfqpoint{0.036082in}{0.062127in}}{\pgfqpoint{0.018417in}{0.069444in}}{\pgfqpoint{0.000000in}{0.069444in}}%
\pgfpathcurveto{\pgfqpoint{-0.018417in}{0.069444in}}{\pgfqpoint{-0.036082in}{0.062127in}}{\pgfqpoint{-0.049105in}{0.049105in}}%
\pgfpathcurveto{\pgfqpoint{-0.062127in}{0.036082in}}{\pgfqpoint{-0.069444in}{0.018417in}}{\pgfqpoint{-0.069444in}{0.000000in}}%
\pgfpathcurveto{\pgfqpoint{-0.069444in}{-0.018417in}}{\pgfqpoint{-0.062127in}{-0.036082in}}{\pgfqpoint{-0.049105in}{-0.049105in}}%
\pgfpathcurveto{\pgfqpoint{-0.036082in}{-0.062127in}}{\pgfqpoint{-0.018417in}{-0.069444in}}{\pgfqpoint{0.000000in}{-0.069444in}}%
\pgfpathlineto{\pgfqpoint{0.000000in}{-0.069444in}}%
\pgfpathclose%
\pgfusepath{stroke,fill}%
}%
\begin{pgfscope}%
\pgfsys@transformshift{3.801968in}{2.971219in}%
\pgfsys@useobject{currentmarker}{}%
\end{pgfscope}%
\end{pgfscope}%
\begin{pgfscope}%
\pgfsetrectcap%
\pgfsetmiterjoin%
\pgfsetlinewidth{0.803000pt}%
\definecolor{currentstroke}{rgb}{0.000000,0.000000,0.000000}%
\pgfsetstrokecolor{currentstroke}%
\pgfsetdash{}{0pt}%
\pgfpathmoveto{\pgfqpoint{0.608025in}{0.554012in}}%
\pgfpathlineto{\pgfqpoint{0.608025in}{5.388426in}}%
\pgfusepath{stroke}%
\end{pgfscope}%
\begin{pgfscope}%
\pgfsetrectcap%
\pgfsetmiterjoin%
\pgfsetlinewidth{0.803000pt}%
\definecolor{currentstroke}{rgb}{0.000000,0.000000,0.000000}%
\pgfsetstrokecolor{currentstroke}%
\pgfsetdash{}{0pt}%
\pgfpathmoveto{\pgfqpoint{6.995910in}{0.554012in}}%
\pgfpathlineto{\pgfqpoint{6.995910in}{5.388426in}}%
\pgfusepath{stroke}%
\end{pgfscope}%
\begin{pgfscope}%
\pgfsetrectcap%
\pgfsetmiterjoin%
\pgfsetlinewidth{0.803000pt}%
\definecolor{currentstroke}{rgb}{0.000000,0.000000,0.000000}%
\pgfsetstrokecolor{currentstroke}%
\pgfsetdash{}{0pt}%
\pgfpathmoveto{\pgfqpoint{0.608025in}{0.554012in}}%
\pgfpathlineto{\pgfqpoint{6.995910in}{0.554012in}}%
\pgfusepath{stroke}%
\end{pgfscope}%
\begin{pgfscope}%
\pgfsetrectcap%
\pgfsetmiterjoin%
\pgfsetlinewidth{0.803000pt}%
\definecolor{currentstroke}{rgb}{0.000000,0.000000,0.000000}%
\pgfsetstrokecolor{currentstroke}%
\pgfsetdash{}{0pt}%
\pgfpathmoveto{\pgfqpoint{0.608025in}{5.388426in}}%
\pgfpathlineto{\pgfqpoint{6.995910in}{5.388426in}}%
\pgfusepath{stroke}%
\end{pgfscope}%
\begin{pgfscope}%
\definecolor{textcolor}{rgb}{0.000000,0.000000,0.000000}%
\pgfsetstrokecolor{textcolor}%
\pgfsetfillcolor{textcolor}%
\pgftext[x=0.927419in,y=3.454660in,left,base]{\color{textcolor}\rmfamily\fontsize{12.000000}{14.400000}\selectfont Consumer Surplus}%
\end{pgfscope}%
\begin{pgfscope}%
\definecolor{textcolor}{rgb}{0.000000,0.000000,0.000000}%
\pgfsetstrokecolor{textcolor}%
\pgfsetfillcolor{textcolor}%
\pgftext[x=0.927419in,y=2.487778in,left,base]{\color{textcolor}\rmfamily\fontsize{12.000000}{14.400000}\selectfont Producer Surplus}%
\end{pgfscope}%
\begin{pgfscope}%
\definecolor{textcolor}{rgb}{0.000000,0.000000,0.000000}%
\pgfsetstrokecolor{textcolor}%
\pgfsetfillcolor{textcolor}%
\pgftext[x=5.573101in, y=3.990313in, left, base,rotate=45.000000]{\color{textcolor}\rmfamily\fontsize{12.000000}{14.400000}\selectfont Supply}%
\end{pgfscope}%
\begin{pgfscope}%
\definecolor{textcolor}{rgb}{0.000000,0.000000,0.000000}%
\pgfsetstrokecolor{textcolor}%
\pgfsetfillcolor{textcolor}%
\pgftext[x=5.463239in, y=1.432232in, left, base,rotate=320.000000]{\color{textcolor}\rmfamily\fontsize{12.000000}{14.400000}\selectfont Demand}%
\end{pgfscope}%
\begin{pgfscope}%
\pgfsetbuttcap%
\pgfsetmiterjoin%
\definecolor{currentfill}{rgb}{1.000000,1.000000,1.000000}%
\pgfsetfillcolor{currentfill}%
\pgfsetlinewidth{1.003750pt}%
\definecolor{currentstroke}{rgb}{0.000000,0.000000,0.000000}%
\pgfsetstrokecolor{currentstroke}%
\pgfsetdash{}{0pt}%
\pgfpathmoveto{\pgfqpoint{0.615515in}{5.040317in}}%
\pgfpathlineto{\pgfqpoint{0.902364in}{5.040317in}}%
\pgfpathlineto{\pgfqpoint{0.902364in}{5.353094in}}%
\pgfpathlineto{\pgfqpoint{0.615515in}{5.353094in}}%
\pgfpathlineto{\pgfqpoint{0.615515in}{5.040317in}}%
\pgfpathclose%
\pgfusepath{stroke,fill}%
\end{pgfscope}%
\begin{pgfscope}%
\definecolor{textcolor}{rgb}{0.000000,0.000000,0.000000}%
\pgfsetstrokecolor{textcolor}%
\pgfsetfillcolor{textcolor}%
\pgftext[x=0.671904in,y=5.146705in,left,base]{\color{textcolor}\rmfamily\fontsize{14.000000}{16.800000}\selectfont a)}%
\end{pgfscope}%
\begin{pgfscope}%
\pgfsetbuttcap%
\pgfsetmiterjoin%
\definecolor{currentfill}{rgb}{1.000000,1.000000,1.000000}%
\pgfsetfillcolor{currentfill}%
\pgfsetlinewidth{0.000000pt}%
\definecolor{currentstroke}{rgb}{0.000000,0.000000,0.000000}%
\pgfsetstrokecolor{currentstroke}%
\pgfsetstrokeopacity{0.000000}%
\pgfsetdash{}{0pt}%
\pgfpathmoveto{\pgfqpoint{7.323380in}{0.554012in}}%
\pgfpathlineto{\pgfqpoint{13.711265in}{0.554012in}}%
\pgfpathlineto{\pgfqpoint{13.711265in}{5.388426in}}%
\pgfpathlineto{\pgfqpoint{7.323380in}{5.388426in}}%
\pgfpathlineto{\pgfqpoint{7.323380in}{0.554012in}}%
\pgfpathclose%
\pgfusepath{fill}%
\end{pgfscope}%
\begin{pgfscope}%
\pgfpathrectangle{\pgfqpoint{7.323380in}{0.554012in}}{\pgfqpoint{6.387885in}{4.834414in}}%
\pgfusepath{clip}%
\pgfsetbuttcap%
\pgfsetroundjoin%
\definecolor{currentfill}{rgb}{0.121569,0.466667,0.705882}%
\pgfsetfillcolor{currentfill}%
\pgfsetfillopacity{0.200000}%
\pgfsetlinewidth{0.000000pt}%
\definecolor{currentstroke}{rgb}{0.000000,0.000000,0.000000}%
\pgfsetstrokecolor{currentstroke}%
\pgfsetdash{}{0pt}%
\pgfpathmoveto{\pgfqpoint{7.323380in}{226.694073in}}%
\pgfpathlineto{\pgfqpoint{7.323380in}{2.971219in}}%
\pgfpathlineto{\pgfqpoint{7.329774in}{2.971219in}}%
\pgfpathlineto{\pgfqpoint{7.336168in}{2.971219in}}%
\pgfpathlineto{\pgfqpoint{7.342563in}{2.971219in}}%
\pgfpathlineto{\pgfqpoint{7.348957in}{2.971219in}}%
\pgfpathlineto{\pgfqpoint{7.355351in}{2.971219in}}%
\pgfpathlineto{\pgfqpoint{7.361746in}{2.971219in}}%
\pgfpathlineto{\pgfqpoint{7.368140in}{2.971219in}}%
\pgfpathlineto{\pgfqpoint{7.374534in}{2.971219in}}%
\pgfpathlineto{\pgfqpoint{7.380928in}{2.971219in}}%
\pgfpathlineto{\pgfqpoint{7.387323in}{2.971219in}}%
\pgfpathlineto{\pgfqpoint{7.393717in}{2.971219in}}%
\pgfpathlineto{\pgfqpoint{7.400111in}{2.971219in}}%
\pgfpathlineto{\pgfqpoint{7.406505in}{2.971219in}}%
\pgfpathlineto{\pgfqpoint{7.412900in}{2.971219in}}%
\pgfpathlineto{\pgfqpoint{7.419294in}{2.971219in}}%
\pgfpathlineto{\pgfqpoint{7.425688in}{2.971219in}}%
\pgfpathlineto{\pgfqpoint{7.432083in}{2.971219in}}%
\pgfpathlineto{\pgfqpoint{7.438477in}{2.971219in}}%
\pgfpathlineto{\pgfqpoint{7.444871in}{2.971219in}}%
\pgfpathlineto{\pgfqpoint{7.451265in}{2.971219in}}%
\pgfpathlineto{\pgfqpoint{7.457660in}{2.971219in}}%
\pgfpathlineto{\pgfqpoint{7.464054in}{2.971219in}}%
\pgfpathlineto{\pgfqpoint{7.470448in}{2.971219in}}%
\pgfpathlineto{\pgfqpoint{7.476843in}{2.971219in}}%
\pgfpathlineto{\pgfqpoint{7.483237in}{2.971219in}}%
\pgfpathlineto{\pgfqpoint{7.489631in}{2.971219in}}%
\pgfpathlineto{\pgfqpoint{7.496025in}{2.971219in}}%
\pgfpathlineto{\pgfqpoint{7.502420in}{2.971219in}}%
\pgfpathlineto{\pgfqpoint{7.508814in}{2.971219in}}%
\pgfpathlineto{\pgfqpoint{7.515208in}{2.971219in}}%
\pgfpathlineto{\pgfqpoint{7.521603in}{2.971219in}}%
\pgfpathlineto{\pgfqpoint{7.527997in}{2.971219in}}%
\pgfpathlineto{\pgfqpoint{7.534391in}{2.971219in}}%
\pgfpathlineto{\pgfqpoint{7.540785in}{2.971219in}}%
\pgfpathlineto{\pgfqpoint{7.547180in}{2.971219in}}%
\pgfpathlineto{\pgfqpoint{7.553574in}{2.971219in}}%
\pgfpathlineto{\pgfqpoint{7.559968in}{2.971219in}}%
\pgfpathlineto{\pgfqpoint{7.566362in}{2.971219in}}%
\pgfpathlineto{\pgfqpoint{7.572757in}{2.971219in}}%
\pgfpathlineto{\pgfqpoint{7.579151in}{2.971219in}}%
\pgfpathlineto{\pgfqpoint{7.585545in}{2.971219in}}%
\pgfpathlineto{\pgfqpoint{7.591940in}{2.971219in}}%
\pgfpathlineto{\pgfqpoint{7.598334in}{2.971219in}}%
\pgfpathlineto{\pgfqpoint{7.604728in}{2.971219in}}%
\pgfpathlineto{\pgfqpoint{7.611122in}{2.971219in}}%
\pgfpathlineto{\pgfqpoint{7.617517in}{2.971219in}}%
\pgfpathlineto{\pgfqpoint{7.623911in}{2.971219in}}%
\pgfpathlineto{\pgfqpoint{7.630305in}{2.971219in}}%
\pgfpathlineto{\pgfqpoint{7.636700in}{2.971219in}}%
\pgfpathlineto{\pgfqpoint{7.643094in}{2.971219in}}%
\pgfpathlineto{\pgfqpoint{7.649488in}{2.971219in}}%
\pgfpathlineto{\pgfqpoint{7.655882in}{2.971219in}}%
\pgfpathlineto{\pgfqpoint{7.662277in}{2.971219in}}%
\pgfpathlineto{\pgfqpoint{7.668671in}{2.971219in}}%
\pgfpathlineto{\pgfqpoint{7.675065in}{2.971219in}}%
\pgfpathlineto{\pgfqpoint{7.681460in}{2.971219in}}%
\pgfpathlineto{\pgfqpoint{7.687854in}{2.971219in}}%
\pgfpathlineto{\pgfqpoint{7.694248in}{2.971219in}}%
\pgfpathlineto{\pgfqpoint{7.700642in}{2.971219in}}%
\pgfpathlineto{\pgfqpoint{7.707037in}{2.971219in}}%
\pgfpathlineto{\pgfqpoint{7.713431in}{2.971219in}}%
\pgfpathlineto{\pgfqpoint{7.719825in}{2.971219in}}%
\pgfpathlineto{\pgfqpoint{7.726219in}{2.971219in}}%
\pgfpathlineto{\pgfqpoint{7.732614in}{2.971219in}}%
\pgfpathlineto{\pgfqpoint{7.739008in}{2.971219in}}%
\pgfpathlineto{\pgfqpoint{7.745402in}{2.971219in}}%
\pgfpathlineto{\pgfqpoint{7.751797in}{2.971219in}}%
\pgfpathlineto{\pgfqpoint{7.758191in}{2.971219in}}%
\pgfpathlineto{\pgfqpoint{7.764585in}{2.971219in}}%
\pgfpathlineto{\pgfqpoint{7.770979in}{2.971219in}}%
\pgfpathlineto{\pgfqpoint{7.777374in}{2.971219in}}%
\pgfpathlineto{\pgfqpoint{7.783768in}{2.971219in}}%
\pgfpathlineto{\pgfqpoint{7.790162in}{2.971219in}}%
\pgfpathlineto{\pgfqpoint{7.796557in}{2.971219in}}%
\pgfpathlineto{\pgfqpoint{7.802951in}{2.971219in}}%
\pgfpathlineto{\pgfqpoint{7.809345in}{2.971219in}}%
\pgfpathlineto{\pgfqpoint{7.815739in}{2.971219in}}%
\pgfpathlineto{\pgfqpoint{7.822134in}{2.971219in}}%
\pgfpathlineto{\pgfqpoint{7.828528in}{2.971219in}}%
\pgfpathlineto{\pgfqpoint{7.834922in}{2.971219in}}%
\pgfpathlineto{\pgfqpoint{7.841317in}{2.971219in}}%
\pgfpathlineto{\pgfqpoint{7.847711in}{2.971219in}}%
\pgfpathlineto{\pgfqpoint{7.854105in}{2.971219in}}%
\pgfpathlineto{\pgfqpoint{7.860499in}{2.971219in}}%
\pgfpathlineto{\pgfqpoint{7.866894in}{2.971219in}}%
\pgfpathlineto{\pgfqpoint{7.873288in}{2.971219in}}%
\pgfpathlineto{\pgfqpoint{7.879682in}{2.971219in}}%
\pgfpathlineto{\pgfqpoint{7.886076in}{2.971219in}}%
\pgfpathlineto{\pgfqpoint{7.892471in}{2.971219in}}%
\pgfpathlineto{\pgfqpoint{7.898865in}{2.971219in}}%
\pgfpathlineto{\pgfqpoint{7.905259in}{2.971219in}}%
\pgfpathlineto{\pgfqpoint{7.911654in}{2.971219in}}%
\pgfpathlineto{\pgfqpoint{7.918048in}{2.971219in}}%
\pgfpathlineto{\pgfqpoint{7.924442in}{2.971219in}}%
\pgfpathlineto{\pgfqpoint{7.930836in}{2.971219in}}%
\pgfpathlineto{\pgfqpoint{7.937231in}{2.971219in}}%
\pgfpathlineto{\pgfqpoint{7.943625in}{2.971219in}}%
\pgfpathlineto{\pgfqpoint{7.950019in}{2.971219in}}%
\pgfpathlineto{\pgfqpoint{7.956414in}{2.971219in}}%
\pgfpathlineto{\pgfqpoint{7.962808in}{2.971219in}}%
\pgfpathlineto{\pgfqpoint{7.969202in}{2.971219in}}%
\pgfpathlineto{\pgfqpoint{7.975596in}{2.971219in}}%
\pgfpathlineto{\pgfqpoint{7.981991in}{2.971219in}}%
\pgfpathlineto{\pgfqpoint{7.988385in}{2.971219in}}%
\pgfpathlineto{\pgfqpoint{7.994779in}{2.971219in}}%
\pgfpathlineto{\pgfqpoint{8.001173in}{2.971219in}}%
\pgfpathlineto{\pgfqpoint{8.007568in}{2.971219in}}%
\pgfpathlineto{\pgfqpoint{8.013962in}{2.971219in}}%
\pgfpathlineto{\pgfqpoint{8.020356in}{2.971219in}}%
\pgfpathlineto{\pgfqpoint{8.026751in}{2.971219in}}%
\pgfpathlineto{\pgfqpoint{8.033145in}{2.971219in}}%
\pgfpathlineto{\pgfqpoint{8.039539in}{2.971219in}}%
\pgfpathlineto{\pgfqpoint{8.045933in}{2.971219in}}%
\pgfpathlineto{\pgfqpoint{8.052328in}{2.971219in}}%
\pgfpathlineto{\pgfqpoint{8.058722in}{2.971219in}}%
\pgfpathlineto{\pgfqpoint{8.065116in}{2.971219in}}%
\pgfpathlineto{\pgfqpoint{8.071511in}{2.971219in}}%
\pgfpathlineto{\pgfqpoint{8.077905in}{2.971219in}}%
\pgfpathlineto{\pgfqpoint{8.084299in}{2.971219in}}%
\pgfpathlineto{\pgfqpoint{8.090693in}{2.971219in}}%
\pgfpathlineto{\pgfqpoint{8.097088in}{2.971219in}}%
\pgfpathlineto{\pgfqpoint{8.103482in}{2.971219in}}%
\pgfpathlineto{\pgfqpoint{8.109876in}{2.971219in}}%
\pgfpathlineto{\pgfqpoint{8.116271in}{2.971219in}}%
\pgfpathlineto{\pgfqpoint{8.122665in}{2.971219in}}%
\pgfpathlineto{\pgfqpoint{8.129059in}{2.971219in}}%
\pgfpathlineto{\pgfqpoint{8.135453in}{2.971219in}}%
\pgfpathlineto{\pgfqpoint{8.141848in}{2.971219in}}%
\pgfpathlineto{\pgfqpoint{8.148242in}{2.971219in}}%
\pgfpathlineto{\pgfqpoint{8.154636in}{2.971219in}}%
\pgfpathlineto{\pgfqpoint{8.161030in}{2.971219in}}%
\pgfpathlineto{\pgfqpoint{8.167425in}{2.971219in}}%
\pgfpathlineto{\pgfqpoint{8.173819in}{2.971219in}}%
\pgfpathlineto{\pgfqpoint{8.180213in}{2.971219in}}%
\pgfpathlineto{\pgfqpoint{8.186608in}{2.971219in}}%
\pgfpathlineto{\pgfqpoint{8.193002in}{2.971219in}}%
\pgfpathlineto{\pgfqpoint{8.199396in}{2.971219in}}%
\pgfpathlineto{\pgfqpoint{8.205790in}{2.971219in}}%
\pgfpathlineto{\pgfqpoint{8.212185in}{2.971219in}}%
\pgfpathlineto{\pgfqpoint{8.218579in}{2.971219in}}%
\pgfpathlineto{\pgfqpoint{8.224973in}{2.971219in}}%
\pgfpathlineto{\pgfqpoint{8.231368in}{2.971219in}}%
\pgfpathlineto{\pgfqpoint{8.237762in}{2.971219in}}%
\pgfpathlineto{\pgfqpoint{8.244156in}{2.971219in}}%
\pgfpathlineto{\pgfqpoint{8.250550in}{2.971219in}}%
\pgfpathlineto{\pgfqpoint{8.256945in}{2.971219in}}%
\pgfpathlineto{\pgfqpoint{8.263339in}{2.971219in}}%
\pgfpathlineto{\pgfqpoint{8.269733in}{2.971219in}}%
\pgfpathlineto{\pgfqpoint{8.276128in}{2.971219in}}%
\pgfpathlineto{\pgfqpoint{8.282522in}{2.971219in}}%
\pgfpathlineto{\pgfqpoint{8.288916in}{2.971219in}}%
\pgfpathlineto{\pgfqpoint{8.295310in}{2.971219in}}%
\pgfpathlineto{\pgfqpoint{8.301705in}{2.971219in}}%
\pgfpathlineto{\pgfqpoint{8.308099in}{2.971219in}}%
\pgfpathlineto{\pgfqpoint{8.314493in}{2.971219in}}%
\pgfpathlineto{\pgfqpoint{8.320887in}{2.971219in}}%
\pgfpathlineto{\pgfqpoint{8.327282in}{2.971219in}}%
\pgfpathlineto{\pgfqpoint{8.333676in}{2.971219in}}%
\pgfpathlineto{\pgfqpoint{8.340070in}{2.971219in}}%
\pgfpathlineto{\pgfqpoint{8.346465in}{2.971219in}}%
\pgfpathlineto{\pgfqpoint{8.352859in}{2.971219in}}%
\pgfpathlineto{\pgfqpoint{8.359253in}{2.971219in}}%
\pgfpathlineto{\pgfqpoint{8.365647in}{2.971219in}}%
\pgfpathlineto{\pgfqpoint{8.372042in}{2.971219in}}%
\pgfpathlineto{\pgfqpoint{8.378436in}{2.971219in}}%
\pgfpathlineto{\pgfqpoint{8.384830in}{2.971219in}}%
\pgfpathlineto{\pgfqpoint{8.391225in}{2.971219in}}%
\pgfpathlineto{\pgfqpoint{8.397619in}{2.971219in}}%
\pgfpathlineto{\pgfqpoint{8.404013in}{2.971219in}}%
\pgfpathlineto{\pgfqpoint{8.410407in}{2.971219in}}%
\pgfpathlineto{\pgfqpoint{8.416802in}{2.971219in}}%
\pgfpathlineto{\pgfqpoint{8.423196in}{2.971219in}}%
\pgfpathlineto{\pgfqpoint{8.429590in}{2.971219in}}%
\pgfpathlineto{\pgfqpoint{8.435985in}{2.971219in}}%
\pgfpathlineto{\pgfqpoint{8.442379in}{2.971219in}}%
\pgfpathlineto{\pgfqpoint{8.448773in}{2.971219in}}%
\pgfpathlineto{\pgfqpoint{8.455167in}{2.971219in}}%
\pgfpathlineto{\pgfqpoint{8.461562in}{2.971219in}}%
\pgfpathlineto{\pgfqpoint{8.467956in}{2.971219in}}%
\pgfpathlineto{\pgfqpoint{8.474350in}{2.971219in}}%
\pgfpathlineto{\pgfqpoint{8.480744in}{2.971219in}}%
\pgfpathlineto{\pgfqpoint{8.487139in}{2.971219in}}%
\pgfpathlineto{\pgfqpoint{8.493533in}{2.971219in}}%
\pgfpathlineto{\pgfqpoint{8.499927in}{2.971219in}}%
\pgfpathlineto{\pgfqpoint{8.506322in}{2.971219in}}%
\pgfpathlineto{\pgfqpoint{8.512716in}{2.971219in}}%
\pgfpathlineto{\pgfqpoint{8.519110in}{2.971219in}}%
\pgfpathlineto{\pgfqpoint{8.525504in}{2.971219in}}%
\pgfpathlineto{\pgfqpoint{8.531899in}{2.971219in}}%
\pgfpathlineto{\pgfqpoint{8.538293in}{2.971219in}}%
\pgfpathlineto{\pgfqpoint{8.544687in}{2.971219in}}%
\pgfpathlineto{\pgfqpoint{8.551082in}{2.971219in}}%
\pgfpathlineto{\pgfqpoint{8.557476in}{2.971219in}}%
\pgfpathlineto{\pgfqpoint{8.563870in}{2.971219in}}%
\pgfpathlineto{\pgfqpoint{8.570264in}{2.971219in}}%
\pgfpathlineto{\pgfqpoint{8.576659in}{2.971219in}}%
\pgfpathlineto{\pgfqpoint{8.583053in}{2.971219in}}%
\pgfpathlineto{\pgfqpoint{8.589447in}{2.971219in}}%
\pgfpathlineto{\pgfqpoint{8.595841in}{2.971219in}}%
\pgfpathlineto{\pgfqpoint{8.602236in}{2.971219in}}%
\pgfpathlineto{\pgfqpoint{8.608630in}{2.971219in}}%
\pgfpathlineto{\pgfqpoint{8.615024in}{2.971219in}}%
\pgfpathlineto{\pgfqpoint{8.621419in}{2.971219in}}%
\pgfpathlineto{\pgfqpoint{8.627813in}{2.971219in}}%
\pgfpathlineto{\pgfqpoint{8.634207in}{2.971219in}}%
\pgfpathlineto{\pgfqpoint{8.640601in}{2.971219in}}%
\pgfpathlineto{\pgfqpoint{8.646996in}{2.971219in}}%
\pgfpathlineto{\pgfqpoint{8.653390in}{2.971219in}}%
\pgfpathlineto{\pgfqpoint{8.659784in}{2.971219in}}%
\pgfpathlineto{\pgfqpoint{8.666179in}{2.971219in}}%
\pgfpathlineto{\pgfqpoint{8.672573in}{2.971219in}}%
\pgfpathlineto{\pgfqpoint{8.678967in}{2.971219in}}%
\pgfpathlineto{\pgfqpoint{8.685361in}{2.971219in}}%
\pgfpathlineto{\pgfqpoint{8.691756in}{2.971219in}}%
\pgfpathlineto{\pgfqpoint{8.698150in}{2.971219in}}%
\pgfpathlineto{\pgfqpoint{8.704544in}{2.971219in}}%
\pgfpathlineto{\pgfqpoint{8.710939in}{2.971219in}}%
\pgfpathlineto{\pgfqpoint{8.717333in}{2.971219in}}%
\pgfpathlineto{\pgfqpoint{8.723727in}{2.971219in}}%
\pgfpathlineto{\pgfqpoint{8.730121in}{2.971219in}}%
\pgfpathlineto{\pgfqpoint{8.736516in}{2.971219in}}%
\pgfpathlineto{\pgfqpoint{8.742910in}{2.971219in}}%
\pgfpathlineto{\pgfqpoint{8.749304in}{2.971219in}}%
\pgfpathlineto{\pgfqpoint{8.755698in}{2.971219in}}%
\pgfpathlineto{\pgfqpoint{8.762093in}{2.971219in}}%
\pgfpathlineto{\pgfqpoint{8.768487in}{2.971219in}}%
\pgfpathlineto{\pgfqpoint{8.774881in}{2.971219in}}%
\pgfpathlineto{\pgfqpoint{8.781276in}{2.971219in}}%
\pgfpathlineto{\pgfqpoint{8.787670in}{2.971219in}}%
\pgfpathlineto{\pgfqpoint{8.794064in}{2.971219in}}%
\pgfpathlineto{\pgfqpoint{8.800458in}{2.971219in}}%
\pgfpathlineto{\pgfqpoint{8.806853in}{2.971219in}}%
\pgfpathlineto{\pgfqpoint{8.813247in}{2.971219in}}%
\pgfpathlineto{\pgfqpoint{8.819641in}{2.971219in}}%
\pgfpathlineto{\pgfqpoint{8.826036in}{2.971219in}}%
\pgfpathlineto{\pgfqpoint{8.832430in}{2.971219in}}%
\pgfpathlineto{\pgfqpoint{8.838824in}{2.971219in}}%
\pgfpathlineto{\pgfqpoint{8.845218in}{2.971219in}}%
\pgfpathlineto{\pgfqpoint{8.851613in}{2.971219in}}%
\pgfpathlineto{\pgfqpoint{8.858007in}{2.971219in}}%
\pgfpathlineto{\pgfqpoint{8.864401in}{2.971219in}}%
\pgfpathlineto{\pgfqpoint{8.870796in}{2.971219in}}%
\pgfpathlineto{\pgfqpoint{8.877190in}{2.971219in}}%
\pgfpathlineto{\pgfqpoint{8.883584in}{2.971219in}}%
\pgfpathlineto{\pgfqpoint{8.889978in}{2.971219in}}%
\pgfpathlineto{\pgfqpoint{8.896373in}{2.971219in}}%
\pgfpathlineto{\pgfqpoint{8.902767in}{2.971219in}}%
\pgfpathlineto{\pgfqpoint{8.909161in}{2.971219in}}%
\pgfpathlineto{\pgfqpoint{8.915555in}{2.971219in}}%
\pgfpathlineto{\pgfqpoint{8.921950in}{2.971219in}}%
\pgfpathlineto{\pgfqpoint{8.928344in}{2.971219in}}%
\pgfpathlineto{\pgfqpoint{8.934738in}{2.971219in}}%
\pgfpathlineto{\pgfqpoint{8.941133in}{2.971219in}}%
\pgfpathlineto{\pgfqpoint{8.947527in}{2.971219in}}%
\pgfpathlineto{\pgfqpoint{8.953921in}{2.971219in}}%
\pgfpathlineto{\pgfqpoint{8.960315in}{2.971219in}}%
\pgfpathlineto{\pgfqpoint{8.966710in}{2.971219in}}%
\pgfpathlineto{\pgfqpoint{8.973104in}{2.971219in}}%
\pgfpathlineto{\pgfqpoint{8.979498in}{2.971219in}}%
\pgfpathlineto{\pgfqpoint{8.985893in}{2.971219in}}%
\pgfpathlineto{\pgfqpoint{8.992287in}{2.971219in}}%
\pgfpathlineto{\pgfqpoint{8.998681in}{2.971219in}}%
\pgfpathlineto{\pgfqpoint{9.005075in}{2.971219in}}%
\pgfpathlineto{\pgfqpoint{9.011470in}{2.971219in}}%
\pgfpathlineto{\pgfqpoint{9.017864in}{2.971219in}}%
\pgfpathlineto{\pgfqpoint{9.024258in}{2.971219in}}%
\pgfpathlineto{\pgfqpoint{9.030653in}{2.971219in}}%
\pgfpathlineto{\pgfqpoint{9.037047in}{2.971219in}}%
\pgfpathlineto{\pgfqpoint{9.043441in}{2.971219in}}%
\pgfpathlineto{\pgfqpoint{9.049835in}{2.971219in}}%
\pgfpathlineto{\pgfqpoint{9.056230in}{2.971219in}}%
\pgfpathlineto{\pgfqpoint{9.062624in}{2.971219in}}%
\pgfpathlineto{\pgfqpoint{9.069018in}{2.971219in}}%
\pgfpathlineto{\pgfqpoint{9.075412in}{2.971219in}}%
\pgfpathlineto{\pgfqpoint{9.081807in}{2.971219in}}%
\pgfpathlineto{\pgfqpoint{9.088201in}{2.971219in}}%
\pgfpathlineto{\pgfqpoint{9.094595in}{2.971219in}}%
\pgfpathlineto{\pgfqpoint{9.100990in}{2.971219in}}%
\pgfpathlineto{\pgfqpoint{9.107384in}{2.971219in}}%
\pgfpathlineto{\pgfqpoint{9.113778in}{2.971219in}}%
\pgfpathlineto{\pgfqpoint{9.120172in}{2.971219in}}%
\pgfpathlineto{\pgfqpoint{9.126567in}{2.971219in}}%
\pgfpathlineto{\pgfqpoint{9.132961in}{2.971219in}}%
\pgfpathlineto{\pgfqpoint{9.139355in}{2.971219in}}%
\pgfpathlineto{\pgfqpoint{9.145750in}{2.971219in}}%
\pgfpathlineto{\pgfqpoint{9.152144in}{2.971219in}}%
\pgfpathlineto{\pgfqpoint{9.158538in}{2.971219in}}%
\pgfpathlineto{\pgfqpoint{9.164932in}{2.971219in}}%
\pgfpathlineto{\pgfqpoint{9.171327in}{2.971219in}}%
\pgfpathlineto{\pgfqpoint{9.177721in}{2.971219in}}%
\pgfpathlineto{\pgfqpoint{9.184115in}{2.971219in}}%
\pgfpathlineto{\pgfqpoint{9.190509in}{2.971219in}}%
\pgfpathlineto{\pgfqpoint{9.196904in}{2.971219in}}%
\pgfpathlineto{\pgfqpoint{9.203298in}{2.971219in}}%
\pgfpathlineto{\pgfqpoint{9.209692in}{2.971219in}}%
\pgfpathlineto{\pgfqpoint{9.216087in}{2.971219in}}%
\pgfpathlineto{\pgfqpoint{9.222481in}{2.971219in}}%
\pgfpathlineto{\pgfqpoint{9.228875in}{2.971219in}}%
\pgfpathlineto{\pgfqpoint{9.235269in}{2.971219in}}%
\pgfpathlineto{\pgfqpoint{9.241664in}{2.971219in}}%
\pgfpathlineto{\pgfqpoint{9.248058in}{2.971219in}}%
\pgfpathlineto{\pgfqpoint{9.254452in}{2.971219in}}%
\pgfpathlineto{\pgfqpoint{9.260847in}{2.971219in}}%
\pgfpathlineto{\pgfqpoint{9.267241in}{2.971219in}}%
\pgfpathlineto{\pgfqpoint{9.273635in}{2.971219in}}%
\pgfpathlineto{\pgfqpoint{9.280029in}{2.971219in}}%
\pgfpathlineto{\pgfqpoint{9.286424in}{2.971219in}}%
\pgfpathlineto{\pgfqpoint{9.292818in}{2.971219in}}%
\pgfpathlineto{\pgfqpoint{9.299212in}{2.971219in}}%
\pgfpathlineto{\pgfqpoint{9.305607in}{2.971219in}}%
\pgfpathlineto{\pgfqpoint{9.312001in}{2.971219in}}%
\pgfpathlineto{\pgfqpoint{9.318395in}{2.971219in}}%
\pgfpathlineto{\pgfqpoint{9.324789in}{2.971219in}}%
\pgfpathlineto{\pgfqpoint{9.331184in}{2.971219in}}%
\pgfpathlineto{\pgfqpoint{9.337578in}{2.971219in}}%
\pgfpathlineto{\pgfqpoint{9.343972in}{2.971219in}}%
\pgfpathlineto{\pgfqpoint{9.350366in}{2.971219in}}%
\pgfpathlineto{\pgfqpoint{9.356761in}{2.971219in}}%
\pgfpathlineto{\pgfqpoint{9.363155in}{2.971219in}}%
\pgfpathlineto{\pgfqpoint{9.369549in}{2.971219in}}%
\pgfpathlineto{\pgfqpoint{9.375944in}{2.971219in}}%
\pgfpathlineto{\pgfqpoint{9.382338in}{2.971219in}}%
\pgfpathlineto{\pgfqpoint{9.388732in}{2.971219in}}%
\pgfpathlineto{\pgfqpoint{9.395126in}{2.971219in}}%
\pgfpathlineto{\pgfqpoint{9.401521in}{2.971219in}}%
\pgfpathlineto{\pgfqpoint{9.407915in}{2.971219in}}%
\pgfpathlineto{\pgfqpoint{9.414309in}{2.971219in}}%
\pgfpathlineto{\pgfqpoint{9.420704in}{2.971219in}}%
\pgfpathlineto{\pgfqpoint{9.427098in}{2.971219in}}%
\pgfpathlineto{\pgfqpoint{9.433492in}{2.971219in}}%
\pgfpathlineto{\pgfqpoint{9.439886in}{2.971219in}}%
\pgfpathlineto{\pgfqpoint{9.446281in}{2.971219in}}%
\pgfpathlineto{\pgfqpoint{9.452675in}{2.971219in}}%
\pgfpathlineto{\pgfqpoint{9.459069in}{2.971219in}}%
\pgfpathlineto{\pgfqpoint{9.465464in}{2.971219in}}%
\pgfpathlineto{\pgfqpoint{9.471858in}{2.971219in}}%
\pgfpathlineto{\pgfqpoint{9.478252in}{2.971219in}}%
\pgfpathlineto{\pgfqpoint{9.484646in}{2.971219in}}%
\pgfpathlineto{\pgfqpoint{9.491041in}{2.971219in}}%
\pgfpathlineto{\pgfqpoint{9.497435in}{2.971219in}}%
\pgfpathlineto{\pgfqpoint{9.503829in}{2.971219in}}%
\pgfpathlineto{\pgfqpoint{9.510223in}{2.971219in}}%
\pgfpathlineto{\pgfqpoint{9.516618in}{2.971219in}}%
\pgfpathlineto{\pgfqpoint{9.523012in}{2.971219in}}%
\pgfpathlineto{\pgfqpoint{9.529406in}{2.971219in}}%
\pgfpathlineto{\pgfqpoint{9.535801in}{2.971219in}}%
\pgfpathlineto{\pgfqpoint{9.542195in}{2.971219in}}%
\pgfpathlineto{\pgfqpoint{9.548589in}{2.971219in}}%
\pgfpathlineto{\pgfqpoint{9.554983in}{2.971219in}}%
\pgfpathlineto{\pgfqpoint{9.561378in}{2.971219in}}%
\pgfpathlineto{\pgfqpoint{9.567772in}{2.971219in}}%
\pgfpathlineto{\pgfqpoint{9.574166in}{2.971219in}}%
\pgfpathlineto{\pgfqpoint{9.580561in}{2.971219in}}%
\pgfpathlineto{\pgfqpoint{9.586955in}{2.971219in}}%
\pgfpathlineto{\pgfqpoint{9.593349in}{2.971219in}}%
\pgfpathlineto{\pgfqpoint{9.599743in}{2.971219in}}%
\pgfpathlineto{\pgfqpoint{9.606138in}{2.971219in}}%
\pgfpathlineto{\pgfqpoint{9.612532in}{2.971219in}}%
\pgfpathlineto{\pgfqpoint{9.618926in}{2.971219in}}%
\pgfpathlineto{\pgfqpoint{9.625321in}{2.971219in}}%
\pgfpathlineto{\pgfqpoint{9.631715in}{2.971219in}}%
\pgfpathlineto{\pgfqpoint{9.638109in}{2.971219in}}%
\pgfpathlineto{\pgfqpoint{9.644503in}{2.971219in}}%
\pgfpathlineto{\pgfqpoint{9.650898in}{2.971219in}}%
\pgfpathlineto{\pgfqpoint{9.657292in}{2.971219in}}%
\pgfpathlineto{\pgfqpoint{9.663686in}{2.971219in}}%
\pgfpathlineto{\pgfqpoint{9.670080in}{2.971219in}}%
\pgfpathlineto{\pgfqpoint{9.676475in}{2.971219in}}%
\pgfpathlineto{\pgfqpoint{9.682869in}{2.971219in}}%
\pgfpathlineto{\pgfqpoint{9.689263in}{2.971219in}}%
\pgfpathlineto{\pgfqpoint{9.695658in}{2.971219in}}%
\pgfpathlineto{\pgfqpoint{9.702052in}{2.971219in}}%
\pgfpathlineto{\pgfqpoint{9.708446in}{2.971219in}}%
\pgfpathlineto{\pgfqpoint{9.714840in}{2.971219in}}%
\pgfpathlineto{\pgfqpoint{9.721235in}{2.971219in}}%
\pgfpathlineto{\pgfqpoint{9.727629in}{2.971219in}}%
\pgfpathlineto{\pgfqpoint{9.734023in}{2.971219in}}%
\pgfpathlineto{\pgfqpoint{9.740418in}{2.971219in}}%
\pgfpathlineto{\pgfqpoint{9.746812in}{2.971219in}}%
\pgfpathlineto{\pgfqpoint{9.753206in}{2.971219in}}%
\pgfpathlineto{\pgfqpoint{9.759600in}{2.971219in}}%
\pgfpathlineto{\pgfqpoint{9.765995in}{2.971219in}}%
\pgfpathlineto{\pgfqpoint{9.772389in}{2.971219in}}%
\pgfpathlineto{\pgfqpoint{9.778783in}{2.971219in}}%
\pgfpathlineto{\pgfqpoint{9.785177in}{2.971219in}}%
\pgfpathlineto{\pgfqpoint{9.791572in}{2.971219in}}%
\pgfpathlineto{\pgfqpoint{9.797966in}{2.971219in}}%
\pgfpathlineto{\pgfqpoint{9.804360in}{2.971219in}}%
\pgfpathlineto{\pgfqpoint{9.810755in}{2.971219in}}%
\pgfpathlineto{\pgfqpoint{9.817149in}{2.971219in}}%
\pgfpathlineto{\pgfqpoint{9.823543in}{2.971219in}}%
\pgfpathlineto{\pgfqpoint{9.829937in}{2.971219in}}%
\pgfpathlineto{\pgfqpoint{9.836332in}{2.971219in}}%
\pgfpathlineto{\pgfqpoint{9.842726in}{2.971219in}}%
\pgfpathlineto{\pgfqpoint{9.849120in}{2.971219in}}%
\pgfpathlineto{\pgfqpoint{9.855515in}{2.971219in}}%
\pgfpathlineto{\pgfqpoint{9.861909in}{2.971219in}}%
\pgfpathlineto{\pgfqpoint{9.868303in}{2.971219in}}%
\pgfpathlineto{\pgfqpoint{9.874697in}{2.971219in}}%
\pgfpathlineto{\pgfqpoint{9.881092in}{2.971219in}}%
\pgfpathlineto{\pgfqpoint{9.887486in}{2.971219in}}%
\pgfpathlineto{\pgfqpoint{9.893880in}{2.971219in}}%
\pgfpathlineto{\pgfqpoint{9.900275in}{2.971219in}}%
\pgfpathlineto{\pgfqpoint{9.906669in}{2.971219in}}%
\pgfpathlineto{\pgfqpoint{9.913063in}{2.971219in}}%
\pgfpathlineto{\pgfqpoint{9.919457in}{2.971219in}}%
\pgfpathlineto{\pgfqpoint{9.925852in}{2.971219in}}%
\pgfpathlineto{\pgfqpoint{9.932246in}{2.971219in}}%
\pgfpathlineto{\pgfqpoint{9.938640in}{2.971219in}}%
\pgfpathlineto{\pgfqpoint{9.945034in}{2.971219in}}%
\pgfpathlineto{\pgfqpoint{9.951429in}{2.971219in}}%
\pgfpathlineto{\pgfqpoint{9.957823in}{2.971219in}}%
\pgfpathlineto{\pgfqpoint{9.964217in}{2.971219in}}%
\pgfpathlineto{\pgfqpoint{9.970612in}{2.971219in}}%
\pgfpathlineto{\pgfqpoint{9.977006in}{2.971219in}}%
\pgfpathlineto{\pgfqpoint{9.983400in}{2.971219in}}%
\pgfpathlineto{\pgfqpoint{9.989794in}{2.971219in}}%
\pgfpathlineto{\pgfqpoint{9.996189in}{2.971219in}}%
\pgfpathlineto{\pgfqpoint{10.002583in}{2.971219in}}%
\pgfpathlineto{\pgfqpoint{10.008977in}{2.971219in}}%
\pgfpathlineto{\pgfqpoint{10.015372in}{2.971219in}}%
\pgfpathlineto{\pgfqpoint{10.021766in}{2.971219in}}%
\pgfpathlineto{\pgfqpoint{10.028160in}{2.971219in}}%
\pgfpathlineto{\pgfqpoint{10.034554in}{2.971219in}}%
\pgfpathlineto{\pgfqpoint{10.040949in}{2.971219in}}%
\pgfpathlineto{\pgfqpoint{10.047343in}{2.971219in}}%
\pgfpathlineto{\pgfqpoint{10.053737in}{2.971219in}}%
\pgfpathlineto{\pgfqpoint{10.060132in}{2.971219in}}%
\pgfpathlineto{\pgfqpoint{10.066526in}{2.971219in}}%
\pgfpathlineto{\pgfqpoint{10.072920in}{2.971219in}}%
\pgfpathlineto{\pgfqpoint{10.079314in}{2.971219in}}%
\pgfpathlineto{\pgfqpoint{10.085709in}{2.971219in}}%
\pgfpathlineto{\pgfqpoint{10.092103in}{2.971219in}}%
\pgfpathlineto{\pgfqpoint{10.098497in}{2.971219in}}%
\pgfpathlineto{\pgfqpoint{10.104891in}{2.971219in}}%
\pgfpathlineto{\pgfqpoint{10.111286in}{2.971219in}}%
\pgfpathlineto{\pgfqpoint{10.117680in}{2.971219in}}%
\pgfpathlineto{\pgfqpoint{10.124074in}{2.971219in}}%
\pgfpathlineto{\pgfqpoint{10.130469in}{2.971219in}}%
\pgfpathlineto{\pgfqpoint{10.136863in}{2.971219in}}%
\pgfpathlineto{\pgfqpoint{10.143257in}{2.971219in}}%
\pgfpathlineto{\pgfqpoint{10.149651in}{2.971219in}}%
\pgfpathlineto{\pgfqpoint{10.156046in}{2.971219in}}%
\pgfpathlineto{\pgfqpoint{10.162440in}{2.971219in}}%
\pgfpathlineto{\pgfqpoint{10.168834in}{2.971219in}}%
\pgfpathlineto{\pgfqpoint{10.175229in}{2.971219in}}%
\pgfpathlineto{\pgfqpoint{10.181623in}{2.971219in}}%
\pgfpathlineto{\pgfqpoint{10.188017in}{2.971219in}}%
\pgfpathlineto{\pgfqpoint{10.194411in}{2.971219in}}%
\pgfpathlineto{\pgfqpoint{10.200806in}{2.971219in}}%
\pgfpathlineto{\pgfqpoint{10.207200in}{2.971219in}}%
\pgfpathlineto{\pgfqpoint{10.213594in}{2.971219in}}%
\pgfpathlineto{\pgfqpoint{10.219989in}{2.971219in}}%
\pgfpathlineto{\pgfqpoint{10.226383in}{2.971219in}}%
\pgfpathlineto{\pgfqpoint{10.232777in}{2.971219in}}%
\pgfpathlineto{\pgfqpoint{10.239171in}{2.971219in}}%
\pgfpathlineto{\pgfqpoint{10.245566in}{2.971219in}}%
\pgfpathlineto{\pgfqpoint{10.251960in}{2.971219in}}%
\pgfpathlineto{\pgfqpoint{10.258354in}{2.971219in}}%
\pgfpathlineto{\pgfqpoint{10.264748in}{2.971219in}}%
\pgfpathlineto{\pgfqpoint{10.271143in}{2.971219in}}%
\pgfpathlineto{\pgfqpoint{10.277537in}{2.971219in}}%
\pgfpathlineto{\pgfqpoint{10.283931in}{2.971219in}}%
\pgfpathlineto{\pgfqpoint{10.290326in}{2.971219in}}%
\pgfpathlineto{\pgfqpoint{10.296720in}{2.971219in}}%
\pgfpathlineto{\pgfqpoint{10.303114in}{2.971219in}}%
\pgfpathlineto{\pgfqpoint{10.309508in}{2.971219in}}%
\pgfpathlineto{\pgfqpoint{10.315903in}{2.971219in}}%
\pgfpathlineto{\pgfqpoint{10.322297in}{2.971219in}}%
\pgfpathlineto{\pgfqpoint{10.328691in}{2.971219in}}%
\pgfpathlineto{\pgfqpoint{10.335086in}{2.971219in}}%
\pgfpathlineto{\pgfqpoint{10.341480in}{2.971219in}}%
\pgfpathlineto{\pgfqpoint{10.347874in}{2.971219in}}%
\pgfpathlineto{\pgfqpoint{10.354268in}{2.971219in}}%
\pgfpathlineto{\pgfqpoint{10.360663in}{2.971219in}}%
\pgfpathlineto{\pgfqpoint{10.367057in}{2.971219in}}%
\pgfpathlineto{\pgfqpoint{10.373451in}{2.971219in}}%
\pgfpathlineto{\pgfqpoint{10.379845in}{2.971219in}}%
\pgfpathlineto{\pgfqpoint{10.386240in}{2.971219in}}%
\pgfpathlineto{\pgfqpoint{10.392634in}{2.971219in}}%
\pgfpathlineto{\pgfqpoint{10.399028in}{2.971219in}}%
\pgfpathlineto{\pgfqpoint{10.405423in}{2.971219in}}%
\pgfpathlineto{\pgfqpoint{10.411817in}{2.971219in}}%
\pgfpathlineto{\pgfqpoint{10.418211in}{2.971219in}}%
\pgfpathlineto{\pgfqpoint{10.424605in}{2.971219in}}%
\pgfpathlineto{\pgfqpoint{10.431000in}{2.971219in}}%
\pgfpathlineto{\pgfqpoint{10.437394in}{2.971219in}}%
\pgfpathlineto{\pgfqpoint{10.443788in}{2.971219in}}%
\pgfpathlineto{\pgfqpoint{10.450183in}{2.971219in}}%
\pgfpathlineto{\pgfqpoint{10.456577in}{2.971219in}}%
\pgfpathlineto{\pgfqpoint{10.462971in}{2.971219in}}%
\pgfpathlineto{\pgfqpoint{10.469365in}{2.971219in}}%
\pgfpathlineto{\pgfqpoint{10.475760in}{2.971219in}}%
\pgfpathlineto{\pgfqpoint{10.482154in}{2.971219in}}%
\pgfpathlineto{\pgfqpoint{10.488548in}{2.971219in}}%
\pgfpathlineto{\pgfqpoint{10.494943in}{2.971219in}}%
\pgfpathlineto{\pgfqpoint{10.501337in}{2.971219in}}%
\pgfpathlineto{\pgfqpoint{10.507731in}{2.971219in}}%
\pgfpathlineto{\pgfqpoint{10.507731in}{10.230099in}}%
\pgfpathlineto{\pgfqpoint{10.507731in}{10.230099in}}%
\pgfpathlineto{\pgfqpoint{10.501337in}{15.069353in}}%
\pgfpathlineto{\pgfqpoint{10.494943in}{19.908606in}}%
\pgfpathlineto{\pgfqpoint{10.488548in}{24.747860in}}%
\pgfpathlineto{\pgfqpoint{10.482154in}{29.587114in}}%
\pgfpathlineto{\pgfqpoint{10.475760in}{34.426367in}}%
\pgfpathlineto{\pgfqpoint{10.469365in}{39.265621in}}%
\pgfpathlineto{\pgfqpoint{10.462971in}{44.104874in}}%
\pgfpathlineto{\pgfqpoint{10.456577in}{48.944128in}}%
\pgfpathlineto{\pgfqpoint{10.450183in}{53.783381in}}%
\pgfpathlineto{\pgfqpoint{10.443788in}{58.622635in}}%
\pgfpathlineto{\pgfqpoint{10.437394in}{63.461888in}}%
\pgfpathlineto{\pgfqpoint{10.431000in}{68.301142in}}%
\pgfpathlineto{\pgfqpoint{10.424605in}{73.140396in}}%
\pgfpathlineto{\pgfqpoint{10.418211in}{77.979649in}}%
\pgfpathlineto{\pgfqpoint{10.411817in}{82.818903in}}%
\pgfpathlineto{\pgfqpoint{10.405423in}{87.658156in}}%
\pgfpathlineto{\pgfqpoint{10.399028in}{92.497410in}}%
\pgfpathlineto{\pgfqpoint{10.392634in}{97.336663in}}%
\pgfpathlineto{\pgfqpoint{10.386240in}{102.175917in}}%
\pgfpathlineto{\pgfqpoint{10.379845in}{107.015170in}}%
\pgfpathlineto{\pgfqpoint{10.373451in}{111.854424in}}%
\pgfpathlineto{\pgfqpoint{10.367057in}{116.693678in}}%
\pgfpathlineto{\pgfqpoint{10.360663in}{121.532931in}}%
\pgfpathlineto{\pgfqpoint{10.354268in}{126.372185in}}%
\pgfpathlineto{\pgfqpoint{10.347874in}{131.211438in}}%
\pgfpathlineto{\pgfqpoint{10.341480in}{136.050692in}}%
\pgfpathlineto{\pgfqpoint{10.335086in}{140.889945in}}%
\pgfpathlineto{\pgfqpoint{10.328691in}{145.729199in}}%
\pgfpathlineto{\pgfqpoint{10.322297in}{150.568453in}}%
\pgfpathlineto{\pgfqpoint{10.315903in}{155.407706in}}%
\pgfpathlineto{\pgfqpoint{10.309508in}{160.246960in}}%
\pgfpathlineto{\pgfqpoint{10.303114in}{165.086213in}}%
\pgfpathlineto{\pgfqpoint{10.296720in}{169.925467in}}%
\pgfpathlineto{\pgfqpoint{10.290326in}{174.764720in}}%
\pgfpathlineto{\pgfqpoint{10.283931in}{179.603974in}}%
\pgfpathlineto{\pgfqpoint{10.277537in}{184.443227in}}%
\pgfpathlineto{\pgfqpoint{10.271143in}{189.282481in}}%
\pgfpathlineto{\pgfqpoint{10.264748in}{194.121735in}}%
\pgfpathlineto{\pgfqpoint{10.258354in}{198.960988in}}%
\pgfpathlineto{\pgfqpoint{10.251960in}{203.800242in}}%
\pgfpathlineto{\pgfqpoint{10.245566in}{208.639495in}}%
\pgfpathlineto{\pgfqpoint{10.239171in}{213.478749in}}%
\pgfpathlineto{\pgfqpoint{10.232777in}{218.318002in}}%
\pgfpathlineto{\pgfqpoint{10.226383in}{223.157256in}}%
\pgfpathlineto{\pgfqpoint{10.221709in}{226.694073in}}%
\pgfusepath{fill}%
\end{pgfscope}%
\begin{pgfscope}%
\pgfpathrectangle{\pgfqpoint{7.323380in}{0.554012in}}{\pgfqpoint{6.387885in}{4.834414in}}%
\pgfusepath{clip}%
\pgfsetbuttcap%
\pgfsetroundjoin%
\definecolor{currentfill}{rgb}{1.000000,0.498039,0.054902}%
\pgfsetfillcolor{currentfill}%
\pgfsetfillopacity{0.200000}%
\pgfsetlinewidth{1.003750pt}%
\definecolor{currentstroke}{rgb}{1.000000,0.498039,0.054902}%
\pgfsetstrokecolor{currentstroke}%
\pgfsetstrokeopacity{0.200000}%
\pgfsetdash{}{0pt}%
\pgfsys@defobject{currentmarker}{\pgfqpoint{7.323380in}{0.554012in}}{\pgfqpoint{10.507731in}{2.971219in}}{%
\pgfpathmoveto{\pgfqpoint{7.323380in}{2.971219in}}%
\pgfpathlineto{\pgfqpoint{7.323380in}{0.554012in}}%
\pgfpathlineto{\pgfqpoint{7.329774in}{0.554012in}}%
\pgfpathlineto{\pgfqpoint{7.336168in}{0.554012in}}%
\pgfpathlineto{\pgfqpoint{7.342563in}{0.554012in}}%
\pgfpathlineto{\pgfqpoint{7.348957in}{0.554012in}}%
\pgfpathlineto{\pgfqpoint{7.355351in}{0.554012in}}%
\pgfpathlineto{\pgfqpoint{7.361746in}{0.554012in}}%
\pgfpathlineto{\pgfqpoint{7.368140in}{0.554012in}}%
\pgfpathlineto{\pgfqpoint{7.374534in}{0.554012in}}%
\pgfpathlineto{\pgfqpoint{7.380928in}{0.554012in}}%
\pgfpathlineto{\pgfqpoint{7.387323in}{0.554012in}}%
\pgfpathlineto{\pgfqpoint{7.393717in}{0.554012in}}%
\pgfpathlineto{\pgfqpoint{7.400111in}{0.554012in}}%
\pgfpathlineto{\pgfqpoint{7.406505in}{0.554012in}}%
\pgfpathlineto{\pgfqpoint{7.412900in}{0.554012in}}%
\pgfpathlineto{\pgfqpoint{7.419294in}{0.554012in}}%
\pgfpathlineto{\pgfqpoint{7.425688in}{0.554012in}}%
\pgfpathlineto{\pgfqpoint{7.432083in}{0.554012in}}%
\pgfpathlineto{\pgfqpoint{7.438477in}{0.554012in}}%
\pgfpathlineto{\pgfqpoint{7.444871in}{0.554012in}}%
\pgfpathlineto{\pgfqpoint{7.451265in}{0.554012in}}%
\pgfpathlineto{\pgfqpoint{7.457660in}{0.554012in}}%
\pgfpathlineto{\pgfqpoint{7.464054in}{0.554012in}}%
\pgfpathlineto{\pgfqpoint{7.470448in}{0.554012in}}%
\pgfpathlineto{\pgfqpoint{7.476843in}{0.554012in}}%
\pgfpathlineto{\pgfqpoint{7.483237in}{0.554012in}}%
\pgfpathlineto{\pgfqpoint{7.489631in}{0.554012in}}%
\pgfpathlineto{\pgfqpoint{7.496025in}{0.554012in}}%
\pgfpathlineto{\pgfqpoint{7.502420in}{0.554012in}}%
\pgfpathlineto{\pgfqpoint{7.508814in}{0.554012in}}%
\pgfpathlineto{\pgfqpoint{7.515208in}{0.554012in}}%
\pgfpathlineto{\pgfqpoint{7.521603in}{0.554012in}}%
\pgfpathlineto{\pgfqpoint{7.527997in}{0.554012in}}%
\pgfpathlineto{\pgfqpoint{7.534391in}{0.554012in}}%
\pgfpathlineto{\pgfqpoint{7.540785in}{0.554012in}}%
\pgfpathlineto{\pgfqpoint{7.547180in}{0.554012in}}%
\pgfpathlineto{\pgfqpoint{7.553574in}{0.554012in}}%
\pgfpathlineto{\pgfqpoint{7.559968in}{0.554012in}}%
\pgfpathlineto{\pgfqpoint{7.566362in}{0.554012in}}%
\pgfpathlineto{\pgfqpoint{7.572757in}{0.554012in}}%
\pgfpathlineto{\pgfqpoint{7.579151in}{0.554012in}}%
\pgfpathlineto{\pgfqpoint{7.585545in}{0.554012in}}%
\pgfpathlineto{\pgfqpoint{7.591940in}{0.554012in}}%
\pgfpathlineto{\pgfqpoint{7.598334in}{0.554012in}}%
\pgfpathlineto{\pgfqpoint{7.604728in}{0.554012in}}%
\pgfpathlineto{\pgfqpoint{7.611122in}{0.554012in}}%
\pgfpathlineto{\pgfqpoint{7.617517in}{0.554012in}}%
\pgfpathlineto{\pgfqpoint{7.623911in}{0.554012in}}%
\pgfpathlineto{\pgfqpoint{7.630305in}{0.554012in}}%
\pgfpathlineto{\pgfqpoint{7.636700in}{0.554012in}}%
\pgfpathlineto{\pgfqpoint{7.643094in}{0.554012in}}%
\pgfpathlineto{\pgfqpoint{7.649488in}{0.554012in}}%
\pgfpathlineto{\pgfqpoint{7.655882in}{0.554012in}}%
\pgfpathlineto{\pgfqpoint{7.662277in}{0.554012in}}%
\pgfpathlineto{\pgfqpoint{7.668671in}{0.554012in}}%
\pgfpathlineto{\pgfqpoint{7.675065in}{0.554012in}}%
\pgfpathlineto{\pgfqpoint{7.681460in}{0.554012in}}%
\pgfpathlineto{\pgfqpoint{7.687854in}{0.554012in}}%
\pgfpathlineto{\pgfqpoint{7.694248in}{0.554012in}}%
\pgfpathlineto{\pgfqpoint{7.700642in}{0.554012in}}%
\pgfpathlineto{\pgfqpoint{7.707037in}{0.554012in}}%
\pgfpathlineto{\pgfqpoint{7.713431in}{0.554012in}}%
\pgfpathlineto{\pgfqpoint{7.719825in}{0.554012in}}%
\pgfpathlineto{\pgfqpoint{7.726219in}{0.554012in}}%
\pgfpathlineto{\pgfqpoint{7.732614in}{0.554012in}}%
\pgfpathlineto{\pgfqpoint{7.739008in}{0.554012in}}%
\pgfpathlineto{\pgfqpoint{7.745402in}{0.554012in}}%
\pgfpathlineto{\pgfqpoint{7.751797in}{0.554012in}}%
\pgfpathlineto{\pgfqpoint{7.758191in}{0.554012in}}%
\pgfpathlineto{\pgfqpoint{7.764585in}{0.554012in}}%
\pgfpathlineto{\pgfqpoint{7.770979in}{0.554012in}}%
\pgfpathlineto{\pgfqpoint{7.777374in}{0.554012in}}%
\pgfpathlineto{\pgfqpoint{7.783768in}{0.554012in}}%
\pgfpathlineto{\pgfqpoint{7.790162in}{0.554012in}}%
\pgfpathlineto{\pgfqpoint{7.796557in}{0.554012in}}%
\pgfpathlineto{\pgfqpoint{7.802951in}{0.554012in}}%
\pgfpathlineto{\pgfqpoint{7.809345in}{0.554012in}}%
\pgfpathlineto{\pgfqpoint{7.815739in}{0.554012in}}%
\pgfpathlineto{\pgfqpoint{7.822134in}{0.554012in}}%
\pgfpathlineto{\pgfqpoint{7.828528in}{0.554012in}}%
\pgfpathlineto{\pgfqpoint{7.834922in}{0.554012in}}%
\pgfpathlineto{\pgfqpoint{7.841317in}{0.554012in}}%
\pgfpathlineto{\pgfqpoint{7.847711in}{0.554012in}}%
\pgfpathlineto{\pgfqpoint{7.854105in}{0.554012in}}%
\pgfpathlineto{\pgfqpoint{7.860499in}{0.554012in}}%
\pgfpathlineto{\pgfqpoint{7.866894in}{0.554012in}}%
\pgfpathlineto{\pgfqpoint{7.873288in}{0.554012in}}%
\pgfpathlineto{\pgfqpoint{7.879682in}{0.554012in}}%
\pgfpathlineto{\pgfqpoint{7.886076in}{0.554012in}}%
\pgfpathlineto{\pgfqpoint{7.892471in}{0.554012in}}%
\pgfpathlineto{\pgfqpoint{7.898865in}{0.554012in}}%
\pgfpathlineto{\pgfqpoint{7.905259in}{0.554012in}}%
\pgfpathlineto{\pgfqpoint{7.911654in}{0.554012in}}%
\pgfpathlineto{\pgfqpoint{7.918048in}{0.554012in}}%
\pgfpathlineto{\pgfqpoint{7.924442in}{0.554012in}}%
\pgfpathlineto{\pgfqpoint{7.930836in}{0.554012in}}%
\pgfpathlineto{\pgfqpoint{7.937231in}{0.554012in}}%
\pgfpathlineto{\pgfqpoint{7.943625in}{0.554012in}}%
\pgfpathlineto{\pgfqpoint{7.950019in}{0.554012in}}%
\pgfpathlineto{\pgfqpoint{7.956414in}{0.554012in}}%
\pgfpathlineto{\pgfqpoint{7.962808in}{0.554012in}}%
\pgfpathlineto{\pgfqpoint{7.969202in}{0.554012in}}%
\pgfpathlineto{\pgfqpoint{7.975596in}{0.554012in}}%
\pgfpathlineto{\pgfqpoint{7.981991in}{0.554012in}}%
\pgfpathlineto{\pgfqpoint{7.988385in}{0.554012in}}%
\pgfpathlineto{\pgfqpoint{7.994779in}{0.554012in}}%
\pgfpathlineto{\pgfqpoint{8.001173in}{0.554012in}}%
\pgfpathlineto{\pgfqpoint{8.007568in}{0.554012in}}%
\pgfpathlineto{\pgfqpoint{8.013962in}{0.554012in}}%
\pgfpathlineto{\pgfqpoint{8.020356in}{0.554012in}}%
\pgfpathlineto{\pgfqpoint{8.026751in}{0.554012in}}%
\pgfpathlineto{\pgfqpoint{8.033145in}{0.554012in}}%
\pgfpathlineto{\pgfqpoint{8.039539in}{0.554012in}}%
\pgfpathlineto{\pgfqpoint{8.045933in}{0.554012in}}%
\pgfpathlineto{\pgfqpoint{8.052328in}{0.554012in}}%
\pgfpathlineto{\pgfqpoint{8.058722in}{0.554012in}}%
\pgfpathlineto{\pgfqpoint{8.065116in}{0.554012in}}%
\pgfpathlineto{\pgfqpoint{8.071511in}{0.554012in}}%
\pgfpathlineto{\pgfqpoint{8.077905in}{0.554012in}}%
\pgfpathlineto{\pgfqpoint{8.084299in}{0.554012in}}%
\pgfpathlineto{\pgfqpoint{8.090693in}{0.554012in}}%
\pgfpathlineto{\pgfqpoint{8.097088in}{0.554012in}}%
\pgfpathlineto{\pgfqpoint{8.103482in}{0.554012in}}%
\pgfpathlineto{\pgfqpoint{8.109876in}{0.554012in}}%
\pgfpathlineto{\pgfqpoint{8.116271in}{0.554012in}}%
\pgfpathlineto{\pgfqpoint{8.122665in}{0.554012in}}%
\pgfpathlineto{\pgfqpoint{8.129059in}{0.554012in}}%
\pgfpathlineto{\pgfqpoint{8.135453in}{0.554012in}}%
\pgfpathlineto{\pgfqpoint{8.141848in}{0.554012in}}%
\pgfpathlineto{\pgfqpoint{8.148242in}{0.554012in}}%
\pgfpathlineto{\pgfqpoint{8.154636in}{0.554012in}}%
\pgfpathlineto{\pgfqpoint{8.161030in}{0.554012in}}%
\pgfpathlineto{\pgfqpoint{8.167425in}{0.554012in}}%
\pgfpathlineto{\pgfqpoint{8.173819in}{0.554012in}}%
\pgfpathlineto{\pgfqpoint{8.180213in}{0.554012in}}%
\pgfpathlineto{\pgfqpoint{8.186608in}{0.554012in}}%
\pgfpathlineto{\pgfqpoint{8.193002in}{0.554012in}}%
\pgfpathlineto{\pgfqpoint{8.199396in}{0.554012in}}%
\pgfpathlineto{\pgfqpoint{8.205790in}{0.554012in}}%
\pgfpathlineto{\pgfqpoint{8.212185in}{0.554012in}}%
\pgfpathlineto{\pgfqpoint{8.218579in}{0.554012in}}%
\pgfpathlineto{\pgfqpoint{8.224973in}{0.554012in}}%
\pgfpathlineto{\pgfqpoint{8.231368in}{0.554012in}}%
\pgfpathlineto{\pgfqpoint{8.237762in}{0.554012in}}%
\pgfpathlineto{\pgfqpoint{8.244156in}{0.554012in}}%
\pgfpathlineto{\pgfqpoint{8.250550in}{0.554012in}}%
\pgfpathlineto{\pgfqpoint{8.256945in}{0.554012in}}%
\pgfpathlineto{\pgfqpoint{8.263339in}{0.554012in}}%
\pgfpathlineto{\pgfqpoint{8.269733in}{0.554012in}}%
\pgfpathlineto{\pgfqpoint{8.276128in}{0.554012in}}%
\pgfpathlineto{\pgfqpoint{8.282522in}{0.554012in}}%
\pgfpathlineto{\pgfqpoint{8.288916in}{0.554012in}}%
\pgfpathlineto{\pgfqpoint{8.295310in}{0.554012in}}%
\pgfpathlineto{\pgfqpoint{8.301705in}{0.554012in}}%
\pgfpathlineto{\pgfqpoint{8.308099in}{0.554012in}}%
\pgfpathlineto{\pgfqpoint{8.314493in}{0.554012in}}%
\pgfpathlineto{\pgfqpoint{8.320887in}{0.554012in}}%
\pgfpathlineto{\pgfqpoint{8.327282in}{0.554012in}}%
\pgfpathlineto{\pgfqpoint{8.333676in}{0.554012in}}%
\pgfpathlineto{\pgfqpoint{8.340070in}{0.554012in}}%
\pgfpathlineto{\pgfqpoint{8.346465in}{0.554012in}}%
\pgfpathlineto{\pgfqpoint{8.352859in}{0.554012in}}%
\pgfpathlineto{\pgfqpoint{8.359253in}{0.554012in}}%
\pgfpathlineto{\pgfqpoint{8.365647in}{0.554012in}}%
\pgfpathlineto{\pgfqpoint{8.372042in}{0.554012in}}%
\pgfpathlineto{\pgfqpoint{8.378436in}{0.554012in}}%
\pgfpathlineto{\pgfqpoint{8.384830in}{0.554012in}}%
\pgfpathlineto{\pgfqpoint{8.391225in}{0.554012in}}%
\pgfpathlineto{\pgfqpoint{8.397619in}{0.554012in}}%
\pgfpathlineto{\pgfqpoint{8.404013in}{0.554012in}}%
\pgfpathlineto{\pgfqpoint{8.410407in}{0.554012in}}%
\pgfpathlineto{\pgfqpoint{8.416802in}{0.554012in}}%
\pgfpathlineto{\pgfqpoint{8.423196in}{0.554012in}}%
\pgfpathlineto{\pgfqpoint{8.429590in}{0.554012in}}%
\pgfpathlineto{\pgfqpoint{8.435985in}{0.554012in}}%
\pgfpathlineto{\pgfqpoint{8.442379in}{0.554012in}}%
\pgfpathlineto{\pgfqpoint{8.448773in}{0.554012in}}%
\pgfpathlineto{\pgfqpoint{8.455167in}{0.554012in}}%
\pgfpathlineto{\pgfqpoint{8.461562in}{0.554012in}}%
\pgfpathlineto{\pgfqpoint{8.467956in}{0.554012in}}%
\pgfpathlineto{\pgfqpoint{8.474350in}{0.554012in}}%
\pgfpathlineto{\pgfqpoint{8.480744in}{0.554012in}}%
\pgfpathlineto{\pgfqpoint{8.487139in}{0.554012in}}%
\pgfpathlineto{\pgfqpoint{8.493533in}{0.554012in}}%
\pgfpathlineto{\pgfqpoint{8.499927in}{0.554012in}}%
\pgfpathlineto{\pgfqpoint{8.506322in}{0.554012in}}%
\pgfpathlineto{\pgfqpoint{8.512716in}{0.554012in}}%
\pgfpathlineto{\pgfqpoint{8.519110in}{0.554012in}}%
\pgfpathlineto{\pgfqpoint{8.525504in}{0.554012in}}%
\pgfpathlineto{\pgfqpoint{8.531899in}{0.554012in}}%
\pgfpathlineto{\pgfqpoint{8.538293in}{0.554012in}}%
\pgfpathlineto{\pgfqpoint{8.544687in}{0.554012in}}%
\pgfpathlineto{\pgfqpoint{8.551082in}{0.554012in}}%
\pgfpathlineto{\pgfqpoint{8.557476in}{0.554012in}}%
\pgfpathlineto{\pgfqpoint{8.563870in}{0.554012in}}%
\pgfpathlineto{\pgfqpoint{8.570264in}{0.554012in}}%
\pgfpathlineto{\pgfqpoint{8.576659in}{0.554012in}}%
\pgfpathlineto{\pgfqpoint{8.583053in}{0.554012in}}%
\pgfpathlineto{\pgfqpoint{8.589447in}{0.554012in}}%
\pgfpathlineto{\pgfqpoint{8.595841in}{0.554012in}}%
\pgfpathlineto{\pgfqpoint{8.602236in}{0.554012in}}%
\pgfpathlineto{\pgfqpoint{8.608630in}{0.554012in}}%
\pgfpathlineto{\pgfqpoint{8.615024in}{0.554012in}}%
\pgfpathlineto{\pgfqpoint{8.621419in}{0.554012in}}%
\pgfpathlineto{\pgfqpoint{8.627813in}{0.554012in}}%
\pgfpathlineto{\pgfqpoint{8.634207in}{0.554012in}}%
\pgfpathlineto{\pgfqpoint{8.640601in}{0.554012in}}%
\pgfpathlineto{\pgfqpoint{8.646996in}{0.554012in}}%
\pgfpathlineto{\pgfqpoint{8.653390in}{0.554012in}}%
\pgfpathlineto{\pgfqpoint{8.659784in}{0.554012in}}%
\pgfpathlineto{\pgfqpoint{8.666179in}{0.554012in}}%
\pgfpathlineto{\pgfqpoint{8.672573in}{0.554012in}}%
\pgfpathlineto{\pgfqpoint{8.678967in}{0.554012in}}%
\pgfpathlineto{\pgfqpoint{8.685361in}{0.554012in}}%
\pgfpathlineto{\pgfqpoint{8.691756in}{0.554012in}}%
\pgfpathlineto{\pgfqpoint{8.698150in}{0.554012in}}%
\pgfpathlineto{\pgfqpoint{8.704544in}{0.554012in}}%
\pgfpathlineto{\pgfqpoint{8.710939in}{0.554012in}}%
\pgfpathlineto{\pgfqpoint{8.717333in}{0.554012in}}%
\pgfpathlineto{\pgfqpoint{8.723727in}{0.554012in}}%
\pgfpathlineto{\pgfqpoint{8.730121in}{0.554012in}}%
\pgfpathlineto{\pgfqpoint{8.736516in}{0.554012in}}%
\pgfpathlineto{\pgfqpoint{8.742910in}{0.554012in}}%
\pgfpathlineto{\pgfqpoint{8.749304in}{0.554012in}}%
\pgfpathlineto{\pgfqpoint{8.755698in}{0.554012in}}%
\pgfpathlineto{\pgfqpoint{8.762093in}{0.554012in}}%
\pgfpathlineto{\pgfqpoint{8.768487in}{0.554012in}}%
\pgfpathlineto{\pgfqpoint{8.774881in}{0.554012in}}%
\pgfpathlineto{\pgfqpoint{8.781276in}{0.554012in}}%
\pgfpathlineto{\pgfqpoint{8.787670in}{0.554012in}}%
\pgfpathlineto{\pgfqpoint{8.794064in}{0.554012in}}%
\pgfpathlineto{\pgfqpoint{8.800458in}{0.554012in}}%
\pgfpathlineto{\pgfqpoint{8.806853in}{0.554012in}}%
\pgfpathlineto{\pgfqpoint{8.813247in}{0.554012in}}%
\pgfpathlineto{\pgfqpoint{8.819641in}{0.554012in}}%
\pgfpathlineto{\pgfqpoint{8.826036in}{0.554012in}}%
\pgfpathlineto{\pgfqpoint{8.832430in}{0.554012in}}%
\pgfpathlineto{\pgfqpoint{8.838824in}{0.554012in}}%
\pgfpathlineto{\pgfqpoint{8.845218in}{0.554012in}}%
\pgfpathlineto{\pgfqpoint{8.851613in}{0.554012in}}%
\pgfpathlineto{\pgfqpoint{8.858007in}{0.554012in}}%
\pgfpathlineto{\pgfqpoint{8.864401in}{0.554012in}}%
\pgfpathlineto{\pgfqpoint{8.870796in}{0.554012in}}%
\pgfpathlineto{\pgfqpoint{8.877190in}{0.554012in}}%
\pgfpathlineto{\pgfqpoint{8.883584in}{0.554012in}}%
\pgfpathlineto{\pgfqpoint{8.889978in}{0.554012in}}%
\pgfpathlineto{\pgfqpoint{8.896373in}{0.554012in}}%
\pgfpathlineto{\pgfqpoint{8.902767in}{0.554012in}}%
\pgfpathlineto{\pgfqpoint{8.909161in}{0.554012in}}%
\pgfpathlineto{\pgfqpoint{8.915555in}{0.554012in}}%
\pgfpathlineto{\pgfqpoint{8.921950in}{0.554012in}}%
\pgfpathlineto{\pgfqpoint{8.928344in}{0.554012in}}%
\pgfpathlineto{\pgfqpoint{8.934738in}{0.554012in}}%
\pgfpathlineto{\pgfqpoint{8.941133in}{0.554012in}}%
\pgfpathlineto{\pgfqpoint{8.947527in}{0.554012in}}%
\pgfpathlineto{\pgfqpoint{8.953921in}{0.554012in}}%
\pgfpathlineto{\pgfqpoint{8.960315in}{0.554012in}}%
\pgfpathlineto{\pgfqpoint{8.966710in}{0.554012in}}%
\pgfpathlineto{\pgfqpoint{8.973104in}{0.554012in}}%
\pgfpathlineto{\pgfqpoint{8.979498in}{0.554012in}}%
\pgfpathlineto{\pgfqpoint{8.985893in}{0.554012in}}%
\pgfpathlineto{\pgfqpoint{8.992287in}{0.554012in}}%
\pgfpathlineto{\pgfqpoint{8.998681in}{0.554012in}}%
\pgfpathlineto{\pgfqpoint{9.005075in}{0.554012in}}%
\pgfpathlineto{\pgfqpoint{9.011470in}{0.554012in}}%
\pgfpathlineto{\pgfqpoint{9.017864in}{0.554012in}}%
\pgfpathlineto{\pgfqpoint{9.024258in}{0.554012in}}%
\pgfpathlineto{\pgfqpoint{9.030653in}{0.554012in}}%
\pgfpathlineto{\pgfqpoint{9.037047in}{0.554012in}}%
\pgfpathlineto{\pgfqpoint{9.043441in}{0.554012in}}%
\pgfpathlineto{\pgfqpoint{9.049835in}{0.554012in}}%
\pgfpathlineto{\pgfqpoint{9.056230in}{0.554012in}}%
\pgfpathlineto{\pgfqpoint{9.062624in}{0.554012in}}%
\pgfpathlineto{\pgfqpoint{9.069018in}{0.554012in}}%
\pgfpathlineto{\pgfqpoint{9.075412in}{0.554012in}}%
\pgfpathlineto{\pgfqpoint{9.081807in}{0.554012in}}%
\pgfpathlineto{\pgfqpoint{9.088201in}{0.554012in}}%
\pgfpathlineto{\pgfqpoint{9.094595in}{0.554012in}}%
\pgfpathlineto{\pgfqpoint{9.100990in}{0.554012in}}%
\pgfpathlineto{\pgfqpoint{9.107384in}{0.554012in}}%
\pgfpathlineto{\pgfqpoint{9.113778in}{0.554012in}}%
\pgfpathlineto{\pgfqpoint{9.120172in}{0.554012in}}%
\pgfpathlineto{\pgfqpoint{9.126567in}{0.554012in}}%
\pgfpathlineto{\pgfqpoint{9.132961in}{0.554012in}}%
\pgfpathlineto{\pgfqpoint{9.139355in}{0.554012in}}%
\pgfpathlineto{\pgfqpoint{9.145750in}{0.554012in}}%
\pgfpathlineto{\pgfqpoint{9.152144in}{0.554012in}}%
\pgfpathlineto{\pgfqpoint{9.158538in}{0.554012in}}%
\pgfpathlineto{\pgfqpoint{9.164932in}{0.554012in}}%
\pgfpathlineto{\pgfqpoint{9.171327in}{0.554012in}}%
\pgfpathlineto{\pgfqpoint{9.177721in}{0.554012in}}%
\pgfpathlineto{\pgfqpoint{9.184115in}{0.554012in}}%
\pgfpathlineto{\pgfqpoint{9.190509in}{0.554012in}}%
\pgfpathlineto{\pgfqpoint{9.196904in}{0.554012in}}%
\pgfpathlineto{\pgfqpoint{9.203298in}{0.554012in}}%
\pgfpathlineto{\pgfqpoint{9.209692in}{0.554012in}}%
\pgfpathlineto{\pgfqpoint{9.216087in}{0.554012in}}%
\pgfpathlineto{\pgfqpoint{9.222481in}{0.554012in}}%
\pgfpathlineto{\pgfqpoint{9.228875in}{0.554012in}}%
\pgfpathlineto{\pgfqpoint{9.235269in}{0.554012in}}%
\pgfpathlineto{\pgfqpoint{9.241664in}{0.554012in}}%
\pgfpathlineto{\pgfqpoint{9.248058in}{0.554012in}}%
\pgfpathlineto{\pgfqpoint{9.254452in}{0.554012in}}%
\pgfpathlineto{\pgfqpoint{9.260847in}{0.554012in}}%
\pgfpathlineto{\pgfqpoint{9.267241in}{0.554012in}}%
\pgfpathlineto{\pgfqpoint{9.273635in}{0.554012in}}%
\pgfpathlineto{\pgfqpoint{9.280029in}{0.554012in}}%
\pgfpathlineto{\pgfqpoint{9.286424in}{0.554012in}}%
\pgfpathlineto{\pgfqpoint{9.292818in}{0.554012in}}%
\pgfpathlineto{\pgfqpoint{9.299212in}{0.554012in}}%
\pgfpathlineto{\pgfqpoint{9.305607in}{0.554012in}}%
\pgfpathlineto{\pgfqpoint{9.312001in}{0.554012in}}%
\pgfpathlineto{\pgfqpoint{9.318395in}{0.554012in}}%
\pgfpathlineto{\pgfqpoint{9.324789in}{0.554012in}}%
\pgfpathlineto{\pgfqpoint{9.331184in}{0.554012in}}%
\pgfpathlineto{\pgfqpoint{9.337578in}{0.554012in}}%
\pgfpathlineto{\pgfqpoint{9.343972in}{0.554012in}}%
\pgfpathlineto{\pgfqpoint{9.350366in}{0.554012in}}%
\pgfpathlineto{\pgfqpoint{9.356761in}{0.554012in}}%
\pgfpathlineto{\pgfqpoint{9.363155in}{0.554012in}}%
\pgfpathlineto{\pgfqpoint{9.369549in}{0.554012in}}%
\pgfpathlineto{\pgfqpoint{9.375944in}{0.554012in}}%
\pgfpathlineto{\pgfqpoint{9.382338in}{0.554012in}}%
\pgfpathlineto{\pgfqpoint{9.388732in}{0.554012in}}%
\pgfpathlineto{\pgfqpoint{9.395126in}{0.554012in}}%
\pgfpathlineto{\pgfqpoint{9.401521in}{0.554012in}}%
\pgfpathlineto{\pgfqpoint{9.407915in}{0.554012in}}%
\pgfpathlineto{\pgfqpoint{9.414309in}{0.554012in}}%
\pgfpathlineto{\pgfqpoint{9.420704in}{0.554012in}}%
\pgfpathlineto{\pgfqpoint{9.427098in}{0.554012in}}%
\pgfpathlineto{\pgfqpoint{9.433492in}{0.554012in}}%
\pgfpathlineto{\pgfqpoint{9.439886in}{0.554012in}}%
\pgfpathlineto{\pgfqpoint{9.446281in}{0.554012in}}%
\pgfpathlineto{\pgfqpoint{9.452675in}{0.554012in}}%
\pgfpathlineto{\pgfqpoint{9.459069in}{0.554012in}}%
\pgfpathlineto{\pgfqpoint{9.465464in}{0.554012in}}%
\pgfpathlineto{\pgfqpoint{9.471858in}{0.554012in}}%
\pgfpathlineto{\pgfqpoint{9.478252in}{0.554012in}}%
\pgfpathlineto{\pgfqpoint{9.484646in}{0.554012in}}%
\pgfpathlineto{\pgfqpoint{9.491041in}{0.554012in}}%
\pgfpathlineto{\pgfqpoint{9.497435in}{0.554012in}}%
\pgfpathlineto{\pgfqpoint{9.503829in}{0.554012in}}%
\pgfpathlineto{\pgfqpoint{9.510223in}{0.554012in}}%
\pgfpathlineto{\pgfqpoint{9.516618in}{0.554012in}}%
\pgfpathlineto{\pgfqpoint{9.523012in}{0.554012in}}%
\pgfpathlineto{\pgfqpoint{9.529406in}{0.554012in}}%
\pgfpathlineto{\pgfqpoint{9.535801in}{0.554012in}}%
\pgfpathlineto{\pgfqpoint{9.542195in}{0.554012in}}%
\pgfpathlineto{\pgfqpoint{9.548589in}{0.554012in}}%
\pgfpathlineto{\pgfqpoint{9.554983in}{0.554012in}}%
\pgfpathlineto{\pgfqpoint{9.561378in}{0.554012in}}%
\pgfpathlineto{\pgfqpoint{9.567772in}{0.554012in}}%
\pgfpathlineto{\pgfqpoint{9.574166in}{0.554012in}}%
\pgfpathlineto{\pgfqpoint{9.580561in}{0.554012in}}%
\pgfpathlineto{\pgfqpoint{9.586955in}{0.554012in}}%
\pgfpathlineto{\pgfqpoint{9.593349in}{0.554012in}}%
\pgfpathlineto{\pgfqpoint{9.599743in}{0.554012in}}%
\pgfpathlineto{\pgfqpoint{9.606138in}{0.554012in}}%
\pgfpathlineto{\pgfqpoint{9.612532in}{0.554012in}}%
\pgfpathlineto{\pgfqpoint{9.618926in}{0.554012in}}%
\pgfpathlineto{\pgfqpoint{9.625321in}{0.554012in}}%
\pgfpathlineto{\pgfqpoint{9.631715in}{0.554012in}}%
\pgfpathlineto{\pgfqpoint{9.638109in}{0.554012in}}%
\pgfpathlineto{\pgfqpoint{9.644503in}{0.554012in}}%
\pgfpathlineto{\pgfqpoint{9.650898in}{0.554012in}}%
\pgfpathlineto{\pgfqpoint{9.657292in}{0.554012in}}%
\pgfpathlineto{\pgfqpoint{9.663686in}{0.554012in}}%
\pgfpathlineto{\pgfqpoint{9.670080in}{0.554012in}}%
\pgfpathlineto{\pgfqpoint{9.676475in}{0.554012in}}%
\pgfpathlineto{\pgfqpoint{9.682869in}{0.554012in}}%
\pgfpathlineto{\pgfqpoint{9.689263in}{0.554012in}}%
\pgfpathlineto{\pgfqpoint{9.695658in}{0.554012in}}%
\pgfpathlineto{\pgfqpoint{9.702052in}{0.554012in}}%
\pgfpathlineto{\pgfqpoint{9.708446in}{0.554012in}}%
\pgfpathlineto{\pgfqpoint{9.714840in}{0.554012in}}%
\pgfpathlineto{\pgfqpoint{9.721235in}{0.554012in}}%
\pgfpathlineto{\pgfqpoint{9.727629in}{0.554012in}}%
\pgfpathlineto{\pgfqpoint{9.734023in}{0.554012in}}%
\pgfpathlineto{\pgfqpoint{9.740418in}{0.554012in}}%
\pgfpathlineto{\pgfqpoint{9.746812in}{0.554012in}}%
\pgfpathlineto{\pgfqpoint{9.753206in}{0.554012in}}%
\pgfpathlineto{\pgfqpoint{9.759600in}{0.554012in}}%
\pgfpathlineto{\pgfqpoint{9.765995in}{0.554012in}}%
\pgfpathlineto{\pgfqpoint{9.772389in}{0.554012in}}%
\pgfpathlineto{\pgfqpoint{9.778783in}{0.554012in}}%
\pgfpathlineto{\pgfqpoint{9.785177in}{0.554012in}}%
\pgfpathlineto{\pgfqpoint{9.791572in}{0.554012in}}%
\pgfpathlineto{\pgfqpoint{9.797966in}{0.554012in}}%
\pgfpathlineto{\pgfqpoint{9.804360in}{0.554012in}}%
\pgfpathlineto{\pgfqpoint{9.810755in}{0.554012in}}%
\pgfpathlineto{\pgfqpoint{9.817149in}{0.554012in}}%
\pgfpathlineto{\pgfqpoint{9.823543in}{0.554012in}}%
\pgfpathlineto{\pgfqpoint{9.829937in}{0.554012in}}%
\pgfpathlineto{\pgfqpoint{9.836332in}{0.554012in}}%
\pgfpathlineto{\pgfqpoint{9.842726in}{0.554012in}}%
\pgfpathlineto{\pgfqpoint{9.849120in}{0.554012in}}%
\pgfpathlineto{\pgfqpoint{9.855515in}{0.554012in}}%
\pgfpathlineto{\pgfqpoint{9.861909in}{0.554012in}}%
\pgfpathlineto{\pgfqpoint{9.868303in}{0.554012in}}%
\pgfpathlineto{\pgfqpoint{9.874697in}{0.554012in}}%
\pgfpathlineto{\pgfqpoint{9.881092in}{0.554012in}}%
\pgfpathlineto{\pgfqpoint{9.887486in}{0.554012in}}%
\pgfpathlineto{\pgfqpoint{9.893880in}{0.554012in}}%
\pgfpathlineto{\pgfqpoint{9.900275in}{0.554012in}}%
\pgfpathlineto{\pgfqpoint{9.906669in}{0.554012in}}%
\pgfpathlineto{\pgfqpoint{9.913063in}{0.554012in}}%
\pgfpathlineto{\pgfqpoint{9.919457in}{0.554012in}}%
\pgfpathlineto{\pgfqpoint{9.925852in}{0.554012in}}%
\pgfpathlineto{\pgfqpoint{9.932246in}{0.554012in}}%
\pgfpathlineto{\pgfqpoint{9.938640in}{0.554012in}}%
\pgfpathlineto{\pgfqpoint{9.945034in}{0.554012in}}%
\pgfpathlineto{\pgfqpoint{9.951429in}{0.554012in}}%
\pgfpathlineto{\pgfqpoint{9.957823in}{0.554012in}}%
\pgfpathlineto{\pgfqpoint{9.964217in}{0.554012in}}%
\pgfpathlineto{\pgfqpoint{9.970612in}{0.554012in}}%
\pgfpathlineto{\pgfqpoint{9.977006in}{0.554012in}}%
\pgfpathlineto{\pgfqpoint{9.983400in}{0.554012in}}%
\pgfpathlineto{\pgfqpoint{9.989794in}{0.554012in}}%
\pgfpathlineto{\pgfqpoint{9.996189in}{0.554012in}}%
\pgfpathlineto{\pgfqpoint{10.002583in}{0.554012in}}%
\pgfpathlineto{\pgfqpoint{10.008977in}{0.554012in}}%
\pgfpathlineto{\pgfqpoint{10.015372in}{0.554012in}}%
\pgfpathlineto{\pgfqpoint{10.021766in}{0.554012in}}%
\pgfpathlineto{\pgfqpoint{10.028160in}{0.554012in}}%
\pgfpathlineto{\pgfqpoint{10.034554in}{0.554012in}}%
\pgfpathlineto{\pgfqpoint{10.040949in}{0.554012in}}%
\pgfpathlineto{\pgfqpoint{10.047343in}{0.554012in}}%
\pgfpathlineto{\pgfqpoint{10.053737in}{0.554012in}}%
\pgfpathlineto{\pgfqpoint{10.060132in}{0.554012in}}%
\pgfpathlineto{\pgfqpoint{10.066526in}{0.554012in}}%
\pgfpathlineto{\pgfqpoint{10.072920in}{0.554012in}}%
\pgfpathlineto{\pgfqpoint{10.079314in}{0.554012in}}%
\pgfpathlineto{\pgfqpoint{10.085709in}{0.554012in}}%
\pgfpathlineto{\pgfqpoint{10.092103in}{0.554012in}}%
\pgfpathlineto{\pgfqpoint{10.098497in}{0.554012in}}%
\pgfpathlineto{\pgfqpoint{10.104891in}{0.554012in}}%
\pgfpathlineto{\pgfqpoint{10.111286in}{0.554012in}}%
\pgfpathlineto{\pgfqpoint{10.117680in}{0.554012in}}%
\pgfpathlineto{\pgfqpoint{10.124074in}{0.554012in}}%
\pgfpathlineto{\pgfqpoint{10.130469in}{0.554012in}}%
\pgfpathlineto{\pgfqpoint{10.136863in}{0.554012in}}%
\pgfpathlineto{\pgfqpoint{10.143257in}{0.554012in}}%
\pgfpathlineto{\pgfqpoint{10.149651in}{0.554012in}}%
\pgfpathlineto{\pgfqpoint{10.156046in}{0.554012in}}%
\pgfpathlineto{\pgfqpoint{10.162440in}{0.554012in}}%
\pgfpathlineto{\pgfqpoint{10.168834in}{0.554012in}}%
\pgfpathlineto{\pgfqpoint{10.175229in}{0.554012in}}%
\pgfpathlineto{\pgfqpoint{10.181623in}{0.554012in}}%
\pgfpathlineto{\pgfqpoint{10.188017in}{0.554012in}}%
\pgfpathlineto{\pgfqpoint{10.194411in}{0.554012in}}%
\pgfpathlineto{\pgfqpoint{10.200806in}{0.554012in}}%
\pgfpathlineto{\pgfqpoint{10.207200in}{0.554012in}}%
\pgfpathlineto{\pgfqpoint{10.213594in}{0.554012in}}%
\pgfpathlineto{\pgfqpoint{10.219989in}{0.554012in}}%
\pgfpathlineto{\pgfqpoint{10.226383in}{0.554012in}}%
\pgfpathlineto{\pgfqpoint{10.232777in}{0.554012in}}%
\pgfpathlineto{\pgfqpoint{10.239171in}{0.554012in}}%
\pgfpathlineto{\pgfqpoint{10.245566in}{0.554012in}}%
\pgfpathlineto{\pgfqpoint{10.251960in}{0.554012in}}%
\pgfpathlineto{\pgfqpoint{10.258354in}{0.554012in}}%
\pgfpathlineto{\pgfqpoint{10.264748in}{0.554012in}}%
\pgfpathlineto{\pgfqpoint{10.271143in}{0.554012in}}%
\pgfpathlineto{\pgfqpoint{10.277537in}{0.554012in}}%
\pgfpathlineto{\pgfqpoint{10.283931in}{0.554012in}}%
\pgfpathlineto{\pgfqpoint{10.290326in}{0.554012in}}%
\pgfpathlineto{\pgfqpoint{10.296720in}{0.554012in}}%
\pgfpathlineto{\pgfqpoint{10.303114in}{0.554012in}}%
\pgfpathlineto{\pgfqpoint{10.309508in}{0.554012in}}%
\pgfpathlineto{\pgfqpoint{10.315903in}{0.554012in}}%
\pgfpathlineto{\pgfqpoint{10.322297in}{0.554012in}}%
\pgfpathlineto{\pgfqpoint{10.328691in}{0.554012in}}%
\pgfpathlineto{\pgfqpoint{10.335086in}{0.554012in}}%
\pgfpathlineto{\pgfqpoint{10.341480in}{0.554012in}}%
\pgfpathlineto{\pgfqpoint{10.347874in}{0.554012in}}%
\pgfpathlineto{\pgfqpoint{10.354268in}{0.554012in}}%
\pgfpathlineto{\pgfqpoint{10.360663in}{0.554012in}}%
\pgfpathlineto{\pgfqpoint{10.367057in}{0.554012in}}%
\pgfpathlineto{\pgfqpoint{10.373451in}{0.554012in}}%
\pgfpathlineto{\pgfqpoint{10.379845in}{0.554012in}}%
\pgfpathlineto{\pgfqpoint{10.386240in}{0.554012in}}%
\pgfpathlineto{\pgfqpoint{10.392634in}{0.554012in}}%
\pgfpathlineto{\pgfqpoint{10.399028in}{0.554012in}}%
\pgfpathlineto{\pgfqpoint{10.405423in}{0.554012in}}%
\pgfpathlineto{\pgfqpoint{10.411817in}{0.554012in}}%
\pgfpathlineto{\pgfqpoint{10.418211in}{0.554012in}}%
\pgfpathlineto{\pgfqpoint{10.424605in}{0.554012in}}%
\pgfpathlineto{\pgfqpoint{10.431000in}{0.554012in}}%
\pgfpathlineto{\pgfqpoint{10.437394in}{0.554012in}}%
\pgfpathlineto{\pgfqpoint{10.443788in}{0.554012in}}%
\pgfpathlineto{\pgfqpoint{10.450183in}{0.554012in}}%
\pgfpathlineto{\pgfqpoint{10.456577in}{0.554012in}}%
\pgfpathlineto{\pgfqpoint{10.462971in}{0.554012in}}%
\pgfpathlineto{\pgfqpoint{10.469365in}{0.554012in}}%
\pgfpathlineto{\pgfqpoint{10.475760in}{0.554012in}}%
\pgfpathlineto{\pgfqpoint{10.482154in}{0.554012in}}%
\pgfpathlineto{\pgfqpoint{10.488548in}{0.554012in}}%
\pgfpathlineto{\pgfqpoint{10.494943in}{0.554012in}}%
\pgfpathlineto{\pgfqpoint{10.501337in}{0.554012in}}%
\pgfpathlineto{\pgfqpoint{10.507731in}{0.554012in}}%
\pgfpathlineto{\pgfqpoint{10.507731in}{2.971219in}}%
\pgfpathlineto{\pgfqpoint{10.507731in}{2.971219in}}%
\pgfpathlineto{\pgfqpoint{10.501337in}{2.971219in}}%
\pgfpathlineto{\pgfqpoint{10.494943in}{2.971219in}}%
\pgfpathlineto{\pgfqpoint{10.488548in}{2.971219in}}%
\pgfpathlineto{\pgfqpoint{10.482154in}{2.971219in}}%
\pgfpathlineto{\pgfqpoint{10.475760in}{2.971219in}}%
\pgfpathlineto{\pgfqpoint{10.469365in}{2.971219in}}%
\pgfpathlineto{\pgfqpoint{10.462971in}{2.971219in}}%
\pgfpathlineto{\pgfqpoint{10.456577in}{2.971219in}}%
\pgfpathlineto{\pgfqpoint{10.450183in}{2.971219in}}%
\pgfpathlineto{\pgfqpoint{10.443788in}{2.971219in}}%
\pgfpathlineto{\pgfqpoint{10.437394in}{2.971219in}}%
\pgfpathlineto{\pgfqpoint{10.431000in}{2.971219in}}%
\pgfpathlineto{\pgfqpoint{10.424605in}{2.971219in}}%
\pgfpathlineto{\pgfqpoint{10.418211in}{2.971219in}}%
\pgfpathlineto{\pgfqpoint{10.411817in}{2.971219in}}%
\pgfpathlineto{\pgfqpoint{10.405423in}{2.971219in}}%
\pgfpathlineto{\pgfqpoint{10.399028in}{2.971219in}}%
\pgfpathlineto{\pgfqpoint{10.392634in}{2.971219in}}%
\pgfpathlineto{\pgfqpoint{10.386240in}{2.971219in}}%
\pgfpathlineto{\pgfqpoint{10.379845in}{2.971219in}}%
\pgfpathlineto{\pgfqpoint{10.373451in}{2.971219in}}%
\pgfpathlineto{\pgfqpoint{10.367057in}{2.971219in}}%
\pgfpathlineto{\pgfqpoint{10.360663in}{2.971219in}}%
\pgfpathlineto{\pgfqpoint{10.354268in}{2.971219in}}%
\pgfpathlineto{\pgfqpoint{10.347874in}{2.971219in}}%
\pgfpathlineto{\pgfqpoint{10.341480in}{2.971219in}}%
\pgfpathlineto{\pgfqpoint{10.335086in}{2.971219in}}%
\pgfpathlineto{\pgfqpoint{10.328691in}{2.971219in}}%
\pgfpathlineto{\pgfqpoint{10.322297in}{2.971219in}}%
\pgfpathlineto{\pgfqpoint{10.315903in}{2.971219in}}%
\pgfpathlineto{\pgfqpoint{10.309508in}{2.971219in}}%
\pgfpathlineto{\pgfqpoint{10.303114in}{2.971219in}}%
\pgfpathlineto{\pgfqpoint{10.296720in}{2.971219in}}%
\pgfpathlineto{\pgfqpoint{10.290326in}{2.971219in}}%
\pgfpathlineto{\pgfqpoint{10.283931in}{2.971219in}}%
\pgfpathlineto{\pgfqpoint{10.277537in}{2.971219in}}%
\pgfpathlineto{\pgfqpoint{10.271143in}{2.971219in}}%
\pgfpathlineto{\pgfqpoint{10.264748in}{2.971219in}}%
\pgfpathlineto{\pgfqpoint{10.258354in}{2.971219in}}%
\pgfpathlineto{\pgfqpoint{10.251960in}{2.971219in}}%
\pgfpathlineto{\pgfqpoint{10.245566in}{2.971219in}}%
\pgfpathlineto{\pgfqpoint{10.239171in}{2.971219in}}%
\pgfpathlineto{\pgfqpoint{10.232777in}{2.971219in}}%
\pgfpathlineto{\pgfqpoint{10.226383in}{2.971219in}}%
\pgfpathlineto{\pgfqpoint{10.219989in}{2.971219in}}%
\pgfpathlineto{\pgfqpoint{10.213594in}{2.971219in}}%
\pgfpathlineto{\pgfqpoint{10.207200in}{2.971219in}}%
\pgfpathlineto{\pgfqpoint{10.200806in}{2.971219in}}%
\pgfpathlineto{\pgfqpoint{10.194411in}{2.971219in}}%
\pgfpathlineto{\pgfqpoint{10.188017in}{2.971219in}}%
\pgfpathlineto{\pgfqpoint{10.181623in}{2.971219in}}%
\pgfpathlineto{\pgfqpoint{10.175229in}{2.971219in}}%
\pgfpathlineto{\pgfqpoint{10.168834in}{2.971219in}}%
\pgfpathlineto{\pgfqpoint{10.162440in}{2.971219in}}%
\pgfpathlineto{\pgfqpoint{10.156046in}{2.971219in}}%
\pgfpathlineto{\pgfqpoint{10.149651in}{2.971219in}}%
\pgfpathlineto{\pgfqpoint{10.143257in}{2.971219in}}%
\pgfpathlineto{\pgfqpoint{10.136863in}{2.971219in}}%
\pgfpathlineto{\pgfqpoint{10.130469in}{2.971219in}}%
\pgfpathlineto{\pgfqpoint{10.124074in}{2.971219in}}%
\pgfpathlineto{\pgfqpoint{10.117680in}{2.971219in}}%
\pgfpathlineto{\pgfqpoint{10.111286in}{2.971219in}}%
\pgfpathlineto{\pgfqpoint{10.104891in}{2.971219in}}%
\pgfpathlineto{\pgfqpoint{10.098497in}{2.971219in}}%
\pgfpathlineto{\pgfqpoint{10.092103in}{2.971219in}}%
\pgfpathlineto{\pgfqpoint{10.085709in}{2.971219in}}%
\pgfpathlineto{\pgfqpoint{10.079314in}{2.971219in}}%
\pgfpathlineto{\pgfqpoint{10.072920in}{2.971219in}}%
\pgfpathlineto{\pgfqpoint{10.066526in}{2.971219in}}%
\pgfpathlineto{\pgfqpoint{10.060132in}{2.971219in}}%
\pgfpathlineto{\pgfqpoint{10.053737in}{2.971219in}}%
\pgfpathlineto{\pgfqpoint{10.047343in}{2.971219in}}%
\pgfpathlineto{\pgfqpoint{10.040949in}{2.971219in}}%
\pgfpathlineto{\pgfqpoint{10.034554in}{2.971219in}}%
\pgfpathlineto{\pgfqpoint{10.028160in}{2.971219in}}%
\pgfpathlineto{\pgfqpoint{10.021766in}{2.971219in}}%
\pgfpathlineto{\pgfqpoint{10.015372in}{2.971219in}}%
\pgfpathlineto{\pgfqpoint{10.008977in}{2.971219in}}%
\pgfpathlineto{\pgfqpoint{10.002583in}{2.971219in}}%
\pgfpathlineto{\pgfqpoint{9.996189in}{2.971219in}}%
\pgfpathlineto{\pgfqpoint{9.989794in}{2.971219in}}%
\pgfpathlineto{\pgfqpoint{9.983400in}{2.971219in}}%
\pgfpathlineto{\pgfqpoint{9.977006in}{2.971219in}}%
\pgfpathlineto{\pgfqpoint{9.970612in}{2.971219in}}%
\pgfpathlineto{\pgfqpoint{9.964217in}{2.971219in}}%
\pgfpathlineto{\pgfqpoint{9.957823in}{2.971219in}}%
\pgfpathlineto{\pgfqpoint{9.951429in}{2.971219in}}%
\pgfpathlineto{\pgfqpoint{9.945034in}{2.971219in}}%
\pgfpathlineto{\pgfqpoint{9.938640in}{2.971219in}}%
\pgfpathlineto{\pgfqpoint{9.932246in}{2.971219in}}%
\pgfpathlineto{\pgfqpoint{9.925852in}{2.971219in}}%
\pgfpathlineto{\pgfqpoint{9.919457in}{2.971219in}}%
\pgfpathlineto{\pgfqpoint{9.913063in}{2.971219in}}%
\pgfpathlineto{\pgfqpoint{9.906669in}{2.971219in}}%
\pgfpathlineto{\pgfqpoint{9.900275in}{2.971219in}}%
\pgfpathlineto{\pgfqpoint{9.893880in}{2.971219in}}%
\pgfpathlineto{\pgfqpoint{9.887486in}{2.971219in}}%
\pgfpathlineto{\pgfqpoint{9.881092in}{2.971219in}}%
\pgfpathlineto{\pgfqpoint{9.874697in}{2.971219in}}%
\pgfpathlineto{\pgfqpoint{9.868303in}{2.971219in}}%
\pgfpathlineto{\pgfqpoint{9.861909in}{2.971219in}}%
\pgfpathlineto{\pgfqpoint{9.855515in}{2.971219in}}%
\pgfpathlineto{\pgfqpoint{9.849120in}{2.971219in}}%
\pgfpathlineto{\pgfqpoint{9.842726in}{2.971219in}}%
\pgfpathlineto{\pgfqpoint{9.836332in}{2.971219in}}%
\pgfpathlineto{\pgfqpoint{9.829937in}{2.971219in}}%
\pgfpathlineto{\pgfqpoint{9.823543in}{2.971219in}}%
\pgfpathlineto{\pgfqpoint{9.817149in}{2.971219in}}%
\pgfpathlineto{\pgfqpoint{9.810755in}{2.971219in}}%
\pgfpathlineto{\pgfqpoint{9.804360in}{2.971219in}}%
\pgfpathlineto{\pgfqpoint{9.797966in}{2.971219in}}%
\pgfpathlineto{\pgfqpoint{9.791572in}{2.971219in}}%
\pgfpathlineto{\pgfqpoint{9.785177in}{2.971219in}}%
\pgfpathlineto{\pgfqpoint{9.778783in}{2.971219in}}%
\pgfpathlineto{\pgfqpoint{9.772389in}{2.971219in}}%
\pgfpathlineto{\pgfqpoint{9.765995in}{2.971219in}}%
\pgfpathlineto{\pgfqpoint{9.759600in}{2.971219in}}%
\pgfpathlineto{\pgfqpoint{9.753206in}{2.971219in}}%
\pgfpathlineto{\pgfqpoint{9.746812in}{2.971219in}}%
\pgfpathlineto{\pgfqpoint{9.740418in}{2.971219in}}%
\pgfpathlineto{\pgfqpoint{9.734023in}{2.971219in}}%
\pgfpathlineto{\pgfqpoint{9.727629in}{2.971219in}}%
\pgfpathlineto{\pgfqpoint{9.721235in}{2.971219in}}%
\pgfpathlineto{\pgfqpoint{9.714840in}{2.971219in}}%
\pgfpathlineto{\pgfqpoint{9.708446in}{2.971219in}}%
\pgfpathlineto{\pgfqpoint{9.702052in}{2.971219in}}%
\pgfpathlineto{\pgfqpoint{9.695658in}{2.971219in}}%
\pgfpathlineto{\pgfqpoint{9.689263in}{2.971219in}}%
\pgfpathlineto{\pgfqpoint{9.682869in}{2.971219in}}%
\pgfpathlineto{\pgfqpoint{9.676475in}{2.971219in}}%
\pgfpathlineto{\pgfqpoint{9.670080in}{2.971219in}}%
\pgfpathlineto{\pgfqpoint{9.663686in}{2.971219in}}%
\pgfpathlineto{\pgfqpoint{9.657292in}{2.971219in}}%
\pgfpathlineto{\pgfqpoint{9.650898in}{2.971219in}}%
\pgfpathlineto{\pgfqpoint{9.644503in}{2.971219in}}%
\pgfpathlineto{\pgfqpoint{9.638109in}{2.971219in}}%
\pgfpathlineto{\pgfqpoint{9.631715in}{2.971219in}}%
\pgfpathlineto{\pgfqpoint{9.625321in}{2.971219in}}%
\pgfpathlineto{\pgfqpoint{9.618926in}{2.971219in}}%
\pgfpathlineto{\pgfqpoint{9.612532in}{2.971219in}}%
\pgfpathlineto{\pgfqpoint{9.606138in}{2.971219in}}%
\pgfpathlineto{\pgfqpoint{9.599743in}{2.971219in}}%
\pgfpathlineto{\pgfqpoint{9.593349in}{2.971219in}}%
\pgfpathlineto{\pgfqpoint{9.586955in}{2.971219in}}%
\pgfpathlineto{\pgfqpoint{9.580561in}{2.971219in}}%
\pgfpathlineto{\pgfqpoint{9.574166in}{2.971219in}}%
\pgfpathlineto{\pgfqpoint{9.567772in}{2.971219in}}%
\pgfpathlineto{\pgfqpoint{9.561378in}{2.971219in}}%
\pgfpathlineto{\pgfqpoint{9.554983in}{2.971219in}}%
\pgfpathlineto{\pgfqpoint{9.548589in}{2.971219in}}%
\pgfpathlineto{\pgfqpoint{9.542195in}{2.971219in}}%
\pgfpathlineto{\pgfqpoint{9.535801in}{2.971219in}}%
\pgfpathlineto{\pgfqpoint{9.529406in}{2.971219in}}%
\pgfpathlineto{\pgfqpoint{9.523012in}{2.971219in}}%
\pgfpathlineto{\pgfqpoint{9.516618in}{2.971219in}}%
\pgfpathlineto{\pgfqpoint{9.510223in}{2.971219in}}%
\pgfpathlineto{\pgfqpoint{9.503829in}{2.971219in}}%
\pgfpathlineto{\pgfqpoint{9.497435in}{2.971219in}}%
\pgfpathlineto{\pgfqpoint{9.491041in}{2.971219in}}%
\pgfpathlineto{\pgfqpoint{9.484646in}{2.971219in}}%
\pgfpathlineto{\pgfqpoint{9.478252in}{2.971219in}}%
\pgfpathlineto{\pgfqpoint{9.471858in}{2.971219in}}%
\pgfpathlineto{\pgfqpoint{9.465464in}{2.971219in}}%
\pgfpathlineto{\pgfqpoint{9.459069in}{2.971219in}}%
\pgfpathlineto{\pgfqpoint{9.452675in}{2.971219in}}%
\pgfpathlineto{\pgfqpoint{9.446281in}{2.971219in}}%
\pgfpathlineto{\pgfqpoint{9.439886in}{2.971219in}}%
\pgfpathlineto{\pgfqpoint{9.433492in}{2.971219in}}%
\pgfpathlineto{\pgfqpoint{9.427098in}{2.971219in}}%
\pgfpathlineto{\pgfqpoint{9.420704in}{2.971219in}}%
\pgfpathlineto{\pgfqpoint{9.414309in}{2.971219in}}%
\pgfpathlineto{\pgfqpoint{9.407915in}{2.971219in}}%
\pgfpathlineto{\pgfqpoint{9.401521in}{2.971219in}}%
\pgfpathlineto{\pgfqpoint{9.395126in}{2.971219in}}%
\pgfpathlineto{\pgfqpoint{9.388732in}{2.971219in}}%
\pgfpathlineto{\pgfqpoint{9.382338in}{2.971219in}}%
\pgfpathlineto{\pgfqpoint{9.375944in}{2.971219in}}%
\pgfpathlineto{\pgfqpoint{9.369549in}{2.971219in}}%
\pgfpathlineto{\pgfqpoint{9.363155in}{2.971219in}}%
\pgfpathlineto{\pgfqpoint{9.356761in}{2.971219in}}%
\pgfpathlineto{\pgfqpoint{9.350366in}{2.971219in}}%
\pgfpathlineto{\pgfqpoint{9.343972in}{2.971219in}}%
\pgfpathlineto{\pgfqpoint{9.337578in}{2.971219in}}%
\pgfpathlineto{\pgfqpoint{9.331184in}{2.971219in}}%
\pgfpathlineto{\pgfqpoint{9.324789in}{2.971219in}}%
\pgfpathlineto{\pgfqpoint{9.318395in}{2.971219in}}%
\pgfpathlineto{\pgfqpoint{9.312001in}{2.971219in}}%
\pgfpathlineto{\pgfqpoint{9.305607in}{2.971219in}}%
\pgfpathlineto{\pgfqpoint{9.299212in}{2.971219in}}%
\pgfpathlineto{\pgfqpoint{9.292818in}{2.971219in}}%
\pgfpathlineto{\pgfqpoint{9.286424in}{2.971219in}}%
\pgfpathlineto{\pgfqpoint{9.280029in}{2.971219in}}%
\pgfpathlineto{\pgfqpoint{9.273635in}{2.971219in}}%
\pgfpathlineto{\pgfqpoint{9.267241in}{2.971219in}}%
\pgfpathlineto{\pgfqpoint{9.260847in}{2.971219in}}%
\pgfpathlineto{\pgfqpoint{9.254452in}{2.971219in}}%
\pgfpathlineto{\pgfqpoint{9.248058in}{2.971219in}}%
\pgfpathlineto{\pgfqpoint{9.241664in}{2.971219in}}%
\pgfpathlineto{\pgfqpoint{9.235269in}{2.971219in}}%
\pgfpathlineto{\pgfqpoint{9.228875in}{2.971219in}}%
\pgfpathlineto{\pgfqpoint{9.222481in}{2.971219in}}%
\pgfpathlineto{\pgfqpoint{9.216087in}{2.971219in}}%
\pgfpathlineto{\pgfqpoint{9.209692in}{2.971219in}}%
\pgfpathlineto{\pgfqpoint{9.203298in}{2.971219in}}%
\pgfpathlineto{\pgfqpoint{9.196904in}{2.971219in}}%
\pgfpathlineto{\pgfqpoint{9.190509in}{2.971219in}}%
\pgfpathlineto{\pgfqpoint{9.184115in}{2.971219in}}%
\pgfpathlineto{\pgfqpoint{9.177721in}{2.971219in}}%
\pgfpathlineto{\pgfqpoint{9.171327in}{2.971219in}}%
\pgfpathlineto{\pgfqpoint{9.164932in}{2.971219in}}%
\pgfpathlineto{\pgfqpoint{9.158538in}{2.971219in}}%
\pgfpathlineto{\pgfqpoint{9.152144in}{2.971219in}}%
\pgfpathlineto{\pgfqpoint{9.145750in}{2.971219in}}%
\pgfpathlineto{\pgfqpoint{9.139355in}{2.971219in}}%
\pgfpathlineto{\pgfqpoint{9.132961in}{2.971219in}}%
\pgfpathlineto{\pgfqpoint{9.126567in}{2.971219in}}%
\pgfpathlineto{\pgfqpoint{9.120172in}{2.971219in}}%
\pgfpathlineto{\pgfqpoint{9.113778in}{2.971219in}}%
\pgfpathlineto{\pgfqpoint{9.107384in}{2.971219in}}%
\pgfpathlineto{\pgfqpoint{9.100990in}{2.971219in}}%
\pgfpathlineto{\pgfqpoint{9.094595in}{2.971219in}}%
\pgfpathlineto{\pgfqpoint{9.088201in}{2.971219in}}%
\pgfpathlineto{\pgfqpoint{9.081807in}{2.971219in}}%
\pgfpathlineto{\pgfqpoint{9.075412in}{2.971219in}}%
\pgfpathlineto{\pgfqpoint{9.069018in}{2.971219in}}%
\pgfpathlineto{\pgfqpoint{9.062624in}{2.971219in}}%
\pgfpathlineto{\pgfqpoint{9.056230in}{2.971219in}}%
\pgfpathlineto{\pgfqpoint{9.049835in}{2.971219in}}%
\pgfpathlineto{\pgfqpoint{9.043441in}{2.971219in}}%
\pgfpathlineto{\pgfqpoint{9.037047in}{2.971219in}}%
\pgfpathlineto{\pgfqpoint{9.030653in}{2.971219in}}%
\pgfpathlineto{\pgfqpoint{9.024258in}{2.971219in}}%
\pgfpathlineto{\pgfqpoint{9.017864in}{2.971219in}}%
\pgfpathlineto{\pgfqpoint{9.011470in}{2.971219in}}%
\pgfpathlineto{\pgfqpoint{9.005075in}{2.971219in}}%
\pgfpathlineto{\pgfqpoint{8.998681in}{2.971219in}}%
\pgfpathlineto{\pgfqpoint{8.992287in}{2.971219in}}%
\pgfpathlineto{\pgfqpoint{8.985893in}{2.971219in}}%
\pgfpathlineto{\pgfqpoint{8.979498in}{2.971219in}}%
\pgfpathlineto{\pgfqpoint{8.973104in}{2.971219in}}%
\pgfpathlineto{\pgfqpoint{8.966710in}{2.971219in}}%
\pgfpathlineto{\pgfqpoint{8.960315in}{2.971219in}}%
\pgfpathlineto{\pgfqpoint{8.953921in}{2.971219in}}%
\pgfpathlineto{\pgfqpoint{8.947527in}{2.971219in}}%
\pgfpathlineto{\pgfqpoint{8.941133in}{2.971219in}}%
\pgfpathlineto{\pgfqpoint{8.934738in}{2.971219in}}%
\pgfpathlineto{\pgfqpoint{8.928344in}{2.971219in}}%
\pgfpathlineto{\pgfqpoint{8.921950in}{2.971219in}}%
\pgfpathlineto{\pgfqpoint{8.915555in}{2.971219in}}%
\pgfpathlineto{\pgfqpoint{8.909161in}{2.971219in}}%
\pgfpathlineto{\pgfqpoint{8.902767in}{2.971219in}}%
\pgfpathlineto{\pgfqpoint{8.896373in}{2.971219in}}%
\pgfpathlineto{\pgfqpoint{8.889978in}{2.971219in}}%
\pgfpathlineto{\pgfqpoint{8.883584in}{2.971219in}}%
\pgfpathlineto{\pgfqpoint{8.877190in}{2.971219in}}%
\pgfpathlineto{\pgfqpoint{8.870796in}{2.971219in}}%
\pgfpathlineto{\pgfqpoint{8.864401in}{2.971219in}}%
\pgfpathlineto{\pgfqpoint{8.858007in}{2.971219in}}%
\pgfpathlineto{\pgfqpoint{8.851613in}{2.971219in}}%
\pgfpathlineto{\pgfqpoint{8.845218in}{2.971219in}}%
\pgfpathlineto{\pgfqpoint{8.838824in}{2.971219in}}%
\pgfpathlineto{\pgfqpoint{8.832430in}{2.971219in}}%
\pgfpathlineto{\pgfqpoint{8.826036in}{2.971219in}}%
\pgfpathlineto{\pgfqpoint{8.819641in}{2.971219in}}%
\pgfpathlineto{\pgfqpoint{8.813247in}{2.971219in}}%
\pgfpathlineto{\pgfqpoint{8.806853in}{2.971219in}}%
\pgfpathlineto{\pgfqpoint{8.800458in}{2.971219in}}%
\pgfpathlineto{\pgfqpoint{8.794064in}{2.971219in}}%
\pgfpathlineto{\pgfqpoint{8.787670in}{2.971219in}}%
\pgfpathlineto{\pgfqpoint{8.781276in}{2.971219in}}%
\pgfpathlineto{\pgfqpoint{8.774881in}{2.971219in}}%
\pgfpathlineto{\pgfqpoint{8.768487in}{2.971219in}}%
\pgfpathlineto{\pgfqpoint{8.762093in}{2.971219in}}%
\pgfpathlineto{\pgfqpoint{8.755698in}{2.971219in}}%
\pgfpathlineto{\pgfqpoint{8.749304in}{2.971219in}}%
\pgfpathlineto{\pgfqpoint{8.742910in}{2.971219in}}%
\pgfpathlineto{\pgfqpoint{8.736516in}{2.971219in}}%
\pgfpathlineto{\pgfqpoint{8.730121in}{2.971219in}}%
\pgfpathlineto{\pgfqpoint{8.723727in}{2.971219in}}%
\pgfpathlineto{\pgfqpoint{8.717333in}{2.971219in}}%
\pgfpathlineto{\pgfqpoint{8.710939in}{2.971219in}}%
\pgfpathlineto{\pgfqpoint{8.704544in}{2.971219in}}%
\pgfpathlineto{\pgfqpoint{8.698150in}{2.971219in}}%
\pgfpathlineto{\pgfqpoint{8.691756in}{2.971219in}}%
\pgfpathlineto{\pgfqpoint{8.685361in}{2.971219in}}%
\pgfpathlineto{\pgfqpoint{8.678967in}{2.971219in}}%
\pgfpathlineto{\pgfqpoint{8.672573in}{2.971219in}}%
\pgfpathlineto{\pgfqpoint{8.666179in}{2.971219in}}%
\pgfpathlineto{\pgfqpoint{8.659784in}{2.971219in}}%
\pgfpathlineto{\pgfqpoint{8.653390in}{2.971219in}}%
\pgfpathlineto{\pgfqpoint{8.646996in}{2.971219in}}%
\pgfpathlineto{\pgfqpoint{8.640601in}{2.971219in}}%
\pgfpathlineto{\pgfqpoint{8.634207in}{2.971219in}}%
\pgfpathlineto{\pgfqpoint{8.627813in}{2.971219in}}%
\pgfpathlineto{\pgfqpoint{8.621419in}{2.971219in}}%
\pgfpathlineto{\pgfqpoint{8.615024in}{2.971219in}}%
\pgfpathlineto{\pgfqpoint{8.608630in}{2.971219in}}%
\pgfpathlineto{\pgfqpoint{8.602236in}{2.971219in}}%
\pgfpathlineto{\pgfqpoint{8.595841in}{2.971219in}}%
\pgfpathlineto{\pgfqpoint{8.589447in}{2.971219in}}%
\pgfpathlineto{\pgfqpoint{8.583053in}{2.971219in}}%
\pgfpathlineto{\pgfqpoint{8.576659in}{2.971219in}}%
\pgfpathlineto{\pgfqpoint{8.570264in}{2.971219in}}%
\pgfpathlineto{\pgfqpoint{8.563870in}{2.971219in}}%
\pgfpathlineto{\pgfqpoint{8.557476in}{2.971219in}}%
\pgfpathlineto{\pgfqpoint{8.551082in}{2.971219in}}%
\pgfpathlineto{\pgfqpoint{8.544687in}{2.971219in}}%
\pgfpathlineto{\pgfqpoint{8.538293in}{2.971219in}}%
\pgfpathlineto{\pgfqpoint{8.531899in}{2.971219in}}%
\pgfpathlineto{\pgfqpoint{8.525504in}{2.971219in}}%
\pgfpathlineto{\pgfqpoint{8.519110in}{2.971219in}}%
\pgfpathlineto{\pgfqpoint{8.512716in}{2.971219in}}%
\pgfpathlineto{\pgfqpoint{8.506322in}{2.971219in}}%
\pgfpathlineto{\pgfqpoint{8.499927in}{2.971219in}}%
\pgfpathlineto{\pgfqpoint{8.493533in}{2.971219in}}%
\pgfpathlineto{\pgfqpoint{8.487139in}{2.971219in}}%
\pgfpathlineto{\pgfqpoint{8.480744in}{2.971219in}}%
\pgfpathlineto{\pgfqpoint{8.474350in}{2.971219in}}%
\pgfpathlineto{\pgfqpoint{8.467956in}{2.971219in}}%
\pgfpathlineto{\pgfqpoint{8.461562in}{2.971219in}}%
\pgfpathlineto{\pgfqpoint{8.455167in}{2.971219in}}%
\pgfpathlineto{\pgfqpoint{8.448773in}{2.971219in}}%
\pgfpathlineto{\pgfqpoint{8.442379in}{2.971219in}}%
\pgfpathlineto{\pgfqpoint{8.435985in}{2.971219in}}%
\pgfpathlineto{\pgfqpoint{8.429590in}{2.971219in}}%
\pgfpathlineto{\pgfqpoint{8.423196in}{2.971219in}}%
\pgfpathlineto{\pgfqpoint{8.416802in}{2.971219in}}%
\pgfpathlineto{\pgfqpoint{8.410407in}{2.971219in}}%
\pgfpathlineto{\pgfqpoint{8.404013in}{2.971219in}}%
\pgfpathlineto{\pgfqpoint{8.397619in}{2.971219in}}%
\pgfpathlineto{\pgfqpoint{8.391225in}{2.971219in}}%
\pgfpathlineto{\pgfqpoint{8.384830in}{2.971219in}}%
\pgfpathlineto{\pgfqpoint{8.378436in}{2.971219in}}%
\pgfpathlineto{\pgfqpoint{8.372042in}{2.971219in}}%
\pgfpathlineto{\pgfqpoint{8.365647in}{2.971219in}}%
\pgfpathlineto{\pgfqpoint{8.359253in}{2.971219in}}%
\pgfpathlineto{\pgfqpoint{8.352859in}{2.971219in}}%
\pgfpathlineto{\pgfqpoint{8.346465in}{2.971219in}}%
\pgfpathlineto{\pgfqpoint{8.340070in}{2.971219in}}%
\pgfpathlineto{\pgfqpoint{8.333676in}{2.971219in}}%
\pgfpathlineto{\pgfqpoint{8.327282in}{2.971219in}}%
\pgfpathlineto{\pgfqpoint{8.320887in}{2.971219in}}%
\pgfpathlineto{\pgfqpoint{8.314493in}{2.971219in}}%
\pgfpathlineto{\pgfqpoint{8.308099in}{2.971219in}}%
\pgfpathlineto{\pgfqpoint{8.301705in}{2.971219in}}%
\pgfpathlineto{\pgfqpoint{8.295310in}{2.971219in}}%
\pgfpathlineto{\pgfqpoint{8.288916in}{2.971219in}}%
\pgfpathlineto{\pgfqpoint{8.282522in}{2.971219in}}%
\pgfpathlineto{\pgfqpoint{8.276128in}{2.971219in}}%
\pgfpathlineto{\pgfqpoint{8.269733in}{2.971219in}}%
\pgfpathlineto{\pgfqpoint{8.263339in}{2.971219in}}%
\pgfpathlineto{\pgfqpoint{8.256945in}{2.971219in}}%
\pgfpathlineto{\pgfqpoint{8.250550in}{2.971219in}}%
\pgfpathlineto{\pgfqpoint{8.244156in}{2.971219in}}%
\pgfpathlineto{\pgfqpoint{8.237762in}{2.971219in}}%
\pgfpathlineto{\pgfqpoint{8.231368in}{2.971219in}}%
\pgfpathlineto{\pgfqpoint{8.224973in}{2.971219in}}%
\pgfpathlineto{\pgfqpoint{8.218579in}{2.971219in}}%
\pgfpathlineto{\pgfqpoint{8.212185in}{2.971219in}}%
\pgfpathlineto{\pgfqpoint{8.205790in}{2.971219in}}%
\pgfpathlineto{\pgfqpoint{8.199396in}{2.971219in}}%
\pgfpathlineto{\pgfqpoint{8.193002in}{2.971219in}}%
\pgfpathlineto{\pgfqpoint{8.186608in}{2.971219in}}%
\pgfpathlineto{\pgfqpoint{8.180213in}{2.971219in}}%
\pgfpathlineto{\pgfqpoint{8.173819in}{2.971219in}}%
\pgfpathlineto{\pgfqpoint{8.167425in}{2.971219in}}%
\pgfpathlineto{\pgfqpoint{8.161030in}{2.971219in}}%
\pgfpathlineto{\pgfqpoint{8.154636in}{2.971219in}}%
\pgfpathlineto{\pgfqpoint{8.148242in}{2.971219in}}%
\pgfpathlineto{\pgfqpoint{8.141848in}{2.971219in}}%
\pgfpathlineto{\pgfqpoint{8.135453in}{2.971219in}}%
\pgfpathlineto{\pgfqpoint{8.129059in}{2.971219in}}%
\pgfpathlineto{\pgfqpoint{8.122665in}{2.971219in}}%
\pgfpathlineto{\pgfqpoint{8.116271in}{2.971219in}}%
\pgfpathlineto{\pgfqpoint{8.109876in}{2.971219in}}%
\pgfpathlineto{\pgfqpoint{8.103482in}{2.971219in}}%
\pgfpathlineto{\pgfqpoint{8.097088in}{2.971219in}}%
\pgfpathlineto{\pgfqpoint{8.090693in}{2.971219in}}%
\pgfpathlineto{\pgfqpoint{8.084299in}{2.971219in}}%
\pgfpathlineto{\pgfqpoint{8.077905in}{2.971219in}}%
\pgfpathlineto{\pgfqpoint{8.071511in}{2.971219in}}%
\pgfpathlineto{\pgfqpoint{8.065116in}{2.971219in}}%
\pgfpathlineto{\pgfqpoint{8.058722in}{2.971219in}}%
\pgfpathlineto{\pgfqpoint{8.052328in}{2.971219in}}%
\pgfpathlineto{\pgfqpoint{8.045933in}{2.971219in}}%
\pgfpathlineto{\pgfqpoint{8.039539in}{2.971219in}}%
\pgfpathlineto{\pgfqpoint{8.033145in}{2.971219in}}%
\pgfpathlineto{\pgfqpoint{8.026751in}{2.971219in}}%
\pgfpathlineto{\pgfqpoint{8.020356in}{2.971219in}}%
\pgfpathlineto{\pgfqpoint{8.013962in}{2.971219in}}%
\pgfpathlineto{\pgfqpoint{8.007568in}{2.971219in}}%
\pgfpathlineto{\pgfqpoint{8.001173in}{2.971219in}}%
\pgfpathlineto{\pgfqpoint{7.994779in}{2.971219in}}%
\pgfpathlineto{\pgfqpoint{7.988385in}{2.971219in}}%
\pgfpathlineto{\pgfqpoint{7.981991in}{2.971219in}}%
\pgfpathlineto{\pgfqpoint{7.975596in}{2.971219in}}%
\pgfpathlineto{\pgfqpoint{7.969202in}{2.971219in}}%
\pgfpathlineto{\pgfqpoint{7.962808in}{2.971219in}}%
\pgfpathlineto{\pgfqpoint{7.956414in}{2.971219in}}%
\pgfpathlineto{\pgfqpoint{7.950019in}{2.971219in}}%
\pgfpathlineto{\pgfqpoint{7.943625in}{2.971219in}}%
\pgfpathlineto{\pgfqpoint{7.937231in}{2.971219in}}%
\pgfpathlineto{\pgfqpoint{7.930836in}{2.971219in}}%
\pgfpathlineto{\pgfqpoint{7.924442in}{2.971219in}}%
\pgfpathlineto{\pgfqpoint{7.918048in}{2.971219in}}%
\pgfpathlineto{\pgfqpoint{7.911654in}{2.971219in}}%
\pgfpathlineto{\pgfqpoint{7.905259in}{2.971219in}}%
\pgfpathlineto{\pgfqpoint{7.898865in}{2.971219in}}%
\pgfpathlineto{\pgfqpoint{7.892471in}{2.971219in}}%
\pgfpathlineto{\pgfqpoint{7.886076in}{2.971219in}}%
\pgfpathlineto{\pgfqpoint{7.879682in}{2.971219in}}%
\pgfpathlineto{\pgfqpoint{7.873288in}{2.971219in}}%
\pgfpathlineto{\pgfqpoint{7.866894in}{2.971219in}}%
\pgfpathlineto{\pgfqpoint{7.860499in}{2.971219in}}%
\pgfpathlineto{\pgfqpoint{7.854105in}{2.971219in}}%
\pgfpathlineto{\pgfqpoint{7.847711in}{2.971219in}}%
\pgfpathlineto{\pgfqpoint{7.841317in}{2.971219in}}%
\pgfpathlineto{\pgfqpoint{7.834922in}{2.971219in}}%
\pgfpathlineto{\pgfqpoint{7.828528in}{2.971219in}}%
\pgfpathlineto{\pgfqpoint{7.822134in}{2.971219in}}%
\pgfpathlineto{\pgfqpoint{7.815739in}{2.971219in}}%
\pgfpathlineto{\pgfqpoint{7.809345in}{2.971219in}}%
\pgfpathlineto{\pgfqpoint{7.802951in}{2.971219in}}%
\pgfpathlineto{\pgfqpoint{7.796557in}{2.971219in}}%
\pgfpathlineto{\pgfqpoint{7.790162in}{2.971219in}}%
\pgfpathlineto{\pgfqpoint{7.783768in}{2.971219in}}%
\pgfpathlineto{\pgfqpoint{7.777374in}{2.971219in}}%
\pgfpathlineto{\pgfqpoint{7.770979in}{2.971219in}}%
\pgfpathlineto{\pgfqpoint{7.764585in}{2.971219in}}%
\pgfpathlineto{\pgfqpoint{7.758191in}{2.971219in}}%
\pgfpathlineto{\pgfqpoint{7.751797in}{2.971219in}}%
\pgfpathlineto{\pgfqpoint{7.745402in}{2.971219in}}%
\pgfpathlineto{\pgfqpoint{7.739008in}{2.971219in}}%
\pgfpathlineto{\pgfqpoint{7.732614in}{2.971219in}}%
\pgfpathlineto{\pgfqpoint{7.726219in}{2.971219in}}%
\pgfpathlineto{\pgfqpoint{7.719825in}{2.971219in}}%
\pgfpathlineto{\pgfqpoint{7.713431in}{2.971219in}}%
\pgfpathlineto{\pgfqpoint{7.707037in}{2.971219in}}%
\pgfpathlineto{\pgfqpoint{7.700642in}{2.971219in}}%
\pgfpathlineto{\pgfqpoint{7.694248in}{2.971219in}}%
\pgfpathlineto{\pgfqpoint{7.687854in}{2.971219in}}%
\pgfpathlineto{\pgfqpoint{7.681460in}{2.971219in}}%
\pgfpathlineto{\pgfqpoint{7.675065in}{2.971219in}}%
\pgfpathlineto{\pgfqpoint{7.668671in}{2.971219in}}%
\pgfpathlineto{\pgfqpoint{7.662277in}{2.971219in}}%
\pgfpathlineto{\pgfqpoint{7.655882in}{2.971219in}}%
\pgfpathlineto{\pgfqpoint{7.649488in}{2.971219in}}%
\pgfpathlineto{\pgfqpoint{7.643094in}{2.971219in}}%
\pgfpathlineto{\pgfqpoint{7.636700in}{2.971219in}}%
\pgfpathlineto{\pgfqpoint{7.630305in}{2.971219in}}%
\pgfpathlineto{\pgfqpoint{7.623911in}{2.971219in}}%
\pgfpathlineto{\pgfqpoint{7.617517in}{2.971219in}}%
\pgfpathlineto{\pgfqpoint{7.611122in}{2.971219in}}%
\pgfpathlineto{\pgfqpoint{7.604728in}{2.971219in}}%
\pgfpathlineto{\pgfqpoint{7.598334in}{2.971219in}}%
\pgfpathlineto{\pgfqpoint{7.591940in}{2.971219in}}%
\pgfpathlineto{\pgfqpoint{7.585545in}{2.971219in}}%
\pgfpathlineto{\pgfqpoint{7.579151in}{2.971219in}}%
\pgfpathlineto{\pgfqpoint{7.572757in}{2.971219in}}%
\pgfpathlineto{\pgfqpoint{7.566362in}{2.971219in}}%
\pgfpathlineto{\pgfqpoint{7.559968in}{2.971219in}}%
\pgfpathlineto{\pgfqpoint{7.553574in}{2.971219in}}%
\pgfpathlineto{\pgfqpoint{7.547180in}{2.971219in}}%
\pgfpathlineto{\pgfqpoint{7.540785in}{2.971219in}}%
\pgfpathlineto{\pgfqpoint{7.534391in}{2.971219in}}%
\pgfpathlineto{\pgfqpoint{7.527997in}{2.971219in}}%
\pgfpathlineto{\pgfqpoint{7.521603in}{2.971219in}}%
\pgfpathlineto{\pgfqpoint{7.515208in}{2.971219in}}%
\pgfpathlineto{\pgfqpoint{7.508814in}{2.971219in}}%
\pgfpathlineto{\pgfqpoint{7.502420in}{2.971219in}}%
\pgfpathlineto{\pgfqpoint{7.496025in}{2.971219in}}%
\pgfpathlineto{\pgfqpoint{7.489631in}{2.971219in}}%
\pgfpathlineto{\pgfqpoint{7.483237in}{2.971219in}}%
\pgfpathlineto{\pgfqpoint{7.476843in}{2.971219in}}%
\pgfpathlineto{\pgfqpoint{7.470448in}{2.971219in}}%
\pgfpathlineto{\pgfqpoint{7.464054in}{2.971219in}}%
\pgfpathlineto{\pgfqpoint{7.457660in}{2.971219in}}%
\pgfpathlineto{\pgfqpoint{7.451265in}{2.971219in}}%
\pgfpathlineto{\pgfqpoint{7.444871in}{2.971219in}}%
\pgfpathlineto{\pgfqpoint{7.438477in}{2.971219in}}%
\pgfpathlineto{\pgfqpoint{7.432083in}{2.971219in}}%
\pgfpathlineto{\pgfqpoint{7.425688in}{2.971219in}}%
\pgfpathlineto{\pgfqpoint{7.419294in}{2.971219in}}%
\pgfpathlineto{\pgfqpoint{7.412900in}{2.971219in}}%
\pgfpathlineto{\pgfqpoint{7.406505in}{2.971219in}}%
\pgfpathlineto{\pgfqpoint{7.400111in}{2.971219in}}%
\pgfpathlineto{\pgfqpoint{7.393717in}{2.971219in}}%
\pgfpathlineto{\pgfqpoint{7.387323in}{2.971219in}}%
\pgfpathlineto{\pgfqpoint{7.380928in}{2.971219in}}%
\pgfpathlineto{\pgfqpoint{7.374534in}{2.971219in}}%
\pgfpathlineto{\pgfqpoint{7.368140in}{2.971219in}}%
\pgfpathlineto{\pgfqpoint{7.361746in}{2.971219in}}%
\pgfpathlineto{\pgfqpoint{7.355351in}{2.971219in}}%
\pgfpathlineto{\pgfqpoint{7.348957in}{2.971219in}}%
\pgfpathlineto{\pgfqpoint{7.342563in}{2.971219in}}%
\pgfpathlineto{\pgfqpoint{7.336168in}{2.971219in}}%
\pgfpathlineto{\pgfqpoint{7.329774in}{2.971219in}}%
\pgfpathlineto{\pgfqpoint{7.323380in}{2.971219in}}%
\pgfpathlineto{\pgfqpoint{7.323380in}{2.971219in}}%
\pgfpathclose%
\pgfusepath{stroke,fill}%
}%
\begin{pgfscope}%
\pgfsys@transformshift{0.000000in}{0.000000in}%
\pgfsys@useobject{currentmarker}{}%
\end{pgfscope}%
\end{pgfscope}%
\begin{pgfscope}%
\pgfsetbuttcap%
\pgfsetroundjoin%
\definecolor{currentfill}{rgb}{0.000000,0.000000,0.000000}%
\pgfsetfillcolor{currentfill}%
\pgfsetlinewidth{0.803000pt}%
\definecolor{currentstroke}{rgb}{0.000000,0.000000,0.000000}%
\pgfsetstrokecolor{currentstroke}%
\pgfsetdash{}{0pt}%
\pgfsys@defobject{currentmarker}{\pgfqpoint{0.000000in}{-0.048611in}}{\pgfqpoint{0.000000in}{0.000000in}}{%
\pgfpathmoveto{\pgfqpoint{0.000000in}{0.000000in}}%
\pgfpathlineto{\pgfqpoint{0.000000in}{-0.048611in}}%
\pgfusepath{stroke,fill}%
}%
\begin{pgfscope}%
\pgfsys@transformshift{7.323380in}{0.554012in}%
\pgfsys@useobject{currentmarker}{}%
\end{pgfscope}%
\end{pgfscope}%
\begin{pgfscope}%
\definecolor{textcolor}{rgb}{0.000000,0.000000,0.000000}%
\pgfsetstrokecolor{textcolor}%
\pgfsetfillcolor{textcolor}%
\pgftext[x=7.323380in,y=0.456790in,,top]{\color{textcolor}\rmfamily\fontsize{10.000000}{12.000000}\selectfont \(\displaystyle {0.0}\)}%
\end{pgfscope}%
\begin{pgfscope}%
\pgfsetbuttcap%
\pgfsetroundjoin%
\definecolor{currentfill}{rgb}{0.000000,0.000000,0.000000}%
\pgfsetfillcolor{currentfill}%
\pgfsetlinewidth{0.803000pt}%
\definecolor{currentstroke}{rgb}{0.000000,0.000000,0.000000}%
\pgfsetstrokecolor{currentstroke}%
\pgfsetdash{}{0pt}%
\pgfsys@defobject{currentmarker}{\pgfqpoint{0.000000in}{-0.048611in}}{\pgfqpoint{0.000000in}{0.000000in}}{%
\pgfpathmoveto{\pgfqpoint{0.000000in}{0.000000in}}%
\pgfpathlineto{\pgfqpoint{0.000000in}{-0.048611in}}%
\pgfusepath{stroke,fill}%
}%
\begin{pgfscope}%
\pgfsys@transformshift{8.600957in}{0.554012in}%
\pgfsys@useobject{currentmarker}{}%
\end{pgfscope}%
\end{pgfscope}%
\begin{pgfscope}%
\definecolor{textcolor}{rgb}{0.000000,0.000000,0.000000}%
\pgfsetstrokecolor{textcolor}%
\pgfsetfillcolor{textcolor}%
\pgftext[x=8.600957in,y=0.456790in,,top]{\color{textcolor}\rmfamily\fontsize{10.000000}{12.000000}\selectfont \(\displaystyle {0.2}\)}%
\end{pgfscope}%
\begin{pgfscope}%
\pgfsetbuttcap%
\pgfsetroundjoin%
\definecolor{currentfill}{rgb}{0.000000,0.000000,0.000000}%
\pgfsetfillcolor{currentfill}%
\pgfsetlinewidth{0.803000pt}%
\definecolor{currentstroke}{rgb}{0.000000,0.000000,0.000000}%
\pgfsetstrokecolor{currentstroke}%
\pgfsetdash{}{0pt}%
\pgfsys@defobject{currentmarker}{\pgfqpoint{0.000000in}{-0.048611in}}{\pgfqpoint{0.000000in}{0.000000in}}{%
\pgfpathmoveto{\pgfqpoint{0.000000in}{0.000000in}}%
\pgfpathlineto{\pgfqpoint{0.000000in}{-0.048611in}}%
\pgfusepath{stroke,fill}%
}%
\begin{pgfscope}%
\pgfsys@transformshift{9.878534in}{0.554012in}%
\pgfsys@useobject{currentmarker}{}%
\end{pgfscope}%
\end{pgfscope}%
\begin{pgfscope}%
\definecolor{textcolor}{rgb}{0.000000,0.000000,0.000000}%
\pgfsetstrokecolor{textcolor}%
\pgfsetfillcolor{textcolor}%
\pgftext[x=9.878534in,y=0.456790in,,top]{\color{textcolor}\rmfamily\fontsize{10.000000}{12.000000}\selectfont \(\displaystyle {0.4}\)}%
\end{pgfscope}%
\begin{pgfscope}%
\pgfsetbuttcap%
\pgfsetroundjoin%
\definecolor{currentfill}{rgb}{0.000000,0.000000,0.000000}%
\pgfsetfillcolor{currentfill}%
\pgfsetlinewidth{0.803000pt}%
\definecolor{currentstroke}{rgb}{0.000000,0.000000,0.000000}%
\pgfsetstrokecolor{currentstroke}%
\pgfsetdash{}{0pt}%
\pgfsys@defobject{currentmarker}{\pgfqpoint{0.000000in}{-0.048611in}}{\pgfqpoint{0.000000in}{0.000000in}}{%
\pgfpathmoveto{\pgfqpoint{0.000000in}{0.000000in}}%
\pgfpathlineto{\pgfqpoint{0.000000in}{-0.048611in}}%
\pgfusepath{stroke,fill}%
}%
\begin{pgfscope}%
\pgfsys@transformshift{11.156111in}{0.554012in}%
\pgfsys@useobject{currentmarker}{}%
\end{pgfscope}%
\end{pgfscope}%
\begin{pgfscope}%
\definecolor{textcolor}{rgb}{0.000000,0.000000,0.000000}%
\pgfsetstrokecolor{textcolor}%
\pgfsetfillcolor{textcolor}%
\pgftext[x=11.156111in,y=0.456790in,,top]{\color{textcolor}\rmfamily\fontsize{10.000000}{12.000000}\selectfont \(\displaystyle {0.6}\)}%
\end{pgfscope}%
\begin{pgfscope}%
\pgfsetbuttcap%
\pgfsetroundjoin%
\definecolor{currentfill}{rgb}{0.000000,0.000000,0.000000}%
\pgfsetfillcolor{currentfill}%
\pgfsetlinewidth{0.803000pt}%
\definecolor{currentstroke}{rgb}{0.000000,0.000000,0.000000}%
\pgfsetstrokecolor{currentstroke}%
\pgfsetdash{}{0pt}%
\pgfsys@defobject{currentmarker}{\pgfqpoint{0.000000in}{-0.048611in}}{\pgfqpoint{0.000000in}{0.000000in}}{%
\pgfpathmoveto{\pgfqpoint{0.000000in}{0.000000in}}%
\pgfpathlineto{\pgfqpoint{0.000000in}{-0.048611in}}%
\pgfusepath{stroke,fill}%
}%
\begin{pgfscope}%
\pgfsys@transformshift{12.433688in}{0.554012in}%
\pgfsys@useobject{currentmarker}{}%
\end{pgfscope}%
\end{pgfscope}%
\begin{pgfscope}%
\definecolor{textcolor}{rgb}{0.000000,0.000000,0.000000}%
\pgfsetstrokecolor{textcolor}%
\pgfsetfillcolor{textcolor}%
\pgftext[x=12.433688in,y=0.456790in,,top]{\color{textcolor}\rmfamily\fontsize{10.000000}{12.000000}\selectfont \(\displaystyle {0.8}\)}%
\end{pgfscope}%
\begin{pgfscope}%
\pgfsetbuttcap%
\pgfsetroundjoin%
\definecolor{currentfill}{rgb}{0.000000,0.000000,0.000000}%
\pgfsetfillcolor{currentfill}%
\pgfsetlinewidth{0.803000pt}%
\definecolor{currentstroke}{rgb}{0.000000,0.000000,0.000000}%
\pgfsetstrokecolor{currentstroke}%
\pgfsetdash{}{0pt}%
\pgfsys@defobject{currentmarker}{\pgfqpoint{0.000000in}{-0.048611in}}{\pgfqpoint{0.000000in}{0.000000in}}{%
\pgfpathmoveto{\pgfqpoint{0.000000in}{0.000000in}}%
\pgfpathlineto{\pgfqpoint{0.000000in}{-0.048611in}}%
\pgfusepath{stroke,fill}%
}%
\begin{pgfscope}%
\pgfsys@transformshift{13.711265in}{0.554012in}%
\pgfsys@useobject{currentmarker}{}%
\end{pgfscope}%
\end{pgfscope}%
\begin{pgfscope}%
\definecolor{textcolor}{rgb}{0.000000,0.000000,0.000000}%
\pgfsetstrokecolor{textcolor}%
\pgfsetfillcolor{textcolor}%
\pgftext[x=13.711265in,y=0.456790in,,top]{\color{textcolor}\rmfamily\fontsize{10.000000}{12.000000}\selectfont \(\displaystyle {1.0}\)}%
\end{pgfscope}%
\begin{pgfscope}%
\definecolor{textcolor}{rgb}{0.000000,0.000000,0.000000}%
\pgfsetstrokecolor{textcolor}%
\pgfsetfillcolor{textcolor}%
\pgftext[x=10.517323in,y=0.277777in,,top]{\color{textcolor}\rmfamily\fontsize{14.000000}{16.800000}\selectfont Normalized Quantity}%
\end{pgfscope}%
\begin{pgfscope}%
\pgfsetbuttcap%
\pgfsetroundjoin%
\definecolor{currentfill}{rgb}{0.000000,0.000000,0.000000}%
\pgfsetfillcolor{currentfill}%
\pgfsetlinewidth{0.803000pt}%
\definecolor{currentstroke}{rgb}{0.000000,0.000000,0.000000}%
\pgfsetstrokecolor{currentstroke}%
\pgfsetdash{}{0pt}%
\pgfsys@defobject{currentmarker}{\pgfqpoint{-0.048611in}{0.000000in}}{\pgfqpoint{-0.000000in}{0.000000in}}{%
\pgfpathmoveto{\pgfqpoint{-0.000000in}{0.000000in}}%
\pgfpathlineto{\pgfqpoint{-0.048611in}{0.000000in}}%
\pgfusepath{stroke,fill}%
}%
\begin{pgfscope}%
\pgfsys@transformshift{7.323380in}{0.554012in}%
\pgfsys@useobject{currentmarker}{}%
\end{pgfscope}%
\end{pgfscope}%
\begin{pgfscope}%
\pgfsetbuttcap%
\pgfsetroundjoin%
\definecolor{currentfill}{rgb}{0.000000,0.000000,0.000000}%
\pgfsetfillcolor{currentfill}%
\pgfsetlinewidth{0.803000pt}%
\definecolor{currentstroke}{rgb}{0.000000,0.000000,0.000000}%
\pgfsetstrokecolor{currentstroke}%
\pgfsetdash{}{0pt}%
\pgfsys@defobject{currentmarker}{\pgfqpoint{-0.048611in}{0.000000in}}{\pgfqpoint{-0.000000in}{0.000000in}}{%
\pgfpathmoveto{\pgfqpoint{-0.000000in}{0.000000in}}%
\pgfpathlineto{\pgfqpoint{-0.048611in}{0.000000in}}%
\pgfusepath{stroke,fill}%
}%
\begin{pgfscope}%
\pgfsys@transformshift{7.323380in}{1.520895in}%
\pgfsys@useobject{currentmarker}{}%
\end{pgfscope}%
\end{pgfscope}%
\begin{pgfscope}%
\pgfsetbuttcap%
\pgfsetroundjoin%
\definecolor{currentfill}{rgb}{0.000000,0.000000,0.000000}%
\pgfsetfillcolor{currentfill}%
\pgfsetlinewidth{0.803000pt}%
\definecolor{currentstroke}{rgb}{0.000000,0.000000,0.000000}%
\pgfsetstrokecolor{currentstroke}%
\pgfsetdash{}{0pt}%
\pgfsys@defobject{currentmarker}{\pgfqpoint{-0.048611in}{0.000000in}}{\pgfqpoint{-0.000000in}{0.000000in}}{%
\pgfpathmoveto{\pgfqpoint{-0.000000in}{0.000000in}}%
\pgfpathlineto{\pgfqpoint{-0.048611in}{0.000000in}}%
\pgfusepath{stroke,fill}%
}%
\begin{pgfscope}%
\pgfsys@transformshift{7.323380in}{2.487778in}%
\pgfsys@useobject{currentmarker}{}%
\end{pgfscope}%
\end{pgfscope}%
\begin{pgfscope}%
\pgfsetbuttcap%
\pgfsetroundjoin%
\definecolor{currentfill}{rgb}{0.000000,0.000000,0.000000}%
\pgfsetfillcolor{currentfill}%
\pgfsetlinewidth{0.803000pt}%
\definecolor{currentstroke}{rgb}{0.000000,0.000000,0.000000}%
\pgfsetstrokecolor{currentstroke}%
\pgfsetdash{}{0pt}%
\pgfsys@defobject{currentmarker}{\pgfqpoint{-0.048611in}{0.000000in}}{\pgfqpoint{-0.000000in}{0.000000in}}{%
\pgfpathmoveto{\pgfqpoint{-0.000000in}{0.000000in}}%
\pgfpathlineto{\pgfqpoint{-0.048611in}{0.000000in}}%
\pgfusepath{stroke,fill}%
}%
\begin{pgfscope}%
\pgfsys@transformshift{7.323380in}{3.454660in}%
\pgfsys@useobject{currentmarker}{}%
\end{pgfscope}%
\end{pgfscope}%
\begin{pgfscope}%
\pgfsetbuttcap%
\pgfsetroundjoin%
\definecolor{currentfill}{rgb}{0.000000,0.000000,0.000000}%
\pgfsetfillcolor{currentfill}%
\pgfsetlinewidth{0.803000pt}%
\definecolor{currentstroke}{rgb}{0.000000,0.000000,0.000000}%
\pgfsetstrokecolor{currentstroke}%
\pgfsetdash{}{0pt}%
\pgfsys@defobject{currentmarker}{\pgfqpoint{-0.048611in}{0.000000in}}{\pgfqpoint{-0.000000in}{0.000000in}}{%
\pgfpathmoveto{\pgfqpoint{-0.000000in}{0.000000in}}%
\pgfpathlineto{\pgfqpoint{-0.048611in}{0.000000in}}%
\pgfusepath{stroke,fill}%
}%
\begin{pgfscope}%
\pgfsys@transformshift{7.323380in}{4.421543in}%
\pgfsys@useobject{currentmarker}{}%
\end{pgfscope}%
\end{pgfscope}%
\begin{pgfscope}%
\pgfsetbuttcap%
\pgfsetroundjoin%
\definecolor{currentfill}{rgb}{0.000000,0.000000,0.000000}%
\pgfsetfillcolor{currentfill}%
\pgfsetlinewidth{0.803000pt}%
\definecolor{currentstroke}{rgb}{0.000000,0.000000,0.000000}%
\pgfsetstrokecolor{currentstroke}%
\pgfsetdash{}{0pt}%
\pgfsys@defobject{currentmarker}{\pgfqpoint{-0.048611in}{0.000000in}}{\pgfqpoint{-0.000000in}{0.000000in}}{%
\pgfpathmoveto{\pgfqpoint{-0.000000in}{0.000000in}}%
\pgfpathlineto{\pgfqpoint{-0.048611in}{0.000000in}}%
\pgfusepath{stroke,fill}%
}%
\begin{pgfscope}%
\pgfsys@transformshift{7.323380in}{5.388426in}%
\pgfsys@useobject{currentmarker}{}%
\end{pgfscope}%
\end{pgfscope}%
\begin{pgfscope}%
\pgfpathrectangle{\pgfqpoint{7.323380in}{0.554012in}}{\pgfqpoint{6.387885in}{4.834414in}}%
\pgfusepath{clip}%
\pgfsetrectcap%
\pgfsetroundjoin%
\pgfsetlinewidth{1.505625pt}%
\definecolor{currentstroke}{rgb}{0.121569,0.466667,0.705882}%
\pgfsetstrokecolor{currentstroke}%
\pgfsetdash{}{0pt}%
\pgfpathmoveto{\pgfqpoint{10.514125in}{5.390926in}}%
\pgfpathlineto{\pgfqpoint{10.520520in}{0.551512in}}%
\pgfpathlineto{\pgfqpoint{10.520520in}{0.551512in}}%
\pgfusepath{stroke}%
\end{pgfscope}%
\begin{pgfscope}%
\pgfpathrectangle{\pgfqpoint{7.323380in}{0.554012in}}{\pgfqpoint{6.387885in}{4.834414in}}%
\pgfusepath{clip}%
\pgfsetrectcap%
\pgfsetroundjoin%
\pgfsetlinewidth{1.505625pt}%
\definecolor{currentstroke}{rgb}{1.000000,0.498039,0.054902}%
\pgfsetstrokecolor{currentstroke}%
\pgfsetdash{}{0pt}%
\pgfpathmoveto{\pgfqpoint{7.323380in}{2.971219in}}%
\pgfpathlineto{\pgfqpoint{13.711265in}{2.971219in}}%
\pgfpathlineto{\pgfqpoint{13.711265in}{2.971219in}}%
\pgfusepath{stroke}%
\end{pgfscope}%
\begin{pgfscope}%
\pgfpathrectangle{\pgfqpoint{7.323380in}{0.554012in}}{\pgfqpoint{6.387885in}{4.834414in}}%
\pgfusepath{clip}%
\pgfsetbuttcap%
\pgfsetroundjoin%
\definecolor{currentfill}{rgb}{1.000000,0.000000,0.000000}%
\pgfsetfillcolor{currentfill}%
\pgfsetlinewidth{1.003750pt}%
\definecolor{currentstroke}{rgb}{1.000000,0.000000,0.000000}%
\pgfsetstrokecolor{currentstroke}%
\pgfsetdash{}{0pt}%
\pgfsys@defobject{currentmarker}{\pgfqpoint{-0.069444in}{-0.069444in}}{\pgfqpoint{0.069444in}{0.069444in}}{%
\pgfpathmoveto{\pgfqpoint{0.000000in}{-0.069444in}}%
\pgfpathcurveto{\pgfqpoint{0.018417in}{-0.069444in}}{\pgfqpoint{0.036082in}{-0.062127in}}{\pgfqpoint{0.049105in}{-0.049105in}}%
\pgfpathcurveto{\pgfqpoint{0.062127in}{-0.036082in}}{\pgfqpoint{0.069444in}{-0.018417in}}{\pgfqpoint{0.069444in}{0.000000in}}%
\pgfpathcurveto{\pgfqpoint{0.069444in}{0.018417in}}{\pgfqpoint{0.062127in}{0.036082in}}{\pgfqpoint{0.049105in}{0.049105in}}%
\pgfpathcurveto{\pgfqpoint{0.036082in}{0.062127in}}{\pgfqpoint{0.018417in}{0.069444in}}{\pgfqpoint{0.000000in}{0.069444in}}%
\pgfpathcurveto{\pgfqpoint{-0.018417in}{0.069444in}}{\pgfqpoint{-0.036082in}{0.062127in}}{\pgfqpoint{-0.049105in}{0.049105in}}%
\pgfpathcurveto{\pgfqpoint{-0.062127in}{0.036082in}}{\pgfqpoint{-0.069444in}{0.018417in}}{\pgfqpoint{-0.069444in}{0.000000in}}%
\pgfpathcurveto{\pgfqpoint{-0.069444in}{-0.018417in}}{\pgfqpoint{-0.062127in}{-0.036082in}}{\pgfqpoint{-0.049105in}{-0.049105in}}%
\pgfpathcurveto{\pgfqpoint{-0.036082in}{-0.062127in}}{\pgfqpoint{-0.018417in}{-0.069444in}}{\pgfqpoint{0.000000in}{-0.069444in}}%
\pgfpathlineto{\pgfqpoint{0.000000in}{-0.069444in}}%
\pgfpathclose%
\pgfusepath{stroke,fill}%
}%
\begin{pgfscope}%
\pgfsys@transformshift{10.514125in}{2.971219in}%
\pgfsys@useobject{currentmarker}{}%
\end{pgfscope}%
\end{pgfscope}%
\begin{pgfscope}%
\pgfpathrectangle{\pgfqpoint{7.323380in}{0.554012in}}{\pgfqpoint{6.387885in}{4.834414in}}%
\pgfusepath{clip}%
\pgfsetbuttcap%
\pgfsetroundjoin%
\pgfsetlinewidth{1.505625pt}%
\definecolor{currentstroke}{rgb}{1.000000,0.000000,0.000000}%
\pgfsetstrokecolor{currentstroke}%
\pgfsetstrokeopacity{0.600000}%
\pgfsetdash{{5.550000pt}{2.400000pt}}{0.000000pt}%
\pgfpathmoveto{\pgfqpoint{7.323380in}{2.971219in}}%
\pgfpathlineto{\pgfqpoint{10.514125in}{2.971219in}}%
\pgfusepath{stroke}%
\end{pgfscope}%
\begin{pgfscope}%
\pgfsetrectcap%
\pgfsetmiterjoin%
\pgfsetlinewidth{0.803000pt}%
\definecolor{currentstroke}{rgb}{0.000000,0.000000,0.000000}%
\pgfsetstrokecolor{currentstroke}%
\pgfsetdash{}{0pt}%
\pgfpathmoveto{\pgfqpoint{7.323380in}{0.554012in}}%
\pgfpathlineto{\pgfqpoint{7.323380in}{5.388426in}}%
\pgfusepath{stroke}%
\end{pgfscope}%
\begin{pgfscope}%
\pgfsetrectcap%
\pgfsetmiterjoin%
\pgfsetlinewidth{0.803000pt}%
\definecolor{currentstroke}{rgb}{0.000000,0.000000,0.000000}%
\pgfsetstrokecolor{currentstroke}%
\pgfsetdash{}{0pt}%
\pgfpathmoveto{\pgfqpoint{13.711265in}{0.554012in}}%
\pgfpathlineto{\pgfqpoint{13.711265in}{5.388426in}}%
\pgfusepath{stroke}%
\end{pgfscope}%
\begin{pgfscope}%
\pgfsetrectcap%
\pgfsetmiterjoin%
\pgfsetlinewidth{0.803000pt}%
\definecolor{currentstroke}{rgb}{0.000000,0.000000,0.000000}%
\pgfsetstrokecolor{currentstroke}%
\pgfsetdash{}{0pt}%
\pgfpathmoveto{\pgfqpoint{7.323380in}{0.554012in}}%
\pgfpathlineto{\pgfqpoint{13.711265in}{0.554012in}}%
\pgfusepath{stroke}%
\end{pgfscope}%
\begin{pgfscope}%
\pgfsetrectcap%
\pgfsetmiterjoin%
\pgfsetlinewidth{0.803000pt}%
\definecolor{currentstroke}{rgb}{0.000000,0.000000,0.000000}%
\pgfsetstrokecolor{currentstroke}%
\pgfsetdash{}{0pt}%
\pgfpathmoveto{\pgfqpoint{7.323380in}{5.388426in}}%
\pgfpathlineto{\pgfqpoint{13.711265in}{5.388426in}}%
\pgfusepath{stroke}%
\end{pgfscope}%
\begin{pgfscope}%
\definecolor{textcolor}{rgb}{0.000000,0.000000,0.000000}%
\pgfsetstrokecolor{textcolor}%
\pgfsetfillcolor{textcolor}%
\pgftext[x=7.642774in,y=3.454660in,left,base]{\color{textcolor}\rmfamily\fontsize{12.000000}{14.400000}\selectfont Consumer Surplus}%
\end{pgfscope}%
\begin{pgfscope}%
\definecolor{textcolor}{rgb}{0.000000,0.000000,0.000000}%
\pgfsetstrokecolor{textcolor}%
\pgfsetfillcolor{textcolor}%
\pgftext[x=7.642774in,y=2.487778in,left,base]{\color{textcolor}\rmfamily\fontsize{12.000000}{14.400000}\selectfont Producer Surplus}%
\end{pgfscope}%
\begin{pgfscope}%
\definecolor{textcolor}{rgb}{0.000000,0.000000,0.000000}%
\pgfsetstrokecolor{textcolor}%
\pgfsetfillcolor{textcolor}%
\pgftext[x=12.433688in,y=2.874531in,,top]{\color{textcolor}\rmfamily\fontsize{12.000000}{14.400000}\selectfont Supply}%
\end{pgfscope}%
\begin{pgfscope}%
\definecolor{textcolor}{rgb}{0.000000,0.000000,0.000000}%
\pgfsetstrokecolor{textcolor}%
\pgfsetfillcolor{textcolor}%
\pgftext[x=10.284019in, y=1.520895in, left, base,rotate=270.000000]{\color{textcolor}\rmfamily\fontsize{12.000000}{14.400000}\selectfont Demand}%
\end{pgfscope}%
\begin{pgfscope}%
\pgfsetbuttcap%
\pgfsetmiterjoin%
\definecolor{currentfill}{rgb}{1.000000,1.000000,1.000000}%
\pgfsetfillcolor{currentfill}%
\pgfsetlinewidth{1.003750pt}%
\definecolor{currentstroke}{rgb}{0.000000,0.000000,0.000000}%
\pgfsetstrokecolor{currentstroke}%
\pgfsetdash{}{0pt}%
\pgfpathmoveto{\pgfqpoint{7.330870in}{5.040317in}}%
\pgfpathlineto{\pgfqpoint{7.628599in}{5.040317in}}%
\pgfpathlineto{\pgfqpoint{7.628599in}{5.353094in}}%
\pgfpathlineto{\pgfqpoint{7.330870in}{5.353094in}}%
\pgfpathlineto{\pgfqpoint{7.330870in}{5.040317in}}%
\pgfpathclose%
\pgfusepath{stroke,fill}%
\end{pgfscope}%
\begin{pgfscope}%
\definecolor{textcolor}{rgb}{0.000000,0.000000,0.000000}%
\pgfsetstrokecolor{textcolor}%
\pgfsetfillcolor{textcolor}%
\pgftext[x=7.387259in,y=5.146705in,left,base]{\color{textcolor}\rmfamily\fontsize{14.000000}{16.800000}\selectfont b)}%
\end{pgfscope}%
\begin{pgfscope}%
\definecolor{textcolor}{rgb}{0.000000,0.000000,0.000000}%
\pgfsetstrokecolor{textcolor}%
\pgfsetfillcolor{textcolor}%
\pgftext[x=6.950000in,y=5.830000in,,top]{\color{textcolor}\rmfamily\fontsize{16.000000}{19.200000}\selectfont Price Elasticity}%
\end{pgfscope}%
\end{pgfpicture}%
\makeatother%
\endgroup%
}
  \caption{Demonstration of ``price elasticity.'' Plot a) shows a typical supply-demand curve where changes in price lead to proportional changes in demand. Plot b) shows an inelastic demand where consumption does not change proportionally with price.}
  \label{fig:inelastic}
\end{figure}

For an elastic good supply and demand are in proportion with each other. An
increase in the supply leads to a proportional increase in demand via a reduced
price, eventually returning to an equilibrium price (shown in Figure
\ref{fig:inelastic}a). However, as Figure \ref{fig:inelastic}b demonstrates, an
inelastic demand does not respond proportionally to changes in price, such that
consumers become ``price- takers,'' paying the price set by producers.
Importantly, in the latter case consumer surplus is infinite and minimizing the
energy cost through policy mechanisms does not create a lost surplus as shown in
Figure \ref{fig:social-max}b. Since electricity demand is highly inelastic,
economic dispatch models minimize the cost of generating electricity.
\textcolor{black}{Although optimizing welfare, rather than the total cost, is
useful for disaggregating multiple demands for the same commodity
\cite{leuthold_elmod_2008}, this thesis adopts the former, simplified, approach
to economic dispatch.}

\subsection{Accounting for Uncertainty}
\label{section:uncertainty}
Due to the complexity of our energy system, handling uncertainty is one of the
most important features for \acp{esom} \cite{yue_review_2018,
decarolis_using_2011}. There are broadly two types of uncertainties: parametric
and structural. The former refers to uncertainty around the value of some
empirical quantity (e.g. price of fuel or the discount rate). In many cases,
these quantities are better represented by \textit{distributions} which may be
sampled using formal methods like \ac{mc} or \ac{pa}
\cite{pfenninger_energy_2014, yue_review_2018}. Deterministic codes such as
TEMOA, TIMES, or ESME use these techniques to generate many model runs. Another
method for handling parametric uncertainty is \ac{sp}, where parameters are
replaced with non-linear risk functions \cite{yue_review_2018,
decarolis_multi-stage_2012}. Although parametric uncertainty is important the
analysis of uncertain values is not a focus of this thesis.

Structural uncertainty relates to \textit{unmodeled objectives}
\cite{yue_review_2018, decarolis_using_2011, decarolis_modelling_2016}. There
are few formal methods to address structural uncertainty due to its qualitative
nature. The most common approach to handling this type of uncertainty is using
\ac{mga} to probe the near-optimal decision space \cite{brill_mga_1990,
jenkins_genx_2022, decarolis_using_2011, neumann_near-optimal_2021,
pfenninger_energy_2014}. DeCarolis wrote, ``[p]olicy-makers often have strong
concerns outside the scope of most models (e.g., political feasibility,
permitting and regulation, and timing of action), which implies that feasible,
sub-optimal solutions may be preferable for reasons that are difficult to
quantify in energy economy optimization models'' \cite{decarolis_using_2011}.
Therefore, an ``optimal solution'' may lie in the model's inferior space
\cite{decarolis_using_2011}. Section \ref{section:mga} details the
implementation of \ac{mga}. \textcolor{black}{However, this approach still
requires an objective function, and the sub-optimal space is still within some
tolerance of the optimal value of the defined optimization space. Further, the solutions generated by \ac{mga}
still admit bias from policy-makers and does not require users to consider the
equity implications of these alternative solutions.} 

Another strategy to handle structural uncertainty is optimizing multiple
objectives simultaneously. However, some researchers dismissed this approach for
the following reasons \cite{decarolis_using_2011}:
\begin{enumerate}
    \item structural uncertainty will always exist, regardless of the number of
    modeled objectives;
    \item traditional \ac{moo} enables the exploration of a set of non-dominated
    solutions (i.e., the Pareto-front), but
    not the near-optimal space;
    \item analyzing tradeoffs for problems with many objectives is tedious.
\end{enumerate}
These critiques may explain the distinct lack of frameworks that apply \ac{moo}
for energy system problems. However, there are important benefits to \ac{moo} (
primarily the opportunity to analyze tradeoffs),
and the lack of an energy system \textit{framework} to apply this technique is
one of the gaps this thesis fulfills.



